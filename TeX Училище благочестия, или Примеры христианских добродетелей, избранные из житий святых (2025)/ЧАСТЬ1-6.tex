
\thispagestyle{empty}
\section[Предисловие составителя (1813 г.)]{Предисловие протоиерея\\Григория Ивановича Мансветова,\\составителя книги (1813~г.)}

Предлагая для пользы твоего сердца, благочестивый читатель, духовные повести, из Четьи"=Минеи и Пролога выбранные, за нужное почитаю объявить тому причину. Не однократно случалось мне видеть, что усерднейшие христиане, начиная читать Жития Святых, утомлялись на нескольких жизнеописаниях не потому, чтоб на их сердца не имели влияния происшествия, тут изображаемые, а потому только, что совершенство нынешнего языка, в рассуждении Четьи"=Минеи и Пролога, есть общий всем, так сказать, недостаток. Таким образом свет лишается лучшей нравоучительной книги "--- разительных примеров для своего поведения в делах веры и общежития.

Но сия потеря не ограничивается недостатком в воспитании; она влечет за собою чувствительнейший вред.

Молодые люди, образуемые по творениям древних и новых мудрецов, почерпают и для поведения своего правила из их жизнеописаний. Имена Сократа и Жан"=Жака, Катона и Дагесса, с первыми началами учения, уже врезываются в памяти их. Это весьма похвально, но бесчестно для христианина то, что, совершив весь круг учения, едва ли знают, кто были Богословы, Златоусты, Николаи, Амвросии...

Сия зараза так распространилась, что даже сочинители чего"=нибудь духовного подтверждают истину своих правил часто примером, какого"=нибудь чудака древности, или славного грешника новейших времен.

Что ж происходит из того? Ах! Пагубное глумление, будто на скрижалях Веры нет славных имен! Будто летописец ее не представляет примеров для добродетели! Какое торжество для нечестивого! И какая горесть для сердца христианского! Напротив того, сколь полезно было бы для юношества, если бы в Российских училищах (по крайней мере в Духовных приходских училищах хотя раз в неделю прочитывали детям по одной повести из Житий Святых. Не смею предлагать для сего предмета мною выбранные; каждый законоучитель, имея свой вкус, может из сего неистощимого, для истинных христиан, источника извлекать то, что, судя по званию воспитанников, покажется ему полезнейшим.

Впрочем, если христиане труд мой примут благосклонно, я постараюсь предлагаемую здесь книжку распространить столько, чтоб дети из житий угодников Господних имели по уроку к какой"=либо добродетели на каждый день. Если же мой выбор не понравится: утешусь тем, что проложил путь другим, которые более, нежели я, имеют способности.

Что касается до заглавия книги, она с начала названа мною была Детскою Четью"=Минеею: но Святейший Синод, удостоив ее своего благословения, благоволил наименовать \textsc{Училищем благочестия}. Коль сильное побуждение к продолжению трудов моих!

\chapter{ЧАСТЬ ПЕРВАЯ}
\section{Твердость Василия Великого\footnote{Свт. Василий Великий (330~-- 379) жил в царствование Юлиана Отступника и Валента. Управлял Церковью 14~лет. Память его празднуется 1~(14) января и 30~января (12~февраля).}}

Евсевий, градоначальник Кесарийский и родственник царский, вздумал одну вдову, молодую и богатую, по имени Вестиана, отдать в супружество своему любимцу, который несколько раз без успеха старался получить руку ее. Чудное дело! Безрассудный Вельможа, сжалившись над страданием одного, хотел заставить, чтоб страдала другая. Едва ли не такова вообще добродетель света!

Как скоро Вестиана узнала, что ей предстоит насилие, немедленно ушла в церковь и объявила святому Василию, сколь бедственны ее обстоятельства. Защищать беззащитных было приятнейшим удовольствием для угодника Божия. Он приемлет вдовицу под свое покровительство.

Разгневанный градоначальник повелевает воинам окружить церковь, исполнители беззаконной воли его повсюду ищут Вестиану, не уважая святыни. Идут в дом святого Василия, но, сколько ни искали ее, труд их был напрасен; ибо святой Василий, опасаясь, чтоб Вестиану с наглостью не исторгли из церкви, приказал тайно проводить ее в девический монастырь к сестре своей, преподобной Макрине.

Раздраженный Евсевий призывает к себе святого Василия; делает ему жестокие укоризны и клянется пытками заставить его сказать, где сокрыта жертва его насилия. «Я готов на все казни, "--- отвечает ему святой, "--- повели растерзать меня, повели источать по капле хладную кровь мою, но тайны не открою; мое тело в твоих руках, но дух мой в руке Божией». Может быть, в самом деле, смертью праведника кончился бы гнев и ярость градоначальника; но граждане узнали, что великий святитель их в опасности. Вдруг не только граждане, но и жены, даже дети, вооружившись кто чем мог, устремились к дому градоправителя, требовали крови его за святого отца и пастыря своего. Злодей ужаснулся, ярость переменил на отчаяние. А Василий Великий, выйдя к народу напомнить ему долг подчиненности, представил, сколь ужасны следствия бунта, и среди благословений народных возвратился в дом свой.

Где столь непоколебимое великодушие против насилия? Где столь надежное покровительство для невинности и, сверх того, где толикая кротость против своего гонителя?.. О христиане! Возьмите пример с Великого Василия.

\section{Непамятозлобие Василия Великого}

«Святитель Господень! "--- воскликнула на пути одна бедная вдовица к Василию Великому. "--- Защити меня, злосчастную, пред градоначальником: ты имеешь у него великую силу, а я, горькая сирота, не могу более переносить обид и притеснений…» Угодник Божий в тот же час написал к нему следующее: «Сия неимущая вдова сказала мне, что моя просьба всегда имеет у тебя силу; если это правда, докажи делом и сотвори милость несчастной…» Вдова с радостью приемлет письмо святителя и, воображая себя уже избавленной от всех притеснений, поспешно идет к градоначальнику.

Но судья не чувствовал счастья исполнить просьбу святого и, прочитав письмо, отвечал ему: «С удовольствием бы сделал для сей бедной женщины все, о чем просит она, но не в силах; ибо закон велит взыскивать дань, которой она подлежит».

Архиерей Божий дал на бедность вдовице что мог, а ему отписал вторично: «Если хотел ты, но не мог, не теряет цены добродетель твоя. Если же мог, но не хотел, то Небесный Судия некогда причтет тебя к требующим; и тогда восхочешь, но не возможешь…» Это предсказание оправдалось вскоре.

Царь услышал о любостяжании и обидах градоначальника, разгневался и велел заключить его в узы. Тогда"=то неправедный судья вспомнил, что приговор святого Василия не был игрой слов, как думал прежде, и, едва поднимая отягченные оковами руки, написал к нему: «Человек Божий! Слово твое свято; я причтен к плачевному лику требующих: преклони к несчастному на милость царя, отца народа. Угодник Божий! Ты всегда можешь, когда хочешь». Святой Василий не отверг просьбы погибающего; царь не отверг просьбы праведника. Градоначальник, получив свободу, повергся к стопам святителя и от него спешил воздать сторицей бедной вдове, которую прежде не хотел и слушать.

Видя несчастного, всегда должно ставить себя на его место; ибо счастье этого света непостоянно. Кто славен сегодня, может быть, завтра сделается последним. Отсюда можно заключить, хорошо ли поступают люди гордые и безжалостные?

\section{О трудолюбии\footnote{Из Пролога, в 18"~й день января.}}

К настоятелю некоего монастыря на горе Синайской, Силуану, однажды пришел принять благословение инок и, увидев братию в трудах, сказал: \textit{Не делайте брашна гиблющего, Мария бо благую часть избра}.\footnote{\textit{Старайтесь не о пище тленной} (Ин. 6, 27); \textit{Мария же избрала благую часть} (Лк. 10, 42).} Игумен, услышав это, приказал смотрителю отвести для гостя отдельную комнату и там оставить его. Чрез несколько часов иноки начали обедать, между тем гость смотрел в окно и ждал, скоро ли позовут его. Иноки вышли из"=за стола, а о нем никто и не позаботился. Наконец, побуждаемый голодом, он пришел к настоятелю и спросил, ели ли сегодня иноки? «Ели», "--- отвечал настоятель. «Да почто ж не повестили меня?» "--- возразил пришелец. «Ты человек духовный и не требуешь пищи, "--- сказал настоятель, "--- а мы одеты во плоть, хотим есть, вот почему и трудимся; ты избрал благую часть: читай же книги и питайся словами духовными».

Старец устыдился своего умствования и, поклонившись до земли, просил у настоятеля прощения. «Любезный о Христе брат! "--- поднимая его, сказал игумен. "--- С какой стороны ни рассматривай, но от трудов всегда есть польза; и Мария от Спасителя заслужила похвалу "--- для сестры своей, Марфы».

Праздность часто бывает причиной пороков; ибо тогда человек, не имея чем заняться, дает волю своим страстям, которые увлекают его в погибель.

\section{Юный благотворитель\footnote{Из Пролога, в 23"~й день января.}}

Святой Климент, сын идолопоклонника и христианки, в нежной юности лишившись отца и матери, был усыновлен Софией, благородной и добродетельной гражданкой Анкиры, подругой его матери. Руководимый ее примером, юный христианин воспитывался в страхе Господнем.

Тогда сделался в Галатии\footnote{\textit{Галатия} (греч. \textsf{Γαλατία}; лат. Galatia) "--- область в Малой Азии (совр. Турция), на Анатолийском нагорье, граничила на севере с Вифинией и Пафлагонией, на востоке с Понтом и Каппадокией, на юге с Писидией и Ликаонией, на западе с Фригией. Получила название от галатов (галлов), кельтских племен, пришедших в эту местность в III~в. до~Р.Х. Неоднократно упоминается в Посланиях и Деяниях св. апостолов, так как здесь при участии ап. Павла и его спутников была создана одна из первых общин христиан.} столь великий голод, что жители, по большей части неверные, не имея чем питать детей своих, оставляли их на произвол судьбы. Что ж делал святой отрок? Он собирал сверстников, повсюду скитающихся без надежды утолить голод свой, и младенцев на распутье, испускающих последний вопль, учил грамоте и вере Евангельской. Таким образом, дом Софии, через святого Климента, сделался гостиницей и училищем, а София "--- более, нежели матерью; ибо, возвращая бедным отрокам и младенцам жизнь временную, уготавливала для них чрез юного благотворителя и вечную жизнь.

\section{Величайшая жертва\footnote{Из Пролога, в 23"~й день января.}}

Святой Павлин, епископ Ноланский\footnote{Свт. Павлин Милостивый, епископ Ноланский (409/411--431). Память его празднуется 23~января (5~февраля).}, был столь милостив, что все свое имущество раздал бедным соотечественникам для искупления у вандалов пленников. Наконец, когда у праведника не осталось не только денег, но и платья, которое бы он мог продать, пришла к нему бедная вдовица и просила у него помощи, чтобы выкупить единственного ее сына. Раб Божий Павлин обыскал весь дом свой, но, не найдя ни полушки, сказал рыдающей матери: «Кроме меня самого, нет у меня ничего; итак, возьми меня и как раба своего отдай за сына…» Бедная вдовица подумала, что святитель над нею смеется, и хотела идти домой, но святой Павлин остановил ее, уверил, что говорит сущую правду, и убедил ее епископа отдать за сына.

Здесь обыкновенные люди удивятся и поступок святого человека почтут неблагоразумным. Будучи не в состоянии выкупить одного юношу, скажут они, Павлин мог бы в другое время освободить из рабства несколько человек. Но пусть они прочтут слова Спасителя: \textit{болши сея любве никтоже имать, да кто душу свою положит за други своя} (Ин. 15, 13). Сверх того, пусть смотрят, сколь спасительны были следствия от этого столь решительного поступка.

Святой Павлин и вдовица скрылись из города и пошли к вандалам. Достигнув места, вдовица пала к ногам князя вандальского, зятя царя Рикса, и просила его, чтобы он раба взял за сына. Долго жестокосердый господин не соглашался, долго не уважал слез матери; но, услышав, что Павлин "--- искусный садовник, принял старца за юношу.

Вдовица с сыном возвратилась в отечество, а праведник начал трудиться в саду господина своего и получил приказание "--- каждый день к столу его приносить различные травы и плоды. Вельможа, будучи охотник до садов, также часто ходил к Павлину и, всегда подолгу разговаривая с ним, узнал его разум и мудрость и, наконец, так полюбил его, что почувствовал необходимость каждый день видеть его. Однажды, между прочими разговорами с господином, святой Павлин втайне поведал ему: «Скоро встретятся с тобою важные обстоятельства: царь Рикс умрет скоропостижно; смотри же, не отлучайся из столицы; в противном случае кто"=нибудь другой захватит власть и венец царский». Удивленный столь важным пророчеством, вельможа верил и не верил словам праведника; но, будучи весьма любим царем, в тот же час пошел к нему и сказал все, что слышал от своего садовника. «Я хочу сам видеть человека этого, "--- отвечал царь. "--- Прикажи ему принести к столу моему трав и плодов».

Едва Павлин по приказанию вошел в палату, Рикс затрепетал. Потом, опомнившись, подозвал к себе зятя и сказал ему: «Слова его справедливы. В прошедшую ночь я видел сон, будто мои вельможи сидят в совете, а сей человек выше всех их: разговаривая между собою некоторое время, вдруг они вынесли приговор, чтобы отнять у меня царство; итак, спроси его, кто он? И откуда? Я не думаю, чтобы сей старец был простой человек». Тогда зять царский отвел святого Павлина в сторону и заклинал его Богом, чтобы тот сказал о себе всю истину. Праведник ни за что не обнаружил бы себя, но для имени Господня, которым заклинал его вельможа, хотя и с душевным прискорбием, сказал, что он епископ.

Рикс и зять его объяты были ужасом. «Прости меня, человек Божий, "--- сказал царь, "--- что я, по неведению, возложил на тебя труд рабский! Требуй от меня чего хочешь и возвращайся в землю твою с великими дарами». "--- «Одного прошу у тебя, "--- смиренно отвечал святой Павлин, "--- отпусти пленников из всех мест моего отечества; это величайший дар, которым можешь меня напутствовать».

Тогда везде разосланы были указы царские. Все пленники собирались в одно место и были отданы святому Павлину. Наконец, праведник с честью, осыпаемый благословениями, отправился в отечество и возвратил спокойствие и радость отцам и матерям, супругам и детям, родственникам и друзьям.

Сам Господь сохраняет жизнь и свободу того, кто ими жертвует для спасения жизни и свободы своих ближних.

\section{Святой Григорий Богослов\footnote{Свт. Григорий Богослов, архиепископ Константинопольский (330~-- 389), сын Григория Старшего, жил в царствование Феодосия Великого; управлял Церковью Божией 12~лет. Память его празднуется 25~января (7~февраля) и 30~января (12~февраля).} отказывается от патриаршества, чтобы даровать мир Церкви}

Святитель Христов Григорий, занимаясь богомыслием, любил жизнь уединенную; но царь и народ призывали его в Царьград на престол патриарший. Как ни отрекался угодник Божий принять на себя столь важную должность, но тщетно. Собрались сто пятьдесят епископов и под председательством святого Мелетия, патриарха Антиохийского, вручили ему жезл пастырский.

Спустя несколько дней святой Мелетий преставился, и епископы Египетские и Македонские вознегодовали на Григория, утверждая, что он, будучи возведен на престол свой не Александрийским патриархом, яко старшим, против правил получил сан великого святителя. Восстала распря и междоусобие. Угодник Божий, видя в сане своем причину раздоров между пастырями духовными, мгновенно приемлет намерение решительное: представ посреди собора, он говорит им: «Пастыри святого стада Христова! Стыдно вам других учить миру и любви, а между собою воздвигать брань; ибо увещания к согласию и единодушию не только недействительны, но и обращаются в посмеяние, когда происходят из сердца, обуреваемого раздорами. Заклинаю вас Пресвятою Единосущною Троицею: имейте между собою мир и любовь и устраивайте дела Церкви согласно. Если я, один я, причина вашего разномыслия и ссор, то я не лучше пророка Ионы; ввергните меня в море, и утолится между вами буря междоусобия; корабль духовный войдет в пристанище мира. Хотя невинен я, но готов претерпеть все, что вам угодно; только примиритесь и будьте единодушны».

Эта твердость святого Григория всех привела в стыд и смущение; никто не мог ответить ему; многие обливались слезами. Увидев это, угодник Божий, сам прослезившись, воскликнул: «Простите, служители Православия; поминайте меня, грешного, в молитвах ваших». Он вышел из собрания, спросил у царя увольнение от должности и немедленно отбыл в отечество на жизнь уединенную.

\section{Милостыня от подаяния\footnote{Из Пролога, в 30"~й день января.}}

Преподобный Зинон, ученик святого Василия Великого, сорок лет в постничестве богоугодно проводивший, сначала ни от кого и ничего принимать не хотел, поэтому щедролюбцы, унося свои подаяния назад, уходили от него с душевной печалью о том, что святой муж не приемлет от них. Между тем приходили к нему бедные, желая что"=нибудь получить от него как от великого старца; но поскольку Зинон не имел, что дать, то и эти уходили от него также печальны. «Что предприму? Что буду делать? "--- наконец сказал сам себе старец. "--- Печалятся и те, которые назад уносят дары свои; печалятся и те, которые хотят что"=нибудь принять от меня. Буду же поступать таким образом: если кто мне принесет что"=нибудь, я возьму; если кто у меня попросит, я отдам». Соблюдая это, святой Зинон и сам был спокоен, и других успокаивал.

\section{По какому случаю установлен праздник Сретения Господня}

В один злополучный год царствования Юстиниана, с последних чисел октября, в Византии и окрестных странах открылось моровое поветрие и свирепствовало так сильно, что стало умирать тысяч по десять каждодневно; в некоторых местах оказалось такое запустение, что \textit{не бе погребаяй} (Пс. 78, 3). Но в Антиохии, сверх этой язвы, продолжалось ужасное землетрясение: ежедневно разрушались дома и церкви, и множество народа погибало под их развалинами. Среди прочих был подавлен и Евфрасий, епископ Антиохийский, в то время приносивший бескровную жертву в храме Господнем. Мизийский город Помпеополь до половины был разрушен, а другая половина его поглощена землей с тысячами жителей. Дабы отвратить этот гнев Господень, по особенному откровению некоторому святому человеку установлено торжественно праздновать \textit{Сретение Господне}. И как скоро, февраля второго дня, начали \textit{всенощное бдение} с крестным ходом, в тот час язва миновала, мор перестал, землетрясение укротилось.

\section{Симеон Богоприимец\footnote{Из Четии"=Минеи, в 3"~й день февраля.}}

Старец Симеон, по свидетельству Священного Писания, был человек \textit{праведный и благочестивый}, просвещенный Духом Святым, \textit{чающий утешения Израилева} (Лк. 2, 25) по следующему откровению.

Когда Птоломей, царь Египетский, основатель славной Александрийской библиотеки, приказал Моисеевы и пророческие книги перевести с еврейского языка на греческий, тогда из израильтян выбраны были семьдесят мудрых и как в том, так и в другом языке искусных мужей\footnote{Они вообще известны под именем 70~толковников.}. В числе их был и праведный Симеон. Занимаясь переложением книги Исаии, когда Симеон достиг пророчества: \textit{Се, Дева во чреве зачнет и родит Сына} (Ис. 7, 14), усомнился и, остановившись, рассуждал сам с собой: как девица может родить отроча? Потом взял нож и в своей рукописи хотел изгладить оное… Вдруг Ангел Господень явился ему и, удержав руку, сказал: «Веруй тому, что написано; ты сам узришь событие величайшего таинства; ибо не увидишь смерти, доколе не узришь имеющего родиться от Пречистой Девы Христа Господня». Утвердившись на словах Ангела и пророка, с того времени Симеон с пламенным желанием ожидал пришествия Христова: жил праведно и непорочно и, не выходя из церкви, молился Богу.

Когда же рожденный Спаситель после сорока дней Богоблагодатной Марией принесен был в церковь, тогда святой, познав свыше, что этот Младенец есть обетованный Мессия и эта Матерь есть Святая Дева, на которой исполнилось пророчество Исаии, с благоговением и радостью принял в объятия свои Господа и воскликнул: «\textit{Ныне отпущаеши раба Твоего, Владыко, по глаголу Твоему с миром, яко видесте очи мои спасение Твое}» (Лк. 2, 29) "--- и, возблагодарив Бога, давшего утешение Израилю, испустил на небо дух свой.

Да подражаем в житии и вере праведному Симеону! Тогда здесь, в мире, удостоимся принять не в объятия, но в сердца наши Христа Господня; а после смерти \textit{восхищены будем на облацех в сретение Господне на воздусе}. Таким образом, \textit{всегда с Господем будем} (1~Фес. 4, 17).

\section{О терпении\footnote{Из Пролога, в 15"~й день февраля.}}

Некоторый инок был любим пятью старцами, но не нравился одному, который и старался оскорблять его разным образом. Инок вышел из терпения и, оставив обитель, удалился в другую. Там восемь человек обходились с ним по"=дружески, но двое ненавидели. Инок ушел оттуда в третий монастырь; но там семь только братьев изъявляли некоторое к нему благорасположение, а пятеро смотрели на него с превеликим неудовольствием. Инок опять решился искать нового убежища.

На пути старец размышлял о своем несчастии и ужасался, воображая, что с ним случится везде то же самое; он искал средств, как бы ему ужиться на одном месте, и после долгого размышления нашел. Это было \textit{терпение}! Обрадовавшись счастливой мысли, он взял свиток бумаги и написал: \textit{терпи}! Потом вошел в первый монастырь, который ему встретился.

Тут нередко случалось доброму старцу принимать обиды не только от одного, от двух, но иногда и от всей братии. Однако он, дав обет Богу, не хотел нарушать его. Никогда не покушался, даже в мыслях, переменить место свое, и, когда весьма чувствительно кто"=нибудь огорчит его, он только вынет хартийцу и прочтет: «Во имя Иисуса, Сына Божия, терплю» "--- и после этого успокоится. Таким образом добродетельный старец проводил жизнь спокойно.

Кто желает спокойно жить в обществе, тот должен равнодушно смотреть на все неудовольствия, которые ему причинять будут; без этой необходимо нужной добродетели человек будет менять место за местом и вся его жизнь будет подобна вихрю.

\section{Пагубно верить снам\footnote{Из Пролога, в 26"~й день февраля.}}

В некоторой обители был инок, украшенный всеми добродетелями и за то весьма уважаемый братией. К несчастию, он всегда верил сновидениям.

Дух"=искуситель весьма радуется, когда узнает в человеке слабую сторону, с которой легко может победить его; дух"=искуситель всею адскою силою вооружился на инока.

Каждую ночь, как скоро инок после обыкновенных молитв задремлет, враг рода человеческого начинает показывать ему сновидения, сначала безвредные, чтобы тем более обольстить несчастного, и, в какую сторону старец ни толковал их, каждый сон оправдывает событием наяву. Наконец, увидев, что заблудший старец всему верит, в одну злополучную ночь тот изобразил пред ним жизнь будущую. Изобразил, что апостолы, мученики, преподобные и христиане сидят в ужасной тьме, терзаемые отчаянием. А в другой стороне вместе с пророками и древними патриархами ликует народ еврейский, и Бог Отец, указывая на них перстом, вещает: «Се чада Моя!» Старец от ужаса пробудился и, не рассуждая ни о чем, ушел в Палестину, в жилища иудейские. Там принял обрезание и стал ревностным защитником убийц Христовых.

Но Бог сколько долготерпелив, столько и правосуден: через три года Бог наслал на него болезнь столь лютую, что сгнили даже кости его, и отступник в ужасных мучениях испустил дух свой.

Кто слепо верит снам, тот погрешает не менее, как если бы ворожил на картах. Лучше надеяться на Промысл Божий и не вверяться тому, что бывает часто действием заблуждающегося воображения.

\section{Наказанный ненавистник}

В Киево"=Печерской Лавре были два инока, священник Тит\footnote{Прп. Тит, пресвитер Печерский, в Ближних пещерах (†~1190). Из Четии"=Минеи, в 27"~й день февраля. Память его празднуется 27~февраля (11~марта).} и диакон Евагрий. Несколько лет они жили между собою так дружелюбно, что прочие братия дивились их единодушию. Но враг рода человеческого искони сеет плевелы посреди пшеницы! Он посеял между ними вражду и гневом и ненавистью так омрачил их, что они не могли без досады даже взглянуть друг на друга. Когда, отправляя Божию службу, один из них шел с кадильницей по церкви, то другой отходил в сторону, чтобы не услышать фимиама, а если иногда этот последний оставался на своем месте, то первый проходил от него как можно далее. Эта взаимная злоба продолжалась весьма долго, и они, не примирившись между собою, дерзали приносить бескровную жертву Богу. Сколько братия ни советовали им, чтобы отложили гнев и жили между собою в мире и согласии, но все было тщетно!

Однажды священник Тит тяжко разболелся. Отчаявшись в жизни, он начал горько плакать о своем согрешении и послал к недругу своему просить прощения. Но Евагрий не хотел слышать о том и начал жестоко проклинать его. Братия, соболезнуя о столь тяжком заблуждении, насильно привлекли его к умирающему. Тит, увидев врага своего, с помощью других встал с одра и пал пред ним, слезно умоляя простить его. Но Евагрий был так бесчеловечен, что отвратился от него и с остервенением воскликнул: «Ни в этой, ни в будущей жизни не хочу примириться с ним!» Он вырвался из рук братии и пал на землю. Иноки хотели поднять его. Но как изумились, увидев его мертвым и настолько охладевшим, как будто бы он умер некоторое время тому назад! Их изумление умножилось еще более, когда священник Тит в то же самое время встал с одра болезни здрав, будто никогда болен не был. В ужасе от столь необыкновенного происшествия, они окружили Тита и один за другим спрашивали: «Что значит это?»

«Будучи в тяжкой болезни, "--- отвечал он, "--- доколе я, грешный, сердился на брата моего, видел Ангелов, от меня отступивших и плачущих о погибели души моей, а нечистых духов "--- радующихся, "--- вот причина, почему я всего более желал примириться с ним. Но, как скоро привели его сюда и я поклонился ему, а он начал проклинать меня, я увидел, что один грозный Ангел поразил его пламенным копьем и несчастный мертвым повергся на землю; а мне этот же Ангел подал руку и восстановил от одра болезни».

Иноки оплакали лютую смерть Евагрия и с того времени более прежнего начали блюстись, да никогда солнце не зайдет во гневе их.

Памятозлобие есть порок ужаснейший и столько же мерзок пред Богом, сколько пагубен в обществе. Христиане! Человек создан по образу и подобию Божию: какое отличие!.. Но поверьте, что памятозлобный не имеет его: он более зверь, нежели человек.

\section{Чудесная встреча}

Некто, по имени Иоанн\footnote{Прп. Иоанн постник, патриарх Цареградский (†~595). Из Жития святых, во 2"~й день сентября. Память его празднуется 2~(15) сентября.}, служил в Царьграде при монетном дворе. Он был благочестив, щедр для нищих и богобоязнен. Однажды Иоанн принял к себе инока, пришедшего из Палестины, как для того, чтобы исполнить приятнейший сердцу его долг странноприимства, так и для того, чтобы в праздное время заниматься с ним разговорами душеспасительными.

В одно время собеседники шли вместе, инок по правую, а Иоанн по левую руку. Вдруг встретился с ними незнакомец и сказал выразительно: «Не должно тебе, авва, идти одесную\footnote{По правую руку.} великого архиерея». Удивленный инок объявил об этом Евтихию, патриарху Царьградскому, и святитель, уважив столь чудное происшествие, облек Иоанна в образ ангельский.

Новопосвященный инок не обманул надежды первосвященника: был образцом братии, Ангелом земным, человеком небесным. Чрез некоторое время умер блаженный Евтихий, и праведный Иоанн с общего согласия духовных и светских был избран на место его. Человек Божий долго отказывался от столь великого сана, но, устрашенный сверхъестественным видением, в котором представилось ему пламенное море, от земли до неба колеблющееся, и сонм Ангелов, в случае непослушания предвещавших ему грозную казнь, наконец согласился вступить на престол патриарший.

Житие чистое, пост совершенный, высокие добродетели украшали его до блаженной кончины; и содеянные им чудеса свидетельствуют, что Иоанн, нареченный \textit{Постником}, был возлюбленный Божий.

\section{Праведное воздаяние отступнику Ватадзису\footnote{Из жития святых 42~византийских мучеников во Амморее, которые жили в IX~в. в царствование Феофила иконоборца. Память их празднуется 6~(19) марта.}}

В царствование греческого царя Феофила агарянский князь Амирмумна разбил христианское воинство и осадил прекрасный и великий град Амморей. Но поскольку неприступные стены и многочисленная сила защищали его и греки на вылазках каждую ночь истребляли великое число измаильтян, то Амирмумна был в сомнении, не лучше ли отступить от столь крепкого града.

Но, к несчастию греков, между царскими военачальниками был некто Ватадзис. Этот изверг, имея личную злобу на некоторых полководцев, вознамерился предать неприятелю город, чтобы вместе с ним погубить тех, кого ненавидел. Выбрав благоприятный случай, злодей пустил к агарянам, уже начинавшим снимать лагерь, стрелу с письмом следующего содержания: «Почто после столького с вашей стороны кровопролития отходите без всякого успеха? Приступите к стене оттуда, где видите столб, на котором стоит мраморный лев. Эту часть города защищаю я и клянусь, что буду вам помощником. Вы знаете, впрочем, какой награды будет стоить ваш благоприятель и друг». Стрела была примечена; письмо отдано Амирмумне, который, прочитав его, не преминул воспользоваться советом злодея.

В глухую полночь, когда амморяне неприятеля не опасались, агарянин совокупил силу свою и с бешенством наступил на ту часть стены, где был Ватадзис. Граждане и войска устремились сделать отпор; но злодей обратил оружие на своих соотечественников, и неприятель беспрепятственно вошел в город. Кровопролитие было ужасно; все обращено в пепел; христианские вожди уведены в плен, а предатель отечества Ватадзис, приняв веру лжепророка, наделен от Амирмумны богатством и возвеличен саном вельможи.

Семь лет злосчастные военачальники сидели в мрачной темнице, занимаясь единым богомыслием, ибо выкуп за них отвергнут был с презрением. Агаряне, пленившие их тела, хотели пленить и души их. Часто посылаемы были вельможи и мудрецы, чтобы преклонить их на свою сторону, но тщетно; приходил и Ватадзис, но они ужасались и воззреть на него. Наконец, Амирмумна ожесточился и изрек им приговор смерти.

Сорок два праведных мужа за веру во Христа Спасителя пролили кровь свою… Но Бог сколько долготерпелив, столько и правосуден.

Агарянский князь, удивляясь их непоколебимости в вере и преданности царю своему, вдруг призывает к себе Ватадзиса, который был равнодушным свидетелем смерти их, и с грозным взором говорит ему: «Ты не соблюл веры во Христа, изменил царю твоему, то уверен я, что при случае оставишь веру в Магомета и будешь моим предателем; посему предупреждаю вторичное от тебя злодейство». Он дал знак исполнителям казни, и в одно мгновение глава изменника была отторгнута от выи\footnote{\textit{Выя} "--- шея.}.

Ужасно быть изменником Царя "--- Отца Небесного! Столь же ужасно быть изменником царя "--- отца отечества!.. Как гражданин, изменивший царю своему, не может быть христианином, так и христианин, отрекшийся от веры Господней, не может быть гражданином. Ибо какая верность царю земному от того, кто изменяет Царю Небесному?

\section{Видение старцу о приемлющих милостыню\footnote{Из Жития святых в 7"~й день марта.}}

Один затворник, в юности отказавшись от всех удовольствий света, жил в тесной хижине и, занимаясь богомыслием, умерщвлял тело свое постом и всенощным бдением. За столь равноангельское житие праведник удостоился откровения свыше.

В один день пришел в монастырь старейшина града, чтоб подать, по обыкновению своему, милостыню братии. Он наделил всех иноков по сребренику, а к затворнику принес златницу и, стоя у окна кельи, усердно просил его, чтобы тот не отвергнул столь малого дара. Старец употреблял в пищу и питие не более как по одному сухарю и по чашке воды на день; хотя никогда не принимал ни злата, ни серебра, но, устыдившись просьбы человека благородного и благодетельного, взял златницу.

Ночью, прочитав обычные молитвы, он уснул. Вдруг в сонном видении представилось ему, что вместе с братией он стоит в пространном поле, которое заросло густым и колючим тернием, а какой"=то юноша, подобный Ангелу, понуждает старцев, чтоб пожинали оное. Потом подходит к нему и говорит: «Жни терны!» Когда же затворник начал отговариваться, то юноша грозно воскликнул: «Не имеешь причины стоять праздно! Ибо вчера вместе с братией ты нанялся, и притом взял златницу, между тем как прочие получили только по сребренику. Как больше взявший, ты больше других должен и трудиться. Эти терны суть дела того, кто вчера подал тебе милостыню… Трудись в поте лица твоего». С этим словом затворник проснулся.

Он размыслил о своем сновидении и сердечно ужаснулся, что при столь непроходимом тернии грехов своих должен исторгать терны чужих прегрешений.

О вы, которые питаетесь милостыней! Познайте обязанность вашу из этого видения, которого Небесный Судия удостоил праведного мужа: молитесь о прощении грехов благодетелей ваших. А вы, которые даете пищу и одежду страждущему нищетой или недугами человечеству, приумножьте щедроты свои, ибо милостыней и верой очищаются грехи.

\section{Ревность по Христе блаженного Кодрата\footnote{Св. мученик Кодрат был родом коринфянин, скончался в царство Деция, в 248~г. Память его празднуется 10~(23) марта.}}

В царствование Деция, когда христиане, в необъятном числе содержась в темницах, были томимы голодом и муками, праведный Кодрат, будучи знатен и богат, покупал вход у стражей к христианским узникам, приносил им пищу и увещевал, чтобы по любви к Богу они дерзали на все ужасы смерти.

В то время некто Перенний, чиновник царский, прибыл в Никомидию\footnote{\textit{Никомидия} (греч. \textsf{Νικομήδεια}, ныне Измит) "--- древний город в Малой Азии, центр области Вифиния. Располагался на подступах к Константинополю, на берегу Мраморного моря.} и потребовал списка богоотступников (так поклонник идольский называл христиан). Видя их столь великое множество, тиран поклялся даже память их истребить из среды живых.

В общем собрании народа представлены были на суд христиане. Судья воскликнул грозно: «Пусть скажет всяк свое имя, звание и отечество». Блаженный Кодрат, как втайне исповедующий веру Христову, был посторонним зрителем и стоял позади всех. Видя робость страдальцев, он устрашился, чтобы кто"=нибудь из них, от угроз или мучений, не отрекся от Христа Спасителя, и решился подать пример страдальчества до смерти: он воззвал велегласно\footnote{Громко, звучно.}: «Мы называемся христиане, чином рабы Христовы; а отечество наше есть Небо». Услышав это, воевода царский удивился дерзости и приказал оруженосцам, чтобы наглого (так думал злобожный) незнакомца взяли и представили к нему ближе. Но дерзающий по Боге Кодрат, не дожидаясь стражи, устремился туда сам и, раздвигая руками теснящийся народ, стал пред Переннием. Все обратили на него взоры свои; а он, оградившись крестным знамением, воскликнул: «Се от имени всех предстою тебе и уведомляю тебя, что мы все, сколько здесь ни видишь, воины Иисуса Христа…» Взирая на столь великого ревнителя Православной веры, все христиане укрепились, все восхотели лучше пролить кровь свою, нежели сказать одно слово в пользу язычества. Так святой Кодрат стал первым из мучеников.

Человек не должен сомневаться, жертвовать ли самим собой, если имеет случай обратить других на путь спасения или поддержать на оном.

\section{Великодушие против обид\footnote{Из Пролога, в 12"~й день марта.}}

Некоторый старец, по имени Кир, будучи из низкого рода и весьма кроток, не понравился братии. Часто случается, что за смирение или за другие хорошие качества, наконец, полюбят того, кого прежде не любили, но участь преподобного Кира была не такова! С продолжением времени умножалась ненависть братии: не только старшие, но и юноши, находящиеся под искусом, оскорбляли его и нередко даже выгоняли из"=за стола. Это продолжалось пятнадцать лет.

В той обители случилось быть Иоанну Лествичнику.

Видя, что кроткий Кир, будучи выгоняем из"=за стола, часто ложился спать голодным, он спросил у него: «Скажи мне, что значат против тебя эти обиды?» "--- «Поверь мне, любезный о Христе гость, "--- отвечал смиренный старец, "--- что братия так поступает не по злобе; они только искушают меня, достойно ли ношу образ ангельский. Вступив в эту обитель, я слышал, что отшельнику должно быть под искусом тридцать лет, а я прожил еще одну половину».

Христиане! Вот примерный поступок против тех, которые обижают нас. Кроткий Кир не хотел мстить на деле своим гонителям и считал за высшее счастье то, что другие бы почли для себя совершенным несчастием.

\section{Ненавистник не способен ни к каким добродетелям\footnote{Из Пролога, в 22"~й день марта.}}

Во время гонений на христианскую веру взяты были под стражу два брата, которые друг на друга имели злобу, и после обыкновенных допросов осуждены на смерть. Тогда один сказал другому: «Примиримся между собой, любезный брат! Мы завтра отойдем к Богу». Но тот и слушать его не хотел.

На следующий день повели их на место казни, и что же?.. Тот, который не перестал ненавидеть брата, отрекся от Христа; другой, который простился с братом своим, умер за веру Христову. Удивленный мучитель спросил у первого: «Почто вчера не отвергся от Христа, чтобы избегнуть пытки?» И услышал от него следующий ответ: «Когда я решительно сказал брату моему, что не хочу с ним помириться, тогда мгновенно оставила меня сила и помощь Божия: вот единственная причина, почему я оставил веру во Христа».

Ненавистник, беспрестанно изыскивая различные средства, как бы сделать зло тому, кого ненавидит, уже не походит на человека, но совершенно уподобляется духу"=губителю. А посему может ли он быть способен к каким"=нибудь добродетелям, наипаче к мученичеству, где нужна в высочайшей степени любовь к Богу, которая не может быть без любви к ближнему?

\section{Наказанный клятвопреступник\footnote{В 23"~й день марта.}}

Некто, по имени Иоанн, бывший прежде боярин киевский, облекшись в ангельский образ в Киево"=Печерской Лавре, имел при себе юного сына Захарию. Видя приближающуюся смерть, Иоанн поручил его покровительству Промысла Божия; а некоторому иноку, по имени Сергий, бывшему прежде также боярину киевскому и с ним вместе постригшемуся, оставил тысячу гривен серебра и сто гривен золота, чтобы сохранил их до совершеннолетия сына и после ему отдал. Распорядившись таким образом, Иоанн умер.

Захария, достигнув восемнадцати лет, потребовал у Сергия наследство отца своего. Но отшельник, сделавшись рабом любостяжания, изумился, как бы не зная, чего юноша от него требует; но, видя неотступность Захарии, притворно оскорбился и сказал ему: «Отец твой отдал все имение Богу; у Него и проси, а не у меня. Виноват ли я, что Иоанн был так безумен, что, обогащая нищих, сделал нищим единородного сына».

Видя такое вероломство, юный Захария заплакал. Он умолял Сергия, чтобы тот отдал ему, по крайней мере, половину имения, хотя бы третью часть; даже просил у него, наконец, десятой доли, но напрасно. Для Захарии осталось последнее средство, чтобы инок доказал невинность свою клятвой перед Богом.

Увы! Порок имеет столь гибельное свойство, что человек, поскользнувшись однажды, спотыкается и в другой раз! Пустынник"=миролюбец, имея свидетелями братию, идет с Захарией в Печерскую церковь. Там, став пред чудотворным образом Пресвятой Богородицы, где прежде вместе с юным Иоанном приял союз братолюбия, клялся небом и землей, что отец Захарии никогда не давал ему ни полушки, не только тысячи гривен серебра и ста гривен золота. Но, как скоро начал приступать к Приснодеве, чтобы облобызать воскрилия\footnote{\textit{Воскрилия} "--- подол, полы верхней одежды.} одежды ее, вдруг невидимая сила удержала преступника. Он не мог двинуться с места, как ни напрягал стопы свои "--- весь трепетал… Наконец, видя, сколь ужасно правосудие Господне, воскликнул: «Преподобные отцы Антоний и Феодосий! Помолитесь Богородительнице, да не даст Ангелу смерти погубить меня… А ты, сын моего собрата и друга! Возьми злато свое и прости меня, грешного».

Иноки ужаснулись и пошли в келью Сергия, славя Бога, такие чудеса сотворившего.

\textit{Не приемли имене Господа Бога твоего всуе} (ср. Исх. 20, 7), сказал Господь чрез Моисея. Христиане! Страшитесь говорить: Ей"=Богу! Вот тебе Христос! Даже тогда, когда на вашей стороне правда. Довольно сказать: Ей"~ей! или Ни"=ни! Тем более да сохранит вас Бог от привычки божиться напрасно.

\section{О молитвах, недостойных Божия величия\footnote{Из Пролога, в 27"~й день марта.}}

Некоторый старец, молившийся Богу и питавшийся от трудов своих, пришел в селение продать свое рукоделие. Уже благочестивый инок хотел возвратиться в обитель, как встретился с ним незнакомец, окруженный нищими, странниками, и пригласил его к себе на ужин.

Обласканный как нельзя лучше, старец простился с добродетельным человеком и, восхищаясь гостеприимством его, старался узнать, кто таков этот страннолюбец. Какую же почувствовал к нему любовь, когда услышал, что он каменосечец, по имени Евлогий, занимается работой весь день, ничего не вкушая! Когда же настанет вечер, получив обыкновенную плату, ведет за собой в дом всех, кого из бедных на пути встретит, и издерживает на ужин все, что заработал; даже крошки, которые останутся от стола, бросает соседним псам. Так надеялся Евлогий на Промысл Небесный, питающий всех "--- от человека до червя!

Добродушный старец, удивляясь добродетелям его, думал: «Евлогий бедный питает каждый день столько нищих; что же бы сделал Евлогий богатый? Ах! Он был бы пища всем голодным, одежда всем нагим!» И начал день и ночь просить Бога, чтобы Евлогий для счастья других сделался богат; соединив молитву свою с постом, он так изнемог, что едва был жив.

Наконец Бог услышал молитву старца и известил его следующим образом: старец заснул, вдруг видит себя в церкви Воскресения, видит некоторого богоподобного отрока, на камне сидящего, подле него стоял Евлогий. Потом отрок сказал: «Если отдашь сам себя порукой за Евлогия, то наделю его богатством». "--- «От рук моих взыщи душу его», "--- отвечал старец и увидел, что двое из предстоявших начали в пазуху Евлогия сыпать золото. Тут старец воспрянул от сна своего, благодаря Бога, что услышана молитва его.

Между тем Евлогий жил по"=прежнему. Однажды поутру, выйдя на работу свою, ударил он киркой по камню и почувствовал, что он пуст; ударил в другой раз и увидел скважину; ударил в третий раз, и из камня посыпалось золото. Евлогий ужаснулся и не знал, что делать с сокровищем. В тот день ни один нищий у Евлогия и даже сам Евлогий не вкушали пищи. Назавтра он купил лошадей и под видом перевозки камней привез домой золото и ужинал один. Долго Евлогий был в беспокойных размышлениях… Наконец, нанял корабль и отправился в Царьград; там задарил всех вельмож и сам сделался вельможей; купил огромный великолепный дом и жил так, как прежде, без всякой бережливости, то есть каждый день угощал людей знатных и сильных.

Старец, который испросил у Бога богатство Евлогию, ничего не знал о том, но спустя два года опять увидел во сне святолепного отрока и подумал: где"=то Евлогий? Но что же узрел после того? Некто злообразный изгоняет Евлогия от лица отрока… Старец пробудился ото сна и, воздохнув, сказал сам себе: «Увы! Я погубил душу мою!» Потом оделся и пошел в то селение, где прежде жил Евлогий; он долго ждал, пока придет питатель бедных и пригласит его в дом свой, но тщетно! Наконец, выйдя из терпения, спросил у одной старушки: «Есть ли здесь кто"=нибудь, принимающий странников?» "--- «Нет, "--- отвечала она с тяжким вздохом. "--- Был у нас каменосечец, который всего более любил странноприимство; но Бог, увидев дела его, дал ему благодать Свою, и он теперь в Царьграде вельможей». Услышав это, старец сказал сам себе: «Я сделал убийство!» "--- и пошел в столицу.

Там он узнал, где живет Евлогий, сел у врат дома его и ожидал, когда выйдет он… Наконец является Евлогий с гордостью на лице, с гордостью в походке, окруженный ласкателями и сопровождаемый рабами. «Помилуй меня, "--- воскликнул старец, "--- я хочу нечто сказать тебе…» Но Евлогий даже не воззрел на старца, а рабы оттолкнули его. Несчастный поручитель опередил Евлогия другой улицей, опять встретился, опять окликнул его, но, получив несколько ударов, вынужден был удалиться. Таким образом, старец сидел четыре недели пред вратами дома Евлогиева, обуреваемый снегом и дождем, и не имел случая с ним объясниться.

Наконец старец отчаялся о спасении Евлогия и, повергшись на землю, просил Бога разрешить его от поручения, но чрез необычный сон был извещен, что безумная молитва и безумное поручение недостойны прощения. Несчастный старец, не зная, что делать, сел в корабль, чтобы отплыть в Александрию, и в это самое время повергся в такое малодушие и отчаяние, что был подобен мертвецу. В этом умоисступлении он задремал и после многих мучительных сновидений услышал глас: «Иди в твою келью! Я возвращу Евлогия в прежнее состояние, но ты, слабый человек, не воссылай к Богу молитв, недостойных Его величия».

Спустя три месяца старец услышал, что царь Иустин умер, что наследник его начал гнать прежде бывших любимцев, что двое из них умерщвлены, а Евлогий бежал. Впрочем, царь назначил великое награждение тому, кто принесет его голову. Старец опять пошел в ту весь, где прежде жил Евлогий. Но как обрадовался он, когда на закате солнца увидел его, идущего с работы и окруженного нищими! Старец хотел было броситься навстречу к нему, но Евлогий предупредил и, целуя руки его, пригласил к себе ужинать.

Тут старец и Евлогий объяснились между собой, причем каменосечец рассказал ему, как возвратился он в отечество свое, как бежали все жители, чтобы видеть его, и поздравляли с саном вельможи. И как он, боясь обнаружить себя, уверил их, что ходил только в Иерусалим поклониться Гробу Господню. «Я вторично взял, "--- продолжал Евлогий, "--- свои орудия и пошел прямо к тому камню, где обрел золото. Я думал, что опять найду клад; но, сколько ни стучал, сколько ни тесал разных камней, не нашел ничего. Наконец вышел из заблуждения и, слава Богу, забыл пагубный сан вельможи».

Молись Господу Богу о том, чтобы простил твои согрешения, и о том, чтобы предохранил тебя от внезапных случаев совершить преступление, но не проси в молитвах твоих богатства, власти и прочего. Ибо Бог всеведущ: Он знает, что нам полезно и вредно. Бог милосерд. Он дарует то, что человеку нужно. Евлогий служит примером, сколь иногда пагубно бедному сделаться богатым.

\section{Единственное сокровище для человека есть Бог\footnote{Из Пролога, в 27"~й день марта.}}

В Царьграде был гражданин, довольно богатый и человеколюбивый. Имея у себя одного сына, он не старался оставить ему великое наследство и свое имение раздавал щедрой рукой неимущим. Но этого еще не довольно было для человека богоугодного. Он хотел иметь и то утешение, чтобы, явившись к страшному Престолу Судии Небесного, мог ему сказать: се аз и чадо мое! Хотел и сына своего сделать столько же милостивым к страждущему человечеству.

В один день, призвав его к себе и показав все свои богатства, он спросил у него: «Любезный сын! Что более желаешь получить от меня в наследство: эти ли сокровища? Или Христа Спасителя?» Юноша, воспитанный в страхе Божием, решительно отвечал ему: «Христа: более ничего». Восхищенный сыновним ответом, отец еще более начал расточать свое имение нищим, так что после смерти своей оставил ему только насущный хлеб. Благонравный юноша, из богача сделавшись почти убогим, не сожалел о том, ибо надеялся на Христа "--- на это великое сокровище, которое у него осталось.

В том же городе жил некоторый вельможа, славный богатством и благочестием. Он имел у себя супругу, также богобоязненную, и единственную дочь, воспитанную в страхе Божием. Когда девице пришло время супружества, родители сказали друг другу: «Какого жениха изберем нашей дочери? Богатого ли и злонравного? Но он будет оскорблять ее, принудит гоняться за суетой света, научит тщеславиться… Нет; изберем лучше человека среднего состояния, но богобоязненного, который примет дочь нашу как особливый дар с неба. Пойдем в церковь Божию, "--- присовокупили они. "--- Будем молиться о счастье нашего дитяти; и кто первый войдет в храм Божий, тот пусть будет супругом ее».

Сотворив пламенную молитву к Богу, отец и мать девицы сели и ждали, чем Промысл Божий решит судьбу любезнейшей их дочери… Вдруг входит вышеупомянутый богобоязненный юноша. Подозвав его к себе, они спросили: кто он и откуда? И, услышав, что это сын такого"=то в Бозе усопшего щедролюбца, возблагодарили Бога и сказали ему: «Христос Спаситель нам усыновил тебя; прими от нас руку нашей дочери, а с ней и богатство наше. Все это дает тебе Сам Бог».

Благочестивый юноша служит примером, что Бог никогда не оставляет в нищете того, кто единственно из любви к Нему расточает свои сокровища.

\section{Как сребролюбивая инокиня сделалась щедролюбивой\footnote{Из Пролога, в 30"~й день марта.}}

Девица Монония, в юности приняв иночество, по"=видимому, была смиренна и благочестива, но на самом деле поклонялась идолу, ибо собирала богатство и никогда не подавала милостыни. Имея у себя племянницу и любя ее сердечно, она думала только о том, чтобы выдать ее в супружество как можно выгоднее, и забыла своего Небесного Жениха.

Святой Макарий, попечитель неимущих, с душевным сокрушением взирая на эту заблудшую инокиню, вознамерился обратить ее от идола к Богу, в чем и преуспел следующим образом. Будучи в молодости своей искусным гранильщиком драгоценных камней, он был известен Мононии, страстной охотнице до редких вещей. Итак, в один день придя к ней, сказал он: «Некто продает два бесценных камня, изумруд и яхонт: я не знаю, его ли они собственные или где украдены, но уверяю тебя, что они будут редким украшением для твоей племянницы, купец отдает их недорого "--- только за пятьсот золотых». Монония обрадовалась несказанно и начала просить святого Макария, чтобы такую редкость не выпускал из рук; не желая сама выходить за монастырь, она в ту же минуту вручила ему деньги.

Благочестивый Макарий, получив золото, ушел и не хотел с ней видеться несколько дней, а старице совестно было послать за ним, ибо Макарий был человек всеми уважаемый и пользовался общим доверием. Наконец Монония, встретив его в церкви, спросила, куплены ли камни или нет. «Они у меня дома, "--- отвечал праведник, "--- не угодно ли тебе самой посетить меня и посмотреть камни? Если они понравятся, то возьми их; если же не понравятся, то получи назад золото». Монония с радостью пошла к нему.

«Что прежде хочешь видеть, яхонт или изумруд?» "--- войдя в дом свой, спросил ее святой Макарий. «Что тебе угодно», "--- отвечала Монония; и праведник ввел ее в одну комнату. Но что увидела тут Монония? Мужеского пола нищих, хромых, слепых, расслабленных, которые, сидя за столом, благословляли имя Мононии… «Вот твой яхонт», "--- указав на них, сказал ей Макарий и повел в другую комнату. Изумленная старица, не говоря ни слова, за ним следовала и увидела женщин, страждущих или от старости, или от болезней, которые, также обедая, молили Бога о здравии Мононии. «Вот твой изумруд, "--- указав на них, опять сказал Макарий. "--- Как ты думаешь, есть ли что драгоценнее этих камней? Если хочешь, возьми их; если не хочешь, оставь у меня и возьми назад твои деньги». Не зная, что отвечать на этот поступок святого Макария, пораженная стыдом, старица безмолвно оставила его.

Будучи наедине, Монония призадумалась, припомнила всю жизнь свою в иночестве и не могла вспомнить ни одного совершенного ею доброго дела; она так начала сокрушаться о том, что впала в тяжкую болезнь. Тут представлялась ее воображению то тьма кромешная, то огонь неугасаемый, а некоторый Муж, сияющий славой, на все это указывая, говорил ей: «Вот ужасы, от которых могут избавить тебя яхонт и изумруд, купленные святым Макарием». Устрашенная этим, Монония как от сна воспрянула и почувствовала облегчение от болезни.

С того времени Монония уважала Макария как своего Ангела"=хранителя и, переменив образ жизни, сделалась щедрой питательницей бедных.

Воистину, тот покупает бесценный яхонт, бесценный изумруд, кто питает бедных, ибо на эти камни можно купить Царствие Небесное.

\section{Примиритель\footnote{Из Пролога, в 6"~й день апреля.}}

Благочестивый пустынник пришел в некоторый скит на короткое время и, не застав свободной кельи, не знал, где остановиться. Другой старец, у которого была пустая хижина, отдал ее пришельцу. В первые дни иноки обходились между собой, как братия о Христе, но скоро их любовь нарушилась.

Узнав, как поучительна беседа пришельца, многие начали посещать его и приносили ему, кто что имел. Старец, давший ему убежище, позавидовал славе инока и сказал сам себе: «Я живу здесь издавна, молюсь и пощусь, однако никто не навестил меня до этого времени, а этот пришелец здесь несколько дней, и все приходят к нему просить благословения». Потом начал роптать и злословить инока. Наконец, дошло до того, что он, надрываясь с досады, послал своего ученика сказать гостю, чтобы тот уходил из кельи, куда хочет. Добродетельный ученик пришел к гостю, вместо того чтобы объявить волю своего наставника, поклонился и сказал ему: «Отец мой приказал спросить у тебя, здоров ли ты». "--- «Скажи отцу твоему, "--- отвечал гость, "--- чтобы помолился обо мне Господу Богу, я несколько болен». Юноша, возвратившись к своему наставнику, объявил другое: будто старец нашел другую келью и скоро уйдет от него.

Прошло несколько дней, и старец опять послал ученика своего сказать пришельцу, что, если он не выйдет из кельи, он сам придет и жезлом изгонит его. Что ж напротив того сделал ученик? «Отец мой, "--- сказал он пришельцу, "--- услышав, что ты болен, старец прислал меня навестить тебя». "--- «Скажи отцу твоему, "--- отвечал гость, "--- что я, молитвами его, выздоровел». Юноша, возвратившись к наставнику своему, сказал, будто гость просит позволения пожить у него только одну неделю. Прошла и неделя… Старец вышел из терпения и, взяв костыль, пошел выгонять пришельца. Но юноша, предупредив его, прибежал к иноку и сказал ему, что учитель идет звать его на ужин… Услышав это, пришелец вышел навстречу старцу и, поклонившись ему, сказал: «Я иду к тебе, любезный о Христе брат! Не трудись идти ко мне». Изумленный кротостью его, старец поверг на землю костыль свой и устремился облобызать его; потом увел в свою келью и угостил чем Бог послал.

Проводив гостя, старец спросил у юноши, сказывал ли он пришельцу то, что было ему приказано. «Нет!» "--- отвечал юноша. Старец душевно тому обрадовался, и, узнав, что зависть его была следствием зависти дьявольской, он припал к ногам ученика своего и сказал ему: «Отныне ты отец мой, а я ученик твой, ибо чрез твой только поступок спасены души обоих нас».

Вот поучительный пример для тех, которые любят из дома в дом переносить вести! Христиане! Если увидите двух между собой ссорящихся, употребляйте все способы, чтобы примирить их: чрез это спасительное дело вы избавите от погибели заблудших и сами к добрым своим делам присовокупите важнейшую добродетель.

\section{Отступник Нирсан ужаснейшею ценою покупает себе жизнь\footnote{В 9"~й день апреля.}}

Когда преподобный Вадим, архимандрит Персидский, за веру Христову содержался в темнице и, будучи томим голодом и терзаем пытками, не хотел оказать даже наружного богопочитания солнцу, в то самое время некто Нирсан, князь арийский, уличенный в христианстве, отдан был под стражу. Но он не имел великодушия и долготерпения Вадимова. Устрашившись угрожающих мук, он повергся наконец в такую робость, что не устыдился сказать: «Почто терять мне блаженство известное в надежде некогда получить блаженство неизвестное?» И в тот же час послал одного из стражей к царю Саворию с просьбой, чтобы он возвратил ему свое благоволение, а он, со своей стороны, вернется к вере отцов своих. Царь обрадовался и велел, чтобы Нирсана к нему представили.

«Я готов доказать тебе мою царскую милость всем, чем пожелаешь, "--- сказал Саворий, "--- но и ты, Нирсан, докажи мне послушание верноподданного». Он шепнул что"=то на ухо одному из царедворцев, и вскоре предстал преподобный Вадим в тяжких оковах на руках и ногах. «Если Нирсан умертвит своей рукой Вадима, "--- сказал Саворий к предстоящим вельможам, "--- то оставлю ему жизнь, возвращу сан, богатства и любовь мою». Злочестивец, не ответствуя ни слова, берет у одного оруженосца саблю и взмахивает ею…

Вдруг раб Божий воззрел на него. Злодея объял трепет, рука его как бы окаменела. «Злобожный человек! "--- сказал ему праведник. "--- До того ли дошло твое неистовство, что не только дерзнул ты отречься от Христа, но еще убиваешь рабов Его? Горе тебе в страшный день Суда Божия! Я охотно умираю за веру Христову; но не хотел бы быть умерщвленным от твоей руки». Богоотступник, не чувствуя укоризны праведника и не имея сил исполнить волю тирана, стоял в трепете, устремив дикий взор на Вадима. Наконец он собрался духом, с напряжением поднял руку свою и спустил удар на выю святого, но вскользь. Он несколько раз повторял удары свои и, нанеся бесчисленные раны, едва мог умертвить мученика. Так дрожали руки убийцы!

Предстоявшие вельможи удивлялись долготерпению мученика, который под многократными ударами стоял неподвижно, как гора каменная, и ругались над убийцей, гнушаясь, что тот так боязлив и малодушен.

Бедственна была жизнь богоотступника! Ибо неверные подозревали его, не обратится ли опять к Христианству, и при всяком случае укоряли, что он, раз изменив вере отцов своих, не может быть добрым гражданином. А христиане, встречаясь с ним, бежали от него или падали ниц, чтобы не осквернить очей своих воззрением на столь лютого нечестивца… Наконец, терзаемый отчаянием, святоубийца сделался самоубийцей.

\section{Сила архипастырского запрещения}

Один епископ за некоторую вину отлучил священника от службы Божией. Вскоре случилось, что несчастный по своей надобности пошел в другую область и на дороге был захвачен неверными. Будучи принуждаем принести жертву идолам, с твердостью отрекся он от злобожного веления и был замучен.

Тело убиенного досталось христианам и было принесено в близлежащий город. Бог не токмо на душе, но и на телах праведных удивляет милость Свою. Градоначальник, видя чудеса от мощей мученика, соорудил церковь, устроил раку и, положив в нее тело его, поставил внутрь святилища. Потом, пригласив епископа и весь причт церковный, просил освятить храм Господень.

Но едва началось всенощное бдение "--- о чудо! Рака двинулась с места и, никем не тронутая, вышла из церкви. Все тут находившиеся ужаснулись и, усердно молясь Богу о своих согрешениях, которые, как думали они, прогневали угодника Господня, опять внесли оную в церковь; но, едва в другой раз началась служба Божия, рака опять вышла.

Градоначальник, пораженный удивлением и страхом, не знал, чему приписать это чудо; епископ болезновал; весь народ плакал, и с этими чувствованиями все разошлись по домам своим.

Настала ночь, и святой мученик явился епископу. «Не скорбите о себе, "--- сказал он, "--- что не остаюсь с вами в храме Господнем. Я умоляю тебя, сходи в такой"=то град к святителю и упроси его, да разрешит и меня от запрещения. Ах! Я отлучен от службы Божией и потому не могу с вами служить в церкви и исхожу вон; хотя и принял венец мученический, но не видел Лица Христова как недостойный совершать святые Таины Его; \textit{елицех бо архипастырь свяжет на земли, будут связаны на небесех} (ср. Мф. 18, 18). Ты не приступай к сему; кто связал, тот пусть и разрешит». Сказав это, мученик стал невидим.

Наутро епископ, взяв с собой некоторых из причта духовного, отправился к нареченному от святого мученика епископу и обо всем случившемся рассказал ему. Святитель немедленно отправился в путь и, преклонившись пред чудотворными мощами, воскликнул: «Христос, связавший тебя через меня, смиренного раба Своего, да разрешит тебя ныне за пролитие крови твоей во славу имени Его! Войди в храм Господень и пребуди с нами».

Тогда с подобающей честью внесли раку святого в церковь и совершили всенощное бдение и литургию. Мощи не двинулись и источали чудотворения.

\section{Человек, рачительный к своей должности, не дожидается приказаний, а особливо понуждений\footnote{В 12"~й день апреля.}}

«В юности моей, "--- говорит о себе преподобный Исаак Сирский, "--- я жил с аввой Кронием, и этот уже дрожащий старец не возлагал на меня никакого дела, но добровольно вставал сам и подавал мне и прочим сосуд воды для омовения. После того я жил с аввой Феодором Ферменским, и этот отец не дал мне ни одного приказания, но сам приносил на стол пищу и говорил мне: “Если хочешь, садись обедать”. Я обычно отвечал ему: “Авва! Я пришел сюда для того, чтобы тебе служить; почто не объявляешь мне воли твоей?” Но старец всегда молчал… Наконец я уведомил об этом прочих. Иноки пришли к нему и сказали: “Авва! Брат пришел сюда единственно для твоих услуг; почто не поручаешь ему никакого дела?” "--- “Разве я начальник киновии\footnote{\textit{Киновия} "--- общежительный монастырь.}, что буду приказывать? "--- отвечал преподобный Феодор. "--- Когда захочет, пусть делает то, что я делаю, как видит он”. Итак, с этого времени я предупреждал его и делал все, что намеревался делать авва».

Дети, ученики, подчиненные! Из этой повести заметьте, как должно поступать вам с родителями, начальниками и учителями.

\section{Пример добродетельных мужей для нас спасителен}

Когда преподобный Исаак приблизился к блаженной кончине, собравшиеся вокруг него старцы спросили: «Что будем делать после тебя, авва?» "--- «Смотрите, "--- отвечал святой муж, "--- како \textit{ходих пред вами} (1~Цар. 12, 2); этого для вас довольно. Если захотите идти в след мой и хранить заповеди Божии, Господь ниспошлет на вас благодать Свою и сохранит место сие, а если не соблюдете оных, не останетесь и на месте этом. Мы, видя приближающихся к смерти отцов наших, также сокрушались; но, соблюдая заповеди Господни и советы их, жили, как будто их самих пред собой имея. Равным образом поступайте вы, и Сам Бог сохранит вас».

Дети! Каждый благочестивый поступок ваших родителей, ваших наставников, ваших родственников, старцев и всех добродетельных людей всегда имейте пред вашими очами не только при их жизни, но и после смерти. Это будет светильником на путях ваших, с которым никогда не преткнетесь о камень соблазна.

\section{Сокращенное содержание Закона Божия}

Святой Иоанн Богослов, достигнув глубоких лет и потеряв силы не столько от старости, сколько от великих, непостижимых для нашего разума трудов, приносим был в собрание верных своими учениками. Апостольская ревность заставляла его проповедовать Слово Божие и поучать христиан, но старец говорить не мог, а потому и довольствовался только такими словами: «Чадца моя! Любите друг друга». Удивляясь всегдашнему повторению одного и того же совета, они спросили его о причине, и любимец Христов отвечал: «Это заповедь Спасителя, исполнение которой, после любви к Богу, есть сокращенное содержание всего Закона».

\section{Милуяй нища, взаим дает Богови!\footnote{В 16"~й день июня. Синодальный перевод: \textit{Благотворящий бедному дает взаймы Господу} (Притч. 19, 17).}}

Святой Тихон, бывший епископом Амафунтским и великим поборником Православия, в молодых летах продавал хлебы, ибо отец его был пекарь. Имея сострадательное сердце, благочестивый отрок часто раздавал по нескольку хлебов нищим безденежно, за что раздраженный отец в одно время строго наказал его. Тогда святой Тихон сказал ему: «Родитель мой, ты огорчаешься напрасно: я отдаю хлебы твои взаймы Богу и имею от Него рукописание в Священных книгах, где именно и говорится: \textit{Даяй Богу, сторицей приимет} (Мф. 19, 29). Если тому не веришь, пойдем в житницу: там увидишь, как Бог возвращает заимодавцам долг Свой». Сказав это, он берет отца своего за руку и ведет к житнице. Но, когда хотели отворить дверь, они нашли, что она наполнена чистой и лучшей пшеницей, хотя прежде в ней почти ничего не было. Отец удивился и ужаснулся; он пал на землю, воздавая благодарение Богу, и с того времени не запрещал сыну своему раздавать милостыню по его воле.

\section{Сила молитвы}

В саду обрезали сухие ветви от виноградных древ и выбрасывали вон. Святой Тихон, собрав оные, рассадил их в своем огороде и помолился Богу, чтобы Он ниспослал на виноград его четыре дарования: первое, чтобы сухие розги приняли влагу, жизнь и силу, укоренились и возросли; второе, чтобы виноград был полон и красив; третье, чтобы ягоды были сладки и здоровы; четвертое, чтобы гроздья его скорее прочих созревали.

Наутро Тихон пошел наведаться о винограде своем и увидал на нем благословение Господне: все розги прозябли. Они начали расти и того же лета сверхъестественно принесли чрезвычайное изобилие плодов; повсюду виноград был зелен, а у святого отрока созрел, имел вкус весьма сладкий и для здоровья полезный.

Этот виноградник не только при жизни, но и после смерти чудотворца не лишился своей чудесной силы, и вино из этих ягод по большей части употреблялось для приношения бескровной жертвы Богу.

Столько пред Господом действительна молитва угодников Его! Но, христиане! Здесь заметьте другое, не менее важное для вас нравоучение. Вы не члены ли Церкви? Не розги ли винограда, его же насадила десница Господня? Вы, без духовных дарований, сухими исходите из чрева матери своей; но святая купель вам дарует влагу, Божественное миро "--- жизнь, Крест Господень "--- силу. Вся Церковь тогда молится за вас устами священнослужителя, да укоренитесь и возрастете в вере Евангельской. Итак, исполнитесь дел добрых; будьте украшением не только ваших семейств, но и общества христианского. Растворяйте сладостью благодеяний жизнь ближних, огорченную злосчастиями. Споспешествуйте здравию недужных. Спешите, один скорее другого, расцвести верой, благочестием, братолюбием! Тогда \textit{будете яко древа насажденные при исходящих вод, еже плод свой даст во время свое, и лист их не отпадет: и вся, елика аще творит, успеет} (Пс. 1, 3).

\section{Сампсон странноприимец\footnote{Память его празднуется 27~июня (10~июля).}}

Святой Сампсон, уроженец римский, еще в юности своей обучился врачебному искусству, не по нужде, не из корысти, но единственно для того, чтобы не быть праздным и помогать страждущим. После смерти родителей он поселился в Царьграде и полностью посвятил себя врачеванию. Но неусыпного врача наиболее украшала благодать Господня, которая всегда благословляла успехом святые труды его, "--- через что человек Божий так прославился, что патриарх против воли рукоположил его в пресвитера.

В то время тяжко разболелся греческий царь Юстиниан и, не получая от врачей своих ни малейшего облегчения, обратился к Богу, единому источнику исцелений, и после усердной молитвы извещен был в сновидении, что только Сампсон может избавить его от болезни. Не скоро и с трудом благодатный врач был отыскан. Обрадованный Юстиниан оказал ему необыкновенную ласковость, на которую святой Сампсон отвечал с кротостью: «Всемилостивейший государь! Я нищ и грешен: сам требую от Христа исцеления грехов моих. Но вера твоя преклонит на милость Царя царей; Он может сотворить все, что восхощет». После этого человек Божий, конечно, не искусством своим, но благодатью Небесного Врача совершенно восстановил от одра болезни столь долго страдавшего Юстиниана.

Благодарный монарх радовался как об исцелении своем, так и о том, что богоугодный человек обитает в царствующем граде его. Он поднес ему богатые дары в сребре и злате, но святой Сампсон ничего не принял. «Если благоволишь оказать мне милость твою, "--- сказал он, "--- то прошу тебя об одном: во славу Бога и для твоего спасения повели близ хижины моей построить гостиницу, в которой бы я, пока жив буду, мог принимать больных и странных и по силе моей успокаивать их. Чрез сие исходатайствуешь и себе вечное воздаяние от Бога и мою старость утешишь». Юстиниан с сердечным удовольствием исполнил просьбу человека Божия: соорудил странноприимницу и больницу и определил для них большие доходы. Здесь"=то неутомимый врач, не только телесно, но и душевно благодетельствуя ближним, достиг глубокой старости и безболезненно перешел в жизнь вечную.

Но и сама смерть этот богоугодный дом не сделала сирым. Когда случалось, что надзиратель странноприимницы был нерадив в своей должности или скуп для бедных, святой Сампсон являлся ему и, если тот не исправится, грозил погибелью. Однажды в Царьграде случился пожар столь свирепый, что отчаялись потушить его; огонь шел прямо на странноприимницу и готов был истребить оную. Вдруг увидели угодника Божия, явившегося над домом. Он ходил вокруг ограды своей и повелевал остановиться пламени. В то самое время сверх чаяния возгремело облако, пролился необыкновенный дождь и погасил пламень.

\section{Смирение, Богом прославленное}

Один старец не выходил из кельи, но имя затворника везде было славно и до благоговения уважаемо. Однажды Дух Святой возвестил ему: «Некто из праведных мужей разрешается от тела, гряди и, прежде нежели умрет он, воздай целование». Старец задумался: «Если выйду днем, стечется народ, будет, может быть, прославлять меня, а это тяжело для моего сердца. Итак, отправлюсь в путь поздно вечером, тайно от всех». Когда стемнело, человек Божий вышел из кельи… Но вдруг явились от Господа два Ангела с лампадами, чтобы освещать путь его. Весь город сбежался увидеть славу праведника. Таким образом, сколько он убегал славы, столько слава за ним следовала, и исполнились слова Священного Писания: \textit{Всяк смиряяй себе вознесется} (Лк. 14, 11).

\section{О важности сана епископского и вообще священнического}

Когда авва Амос принял патриарший престол в Иерусалиме, тогда иноки из разных обителей и пустынь по обыкновению пришли воздать поклонение первосвященнику Божию. «Молитесь за меня, грешного, "--- облобызав их, сказал патриарх, "--- непрестанно молитесь, отцы и братия! Великое и тяжкое бремя возложено на меня, и ужасают меня подвиги старейшинства. Я читал негде, что Лев, первый наставник римский, сорок дней пребыл в церкви в молитве и посте и слезно просил Бога: да простит согрешения его. Наконец услышал глас: «Господь отпустил все грехи твои, кроме тех, которые касаются обязанностей, возложенных на тебя рукоположением. Об этом едином истязан будешь».

Вы, которые готовитесь быть пастырями и учителями словесного стада Христова! Познайте из этого, сколь важное дело "--- принять рукоположение. Не забывайте, что на Страшном Суде Христовом взыщется от вас каждая душа, вам порученная. Пасите их так, чтобы тогда дерзновенно могли сказать: «Господи! Се аз и дети, яже ми дал еси».

\section{Неблагодарность царя Савория}

Когда святой Симеон, епископ Персидский\footnote{Священномученик Симеон пострадал в 344~г. Память его празднуется 17~(30) апреля.}, пренебрегая ласками и угрозами мучителя, показал готовность на все ужасы смерти, окованный и терзаемый оруженосцами, был выведен из царской палаты, евнух Усфаган, первый вельможа и в юности царя бывший его дядька, увидев Симеона, встал и поклонился угоднику Божию. Но Симеон как принявшего некогда христианство, а потом сделавшегося богоотступником Усфагана не удостоил даже взора своего и, укорив его, как беззаконника, равнодушно пошел на мучения, для него уготованные.

Усфаган затрепетал… Он вспомнил Божественные надежды, которые имел некогда на заслуги Христа Спасителя. Искра Христовой веры пробудилась в душе его, и Усфаган, сокрушаясь сердцем, начал плакать и рыдать. Он совлек с себя драгоценные одежды и, надев рубище, сел у врат двора царского и, ударяя себя в перси, вопиял: «Увы мне, окаянному! Как явлюсь к Богу моему, Которого отвергнул? Какой дам ответ на Суде Страшном?..»

Царь Саворий вскоре узнал о столь внезапной перемене своего любимца и благодетеля. Он призывает его к себе, истощает все средства, чтобы возвратить его на прежний путь гибели, удивляется, увещевает, укоряет, сожалеет, угрожает. Называет волшебством все поступки христиан; проклинает веру Христа Спасителя и потом опять умоляет Усфагана, как отца своего, чтобы он, именитый и мудрый старец, не делал уничижения богам, ему бесчестия и печали всему царскому дому. Но Усфаган на все отвечал только следующими словами: «Довольно ужасно и прежде содеянное мною безумие; теперь свидетельствуюсь небом и землею: не почту тварь паче Творца».

После борьбы между царем и Усфаганом, в конце концов неблагодарный Саворий решился предать смерти своего благодетеля и второго отца… Он повелел отсечь голову праведному мужу. Идучи на казнь, Усфаган просил царя обнародовать, что умирает не за гражданское какое"=нибудь преступление, но за веру христианскую. Он без труда получил на то согласие царское, ибо Саворий, наказывая своего друга и благодетеля, думал тем самым устрашить прочих христиан, а Усфаган надеялся, что христиане, услышав о страдании его, более укрепятся на все ужасы смерти за веру Христову. С этими мыслями он умер спокойно.

Вот пример «добродетели» языческой! Этот тиран, осудивший на смерть своего воспитателя, наставника и благодетеля, второго отца и этим поправший все священное, мог ли иметь в сердце своем какие"=либо благодетельные чувствования? Душа христианина не такова: она сочувствует в несчастии и иноверцу.

\section{Пустыннолюбец, или Начало Соловецкого монастыря\footnote{В 17"~й день апреля.}}

В царствование великого князя Василия Васильевича в Белозерском Кирилловском монастыре был старец, по имени Савватий. Работая Господу день и ночь, умерщвляя тело свое постом, всенощным бдением и трудами на пользу обители и наставляя в благочестии приходящих туда богомольцев, он так прославился, что не проходило ни одного дня, чтобы не посетили его несколько человек для принятия от него благословения.

Но Савватий, ища славы не от людей, а от Бога, сокрушался о том, что везде говорят о нем, ибо, зная слабость человеческую, праведник не надеялся на свое сердце и боялся, чтобы не проникли в него чувствования гордости и высокомерия. Святой муж решился отыскать для себя место пустынное, никем не посещаемое.

Утвердившись в этом намерении, он принял благословение от настоятеля, простился с братией и ушел на остров Валаам, лежащий на Ладожском озере, где иноки, ночь трудясь в молитвах, днем снискивают себе пищу милостыней и трудами рук своих. Но слава всегда следует за тем, кто убегает от нее… Всех превосходя подвигами постничества, святой Савватий и здесь просиял именем не только в местах окрестных, но и в странах далее лежащих.

Опять человек Божий начал скорбеть и снедаться печалью, видя почтение от братии и от приходящих туда богомольцев. Но, к сердечному удовольствию, услышал он, что на море есть остров, называемый Соловецким, пустой и ненаселенный, отстоящий от берега на два дня плавания. Другой при первом слухе вообразил бы ужас пустыни, хлад севера, рев зверей и в трепете отвратил бы оттуда очи свои; святой Савватий, напротив того, воскликнул: «Здесь благоприятное для меня пристанище! Бог милосерд: Он пятью хлебами напитал в пустыне пять тысяч народа, пропитает и меня на острове хладном и пустом и даже пошлет для меня, чем накормить и пришельцев, если какие посетят обитель мою». Но настоятель и братия ни под каким видом не хотели расстаться со столь праведным мужем, ибо святой Савватий был венцом их братства.

Человек Божий решился уйти тайно. В глухую полночь он вышел из обители и направил путь свой к стороне моря; от живущих на берегу рыболовов вторично услышал он, сколь ужасная пустыня будет жилищем его, но все это святого отшельника только более восхищало. Идучи вдоль по берегу морскому, он нашел инока, по имени Герман, при часовне живущего, и соорудил с ним малую ладью.

Взяв с собой немного хлеба и необходимое орудие, с надеждой на Бога, они пустились в море. Тишина и попутный ветер способствовали им достигнуть желанного места. Там пустыннолюбцы водрузили крест, поставили келью и начали жить, работая Господу и снискивая себе пищу в поте лица своего. Их руки были заняты трудом, а уста благословляли Господа.

\section{Христианская любовь к ближним}

В Антиохии по случаю необходимо нужных новых податей для наступающего тогда военного времени произошел бунт, в котором даже статуи Феодосия Великого и покойной супруги его Флакиллы были с ругательством низвергнуты. Суд о столь неслыханной дерзости производился с надлежащею строгостью. Посланные от Феодосия чиновники лишили Антиохию наименования Восточной столицы; запретили даже общенародные бани; наполнили темницы не только виновными, но и теми, которые показались им подозрительными; описали имения у большей части знатных особ и повсюду расставили вооруженную стражу. Пораженные ужасом граждане непрестанно имели пред очами своими образ смерти и ожидали часа казни.

Здесь"=то сострадание пастырей и учителей христианских проявилось в высшей степени. Флавиан, архиепископ злосчастной Антиохии, невзирая на худую погоду, на беспокойства в пути, на свою старость, отправился в Царьград ходатайствовать за преступников у Феодосия. Святой Иоанн Златоуст утешал народ душеспасительными поучениями: представлял им необходимость истинного покаяния после столь жестокого злодейства и пагубные следствия отчаяния, которому после минутной дерзости предались антиохийцы. Пустынники, жившие близ города, стекаясь отовсюду, уверяли граждан, что они или испросят им милость, или умрут с ними. Проводя целые дни у входа в судилище для умилостивления судей, ночью они ложились около дверей темниц и готовы были предать жизнь за избавление братий своих: то обнимали колена судей, то обращались к ним с именем Божиим.

Один из таких человеколюбивых старцев, по имени Македоний, человек простой и не знавший светского обращения, но славный благочестием и святостью, встретил двух судей посреди города и приказал им сойти с лошадей. Судьи рассердились было, но, услышав о беспримерной жизни пустынника, исполнили его требование, обняли его и просили прощения. Тогда этот старец, исполнившись Божественной премудрости, воздвиг глас свой: «Пойдите, друзья мои, и сделайте государю от меня следующее представление: ты "--- император, однако человек; ты обладаешь людьми, которые суть образ Божий. Бойся гнева Создателя, если истребишь Его создание. Ты раздражен, что твои образы разрушены: менее ли прогневается Бог, когда ты разрушишь Его образы? Твои изображения бездушны и бесчувственны, а Его одушевлены и разумны. Твои медные образы опять восставлены и учреждены; но, когда умертвишь людей, чем можешь вознаградить твое преступление? Воскресишь ли их?» Сей глас ревности и любви сильно потряс сердца судей, равным образом и просьбы прочих иноков нередко исторгали у них слезы. Но они должны были в ответ и свое оправдание представлять им волю императора, опасность послабления государственным преступникам и необходимость исполнять требования правосудия.

Уже настал ужасный день, когда хотели читать решительный приговор обвиненным… Пустынножители опять собрались в судилище и неотступно просили отложить дело на некоторое время и ожидать новых от Двора повелений. Они принимали на себя идти к императору и умилостивить его своими слезами и, наконец, столько успели, что получили желаемое. Судьи дозволили им подать представления свои на письме и отослали оные к императору.

Достойные удивления пустынники, увидев, что дело, столь близкое их сердцу, находится в лучшем состоянии, возвратились в свои кельи. Их слезы и моление пред судьями имели то спасительное действие, что Флавиан успел между тем явиться пред Феодосием и преклонить сердце его на жалость. Это происходило следующим образом.

Прибыв в палаты, где находился Феодосий, архиепископ остановился от него вдали, как бы удерживаемый страхом, стыдом и печалью. Он стоял безмолвно, потупив глаза в землю; казалось, сам был виновен и просил помилования себе самому; только слезы и воздыхания говорили за него. Пораженный этим, Феодосий подошел к нему и с кротостью сказал: «Не погрешая против Неба, могу жаловаться на жителей антиохийских; я предпочитал их город всем городам моего государства: за столь многие милости и благодеяния мог ли ожидать от них столь злодейского воздаяния? Я не думаю, что сделал им какую"=либо несправедливость. А если иногда и был столько несчастлив, то могли бы они восстать на меня одного. Для чего же оскорблять умерших, которые против них не погрешили?» Он остановился на этих словах, и архиепископ, отерев слезы, наконец прервал молчание.

Он начал речь истинным признанием преступления антиохийцев, исповедуя, что нет соразмерного оному наказания. Но, сравнив их неблагодарность с благодеяниями императора, представил ему, что чем больше оскорбление, тем славнее прощение; привел в пример Константина Великого\footnote{Флавий Валарий Аврелий Константин, Константин I, Константин Великий (27~февраля 272~г., Нанес, Мезия "--- 22~мая 337~г., Никомедия) "--- римский император. После смерти отца, в 306~г., был провозглашен войском Августом, после победы над Максенцием в 312~г. в битве у Мульвийского моста и над Лицинием в 323~г. стал единственным полновластным правителем Римского государства, христианство сделал господствующей религией, в 330~г. перенес столицу государства в Византий (Константинополь), организовал новое государственное устройство. Константин прославлен в лике равноапостольных \textit{(святой равноапостольный царь Константин)}.}, который, будучи побуждаем царедворцами жестоко наказать мятежников, обезобразивших одну из его статуй, осязая рукой лицо свое и улыбаясь, ответствовал: я не ощущаю никаких язв; напомнил ему собственную его милость, когда он, на праздник Святой Пасхи, даровав жизнь заключенным в темницах преступникам, воскликнул: «О, если бы Господь дал мне силу воскрешать и мертвых!»; представил, что иудеи, язычники и варвары теперь смотрят на него и ожидают приговора виновным, чтобы, судя по его милосердию или жестокосердию, заключить о силе и святости веры Евангельской. Но, дабы истребить из сердца его опасение дурного примера, если столь великое преступление останется без наказания, представил ему, что прощение в этом случае не будет следствием малодушия и невозможности мстить, но действием веры и милосердия. И что злосчастная Антиохия страхом и угрызением совести уже наказана более, нежели сколько могли бы наказать ее меч и огонь.

Слезные просьбы и угрозы Суда Господня в устах человека Божия были столь действенны, что Феодосий не мог противиться силе убеждений. Не в силах удержать слез своих и сколько можно скрывая внутреннее смущение, он сказал только: «Если Господь благоволил простить убийц Своих, я ли не должен простить моих оскорбителей, "--- я, который такой же смертный, как и они, и раб Того же Господа?» Тогда благочестивый старец повергся к ногам государя и умолял Небо о его благоденствии. «Иди, отец мой, "--- обняв его, продолжал Феодосий, "--- и принеси скорее утешение сетующему народу; обрадуй его на праздник Христова Воскресения\footnote{Это происходило на последних неделях Святого Великого поста.} отпущением преступления его. Моли Бога, да благословит оружие мое, и будь уверен, что я по окончании войны приду сам утешить граждан антиохийских». Оказав всевозможное благоволение Флавиану, Феодосий вскоре отпустил его.

Возвращение святителя в Антиохию было подобно торжеству. Весь город встретил его за несколько стадий\footnote{\textit{Стадия} (\textit{греч}. stadion, \textit{лат}. stadium) "--- древнегреческая путевая мера в 140~шагов или 600~футов. 1~фут = 2~дюймам (30,48~см).}; дорога усыпана была цветами. Радость и молитвы были единое и общее чувство граждан.

\section{Празднословие и беседа духовная\footnote{Из Пролога, в 25"~й день апреля.}}

Состарившийся в постничестве Кассиан разговаривал с одним старцем о пользе души, и поскольку разговор их продолжался довольно долгое время, то старец начал дремать. Преподобный Кассиан, сожалея о нерадении к Богу инока, хотел обнаружить пред ним действие духа"=искусителя и произнес нечто, служащее только веселому времяпрепровождению… Вдруг инок очнулся и весь был во внимании, чтобы не проронить ни одного слова. Тогда Кассиан, вздохнув тяжко, сказал ему: «Как несчастны мы! Пока беседовали о вещах небесных, то дремота сжимала очи наши; а как скоро вылетело из уст веселое слово, мы воспрянули, и сон прошел».

Пораженный до глубины сердца наставлением, старец после этого всегда молился Богу, чтобы Он никогда не допускал его заснуть, когда идет разговор духовный, и немедленно ниспосылал бы на него крепкий сон, когда начнут празднословить, а тем более уязвлять имя других.

И мы, подобно беспечному иноку, все слушаем охотно, кроме нравоучений. Разница только в том, что инок исправился, а мы по большей части как живем, так и умираем.

\section{Ходи в церковь и никогда не выходи из нее, пока не кончится служба Божия\footnote{Из Пролога, в 6"~й день апреля.}}

Один добрый, но бедный человек имел сына. К несчастию, в их стороне наступило голодное время. Тогда, видя себя, жену и сына, истаивающих от голода, несчастный отец сказал: «Любезное дитя! Согласись, чтобы я продал тебя; мучительно с тобой расстаться, но этим только и мы, и ты можем избегнуть голодной смерти». "--- «Делай, что хочешь, батюшка, "--- отвечал юноша, "--- я повинуюсь с охотой». "--- «Но сын мой! "--- возразил отец. "--- Помни совет, который дам тебе: ходи всегда в церковь и никогда не выходи оттуда, не дослушав Божественную службу». После этого он отвел его и продал одному вельможе, а сын, со своей стороны, никогда не нарушал повеления родительского и, служа верой и правдой своему господину, всегда ходил в церковь.

Но добродетель беспрестанно искушается. По какому"=то несчастному случаю юноша подвигнул на гнев и мщение госпожу. Один молодой слуга во всем тому способствовал. Мстительная женщина решилась погубить невинного юношу. «Сей новокупленный, "--- с видом испуганным однажды сказала она своему мужу, "--- имеет злые мысли на жизнь твою». И с обыкновенною в таком случае хитростью так уверила его, что он, не разобрав дела, согласился верного раба предать смерти. В тот же день, увидев уголовного судью, безрассудный господин сказал ему: «Я пришлю к тебе одного из моих слуг с убрусом\footnote{\textit{Убрус} "--- покрывало, платок, полотно.}, ты прикажи отсечь ему голову и, завернув в убрус, отдай тому, кто придет после него», "--- по имени не назвав ни того ни другого.

Таким образом, добрый юноша, не зная ничего, по приказанию господина пошел на смерть свою, но по дороге увидел, что пред одной церковью отправляют молебствие; он вспомнил наставление отца своего и остался тут до окончания службы.

Между тем госпожа вышла из терпения, что так долго принуждают ждать смерти раба, и послала к судье своего любимца спросить у него, какой даст ответ на то, с чем послан новокупленный раб. Соблазнитель, идя дорогой и услышав молебствие, также остановился "--- из любопытства. Юноша, увидев его, спросил: «Куда идешь?» "--- «К уголовному судье, "--- отвечал он, "--- требовать ответа на то, зачем ходил ты: был ли у него?» "--- «Нет еще, "--- отвечал юноша, "--- но сделай милость, переменим наши препоручения, мне весьма хочется дослушать Божественную службу; ты отнеси этот убрус вместо меня, а я схожу за ответом». Молодой человек, взяв убрус, пошел, куда должно, и в тот же час был обезглавлен.

Служба Господня кончилась, и юноша по уговору пошел к судье уголовному. Там получил он что"=то завернутое в убрус и, не любопытствуя, принес домой. Как изумились господин и госпожа, увидев в живых того, кого послали на смерть! Они обмерли от ужаса, когда увидели в убрусе голову любимого слуги, которого послали за головой юноши. Потом спросили, как это происходило, и, услышав от него все подробно, увидели тут действие Промысла Божия. Супруг видел только свою неосмотрительность в обвинении одного и невинную смерть другого; а злобная жена, сверх того, видела тут праведный Суд Господень и, умилившись душой, перестала гнать юного раба своего.

Храм есть земное Небо, где невидимо присутствует Сам Бог. Христиане! Приходите в церковь, как в тихое пристанище от бури, от печалей житейских: там \textit{просите, и дастся вам; ищите, и обрящете; толцыте, и отверзется вам} (Мф. 7, 7).

\section{Смирение преподобного Феодосия\footnote{Прп. Феодосий, игумен Киево"=Печерский, постригся в 1032~г., скончался в 1074~г. Память его празднуется 3~(16) мая.}}

Однажды нужно было преподобному Феодосию Печерскому посетить великого князя Изяслава, находившегося тогда в дальнем расстоянии от Киева. Промедлив до вечера, угодник Божий хотел по обыкновению пешком идти в обитель, но князь приказал отвезти его в коляске.

Провожатый, человек молодой, видя преподобного Феодосия в весьма простой одежде, почел его за какого"=нибудь собирателя милостыни и позавидовал, что он, сидя в коляске, может спокойно заснуть. Наконец, он был так раздосадован, что вышел из терпения и сказал ему: «Черноризец! Время бы и мне отдохнуть на твоем месте». Святой муж, не сказав ни слова, вышел из коляски и посадил туда слугу княжеского.

Преподобный Феодосий, приняв на себя должность провожатого, то шел подле коня, то, утомившись, садился на него. Это продолжалось до рассвета, между тем молодой человек спал спокойно…

Наконец, с преподобным Феодосием начали встречаться вельможи, едущие из Киева к Изяславу: увидев праведника, они сходили с коней и ему кланялись. Слуга княжеский, пробудившись на голос Феодосия, изумился, увидев, что знатные люди воздают столь великое почтение старцу, которого он обидел; в ужасе он выскочил из коляски… Но этот ужас еще умножился, когда, приближаясь к монастырю, увидел он, что вся братия вышли к нему навстречу и с благоговением принимали от него благословение. «Кто сей старец?» "--- думал слуга княжеский и почитал себя погибшим, если Изяслав узнает о его дерзости… Но праведный муж вскоре успокоил его. Он ввел своего оскорбителя в трапезную и приказал поставить пред ним питье и яства; потом, дав немного денег, ласково отпустил домой.

Вот смиренномудрие истинно христианское! А мы везде любим быть первыми: не только стараемся в дружеских собраниях занять высшее место, но даже в храме Божием, где невидимо присутствует Царь царей, пред Которым в равном достоинстве раб и князь, мы оскорбляемся, если кто"=нибудь низшего против нас состояния станет впереди, "--- иногда с негодованием отталкиваем его.

\section{Царь"=отец}

Преподобный Арсений Великий\footnote{Прп. Арсений Великий жил в IV~в. Память его празднуется 8~(21) мая.}, не приняв еще образа иноческого, как человек мудрый и добродетельный был призван царем Феодосием в Царьград и сделан учителем Аркадия и Онория. Будучи кроток душою и уважая их как детей царских, он преподавал уроки всегда стоя, между тем как ученики его сидели в мягких и роскошных креслах.

Однажды случайно пришел к ним Феодосий. Увидев Аркадия и Онория сидящих, а Арсения пред ними стоящего, он оскорбился и сказал: «Такое ли от меня было тебе наставление? Не сказал ли я с самого начала, чтобы ты, как второй отец, почитал их своими детьми, а не царскими?» "--- «Все должно быть на своем месте, "--- с кротостью отвечал Арсений, "--- юношам нужно учение, а царевичам прилична честь царская…»

Феодосий, будучи еще более недоволен ответом Арсения, немедленно снял с детей своих знаки царские и, против воли посадив учителя, ученикам приказал стоять пред ним… «Если научатся бояться Бога, "--- сказал он, выходя от детей своих, "--- и будут достойны владычествовать над народом, то Промысл Божий даст им царствие на земле; если же будут горды и злобны, то лучше им не царствовать, нежели царствовать безумно. Молю Бога, чтобы дети мои лучше умерли в юности, нежели бы возросли на гибель души своей и на злосчастие народа».

Арсений, благословив в душе своей Феодосия, с того времени начал преподавать уроки сидя, а царевичи стояли перед ним.

Без сомнения, не желает счастья детям своим тот отец, который подает им хоть малейший повод гордиться своим богатством или знатностью.

\section{Смиренномудрие Арсения Великого}

Преподобный Арсений Великий был так смиренномудр и столько не надеялся на свой разум, что всегда искал чему"=нибудь научиться "--- даже от людей неученых.

Однажды этот праведник, разговаривая с некоторым старцем, пришедшим из Египта, просил у него наставления, как лучше и надежнее отгонять от сердца своего помыслы небогоугодные? Некто из иноков, слышавший их беседу, удивился тому и спросил у Арсения: «Почто ты, будучи искусен в греческом и латинском языках и зная разум всех книг, принимаешь уроки от невежды?» "--- «Я знаю греческий и римский языки, "--- отвечал праведник, "--- но по сию пору не могу выучить азбуки, которую сей простец знает весьма твердо».

Под этим святой Арсений разумел смирение, которое есть начало всех добродетелей, как азбука есть начало всех книг.

\section{Осуждение падает на осуждающего}

Некогда преподобный Арсений тяжко разболелся. Иноки перенесли его в церковную больницу и положили на мягкую постель.

Старец из другого монастыря пришел посетить больного и, увидев Арсения на мягком ложе, соблазнился и подумал: «Это ли отец Арсений? Как можно ему так покоиться?»

Но как пришелец изумился тому весьма неосторожно, то бывший тут священник отгадал мысли его и, отведя в сторону, спросил, кто таков был он, живучи в миру, и какова была жизнь его. «Я был пастух, "--- ничего не подозревая, отвечал инок, "--- и жил в чрезмерной бедности и печали». "--- «А теперь хорошо ли живешь?» "--- опять спросил у него священник. «Весьма спокойно, "--- отвечал он, "--- и имею все, что мне нужно». "--- «Знай же, "--- сказал священник, "--- что сей старец, которого видишь, был второй отец царским детям. Ему принадлежали тысячи рабов; его постель была мягка и роскошна; его богатства были бесчисленны: итак, не правда ли, что ты после трудов мирских ныне покоишься, а отец Арсений после покоя мирского ныне трудится?»

Старец признался в своем заблуждении и, поклонившись в ноги священнику, просил у него прощения, что в сердце своем оскорбил праведника.

\textit{Не осуждайте, да не осуждены будете} (Лк. 6, 37), сказал Христос Спаситель. Мы часто осуждаем ближнего, но сами на себя и оглянуться не хотим.

\section{Истинный пустынножитель}

Один римский вельможа, близкий родственник преподобного Арсения, при смерти своей завещал ему великое богатство. Праведник, получив духовную, едва успел узнать содержание, как хотел разорвать ее. Но слуга, который принес оную, пал к нему в ноги и умолял не делать этого; в противном случае строго взыщут с него. Тогда преподобный Арсений, отдавая ему назад завещание своего родственника, сказал: «Я умер прежде его: удивительно, как он, недавно умерший, делает меня, мертвеца, наследником богатства. Отдай, друг мой, бумагу назад тому, кто тебя с ней послал сюда».

Вот пример высокого пустынножительства! Кто посвятил себя на служение Единому Богу, тот не имеет нужды в богатстве.

\section{Излишняя на себя надежда пагубна\footnote{Из жития прп. Пахомия Великого (†~ок. 348), память которого празднуется 15~(28) мая.}}

Один инок желал быть мучеником за Христа, но поскольку Церковь наслаждалась миром, то и не знал он, как удовлетворить желание своего сердца. Видя в святом Пахомии дар чудотворения, он приходил к нему многократно и просил помолиться за него Богу, чтобы Тот удостоил его венца страдальческого. Сколько угодник Божий ни советовал ему, чтобы не принимал в сердце свое столь странных помышлений, сколько ни говорил ему: «Любезный о Христе брат! Переноси мужественно подвиг иночества; день и ночь занимайся богомыслием; непорочной жизнью подавай душеспасительный пример братии: и будешь иметь участие на Небеси с мучениками», "--- но все увещания святого Пахомия были тщетны. Старец с сокрушением и слезами умолял его.

Наконец праведник, желая избегнуть непрестанных просьб инока, сказал ему: «Хорошо, я помолюсь о тебе к Спасителю мира, да подаст тебе случай прославиться в страдальчестве. Но ты, как можно, берегись, чтобы вместо исповедания имени Христова не отвергнуть его в лютый час мучений. Я уверял тебя и теперь уверяю, что заблуждаешься, без нужды желая повергнуться в напасть, когда Сам Господь учит нас молиться, да не будем введены во искушение (см. Мф. 26, 41; Мк. 14, 38); но если желание твое так сильно, то пусть по"=твоему будет».

Через два года после их разговора случилось некоторым из братии быть в дальней веси для собирания камыша на постели. Но так как они там надолго задержались, то святой Пахомий, призвав этого старца, сказал ему: «Иди, посети трудников и отнеси им пищу» "--- и, благословляя его в путь, присовокупил: «\textit{Се ныне, время благоприятно, се ныне день спасения, ни едино ни в чем же дающе претыкание, да служение беспорочно будет}» (2~Кор. 6, 2~-- 3). Инок не мог догадаться, к чему относились слова угодника Божия, и отправился в путь свой.

Но, проходя чрез пустыню, он встречен был варварами, которые обитали в неприступных горах. Они окружили его, взяли в плен и увели в жилища свои. «Приди, старче, "--- ругаясь, кричали они, "--- поклонись богам нашим». Долго отрекался инок принести жертву идолам; но обнаженные над головой его сабли, угрозы неслыханных мук принудили несчастного сделать коленопреклонение злобожное. Убоявшись смерти, он забыл душу бессмертную и, отрекшись от Христа, ел и пил от жертвы их. После этого, как единоверец, он был отпущен без всякого вреда.

Сколь опасно полагаться на свои силы, когда человеческое сердце столь непостоянно и превратно, а плоть столь слаба и немощна!

\section{Неосмотрительность в обвинении}

В одной из Египетских пустынь жил благочестивый инок, по имени Дула\footnote{Прп. Дула страстотерпец, Египетский. Память его празднуется 15~(28) июня.}, честный, кроткий, против клеветы безответный, "--- истинный раб Христов. В этом терпении богобоязливый старец препроводил двадцать лет.

Но враг рода человеческого искони устраивает погибель тем, которые служат не ему, но Единому Богу. Будучи не в силах низложить огражденное щитом Небесной благодати сердце отшельника, он вооружился на него другим образом.

Один старец, не имевший страха к заповедям Господним, руководимый духом"=искусителем, ночью, как тать, вошел в церковь, похитил святые сосуды и, сокрыв их в тайном месте, затворился в келье своей, из которой прежде того выходил весьма редко. Когда пришло время утреннего пения, инок, вошедший в церковь, чтобы разжечь кадило, приметил с первого взгляда, что в храме неблагополучно; он немедленно поведал об этом настоятелю, настоятель объявил всей братии.

К несчастию, праведный Дула в то время из"=за болезни не пришел на утреннее славословие. Воспользовавшись этим, некоторые из его недоброжелателей подали свой голос, что святыню украл не кто иной, как Дула: лучшее доказательство, что он, против обыкновения своего, сегодня не пришел к утрени… Легкомысленные люди поверили. Вся братия ужаснулась, вознегодовала и велела невинного Дулу привести на собор.

Удрученного болезнью праведника насильно привлекли в церковь. Там в присутствии настоятеля и отцов, состарившихся в постничестве, не спрашивая, он ли точно святотатец, злые люди начали клеветать на него: иной говорил, что он тайно от всех ел; иной кричал, что он крал хлеб и продавал за монастырем. Слыша это, настоятель и старцы поверили клевете и с угрозами начали допрашивать невинного Дулу, где тот скрыл похищенное из церкви сокровище. Сначала раб Божий оправдывался и представлял убедительнейшие доказательства своей невинности. Но, видя, что ему не верят, скрепил сердце свое и замолчал, говоря только изредка: «Простите, отцы святые, меня, грешного».

После этого настоятель повелел совлечь с него монашеский образ и облечь в мирские одежды; невинный Дула был заключен в тяжкие узы и отдан на допрос эконому… Чего страдалец не вытерпел тут! Но на все угрозы, на все мучительства отвечал только: «Простите меня, грешного!»

Наконец, раздраженный настоятель и братия отослали его к градоначальнику "--- там новые пытки; но на каждый вопрос: «Где сокрыл похищенное из церкви сокровище?» "--- невинный Дула повторял одно и то же: «Не имею у себя ни серебра, ни злата». Истощив весь ужас мучений, градоначальник повелел ввергнуть его в глухую и смрадную темницу.

На следующий день судья призвал к себе настоятеля и прочих иноков и объявил им, что Дула ни в чем не признается; но ожесточенные люди представили на него новые лжедоказательства, и обманутый градоначальник подписал приговор, чтобы у Дулы как у святотатца отсечь руки.

И повели невинного страдальца на место казни… В то самое время настоящий похититель почувствовал в душе своей угрызение совести. Он ужаснулся и сказал сам себе: «Увы! Что будет мне, если обнаружится злодейство мое? Но положим, что тайна моя здесь, на этом свете, останется тайной: какой же дам ответ я на Страшном Суде Христовом? О, горе мне, злочестивцу! Я "--- святотатец, я кровопийца!» Он немедленно идет к настоятелю и говорит ему: «Отче, церковное сокровище нашлось; клянусь, что Дула невинен!» Немедленно настоятель послал с этим известием к градоначальнику, и страстотерпец возвратился в обитель свою.

В тот же час иноки узнали о невинности его, узнали настоящего похитителя и припали к стопам праведника, умоляя его, да простит им столь лютый грех. Но будет ли искать удовлетворения за обиды свои тот, кто живет по примеру Христа Спасителя, невинно приявшего оплевания, заушения, венец терновый и смерть крестную?..

Никого не подозревай в преступлении, пока не будешь иметь несомненные на то доказательства. Неосмотрительность в этом случае губит человека невинного, меж тем преступник остается не наказанным за большие злодеяния.

\section{Исправленный мститель\footnote{Из Пролога, в 20"~й день июня.}}

Один инок, будучи обижен другим иноком, пришел к некоторому старцу и, рассказав, что и как между ними произошло, с гневом воскликнул: «Я не премину отомстить ему!» Сколько старец ни уговаривал его, сколько ни советовал ему, чтобы он мстить за себя предоставил Богу, но огорченный инок не слушал его и беспрестанно повторял: «Не могу быть спокоен, пока за оскорбление не воздам оскорблением».

Оплакивая заблуждение инока, старец несколько умолк; потом сказал ему: «Время молитвы; принесем, любезный брат, жертву Богу». Инок восстал, и старец начал читать следующее: «Господи Боже наш! Мы на Тебя не уповаем; не пекися о нас, не мсти за нас; мы сами все начнем и кончим». Услышав это, инок ужаснулся и пал в ноги старцу. «Не хочу иметь ссоры с моим братом, "--- сквозь слезы говорил он, "--- не хочу мстить ему». Потом, обратившись к Богу, воскликнул: «Един мститель неправды! Прости мое согрешение!» Старец, веселясь о покаянии инока, сказал ему: «Будь уверен, сын мой, что претерпевший досаду без труда спасется; напротив того, кто гневается на брата своего, тот погубляет всю свою добродетель и рабствует дьяволу».

Христиане! Оставьте Господу Богу отмщать за те обиды, которые нанесет вам ближний; Он воздает всякому по делам его; а ваше дело "--- простить оскорбителя.

\section{Повиновение власти}

Греческий царь Валент, держась Ариевой ереси, осудил на изгнание за реку Истр Евсевия, епископа Самосатского\footnote{Священномученик Евсевий, епископ Самосатский (†~380). Память его празднуется 22~июня (5~июля).}. Посланный для исполнения этого приговора, прибыв в Самосаты, не знал, как приступить к делу, ибо народ любил святителя и готов был лучше решиться на общий бунт, нежели отдать его в руки правосудия. Узнав это, святой Евсевий поздно вечером призывает к себе царского посланника и, приняв его ласково, говорит ему: «Молчи и никому не объявляй причины твоего сюда прибытия; в противном случае народ взбунтуется и убьет тебя; я не хочу быть виновником твоей смерти». После того Евсевий совершил вечернее славословие Богу и, когда настала ночь, открыл тайну и намерение одному из слуг. Рано поднялся с ложа своего и вышел из дома архиерейского, сопровождаемый верным спутником, который нес за ним возглавие\footnote{\textit{Возглавие} "--- изголовье, подушка.} и книгу "--- все богатство, которое святитель взял с собой. Достигнув реки Евфрата, близ стен градских текущей, он отдал себя в руки царского чиновника и, сев в ладью, поплыл к городу Зевгме.

Как ужаснулся народ самосатский, узнав о судьбе святителя своего! Все пришли в волнение, все рыдали, все спрашивали, в которую сторону святой Евсевий направил путь свой, и, когда узнали, что целью его путешествия изначально был город Зевгма, с поспешностью устремились на лодках за архиереем Божиим. Они настигли его там, где надеялись; их плач и рыдание слышны были издалека.

Каких не истощили тут они просьб и молений, чтоб возвратить к себе пастыря своего и учителя! Каких не употребили убеждений! Они припадали к ногам его, омывали их своими слезами, уверяя, что паства его достанется какому"=нибудь хищному волку "--- арианину; но святой Евсевий на каждое новое убеждение отвечал этими только словами: «\textit{Всяка душа властем предержащим да повинуется, несть бо власть аще не от Бога, сущия же власти от Бога учинены суть. Темже противляяйся власти, Божию повелению противляется, противляющиися же себе грех приемлют} (Рим. 13, 1~-- 2). Так, дети мои, "--- каждый раз присовокуплял он, "--- невозможно мне, подданному, противиться царскому повелению; также и вы, как подданные, будете несчастны здесь и в будущей жизни, если не будете послушны воле царской. Я первый в этом случае готов обвинить вас».

Народ, видя непреклонность своего пастыря, предлагал ему в путь: иной злато и серебро, иной одежды, иной рабов; но святому человеку, кроме книги и самой простой одежды, ничего было не нужно. Единственно в угождение духовных чад своих он взял с собой несколько маловажных вещей; напомнил всем, чтобы твердо держались догматов Православной веры, помолился о них, благословил и отправился в предлежащий путь.

\begin{center}\small\textsc{Конец первой части.}\end{center}

\chapter{ЧАСТЬ ВТОРАЯ}
\section{Косма и Дамиан, безмездные целители\footnote{Бессребреники Косма и Дамиан, в Риме пострадавшие (†~284). Память их празднуется 1~(14) июля.}}

Святые страстотерпцы Косма и Дамиан были единокровные братья, родом римляне, воспитанные в христианском благочестии. Смолоду научившись врачебному искусству, они с изумительным успехом исцеляли болезни, даже самые опасные, ибо содействовала им благодать Святого Духа. Каждый день стекались к ним страждущие и тем в большем количестве, что богобоязливые врачи ни от кого не требовали награды за труд свой; они об одном только просили исцеляемых "--- чтобы те веровали во Христа Спасителя. Всегда имея успех свыше, они трудились не в одном Риме, но ходили по окрестным странам и, исцеляя недуги, обращали людей на путь истины.

Но благочестивым юношам казалось, что они, исцеляя тело от болезни и душу от злобожия, делали не все, и к этим великим благодеяниям присовокупили третье: благодетельствовали бедным. Получив богатое наследство от родителей своих, они продавали сокровища и питали голодных, одевали нагих, являли всякую милость страждущим.

Образ их целения был не менее боголюбезен. Они говорили всякому больному: «Мы только возлагаем на тебя руки, но сами по себе сделать ничего не можем; действует всемогущая сила Христа, Единого Истинного Бога. Если обещаешься в Него веровать, здрав будешь». Таким образом, расслабленный язычник отходил от них здрав "--- христианином.

О вы, которые готовитесь быть пастырями и учителями словесного стада Христова, дети священно"= и церковнослужителей! Возьмите себе в пример угодников Божиих Косму и Дамиана: будьте целителями душ и телес.

\section{Образ подаяния милостыни\footnote{Из жития прп. Памвы пустынника (IV~в.), память которого празднуется 18~(31) июля.}}

Святая Мелания, расточая ради Христа сокровища свои, достигла Александрии и, услышав о богоугодном житии святого Памвы, пришла в его пустыню принять благословение. И принесла с собой триста литр\footnote{\textit{Литра} "--- церковная мера веса (см. Ин. 19, 39).} серебра, умоляя его, чтобы из этого сокровища взял, сколько пожелает. Но человек Божий, занимаясь своим обычным рукоделием, даже не воззрел на серебро и только сказал: «Да воздаст тебе Господь Бог по усердию твоему». Когда же благодетельница бедных не переставала умолять его, чтобы принял от ее усердия хоть немного, то святой Памва сказал служащему собрату: «Возьми у праведницы, что даст тебе, и раздай старцам, живущим в Ливии и на островах, ибо земля их бесплодна; но ничего не отделяй инокам египетским: живучи на земле плодоносной, они могут быть сыты от труда рук своих…»

Святая Мелания, отдавая серебро свое иноку, сказала преподобному Памве: «Отче! Здесь серебра ровно триста литр; посмотри его». "--- «Бог, Которому ты, дочь моя, принесла в жертву твое сокровище, "--- отвечал человек Божий, "--- не имеет нужды спрашивать у тебя, сколько оного. Тот, Который перстом измеряет землю и взвешивает горы, неужели не знает количества серебра твоего? Если бы ты подавала мне, я пересчитал бы оное; но ты даешь Богу, Который не презрел и две лепты\footnote{\textit{Лепта} "--- еврейская мелкая монета, грош, денежка.} от вдовицы, но принял их лучше бесчисленных сокровищ: итак, молчи и не воструби пред тобой».

Поистине, что человек отдает бедному, то отдает Самому Богу. Сколь великое побуждение быть благодетельным!

\section{Пустыннолюбец и страннолюбец\footnote{Из Пролога, в 18"~й день июля.}}

В Египте были два единоутробных брата, Паисий и Исаия, дети богатейшего в той стране купца. После смерти родителей, разделив все наследство на две равные части, они начали между собой рассуждать, какую жизнь избрать им. «Если будем заниматься торговлей, "--- говорили они, "--- то после смерти нашей кто знает, кому достанутся труды наши? Притом всегда должно будет бояться, чтобы не обнищать, чтобы не попасть к разбойникам, чтобы не потонуть в море». Итак, после некоторого размышления они решились положить богатство свое в сокровище"=хранительницу Небесную на пользу душ своих.

В этом намерении один из них роздал имущество свое нищим, Божиим храмам, отшельническим обителям и, ничего для себя не оставив, ушел в пустыню. Там, питаясь трудами рук своих, молился Богу и умерщвлял страсти свои. Другой, построив для себя небольшой монастырь близ мирских селений, занимался странноприимством и питал нищих; он соорудил кельи для приходящих и больницу, где всех успокаивал, всем служил с усердием, а в субботу и воскресенье учреждал для нищих две, три и четыре трапезы. Таким образом оба брата жили до конца дней своих.

После смерти их между иноками произошло рассуждение и невинный спор, кто более угодил Богу "--- Паисий или Исаия? Одни величали того, кто в один раз благорасточил имение свое и отошел в пустыню на безмолвие; другие ублажали того, кто сокровище свое употреблял во всю жизнь на пользу странников, нищих и больных.

Будучи не в состоянии решить сами этой духовной распри, они прибегли к преподобному Памве и спросили, который из братьев, Паисий или Исаия, получил большую награду от Бога? «Они оба равно любезны Богу, "--- отвечал святой старец, "--- ибо странноприимец уподобился праведному Аврааму, а пустынник "--- пророку Илии». "--- «Но пустынник, "--- возразили некоторые из братий, "--- исполнил заповедь Евангельскую: \textit{Продаждь имение свое и раздаждь нищим, и взем крест последуй Христу, во алчбе и жажде пребывая по вся дни} (ср. Мк. 10, 21)\footnote{\textit{Пойди, все, что имеешь, продай и раздай нищим, и будешь иметь сокровище на небесах; и приходи, последуй за Мною, взяв крест} (Мк. 10, 21).}; а странноприимец, хотя от своего имения и награждал нищих, однако и сам имел покой: ел и пил с больными и странниками». Напротив того, другая сторона утверждала, что и странноприимец исполнил слово Христово: \textit{Не приидох, да послужат Ми, но да послужу им} (ср. Мк. 10, 45)\footnote{\textit{Ибо и Сын Человеческий не для того пришел, чтобы Ему служили, но чтобы послужить} (Мк. 10, 45).}; исходя во все дни на распутья, ища странных, нищих и больных, вводя их в дом свой и там упокоивая, Исаия служил бесчисленному множеству людей; если и за едину чашу студеной воды, жаждущему данной, обещана мзда от Бога, то сколь великую получил награду этот странноприимец! Видя такое разномыслие иноков, преподобный Памва сказал им: «Братия! Подождите, доколе Сам Бог разрешит вопрос ваш; я буду о этом молиться».

Через несколько дней вторично пришли к нему братия и спрашивали о будущем жребии двух благодетельных братьев. «Свидетель Бог, "--- сказал им преподобный Памва, "--- что обоих братьев, пустыннолюбца Паисия и страннолюбца Исаию, я видел вместе в Раю стоящих». Услышав это, те и другие иноки между собой согласились и, хваля Бога, разошлись по своим кельям.

Добродетельный пустынник молится о грехах наших и тем облегчает нам подвиг на трудном пути спасения. Добродетельный мирянин разливает милость на бедных и тем облегчает судьбу страждущего человечества. Не оба ли они достойны блаженства, которое уготовил Бог любящим Его?

\section{Презрение богатства\footnote{Из Пролога, в 25"~й день июля.}}

Святая девица Олимпиада\footnote{Св. жена Олимпиада диакониса (†~409). Память ее празднуется 25~июля (7~августа).}, дочь славных и благородных родителей и родственница императора Феодосия Великого, в нежной юности обручена была с Невредием, сыном знатнейшего вельможи. Но так как жених ее еще до брака умер, то Олимпиада, оставшись девицей, решилась проводить жизнь свою в девстве. Вскоре последовала смерть ее родителей, и Олимпиада, сделавшись наследницей бесчисленных богатств, все посвятила в жертву Богу: обогащала церкви, наделяла пустынножителей; нищие и больные, странники и разорившиеся, вдовы и сироты "--- все называли ее матерью и никогда не выходили из дома ее с пустыми руками.

Между тем Феодосий Великий, видя красоту ее и благонравие, вздумал отдать ее в супружество одному из своих родственников "--- Елпидию, но Олимпиада не хотела этого. Царь неоднократно посылал к ней бояр своих, советуя и увещевая, чтобы не отвергала счастья быть супругой человека знатного; но она всегда отвечала: «Если бы Господь хотел, чтобы я была супругой, то не взял бы у меня первого жениха». Наконец, Феодосий разгневался и повелел взять в опеку все ее имение и держать дотоле, пока Олимпиаде исполнится тридцать лет. К этому побудило его, может быть, то, чтобы она чрез беспрестанные милостыни сама наконец"=то не была принуждена просить милостыни.

Олимпиаде не позволено было видеться с богоугодными пастырями душ, запрещено ходить в церковь. За все это святая девица благодарила Бога, а к царю написала следующее: «Ты оказал мне поистине царскую милость и честь, принадлежащую одним святителям, повелев другому хранить тяжкое бремя "--- мои сокровища; я прежде много заботилась: теперь спокойна. О, государь! Еще более облагодетельствуешь меня, если повелишь все раздать церквам и нищим: я избегну чрез то суетной славы прослыть благодетельницей. Избегну непредвидимых случаев "--- пренебречь \textit{Сокровище} благих. Чего не может сделать мятеж мира этого?»

Царь, прочитав письмо ее, удивился сердцу и уму Олимпиады и возвратил ей право располагать своим богатством. «Столь добродетельная и богоугодная девица, "--- сказал он, "--- лучше всех нас знает, как употреблять блага мира этого». Вскоре Олимпиада была посвящена в сан диаконисы и, совсем оставив суетный мир, до смерти служила Единому Богу.

Горе тем, которые стараются отклонить нас от пути спасения! Святая Олимпиада имела столько мужества, что могла устоять против ласк и угроз. Но не всякий человек имеет сердце этой святой девицы; не всякий может отринуть искушение. Тогда Бог взыщет на душе соблазнителя.

\section{Истинная добродетель имеет великую силу и над неверными\footnote{Из Пролога. В 25"~й день июля.}}

Преподобный Макарий\footnote{Преподобный Макарий Желтоводский, Унженский (†~1444). Память его празднуется 25~июля (7~августа).}, обитая в тесной пещере у Желтых вод, молился Богу и благодетельствовал ближним: не только единоверных себе христиан, но и приходящих по какому"=нибудь случаю татар успокаивал и довольствовал пищей и питьем "--- за что имя его известно было повсюду.

В то время Улу"=Махмет, царь Казанский, устремился с воинством на Нижний Новгород и все, что ни встречалось ему, опустошал огнем и мечом. Наконец, разъяренные татары напали на пустыню преподобного Макария и находившихся там иноков и бельцов "--- иных изрубили, иных увели в плен, а обитель сожгли. Четыреста мужей, кроме жен и детей, обремененные оковами, были уведены варварами в страну дальнюю, в рабство народу дикому. Между ними находился и святой Макарий.

Когда вместе с прочими пленниками представили его пред гордым Улу"=Махметом, этот вождь, увидев достопочтенный и кроткий взор его, украшенную сединами главу, сверх того узнав о его благочестии и добродетелях, умилился в душе своей и с гневом сказал воинам: «Почто оскорбили столь доброго и святого мужа, который не стоял против вас с оружием? Почто разорили обитель его? Или не знаете вы, что за этих кротких людей гневается Бог, Единый над всеми царствами и народами?» Умягченный Самим Богом, вождь татарский дал свободу святому Макарию и прочим пленникам из одного с ним места, возвратил всем имущество и отпустил в Русь, с тем только условием, чтобы ушли далее от Желтых вод, которые по праву завоевания, как говорил он, принадлежали татарам. Впрочем, по просьбе праведника позволил ему там остаться на столько времени, сколько потребно для погребения избиенной братии.

Зрелище столько же прекрасное, сколько прежде того было плачевное! Идет старец, как отец, радующийся о детях своих, как пастырь, обретший овча погибшее; за ним "--- множество людей, которые один пред другим стараются облобызать воскрилия риз его… В древние и нынешние времена шел ли в таком торжестве какой"=нибудь царь, победитель народов?

\section{Как святой Александр сделался епископом Команским\footnote{Из Пролога. В 12"~й день августа. Священномученик Александр, епископ Команский (†~III~в.). Память его празднуется 12~(25) августа.}}

Святой Александр, уроженец команский, был человек сколько благочестивый, столько и просвещенный. Обладая в высшей степени этими преимуществами, он мог бы иметь богатства и приобрести славу. Но избрал, напротив того, самовольную нищету: жег угли, привозя на торжище, продавал их и тем снискивал для себя кусок хлеба. Всегда с замаранным лицом и худой одеждой, он не имел другого имени, как Александр Угольник. Но Господь, на смиренных глядя и вознося их, удивил на нем милость Свою следующим образом.

В городе Комане умер бывший пред тем епископ. Граждане послали от себя поверенных в Неокесарию к чудотворцу Григорию, умоляя его, чтобы тот пришел к ним для избрания и посвящения епископа. Святой Григорий исполнил их желание немедленно. На установленном для этого соборе начали рассуждать, кого избрать своим пастырем и учителем: иные представляли благородных, иные богатых, иные красноречивых, иные благообразных; но святой Григорий в этом случае не был скор: в уповании на Бога, что Он Сам покажет человека, достойного получить сан святительский, напоминал собору, как избрал Бог Давида, да пасет Израиля: ибо, когда Иессей привел своего старшего сына Елиава к святому Самуилу пророку и он вопросил Господа: «\textit{Сей ли пред Господем помазанник Его?}» "--- тогда Господь сказал Самуилу: \textit{Не зри на лице его, ниже, на возраст величества его} (1~Цар. 16, 6~-- 7). «Должно и нам, "--- говорил святой Григорий, "--- избрать пастыря граду этому, не на лицо взирая; ибо наружность не есть достоинство, но истинное величие человека есть сердце его».

Этот совет праведного мужа некоторым из граждан показался неприятен. Они начали роптать и, усмехаясь между собой, говорили: «Если не должно уважать наружность, то пусть будет избран и посвящен в епископы угольник Александр». Услышав имя человека, по их мнению ничего не значащего, все засмеялись, а святой Григорий немедленно спросил, кто такой Александр, и велел представить его на собор. Как скоро вошел он в собрание, все на него обратились и начали снова смеяться, ибо от угля он был весь черен, в разодранной, измаранной одежде "--- настоящий эфиоп. При всем том всеобщий смех не смутил его: ничему не внимая, он стоял пред святителем благопристойно. Тогда"=то святой Григорий познал Духом живущую в нем благодать Божию и, восстав с места своего, отвел в особливый покой. Там наедине вопросил Александра, кто он, и заклинал именем Божиим сказать о себе истину. Александр хотя и желал остаться в неизвестности, однако не смел солгать пред столь великим святителем и открыл все: кто был и для чего принял образ нищеты и уничижения. Беседуя с ним, святой Григорий открыл в нем великое знание не только в любомудрии света, но и в Божественном Писании. Наконец, повелел служителям своим увести его в дом свой, там омыть, облечь в приличные одежды и опять представить на собор.

Между тем святой Григорий сидел на своем месте, упражняясь в богодухновенной беседе. Вдруг был введен Александр в одежде светлой, лицом благообразный. Увидев его, все изумились; святой Григорий начал предлагать ему вопросы из Священного Писания, на которые Александр отвечал благоразумно, удовлетворительно. Тогда все познали заблуждение свое о столь мудром муже и устыдились тех насмешек и оскорблений, которые только перед этим ему сделали; тогда"=то увидели они исполнение словес Господних: \textit{Яко человек зрит на лице, Бог же зрит на сердце} (1~Цар. 16, 7). Наконец, все, сколько ни было граждан на соборе, с единодушием и радостью избрали угольника Александра своим пастырем и учителем. И не обманулись они! Ибо этот святитель был богобоязлив, благотворителен, кроток, великодушен, в трудах неутомим; а когда говорил поучение к народу, \textit{течаше, аки река, от у стен его благодать Духа Святого, всех сердца в умиление приводящая}.

\section{Пустынник\footnote{Из Пролога, в 27"~й день августа.}}

Некоторый Египетский князь, удивляясь пустынному безмолвию преподобного Пимена\footnote{Прп. Пимен Великий (†~ок. 450). Память его празднуется 27~августа (9~сентября).}, просил у него позволения посетить его. Старец опечалился и сказал сам себе: «Если вельможи начнут посещать меня, то и весь народ будет ходить ко мне. А это нарушит пустынную жизнь и, может быть, ввергнет меня в сеть гордыни». Размыслив таким образом, святой Пимен отвечал решительно: «Умоляю князя, чтобы не приходил ко мне; ибо скорее изгонит меня от пределов своих, нежели увидит».

Князь опечалился и сказал: «Видно, грехи мои причина, что Бог не сподобил меня видеть человека Божия». Но, желая каким бы то ни было образом получить столь великое счастье, он взял под стражу и посадил в темницу сына сестры его в надежде, что святой Пимен будет ходатайствовать о юноше. «Если угодник Божий придет ко мне, "--- сказал он, "--- то молодого человека освобожу; если же нет, предам его казни». Несчастная мать, услышав этот отзыв, без памяти побежала к брату своему и с рыданием объявила ему о бедственной участи своего сына и об условии, с которым князь освобождает его от смерти; но праведник даже не отверз ей дверей, не дал никакого ответа. Наконец, растерзанная горестью мать начала поносить и укорять его. «Немилосердный, злонравный человек! "--- вопияла она. "--- Ужели не трогает тебя слезное рыдание единоутробной сестры? Ужели не трогает тебя неминуемая смерть племянника?» На все укоризны старец чрез ученика своего сказал ей только это: «Пимен чад не имеет и потому не болезнует». С горьким ответом сестра возвратилась, рыдая и проклиная брата своего.

Наконец, князь сказал: «По крайней мере, пусть хоть напишет ко мне что"=нибудь».

И старец написал к нему следующее: «Да повелит власть твоя лучше рассмотреть преступление юноши; и если оно достойно смерти, пусть умрет он, чтобы чрез казнь временную избег вечных мук; если же преступление ниже смерти, то, наказав его по закону, отпусти».

Князь, прочитав письмо, удивился великодушию и разуму богобоязливого и правосудного пустынника. Не получив счастья видеть его, он радовался тому, что, по крайней мере, получил письмо от руки его, и отпустил юношу.

Лжечувствительные люди нынешнего света почтут, может быть, преподобного Пимена жестокосердым. Но какая нужда до их суждений "--- они суетны! Праведник сказал, что преступление, достойное смерти, должно наказать смертью, чтоб преступник избег чрез то вечной смерти; а преступление, не стоящее смерти, также должно наказать по закону для исправления преступника и в пример другим. Решившись на просьбу, он поступил бы против правосудия.

\section{Средство сделать врага другом\footnote{Из Пролога, в 27"~й день августа.}}

В Египте был некий старец, совершенный пустынножитель, которого весь народ почитал совершенным в вере и благочестии. Все называли его своим богомольцем; все приходили к нему просить благословения. Когда слава об этом иноке более и более распространялась и из разных стран привлекала к нему посетителей, в то самое время пришел в Египет преподобный Пимен и там основал для себя пустынную обитель. Вдруг большая часть людей обратилась к нему, и келья святого Пимена всегда наполнена была народом. Старец разгневался на пришельца, начал завидовать и все поступки его толковать в худую сторону. Святой Пимен вскоре узнал это. «Братия мои! "--- сказал он ученикам своим. "--- Что делать нам с этими легкомысленными и досадными людьми, которые, оставив столь святого мужа, приходят к нам, ничего не значащим? Чем уврачуем гнев великого отца? Пойдем, будем умолять его; может быть, умилостивим».

Что сказал, то и сделал человек Божий. Они пришли к удрученному горестью старцу и постучались в двери его. «Кто там?» "--- спросил ученик пустынника. «Скажи отцу твоему, "--- отвечал угодник Божий, "--- что Пимен пришел с братией принять от него благословение». Услышав столь несносное имя, старец воспылал гневом. «Скажите Пимену, "--- с досадой воскликнул он, "--- что у меня нет времени видеть его с братией». "--- «Не отойдем отсюда, "--- отвечали на это в один голос Пимен и братия, "--- не отойдем от кельи, пока не сподобимся принять благословение у святого мужа…» Палимые зноем, остались они у дверей кельи.

Наконец, старец, видя смирение и терпение пришельцев, умилился, отверз им двери, принял с лобзанием братним и беседовал с любовью. «Поистине, не только справедливо все то, "--- сказал пустынник своим посетителям, "--- что я до этого времени слышал о вас, но я вижу в вас добрых дел стократ более…» И с того времени этот старец был другом и собеседником преподобного Пимена.

Христиане! Если приметите, что на вас кто"=нибудь гневается, всемерно старайтесь угождать ему и все ваши поступки устройте так, будто ничего не знаете о его к вам недоброжелательстве; тогда враг, без всякого напоминания с вашей стороны, узнает свою несправедливость и полюбит вас. Если же за вражду будете воздавать ему также враждой, то этот недоброжелатель вскоре сделается вашим врагом.

\section{Предмет разговоров\footnote{Из Пролога, в 27"~й день августа.}}

Некоторый инок, слыша о добродетелях преподобного Пимена, пришел к нему из дальних стран, чтобы воспользоваться поучением его. Праведник принял его ласково, и началась духовная беседа. Посетитель начал рассуждать от Божественного Писания о вещах таинственных, о служении на Небе Ангелов. Он говорил много и долго, а святой Пимен, потупив взор свой, молчал и не дал ему ни одного ответа. Таким образом, пришелец распростился с ним и, выйдя из кельи, сказал ученику святого Пимена: «Ах! Я напрасно прошел столь дальний и трудный путь: я хотел чему"=нибудь научиться и не услышал ни одного слова». Ученик объявил об этом преподобному Пимену. «Он говорил о таких вещах, которые превышают разум человека, "--- сказал праведник. "--- Чему тут могу я научить? Если бы он начал рассуждать о слабостях человеческих, я бы с удовольствием отвечал ему». Ученик опять вышел из кельи и все пересказал пришельцу.

Тогда"=то высокоумствующий инок узнал ошибку свою. Он вторично вошел к старцу и спросил у него: «Отче! Что должен делать я? Страсти всего меня взяли в плен свой!» Святой Пимен взглянул на него веселым взором. «Теперь за добрым и полезным делом пришел ты, любезный о Христе брат, "--- сказал он, "--- теперь отверзу уста мои». Опять началась духовная беседа и продолжалась до вечера.

Старец, воспользовавшись мудрым поучением Пимена, был вне себя от радости и возвратился в пустыню свою, благодаря Бога, что Он удостоил его увидеть столь великого нравоучителя.

На что и нам рассуждать о вещах таинственных, превышающих разум человека? \textit{Не испытуй, но веруй} (Мк. 5, 36), "--- сказал Сам Господь. Итак, в дружеских беседах будем лучше рассуждать о добродетелях, которых наиболее требует звание наше.

\section{Нравоучительные правила преподобного Пимена}
\subsection*{Не обнаруживай чужие грехи.\\Не вдруг верь, когда слышишь о ком худое}

\textit{Старец}. Отче! Я смущаюсь и хочу уйти отсюда.

\textit{Святой Пимен}. Для чего так?

\textit{Старец}. Слышу постыдные слова об одном из живущих здесь братий "--- и соблазняюсь.

\textit{Святой Пимен}. Ты слышал неправду.

\textit{Старец}. Но мне сказывал верный человек.

\textit{Святой Пимен}. Верный? Нет, если бы он был таков, то не объявил бы тебе того, что видел; не верь ему, пока сам не увидишь; ибо и Бог, слыша вопль содомский, не поверил, доколе Сам не увидел Своими очами. \textit{Вопль}, сказал Господь, \textit{Содомский и Гоморрский умножися, и греси их велицы зело: сошед убо узрю, аще по воплю их совершаются} (Быт. 18, 21~-- 22).

\textit{Старец}. Ах! Я и сам видел его прегрешение "--- сам, своими очами.

\textit{Святой Пимен (подняв с пола сучок)}. Что это?

\textit{Старец}. Это сучок.

\textit{Святой Пимен (воззрев на потолок и указав на бревно)}. А это что?

\textit{Старец}. Это бревно.

\textit{Святой Пимен}. Положи же в сердце твоем, что твои грехи "--- это бревно, а грех брата твоего "--- этот сучок.

\subsection*{О покаянии}

\textit{Старец}. Я сделал тяжкий грех и хочу три года быть в покаянии. Этого времени довольно ли для того, чтобы очистить грехи мои?

\textit{Святой Пимен}. Много.

\textit{Старец}. Так благослови, отче, на покаяние один год.

\textit{Святой Пимен}. Много.

\textit{Старец}. Итак, довольно сорока дней для покаяния?

\textit{Святой Пимен}. Много. Я думаю так: «Если человек покается от всего сердца и положит твердое намерение, чтобы на грех не возвращаться, то Бог приимет и трехдневное покаяние».

\subsection*{Об удалении от худых помыслов}

\textit{Вопрос}. Как можно предохранить себя от вражеских наветов?

\textit{Ответ}. Когда котел, снизу поджигаемый, кипит, тогда муха или другое насекомое не смеет прикоснуться к нему; когда же простынет, то и мухи, и все насекомые садятся на него. Равным образом и к человеку, на духовные дела ум свой устремившему, не смеет приступить враг рода человеческого. Кто же живет в небрежении и лености, того низлагает он без всякого труда.

\subsection*{О молитве}

Мы подобны тому, кто по левую сторону имеет огонь, а по правую сторону воду; если загорится от огня, берет воду и тушит огонь: огонь "--- злые помыслы, а вода "--- молитва.

\subsection*{О молчании}

\textit{Вопрос}. Молчать или говорить лучше?

\textit{Ответ}. Кто говорит во славу Божию "--- делает хорошо, и кто молчит во славу Божию "--- делает хорошо.

Некогда святой Пимен назвал юного брата Агафона отцом. Бывшие тут иноки сказали: «Как можно назвать отцом того, кто столь молод?» "--- «Молчаливые уста, "--- отвечал праведник, "--- сделали его старцем».

Есть люди, которые, никогда не говоря, говорят беспрестанно; и есть люди, которые, говоря с утра до вечера, всегда молчат. Под первыми разумею тех, которые молчат языком, но в сердце своем осуждают ближнего. Под вторыми разумею тех, язык которых говорит беспрестанно, но сердце ни о ком не помыслит худого.

\subsection*{О спокойствии душевном}

Дымом отгоняют пчел и берут сладость трудов их; равным образом порочные страсти отгоняют от души страх Божий.

\subsection*{О назидании души}

Когда хочет кто строить дом, то собирает разные к тому потребности; так и человек должен взять от всех добродетелей по некоторой части, чтобы соорудить в себе дом духовный.

\section{Истинный трудник}

Преподобный Моисей\footnote{Прп. Моисей Мурин (†~ок. 400). Память его празднуется 28~августа (10~сентября).}, из разбойника сделавшийся возлюбленным рабом Божиим, так по благочестию своему был славен, что не только молодые, но и состарившиеся в постничестве иноки приходили к нему учиться совершенству пустынной жизни и подвигам иночества. Вот один из примеров его трудничества.

Избрав самый жестокий образ жизни, этот мужественный подвижник ночью оставлял свое ложе, обходил всех своих собратий и, если находил при их хижинах водоносы, приносил им воду. Ибо от некоторых пустынь река была далеко, а многие из старцев были дряхлы и не могли подъять столь чрезмерного труда, чтобы к себе носить воду. Таким образом, святой Моисей молился Богу, трудами снискивал себе пищу и помогал другим.

Христиане! Помогайте ближним своим, по крайней мере, столько, сколько позволяют обстоятельства вашей жизни.

\section{Словопрение святого Александра, патриарха Царьградского\footnote{Из Пролога, в 30"~й день августа. Свт. Александр, патриарх Константинопольский (†~340). Память его празднуется 30~августа (12~сентября).}}

Угодник Божий Александр, достойно занимая престол патриаршеский, должен был бороться не только с арианами, которые, как волки, устремлялись на стадо Христово, но и с эллинскими мудрецами.

Однажды некоторые из них отважились напомнить царю, что он, оставив древнюю отеческую веру и приняв новую, какую"=то неизвестную, чрез то ускорит падение царства. Они поступили еще далее: просили царя, чтобы позволил им иметь прение с его святителем Александром, которая вера лучше: древняя или новая? И царь дал на то свое позволение.

Святитель Христов Александр хотя и не упражнялся в эллинском любомудрии, но, уповая на благодать Духа Святого, не отрекся от состязания. Собрались философы и все те, которые ненавидели веру Евангельскую. Святой Александр, видя несметное множество противников, уговорил их, чтобы избрали между собой мудрейшего и красноречивейшего, с которым он и будет состязаться; ибо невозможно, говорил он, отвечать сразу на тысячи голосов. После чего представлен был мудрец, славнейший по всей Греции; все прочие приготовились слушать их и, в случае необходимости, помогать философу языческому.

Святейший патриарх начал "--- и чем же?.. Следующими словами: «Именем Господа моего Иисуса Христа повелеваю тебе безмолвствовать…» Мгновенно у мудреца отнялся язык, так что не мог выговорить ни единого слова. Увидев это, собрание философов поражено было ужасом и стыдом: одни из них бежали, другие уверовали во Христа; онемевший мудрец припал к ногам святителя и знаками утверждал святость веры Евангельской, за что молитвами угодника Божия был исцелен и крестился в числе прочих. Все христиане прославили имя Господне. Этим кончился спор христиан и язычников.

\textit{Аще речете горе: двигнися, и двигнется} (ср. Мф. 21, 21), сказал Господь. Христианину все возможно, только бы не вмешивалась в его веру и капля сомнения: иначе Бог отвергнет молитву нашу, ибо это будет молитва неверующего.

\section{Святой Анфим, епископ Никомидийский, хочет лучше принять казнь, нежели сделать лжецами воинов\footnote{Из Пролога, в 3"~й день сентября. Священномученик Анфим, епископ Никомедийский (†~302). Память его празднуется 3~(16) сентября.}}

Когда из"=за одного подозрения, будто христиане зажгли дворец римского императора Диоклетиана, жившего тогда в Никомидии, восстало на них столь ужасное гонение, что не только рассекали их на части, сжигали в пламени, топили в море, но и тела умерших христиан исторгали из могил, чтобы с ними поступить так же, как с живыми, "--- тогда святитель их, Анфим, жил в неизвестной веси "--- Семане и, не имея возможности лично беседовать с заключенными в темницах христианами, через письма укреплял их в подвигах веры. Не от смерти убегал угодник Божий, но страшился, чтоб, приняв прежде всех венец мученический, не оставить без пастыря малое стадо свое; ибо тогда многие из страха перед наказанием принесли бы жертву идолам. Итак, Небесному Промыслу угодно было, чтобы пастырь тайно боролся с хищными волками за духовную паству.

Наконец убежище его было открыто. Мучитель обрадовался и послал немедленно воинов, чтобы взять его и, как ужаснейшего преступника, представить на суд царский. Воины достигли веси. Встретившись с угодником Божиим, у него самого спрашивают: «Где Анфим, учитель христианский?» "--- «Я отдам его в ваши руки, "--- отвечает он, "--- только отдохните у меня несколько минут от пути». Потом, взяв старейшину их за руку, он всех повел в келью свою и предложил пищу и питье. «Богу угодно было, "--- думал праведник, "--- открыть мое убежище; вижу, что угодно Его благому Промыслу, чтобы я, проводив мое стадо в место злачное, в место покойное, и сам последовал за ним… Пришло время мое… Благодарю Тебя, Боже Вседержителю!» По окончании умеренного стола вдруг встает он и говорит: «Теперь делайте то, зачем пришли сюда; я тот, кого ищете: я "--- Анфим; возьмите меня и ведите к пославшему вас».

Услышав это, воины изумились; устыдившись, они не могли воззреть на седую главу старца. Так дружеский прием и неустрашимость подействовали на них! Они душевно сокрушались и посоветовали архипастырю где"=нибудь укрыться, принимая на себя ответственность, что нигде не могли найти его. «Нет! "--- отвечал угодник Божий. "--- Великий пред Богом грех "--- обмануть всякого человека, тем более ужаснейший грех "--- обмануть царя своего. Исполняя волю его, вы невинны: виновен царь, делая поручения беззаконные». Сказав это, праведник пошел с ними в предлежащий путь.

Чем же занимался он, идя на смерть? Проповедовал им Слово Божие, поучал вере в Господа нашего Иисуса Христа, "--- и семя учения его пало на добрую землю, прозябло и принесло плод. Ибо, достигнув реки, святой Анфим сотворил о них молитву и крестил их во имя Отца и Сына и Святого Духа.

Вот пример великодушия, неустрашимости, любви к истине и усердия христианского! Не довольно было для святого мужа отдать себя на мучение и казнь: он угостил врагов своих, наставил их свято исполнять долг свой и ни в чем не обманывать власть, сущую от Бога. Шествуя на смерть, он еще усыновил их Господу Богу своему.

\section{Обращенный мучитель\footnote{Из Пролога, в 3"~й день сентября.}}

Святая мученица Василиса\footnote{Мученица Василиса Никомидийская (†~309). Память ее празднуется 3~(16) сентября.}, будучи девяти лет от рождения своего, столь неустрашимо предстала на суд, что сами мучители, сколь неистовы ни были, пришли в изумление. Сначала поступали с ней как с младенцем, сердце которого, на все удобопреклонное, без труда сдается на лестные обещания, но, увидев ее мужество и решимость умереть за веру Христову, сбросили личину кротости и из агнцев сделались тиграми. Однако ни пытка, ни огонь, ни лютые звери, впрочем кротчайшие этих мучителей, не могли поколебать в сердце ее любви к Жениху Небесному. Покровительствуемая Небом, она прошла сквозь весь ужас мучений и осталась жива и невредима.

Военачальник, ее мучитель, ужаснулся и был долгое время как бы в исступлении ума. Потом воскликнул: «Кто устоит против Господа!» "--- и повергся к ногам святой девицы. «Помилуй меня, раба Бога и Царя Небесного, "--- продолжал он, "--- и прости мне свирепства, которые, в слепоте моей, я истощил на тебя, упроси Бога твоего, чтобы за тебя не погубил меня; отныне верую в Него». Агница Христова велегласно прославила Бога, озарившего лучом истины душу мучителя, вскоре принявшего Святое Крещение.

Так действует Небесный Промысл, избирая человека орудием своей благости, своего могущества!

\section{Благочестивый художник\footnote{Из Пролога, в 5"~й день сентября.}}

Богатый вельможа заказал молодому золотых дел мастеру сделать крест и усыпать драгоценными камнями, чтобы приложить его в церковь Божию. Отвесив, сколько должно, золота, он отпустил мастера от себя.

Юноша был благочестив. Занимаясь работой, он размышлял: «Сколь великое средство спастись имеет этот богатый человек! Сколь великую получит он благодать от Бога, дав Ему столько золота! Для чего же и мне, "--- подумав, продолжал он, "--- не прибавить своего золота в этот крест, чтобы с ним вместе принять мзду на Небе?»

Что вздумал, то и сделал: приложил в крест свои последние десять златниц и, кончив работу, принес к вельможе, чтобы свесить золото и потом вставлять камни.

Как удивился вельможа, увидев, что крест против данного золота тянет более. «Что значит это? "--- с гневом сказал он. "--- Конечно, ты, утаив от меня часть золота, положил примеси и ошибся в весе? Признайся, неопытный обманщик!» "--- «Сердцеведец Бог свидетель, "--- отвечал молодой человек, "--- что ни зерна не взял я твоего золота, но еще приложил своего, чтоб и мне вместе с тобой сделать много добра; я надеюсь, что и у меня Христос примет это маловажное приношение, как принял две лепты от вдовицы». "--- «Но точно ли с этим намерением ты сделал это?» "--- возразил удивленный вельможа. «Клянусь небом и землей!» "--- сказал юноша.

Тогда богобоязливый вельможа обнял молодого человека и сказал ему: «Когда было у тебя столь доброе намерение, чтоб иметь со мною часть на Небеси, то от этого часа имей со мною часть и на земле: я усыновляю тебя и делаю наследником всего моего имения. Вместе прославим Бога, дающего награду и здесь, и в будущей жизни тем, которые с верой приносят Ему дары свои».

После этого вельможа и юноша жили неразлучно, как отец и сын. Так награждает Бог всех тех, которые с усердием приносят Ему дары свои!

\section{Чудо от Архистратига Михаила, бывшее в Колоссаях, еже есть в Хонех\footnote{В 5"~й день сентября.}}

В Колоссаях Фригийских, близ Иераполя\footnote{На этом месте некогда было капище, в котором язычники поклонялись какой"=то ехидне. Св. Иоанн Богослов, будучи в Иераполе, истребил оную и пророчествовал, что тут воссияет благодать Божия и будет присутствовать Архистратиг Михаил. Через некоторое время открылся чудотворный источник, и тем оправдалось пророчество: в честь горнего воеводы соорудили храм.}, был храм во имя Архистратига Михаила, сооруженный над источником, чудотворная сила которого отовсюду призывала многочисленный народ, ибо все пьющие и моющиеся в струях его получали исцеление от болезней, а неверные, испытав на себе могущество Небес, принимали Святое Крещение. Ожесточенные язычники от ярости скрежетали зубами. Но, к вящему их огорчению, некто богобоязненный Архипп жил при этой церкви и при этом источнике. Посвятив сердце и ум Единому Богу, он имел попечение не только о своем, но и об общем спасении: наставлял на путь истинный неверных и крестил их в струях живоносного источника.

Кого бы не могли примирить с Небом Евангельское красноречие уст его и сила вод, посвященных воеводе Небесных Сил? Но поклонники идолов изыскивали все средства истребить чудотворный источник, и после нескольких тщетных покушений наконец дьявол вложил в сердца их следующий совет.

Недалеко от святого места текли две реки, Ликопапер и Куфос, и, достигнув горы, соединялись в одно русло. А так как местоположение этих рек было несколько выше церковного, то злочестивым не трудно было потопить храм и залить источник. Собравшись в великом числе, они начали копать ров "--- от одного весьма огромного, вросшего глубоко в землю камня, который лежал близ алтаря церковного. Между тем святой Архипп неусыпно молился Богу, да не допустит врагов Своих поругаться над святыней. Уже в их злонамеренном труде минул десятый день; от трех стадий суши остался весьма узкий перешеек, который быстрота вод могла промыть в несколько часов. Эллины стояли на высоком месте, чтобы видеть истребление храма и источника.

В третьем часу ночи вдруг восшумели воды и устремились на храм… Святой Архипп пал на колена и воскликнул: «\textit{Воздвигоша реки, Господи, воздвигоша реки гласы своя: возмут реки сотрения своя от гласов вод многих. Дивны высоты морския: дивен в высоких Господь. Дому Твоему подобает святыня, Господи, в долготу дний} (Пс. 92, 3~-- 6)». В это мгновение праведник услышал глас, повелевавший ему выйти из храма; он повиновался и узрел Архистратига в образе человеческом, но во славе Небесной. Старец пал от страха на землю. «Восстань "--- и узришь силу Божию в водах этих», "--- сказал Михаил. Мгновенно святой муж увидел столп огненный от земли до неба. Как скоро воды приблизились к храму, Архистратиг оградил их крестным знамением "--- и воды, вознесшись на высоту, остановились. Михаил обратился к камню, лежавшему близ алтаря, и ударил крестообразно жезлом своим. Вдруг загрохотал гром, земля потряслась, камень расселся, и открылась бездна. «Войдите в тесноту эту», "--- воскликнул Архистратиг, и воды с шумом устремились в расщелину камня. Эллины, смотря на чудо, окаменели от ужаса, а воевода бесплотных Сил вознесся на небо.

Христиане, радуясь победе над врагами веры, установили праздновать этот день. Место, где совершилось чудо, нарекли \textit{Хони}, то есть \textit{погружение}; ибо воды погрязли в камень, а реки Ликопалер и Куфос с того времени имеют свое течение в эту расщелину.

\section{Бог приемлет молитвы, не смотря на лица\footnote{Из Пролога, в 8"~й день сентября.}}

Некогда на острове Кипре было столь долговременное бездождие, что все травы и плоды иссохли, и всей стране угрожал ужасный голод. Епископ собрал народ и принес торжественное моление Господу Богу, чтобы Он даровал благорастворение воздуху. Посреди самого пения он услышал глас Небесный: «Иди к таким"=то вратам града и кого прежде всех увидишь входящего, того и проси, чтоб помолился. Он будет услышан на Небеси».

Епископ после утреннего славословия вышел с причтом церковным и сел у врат градских. Вдруг входит какой"=то старец, родом эфиоп, неся на плечах бремя дров, чтобы продать их в городе. Епископ, восстав с места, удержал его. Старец, сложив с себя ношу и поклонившись до земли святителю, просил у него благословения, но епископ сам поклонился ему и сказал: «Старче! Прошу тебя именем Божиим, помолись Господу, чтобы даровал милость людям Своим и оросил дождем лицо земли». Удивленный дровосек не знал, что отвечать епископу; но, слыша то же от всего причта, сказал: «Я грешник, как молиться мне за людей?» Но святитель настаивал, и, лишь только старец, принуждаемый к тому именем Господним, преклонил колена и воздел руки к небу, вдруг стали находить облака, и зашумел дождь.

По окончании молитвы епископ сказал старцу: «Ради любви к ближним и для общей пользы, скажи нам, как препровождаешь жизнь свою, чтобы и мы, взяв тебя в пример, могли жить так же». "--- «Прости меня, Господи! "--- отвечал смиренный старец. "--- Я грешный человек; родился на то, чтобы жить в суетах: каждый день выхожу ночью из города и, нарубив ношу дров, продаю "--- этим только трудом снискиваю себе насущный хлеб. Не имею сродников, не имею дома и сплю у церкви. Когда случится холод или дождь, день или два не ем. Время, оставшееся от трудов моих, препровождаю в молитвах».

Епископ и весь причт церковный, получив пользу для души своей, прославили Бога. «Ты доказал на себе истину Писания, "--- сказали они старцу, "--- которое говорит: \textit{Аз пришлец есмь на земли}». Потом епископ взял его к себе и, дав спокойную келью, содержал, пока не отошел он в жизнь вечную.

Этот старец да послужит для нас уроком, чтобы не судить о добродетелях человека по его положению или наружности.

\section{Святая Пульхерия избирает невесту царю, своему брату\footnote{Из Пролога, в 10"~й день сентября. Благоверная царица Греческая Пульхерия (†~453). Память ее празднуется 10~(23) сентября.}}

Когда греческому императору Феодосию Второму исполнилось двадцать лет, то святая Пульхерия, как старшая сестра и соправительница, почла нужным избрать ему супругу, достойную царского сана. Но так как богатство и знатность рода не считала она достоинствами, то и не могла долгое время через супружество составить счастья царю, своему брату.

Наконец непредвиденный случай удовлетворил желание сердца ее. В Царьград приехала девица, по имени Афинаида, дочь афинского философа Леонтия, прекрасная, разумная и кроткая. Причина ее приезда в столицу была довольно примечательна: отец Афинаиды, умирая, разделил все наследство сыновьям, ничего не отказав своей дочери, кроме нескольких златниц; и, когда родственники спрашивали его, зачем без всего оставляет дочь свою, он сказал только: «Довольно для нее красоты и разума» "--- и с этим словом скончался. Афинаидины братья были столь жестокосерды, что ни в чем не хотели нарушить духовную отца своего. Растерзанная горестью, девица уважала память родителя, но не могла простить столь непонятного равнодушия братьям; она решилась искать правосудия у престола и принесла жалобу святой Пульхерии. Царевна, увидев красоту Афинаиды, ее разум и невинность сердца, вознамерилась эту сирую девицу, лишенную части в своем достоянии, сделать участницей трона, и поскольку Афинаида была язычница, то она просветила ее крещением, сама была матерью при купели и вскоре сочетала с царем, своим братом.

Как заблуждаются люди в важнейшем случае жизни "--- при выборе супруги! Один ищет богатств, другой "--- красоты, третий "--- благородства. Но сколь редкие имеют в виду разум и сердце! Возьмем в пример святую Пульхерию и не будем суетны в выборе подруги, которая должна решить счастье или несчастье наше на всю жизнь!

\section{Святая Пульхерия уличает в беспечности царя, своего брата}

Феодосий Второй был царь благочестивый, добродушный и кроткий, но довольно нерадивый. Он имел дурное обыкновение подписывать дела не читая и даже иногда не спрашивая, какого содержания подносимый ему доклад. От этого происходило, что близкие к царю вельможи делали все, что хотели: грабеж в казне государственной, возвышение недостойных, страдание невинных, ссылки, казни "--- все это было делом обыкновенным. Злодейства наносили вред государству и стыд государю; но Феодосий ни о чем не думал.

Святая Пульхерия сокрушалась сердцем, видя, что царь, ее брат и воспитанник, предан столь глубокому сну. Не однажды покушалась она напомнить ему о долге самодержца, но советы не вдруг могут иметь свое действие. Пульхерия вознамерилась вразумить его самим опытом, сколь пагубна беспечность и доверчивость царская, и поднесла ему бумагу, в которой всеподданнейше просила, чтобы супругу свою Евдокию он отдал ей в рабство. Царь, не спросив о содержании доклада, подписывает его и возвращает сестре своей.

После этого она приглашает к себе Евдокию, принимает как сестру и царицу и, занимаясь разговорами о разных предметах, удерживает у себя весь день. Вечером за Евдокией приходят и зовут ее к царю; Пульхерия не отпустила ее. Приходят в другой раз "--- Пульхерия с улыбкой сказала: «Царь не имеет права над своей супругой, ибо отдал ее мне в услужение и это подписал своею царской рукой». Отозвавшись таким образом, она сама идет к Феодосию и, показывая ему грамоту за его подписанием, с кротостью выговаривает: «Видишь ли теперь, сколь пагубно не читая подписывать доклады, которые подносят тебе? Я не хочу думать, чтобы так поступал ты от беспечности и нерадения. Нет; ты столь равнодушен, что и других почитаешь неспособными обманывать тебя и делать зло. Как бы то ни было, опомнись, государь, и будь осторожнее… Ты "--- отец отечества!»

Феодосий, будучи поражен столь чувствительным примером своей беспечности, устыдился и с того времени начал рассматривать все, что предлагали ему.

\section{Обновление храма Воскресения Христова, находящегося в Иерусалиме\footnote{13"~й (26"~й) день сентября, память обновления (освящения) храма Воскресения Христова в Иерусалиме (Воскресение словущее).}}

По разорению Иерусалима римляне распространились по всему царству Иудейскому; так как они были язычники, то, естественно, должны были умножаться в Иерусалиме капища идольские. Более прочих императоров римских занимался устройством капищ Елий Адриан. Презирая иудеев и ненавидя христиан, он решился изгладить из памяти человеческой все места, которые те и другие почитали святыми. Иерусалим во имя его был назван Елией; в Вифлееме, где родился Спаситель мира, поставлен кумир Адониса; а на Голгофе, которую оросила кровь Сына Божия, сооружен языческий храм в честь богини Венеры. Это оскорбление святым местам продолжалось до Константина Великого, который, стараясь распространить и утвердить веру Евангельскую, принял святое намерение очистить от идолопоклонства страну богошественную. Мать его, равноапостольная Елена, сама прибыла в отечество пророков, разорила капища и столько была обрадована Небом, что обрела Крест и Гроб Господень. На этом"=то месте она воздвигла храм Воскресения Господня, который освящен был, уже после ее смерти, Великим Константином.

Этот праздник назван \textit{Обновлением храма Воскресения Господня}. Потому, что на том месте был мысленный храм, который Иисус Христос, по словам Евангелия, создал в три дня и который, будучи осквернен жертвами идольскими, в это время обновлен Константином и Еленой.

Да обновляется сердце наше, яко нерукотворный храм Бога живого, столь часто оскверняемый грехами! Да обновляется, отлагая ветхого человека и облекаясь в нового!

\section{Воздвижение Честного и Животворящего Креста Господня\footnote{14"~й (27"~й) день сентября.}}

Когда святая равноапостольная царица Елена разрушила храм, в честь порока на лобном месте сооруженный, и, раскопав основание оного, обрела три креста, тогда все были в недоумении, на котором из них распят был Спаситель мира и на которых "--- разбойники, одесную и ошуюю\footnote{\textit{Одесную и ошуюю} "--- по правую и по левую руку.} Его повешенные. В это время случилось нести мертвеца на погребение. Патриарх Макарий остановил шествие и повелел каждый крест, один после другого, прикладывать к бездыханному телу. Два креста не имели действия; но, как скоро Животворящее Древо Господне было поднесено, мертвец восстал из гроба.

Святая царица в небесном восторге поверглась на колена пред Крестом Спасителя и стократ облобызала его. То же сделал весь государственный синклит. Но так как, по причине бесчисленного стечения и тесноты народа, большая часть христиан не могли не только коснуться, но и видеть столь драгоценного предмета и все громогласно просили, чтобы хоть издали воззреть на оный, то святейший патриарх, взойдя на некоторое возвышение, воздвиг или поднял вверх Древо, произрастившее нам жизнь, и показал народу, который воскликнул: «Господи, помилуй!»

С того времени установлен праздник \textit{Воздвижения Животворящего Креста Господня}; и, с этим действием сообразуясь, Греко"=Российская Церковь священный обряд продолжает и поныне.

\section{О сомнении в делах веры\footnote{Из Пролога, в 18"~й день сентября.}}

Один десятилетний юноша за литургией услышал от старца, проповедующего Слово Божие, что милостыню, которую даем нищему, даем не ему, но Самому Христу. Молодой человек усомнился в истине учения старческого и, возвращаясь домой, думал: «Я знаю, что Христос на небесах, одесную Отца, как же может Он принимать, например, ломоть хлеба, который даю нищему?» Размышляя таким образом, он увидел нищего в разодранной одежде, просящего что"=нибудь ради Христа подать ему; а над главой его увидел обличие Христа Спасителя. Юноша испугался… Вдруг из ближайшего дома кто"=то вынес бедному старцу хлеб, и, как скоро протянул руку, чтобы подать оный, Спаситель простер десницу, чтобы принять его, и после того подавшего милостыню назнаменовал благословением.

Тогда"=то безрассудный юноша поверил Слову Божию и с того времени сделался первым в Царьграде благодетелем нищих.

\textit{Не испытуй, но веруй} (ср. Ин. 14, 1), сказал Спаситель; каждое Евангельское слово столь свято, что скорее прейдет небо и земля, нежели изгладится едина черта от Небесного учения.

\section{Евстафий Плакида\footnote{Великомученик Евстафий Плакида. Память его празднуется 20~сентября (3~октября).}}

Плакида, славный римский полководец, живший в царствование Траяна, будучи призван Богом на путь истины, провождал дни свои в подвигах богоугодных. Но Промысл Божий восхотел искусить его, как человека новопросвященного, злосчастием и страданием: болезнь и смерть водворились в доме его, и в короткое время не стало у него ни рабов, ни стад. Сверх того, злые люди расхитили сокровища его, и Евстафий (христианское имя Плакиды) из богача сделался нищим.

Трудно в таком случае не поколебаться сердцу, недавно озаренному Евангелием, но Евстафий был тверд. Он утешал супругу свою, чтобы не скорбела, она, со своей стороны, утешала Евстафия, оба говорили друг другу: «\textit{Господь даде, Господь и отъят: буди имя Господне благословено}» (Иов. 1, 21).

Но, чтобы более пред Богом доказать веру свою, они вознамерились удалиться от мира. Для этого ночью, от всех тайно, облеклись в рубища и, взяв с собой двух малолетних сынов, некоторое время скрывались в окрестных селах; наконец, будучи близ Средиземного моря, решили отплыть в Египет, где наиболее процветала тогда Церковь Христова. Они наняли корабль, но, к несчастью, хозяин корабля был морской разбойник.

Как только пристали они к берегам Африки, варвар, высадив Евстафия с детьми с корабля своего, объявил, что супруга его остается у него рабой. Супруг обмер, дети зарыдали; Евстафий напомнил о праве своем, но его не хотели и слышать. Супруг и дети бросились к ногам варвара… но тщетно! Злодей обнажил меч свой и закричал страшным голосом: «Избирай одно: или иди куда хочешь с детьми своими, или море будет вам гробом».

Плачевное зрелище! Супруг и дети, стоя на берегу, простирали к ней свои объятия; она с корабля простирала трепещущие руки к супругу и детям. Там и здесь жалобы прерываемы были рыданием. Наконец хищник с добычей скрылся из глаз… Для Евстафия и для юных чад день переменился на ночь… Но праведник всегда возлагает надежду свою на Бога.

Промысл хотел еще раз испытать душу праведника. Вскоре им встретилась река, не очень глубокая, но быстрая. Не имея челнока и нигде не видя моста, Евстафий решился перейти оную. А так как обоих детей сразу перенести было невозможно, то он, посадив одного на берегу, другого взял на руки и пошел на ту сторону. Перенеся его, посадил на траву и пошел за первым. Но, едва достиг середины реки, вдруг услышал крик его, воззрел и увидел, что старшего сына схватил лев и побежал в пустыню… Евстафий обмер, бросился за ним; но зверь в мгновение ока скрылся. Горько рыдая, Евстафий пошел назад к младшему сыну, но вдруг подбежал волк, схватил его и понес в близлежащие кустарники. Злосчастный отец почти лишился ума. Стоя посреди реки, он не знал, что делать: идти ли далее или кончить бедственную жизнь свою во гробе этой же реки?… Но Бог был подпорой его, и праведник не возроптал на Промысл Небесный, но с кротостью подвергся неисповедимым судьбам Его. Изливая пред Богом молитву сокрушенного сердца, он пошел дальше, достиг некоторой веси, называемой Вадизис, и, нанявшись у жителей стеречь их житницы, питался от трудов своих, жил в нищете и упражнялся в молитвах.

Всякое несчастие, ниспосылаемое свыше, есть не что иное, как испытание в подвигах добродетели, предохраняющее от преступлений, которые сделал бы человек, возгордившись в счастье своем, "--- или наказание за зло, которое сделал. В первом случае мы должны благодарить Бога милосердого, во втором случае мы должны благодарить Бога правосудного.

\section{Христианин"=патриот}

Занимаясь трудами, о которых прежде и вообразить не мог, Евстафий Плакида провел в селе Вадизисе более пятнадцати лет. Часто сокрушался он о супруге своей, часто оплакивал детей своих, но всегда в минуту горести находил утешение в Боге, содержащем жизнь и смерть в деснице, и не поступал по образу не имеющих упования. В Риме удивлялись, что сделалось с Евстафием. Везде искали его… наконец забыли.

Между тем случилось нашествие иноплеменников. Траян послал против них многочисленное войско, искусных военачальников, но все были разбиты, несколько городов разорено, а граждане уведены в плен. При столь неблагоприятных обстоятельствах император вспомнил о Евстафии и сказал при всех вельможах: «Если бы в живых был Плакида, то дела шли бы, без сомнения, лучше».

Известно, что всякое слово государя принимается с особливым вниманием. Немедленно два чиновника, Антиох и Акакий, вызвались пред Траяном, что они, может быть, успеют сыскать его, если на то будет воля царская. «Это лучшее желание моего сердца», "--- отвечал император и, наградив их щедро, отпустил в путь.

Эти царедворцы прежде были друзья и сверстники Евстафию летами и службой и всегда принимали живейшее участие в судьбе его. Руководимые, без сомнения, Небесным Промыслом, они направили путь свой к той стороне, где находился друг их, по дороге везде спрашивали: нет ли у них странника по имени Плакиды? Описывали его лицо, возраст, поступь и прочее. Наконец, достигли селения Вадизис "--- и хотели пройти мимо.

В то самое время Плакида был в поле, занимаясь обыкновенным трудом своим. Увидев римских воинов, он узнал в них прежних друзей и от радости заплакал. Желая ближе увидеть их, он стал при пути. Между тем римляне, приблизившись к нему, спросили: не знает ли он Плакиды? «На что он вам?» "--- сказал Евстафий. «Мы друзья его», "--- отвечали воины. «Нет, я не знаю его, "--- сказал Евстафий, "--- и даже никогда не слыхал о нем… но, государи мои! Прошу вас зайти в селение и отдохнуть у меня от столь трудного пути, а сверх того, будем иметь время разведать о том, кого ищете». Они согласились на просьбу его, и Евстафий привел их к одному зажиточному крестьянину, прося угостить их, будто в его доме, и обещаясь за все издержки заплатить ему трудами рук своих.

Евстафий сердечно радовался, что мог узнать от них, что делается в Риме, ибо сердце его, возожженное любовью к Богу, не охладело для пользы во славу отечества. Евстафий служил им и угощал их, часто вырывались из его очей слезы, но, не желая изменить тайне, он выходил из храмины и, отерев глаза, опять возвращался к ним. Антиох и Акакий, глядя на лицо его и приметив слезы, начали мало"=помалу узнавать его и говорить между собой тихо: «Этот человек подобен Плакиде или сам Плакида». Они вспомнили, что у него на шее была рана, полученная им некогда в сражении, начали искать ее и увидели… Воспрянули из"=за стола, бросились ему на шею и зарыдали от радости. «Ты Плакида, "--- воскликнули они, "--- которого мы столь долго ищем! Ты "--- любимец цезарев! Ты "--- вождь римский, о котором царь и народ так долго сетуют!» Тогда объявили ему причину своего путешествия, злосчастие Рима и гордость врагов. Ревность к прежним подвигам запылала в сердце Евстафия, сыновняя горячность к родной стране в нем обновилась. «Так! Я тот Плакида, "--- сквозь слезы сказал он, "--- с которым вы столько раз были победителями, я тот, который некогда был в Риме славен, но, ах! Я ныне слаб и презрен… однако клянусь Богом Искупителем, что всегда готов пролить кровь мою за отечество, если правда, что я для пользы его нужен». С обеих сторон недоставало слов все выспросить, на все отвечать.

Евстафий облекся в одежды военачальнические, препоясал меч по бедру своему. Все жители были в изумлении, видя столь нечаянную перемену. Хозяин дома повергся к стопам его и просил извинения, что по незнанию держал его рабом в доме своем.

Истинный христианин, наследник Неба, есть тот, кто воздает \textit{Божия Богови} и \textit{кесарева кесареви} (Мф. 22, 21).

\section{Награда благочестивому}

Окончив со славой брань против врагов римских, Евстафий Плакида с победоносным воинством возвращался в отечество. В один жаркий день, проходя мимо какого"=то селения, стоявшего над рекой и имевшего красивое местоположение, Плакида остановился, и так как воинство его от дальнего похода утомилось, то и решился он остаться тут на три дня и велел раскинуть шатры для себя и для воинов.

Между тем два молодых воина поставили для себя одну палатку подле некоторого сада. Расположившись на траве, под тенью свесившихся через ограду дерев, они разговорились друг с другом, и один спросил другого, кто у него отец. «Не знаю, "--- отвечал он, "--- я помню только, что отец мой был известный человек в Риме и Бог знает, для чего вдруг взял мою мать и меня с меньшим братом, пришел к морю и пустился по водам. Потом, когда пристали к берегу, отец мой пошел с нами далее, а где девалась мать моя "--- не знаю. Помню только, что отец мой и мы весьма плакали о ней» (здесь товарищ его смутился, и покатились у него слезы). «Потом, "--- продолжал он, "--- мы дошли до какой"=то реки, меня посадил отец на берег, а меньшего брата понес через нее на плечах. Вдруг откуда ни взялся лев и подхватил меня, но пастухи бросились на него, отняли меня и воспитали в деревне своей». "--- «Ах, брат мой! Любезнейший брат! "--- воскликнул другой юноша и бросился к нему на шею. "--- Я помню это! Я видел, как зверь похитил тебя… увы! И меня тогда же похитил волк и унес в леса, но также пастухи меня отняли у него и у себя воспитали». Они обнимали друг друга, целовались и от радости плакали.

Это были Агапий и Феопист, дети Евстафия Плакиды! Но Промысл Небесный готовил им вящую радость: мать их находилась в том самом саду и, слыша их разговор, возводила с воздыханием и слезами на небо очи свои. Она хотела бежать и заключить их в объятия, но, видя себя в рубищах, остановилась… Притом, зная ветреность и гордость вообще молодых воинов, она опасалась, чтобы они не отреклись от нее и тем бы не причинили жесточайшего удара сердцу ее. Итак, она решилась подготовить их к этому свиданию мало"=помалу.

В этом намерении она пошла к военачальнику, чтобы выпросить у него позволение отправиться в войске его в Рим. Но, представ пред Евстафием, как изумилась, узнав в нем того, о ком столько лет сокрушалась! Она стояла долго, как бы в забвении, не могла вымолвить ни слова, хотела повергнуться пред ним, но, видя стоящих вокруг него вождей и оруженосцев, усомнилась, точно ли это супруг ее. И как он опять сделался вельможей? Наконец, объяснила ему просьбу свою и получила желаемое.

Но сердце ее исполнилось нетерпения узнать тайну эту… Итак, спустя несколько часов она вторично пришла к Евстафию и, заставши его одного, сказала ему: «Не разгневайся на меня, милостивый государь, если я спрошу тебя о некоторой важной для меня вещи». "--- «Говори обо всем, добрая старушка», "--- отвечал Плакида. «Не ты ли Плакида, нареченный при Святом Крещении Евстафием? "--- спросила она. "--- Не у тебя ли варвар отнял жену?» Евстафий изумился. «Что касается до меня, "--- продолжала она, "--- я осмеливаюсь назвать себя твоей супругой, я Божеским милосердием сохранена от насилия, ибо варвар в тот же день погиб, и мне дана свобода». Тогда Евстафий как бы пробудился от сна и, узнав супругу свою, воскликнул: «Благодарю Тебя, Господи Боже мой!» Оба радовались и плакали…

Потом, когда утолились их рыдания и первые восторги, она спросила у Евстафия: «Где же дети наши?» Военачальник вздохнул из глубины сердца, и опять слезы ручьем полились из глаз его. «Увы! "--- уныло сказал он. "--- Их пожрали лютые звери». "--- «Не отчаивайся, "--- отвечала она, "--- как нашел супругу твою, так же найдешь и детей твоих». Тогда она рассказала ему, что слышала в саду от двух юношей.

Немедленно Евстафий посылает за молодыми воинами, идет туда сам, чтобы скорее обнять детей своих… Стократно лобызает их и потом уже спрашивает, кто они… Удивленные юноши рассказывают историю своей жизни и узнают, что они в объятиях отца своего… Идут с ним в шатер его и видят себя в объятиях матери… Кто может описать восхищение отца и супруга, матери и супруги, детей и братьев?..

Надежда на Промысл Божий никогда не обманет благочестивого. Поэтому не ропщите на святую волю Его, но с кротостью подвергайтесь неисповедимым судьбам Его…

\section{Непоколебимость в вере\footnote{Из Пролога, в 20"~й день сентября. Память мучеников и исповедников Михаила, князя Черниговского, и болярина его Феодора, чудотворцев (†~1245).}}

Когда татарский хан Батый опустошил Россию, а избежавшие убийства и плена христиане скитались по горам и в непроходимых пустынях, тогда жилища человеческие сделались жилищами зверей, а вертепы зверей превратились в жилища человеческие.

Киевский и Черниговский князь Михаил, славный благочестием и царскими добродетелями, в это лютое время жил в земле Угорской с боярином своим Феодором. Он сердечно сокрушался о страждущих единоверцах своих и соотечественниках, но делать было нечего, как только ожидать, \textit{дондеже мимоидет гнев Господень} (Ис. 26, 20).

Наконец Михаил услышал, что татарский хан, наложив дань на Русскую землю, оставил ее в покое. Русские князья возвращались из бегства и, делая коленопреклонения перед царем злочестивым, принимали от него княжества. На такой же поступок решился и Михаил, но не по честолюбию своему, ибо какая честь быть князем и вместе с тем рабом? Но единственно для того, чтобы своей мудростью хоть сколько"=нибудь облегчить бремя всеобщего рабства. В этом намерении, взяв с собой боярина Феодора, он прибыл в Чернигов. Сердце князя облилось кровью, когда он увидел повсюду опустошение и плач народа.

Приняв благословение от отца духовного и получив от него благочестивое напутствие, Михаил и боярин Феодор отправились в Золотую Орду; но там готовился им венец мученический.

Едва явились они в престольный улус Батыя, к ним пришли жрецы и волхвы татарские и объявили именем великого царя, что они должны перейти сквозь священный огонь, близ дворца по обеим сторонам пути разложенный, и, таким образом очистившись от скверны христианской, предстать царю Батыю. Князь и боярин ужаснулись и сказали решительно, что они этого сделать не могут. Услышали об их отказе бывшие на тот раз в Орде русские князья и, почитая оный необдуманным, пришли с поспешностью к своему сроднику и слезно просили его не подвигать на гнев хана и, по крайней мере, для вида сделать поклонение солнцу, в их огне обожаемому. Но они ушли от князя с тем же ответом.

Батый, привыкший видеть одно только повиновение, с угрозами требовал, чтобы данник сделал все, что ему приказано. Он послал объявить волю свою одного из царедворцев, Елдегу, но все ласки и прещения были напрасны. Михаил отвечал ему спокойно: «Скажи хану, что ему, как царю, я поклоняюсь и воздаю достодолжную честь, ибо десница Всевышнего покорила меня власти его, но отнюдь не хочу оскорбить Бога исполнением обрядов языческих». Батый освирепел и повелел мучительски умертвить Михаила.

Как волки на кроткую лань, устремились на князя исполнители смертной казни, но святой мученик пребывал бесстрашен. Стоя на одном месте, он молился Спасителю, в то время как все члены его были раздробляемы от ударов; наконец, глава была отторгнута от выи. Святой Феодор был свидетелем смерти его и в тот же день, подобно ему, получил венец мученический.

Одна вера укрепляет человека в бедствиях, одна она может подвигнуть человека к перенесению страданий и не убояться самой смерти.

\section{Кто не имеет добрых дел, для того и молитва других за него бесполезна\footnote{Из Пролога, в 24"~й день сентября.}}

Настоятель одного монастыря, человек благочестивый, к нищим милостивый и украшенный всеми добродетелями, день и ночь изливал Богу следующую молитву: «Господи! Я "--- грешник, но надеюсь на Твои щедроты, надеюсь спастись по милости Твоей. Молюсь Тебе, Владыка! Не разлучи меня от моей братии и в будущем веке; удостой их вместе со мною жизни райской!»

Долго молился чадолюбивый отец старцев, и Господь открыл ему волю Свою следующим образом: в другой обители, неподалеку от них, наступал праздник, настоятель зван был туда вместе с братией. Он не хотел выходить из своего монастыря, но во сне услышал глас Небесный: «Иди, только пошли прежде себя учеников твоих, а сам последуй за ними».

Когда настал день праздника, братия пошли на обед и на пути увидели нищего, израненного, едва переводящего дыхание. На вопрос их: «Чем болен ты?» "--- несчастный едва мог сказать голосом полумертвым: «Я шел по делу, но здесь напал на меня зверь, всего истерзал и ушел "--- некому взять меня и отнести в селение». "--- «Мы пешии, "--- отвечали старцы, "--- не можем ничего сделать для тебя», "--- с этим словом они пошли далее.

Вскоре на то же место пришел настоятель и увидел стенающего страдальца. Узнав, что случилось с ним, спросил он: «Не проходили ли мимо тебя незадолго пред этим иноки? И видели ли тебя?» "--- «Они стояли надо мною, "--- отвечал старец, "--- узнали мое несчастие и пошли далее, отозвавшись, что тяжело меня нести на себе». "--- «Можешь ли ты, "--- спросил у него огорченный поступком старцев настоятель, "--- тихонько идти за мною?» "--- «Не могу!» "--- отвечал нищий. «Так садись же на мои плечи, "--- сказал настоятель. "--- С Божией помощью я донесу тебя…» Сколько расслабленный старец ни отрицался от того, сколько ни говорил, что путь далек, а он тяжел, сколько ни просил, чтобы он из селения кого"=нибудь послал за ним, но благочестивый настоятель, невзирая ни на что, взял его на руки и понес. Он чувствовал сначала обыкновенную тяжесть человека, а потом легче, легче и, наконец, как будто ничего не имел на себе. Настоятель не знал, что думать об этом, как вдруг не стало у него странника, и свыше он услышал глас: «Ты всегда молишься о твоих учениках, чтобы вошли вместе с тобой в Царствие Небесное, между тем их дела другие, нежели твои. Итак, понудь их, чтобы ходили по стопам твоим, иначе они не достигнут Царствия Божия. Я праведен и воздаю каждому по делам его».

\section{Смирение преподобного Сергия, игумена Радонежского\footnote{В 25"~й день сентября. Преставление прп. Сергия, игумена Радонежского, чудотворца (†~1392).}}

Один сельский житель, наслышавшись о добродетелях и чудотворениях святого Сергия, издалека пришел в обитель его, чтобы увидеть праведника и принять от него благословение. В то время человек Божий возделывал землю в саду своем. Будучи извещен о пришельце, он велел допустить его к себе.

Крестьянин, видя человека, слишком просто одетого, копающего гряды, не мог вообразить, чтобы это был столь славный Сергий, и подумал, что в насмешку ему вместо святого указали на работника. Он возвратился в монастырь и вторично спросил: «Где преподобный Сергий?» Сколько ни уверяли его, что, будучи в саду, он видел его, но крестьянин не хотел и слышать.

Между тем угодник Божий вышел из сада. Крестьянин, гнушаясь им, отворотился в сторону и не хотел взглянуть на него. «Ах! Какой труд подъял я "--- и понапрасну! "--- думал он. "--- Я пришел для великого пророка, надеялся узреть славу его, но что же вижу? Нищего!» Святой узнал его мысли и возблагодарил Бога: ибо как горделивец восхищается хвалой и честью, так смиренномудрый радуется уничижению. Он взял крестьянина за руки, увел в свою келью и поставил перед ним пищу и питье. «Не печалься, добрый человек, "--- сказал он, "--- ты вскоре увидишь Сергия».

Едва святой выговорил это, вдруг объявили ему, что в монастырь прибыл великий князь; Сергий вышел к нему навстречу… Как удивился крестьянин, увидев, что государь к этому, как называл он, нищему приступил с благоговением и, поклонившись до земли, просил благословения! Как удивился он, что этот старец сел вместе с князем, между тем как все прочие, даже вельможи, стояли пред ними!

Бедного крестьянина гнушались и слуги их… Едва выпросив для себя уголок в сенях, он не спускал глаз с того, на кого прежде и взглянуть не хотел. «Сделайте милость, скажите мне, "--- тихонько спросил он, "--- кто этот старец, подле князя сидящий?» "--- «Это Сергий», "--- отвечали ему. Обманутый самим собой, крестьянин заплакал и укорял себя в невежестве. «Конечно, слеп я, "--- говорил он, "--- что не мог узнать человека Божия и не отдал ему подобающей чести… Увы! Как теперь предстану лицу его?» Но Сергий вскоре утешил его.

Когда князь простился с праведником, крестьянин припал к ногам его и раскаивался в безумии своем. «Ты один прав, "--- поднимая его, сказал святой муж, "--- что почел меня старцем, ничего не значащим; все прочие обманываются».

Уважай человека в худом платье так же, как богато одетого, дабы не подумали, что отдаешь почтение не человеку, но платью его. Сверх того, под златотканой одеждой часто скрывается невежество и злое сердце, а под рубищем "--- великий ум и добродетель.

\section{Ревность о спасении ближнего\footnote{Из Пролога, в 26"~й день сентября. Преставление апостола и евангелиста Иоанна Богослова (начало II~в.).}}

Святой Иоанн Богослов, будучи в Эфесе, взял к себе юношу, по"=видимому добродушного и подававшего большие надежды. Возлюбив его душевно, он захотел воспитать его. Вскоре Иоанну должно было отлучиться в дальние места для проповеди Слова Божия, почему он, отходя из Эфеса, поручил молодого человека епископу с тем, чтобы он как можно этого отрока соблюдал от всякого зла. Епископ охотно согласился, хранил юность его и наставлял на путь добродетели. Через некоторое время он крестил его и, думая, что через это таинство совершенно утвердил его в вере, прервал душеполезные наставления.

Но сколь пагубно молодому человеку отдавать на волю поведение его! Бедный юноша познакомился с некоторыми сверстниками, не так благонравными, начал с ними ходить по ночным беседам, вскоре пристрастился к пьянству, к разорительным играм и другим порокам. А как на это нужны были деньги, то развратник сделался хищником и наконец дошел до такого жестокосердия, что одна разбойническая шайка избрала его над собой атаманом. С тех пор не только грабеж, но и убийство для него было шуткой. Когда святой Иоанн возвратился, то, увидевшись с епископом, в присутствии всех спросил его: «Где юноша? Приведи его сюда…» Епископ вздохнул из глубины сердца, облился слезами и едва мог сказать ему: «Он умер». «Как? "--- возразил святой Иоанн. "--- Душевной смертью или телесною?» "--- «Душевной, "--- отвечал епископ, "--- он теперь разбойник». "--- «Двое несчастных! "--- воскликнул святой Иоанн. "--- Беспечный воспитатель и нерадивый питомец! Горе ему! Но что сделал ты? Не тебя ли я, отходя отсюда, поставил хранителем души отроческой? Ах! Дай мне коня».

Святой Иоанн долго ездил по лесам и горам, ища блудного сына, и наконец был взят разъездом разбойническим. Вместо того чтобы стараться избавить себя от злодеев, он просил их, чтобы представили его своему атаману.

Затрепетал злочестивый юноша, увидев святого Иоанна, и побежал от него, но Богослов, забыв старость свою, погнался за злодеем. «Почто бежишь от меня, о чадо мое? "--- вопиял он. "--- Почто причиняешь мне столь несносный труд, о сын мой? Остановись, сжалься над дряхлым старцем. Остановись и ничего не бойся: не исчезла надежда спасения… Я отвечаю за тебя пред Богом! Пусть на мне будет кровь, которую ты пролил». Наконец, против воли убежденный, юноша остановился, бросил свое оружие, зарыдал и, закрыв руками лицо свое, устремился к нему. Несчастный трепетал весь, как под взмахом смертоносного железа.

Богослов взял его с собой в город и привел в церковь, подавая всем образ покаяния, чтоб, низвергшись в прегрешения, никто бы не отчаивался получить спасение, ибо Господь хочет всем нам \textit{спастися и в разум истины приити} (1~Тим. 2, 4).

Родители! Учите страху Божию детей своих в младенчестве, учите их в отрочестве, учите их и в зрелом возрасте, ибо человек слаб и всегда подвержен заблуждению.

\section{Казнь богоотступнику\footnote{См. страдание великомученика Артемия, в 20"~й день октября (2~ноября).}}

Император Юлиан, из христианина сделавшись идолопоклонником, для православных чад Церкви был жесток не менее, чем языческие мучители. Господь наказал его следующим образом.

После разных тиранств за веру Христову умертвив святого Артемия, Юлиан с многочисленным воинством пошел в Персию, чтобы завоевать или, по крайней мере, обессилить оную. Уже царь и воинство достигли города Ктезифона, как вдруг предстал пред ним некий старец, с виду благородный и весьма умный, и обещал Юлиану показать ближайшую и лучшую дорогу к самому сердцу Персии, "--- чему обрадовался он до безумия.

Сначала под руководством старца шли они по пути хорошему, но скоро увидели себя в ужаснейшей пустыне. При легком ветре песок поднимался и клубился, наподобие моря; солнце палило их; голод и жажда каждый день похищали множество жертв; вскоре погибло более половины воинства… Старец был доволен сам собой и, будучи спрошен императором, для чего завел их в непроходимое место, сказал с неустрашимостью: «Для того чтобы не видеть опустошенным от врага мое любезное отечество». После этого он с радостью претерпел ужаснейшую смерть.

Блуждая с остатком воинства, Юлиан, наконец, встретил персидские полчища и по необходимости вступил в сражение. Греки и римляне были разбиты, и стрела, неизвестно откуда пущенная, пронзила сердце богоотступника. Заскрежетав зубами, он захватил горсть крови своей вместе с пылью и, бросив вверх, воскликнул: «Победил, галилеянин! Насыться…» "--- и, как злодей, испустил дух свой.

Здесь два человека представляют два примера различные, но разительные… Всевышний Промысл долго терпел злочестие Юлиана, но Он правосуден: смотрите, как наказал богоотступника! Ужасайтесь даже в одной черте нарушить святейший Его закон. Другой человек "--- персиянин: он претерпел лютейшую смерть для того, чтобы спасти отечество. Подражайте примеру его и жертвуйте всем, даже кровью, для блага своего отечества.

\section{Разум младенца\footnote{В 24"~й день октября.}}

Когда Омиритский князь Дунаан, стараясь всех христиан, своих подданных, обратить в иудейство, после жесточайших мучений повелел умертвить святого Арефу и с ним множество других, тогда одна христианка, оплакивая смерть праведных, исполнилась Божественного дерзновения и, держа на руках сына своего, пятилетнего младенца, подошла к исполнителям смертной казни и громко укоряла князя"~мучителя, моля Бога, чтобы Он наказал его, как древле фараона.

Уличенная в христианском исповедании и вместе с тем в оскорблении князя, она была схвачена, представлена Дунаану и, подтвердив пред ним то же, осуждена на сожжение. Младенец, облившись слезами, бросился к ногам князя и молил его освободить злосчастную мать, а так как он был собой прекрасен и благоразумен, то Дунаан взял его к себе на колени и спросил: «Кого любишь более: меня или мать?» "--- «Без сомнения, матушку, "--- отвечал младенец, "--- я для нее пришел к тебе и с ней лучше хочу умереть, нежели без нее жить».

\textit{Дунаан}. Останься у меня, друг мой! Я дам тебе самых лучших плодов.

\textit{Младенец}. Сохрани меня, Боже! Я лучше пойду на мучения вместе с матушкой, если не могу спасти ее.

\textit{Дунаан}. Да знаешь ли ты, что есть мучение?

\textit{Младенец}. Умереть за Христа, чтобы жить с Ним в будущем веке.

\textit{Дунаан}. А кто Христос твой?

\textit{Младенец}. Пойди за мной в церковь, и я Его покажу тебе… "--- Взглянув на мать свою, опять заплакал. "--- Отпусти меня, я пойду к матушке.

Князь удивился разуму младенца, но был столь жестокосерд, что не внял мольбам его… Между тем мать его связали и повергли в огонь. Младенец, увидев это, крепко укусил Дунаана "--- одно оружие, которым мог отмстить мучителю!

Дунаан оттолкнул его от себя и велел одному из предстоявших вельмож увести его домой и воспитать в законе еврейском. Но младенец успел вырвать у него руку и бросился прямо в огонь… Там, обняв мать свою, сгорел вместе с ней и оправдал пророчество царя Давида: \textit{Из уст младенец и ссущих совершили еси хвалу, враг твоих ради, еже разрушити врага и местника} (Пс. 8, 3).

Дети! Возьмите этого разумного младенца в пример любви к родителям и будьте всегда их утешением.

\section{О том, сколь грешно осуждать других\footnote{Из Пролога, в 25"~й день октября.}}

Однажды Иоанн Савватийский сидел в пустыне и размышлял о делах богоугодных. Вдруг приходит к нему из одной обители старец, чтобы посетить праведника и принять от него благословение. «Как живут твои собратья?» "--- спросил у него Иоанн. «Хорошо, молитвами твоими», "--- отвечал инок. «А как живет такой"=то черноризец?» "--- опять спросил его угодник Божий об одном иноке, о котором носилась худая слава. «Он нисколько не переменился», "--- сказал посетитель. «Горе ему!» "--- воскликнул Иоанн и с этим словом объят был каким"=то чудесным сном: он видит себя стоящим пред Голгофой, видит Иисуса Христа между двумя разбойниками. Иоанн устремился поклониться Ходатаю мира. Но, едва успел приблизиться, вдруг Иисус обратился к предстоящим Ангелам и сказал им: «Изриньте его вон! Он осудил своего брата прежде Моего суда». Когда Иоанн, будучи изгоняем, бежал из дверей, задержалась его мантия так крепко, что вынужден он был ее оставить. С этим пробудился он и, будучи объят трепетом, с глубоким вздохом сказал посетителю: «Ужасен для меня день этот!» "--- «Отчего так?» "--- спросил старец. Тогда святой Иоанн, рассказав ему свой сон, присовокупил: «Оставленная мантия значит то, что я за осуждение брата лишился Божеского покровительства и благодати».

С того времени святой Иоанн семь лет молился Богу в пустыне, не вкушал хлеба, не входил в келью, не говорил с людьми. Наконец, опять в чудотворном сновидении узрел он, что Господь отдал ему мантию, и через это познал, что отпущен ему тяжкий грех осуждения.

\textit{Не осуждайте, да не осуждены будете} (Мф. 7, 1), сказал Спаситель. Важны слова эти: ибо кто осуждает ближнего, тот как бы хочет отнять право у Судии Небесного.

\section{О послушании\footnote{Из Пролога, в 25"~й день октября.}}

Однажды к преподобному Исидору пришел ученик одного великого старца, чтобы спросить его о каком"=то важном для иночества деле. В то время как они занимались разговором, настала ночь и поднялась буря, молния и гром были ужасны. Несмотря на это, ученик собрался идти к своему старцу. Святой Исидор и прочие иноки приглашали его ночевать у них; представляли, что ночь ненастна, а ему должно переходить через реку, которая хотя неглубока, но валы плещут весьма сильно. Однако пришлец не согласился ни на что. «Мне велел настоятель мой, "--- сказал он, "--- непременно возвратиться сегодня: как можно его ослушаться?» "--- и простился с ними.

Преподобный Исидор и прочие иноки пошли провожать его. Когда достигли реки, ученик скинул с себя одежды, навязал их на голову и пошел на ту сторону. Иноки, смотря на него, трепетали при каждой волне, которая на него упадала. Но ученик перешел бурную реку благополучно и, облекшись в одежды, поклонился им и пошел к своему старцу. Преподобный Исидор и все иноки, благословляя его, удивлялись, сколь велика сила каждой добродетели!

\section{Ревность к проповеди\footnote{В 13"~й день октября.}}

Хотя Греция озарена была светом Евангелия от времен проповеди святых апостолов, несмотря на это, над некоторыми местами расстилалась тьма язычества, и народ не мог видеть Бога, Искупителя мира. Между прочими была одна обширная и многолюдная весь, в которой жители все до одного были идолопоклонниками, приверженными к своему заблуждению столь крепко, что никто не мог обратить их к Богу. Сколько тамошние епископы ни посылали проповедников слова Господня, все возвращались от них без успеха. Наконец, содействием Святого Духа пал жребий на преподобного Авраамия Затворника\footnote{Прп. Авраамий Затворник (†~ок. 360). Память его празднуется 29~октября (11~ноября).}.

Долго он отказывался от возлагаемой на него должности, почитая себя недостойным столь великого служения, но, убежденный истиной, что лучше спасти многих, нежели себя одного, вышел из своей кельи и, будучи рукоположен в священника, отправился в путь свой. Сколь глубоко восстенал святой муж, увидев народ, пред идолами колена преклоняющий! «Единый без греха Боже! "--- воздвигши очи на небо, воскликнул он. "--- Не презри дела рук Твоих!» С помощью имения, оставшегося от родителей своих, которым на пользу нищих распоряжался один из сродников, начал он созидать храм Божий, вседневно ходил на место строения и среди стоявших там идолов молился Богу, не произнося ни единого слова. Народ на созидаемую церковь и на святого человека смотрел равнодушно и довольствовался насмешками, иногда ругательствами.

Когда храм Господень был отстроен и освящен, преподобный Авраамий принес в нем слезную молитву к Богу: «Собери, Господи, люди расточенные и введи их в церковь эту, просвети умные очи их, да познают Тебя, Единого Истинного Бога». После этого он с церковнослужителями вышел из храма, устремился на капище и сокрушил поставленных там идолов.

Народ пришел в бешенство. Все мужи и жены восстали на святого человека, как лютые звери; били немилосердно и изгнали из селения. Но Авраамий при наступлении ночи туда возвратился, вошел в церковь и с плачем и рыданием молился Богу Искупителю, да спасет народ погибший.

Едва настал следующий день, некоторые из жителей пришли в храм Господень "--- не для молитвы, но из любопытства "--- и увидели Авраамия. Преподобный умолял их, чтобы обратились к Богу. Но сердца их были так ожесточены, что не могли принять Духа Святого; варвары устремились на него с дреколием, повергли замертво на землю и, вытащив на поле, закидали каменьями. Там праведник, едва переводя дыхание, ночью опомнился и, восстав, начал плакать горько. «Господи Боже мой! "--- восклицал он. "--- Призри на раба Твоего и укрепи меня на подвиг Твой». С этими словами, опять возвратившись в селение, он вошел в церковь и опять на другой день был мучим и терзаем; потом, как мертвый, извержен был в близлежащий лес.

Эта ревность праведника, эти страдания его продолжались три года. Наконец мужество святого человека преодолело препятствия, святая вера восторжествовала.

В один день жители, собравшись вместе, начали разговор о святом Авраамии и удивлялись его великодушию и терпению. «Видите ли его беспримерную к вам любовь? "--- сказали мудрейшие из старцев. "--- Сколько мы оскорбляли его, сколько мучили, но он даже не сказал никому худого слова; поистине, он послан к нам от Бога». "--- «Мы то же думаем, "--- отвечали почти все в один голос, "--- но боялись обнаружить свои мысли». "--- «Когда так, "--- продолжали старцы, "--- то зачем же медлить? Пойдем к нему и объявим себя верующими проповедуемому им Богу». Все приняли совет старцев: единодушно вошли в церковь и воскликнули: «Слава Царю Небесному, пославшему к нам мужа святого!»

Сердце праведника взыграло радостью небесной. «Отцы мои! Братия мои! Чада мои! "--- воскликнул он. "--- Приидите, дадим славу Богу, просветившему сердечные очи в познание истины… Веруйте в Бога, Творца неба и земли, веруйте в Его Единородного Сына, Агнца, вземшего грех мира. Веруйте в Пресвятого Духа Его, все оживотворяющего; веруйте "--- и получите жизнь небесную, вечную».

С того времени святой муж неусыпно старался предуготовить их к возрождению водой и Духом; учил их всему, что касается до Царствия Небесного, "--- что есть вера, надежда и любовь. Наконец усыновил их Богу.

\section{Одна истинная вера непоколебима в злостраданиях}

Когда священник Иосиф и диакон Аифал\footnote{Память Иосифа пресвитера и Аифала диакона (†~IV в.) празднуется 3~(16) ноября.} непоколебимо стояли против ужаснейших мучений, которые в Персии свирепствовали, тогда один христианин, манихейского раскола, всенародно исповедал свою веру и укорил персов за их нечестие.

Тотчас схватили его и на глазах у Аифала начали мучить… Сначала манихей казался великодушным и неустрашимым, но, когда мучители испытали над ним жесточайшие казни, он громко воскликнул: «Отвергаю веру отцов моих; поклоняюсь богам царя моего». И мгновенно прекратились мучения лжестрадальца.

Видев это, святой Аифал исполнился радости духовной, что Бог удивляет славу Свою на православных только чадах Церкви, и, соболезнуя о злочестивом отступнике, начал укорять его: «Заблудший человек! Ты едва почувствовал мучения, уже отвергся от Бога твоего… Вот доказательство, что вера ваша "--- не Православие, но буйный, богопротивный раскол! Благословен Христос, Бог наш, укрепляющий нас в лютейших мучениях и дающий непоколебимость в благочестивой вере нашей!»

Судья, услышав укоризны праведника, так возъярился на него, что повелел мучить его дотоле, пока не пресеклось в нем дыхание.

\section{О пользе поминовения усопших\footnote{В 9"~й день ноября.}}

Один юноша с острова Кипра был пленен, отведен в Персию и там заключен в темницу. Родители сначала не знали, где находится сын их, потом услышали, что он умер, и начали по душе его делать поминовения в Рождество Христово, в Святую Пасху и в день Святой Троицы.

Через четыре года юноша убежал из плена и пришел к родителям. Восхищенные отец и мать после первых объятий и слез сказали ему: «Мы услышали, что ты, любезный сын, умер, и молились об упокоении души твоей». "--- «В какой месяц и в какой день это делали вы?» "--- спросил юноша и, когда услышал от них время молитв, с удивлением воскликнул: «Ах, любезные родители! Вы не знали, сколько чрез то делали мне добра. Ибо в каждый из этих дней приходил ко мне некто в белых ризах, снимал с меня оковы и выпускал из темницы, так что этого никто видеть не мог; а на другой день я опять оказывался в темнице и в оковах».

Христиане! Молитесь об усопших ваших сродниках "--- молитесь о каждом усопшем христианине, ибо через это душа его получает утешение.

\section{Поступок Иоанна Милостивого с племянником своим\footnote{Память свт. Иоанна Милостивого, патриарха Александрийского (†~620), празднуется 12~(25) ноября.}}

Георгий, племянник святого Иоанна Милостивого, будучи оскорблен и обруган одним простолюдином, пришел к дяде своему и слезно жаловался на столь тяжкую обиду. Человек Божий, видя Георгия в сильной досаде и огорчении, показался тронутым и разгневанным. «Как дерзнул человек низкого происхождения, "--- воскликнул он, "--- обесчестить племянника патриаршего! Бог свидетель, что отомщу оскорбителю; я поступлю с ним так, что дивиться будет вся Александрия». Молодой человек принял слова дяди за истину и был чрезвычайно рад.

Но вдруг Иоанн переменил голос свой и, с нежностью соединяя выговор, сказал ему: «Послушай, Георгий! Если хочешь называться моим племянником, то будь готов терпеть не только досады, но и язвы; для Бога все прощай ближнему. Если хочешь казаться благородным, то ищи благородства от добродетели, а не от крови, ибо не предки, но жизнь богоугодная украшает нас; сын или родственник какого"=нибудь знатного человека, не имея добродетелей, есть то же, что сын богача, сделавшийся нищим». Увещевая Георгия таким образом, праведник тронул его до слез, которые не значили уже, как прежде, гнев и мстительность, но были следствием стыда и раскаяния.

Но Иоанн тем не удовольствовался. Он призвал строителя церковного и приказал с простолюдина, который обесчестил Георгия, не брать церковной дани. Таким"=то образом святой муж поступил с оскорбителем по своему обещанию, и, без сомнения, этому удивилась вся Александрия.

Всякое оскорбление почитай благодеянием, ибо оно смиряет душу, поползновенную к гордости, и тем приближает ее к Богу.

\section{Напоминание о смерти}

Святой Иоанн Милостивый был истинный сын Неба, непоколебимый поборник Православия, пример всех добродетелей; но он был человек… Страшась, чтобы какие"=нибудь суетные мысли иногда не коснулись его сердца, он всегда имел при себе твердый щит "--- память смертную, которая есть непререкаемое доказательство человеку, что он не более как горсть пыли. На этот случай святой муж велел приготовить себе гроб, до половины только устроенный, и художникам приказал, чтобы всякий праздник приходили к нему и пред собравшимися посетителями вслух сказывали: «Гроб твой, владыка, еще не доделан, прикажи окончить его; ибо смерть приходит, как тать, "--- не знаешь, в который час». Таким образом, Иоанн Милостивый и сам всегда имел смерть пред очами, и другим в разительном виде показывал оную.

О, если бы мы как можно чаще мыслями своими ложились на одр смерти! Если бы чаще воображением своим представляли то ужасное мгновение, когда настоящее примет конец и наступит будущее! Тогда бы мы смотрели на все вещи другими глазами. Мы ужасались бы тех дел, в которых теперь себя нимало не укоряем; мы бы находили страшные пропасти на тех путях, где теперь ничего не видим, кроме равнины и цветов.

\section{Благое употребление подарка}

Однажды к Иоанну Милостивому пришел принять благословение богатый вельможа и увидел его постель, покрытую худым и уже полуизношенным одеялом. Возвратившись домой, он послал ему новое, которое стоило тридцать шесть златниц.

Человек Божий не хотел оскорбить усердия вельможи и в следующую ночь уснул под этим драгоценным покрывалом, но, пробудившись, мгновенно раскаялся в том. «Горе тебе, грешному, "--- воскликнул он, "--- что спишь под нежной тканью, а братья Христовы, нищие, цепенеют от холода. Ах, сколько есть убогих, которые не имеют и лепты, чтоб прикрыть наготу свою! Сколько есть странников и пришельцев, которые спят на стогнах под открытым небом! А ты имеешь пышные храмины, толпу рабов, разную пищу, всякое питье "--- и еще вздумал более укрепить и разнежить сон твой под этим мягким шелком… Бойся, чтобы, вкусив все наслаждения в жизни твоей, не получить одну горесть в будущем веке. Нет! Не покрывайся этим пышным одеялом в другую ночь! Пусть ценой его покроются несколько бедных».

В таких размышлениях святой Иоанн, лишь только дождался утра, немедленно послал подаренное ему одеяло на торжище для продажи. Случай привел туда и вельможу… Он увидел свое покрывало, купил его и вторично отослал святому Иоанну, усердно прося его, чтоб тот пользовался им сам. Человек Божий принял, поблагодарил и опять велел продать оное. Вельможа опять купил его и возвратил Иоанну. Святой в третий раз сделал то же, и вельможа сделал то же в третий раз. Наконец, праведник, удивляясь его щедрости, послал к нему следующий отзыв: «Увидим, кто из нас прежде наскучит: я ли, решившись продавать, или ты, решившись покупать и всегда мне возвращать?» Сам между тем он продолжал благочестивую торговлю.

Таким образом, Иоанн приверженность к себе вельможи обратил в пользу страждущих. И против воли сделал его благотворителем.

Не украшай тела твоего одеждой слишком нежной и великолепной. Христианское одеяние есть одеяние души, ее должно украшать, а не тело: ибо красота души есть образ Божий, по которому человек создан; напротив того, чему научает нас роскошь тела?.. Похищать чужое и расточать свое, обижать ближнего и свою руку удерживать от милостыни. Она ненасытна: ей все надобно. Нехорош у нее дом: надобно соорудить новый. Худо у нее платье: надобно сшить лучшее. Не по вкусу кушанье, не по вкусу вина: надобно купить усладительнейшие. Посудите же, есть ли время, есть ли возможность подумать о страждущем человечестве?

\section{Поступок Иоанна Златоустого с Евтропием\footnote{Память свт. Иоанна Златоустого, архиепископа Константинопольского (†~407), празднуется 13~(26) ноября.}}

В Греции был закон, необходимо нужный для первых времен христианства, по которому всякий преступник почитаем был неприкосновенным для правосудия, если успеет скрыться в церкви. Узаконение это, без сомнения, было издано с той целью, чтобы пред народом тем более проповедать и прославить благодать Бога Искупителя: ибо злодей, в смертный час нашедший защиту у подножия престола Господня, почти всегда выходил оттуда человеком добрым на всю жизнь. Так глубоко запечатлевалось в душе его чувство всемогущества и милосердия Божия!.. Несмотря на это, некоторые из сильных мира имели намерение этот закон уничтожить. Таков был Евтропий, вельможа при царе Аркадии. Он уговорил царя издать указ, чтобы никто не мог иметь убежища в церкви, когда преследует его правосудие, а если кто и успеет скрыться, то будет извлечен оттуда и, как преступник царской воли, будет строго наказан. Сколько Иоанн Златоуст, бывший тогда патриархом Царьградским, ни старался отвратить это оскорбление от храма Господня, но через то возбудил на себя гнев и ненависть Евтропия и часто терпел от него обиды.

Но \textit{в нюже меру мерите, возмерится вам} (Мф. 7, 2), сказал Спаситель. Народ, будучи не в силах стерпеть притеснения и жестокости от Евтропия, начал роптать вслух: неотступно просил наказать, в пример всем, судью неправедного, и Аркадий, который часто возводил на степень величия людей недостойных и никогда не умел поддерживать своих любимцев, изрек на Евтропия смертный приговор.

Злосчастный бежит в церковь Господню и скрывается в алтаре… Это случилось в то самое время, как святой Златоуст проповедовал тут слово Господне. Как поступил в этом случае святитель Божий? Изгнал ли злосчастного из церкви как разорителя правил ее? Нет, он только представил народу, раздраженному на Евтропия, сколь непостоянно человеческое счастье, и, изображая жалостную судьбу любимца царского, тогда на смерть осужденного, старался проникнуть жалостью к нему сердце народа.

Вот пример поучительный! Другой на месте Иоанна возрадовался бы, увидев так глубоко низверженным врага высоковыйного, и к народному гласу присоединил бы свой глас. Но праведник первый дал убежище и защиту жесточайшему своему гонителю. Он видел, что Евтропий довольно наказан и тем, когда мог найти спасение своей жизни только там, куда возбранял убежище от казни другим, столько же несчастным, но, может быть, менее преступным.

\section{Иоанн Златоустый}

Иоанн, архиепископ Царьградский, будучи пресвитером, проповедовал пред многочисленным народом Слово Божие. Вдруг какая"=то женщина, не уразумев высоких мыслей и сильных выражений, воздвигла глас из"=за народа: «Учитель духовный! Или, лучше сказать, Иоанн Златоустый! Глубокомысленно твое учение, а ум наш бессилен и не может постигнуть твоих спасительных слов». Услышав это, весь народ обратился на ту сторону, откуда услышал голос, и наконец все вообще воскликнули: «Неизвестная женщина произнесла имя это: но без сомнения Сам Бог нарек тебя Златоустым!» С того времени начали называть святого Иоанна Златоустым, что продолжается и доныне.

Между тем великий проповедник и сам согласился, что хитро сплетенное слово, будучи действительно для людей просвещенных, не принесет пользы простому народу, и начал учить людей Христовых просто и вразумительно.

\section{О неисповедимости судеб Господних\footnote{Из Пролога, в 21"~й день ноября.}}

Некий отшельник, кроткий, благочестивый, добродетельный, день и ночь молился Богу, чтобы просветил разум его познать судьбы Промысла Небесного; но Господь не внимал его молитве. Пустынник, почитая себя грешником, недостойным принять откровение свыше, решился идти к ветхому старцу, который жил довольно далеко, и от него узнать то, что постоянно занимало его ум. Взяв с собой пищу и питье, он отправился в путь.

Вдруг встретился с ним черноризец и спросил: «Куда идешь, раб Христов?» "--- «К такому"=то старцу», "--- отвечал пустынник. «Я иду туда же», "--- сказал черноризец. Спутники, обрадовавшись друг другу, пошли вместе.

Когда настал вечер, странники остановились у одного богатого человека, который принял их с сердечной радостью и угощал ужином из серебряной посуды… Как скоро вышли они из"=за стола, черноризец взял одно блюдо, вышел из дома и бросил в близтекущую реку. Хозяин не сказал ни слова, а пустынник не знал, что думать о том.

Поутру странники продолжали путь свой далее и на другой вечер пришли ночевать к другому странноприимцу, который оказал им всевозможное почтение. Что же сделал черноризец? При наступлении утра, когда должно было благодарить за гостеприимство, хозяин привел к ним единородного сына и просил ему благословения. Вдруг черноризец схватил его за гортань и задушил. Пустынник ужаснулся, хотел закричать, но голос его пресекся. А отец сказал только: «Буди воля Господня!»

Таким образом, пешеходы пошли далее, но на третий вечер не нашли никого, кто бы их принял и упокоил: почему и вошли в один ветхий, опустевший дом и тут препроводили ночь. Утром, когда должно было идти, черноризец начал разрушать дом и, разломав до основания, начал созидать снова. Пустынник вышел из терпения и воскликнул: «Заклинаю тебя, скажи мне: Ангел ли ты или дьявол? Дела твои для меня непостижимы!» "--- «А что сделал я?» "--- возразил черноризец. «Третьего дня, "--- отвечал ему пустынник, "--- ты утопил серебряное блюдо у добродетельного странноприимца, вчера задушил сына у того, кто так ласково угостил нас, а сегодня без всякой причины разломал дом и начал строить снова, не зная сам для кого». "--- «Не дивись этому, добродушный старец, "--- сказал ему спутник, "--- и не соблазняйся в рассуждении моих поступков, но послушай, что я скажу тебе… Первый страннолюбец "--- человек совершенно богоугодный, но блюдо, которое мной утоплено, он приобрел неправдой. Знай же, что я, невзирая на его ласковость, из благодарности за гостеприимство истребил оное, чтобы через эту ничтожную вещь добрый человек не потерял награды на Небесах. Другой странноприимец также богобоязлив, но малолетний сын его, достигнув совершенного возраста, сделался бы злодеем, посрамлением человечества, и тем отцу своему нанес бы стыд и смертоносное сокрушение. Сверх того, несчастный отец тогда бы должен был дать за сына своего ответ на Суде Страшном. Итак, знай, что я, благодаря за добродетель отца, умертвил сына». "--- «Но здесь, в пустом месте, "--- возразил старец, "--- какую имеешь причину разрушать дом и снова сооружать оный?» "--- «Не соблазняйся и в этом случае, "--- отвечал спутник. "--- Господин этого дома был хищник и человекоубийца, но он обнищал и оставил дом свой; дед его, строя дом, скрыл в стене золото: знай же, что дом разрыт для того, чтоб кто"=нибудь, ища клада, не погубил души своей. Итак, возвратись, добродушный старец, "--- присовокупил он, "--- в твою келью и впредь не принимай на себя безрассудного труда испытывать судьбы Господни: Сам Святой Дух говорит, что они "--- \textit{глубина непостижима и недоведома для человека}. Не старайся понапрасну узнать их, ибо нет в том никакой пользы». Сказав это, черноризец мгновенно стал невидим. Старец ужаснулся. Он узнал, что этот пустынник был Ангел, посланный к нему от Бога для того, чтоб дать ему спасительный урок; покаялся в неразумии своем и дал обет Господу никогда не испытывать судеб Его.

\section{Поступок Амфилохия, епископа Ионийского\footnote{Память свт. Амфилохия, епископа Иконийского (†~после 394), празднуется 23~ноября (6~декабря).}, против Феодосия, царя греческого}

Святой Амфилохий, с мужеством и неусыпностью подвизавшийся против богохульника Евномия и духоборца Македония, не мог ничем удержать раскола Ариева, который разливался, наподобие стремительного потопа, по всей Греции. Добропобедный воин Христов прибегнул к помощи императора и просил, чтобы он царской властью изгнал из всех городов Греции скопища арианские. Но император, видя повсюду множество еретиков и, может быть, опасаясь от них бунта, просьбу святого оставил без внимания.

Через некоторое время Амфилохий имел случай быть у царя. Он поклонился Феодосию, сидевшему на престоле, и как верноподданный приветствовал его с глубоким почтением; между тем будто не видел сына его, Аркадия, незадолго перед тем нареченного Августом и сидевшего близ царя, отца своего. Приметив это невнимание к соцарствующему сыну, Феодосий разгневался и, не в силах стерпеть бесчестия его, повелел святого Амфилохия как неучтивца удалить от лица своего. «О, государь! "--- воскликнул тогда праведник. "--- Теперь собственным опытом ты узнал, сколь трудно стерпеть бесчестие сыновнее. Ты гневаешься на меня: точно так же Бог Отец не может стерпеть бесчестия Сына Своего. Он отвращается и ненавидит хулящих Его и жестокую готовит казнь тем, которые приобщаются проклятой ереси Ариевой». Тогда царь узнал, что неуважение к сыну его оказано святым Амфилохием с благонамерением и служит притчей, что Богу Сыну должно воздавать равную честь с Богом Отцом.

Удивляясь столь мудрому поступку, столь благочестивому уроку, он восстал с престола своего, преклонился перед ним и испросил прощение в оскорблении праведника.

Немедленно повсюду разосланы были царские предписания, чтобы гражданская власть неусыпно содействовала истреблению ереси арианской. Таким образом, ходатайством святого Амфилохия Церковь Христова получила покой и благоденствие.

\section{Обращение святой великомученицы Екатерины\footnote{Память великомученицы Екатерины (†~305~-- 313) празднуется 24~ноября (7~декабря).}}

Святая великомученица Екатерина была дочь Консты, римского цесаря. После смерти отца ее скипетр царства перешел в руки Максима. Екатерина об этом, кажется, и не думала, ибо полностью посвятила себя любомудрию: читала Гомера, Платона, Демосфена и восхищалась только своей ученостью. Но мать ее заботилась о том, как бы, через супружество дочери с каким"=нибудь предприимчивым вельможей, опять увидеть род свой на престоле вселенной. Она беспрестанно советовала ей избрать жениха из благороднейших юношей, но Екатерина всегда старалась отклонить намерение матери, и родительница ее, сердечно сокрушаясь об этом, решилась призвать к себе в помощь других.

В безвестном месте города жил богоугодный старец, духовный отец ее матери, ибо она была христианка. Но или по какому"=то непонятному равнодушию к вере, или боясь, чтобы дочь ее, сделавшись христианкой, по юношеской ревности не обнаружилась и через то бы не пострадала, до времени скрывала от нее веру свою. К этому"=то старцу пришла она с Екатериной и, объявив ему о дочерней холодности к супружеству, просила уговорить упрямую девицу, чтоб послушалась ее совета, но все увещания старца, все убеждения матери для нее были напрасны. Екатерина сказала решительно: «Если хотите видеть меня в супружестве, то найдите мне юношу, который бы не уступал мне ни в благородстве, ни в учености, ни в богатстве, ни в красоте; если же из этих совершенств хотя бы в одном будет иметь недостаток, то вечно останусь девицей». Услышав это, мать отчаялась исполнить желание своего сердца; а старец, восхищенный ее благоразумием и привязанностью к девству, подумал сам себе: «Сколь приятна будет жертва эта Христу Спасителю» "--- и решился обручить ее Жениху Небесному.

Несколько подумав, он сказал Екатерине: «Я знаю Чудного Юношу, Который без сравнения лучше тебя всеми дарованиями: красота Его превосходит сияние солнечное; мудрость Его управляет всем чувственным и духовным; сокровища Его изливаются на весь мир; благородство Его выше всех царей». Вообразив, что старец говорит о каком"=нибудь самодержце, Екатерина смутилась, переменилась в лице и спросила с некоторым сомнением: «Правда ли это?»

\textit{Старец}. Представь себе глубокую тьму и лучезарное солнце; я клянусь Богом, что мое описание Юноши есть та глубокая тьма, а Его совершенство есть то лучезарное солнце. Это такой Жених, Который на тебя и взглянуть не восхощет.

\textit{Екатерина}. Чей же сын этот столько прославляемый тобой Юноша?

\textit{Старец}. Он не имеет отца на земле, но родился сверх естества от Пречистой и Пресвятой Девы, наитием Духа Святого.

\textit{Екатерина}. Конечно, ты говоришь о каком"=нибудь боге или полубоге? Но ты ведь знаешь, что они все имеют по нескольку жен.

\textit{Старец}. О нет! Этот Жених одним дуновением может обратить в прах всех ваших богов и полубогов.

\textit{Екатерина}. Не понимаю! Но можно ли мне увидеть столь Чудного Юношу?

\textit{Старец}. Если сделаешь все, что предпишу тебе, то удостоишься увидеть лицо Его.

\textit{Екатерина}. Божусь, я на все готова.

Тогда старец дал ей образ Пресвятой Богородицы, держащей в объятиях Предвечного Младенца, и сказал: «Се Дева и Матерь, и се Ее Сын!

Этот образ унеси с собой и, заключившись в чертоге, с благоговением молись всю ночь, чтобы Она показала тебе Сына Своего. Если будешь просить с истинной верой, то надеюсь, что увидишь Того, к Кому страстно привержено сердце твое».

Мудрая девица приемлет неоцененный дар и спешит в чертог свой. Там по наставлению старца она молилась дотоле, пока, утрудившись от столь необыкновенного для нее подвига, не заснула крепким сном. Вдруг явилась перед ней Матерь Божия в таком точно виде, как была изображена на картине: в объятиях Ее был Предвечный Младенец. Он от лица Своего испускал лучи, светлейшие солнца, но взор Свой обращал всегда к Матери, так что Екатерина не могла видеть оного. Она переходит на другую сторону, но Младенец снова отвращает лицо Свое…

Между тем как борьба с обеих сторон продолжалась несколько раз, Царица Ангелов сказала Сыну Своему: «Воззри, Чадо мое, на рабу Твою, сколь она разумна, сколь благородна, сколь богата и сколь прекрасна!» "--- «Она так безумна, "--- отвечал Святейший Младенец, "--- так низка, так бедна, так безобразна, что не могу и воззреть на нее». "--- «По крайней мере, "--- возразила Матерь Божия, "--- научи ее, что должно сделать, чтобы удостоиться видеть лицо Твое». "--- «Пусть идет к старцу, "--- сказал Иисус, "--- и, что повелит он, сделает в ту же минуту, тогда увидит Меня и обрящет благодать». Встревоженная видением, Екатерина проснулась.

Сколь долог показался ей остаток ночи! Едва явилась на востоке заря утренняя, эта девица, взяв с собой верную рабыню, пошла к благочестивому старцу и просила его, чтобы научил ее, как заслужить любовь Жениха Небесного. Тогда святой муж подробно растолковал ей таинства христианской веры, блаженство души, приверженной к Богу, неизреченную славу райской жизни и ужас будущих мучений. Екатерина как девица мудрая и просвещенная все поняла, все уразумела; уверовала от всего сердца и приняла от него Святое Крещение. Старец дал ей наставление, с каким умилением должно молиться Богородительнице, чтобы насладиться тем же видением.

Совлекши с себя ветхого человека и облекшись в нового, мудрая отроковица молилась до глубокой ночи. Вся душа ее, все сердце были углублены в священные думы. О едином помышляла "--- чтобы увидеть Жениха Небесного. Утрудившись бдением и молитвой, она опять заснула, и прежде бывшее видение открылось пред нею. Уже Небесный Младенец смотрит на нее веселыми очами. Божия Матерь вопросила Сына Своего: «Благоугодна ли Тебе девица эта?» "--- «Теперь она для Меня столь же благородна, славна, богата и прекрасна, "--- отвечал Спаситель, "--- сколько прежде была безумна, бесславна, бедна и безобразна, и Я столько ее возлюбил ныне, сколько прежде ненавидел. Так, мудрая девица, "--- обратившись к Екатерине, сказал Сын Божий. "--- Я избираю тебя Моей невестой и в залог Моей к тебе любви вручаю этот перстень. Сохрани его и не приемли другого жениха». Святая Екатерина, исполнившись радости и восхищения, поверглась пред Ним на колени… «Я недостойна быть Твоей невестой! Пусть буду Твоей рабой!» "--- воскликнула она и от сильного потрясения проснулась… Бесценный перстень был на руке ее.

Человеческие намерения суетны, иногда гибельны; намерения Божеские всегда клонятся к тому, чтоб нам обратиться к Богу и в разум истины прийти. Следственно, мы при всяком предприятии должны слушаться не честолюбия своего или другой какой"=нибудь страсти, но Единого Бога, Который столь ясно в Священном Писании открыл нам волю Свою.

\section{Святой Петр, архиепископ Александрийский, предает сам себя на смерть, чтобы предупредить бунт\footnote{Память священномученика Петра, архиепископа Александрийского (†~311), празднуется 25~ноября (8~декабря).}}

Когда угодник Божий Петр за проповедь Господню заключен был в темницу и приговорен к смерти, тогда все христиане, жившие в Александрии и окрестных местах, стеклись к невинному страдальцу и при вратах дома, где содержался святитель, проводили дни и ночи. Воевода, долженствовавший совершить казнь, сколько ни выжидал времени, сколько ни покушался разогнать силой или отвлечь хитростью народ, стерегущий своего пастыря, но всегда видел намерения свои тщетными. Ибо каждый раз, как только приближались воины, все вопияли, что скорее положат душу свою за пастыря и учителя, нежели оставят его. Воеводе должно было прибегнуть к средствам насильственным, а от этого мог произойти ужасный бунт.

Святой Петр видел в царском вожде решимость умертвить его, а в христианах решимость защищать до последней капли крови. Другой на его месте всего бы легче мог найти средство избегнуть смерти, но праведник думал не так. Он, с одной стороны, страшился всенародного смертоубийства, если останется долее под защитой усердных чад Церкви, с другой стороны, и в самом гонителе христианства уважал царя и свято повиновался воле его. Одушевляемый этими чувствами, он увещевал многократно народ свой, чтоб отдали его в руки закона отечественного, но всегда слышал один голос: «Человек Божий! Что будем без тебя мы, бедные, беззащитные христиане?» Святитель Христов, поручив Промыслу духовное стадо свое, принял твердое намерение умереть один, отвращая смерть от многих.

Он посылает одного из стражей к воеводе, тайно от народа, с таким предложением, чтобы он, если хочет совершить волю цареву, пришел ночью с другой стороны и, подкопавшись под стену, взял бы его и поступил как будет угодно… Военачальник удивился столь невероятной неустрашимости праведника и вместе обрадовался совету его, ибо надеялся через то удобным образом угодить царю своему.

Тогда было зимнее время. Ночь была темная, ветер шумел ужасно. Никто из духовных детей его не мог услышать, как сделали подкоп в темницу, вывели угодника Божия и отсекли главу его. Уже на другой день христиане узнали, что пастырь их и учитель принял венец мученический.

\section{Покаяние святого Иакова\footnote{Память великомученика Иакова Персянина (†~421) празднуется 27~ноября (10~декабря).}}

Святой Иаков был родом персианин. Рожденный отцом и матерью "--- христианами, воспитан он был в благочестии и вступил в супружество также с христианкой.

Облеченный в сан царедворца, он должен был сопровождать царя своего в некоторый путь, и так как встретились обстоятельства, благоприятные для государства, то персидский царь Истигерд принес богам своего отечества торжественную жертву. Иаков, как ближний к царю вельможа, не мог быть посторонним зрителем обряда без того, чтобы не обнаружить веры своей, и сделал коленопреклонение идолам.

Мать и супруга его, услышав о таком безбожном поступке, ужаснулись и немедленно отправили к нему письмо следующего содержания: «Злосчастный! Зачем ты в угождение человеку оставил Бога, Царя Небесного? Ублажая временную жизнь, погубил жизнь бессмертную? Истину переменил на ложь? Мы плачем и рыдаем, что ты, преклонив колена свои пред идолами, уподобился убийцам Христа Спасителя, которые, издеваясь над Божественным страдальцем, преклоняли пред Ним колена и били в то же время по ланитам. Заклинаем тебя, обратись к Ходатаю мира и тем избегни гнева и ярости Бога мстителя! Но если ты по упрямству или по робости не перестанешь приносить сердце твое в жертву идолам, то знай, недостойный сын и супруг, что мы не будем более для тебя тем, чем были прежде: мы не можем даже видеть тебя. Восстенаешь некогда, но будет поздно». Прочитав письмо супруги и матери, Иаков смутился, повергся в глубокую думу, и слезы из очей его покатились градом. «Увы, "--- с печалью сказал он. "--- Если мать моя, супруга моя так чуждаются меня, что же мне будет в будущий век, когда придет Бог судить живых и мертвых и воздаст каждому по делам его?» Он вторично прочитал письмо их и воскликнул: «Заблудший! Моли твоего Искупителя, чтобы простил страшный грех твой! Моли! Он милосерд и не хочет смерти грешника, но с радостью приемлет кающегося».

С того времени святой Иаков день и ночь плакал и рыдал, повергшись пред распятием Господним; и это раскаяние сокрушенного сердца продолжалось дотоле, пока некоторые из нечестивцев не увидели его. Они донесли об этом царю, гонителю веры Христовой, и праведник принял венец страдальческий.

Вот пример истинного покаяния! Святой Иаков, повинуясь матери и супруге, пренебрег благами мира и лучше захотел умереть, нежели обратиться опять к идолам. А мы ежедневно слышим глас Предтечи Господня: \textit{Покайтеся, приближи бо ся Царство Небесное} (Мф. 3, 2), "--- и всегда пропускаем мимо ушей, иногда умиляемся, но, выйдя из церкви, опять предаемся суетной и греховной жизни. Мы ежегодно исповедуем пред Богом грехи свои; но для чего делаем это? Едва ли по большей части не из одной привычки, а часто из одного только принуждения.

\section{Притча преподобного Варлаама о богатых и убогих\footnote{Из Пролога, в 28"~й день ноября.}}

Один славный римский царь шествовал на златозарной колеснице, окруженный оруженосцами и вельможами, и встретил двух старцев в грязных и разодранных одеждах. Зная, что это благочестивые постники, царь сошел с колесницы, поклонился и, обняв, облобызал их. Удивленные вельможи вознегодовали и говорили друг другу на ухо: «До какого унижения дошел царь наш!» Но, не смея обнаружить мыслей своих, просили царского брата сказать ему, что неприлично так бесчестить венец царский. Брат царский согласен был с мыслями вельмож и объяснился пред ним, но царь дал ему ответ, которого он, сколько ни старался, понять не мог.

У этого царя было следующее обыкновение: когда хотел он кого"=нибудь наказать смертью, то посылал вестника, который пред вратами преступника трубным гласом объявлял ему гибель. Сообразно с этим обыкновением в следующую ночь послал он трубу смертную к дому брата своего. Несчастный царевич затрепетал, услышав глас смерти, и всю ночь употребил на то, чтобы к ней приготовиться.

Поутру облекся он в печальные одежды, вместе с женой и детьми пришел в царский дворец и, став у дверей, плакал и рыдал. Услышав это, царь призвал его к себе и, видя смертное отчаяние брата, сказал ему: «Неразумный человек! Ты устрашился провозвестника, которого послал к тебе брат, тебе подоборожденный и не видевший от тебя никакого оскорбления: как же мог ты зазирать\footnote{\textit{Зазирать} "--- порицать, хулить, осуждать.} меня, когда я в смирении сердца целовал проповедников Бога моего, которые предвозвещали мне смерть громогласнее трубы? Я с тобой так поступил только для того, чтобы научить тебя. Иди спокойно в дом свой, а тех, которые о моем унижении перед нищими были одних с тобой мыслей, я научу другим образом».

После этого царь повелел сделать четыре ящика и два из них оковать золотом, а другие два обмазать смолой и пеплом. Когда все было готово, царь призвал тех вельмож, которые ставили ему в бесчестие унижение пред нищими, и, поставив пред ними четыре ящика, спросил, чего стоят два золотые и чего стоят два смоленые. Вельможи, как и все люди, обыкновенно судящие по наружности, думая, что в златокованых ящиках лежит утварь царская, объявили им самую высокую цену, а взглянув на ящики, замазанные смолой и пеплом, почли недостойными даже того, чтобы ценить их. «Я уверен был, "--- сказал им тогда царь, "--- что вы будете судить по внешности, но посмотрим, не обманулись ли вы?»

Тогда повелел он раскрыть золотые ящики, и из них распространилось зловоние. «Смотрите, что там лежит!» "--- сказал он. Вельможи взглянули и, увидев кости человеческие, тотчас с ужасом отступили назад. «Вот изображение тех, "--- сказал им царь, "--- которые, облекшись в драгоценные, блестящие ризы, гордятся славой, а в душе своей исполнены зловонных пороков и беззаконий». Потом велел открыть другие два ящика, замазанные смолой, и вдруг повеяло такое благоухание, что более не чувствовали смрада от костей мертвеца. «Посмотрите, что там лежит!» "--- опять сказал им царь. Вельможи увидели драгоценные камни, пересыпанные разного рода благоуханиями. «Вот изображение тех, "--- сказал царь, "--- которым я кланялся, которых целовал и за которых вы меня поносили только потому, что они были в худых рубищах. Но вы не знали, что душа их не потеряла ни одной черты от образа Божия». Дав столь поучительный урок своим вельможам, царь отпустил их.

Под драгоценным платьем часто скрывается порочное сердце, а под рубищем "--- чистая и добрая душа.

\section{Притча святого Ефрема о долготерпении Божием\footnote{Из Пролога, в 30"~й день ноября.}}

Один богатый человек купил обширное поле по другую сторону реки и, призвав своих рабов, сказал им: «Я даю всем вам по равной части земли, идите и возделайте оную, а я чрез некоторое время вас навещу и посмотрю на труды ваши». Усердные рабы, выслушав приказание господина, в тот же час переправились чрез реку и принялись за свое дело, а другие из них, непокорные, жестокосердые, начали представлять разные причины, почему не могут идти туда, и, наконец, совсем отказались.

Господин, вместо того чтобы наказать ослушников, сделал для них пир, упоил допьяна и в этом беспамятстве велел их перевезти за реку и оставить каждого на том месте, которое должно ему возделывать. Когда хмель вышел из головы их, то, пробуждаясь ото сна, некоторые удивились великодушию и долготерпению господина своего и с тех пор всемерно старались загладить поступок свой неутомимыми трудами. А другие, напротив, досадуя за насильственный с ними поступок, не хотели приняться ни за что и только спали, отчего на их земле выросло терние и крапива.

Через какое"=то время пришел к ним господин и, увидев неусыпность рабов послушных, благословил их. Потом, осмотрев труды тех, которые, будучи перевезены через реку против воли, принялись за свое дело, также благословил и их. Наконец, дошел он до рабов ленивых, нераскаянных и, увидев их в крепком сне, а поле в тернах и крапиве, с гневом воскликнул: «Лукавые рабы! Зачем оставили в пустоте часть винограда, которая назначена для ваших трудов? Или не помните, как я, простив ваше непокорство, напоил вас, спящих перевез через реку и оставил здесь? За мое благодеяние "--- не должно ли было вам сравниться с вашими сотрудниками?» Злобные рабы трепетали и не знали, что отвечать ему.

Наконец, господин, как праведный судья, воздал каждому по делам его: прилежных рабов наградил, а рабов ленивых предал смерти.

\section{Филарет Милостивый\footnote{Память праведного Филарета Милостивого (†~792) празднуется 1~(14) декабря.}}

Праведный Филарет был богат, благороден и всеми уважаем, а что более всего возвышало его, был добродетелен и настолько милостив, что ни разу за всю жизнь не отказывал, если бедный что"=нибудь просил у него.

Но Бог, все на пользу нашу устраивающий, восхотел испытать его твердость. Агаряне напали на его отечество и опустошили все, что им ни встретилось: людей и скот увели с собой, нивы потопили, сады выжгли. Этой ужасной участи более всех подвергся праведный Филарет и сделался беднейшим по всей Пафлагонии.

Имея супругу, детей и внучат, он начал заниматься хлебопашеством, чтобы пропитать себя и семейство. К несчастию, у одного бедного крестьянина издох чужой вол, и праведник для расплаты отдал ему последнего вола. Феозва, жена его, обливаясь слезами, горько укоряла его за расточительность. У одного воина издох конь, и праведник отдал ему последнего коня. Феозва осыпала его ругательствами. У одного бедного человека, который имел грудного младенца, пала корова; праведник отдал ему последнюю корову. Феозва прогнала его с глаз своих и с того времени ела с детьми своими, как будто не зная, что у нее есть муж.

Уже несчастное семейство питалось милостыней, но Филарет столько же был благодетелен. Если кто"=нибудь из прежних знакомцев посылал ему немного хлеба, он разделял на части жене, детям, внучатам, а что доставалось ему, еще разделял с нищими. Жена и дети называли его безжалостным, извергом, чадоубийцей, но святой муж всегда отвечал с кротостью: «У меня есть в одном тайном месте бесчисленные сокровища, их хватит на всю жизнь не только нам, но и внучатам нашим».

Феозва думала, что супруг ее говорит вне ума, но время вскоре оправдало слова его. Христолюбивая Ирина, царствовавшая тогда в Греции с сыном своим Константином, вознамерилась сыскать ему достойную невесту в пределах своего отечества. Посланные для этого несколько знаменитых и благородных мужей прибыли в Пафлагонию. Увидев дом праведного Филарета, отличавшийся от прочих обширностью и красотой, они избрали его себе для пристанища, хотя некоторые из граждан говорили им, что этот дом великолепен снаружи, а внутри пуст, но через это возбудили в них большее любопытство. Они отослали наперед от себя все, что нужно было к столу.

Праведный Филарет с сыновьями и внуками угощал их. Но как из всех его поступков обнаружилось какое"=то благородство, а молодые люди весьма благопристойно обходились с вельможами, то эти последние полюбопытствовали, имеет ли он супругу и дочерей. Филарет вывел к ним Феозву, увидев которую они удивились, что она, будучи при старости, обладает красотой. Что же касается дочерей его и внучек, они пришли в восхищение. «Благодарим Бога, что нашли желаемое!» "--- сказали они. Потом объявили праведному Филарету намерение, с которым они прибыли в его отечество, и Марию, старшую его внучку, поздравили невестой царя своего.

Праведный Филарет и все его семейство отправились в Царьград. Там Мария обручилась с царем Константином, другая сестра ее вступила в супружество с одним из первых вельмож, третья сделалась королевой Лонгобардской… Таким образом, на праведном Филарете сбылись слова царя Давида: \textit{Блажен человек, разумеваяй на нища и убога} (Пс. 40, 2).

Будь только добродетелен, все прочее оставь на попечение Богу, но при всем том никогда не забывай и того, чтобы надежда твоя не была запятнана сомнением.

Совершив торжество брачное, праведный Филарет сказал своей супруге: «Сделаем великолепный обед; я хочу угостить царя и знатнейших вельмож его». Феозва употребила все, чтобы сделать пиршество как можно роскошнее.

Когда все было готово, Филарет приказал сыну своему Иоанну, занимавшему тогда уже видное положение, и прочим внукам, чтобы столь знаменитым гостям при столе служили сами, а о себе сказал, что идет звать царя и вельмож.

Но как удивилось семейство его, увидев святого человека, возвращавшегося домой! Он был окружен нищими, слепыми, хромыми, дряхлыми, число которых простиралось до двухсот человек. «Это царь приближается с вельможами, "--- сказал он. "--- Все ли у вас готово?» Тогда"=то домашние его поняли, что праведник царем называл Самого Иисуса Христа, а вельможами "--- нищих, которых молитвы имеют у Него столь великую силу, и с кротостью исполняли все, что ни повелевал он.

Святой Филарет, угостив царя своего и вельмож, еще дал каждому из них по златнице.

Напитав жаждущего, мы напитаем Самого Христа; одев нагого, мы оденем Самого Христа; упокоив странника, мы упокоим Самого Христа; посетив страждущего в темнице, мы посетим Самого Христа. Христиане! Сколь великое утешение быть благодетелем бедных!

\section{Притча преподобного Даниила о делах человеческих\footnote{Из Пролога, в 4"~й день декабря.}}

Один инок, сидя в келье, услышал глас свыше: «Приди, я покажу тебе дела человеческие». Инок вышел, и явившийся Ангел указал ему на Мурина, который насек дров и, связав их, пытался поднять на плечи, но был не в силах. Вместо того чтобы убавить дров, он нарубил их еще более и, увеличив ношу, опять стал поднимать и опять не мог "--- и это повторял многократно.

Пройдя несколько поприщ, опять Ангел указал ему на другого человека, который, стоя у колодезя, черпал воду и лил в утлую кадь, из которой вода опять бежала в колодезь.

Удивляясь этому, инок пошел далее; вдруг увидел он церковь, и какие"=то два человека, сидя на конях и держа бревно поперек дверей, силились внести его прямо в церковь; но, сколько ни старались, труд их был напрасен. Ибо бревно было гораздо длиннее, нежели ширина двери, а повернуть его вдоль по дверям они не хотели.

\subsection*{Толкование}

Секущий дрова есть человек, живущий во грехах, который вместо покаяния беспрестанно умножает оные. Черпающий воду есть тот, кто хотя и много делает добра, но не оставляет грехов своих и потому не имеет части с избранными Божиими. А два человека, несущие бревно, суть те, которые добродетельны, но горды.

\section{Творец в Своем творении\footnote{Память великомученицы Варвары (†~ок. 306) празднуется 4~(17) декабря.}}

Каждый взгляд на природу силен убедить ум и сердце, что создание мира не зависело от слепого случая. \textit{Небеса поведают славу Божию, творение же руку Его возвещает твердь} (Пс. 18, 2). Разительный этому пример представляет святая великомученица Варвара.

Эта праведница, дочь богатого и славного сановника в Илиополе, родилась в идолопоклонстве. Диоскор, отец ее, будучи вдов, единственное утешение жизни находил в Варваре, ибо она сколько была прекрасна, столько и благоразумна. Она жила в отдельной части дома, возвышавшейся наподобие башни, из которой видны были прелестные окрестности. Там не занималась она ни нарядами, ни играми, которые столь свойственны юношескому возрасту. Любимое ее занятие было любоваться природой.

Может быть, сначала утешали ее красота, благоухание и нежность цветов. А кто знает силу пристрастий, тот не удивится, если юная девица от прелестной земли вознеслась до красоты неба и, наконец, начала размышлять о явлениях воздушных. Она видела великолепное солнце, тихий месяц, бесчисленные звезды и не могла понять, кто поселил их в необозримом пространстве неба. Слышала гром "--- и не знала, чьи уста глаголют с такой силой, которая все творения поражает ужасом… Ей с младенчества натвердили, что все производят боги ее отечества, но Варвара знала и то, что их изображали подобострастными слабому человечеству, знала, что они часто предаются порокам. «Боги отца моего, "--- часто говорила она, "--- беззаконны в поступках своих: как же могли они основать, как могут поддерживать всеобщий закон природы?» Разум ее переходил от мысли к мысли и между тем все оставался в том же сомнении. Она представляла Бога, Творца вселенной, то в том, то в другом виде, но весь ее успех был только тот, что она из одного заблуждения всегда переходила в другое, чувствовала бессилие свое и не могла помочь оному. Побуждаемая любопытством, она желала удовлетворить его в беседе с людьми мудрыми, но жизнь уединенная являлась тому преградой, которую праведница преодолеть была не в силах.

Но Бог милосерд! Дух Святой разверзает сердце человека к приятию Небесной истины, если оно с пламенным рвением старается постигнуть оную. Дух Святой озарил душу юной девицы, и она воскликнула: «Должен быть Бог, Который все устроил и всем в мире управляет! Но не хочу думать, чтобы этот Бог был подобен богам отца моего». Эти размышления столько занимали мудрую Варвару, что она забыла все на свете, кроме Творца и творения, и как ужаснулась, когда отец представил ей на выбор славных женихов, которые давно искали ее руки и сердца! Девица облилась слезами и сказала ему решительно, что лучше готова умереть, нежели лишиться счастья жизни девической.

Бог, всегда споспешествующий нашему спасению, все обстоятельства располагает нам в пользу. Вскоре отцу Варвары должно было отлучиться на некоторое время в другое место. Тронутый нечувствительностью ее к супружеству, он дал ей в отсутствие своем полную свободу проводить время с подругами, в той надежде, что она, видя тихую радость невест и счастье супругов, переменит свои мысли и отдаст сердце свое достойнейшему из благородных юношей. Среди подруг ее было несколько девиц, которые, будучи христианками, скрывали веру свою, страшась гонений, которые тогда свирепствовали повсюду в Римском государстве. От них"=то услышала она имя Христа Спасителя и с того времени препровождала с ними дни и ночи: училась всему, чему они могли научить ее. Но к совершенному ее благополучию послужил случай особливый: в Илиополь прибыл один александрийский священник под именем купца. Несказанна была радость праведницы, когда она в первый раз воспользовалась беседой его. С тех пор благочестивые наставления старца были единственным занятием ее сердца. Тут познала она бытие и существо Того, Кого прежде усматривала только издали "--- как бы в некотором тумане; узнала воплощение, вольную страсть и восстание от гроба Христа Спасителя, Агнца, взявшего грех мира; узнала добродетели, отличающие душу христианина, и "--- отродилась водой и Духом.

Ни трогательные убеждения, ни зверская ярость отца ее, ни огонь и сталь не могли поколебать в ней любви к Жениху Небесному, а мученическая смерть ее доказала истину слов Евангельских: \textit{Предаст же брат брата на смерть, и отец чадо} (Мк. 13, 12).

\section{Пример подаяния милостыни\footnote{Память свт. Николая, архиепископа Мир Ликийских, чудотворца (†~ок. 335), празднуется 6~(19) декабря.}}

Милостыня есть добродетель боголюбезнейшая, но она теряет едва ли не всю цену, когда честолюбивый благотворитель для столь доброго дела выходит на торжище, с гордостью простирает руку к бедным и раздает милостыню как можно медленнее, чтобы все мимоходящие успели увидеть сострадательность его и щедролюбие. Она не будет тогда общеполезна: ибо лишаются ее те, которые по каким"=нибудь обстоятельствам не могут принимать оную всенародно; не будет означать милосердия: ибо тогда оскорбляется чувствительность приемлющих; не будет достойна награды Небесной: ибо гордый податель милостыни тогда приемлет мзду от мира. Она будет служить только на пользу тунеядцам, к поощрению их праздности. Итак, если сердце наше благорасположено, то будем благотворить втайне. Угодник Божий Николай в этом случае "--- наш наставник.

В ликийском городе Патаре, где святитель Христов Николай приносил бескровную жертву Богу Вседержителю, был один благородный, богатый и славный гражданин, но что более утешало его "--- имел он у себя трех дочерей, прекрасных и благовоспитанных. Уже приближалось время их супружества. Лучшие из патарских юношей один пред другим старались понравиться им и заслужить сердце и руку девиц. Отеческое сердце радовалось, воображая будущую участь дочерей и спокойствие своей старости. Но жизнь века этого непостоянна. Внезапное бедствие постигло дом счастливца, и вдруг богач делается нищим, славный приходит в презрение.

В этом"=то горестном состоянии услышал о нем святитель Христов Николай и положил на сердце мгновенно помочь ему, помочь так, чтобы не ведала шуйца\footnote{Левая рука.}, что сотворит десница его. Праведник дожидается глухой полночи, берет кошелек золота, идет к дому несчастливца и, приблизившись к окну, сквозь разбитую оконницу, потрясаемую ветром, бросает дар свой и уходит поспешно.

Как удивился бедный человек, когда, пробудившись ото сна, увидел на полу кошелек и в нем нашел деньги! «Не мечтание ли это?» "--- думает он, стократ пересыпает их с руки на руку, страшится, чтобы столь же внезапно не исчезли они, сколь внезапно пришли к нему, потом от радости восклицает: «Ах! Зачем не знаю, кому обязан я спасением моей чести и жизни?»

Святитель Николай слышит, что облагодетельствованный им несчастливец счел за первый долг выдать в супружество старшую дочь и, похвалив благое употребление дара в душе своей, повторяет его точно таким же образом. Бедный человек вторую милостыню посвящает счастью второй дочери и догадывается, что неизвестный благодетель без сомнения подаст руку помощи и третьей его дочери. Он чувствует сердечную нетерпеливость узнать своего благодетеля, не дремлет ночи "--- и вдруг слышит, что на пол его хижины упало что"=то в третий раз.

Счастливец (ибо ему тогда грешно было почитать себя несчастливцем) выбегает на улицу, стремится в намерении кого"=нибудь достигнуть и "--- падает в ноги святому Николаю. «Человек Божий! "--- восклицает он. "--- Чем могу возблагодарить тебя?» "--- «Молчание, "--- отвечает угодник Божий, "--- глубокое молчание и благое употребление посильного дара для меня будут приятнейшей от тебя благодарностью». Потом, сделав ему несколько душеполезных наставлений, оставляет в недоумении.

Вот трогательная картина истинной благотворительности. Сколь мало таковых добродетелей представляет этот свет! Тем более немерцаемо их сияние.

\section{Святитель Николай укрощает бунт}

Для укрощения мятежа посланы были от царя Константина трое вождей с немалым числом войска, но так как поднявшаяся на море буря не позволяла им продолжать путь свой, то в ожидании благоприятного ветра они остановились в одном приморском городе, который принадлежал пастве святителя Николая.

Должно думать, что военачальники не строго наблюдали за порядком и благонравием в войске, ибо граждане вскоре почувствовали, что эти защитники отечества не что иное, как разбойники. В домах показались обиды и оскорбления, на торжище "--- грабеж, отчего произошел ропот, потом распри, далее драки, наконец, показались искры всеобщего мятежа. Каждый час страшились увидеть кровопролитное междоусобие.

Едва услышав о столь плачевном происшествии, угодник Божий немедленно отправился в путь, чтобы предупредить или потушить пламя бунта. Одно имя праведника имеет такую силу, которая останавливает злодеяния в стремительнейшем их шествии. Точно таким образом случилось и здесь: весь народ идет ему навстречу, военачальники и воины следуют за народом. Сердца, одно против другого враждебные, на тот раз воспламенились единодушием.

«Воинство, Христу тезоименитое!{Тезоименитый "--- имеющий одинаковое с кем"=либо имя.} "--- сказал святитель Николай, обращаясь к войску. "--- Вы дети одного отечества, подданные одного государя, братия одной Церкви, наследники одного Небесного Царствия, зачем вам оскорблять друг друга? Но этого мало: на вас возложен от царя долг укротить мятеж в отечестве, а вы, не успев достигнуть своей цели, делаете бунт ужаснейший в самом сердце оного!» Так говорил праведник, и военачальники раскаялись в послаблении своем, воины "--- в самовольствах своих; те и другие поклялись свято исполнять обязанности своего звания. Мир и дружелюбие заступили место раздора и междоусобия.

Таков должен быть истинный пастырь и учитель словесного стада Христова.

\section{Невинность освобождается от смертной казни}

Едва угодник Божий Николай укротил бунт между гражданами и войском, явились пред ним несколько знатных мужей из Мир"=Града, где он был епископом. И со слезным сетованием представили ему, что градоначальник их, привыкнув простирать руку свою к беззаконной мзде, осудил на смерть трех граждан, которые ни в чем не повинны и только имели несчастие возбудить на себя зависть и злобу в некоторых злонамеренных богачах, что весь город им соболезнует, но, кроме его предстательства, не имеет средства спасти их. Что, сверх того, этот злобный судья, пользуясь его отсутствием, беспрестанно злоупотребляет своей властью. Облилось кровью сердце угодника Божия. Он с поспешностью возвращается в престольный град свой и слышит, что несчастных ведут на место казни. Гнев и соболезнование дают крылья старцу. Народ пред ним раздвигается с шумом.

Руки злосчастных были связаны, поникшие к земле главы их закрыты, обнаженные выи простерты. Уже исполнитель смертного приговора с суровым взором поднял секиру для удара. В это"=то мгновение предстает тут угодник Божий, с гневом и вместе с кротостью исторгает меч неправды, повергает на землю и с дерзновением разрешает узы невинных. Никто не дерзнул противоречить ему, ибо слово его было властно, поступок его знаменовал силу Божественную.

Спасенные им граждане падают к стопам его и, орошая слезами, с благоговением лобызают их. Весь народ благословляет святого человека, что столь славно Небесный Судия явил на нем милость Свою. Неправедный градоначальник старается пред ним оправдать суд свой, уверяя, что суд его есть суд царя, но человек Божий не удостоил злодея и взгляда своего. Он сказал только: «Я знаю сердце царя моего: он праведен и отметит неправду твою…»

Если имеешь хоть малейшее средство избавить от погибели невинного или, по крайней мере, облегчить горестную судьбу его, не медли сделать доброе дело. Если и не успеешь в том, Господь наградит твое доброе намерение.

\section{Избрание святого Амвросия на епископство Медиоланское\footnote{Память свт. Амвросия, епископа Медиоланского (†~397), празднуется 7~(20) декабря.}}

После смерти арианского епископа Авксентия император Валентиниан поручил духовенству в общем собрании избрать нового архипастыря Медиолану. Он требовал человека беспорочной жизни и с глубокими знаниями, чтобы, как говорил он, царствующий град святился его наставлениями и примерами и чтобы императоры могли принимать советы его с доверием и увещания с почтением.

Епископы собрались, но так как на соборе было столько же ариан, сколько и православных, то, предвидя несогласие, они просили Валентиниана, чтобы он сам нарек из них того, которого желает. Император отвечал им, что это дело сверх сил его, что он не имеет столько ни знания в церковных правилах, ни благочестия, чтобы приступить к оному. «Этот выбор, "--- говорил он, "--- принадлежит единственно пастырям Церкви, потому что они имеют руководителем своим Духа Святого». Вот истинное благочестие просвещенного государя! Архипастырь за порученное ему словесное стадо должен ответствовать Богу: кто же может без трепета принять на себя этот выбор, который и его делает участником того же осуждения?

Наконец, вследствие императорского отзыва епископы собрались в назначенном месте с прочими духовными особами; народ, согласие которого в этом случае было необходимо, был приглашен туда же. Ариане нарекли одного человека из своей секты, а православные хотели избрать одного из своего общества. Обе стороны разгорячились, и этот спор по мятежническому духу ариан уже превращался в открытую войну. Амвросий, начальник области и города, человек разумный и честный, услышал об этом беспорядке и пришел в церковь, чтобы воспрепятствовать оному. Но едва он начал говорить с народом, как один младенец, сидевший на руках у матери, воскликнул: «Амвросий епископ!» Весь народ изумился гласу младенца и, почитая оный за вдохновение Божие, проникся единодушием. Православные и ариане потребовали, чтобы Амвросий был их архипастырем.

Между тем Амвросий сначала почитал этот общий глас вздорным. Но так как народ настаивал на своем требовании, то он объявил собранию, что еще не крестился\footnote{В первые времена христианства было обыкновение отсрочивать крещение до тридцатилетнего возраста; эта отсрочка могла иметь побуждением желание приготовиться достойным образом к принятию крещения.} и знает за собой такие грехи, которые не позволяют ему быть пастырем, но общий голос отвечал ему: «Грех твой на нас да будет!» Амвросий принял решительные меры, чтобы отклонить избрание в должность, превышавшую силы его, и употребил умышленную строгость на суде, которому подпали наиболее обличенные в этом мятеже, чтобы через то охладить к себе сердце народа, но народ проник в его намерение. Наконец, Амвросий тайно бежал из города, с тем чтобы удалиться в неизвестное место, но, блуждая всю ночь, зашел в незнакомые пути, которые внезапно привели его опять к вратам Медиолана. К нему приставили стражу, чтобы он не ушел вторично, и подали челобитную императору, дабы подтвердил избрание их.

Сколько этому противился Амвросий, столь же радостно согласился на то Валентиниан и приказал крестить его немедленно. Избранник Божий в семь дней прошел все степени священства и в восьмой был рукоположен в епископа. Народ радовался неизреченно; сам государь был при посвящении, и сказывают, что по окончании обряда, подняв взор и руки к небу, он воскликнул: «Благодарю Тебя, Господи, что подтвердил выбор мой Твоим избранием, поручив попечение о душах наших тому, кому я поручил правление этой области».

Богоизбранный епископ совершенно посвятил себя изучению Священного Писания и более всего старался восстановить веру и благочестие в духовной пастве.

\section{Нелицеприятие святого Амвросия против грешников}

В народном возмущении, случившемся в Фессалонике по весьма маловажной причине, был умерщвлен Боверик, наместник Иллирийский. Император Феодосий Великий, услышав об этом возмущении, разгневался и решил предать ужасной казни злосчастный город. Хотя святой Амвросий и успел укротить первое движение царского сердца, вообще доброго, но вспыльчивого, однако царедворцы, а особенно Руффин, главный дворецкий, нашли время обратить Феодосия к прежней мысли, и император повелел примерно наказать жителей Фессалоники.

Нельзя без содрогания описать ужас, который был следствием необдуманного суда. Повеление царское было исполнено с коварством и жестокостью: злополучный народ забавляли некоторыми приготовлениями к общенародным играм и, собрав множество граждан в назначенное для этого место, дали условный знак. Вдруг воины устремились на убийство: на площадях, в домах, особенно в цирке, куда собрался народ, кровь лилась ручьями; невинные погибали с виновными; не было пощады ни полу, ни состоянию. Город предан был мечу на три часа, и в это время погибло около семи тысяч человек. Слух об этом ужасном происшествии разнесся повсюду, и святой Амвросий узнал все подробности оного.

Между тем Феодосий намеревался посетить его. Человек Божий, узнав о том, написал к нему следующее: «Сколь ни сильно в душе моей чувство благодарности за любовь и благодеяния монарха, но я после происшествия Фессалоникского уже не могу иметь той радости при виде тебя, которую ощущал прежде. И потому желаю лучше оставить тебя в покое, чтобы ты имел время подумать о своем поведении».

Сверх этого, святой Амвросий представил ему дело Фессалоникское в виде неслыханного мщения и заявил, что Церковь воздыхает о том и первосвященники ее почитают за необходимость, чтобы государь покаялся, а без того не может приступить к Тайной Вечере. Что это предлагают ему не для того, чтобы посрамить его, но дабы побудить его познать в царе человека и смириться пред Богом. «Я пишу к тебе своеручно, "--- продолжал он, "--- чтобы ты, видя мое усердие, рассудил сам о себе втайне».

Император, получив письмо от святого, очувствовался; мгла предрассуждений исчезла, и мщение показалось столько злобным, сколько и безрассудным, ибо основанием оного не была даже и ложная политика, но одна своекорыстная лесть придворных. Феодосий раскаялся и немедленно отправился в Медиолан.

Чтобы истребить худое о себе мнение в народе, он хотел прежде всего оказать знаки благоговения и пошел в соборную церковь, чтобы причаститься Святых Таин. Епископ, услышав это, сошел со своего места и ожидал государя вне притвора. Как скоро увидел его, подошел и начал говорить с той важностью, какую давали ему сан и святость жизни: «Должно думать, государь, что ты не имеешь еще понятия о важности преступления твоего, когда дерзаешь здесь показаться. Может быть, гордясь величеством царского сана, ты утаиваешь сам пред собой твои слабости, но помысли, что ты взят от земли, как и другие люди, и опять возвратишься в землю, как они. Не ослепляйся сиянием порфиры, которая покрывает слабое и мертвенное тело. Подвластные тебе народы одного с тобой естества и служат тому же Богу, Который есть Господь и подданных, и монархов. Ты умертвил тысячи подобных тебе, как же дерзаешь войти в храм общего Отца? Дерзнешь ли простереть обагренные невинной кровью руки к принятию святейшего Тела Христова? Дерзнешь ли Святую Его Кровь принять в эти уста, которые повелели убить твоих собратий? Удались отсюда и не усугубляй новым преступлением того, которое тобой сделано. Но лучше прими с христианским смирением приговор, который Церковь делает тебе на земле и который подтверждает Иисус Христос на небеси. Прими его, ибо это делается для твоего спасения».

Феодосий, пораженный гласом Церкви, стоял некоторое время, потупив очи и не говоря ни слова, потом сказал: «Я признаю свое согрешение, но надеюсь, что Бог призрит на мою слабость и простит меня, как Давида, соделавшего прелюбодеяние и убийство». "--- «Но ты, "--- возразил святой епископ, "--- подражал ему в грехах его, почему же не последуешь ему и в покаянии?» Феодосий, облившись слезами, возвратился в свои чертоги и жил как кающийся, почти не думая, что он "--- император.

Между тем наступил праздник Рождества Христова. Феодосий, удручаемый печалью, встал против своего обыкновения весьма рано и, поскольку как отлученный не мог быть участником торжествующих христиан, готовился проводить это время в глубокой печали. Руффин всемерно старался развлечь его… Таковы льстецы! Этот коварный царедворец, побудив прежде своего государя учинить столь великое прегрешение, теперь старается отвлечь его от покаяния. Слава монарху, что он, будучи другого мнения, пребыл тверд в спасительном намерении! «Перестань издеваться над моим сетованием, "--- вознегодовав, сказал он. "--- Я рассуждаю лучше о состоянии, в котором нахожусь, нежели ты. Не имею ли причины печалиться, когда и последние из моих подданных будут сегодня отправлять молитву свою пред алтарем, а мне одному возбранен вход не только в церковь, но и в самые небеса, по словам Евангелия: \textit{Елика аще свяжете на земли, будут связана на небеси}? (Мф. 18, 18)».

Руффин, не надеясь истребить из мыслей государя священного страха, решился идти к святому архиерею и просить его, дабы уничтожил он приговор отлучения. Хотя Феодосий уверял, что в рассуждении его как величайшего грешника это наказание справедливо, а в рассуждении непреклонности архиерея Божия этот труд бесполезен, но, видя, что Руффин подает несомненную надежду, дозволил ему идти к епископу и через несколько минут сам последовал за ним. Как ни искусно хитрый вельможа исполнял свое дело, но святой Амвросий отвечал ему с обыкновенной неустрашимостью: «Тебе как первому виновнику этого преступления нельзя быть ходатаем разрешения, и если имеешь хотя немного стыда и страха Божия, то помышляй о деле Фессалоникском не иначе, как оплакивая худые советы, которые подавал твоему государю». Руффин, видя безуспешность своего предприятия, объявил, что сам император скоро придет в церковь, но святитель ответствовал: «Я сам буду ожидать его пред дверьми, чтобы возбранить ему вход. Если Феодосий придет как государь христианский, то не преступит законов веры своей; если же захочет сделаться тираном, то к смерти такого числа невинно убиенных может прибавить смерть епископа».

Руффин возвратился, Феодосий, уже на пути бывший, услышав такой ответ святителя, остановился, но, подумав несколько, пошел далее; он готов был претерпеть всякое уничижение и, встреченный святительскими укоризнами, смиренно отвечал: «Я пришел к тебе исцелиться, врач души моей! В твоей власти состоит приказать, что мне делать должно».

Тогда человек Божий описал ему несчастие государя, который не умеет управлять своими страстями и, делая необдуманные приговоры, проливает кровь невинную. В заключение он повелел ему издать закон, чтобы государи, когда принуждены будут употребить чрезвычайную против кого"=либо строгость, по изречении смертного приговора исполнение его отлагали на месяц. Наставление, достойное мудрого, благочестивого священнослужителя Господня. Когда страсти успокоятся, судья лучше может рассматривать свои приговоры и различать невинных от виновных.

Феодосий тут же приказал изготовить это спасительное узаконение, подписал и дал клятву свято соблюдать оное, после чего был разрешен и принят в Церковь. Богобоязненный государь пал на землю и начал молиться: «\textit{Прильпе земли душа моя: Господи, живи мя по словеси Твоему}» (Пс. 118, 25). Он бил в перси свои и, вознося глас молитвы к Нему, оплакивал грехи свои пред всем народом, который через то пришел в сокрушение и вместе с ним плакал. Святитель Господень проливал за него слезные молитвы…

Таков был Феодосий Великий, единодержец всего известного тогда мира. Вся Церковь и поныне назидается его верой, благочестием и послушанием глаголу Евангельскому.

\section{Архипастырь, воздающий Божия Богови и кесарева кесареви}

Святой Амвросий за непоколебимую твердость в православной вере не только терпел непрестанные гонения, но часто подвергался опасности и самой смерти от императрицы Иустины, глубоко зараженной арианской ересью. Но он, воздавая Божия Богови, всегда воздавал и кесарева кесареви.

Максим, предводитель римского войска в Англии, после умерщвления императора Грациана овладел Галлией и, возгордившись успехом злодейств своих, готовился учинить нападение на Италию, где царствовал Валентиниан, младший брат Грациана. Соцарствующая ему мать, Иустина, для сопротивления тирану не имела войска и не надеялась получить помощь от союзников, почему и вознамерилась отправить к Максиму посольство, чтобы смирением своим удержать его за Альпийскими горами. Но при дворе своем не нашла человека, который бы мог или хотел принять на себя столь важное дело. Арианка прибегла к гонимому от нее святителю и, будто отложив ненависть свою, просила его именем императора, своего сына, идти к Максиму, одно имя которого ужасало всех вельмож.

Угодник Божий принял с удовольствием предложение и отправился в путь с намерением пожертвовать даже своей жизнью за государя и отечество. Он нашел Максима, готового вторгнуться в Италию.

Но сила благочестивой мудрости и священного красноречия непостижима. Святой Амвросий довел тирана до того, что оружие пало из рук его или потому, что почтение и благоговение к великому мужу внушали ему некоторую кротость, или Сам Бог, Царь царей, утоляющий ярость мучителей, когда Ему угодно, предписал ему на тот раз пределы. С того времени тиран всегда раскаивался, что упустил столь благоприятный случай для властолюбия и хищничества, и часто жаловался на мудрого первосвященника, а более на себя, что послушался его увещаний.

Столь великого пастыря должно бы почитать отцом юного царя и спасителем царства, но Иустина думала только о восстановлении утесненного им арианства "--- и решилась погубить святого Амвросия. Она издала именем сына своего указ, которым объявила мятежниками, нарушителями церковного мира, оскорбителями царского величества и преступниками, достойными смертной казни, всех дерзающих прекословить всенародному отправлению арианского лжесвященнослужения. Избрала в епископы какого"=то Авксентия, родом скифа, изгнанного из своей земли за порочную жизнь, и, наконец, дошла до такой степени ожесточения, что велела тайно схватить человека Божия и увезти в заточение, но, к счастью, предприятие злобы открылось.

Досадуя на безуспешность, Иустина приказала святому епископу и всему православному духовенству отдать свои церкви арианам. Но так как они своего права уступить не хотели, то она приказала войску окружить священные храмы. Небу угодно было, чтобы это войско приняло сторону архиерея Божия. Между тем Максим под видом защиты веры опять устремился на Италию, и жесточайшая гонительница святого Амвросия опять прибегла к нему и просила забыть прошедшее. Архипастырь, предпочитая пользу общую и службу царскую своему покою, немедленно отправился к тирану. Но Господь, наказуя утеснителей веры, не благоволил смягчить сердце их утеснителя.

Святитель Амвросий благочестием и гражданскими добродетелями столько прославился по всему свету, что в одно время мудрейшие из персидских вельмож приходили в Медиолан нарочно, дабы видеть его, как некое чудо. Но это благоговение к угоднику Божию еще более явствует из отзыва французских и немецких королей. Будучи на пиршестве с Арбогастом, славным и почти самовластным предводителем римских войск в Галлии, они спросили, знает ли он епископа Амвросия. «Имею честь быть его другом, "--- отвечал вельможа, "--- и нередко кушал за его столом». "--- «Итак, не должно удивляться, "--- воскликнули цари, "--- что ты одержал столь много знаменитых побед, когда любил тебя тот, который может удержать и самое течение солнца, если захочет».

\begin{center}\small\textsc{Конец второй части.}\end{center}

\chapter{ЧАСТЬ ТРЕТЬЯ}
\section{Богодухновенное происхождение Литургии святого Василия Великого\footnote{Память свт. Василия Великого, архиепископа Кесарии Каппадокийской (†~379), празднуется 1~(14) января.}}

Святитель Христов Василий Великий просил у Господа, да снидет на него Дух благодати, премудрости и разума, чтобы своими глаголами\footnote{\textit{Глагол} "--- слово, речь.} мог совершать бескровную службу. Через некоторое время явился ему Господь в сновидении с апостолами, совершая «предложение хлеба и чаши» на святом жертвеннике, и, восставив Василия, изрек ему: «По твоему прошению да исполнятся уста твои хваления». Архиерей Божий пробудился, с благоговейным трепетом пошел в церковь и, приступив к жертвеннику, начал глаголать и вкупе писать все те молитвы, которые находятся в его литургии. После того он воздвиг хлеб, молясь прилежно: \textit{Вонми, Господи Иисусе Христе, Боже наш, от святого жилища Твоего}, и прочее. Священнослужители и народ видели Небесный свет, озарявший алтарь, и святителя, и некоторых мужей, в белые ризы одеянных и великого архиерея окружающих; в ужасе поверглись они лицом на землю и, проливая слезы, благословляли Господа.

\section{Смерть Василия Великого}

Один еврей, по имени Иосиф, был столь искусный врач, что по биению пульса за пять дней предсказывал больному час кончины его. Богоносный Василий, предвидя духом, что Иосиф некогда обратится к Спасителю мира, любил его и, часто призывая к себе, советовал принять Святое Крещение. Но еврей всегда отзывался тем, что в которой вере рожден, в той и умереть хочет. На что угодник Божий отвечал ему: «Поверь мне, что не умрем ни я, ни ты, доколе не родишься от воды и Духа. Без этой благодати невозможно войти в Царствие Божие. Не крестились ли и отцы твои в облаке и море? Не пили ли от камени, который был прообразованием духовного камени, Иисуса Христа?»

Наконец приспело время отшествия его к Богу. Святитель разболелся и, призвав Иосифа будто для своего врачевания, спросил у него: «Что думаешь обо мне?» Еврей, ощупав пульс у святого, сказал служителям: «Уготовьте все нужное к погребению, святитель скоро умрет».

«Сам не ведаешь, что говоришь», "--- возразил Василий Великий. Но Иосиф отвечал с уверенностью: «Ей"~ей, владыка! Ты скончаешься прежде, нежели зайдет солнце». "--- «А что сделаешь, если останусь жив до шестого часа утра?» "--- спросил у него Василий. «Пусть умру сам», "--- отвечал Иосиф. «Заклинаю тебя умереть греху, да оживешь Богу», "--- сказал святитель. «Знаю, о чем говоришь, владыка! "--- отвечал еврей. "--- И свидетельствуюсь Богом, что сотворю волю твою». Святой Василий отпустил Иосифа и помолился Богу, да продолжит жизнь его для спасения заблудшего.

Утром на другой день святитель послал за ним. Врач, столько на себя надеявшийся, не поверил служителю, что Василий жив, однако, желая проститься с умирающим, он пошел немедленно. Но как изумился он, увидев человека Божия! Иосиф пал ему в ноги и чистосердечно воскликнул: «Велик Бог христианский! И нет Бога, кроме Него! Повели, святой отче, не отлагая, подать мне и всему семейству моему Святое Крещение». "--- «Я сам, моими руками, отрожду тебя», "--- отвечал святитель Христов и, к общему удивлению, восстав с одра болезни, пошел в церковь. Там, перед всем народом, крестил Иосифа и весь дом его сам. Совершил литургию и новопросвещенных приобщил животворящих Таин Христовых. Поучая их, он пребыл в церкви до девятого часа. Наконец, воссылая благодарение Богу за неизреченные Его благодеяния, всенародно предал дух свой в руки Господни.

\section{Сила имени Иисусова. Повесть преподобного Илии, пустынника Египетского\footnote{Память прп. Илии Египетского (†~IV~в.) празднуется 8~(21) января.}}

Один старец обитал в опустевшем капище. В одно время пришли к нему нечистые духи и сказали: «Выйди из нашего места». "--- «Вы не имеете места», "--- отвечал старец. Тогда начали они повсюду разбрасывать пальмовые ветви, приготовленные для плетения корзин, но старец неутомимо собирал оные. После того враги человеческого рода, схватив его за одежду, повлекли вон, но старец уперся в двери и возопил: «Иисусе, помоги мне!» "--- и бесплотные злодеи мгновенно разбежались. «О чем плачешь?» "--- тогда спросил у него невидимый голос. «О том, что ненавистники веры и добродетели дерзают издеваться над рабом Господним», "--- отвечал старец. «Ты сам виноват, "--- возразил небесный голос, "--- поскольку забыл обо Мне; сам видишь: когда призвал Меня, Я не укоснил прийти на помощь к тебе». Старец, ощутив присутствие Божие, повергся на колена и приник лицом к земле. Сколь разительное и необходимо нужное для каждого человека тут заключается наставление!.. Мы вопием к Господу, молимся Ему, даже бываем иногда столь наглы, что ропщем на Него, когда приключится с нами какое"=либо несчастье. Но для чего не просили у Него помощи тогда, когда страсти, так сказать, рассыпали по воздуху наш разум, нашу веру, наши добрые дела? Видно, мы надеялись тогда на самих себя. Но сколь безумно быть уверенными, что можем выгнать сами собой врагов, \textit{воюющих на душу}.

\section{Друзья по вере Христовой}

В царствование Деция были два римлянина, Неарх и Полиевкт\footnote{Память мученика Полиевкта (†~259) празднуется 9~(22) января.}, которые имели между собой такую дружбу, какую редко можно видеть и между родными братьями. Неарх был христианин, ходящий по заповедям Господним, а Полиевкт язычник, впрочем добродетельный. Часто Неарх слезно уговаривал его присоединиться к Церкви Христовой, читал ему Божественное Писание, объяснял таинства веры, показывал мерзость идолопоклонства. Но Полиевкт, не отказывая своему другу ни в чем, в этом только случае, по"=видимому, был равнодушным слушателем его советов. Между тем настало гонение, по стогнам\footnote{\textit{Стогны} "--- площади, широкие улицы.} града читали нечестивое повеление Деция, чтобы все поклонялись идолам, как богам "--- хранителям отечества. Тогда христолюбивый Неарх, приготовляясь к смерти, начал сокрушаться, что друг его навсегда останется в нечестии, и, беспрестанно сетуя о погибели души его, даже изменился в лице своем. Это было сердечным ударом для Полиевкта. При каждом свидании он спрашивал его о причине столь глубокой печали, но Неарх, страшась услышать решительный отказ, не смел объясниться. Наконец, после неотступных просьб Полиевкт сказал ему: «Друг мой, конечно, чем"=нибудь я оскорбил тебя, оскорбил столь тяжко, что не хочешь и простить меня!» Тогда Неарх, облившись слезами, отвечал ему: «Ах! Я предвижу разрыв нашего дружества и прихожу в отчаяние». Полиевкт затрепетал. «Этого быть не может, "--- сказал он. "--- Я клянусь, что и смерть разлучить меня с тобой не сильна». "--- «Но причиной разлуки нашей будет нечто ужаснейшее, нежели естественная смерть», "--- возразил Неарх. Полиевкт, не уразумев ответа, обнял его и, прижимая к сердцу, сказал решительно: «Не могу более переносить твоего молчания и загадок: или объясни все, или сейчас же увидишь смерть мою». "--- «Царское повеление, "--- воскликнул Неарх, "--- должно разлучить нас: я "--- христианин, а ты "--- поклонник идолов; итак, когда осудят меня на смерть, ты отречешься от меня».

Услышав это, Полиевкт узнал, чего хочет Неарх, и, ощутив наитие Духа Святого, сказал спокойно: «О! Если так, то стесненное сердце мое пусть отдохнет; не бойся, любезный друг! Мы не разлучимся… Буди слава Христу Богу Истинному, "--- продолжал он. "--- От этого часа оставляю суетный мир и исповедаю в себе раба Христова; пойду, прочитаю повеление Деция, на христиан обнародованное, после этого знаю, что мне делать должно!»

Ни ласки и угрозы начальников, ни слезы родителей, супруги и детей не могли принудить его к нарушению обета неразрывной дружбы с Неархом. Полиевкт крестился кровью своею и был ему предшественником в жизнь вечную.

\section{Святитель Филипп, митрополит Московский\footnote{Память страдальца Божия, свт. Филиппа, митр. Московского и всея России чудотворца празднуется 9~(22)~января.}}

Из Российской истории известно, что царь Иоанн Васильевич Грозный, к приближенным боярам всякое доверие потеряв по причинам отчасти справедливым, а отчасти и несправедливым, учредил особенную телохранительную стражу, под именем опричников, и одел их совершенно по"=татарски. Он присвоил в удельную себе собственность богатые волости для содержания сих телохранителей и сквозь пальцы смотрел на их наглости, а что всего хуже, принимал от них всякие доносы.

Усердные сыны отечества и веры Христовой с неудовольствием смотрели на поступки государя, но, опасаясь неумолимого деспотизма, с сердцем стесненным удалялись от престола и молчали. Один святитель Христов Филипп, будучи вызван из Соловецкого монастыря, где был игуменом, на престол Российской митрополии, единственно за святость жизни, вопреки его желанию, еще до посвящения своего осмелился грозному царю сказать, что ему весьма не нравится опричнина, отделяющая государство от государя. Получив сан верховного в России святителя, уже торжественно предлагал он уничтожить сих телохранителей и не называть всех других бояр «земскими».

Но грозный царь имел сердце не Феодосия Великого\footnote{Этот государь с примерною для христиан кротостью подвергся епитимии, возложенной на него св. Амвросием Медиоланским за убийства, учиненные в Фессалонике.} и на справедливые упреки святого Филиппа отвечал с яростью: «Что за дело тебе, чернецу, до наших царских советов? Ужели не знаешь, что меня мои же присные\footnote{Приближенные.} хотят поглотить? \textit{Друзи}\footnote{Друзья.}\textit{ мои и искреннии}\footnote{Ближние.}\textit{ мои прямо мене сташа}\footnote{Встали напротив меня.}\textit{, и нуждахуся}\footnote{Употребляют усилия.}\textit{ ищущии душу мою и ищущии злая мне}». "--- «Государь! "--- возразил митрополит, "--- это внушают тебе льстецы и лукавые люди. Приемли от благих благие советы, а не от ласкателей, ищущих своей, а не твоей пользы. Почто единых избрал? Их единых слушаешь, а других всех, яко враждебных, до себя не допускаешь? Государь! Это служит причиною разделения, а не соединения». "--- Царь на это отвечал: «Филипп! Не прекословь нашей державе, да не постигнет тебя гнев мой, или добровольно оставь твой сан». "--- «Благочестивый царь! "--- отвечал Филипп, "--- я сего сана не искал, но еще умолял тебя, да избавишь меня от оного. Почто лишил меня пустыни и братий? Впрочем, если хочешь поступить против правил святой Церкви, твори еже хощеши\footnote{То, что хочешь.}». "--- После этого начальник опричников, Малюта Скуратов, начал еще более наговаривать государю на святого митрополита, как на главного противника предержащей власти.

Но первосвященник не устрашился гнева грозного самовластителя. В один праздник государь, в соборе подошедший к его месту, трижды просил у него благословения. Святитель Христов в досаде, что царские приближенцы вошли в церковь в черном одеянии с высокими остроконечными колпаками, стоял, как будто не видя Иоанна. Когда же приближенные сказали ему: «Владыко святой! Благочестивый царь пришел к твоей святыне и требует у тебя благословения», "--- тогда святой возгласил: «Царю благий! Кому поревновав, ты изменил образ царского своего благолепия и в столь непристойное облекся одеяние? Убойся суда Божия! Неужели ты не отец народа? Неужели не знаешь, сколько страждут православные христиане! О, государь! Мы здесь, в священном храме, приносим бескровную жертву Богу, а за алтарем неповинно льется кровь христианская! Вспомни, что ты, хотя носишь подобие Божие на земле, яко человек, причастен персти земной». "--- Государь на это сказал с угрозами: «Филипп! Моей ли державе противишься? Посмотрим, какова крепость твоя». "--- Он тотчас вышел из церкви; однако же на тот раз не излил на него неправедный гнев свой.

Спустя некоторое время, святитель Христов, отправляя обедню в Новодевичьем монастыре, увидел многих, пришедших с государем, в татарских тафьях\footnote{Шапочка, употреблявшаяся у татар, которую и в России с татарского обычая мирские и сановные люди носили. А при царе Иоанне Васильевиче соборно запрещено было их носить, особенно в церковь святую ходить в тафьях. [Стоглавый собор 1551~г., гл.~39].} на голове, и немедленно обратил речь свою к государю: «По Евангельскому слову на молитве должно быть с непокровенною главою. Почто же сии, будучи единоверными, соделались татарами и дерзают стоять здесь, пред Всемогущим Богом, с покрытою главою?» "--- «А кто они таковы?», "--- спросил на это царь. Но тафьи еще при первых словах митрополита были сняты, и приближенцы донесли Иоанну, что митрополит лжет единственно для того, чтобы насмеяться над его царскою державою. Тогда"=то обманутый самодержец вышел сам из себя, позволил на святого человека говорить кому и что угодно, и по некоторым доносам повелел в соборе при всем народе снять с него одеяние и сан святительский. Исполнителями злочестивого приказания были опричники. Эти лютые враги, растерзав на архиерее Божием одежды, ругались над ним ужасным образом. Подвижник Веры и отечества, влачимый ими, сказал: «Радуюсь, что для Церкви Господней терплю это; но близко время, в которое она приимет вдовство, и пастыри, яко наемники, презираемы будут».

Святитель Христов сослан был в Богоявленский, а потом в Отроч монастырь; но враги преследовали его и в самом уединении: приставленная к нему стража обходилась с ним по"=варварски; на ежедневное содержание выдаваемо было только по две деньги; а наконец подозрения, беспрестанно опричниками внушаемые Иоанну, будто человек по сердцу Божию имеет вредную для царя переписку с Новгородскими мятежниками, довершили его участь. Грозный Иоанн, на пути к Новгороду, дал Скуратову бесчеловечный приказ, умертвить невинного страдальца, и сей варвар задушил Филиппа.

\section{Нелицеприятие, снисхождением растворяемое}

Греческий царь Анастасий, заразившись ересью Евтихия, старался преклонить на свою сторону все православное духовенство "--- то угрозами и заточением, то лаской и дарами. В число последних лжемудрствующий царь поместил и Феодосия Великого\footnote{Прп. Феодосий Великий, родом каппадокиянин, жил в IV~столетии, скончался 105~лет от рождения в 529~г. Память его празднуется 11~(24) января.}. Он послал к нему тридцать литр\footnote{\textit{Литра} "--- греческая мера веса, равная 0,34~кг; также \textit{литра} "--- церковная мера веса (см. Ин. 19, 39).} злата, будто для пропитания и одеяния нищих, а на самом деле для того, чтобы уловить согласие великого отца, на которого греки и римляне взирали как на апостола. Поборник Православия уразумел коварство Анастасия и восхотел богоугодным образом сам уловить его. Он не отверг даров, чтобы не показаться, будто имеет подозрение на веру царскую, и тем не подать причины к его гневу. К этому побуждало его благочестивое, истинно пастырское желание "--- через милостыню исходатайствовать у Бога Анастасию благодать, наставляющую на путь истины. Между тем Анастасий думал о Феодосии Великом как об одном из тех лжепастырей, которые, подобно наемникам, не радят о стаде своем, и, поскольку дары были приняты, почитал его своим единомышленником, но обманулся злобожный властелин. Ибо, когда через особливых посланников захотел он испытать Феодосиево исповедание веры, великий отец, собрав всех пустынных граждан, крепко стал против злочестия, и царю, между прочим, ответствовал этими достопамятными в устах отшельника словами: «Как за отечество, так и за правоверие сладостно положить душу! Сладостно, хотя и святые места наши узрим истребляемыми огнем!» Через это угодник Божий отнял у Анастасия всякую надежду когда"=либо преклонить его на свою сторону.

Кто после этого скажет, что иноки, отказавшиеся от мира, занимаясь только богомыслием, не умеют верноподданнически поступать с предержащей властью и забывают отечество?

\section{Новозаветный Елисей}

Преподобный Иаков Низибийский\footnote{Прп. Иаков, епископ Низибийский (†~350), уроженец персидского города Низибия, который тогда платил дань римлянам, жил в III -- IV~вв. Память его празднуется 13~(26) января.}, увидев некогда персидского градоначальника, только что свершившего неправедный суд над одним из невинных граждан, горько вознегодовал. Человек Божий имел свыше силу пророков и апостолов и мог бы жестоко наказать притеснителя, но поступать сообразно с Новым Заветом и благодатью Христовой было первым правилом всех его деяний. Святой старец воззрел на небо, и мгновенно камень, близ судьи лежавший, сокрушился и рассыпался в персть, обличая тем суд неправедный и знаменуя, что так истребятся нечестивые в день гнева Господня. Все ужаснулись, градоначальник вострепетал и, познав грех свой, отменил несправедливый приговор.

Какой урок для тех, которые имеют власть и силу других наказывать!

\section{Не должно мечтать высоко о своих добродетелях}

Преподобный Антоний, обитая в пустыне Египетской с учениками своими, однажды помыслил, что нет инока столь совершенного, как он, поскольку всех прежде поселился он в пустыне и избрал место всех далее от сует мира. Будучи погружен в такие думы, великий старец вдруг услышал: «Антоний! Есть раб Божий, который совершеннее тебя; если хочешь, ты можешь найти его посреди внутренней пустыни». Устрашенный старец, как только пришел в себя, взял жезл свой и пошел в путь. Будучи руководствуем сверхъестественными и чудными встречами, Антоний наконец узрел пещеру, в которой обитал пустынник, свыше откровенный. Но он, узнав о приходе посетителя, затворил двери. Сколько Антоний ни стучался, ответа не было. Антоний стоял до шестого часа и умолял, да будет ему позволено увидеть того, кого Сам Бог повелел обрести ему. «Если не отверзешь мне, "--- говорил он, "--- то умру на пороге твоем, и здесь погребешь меня». Но старец из хижины отвечал ему: «Кто просит с угрозой и, проливая слезы, укоряет? Ты жалуешься, что не отверзаю тебе, но для чего пришел ты сюда? Разве хвалиться, что умрешь здесь?» Тогда он отверз ему.

Этот великий старец был Павел Фивейский\footnote{Преподобный Павел был первый пустынножитель, оставил мир при Деции, в 251~г. Скончался при Констанции, сыне Константина Великого, в 341~г. Память его празднуется 15~(28) января.}, имевший от рождения своего сто тринадцать лет и от самой юности в пустыне поселившийся. Он рассказал преподобному Антонию историю своей жизни и возвестил о смерти своей, которая на другой же день и последовала. Тогда "--- сколь торжественна, сколь прекрасна кончина праведника! Тогда Антоний узрел на воздухе лики Ангелов, пророков и апостолов, а посреди них узрел душу святого Павла, чистейшую солнца: она восходила на Небо. От страха и радости Антоний пал на землю. «Душа праведная! "--- вопиял он. "--- Прими меня в сопутники твои». Но воздушный путник отвечал ему: «Не должно наблюдать тебе свою только пользу, твоя обязанность "--- пещись о ближних; будет время, что предстанешь Агнцу Небесному, но прежде научи других непреткновенно шествовать по стезе, которая ведет к престолу Его».

Преподобный Антоний похоронил святые останки великого старца и, возвратившись в пустыню свою, едва узрев братию, воскликнул: «Горе мне, грешнику, имя только инока носящему!» И когда у него спрашивали, где так долго замедлил, он отвечал в восхищении духа: «Я видел Илию, видел Иоанна в пустыне; ей"~ей, видел Павла в раю Господнем».

Христиане! Вспомните, сколь велик верой и добродетелями был Антоний, и не забудьте, как Господь уличил его в минутном, непроизвольном высокоумии. Не мечтайте высоко о своих совершенствах. «Добродетельный человек, "--- часто говаривал Исидор Пилусиотский, "--- без сомнения получит венец светлый. Но кто, будучи добродетелен, думает, что сделал весьма мало, тот за смиренномудрое о себе мнение приимет венец светлейший». Хотите ли быть истинно добродетельными "--- не верьте, что вы добродетельны, и через то добродетель ваша будет возрастать более и более.

\section{Труд и молитва}

Антоний Великий\footnote{Прп. Антоний Великий удалился в пустыню при Константине Великом, в 310~г., преставился в 356~г., в глубокой старости. Память его празднуется 17~(30) января.} однажды предался унынию и каким"=то мрачным помышлениям, но вскоре, испугавшись сам себя, воскликнул к Богу: «Господи! Я хощу спасен быть; но и сам не знаю, что развлекает меня. Покажи мне средство избегнуть противоборствующих помыслов; научи меня спастись». Через какое"=то время святой Антоний вышел из кельи и увидел незнакомого человека, который сначала сидел за рукоделием, потом встал и молился. Опять садился и вязал кошницу\footnote{\textit{Кошница} "--- корзина.}, после того опять встал на молитву… Это был Ангел, в образе инока ниспосланный от Бога, чтобы исправить и предостеречь Антония. Между тем как святой пустынножитель смотрел на него, бесплотный хранитель сказал ему: «Так поступай "--- и спасен будешь». Услышав это, преподобный Антоний исполнился радости и надежды и с того времени никогда не нарушал правила Ангельского.

Какое бы звание Господь ни возложил на тебя, какие бы обстоятельства ни окружали тебя, молись для спасения души твоей и трудись для пользы общества. Молитва и труд друг друга подкрепляют взаимно: молитва дарует силы к понесению труда, на который каждый человек родится, а труд, разгоняя зловредные помыслы, эти исчадия праздности, предуготовляет уму и сердцу благоговейное спокойствие и глубокое внимание для часов молитвенных. Они служат друг для друга как бы отдохновением, и блажен человек, в чьем сердце эта чреда не нарушается.

\section{Пустынник и языческие мудрецы}

К преподобному Антонию, безмолвствовавшему в нагорной пещере, пришли два языческих философа с тем намерением, чтобы поговорить о таинствах христианской веры и посмеяться над жизнью пустынников. Мудрый старец, при первом взгляде познав, кто они, через переводчика сказал: «Для чего вы, люди мудрые, приняли на себя столь дальний и трудный путь и пришли к человеку безумному?» "--- «Ты не безумный, "--- отвечали они, "--- но совершенный мудрец». "--- «Если так, "--- возразил старец, "--- то обязанность и польза ваша требуют принять от меня наставление как от мудреца, к которому пришли научиться, и я даю вам оное: будьте христианами!» Пораженные столь нечаянным приветствием, философы пошли назад, наиболее удивляясь остроте и разуму пустынножителя.

\section{Общее правило для начальников и отцов семейства}

Когда Антоний Великий после уединенных подвигов отшельничества против воли своей поставлен был пастырем и наставником других, тогда братия, старцы и юноши, собравшись к нему, начали просить, чтобы он предписал им уставы иночества. Великий отец сказал им: «Чтобы исполнять в точности все заповеди Господни, на это довольно Божественного Писания: но помните, что все добродетели могут поколебаться в основании своем, если братия не будет взаимно утешать друг друга. Итак, вы, как отцу, исповедуйте мне все, что знаете, а я, наученный временем и опытами, как детей, буду наставлять вас. Не ослабевайте в предпринятом подвиге "--- вот первое общее для всех сословий наставление!» Потом, поучив их всем добродетелям гражданства небесного, отпустил с миром.

Уроки преподобного отца имели спасительное действие, ибо очень скоро общество пустыннолюбцев так сильно умножилось, что не было числа братии, повсюду были монастыри, как сени расставленные. Поющие, молящиеся, уповающие на \textit{Сокровище благих}, сверх того занимались трудами рук своих, чтобы могли быть не только ходатаями за грешников, но и подавать милостыню бедным. Не было между ними ни вражды, ни клеветы, ни ропота; любовь и согласие соделывали из селений их как бы град, от мирского мятежа далече отстоящий, правдой и благочестием населенный. Взирая на них, можно было воскликнуть: \textit{Коль добри доми твои, Иакове! и кущы твоя, Израилю! Яко дубравы осеняющыя, и яко садие при реках, и яко кущы, яже водрузи Господь} (Чис. 24, 5~-- 6).

Облеченные свыше чиноначалием, а также и вы, отцы семейства! Наставляйте порученных вам людей тому, чему наставлял преподобный Антоний Великий: тогда "--- \textit{яко роса Аермонская, сходящая на горы Сионския, будет над вами благословение и живот до века} (Пс. 132, 3).

\section{Пределы смирения}

Антоний Великий, яко истинный раб Христов, всегда готов был понести на себе все уничижения; но, если клевета касалась чистоты его вероисповедания, тогда поборник Православия не оставался в пределах смирения, что можно видеть из следующего поступка.

Однажды ариане оболгали его, будто дружески с ними обращается. Удивленный их дерзновением, святой Антоний воспылал праведным гневом и немедленно пошел в Александрию, там, пред епископом и всем народом, предал их проклятию, нарицая предтечами антихриста. Но такова сила правды и святости! Хотя начальники и учители арианства скрежетали на Антония Великого, однако народ, не только еретики, но и служители идолов, повсюду теснился вокруг него, стараясь принять благословение или, по крайней мере, прикоснуться к одеждам его.

Христиане! Не подумайте, чтобы Антоний через этот поступок нарушил урок христианского терпения! Если бы он не обличил злостной клеветы, он подал бы случай другим о себе соблазняться, а что ужаснее, невежды, которые не могут проникать намерений великого человека, тогда могли бы предаться пагубному расколу. Хотя молчание почитается, по всей справедливости, в числе добродетелей, но свет почитает оное не иначе, как согласием на то, что клеветники нам приписывают.

\section{Сострадательность к заблуждающимся}

Один инок был соблазнен духом"=искусителем. Будучи выгнан из обители, несчастный пришел к преподобному Антонию и пребыл с ним некоторое время. Но, когда по приказанию святого отца возвратился он на прежнее место, иноки не хотели принять его. Изгнанник вторично прибег к Антонию. Святой старец опять послал его к братии и велел от своего имени сказать им следующее: «Корабль от бури повредился, растерял товары и едва достиг берега: неужели хотите погребсти в волнах и то, что не погибло?» Иноки, услышав это, немедленно приняли изгнанника опять в число братии.

Наказывая подчиненного, должно соблюдать меру, чтобы не довести его до отчаяния.

\section{Благоразумная снисходительность}

Один охотник, находясь в пустыне на охоте за сернами, приметил, что авва Антоний не по"=начальнически, но весьма ласково обращается с братией. Старец познал невыгодные мысли охотника о строгости иноческого жития и, желая вывести из заблуждения, сказал ему: «Наставь стрелу и напряги лук». Охотник сделал это. «Еще напряги», "--- продолжал старец. И охотник напряг туже. «Еще напряги», "--- сказал старец в третий раз. Но охотник возразил: «Если натяну чрезмерно, переломится лук». Тогда святой Антоний сказал ему: «Того же должно опасаться и в делах веры: если с братией будем поступать с чрезвычайной строгостью, они почувствуют перелом; везде нужны снисходительность и некоторое послабление, а иноки также люди». Тогда охотник раскаялся и пошел от святого Антония с пользой для души своей, а братия через то укрепились на большие подвиги.

\section{Слава подаяния милостыни}

Преподобный Маркиан\footnote{Прп. Маркиан жил в V~в. Память его празднуется 10~(23) января.}, пресвитер великой Константинопольской Церкви, идя на службу Божию и увидев нищего, почти нагого, от холода трясущегося, снял с себя одежду и прикрыл наготу его, а сам остался только в нижнем платье. Но посмотрите, как Господь за это прославил его! Когда святой Маркиан священнодействовал, то весь народ, причт и сам патриарх под церковными ризами приметили блистающую порфиру царскую.

Не видно ли из этого, что мы, снабжая убогих от имения своего, не теряем оное, но отдаем его в обмен на драгоценнейшие сокровища Тому, Кто и за чашу студеной воды обещал наградить нас вечным блаженством! Царская одежда, преподобному Маркиану данная, не есть ли доказательство, что подающий милостыню делает богоугодное.

\section{Посрамленная клевета}

Афанасий Великий\footnote{Свт. Афанасий Великий жил в царствование Константина Великого и его преемников; за Православие страдал в изгнании 46~лет. Память его празднуется 18~(31) января.}, архиепископ Александрийский, как неусыпный и мудрый защитник Православия более всех был угнетаем и гоним арианами. Исступленные и неистовые богоборцы, желая погубить человека Божия, вымышляли на него самые бессовестные и безумные преступления и, между прочим, донесли царю Констанцию, что святитель Афанасий делал чары и чудеса рукой, которую будто отсек у клирика, Арсения.

Дело началось, по"=видимому, весьма выгодно для ариан. Указом царским предписано в Тире быть собору, куда призван и Афанасий. Двор и почти все чиновники держали сторону ереси, лжесвидетелей нашлось довольно; для злодеев не трудно было представить и мертвую руку выдать за руку Арсения, ибо его незадолго до этого, в самом деле, вдруг в Александрии не стало.

Но Бог не \textit{оставляет жезла грешного на жребии праведных} (Пс. 124, 3). Арсений, который скрывался единственно от страха, чтобы не понести наказания за некоторое преступление, втайне им учиненное, услышав о клевете, его рукой подкрепляемой, восчувствовал глубокое сокрушение об участи отца и благодетеля и, пришедши тайно в Тир, явился к Афанасию с сердечным раскаянием о грехе своем. Святитель возблагодарил Промысл, его спасающий, и приказал Арсению до времени жить скрытно.

Между тем Архелай, один из царедворцев Констанциевых, и Нон, князь финикийский, открыли суд. Тогда "--- о зрелище ужасное и богомерзкое! И чье сердце не обольется кровью, видя злобу нечестивых! Тогда внесли посреди судилища мертвую руку и, показывая оную святому Афанасию, в один голос воскликнули: «Вот безмолвная на тебя свидетельница! Она обличает тебя, она держит тебя; ни велеречием, ни хитростью не вырвешься от нее! Все знают Арсения, у которого ты, лицемерный святотатец, отсек ее; скажи, для чего она тебе потребна?» На все вопли человек Божий отвечал кротко: «Кто из вас лично знает Арсения? И все ли уверены, что эта рука именно ему принадлежит?» Но, когда богоборцы начали ложь свою утверждать клятвой, Афанасий подал знак: вдруг из среды его защитников вышел Арсений и снял с лица своего покрывало. Тогда святитель с гневом воззрел на клеветников и, кивая головой, сказал: «Не Арсений ли это? Не тот ли церковнослужитель, у которого, как утверждаете, я отсек руку?» Он повелел Арсению простереть и показать судьям десницу, потом шуйцу и обратился к неприятелям своим укоризненно: «Здесь обе руки! Покажите же мне вашего Арсения и скажите, чью руку здесь показываете? Она обвиняет вас в том злодействе, в котором единственно по ненависти сами меня обвиняете».

Арианские епископы и клеветники, сколько бессовестны ни были, не могли перенести стыда своего и, закрывая руками лица, в безмолвии начали выходить из судилища.

\section{Ревность о сбережении доброго имени ближнего}

Хотя Тирский собор, к стыду ариан и чести святителя Афанасия Великого, был распущен, но лжеепископы не могли быть спокойны. По их проискам приказано вторично быть собору в Медиолане\footnote{Современный Милан.}.

Православные епископы с трепетом взирали на последствия и всемерно уклонялись от сонмища, подобного христоубийцам, и потому арианское духовенство, в числе тридцати особ, одно совершило суд над архиереем Божиим. Когда начали подписывать имена свои, то Дионисий, православный епископ Медиоланский, летами всех младший и недавно возведенный на престол архипасторства, был склонен к общему рукоприкладству. Не страсть, не зловерие водили пером его; к несчастью, долго отрицаясь от этого, он устыдился толикого числа престарелых и уважаемых епископов.

Сколь иногда вредны и добродетели, не руководствуемые благоразумием!

Вскоре после того прибыл в Медиолан Евсевий, епископ Виркеллинский, которого по летам и благочестию Дионисий почитал не иначе, как отцом; и, когда спросил он, что делается на соборе, Дионисий, рассказывая все обстоятельства, с сожалением и раскаянием исповедал перед ним и свой грех против праведника. Блаженный Евсевий сделал ему строгий выговор, но, увидев слезы сокрушения, начал утешать. «Я знаю, "--- сказал он, "--- чем поправить дело; может быть, из среды злочестивых имен изглажу твое рукописание».

Между тем арианские епископы, узнав о прибытии Евсевия, пригласили его в общее собрание и, показав судопроизводство, просили, чтобы и своей рукой подтвердил оное, поскольку имя столь великого святителя обещало всевозможный успех в их предприятии. Евсевий притворился, будто приемлет их сторону, и, читая имена подписавших епископов, вдруг сказал с досадой: «А где же приложить мне свою руку? Неужели ниже Дионисия? Вы сами говорите, что Сын Божий не может быть равен Богу Отцу\footnote{Это был злочестивый догмат арианства.}; зачем же моего сына (я не иначе почитаю Дионисия) мне предпочитаете?» Сколько ариане ни старались уговорить его, но Евсевий не хотел подписаться, пока не будет изглажено имя Дионисиево. Ариане с удовольствием согласились, и Дионисий своей рукой выскоблил оное, будто бы уступая место старшему, а сам желая подписаться ниже всех. Тогда Евсевий вдруг переменил голос и, воззрев на небо, сказал: «Благодарю Тебя, Господи, что от беззаконного сонмища избавил Дионисия! Благодарю Тебя, Господи, что возложил в сердце мое благочестивый умысел от вечного проклятия спасти имя его! Так, "--- обратившись к арианам, продолжал он, "--- я вашими беззакониями не оскверняюсь, и сыну моему Дионисию не попущу быть участником злобы вашей». Услышав это, посрамленные лжеепископы воспылали бешенством и, осыпав ругательствами архиерея Божия, возложили на него руки свои. Вместе с Дионисием он послан был в заточение, где от чрезвычайных утеснений скончался страдальчески.

\section{О том, что дарования должно употреблять единственно на пользу ближних}

Преподобный Макарий Египетский\footnote{Преподобный Макарий Великий, Египетский (†~390/391) жил в царствование Феодосия Великого, был собеседником Антония Великого; преставился 97~лет от рождения. Память его празднуется 19~января (1~февраля).} получил такую благодать от Бога, что на глас его ответствовали и мертвые, если требовала этого польза веры или счастье невинно страждущих. Вот пример.

В одном селении случилось убийство, которое по ложным подозрениям возложили на человека совершенно невинного. Бедный, будучи преследуем гонителями, в ужасе прибежал к келье преподобного Макария и тут был схвачен. Ни слезы, ни клятвы не могли уверить служителей правосудия в его невинности: сбежавшийся народ единогласно называл его убийцей. Святой Макарий вышел на этот шум и, узнав, в чем состоит дело, спросил, где погребен убиенный. И когда объявили место, он посреди народа пошел к его гробу. Там, преклонив колена и пролив к Небесам теплую молитву, праведник сказал к предстоящим: «Ныне явит Господь, этот ли человек соделал убийство». Потом велегласно воззвал убиенного по имени, и мертвец отозвался из гроба. «Верой Христовой повелеваю тебе, "--- сказал Макарий, "--- открой нам, этот ли, народом обвиняемый, умертвил тебя?» "--- «Не этот убийца мой, "--- отвечал громогласно мертвец из"=под земли, "--- не этот, которого осуждают». Народ изумился и от ужаса пал на землю. Потом, повергшись к стопам праведника, все начали умолять его, чтобы спросил у мертвеца, кто убил его, но святой Макарий отвечал: «Этого принять на себя не могу; довольно для меня избавить от напасти невинного, а предать суду виновника не мое дело».

Христиане! И мы имеем разные дарования, от которых ближние наши могут получить пользу и вред, смотря по тому, на что будут употреблять их. Возьмите же в пример святого Макария, который благодать Божию употребил на избавление невинного человека от погибели, но не хотел употребить оную на погубление даже преступника!

\section{О том, сколь любезны Богу добродетели семейственные}

Однажды преподобный Макарий, углубившись всем сердцем в молитву, услышал глас Небесный: «Макарий! Ты до сих пор еще не уподобился двум женщинам, которые живут в близлежащем граде». Старец взял жезл и пошел в город. Отыскав дом, постучался в двери, и немедленно вышла женщина, с великой радостью приняла праведника и ввела в комнаты, где встретила его другая женщина.

Тогда святой старец сказал им: «Единственно для вас я принял на себя великий труд; я пришел из дальней пустыни, чтоб узнать дела ваши; не таясь, откройтесь предо мной». "--- «Человек Божий! "--- отвечали на это со скромностью обе женщины. "--- Можно ли чего"=нибудь богоугодного требовать от тех, которые беспрестанно заняты домашними попечениями и должны исполнять обязанности супружества?» Но святой Макарий настоял, чтобы объявили ему, как они проводят жизнь свою. Тогда добродетельные хозяйки смиренно сказали ему: «Мы две снохи, супруги родных братьев; пятнадцать лет живем вместе и в это время ни одного досадного слова друг от друга не слышали. Не имеем детей, а если Господь даст их, то первое наше попечение будет молить Его, чтобы помог нам воспитать их в вере и благочестии. С рабами поступаем ласково. Неоднократно советовались между собой вступить в лик святых дев, но не могли испросить на то позволения у супругов. Чувствуя их к нам любовь, мы решились остаться с ними и служить им утешением. А чтобы жизнь наша хотя бы немного походила на жизнь святых пустынниц, мы положили на сердце своем избегать шумных бесед, чаще быть дома и заниматься хозяйством».

Выслушав это, святой Макарий сказал: «Поистине, Бог не смотрит, дева ли кто или супруга, инок или мирянин, но ищет только сердечного произволения на добрые дела: это произволение Господь приемлет и, взирая на оное, всякому ниспосылает Святого Духа, действующего и управляющего житием каждого, желающего спастись».

Супруги! Матери семейства! Возьмите в пример двух благочестивых жен "--- и тогда жребием вашим будет благоденствие в этой жизни и блаженство в будущем веке.

\section{Сила дружеского приветствия}

Однажды преподобный Макарий, идя с учеником своим на гору Нитрийскую, велел ему следовать несколько впереди. Инок, исполняя волю его, встретился с языческим жрецом, который казался весьма утомленным, и от излишней ревности сказал ему: «Куда идешь, демон?» Жрец рассердился и, ударив его жезлом своим столь сильно, что оставил полумертвого, пошел далее. Через несколько минут встретился с ним святой Макарий и сказал: «Будь здрав, любезный трудник, будь здрав!» Удивленный приветствием святого, жрец спросил у него: «За какое доброе дело желаешь мне здравия?» "--- «Я вижу, "--- отвечал Макарий, "--- что ты утомился». "--- «Твое приветствие меня поразило, "--- сказал тогда жрец, "--- и я познаю в тебе человека Божия, напротив, другой, злобный инок, встретившись со мной, жестоко оскорбил меня, и за это я поразил его до смерти». Преподобный Макарий догадался, что речь идет о его ученике, а жрец, обняв колена старческие, сказал: «Не отпущу тебя, доколе не сделаешь меня подобным тебе иноком». Они пошли далее на гору и, взяв полумертвого ученика, отнесли в церковь. Нитрийские старцы, увидев с преподобным Макарием жреца идольского, изумились. Вскоре служитель дьявола сделался служителем Иисуса, и множество язычников по этому случаю приняли христианство. После этого святой Макарий часто говорил братии, что злословие иногда и добрых людей делает злыми, а доброе слово и злых делает добрыми.

\section{Любовь к предметам ненавистным и ненависть к предметам достолюбезным}

В одно время преподобный Макарий, идя с братией, услышал мальчика, который так говорил своей матери: «Один богач любит меня, но я, не знаю почему, ненавижу его. А бедный человек ненавидит меня, однако я люблю его». Услышав это, Макарий изумился, потом задумался. «Что сделалось с тобой, авва?» "--- спросили у него братия. «Поистине, Господь наш богат, "--- отвечал Макарий, "--- и любит нас, а мы, безумные люди, не хотим любить Его, напротив того, враг наш, дух соблазна, и беден, и ненавидит нас, а мы, к несчастью нашему, любим его!»

\section{О том, сколь горек плод непослушания}

Когда пустынная и безмолвная келья преподобного Евфимия\footnote{Прп. Евфимий Великий родился в 377~г., скончался в 473~г. Память его празднуется 20~января (2~февраля).} по изволению Божескому обратилась в многолюдную Лавру, тогда эконом, закупив для потребы монастырской несколько рабочего скота, предложил брату Авксентию, родом из кочующих азиатцев, чтобы принял на себя обязанности пастуха, но инок отказался. К просьбе эконома присовокупили свою просьбу пресвитер обители и другие, уважаемые по благочестию, старцы, но Авксентий и тогда не послушался. Преподобный Евфимий, узнав о том, призвал непокорного инока и всемерно увещевал его, чтобы он, как человек, способный к скотоводству, послужил братии, но Авксентий отвечал: «Не могу исполнить желания братии, никак не могу, поскольку, будучи грешником, страшусь вне святой обители еще более согрешить». "--- «Мы будем за тебя молиться Богу, "--- сказал преподобный Евфимий, "--- чтобы послал Ангела Своего для сохранения тебя. Всевышний знает, что ты из"=за любви к Нему будешь работать братии. Вспомни, что и Сам Христос сказал о Себе: \textit{Не ищу воли Моея, но воли пославшего Мя Отца} (Ин. 5, 30)». Но Авксентий ожесточился и, сколько ни убеждал его святой Евфимий, не хотел ему покориться. Тогда великий старец разгневался и сказал ему: «Чадо! Мы советуем тебе то, что послужило бы тебе же на пользу, а ты не слушаешься, я душевно жалею о тебе и хотел бы простить тебя, но Бог…» В это мгновение Авксентий затрепетал, как беснующийся, и повергся на землю. Предстоявшие отцы ужаснулись и начали умолять Евфимия, чтобы он испросил ему от Бога прощение.

Долго святой старец не соглашался на просьбы их, не по жестокости сердца, но чтобы показать им, сколь непростителен грех непослушания; наконец, как бы смягчившись, он оградил его крестным знамением и исцелил от болезни. Авксентий пал пред ним на колена. «Послушание есть великая добродетель, "--- сказал ему преподобный Евфимий. "--- Послушания Бог требует паче, нежели жертвы; напротив того, непослушание есть источник смерти. \textit{Се, здрав} был еси: \textit{ктому не согрешай} (Ин. 5, 14), да не случится с тобой что"=либо горшее». Тогда Авксентий начал уже сам умолять великого отца и братию, чтобы никому, кроме него, не поручали должности пастушеской.

Дети! Слушайтесь отца и матери, слушайтесь родителей ваших, слушайтесь старших людей, дающих вам добрые советы. А когда возложена будет на вас должность общественная, без прекословия повинуйтесь начальству: без послушания к высшим не может благостоять ни одно семейство, ни одно государственное сословие.

\section{О том, сколь грешно с легкомыслием принимать слово Господне}

Два инока, Маро и Климатий, которым наскучила строгость монастырской жизни, сговорились в следующую ночь убежать, но богопротивное намерение было открыто. Преподобный Евфимий, призвав их, советовал оставить пагубные мысли. «Везде должно хранить себя от искушений, "--- сказал он. "--- Адам преступил заповедь Господню в раю, Иов и на гноище сохранил ее. Дело состоит только в том, чтобы не принимать в сердце свое злых помышлений: они"=то причиняют уныние и печаль или ненависть к этому месту и к жителям оного. Итак, будьте трезвы и остерегайтесь от козней соблазна, который внушает вам перейти на другое место. Древо, часто пересаживаемое, плода не приносит; так и человек, который часто переменяет место или состояние, бесполезен для себя и для общества. Хочет ли кто быть добродетельным, но чувствует в себе к тому слабость, да не мыслит, чтобы сделался он способнее к тому на другом месте.

Добродетель усовершается не местом, но изволением делать добро. Это подтверждается опытом. Один египетский инок, имея очень сварливый нрав, не мог ужиться на одном месте, хотя сам был виновен, но обычно всегда обвинял других. Наконец он вышел из монастыря и поселился в пустыне один "--- в тех мыслях, что некому будет раздражать его. Что же вышло? В одно время у него опрокинулся кувшин с водой; старец рассердился и в другой раз наполнил оный; но, как нарочно, этот кувшин опрокинулся и в другой раз. Инок вышел из терпения и, схватив кувшин, ударил о землю». Услышав последние слова, Климатий рассмеялся. Тогда преподобный Евфимий, бросив на него значительный взгляд, сказал: «Сын мой! Смех твой доказывает, что дух"=соблазнитель совершенно овладел тобой; неужели не слышал ты, что Господь без ума смеющихся называет окаянными, а плачущих ублажает?» "--- и, оставив Климатия, он удалился в другую келью. Тогда на кощунствующего инока напал непостижимый ужас. Только молитва преподобного Евфимия и знамение креста могли исцелить его. «С этого времени будь осторожен, "--- сказал святой отец, "--- и внимателен к словам отеческим, а еще более к словам Господним; будь, так сказать, весь око, как описывают нам Херувимов, ибо ходишь посреди сетей».

Какой Маро, какой Климатий после столь видимого наказания Господня не исправится? Христиане! Мы слышим каждодневно Слово Божие, в храмах проповедуемое, и "--- ах! С каким чувствованием принимаем оное? Иногда хвалим только красноречие проповедника; чаще того "--- желаем, чтобы скорее кончилась проповедь; а иногда кощунствуем над строгостью мыслей Евангельских. Устрашимся, да меч гнева Господня \textit{не пояст ны} (Ис. 1, 20)… Мы часто слышим опытных и добродетельных старцев, беседующих между собой о делах веры и общежития, наставляющих юношей любви к отечеству, приверженности к благочестивым обыкновениям предков, "--- и зеваем от скуки или с насмешкой от них удаляемся. Убоимся, да не будем некогда сожалеть, что, имея уши, не слышали.

\section{Благодарность новопросвещенных к Евфимию Великому}

Некоторый князь, происходивший от древнего и знаменитого рода, верой огнепоклонник, именем Аспевет\footnote{Этот Аспевет прежде был военачальником в Персии, но, не исполнив повелений царя Издигерда, воздвигшего гонение на христиан, волхвами был пред ним оклеветан; убоясь казни, он бежал в Грецию.}, поставленный старейшиной над сарацинскими племенами, которые жили под Греческой державой, имел юного сына, которого дух"=губитель поразил неисцельным недугом, так что Тревон (имя Аспеветова сына) имел половину тела, от головы до ног иссохшую и совершенно омертвелую. Сколько отец ни старался подать помощь сыну, но все средства врачей, все чары персидских мудрецов были тщетны\footnote{Этот недуг Тревон получил, еще будучи в Персии.}.

В одну ночь юный Тревон, лежа на одре болезни и рассуждая о бедственном состоянии, в котором он находился, от чрезмерной горести воскликнул: «Вот что значит эллинское и персидское врачебное искусство! Вот что значат волшебства и заклинания волхвов! Сколько принес отец мой жертв в капища богов! Сколько советовался со звездочетами! Но все оканчивалось обманом, достойным посмеяния. Кроме Истинного Бога, никто исцелить не может». Размышляя таким образом, юноша начал молиться, призывая имя Иисуса Христа, о Котором часто имел случай слышать от тамошних христиан, и обещаясь немедленно креститься, когда Господь дарует ему здравие.

После этого Тревон неприметно погрузился в сон и увидел инока с длинной бородой и седыми волосами, который спросил: чем страждет он? Тревон объявил все, и старец опять спросил у него: «Исполнишь ли обет, данный тобой Богу?» Юноша немедленно поклялся. Тогда старец сказал ему: «Я "--- Евфимий, живу в пустыне недалеко от пути, ведущего к Иерихону; если хочешь быть здрав, приди, и Бог чрез меня исцелит тебя». От сильного движения, которое произвел столь радостный сон, юноша проснулся и, что видел, рассказал отцу своему.

Аспевет, с нетерпением дождавшись утра, взял отрока и, сопровождаемый сродниками, слугами и всем домом, отправился в путь. Увидев толикое множество варваров, сначала иноки ужаснулись, но Аспевет воскликнул к ним: «Слезно просим вас, покажите нам чудного и Божественного врача вашего, которого Сам Бог открыл сыну моему в сновидении». Когда приблизились к кельям и Тревон рассказал все обстоятельства Небесного откровения, тогда старец вышел из уединенной и безмолвной кельи; видя действующий тут перст Божий, он излил пред Богом теплую молитву за отрока и, оградив его крестным знамением, даровал совершенное здравие.

Пораженные этим чудом, варвары припали к стопам Евфимия; их души столь же скоро исцелились от заблуждения, как тело Тревоново "--- от болезни. «Сочетай нас Христу, человек Божий, "--- вопияли они, "--- да сподобимся получить имя и часть избранных». Святой Евфимий внял молению их и, немедленно огласив, вскоре просветил их крещением. В одно время несколько душ умножили стадо Христово, а Марин, по матери дядя Тревонов, так был проникнут благодатью Небесной, что не захотел возвратиться в дом свой и, приняв иночество, при святом Евфимии остался навсегда. Он жил столь примерно, что, наконец, был сделан настоятелем.

Но ревность новопросвещенных этим не ограничилась. Аспевет, при святой купели нареченный Петром, через некоторое время собрав множество агарян с их женами, детьми и рабами, привел к преподобному Евфимию, чтобы он наставил их на путь спасения: этот истинный христианин хотел, чтобы и все наслаждались тем благом, которого он сам сделался причастником. Святой Евфимий был восхищен их Небесным желанием. Петр, видя, что старец Божий не имеет кельи и обитает в тесной хижине, призвал каменоздателей и соорудил для него церковь и дом для жительства. Так новопросвещенные чувствовали величие дара Духа Святого! Их благодарность, хотя несоразмерная благодеянию, по крайней мере, показывает признательность их и благоговение к угоднику Божию.

Между тем чада Агари, соделавшиеся сынами Сарры, желая всегда слышать душеспасительные поучения из уст Евфимия, просили у него позволения жить близ него. Человек Божий, любя безмолвие, хотя не согласился на их желание, однако назначил им другое место, удобное для поселения. Велел построить дома и церковь, дал священнослужителей и сам часто посещал духовную паству. Вскоре от приходивших туда и принимавших Святое Крещение агарян малое селение сделалось столь великим городом, что преподобный Евфимий почел за нужное быть тут епископу, и Ювеналий, патриарх Иерусалимский, по его предложению, вручил жезл пастырский Петру, называвшемуся некогда Аспеветом. Сколько потребно благодати Божией, чтобы из новопросвещенных идолопоклонников соделать великих поборников Православия!

Читатели! Святой Евфимий, обратив агарян и язычников в христианство, не соделал ли их и ревностными сынами отечества? Подвластные какой"=либо державе народы ничто так не соединяет, как вера. А вы, безбожники и вольнодумцы, не дерзайте говорить, что святые люди для государства не нужны, или, по крайней мере, не столько нужны, как искусные градоправители и храбрые воины.

\section{Отеческое наследство}

Преподобный Евфимий, предузнав отшествие свое к Богу, призвал к себе всех иноков и вместо наследства оставил им следующее завещание: «Отцы, братия и чада мои о Господе! Я отхожу от вас в жизнь вечную. Умоляю вас "--- сохраните заповеди мои, а особенно любовь: ибо что соль для пищи, то любовь для всех добродетелей, даже смирение, сколь ни свято и боголюбезно, без любви быть не может. Вы знаете, что и Сам Господь единственно из любви к нам смирился и бысть человек (см. Ин. 1, 6), смирение возносит нас наверх всех добродетелей, а любовь поддерживает и не допускает упасть с оного. В чистоте соблюдайте не только душу, но и тело: ибо как человек не может жить опрятно в доме смрадном и темном, так и в оскверненном теле не может быть душа чистая. Все уставы и законы обители храните: от этого зависит благосостояние каждого общества, духовного и гражданского. Несчастным по силе помогайте; если кто из братии борется с нечистыми помыслами, того наставляйте, утешайте, укрепляйте. Но вот последняя и Богу приятнейшая добродетель: врата монастырские для странников никогда не запирайте; пусть они будут открыты для каждого пришельца; пусть будет у вас все с ними общее: тогда изобильно прольется на вас благословение от Бога, и если я обрету дерзновение у моего Искупителя, то паче всего буду просить у Него той благодати, да всегда духом буду с вами». После этого, три дня пробыв в церкви в молитвах и бдении, человек Божий почил с миром.

Родители! Живите единственно для того, чтобы юную душу чад ваших обогатить примерами разных добродетелей; а если Господь воззовет вас к Себе, то завещание преподобного Евфимия да будет вашим завещанием. Живите всегда так, чтобы дети ваши, и в юности, и на \textit{седалищи старцев} (Пс. 106, 32), при каждом добром деле могли сказать: «Отец мой, мать моя так поступали, и их дух носится надо мной и внушает мне делать это».

\section{Видение святителя Григория Богослова еще в юности его}

Святой Григорий\footnote{Память свт. Григория Богослова, архиепископа Константинопольского (†~389), празднуется 25~января (7~февраля) и 30~января (12~февраля).}, по глубоким сведениям, в науке Божественной нареченный Богословом, имея благочестивых родителей, с юных лет успевал в добродетелях. Но не один пример имел влияние на душу Григория; следующее чудотворное сновидение еще более укрепило его на подвиги веры.

Однажды, заснув, благочестивый юноша увидел близ себя двух девиц, облеченных в белые одежды, лицом прекрасных, ростом и летами равных. Не золото, не жемчуг, не шелк блистали на них; не распущенные волосы, не пламень в очах составляли их прелесть: полотно было их одеждой; сквозь тонкие покрывала видимы были потупленные очи, ланиты, девической стыдливостью украшенные; уста розовые, которым молчание придавало что"=то святоподобное. Григорий почувствовал в сердце своем Ангельскую радость. Девицы также возлюбили его, и на вопрос, кто они и откуда пришли, первая назвала себя \textit{невинностью}, другая \textit{кротостью}. «Мы предстоим престолу Царя славы, Иисуса Христа, "--- продолжали они, "--- будь нам единомыслен; уподобь ум твой нашему уму, очи твои "--- нашим очам, тогда вознесем тебя на Небеса и поставим близ Троичного Света». Сказав это, они понеслись по воздуху на Небеса. Святой юноша провожал их очами, пока они не скрылись от него, и тогда же проснулся, ощущая в сердце своем неизреченную радость и веселие. С того времени святой Григорий ужасался паче смерти нарушить добродетели, им соименитые.

\section{Невидимая десница, отвратившая смертный удар от Григория Богослова}

Ариане, раскол которых господствовал на Востоке и Западе почти восемьдесят лет в царствование Феодосия Великого, не только лишились всех преимуществ, но и почитались не иначе, как заразой государства. Принужденные оставить все церкви в Царьграде, они потеряли средоточие, из которого столь удачно распространяли заблуждение. Их лжеепископ жил за городом; проповедь арианская умолкла, даже не позволено им было сооружать зданий, которые бы имели подобие храма. Это"=то и было причиной того, что некоторые из ариан, приписывая все несчастия свои Григорию, архиепископу Царьградскому, как первому поборнику Православия, вознамерились погубить его.

Они подкупили одного возмутительного, предприимчивого и дерзкого молодого человека, который взялся убить архиерея Божия в епископском доме. Быть близ него нетрудно было в то время, когда множество народа стекалось к нему для поздравления с благополучным успехом в делах, касающихся веры. Злодей с прочими посетителями вошел в комнаты архиерея, который, будучи тогда болен, сидел на постели, и, в то время как другие, засвидетельствовав усердие и почтение, выходили, остался один.

Не было удобнейшего времени для злодейства, но молодой человек вдруг задрожал и повергся к ногам святого Григория, рыдая и как будто испрашивая у него милости. Страх овладел им так, что он в этом положении пробыл несколько минут, не имея силы промолвить ни одного слова. Святитель, испугавшись столь нечаянного явления, наклонился, чтобы поднять его, и долго спрашивал, кто он и чего требует, но, услышав только беспорядочные слова, воплем и рыданием прерываемые, пришел в сожаление о нем и сам начал плакать.

Люди сбежались на шум и, не сумев уговорить молодого человека, чтобы он вышел вон, насильно вынесли его в переднюю. Там он пришел несколько в себя, воздел руки к небу и, исповедуя зловредный свой умысел, обнаруживал знаки величайшего сокрушения. После того повели его опять к епископу, и один из келейников в страхе и ужасе сказал: «Этот незнакомец есть убийца, хотевший погубить тебя, владыка святой! Один Господь мог привести его к раскаянию». Григорий велел ему к себе приблизиться и, обнимая его, с кротостью сказал: «Бог да помилует тебя, сын мой! И как Он сегодня спас мою жизнь, так должно и мне спасти твою жизнь. Все мздовоздаяние, которого от тебя требую, состоит в том, да оставишь ересь и попечешься о своем спасении». Этот поступок, пример кротости и милосердия, привел в удивление даже неприятелей Григориевых.

\section{Испытание самого себя}

Преподобный Зинон\footnote{Прп. Зинон, ученик свт. Василия Великого, жил в V~в. Память его празднуется 30~января (12~февраля).}, путешествуя по Палестине, утомился и сел подле овощной гряды, чтобы съесть кусок хлеба. Вдруг пришла ему мысль сорвать один огурец. «Это дело не важное!» "--- думал он, но в то же мгновение, убоясь сам себя, сказал: «Хищники осуждаются к наказанию; итак, Зинон, испытай сам себя, можешь ли перенести мучение!» После этого старец стал против зноя солнечного и в этом положении пробыл пять дней.

Наконец, когда почувствовал, что силы от жары совершенно, так сказать, иссохли, сделал сам себе приговор: Зинон не может сносить казни; следственно, для насыщения своего не должен похищать чужого.

\section{Скромность и самохвальство}

Египетский инок пришел посетить преподобного Зинона, подвизавшегося в Сирии, и осуждал пред ним свои помыслы. Старец удивился и сказал: «Египтяне утаивают добродетели, которые имеют, и беспрестанно обличают недостатки, которых не имеют. А сирийцы и греки выставляют добродетели, которых не имеют, и скрывают пороки, которые имеют».

\section{Почему установлен праздник трех святителей: Василия Великого, Григория Богослова и Иоанна Златоустого}

В царствование Алексея Комнина об этих трех святителях возникла распря: который из них был общеполезнее и боголюбезнее? Иные превозносили Василия Великого, называя его по уму и делам совершенным гением, человеком ангелонравным. Иные возвышали Иоанна Златоуста, нарицая его пастырем человеколюбивым, к слабости естества человеческого снисходительным и в слове сильнейшим, когда должно тронуть ожесточенного грешника; иные держали сторону Григория Богослова как пастыря, в науке Божественной всех превосходящего. От этого произошло разделение: одни назывались василианами, другие иоаннистами, третьи григорианами; и этим заблуждением был заражен, к удивлению, не столько простой народ, сколько ученые. Господь восхотел, чтобы великие святители сами утолили раздор, о них восставший; и они, сначала каждый по одному, а потом все вместе явились Иоанну, епископу Евхаитскому, в то время по учености и благочестию знаменитейшему, и единогласно сказали: «Мы имеем одно достоинство пред Богом; нет в нас противоположностей: каждый из нас в свое время, подвизаемый Духом Святым, проповедовал спасение; нет из нас ни первого, ни второго, ни третьего. Итак, скажи, чтобы православные люди раздор оставили; скажи им, что мы, как в жизни, так и по кончине, более всего стараемся о том, чтобы всех привести в единомыслие. Совокупи нас в один день и составь один праздник; а мы всем, нас поминающим, будем споспешествовать ко спасению». Сказав это, они, светом Небесным озаренные, вознеслись на Небо. Преподобный Иоанн всенародно объявил чудотворное видение и, поскольку память трех великих святителей: Василия Великого, Григория Богослова и Иоанна Златоуста "--- празднуется в январе, установил торжествовать оную вместе тридцатого числа января и тем примирил неблагоразумное разногласие христиан.

\section{Неусыпный страж веры Христовой}

Преподобный Исидор Пилусиотский\footnote{Прп. Исидор, нареченный Пилусиотским (†~ок. 436~-- 440), потому что иноческое житие восприял и подвизался на горе Пилусиотской, жил в царствование Феодосия Младшего. Память его празднуется 4~(17) февраля.}, высочайшим и таинственным учением согревая душу свою, старался просветить прочих. Священнонастоятель, в добродетелях совершенный, мудростью славный и всеми уважаемый, какую ревность имел он к вере Христовой, явствует из следующего письма.

«Я спрашиваю у тебя самого, "--- пишет святой Исидор к арианину Ферасию, "--- у тебя, который над нами ругается и хочет показаться проницательным и суровым судьей, скажи, что бы ты сделал, если бы государь, отдав под твое охранение город, поставил тебя на стенах оного? Неужели бы ты, видя супостата, устраивающего подкопы, чтобы взорвать твердыню и войти в город, не напряг всех усилий, не употребил всех орудий для отвращения погибели? Нет! Ты бы испытал все средства, чтобы защитить крепость и через то доказать свою верность и усердие государю… Не должно ли и нам, которых Бог поставил пастырями и воинами царства благодати, крепко стоять против Ария, не только воздвигшего брань на Церковь Христову, но и погубившего множество душ? Для этой"=то причины я презираю все бедствия и ничего более не желаю, как пострадать даже до смерти за Православие».

Но при столь необоримой ревности великий Пилусиот не был гонителем иноверцев. «Бог одарил человека совершенной волей, "--- говорит он в письме к епископу Аполлонию, "--- следственно, противно закону употреблять насилие, привлекая людей к Православию: итак, если хочешь просветить сущих во тьме, то единственное в этом случае средство "--- благие советы, непорочная жизнь, добрые дела и терпение».

Этот мудрый старец, чему ни учил, все подтверждал примером своих деяний. «Жизнь без слова, "--- говорит он, "--- более приносит пользы, нежели слово без жизни; ибо кто живет праведно, тот и в безмолвии споспешествует счастью других своим примером; а кто говорит, и более ничего, тот подает только соблазн; некоторые будут о нем сожалеть, а большая часть "--- смеяться. Но если слово и жизнь помогают друг другу, тогда видим образ истинной философии».

Один приходский священник жаловался на худое поведение духовных детей. Святой Исидор сказал ему: «Некоторые из людей ищут добродетели, но лениво идут по пути, ведущему к оной; другие не верят и тому, что есть добродетель. Итак, должно первых убедить, чтобы оставили беспечность и нерадение, а вторым доказать, что добродетель не есть вымысел».

Неусыпный поборник веры, образ жития добродетельного, святой Исидор нетрепетно (без страха) защищал Иоанна Златоуста пред Аркадием и, после страдальческой его кончины, просил святого Кирилла Александрийского, чтобы тот причислил его к лику святых исповедников; подал совет Феодосию Второму собрать Третий Вселенский Собор в Эфесе. Наконец, исполнив вселенную богомудрыми правилами, угодник Божий почил с миром.

\section{Чувствования святой Агафии в темнице}

Святая Агафия\footnote{Св. мученица Агафия пострадала в 251~г. Память ее празднуется 5~(18) февраля.}, претерпев мучения, какие только могли изобрести против христиан тиранство и ненависть, была ввержена в глубокую и смрадную темницу. В полночь внезапно отверзлись сами собой двери, возблистал неизреченный свет, и явился пред ней старец, который предложил ей целебные травы. Святая девица, думая, что это какой"=нибудь врач, решительно сказала ему: «От всего сердца благодарю тебя, добрый человек, за милосердие, но не могу воспользоваться твоим благодеянием: я уповаю на Единого Господа моего Иисуса Христа, Который, если восхощет, в одно мгновение может исцелить меня». Обрадованный столь великой верой, старец улыбнулся и сказал ей: «Аз есмь апостол Петр: вижду убо тебя испиленною» "--- и после этого стал невидим… Святая мученица поверглась на колена. «Благодарю Тебя, Господи Боже мой, "--- воскликнула она, "--- что помянул меня, рабу Твою» "--- и, воззрев на свое тело, видела себя совершенно здравой. Между тем Небесный свет и без апостола всю ночь озарял темницу ее. Устрашенные стражи разбежались; темница осталась незапертой. Тогда прочие узники, обрадовавшись столь неожиданному случаю, один за другим начали выходить и святой Агафии говорили: «Двери открыты, стерегущих нет: беги от мучений». "--- «Не дай мне, Господи, учинить это, "--- отвечала им страдалица. "--- Не дай, Господи, чтобы за меня претерпели что"=нибудь стражи темничные! Имея помощником моим Господа, исцелившего рабу Свою, пребуду до конца в Его исповедании». После чего с ангельским спокойствием она осталась в темнице, не страшась уготовляемых ей наутро мучений и смерти.

\section{Необоримая Заступница}

Преподобный Палладий Александрийский однажды рассказал посетителям своим следующую повесть. Один благочестивый и странноприимный человек имел супругу, кроткую и богобоязненную, и шестилетнюю дочь, которую воспитывал в благочестии и страхе Божием. В одно время по купеческим делам должно было ему из Александрии отправиться в Царьград. Супруга, провожая его на корабль, между прочим, спросила: «Кому на руки оставляешь меня и дочь твою?» "--- «Госпоже нашей Богородице», "--- отвечал истинный отец семейства и пустился по водам.

Вскоре после того, когда добродетельная мать, сидя с дочерью, занималась домашним рукоделием, ненавистник рода человеческого научил слугу их убить мать и дочь и, похитив лучшие вещи, бежать. Злодей взял из поварни нож и пошел в горницу, но, едва коснулся порога, вдруг поражен был слепотой и так помрачился в рассудке, что не мог двинуться ни назад, ни вперед. Долго борясь сам с собой и силясь войти, наконец начал он кликать госпожу свою, но она, удивившись столь необыкновенному поступку и в то же время не подозревая тут никакого зла, сказала ему, чтобы подошел к ней сам. Слуга просил, умолял ее, чтобы приблизилась к нему или, по крайней мере, послала дочь свою, но она, назвав его безумцем, замолчала. Тогда злодей, будучи не в силах услужить дьяволу, в отчаянии выхватил из"=за пазухи нож и поразил сам себя. На крик госпожи немедленно сбежались соседи, пришла стража и, застав злодея еще в живых, узнали, что и как случилось. Все прославили Бога, творящего непостижимые чудеса, и Заступницу верных, Приснодеву Марию.

\section{Молитва о счастии врагов}

Святая девица Фавста\footnote{Св. мученица Фавста жила в III~в. Память ее празднуется 6~(19) февраля.}, дочь верных и благородных родителей, уроженка кизическая, на тринадцатом году возраста своего отдана была за веру Христову в мучительские руки жрецу Евиласию. Но ее невинность, терпение и кротость так подействовали на зверское сердце идолослужителя, что он, познав силу Божию и умилившись над Фавстой, даровал ей свободу, за что сам восприял венец мученический.

После того император Максимиан повелел другому лицу истощить над Фавстой весь ужас мучений. Что же и тогда сделала Богом укрепляемая девица? Просила ли на своего тирана гнева Небесного? Нет, она восклицала только: «Благодарю Тебя, Сладчайший Иисусе, Свидетелю сердец, что сподобил меня исповедать святое имя Твое! Еще молюсь Тебе, возлюби Максима (имя ее мучителя), просвети его верой и утверди в страхе Твоем». Сколько тиран ни был глух к гласу невинности и к чудесам, которыми Бог прославил страдальцев, но ангельское желание святой Фавсты поразило каменное сердце его. «Помощник и Покровитель невинной девицы! "--- наконец воскликнул он. "--- Яви Твою благодать и на мне, недостойном, прими меня в раба Твоего, воистину, Ты Един Господь и Бог сил!» В это мгновение отверзлись небеса, и явился на облаках посреди Ангелов и всех святых Сын Божий, благословляющий новообратившегося Максима.

Восхищенный небесным видением, исполнитель царского веления последовал Евиласию и Фавсте и вместе с ними крестился \textit{своею кровью}. А вы, христиане, познайте из этого, какую силу и над злодеями нашими имеет молитва об их счастье.

\section{О том, сколь спасительно не слушать клеветы\footnote{Из Пролога, в 10"~й день марта.}}

Преподобный Марк Египетский, подвизаясь в молитве и постничестве тридцать лет, не выходил из кельи. Братия уважали его, а пресвитер в известные дни приобщал его Святых Таин на дому. Но дьявол, завидуя добродетелям старца, употреблял всевозможную хитрость, чтобы соблазнить его на грех осуждения. Он вселился в одного грешника и привел его к старцу под видом, будто просить благословения. Недугующий, только переступил порог кельи, вдруг закричал: «Твой пресвитер "--- великий грешник! Не позволяй ему приступать к тебе». Но Богом вдохновенный муж сказал на то: «Чадо, \textit{не суди, да не судим будеши} (Мф. 7, 1). Если он и грешник, Господь исцелит его, написано бо есть: \textit{молитеся друг за друга, яко да исцелеете} (Иак. 5, 16)». После чего, сотворив молитву, старец изгнал нечистого духа и пришельцу даровал здравие.

На следующий день пришел к нему пресвитер, и старец принял его с радостью. Тогда милосердый Бог, благословляя непорочность его, показал знамение: ибо, когда священник начал подавать ему Святое Причастие, преподобный Марк увидел Ангела, сходящего с небес и возлагающего десницу на главу пресвитера. Марк ужаснулся, а Ангел Господень сказал ему: «И земной царь не терпит, чтобы вельможи предстояли ему в нечистоте; кольми паче могущество пренебесное». Таким образом великий подвижник Христов Марк удостоен был благодати Божией за то, что не послушал клеветы и не осудил священнослужителя.

\section{Сила истинной веры}

Когда святая мученица Дорофея\footnote{Память святых мучениц Дорофеи, Христины, Каллисты (†~300) празднуется 6~(19) февраля.} за исповедание христианской веры представлена была на суд к Саприкию, градоначальнику кесарийскому, то мучитель, видя юность ее, сначала не хотел употребить орудий казни, но отдал ее для искушения и соблазна двум сестрам, Каллисте и Христине, которые прежде были христианками, но, устрашившись мучений, предались идолопоклонству и, поскольку такие люди почти всегда бывают отчаянны, жили развратно.

Эти"=то женщины, взяв святую Дорофею в дом свой, начали уговаривать ее, чтобы она поклонилась идолам. «Покорись воле правительства, "--- говорили они, "--- и тем избавь себя от погибели, мы сделали разумно, что наконец опомнились от христианского исступления. Рассуди и ты, какая польза умереть посреди мучений». Но святая девица отвечала им: «О, если бы и вы покорились воле Иисуса Христа и опять обратились к Нему! Вы бы тогда спасли себя от мук гееннских». "--- «Раз мы навсегда решились служить греху, "--- отвечали ей пораженные гласом веры и христианского усердия соблазнительницы, "--- то можем ли возвратиться на путь истины и обрести благодать Божию?» "--- «Напрасно так думаете, "--- сказала им святая Дорофея, "--- отчаяться в милосердии Господнем есть грех ужаснейший, нежели поклониться идолам. Небесный Врач всегда может исцелить раны нашего сердца. Он для того и Спасителем нарицается, что спасает. Покайтесь, и Господь простит вас». Тогда обе сестры припали к стопам юной мученицы и, рыдая, просили ее, чтобы помолилась о них Ходатаю мира.

Когда после Саприкий спросил у них об успехе своего поручения, то Христина и Каллиста в один голос сказали ему: «Великое преступление сделали и мы, что не имели твердости Дорофеиной и убоялись угроз твоих! Теперь Господь простил грехи наши, жизнь наша в твоих руках. Но не предадим никому, кроме Христа, душ наших». Изумленный тиран долго стоял безмолвным, потом, воспылав яростью, осудил их на одну смерть с Дорофеей.

И мы из святой купели выходим христианами, но, к несчастью нашему, поступаем хуже Каллисты и Христины, ибо они, убоявшись ужаснейших мук, поклонились идолам. Мы же, часто из одного суетного благоприличия, а чаще того "--- нарушая и самое благоприличие света, идолопоклонству ем страстям, \textit{на душу воюющим…} О, если бы мы послушались гласа, вопиющего в пустыне: \textit{покайтеся, приближибося Царствие Небесное!} (Мф. 3, 2), тогда с тремя мученицами, днесь воспоминаемыми, мы умерли бы греху, да оживем Богови.

\section{Болезнь и исцеление сребролюбца}

Некогда святитель Христов Парфений\footnote{Прп. Парфений, епископ Лампсакийский (†~IV~в.), жил в царствование Константина Великого. Память его празднуется 7~(20) февраля.} посетил Ираклийского архиепископа Ипатиана, который тогда был весьма болен. Когда угодник Божий, проведя вечер в разговорах о естественных причинах болезни и о средствах исцелить оную, лег спать, Господь в сонном видении открыл ему, что Ипатиан наказан за сребролюбие и насилие над бедными.

Удивленный Парфений поутру опять пришел к архиепископу и сказал ему: «Владыка! Ты страждешь душевной болезнью. Отряси ее "--- и здрав будешь». Ипатиан ужаснулся и, видя в Парфении пророка Божия, сквозь слезы отвечал: «Исповедую грехи мои, помолись обо мне, человек по сердцу Божиему, да Господь поможет мне очиститься от беззаконий моих». Тогда святой Парфений сказал ему: «Есть грехи, которые могут очиститься одной молитвой, но твои грехи другого рода: ты в лице нищих отнимал у Бога имущество. Возврати Ему также в лице нищих "--- и будешь здрав душой и телом».

Тогда архиепископ совершенно очувствовался. «Господи! Согрешил пред Тобой, "--- воскликнул он, "--- но Ты сколько праведен, столько и милосерд». Потом, призвав домоуправителя, велел принести к себе все неправедно собранное и усердно просил святого Парфения, чтобы принял на себя труд расточить его имение бедным, но человек Божий отказался. «Ты должен раздать сам, "--- сказал он, "--- если помнишь, у кого отнял, тому и возврати с лихвой. Если не помнишь "--- отдай наиболее нуждающимся».

Ипатиан последовал совету праведника, вскоре получил здравие и с того времени не прилеплял к сокровищам сердца своего.

\section{Сила родительских молитв}

Преподобный Лука\footnote{Память прп. Луки Елладского (†~ок. 946) празднуется 7~(20) февраля.}, уроженец элладский, от юных лет возлюбив иноческие подвиги, ушел тайно от своей матери и в одном из афинских монастырей постригся в новоначалие.

Евфросиния (имя его матери), зная благонравие сына, хотя и догадывалась о святом его поступке, но, будучи вдовой, не могла переносить разлуки и беспрестанно жаловалась Богу: «Господи! Ты свидетель вдовства моего и всех огорчений, с ним соединенных, не забудь меня до конца, не отними у меня единственную отраду жизни, любезного сына. Ах! Я не подала ему ни малейшей причины оставить меня… Господи! Ты ведаешь мое с ним обхождение: я не запрещала ему жить по закону Твоему, я была ему матерью не только по плоти, но и по душе, хотела видеть его совершенным в добродетелях. Господи! Не презри слез моих, возврати мне сына, да прославлю имя Твое святое». Так молилась осиротевшая мать, и Господь услышал молитву ее.

В одно время игумен обители, в которой подвизался святой Лука, узрел в сонном видении плачущую мать. «Зачем меня, бедную вдову, так жестоко обижаешь? "--- вопияла она к настоятелю. "--- Зачем отнял у меня единородного сына, утеху моей старости? Это единый \textit{свет очию моею, и той несть со мною}! (Пс. 37, 11). Возврати мне сына, или не перестану жаловаться на тебя к Богу». Устрашенный игумен пробудился, долго размышлял о сновидении и, наконец, почел за пустое мечтание. Но в следующую ночь увидел то же; в третью ночь услышал тот же вопль сетующей матери. Тогда познал он, что это не есть действие духа"=искусителя, но явление, Богом устрояемое. Долго рассуждал о каждом иноке порознь и, поскольку святой Лука всегда скрывал от него и от всей братии свой род и отечество, наконец, уверился, что сновидение относится к нему.

На следующий день, когда, совершив Божественную службу, братия отобедали, он призывает к себе Луку и говорит с гневом: «Как осмелился ты обмануть нас, уверив, что не имеешь родителей? Как дерзнул жить с нами, имея обман на сердце твоем? Знаешь ли, кто начальник лжи?.. Уйди от нас; уйди даже из пределов афинских и возвратись к твоей родительнице. Уже третью ночь она укоряет и смущает меня!» Пораженный, Лука стоял недвижно. Потупив глаза свои, не смея выговорить ни единого слова; он плакал, не желая оставить святую дружину. Тогда настоятель сжалился над ним и сказал с кротостью: «Сын мой! Теперь непременно возвратись к твоей матери, а после Сам Господь поможет тебе в добром намерении. Послушайся: я вижу, что ее молитва имеет великую силу у Бога и может уничтожить все твои молитвы. Иди с миром и помни, что без благословения родительского ничего предпринимать не должно». Выслушав это, святой Лука в молчании поклонился настоятелю, простился с братией и отправился в путь свой.

Кто может описать удивление и радость Евфросинии? Она бросилась обнять его, но вдруг остановилась, возвела очи и руки к небу и воздала благодарение Богу. «Благословляю Тебя, Господи, "--- сказала она, "--- что не оставил молитвы моей». Потом облобызала сына своего.

Святой Лука пробыл у матери четыре месяца, но, стремясь сердцем и душой к Богу и житию безмолвному, начал просить у нее позволения удалиться в пустыню. Богобоязненная Евфросиния благословила его, ибо знала она, что Бога должно почитать и любить более, нежели родителей, и ее молитвы для праведного Луки были вторым вождем на пути спасения.

\section{Мученик, защищающий мучителя}

Когда злочестивый Лициний, истощив все мучения на святого Феодора Стратилата\footnote{Св. великомученик Феодор Стратилат (†~319), родом из Евхаит, был воинский начальник. Память его празднуется 8~(21) февраля.}, изрек ему смертный приговор, тогда для предупреждения мятежа со стороны христиан, которые "--- как думал он "--- будут защищать страдальца, отряжен был воинский начальник, именем Сикст, с тремястами воинов. Но все они, увидев чудеса, совершаемые Феодором, уверовали в Господа нашего Иисуса Христа. К ним присоединилось множество народа; повсюду гремели восклицания: «Един есть Бог, Бог христианский, и нет Бога, кроме Него! Не хотим повиноваться тому, кто не повинуется Иисусу Христу». Смятение ежеминутно усиливалось; один из язычников бросился с кинжалом на Стратилата, но Сикст отразил удар и поверг его без головы; другой идолопоклонник поразил Сикста. Должно было ожидать величайшего кровопролития. Но кто из святых мучеников употреблял преступления других в свою пользу?.. И Феодор воззвал велегласно: «Возлюбленные о Христе братия! Перестаньте защищать меня; когда распинали Господа моего Иисуса Христа, Он не восхотел, чтобы Ангелы творили отмщение убийцам Его. Не воздвигайте брани на Лициния: хотя он, по своему жестокосердию, восхотел быть слугой гееннского могущества, но вам как христианам должно внимать гласу Господню, возвестившему, что \textit{несть власть, аще не от Бога} (Рим. 13, 1), вам должно в царе своем почитать слугу Божия». Этим и тому подобным образом уговаривая народ, святой великомученик преклонил под меч главу свою, а мятущийся народ, почитая совет праведника не иначе, как заповедью Господней, облегчил горесть свою торжественным погребением тела его, чему не дерзнул воспрепятствовать и мучитель Лициний.

\section{Дух клеветы}

Когда святитель Христов Харлампий\footnote{Священномученик Харлампий пострадал в 202~г. Память его празднуется 10~(23) февраля.} из Магнезии веден был под стражей в Антиохию к императору Северу, который хотел лично допросить его, тогда дух злобы, преобразившись в старца, предстал тирану и, сетуя о своем несчастии, сказал: «Повелитель вселенной! Я царь скифский, но теперь скитаюсь, как последний из рабов; горестна участь моя, но всего тягостнее для меня, что человек низкий и ничтожный был причиной моего злосчастия… Некто Харлампий, великий волшебник, взбунтовал мое воинство и возмутил против меня весь народ. Всеми оставленный, я прошу твоего покровительства; и если благоволишь принять советы изгнанника, берегись опаснейшего из чародеев: имя какого"=то Иисуса в устах его имеет такую силу, которая всех привязывает к нему до безумия». Еще дух клеветы не кончил речи, как святого Харлампия представили к императору. Демон, будто пораженный внезапным ужасом, отступил далее, а Север затрепетал от ярости. Нельзя исчислить мучений, которые человек Божий должен был вытерпеть!

Христиане! Такова сила и хитрость каждой порочной страсти! Они овладевают нашим сердцем, захватывают над ним всю власть, так что буйный рев их заглушает не только совесть, разум и веру, но и голос собственной нашей пользы, хотя люди всего более к ней привязаны. Вспомните: не властолюбие ли Ирода было причиной избиения вифлеемских младенцев? Не мщение ли Иезавели истребило столько пророков? Не зависть ли Каина соделала первое братоубийство? Всякая буйная страсть есть не что иное, как тот самый демон, который пред Севером оклеветал невинного Харлампия.

\section{Любовь к отечеству посреди лютых тиранств}

Святой Харлампий, после разных продолжительных страданий осужденный на отсечение главы, придя на место казни, начал молиться: «Помяни, мя, Господи во Царствии Твоем!» И посреди молитвы своей узрел на облаках в неизреченной славе Сына Божия и услышал глас Его: «Прииди, друже Мой, столько имене Моего ради претерпевши: проси у Мене, чего хощеши, и дам тебе». Тогда мученик Христов в Небесном восхищении воскликнул: «Господи! Несказано и это благодеяние Твое, что сподобил меня узреть славу Твою! Но, Боже мой, если угодно будет святой воле Твоей, даждь славу имени Твоему, да в отечестве моем, где я родился и где почиют кости мои, не будет ни глада, ни тлетворного ветра; да цветет вечный мир и всех плодов изобилие; да царствует в душе моих соотечественников любовь к порядку и благочестие; да водворится ревность о спасении! Господи, Ты ведаешь слабость человеческую: остави им грехи их, излей на всех благодать Твою, да прославят Тебя, Единого Истинного Бога, всех благ Подателя». Сказав это, святой старец услышал глас Господень: «Буди по твоему прошению» "--- и, не дождавшись взмаха смертной секиры, испустил в руце Христовы дух свой.

Юноши, обучаемые по творениям мудрецов света и по примерам великих людей древности! Вы превозносите Аристида, который, выходя из Афин, простер руки свои к небу и молился, чтобы отечество не возымело в нем, для избавления своего от опасности, когда"=либо нужды. И желание Аристида похвально! Но молитва святого Харлампия заставляет забывать всех страдальцев языческих. Сколько Евангелие чище и возвышенней умствований человеческих, столько глас святителя Христова чище и возвышенней гласа Аристидова.

\section{Вражда и дружество великого князя с иноком}

Преподобный Прохор\footnote{Память прп. Прохора Печерского (†~1107) празднуется в 10~(23) февраля.}, уроженец смоленский, пришел в Киев при великом князе Святополке Изяславиче и принял образ ангельский в Киево"=Печерской Лавре от игумена Иоанна. Он столько угодил Богу, что удостоился дара чудотворений.

Вскоре после этого настало голодное время от неурожая, а более от того, что неустройство и грабежи препятствовали привозить в Киев хлеб и соль из Галича и Перемышля. Жители отчаивались в жизни; повсюду были слышны плач и рыдание; бедные умирали на распутьях; к несчастью, если и было сколько"=нибудь хлеба и соли, то и это небольшое количество находилось у жестокосердых откупщиков, которые не могли насытиться золотом. Тогда"=то блаженный Прохор доказал, сколь человеколюбиво сердце его и сколько любит его Господь, всея твари Питатель. Чудотворец, своими молитвами претворяя горькую лебеду в здоровый и вкусный хлеб (что и прежде делал, только сам для себя) и пепел превращая в соль, раздавал бедным гражданам: келья его от утра до ночи окружена была множеством народа.

Вот благодеяние, которое верховная власть должна бы увенчать славой и благословением! Но с преподобным Прохором случилось противное. Соляные откупщики, видя подрыв своему лихоимству, так жестоко оклеветали его пред Святополком, что он вместе с чудотворцем возненавидел и всю Печерскую Лавру. Не верил чудесам блаженного Прохора, начал гнать игумена Иоанна как начальника ненавистной ему братии и обо всей Лавре думал не иначе, как о таком месте, которое старается сделать подрыв казне государственной. Но истинная добродетель рано или поздно всегда восторжествует: ибо если Святополк и слушал наветы откупщиков, то, кажется, только потому, что на военные издержки имел нужду в их деньгах. Великий князь наконец переменился и так возлюбил святого инока, что поклялся ему торжественно впредь не делать обиды ни одному из ближних. Этого еще мало: князь и отшельник дали друг другу обещание, чтобы тот, кто прежде умрет, непременно был несен к месту погребения на раменах\footnote{\textit{Рамени} "--- плечи.} другого.

Через некоторое время великий князь отправился на войну против половцев. Между тем святой Прохор разболелся и, когда почувствовал конец жизни, написал к Святополку: «Государь! Если хочешь исполнить твое обещание, то приди в Киев. Не чести и славы желая, об этом напоминаю тебе: я страшусь, что, нарушив клятву, ты не так благополучно кончишь войну против неверных». Прочитав это, Святополк немедленно оставил поле брани. Столь крепко держались своего слова наши предки! Хотя великий князь знал, что его присутствие нужно для воинства, но, с другой стороны, уверен был, что молитвы праведника воодушевят оное сильнее его присутствия.

Святой Прохор, поучив Святополка вере и всем добродетелям, простился с ним и, возведя очи свои к небу, предал дух свой в руце Божии. Великий князь с иноками положил в гроб тело его и внес в пещеру. Потом, в надежде на помощь Божию, пошел на войну и получил славную победу над неверными.

После этого великий князь, когда отправлялся против неприятелей отечества, всегда приходил в Печерскую обитель и, помолившись Пресвятой Богородице и усопшим отцам, предпринимал дело свое.

\section{Исповедник Святой Троицы}

Святой Мелетий\footnote{Память свт. Мелетия, архиепископа Антиохийского (†~381), празднуется 12~(25) февраля.}, епископ Севастийский, возведен был на престол Антиохийского архиепископства с общего согласия православных и ариан.

Трудно угадать, почему те и другие почитали его своим сообщником. Может быть, православные епископы знали его сердечные чувствования, а епископы арианские были обмануты терпимостью и даже снисхождением к арианству, которые человек Божий оказывал с тем же намерением, с каким защитники отечества иногда завлекают неприятеля в самое сердце оного.

Но поскольку архиерей Божий был не наемник, а истинный пастырь стада Христова, то через тридцать дней не только опять свергнут был с престола, но и заточен в Армению "--- что происходило следующим образом. Приняв жезл архипастырства Антиохийского, святой Мелетий прежде всего старался научить порученный ему народ добродетельному житию, благонравию и миролюбию, углаждая через то в сердцах их путь к Православию. Святой муж был уверен, что, не исторгнув терния пороков, невозможно ожидать плода от семени Евангельского, и до времени не обнаруживал своего вероисповедания.

Между тем граждане антиохийские, по большей части ариане, нетерпеливо желали узнать мысли архиепископа, поскольку неправда всегда подозрительна. Иногда через искусные и хитрые вопросы, иногда обнаруживая свое беспокойство, искушали его столь часто, что святой Мелетий, наконец, решился сказать правду. В один великий праздник, в соборной церкви, при многочисленном народе, он выходит на кафедру и начинает проповедь Слова Господня. Но как изумились ариане, когда архипастырь вдруг начал прославлять веру, утвержденную на Первом Никейском Соборе, всенародно исповедуя, что Сын соприсносущен Отцу, соестествен, равен, несоздан и Творец всех тварей, видимых и невидимых! Православные возрадовались неизреченно, ариане уныли. Но их уныние тогда же обратилось в столь необузданное бешенство, что архидиакон церковного причта, зараженного ересью, подошел к архиепископу и своей рукой дерзнул зажать уста его. Тогда святой простер к народу руку свою и, потрясая оной, громогласнее языка проповедовал Святую Троицу. Он показывал три перста, изображая три Божественные Лица, потом, пригнув два перста, одним показывал в трех лицах единое Божество.

Наглый архидиакон, оставив уста, схватил архиепископа за руку, тогда святой Мелетий начал прославлять языком и голосом Троицу в Единице и Единицу в Троице. Эта борьба между святым проповедником и злочестивым служителем продолжалась какое"=то время. Наконец ариане пришли в ярость и выгнали человека Божия с ругательствами из церкви.

Заметим из этого, что истинный христианин лучше согласится потерять все на свете, даже самую жизнь, нежели поступить против совести и веры.

\section{Слава святого Мелетия}

В то время как Римское государство и Святая Церковь приходили в расстройство, воцарился Феодосий Великий, кажется для того и рожденный, чтобы все восстановить и привести в порядок. Он был сын Феодосия же, военачальника Валентинианова, воспитанник Анатолия, одного из славнейших ученых мужей тогдашнего времени, и по отцу своему происходил от императора Траяна; предводительствовал войсками Грациановыми и оказал великие услуги государству. Этого"=то Феодосия Господь назначил скипетродержцем старого и нового Рима, а святого Мелетия, как славнейшего между епископами, избрал к тому, чтобы он известил храброго военачальника о чести, ему предназначенной: ибо, когда Феодосий однажды спал, явился ему епископ, возложил на главу его диадему и облек в порфиру. Проснувшийся Феодосий открыл сновидение свое одному из искренних друзей, который и уверял его, что это есть неложное предзнаменование величия, к которому Бог призывает его, а по описанию явившегося святителя, указал именно на святого Мелетия.

Это предвещание Господне, которого святой архиепископ был орудием, может быть, и сам не зная того, вскоре исполнилось. Ибо император Грациан, видя государство свое окруженным бесчисленным множеством варваров, принял Феодосия к соцарствованию, а по смерти Грациана и брата его младшего, Валентиниана, сделался он полным обладателем вселенной и восстановил мир в Церкви.

Этот мудрый и благочестивый император любил святого Мелетия, как своего отца, и почитал не иначе, как будто от него получил государство. Но великий святитель жил недолго и закончил многотрудную жизнь свою в Царьграде, куда прибыл на Второй Вселенский Собор. К несчастью Церкви, он умер тогда, когда Православие только начало воскресать и когда еще не совсем были кончены дела Собора, председателем и душой которого он был. Вся Восточная Церковь о нем восплакала; сетующий Феодосий приказал сделать ему погребение, которое можно назвать торжеством. Тело святого мужа перенесли в храм святых Апостолов, где воспеваемы были псалмы от нескольких ликов на разных языках; множество людей, стекаясь туда, приносили свечи и, прикладывая к лицу его полотенца, уносили их в дом как бесценные сокровища. Красноречивейшие витии\footnote{\textit{Витии} "--- ораторы.} архиереев, бывших на Соборе, говорили в честь его похвальные речи, описывая добродетели его и гонения, претерпенные за веру. По окончании всех почестей, которые надлежало оказать ему, Феодосий велел эти славные и драгоценные мощи перенести в Антиохию и показывать их повсюду, хотя этого в обыкновении у римлян и не было. Отовсюду стекались народы, чтобы сопровождать исповедника Господня с пением псалмов. Наконец, он был погребен в престольном граде своем, подле гроба святого мученика Вавилы, славнейшего епископа Антиохийского.

\section{О происхождении обряда готовить коливо в память святого великомученика Феодора Тирона\footnote{Память великомученика Феодора Тирона (†~306) празднуется 17~февраля (2~марта).}}

Император Юлиан всемерно старался истребить христианскую веру. Но поскольку имя философа всего более льстило его самолюбию, то богоотступник по большей части употреблял не суровые и бесчеловечные, а тайные и хитрые к тому средства. Сверх того, он знал из опытов, что тиранства не уменьшали, но умножали число мучеников и христиан.

Таким образом, издеваясь над верными, Юлиан умыслил осквернить их в первую седмицу Великого поста, когда христиане наиболее очищаются и благоговеют: он приказал царьградскому губернатору все съестные и питейные припасы, которые продаются на торжище, смесить с кровью от жертв идольских или, по крайней мере, окропить оной. Нечестивое повеление готовились исполнить. Но всевидящий Промысл, препинающий мудрых в коварстве и направляющий рабов своих, разрушил злонамерение отступника.

Ночью на чистый понедельник Царьградскому архиепископу, не в сонном видении, но наяву, велением Господним является воин в полном облачении, которого требует поле брани, и говорит ему: «Восстань и, немедленно собрав стадо Христово, строго заповедай, чтобы никто не покупал продающихся на торжище брашен: все они осквернены кровью от жертв идольских». Архипастырь усомнился и спросил явившегося чудотворца: «Ты кто еси?» "--- «Я "--- Феодор Тирон, "--- отвечал воин, "--- и послан к тебе от Иисуса Христа». "--- «Чем же пропитаются люди бедные, никаких запасов в доме не имеющие?» "--- с кротостью и благоговением возразил архиепископ. «Изготовь коливо (кутью или пшеницу, с медом приготовленную), "--- сказал Феодор, "--- и тем отврати недостаток убогих». С этим словом заступник христиан стал невидим.

Архиепископ в точности исполнил повеление святого великомученика. Богоотступник, подозревая придворных в открытии своего намерения, отменил приказание. С того времени и доныне люди Христовы в первую пятницу Великого поста обновляют память бывшего тогда чуда коливами, да незабвен будет столь милосердый Божий о нас Промысл и помощь святого великомученика Феодора Тирона.

\section{Покаяние грешницы}

Святая преподобномученица Евдокия\footnote{Святая преподобномученица Евдокия пострадала ок. 160~-- 170~гг. Память ее празднуется 1~(14) марта.}, родом и верой самарянка, обитала в Илиополе в царствование Траяна и была сколько прекрасна, столько обесславлена своим поведением; она собрала бесчисленные богатства, имела большое знакомство и жила весьма роскошно. Но Бог, \textit{не хотяй смерти грешника, но еже обратитися ему и живу быти} (ср. Иез. 33, 11), воззрел на погибающую девицу и исхитил ее от соблазнов следующим образом.

Инок, по имени Герман, из путешествия возвращался в обитель свою и, поскольку в Илиополе застигнут был ночью, решился тут остановиться до утра. Без сомнения, руководимый Небесным Промыслом, он избрал для ночлега тот же дом, где жила Евдокия, и с дозволения дворецкого занял небольшую каморку, которую от ее спальни отделяла только деревянная стена. Немного отдохнув, старец восстал на свою молитву и начал петь\footnote{То есть читать с напевом, что было тогда в употреблении.} обыкновенное «правило», по окончании оного сел и, вынув из"=за пазухи книжку, которую носил всегда с собой, читал почти до рассвета. К счастью грешницы, тут описывалось, какая слава ожидает праведных на Страшном Суде Христовом и какие мучения предуготовлены для беззаконников. Богобоязненный инок читал с чувством и часто прерывал чтение глубоким воздыханием.

На тот раз (и здесь действовала благодать Господня) Евдокия проснулась. Не подозревая, чтобы кто"=нибудь так близко подле нее находился, и вдруг услышав чтение старца, она изумилась и, как все Божественное для нее было совершенной новостью, старалась, единственно из любопытства, не проронить ни одного слова. Выражения: «грех», «строгий Судия», «ад», «вечное мучение», а с другой стороны: «праведник», «любовь Господня», «рай», «вечное блаженство» "--- глубоко потрясли душу ее.

В первый раз, как начала помнить себя, почувствовала она ужас своего положения. Невольным образом вздохнула, умилилась в душе своей и, приводя на память все дела свои, не могла уснуть всю ночь.

На рассвете следующего дня Евдокия призвала к себе инока и, скрывая сердечное смущение, спросила: «Скажи мне, добрый старец, кто ты? Для чего здесь ночевал? И что читал в прошедшую ночь? Ты меня удивил и напугал до крайности. Неужели правда, что грешники осуждены на огонь вечный? И если так, скажи мне, кто может спастись?» Мудрый пустынножитель на все отвечал ей удовлетворительно, не оскорбляя самолюбия, заставил признаться в слабостях. Описал гибельные плоды богатств, неправедно приобретаемых и на худое расточаемых, и, наконец, довел Евдокию до того, что она пала к его ногам и слезно просила, чтобы он научил ее, как угодить Богу.

Сколь великая бывает радость на Небеси и о едином кающемся грешнике, Господь показал на Евдокии. Обратившись на путь истинный, девица пребыла, по наставлению святого Германа, семь дней в молитвах и посте; в последнюю ночь, когда, простершись на земле, она оплакивала грехи свои, вдруг озарил ее свет Небесный. Евдокия, думая, что взошло солнце, встала; но как ужаснулась, увидев пред собой юношу прекрасного и вместе с тем грозного, в белую одежду облеченного, который, взяв ее за руку, восхитил на облака! Мгновенно очутились они посреди райского жилища. Бесчисленное множество святых душ, радуясь друг перед другом, встречали Евдокию и приветствовали с неизреченным счастьем быть невестой Жениха Предвечного. В это время в воздухе показалось страшилище: вид его был мрачен, как ночь; глаза горели, как угли; оно скрежетало зубами, рыкало, как лев, и, с наглостью порываясь на Евдокию, силилось исторгнуть ее от рук бесплотного вождя; укоряло его как похитителя рабы, единственно ему принадлежащей. Но, услышав глас, исходящий от света, на который и воззреть было невозможно: «Аз с ней буду во вся дни живота ее», "--- злобный дух исчез, а Ангел"=хранитель в ту же минуту поставил Евдокию в ее доме и, оградив крестным знамением, вознесся на Небо.

После этого видения Евдокия как бы разрешилась от уз плоти: все богатство через благочестивого епископа раздала нищим, просветилась крещением и вскоре удалилась в один из девических монастырей, где столько успела в добродетелях, что через одиннадцать месяцев поставлена была игуменией. Избранница Божия "--- сколько чудес соделала! Она смягчала сердца звероподобных тиранов. Воскрешала мертвых, ею же, в предупреждение злодейств, единым дуновением убиенных. Внезапно поражала молнией мучителей христианства; крестным знамением умерщвляла ядовитых змиев. И наконец, скончалась страдальчески, как смиренная агница Христова, в царствование Антонина, для всех благодетельного, но, к несчастью его, для Христовых людей злотворного.

Чада Церкви! Благословим память святой преподобномученицы Евдокии и научимся от нее истинному покаянию! И мы грешники, и на нас скрежещет злообразный дух. Опомнимся, хотя в единонадесятый час!

\section{Разность преступлений}

Однажды к преподобному Агафону\footnote{Прп. Агафон жил в IV~в. Память его празднуется 2~(15) марта.} пришли иноки с тем, чтобы узнать его разум в делах веры и незлобие сердца. «Ты ли отец Агафон?» "--- спросили они, как незнакомцы, издалека прибывшие. «Видите пред собой грешного раба Господня», "--- отвечал старец. «Носится слух, "--- сказали иноки, "--- будто ты человек гордый и невоздержный». "--- «Совершенная правда!» "--- отвечал праведник. «Мы слышали также, "--- продолжали иноки, "--- будто ты лжец и любишь пересуживать других». "--- «И это правда!» "--- сказал святой Агафон. «Сверх того, говорят, "--- присовокупили иноки, "--- будто ты еретик». "--- «Напрасно! "--- возразил старец. "--- Я не еретик». Тогда пришельцы спросили у него: «Для чего же прочие пороки на себя принимаешь, а от последнего отрицаешься?» "--- «Оных пороков нельзя и не приписывать себе, "--- отвечал старец, "--- поскольку они естественны человеку, а ересь есть богоотступничество, чего я, хотя и грешный человек, никак на себя принять не могу и не хочу». Услышав это, иноки удивились разуму и незлобию святого Агафона и, нарицая его учителем, а себя учениками, пошли обратно.

Христиане! Всякий человек есть грешник; всякий может подвергнуться самовольству страстей, против нас воюющих. Но сохрани, Боже, лжемудрствовать о святой вере! Слабости сердца исцеляет признание; а суемудрие, как дочь гордости, умирает, не опомнившись, и низвергается до ада.

\section{Телесное и душевное занятие}

Братия, собравшись к преподобному Агафону, спросили: «Что важнее, телесные подвиги или душевные?» "--- «Человек подобен древу, "--- отвечал старец, "--- телесное занятие приносит листвия, а душевное произращает плод; и поскольку Священное Писание уверяет, что \textit{всяко древо, еже не творит плода добра, посекаемо бывает и во огнь вметаемо} (Мф. 3, 10), то и видно из этого, что более внимания должно устремлять на плод, то есть на занятие души. Но древо также имеет нужду и в листвиях, чтобы тень их подавала защиту плодам от чрезмерного зноя. Этого мало еще: листвия умножают красоту древа и доставляют прохладу путешественникам».

Святой старец сказал сущую правду. Сколь совершен, сколь любезен Богу и людям, сколь спокоен и доволен сам собой тот человек, который, чрез занятие душевное познав истину, может подавать спасительные советы и другим, а чрез занятие телесное снискивая житейские потребности, может благодетельствовать страждущему человечеству! Сверх того, и стремительные, пламенные страсти едва ли могут овладеть сердцем того, кто любит труды.

\section{Человеколюбие отшельника}

Некогда преподобный Агафон, придя в город для продажи сосудов своего рукоделия, увидел больного странника, лежащего на площади и всеми оставленного. Почувствовав глубокое сострадание, он остановился над ним и, расспросив о случившемся, нанял для него хижину; потом, продав свое рукоделие, расплатился с хозяином дома. А остальное употребляя для пищи и врачевания больного, прожил тут три месяца, пока незнакомец не выздоровел, и, радуясь соделанному добру, возвратился в пустыню.

Сострадание к несчастным уподобляет человека Ангелу"=хранителю.

\section{Умирающий праведник}

Преподобный Агафон, видя приближение смерти, трое суток смотрел неподвижными очами на небо. Сидящие вокруг одра его братия, удивляясь тому, спросили у него: «Что так смотришь, авва?» "--- «Стою пред Судилищем Христовым», "--- отвечал Агафон. «Неужели и ты, человек Божий, страшишься Суда?» "--- опять спросили иноки. «Сколько мог, я соблюдал заповеди Господни, "--- сказал им старец, "--- но как человек могу ли знать, что совершенно угодил Богу?» "--- «Авва, "--- возразили иноки, "--- тебе можно надеяться на добрые дела твои». "--- «Не надеюсь, доколе не увижу Самого Господа, "--- вздохнув, отвечал им святой Агафон, "--- ибо Суд Божий не таков, как суд человеческий».

В самом деле, жизнь наша подвержена столь многим искушениям, разум так иногда слаб, а сердце столь непостоянно, что и праведник едва спасется. Тем более мы, грешные люди, должны помнить Суд Божий!

\section{Трудолюбивые постники}

Преподобный Герасим\footnote{Прп. Герасим (†~475) жил в царствование Маркиана и Пульхерии. Память его празднуется 4~(17) марта.} был настоятелем монастыря, стоящего близ Иордана. Но, по уставу его, в нем обитали только те, которые недавно приняли иноческий образ, а старцы, в подвигах совершенные, имели каждый особую келью в пустыне.

Закон, который им от святого Герасима был предписан, достоин примечания: пять дней в седмице отшельники пребывали уединенно, занимаясь рукоделием, и, не приготовляя вареной пищи, вкушали каждые сутки только понемногу хлеба с водой и финиками; даже запрещено было иметь в кельях огонь. Но в субботу и воскресенье приходили они в монастырь, собирались на Божественную службу, приобщались Святых Таин и вкушали вино и вареную пищу; потом каждый из старцев показывал настоятелю свое рукоделие, которым занимался в течение пяти дней. Вечером на понедельник, взяв с собой немного пищи, опять удалялись они в безмолвные кельи.

\section{Христианское братолюбие}

Святой Павел и святая Иулиания\footnote{Святые мученики Павел и Иулиания жили в III~в. Память их празднуется 4~(17) марта.}, родные брат и сестра, жили в финикийском городе Птоломаиде и, будучи ревностными христианами, хотя не скрывали от граждан своего исповедания, но, благодетельствуя верным и неверным, долго скрывались от бдительных очей христоненавистного правительства. Наконец, предуготовил им все ужасы страданий такой поступок, который ныне почитается знамением истинной любви народа к благодетельному монарху.

Император Аврелиан, обозрев Ассирию, прибыл в Птоломаиду; весь народ вышел ему навстречу, возглашая «многая лета»; там же находился и святой Павел.

Веруя учению Иисусову, что \textit{несть власть, аще не от Бога} (Рим. 13, 1), он перекрестился, умоляя Господа, чтобы прибытие царя было всем во благо. Это"=то Аврелиан приметил и тут же приказал взять святого Павла, как христианина, под стражу. Какое ослепление обладало тогда умом даже мудрых государей! Поклонники солнца, варвары, не знавшие никакого богопочитания, и даже иудеи могли свободно соблюдать свои обряды: одни христиане за все были мучимы и убиваемы!

На следующий день Аврелиан потребовал к себе святого Павла и прежде всего грозно спросил, как смел он, встречая государя, оградиться христианским знамением? «Почитая в особе твоей власть и силу Царя царей, Господа моего Иисуса Христа, "--- отвечал святой юноша, "--- я не знал, чем более засвидетельствовать радость мою согражданам, всяких благ от тебя, о, государь, ожидающим». "--- «А разве не обнародованы против галилеян мои указы?» "--- с гневом воскликнул Аврелиан. «Я читаю все повеления вашего величества, "--- отвечал Павел, "--- но исполняю только те, которые не делают насилия моей совести в делах веры, в противном случае я готов вытерпеть все мучения, но Христа не оставлю». Тогда раздраженный Аврелиан повелел приступить исполнителям казни: святой юноша был повешен на древе, ударам не было числа. Но вдруг поражает тирана зрелище другого рода: весь народ удивляется, недоумевает… Посреди мучителей является юная девица, обнимает окровавленного Павла, лобызает раны его (то была Иулиания, сестра его). «Ах! За что так люто терзаешь брата моего? "--- обратившись к Аврелиану, воскликнула она. "--- Оскорбил ли он, хотя раз в жизни, величество царя? Обидел ли своих сограждан? Мое сердце ручается за его поступки; весь народ ручается за его добродетели. Государь! Пощади невинного». Какое бы сердце не тронулось молением невинности! Но Аврелиан был тверже камня. Мы видим язычников чувствительных, по крайней мере наружно человеколюбивых, но, когда дело касалось христиан, они были глухи, как аспиды. Так для идолопоклонников противна была Небесная истина! Так противна истина и для нас, когда раболепствуем пороку!

Таков был и Аврелиан. «Сбросьте покрывало с дерзкой женщины, "--- воскликнул он, "--- и бейте по лицу немилосердно, чтобы умела наблюдать перед царем своим скромность, а над молодым изувером усугубите удары, чтобы какого"=то Христа не предпочитал императору». Святая Иулиания улыбнулась над безумием Аврелиана, и эта ангельская улыбка проникла в сердце его. Тиран смягчился и с кротостью сказал ей: «Для твоей юности, для твоей красоты щажу тебя и погубить не желаю, но поклонись богам, хранителям отечества. Тогда и ты увидишь от меня все милости, и брат твой будет славен». "--- «Молю Иисуса, да сохранит от всяких зол мое отечество, "--- отвечала Иулиания, "--- но идолам жертвы не принесу». После этого она возвела очи свои к небу и оградилась крестным знамением. Тогда весь ад, казалось, овладел душой мучителя. «Испытайте над ней все казни вместе с братом ее», "--- воскликнул он и подтвердил, что малейшая пощада им будет стоить жизни исполнителям воли его.

Три дня продолжались страдания святых мучеников, и Господь все более укреплял их. Угли и кипящая смола, змеи и гады, железо и все орудия казни были для них изысканы; но брат поддерживал сестру свою, сестра поддерживала брата своего. Наконец, святые страстотерпцы преклонили под секиру главы свои и предали дух свой в руце Господни.

\section{Отец и сын посреди страданий}

Святой преподобномученик Конон\footnote{Святые преподобномученики Конон и сын его Конон жили в III~в. Память их празднуется 6~(19) марта.}, уроженец иконийский, оплакав смерть супруги своей, с семилетним сыном удалился в монастырь. Там, подвизаясь в молитвах и постничестве, он удостоился принять от Бога дар чудотворения и в одно время, по просьбе иконийских жителей, именем Христовым повелел возвратиться и течь в берегах своих реке, которая при разлитии вод потопила пашни их и пажити.

Вскоре настало время гонения. Некий князь, по имени Домициан, прибыл в Иконию, чтобы истребить поклонение Иисусу Христу. Но поскольку имя Конона везде было славно, то человек Божий взят был первым на истязание. «Я простой человек, "--- отвечал Конон, "--- ибо недостоин, по грехам моим, ангельского сана; однако и я, наравне с прочими, поклоняюсь Иисусу и живу в Нем». "--- «Имеешь ли детей?» "--- опять спросил Домициан. «У меня один только сын, и остался в келье: если угодно, я представлю его сюда же». "--- «Верно, и он кланяется тому же Иисусу!» "--- с насмешкой сказал Домициан. «Каково древо, таковы и ветви его, "--- отвечал Конон, "--- благодатью Господней он в числе слуг Его и уже диакон». Тогда за юным Кононом посланы были воины, которые и привели его на судилище.

«Любезный юноша! "--- сказал Домициан. "--- Отец твой стар, довольно жил на свете, пресытился всеми удовольствиями, и неудивительно, что сам себе желает смерти. А ты еще молод, только жить начинаешь; доселе видел одну какую"=то монастырскую строгость, а впереди ожидает тебя свобода: так неужели хочешь последовать отцу твоему?» "--- «Родитель мой научил меня знать Единого Истинного Бога, "--- отвечал Конон, "--- и жить для прославления имени Его. Он наставил меня, что эта жизнь есть только поприще трудов и болезней, а истинная жизнь "--- по ту сторону гроба, в царстве Иисуса Христа. Он научил меня, что идолослужителям предуготована вечная мука, а рабам Христовым вечное блаженство: как же мне не идти по стопам его? Сам Господь мой Иисус Христос сказал: \textit{Еже Отец Мой делает, и Аз делаю: и не может Сын творити о Себе, ничесоже, аще не еже видит Отца творяща: яже бо Он творит, сия и Сын такожде творит} (Ин. 5, 17, 19)». "--- «А что, если ужаснейшими мучениями погублю отца твоего? "--- возразил тиран. "--- Захочешь ли и ты погибнуть с ним?» "--- «Я решился умереть с отцом моим, да с ним и жив буду, "--- отвечал юноша. "--- Ибо умереть за Христа есть то же, что жить вечно».

Услышав ответ юноши, Домициан обратился к старцу и сказал: «Сын твой разумен, даже разумнее тебя; жаль только, что суеверное воспитание не позволяет ему познать истины и воздать поклонение богам нашим!» "--- «Он знает истину, "--- отвечал Конон, "--- и поклоняется глаголющему: \textit{Аз есмь истина} (Ин. 14, 6)». "--- «Нечестивый старик! "--- воскликнул тогда раздраженный Домициан. "--- Тебе бы надлежало обращать молодых людей на добрые дела, а ты, безумец, еще более развращаешь сына твоего! Обнажите их, "--- обратившись к служителям, продолжал он, "--- и пытайте, доколе не откажутся от их распятого Бога».

Много страданий вытерпели святой родитель и святой сын! Наконец, когда готовили для них новые орудия казни, старец произнес теплую молитву к Небесам: «Боже всесильный! Благодарю Тебя, что сподобил нас пострадать за имя Твое святое и укрепил посреди мучений; не попусти врагам Твоим посмеяться над нами и над всем христианством; не страшусь о себе, страшусь о юности сына моего: приими убо в мире дух наш!» Юноша отвечал: «Аминь!» "--- и в то же мгновение оба заснули сном блаженной кончины.

\section{Страдание сорока воинов, замученных в Севастийском озере}

Агриколай, военачальник севастийский, известившись, что в полках, находящихся под его ведением, есть офицеры, тайно исповедующие веру Христову, нарядил следствие и вскоре нашел христиан\footnote{Эта святая дружина пострадала ок. 320~г. Память 40~мучеников, в Севастийском озере мучившихся, празднуется 9~(22) марта.}, из которых Кирион, Кандид и Домн были сколько неустрашимы на поле брани, столько благочестивы и сведущи в Божественном Писании.

Сначала Агриколай почел за лучшее средство употребить ласковость и обещание царской милости, но, видя непреклонность воинов Христовых, повелел приготовлять орудия казни. Тогда святой Кирион, приступив к нему, сказал: «Не забывай своей должности: ты не имеешь власти наказывать нас телесно; донеси государю и жди повеления». Агриколай от ярости заскрежетал зубами; но должно было повиноваться законам.

Долго воины Царя Небесного и царя земного сидели в темнице! Наконец прибыл в Севастию князь Лисия, уполномоченный императором, и потребовал их к ответу. Святых мучеников повели под стражей в военный суд. Чем же дорогой занимался святой Кирион? «Братья мои! "--- говорил он прочим сподвижникам. "--- Не убоимся тиранов. Вы помните, как Господь помогал нам посреди кровавой брани, когда мы призывали имя Его; помните, как мы, оградившись крестным знамением, одолевали тысячи врагов? Помните, как все воинство обратилось в бегство, а мы, принеся теплую молитву Господу, остались неподвижны и не только удержали неукротимое стремление врагов, но совершенно поразили их, сами не получив ни одной раны? А теперь кто восстает на нас? Тот, кто за гордыню низвержен с Неба в геенну! Сколь бессилен он против Иисуса, подвигоположника нашего! Он бессилен и против одной души христианской, а нас "--- сорок воинов Христовых. Итак, воспоем: \textit{Боже, во имя Твое спаси мя и в силе Твоей суди ми! Боже, услыши молитву мою, внуши глаголы уст моих} (Пс. 53, 3)».

Когда святые страстотерпцы предстали на суд, Лисия, ласково воззрев на них, сказал: «О! Эти воины стоят того, чтобы дать им важнейшие чины в сословии военном! И император сделает для вас все, "--- обратившись к мученикам, продолжал он, "--- если, покорившись воле его, обратитесь к богам отцов ваших; в противном случае вы должны лишиться всего». "--- «Не только чин воинский, но и тела наши возьми от нас», "--- возразил святой Кандид. Услышав это, князь воспылал бешенством и повелел бить их камнями. Этого мало еще: обвиняя в бессилии и нерадении служителей, он сам схватил камень и бросил в одного из мучеников; но рука Господня дала ему другое направление, и камень разбил лицо Агриколая, а святой Кирион воскликнул: «\textit{Борющиися с нами врази наши изнемогоша и падоша: воистину меч их внидет в сердца их, и луцы их сокрушатся}» (Пс. 26, 2; 36, 15).

Смущенный Лисия повелел поставить исповедников Христовых в озеро посреди льда, отовсюду напирающего (была глубокая осень). Но из сорока мужей поколебался только один: в то время как прочие в лютый мороз стояли в воде, так что раздроблялись члены их, вошел он в теплую баню, поставленную на берегу для искушения их. Малодушный был наказан вечной смертью. Но его место в то же мгновение заступил страж: ибо узрел он, что тридцать девять венцов свыше нисходят на праведников, и возжелал сорокового.

\section{Неустрашимый воин Христов}

Император Максимиан, жесточайший из гонителей христианства, не мог терпеть исповедников Иисуса ни при дворе, ни в воинстве. Однажды, будучи в Никомидии, он потребовал к себе всех воинских и гражданских чиновников и грозно сказал: «Если кто из вас заражен зловерным христианством и не хочет обратиться к богам царя и отечества, пусть снимет с себя все знаки чиноначалия и отличия и удалится не только из палаты, но и из города нашего. Крепко содержа веру предков моих, я не хочу иметь при себе иноверных». Тогда сколь разнообразное открылось зрелище! Страх и трепет был всеобщий. Иные старались оправдаться совестью и чувствованиями, которые внушены им от родителей; иные уверяли, что они не ведают Христа Спасителя, и проклинали веру, доселе ими содержимую; иные в безмолвии снимали с себя знаки отличия и спокойно выходили из комнат царских. Но мудрый и твердый душой Урпасиан\footnote{Св. мученик Урпасиан жил в конце III~в. Память его празднуется 9~(22) марта.}, оказавший отечеству великие услуги, поступил всех небоязненнее: он подошел к трону, на котором сидел Максимиан, и, сбросив с себя одежду сановника, сказал: «Государь! Если мои услуги не нужны (как горестно, что на этот раз, перестав служить тебе, перестану служить и отечеству!), если Урпасиан не нужен для тебя, то объявляю себя воином Царя Небесного и Бессмертного, Господа моего Иисуса Христа. Прими обратно отличия, тобой мне дарованные; видно, служитель Христов не может быть служителем Максимиана». Услышав это, император изменился в лице и какое"=то время оставался безгласен; потом, потирая чело свое и сурово взглядывая на воина Христова, зарыкал, как лев: «Возьмите этого грубияна, этого мятежника и замучьте до смерти».

Тиран с удовольствием смотрел на все страдания святого Урпасиана. Наконец внезапно разлилось благоухание, и дух его в образе блистающей звезды вознесся на Небо. Все христиане видели торжество мученика и благословляли Господа.

\section{Младенец, Небесной рукой воспитанный}

Святой мученик Кодрат\footnote{Св. мученик Кодрат пострадал в 258~г. Память его празднуется 10~(23) марта.} родился и был воспитан следующим образом. Когда по приказанию злочестивых царей и князей свирепствовали ужасные гонения против христиан и чада Церкви, боясь нестерпимых мук, оставляли города и все имущества и, кроясь в пустынях и горах, лучше хотели жить посреди зверей, нежели с нечестивыми, подвергаясь беспрестанно опасности умереть в мучениях и принести жертву идолам, "--- в то время одна благоверная женщина, по имени Руфина, будучи преследуема тиранством, ушла из Коринфа и скиталась по непроходимым местам. Не своей смерти боялась истинная христианка: нежная мать хотела спасти младенца, в утробе ее зачатого. Наконец приспело время разрешения, и она родила сына, но, к несчастью, через несколько дней скончалась.

Всего естественнее подумать, что беспомощный сирота умрет от голода или сделается пищей зверей… Но \textit{отверзаяй руку Свою и насыщаяй всякое животно благоволения} (ср. Пс. 144, 16) не презрел пеленами повитого Кодрата и нарекся отцом его и матерью. Бог заповедал облакам, и они, на голос младенца спускаясь с высоты и преклоняясь долу, источали на уста его сладкую росу и питали млеком и медом, доколе он не подрос и сам не мог снискивать для себя пустынные плоды. Таким образом, отрок жил в пустыне, подобно Иоанну Крестителю, Богом хранимый, Святым Духом наставляемый и вразумляемый к боговедению, и, уже достигнув юношеского возраста, был найден христианами.

Скоро святой Кодрат научился грамоте, еще скорее "--- врачебному искусству и отчасти с помощью трав, известных ему как воспитаннику пустыни по собственному опыту и употреблению, а более "--- благодатью Господней исцелял всякие недуги. Но, привыкнув с младенчества к пустынному безмолвию, человек Божий не мог навсегда остаться в городе и по большей части жил в горах, упражняясь в богомыслии. Если недуги страждущего человечества и призывали его иногда в сообщество людей, то разврат идолопоклонников опять изгонял его в пустыню. Но и уединение его было общей врачебницей и храмом проповеди Господней: беспрестанно стекалось к нему множество народа, требующего то советов, то исцеления.

Этот второй Предтеча жил спокойно до старости. Но, к несчастию христиан, воцарился злочестивый Деций, и святой Кодрат, как муж по ревности к Иисусу известный, первый принял венец мученический с друзьями своими Киприаном, Дионисием, Анектом, Павлом и Крискентом.

\section{Благодать за посещение святых мест}

Преподобный Феофан\footnote{Этот преподобный отец жил в конце VIII "--- начале IX~в. и уморен гладом в изгнании по приказанию иконоборца Льва Армянина. Память его празднуется 12~(25) марта.} был родственник греческого царя Льва Исавра и с юных лет исправлял при дворе и в правительстве важные должности. Вступив в супружество с одной из благороднейших девиц, Феофан был очень счастлив, что обрел в ней не только добродетельную помощницу, но и совершенного друга в деле спасения.

Вскоре после того Феофан, по указу царскому, должен был отправиться в страну Кизическую для общественного благоустройства; благочестивая супруга следовала за ним же. Хотя в Кизику можно было ехать и сухим путем, но Феофан, по вдохновению Господню, отправился по воде и плыл между страной Олимпийской и Сигрианской, где было множество монастырей и отшельнических келий. Тогда единственным занятием Феофана было: половину дня размышлять, как бы с лучшим успехом исполнить порученное ему дело, а другую половину времени употреблять на посещение пустынников, чтобы в их беседах усовершенствовать себя в вере и благочестии.

Столь мудрому человеку нетрудно было устроить благоденствие народа кизического; он исправил судилища, открыл новые источники народного довольства, именем государя предписал законы благонравия и заслужил общее благоволение от всех государственных сословий.

Заметим поступки преподобного Феофана. Пока жил он в Кизике, все дела, касающиеся общественного благоустройства, предпринимал не прежде, как посетив бесстрастных пустынников; по их равноангельному совету располагал он свое поведение в делах общих и частных. За то не только угодил государю и всеми почитаем был, \textit{как муж силен, Бога боящийся и праведен, да судит на всяк час}, но и удостоился сверхъестественной благодати от Господа.

В одно время, посещая святую и богомудрую братию, обитающую в горах Сигрианских, Феофан опоздал, так что на обратном пути должен был ночевать в пустыне. Зной был чрезвычайный, место безводное, слуги и скот изнемогли от жажды; святой Феофан, почти полумертвый, по обыкновению своему помолившись Богу, сел под холмом, чтобы немного отдохнуть и, по крайней мере, во сне избавиться от мучительной жажды. Едва задремал он, внезапно над головой его зажурчал источник и окропил Феофана. Человек Божий, почувствовав журчание и росу ключа, воспрянул с места и, воздав благодарение Богу, созвал своих спутников. Все, удивляясь внезапному чуду, прославили Бога и, напившись из источника, остались тут на всю ночь.

Исполнив в точности царскую волю, святой Феофан с честью возвратился в Царьград и вскоре, приняв на себя образ ангельский, удалился в пустыню. Супруга его вступила в лик преподобных дев и там, в молитве и постничестве, пребыла до блаженной кончины.

\section{Милость Господня за милость к нищим}

Святой Григорий Двоеслов\footnote{Св. Григорий, нареченный Двоесловом (†~604) или Собеседником по причине душеспасительных его разговоров, есть боговдохновенный сочинитель Преждеосвященной литургии. Он родился в Риме от сенатора Гордиана, был славный вития и философ и уже имел чин претора; но всем отличиям света предпочел иночество. После папы Пелагия против воли своей он был возведен на престол Римской Церкви при царе Маврикии. Пас стадо Христово тринадцать лет, шесть месяцев и десять дней. Преставился в царствование Фоки мучителя. Его благочестие и добродетели были, так сказать, наследственными. Феликс Третий, святой Папа Римский, был ему дед; святая Тарсилла, при кончине своей видевшая Господа, к ней грядущего, и Емилиана, ее сподвижница, были тетки. Мать его Сильвия также причтена Римской Церковью к лику святых. Память их празднуется 12~(25) марта.} был столько милостив, что не щадил и последнего, только бы удовлетворить просьбу нищего или каким"=нибудь случаем разоренного. Следующее происшествие представляет разительный пример его милосердия и к нему милосердия Небесного.

Однажды, когда Григорий, сидя в келье, по обыкновению своему читал книгу, пришел к нему бедный человек и жалобно сказал: «Помилуй меня, раб Бога Вышняго, и пособи моему несчастию! Я был хозяином корабля: но Богу за грехи мои угодно было, чтобы море поглотило все мое имение; оставшись почти наг, я имею сверх того немало на себе долгу». Святой Григорий, пролив о нем слезы соболезнования, призвал служащего инока и велел дать несчастному шесть златниц. Нищий, поклонившись ему, ушел, но через час возвратился и опять сказал: «Сжалься надо мной, раб Господень! Ты дал мне мало, а я потерял весьма много». Святой Григорий приказал келейнику дать ему другие шесть златниц. Но вечером нищий пришел в третий раз и, проливая слезы, говорил: «Человек Божий! Не почти за наглость, что опять прибегаю к тебе: ах! я погубил весьма много чужого богатства, и хозяева грозят мне темницей; умилосердись над несчастным!» Святой Григорий, кликнув инока, велел ему дать еще златниц, но инок отвечал, что у них не осталось ничего. «Дай же что"=нибудь другое, "--- возразил Григорий, "--- или из одежды, или из посуды!» "--- «Божусь, что нет у нас ничего, "--- сказал инок, "--- кроме серебряного блюда, которое твоя мать прислала с сочивом». «Отдай же его, "--- сказал Григорий, "--- чтобы бедный человек не ушел от нас печален». Незнакомец, взяв блюдо, пошел с радостью и более не возвращался.

Но через несколько лет, когда святой Григорий был уже первосвященником Римской Церкви, этот нищий явился ему в другом образе. По обыкновению своему всегда приуготовляя стол для странников, в одно время приказал он казначею угостить двенадцать человек. Когда странники сели за стол, святой папа, смотря на них, увидел тринадцать, почему и спросил у казначея на ухо: «Отчего, против моего приказания, здесь находится лишний человек? Клянусь Богом, я бы с сердечной радостью угостил и вдвое больше, но грех тебе преступать волю начальника». Устрашенный казначей божился, что за столом сидят только двенадцать нищих. Удивленный папа, беспрестанно смотря на обедающих странников, а особенно на тринадцатого, сидящего при конце стола, увидел, что лице его, изменяясь, представляет то старца, то юношу.

Когда обед кончился, святой Григорий, отпустив всех, оставил лишнего странника у себя и, взяв за руку, увел в спальню. Там, посмотрев на него пристально, сказал: «Заклинаю тебя силой Бога Вседержителя, скажи мне, кто ты?» "--- «Я тот бедный мореплаватель, "--- отвечал незнакомец, "--- который некогда принял от тебя двенадцать златниц и серебряное блюдо; познай, что от того дня Господь нарек тебя архипастырем Своего стада и первопрестольником Святой Церкви. Я Ангел Господень, к тебе посланный, чтобы узнать твои чувствования, не для тщеславия ли ты подаешь милостыню». Святой Григорий ужаснулся и повергся лицом на землю, говоря: «Прости меня, служитель Иисусов, что я, грешный человек, беседовал с тобой, как с одним из смертных!» «Не бойся! "--- отвечал Ангел. "--- Господь повелел мне, да всегда с тобой пребываю и приношу к Нему молитвы твои; все, что с упованием просишь, от Него приимешь». Сказав это, небесный гость стал невидим, а святой Григорий, воздав благодарение Богу, воскликнул: «Господи! Верую, что милость Твоя к милостивым неизреченна, когда и за малое подаяние столько возлюбил меня, что приставил ко мне Ангела"=хранителя».

\section{О том, сколь трудно и мудрому начальнику угодить своенравию подчиненных}

Преподобный Венедикт\footnote{Прп. Венедикт Нурсийский (†~543). Память его празднуется 14~(27) марта.} в юности своей обучался в Риме, в одном из общественных училищ, но, видя разврат своих товарищей, лучше захотел быть мудрым невеждой, нежели ученым грешником, и удалился в пустыню.

Как святой Венедикт там подвизался для прославления имени Божия, может описать разве перо Ангела. Но не может град, стоящий наверху горы, укрыться, и этого праведника Господь восхотел поставить во спасение прочим. В одном из монастырей умер настоятель. Осиротевшие иноки, по слуху зная о благочестии святого Венедикта, единодушно пришли к его келье и умоляли, да будет их настоятелем. Долго отрицался святой Венедикт, называя себя грешником, недостойным пастырства, но, видя их неотступность, наконец сказал: «Братия мои! Будьте уверены, что мои нравы не будут согласны с вашими» "--- и, предав себя воле Господней, согласился быть их настоятелем.

Что сказал угодник Божий, то и сделалось. Неопустительно наблюдая уставы постничества, уединения и молитвы, святой Венедикт скоро пришел в ненависть у братии. Без сомнения, некоторые из иноков ублажали своего начальника и старались сообразоваться с его поведением, но большая часть желала, чтобы и следов Венедикта не было в их обители. Эта злоба простерлась до того, что порочнейшие из иноков решились умертвить его ядом, но смертоносная чаша, огражденная знамением креста, вдруг распалась и сокрушилась, как от удара камнем. Тогда угодник Божий познал, что от братии ему готовится. Он призвал к себе всех иноков и, улыбнувшись дружески, сказал им: «Да помилует вас милосердый Бог, чада мои, и да отпустит грехи ваши.

Я прежде сказал вам, что мои нравы не согласны с вашим поведением; итак, сыщите начальника по вашему обычаю, а я не могу жить с вами». После этого, благословив их, возвратился в пустыню и там начал подвизаться пред очами Единого Всевидца Бога.

\section{Сила повиновения}

Не долго преподобный Венедикт наслаждался уединением. Господь вместо малого стада поручил ему большее, вместо одного монастыря, оставленного им, дал двенадцать, ибо слава его равноангельной жизни привлекла к нему столько пустыннолюбцев, что должно было разделить их на несколько обителей. В числе этих пришельцев был отрок Мавр и юный Плакида, дети римских сенаторов, порученные родителями духовному воспитанию святого Венедикта.

Однажды юный Плакида, взяв водонос, пошел на реку и, черпая воду, нечаянно упал в самую быстрину, ибо мальчик из желания угодить духовному отцу хотел принести самой чистой воды. Волны подхватили его и понесли вниз по реке. Преподобный Венедикт, сидя в келье, увидел душевными очами несчастье Плакиды и, кликнув к себе Мавра, сказал: «Беги, беги! Плакида упал в реку». Не отвечая ни слова, отрок побежал и, увидев Плакиду, утопающего посреди реки, устремился по воде, как посуху, и, схватив его за волосы, извлек на берег. Тогда уже, оглянувшись назад, узнал он, что бежал по волнам, а не посуху.

Возвратившись с Плакидой в келью, Мавр рассказал святому Венедикту обо всем, что случилось, и, припадая к стопам старца, приписывал чудо немокренного по водам шествия его молитвам, но Венедикт отвечал на это: «Не забудь, любезный юноша, какова сила беспрекословного послушания».

\section{Недостойно принявший священство}

Некоторый церковнослужитель, за грехи свои преданный нечистому духу, сошел с ума и сделался беснующимся. Ни врачевание, ни молитвы Аквилейского епископа Констанция не могли исцелить несчастного. Наконец, этот грешник приведен был к человеку Божию Венедикту.

Святой Венедикт, сотворив над ним знамение креста, мгновенно изгнал духа"=губителя и церковнослужителю крепко заповедал, чтобы не домогался и, если будут давать, не принимал священства. Долго исцелившийся клирик соблюдал повеление святого Венедикта, но, видя умирающих пресвитеров, видя, что на их места поступали некоторые, летами его моложе и дарованиями слабее, почел себе за обиду и, как бы забыв заповедь святого Венедикта, начал требовать священства. Но сколь ужасен грех недостойно принять сан пастырский! Едва церковнослужитель узрел себя рукоположенным, мгновенно дух бешенства напал на него, и несчастный в ужаснейших страданиях испустил последнее дыхание.

\section{Изобличенный хищник}

Один благочестивый помещик послал к преподобному Венедикту два небольших бочонка вина, но слуга, утаив один сосуд, спрятал под кустом на половине дороги, а другой принес Венедикту. Прозорливый старец узнал все и, принимая дар усердия, издалека начал речь о верности рабов к господам и, довольно поучив его, наконец, сказал: «Иди с миром, чадо, и скажи своему господину, что я, молясь о грехах моих, буду молиться и о его здравии и оставлении грехов, но смотри, не пей из сосуда, оставшегося на дороге, все вино вылей на землю, ты увидишь, что в нем находится».

Изобличенный раб от стыда не мог отвечать ни слова, поклонился и ушел. Когда же приблизился он к месту, где оставлено было украденное вино, хотя рассмеялся над приказанием человека Божия, но из любопытства наклонил сосуд и начал лить вино. Что же? Вдруг с вином излиялся змий! Устрашенный раб пал на колена и, обливаясь слезами, благодарил Бога и угодника Божия за избавление себя от смерти.

\section{Польза поминовения на проскомидии\footnote{\textit{Проскомидия} "--- часть литургии, при которой готовятся Евхаристические дары для освящения.} усопших}

Две благородные девицы дали клятвенный обет Богу жить в постничестве и молитвах. К несчастью, сколь чисто было их житие, столь необуздан язык: осудить, оболгать, укорить было обыкновенным делом для постниц. Преподобный Венедикт, узнав об этом, нарочно послал к ним ученика своего и отечески советовал удержать язык свой, а в случае непослушания грозил отлучить их от Божественного причащения. Но безумные девицы начали издеваться и над святым Венедиктом.

Вскоре после этого они умерли и, как великие постницы, были погребены в церкви. Но поскольку \textit{языки} клеветника, по выражению апостола, \textit{есть огнь, лепота неправды, неудержимо зло, исполнь яда смертоносна} (Иак. 3, 6, 8), то можно ли ему быть там, где истина и правота невидимо восседят на престоле своем? Когда совершалась Божественная служба и диакон возглашал: «Елицы оглашеннии, изыдите!» "--- некоторые из христиан видели двух девиц, выходящих из гроба и из церкви. Это чудное явление продолжалось некоторое время. Наконец услышал об этом и преподобный Венедикт: он пролил о них слезы сердечного сокрушения и, послав просфору в оную церковь, велел вынуть части в жертву Богу об упокоении душ их. С того времени никто уже не видел их исходящими вон, и христиане уверовали, что злоязычные постницы жертвой и молитвами преподобного Венедикта получили от Бога прощение.

\section{Добродетельный пустынник}

Святой Анин\footnote{Прп. Анин монах. Этот великий старец был родом из Халкидона, умер в 110~лет. Память его празднуется 18~(31) марта.}, инок и чудотворец, обитая в уединении на границе Сирии с Персией, имел обыкновение часто уходить внутрь пустыни и там по нескольку дней безмолвствовать. А поскольку Анин имел сугубый дар утешать печальных и врачевать недужных, то к его келье из разных стран стекалось всегда множество народа, но, не улучив человека Божия дома, люди претерпевали зной и жажду, дожидаясь его, а некоторые с прискорбием возвращались. Видя это, святой Анин, сколь ни великое находил удовольствие в пустынном безмолвии, немедленно для пользы общей переменил свой образ жизни и с того времени, беспрестанно ожидая посетителей, никуда не отлучался из кельи.

Но не в одних советах и врачевании состояла благодетельность праведника. Обитая далеко от реки Евфрата, каждый день ходил он по нескольку раз за водой для напоения странников. Узнав об этом, епископ Неокесарийский послал ему осла для водоношения. Но вскоре пришел к нему бедный человек, не имевший чем удовлетворить заимодавца, и святой Анин отдал ему своего водоносца. Епископ прислал к нему другого осла, уже не в дар, а только с тем, чтобы до времени тот работал на него, но преподобный Анин не мог удержаться, чтобы не отдать и его одному разоренному семейству. Наконец, епископ увидел, что щедролюбивый пустынник не перестанет ходить за водой, и сам поступил иначе: он приказал подле кельи преподобного Анина выкопать колодезь и время от времени наполнять его свежей водой, чтобы он и посещающие его боголюбцы имели питье всегда готовое.

\section{Иноки, полагающие душу свою за Церковь Божию и за братию\footnote{Из жития прп. отцов, в обители святого Саввы убиенных (†~796). Их память празднуется 20~марта (2~апреля).}}

В царствование Константина и Ирины, когда Иерусалим, уже отторгнутый от Греческой державы, находился под властью агарян, Палестина наполнена была разбойниками, которые, собираясь тысячами, грабили все, что оставалось беззащитным. Земледельцы убегали с полей, иноки оставляли обители, где давали обет Богу жить до смерти; все собрались в Иерусалим как в единое убежище, которое могло спасти их от мучений и смерти. Только иноки обители святого Саввы не хотели ее оставить и среди общего мятежа и шума оружий говорили друг другу: «\textit{Не убойтеся от убивающих тело, души же не могущих убити} (Мф. 10, 28); мы не имеем оружия, \textit{но ополчится Ангел окрест боящихся Господа и избавит их} (Пс. 33, 8)».

Долго святая обитель посреди повсеместного опустошения и кровопролития пребывала невредимой. Варвары часто приходили к ней, но, истребовав на дорогу съестные припасы, оскорбляли иноков одними угрозами. Наконец пришло время покушения и на этих рабов Господних. На шестой неделе Великого поста эфиопские разбойники, в числе шестидесяти человек, напали на святое жилище и нескольких старцев умертвили, но показавшийся вблизи многолюдный караван заставил злодеев разбежаться. Наступила седмица страданий Господа нашего Иисуса Христа, и вместе с тем наступило время пострадать невинным рабам Его. В Великий четверг варвары напали на Святую Лавру и проявили свою жестокость: иных расстреливали, иных поднимали на копья, иным отсекали руки и ноги, иных побивали камнями. Нигде не было для страдальцев убежища, ибо варвары окружили обитель и поставили стражу по горам и ущельям. Кто убегал из ограды монастырской, тот убиваем был на открытом месте. Но сколь ужасны были тиранства злодеев, столь велико и непреклонно было терпение святых пустынножителей.

Юный Иоанн, начальник странноприимницы, бежал в горы, но, будучи захвачен злодеями, вытерпел все ужасы мучений. Эфиопы хотели у него выпытать золото и серебро, но Иоанн предлагал им одну пищу, для странников приготовленную, и злочестивцы в бешенстве перерезали у него жилы у рук и у ног.

Преподобный Сергий, церковный сосудохранитель, опасаясь посреди мучений открыть, где хранится утварь церковная, также бежал из обители и также был захвачен. Варвары повели его в Лавру, требуя, чтобы сказал, где скрыто серебро и золото; но Святой Отец не хотел идти, боясь слабости человеческой. Варвары отсекли ему голову.

Несколько иноков убежали от убийственных рук и скрылись в вертепе, но были замечены и окружены злодеями. Здесь"=то исполнились слова Спасителя: \textit{Больши сея любве никтоже имать, да кто душу свою положит за други своя} (Ин. 15, 13). Когда в темной пещере все трепетали и друг за друга прятались, преподобный Патрикий на ухо сказал им: «Не бойтесь, братия!

Я один выйду, а вы сидите как можно тише», "--- и вышел человек поистине великий и уверил, что в пещере никого более не было.

Наконец варвары, еще не насытившись кровью невинных старцев, всех, кого удержали в Лавре и кого захватили в бегстве, собрали в соборную церковь и начали вымучивать, который из них начальник обители, чтобы через него как главу братии все узнать, но иноки под смертными ударами отвечали единодушно: «Мы все равны; все уверяем, что нет у нас ни золота, ни серебра, а начальник наш для общих потребностей находится в Иерусалиме». После этого ни огонь, ни железо, ни разные пытки не могли принудить их, чтобы из среды своей они выставили настоятеля. Варвары заключили их в пещеру, где прежде подвизался святой Савва, и начали морить дымом; но восемнадцать старцев геройски там скончались, не объявив утвари, Богу принадлежащей, не обнаружив игумена, с ними же пострадавшего. Только слышимы были молитвы к Богу: «Господи! Приими в мире дух мой… Помяни мя, Господи, егда приидеши во царствии Твоем».

Варвары разграбили святую обитель. Но правосудный Бог наказал их на обратном пути. Поссорившись между собой (такова дружба злодеев и вообще порочных людей!), они истребили друг друга и пали на съедение зверям и птицам, а окаянные души их низверглись в геенну.

\section{Необоримое оружие против врагов видимых и невидимых}

Святой Никон\footnote{Преподобномученик Никон подвизался за отечество в воинском звании и принял за веру Христову венец мученический, будучи епископом, в половине III~в., в царствование Деция. Память его празднуется 23~марта (5~апреля).}, сын язычника и христианки, служил офицером в одном из римских полков, расположенных в Неаполе. Отличаясь ростом и красотой, он не менее отличался и на поле брани. Не было сражения, в котором бы Никон не получил новых ран и новых почестей.

Между тем отец его умер. Никон, стараясь поддержать славу своего дома, продолжал проливать кровь свою за отечество. Материнское сердце непрестанно страшилось смерти его, но еще более страшилось того, что сын ее умрет поклонником идолов, ибо, воспитанный в правилах отца своего, в младенчестве не имел он позволения, а в юности "--- времени быть учеником христолюбивой матери. Мудрено ли, что изуверие язычества глубоко укоренилось в его сердце? Сколько нежная мать ни старалась обратить его на путь истины, шум брани и слава воинских подвигов всегда заглушали голос веры. Наконец, чадолюбивая мать, совершенно положившись на Бога, \textit{хотящего всем спастися и в разум истины приити} (1~Тим. 2, 4), сказала ему: «Не могу внушить тебе любви к жизни вечной. По крайней мере, любезный сын, пощади для меня жизнь временную и, если на брани случится тебе быть в опасности, оградись крестным знамением. Тогда увидишь, кто есть Бог Истинный, Ему же подобает кланяться».

Спасительное действие материнских наставлений оправдалось вскоре. Римские войска выступили против варваров, произошла кровопролитная битва, победа начала клониться в пользу неприятелей, уже римские войска поражены были с тылу. Никон, имея в виду одну славу отечества, вдруг очутился один посреди врагов. Готовясь упасть под тучей стрел вражеских, бестрепетный воин вспомнил увещание матери и, устремив очи к небу, вооружился знамением креста. «Иисусе Всесильный! "--- воскликнул он. "--- Яви в сей час силу Твою, да буду и я раб Твой». Сказав это, он мужественно напал на врагов. Около ста восьмидесяти воинов со стороны вражеской пали под его ударами, и Никон не только вышел невредим, но и дал время ободриться своим соотечественникам.

Победа осталась на стороне римлян. Все воинство прославляло неустрашимость святого Никона, а он, возвратившись в объятия матери, воскликнул: «Велик Бог христианский! Я отныне раб Его и вместе с тобой, возлюбленная родительница, проклинаю веру язычества».

Христолюбивые русские воины! \textit{Аще ополчится} на вас \textit{полк} (см. Пс. 26, 3), оградитесь оружием креста и воззовите \textit{к Богу вышнему, к Богу благодеявшему} вам (см. Пс. 56, 3) "--- и Господь с Небеси пошлет помощь Свою, спасет вас и даст \textit{в поношение попирающим} вас (см. Пс. 56, 4).

\section{Божие мщение за гонимую невинность}

Преподобный Малх\footnote{Память прп. Малха Сирийского (†~IV~в.) празднуется 26~марта (8~апреля).}, захваченный на пути сарацинскими разбойниками, отведен был в Аравию и по жребию достался некоторому Мурину. Долго там святой отец смотрел за чистотой дома и пас стада господина, но, будучи не в силах более сносить разлуки с христолюбивой братией, по внушению Божию решился бежать. За ним последовала одна благочестивая старица, перенесшая одинаковую с преподобным участь.

Опаляемые зноем, томимые голодом, шли они трое суток, беспрестанно оглядываясь, нет ли за ними погони, и, наконец, увидели двух сарацин, на верблюдах за ними едущих. Несчастные обмерли от страха, но Господь милует праведников! Внезапно близ себя узрели они глубокий вертеп. Должно было выбирать одно: или опасность погибнуть от змия или зверя, которые обыкновенно гнездятся в вертепах, или неминуемую смерть от варваров. Малх и старица перекрестились и бросились в глубину логовища и, притаившись в стороне близ входа, стояли как мертвые, едва имея силы произносить только имя Христово. Немедленно варвары, приметившие их, остановились у вертепа и закричали страшным голосом: «Выходите вон, злодеи!» "--- но, не получив ответа, излили все угрозы и ругательства, потом господин с обнаженной саблей стал у входа, а раб кинулся в пещеру и миновал их, поскольку от света солнечного вошедшему вдруг в темноту видеть невозможно. Но, о Преблагий Господи! Сколь велик Промысл Твой о рабах Твоих и скорая в неизбежной гибели помощь! Внезапно из глубины вертепа выбежала львица и, схватив раба за гортань, умертвила и увлекла в логовище. Малх и старица едва смели креститься. Господин, не слыша голоса раба, подумал, что он умерщвлен беглецами, и со скрежетом устремился в пещеру, но, едва ступил несколько шагов, та же участь постигла и его: львица растерзала его, потом, взяв свое детище в зубы, спокойно вышла из логовища.

Долго в глубоком молчании трепетали Малх и старица, но, уверившись, что львица удалилась, вышли на свет. Верблюды убитых варваров были навьючены пищей и питьем и, будто именно их дожидаясь, преклонили колена.

Праведники возблагодарили Бога и, подкрепившись яствами, сели на них (на животных) и отправились в путь. Через десять дней прошли они пустыню и, встретив греческое войско, явились к полковнику, который принял их ласково.

Тогда старица постриглась в одном из девических монастырей, а преподобный Малх возвратился к своей братии, о которой столь долго сокрушался. Оба столь чудесным с ними происшествием доказывают нам, как Господь милует всех, из духовного Египта убегающих в землю, Христом обетованную.

\section{Инок, изобличивший тайные замыслы мятежного вельможи}

Преподобный Василий, так называемый Новый\footnote{Прп. Василий Новый, от самой юности поселившись в пустыне, жил в ней уединенно пятьдесят лет, другие пятьдесят препроводил в Царьграде. Часто терпел гонение и муки, иногда уважаем был не только сановниками, но и государями. Скончался в 944~г. по~Р.Х. и погребен торжественно. Память его празднуется 26~марта (8~апреля).}, имел свыше дар проникать в сокровеннейшие мысли человеческие и, если они имели целью причинить вред ближнему, невзирая на лицо, с неустрашимостью изобличал преступника, что видно из следующей повести. Роман, тесть и соправитель греческого царя Константина Багрянородного, имел у себя другого зятя, Саронита, который, как родственник императора, занимал важные должности в государстве. Но страсть властолюбия ненасытна!

Этот гордый вельможа, надеясь на свое богатство и силу, с жадностью взирал на корону и только ждал случая, чтобы сорвать ее с Константина для себя. Прозорливый старец с первого взгляда узнал намерение Саронита и сказал сам себе: «Бог взыщет с души моей, если вижу злой умысел и не покушусь уничтожить его. Пойду, изобличу злодея; может быть, прекратит безумные свои начинания… Пойду! Сам Бог хочет этого, иначе не открыл бы предо мною сердца Саронитова».

Решившись таким образом, преподобный Василий вышел на распутье в то самое время, когда Саронит по обыкновению своему ехал во дворец; Василий зашел вперед и велегласно воскликнул: «Зачем сердце твое помышляет лукавство на Христово наследие? Нет тебе части в жребии царском! Перестань и больше не мучь себя; опомнись, да не прогневается на тебя Господь за злоумышление против помазанника Его…» Изумленный Саронит сначала затрепетал, но затем, ободрившись, с яростью наскочил прямо на святого старца и, дав ему множество ударов бичом, плюнул и поехал далее.

Но бестрепетный защитник престола, сделав первый шаг, не хотел остановиться. В следующее утро опять вышел он туда же, дождался предателя и теми же словами изобличил его.

Тогда приказано было вооруженным служителям взять старца и отвести в дом его. «Безумный бродяга! "--- закричал Саронит, возвратившись из дворца. "--- Какой дьявол научил тебя так злобно нападать на меня? Разве не знаешь, кто я и что с тобой сделать могу?» Старец посмотрел на него с сожалением и отвечал спокойно: «Неужели ты думаешь, что твое злонамерение есть такая тайна, которую никто знать не может? Сам Господь открыл мне дерзость твою. Заклинаю тебя, оставь пагубное желание, чтобы вскоре не истребилась и память твоя от среды живых». Безумный Саронит, вместо того, чтобы убояться и послушать человека Божия, воспылал яростью и грозно закричал служителям: «Пытайте, мучьте его дотоле, пока не изойдет от него дух».

Три дня слабый старец бит был немилосердно и три ночи лежал еле жив в холодной яме. На четвертую ночь Господь восхотел оправдать пророчество праведника: спящий Саронит узрел в сонном видении высокий и ветвистый дуб, на котором хищный ворон имел гнездо и, сидя на нем, крылом покрывал птенцов своих; вдруг пришли два человека с секирами, и один сказал: «Этот ворон, непрестанно каркая, не дает государю уснуть спокойно». "--- «Да, он совершенно измучил и угодника Божия, который ему же хотел сделать добро», "--- сказал другой. «Должно срубить дерево!» "--- повторили они оба в один голос и подсекли огромный дуб в одну минуту, потом развели огонь и сожгли дуб.

Устрашенный Саронит от сильного трепета пробудился и почувствовал себя в нестерпимой горячке. Рассуждая о пророчестве святого Василия и о своем сновидении, он приказал немедленно выпустить святого и, повергшись в столь же великое отчаяние, сколь велико прежде было его надмение, вскоре умер.

\section{Разборчивость Иоанна пустынника в приеме посетителей}

Преподобный Палладий, однажды навестив пустынника Иоанна\footnote{Преподобный Иоанн пустынник (IV~в.), которого греко"=римский император Феодосий Великий всегда вопрошал об успехах, важнейших на пользу государства своих предприятий. Память его празднуется 29~марта (11~апреля).}, с величайшим удовольствием слушал наставление святого старца. В то время пришел один воевода, и дружелюбный Иоанн, приняв его с радостью, оставил Палладия. Но так как их разговор продолжался весьма долго, то Палладий, сидя в отдалении, наконец начал досадовать и приписывать суетности Иоанновой, что тот, презрев инока, уважает сановника. «Заплачу ему таким же презрением, "--- думал он, "--- и уйду отселе не простившись». Прозорливый старец узнал мысли посетителя, кликнул ученика своего и велел сказать Палладию, чтобы он не малодушествовал.

Когда вельможа ушел, Иоанн, обратившись к Палладию, сказал ему: «За что ты разгневался на меня? Разве нашел в поступке моем что"=нибудь для тебя оскорбительное? Любезный мой о Христе брат! Ты подумал о том, чего нет в сердце моем и что не прилично твоему сердцу. Ужели не знаешь, что \textit{не требуют здравии врача, но болящии}? (Мф. 9, 12). С тобой я увижусь, когда хочу, и ты, когда хочешь, также можешь меня увидеть. А этот воевода, связанный мирскими попечениями, которых требует от него сама его должность, может быть, едва нашел время прийти сюда и принять некоторую пользу для души своей. Итак, справедливо ли оставить его и беседовать с тобой, занимающимся единственно своим спасением?»

Удивленный Палладий познал, что этот старец вдохновлен Богом, и с восхищением рассказывал о поступке его всем пустынножителям, дабы научить их, что не должно отказывать пришельцам, сущим от мира.

\section{Непоколебимость исповедника Христова}

Мученик Марк, епископ Арефусийский\footnote{Мученик Марк, епископ Арефусийский (†~ок. 364). Память его празднуется 29~марта (11~апреля).}, был один из тех человеколюбивых и сострадательных людей, которые спасли Юлиана в его юности, когда по повелению Констанция, сына Константина Великого, истребляем был род его\footnote{Из жития св. великомученика Артемия (†~362). Память его празднуется 20~октября (2~ноября).}, нечестивый и царствующему колену враждебный.

Будучи ревностным проповедником Слова Господня еще при языческих тиранах, по крещении равноапостольного Константина святой Марк разорил некоторое идольское капище, обратил множество народа от заблуждения на путь истины и, невзирая на лица, обличал язычников "--- чем и возбудил против себя общую их ненависть. Но вера Христова и ее поборники были тогда под покровительством величайшего из царей земных.

Наконец настало время искушения: воцарился богоотступник Юлиан и арефусийские жители восстали на святого Марка. Нельзя исчислить тиранств, которые претерпел праведник. Юлиану должно было бы спасти своего избавителя, но кто пренебрег благодеяниями Иисуса, тот может ли помнить благодеяние человеческое? Богоотступник в уничижение христианства повелел ему непременно заплатить ущерб, разорением «требища»\footnote{\textit{Требище} "--- жертвенник, место приношения жертвы.} нанесенный. Если бы святой Марк сделал послабление, то вскоре бы и все храмы Господни обратились в идольские капища, а христиан, может быть, употребили бы на работы, каковыми удручал их злочестивый Диоклетиан.

Святитель Господень предвидел это и решительно отказался внести убытки, а на все муки отвечал терпением; идолопоклонники день ото дня сбавляли цену, но праведник был тем непоколебимее. Посреди всех ужасов смерти он только вопиял на небо: «Господи! Я стражду не за серебро и золото, которого у меня нет, я стражду за благочестие. Но да будет воля Твоя! Я достоин и более пострадать за то, что по неведению моему не дал истребить столь великое для всего света зло». Многие из христиан предлагали свои сокровища, чтобы удовольствовать бешенство неверных; но святой Марк не хотел благословить их даяния и даже грозил им за то отлучением.

Наконец, как повествует Феодорит, идолопоклонники так поражены были его терпением и непреклонностью, что не только согласились оставить его в покое, но некоторые и крестились. Даже градоначальник арефусийский, хотя был также язычник, совершенно преданный Юлиану, просил его пощадить Марка и представлял, сколь срамно для них истощать все средства к преодолению единого старца.

\section{О том, сколь спасительно жить в незлобии сердца и не осуждать других\footnote{Повесть прп. Анастасия, игумена Синайской горы (†~685), память которого празднуется 20~апреля (3~мая).}}

Один инок не имел всех добродетелей, предписанных уставами святых обителей, и был довольно нерадив на пути спасения в постничестве и молитвах. Наконец пришло время расстаться ему с жизнью, и собравшиеся вокруг одра его братия чрезвычайно удивились, видя, что этот, по мнению их, беспечный инок оставляет мир не только без трепета, но благодаря Бога и даже улыбаясь. «Отчего ты в грозный час Суда Божия так беспечален? "--- спросили они. "--- Мы знаем твою жизнь и не понимаем твоего равнодушия; укрепись силой Христа, Бога нашего, и скажи нам, да прославим Его милосердие». Тогда умирающий инок, приподнявшись с одра, сказал им: «Так, отцы и братия! Я жил нерадиво, и ныне все мои дела представлены мне и прочтены Ангелами Божиими; я с сокрушением признался в них и ожидал всей строгости Суда Господня. Но вдруг Ангелы сказали мне: “При всем небрежении, ты не осуждал и был незлобив” "--- и с этим словом раздрали рукописание грехов моих. Вот источник моей радости». Сказав это, инок предал с миром душу свою Господу.

\textit{Не осуждайте, да не осуждены будете; отпущайте, и отпустят вам} (Лк. 6, 37), сказал Господь. «Кто не может быть совершенным постником, "--- говорит святой Анастасий Синайский\footnote{Из Пролога, в 22"~й день марта.}, "--- не может оставить суетности мира, тот пусть не осуждает ближнего. Другой по своей бедности не может подавать милостыни или по своим занятиям часто ходить в церковь: пусть оставит ближнему, и ему оставится. Един Христос есть общий Судия: да не будем убо антихристами! Мы видим грехи брата, но видим ли его добродетели, его покаяние? Многие согрешили явно, но втайне покаялись. Разбойник, одесную Христа распятый, был человекоубийца, а Иуда был апостол: но разбойник "--- в раю, а спутник Христов "--- в геенне. Итак, осуждать может один только Сердцеведец».

\section{Казнь немилосердному}

У святого Ионы, Всероссийского митрополита\footnote{Свт. Иона, митрополит Московский и Всея России, чудотворец (†~1461). Память его празднуется 31~марта (13~апреля).}, был эконом, по имени Пимен, который, смотря за погребом, имел повеление давать пить всем пришельцам. В одно время бедная и нездоровая женщина попросила у него немного меду, чтобы укрепиться от болезни, но Пимен сердито закричал на нее: «Поди прочь! Еще не пришло время угощать вас». Бедная старица удалилась с горестью на сердце; но святитель узнал о том. Он призывает эконома и с кротостью выговаривает: «Ты не знаешь, кого оскорбил: вдовица, тобой отвергнутая, лучше всех нас угодила Богу; мне жаль тебя, но Господь послал на тебя удар смертный. Итак, пойди с миром и немедленно покайся в грехах твоих». На следующий день святой Иона повелел духовнику своему постричь его в схиму, и Пимен в тот же день умер.

\section{Злоба Юлиана на христиан и мщение, Богом на него излиянное}

Святой Григорий Богослов еще в юности своей предвидел, сколь пагубен для христианской веры Юлиан, после нареченный Отступником, и, обучаясь с ним вместе в Афинах, часто говаривал с глубоким воздыханием: «Ах! Величайшее из всех зол питает в себе земля Римская и Греческая!»\footnote{См. в Четии"=Минее житие свт. Григория Богослова. Память его празднуется 25~января (7~февраля).} Это пророчество совершенно оправдалось, когда Юлиан сделался императором. Благоволение к еврейскому народу и упорное желание построить храм Иерусалимский, уважение к языческим жрецам и презрение к христианским епископам, непрестанные кощунства над Евангельской верой и казни, если кто скажет хоть что"=нибудь предосудительное об идолопоклонстве, "--- все предвещало падение христианства.

Но богоотступник особенно ненавидел Кесарию Каппадокийскую. Причиной такого ожесточения было не что иное, как ревность по Иисусе тамошних христиан, бывших под архипастырством Василия Великого. А разорение одного из идольских капищ, в котором он, приезжая в Кесарию, обыкновенно приносил жертвы Фортуне, дополнило чашу его бешенства\footnote{См. в Четии"=Минее житие святого мученика Евпсихия (†~362), Память его празднуется 9~(22) апреля.}. Тогда святой Евпсихий умер в жесточайших мучениях. Множество граждан взято под стражу; одни наказаны смертью, другие посланы в заточение. У многих отобраны имения; церкви ограблены; церковнослужители расписаны по дальним полкам, а город был обременен чрезвычайными налогами.

Василий Великий трепетал от ужаса, видя тирана, готового истребить веру Христову, и сокрушался о злополучии граждан; но в особе царя земного не переставал почитать власть и силу Царя Небесного. Воздавая \textit{Божия Богови} и \textit{кесарева кесареви}, он молился и побуждал молиться духовных чад своих к Господу сил, чтобы умудрил и обратил на правый путь Юлиана, но \textit{в злохудожну душу не внидет премудрость} (Прем. 1, 4). Богоотступник не чувствовал, сколь великого имеет за себя молитвенника, и восхотел до конца посрамить Кесарию в лице ее святого архипастыря, что происходило следующим образом.

Отправляясь на войну против Персии, Юлиан шел с воинством через Каппадокию. Василий Великий вышел ему навстречу с освященным собором и с лучшими гражданами, в сопровождении многочисленного народа, который, ненавидя христоненавистника, подвигся единственно на глас своего архипастыря. Угодник Божий вынес ему на благословение три чистых хлеба, из ячменной муки приготовленные, каковыми и сам обыкновенно питался. Что же сделал Юлиан? Он велел оруженосцам принять оное, а святому Василию дать горсть сена, с ругательством сказав: «Ты поднес нам ячмень "--- пищу, скотами употребляемую; прими же от нас сено». «О, государь! "--- отвечал огорченный до глубины сердца Василий. "--- Мы принесли тебе то, что сами едим, а ты, воздавая скотской пищей, воистину, ругаешься, ибо не можешь властью своей сено претворить в хлеб и пищу человеческую». Тогда Юлиан, с яростью воззрев на святителя, воскликнул: «Будь же уверен, что стану тебя кормить сеном, когда возвращусь из Персии, разорю город этот до основания, все место изглажу плугом, да родит жито, а не людей. Но нет! Этого для вас мало! Я повелю самые развалины посыпать солью: тогда некому будет внимать твоим советам, некому будет восставать против моих намерений. Восчувствуешь, дерзкий старец, всю силу гнева Юлианова!» Упоенный славой будущих побед, Юлиан пошел далее, а святитель, отринутый и поруганный, продолжал молиться Господу, да благодатью Своей просветит преступника и пошлет на него дух умиления и покаяния.

Между тем Юлиан, вступив в персидские пределы, рассеял несколько неприятельских отрядов. Принося за это благодарственные жертвы идолам, он объявил воинству и поклялся всенародно пред истуканом Зевса, что по окончании войны непременно истребит христианство. Узнав об этом, Василий Великий потерял всякую надежду видеть обращение отступника.

Должно было или ожидать снова курящихся по всей земле жертв идолам или потребить от лица земли сильного и хитрого безбожника. Праведник, оставшись один в церкви, повергся пред образом Пресвятой Богородицы, где изображен был с копьем и воин Христов Меркурий\footnote{Из жития св. великомученика Меркурия (†~III~в.), память которого празднуется 24~ноября (7~декабря).}; он молился, чтобы гонитель христианства живым не возвратился с войны. Хотя Всемирная Заступница, не хотящая смерти грешников, запрещает молиться о погибели их, но, поскольку Юлиан не хотел обратиться и жив быти, Матерь Божия Своим благословением утвердила смерть его. Посреди слезных молитв Василия Великого вдруг святого страстотерпца на иконе не стало. Архипастырь ужаснулся. Но через некоторое время воин Христов показался опять "--- с копьем окровавленным. В тот самый час, как после узнали, Юлиан поражен был в битве рукой невидимой.

\section{Свои и чужие грехи}

Некий инок учинил прегрешение. Братия, собравшись рассуждать о его деле, послали за аввой Моисеем, но смиренный старец отказался быть в их совете. Скитский пресвитер приказал сходить за ним вторично и сказать, что братия ожидают его всем собором. Тогда Моисей пошел, но каким образом? Он насыпал в ветхую кошницу песку и понес с собой. «Что это значит, авва?» "--- встречая его, спросили иноки. «Видите, сколько за мною грехов? "--- указывая на сыплющийся песок, отвечал святой Моисей. "--- Я не вижу их, а между тем иду совершать суд над другим!» Услышав это, иноки ничего не сказали согрешившему брату и безмолвно простили его.

\section{Видение преподобного Зосимы об участи новгородских бояр\footnote{Прп. Зосима, игумен Соловецкий (†~1478), родившийся в селе Толвуе Новгородской области, при Онежском озере, устроил Соловецкую обитель, собрал братию, построил церковь и иноческие кельи и умер в глубокой старости. Память его празднуется 17~(30) апреля (см. Жития святых).}}

Когда Соловецкая обитель украсилась зданием, а что важнее всего, добродетельными иноками и прославилась подвигами и чудесами святого настоятеля, враг рода человеческого, ненавидя добро, покусился уничтожить ее и подвигнул на брань злобных людей. Слуги новгородских бояр и карельских помещиков, приходя на рыбную ловлю, называли себя единственными господами острова и не допускали иноков ни к одному озеру. Благочестивые старцы с кротостью говорили им о своей нужде в житейских потребностях, но злые люди отвечали одним ругательством, делали разные обиды и наглости и угрожали разогнать иноков.

Последняя угроза побудила преподобного Зосиму идти в Новгород, чтобы испросить защиту святой обители. Там, по совету архиепископа Феофила, он был у всех бояр и от всех получил уверение в их покровительстве. Одна гордая Марфа, новгородская посадница, с бесчестием выгнала его из дому, и раб Божий, уходя от нее, сказал ученикам своим: «Се дние грядут, в няже жителю дому этого не исследят стопами своими двора этого».

Архиепископ Феофил, опасаясь, чтобы Марфа, которая сильно действовала на умы новгородцев, не сделала для обители чего"=нибудь худшего, принял дело Зосимы на себя и довел бояр до того, что они с общего согласия уступили в церковное владение весь остров и дали «жалованную грамоту». Сверх того, одарили храм Господень церковными сосудами, священническими одеждами и окладами на святые образа.

Тогда и честолюбивая посадница почувствовала стыд, а общий отзыв о благочестии и святости Зосимы заставил ее раскаяться в оскорблении, ему причиненном: она пригласила его на обед. Кроткий старец, подавая пример незлобия, пришел в дом ее, но посреди общего веселья вдруг удивился и в безмолвии опустил очи свои долу; опять воззрел на пирующих бояр и опять преклонил главу свою; то же сделал и в третий раз; потом вздохнул и прослезился. Сколько ни просили его, старец не мог вкушать пищи и питья.

По окончании стола Марфа извинялась пред ним в своей, как говорила она, неосторожности и подарила Соловецкой обители деревню, лежащую при реке Суме. Когда преподобный Зосима возвращался домой, сопровождавший его ученик Даниил осмелился спросить о причине удивления и слез за трапезой посадницы. «Чадо! "--- отвечал старец. "--- Горестно и самому мне знать ужасную тайну, не только открывать другим, однако познай судьбы Божии, имеющие сбыться в свое время: я видел шесть знаменитейших бояр новгородских, без глав на пиршестве сидящих, и ужасался. После этого можно ли мне было участвовать в трапезе? Думаю, что они будут некогда обезглавлены; но, чадо! Сохрани в сердце твоем, что слышал от меня».

Вскоре пророчество святого Зосимы оправдалось на деле. Великий князь Иоанн Васильевич, утверждая в России благословенную власть самодержавия, овладел Новгородом и за непокорство казнил тех бояр, которых святой старец видел без глав на пиршестве. Марфа Борецкая была отослана с детьми своими в заточение, дом ее разрушен, и само место, где он стоял, поросло травой.

\section{Святыня просфоры}

Преподобный Зосима Соловецкий в одно время дал приехавшим на остров купцам от своего священнодействия просфору. Но они, идучи из церкви, по неосторожности обронили ее. Инок Макарий, случайно проходя мимо, увидел стоящего над просфорой пса, который всемерно старался схватить ее зубами, но при каждом покушении опаляем был огнем, от святого хлеба исходящим. Макарий приблизился "--- и огня не стало; перекрестившись, поднял он просфору и, принеся к преподобному Зосиме, рассказал о чудном видении. Все удивились столь поражающему уроку Христа Спасителя, в Его же славу совершается \textit{Святая Святых}.

Христиане! Из этого познайте святость хлебов, приемлемых вами от алтаря Господня, и вкушайте их с чистым сердцем, с благоговением к Богу; в противном случае бойтесь, да огонь гнева Божия не опалит вас.

\section{Благодарность уволенного раба}

Некоторый старец, бывший прежде рабом, обитая в ските, каждый год ходил в Александрию с тем, чтобы отдать оброчные деньги прежним господам, для расплаты с другим, который должен был вместо него служить. Благочестивые господа, воздавая должную справедливость святому человеку, всегда с глубоким почтением встречали его и, наконец, едва упросили впредь не делать этого. Однако старец и тогда не перестал платить им "--- только другим образом: принося в сосуде воду, он умывал ноги их. «Авва! Ты обременяешь себя напрасно», "--- говорили благочестивые хозяева. Но авва отвечал им: «Я раб ваш и не могу по достоинству возблагодарить вас за то, что позволили мне беспрепятственно служить Богу. Вы не принимаете денег: по крайней мере, вот мой оброк, который я могу воздать вам!» "--- указывая на сосуд с водой, продолжал он. Добрые господа делали ему возражения и доказывали, что они не могут быть спокойны, пока будут видеть его беспокойство, но старец сказал решительно: «Если не хотите принимать этой слабой жертвы, я останусь по"=прежнему у вас слугой». Благочестивые хозяева, опасаясь этого, наконец позволили ему делать, что хочет, и отпускали его с надлежащей честью, наградив всем потребным для жизни, так что он мог подавать и за них милостыню. Этот старец был славен и любим в ските.

\section{Божия помощь милосердному}

Некогда к преподобному Феодору Сикеотскому\footnote{См. Житие прп. Феодора Сикеота, епископа Анастасиупольского (†~613). Память его празднуется 22~апреля (5~мая).} пришел из Илиополя Галатийского церковный эконом и, обливаясь слезами, сказал: «Помилуй меня, раб Божий, и помоги моему несчастию! Я поручил служителю собрать церковные доходы, а этот бесчестный человек бежал с деньгами; сколько ни искали его, но найти не могли. Помолись Господу Богу, чтобы показал мне, где скрылся он, "--- иначе я погибну, ибо все мое имение не довлеет к тому, чтобы воздать церкви за похищенное им». "--- «А не будешь ли бить твоего служителя? "--- спросил преподобный Феодор. "--- Не будешь ли от него требовать с лихвой? Если дашь верное слово, то Господь тебя обрадует и беглеца предаст в руки твои». Эконом поклялся даже наградить его. Тогда Сикеотский чудотворец сказал ему: «Иди с миром в дом твой и будь благонадежен; уповаю на Бога, что скоро поможет тебе».

В то самое время Господь молитвами святого Феодора связал хищника. Он остановился близ некоторой веси и не мог ступить далее, думал, что бежит, и пребывал на одном месте. Тут он был узнан, взят и приведен к эконому, все церковное имущество возвращено, и хищник наказан только своей совестью. Таким образом, по требованию преподобного Феодора счастье одного не сделалось видимым злосчастием другого.

\section{Сила крестного знамения}

Когда святой Иулиан был возведен на епископский престол в Бостре, тогда некоторые из граждан, ненавистники имени Христова, вознамерились умертвить ревностного архипастыря. Они подкупили его кравчего, и этот злочестивец, подавая святителю пить, в чашу положил яду, но святой Иулиан, едва принял смертоносное питье в руки, познал свыше не только о злоумышлении, но и обо всех злоумышленниках. Он поставил пред собой чашу и, не сказывая о том служителю, велел позвать к себе всех граждан, бывших в заговоре. Когда пришли они, праведник, не желая обнаружить их злодеяния, сказал тихо: «Если хотите умертвить смиренного Иулиана ядом, се пред вами пью чашу сию». Потом трижды перекрестил оную и, сказав: «Во имя Отца и Сына и Святаго Духа», выпил "--- и остался невредим. Пораженные ужасом, злодеи припали к стопам его и умоляли, чтобы он простил их.

Христиане! И Российская история представляет разительный пример, сколь необорима сила крестного знамения. В смутные времена самозванцев, когда злонамеренные бояре тщетно принуждали святейшего патриарха Гермогена, чтобы тот согласился принять на Всероссийский престол польского короля Сигизмунда, один из вельмож устремился на архиерея Божия с кинжалом. Гермоген тотчас оградил его крестным знамением и едва сказал: «Буди на тебе сила крестная», "--- оружие выпало из рук убийцы, и злоумышленники хотя скрежетали зубами, но к человеку Божию на тот раз прикоснуться не смели.

\section{Чудотворная сила святых икон\footnote{См. в чудесах св. великомученика Георгия. Память его празднуется 23~апреля (6~мая).}}

Когда сарацины овладели сирийским городом Рамеллией, несколько этих варваров из любопытства вошли в церковь святого великомученика Георгия. Один из них, видя священника, молящегося на коленах пред образом Победоносца, с насмешкой сказал товарищам: «Вот безумец, доске поклоняющийся!» "--- и, вынув из колчана стрелу, пустил в икону, но стрела, вопреки направлению, устремилась вверх и, с силой низвергшись, пронзила руку у варвара. Почувствовав нестерпимую боль, он поспешно ушел домой; болезнь ежеминутно усиливалась, и вскоре вся рука опухла.

Сарацин ожидал пагубных последствий, но, к счастью, в услужении у него жила христианка. Узнав, что случилось в церкви, она сказала ему: «Я человек простой, о святости и чудесах веры говорить не умею, но знаю, что проступок твой заслуживает смерти. Если хочешь быть здрав, то призови священника и спроси у него, что делать должно». Сколько сарацин ни был ожесточен против христиан, но, видя опасность, с радостью согласился на предложение служанки, и священник пришел. «Какую силу имеет образ, которому ты в бытность нашу в церкви столь усердно поклонялся?» "--- спросил у него сарацин. «Всю силу имеет только один Бог, "--- отвечал священник. "--- Ему я и поклонялся, а святого великомученика Георгия, тут изображенного, молил о том, да будет о мне ходатаем к Богу». "--- «Итак, Георгий не Бог ваш; кто же он?» "--- опять спросил варвар. «Он возлюбленный слуга Божий, "--- сказал священник. "--- Был человек, нам подобный, но, за славу имени Христова претерпев мучение и смерть, получил от Бога благодать чудодействовать. Любя угодника Божия, мы почитаем его икону и, взирая как бы на самого, поклоняемся ей, лобызаем ее. Дерзость твоя, столь очевидно наказанная, свидетельствует, что христиане с умилением припадают не к простому дереву». "--- «Что же мне присоветуешь сделать? "--- спросил поколебавшийся в неверии варвар. "--- Посмотри на мою руку! Я мучаюсь и, без сомнения, умру». "--- «Если хочешь быть жив, "--- отвечал пресвитер, "--- то вели принести в дом твой образ святого Георгия, поставь над твоим ложем и возжги лампаду елея. Пусть она горит всю ночь; а на другой день помажь им руку твою и веруй, что исцелеешь». Немедленно святая икона была принесена. Сарацин поступил так, как приказал священник, и увидел, что опухоль опала и рана закрылась. В удивлении и восхищении он хотел выслушать повесть о святом великомученике, и, в то время как священник читал его житие и страдания, сарацин, держа в руках икону, сквозь слезы говорил: «О святой Георгий! Ты был юн, но премудр, а я стар, но безумен; млад, но Богу любезен, а я долголетен, но гнусен пред Богом, моли обо мне Господа твоего, да учинит меня рабом Своим». Потом припал к ногам священнослужителя и умолял даровать ему Святое Крещение.

В тот же вечер этот варвар, как агнец из волка, из язычника сделался христианином, а на следующий день, нетрепетно проповедуя имя Христово, от рук прежних своих единоверцев скончался смертью мучеников.

\section{Георгий Победоносец}

В стране Сирофиникийской из одного великого озера, которое простиралось от горы Ливанской, выходил огромный, крылатый и когтями вооруженный змий и, кого ни встречал, всех, увлекая в воды, пожирал, отчего близлежащий город Верит\footnote{Древнее название города Бейрута.} находился в беспрестанном страхе. Много раз народ вооружался, чтобы убить его, но змий умерщвлял отважнейших издалека одним дыханием, и народ всегда с ужасом убегал назад. Общее уныние овладело сердцами, ибо чудовище, своим ядом заражая воздух, по их мнению, вливало смерть даже в плоды земные.

Наконец, жители (они все были идолопоклонники) оставили домашние и общественные занятия, непрестанно собирались на городскую площадь и рассуждали, что им в общем злосчастии должно делать. Не зная, на что решиться, приступили они к своему князю и громко вопияли: «Погибаем от змия! Спаси нас! Ты начальник и отец!» "--- «Не знаю, что повелят мне предпринять боги, "--- отвечал им князь. "--- Я буду молить их, чтобы открыли мне волю свою». "--- И пошел советоваться с изуверными кумирослужителями.

Как обрадовался дух геенский, услышав намерение князя веритского! Он весь вгнездился в ум и сердце жрецов своих и устами их дал следующий ответ: «Если не хотите погибнуть все, то жителям Верита, не исключая и князя, должно каждый день по жребию отдавать на съедение змию детей своих, сына или дочь». Проклятый совет был принят, и, поступая по оному, начали приносить богомерзкую жертву прежде сановники, потом народ. Каждое утро несчастный город наполнялся воплем и рыданием, каждое утро оплакивали отрока или отроковицу, обреченных на смерть. Родители и сродники, терзая волосы свои, приводили их к озеру и тут оставляли. Немедленно из глубины вод являлось чудовище и пожирало несчастную жертву.

Наконец, пришла очередь и князю отдать единородную дочь. Он приказал ей облечься в лучшие одежды и, оплакав, как умершую, выслал на берег озера, а сам, скрепив сердце свое, смотрел с высокой башни, как будет погибать отроковица… Стоявшая на берегу княжна, ожидая лютого часа, рыдала и, ломая руки свои, возводила к небесам помертвевшие очи.

Но Бог, желающий всем спастись, восхотел избавить град этот от власти змия и ада. По Его всесильному мановению вдруг является святой великомученик Георгий в образе молодого всадника, копьем вооруженный, и, приблизившись к скорбящей девице, спрашивает ее, зачем она стоит у озера и так горько плачет. «Беги отсюда как возможно скорее, добрый юноша! "--- отвечала княжна, не зная, что это святой великомученик Георгий. "--- Беги, чтобы не погибнуть вместе со мною!» "--- «Не страшись о мне, "--- сказал Победоносец, "--- но скажи, чего ожидаешь здесь, между тем как из города весь народ смотрит на тебя?» Тогда девица вкратце сказала ему о лютости змия, о несчастии их города, о ежедневном жертвоприношении детей и опять начала просить его, чтобы без нужды не подвергал опасности жизнь свою: «Вижу, что ты, добродушный юноша, мужествен и силен, но против лютого чудовища и тысячи устоять не могут; итак, беги и спасайся, иначе сам погибнешь, а меня не избавишь». "--- «Во имя Господа моего Иисуса Христа все возможно!» "--- отвечал Георгий. В это мгновение ужасный змий показался на поверхности вод, быстро приближаясь к добыче. Девица испустила отчаянный вопль, а святой Победоносец, оградив себя крестным знамением, сказал: «Во имя Отца и Сына и Святого Духа» "--- и устремился на чудовище, потрясая копьем своим. Мгновенно ударил его в самую гортань и, притиснув к земле, попирал разъяренным конем; потом приказал девице привязать его к своему поясу и вести в город, как кроткого пса… Народ ужаснулся при столь внезапном зрелище, но Победоносец воодушевил их бодростью. «Не бойтесь, "--- говорил он, "--- но уповайте на Господа и веруйте в Него. Сей Господь послал меня истребить чудовище, вас пожиравшее». После этого посреди города умертвил змия и, повелев сжечь труп его, стал невидим.

Князь и весь народ, познав, что это был святой великомученик и Победоносец Георгий, приняли Святое Крещение. Двадцать пять тысяч народа, кроме жен и детей, возродились водой и Духом.

\section{Непрестанная молитва}

Некогда к Лукию, старцу Енатскому, пришли иноки, так называемые евхиты, или молитвенники\footnote{Одна из еретических сект. Последователи ее, между множеством заблуждений, отвергали все добродетели, даже силу таинств, в рассуждении нашего спасения, и всю важность приписывали одним молитвам.}. «Каким занимаетесь рукоделием?» "--- спросил у них Лукий. «Мы не занимаемся работой рук, "--- отвечали они, "--- но, по заповеди апостольской, непрестанно молимся». "--- «А едите ли?» "--- опять спросил старец. «Едим», "--- отвечали они. «Итак, когда едите, кто тогда за вас молится? "--- возразил Лукий. "--- Сверх того, спите ли вы?» "--- «Как же можно не спать!» "--- сказали евхиты. «Итак, когда спите, кто тогда за вас молится?» "--- опять возразил старец. Пришельцы не могли дать никакого ответа. Тогда преподобный Лукий сказал им: «Простите мою откровенность: вы сами себе противоречите. Я, напротив того, докажу вам, что, и занимаясь рукоделием, молюсь непрестанно. Например, плету из камыша корзины и читаю: \textit{Помилуй мя, Боже, по велицей милости Твоей, и по множеству щедрот Твоих очисти беззаконие мое…} (Пс. 50, 3). Не молитва ли это?» "--- «Конечно!» "--- отвечали евхиты. «Таким образом, проведя весь день в труде и молитве, "--- продолжал старец, "--- я вырабатываю немного денег: половину из них отдаю нищим, а другую употребляю на свои потребности. Когда же ем или сплю, тогда молятся за меня приемшие милостыню. Не видите ли, что я, с помощью Божией, исполняю правило апостола и молюсь непрестанно?»

\section{Великопермский апостол}

Святой Стефан\footnote{Свт. Стефан, епископ Великопермский (†~1396). Память его празднуется 26~апреля (9~мая).}, епископ Пермский, будучи иеромонахом, восприял твердое намерение "--- просветить крещением страну Пермскую. Слыша, что жители ее, покрытые мраком идолопоклонства, чрезвычайно привержены к волшебству и чародействам, он объят был священной ревностью апостолов "--- спасти от погибели души человеческие, изведя их от тьмы и сени смертной к солнцу истинного Богопознания. На этот конец начал он учиться пермскому языку и через неутомимое прилежание успел столько, что вскоре изобрел пермские буквы и на их наречие переложил несколько Священных книг. Но это для великого пастыря казалось недостаточным: желая паче научиться Богомудрию, он после этого устремил все свое внимание на язык греческий, в котором также достиг успехов.

Между тем Божественное рвение умножалось в сердце его более и более. Святой Стефан с пощением и слезами молился день и ночь Господу, да по Своему произволению устроит путь его. Потом испросил благословение от епископа Коломенского Герасима, бывшего тогда блюстителем Всероссийской митрополии, и, по словам апостола: \textit{Вам даровася, еже о Христе, не токмо еже в Него веровати, но и еже по Нем страдати} (Флп. 1, 29), "--- отправился в Пермь. Положив на сердце своем или обратить неверных к Спасителю мира, или принять за Него венец мученический, неусыпный священноинок начал проповедовать Истинного Бога, один посреди рода строптивого и развращенного, как агнец посреди волков. Неверные удивлялись неслыханному учению, мало"=помалу познавали истину и принимали Святое Крещение. И хотя большая часть идолопоклонников не хотели слышать угодника Божия, делали ему разные пакости, ругались и не однажды намерены были убить его, но Евангельская ревность все преодолела. Преподобный Стефан соорудил храм Господень, по новоизобретенной пермской азбуке обучил грамоте нескольких тамошних уроженцев и через них более и более распространял веру Христову.

Но едва неусыпный делатель винограда Господня начал пожинать плоды своих подвигов, вдруг появился «кудесник», старейшина всех чародеев, которого идолопоклонники почитали отцом, наставником и главой их суеверных обрядов при жертвоприношениях. Он начал новопросвещенных людей совращать с пути истины. Но, к счастью, большая часть христиан не поверила его увещаниям и предложила ему состязаться с преподобным Стефаном. Волхв согласился, но все его бредни опровергаемы были учением Евангельским, все его чарования и волшебства низлагались молитвой равноапостольного пастыря. Наконец, отчаянный жрец прибегнул к самому последнему средству: хотел устрашить. «Перейдем оба сквозь огонь и воду, "--- сказал он Стефану, "--- и кто не сгорит и не утонет, того вера, как истинная и Богу приятная, должна в Пермской стране господствовать». Но святой пастырь отвечал нетрепетно: «Ты от меня требуешь вещи, которая превосходит силы человеческие; однако уповаю на милость Всемогущего Бога, \textit{Иже всем хощет спастися и в разум истины приити}» (1~Тим. 2, 4). Благословен Господь Бог! "--- продолжал он, обратившись к народу. "--- Принесите сюда огня и зажгите пустую храмину, я и чародей, взяв друг друга за руку, войдем в нее».

Немедленно по приказанию святого Стефана все было исполнено. Тогда, воздев руки свои к Богу, он излил слезную молитву: «Владыка Всемилостивый и Всемогущий! Яви человеколюбие Твое, покажи силу Твою, да разумеют предстоящие люди, яко вера Твоя истинна; \textit{се врази Твои возшумеша, и ненавидящии Тя воздвигоша главу} (Пс. 82, 3). \textit{Положиша на небеси уста своя, и язык их прейде по земли} (Пс. 72, 9). Сего ради \textit{сотвори со мною знамение во благо: и да видят ненавидящии мя, и постыдятся, яко Ты, Господи, помогл ми и утешил мя еси} (Пс. 85, 17)». Совершив молитву, святой пастырь воскликнул к предстоящему народу: «Мир вам, братия и чада! Простите и молитесь обо мне, я готов умереть за святую веру». Потом, крепко взяв волшебника за руку, повлек в пылающий огонь. Но куда исчезла сила чародейства?

Кудесник затрепетал: он падал на землю, кричал, просил пощады, между тем народ вопиял с ругательством: «Да идет в огонь, поскольку сам предложил этот опыт!» Таким же образом языческий волхв поруган был и у реки, где положено было спуститься в одну прорубь и выйти из другой.

Тогда преподобный Стефан сказал ему: «Хочешь ли веровать во Иисуса Христа и просветиться крещением?» Но так как ожесточенный изувер не хотел исполнить данного слова, то народ возопил единогласно: «Злодей повинен смерти!» "--- и хотел растерзать его. Одно милосердие святого Стефана спасло его. «Не буди рука наша на враге нашем, "--- удерживая раздраженный народ, говорил он, "--- ибо Христос не убивать послал меня, но благовестить, велел не мутить, но учить с кротостью, по словам порфироносного пророка: \textit{Накажет мя праведник милостью и обличит мя} (Пс. 140, 5), а если он, ожесточенный злобой, веровать не хочет, то вечная казнь готова ему в геенне огненной». После этого он повелел изгнать его из пределов пермских.

Тогда новонасажденная Церковь осенилась миром, и повсюду на развалинах идольских капищ созидаемы были храмы Господни. Наконец, преподобный Стефан почел за необходимое, чтобы Пермская страна имела епископа, и с общего согласия христиан послал прошение к великому князю Димитрию Иоанновичу и преосвященному митрополиту Пимену, возвещая об успехах православной веры и требуя архипастыря. Но кто более достоин был сана архиерейского, как не тот, кто совершил столько трудов и подвигов? Равноапостольный проповедник Евангелия вызван был в Москву и, рукоположенный во епископа, возвратился к любезной пастве, от Небесного Пастыреначальника ему врученной\footnote{Святитель Христов Стефан преставился в Москве, прибыв туда по делам своей паствы.}.

\section{Клятва, которую нарушить можно}

К одному старцу пришел незнакомый человек и сказал: «Я и брат мой друг с другом поссорились, и, к несчастью, он не хочет примириться, хотя я всеми мерами стараюсь о том. Человек Божий, сделай милость, уговори его». Старец с радостью принял на себя дело пришельца и, призвав к себе брата его, начал говорить о любви и согласии. Сначала казалось, что ожесточенный брат смягчается, но вдруг он сказал: «Не могу примириться, ибо крестом поклялся вечно враждовать с ним». Тогда старец, улыбаясь, сказал ему: «Клятва твоя имеет такую силу: Сладчайший Иисусе! Заклинаю себя Крестом Твоим, что не буду исполнять Твоих заповедей и хочу повиноваться воле врага Твоего "--- дьявола. Друг мой! Не только должно отвергнуть то, на что мы в злой час решились, но должно раскаяться в оном; должно сокрушаться о том, в чем против души своей согрешили. Если бы Ирод раскаялся и не поступил по своей клятве, то не сделал бы величайшего на свете злодеяния, не умертвил бы Предтечу Христова. Вспомни, что было тогда, когда Иисус хотел умыть ноги святого Петра, а он отрицался! (см. Ин. 13, 8--10). Вспомни, что этот же апостол с клятвой утверждал: \textit{Не знаю Человека} (Мф. 26, 74). Сдержал ли он свое клятвенное слово? Долго ли продолжал свое неведение о Человеке? \textit{И изшед вон плакася горько} (Мф. 26, 75)». Выслушав истину из уст старца, незнакомец в тот же час примирился с братом своим.

Христиане! Если данная нами в какой"=либо шумной и порочной страсти клятва влечет за собой гибельные последствия для нашего спасения или для счастья ближних, немедленно смойте ее вашими слезами: пагубно исполнять оную. Но чтоб избегнуть опасности, в которой находился брат, Крестом поклявшийся враждовать на брата, то лучшее средство "--- никогда не клясться без нужды и в маловажных случаях. Если ты человек совестный, справедливый, благоразумный и чистосердечный, то одному слову «да» или «нет» поверит каждый. Напротив того, дурной человек хотя надрывайся от клятв "--- все будет напрасно. На что же без разбора столь легкомысленно призывать имя Того, Кому предстоят с трепетом чины Ангельские и, от страха закрывая лица свои, взывают: \textit{Свят, Свят, Свят Господь Саваоф: исполнь вся земля славы Его} (Ис. 6, 3).

\begin{center}\small\textsc{Конец третьей части.}\end{center}

\chapter{ЧАСТЬ ЧЕТВЕРТАЯ}
\section{Видение воина"=патриота\footnote{Из жития св. пророка Иеремии (VI~в. до~Р.Х.). Память его празднуется 1~(14) мая.}}

Когда Иуда Маккавей за отечество и Ветхозаветную Церковь ополчался против злочестивого Антиоха, то однажды, в глубоком размышлении воспоминая славу Израиля и чудесную помощь, много раз с Небес ему ниспосылавшуюся, потом рассуждая об ужасных обстоятельствах настоящего времени и изыскивая средства спасти народ свой, вдруг узрел в видении архиерея Онию (см. 2~Мак. 15, 12), мужа благоговейного, кроткого и благодетельного. Первосвященник (так видел Маккавей), воздевая руки к небу, молился о мире и спасении всего народа иудейского. Благочестивый военачальник только что хотел повергнуться на землю и с ним же излить сердце свое к Богу, как вдруг явился другой муж, сединами убеленный, Небесной славой сияющий и подобный более Божеству, нежели человеку. Тогда Ония сказал: «Се Иеремия, пророк Божий», "--- и Иеремия в то же мгновение, простерши руку свою к Маккавею, подал ему златой меч и возгласил: «Приими дар Божий! Им сокрушишь всех врагов веры и отечества». После этого пророк и архиерей стали невидимы.

Христолюбивые воины! Из этого видения, которого свыше удостоился величайший из военачальников, познайте, что святые угодники Божии и по кончине своей молятся за вас к Господу и вспомоществуют вам против супостатов.

\section{Верность своим господам}

Благочестивые супруги Еспер и Зоя и дети их Кириак и Феодул\footnote{Святые мученики Еспер, Зоя, Кириак и Феодул жили в царствование императора Адриана, во II~в. См.: Жития святых, 2~(15) мая.} были рабами одного римлянина, по имени Катал. Невзирая на то, что господин их и весь дом поклонялись идолам, они жили по заповедям Иисуса Христа "--- были трудолюбивы: что возлагали на них, исполняли с усердием и верностью отличались пред всеми домочадцами.

В одно время блаженный Еспер по приказанию господина отправлен был в дальнее поместье, отчего Кириак и Феодул почувствовали огорчение и вскоре начали задумываться. Святая Зоя приписывала печаль детей своих то их нездоровью, то беспрестанным работам, то разлуке с родителем, но, к удивлению своему, вскоре узнала другое; ибо, когда с материнской нежностью начала выведывать у них причину душевного расстройства, дети сказали ей: «Любезная родительница! Мы не можем более жить с нечестивым Каталом. Ты сама из Священного Писания говорила нам: \textit{Не бывайте удобь преложни ко иному ярму, якоже невернии} (2~Кор. 6, 14). Как же сохраним заповеди Господни, если не уйдем от идолопоклонников? Мы боимся, чтобы, обращаясь между ними, не были от Бога поставлены с ними вместе ошуюю и не погибли навеки. Благослови нас: мы хотим бежать». Благочестивая мать изумилась, услышав предложение детей своих, и, с гневом воззрев на них, отвечала: «Как могло прийти вам на мысль столь нелепое намерение? Вы толкуете неправильно слова Спасителя нашего: от вас только требуется, чтобы от чистоты христианской вы не прилагались к нечистотам языческим. А что касается до повиновения господам своим, то послушайте, что заповедует ревностнейший ученик Христов: \textit{Раби}, "--- говорит он, "--- \textit{послушайте господий своих по плоти со страхом и трепетом, в простоте сердца вашего, якоже и Христа, не пред очима точию работающе яко человекоугодницы, но якоже раби Христовы, творяще волю Божию от души: ведяще, яко кийждо, еже аще сотворит благое, сие приимет от Господа, аще раб, аще свободь} (Еф. 6, 5~-- 6, 8). Подумайте же, имеете ли вы хоть малейшее право уйти от господ своих, когда они имеют полную власть над телом нашим? Ваше дело, служить им до смерти, а кого поставить одесную и кого ошуюю "--- это, судя по делам их и вашим, решит Сам Господь после смерти».

Тогда послушные дети познали свое заблуждение и, умоляя мать свою, чтобы простила их, сказали: «По крайней мере, из подражания Христу, Который предал Себя в руки иудеев, дабы через то искупить от геенского ига весь род человеческий, мы просим тебя, любезная родительница, позволить и нам искупить себя от язычника Катала своею кровью, если будет принуждать нас к жертвоприношению своим идолам». Святая Зоя благословила Кириака и Феодула на подвиг страдальчества и страшилась только, чтобы мучения не поколебали твердости их в вере Евангельской.

\section{Каждое дело должно начинать с благословением}

Однажды преподобного Феодосия Печерского\footnote{Прп. Феодосий, игумен Киево"=Печерский, постригся в 1023~г., преставился в 1074~г. Память его празднуется 3~(16) мая и 14~(27) августа.} посетил великий князь Изяслав и остался у него обедать. На простом столе поставлена была простая пища, какую употребляли в праздничные дни иночествующие постники, но эта пища великому князю показалась чрезвычайно вкусной. Видя хлеб не очень белый и обыкновенные плоды, Изяслав не понимал, отчего с такой сладостью вкушает он, и, наконец, с удивлением спросил о причине своего удовольствия за столом старческим. «У меня, "--- сказал он, "--- всего довольно: припасы лучшие, разные приправы выписываются из чужих земель, повара свое дело знают, но редко могу есть с некоторым только побуждением к пище». "--- «Государь! "--- отвечал Феодосий. "--- У нас братия, когда хотят варить яства или печь хлебы, имеют следующий устав.

Сначала они приходят к настоятелю и берут у него благословение. Потом делают три поклона пред святым алтарем и, возжегши свечу от елея, горящего пред образом Господним, ею разводят огонь в поварне или в хлебне. Также, когда надобно лить в котел воду, служитель говорит старейшине: \textit{Благослови, отче!} "--- а старейшина отвечает: \textit{Бог да благословит вас!} Таким образом, у нас все делается с благословением, и вот причина, что пища наша имеет вкус и сладость. Напротив того, ваши рабы каждое дело начинают с ропотом, досадуя друг на друга, а где есть грехи, там может ли быть удовольствие? Сверх того, ваши дворецкие иногда бьют служителей за малейшую неисправность; их слезы также прибавляют много горькости в ваши брашна, сколь бы они драгоценны ни были».

Великий князь согласился с тем, что святой старец говорит правду, и, приняв от него благословение, возвратился в дом свой. Согласимся и мы, христиане, что урок Феодосиев особенно для нас нужен. Ибо мы по ветрености или, лучше сказать, в угождение иностранным лжеучителям не только без благословения принимаемся за каждое дело, но и, садясь за стол, чтобы насытиться дарами Господними, не только оставили благочестивое обыкновение предков наших "--- помолиться Богу, но, к несчастью, иногда почитаем за стыд даже перекреститься.

\section{Забвение обетов, данных Богу, есть великий грех\footnote{Из жития прп. Феодосия Печерского.}}

В игуменство преподобного Феодосия Печерского боярин Судислав Гедевич, отправляясь на брань с великим князем Изяславом, дал обет Богу "--- если здрав возвратится в дом свой, приложить в Печерскую Лавру две гривны золота и сделать из того же металла венец на образ Пресвятой Богородицы. Брань кончилась для нас, русских, благополучно, враги были рассеяны и обращены в бегство, но Климент (христианское имя Судислава), возвратившись в Киев здравым и невредимым с отрядом своим, забыл обещание.

Через несколько дней, когда Судислав после обеда лег отдохнуть и заснул, вдруг загремел над ним страшный голос, по имени его называющий. Климент воспрянул и, узрев перед собой икону Пресвятой Богородицы, услышал укоризну, от Нее исходящую: «Почто, Клименте, еже обещался еси дати, не дал еси? Но се ныне глаголю тебе: потщися исполнити обещание твое». Посреди этого гласа святая икона стала невидима, а боярин, пораженный ужасом, в то же мгновение бросился в сокровищницу и, взяв трепещущими руками нужное количество золота, пошел в Печерскую Лавру.

Каждый из нас "--- великий охотник давать обеты Богу, когда предстоит опасность, и не исполнять их, когда опасность минует: богач обещает Богу золото и серебро, бедный "--- свечу, порочный клянется, что будет беспорочен. Но где обеты? Где клятвы? Все разносит попутный ветер. Христиане! Тот, Кто все создал и всем управляет, Тот, Кому небо служит престолом, а земля подножием, не имеет нужды в наших приношениях. Он только хочет, чтобы мы, как чада Его, были благосерды, но какое благосердие может иметь лжец, обманщик? \textit{Уне есть тать, нежели присно лжай: оба же пагубу наследят} (Сир. 20, 25). А тот, кто не исполняет своих обетов, есть и тать, и лжец: лжец, поскольку не сдержал слова; тать, поскольку вещь, другому обещанная, уже не наша. Итак, если грешно обмануть ближнего своего, то сколь ужаснейший грех обмануть Бога!

\section{Пример избрания наставников для юношества и уважения к оным\footnote{Из жития прп. Арсения Великого (†~449~-- 450), память которого празднуется 8~(21) мая.}}

Император Феодосий Великий, видя, что сын его Аркадий приходит в возраст, торжественно объявил его Августом. Народ восклицанием своим и празднествами изъявлял радость, но порфироносный отец более заботился о воспитании, нежели о возвышении детей своих. Приятная для родительского сердца мысль, что кровь его будет царствовать на престоле вселенной, подавляема была опасением, что дети его могут быть недостойны столь священного сана. Он уверен был, что не только бесполезно, но и пагубно оставить им обширные области, не снабдив мудростью управлять оными, и потому неусыпно искал добродетельного и просвещенного человека.

Долго благочестивое желание Феодосия не могло исполниться, ибо хотя и много было мудрецов и витий, но они или привязаны были к язычеству, или, будучи христианами, не имели христианских добродетелей. Видно, царь"=отец думал о воспитании юношества не так, как думает большая часть нынешних родителей, которые, оставляя в стороне не только любовь к Богу, но и любовь к отечеству, богатством и почестями их осыпающему, ищут в учителе только учености и, что хуже всего, иноземного лжеумствования.

Наконец, Феодосий был вынужден просить западного императора Грациана и папу Дамаса, чтобы они избрали и прислали в Царьград, кого признают способным к столь великому званию. Выбор пал на Арсения\footnote{Прп. Арсений, уроженец римский, происходил от знаменитого дома и был сведущ в греческой и латинской словесности, философии и Священном Писании. Но сколь ни достоин он был хороших чинов и первых достоинств в Церкви, однако не имел других попечений, кроме своего спасения. Впрочем, от честного сообщества не удалялся, за что как пустынникам, так и светским людям был равно любезен.}, диакона Римской Церкви, добродетели которого и ученость папе были хорошо известны. И этот"=то священнослужитель представлен был Феодосию как мудрый и благочестивый человек, который может жить при дворе, не повреждая своих нравов, и будет подавать не только князьям достодолжное наставление, но и придворным хороший пример.

Император принял Арсения как особый дар Неба и просил его прилагать всевозможное попечение о воспитании юных царевичей и употреблять над ними отеческую власть, чтобы сделать их мудрыми и благочестивыми государями. Он повелел Аркадию и Онорию быть ему послушными и иметь к нему почтение, часто повторяя: «Дети мои! Помните, что вы более будете одолжены учителю, нежели мне самому: от меня получили вы жизнь и получите царство, а через него получите вечную жизнь и Царствие Небесное».

Святой Арсений старался не только научить учеников своих мудрости и всем изящным знаниям, но и наставить их в вере и христианских добродетелях: изведывал их склонности и, стараясь искоренять злые, утверждал добрые. Благодарный Феодосий сделал его сенатором, всегда уважал голос его в тайных советах и называл его не иначе, как отцом императорского дома; а некоторые из историков утверждают, что Арсений был восприемником обоих царевичей от купели Святого Крещения\footnote{В то время принимали Святое Крещение не в младенчестве, но в юношеском возрасте.}.

Христиане! Вот поучительный пример, как избирать учителей для детей ваших и какие им внушать к наставнику своему чувствования! В противном случае бойтесь, чтобы чада ваши когда"=нибудь не укорили вас нерадением об их воспитании.

\section{Бегство Арсения от двора царского}

Аркадий, старший сын Феодосия Великого, имел острый разум, нрав веселый и приятный, благородные мысли и душу, по природе склонную к благочестию и правосудию. Но труды ненавидел, был непостоянен в дружбе, склонен к принятию всяких внушений и верил более тем, которые льстили его порокам, нежели тем, которые желали исправить оные.

Арсений, предвидя печальные следствия, которые могли произойти в наследнике престола от столь худых склонностей, и тщетно употребляя благоразумие к отвращению оных, принужден был принять строгие меры и к выговорам, наконец, присовокупил наказание. Аркадий почел усердие за обиду и захотел от него освободиться. Он открыл предприятие свое одному из придворных чиновников, к которому имел великое доверие, и приказал ему избавить себя от столь несносного, по его мнению, человека. Этот чиновник (видно, боялся Бога), опасаясь, чтобы другому не дали того же поручения, обещался исполнить волю его, а сам тайно пошел к Арсению и сказал, чтобы он, не теряя времени, помышлял о своей безопасности.

Арсений хотя и видел, что приказ царевича был только следствием детского гнева, но, рассуждая о несчастии государей, которые почти от рождения своего любят льстецов и почитают неприятелями наставников, не мог удержаться от слез и принял твердое намерение оставить звание, которое угрожало ему смертью, если пребудет постоянен в своей твердости, и вечным наказанием, если станет поступать слабо и нерадиво. Само Небо благословило его предприятие; ибо, когда он в пламенной молитве просил Бога, дабы научил его, что должно делать для своего спасения, услышал глас Небесный: «Убегай мира, се единое средство к твоему спасению».

Арсений возблагодарил Промысл Божий и через несколько дней, переодевшись странником, вышел из Царьграда и удалился в египетскую пустыню, где жил около пятидесяти лет, не имея сообщения с миром. Питался одними травами, день и ночь проводил в молитвах и слезах и, потеряв надежду в особе детей Феодосиевых спасти свое отечество от угрожающего ему падения, пекся единственно о своем спасении до девяноста пяти лет своей жизни.

Император услышал о бегстве Арсения тем с большим прискорбием, что не знал оному причины. Он велел искать его по всему государству, но Бог восхотел сокрыть его от мира, чтобы представить в нем совершенный образ Своих угодников. Аркадий не понимал полученного им через то урока, но народы почувствовали оный, когда этот государь, заматерев в страстях своих и будучи управляем женщинами и евнухами, возвышая и низвергая своих любимцев, дал повод к переменам, которые, наконец, разорили Римское государство до основания\footnote{Аркадий, изгнав Арсения Великого, через то погубил и себя, и своего брата Онория, ибо они были один другого не лучше. В доказательство, чего они в Арсении лишились, приведем слова Георгия Кедрина. Онорий, говорит он, имел попугая, которого чрезвычайно любил и называл его Римом. Однажды пришел к нему министр, весь в трепете, и печально сказал: «Государь! Рим пропал». "--- «Как! "--- воскликнул царь, больше его испугавшись. "--- Я теперь только видел его». "--- «Государь! "--- возразил министр. "--- Я говорю не о попугае, но о Риме, столице вселенной: его осадили неприятели». "--- «Слава Богу, теперь я отдохнул! "--- сказал Онорий. "--- Пропал город Рим? Так и быть, лишь бы жив был попугай Рим».}.

Любезные дети! Не сердитесь на ваших учителей, если они иногда и строго накажут вас. Послушайте, как добродетельные старые люди отзываются о наставниках своей юности! Вспоминая о наказании своих детских шалостей, они всегда приговаривают: «Учитель желал мне добра». Придет время, когда и вы сделаетесь отцами или воспитателями и должны будете также наказывать других. Непослушный в юности своей будет некогда иметь то неудовольствие, что и его не будут слушать в старости. Это Божие наказание для нас неприметно, но спросите у несчастных стариков, сколь оно чувствительно их сердцу!

\section{Разговор с самим собой}

Арсений Великий, обитая в пустыне, часто сам у себя спрашивал: «Арсений, почто сюда пришел ты?» И потом отвечал: «Не для покоя, но для трудов; не для лености, но для подвигов. Подвизайся убо: леность Арсения соделает не Арсением».

О, если бы носители всякого звания, от вельможи до раба, чаще сами у себя спрашивали: для чего они вступили в звание свое? Тогда бы все были тем, чем быть должны, тогда не говорили бы: «Он имеет выгодное место, он имеет спокойную должность!»

\section{Посещение по пути}

Несколько иноков пошли в Фиваиду для покупки пряденого льна и по пути вздумали посетить святого Арсения. Но, когда служащий ему ученик доложил, что братия, прибывшие из Александрии, желают видеть его, великий старец приказал спросить у них: какая была причина столь дальнего путешествия? Узнав, для чего отлучились они из своей обители, ученик объявил о том Арсению. «Итак, не увидят они лица моего, "--- сказал старец, "--- поскольку вышли из дому не для меня, но для своего дела. Прими их, угости и, когда отдохнут, отпусти с миром, а обо мне скажи, что не могу видеться с ними».

Не так ли и мы приходим иногда в церковь Божию? Убоимся, дабы Господь не отвратил от нас лица Своего!

\section{Ругательство, обращенное на главу ругателей}

В Самаре, столице сарацинского князя Амирмумны\footnote{Убийца 42~мучеников, иже во Аммории, пострадавших в царствование Феофила иконоборца.}, более, чем в других неверных странах, были ненавидимы христиане. Злобожные агаряне гнушались людьми Христовыми, как демонами, почему на вратах каждого христианина по приказанию варварского правительства был изображен этот дух лжи и злобы. Святой Константин философ\footnote{Св. Константин, в схимонашестве нареченный Кириллом, с братом своим, св. Мефодием, проповедовал Евангелие народам славянского племени, изобрел азбуку, доселе у нас употребляемую, перевел на славянский язык несколько Священных книг и начал отправлять Богослужение на языке славянском. Память равноапостольных Кирилла (†~869) и Мефодия (†~885), учителей словенских, празднуется 11~(24) мая (см. Жития святых).}, бывший в Самаре по делам, до веры касающимся, с сердечным сокрушением удивлялся ненависти против христиан, но сарацины хотели посмеяться и над философом христианским и, указывая на изображения демонов, спросили у него: «Мудрый Константин, можешь ли угадать, что значит это?» "--- «Я вижу адских духов, "--- отвечал философ, "--- и думаю, что тут живут христиане». Агаряне хотели продолжать коварные вопросы, но Константин немедленно присовокупил: «Эти демоны не смеют войти ни в один христианский дом и потому остаются вне врат, а где этих демонских изображений нет, там они, как друзья и родные, живут вместе с людьми». Сказав это, святой Константин бросил значительный взгляд на вельмож сарацинских.

\section{Доказательство Пресвятой Троицы от солнца}

Однажды сарацинские мудрецы спросили у святого Константина: «Как вы, христиане, Единого Бога разделяете на три Бога? У вас есть Отец, Сын и Дух Святой». "--- «Не злословьте Пресвятой Троицы! "--- отвечал богодухновенный мудрец. "--- Отец и Сын и Дух Святой суть три Ипостаси, существо же едино. Воззрите на солнце, от Бога в образ Святыя Троицы на Небеси поставленное: в нем три вещи: круг, сияние и теплота; также и в Пресвятой Троице Отец, Сын и Дух Святой. Солнечный круг есть подобие Бога Отца: ибо как круг не имеет ни начала, ни конца, так и Бог есть безначален и бесконечен; и как от круга солнечного происходит сияние и теплота, так от Бога Отца рождается Сын и исходит Дух Святой. Сияние, от солнца происходящее и всю поднебесную просвещающее, есть подобие Бога Сына, от Отца рожденного и весь мир Евангелием просветившего, а теплота солнечная, от того же круга вместе с сиянием происходящая, есть подобие Бога Духа Святого, Который от того же Отца исходит предвечно. Итак, рассмотрите солнце и познайте Пресвятую Троицу. В солнце одно есть круг, другое "--- сияние, третье "--- теплота, но кто скажет, что не одно, а три солнца? Так и Пресвятая Троица хотя имеет три лица: Отца и Сына и Святого Духа, однако Божеством не разделяется на три Бога, но Един есть Бог». Сарацинские мудрецы не знали, что сказать против доказательства Константинова, и обратили разговор свой на другие предметы.

\section{Разность между обидами частными и обидами общественными}

«Христос есть Бог ваш, "--- сказали сарацинские мудрецы святому Константину, "--- для чего же вы не поступаете так, как Он повелевает вам? Иисус заповедал вам молиться за врагов, добро творить ненавидящим и гонящим вас, биющим в ланиту обращать другую ланиту\footnote{\textit{Ланита} "--- щека.}, а вы "--- что делаете? Если кто обидит вас, изощряете оружие, исходите на брань, убиваете». Выслушав это, святой мудрец спросил у них: «Если в каком"=либо законе будут написаны две заповеди, то который человек будет совершенный хранитель закона: тот ли, кто исполняет одну, или тот, кто исполняет обе заповеди?» Когда агаряне указали на второго, он продолжал: «Христос Бог наш, повелевший нам молиться за обидящих нас и им благотворить, сказал также, что большей любви никто из нас в жизни этой явить не может, разве кто положит душу свою за други своя. Итак, мы великодушно терпим обиды, причиняемые нам как людям частным, но в обществе друг друга защищаем, жертвуя своей жизнью, чтоб вы, пленив наших сограждан, вместе с телом не пленили и душ их, принудив к богопротивным деяниям. Наши христолюбивые воины с оружием в руках охраняют Святую Церковь, где Спаситель мира присутствует невидимо; охраняют государя, в священной особе коего почитают власть и силу Царя Небесного; охраняют отечество, с разрушением которого неминуемо падет отечественная власть и поколеблется вера Евангельская. Вот драгоценнейшие залоги, за которые до последней капли крови должны сражаться воины; и если они на поле брани положат души свои, Церковь причисляет их к лику святых мучеников и нарицает молитвенниками пред Богом о спасении их отечества».

Христолюбивые всероссийские воины! Какое для вас побуждение подвизаться против врагов отечества! Какая неизреченная награда, если за Церковь, царя и родимую страну прольете кровь вашу! Любовь есть легчайший путь в Царство Небесное; но \textit{больши сея любве никтоже имать, да кто душу свою положит за други своя} (Ин. 15, 13).

\section{Необходимость иметь и читать с благоговением сочинения святых Отцов}

Святой Константин имел пламенную любовь и благоговение к угоднику Божию Григорию Богослову, непрестанно читал его книги и некоторые места учил наизусть. Будучи семи лет от рождения, он изобразил на стене крест и под ним написал: «Святитель Христов Григорий! Ты телом был человек, но житием Ангел; уста твои, яко уста серафимские, прославили Бога, и твое правоверное учение просветило вселенную. Умоляю тебя, приими меня, с верой и любовью к тебе прибегающего, и буди моим наставником и просветителем!»

Так чувствовали и поступали все христиане, сподобившиеся быть участниками блаженства Ангельского! Преподобный Косма\footnote{См. в житии прп. Космы отшельника (†~IV~в.), 3~(16) августа.} столько благоговел к отцам и учителям церковным, что, разговаривая однажды с другим старшим о спасении души и в подтверждение истины приведя свидетельство Афанасия, архиепископа Александрийского, сказал: «Если где найдешь что"=либо из сочинений святых Отцов и не имеешь бумаги, то напиши на твоих одеждах». Если мудрецы нынешнего века спросят, что в этих книгах заслужило столь великое благоговение, "--- я буду отвечать устами святого страстотерпца Тимофея Чтеца: «Не веси ли, "--- ответил он\footnote{См. в его страдании, в 3"~й день мая.} точно на такой же вопрос мучителя Ариана, "--- яко егда книги о Бозе чту, обстоят мя Ангели Божии».

Так, христиане! Кто хочет сделать приобретение для души, тому необходимо иметь христианские книги. «Самый взгляд на эту бесценную собственность, "--- говорит святой Епифаний, епископ Кипрский, "--- производит в нас омерзение к пороку и воспламеняет любовью к добродетели. Надпись, выставленная в заглавии, уже предостерегает, советует, укоряет, утешает, ободряет. Напротив того, неведение Святых Писаний есть, так сказать, измена для души, желающей спастись; неуважение к оным есть глубокая бездна».

Горе человеку, который мечты воображения, бред страстей, лжеумствования разума, юродство иноземной мудрости предпочитает этим душеспасительным наставлениям! Там под розами кроется змея, с сотами растворен яд. Здесь слово и жизнь помогают друг другу, здесь добродетель объемлется с мудростью, здесь "--- воплощенная философия\footnote{Выражения св. Исидора Пилусиота.}. Произведения светские имели началом своим разные страсти; святые отцы писали, будучи осеняемы наитием Духа Святого. Это такая истина, которую осмелится отвергать разве совершенный безбожник, ибо она утверждается бесчисленными историческими доказательствами. Приведем из них некоторые.

\textbf{Преподобный Иосиф Песнописец}\footnote{Прп. Иосиф Песнописец (†~883). Память его празднуется 4~(17) апреля.}, создав в обители своей церковь во имя святого апостола Варфоломея, желал возвеличить память его похвальными песнями, но сомневался, угодно ли будет апостолу его усердие. Посему он прилежно молил святого Варфоломея, дабы тот открыл ему свою волю и свыше испросил премудрость составить пения богоугодные. Иосиф провел сорок дней в посте и слезах. Когда настал день памяти апостола, в навечерии праздника святой Варфоломей, одеянный в белые ризы, внезапно явился в алтаре и, отняв завесу от царских врат, мановением руки призвал к себе Иосифа. Иосиф с благоговейным трепетом приблизился, и святой Варфоломей, взяв со Святого Престола Евангелие, возложил ему на перси и сказал: «Да благословит тебя десница Всесильного Бога, и да потекут на язык твой воды Небесной премудрости; буди сердце твое жилищем Духа Святого, и твои песни да усладят вселенную». Святой апостол стал невидим; а преподобный Иосиф, исполнившись неизреченной радости и ощутив в себе благодать премудрости, простерся на землю и со слезами благодарил Бога. С этого времени он начал писать церковные гимны и песенные «каноны», которыми украсил не только празднество святого Варфоломея, но и других угодников Божиих, а особенно возвеличил Пречистую Богоматерь и святителя Христова Николая, отчего и получил название Песнописца.

\textbf{Святой папа Лев}\footnote{Свт. Лев, Папа Римский (†~461). Память его празднуется 18~февраля (2~марта) (см. Жития святых).}, написав послание к святому Флавиану, патриарху Царьградскому, в обличение лжемудрия Евтихиева и Несториева, положил это послание на гроб первоверховного апостола Петра и, пребыв тут несколько дней в коленопреклонении и постничестве, молился великому ученику Христову: «Ежели я, как слабый человек, что"=либо опустил, сказал превратно, не довершил, "--- благоволи исправить ты, яко апостол Господа нашего». По прошествии сорока дней явился ему апостол Господень и сказал: «\textit{Читах и исправих}». Святой папа с благоговейным восторгом взял с гроба апостольского рукопись, развернул и узрел поправки и дополнения, рукой ученика Христова сделанные.

Когда \textbf{святой Иоанн Златоуст}\footnote{См. в житии cв. Иоанна Златоуста, в 13"~й день ноября.} писал толкования на послания Павловы, в одну ночь некто из граждан, имея до святого нужду, просил доложить о нем. Святой Прокл (бывший тогда келейником, а после патриархом Царьградским), решившись войти к патриарху, сначала посмотрел сквозь скважину двери, дабы увидеть, что делает Иоанн. И что же узрел? Христиане! Благоговейте к творениям Иоанновым… Златоуст сидит при возожженной свече и пишет, а какой"=то благообразный старец, стоя позади него и преклонившись к уху патриаршему, тихо беседует с ним. Прокл, объятый священным ужасом, не знал, что об этом думать, кто с Иоанном беседует и как вошел он, ибо двери отовсюду были заперты. Но изумление его еще увеличилось, когда он приметил, что этот старец совершенно похож на образ святого апостола Павла, который стоял на стене пред Иоанном. Это видение было не один раз, но продолжалось какое"=то время. Часто святой Прокл дожидался всю ночь, не выйдет ли этот таинственный собеседник, но, как только начинали благовестить к утреннему славословию и Златоуст вставал из"=за стола, чтобы идти в церковь, старец делался невидим. Наконец Прокл осмелился спросить у самого Иоанна, кто ночью с ним беседует. «У меня никого не было, и быть не может», "--- отвечал Иоанн. Тогда Прокл рассказал ему подробно, как и когда он смотрел сквозь скважину двери и видел человека старого и благообразного, который что"=то шептал ему на ухо, в то время как писал он. Иоанн изумился, а святой Прокл, воззрев на образ Павлов, сказал: «Старец был точно таков».

Златоуст познал из этого, что труд его исправляет сам святой Апостол, возблагодарил Бога и после того с большим дерзновением начал толковать Божественные книги, через что Церкви оставил по себе бесценное сокровище.

\section{Пренебрежение к начальству, Богом наказанное}

В начале епископства Епифаниева\footnote{Св. Епифаний родился в иудействе, принял Святое Крещение в отрочестве. Будучи 60~лет, удостоен архиерейского сана по особому избранию Господню. Ибо, плывя на корабле мимо Саламина, к этому острову был прибит бурей. Здесь по откровению Божию остановлен собором епископов и против воли рукоположен в епископа. Умер в царствование Аркадия, от рождения своего 116~лет. Память святителя Епифания, епископа Кипрского (†~403), празднуется 12~(25) мая.} один римлянин, по имени Евгномон, человек добропорядочный и честный, по некоторым обстоятельствам задолжал саламинскому жителю Драконтию. И поскольку должник долго не получал денег из Рима, а заимодавец не хотел ждать, то Евгномон и заключен был в темницу. Никто за странника не хотел поручиться, один только святитель Христов Епифаний над ним сжалился и, не имея собственных денег, заплатил долг из церковной сокровищницы и выкупил его из темницы.

Весь город благословлял милосердие Епифания. Только диакон Карин, человек гордый, непокорный, злобный и завистливый, начал роптать на святителя и всех клириков подвигнул к роптанию. «Посмотрите, "--- беспрестанно говорил он, "--- что делает епископ! Как пришелец, он не жалеет церковной собственности. Страшитесь, чтобы после не заставили нас отвечать за его расточительность». Хотя священнослужители сначала не слушали мятежника и уверяли его, что архипастырь имеет полную власть над церковными суммами, особенно когда пожелает расточать оные на дела милосердия, но хитрый Карин успел столько, что все заговорили вслух, что с таким епископом Церковь скоро обнищает. «Пусть отдаст златницы, "--- вопияли они, "--- или пусть уйдет, откуда пришел!» Святитель слушал мятежную молву равнодушно и молчал, пока Евгномон не возвратился из Рима. Этот благочестивый муж не только с лихвой возвратил свой долг, но всю жизнь свою посвятил на служение Богу. Тогда все церковнослужители раскаялись, что по своей глупости столь жестоко оскорбили угодника Божия. Но Карин с гордостью везде хвалился: «Истязанием истязах злато церковное».

Мятежная душа и этим не удовольствовалась, но свое буйство простерла далее. Однажды святитель Христов пригласил всех священнослужителей к обеду и, сидя за столом, объяснял им некоторые места из Священного Писания. Случайно к самому окну кельи прилетел ворон и начал громко каркать. Злобный Карин, издеваясь над учением святительским, спросил у собеседников, знает ли кто из них, о чем говорит этот ворон. С великим вниманием слушая епископа, никто не отвечал Карину. Но презритель власти и святыни тот же вопрос задал в другой и в третий раз. За бессовестного человека устыдились все беседующие. Тогда святой Епифаний, воззрев на него с жалостью, сказал: «Я знаю, что говорит ворон: он говорит, что ты, несчастный, отселе не будешь диаконствовать». Мгновенно от слова святительского Карина объял ужас, все тело поразила тяжкая болезнь, так что он не мог сидеть за столом. Рабы унесли его на руках в дом свой, а на другой день Карин умер.

\section{Милосердие к бедным, не у места оказанное}

Святой Епифаний имел при себе иеродиакона, по имени Савин\footnote{Этот Савин по смерти свт. Епифания был архиепископом Кипрским.}, человека благоразумного, целомудренного и притом весьма искусного в писании книг, почему и поставил его судьей духовных дел.

Однажды пришли к Савину два человека, бывшие между собой в ссоре, один богатый, другой бедный, и каждый, жалуясь на другого, объяснял, в чем состояло его требование. Все обстоятельства дела показывали, что первый был прав, а последний виноват, но Савин, по врожденному милосердию к бедным, не мог проникнуть истины и, почитая их беззащитными, хотел оправдать бедного и обвинить богатого. На тот раз Епифаний, желая знать, как Савин исправляет свою должность, пришел туда и, став в сокровенном месте, к удивлению своему, услышал, что судья его поступает неправедно. Святитель Христов немедленно показывается и тихо уличает Савина: «Сын мой! Иди, пиши книги и содержи Божественные словеса в уме твоем, да научишься судить праведно. Пророк и законодатель Моисей сказал: \textit{Не милуй нищаго и не презри лица сильного, но по правде да судиши ближнему твоему} (Лев. 19, 15). Конечно, бедные требуют от нас милосердия и заступления, ибо часто бывают обижены богатыми, но есть и нищие с дурными нравами, есть и богатые с добрым сердцем. Даже не взирая, добродетельна ли, порочна ли жизнь тех и других, мы часто видим случаи, в которых добрый человек бывает виноват, а худой "--- прав. Вот для чего судья должен быть сколько милосерд, столько же и нелицеприятен». Сказав это, святой Епифаний сам начал разбирать дело, и с того времени, не возлагая на других должности судейской, от утра до вечера сам принимал требующих суда; всех выслушивал, как отец, и отпускал, как судья.

\section{Узы Господних заповедей}

Когда святая мученица Гликерия\footnote{Гликерия была сирота знатного происхождения, родилась в Риме и в нежной юности пострадала в Траянополе Фракийском в царствование Антонина. Память св. мученицы Гликерии девы (†~ок. 177) празднуется 13~(26) мая.} за исповедание имени Христа представлена была на суд, Савин, ахайский наместник, приказал побить ее камнями. Но кому \textit{Небесный Помощник и Покровитель} восхощет быть \textit{во спасение}, над тем изнемогает вся сила тиранов. Сколько изуверный народ ни метал на нее камней, они, не прикасаясь к мученице, падали близ нее и, как бы руками человеческими слагаясь в ограду, окружали ее. Ожесточенный Савин, не признавая силы Господней, непременно хотел умертвить святую девицу. И, обещаясь вымыслить новую, неслыханную казнь, повелел отвести ее в темницу. «Свяжите эту волшебницу, "--- сказал он стражам, "--- и содержите крепко. Она посредством чародейств может убежать и после скажет, что Иисус избавил ее, чем прельстит множество из легкомысленной черни». "--- «Ослепленный человек! "--- отвечала с кротостью святая Гликерия тирану. "--- Ужели не понимаешь ты, что я и без твоих уз связана заповедями Бога моего и не хочу разрешиться от этих любезных мне уз, ни бежать от страдальческого подвига, на который иду добровольно за Христа Спасителя моего?» Сказав это, страдалица спокойно пошла в темницу.

Христиане! Мы живем в блаженные времена: у нас не может быть страдальчества за веру Христову. Но всяк, в нем же призван есть, "--- в том да пребывает пред Богом. Каждый человек по своему званию имеет подвиги, на которые должен идти добровольно, с радостью. О, если бы все, подобно святой Гликерии, имели узы заповедей Господних! Тогда бы не было преступников; никто бы не оставлял своих обязанностей; все бы государственные сословия, от вельможи до раба, имели не наемников, но усердных носителей своего звания.

\section{Совершенный бессребреник}

Преподобный Серапион\footnote{Прп. Серапион Синдонит (†~V~в.). Память его празднуется 14~(27) мая.}, с юности возлюбив жизнь пустынническую, был совершенным образцом иночества "--- был так несребролюбив, что ничего не имел у себя. Вертеп в пустыне был его кельей, мантия "--- всегдашней одеждой, Святое Евангелие "--- единственным сокровищем.

Однажды, идя в Александрию, человек Божий встретился с нищим, который дрожал от холода. Серапион остановился и подумал сам себе: «Меня почитают постником и Христовым последователем; я между тем ношу ризу, а этот раб Христов погибает от холода! Без сомнения, буду убийцей, если бедный человек замерзнет». Немедленно снял он с себя мантию и отдал нищему.

Серапион сел на распутии, держа за пазухой Святое Евангелие, которое повсюду носил с собой; и когда один мимо проходящий спросил у него: «Что ты, отец Серапион, без мантии?» "--- он отвечал, показав Святое Евангелие: «Эта Божия книга раздела меня».

Едва бессребреник выговорил это, как увидел, что мимо него ведут должника в темницу. Серапион не думал много: тотчас продал Евангелие и удовлетворил заимодавца. Несчастный пошел домой, благословляя равноангельного незнакомца.

Когда Серапион пришел в келью, ученик спросил у него: «Где одежда твоя?» "--- «Я отослал ее туда, "--- отвечал бессребреник, "--- где можно вместо нее получить лучшую». "--- «А где Святое Евангелие?» "--- опять спросил ученик. «Сын мой! "--- сказал праведный муж. "--- Оно беспрестанно твердило мне: \textit{Продаждь имение твое и даждь нищим} (Мф. 19, 21); а ты знаешь, что оно только одно составляло все мое имение; итак, я продал его и отдал нищим».

Христиане! Подражайте, хотя несколько, добродетели преподобного Серапиона "--- и вы спасетесь.

\section{Смирение правоверных и гордость зловерных\footnote{Из жития прп. Пахомия Великого (†~ок. 348), память которого празднуется 15~(28) мая.}}

Некогда к преподобному Пахомию пришли еретики, так называемые «постники», власяными одеждами внутренний яд прикрывающие, и, остановившись за монастырскими вратами, велели доложить ему: «Мы присланы от нашего отца спросить у тебя: воистину ли ты человек Божий и совершенно ли уповаешь, что Бог послушает тебя? Если так, то перейди с нами, как посуху, находящуюся близ монастыря реку, да разумеем, кто из нас имеет большее дерзновение: мы или ты?» Услышав столь гордое предложение, преподобный Пахомий почел за грех даже видеться с этими волками в овечьей одежде. Но чрез одного из учеников отвечал им: «Попущением Божиим и еретики могут чрез реку перейти немокренно, но только с помощью дьявола, который этим чудом успеет очаровать слабых и утвердить соблазны еретичества. Что касается до меня, я не прошу у Бога чуда, дабы ходить по водам: этот помысл не иноческий и не христианский. Пусть узнают они, что я на Бога надеюсь, но на дела мои надеяться не могу; признаю себя грешником и не хочу искушать Господа Бога моего, подобно дьяволу. Пусть узнают они, что Пахомий просит помощи у Бога только в том, дабы без вреда пройти вражеские искушения». Услышав столь поучительный ответ, еретики устыдились и пошли безмолвно.

\section{Наказанное ослушание}

Преподобный Пахомий Великий, отлучаясь в Панополь для устроения там иноческого общежития, приказал, чтобы в отсутствие его ни в чем не нарушали уставов монастырских. Но, когда он возвратился, отрок, бывший под искусом, пришел к нему жаловаться, что братия без него не варили ни зелени, ни каши. Святой старец с улыбкой сказал ему: «Не досадуй, сын мой! Я велю, чтобы все было по"=прежнему». Потом пошел к иноку, служащему при поварне, и спросил, зачем без него оставлен был устав в рассуждении пищи. Инок начал извиняться тем, что старцы довольствуются сухоядением, а для одних детей, находящихся в новоначалии, ему не хотелось держать круп и масла. «Впрочем, чтобы не сидеть праздно, "--- продолжал он, "--- я делал корзины». "--- «Принеси же их сюда», "--- сказал Пахомий. Инок повиновался. Тогда святой настоятель приказал развести огонь и, сжигая труды двух месяцев, говорил: «Ты пренебрег мою заповедь о пищи братии, и я не щажу твоего рукоделия. Не уважать повеления отеческие, служащие к общей пользе, великий грех. Разве не знаешь, что имеет от Бога воздаяние только тот, кто постится по своей воле, а кто не ест поневоле, тот не должен ожидать награды. Твоя леность, или упрямство, или скупость сделали то, что воздержание братии потеряло всю цену». Инок, почувствовав всю силу наставления Пахомиева, с покаянием припал к ногам его и сам подавал ему свое рукоделие для сожжения.

\section{Лукавство и дерзость духа"=искусителя}

Однажды преподобный Пахомий, удалившись из монастыря, некоторое время пребывал в пустынной келье. Дух соблазна почел уединение праведника удобнейшим для себя случаем, чтобы искусить его, и, приняв на себя богоподобный образ, явился Пахомию. «Радуйся, старец, столько мне угодивший! "--- сказал он. "--- Аз есмь Христос и пришел к тебе, яко к другу моему». Пахомий изумился и, без робости смотря на привидение, начал рассуждать: «Христово пришествие к человеку сопровождается радостью; сердце не чувствует никакого страха; все помышления тотчас исчезают; ум делается очами Серафимскими и весь вперяется в зрение славы Господней; душа забывает время; человек тогда делается бесплотным. А я теперь смущаюсь, боюсь… нет, это не Христос!» Потом, оградившись крестом, с дерзновением сказал: «Отыди от меня, дух злобы! Будь проклято лукавство всех твоих начинаний!» Мгновенно призрак стал яко прах, храмина исполнилась смрада, по воздуху восшумел ветер.

Христиане! Столь же лукавы и дерзновенны страсти, \textit{на душу воюющия}. Каждый из нас знает, какой иногда блистательный и любезный принимают они на себя образ. Гнуснейшие из них обещают нам блаженство и кажутся споспешествующими нашей пользе и спокойствию, но когда войдем в сердце наше, что почувствуем?.. Смущение, грусть, угрызение, тайный страх и вместе с тем какую"=то ненасытимость, как червь, поедающую наше сердце. Это ли блаженство человеческое? Одна только добродетель, всегда сама собой довольная, истинно спокойна и счастлива. Бурные страсти возмущают душу и наполняют ее нечистотами, после них остается в сердце одна пустота и раскаяние. Блажен человек, который, почувствовав поползновение к пороку, подобно святому Пахомию, скажет: «Отыди от меня, дух погибельный!»

\section{Польза от Слова Божия, проповеданного новоначальным иноком}

Преподобный Феодор\footnote{Прп. Феодор Освященный (†~368). Память его празднуется 16~(29) мая.}, ученик святого Пахомия, был добродетелен, благоразумен, мог толковать Священное Писание и книги святых отцов и имел способность убеждать разум и приводить в умиление сердце слушателей, "--- почему святой Пахомий, уважая столь полезные дарования, однажды велел ему сказать Слово Божие изустно. Двадцатилетний юноша, не дерзая прекословить великому старцу, стал посреди собора и начал от Божественного Писания проповедовать, показуя путь спасения. Братия услаждались его беседой, но некоторые из старцев, ничего в Феодоре не видя, кроме юности, не хотели слушать его и начали роптать: «Вот новоначальный учит нас! Юноша наставляет! Стыд для старцев слушать его!» С досадой они вышли из церкви.

Преподобный Пахомий с сокрушением приметил соблазн братии и, когда святая служба кончилась, призвав к себе, спросил у них: «Почему вышли из церкви и не слушали Слова Божия?» "--- «Потому, "--- отвечали иноки, "--- что старцам, в монастыре столько лет препроводившим, поставлен учителем юноша». Пахомий вздохнул из глубины сердца и сказал: «Неразумные! Вы чрез свое высокомерие погубили всю вашу добродетель: вы презрели не Феодора, но Слово Иисуса Христа и лишились Духа Святого. Разве не видели вы, с каким вниманием и я, старший вас в иночестве, слушал его? Язык был Феодоров, но учение "--- Христово, и я получил великую от него пользу».

\section{Гордость, сама собой низверженная\footnote{См. в житии святых мучеников Петра, Дионисия, Павла и других с ними (†~249~-- 251), 18~(31) мая.}}

Когда святой мученик Петр и прочие воины Христовы мужественно подвизались за православную веру, тогда представлен был на судилище месопотамец, по имени Никомах, который беспрестанно и громко вопиял: «Я христианин! Я христианин!» "--- и, когда мучитель начал принуждать его, чтобы он принес жертву языческим богам, тот отвечал надменно: «Разве ты не знаешь, что христиане не приносят жертв идолам?» Тогда дано было повеление испытать над ним наравне с прочими христианами различные мучения.

Несколько времени Никомах равнодушно принимал все раны и в то время, как другие молились Господу, дабы ниспослал им Свою помощь и укрепил совершить подвиг страдания, дерзновенно восклицал: «Я христианин! Христианской твердости поколебать невозможно!»

Уже Никомах был близ конца своих подвигов, уже Небесный венец имел как бы в руках своих, но сила Божия не могла быть в сосуде, гордостью оскверненном. Ибо не любовь к Сладчайшему Иисусу, а желание прославиться великодушием и терпением против ужасов смерти вызвало его на священный подвиг страдальчества. Никомах внезапно переменил свой голос и закричал: «За что столь нещадно бьете меня? Я никогда не был христианином». В то же мгновение перестали его мучить, разрешили узы, осыпали похвалами. Но, как только отступник начал покланяться идолам, внезапный удар поверг его на землю. Никомах взбесновался, начал грызть язык свой, точить из уст пену и тогда же умер.

И мы не должны хвалиться ни дарованиями, ни твердостью в делах веры и общежития, но должны просить помощи Небесной, которая одна может просветить наш разум и укрепить сердце на труды богоугодные и общеполезные. У самохвала всегда действует один только язык.

\section{Непорочность не имеет нужды в притворстве\footnote{В Прологе, в 16"~й день мая.}}

В один жаркий полдень, когда преподобный Феодор сидел при вратах своей кельи в разодранной одежде, пришел один мирянин принять от него благословение. Ученик тотчас вынес другую одежду, но святой Феодор с неудовольствием отослал его назад и начал разговаривать с пришельцем о средствах \textit{спастися}. Когда мирянин ушел, ученик сказал святому Феодору: «Что ты, авва, делаешь? Ведь этот человек приходил получить пользу, а не соблазниться!» "--- «Я свое дело исполнил, "--- отвечал Феодор, "--- а прочее да идет мимо. Кто хочет пользоваться, пусть пользуется, а кто хочет соблазняться, пусть соблазняется. Я не хочу применяться к лицам и времени, как меня застанут, так и встречаю всех приходящих. И впредь повелеваю тебе всем сказывать обо мне сущую правду и не поступать так, как поступают в мире. Когда ем, скажи, что Феодор ест, когда сплю, скажи, что Феодор спит; словом, что бы ни делал я, пусть дела мои всем будут известны. Того нельзя назвать совершенно честным в душе своей, кто имеет причину скрываться».

\section{Выбор верных слуг царю и отечеству\footnote{Из жития св. равноапостольного царя Константина Великого, память которого празднуется 21~мая (3~июня) (см. Жития святых).}}

Повествуют, что император Констанций, отец Константина Великого, хотя по наружности казался идолопоклонником, но втайне благоговел пред Иисусом Христом, на Него только надеялся и любил христиан, что и было побуждением к следующему поступку.

Желая видеть на всех государственных местах людей добродетельных и благонадежных, этот государь хотел испытать любовь к отечеству и верность к верховной власти любовью к Богу и постоянством в вере. Для этого, созвав к себе всех чиновников, сказал им: «Кто мне верен и хочет служить отечеству, тот да поклоняется богам моим и приносит им жертву: вот единственное условие, с которым все могут быть моими друзьями и истинными служителями государства; иноверные пусть от меня удалятся».

Немедленно чиновники разделились на две части: истинные христиане, оставляя чины и знаки отличия, удалялись в передние комнаты, а малодушные наперерыв старались засвидетельствовать свою покорность и усердие к государю. Но Констанций вдруг оставил притворство. «Вы Богу вашему служите верно, "--- остановив непоколебимых христиан, сказал он, "--- вас только хочу иметь слугами, советниками и друзьями, поскольку надеюсь, что будете и мне столь же верны, сколько верны вашему Богу. А вы, "--- обратясь к отступникам, продолжал он, "--- вы не можете быть при дворе моем: кто изменяет своему Богу, тот не будет верен своему государю».

Язычники, взирая на поступок императора, скрежетали зубами. Но вера христианская в то время уже так усилилась и все государственные сословия имели столько христолюбивых членов, что зловерные не могли предпринять ничего для государя вредного.

\section{Желание инока}

Святой Михаил\footnote{Прп. Михаил Исповедник, епископ Синадский (†~821). Память его празднуется 23~мая (5~июня).}, черноризец обители святого Саввы, будучи по делам монастырским в Иерусалиме, который тогда находился уже под властью агарян, был оклеветан перед сарацинским князем, будто хулит магометанскую веру. Князь в силу закона, который заставлял хулителя веры принимать ту самую веру, и притом видя, что благовидный и рослый Михаил может быть храбрым оруженосцем, сказал ему: «Проси у меня, чего хочешь, только будь мусульманином». "--- «Из трех прошу у тебя одного, "--- отвечал святой инок, "--- или отпусти меня к моему старцу, или во имя Бога моего крестись, или мечом твоим пошли меня ко Христу».

Раздраженный князь в ту же минуту исполнил последнее требование святого Михаила.

\section{Урок о постничестве}

Авва Силуан и ученик его Захария пришли в одну обитель для посещения братии. Когда они, проведя там целые сутки, начали собираться домой, то принуждены были как гости вкусить немного пищи. Когда они были в дороге, Захария, увидев источник, хотел напиться, но Силуан сказал ему: «Сегодня пост!» "--- «А разве мы не ели, авва?» "--- возразил Захария. «Это была трапеза страннолюбия, "--- отвечал Силуан, "--- а мы, чадо, должны соблюдать пост свой. Если бы мы не уважили усердия гостеприимной братии, то в один раз согрешили бы дважды: оскорбили бы своих ближних и вострубили бы пред всеми, что Силуан и Захария "--- постники. И впредь, сын мой, втайне постись».

\section{Не должно просить Бога о погублении даже лютейших беззаконников\footnote{См. в житии св. апостола Карпа, 26~мая (8~июня), а пространнее "--- 3~(16) октября, где повествует об этом св. Дионисий Ареопагит.}}

Один идолопоклонник отвратил от Церкви христианина и преклонил его к своему зловерию. Святой Карп, ученик апостола Павла, у которого в духовной пастве это случилось, в подобных обстоятельствах был всегда терпелив и обыкновенно старался отступника обратить на путь истины увещаниями, а соблазнителя преодолеть своей благостью. Но на этот раз человек Божий чрезмерно оскорбился и, стоя на молитве, неотступно просил Господа, да спадет огонь с неба и попалит нечестивых. В полночь внезапно потряслась храмина и расступилась надвое. Воззрев вверх, святой Карп увидел небо отверстое и Иисуса на престоле, окруженного сонмами Ангелов. Пораженный Божественной славой, апостол преклонил очи свои долу и узрел, что на некотором расстоянии земля расселась и открыла пропасть глубокую и темную. Увидел, что над бездной стоят виновники его скорби, соблазнитель и отступник: они ужасались и трепетали и, поражаемые невидимой силой, уже готовы были низвергнуться в пропасть, где пресмыкался посреди пламени и скрежетал на них великий и страшный змий. Святой Карп почувствовал удовольствие и не столько смотрел на небо и на Иисуса, сколько на погибель двух грешников; ждал казни их и продолжал молиться о том. Но вот Иисус восстал от Престола славы Своея и, спустившись долу, приступил к пропасти; Он простер к погибающим руку помощи, а Ангелы, восприяв их, отвели далее от бездны. Тогда Иисус сказал святому Карпу: «Мучь Меня, если хочешь; Я готов за спасение человеческое паки\footnote{Паки — опять.} распят быти. Мне любезны страдания, только бы люди Мои возненавидели грехи свои. Для чего же ты желаешь погибели братьям? Подражай Мне и не желай \textit{смерти грешника}, но \textit{еже обратитися и живу быти ему} (Иез. 18, 23). Ужели не видишь, где лучше жить: с этим ли ужасным змием или с Богом и человеколюбивыми Его Ангелами? Желай и другим того же, чего себе желаешь».

Эта священная повесть учит и нас желать грешникам не казни и смерти, но покаяния и молить Бога не о том, да погубит их, но Своей благодатью да обратит их и помилует.

\section{Неустрашимый защитник Православия}

Преподобный Исаакий Далматский\footnote{Прп. Исаакий Исповедник, игумен обители Далматской (†~383). Память его празднуется 30~мая (12~июня).}, подражая святому пророку Илии, почти не показывался миру, но Бог, воздвигший Даниила на защищение невинной Сусанны, воздвиг и его на защищение Своей невесты, Православной Церкви, которая терпела ужасное гонение от императора Валента, глубоко зараженного арианством. Руководимый Духом Святым, Исаакий пришел в Царьград.

В то время у греко"=римлян загорелась война с готами. Варвары под предводительством царя Фритигерна осадили Адрианополь и угрожали самой столице. Император Валент, известившись, что войска его разбиты, собрал новые силы и готовился к походу. Преподобный Исаакий, провидя духом гнев Божий, грядущий на гонителей веры, приступил к Валенту в то время, когда он с отборными полками выступал из Царьграда, и с дерзновением святого человека сказал: «Государь! Перестань утеснять православных, отвори им церкви "--- и Господь добре устроит путь твой пред тобой». Но император, презрев его, как невежду, даже не хотел отвечать и пошел далее. На другой день блаженный старец опять вышел навстречу Валенту на следующем ночлеге и опять сказал: «Государь! Отвори церкви православным "--- и война будет благополучна: ты победишь врагов, возвратишься здрав и принесешь мир твоим народам». Император поколебался в зловерии и велел было заготовить указ для обуздания дерзости арианской, но окружающие его еретики употребили все лукавства, чтобы переменить мысли Валента, и так его ожесточили, что святой Исаакий получил несколько заушений. Но ревность человека Божия чрез это не упала: он опередил императора и на третий день остановил на дороге и, взяв за узду коня его, начал просить и умолять: «Государь! Умири Церковь, дай отраду православным; мщение Божие не укоснит, если не оставишь заблуждения». Тогда раздраженный Валент повелел бросить святого Исаакия в зыбкое болото, обросшее тернием, которое тогда встретилось им на пути.

Погиб бы тут угодник Божий, но невидимая десница извлекла его из топкой дебри. Исаакий пал на колена и, воздав благодарение Господу, в четвертый раз пошел за императором. Валент, увидев его, ужаснулся и удивлялся в безмолвии, а преподобный старец с апостольской ревностью воскликнул: «Ты хотел меня уморить в тернии и болоте, но Господь сохранил меня. Государь! Послушай меня, перестань гнать Православие "--- и тогда победишь врагов и возвратишься увенчанным славой и честью. За ослушание и сам погибнешь, и все воинство погубишь. Государь! Если не трогает тебя собственная смерть, то пожалей отечество: оно гибнет за грехи твои».

Хотя на тот раз император принял без гнева дерзновение инока, даже удивлялся светлости лица его, но жестокое сердце, от Бога отлученное, не могло принять истины. Прозорливый Исаакий был под стражей отправлен в Царьград. Уходя от лица Валента, он воодушевился ревностью пророческой и воззвал к нему: «\textit{Аще возвращаяся возвратишися в мире, то не глагола Господь мною} (3~Цар. 22, 28), глаголю же тебе, яко сведении вои на брань, и не возможеши варваров одолети, но \textit{побегнеши от лица их, и ят будеши, и жив огнем сожжешися}»\footnote{Выражения Минеи"=Четии.}.

Вскоре исполнилось пророчество человека Божия. Валент, по прибытии в Адрианополь, дал битву и проиграл ее. Две трети греко"=римских войск изрублено, остальные рассеялись по лесам или ушли в разные города. Лучшие полководцы убиты; император обратился в бегство, но, будучи ранен стрелой, упал с коня и был перенесен служителями в находившийся на поле небольшой дом, и этому кратковременному спасению способствовала одна только темнота ночи. Но едва уняли ему кровь и перевязали рану, как толпа готов, отделившихся от войска, прибежала туда в намерении ограбить дом, не зная, кто там находится; они хотели отбить двери, но, видя сопротивление, зажгли его со всех сторон и пошли далее.

В этом"=то месте Валент, удрученный болезнью и угрызением совести, сгорел живой. Он получил жребий безбожных, будучи в жизни своей от всех ненавидим, а по смерти никем не оплакан.

\section{Врачующий инок}

Более семисот лет тому назад Россия имела уже великих и святых врачей, имена которых достойны вечной памяти. Агапит\footnote{Прп. Агапит Печерский, врач безмездный, в Ближних Пещерах (†~XI~в.). Память его празднуется 1~(14) июня.}, из Киева пришедший в пещеру преподобного Антония и там постригшийся, врачевал иноков и всех бедных людей. Его сострадание простиралось так далеко, что, услышав о болезни кого"=нибудь из братии, он немедленно оставлял свою келью и оставался у недужного дотоле, пока искусством и молитвами не возвращал ему здравия. Сверх того, Печерский монастырь всегда видел множество болящих пришельцев, которые возвращались домой здоровыми.

Вскоре успехи врачевания так прославили Агапита, что Черниговский князь Владимир Всеволодович Мономах, находясь в тяжкой болезни и уже не надеясь получить помощи от придворных врачей, отправил нарочного просить святого Антония, чтобы прислал к нему Агапита. Но преподобный отшельник отрекся идти в Чернигов. «Я не могу оставить святой обители, "--- сказал он, "--- для снискания славы человеческой, которой убегать до последнего издыхания дал клятву Самому Богу». И с посланным от Мономаха боярином отправил лекарства, вместо диеты "--- воздержание, молитву и добрые дела. Князь, скоро и совершенно выздоровев, сам приехал в Печерский монастырь, чтобы щедро наградить своего избавителя. Но святой Агапит скрылся, и Владимир принужден был привезенные для него дары оставить игумену. В изъявление своей благодарности он вторично послал одного из своих придворных также с дарами, но истинный и бескорыстный врач бедных и богатых и тогда, не приняв оных, сказал посланному: «Я никогда ни от кого и ничего не брал, ибо не мое искусство, но сила Христова исцеляет болящих. Пусть государь твой раздаст нищим, просящим милостыни именем Христа, не только эти дары, но и все избытки от сокровищ княжеских, из которых мы, отходя в вечность, ничего взять с собой не можем».

Этот преподобный врач и молитвенник, за три месяца предузнав свою кончину, не переставал лечить болящих братьев и пришельцев до последнего часа. Умер он в глубокой старости и погребен был в пещере преподобного Антония.

\section{Гнев Божий за удаление от обязанностей, свыше возлагаемых}

Преподобный Зосима Киликийский, инок Синайской горы\footnote{Этот преподобный отец жил в IV~в. Память его празднуется 4~(17) июня.}, с юности возжелав уединенной жизни, удалился в Ливийские пустыни и там служил Единому Богу. Однажды, ходя по необитаемым местам, он увидел старца, одеянного власяницей, и, приблизившись к нему, хотел принять благословение. Но старец, предупредив Зосиму, сказал: «Зосима, зачем ты сюда пришел? Возвратись в Синай. Бог не благоволит твоему здесь пребыванию». Удивленный Зосима спросил, почему он знает его. «За два дня перед этим, "--- отвечал пустынник, "--- явился мне некто, по образу чудный, и сказал: к тебе идет синайский инок Зосима. Вели ему отсюда уйти на прежнее место, поскольку хочу поручить ему Вавилонскую Церковь в Египте»\footnote{В древности было два Вавилона: один в Халдее, где царствовал Навуходоносор, другой "--- меньший, построенный в позднейшие времена в Египте переселенцами из Великого Вавилона.}. Потом, немного отступив, пустынник помолился Богу, облобызал Зосиму и, сказав: «Мир тебе, чадо! Погреби тело мое и молись о душе моей», "--- возлег на землю и скончался. Зосима отдал последний долг усопшему и, помня его пророчество, возвратился в Синай, чтобы ожидать там исполнения оного.

Но, сколько преподобный отец ни принуждал себя жить в иноческом обществе, не мог преодолеть стремления к пустынному безмолвию и опять дерзнул оставить гору Синайскую. Взяв ученика своего Иоанна, он ушел в пустыню, нарицаемую Порфирот, и, встретив там двух отшельников, Павла и Феодора, близ них поселился. Но Бог вскоре наказал его непослушание. В один день ученика его, Иоанна, ужалила змея столь жестоко, что от яда в тот же час повредилось все тело, и юноша умер. Сетующий Зосима пошел к Павлу и Феодору, чтобы в беседе с ними несколько утешиться, но святые пустынники, встречая его, наперед спросили: неужели умер ученик его? Зосима удивился их прозорливости и, рассказав о горестном приключении, повел их к мертвецу. Пустынники советовали ему не сокрушаться, увидев же усопшего, воскликнули: «Юноша, восстани! Старец имеет в тебе нужду, поскольку ему должно идти в Синай, чтобы принять там епископство Вавилонской Церкви в Египте». Мгновенно Иоанн ожил, а Зосима, воспомянув волю Господню, открытую ему прежним пустынником, узрел в смерти ученика своего начало Божеского гнева за непокорство. Он не смел задать никакого вопроса преподобным чудотворцам и, тогда же приняв от них благословение, опять возвратился в Синай.

Вскоре настоятель послал святого Зосиму за некоторым монастырским делом в Александрию. Там патриарх Аполлинарий удержал его и рукоположил в епископа Вавилону Египетскому.

Удаление от обязанностей, свыше на нас возлагаемых, по большей части имеет причины неизвинительные. Да убоимся гнева Божия!

\section{Снисходительность к грешникам и строгость против грехов}

Некоторый брат за согрешение отлучен был от святого общества. Когда же начали изгонять его из обители, преподобный Виссарион\footnote{Прп. Виссарион Египетский, чудотворец (†~IV -- V~в.). Память его празднуется 6~(19) июня.} встал и пошел с ним вместе, сказав: «И я такой же грешник». Но сколько человек Божий наставлял учеников своих милосердию к грешникам, столько же советовал им мужественно противоборствовать грехам. «Иноку должно стараться, "--- говорил он, "--- да будет весь око, якоже Херувим и Серафим: ибо грех пресмыкается невидимо. Против кого не воюют страсти, тот еще более да блюдется и смиряет себя пред Богом, чтобы надежда на самого себя не свергнула его тем стремительнее. За самомнение многие преданы были на брань, напротив того, Господь часто отвращает брань плоти от немощных, чтобы до конца не погибли».

Подражая преподобному Виссариону, не должно осуждать ближнего, видя сучок в глазу его, а лучше смотреть, нет ли у нас чего"=нибудь более!

\section{Правило, как жить с другими}

Один брат спросил у преподобного Виссариона: «Как должно вести себя, живя с другими?» "--- «Говори только тогда, когда нужно, "--- отвечал старец, "--- а особенно не соперничай с братией».

Этот совет весьма полезен в каждом духовном и светском состоянии, даже в каждом семействе.

\section{Небесная стража}

Когда преподобный Кирилл Белоезерский\footnote{Этот чудотворец родился в Москве от благородной и благочестивой четы и, осиротев, постригся в отрочестве. Был у братии служителем, потом архимандритом в Симоновом монастыре. На шестидесятом году от рождения он ушел оттуда и хотел остаток дней своих провести в пустынном безмолвии. Но, видя собирающуюся к нему из разных мест братию, построил монастырь, нареченный Белоезерским. Скончался 90~лет, в 1427~г. Память его празднуется 9~(22) июня.} созидал и украшал монастырь, то окрестные жители весьма удивлялись успеху работ и думали, что старец имеет великие богатства, а один корыстолюбивый и злонравный помещик отважился учинить разбой и послал слуг своих, чтобы святого Кирилла и его обитель ограбили. Вечером злодеи собрались в близлежащем лесу и дожидались ночи, чтобы неожиданно напасть на спящих иноков, но, подъехав к монастырю, увидели вокруг ограды великое множество вооруженных людей, из которых иные стреляли из лука, иные ходили с копьями, иные держали в руках обнаженные сабли. Разбойники думали, что какой"=нибудь воевода с отрядом оруженосцев пришел к преподобному Кириллу ради благословения и молитв, почему опять ушли в лес и смотрели, когда воины удалятся от обители. Но, тщетно прождав почти до рассвета, решили, наверное, что богомолец, замедлив, остался тут ночевать, и, возвратившись домой, рассказали господину своему все, что видели. Злочестивый помещик приказал им сделать то же покушение в следующую ночь, но разбойники увидели тогда еще больше вооруженного народа и в трепете опять прибежали домой. Тогда жадный корыстолюбец послал в монастырь одного из сообщников разбоя, чтобы искусным образом спросил у иноков, кто вчера и третьего дня был у них. Но когда служитель принес ответ, что в монастыре более недели не было ни богомольцев, ни посетителей, то Феодор (так назывался помещик) удивился и ужаснулся. Он познал, что Бог ополчает Ангелов, и, страшась, да не постигнет его Суд Господень, немедленно пошел к святому Кириллу и, обливаясь слезами, исповедал свой грех и чудное видение двух ночей. Святой настоятель, возблагодарив Бога, спасающего Своих избранных, довольно поучил Феодора от Божественного Писания и, даровав ему прощение, сказал: «Поверь мне, чадо, и другим скажи, что я с самого вступления в иночество ничего у себя не имею, кроме одежды, которую на мне видишь, и нескольких книг; а деньги на сооружение и украшение святой обители доставляет Тот же, Кто сохраняет оную». Раскаявшийся помещик возвратился домой, радуясь и благодаря Бога, что не допустил его оскорбить угодника Своего, и с того времени сделался первым почитателем добродетелей Кирилла и щедродателем для его святой обители.

Христиане! Если чистое сердце есть достойнейшая обитель и великолепнейший храм для Бога, то и все добродетельные люди должны ожидать от Него такой же помощи, какую получил преподобный Кирилл. Он \textit{Ангелам Своим} заповедует о каждом праведнике "--- \textit{хранить} его \textit{во всех путех его: яко на Мя упова}, вещает Он устами порфироносного пророка, \textit{и избавлю его, покрыю его, яко позна имя Мое: воззовет ко Мне: и услышу его: с ним есмь в скорби, изму его и прославлю его} (Пс. 90, 11, 14~-- 15).

\section{О том, сколь спасительно в несчастиях призывать на помощь угодников Божиих}

Когда святой Вассиан\footnote{Свт. Вассиан, епископ Лавдийский (†~409), был сын наместника Сиракузского; язычником отправившись в Рим для усовершенствования себя в науках, принял там Святое Крещение вместе с дядькой. Потом был пресвитером в Равенне и, наконец, епископом. Жил в царствование Валентиниана, Феодосия Великого, Аркадия и Феодосия II. Умер в 90~лет. Память его празднуется 10~(23) июня.}, бывший после епископом Лавдийским, жил в Равенне при церкви священномученика Аполлинария, "--- в то время от императора Валентиниана к тамошнему градоначальнику пришло повеление предать смерти чиновника, по имени Вифимний, единственно по клевете злобных людей. Немедленно несчастный был окован и без суда приведен на место казни. Чувствуя свою невинность, Вифимний воспомянул святого Вассиана и, сказав сам в себе: «Раб Божий, благодатью, тебе от Бога данной, будь мне ныне помощником!» "--- преклонил главу свою. Но, едва исполнитель смертного приговора поднял секиру, чтобы ударить по шее Вифимния, секира вывалилась из рук его и упала на землю. Изумленный палач схватил оную и опять, держа крепко, направил удар. Но секира невидимой силой опять исторглась и упала. То же случилось и в третий раз. Градоначальник, думая, что тут кроется умысел, чтобы пощадить преступника, повелел другому исполнить суд царев. Но и тот, к общему удивлению, равным образом трижды тщетно поднимал руку свою на Вифимния: секира всегда исторгалась и упадала без действия. Народ, громко вопия, требовал освобождения узнику, столь явно Богом покровительствуемому. Градоначальник не знал, чему приписать это чудо, и, приказав содержать его крепко, но человеколюбиво, под стражей, обратился к императору о столь чрезвычайном происшествии и в истине оного свидетельствовался всем городом. Валентиниан, не менее удивляясь тому, предписал рассмотреть доносы на Вифимния и, увидев одни происки неприятелей, повелел немедленно освободить невинного и возвратить ему все чины и достоинства.

Вифимний всенародно прославил имя угодника Божия Вассиана и, прямо из темницы прибегнув к нему, омыл слезами благодарности его руки и ноги. С того времени народ почитал святого Вассиана не иначе, как Ангелом"=хранителем.

\section{Святые мученики Мануил, Савел и Исмаил}

Святые мученики Мануил, Савел и Исмаил\footnote{Мученики Мануил, Савел и Исмаил (†~362). Память их празднуется 17~(30) июня.}, родные братья, были персияне, тайно в христианском благочестии воспитанные. Занимая важные государственные места в отечестве, они отправлены были государем своим к греко"=римскому императору Юлиану Отступнику для утверждения мира между обеими державами. Посланники сначала приняты были с честью, но презревший святую веру вскоре презрел и святость прав народных, что происходило следующим образом.

Наступил языческий праздник в честь Аполлона. Юлиан, как усердный поклонник идолов и желая притом показать чужестранцам пышность двора своего, захотел отправить этот праздник как можно великолепнее и приказал царедворцам, чиновникам и народу быть в Вифинии. Там воскурились жертвы, загремела музыка, начались пиршества и показались все мерзости, какими только когда"=нибудь отличались язычники. Рабы Христовы "--- Мануил, Савел и Исмаил, хотя находились также в Вифинии по обязанности всюду быть с императором, однако не хотели и смотреть на богопротивное празднество, но в уединении оплакивали заблуждение народа, а наипаче государя, рожденного в христианской вере. Когда Юлиан приказал спросить у них: «Почему не принимают участия в молениях и удовольствиях государя?» "--- святые братья отвечали кротко, что они, как христиане, не могут воздавать чести идолам. Греческий чиновник намекнул им, что за такую грубость неминуемо подвергнутся они гневу императорскому. Тогда посланники, уважая свой сан, сказали: «Мы имеем необходимость быть у его величества только тогда, когда дело идет о взаимной пользе наших государств. Доложите вашему государю, что и мы здесь "--- наместники великого государя». Услышав это, Юлиан воспылал гневом и повелел взять благочестивых персиян под стражу. Видно, богоотступник давно искал случая оскорбить Персию и вызвать против себя государя, который искал мира. Иначе нельзя объяснить в царе, отличавшемся образованием, этот поступок, на который редко отваживаются и варварские правители.

На следующий день Юлиан потребовал их к себе и начал укорять в зловерии (так богоотступник называл правоверие). «Возможно ли вам, "--- говорил он, "--- заступать при мне место вашего государя, когда не наблюдаете веры его? Возможно ли стараться о пользе отечества, когда презираете богов отечественных? Будьте единодушны с нами в вере, тогда будем единодушны и в договорах о пользе государств». "--- «Государь! "--- отвечали святые юноши. "--- Обязанности посланников известны тебе и нам; мы имеем наставление от своего двора, о чем с тобою договариваться: безопасность границ, дружественное сношение и торговля "--- вот предметы, для которых наш государь отправил к тебе посольство. Препираться о вере, которой должно держаться, не есть наша обязанность. Что же касается до нашей верности государю и усердия к отечеству, узнаешь тогда, когда будешь рассуждать о выгодах Рима и Персии». Защищая свой сан, особенно ревнуя по Христианскому благочестию, они употребили несколько выражений, неприятных для богоотступника, и этим так раздражили его, что Юлиан воспылал гневом. «Я угожу приятнейшим образом царю вашему, "--- сказал он, "--- если предам пыткам и казни предателей его веры и оскорбителей моего союзника» "--- и повелел истощить на святых братьях все мучения, а потом отсечь им головы. Так пострадали и умерли за Христа Спасителя святые мученики Мануил, Савел и Исмаил. Но тиран, поправший все права Божественные и человеческие, не был доволен оскорблением Персии в лице ее посланников: он собрал силы почти всех известных тогда народов и пошел мстить, не зная за что, той державе, которая имела полное право отмстить ему жесточайшим образом.

Всякому известно, какой конец имело это предприятие богоотступника, запечатленное кровью святых мучеников: Юлиан погиб.

\section{Попечение о спокойствии ближнего}

Преподобный Иоанн Колов\footnote{Он был наставником прп. Арсения Великого в иноческих подвигах и написал жития некоторых святых пустынножителей.} однажды с некоторыми братьями пошел из скита. Дорогой застигла их ночь, и провожатый, к несчастию, сбился с пути. «Авва, что нам делать? "--- сказали Иоанну спутники. "--- Скитаясь здесь, можно погибнуть». "--- «Не говорите провожатому ничего, "--- отвечал старец. "--- Бедный устыдится и будет печалиться, досадовать и, может быть, роптать. Лучше сделаем вот что: я притворюсь, будто болен; скажу, что не могу идти далее, и останусь здесь, пока не рассветет». Таким образом и поступил. Прочие объявили, что останутся с ним; все пробыли тут до утра и не причинили сокрушения брату.

Вот высокий пример, как должно каждому из нас дорожить спокойствием ближнего!

\section{Благоговение пред святым старцем}

Преподобный Паисий Великий\footnote{Прп. Паисий Великий (†~V~в.) родился в Египте, постригся в отрочестве по указанию Ангела, который в сновидении явился его матери, и был величайшим в постниках того времени. Память его празднуется 19~июня (2~июля).} был столь славен благочестием и всеми добродетелями, что современники называли его не иначе, как «божественным» и почитали за высшее счастье, если сподобятся слышать от него священные уроки и принять благословение. Блаженный Пимен\footnote{Прп. Пимен Великий (†~ок.~450). Память его празднуется 27~августа (9~сентября).}, имея тогда не более семи лет от рождения, также чрезвычайно желал увидеть святого Паисия, но долго не мог удовлетворить желание своего сердца, поскольку, думая о нем как об Ангеле, не смел приступить к нему. Наконец, он начал просить преподобного Павла\footnote{См. в житии прп. Паисия.}, друга Паисиева, чтобы он сходил с ним к человеку Божию, но и этот старец сказал ему: «Ты еще дитя; как поведу тебя к столь великому мужу, когда и мы, состарившиеся в постничестве, приходим к нему с большой осторожностью, в известное только время, единственно для пользы душевной и как можно строже испытав свою совесть? Поверь, чадо, я стыжусь идти с тобой к великому Паисию». "--- «Но я, "--- отвечал юный Пимен, "--- по крайней мере, постоял бы у его кельи. Ты войди к нему один и начни душеспасительный разговор: счастлив буду, если услышу его мудрые наставления, а если невозможно и это, спокоен останусь, если прикоснусь к его келье и облобызаю следы ног его! Отец мой! Не лиши меня хоть этого благословения». Преподобный Павел, видя столь пламенную ревность в дитяти, наконец, согласился взять его с собой и, достигнув кельи Паисиевой, оставил за дверьми.

Но великий пустынник едва встретил его, как вдруг спросил: «Где твой юный спутник?» "--- ибо святой душевными очами издалека увидел Пимена. «Он остался вне кельи, боясь беспокоить тебя», "--- отвечал Павел. Паисий немедленно кликнул отрока, сам ввел его и, благословляя, сказал: «И нам, старым людям, вещает Спаситель: \textit{Аще не будете яко дети, не внидете в Царство Небесное} (Мф. 18, 3). Поверь мне, любезный о Христе брат, "--- обратясь к Павлу, продолжал он, "--- это дитя спасет много мужей и старцев, ибо с ним десница Господня». Потом, возложив руки свои на главу Пимена, вторично благословил его и после продолжительной беседы о спасении отпустил их с миром.

Дети! Этот малолетний Пимен, который столь пламенно желал слушать поучения от великого Паисия, сам после назван был Пименом Великим, и его старались увидеть христолюбивые люди с таким же нетерпением. Любезные дети! Если хотите быть утешением родителей, надеждой отечества, наследием Христовым, то более всего ищите случая беседовать с людьми благочестивыми.

\section{Российские первомученики}

Святой мученик Феодор\footnote{Память святых мучеников Феодора Варяга и сына его Иоанна, умерших в Киеве (†~983), празднуется 12~(25) июля.}, родом варяг, в бытность свою в Греции принявший христианскую веру, по торговым делам жил с сыном своим Иоанном в Киеве и, жалея о заблуждении столь храброго и вместе с тем добродетельного народа, немолчно вопиял против многобожия. За что идолопоклонники ненавидели его, а жрецы искали случая отомстить ему, не подвергая опасности свою честь.

В то время великий князь Владимир Святославич, еще не просветившийся крещением, возвратился в Киев с победой, которую одержал над ятвягами, жившими на реке Бугу. И поскольку покорение многочисленного и воинственного народа\footnote{Из этого воинственного народа никто в плен живым не отдавался, даже женщины смерть предпочитали плену. Владимир победил их в 983~г.} было весьма приятно Владимиру, то определено было принести богам в благодарность чрезвычайную жертву: заклать на алтаре изуверства юношу или девицу. В собрании знатных людей кинули жребий, кому пролить кровь свою; хитрые жрецы направили жертвенный нож на Иоанна, сына Феодорова. Немедленно были посланы к нему нарочные, которые именем богов, князя и народа потребовали юного христианина. «Боги возлюбили твоего сына, "--- говорили они, "--- и им угодно взять его к себе». Ревнующий по Христе Феодор вместо того, чтобы искать способа уклониться от ярости неверных, почел этот случай удобнейшим, дабы сильнее прежнего уличить их в постыдном многобожии. «Боги ваши, "--- сказал он, "--- не суть боги, но идолы, сделанные из дерева; ныне есть, а вскоре сгниют. Сами видите, что они бездушны и бесчувственны, не видят, не слышат, не говорят: как же могли они потребовать моего сына? Они не едят, не пьют: на что им Иоанн? Несчастные! Познайте наконец, что Един есть Бог, Безначальный и Вечный, в трех лицах поклоняемый, в Которого веруют греки и я с сыном. Он сотворил небо и землю, солнце, луну, звезды и человека. Он дал нашему князю и нынешнюю победу над супостатами». Но так как посланцы начали угрожать ему, что гнев их богов прольется на его непокорную главу, то святой Феодор изгнал их с бесчестием.

Народ, узнав ответ христианина, воспылал гневом и, подстрекаемый жрецами, с бешенством бросился к дому Феодора с оружием и дрекольями, начал ломать ворота и нагло кричать: «Выдай нам сына, и мы принесем его в жертву богам».

Нетрепетный воин Христов вышел на крыльцо. Величественная картина! Он стоит, подобно кедру, младой порослью обвитому, "--- стоит, обхваченный руками юного сына, и, спокойно взирая на бешенство народа, отвечает: «Если боги ваши захотели есть, то пусть пришлют от себя одного которого"=нибудь за моим сыном; а вам что стараться о них?» Тогда"=то совершенно обезумела бессмысленная толпа: они зарыкали, как тигры, подсекли крыльцо, и святой Феодор первый пал под их ударами; святой Иоанн своей кровью обрызгал мерзкий истукан Перуна; весь дом разметан был до основания.

Так пострадали российские первомученики. Но их святая кровь, яко дождь, нисшедший на руно, оросила семя Христовой веры, посеянное равноапостольной Ольгой: великий Владимир вскоре просветился и просветил весь народ крещением. На том месте, где стоял дом Феодора, воздвигнут был храм в честь и славу Пресвятой Богородицы.

\section{Сила Господних заповедей\footnote{Повесть прп. Даниила Фаранского.}}

Дочь некоторого чиновника в Вавилоне одержима была нечистым духом так, что изгнать его не могли никакие старания. К счастью, однажды к отцу ее пришел знакомый инок и, узнав о страдании девицы, сказал: «Должно испытать последнее средство "--- призвать на помощь молитвы известных мне благочестивых пустынников: только, если будешь о том просить их, они по своему смирению откажутся от столь славного дела. Разве поступим следующим образом. Когда придут они на рынок, притворись, будто желаешь купить их рукоделия; как скоро вступят в дом твой, попроси у них благословения дочери твоей. Я уверен, что возложение рук их на главу девицы исцелит ее». После этого отец болящей с иноком вышли на площадь, увидели одного из учеников старческих, продающего корзины, купили и, по уговору, для расплаты повели его в дом. Но, едва пустынник переступил порог, беснующаяся вскочила с места и ударила его по ланите. Инок, по заповеди Господней, подставил другую. Вдруг нечистый дух почувствовал мучение и возопил: «О непостижимая сила! Заповедь Христова изгоняет меня!» Тотчас же девица исцелилась. Все прославили Бога и уверились, что заповедь смирения низвергает гордость дьявольскую.

Если какая"=нибудь порочная страсть начнет терзать наше сердце, расстраивать ум, вооружимся заповедями Христа Спасителя. Тогда непременно будем победителями.

\section{Молитва об удалении смертного часа}

Преподобная Евпраксия\footnote{Преподобная Евпраксия, дева, Тавеннская (†~413). Память ее празднуется 25~июля (7~августа).}, узнав из откровения свыше, что час ее смерти наступает, так испугалась и ослабела, что пала на землю, как бездыханная. Чада веры, Христу спогребающиеся! Не обвиняйте праведницу в малодушии, но познайте причину такой великой горести.

Когда подвижница Господня опомнилась, то излила слезную молитву в следующих словах: «Зачем, Царю и Боже мой, возгнушался меня, сироты и грешницы? Се время, в которое бы мне наипаче должно трудиться для ближних и противоборствовать врагам, на душу воюющим. Ты же смерть ниспосылаешь мне. Умилосердись о мне, Господи, и оставь для жизни моей хотя один год, чтобы могла я покаяться в моих согрешениях и совершить служение, для которого Промысл Твой призвал меня в это малое, но возлюбленное Тобою общество! Ах! Я не имею добрых дел и не могу получить награды, уготованной праведникам».

Вот почему ужасалась смерти святая Евпраксия, дочь знаменитых родителей, воспитанница императора Феодосия Великого, которая, будучи десяти лет, восприяла на себя образ ангельский и более двадцати лет, в пример святой обители, подвизалась в молитвах, постничестве и послушании.

Христиане! Мы не знаем, когда придет час смертный, но всякому известно, что мимо нас не пройдет он. Этого"=то часа хотелось бы нам не видеть! Для чего же? Верно, еще родители не получили от нас утешения в старости? Еще дети не воспитаны в страхе Господнем? Еще несколько бедных, по праву, данному им Спасителем, не получили от нас помощи? Еще отечество не получило от нас никакой пользы? О! До этого, как до дела постороннего, нам и нужды нет, а что касается грехов, их мы и вспомнить не хотим. Мы не желаем смерти потому только, что хотим жить, и жить хотим для того, чтобы вкус свой притупить в бесчисленных удовольствиях света и собрать на главу свою обильнейшую жатву гнева Божия. Но, если бы каждый из нас желал отдалить смерть свою с чувствованиями преподобной Евпраксии, тогда бы и семейства, общественные сословия и сами государства были во сто крат благополучнее.

\section{Три брата, знаменитые в российской древности}

Три брата: Георгий, Ефрем и Моисей, уроженцы угорские, в древних российских летописях сияют: первый "--- как герой верности своему государю, а два другие "--- как светильники веры и всех добродетелей. Все они были придворные бояре благоверного князя и страстотерпца Бориса.

В то время, когда убийцы, посланные Святополком, напали на князя Бориса при реке Альте, верный Георгий\footnote{Этого мученика верности государю в числе святых не обретается; но Церковь, воспевая св. страстотерпца Бориса, вместе ублажает и всех верных подданных, живот свой за него положивших.} повергся на него всем телом, чтобы прикрыть от наносимых ему ударов. Он был первой жертвой злодеяния Святополкова, первый устремил на себя ярость убийц и первый лишен главы, которую отсекли для снятия с шеи золотой гривны, пожалованной ему князем Борисом.

Старший из братьев, Ефрем\footnote{Память прп. Ефрема Новоторжского (†~1053) празднуется 28~января (10~февраля).}, услышав об убиении Георгия, немедленно отправился на место несчастного происшествия, чтобы взять тело его, но нашел только главу, которую хранил у себя до самой смерти. Он оставил боярский сан и дом, ушел в Торжок и, избрав близ оного прекрасное местоположение на берегу реки Тверцы, построил храм во имя своих государей Бориса и Глеба, собрал иноков, основал обитель и был архимандритом ее. Георгий пожертвовал жизнью за благотворения своего государя, а преподобный Ефрем посвятил дни свои единственно на то, чтобы несчастиям и добродетелям его воздать вечную память.

Моисей\footnote{Память прп. Моисея Угрина, Печерского, в Ближних Пещерах (†~ок. 1043), празднуется 26~июля (8~августа).}, младший брат, во время убиения князя Бориса и всех с ним находившихся, один избавился от смерти и, прибежав в Киев, скрылся у Предиславы, сестры Ярославовой, но Болеслав, по взятии Киева, в числе пленных отослал в Польшу и Моисея. Руководимый Богом, Моисей, по прошествии некоторого времени, ушел из Польши, опять водворился в пещеру великого учителя веры, святого Антония, безмолвствовал там десять лет и там же блаженно преставился.

\section{Снисхождение к ближнему}

Брат некоторой обители учинил прегрешение. Настоятель объявил о том живущему в окрестности, славному по добродетелям пустыннику, который и осудил его на изгнание. Несчастный вышел из обители и, сидя в пещере, горько плакал. По случаю, кто"=то из братии шли к преподобному Пимену\footnote{Память прп. Пимена Великого (†~ок. 450) празднуется 27~августа (9~сентября).} и, увидев рыдающего инока, сжалились и начали звать его к великому старцу. Но грешник, усугубив слезы, сказал им: «Я должен здесь умереть». Братия уведомили об этом святого Пимена, который, благословив их сострадание, отпустил с тем, чтобы они именем его пригласили к нему удрученного горестью брата. Брат пришел, преподобный Пимен обнял его и, обходясь с ним весьма ласково, просил укрепиться пищей. Потом одного из учеников своих послал к пустыннику сказать: «Уже несколько лет слыша о твоих великих делах, желаю с тобой видеться, но, видно, по общей моей и твоей лености, мы по это время друг у друга не бывали. Теперь Сам Бог послал к тому случай. Итак, возьми на себя труд, приди "--- и увидимся». Пустынник хотя никогда не выходил из своей кельи, но, подумав, что Пимен не послал бы за ним, если бы Сам Господь ему не внушил того, немедленно отправился в путь. Когда они сделали друг другу радостное приветствие и сели, тогда авва Пимен сказал: «В одном месте жили два человека, и случилось обоим иметь у себя по мертвецу, но один из них, оставив своего мертвеца, пошел оплакивать чужого». Пустынник, услышав это, был поражен притчей святого Пимена и, воспомянув свой поступок против согрешившего инока, сказал: «Пимен всегда выше и выше! А я всегда ниже и ниже!»

\section{Общая всем слабость}

Один христианин жаловался преподобному Пимену, что дурные помыслы возмущают и подвергают великой опасности душу, желающую спастись. Старец взял его за руку и, поставив на открытом воздухе, велел, распахнув пазуху, удержать ветер. «Я не могу сделать этого», "--- сказал христианин. «Если не можешь сделать этого, "--- возразил святой Пимен, "--- то не можешь воспрепятствовать и тому, чтобы не возмущали тебя помыслы. Но обязанность наша "--- противиться оным».

\section{Умерщвление тела}

Однажды авва Исаак, увидев, что преподобный Пимен возливает на ноги свои воду, сказал: «Другие ведут себя строже и удручают тело свое». "--- «Мы научены умерщвлять не тело свое, но страсти», "--- отвечал святой старец.

\section{Пример вместо приказания}

Один старец спросил у преподобного Пимена: должно ли ему приказывать братьям, которые с ним живут? «Нет! "--- отвечал Пимен. "--- Лучше делай сам; если они захотят с тобой жить, сами увидят, что и им должно делать!» "--- «Но они сами желают, "--- возразил старец, "--- чтоб я, как старший, приказывал им». "--- «Не советую, "--- сказал Пимен, "--- будь для них примером, а не законодателем! Тогда все будет исполняемо в точности, и, сколь бы труд велик ни был, не будет даже тени роптания».

\section{Богоотцы Иоаким и Анна\footnote{Рождество Пресвятой Богородицы празднуется 8~(21), а память святых Иоакима и Анны "--- 9~(22) сентября.}}

Богом возлюбленные супруги Иоаким и Анна провождали жизнь святую и добродетельную. Изобилуя богатством вещественным, они более прославились богатством духовным. Кроткие, целомудренные, благочестивые, они на каждый праздник отделяли из своего стяжания две части: одну "--- Богу на церковные потребности, другую "--- нищим. Но этим супругам Бог долго не давал чада, не давал для того, чтобы явить на них Свою милость, от века неслыханную: ибо в глубокой старости, после пятидесятилетнего безплодства, они произвели на свет «радость и спасение» всему человечеству.

В великий праздник, когда сыны Израилевы приносили дары Богу, сетующий о безчадии своем Иоаким также пришел в храм Иерусалимский; но первосвященник Иссахар не хотел принять от него жертвоприношения и с укоризной сказал: «Ты не стоишь того, чтобы принять от тебя дар Богу; будучи безчаден, конечно, не имеешь ты благословения Божия за какие"=нибудь тайные грехи». Эту укоризну усугубил другой евреянин, с досадой сказав Иоакиму: «Зачем прежде меня приступаешь к Богу? Разве не знаешь о своем неплодстве?» Тогда"=то Иоаким ужаснулся своего безчадства; посрамленный и униженный, в лютой скорби пошел он от храма Господня. «Увы! "--- сказал он. "--- Всем праздник великий, а мне время сетования!» "--- и не хотел возвратиться в дом свой, но удалился в пустыню к пастырям стад своих и там оплакивал свое поношение в людях. «Боже отцов моих! "--- неусыпно молился скорбящий праведник. "--- Ты праотцу Аврааму, уже состарившемуся, даровал сына, сподоби и меня благословения Твоего, даруй плод моему супружеству хотя в старости, да нарекуся отец, да не буду яко отверженный от Тебя, Господа моего». Святой Иоаким с молитвой соединил сорокадневный пост.

Между тем праведная Анна, услышав о столь чувствительном оскорблении супруга и об удалении его в пустыню, не могла утешиться. «Горе мне, несчастнейшей на свете! "--- вопияла она. "--- Богом я отвержена, от людей укоряема, супругом оставлена! О чем прежде начну плакать? О вдовстве ли моем? О безчадии или о бесславии в женах израильских?» Сколько Юдифь, верная рабыня, ни утешала ее, но тщетно: праведная Анна рыдала и день и ночь.

Однажды, истерзанная печалью, она пришла в сад и, севши под тенью лавра, вздохнула из глубины сердца; потом, возведя к небу наполненные слезами очи, увидела на дереве птичье гнездо, в котором лежали едва оперенные птенцы: тут болезнь ее сердца перешла меру. «Все в природе рождает и воспитывает! "--- воскликнула она. "--- Я одна лишена радости сердечной. Птицы небесные! Вас благословляет Господь, ибо вы утешаетесь детьми своими. Звери дубравные! Вас благословляет Господь, ибо вы кормите детей своих. Земля, общая мать наша! И ты Богом благословенна, ибо на лоне твоем растут древа и цветы, и на них рождаются плоды. Я одна безчадна на земле! О Ты, Который разверз утробу матери пророка Твоего Самуила, Адонаи Господь! Умилосердись надо мной, да, благословляя благоутробие Твое, посвящу Тебе рожденное дитя». В это мгновение вдруг является пред ней Ангел Божий. «Анна! Анна! "--- вещает он. "--- Услышана молитва твоя; воздыхания твои проникли сквозь облака; слезы твои канули пред Господом. Се зачнешь и родишь дочь Богоблагодатную, о которой возрадуются все племена земные; ибо ею дастся спасение всему миру». Праведница преклонила колена. «Жив Господь Бог! "--- воскликнула она. "--- Имеющее родиться да будет Ему в жертву благоприятную». Исполнившись неизреченной радости, святая Анна немедленно пошла в Иерусалим воздать благодарение Богу за дар, внезапно обетованный.

В то же самое время благовеститель Небесный обрадовал и праведного Иоакима. Возвестив ему рождение дочери, он сказал: «Иди в Иерусалим к церкви Господней; там, у златых врат, увидишь супругу свою». Удивленный и обрадованный старец спешит в Святой град, встречает Анну, по словам ангельским, и объявляет ей благовестие свыше. Праведница рассказывает то же. Оба прославили Бога и, поклонившись Ему в храме, возвратились в дом свой.

Вскоре оправдались слова Господни. Анна зачала в девятый день декабря, а в восьмой день сентября родила Пречистую и Преблагословенную Деву Марию, Ходатаицу нашего спасения.

\section{Святая великомученица Евфимия\footnote{Память великомученицы Евфимии всехвальной (†~304) празднуется 16~(29) сентября.}}

Эта страдалица, дочь благородного и богатого человека, именем Филофрона, за исповедание Христовой веры представлена была на суд в числе сорока девяти других христиан, которые под разными мучениями испустили дух свой.

Но святая Евфимия, как юная и прекрасная девица, казалась драгоценнейшей добычей для мучителя: посему, отдалив ее от прочих страдальцев, тиран употреблял всевозможные средства, чтобы уговорить ее к нечестию: ласками, обещаниями, пожертвованием своего сердца уловлял он сердце девическое, но все было безуспешно! Тогда Приск (так именовался мучитель) приказал приготовить орудия казни. Но усеянное серпами колесо останавливала десница Всемогущего; раскаленную печь окропили Ангелы росой прохладной; змеи и все ядовитые гады не имели для нее смертоносных жал; лютые звери претворялись в кротких агнцев. Неверные были в недоумении, но окаменело сердце их и не чувствовало, что тут присутствовал Сам Бог, неизреченные чудеса называли они чародейством.

Наконец, святая великомученица излила теплую молитву к Богу: «Господь всех сил! Ты удивил на мне Твое непобедимое всемогущество; Ты обличил немощь идолов и безумие мучителя; приими и мою жертву "--- сердце смиренно и дух сокрушен, как принял кровь прежде бывших мучеников; упокой меня в скиниях святых Твоих». Праведница умолкла, взирая на небо. Тогда один зверь сделал легкую рану на ноге ее; истекла кровь; раздался голос с неба, призывающий ее к Жениху Небесному, и мгновенно святая душа оставила чистое тело.

\section{Потребна осторожность, когда на кого"=нибудь жалуются другие}

У некоторого старца Силуана был ученик Марк, искусный в рукописании книг\footnote{Тогда не было книгопечатания, а потому искусство переписчиков было важным ремеслом.} и за послушание весьма любимый своим наставником, на что другие двенадцать учеников смотрели с большим неудовольствием и, наконец, везде начали жаловаться на пристрастие Силуана. Некоторые из пустынников поверили им и, однажды придя к преподобному Силуану, начали обвинять его в чрезвычайной приверженности к Марку и в холодности к прочим. Старец, выслушав их, вышел вон и, подходя к каждой келье, стучался: «Такой"=то брат (называя по имени), выйди! Мне есть до тебя дело». Но ни один из них не выходил, ни один не откликался на голос Силуана. Когда же он достиг кельи Марка и, толкнув в двери, сказал: «Марк!» "--- в то же мгновение Марк выскочил, и старец послал его на службу. Тогда Силуан сказал братии: «Видите сами, каковы все двенадцать учеников! Бог знает: или они дома и не хотят слушать меня? Или ушли, куда вздумалось?» Потом вошел с ними в келью Марка и, взглянув на его тетради, нашел, что он, услышав голос наставника, не хотел дописать даже буквы «\textsf{ω}» и, бросив перо, к нему выбежал. Видя это, удивленные старцы сказали: «Авва! Люби Марка еще более; Сам Бог, без сомнения, любит его».

\section{Дочери премудрости: Вера, Надежда и Любовь\footnote{Память мучениц Веры, Надежды, Любови и матери их Софии празднуется 17~(30) сентября.}}

В царствование римского императора Адриана была некоторая вдовица, родом итальянка, именем София, то есть премудрость. Она имела у себя трех дочерей: Веру, Надежду и Любовь, тезоименитых трем богословским добродетелям. В самом деле, что христианская мудрость произвести может, как не веру во Христа, надежду на Христа и любовь ко Христу?

Тогда гонение на христиан свирепствовало ужаснейшим образом; тираны не щадили ни рода, ни пола, ни возраста; по малейшему подозрению в Христовой вере схватывали каждого, и жребий узников решался всегда одинаково: должно было или поклониться идолам, или умереть в лютых мучениях. После этого удивительно ли, что неверные столь зверски поступили с юными отроковицами Верой, Надеждой и Любовью? Когда отреклись они принести жертву Артемиде, мучитель хотел поколебать твердость матери свирепством над ее дочерями. Но тщетно! Все они, одна после другой, предали дух свой в руки Божии.

Сначала Вера, имевшая от рождения двенадцать лет, как старшая сестра, должна была вытерпеть все казни, но, укрепляемая благодатью, не испустила ни единого болезненного вздоха. «Помолись обо мне, дражайшая родительница, "--- только говорила она, "--- да совершу подвиг и удостоюсь видеть лицо Господне; а вы, любезные сестры, до конца пострадайте за Того, Кому уневестились». Потом, облобызавшись с ними, святая Вера преклонила под меч главу свою.

После нее вышла на священный подвиг десятилетняя Надежда. Тиран начал увещевать ее, чтобы отреклась от Распятого, но она отвечала: «Неужели я не сестра той, которую ты умертвил? Не от одной ли с ней утробы родилась? Не одним ли крещением просветилась? Не одно ли Священное Писание научило нас верить, надеяться и любить Христа Спасителя? Как же можно мне другое мыслить и другого желать, нежели сестра моя Вера?» Потом, обняв бездыханное тело Веры и облобызав матерь и Любовь, приняла ту же кончину. Наконец, предстала юная Любовь, имевшая от роду девять лет, и претерпела те же мучения, ту же казнь и с такой же неустрашимостью: ибо кто может поколебать любовь к Жениху Небесному, \textit{краснейшему всех сынов человеческих}? (ср. Пс. 44, 3). Эта любовь \textit{крепка, яко смерть; вода многа не может угасити любве, и реки не потопят ея} (Песн. 8, 6~-- 7).

Так страдали и умерли святые девицы, в вечную славу своего имени и в посрамление тиранов! Матерь их, взирая на муки, воссылала усерднейшие молитвы к Богу, дабы укрепить их терпение. Ни единой слезы не выронила она, ибо верила в те небесные награды, которые Бог обещает пострадавшим за Него. Мучитель не хотел умертвить ее "--- не по милосердию своему, не из уважения к ее несчастному сиротству, но по жестокости нового рода: он хотел, чтобы мать страдала всю жизнь свою, вспоминая погибель, как называл нечестивый, дочерей своих.

Святая София заключила тела их в одну гробницу и погребла за несколько поприщ\footnote{\textit{Поприще} "--- путевая мера, и вероятно суточный переход, около 20 верст. — В. И. Даль.} от города. Нежная мать три дня молилась над священным их прахом и уснула смертью святых. Верные погребли ее вместе с чадами, а Церковь вместе с ними включила ее в лик мучениц, ибо хотя не плотью, но сердцем она вытерпела за Христа весь ужас мучений.

Божественная Премудрость! Облистай свыше наш разум, да не заблудится, да не падет он под мраком тех туч, которые застилают свет горний от мира сего. А вы, боголюбезные дочери ее, будьте всюду нашими спутницами. Святейшая Вера! Вещай нам ежечасно, что эта жизнь есть только преддверие настоящей, блаженной жизни, которая начинается за гробом, есть поприще подвигов, неусыпность или нерадение в которых будет иметь последствием наши награды или наказание в вечности. Несомненная Надежда! Будь твердым для нас якорем в житейском море, воздвигаемом бурей напастей; не дай нам погрязнуть в унынии или отчаянии. А ты, Любовь нелицемерная, душа Серафимов! Обними твоим пламенем все наше сердце, да, соделавшись жилищем Единого Христа, за нас пострадавшего, будет оно неприступно для развратов и сует, которые воздвигает на него князь тьмы.

\section{Смиренномудрие преподобного Сергия, игумена Радонежского\footnote{Память прп. Сергия, игумена Радонежского, чудотворца (†~1392), празднуется 5~(18) июля и 25~сентября (8~октября).}}

Святитель Христов Алексий, митрополит Всероссийский, изнемогая от старости и чувствуя приближение кончины, призвал к себе преподобного Сергия и, сняв с персей своих драгоценную панагию, хотел на него возложить. Смиренный старец весь смутился от столь неожиданного поступка и едва мог сказать: «От юности моея не бых златоносец, ей старости же наипаче хощу в нищете пребыти». "--- «Знаю, возлюбленный о Христе брат! "--- отвечал архиерей Божий. "--- Но будь известен, какое о тебе имею намерение, и окажи послушание. Я содержал российское первосвятительство, сколько угодно было Богу, Который мне даровал оное; ныне же, видя себя близким к кончине, желаю при жизни моей обрести мужа, могущего после меня пасти стадо Христово, и в тебе вижу достойного преемника». Услышав это, преподобный Сергий восчувствовал чрезвычайную горесть. «Прости, владыка святой, мое непокорство! "--- отвечал он. "--- Я никогда не возьму на себя бремени, превышающего силы мои; и я ли, грешный, осмелюсь коснуться сана архипастырского?» Сколько блаженный Алексий ни представлял ему убеждений из Священного Писания, но смиреннолюбец остался непреклонен и просил святителя впредь об этом даже не напоминать ему, а иначе он убежит и скроется в неизвестных и непроходимых местах.

Блаженный Алексий в этой духовной борьбе, наконец, должен был уступить преподобному; он опасался, чтобы столь великий светильник веры на самом деле не удалился от сообщества человеческого, и, благословив его, отпустил с миром.

\section{Добродетель в очах истинного христианина есть высочайшее достоинство\footnote{Из жития прп. Кирилла Белоезерского (†~1427), память которого празднуется 9~(22) июня.}}

Когда преподобный Сергий, Радонежский чудотворец, уже достиг глубокой старости и славился не только по всей России, но и в Греции\footnote{Так, например, Царьградский патриарх Филофей через нарочных прислал в дар прп. Сергию крест, параман°, схиму°° и грамоту, в которой, восхваляя жизнь и подвиги чудотворца, советовал ему установить в своей обители общее житие, что праведник тогда же и исполнил. || °~Параман (или аналав) — платок, носимый монахами на груди, с изображением восьмиконечного с основанием креста. °°~Схима — великий ангельский образ (икона), а также монашеский чин, налагающий самые строгие правила.} примерной жизнью и Господним к нему благоволением, тогда преподобный Кирилл, после нареченный Белоезерским, будучи еще в средних летах, отправлял в Симоновом монастыре должность братского повара. Но добродетельные всегда уважают добродетельных, как бы ни велика была разность в их состоянии, а от преподобного Сергия не могла утаиться святость преподобного Кирилла. Радонежский чудотворец, приходя иногда в Симонов монастырь для посещения архимандрита Феодора, своего родственника, каждый раз прежде заходил в поварню к блаженному Кириллу и по нескольку часов беседовал с ним наедине о пользе душевной. После того навещал архимандрита и прочих благочестивых старцев, но чаще сам архимандрит с братией, узнав о его прибытии, приходили в поварню. Там, приняв друг от друга благословение, они начинали между собой разговаривать, а святой Кирилл подходил к очагу и принимался за свою работу.

\section{Без правой веры спастись невозможно}

В тесной келье близ реки Иордана жил боголюбезный старец Кириак\footnote{Память прп. Кириака отшельника (†~556) празднуется 29~сентября (12~октября).}. Однажды пришел к нему издалека один брат, по имени Феофан, чтобы принять уроки строжайшего иночества. Но когда мудрый Кириак начал поучать его чистоте и смирению, то пришлец, восхищенный боговдохновенными речами старца, сказал ему: «Отче! Не расстался бы с тобой до смерти, но как единомышленник несториан\footnote{Так назывались одни из древних еретиков.} я должен возвратиться к моей братии». Услышав имя Нестория, Кириак затрепетал и, сокрушаясь о погибели гостя, начал умолять его, дабы он присоединился к Апостольской Церкви. «Един путь к нашему спасению, "--- говорил он, "--- есть тот, чтобы мыслить и веровать, как мыслили и веровали святые отцы». Но старец отвечал: «И все ереси также утверждают, что человек может спастись только с их вероисповеданием. Что же после этого делать мне, окаянному? Все называют учение свое боговдохновенным, а мой разум так слаб, что не может постигнуть истины и различить ее от заблуждения. Человек воистину Божий! Помолись Господу, дабы Он каким"=либо откровением свыше просветил мои душевные очи». Святой Кириак неизреченно возрадовался о приближающемся спасении инока. «Останься в моей келье, "--- сказал он, "--- и веруй от всего сердца, что Всеблагий Бог откроет тебе истину». Потом удалился от него и в уединении начал за своего брата молиться Богу. В девятом часу следующего дня несторианин внезапно увидел перед собой боголепного, но взором грозного юношу, который повелел ему идти за ним. Вдруг несторианин очутился в преддверии темного и смрадного места, где посреди тусклого, один только мрак несколько озаряющего пламени увидел он Нестория, Диоскора, Ария и прочих еретиков, испускающих стенание и скрежет нестерпимой муки. «Се обиталище тех, "--- указывая на тартар, сказал таинственный юноша, "--- которые нечестиво умствуют, а также тех, которые следуют их учению! Итак, если это место тебе нравится, держись раскола Нестория или других злочестивцев, если же этой казни ужасаешься, приступи к святой Апостольской Церкви, в которой пребывает Иорданский пустынножитель. Я говорю тебе истину, не забудь ее: если человек имеет все добродетели, но мыслит нечестиво, низвержен будет в это место». После того юноша стал невидим.

Несторианин едва пришел сам в себя и, когда возвратился святой Кириак, рассказал ему все, что происходило. Немедленно присоединился он к Православной Церкви и, прожив несколько лет у своего наставника, скончался в мире.

\section{Покров Пресвятой Богородицы}

В царствование греческого императора Льва Премудрого, в первый день октября, в славной Влахернской церкви, сооруженной в Царьграде в честь Приснодевы, совершалось всенощное бдение. Храм наполнен был народом, между которым находился и святой Андрей, Христа ради юродивый. В четвертом часу ночи прозорливый старец внезапно возвел очи свои к небу и увидел, что храм как бы не имел сводов, увидел Небесную Царицу, Матерь Божию. Сияя паче солнца, Она стояла на воздухе, окруженная пророками, апостолами, святителями, мучениками и всеми Небесными Силами, и, покрывая христоименитых людей чистым своим омофором\footnote{\textit{Омофор} "--- часть архиерейского облачения.}, молилась за них к Сыну Своему и Богу. Святой Андрей спросил у стоявшего тут же ученика своего, блаженного Епифания: «Видишь ли, брате, Ходатаицу мира?» "--- «Вижу, святой отче! "--- отвечал он. "--- И ужасаюся».

С того времени установлен праздник Покрова Пресвятой Богородицы, дабы христиане, вспоминая чудо это, знали, к Кому должно прибегать в несчастии и Кого благодарить, освободившись от оного.

\section{Святой Роман Сладкопевец}

Святой Роман\footnote{Св. Роман Сладкопевец (†~ок. 556) жил в царствование Иустиниана~I. Память его празднуется 1~(14) октября.} был церковнослужителем при храме Святой Софии, Премудрости Божией. Но угодник Божий сколько был добродетелен и благочестив, столько несведущ. Несмотря на это, патриарх Евфимий за неусыпность, усердие и послушание так возлюбил его, что давал равную часть с прочими клириками, составлявшими хор церковный, а это и было причиной их зависти и злобы на святого. Все с роптанием говорили: «Как не совестно патриарху равнять с нами невежду!» "--- и Романа оскорбляли самым наглым образом.

Но Бог часто избирает \textit{буяя міра, да премудрыя посрамит} (1~Кор. 1, 27). В навечерии Рождества Христова, когда сам государь присутствовал в церкви и святой Роман по обыкновению зажигал свечи, певцы, толкая его с ругательством, говорили: «Ты получаешь равную с нами часть, поди же на амвон и пой с нами Богохвальную песнь». Приняв при всем народе столь несносное посрамление, он не мог удержаться от слез и по окончании службы Божией, когда все вышли из церкви, пал пред образом Пресвятой Богородицы, рыданием прерывая молитвы свои. Возвратившись домой, от печали он не вкусил пищи и заснул. Вдруг явилась ему Богородительница, утешение всех скорбящих, держа в деснице Своей небольшой свиток хартии, и тихо рекла ему: «\textit{Отверзи уста и снеждь сие}». Святой Роман проглотил хартию и пробудился. Приснодева стала невидима, а он, восчувствовав в сердце своем неизреченную радость и в уме своем внезапный луч ведения и премудрости, повергся на колена и, воздав благодарение Небесной Наставнице, с душевным восхищением пошел на всенощное бдение. Когда должно было петь кондак\footnote{\textit{Кондак} "--- перечень похвал.}, Богом вдохновенный Роман взошел на амвон\footnote{Тогда было в обыкновении одному певцу возглашать кондак на амвоне.} и сладкогласно воспел составленное им в уме своем: «\textit{Дева днесь Пресущественного рождает, и земля вертеп Неприступному приносит}» и прочее. Все изумились столь чудесной внезапности и с величайшим вниманием слушали пение. После того патриарх спросил у него: «Откуда премудрость сия?» Святой певец, прославляя Богородительницу, рассказал сновидение, и посрамленные клирики должны были раскаяться в причиненных ему оскорблениях.

Вскоре патриарх посвятил его в диакона, и премудрость всегда, \textit{яко река}, текла из уст человека Божия. Те, которые издевались прежде над его невежеством, теперь почитали за счастье быть его учениками.

\textit{Насладися Господеви, и даст ти прошения сердца твоего} (Пс. 36, 4), "--- сказал царь и пророк. Любезные дети! Если хотите просветить разум ваш в познание истины, то просите помощи Небесной. \textit{Просите, и дастся вам} (Мф. 7, 7). Если ваши молитвы будут подобны молитвам святого песнопевца Романа, то будет \textit{светлость Господа Бога нашего} на вас, \textit{и дела рук} ваших \textit{исправит} (ср. Пс. 89, 17). Не призывая в помощь сил небесных, полагаться на свои только силы "--- бесполезно; а полагаться на чужеземных просветителей "--- опасно. Они могут "--- даже стараются дать вам (и сколь часто дают вам!) вместо хлеба камень, и вместо рыбы змию.

\section{Змий сребролюбия}

Один старец был набожен и строг в жизни, но, к несчастию, поработился духу сребролюбия. Он жил близ торжища будто для того, чтобы народ, каждый день собираясь туда, имел вблизи наставника и отца духовного, а на самом деле для того, чтобы посетители, возвращаясь с торжища, приходили к нему всегда не с пустыми руками. Святой Андрей, Христа ради юродивый\footnote{Блаженный Андрей, Христа ради юродивый (†~936), жил в X~в. Память его празднуется 2~(15) октября.}, идучи однажды мимо сребролюбца, увидел по откровению Божию, что страшный змий обвился вокруг его шеи. Прозорливец остановился и смотрел с изумлением, а чернец, думая, что он, будучи одним из нищих, ожидает милостыни, сказал: «Бог даст!» Святой Андрей отступил на несколько шагов и увидел над ним воздушную надпись: «Змий сребролюбия "--- корень всякому беззаконию». Соболезнуя о погибели брата, Андрей хотел было удалиться, но едва обратился вспять, как увидел двух юношей, спорящих между собой: один был черен, с очами мутными и свирепыми, а другой прекрасен, как свет Небесный; первый утверждал, что старец принадлежит ему, ибо не Богу, а ему работает, но другой говорил, что этот старец, будучи смирен и кроток, пребывая в постничестве и молитвах, Божий слуга есть. Так препирались они, как вдруг раздался глас Небесный к светоносному юноше: «Ты не имеешь части в старце: оставь его», "--- и Ангел мгновенно удалился. Удивленный Андрей от жалости заплакал и пошел в путь свой. Но сердце Андрея "--- в глазах человеческих юродивого, а в очах Божиих мудрого Андрея "--- не могло быть спокойно: желая обратить несчастного старца к истине, он через некоторое время опять пошел мимо торжища и, увидев старца, взял за руку. «Раб Божий! "--- смотря пристально, сказал он. "--- Выслушай без гнева меня, раба твоего. Ах! Я весьма сокрушаюсь о тебе и не могу без ужаса помыслить, что ты, прежде этого друг Божий, ныне сделался слугой вражеским. Душа твоя имела крылья, как Серафим: зачем неприязненному духу дал обрезать оные? Ты имел образ ангельский: зачем восхотел быть темнообразен? Зачем подружился со змием сребролюбия и греешь его на лоне твоем? Зачем собираешь сокровища? Неужели с ними погребен будешь? Неусыпно собирая сребреники от детей духовных и увеселяясь ими, ты собираешь на главу твою все их преступления. Как же с таким бременем предстанешь Страшному Суду Христову?

Познай, что Ангел"=хранитель твой далече отступил от тебя и плачет о твоей погибели, а дьявол куда хочет ведет тебя. Я говорю правду: послушай меня и раздай имение твое нищим, вдовицам, сиротам и странникам, в противном случае зле погибнешь. Видишь ли дьявола?» Тогда отверзлись душевные очи старца. Черный, зверообразный исполин стоял на дальнем от него расстоянии и скрежетал зубами, но, ужасаясь святого Андрея, не смел к нему приблизиться. Старец затрепетал, от ужаса обмер и едва мог произнести обещание жить по заповедям Господним. Но, едва он дал клятву, вдруг от востока блеснула молния, нечистый дух исчез, и Ангел Господень приблизился к иноку. «Не открывай никому, что случилось, "--- сказал святой Андрей, отходя от него, "--- а я буду поминать тебя каждый день и ночь в молитвах, чтобы Господь управил путь твой на доброе».

Когда чернец раздал золото свое и начал жить совершенно по заповедям Господним, раб Божий Андрей явился ему в сновидении и, улыбаясь от удовольствия, показал на противолежащем поле сеннолиственное древо, усыпанное прекрасным цветом. «Благодари Бога, "--- сказал он, "--- что исторгнул Он тебя от зубов змеиных и оросил Своей благодатию душу твою: это цветущее древо "--- ее изображение; постарайся же, чтобы этот цвет принес сладкие плоды». Старец, воспрянув ото сна, еще более утвердился на добрые дела и всегда приносил благодарение Богу и угоднику Его святому Андрею.

\section{Обращение в христианскую веру святого священномученика Дионисия Ареопагита\footnote{Память священномученика Дионисия Ареопагита (†~96) празднуется 3~(16) октября.}}

Святой Дионисий родился в Афинах от благородной крови. На двадцать пятом году жизни, изучив весь круг известных тогда наук, Дионисий отправился в Илиополь, чтобы познать и египетскую мудрость.

Это"=то время ознаменовано было тем ужасным днем, когда солнце померкло в полдень, дабы не видеть земли, оскверненной лютейшим из всех злодеяний "--- убиением Господа нашего Иисуса Христа. Удивляясь столь необыкновенному мраку, Дионисий сказал: «Или страждет Бог, всего мира Создатель, или пришла кончина миру!» Конечно, не мудрость века этого, а благодать Духа Святого вдохнула в него дар пророчества о страсти Владычней.

Спустя некоторое время после того Дионисий возвратился в Афины и, как гражданин, знаменитый мудростью и благородством происхождения, причтен был в сочлены древнего судилища, называемого ареопагом. В то время святой апостол Павел прибыл в Афины и перед старейшинами города проповедовал Христа Господня. Мудрейшие люди внимательно слушали посланника Христова, но сколь сильна приверженность к миру и его идолам! Вместо того, чтобы тогда же насытиться от \textit{воды живые}, после которой человек \textit{не вжаждется во веки}, они отложили рассмотрение истин, проповедуемых Павлом, до другого времени. Не так ли поступаем и мы, откладывая день за днем покаяние наше и спасение?

Но Дионисий был не таков. Глубоко проникнутый неслыханным учением, наедине вступил он в разговор с Павлом и, рассуждая о богах своего отечества, вызвался показать ему капища их, на что святой Павел согласился "--- не из обыкновенного любопытства, но для того, чтобы этот случай употребить в пользу веры Христовой. Намерение апостола тогда же исполнилось. Между прочими божницами он увидел храм, над которым было написано: «Неведомому Богу». «Кто же есть этот неведомый Бог?» "--- спросил он у Дионисия. «Он еще не явился, "--- отвечал Ареопагит, "--- но в свое время придет. Этот Бог будет владычествовать над небом и землей, и владычеству Его не будет конца». Эта минута была удобным временем для того, чтобы Апостолу начать сеять на добрую землю Слово Божие. «Сей Бог, "--- сказал он, "--- уже пришел и родился от Пречистой Приснодевы Марии и претерпел для нашего спасения смерть крестную. На страдание Его даже солнце взирать не могло и на три часа погрузилось в глубокую тьму для всей вселенной. Сей Бог воскрес из мертвых и вознесся на Небо. Сего познай, о Дионисий! В Сего веруй, Сему на службу посвяти сердце твое». Тогда"=то Дионисий вспомнил тьму, которой удивлялся в Египте, вспомнил тогдашние свои чувствования и, приписывая их вдохновению Божества, немедленно уверовал в Господа нашего Иисуса Христа и крестился с супругой и детьми.

Святой Дионисий был настолько счастлив, что слышал проповедь не одного Павла, но всех апостолов, и удостоился быть с ними на погребении Пречистой Богородицы и Приснодевы Марии. Был какое"=то время епископом в Афинах, потом в Галлии и там, в царствование Домициана, принял венец мученический.

\section{Небесное руководство}

Однажды преподобный Аммон\footnote{Память прп. Аммона (†~ок. 350) празднуется 4~(17) октября.} пошел к Антонию Великому и сбился с дороги; севши на камень, он ненадолго уснул; потом, пробудившись, начал молиться к Богу, да не допустит погибнуть созданию рук Своих. Тогда явилась из облаков рука, указующая ему путь, и шествовала над ним дотоле, пока святой Аммон не увидел пещеры Антониевой.

Христиане! Вот что делает молитва, от чистого сердца воссылаемая к Богу в каком"=либо несчастии! Но не забудьте, что преподобный Аммон шел тогда к великому наставнику веры, благочестия и всех добродетелей.

\section{Видение о грешнике, принявшем твердое намерение жить непорочно}

Блаженный авва Павел Препростый\footnote{Память прп. Павла Препростого (†~IV~в.) празднуется 4~(17) октября.}, ученик Антония Великого, пришел в одну обитель посетить братию. После обыкновенных разговоров все иноки пошли в церковь к вечернему славословию. Тогда святой Павел стал у дверей храма и примечал, кто с каким расположением духа входил в церковь Божию: ибо Господь даровал ему благодать видеть душу всякого человека, как другие видят лицо. Каждый инок проходил с веселым взором, и каждого из них сопровождал Ангел, также радующийся. Между ними был только один мрачен и расстроен: дух соблазна, хватая его за руки, влек от храма Господня, Ангел же хранитель шел за ним вдали, печальный и с поникшей головой. Преподобный Павел облился слезами и, ударяя себя в грудь, сел у порога церковного. Иноки, увидев столь внезапную перемену и опасаясь, не приметил ли человек Божий какого"=либо греха за всеми, начали просить его, дабы открыл причину скорби и рыдания. Но Павел не отвечал им ни слова и, оставаясь в том же положении, беспрестанно обращал исполненные слез очи на несчастного брата. Наконец, святая служба совершилась, иноки пошли из церкви, и преподобный Павел опять начал с великим вниманием смотреть на них. Что же он увидел? Тот, который прежде был так мрачен и расстроен, шел с мирным и веселым взором; Ангел Божий радостно вел его за руку, а бесплотный соблазнитель, заходя со всех сторон, старался, но не смел приступить к нему. Тогда прозорливый старец воспрянул от радости и, благословляя Бога, воскликнул: «Сколь неизреченна благодать и любовь Господня к человеку!» Он стал посреди братии и сказал велегласно: «Приидите и видите дела Божии; видите Того, \textit{Иже всем человеком хощет спастися и в разум истины приити} (1~Тим. 2, 4). Придите, поклонимся и припадем Ему и скажем: Ты Един можешь очистить грехи наши!» Вся обитель стеклась на голос его, и святой старец объявил, что видел, когда иноки входили в церковь, и что видел, когда выходили они из церкви. Потом спросили у брата, почему Бог даровал ему столь внезапное обращение? Будучи для общей пользы обличен старцем, он в присутствии всех начал рассказывать дерзновенно: «Я человек грешный; издавна и до этого времени слушался врага души моей и жил порочно. Но сегодня вошел в церковь, услышал слова пророка Исаии, или, лучше сказать, Самого Бога: \textit{Измыйтеся и чисти будите, отымите лукавства от душ ваших пред очима Моима. Научитеся добро творити. И аще будут греси ваши яко багряное, яко снег убелю: аще же будут яко червленое, яко волну убелю} (Ис. 1, 16~-- 18). Слушая столь чадолюбивое увещание свыше, я восчувствовал все укоризны моей совести и, воздыхая из глубины сердца, сказал: “Боже, \textit{пришедый в мир грешники спасти, от нихже первый есмь аз}! (1~Тим. 1, 15). Исполни на мне, унывающем, яко обещал еси чрез пророка Твоего. От этого времени я приемлю твердое намерение, обещаюсь от чистого сердца, клянусь, что не сделаю зла пред Тобой; отрицаюсь от всякой скверны и буду служить Тебе в чистоте совести. Днесь, Господи, в это мгновение, приими меня кающегося!”» Братия, услышав это, возблагодарили Бога, а свыше вдохновенный Павел сказал: «О христиане! Зная из Божественного Писания и Святых откровений, сколь великую благодать являет Господь к тем, которые искренно прибегают к Нему и покаянием очищают прежние грехи свои, "--- зная, что этот нежный Отец повелел Ангелу"=хранителю снова принимать каждого грешника под свое покровительство в самую минуту его сердечного покаяния, не отчаивайтесь о вашем спасении. \textit{Живу Аз}, глаголет Господь, \textit{не хощу смерти грешника, но еже обратитися нечестивому от пути своего и живу быти ему} (Иез. 33, 11)».

\section{Слава и безславие}

Авва Авраам спросил у аввы Феодора, что лучше и полезнее: славы ли искать или бесславия? «Я более желаю славы, нежели бесславия, "--- отвечал старец, "--- ибо, если совершу что"=нибудь доброе и приобрету славу, могу обвинять только сам себя, что недостоин ее. А бесславие происходит от дурных дел. Каким же образом утешу мое сердце, когда буду служить в соблазн и преткновение людям? Итак, лучше приобретать славу». "--- «Почему же все великие отцы ничего так не боялись, как славы?» "--- возразил Авраам. «Они боялись ложной славы, "--- отвечал Феодор, "--- или, лучше сказать, они боялись, чтобы их сердца не заразило славолюбие. Между славой, которая необходимо следует за добрыми делами, и славолюбием, которое все заповеди Господни меняет на суетность мира, есть великая разность. Делай добро единственно для угождения Богу, тогда Бог не только не осудит славу, которую человек приобретает без всякого о ней попечения, но к этой славе присовокупит еще ту славу, которую от века уготовал Он любящим Его».

\section{Забывать добродетели предков есть грех, достойный наказания Божия\footnote{Из Пролога, в 29"~й день мая.}}

В одном монастыре с древних лет наблюдаем был следующий благочестивый и достохвальный обычай: ежегодно в Великий четверг приходили туда из всех окрестных мест убогие, вдовицы и сироты и брали из общего имущества иноков по установленной мере пшеницы, по нескольку вина и меда и по пяти медниц, чтобы Светлое Христово Воскресение могли провести без нужды и в радости. Однажды случился неурожай, и цена на хлеб возвысилась чрезвычайно. Братия, хотя съестных припасов имели довольно, но, думая, что подаяния щедролюбцев на то время для них прекратятся, предложили настоятелю, чтобы в этот год нарушить благочестивое обыкновение и в Великий четверг не давать бедным пшеницы. Долго добродетельный настоятель не соглашался на просьбу братии. «Грех "--- нарушать уставы, данные нам святым основателем обители, "--- говорил он, "--- грех "--- не надеяться, что Господь пропитает нас». Но поскольку братия решительно сказали, что не хотят сами быть голодными для того, чтобы кормить других, то он с душевным прискорбием отвечал им: «\textit{Творите, якоже хощете}», "--- и бедные, которые шли в обитель с надеждой, возвратились оттуда с отчаянием.

Но Небесный Отец убогих видимым или невидимым образом не минует наказать немилосердных людей. В Великую субботу монастырский ключарь пошел в житницу, чтобы на Пасху выдать для хлебов чистой муки; но едва отворил двери, как почувствовал дурной запах: вся пшеница так сгнила, что должно было бросить ее в реку. Братия удивлялись, жалели и не знали, что делать. А благочестивый настоятель, спокойно посмотрев на испортившийся хлеб, сказал им: «Кто преступает заповеди святого отца, основателя обители, не надеется на Промысл Божий и не милосердствует о бедных, тот непременно должен быть наказан за преслушание. Вы пожалели пятисот мер "--- и погубили пять тысяч. Вперед знайте, на Бога ли уповать должно или на свои житницы».

К несчастию, и мы начали забывать благочестивое обыкновение наших предков "--- кормить, по крайней мере, несколько раз в году не имеющих насущного хлеба! Не оттого ли у нас и бедные умножаются беспрестанно?

\section{Видение матери, сетующей о смерти сына\footnote{Из страданий св. мученика Уара (†~ок. 307), память которого празднуется 19~октября (1~ноября).}}

Когда святой мученик Уар, после ужаснейших страданий предав душу свою в руки Господни, выброшен был за город на съедение псам (так идолопоклонники старались посрамить христиан), тогда одна вдова, родом из Палестины, по имени Клеопатра, с малолетним сыном Иоанном и верными слугами в глубокую полночь унесла святые останки Уара в дом свой. И вскоре под видом тела мужа, который незадолго перед тем умер, с позволения правительства увезла их в отечество. Там на кладбище своих предков она соорудила храм во имя страдальца и со священным собором, при многочисленном народе, поставила в нем чудотворные мощи.

По окончании литургии и молебного пения Клеопатра приготовила угощение духовенству и народу, нищим и странникам, и как она, так и сын ее сами служили при столе. Но к вечеру Иоанн вдруг разболелся так тяжко, что не мог даже отвечать на вопросы матери; пламень ежечасно увеличивался, и в полночь юный Иоанн умер. Отчаянная мать побежала в церковь святого Уара и, повергшись пред гробом, горько вопияла: «Угодник Божий! Возврати мне дитя мое. Без него некому при смерти закрыть очей моих, некому предать гробу тело мое. Ах! Лучше бы мне самой умереть, нежели видеть умершим сына моего. Угодник Божий! Или умилосердись надо мной, как Елисей некогда умилосердился над вдовицей Соманитской, или возьми теперь же и мою душу. Не могу жить без возлюбленного Иоанна». Стеня и рыдая таким образом у гроба Уарова, от усталости и чрезмерной скорби Клеопатра заснула. Вдруг явился ей святой мученик, держа на руках своих Иоанна. Оба они, имея на себе белые как снег ризы, на чреслах златые пояса, а на главах драгоценные венцы, блистали, подобно солнцу. Восхищенная Клеопатра поверглась на колена, но святой Уар поднял ее. «Усердная почитательница страстотерпцев! "--- сказал он. "--- Я не забыл благодеяний, оказанных тобой моему телу в Египте, и здесь всегда слышу твои молитвы и молюсь о тебе к Богу. Сначала я умолил Господа о твоих предках, вокруг меня почивающих, да отпустит им согрешения их; потом, желая в полной мере вознаградить тебя за добродетели, я умолил Бога, чтобы сопричислил и сына твоего к лику святых. Увидь славу его: он ныне един от предстоящих Престолу Господню. О чем же столь горько сетуешь? Или не хочешь, чтобы Иоанн наслаждался тем блаженством, еже \textit{око не виде, ухо не слыша, и на сердце человеку не взыдоша}? (1~Кор. 2, 9). Если так, возьми его обратно». При этом слове юный Иоанн обнял Уара и, крепко держась за шею его, сказал: «Нет! Не отдавай меня родительнице, хотя и всем сердцем люблю ее, не отпускай меня опять в мятежный и развратный мир, оставь меня навсегда с собой. А ты, любезная родительница, "--- обратясь к Клеопатре, продолжал он, "--- зачем плачешь обо мне? Если любишь меня, то радуйся моему счастью: я сожитель всех святых и воинов Царя Небесного». Безмолвствовавшая доселе Клеопатра вокликнула: «Блаженные души! Жители Неба! Возьмите и меня с собой, сделайте и меня участницей блаженства ангельского». Но святой Уар отвечал ей: «Теперь иди с миром на богоугодные дела твои; живи по"=прежнему. Будет время, когда Господь возьмет душу твою на Небо: там уготовано для тебя такое же место». После этого святой Уар и святой Иоанн стали невидимы, а Клеопатра проснулась. Почувствовав себя исполненной неизреченной радости, она отерла слезы и, поведав чудотворное видение епископу, торжественно погребла Иоанна при гробе святого Уара.

Она сопровождала его не как мертвеца на лоно сени смертной, но как жениха в чертог брачный. Потом раздала имения свои святым обителям и бедным семействам, а сама, удалившись от мира, соорудила келью при церкви святого Уара и там служила Богу день и ночь в посте и молитвах "--- до блаженной кончины. Часто в сонном видении или посреди молитвенного восхищения являлся ей святой Уар в славе Небесной, держащий на руках сына ее.

Родители! Если кто"=либо из вас имел несчастие лишиться сына или дочери, оплачьте драгоценную потерю; только жестокосердый может осудить вашу чувствительность. Но \textit{не скорбите, якоже и прочии не имущии упования} (1~Фес. 4, 13). Послушайте, что сказал Господь: \textit{Оставите детей приити ко Мне, таковых бо есть Царство Небесное} (Мф. 19, 14). \textit{Храняй младенцы в нынешнем житии "--- и в будущем уготовал им пространство, Авраамово лоно, и по чистоте ангельские светообразные места, в нихже водворяются праведных дуси}\footnote{См. в Требнике чин погребения младенцев.}.

\section{Мысли святых Отцов}

Для чего заповедей закона десять, а блаженств девять? Число заповедей равняется числу казней Египетских, а число блаженств есть изображение Троицы, троекратно приемлемой\footnote{Мысли св. Епифания, епископа Кипрского.}.

Жена Хананейская вопиет "--- и Господь услышал ее; жена кровоточивая молчит "--- и Господь исцелил ее. Напротив того, фарисей вопиет "--- и осуждается; мытарь молчит, однако пошел оправдан в дом свой\footnote{Св. Епифаний, епископ Кипрский.}. Из этого ясно, что Бог не смотрит, с воплем ли, с ударением ли себя в перси возносится к нему молитва, а требует сокрушенного сердца и совершенной надежды на Его помощь.

Апостольское изречение: \textit{время искупуйте} (ср. Еф. 5, 16) "--- предполагает духовную корысть. Например, тебе пришло время оскорблений: это время искупи смирением и терпением и тем приобрети корысть. Пришло время бесславия: купи оное великодушием и тем приобрети корысть.

Таким образом, если захотим, то даже все противные обстоятельства принесут нам корысть, то есть пользу, душевную\footnote{Ответ св. папы Феофила на вопрос прп. Феодоры.}.

Нет животного, которое было бы столь сильно, как лев, но из"=за угождения чреву и лев попадает в сети; вся крепость его тут погибает\footnote{Прп. Иоанн Колов (†~V~в.) был наставником прп. Арсения Великого (†~449/450).}.

Дом не может быть созидаем от верха книзу, но от основания идет до высоты. Основание есть ближний; устраивая пользу его, мы полагаем несокрушимое начало, на коем утверждаются все заповеди Христовы\footnote{Прп. Иоанн Колов.}.

Все добродетели между собой равны. Священное Писание объявляет, что Авраам был страннолюбец и Господь возлюбил его. Илия провождал житие пустынное, и Господь возлюбил его. Давид был кроток и смиренномудр, и Господь возлюбил его. Нееман был праведный вельможа, и Господь возлюбил его. Иуда Маккавей проливал кровь свою за отечество, и Господь возлюбил его. Мардохей был верный раб царя персидского, и Господь возлюбил его. Итак, к какой добродетели более наклонна душа твоя, в оной и упражняйся во славу Божию "--- и спасен будешь\footnote{Авва Нистерий, друг прп. Антония Великого.}.

Люди каждый день поутру и вечером должны размышлять сами с собой, что сделали они и чего не сделали угодного Богу, чтобы в следующий день быть исправнее\footnote{Авва Нистерий.}.

Разбойник был на кресте, и одна молитва оправдала его; Иуда был несколько лет апостолом и, потеряв в одну ночь все труды, от неба нисшел до ада. Да не возносятся слабые смертные своими добродетелями, ибо все надеющиеся на самих себя погибали.

«Быть скупым, "--- говорит преподобный Исаия\footnote{См. о нем в житии прп. Памвы пустынника (†~IV~в.), 18~(31) июля.}, "--- есть не веровать, что Бог о нас печется».

«Прежде между людьми, "--- поучая молодых иноков, сказал преподобный Феодор\footnote{Прп. Феодор Освященный (†~368), ученик Пахомия Великого (†~ок. 348), память которого празднуется 16~(29) мая.}, "--- занятие души было единственным занятием, а занятие рук было только дополнением к занятию, а ныне занятие души обратилось в дополнение к занятию, дополнение же к занятию сделалось занятием». "--- «А что значит занятие души, которое ныне почитаем только прибавлением? "--- спросили иноки. "--- И что есть прибавление, которое употребляем вместо занятия?» "--- «Я называю занятием души то, что делается по заповеди Господней, "--- отвечал Феодор, "--- а трудиться и собирать для себя самого есть прибавление». Иноки потребовали объяснения, и авва Феодор продолжал: «Например, я слышу, что известный мне человек болен, и, поставляя за долг навестить его, говорю сам с собой: оставить ли мне дело свое и пойти к нему? Нет; прежде исправлюсь дома; еще успею сходить. Встретилось другое упражнение, и я опять нейду. Потом другой брат говорит мне: сделай милость, помоги мне. А я отвечаю: недосуг! Таким образом, если не пойду к больному и не помогу брату, то оставлю заповедь Божию, то есть занятие души, и выполню прибавление, то есть занятие рук».

Ныне большая часть людей принимают успокоение прежде, нежели им Господь даст оное\footnote{Прп. Феодор Освященный.}. Если военачальник хочет взять осажденный город, прежде всего старается не допускать туда жизненных припасов. И враги, погибая от голода, принуждены бывают сдаться. Поставь себя на месте военачальника; плотские страсти "--- твои неприятели. Живи же воздержно, и тогда все враги, \textit{воюющие на душу}, ослабеют и падут\footnote{Прп. Иоанн Колов.}.

Всякую заповедь отеческую (следственно, и всякое повеление вообще начальства) должно исполнять по апостольскому учению \textit{без гнева и размышления} (1~Тим. 2, 8), ибо какая польза послушнику, имеющему в руках дело, а в устах роптание? Какая польза приказание исполнять, а языком или умом прекословить? И будет ли такой человек когда"=либо истинным членом братии (следственно, и всякого государственного сословия)? Нет. Кто вместе с послушанием носит в недрах своих и змею роптания, тот самый ленивый, самый злой человек, носящий только личину трудолюбия и добродетели\footnote{См. в житии св. Иоанна Дамаскина.}.

\section{Нечто о Святых Таинах}

В житии преподобной Феоктисты\footnote{Память прп. Феоктисты (†~881) празднуется 9~(22) ноября.}, между прочим, упоминается, что один благоговейный простолюдин, пристав на корабле к пустынному острову, где подвизалась она в молитвах и постничестве, получил от нее поручение привезти в чистом ковчежце часть пречистых и животворящих Христовых Таин. Так что на другое лето, отправляясь туда же для ловли зверей, он с величайшим усердием исполнил, и святая Феоктиста причастилась Тела и Крови Христовой.

Здесь некоторые из православных удивятся, как Божественные Таины, которые могут носить и подавать требующим одни священники, были вручены зверолову? Но таков был обычай первенствующей Церкви: лжиц\footnote{Лжица "--- небольшая ложечка с крестом на конце рукоятки, при помощи которой совершается причащение.} тогда не было; приступающие к Тайной Вечери принимали части Тела Христова в руки, а Святую Кровь вкушали из чаши. Отправляющиеся в путь часто брали с собой \textit{Святая Святых}, приготовленные так, как ныне приготовляют у нас запасные Дары, и, достойно испытав совесть свою, в случае нужды приобщались. В 79"~й главе Лимонара повествуется, что благочестивый раб одного еретичествующего купца, взяв Святое Причастие, хранил в ковчежце, но когда узнал о том господин его и хотел сжечь оное, то Тело Христово, хранимое под видом хлеба, мгновенно произрастило колосья. Этот обряд причащения продолжался до святого Иоанна Златоуста и отменен им был по следующей причине: одна женщина, приняв в церкви часть Пречистого Тела Христова, унесла домой и, по какому"=то прямо демонскому суеверию, смесила с чародейственными, по ее мнению, травами. Святитель, узнав о том, заповедал всем Церквам, чтобы Тело Христово подавали лжицей в уста вместе с Божественной Кровию. С того времени такой способ причащения в Греко"=Российской Церкви продолжается неизменно.

\section{Иоанн Милостивый\footnote{Память свт. Иоанна Милостивого (†~620) празднуется 12~(25) ноября.}}

Святой Иоанн, нареченный Милостивым, будучи пятнадцати лет от рождения, видел следующий сон. Девица красивая, богато одеянная и на главе имеющая венец масличный\footnote{\textit{Венец масличный} "--- венок (венец) из листьев оливкового дерева.}, подошла к одру его и, взяв за руку, разбудила юношу. Видя ее уже наяву, удивленный Иоанн спросил: «Кто она и как осмелилась войти к спящему юноше?» "--- «Я старшая дочь Царя великого и любезнейшая между Его дочерями», "--- с веселым взором и тихой улыбкой отвечала девица. Услышав это, Иоанн поклонился ей, а она продолжала: «Если будешь моим другом, я исходатайствую тебе у Царя милость и представлю к Его Престолу, ибо никто пред Ним не имеет столько дерзновения, как я: для меня Он с неба сошел на землю и облекся в плоть человеческую». Сказав это, девица стала невидима.

Удивляясь столь чудному видению и размышляя в себе, Иоанн сказал: «Воистину, милосердие, в подобии девицы, явилось мне; это свидетельствует масличный на главе ее венец; это свидетельствуют и слова ее, что она свела на землю Бога и заставила воплотиться. Итак, должно иметь милосердие к ближним, должно творить милостыню, иначе и от Бога не получишь милости». Рассуждая таким образом, святой Иоанн встал и на рассвете дня пошел в церковь. Увидев на дороге нагого нищего, трясущегося от мороза, Иоанн немедленно снял верхнюю одежду и отдал ему, в юношеском уме рассуждая: «Теперь узнаю, Божественное ли видение или мечту воображения видел я в прошедшую ночь». Но прежде, нежели дошел он до церкви, встретил его незнакомец, одетый в белые ризы, и, подавая ему узел, сказал: «Друг мой, возьми для твоих нужд эти сто златниц». Иоанн сначала принял с радостью; потом, тотчас же раскаявшись, что взял, не будучи нищим, хотел возвратить деньги, но никого пред собой не увидел и, сколько ни искал столь нечаянного благодетеля, найти не мог. Тогда познал, что его видение было откровением Неба, и сказал сам себе: «Перестань же, душа моя, искушать Господа Бога твоего».

С того времени святой Иоанн сделался так милостив к ближним, что не мог без огорчения вкусить пищи, если прежде не успевал сделать кому"=либо благодеяния. Когда же он был поставлен патриархом Александрийской Церкви, то, призвав к себе церковных экономов, поручил им пройти весь город и переписать поименно, как говорил он, «господ его», и, когда у него спросили, кто таковы господа его, он отвечал: «Те самые, которых вы называете убогими и нищими, "--- это господа мои, потому что они "--- настоящие хозяева того, что Небесный Господь даровал мне и над чем уполномочил меня, а я только страж». Воля его немедленно была исполнена, и людей по сердцу Иоаннову найдено семь тысяч пятьсот человек, которым в пропитание установил он дневной оброк из имений церковных.

Но поскольку служители, как водится и у нас, докладывали не обо всех приходящих, то Иоанн поставил для себя законом каждую среду и пятницу, с утра до вечера, сидеть с кем"=нибудь из добродетельных и умных мужей при дверях церковных, чтобы все имеющие до него дело невозбранно могли приходить к нему. Тут святитель Христов всех выслушивал, помогал в нуждах, укрощал распри и свары, судом взаимной совести решал тяжбы, научал обязанностям звания приводимых к нему не так важных нарушителей общественного порядка, примирял супругов, братьев и детей и раздавал неимущим деньги, одежды и прочее. Когда же некто из знатных граждан изъявил пред ним свое удивление, как можно иметь столько терпения, то Иоанн отвечал ему: «Я имею всегда доступ к Господу моему Иисусу Христу, в молитвах беседую с Ним и прошу у Него все, что хочу. Как же могу не дать ближнему моему невозбранного ко мне прихода? Как не позволю объявлять мне их обиды и нужды и просить у меня, чего восхощут? Должно бояться приговора Господня: \textit{в нюже меру мерите, возмерится и вам} (Мф. 7, 2)». Однажды (в бытность у него Софрония Премудрого, бывшего потом патриархом Иерусалимским) случилось, что к Иоанну, препроводившему таким образом весь день на церковном крыльце, никто не пришел, никто не объявил своей нужды, тогда он в скорби и слезах пошел домой. «Ныне смиренный Иоанн не сделал для души своей приобретения, "--- сказал он, "--- ныне Господу ни за один грех свой не принес очистительной жертвы». Услышав это, святой Софроний отвечал ему: «Днесь"=то, владыка, тебе и радоваться должно, поскольку дети твои всем довольны и живут мирно, как Ангелы Господни». Этим только мудрым замечанием святой Иоанн мог успокоиться.

\section{Обязанности матери по отношению к сыну и сына по отношению к матери}

Святой Иоанн Златоуст\footnote{Память свт. Иоанна Златоустого, архиепископа Константинопольского (†~407), празднуется 13~(26) ноября.} еще в отроческих летах намерен был оставить дом родительский и с одним из друзей своих, по имени Василий, проводить жизнь пустынную и уединенную. Как скоро сетующая о вдовстве своем Анфуса (имя его матери) узнала об этом, она взяла Иоанна за руку, увела в спальню и, посадив подле себя на том самом ложе, на котором его родила, облилась слезами. Иоанн сначала изумился и не знал, чему приписать этот поступок, но когда догадался, то и сам не мог удержаться от слез. «Сын мой! "--- начала тогда говорить рыдающая мать. "--- Бог не благоволил, чтобы добродетели отца твоего долее составляли счастье моей жизни. Смерть его, последовавшая вскоре за болезнями, которые претерпела я, рождая тебя, "--- тебя сделала сиротой, а меня вдовицей весьма рано для нашего счастья. Я приняла на себя все труды и горести вдовства, непонятные для тех, которые не изведали оных. Невозможно выразить затруднительного положения молодой женщины, которая едва вышла из дома родительского, не имела времени узнать всех обязанностей хозяйства и, однако же, погрузившись в тоску, должна приняться за новые попечения, по слабости возраста и пола столь для нее тягостные. Необходимо, чтоб она сама дополняла то, что опустят нерадивые слуги, и предохраняла себя от их злости, чтобы защищалась от родственников, которые иногда имеют злые намерения о наследстве, чтобы постоянно терпела обиды от тех, которые почему"=нибудь не любили ее супруга. Однако же все эти злоключения не принудили меня выйти за другого мужа: я пребыла тверда среди этих бурь и напастей и, уповая особенно на милость Божию, вознамерилась претерпеть все невыгоды вдовства. В этих бедствиях единственным для меня утешением было то, чтобы смотреть на тебя непрестанно и в твоем лице видеть живое изображение усопшего моего супруга. Это утешение началось с твоего младенчества, когда язык твой еще не мог выговаривать даже моего имени. И я в глубине сердца своего уверена, что не подала тебе причины обвинять меня в утрате наследия родительского, а это несчастие часто случается с детьми, состоящими под опекой. Я сохранила для тебя в целости это наследие, хотя между тем не пожалела ничего, что нужно было для твоего воспитания; я делала расход из моего добра, которое получила в приданое от моих родителей. Но, любезный сын, я говорю об этом не с целью укорить тебя тем, чем одолжила тебя. За все прошу у тебя только одной милости: не делай меня вторично вдовой; не растравляй раны, которая начала заживать; дождись, по крайней мере, моей смерти: этот день, может быть, недалек. Когда похоронишь меня во гробе отца твоего и прах мой соединишь с его прахом, тогда иди с Богом, куда хочешь: никто тебе не попрепятствует. Но, пока дышу, не отлучайся от меня; не поскучай моим присутствием, не привлеки на себя гнева Божия, причиняя столь чувствительную скорбь матери, которая не заслужила оной. Если я когда"=нибудь помыслю о том, чтобы вовлечь тебя в суетные попечения, или покушусь хоть несколько удалить от Бога, тогда не смотри ни на законы естества, ни на труды, которые употребила я для твоего воспитания, ни на почтение, которым ты, как сын, обязан матери: тогда беги от меня, как от неприятельницы твоего спокойствия, как от врага, который расставляет тебе опасные сети. Но если я делаю все, что зависит от меня, дабы доставить тебе совершенный покой, то, по крайней мере, это должно удержать тебя, когда бесполезны будут все другие убеждения. Сколь ни много у тебя друзей, но поверь мне, любезный сын, что ни один не допустит тебя жить столь свободно, как я. Да и нет из них ни одного, который бы имел столько усердия к истинному счастью твоей жизни». Святой Иоанн не мог противиться столь трогательным убеждениям. Как ни побуждал его Василий удалиться от света, однако же нежный сын никогда не хотел оставить столь чадолюбивую и благоразумную родительницу.

Отцы и матери! Научитесь из этого примера вашим обязанностям в отношении детей: каждое слово Иоанновой матери заключает в себе прекраснейший урок. А вы, любезные дети, почитайте за первый долг утешать старость ваших родителей: вашими желаниями жертвуйте их желаниям, вашими удовольствиями "--- их удовольствиям, вашим покоем "--- их покою.

\section{Ужасное вероломство, Богом наказанное, или чудо святых мучеников Гурия, Самона и Авива\footnote{Святые мученики Гурий и Самон пострадали при Диоклетиане, в 299~-- 306~гг., а св. Авив "--- при Лицинии, в 322~г. Чудотворные их мощи покоятся в Эдесе в одном храме. Память их празднуется 15~(28) ноября.}}

Некогда на Греческую державу учинили нападение варвары, обитавшие близ пределов персидских, которых древние историки называют ефалитами. Греческое правительство отправило для отражения их войско, одна часть которого осталась для защиты Эдеса, где почивали мощи святых мучеников Гурия, Самона и Авива. В этом отряде находился воинский чиновник, родом гот, на то время вступивший в греческую службу; и ему случилось жить в доме добродетельной вдовы, по имени София, которая имела у себя единородную дочь Евфимию, прекрасную, благонравную и богобоязненную. Варвар с первого взгляда возымел преступные мысли против юной девицы и начал употреблять разные средства, чтобы поколебать ее сердце. Но так как благовоспитанная Евфимия не хотела слушать его льстивых речей, то варвар принял другие меры: при случае выхвалял знатность своего рода; говорил, как любят его родители и какая радость по окончании войны ожидает его в объятиях семейства; мимоходом давал знать, что отец и мать непрестанно предлагали ему знатных невест, но сердце его доселе оставалось нечувствительно к приятностям супружеской жизни, и, наконец, все эти ухищрения довершил тем, что начал просить руки Евфиминой. Нежная мать, устрашившись одной мысли, что дочь будет тосковать на чужой стороне и, может быть, вечно с ней не увидится, сделала ему решительный отказ. Но он не хотел оставить своего намерения: то подсылал к Софии своих товарищей, которые выхваляли его род, заслуги и хорошее поведение, то вызывался поддерживать ее состояние (поскольку София была человек небогатый). То притеснял ее через тех, которым от правительства поручено было взыскивать городские повинности, не подавая, впрочем, на себя ни малейшего подозрения, то говорил, что Эдес имеет столько приятностей и столько выгод для жизни, что он, может быть, останется тут навсегда. И наконец, довел Софию до того, что она согласилась выдать за него Евфимию, но с тем условием, чтоб он прежде бракосочетания дал ей клятву пред мощами святых мучеников Гурия, Самона и Авива, что он не женат и что будет жить с Евфимией так, как повелевает закон Божий. Наглый чужестранец согласился на все, немедленно пошел в церковь и, безбоязненно припав к гробницам святых страстотерпцев, воззвал с умилением: «От вас, рабы и друзья Господни, приемлю девицу эту. Вас осмеливаюсь избрать поручителями ее матери и свидетелями, что я прежде Евфимии ни с кем супружеских обязательств не имел и буду наблюдать заповедь моего Спасителя: \textit{Мужие, любите своя жены, якоже и Христос возлюби Церковь и Себе предаде за ню} (Еф. 5, 25). Не оскорблю ее никогда, но буду почитать ее, как госпожу дома, как мою помощницу и подругу. В противном случае взыщите от меня каждый вздох ее и слезу». Могла ли богобоязненная София не поверить клятвам будущего зятя? «Вам по Боге, о святые страстотерпцы, вручаю дочь мою и через ваши руки отдаю этому иностранцу!» "--- воскликнула она и отдала Евфимию в супружество готу. Новобрачные жили мирно, и Евфимия, по"=видимому, была счастлива. Между тем неприятели, будучи всегда храбро отражаемы, оставили осаду Эдеса и, потерпев несколько поражений, удалились в свои степи. Войска получили приказание возвратиться в Царьград, и зять Софии начал собираться к отъезду. Тогда"=то почувствовала она всю горесть одиночества! Неутешно рыдая над своей дочерью, она покушалась даже расторгнуть союз брачный; но \textit{яже Бог сопряже, человек да не разлучает}. Евфимия равномерно оплакивала свою разлуку с нежной родительницей, но любовь супружеская возлагала на нее священнейшую обязанность везде сопутствовать своему супругу. Таким образом, София должна была разлучиться с Евфимией. Супруги до Царьграда путешествовали благополучно. Там гот получил от императора новые отличия, новые награды и предложение остаться навсегда в греческой службе. Но, сколько Евфимия ни просила его, чтобы не разлучал ее, по крайней мере, с отечеством, гот остался непреклонен и отправился на свою сторону. По дороге без всякой причины он дал увольнение своему служителю, и притом, к удивлению Евфимии, с тем условием, чтобы он, под опасением жесточайшего наказания, не появлялся в его отечестве. Но непонятные поступки варвара объяснились вскоре. В одно время сделался он необыкновенно задумчив и к Евфимии свиреп; когда она с робостью начала спрашивать его о причине столь внезапной перемены, злодей схватил саблю и зверски закричал: «Мы скоро будем дома, а я женат! Итак, послушай: если хочешь быть жива, то страшись называться пред кем бы то ни было моей супругой, но отныне носи имя моей пленницы и будь рабой моей жены. Если же дерзнешь вымолвить хотя одно слово о наших действительных отношениях, в ту же минуту этот меч омоется в крови твоей». Как громом пораженная, Евфимия упала без чувств, и варвар равнодушно оставил ее умирать. Но, к несчастию своему, Евфимия очнулась; она поверглась к ногам его, но злодей оттолкнул ее и, повторив те же угрозы, приказал не напоминать ему более, что она когда"=нибудь была его супругой. Обманутой, поруганной Евфимии осталось только втайне оплакивать свою горькую участь и положиться на Единого Бога, Покровителя всех скорбящих. «Боже родителей моих! "--- возводя очи и руки свои к небу, с глубоким воздыханием и горькими слезами восклицала она. "--- Вонми гласу молитвы моея и воззри, сколь бесчеловечно обманул меня и мою родительницу сей клятвопреступник! Господи, избави меня от столь ужасного бедствия молитвами святых Твоих угодников, за Тебя пострадавших! А вы, святые страстотерпцы, помогите мне, несчастной! На вас возлагая надежду, я вверилась чужестранцу; вы за меня будьте и ходатаями».

Оказалось, что гот в самом деле был женат на гордой, своенравной женщине, которая имела знатную родню и, доставив ему богатство, поступала с ним так, как хотела. Преступный муж не только боялся оскорбить родственников своей супруги, но захотел еще сделать ей приятный подарок, отдав ей в рабыни благовоспитанную и наученную всякому рукоделию гречанку. Жестокость, редкая и в варварской стране!

Непрестанно сетуя о своем злополучии, Евфимия, наконец, достигла дома вероломного супруга и была принята, как пленница и раба. Жестокая готянка, заметив красоту Евфимии, заподозрила прежние близкие отношения между нею и своим мужем и стала поступать с нею по"=варварски: поручала ей работы тяжелые и низкие, непрестанно ругала, часто прибегала к побоям. Бедная Евфимия неоднократно намеревалась открыть ей всю истину, но обнаженный некогда пред ней меч и незнание языка готского всегда связывали язык ее. К большему несчастию, она разрешилась от бремени. Готянка осыпала ругательствами своего мужа; но варвар, который клятву ценил столь же мало, как и честь свою, свидетельствуясь Богом, что не знает за собой никакого порока, дал ей полную власть над жизнью и смертью матери и сына. Тогда злобная женщина положила в уме своем удовлетворить мщению и однажды, когда Евфимия занималась обыкновенною работой, отравила невинного младенца. Кто опишет ужас и жалость матери, когда она, подойдя к колыбели, увидела сына своего мертвым, особенно когда приметила, что на устах его пенится что"=то подобное яду?!

До этого времени Евфимия была невинной жертвой обмана и клятвопреступления. О, если бы она таковой осталась и с полным упованием на Промысл Божий ожидала лучшей участи! Но человек вообще слаб. Евфимия, потеряв терпение, не захотела платить своим мучителям одними слезами и сделалась почти столь же виновной, сколько была несчастна. В отчаянии, забыв самое себя, она обтерла сыновние уста волною\footnote{\textit{Вулна (волна)} "--- шерсть (обыкновенно овечья).} и, спрятав ее, решилась увериться в истине своего подозрения опытом над тою, которой приписывала смерть своего сына. На другой же день открылся удобный случай: Евфимия, прислуживая за столом, успела омочить напитанную ядом волну и выжать в питье, которое должно было подносить госпоже ее. Таким образом, преступление было наказано преступлением, и готянка в ту же ночь внезапно умерла.

Соболезнуя об участи Евфимии как обманутой девицы, отвергнутой супруги, осиротевшей матери, благовоспитанной женщины, угнетенной несносным рабством, "--- чувствительные сердца желали бы простить ее поступок с госпожой. Но у Бога нет лицеприятия! Сколь бы велики ни были наши несчастия, они не могут оправдать преступления. Достойная жалости, Евфимия в этом случае представляет нам разительный пример: ибо Господь наказал ее кратковременно, но ужасно, и если не до конца прогневался, то потому только, что ходатаями ее были сердечное раскаяние и святые мученики Гурий, Самон и Авив. Но обратимся к повести.

Услышав о смерти госпожи, Евфимия поражена была смертным ужасом, беспрестанно обливалась слезами и в наказание просила себе у Бога смерти. Между тем прошло семь дней после погребения готянки "--- и над Евфимией начал совершаться Суд Божий. Сродники и друзья умершей, собравшись вместе, начали рассуждать о ее внезапной кончине и обратили мысли свои на Евфимию; все утверждали, что она умертвила госпожу свою, и несчастную обрекли на смерть. По варварскому закону, который давал господам полное право над жизнью рабов, они приговорили погребсти ее живую в одном гробе с умершей. Безжалостный супруг первый возложил на нее руки. Злодеи в глухую полночь открыли склеп, бросили туда бедную жертву и, привалив камень к дверям гроба, поставили стражу. Полумертвая Евфимия, повергшись на землю, не могла произнести ни слова; у нее недоставало голоса даже для стона и рыдания. Она могла только устремить мысли свои к Богу и привести на память святых мучеников Гурия, Самона и Авива и совершенно лишилась чувств.

Но нет столь мрачной и глубокой бездны, где бы уповающему на Бога не было спасения. Вдруг Евфимия видит пред собой страстотерпцев, поручителей ее жизни и здравия. «\textit{Радуйся, дщи, и познай, где еси ныне!} "--- вещают они. "--- \textit{Приспело твое спасение; иди с миром к матери твоей}». После этого угодники Господни стали невидимы, а Евфимия, как будто пробуждаясь от сна, слышит церковное пение; с изумлением видит стены храма, ей известного; узнает знакомые лица молящихся и уверяется, что она в Эдесе, в храме, и подле гробницы святых мучеников Гурия, Самона и Авива. От неизреченной радости она воспрянула и, припав к святой гробнице, воскликнула: \textit{Бог наш и на небеси и на земли, вся елика восхоте, сотвори} (Пс. 113, 11). \textit{Посла с небесе и спасе мя} (Пс. 56, 4): благословен еси, Господи, \textit{спасаяй уповающих на Тя} (ср. Пс. 16, 7). \textit{Вечер водворится плач, и заутра радость} (Пс. 29, 6). Все находившиеся в церкви, внезапно увидев подле мученической гробницы женщину и услышав вопль ее, с удивлением обступили Евфимию. Священник спросил: кто она? Но, едва начала она рассказывать ужасную и чудную о себе повесть, все узнали ее. Послали за Софией и, не объявляя зачем, приказали немедленно прийти в церковь. Какая трогательная встреча! Обрадованная Евфимия поверглась в объятия матери. Изумленная София остановилась, внезапно увидев дочь свою одетой в рабские рубища. Обнявшись, обе они рыдали и не могли промолвить ни единого слова. Потом Евфимия рассказала обо всех злодеяниях варвара, обо всех несчастиях, с ней случившихся, и о внезапной помощи святых мучеников Гурия, Самона и Авива. Сердце нежной матери растаяло от жалости, и все слушавшие, удивляясь, прославляли всемогущую Божию силу и помощь. Припавши к гробнице святых мучеников, София и Евфимия пребыли тут в молитвах до самого вечера. Вся Греция узнала о славном чуде, и все восхвалили имя Господне.

Но Бог не оставил без наказания и клятвопреступного гота. Чрез некоторое время греческому воинству опять случилось идти чрез Эдес, и варвар, привыкший к грабежам, неразлучным с войной, находился тут же. Не сомневаясь, что Евфимия погибла лютою смертью, он без всякого стыда посетил свою тещу. София, увидев его, скрыла Евфимию, приняла гостя с притворной ласковостью и начала спрашивать, здорова ли дочь ее. «Молитвами твоими живет благополучно, "--- отвечал варвар, "--- родила сына и чрез меня целует тебя. Если бы я, будучи дома, знал, что придется быть в Эдесе, то непременно взял бы ее с собой, но поход назначен был в другую сторону, и уже в дороге получили приказание идти к Эдесу». София будто поверила словам его, пригласила родственников и приятелей и вывела пред ним Евфимию. «Злодей! "--- сказала она. "--- Знаешь ли эту отроковицу? Знаешь ли, кто она? Знаешь ли, где заключил ты и какой смерти предал ее?» Затрепетал преступный гот и сделался безгласен, как мертвый, а раздраженные эдесяне схватили его, заключили в темную комнату и обо всем возвестили блаженному епископу Евлогию. Не нужно было новыми доказательствами удостоверяться в его злодеяниях: чудо святых страстотерпцев было в свежей у всех памяти, да и варвар признался немедленно. Преосвященный Евлогий обо всем донес воеводе, начальствовавшему над греческим войском. Немедленно наряжен был суд, и гот, как нарушитель обязанностей супружеских, губитель невинности, клятвопреступник и убийца, приговорен к смерти. Сколько боголюбивый епископ и София вместе с дочерью, единственно из христианского милосердия, ни просили военачальника, чтобы пощадил жизнь его, но тщетно. «Боюсь прогневать святых страстотерпцев, если помилую столь ужасного злодея», "--- сказал воевода и приказал отрубить голову клятвопреступнику.

\section{Наказанное кощунство}

Однажды, когда святитель Христов Григорий Неокесарийский\footnote{Свт. Григорий Чудотворец, епископ Неокесарийский (†~ок. 266~-- 270), жил в III~в. Сколько был он ревностен в проповедании Слова Господня, явствует из того, что он, сделавшись епископом, в пастве своей застал христиан только 17~человек, а умирая, оставил идолопоклонников только 17~человек. Память его празднуется 17~(30) ноября.} возвращался из Коман, где был для избрания и посвящения епископа\footnote{Тогда избран был и посвящен в епископа священномученик Александр (†~III~в.), память которого празднуется 12~(25) августа.}, трое евреев захотели посмеяться над ним: на пути, где должно было проезжать человеку Божию, положили одного из своих товарищей, якобы теперь только умершего, а сами, сидя над ним, горевали. Едва святой Григорий повстречался с ними, евреи начали молить его, да покажет милость над усопшим, да покроет тело его и даст что"=нибудь на погребение. Святой Григорий снял с себя верхнюю одежду и, прибавив денег, отдал просившим и поехал далее. Евреи начали смеяться над чудотворцем Божиим. «Если бы имел в себе Духа Господня, "--- сказали они, "--- то знал бы, что лежит пред ним живой человек». Потом кликнули своего товарища, но он не встал. Думая, что их спутник уснул, начали толкать его, но он не отвечал, ибо уснул вечным сном. Так Господь отмстил им поругание над святым человеком и смех их обратил в плач! Кощунники, не думая о том за минуту, должны были приготовляться к погребению мертвеца своего.

\section{Христианский Сцевола\footnote{Римлянин Сцевола, будучи захвачен в неприятельском стане и представлен к царю Порсене, протянул правую руку свою над пылающим жертвенником и, стоя таким образом, разговаривал с ним.}}

Святой мученик Варлаам\footnote{Св. мученик Варлаам (†~304), уроженец Антиохии Сирийской, жил в царствование Диоклетиана. Память его празднуется 19~ноября (2~декабря).} уже в глубокой старости предан был суду и претерпел много мучений за исповедание имени Христова. Наконец, нечестивые, принуждая его к идолопоклонству, привели в капище Аполлона и, повелев протянуть руку над пылающим жертвенником, положили на оную горящие угли с ливаном (ладаном) и смирной для того, чтобы он от нестерпимой боли вынужден был отрясти эту жертву на идольский жертвенник. Если бы он это сделал, они хотели восплескать руками пред ним и с хохотом сказать: «Уже принес жертву богам нашим!» Но злочестивые не могли восторжествовать над твердостью мученика. Старец, держа в руке своей горящий огонь, явился тверже металла и, не тронувшись с места, не испустив ни единого вздоха, стоял, доколе отгорели персты его и с углями упали на жертвенник. Он произносил только тихим голосом: «\textit{Благословен Господь Бог мой, научаяй руце мои на ополчение, персты моя на брань} (Пс. 143, 1)». Так мужествен и непобедим оказался крепкий страдалец и воин Христов Варлаам! Имея десницу свою вместо алтаря Господня, как говорит святой Василий Великий\footnote{В слове о св. мученике Варлааме.}, он принес себя в жертву всесожжения Господу и в руки Его предал святую душу свою.

\section{Жизнь по правилам}

Великий в постниках авва Диоскор\footnote{Имени прп. Диоскора в греко"=российских святцах не обретается, но сколько есть сокровенных избранников Божиих, даже имена которых неизвестны! Если бы всех можно было счесть, то число их равнялось бы числу звезд небесных.} имел обыкновение каждый год назначать себе какой"=нибудь трудный подвиг, усовершающий инока. Например, он давал обет Богу: «Я весь год не покажусь в общежитии, чтобы беспрепятственно заниматься богомыслием; я налагаю на целый год молчание на уста мои; я весь год не буду вкушать плодов и елея» и тому подобное "--- и никогда не нарушал своего обета. Таким образом праведник поступал в продолжение всей жизни: совершив один подвиг, начинал другой.

Христиане! Жизнь преподобного Диоскора, без сомнения, была украшена множеством великих добродетелей, ибо он избрал лучший способ сделаться совершенным подвижником Христовым. Мы, напротив того, если делаем добро, то по большей части по случаю, когда дунет благоприятный ветер. Может ли быть оно прочно? Что же касается удовлетворения своих страстей, мы не только за год, но иногда и за десять лет предполагаем насытить их.

\section{Слезы святых}

Однажды преподобный Диоскор, сидя в келье, плакал. Увидев это, ученик старца спросил у него: «О чем ты плачешь, авва?» "--- «Оплакиваю грехи мои», "--- отвечал Диоскор. «Ты не знаешь за собой никаких грехов», "--- возразил ученик. «Ах, сын мой! "--- сказал старец. "--- Если бы я, к несчастию, дошел до того, чтобы мог видеть грехи мои, то мало было бы трех или четырех для меня помощников, чтобы достойно оплакать оные!»

Как справедливы слова преподобного старца! Грехи наши подобны змиям, которые редко показываются на распутье, но всегда почти скрываются под мягкой травой, под душистыми цветами. Тем более осторожности должно иметь страннику жизни этой!

\section{Страдания святого Михаила, великого князя Тверского\footnote{Память благоверного князя Михаила Тверского (†~1318) празднуется 22~ноября (5~декабря).}}

Святой Михаил, великий князь Тверской, по злобным проискам Георгия Даниловича, князя Московского, домогавшегося получить великокняжеский престол, вызван был ханом Узбеком в Орду и отдан под суд за то, будто бы не хочет повиноваться татарской власти и, собирая с подданных дань для хана, намерен бежать с ней в немецкие земли. А что важнее всего, будто бы отравил сестру ханскую, бывшую супругу Георгиеву, которая взята была в плен и там скончалась. Клевету Георгия всех более поддерживал некто Кавгадый, вельможа татарский, которого побуждало к тому мщение\footnote{Этот татарский вельможа, находясь в войске Георгия, имевшего тогда войну с Михаилом, взят был в плен, равно как и супруга Георгиева. Хотя князь немедленно освободил его, даже одарил щедро, однако не утолил тем мстительности Кавгадыя за причиненное будто бы ему бесчестье.}.

Сколько великий князь ни старался доказать свою невинность и беспрекословное повиновение победителям, но властолюбие и мстительность взяли верх, и татарские вельможи не только не хотели внимать силе убеждений, но, едва удостоив выслушать оные, наконец, объявили, что князь должен быть казнен. Здесь правосудие хана достойно внимания. Не довольствуясь первым судом, он повелел своим вельможам вторично рассмотреть поступки русского князя. Но Кавгадый, поджигаемый злобой и Георгием, будто бы храня честь прежнего суда и славу государя, уговорил Узбека, чтобы Михаил предстал на суд уже в оковах, как виновный\footnote{У ордынцев в таком случае подсудимых связывали веревкой.}.

Измена, оскорбление верховной Ордынской власти в лице Кавгадыя, смерть сестры ханской, по мнению татар и Георгия, не требовали подтверждения. Хотя невинный клялся Богом и своею совестью, что измена и злоумышление на жизнь княгини никогда и на ум не приходили ему, а оскорбление Кавгадыя было неумышленное, вполне возможное на войне и немедленно поправленное. Но чем более справедливость обнаруживается перед теми, которые опровергают ее, тем более приходят они в озлобление. Вельможи татарские, приведенные в стыд силой оправданий, решительно сказали, что князь по всем законам достоин смерти; Кавгадый доложил о том хану. Властолюбивый Георгий возрадовался адской радостью.

На другой день после этого хан со всей ордой пошел на звериную ловлю к реке Терек\footnote{Походы ханов татарских на звериную охоту были движением всей орды; за ними следовали не только двор и судилища, но также иностранные купцы. Так ныне китайские императоры иногда выезжают на охоту в степи Великой Татарии.}, и этот знаменитый невольник, имея оковы на руках и ногах и цепь на шее, должен был идти за ханом. Двадцать пять дней находился он в этом мучительном путешествии, после чего надменный и грубый Кавгадый представил его на позор народа, иностранцев и данников и, обругав поноснейшим образом, по обычаю татарскому приказал перед смертью облегчить ему участь: ему позволено было видеться с сыном, который находился тут же, с духовником и с верными служителями. Все это время князь"~мученик употребил на приготовление себя к смерти: читал Божественные книги, непрестанно молился, еженедельно приобщался Святых Таин и рассуждал с духовным отцом о вечности.

Между тем Узбек находился с ордой близ Дербента, у так называемой «Темировой богатырской могилы». Здесь должна была решиться участь Михаила. В один день, когда великий князь преподавал сыну своему наставления в христианских и царственных добродетелях, вдруг вбегает в его комнату молодой служитель, бледность и трепет которого довольно и без слов показывали, какую весть принес он. Немедленно явились убийцы, разогнали всех служителей, насильно вывели рыдающего Константина (имя сына Михаила) и, по варварскому обычаю, прилагая мучения к смерти, начали бить его. Потом, взявши за цепь, которая была у него на шее, повесили его на стену, но стена, не сдержав тяжести, обрушилась; великий страдалец упал, но, еще сохраняя немного силы, вскочил на ноги. Варвары, вторично ударив его о землю, начали тиранить "--- и это продолжалось до тех пор, пока некто, по имени Романец, вероятно слуга Георгиев, не пронзил его ножом в бок и, поворачивая нож в ране, прекратил мучение и жизнь святого Михаила. Тогда убийцы, разграбив его имущество, пошли на торжище, где Кавгадый и Георгий, сидя на конях, ожидали, увы, столь приятного для них известия.

Лютейший кровожадного тигра родственник едет с Кавгадыем и равнодушно смотрит на тело убиенного князя, поверженное на землю, нагое, облитое кровью, между тем как это зрелище и в самом суровом сердце татарина произвело некоторую чувствительность: видимо, преступление потрясло его душу! Объятый ужасом, с яростью воззрел он на Георгия и закричал: «Как можешь ты с таким хладнокровием смотреть на обнаженное тело твоего брата? Или еще не насытился его кровью?» Преступник затрепетал; из такого сурового вопроса он ясно почувствовал, чего должно ожидать ему от злобного татарина, и, не ответствуя, велел покрыть труп мученика. Потом, показывая притворную печаль, испросил его у татар, чтобы отвезти для погребения на Русь. Так пострадал и скончался великий князь Михаил.

Христиане! Оплачьте смертоносные следствия властолюбия и берегитесь, чтобы этот идол даже издали не дохнул на сердца ваши: это неисцельная зараза! Сколько история ни представляет примеров оного, везде следы его облиты кровью ближних.

\section{Внезапное покаяние преступника\footnote{Из Пролога, в 25"~й день ноября.}}

Преподобный Пафнутий, как совершенный постник, никогда почти не пил вина. Но если апостол Христов \textit{всем вся бысть, да большия приобрящет}, то святой Пафнутий, по крайней мере, в следующем случае поступил так же. В некотором путешествии встретился он с шайкой разбойников, которые пили вино. Разбойнический атаман знал преподобного Пафнутия и знал его постничество, но, захотев над ним посмеяться, поднес чашу вина и, потрясая мечом, сказал ему: «Пей или умри!» Святой старец не боялся смерти, но, желая приобрести Господу овча погибшее, повиновался и выпил.

Тогда разбойник поклонился ему. «Прости меня, раб Божий! "--- сказал он. "--- Я жестоко оскорбил тебя!» "--- «Уповаю на Господа, "--- отвечал Пафнутий, "--- что Он за эту чашу покажет над тобой Свое милосердие в этом и в будущем веке». "--- «А я, "--- воскликнул тогда начальник грабителей, "--- клянусь Господом, что от этого времени никому не сделаю зла», "--- и сдержал свое обещание. Таким образом, святой старец обратил на путь спасения всю шайку разбойников.

\section{Цель истинного просвещения\footnote{Из жития прп. Иоанна Дамаскина (†~ок. 780), память которого празднуется 4~(17) декабря.}}

Варвары, жившие в Дамаске, выходя часто морем и сушей в окрестные страны, брали в плен христиан и торговали ими, как обыкновенным товаром. Однажды случилось им захватить черноризца, родом из Италии, по имени Косма. Когда по обыкновению вывели его на продажу, инок горько плакал. На тот раз шел мимо торжища один христианин, человек благочестивый, богатый и благородный, который, имея по своему характеру и знаменитости большое влияние на умы своих сограждан, был уважаем и от варварского правительства. Увидев черноризца, проливающего слезы, он подошел и, желая утешить его в скорби, сказал: «Зачем ты, человек Божий, плачешь о потере мира сего, для которого ты, как видно по одеянию, давно умер?» "--- «Ах! Я плачу не о мире, "--- отвечал старец, "--- ибо знаю, что есть другой, лучший мир, бессмертный и вечный, уготованный рабам Христовым, и надеюсь благодатью Господней достигнуть оного, но рыдаю о том, что отхожу от мира этого бесчадным, не оставив по себе наследника». "--- «Но ты, если не обманываюсь, инок, обрекшийся пред Богом хранить чистоту, а не детей рождать! "--- возразил христианин. "--- Как же можно сетовать о бесчадии?» "--- «Государь мой! "--- сказал Косма. "--- Не разумеешь, о чем говорю. Я плачу не о плотском наследнике, но о духовном. Хотя я и убогий инок, как видишь, но старался научиться всему, что по это время мог изобрести ум человеческий. Красноречие и поэзия, любомудрие и богословие, равно и прочие науки были всегдашними предметами моего занятия; учась всему, я имел единственной целью познание моего Создателя. К несчастию моему, имея такие дары, я никому не был полезен; чему сам научился, никого тому не научил и "--- ах! "--- уже не могу научить, ибо приспело время тяжкого моего рабства между варварами, ненавидящими всякое просвещение. Итак, я должен явиться пред Господом моим, как древо неплодное, как раб, скрывший талант господина своего. Об этом плачу, об этом рыдаю! Родители по плоти скорбят, когда не рождают чад: не должен ли и я скорбеть, что не родил духовного сына и наследника дарований Господних?»

Выслушав это, дамасский гражданин перекрестился и сказал: «Слава Тебе, Господи, что обрел желаемое сокровище! Не скорби, добродетельный старец! "--- обратившись к иноку, продолжал он. "--- Господь поможет тебе исполнить желание твоего сердца». После этого пошел он прямо к правителю Дамаска, испросил свободу преподобному Косме и, приведя его в дом свой, представил ему двух отроков. «Вот два воспитанника, "--- сказал он, "--- которым ты можешь даровать духовную жизнь и оставить по себе наследниками премудрости: один "--- родной мне сын, Иоанн, а другой "--- усыновленный мною сирота и тебе соименитый, Косма. Итак, умоляю тебя, святой старец, научи их твоей премудрости и благим нравам, сделай их твоими сыновьями и оставь их по себе наследниками некрадомого богатства духовного». Блаженный инок обрадовался несказанно и, прославив Бога, начал заниматься их просвещением.

Эти"=то два отрока были Иоанн, после нареченный Дамаскиным, и Косма Святоградец, бывший после епископом Маиумским\footnote{Память прп. Космы, епископа Маиумского, творца канонов (†~ок. 787), празднуется 12~(25) октября.}. Оба имели высокие дарования, были остроумны и успехами своими приносили великую радость сердцу преподобного Космы. Иоанн, как орел, по воздуху парящий, постигал высочайшие таинства богословия, а духовный брат его Косма, хотя имел разум медленнее Иоаннова, но обдумывал предметы всегда совершенно. Оба были добродетельны, кротки и совершенны в мудрости, как духовной, так и светской. Особенно Иоанн оказал столь великие успехи, что сам учитель смотрел на него с удивлением: он был великий песнопевец и великий богослов, а что наипаче украшало его "--- не гордился своею мудростью. Как благоплодное дерево чем более приносит плодов, тем ниже приклоняется к земле своими ветвями, так любомудрый Иоанн чем более возрастал в премудрости, тем менее думал о себе и укрощал мысли, свойственные юным умам.

Наконец, преподобный Косма сказал отцу Иоаннову: «Общее желание наше исполнилось: твои дети по плоти, а мои по духу научились всему и уже превосходят меня самого премудростью, ибо Господь чудесным образом умножил их дарования; теперь не нуждаются они в моих уроках, будучи сами способнее меня наставлять других. Итак, умоляю тебя, отпусти меня в монастырь, да там буду сам учеником и научусь от совершенных иноков высшей премудрости; ибо мирская философия предпосылает каждого человека, совершенно ею просвещенного, к философии духовной, которая чище и бесценнее оной и спасает душу».

Долго родитель Иоанна не отпускал и уговаривал старца, чтобы не лишал дом его славы иметь столь великого наставника, но, опасаясь оскорбить его, исполнил желание. Тогда истинно просвещенный муж, великий учитель, ушел в обитель святого Саввы, там пребыл до блаженной кончины и перешел к совершенной премудрости "--- Богу.

\section{Встреча императора с архипастырем}

Святитель Христов Амвросий Медиоланский насколько был строг к императору Феодосию Великому, когда тот, по внушению хитрых царедворцев, поступал против священного сана царей и забывал кротость веры Евангельской, настолько же благоговел пред его добродетелями. С другой стороны, и христолюбивый самодержец все успехи и свою славу приписывал молитвам архиерея Божия, а свои горести считал не чем иным, как следствием преслушания наставлений его. Сколько они взаимно друг друга уважали, явствует из их встречи.

Феодосий Великий, победив тирана Евгения, почел первым долгом уведомить об этом Амвросия. Великий архипастырь, получив письмо великого царя, принес от его имени священную жертву, писание его положил на жертвенник, представляя оное Богу как залог веры императора, и после того немедленно сам отправился для посещения его в Аквилею. Встреча двух великих особ была исполнена радости и любви. Епископ пал к ногам государя, благочестие которого и явное покровительство Божие сделали достойным почтения более, нежели все его победы и лавровые венцы, и молил Бога, дабы даровал ему блаженство на Небе, как даровал счастье на земле. Император со своей стороны пал к стопам епископа, приписывая его молитвам милость, полученную им от Бога, и прося молитвы о его спасении. Потом начали они разговор о средствах совершенно уничтожить арианство и утвердить на незыблемом основании Православие.

\section{Истина, просто излагаемая, в устах праведника сильнее и убедительнее всякого красноречия}

На Первом Вселенском Соборе, созванном против злочестивого Ария, между святыми отцами находился святитель Спиридон Тримифунтский\footnote{Этот великий архипастырь, в молодые годы бывший пастырем овец, жил в царствование Константина Великого и сына его, Констанция. Память святителя Спиридона Тримифунтского (†~ок. 348) празднуется 12~(25) декабря.}. Последователи Ария, не надеясь своим лжеумствованием преодолеть истину, утверждаемую таким множеством богоносных пастырей, под разными предлогами испросили у Константина Великого позволение присутствовать на Соборе и еллинским философам, из которых один, наиболее искусный в словопрении, с гордостью и бесстыдством издевался над учением православных. Преподобный Спиридон, будучи прост и неучен и зная только, что Христос есть Сын Божий и распят за нас, грешных, пожелал выступить против философа. Хотя святые отцы, стараясь общими силами низложить столь ловкого врага, возбраняли святому Спиридону принимать на себя это трудное дело, но человек Божий, уповая на дары Духа Святого, дерзновенно приступил к философу и с простосердечием сказал: «Во имя Иисуса Христа послушай меня. Един есть Бог, Который сотворил небо и землю, создал из брения человека и Словом и Духом Своим устроил все видимое и невидимое. Это Слово и есть Сын Божий и Бог, милосердовавший о нашем заблуждении, родившийся от Девы, за нас пострадавший, воскресший в третий день и нас совоскресивший. Этот Бог некогда придет судить нас и воздаст каждому по делам его. Он Отцу единоестествен, сопрестолен, равночестен. Вот чему веруем мы без всякого испытания. И ты не дерзай испытывать, ибо таинства превышают твой разум и всякую человеческую мудрость». После этого, несколько помолчав, спросил у него самым невинным образом: «Не так ли думаешь и ты, самый искусный из всех мудрецов мира?» Хитрословесный эллин, как бы совершенно не зная философских наук, не отвечал ни слова; какая"=то Божественная сила действовала против него. Наконец, собравшись с духом, сказал: «Верю и я, что уста твои вещают истину». "--- «Итак, иди и приими знамение святыя веры», "--- возразил преподобный Спиридон. Тогда философ, обратившись к арианам и своим ученикам, сказал: «Доколе препирались со мною словами, словам противополагал я слова и искусством красноречия опровергал мысли противников. Но, когда вместо слов некоторая сила вышла из уст этого старца, я признаю слабость мою, ибо слова против этой силы суть ничто, так же как и человек против Бога есть также ничто. Если кто из вас думает одинаково со мною, да верует во Христа и да верует так, как учит этот мудрый старец, устами которого глаголет Сам Бог». Вскоре после того философ принял христианскую веру и крестился в Церкви православных. Это происшествие весьма унизило гордость ариан.

\section{Наглядное подобие Пресвятой Троицы}

Преподобный Спиридон, доказывая арианам на том же Соборе единство Пресвятой Троицы, взял в руку плинфу (кирпич) и стиснул. Мгновенно вышел из нее огонь к небу, вода потекла вниз, а брение осталось в руке святителя. «Се три стихии, а одна плинфа, "--- сказал преподобный Спиридон, "--- так и в Пресвятой Троице три лица, а Един Бог».

\section{Чудотворное превращение золота в змею}

В архипастырство преподобного Спиридона Тримифунтского однажды на острове Кипре свирепствовал чрезвычайный голод. Все бедные люди стекались к святому епископу просить или хлеба, или денег, или заступления у богачей. Вместе с другими пришел к нему знакомый издавна земледелец и, прося что"=нибудь взаймы для покупки жита, слезно жаловался на немилосердного богача, который, повсюду забирая хлеб в свои руки, продает только на золото и по весьма высокой цене. Святитель, утешив и обнадежив земледельца, отпустил домой, а на другой день пришел к нему сам и принес большой слиток золота. «Отнеси это к корыстолюбивому хлеботорговцу в залог и бери хлеба, сколько тебе будет потребно, "--- сказал он. "--- Когда Господь даст обильную жатву, выкупишь оное». Земледелец удивился, где преподобный безсребреник взял столько золота, и, поступив по его приказанию, не только был сыт и доволен, но и другим помогал. Скупец сделался для него так щедр, что часто сам напоминал, не надобно ли ему хлеба.

Наконец голодное время миновало, благословенная жатва утешила сетующий народ, и земледелец, к неудовольствию ростовщика, выкупив залог, возвратил преподобному Спиридону. Чудотворец, приняв золото, повел земледельца в сад и, положив оное у забора, возвел очи свои к небу и сказал: «Господи, Боже мой, претворивший некогда в змия жезл Моисеев! Претвори опять это сокровище в то же пресмыкающееся животное, из которого соделал его». Мгновенно слиток зашевелился и превратился в змия, который, извиваясь, пополз и скрылся в нору. Изумленный земледелец трепетал от ужаса и, повергшись к стопам святого старца, называл себя недостойным столь чудесного благодеяния. А преподобный Спиридон, поднимая его, сказал: «Бог долготерпелив и многомилостив. Вот почему это золото не превратилось в змия в ту самую минуту, когда жестокосердый хлебопродавец, с жадностью приняв оное, положил за пазуху. Но будь уверен, что этот змий первый начнет грызть сердце каждого безжалостного богача там, где вечный вопль и скрежет зубов будет жребием грешников».

Если, по несчастию, и между нами есть такие же корыстолюбцы и ростовщики, то пусть прочитают эту повесть как можно внимательнее.

\section{Явление Царя Славы\footnote{Из жития свт. Григория, архиепископа Омиритского (†~ок. 552), память которого празднуется 19~декабря (1~января).}}

Под державой Авраамия\footnote{Этот государь возведен на престол по смерти злочестивого Дунаана, убийцы св. Арефы.}, благочестивого царя Омиритского, иудеи, будучи принуждаемы принять христианскую веру, просили у него позволения иметь с христианами прение: который Закон справедлив и Божествен "--- Ветхий или Новый? Авраамий воспылал гневом, но по совету святого Григория, архиепископа Омиритского, согласился удовлетворить их желание. Вследствие этого были избраны с православной стороны архиерей Божий, а от еврейского сонмища "--- некто Ерван, искуснейший законоучитель, подкрепляемый притом множеством раввинов.

В присутствии Авраамия, духовенства, царского синклита и народа на обширном поле продолжалось состязание их несколько дней. Сколько Ерван ни был хитер в оборотах, сколь ни искусно употреблял в свою пользу каждое место из Ветхого Завета, но всегда был побеждаем святым архиепископом. Рассуждали о трех Лицах Божества, о Приснодеве Марии, о воплощении от Духа Святого и о тридневном Воскресении Иисуса, словом "--- о всех таинствах и обрядах христианской веры, но Ерван каждый раз оставался в посрамлении. Христиане радовались, иудеи унывали.

Уже наступил последний день прения о вере. Ерван колебался и не находил слов в оправдание упорства своих единоверцев. Но, поскольку должно было на что"=нибудь решиться, то, наконец, воззвал он: «Для чего столь долго утруждать себя рассуждениями и спорами? Я имею надежнейшее средство кончить общее дело. Учитель христианский! Если хочешь, чтобы я и все единоверцы мои без всякого притворства веровали твоему Иисусу, то покажи Его как живого; соделай, да увижу Его и возглаголю с Ним "--- тогда только можем признаться, что христиане победили нас». Услышав это, иудейское сонмище вострепетало; все начали умолять Ервана, чтобы в чем"=нибудь не обманулся, а святой архиепископ сказал ему: «О Ерван! Свыше сил человеческих твое требование, но жив Господь! Для уверения вашего и для утверждения христиан Он силен и это сотворить. Итак, скажи решительно, чего желаешь от меня?» "--- «Умоли Владыку твоего, да снидет сюда и явится пред нами!» "--- отвечал Ерван. «Да явится чувственно, "--- голосом боязни повторило сонмище иудейское, "--- тогда только примем крещение».

Святой архиепископ знал, сколь важный подвиг предстоит ему, но знал и то, что \textit{аще имамы веру, яко зерно горушно, речем горе сей: прейди отсюду тамо, и прейдет: и ничтоже невозможно будет нам} (Мф. 17, 20). Все упование свое возложив на Бога, он восстал с престола своего и несколько отступил от собрания. Царь и все христиане удивились его великому дерзновению к Богу. Смиренно и с умиленной душой преклонив колена, долго молился угодник Божий; вспоминал вслух народа все таинства воплощения Бога Слова, Его житие, страсти, восстание от гроба и славное вознесение, наконец, воскликнул: «Явися, о Владыка, чувственно этим окамененным и ослепленным злобой людям, явися ради славы имени Твоего, да узрят телесными очами животворное человечество, в которое для нас Ты, Сладчайший Иисусе, облекся и с которым вознесся на Небо, да, узрев Тебя, веруют в Тебя, Единого Истинного Бога». Едва святой архиепископ совершил молитву свою, вдруг от востока поднялась буря, и покатился столь ужасный гром, что вся земля поколебалась. Весь народ от страха повергся на землю, но, когда один после другого, как бы пробуждаясь, встали и посмотрели на восток "--- отверзлось небо, и облако, испускающее пламень и лучи солнечные, нисходило на землю, а посреди его узрели Мужа, краснейшего паче сынов человеческих: Господь наш Иисус Христос, неизреченно сияя лицом, в одеждах молниезрачных, ступая по облакам, приближался книзу и стал на воздухе против архиепископа. Небесная красота привлекала сердца и взоры как христиан, так и иудеев, но от страха славы Его и величества, как некогда на Фаворе, все опять поверглись ниц, царь и народ. Иудеи, пораженные трепетом, устремились в бегство, но колена их подогнулись; Божественное блистание опаляло их, невидимая сила отнимала у них движение. Тогда святой архиепископ, укрепившись и собрав мысли свои, вперенные в Небесную славу, громогласно воззвал к Ервану: «Виждь и веруй, яко \textit{един свят, един Господь Иисус Христос во славу Бога Отца, аминь!»} Ерван стоял как мертвый и не мог выговорить ни слова. Тогда от велелепной славы Господней услышан был глас: «\textit{Ради молитв архипастыря, исцелит вас Распятый отцами вашими}», "--- от чего иудеи более прежнего вострепетали и пали на землю, имели открытые глаза, но утратили зрение.

Между тем пред лицом Господним снова начался шум; блистающее облако, поднимаясь кверху, начало мало"=помалу свиваться, наконец, Господь стал невидим. Христиане долго вопияли вслед Его: «Господи, помилуй!» Святой архиепископ, приникнув лицом к земле, воссылал к Богу слезные молитвы о людях своих. А иудеи, рыдая об утрате зрения, припадали к ногам его, да подаст им исцеление по слову Христа Спасителя. Они поклялись, что от чистого сердца приемлют веру Евангельскую, и в тот же день все, от Ервана до последнего младенца, крестились водой и Духом. Эллины последовали их примеру. Вся Омиритская страна просветилась, о чем возрадовались не только царь и святой архиепископ, но и Ангелы Небесные.

\section{Богоносец}

Священномученик Игнатий, нарицаемый Богоносцем\footnote{Память священномученика Игнатия Богоносца (†~107) празднуется 20~декабря (2~января).}, был епископ Антиохийской Церкви. Повествуют, что этот архиерей Бога Вышнего в младенчестве своем был то благословенное дитя, которого Спаситель мира, поучая народы о Царствии Небесном, восприяв на руки, сказал: \textit{Аще не обратитеся и будете яко дети, не внидете в Царство Небесное. Иже аще едино таковых отрочат приимет во имя Мое, Мене приемлет} (Мф. 18, 2, 3, 5; Мк. 9, 36~-- 37). С того времени святой Игнатий, как носимый руками воплощенного Бога, наречен был Богоносцем.

Но этот священномученик не менее был Богоносцем и потому, что сам носил в сердце и в устах Бога и был избранный сосуд Христов, подобно святому апостолу Павлу, что чудодеющая десница Господня показала и по страдальческой кончине его. Ибо, будучи предан по повелению Траяна на съедение зверей, непрестанно имел он в устах своих имя Иисуса Христа, и, когда нечестивые спросили у него, зачем он немолчно твердит имя это, святой отвечал: «Оно написано в сердце моем; как же мне не исповедовать его устами?» Наконец, когда праведник был растерзан и съеден зверями, а сердце его Божеским произволением осталось цело и невредимо между костями, тогда злочестивые невежды, помня слова его, разрезали оное и, к изумлению своему, увидели, что внутри сердца начертано золотыми буквами: «Иисус Христос».

Христиане! Если мы, подобно священномученику Игнатию, всегда будем носить в сердце и устах Спасителя нашего Иисуса Христа, то на Втором Пришествии удостоимся быть в отеческих Его объятиях.

\section{Отчего церковную службу поют на два хора}

Святой Игнатий Богоносец, будучи однажды в Небесном восхищении и сподобившись Божественного откровения, видел Ангельские лики, поющие один с другим попеременно. Так что, когда славословил Господа первый лик, безмолвствовал другой, а когда этот начинал хвалебную песнь, то с глубоким вниманием слушал первый, и таким образом оба лика, как бы передавая друг другу песнопения, славословили Пресвятую Троицу.

Святой Игнатий, подражая Ангелам, установил этот же порядок в Антиохийской Церкви, а оттуда это величественное и ангелоподобное обыкновение распространилось и по всему царству благодати.

\section{Святой архидиакон и первомученик Стефан\footnote{Память св. апостола архидиакона и первомученика Стефана (†~34) празднуется 27~декабря (9~января).}}

Когда, по вознесении Господнем, везде начало возрастать Слово Евангельское, то с распространением оного возрастали братолюбие и странноприимство. Святые апостолы были руководителями предназначенных для этой цели учреждений. Но поскольку служение трапезам вдовиц и сирот несовместно было с великой обязанностью проповедовать Евангелие всему миру, то и советовали они верному народу избрать семь мужей, исполненных Духа Святого и премудрости\footnote{Их имена следующие: святые Стефан, Филипп, Прохор, Никанор, Тимон, Пармен и Николай.}.

Посланники Господни возложили на них руки свои и нарекли диаконами. А Стефан, как первый из них по мудрости, благочестию и добродетелям, был наречен архидиаконом. Будучи силен словом и делом, этот равноапостольный муж утверждал верных в вере, обличал неверных и, возлагая руки на недужных, исцелял их от болезней. А особенно гремел против иудеев, укоряя их, что по единой зависти умертвили Сына Божия, Мессию, от всех веков ожидаемого, "--- за что и был ими ненавидим.

В одно время, когда между иудеями, фарисеями, саддукеями и эллинскими евреями\footnote{Иерусалимляне называли эллинами тех иудеев, которые были рассеяны по Римскому государству; их вера и нравы были иудейские, но общеупотребительный язык был греческий. Св. Стефан и его сослужители были от рода этих же иудеев, что показывают их греческие имена.} было состязание о Господе нашем Иисусе Христе, святой Стефан, став на высоком месте, воскликнул: «Братья мои! Всуе умножилась в вас злоба, и разделился весь Иерусалим. Блажен тот, кто уверовал в Господа нашего! Сей Господь приклонил небеса, сошел для искупления нас от рабства греховного, родился от Святой Девы, избранной прежде сложения мира». Нечестивые иудеи, будучи не в силах стоять против премудрости и Божественного красноречия, воспылали гневом и яростью на святого Стефана и донесли великой архисинагоге еврейской, будто архидиакон изрыгает хулы на Моисея и Самого Бога, воздвигли на него старцев и народ и предали суду архиереев.

Святой архидиакон стоял посреди сонмища, как Ангел Божий, сияя благодатью Духа Святого. С великим благоговением воспоминая о праотцах и пророках, он старался опровергнуть клевету лжесвидетелей, но, видя ожесточение архиереев и книжников, исполнился Божественной ревности и укоризненно воскликнул: «\textit{Жестоковыйнии и необрезаннии сердцы и ушесы, вы присно Духу Святому противитеся, якоже отцы ваши, тако и вы} (Деян. 7, 51). Кого из пророков не изгнали отцы ваши? Кого из провозвестников Мессии не убили?» Тогда весь сонм распылался сердцем и заскрежетал зубами. А равноапостольный муж, воздев очи свои к небу, узрел славу Божию, узрел Иисуса, простирающего к нему десницу, и велегласно воскликнул: «Се вижу небеса отверсты и Сына Человеческого, стоящего одесную Бога». Услышав это, иудеи затыкали уши свои и, возложив на него руки, вывели из города, чтобы побить его камнями.

Сколь сладостная, хотя и посреди лютых мучений готовилась смерть первострадальцу! От небесной высоты призывал его Сладчайший Иисус, а с ближайшей горы на подвиг его смотрела Богородительница с Иоанном Богословом и молилась Сыну Своему, да приимет душу его в руце Свои. Наконец, посыпался на святого Стефана каменный дождь. Весь окровавленный, до смерти изнеможенный, он повергся на колена и велегласно воскликнул: «\textit{Господи! Не постави им греха сего} (Деян. 7, 60)» "--- и, преклонив главу свою, скончался; а святая душа его воспарила в отверстые небеса ко Господу и Царю Славы, Которого он проповедовал и за Которого пострадал.

\begin{center}\small\textsc{Конец четвертой части.}\end{center}

\chapter{ЧАСТЬ ПЯТАЯ}
\section{Знаменитые и святые друзья, или пример, как дети должны вести себя в общенародных училищах}

Святой Василий Великий и святой Григорий Богослов происходили от поколений сколько благородных, столько и благочестивых. Они родились в одно почти время и своим рождением обязаны особенно молитвам своих матерей, которые и принесли их, как дар Божий, в жертву Богу. Мать святого Григория, принесши его в сороковой день в церковь Господню, освятила младенческие руки прикосновением к Евангелию и тем утвердила обет, данный Спасителю мира.

Оба святые отрока имели все то, что делает детей привлекательными: острый ум, добродушие и чистосердечие, тихость и учтивость в поступках, и красоту телесную. Но счастливые природные дарования еще более украшены были воспитанием, ибо оно было таково, какое можно вообразить посреди семейств, где вера и благочестие переходили от родителей к детям и от старших к младшим\footnote{Отец св. Григория Богослова, тоже Григорий, был епископом Назианзинским. Мать его, праведную Нонну (†~374), Церковь воспоминает 5~(18) августа. Бабка св. Василия Великого по отцу, именем Макрина, претерпела гонение вместе с мужем в царствование Максимиана Галерия и семь лет скиталась в пустынях. Его сестра и все братья причислены к лику святых.}.

После домашнего\footnote{Святители Василий Великий и Григорий Богослов были родом из Каппадокии, только из разных городов.} учения послали их в греческие города, которые наиболее славились просвещением, и там принимали они уроки от лучших учителей. Наконец, сошлись в Афинах. Известно, что этот город был средоточием словесных наук и всей учености: он был и колыбелью славной дружбы двух святителей. Одно происшествие подало к тому первый случай\footnote{Об этом повествует сам св. Григорий Богослов, описывая свое пребывание в Афинах.}. В Афинах было вздорное обыкновение в отношении новоприбывших учеников: с самого начала вводили их в многочисленное собрание юношей, таких же, как и они, и тут заставляли их терпеть разные насмешки и ругательства. После того, соблюдая не менее смешные обряды, водили их в общественные бани чрез весь город среди тех же молодых людей, идущих по два в ряд. Здесь вся толпа останавливалась, поднимала великий крик и делала вид, что хочет выломать ворота, как будто не хотели отворить оных. Когда новоприбывший ученик был туда впущен, тогда начинали обходиться с ним запросто и принимали в число товарищей. Святой Григорий, который прибыл в Афины прежде и который чувствовал, сколько этот достойный посмеяния обряд будет противен и досаден строгому нраву святого Василия, уговорил, хотя с трудом, своих товарищей, чтобы его уволили от оного обряда. Это"=то было искрой благочестивой их дружбы и воспламенило в них огонь, который после никогда не угасал. «О счастливые для меня Афины! "--- восклицает по этому случаю святой Григорий. "--- Я пришел сюда единственно для приобретения знаний, но нашел драгоценнейшее из всех сокровищ "--- верного и нежного друга, будучи счастливее Саула, который пошел на паству, обрел царство».

Начатый таким образом союз укреплялся день от дня более, особенно же когда эти два друга, открывая взаимно свои сердца, узнали, что они оба имели одну цель и искали одного сокровища "--- мудрости и добродетели. Они жили в одном доме, пользовались одним столом, имели одни упражнения, одни удовольствия и, так сказать, одну душу. Оба равно стремились обогатить разум свой познаниями, и, хотя это наиболее может возбудить зависть, они, будучи неприступны для этой злобной страсти, не ведали и не ощущали между собой ничего, кроме благородного соревнования. Каждый из них, любя больше славу своего друга, нежели свою собственную, старался не о том, чтобы взять над ним преимущество, но чтобы за ним следовать.

Но главное их учение и единственная цель была "--- добродетель. Они помышляли учинить свое дружество вечным, приуготовляя себя к блаженной вечности и отвлекая себя понемногу от сует мира. Они имели руководителем Слово Божие и служили сами себе вместо учителей и надзирателей, увещевая взаимно один другого к благочестию. Зная, что дурные примеры подобны заразительным болезням, они не имели никакого общения с теми из товарищей, которые были развратны, дерзки и бесчинны. А вели знакомство только с умеренными, скромными и благочестивыми, которые могли подкреплять их в добром намерении.

Эти два святых юноши блистали всегда среди сверстников своих красотой и живостью ума, охотой к трудам, отменным успехом в науках, которые тогда преподавали в Афинах, но они отличались еще более невинностью нравов, ужасаясь даже тени зла. И это было необходимо посреди Афин, опаснейшего для нравов города, по причине необычайного сборища молодых людей, которые приходили туда со всех сторон и приносили с собой не только свои пороки, но и пороки своего отечества. Чем же они защищались от бесчисленных покушений, их окружавших? «Мы знали в Афинах только две дороги, "--- говорит святой Григорий, "--- одну, которая вела нас в церковь и к святым наставникам, в оной проповедующим, другую, которая вела нас в Академию, к учителям словесности и любомудрия. Что касается до тех дорог, по коим ходят на мирские праздники, на зрелища, на пиршества, мы их не знали и знать не хотели».

Кажется, юноши, которые удалялись от всех увеселений, которые не принимали участия в удовольствиях своих сверстников, которых жизнь, чистая и беспорочная, была всегдашним обличением их разврата и шалостей, "--- эти юноши должны были возбудить против себя их ненависть или, по крайней мере, их насмешки. Однако же воспоследовало противное этому, и не было в Афинах ничего столь славного и всеми уважаемого, как имена этих юных особ. Без сомнения, их добродетель была беспримерно чиста и поведение благоразумно и умеренно, когда они при всех успехах, освящаемых добродетелью, умели не только избежать холодности, но и привлечь к себе почтение и любовь со стороны всех сотоварищей.

Это обнаружилось всего яснее, когда узнали, что юноши намерены оставить Афины и возвратиться в свое отечество. Соболезнование было всеобщее, жалобы слышны были со всех сторон, слезы текли у всех из очей. «Мы лишились, "--- говорили афиняне и чужестранцы, "--- всей чести города и славы наших училищ». Учителя и ученики, присоединяя к просьбам и жалобам насилие, утверждали, что они их не отпустят и никогда не согласятся на их отъезд.

И действительно, один из этих знаменитых и святых юношей должен был остаться в Афинах. Это был святой Григорий. В то время как Василий посещал славных пустынножителей, ходил в Иерусалим поклониться Гробу Господню и отовсюду собирал сокровища духовной мудрости, "--- Григорий, сердечно сокрушаясь о разлуке с другом своим, был учителем красноречия в Афинах.

Любезные юноши! Вот для вас образец, в котором находятся совокупно все совершенства, могущие сделать вас любезными и достопочтенными. Красота ума, чистейшие нравы, невероятная жажда к учению, чудный успех в науках, честные и скромные поступки, удивительное смирение посреди похвал и рукоплесканий народных и "--- что еще более возвышало качества святого Василия и святого Григория "--- благочестие и страх Господень посреди всеобщего разврата "--- это такие достоинства, с которыми ничто сравниться не может. Старайтесь же приобрести оные и тогда привлечете на себя благословение отечества и благословение Божие.

\section{Суетность человеческих начинаний, противных воле Божией}

Богоотступник Юлиан, стараясь поколебать христианскую веру в ее основании, хотел доказать, что пророчество Иисуса Христа о разорении Иерусалима (см. Мф. 24, 1~-- 2; Мк. 13, 1~-- 2; Лк. 21, 5~-- 6) ложно. Для этого он вознамерился привести иудейскую столицу опять в цветущее состояние, соорудить храм, истребленный римлянами, и возвеличить богослужение Ветхого Завета. Это предприятие в мыслях отступника не находило препятствий. Иудея была его область, и ее жители, рассеянные по обширным пределам его империи, воздыхая о своем отечестве, должны были не иначе принять повеление возвратиться в оное, как с усердием и радостью.

Обдумав таким образом свой богопротивный план, Юлиан обратился с ласковым воззванием к иудейскому народу. Он жаловался на бедствия и притеснения, которые евреи столь долго терпели, а особенно вопиял против оскорбительных налогов, которые с них взыскивали. «Эту несправедливость, "--- говорил он, "--- вы должны приписывать не столько правительству, сколько христианам, которые свои больницы и странноприимные дома в изобилии содержат вашим потом и кровью. И чем же благодарят за это? Всегда более и более ожесточают на вас свое варварское сердце. С этого времени я освобождаю вас, "--- продолжает Юлиан, "--- от всех бесчеловечных узаконений и вместо того хочу оказать вам добро, чтобы вы от чистого сердца молились о моем благоденствии, о благосостоянии отечества, об успехах на войне и о счастливом возвращении из Персидского похода\footnote{Юлиан тогда готовился идти в Персию.}. Тогда"=то, наконец, "--- говорит Юлиан, "--- я вместе с вами буду обитать в священном городе, мною воздвигнутом, украшенном и возвеличенном. Вместе с вами буду приносить жертву всесожжения Всевысочайшему Богу».

Мог ли столь сильный государь сказать что"=либо более, дабы обрадовать народ, который по природе был горд и суеверен и не выпускал из своих мыслей владычества над всеми народами? Иудеи от всех пределов государства обратили взоры свои на Юлиана и ждали решительного мановения, чтобы составить опять сильное и славное царство. А эта готовность их и усердие имели следствием то, что император не остался при одном приятном обещании: он немедленно дал своему наместнику повеление строить на царские средства храм иудейский. Мало того: Алипий, искренний друг и любимец Юлиана, нарочно отправлен был из Антиохии в Иерусалим, чтобы с возможной поспешностью начать и в кратчайшее время совершить строение храма.

Тогда иудеи отовсюду начали стекаться на развалины Иерусалима, и невозможно описать, какие напасти вынуждены были терпеть палестинские христиане! Но, к вящему их ужасу, Юлиан, выступая с воинством из Антиохии, дал Алипию повеление, чтобы он по сооружении храма немедленно начал строить в Иерусалиме амфитеатр, в котором он, возвратившись из Персии, хочет наслаждаться с иудеями зрелищем, как дикие звери будут терзать епископов, иноков и всех, кто дерзнет защищать христианство\footnote{Орозий в VII книге.}. Единому Богу известно, что сделалось бы с несчастными христианами, которые начали отчаиваться, если бы не подкрепляли их неустрашимые поборники Евангелия! Святой Кирилл, архиепископ Иерусалимский, и прочие святые христиане в это лютое время подавали удивительный пример веры, которая сама себя утешает, и совершенной надежды, что \textit{небо и земля прейдут, словеса же} Господни \textit{не прейдут} (Мк. 13, 31). Будучи тверды в уповании на скорое заступление Господне, они напоминали колеблющимся о \textit{мерзости запустения, реченной Даниилом пророком} (Мк. 13, 14), и из пророчества Самого Иисуса Христа утешали, что злочестивое предприятие совершиться не может, только бы они уповали на Господа, Который в свое время посрамит ненавистников Своего святого имени.

Между тем художники и рабочие ревностно продолжали свое дело. Очищено было место, где стоял прежний храм и где должно было соорудить новый; материалы приготовлены были в великом множестве; тысячи иудеев неусыпно занимались работой. Серебро, золото и драгоценные камни были собраны в изобилии. Даже иудеянки, прежде того празднолюбивые и изнеженные, приготовляли оправленные серебром носилки и заступы, чтобы носить известь и камни. Наиболее привязанные к роскоши почли бы за несчастие, если бы отец или муж не отдали их нарядов на постройку храма.

Вскоре приступили к основанию храма. Иудеи и язычники начали издеваться над пророчествами Евангелия. Юлиан мог сказать: \textit{Виждь, каково камение и какова} будут \textit{здания} (Мк. 13, 1)! Но возгремел глас грома в ответ богоотступнику: \textit{Не имать остати зде камень на камени, иже не разорится} (Мк. 13, 2). Внезапное землетрясение, сопровождаемое бурей, громом и молнией, потрясло и рассыпало все, что было основано, так что под щебнем погибло несколько рабочих. Это чудо хотя и устрашило Алипия и прочих лиц, наблюдавших за постройкой, однако не привело их в отчаяние. Через несколько дней они принялись опять за строение. Но едва положены были первые камни, как новое землетрясение, более ужасное, чем первое, не только уничтожило начатую работу, но извергло из земли даже старый, времен Соломона, фундамент\footnote{См. в Четии"=Минее, в житии св. Кирилла, архиепископа Иерусалимского.}. Порывистый вихрь развеял известь и унес орудия каменщиков. А вышедший из земли огонь и спустившееся с неба пламя окончательно все уничтожили. Множество народа сгорело и было засыпано. Стихии преследовали бегущих иудеев и язычников до самых их жилищ, запечатлевая не только на их одеждах, но и на теле знаки креста, которые сперва были светящиеся, а потом чернели так, что невозможно было их отмыть\footnote{Об этом повествуют как святые отцы, так и все историки тогдашнего времени.}. Самое место, где столь ужасно чудодействовала рука Господня, совсем запустело. Тогда все познали суетность человеческих начинаний, противных воле Божией. Иудеи боялись выйти из своих домов и были в непрестанном ужасе, дабы десница Господня не поразила их еще более. Большая часть из них признали Богом Того, Кого предки их повесили на кресте\footnote{\textit{Сократ}. Церковная История. Кн. V, гл. 20.}. Язычники, ожесточенные менее иудеев, поражены были несказанно при виде столь непостижимого чуда. Их изумление и ужас были так велики, что все они в ту же минуту воззвали к Иисусу Христу о милосердии и старались умилостивить Его молитвами и песнопениями. Многие из них в то же время бежали к христианским священникам и неотступно просили принять их в недра Святой Церкви\footnote{\textit{Свт. Григорий Богослов}. Слово о язычниках.~X.} и даровать Святое Крещение.

Одного Юлиана не поразил глас гнева Небесного. Донесение Алипия произвело в нем только досаду и огорчение, которые он пред народом и своими приверженцами старался прикрыть невниманием. Отвергать этого происшествия было невозможно, ибо оно случилось пред очами разных племен. Приписать злобе и мщению христиан, как то сделали при сожжении Аполлонова храма\footnote{См. в Четии"=Минее под 2"~м числом сентября.}, также было нельзя, ибо все рабочие и зрители видели своими глазами подземное и небесное пламя. Признать истину и победу Иисуса Христа не позволяло ему упорное и закоренелое неверие. Итак, отступник решился это чудодействие Божие считать маловажным, ничего о нем не говорить, смотреть как на обыкновенный случай, на действие природы и, наконец, предать забвению.

Но вскоре сами язычники и друзья Юлиана вынуждены были признаться, что это чудо было грозным вестником его погибели. Отступник поражен был на войне рукой невидимой. Суд Божий над ним был так достопримечателен, что один из язычников при известии о его смерти воскликнул: «Теперь христиане не могут прославлять пред нами долготерпение своего Бога!» А Феодорит повествует, что антиохийские жители тогда в один голос взывали: «Победил Бог и Его Помазанник!»\footnote{\textsf{Ένίκησεν ο̉ Θεός καί ο̉ Χριστός Αύτοῡ}.} Кто не видит, что Юлиана поразила\footnote{О смерти Юлиана см. выше.} та самая рука, которая рассыпала и пожгла основание Иерусалимского храма?

\section{Мысли о раздаче нищим своего имения}

Когда весь свет удивлялся великой жертве, которую святой Павлин\footnote{Память св. Павлина, епископа Ноланского (†~431), 23~января (5~февраля).}, епископ Ноланский, принес Иисусу Христу, раздав все свои богатства неимущим, "--- человек Божий думал, что он \textit{Отцу, Иже есть на небесех, и братиям Христовым} ничего не сделал. «Я подобен бойцу, приготовляющемуся к борьбе, или человеку, хотящему плыть через реку, "--- обыкновенно говорил он, "--- они оба ничего еще не сделали чрез то, что скинули с себя одежду».

\section{Разлука и свидание благочестивого семейства}

Ксенофонт\footnote{Прп. Ксенофонт жил в V~в. в царствование Иустиниана Великого. Память его празднуется 26~января (8~февраля).}, боярин царьградский, украшенный семейственными, гражданскими и Евангельскими добродетелями, и Мария, его благочестивая супруга, имели двух сыновей, Иоанна и Аркадия, которых воспитывали в страхе Божием. Но поскольку знатность рода и положение, к которому готовятся дети государственных сановников, требовали высшего просвещения, а особенно познания нравов народных, то родители и отправили их в Финикию, где тогда было знаменитое училище греческого любомудрия. Отец благословил их христианскими наставлениями, мать оросила лица их слезами родительской нежности, и Аркадий с Иоанном пустились в море.

Сначала их плавание было благополучно. Но внезапно поднялась буря: вскоре паруса были изорваны, мачты изломаны, и корабль стал игрушкой волн. Мореплаватели видели пред собой неизбежную смерть. Иоанн и Аркадий оплакивали вечную разлуку с родителями и, представляя себе их горесть, забыли о собственной опасности. Наконец разбитый корабль начал тонуть. Каждый должен был решиться на последнее средство, не разбирая, спасет ли оно жизнь его. Два родных отрока скинули с себя одежды и, воскликнув: «Простите, дражайшие родители!», "--- бросились в море. Рабы их сделали то же. Кто успел, ухватился за обломки корабля, и все отдались направлению волн.

Но Промысл Божий не восхотел погубить благочестивых юношей. Вскоре Иоанн был выброшен на сушу и опомнился, но, увидев наготу свою, не знал, что делать. Стыдясь показаться пред людьми, наконец решился он идти вдоль по берегу и, к счастью, увидел монастырь. «Благодарю Тебя, Господи Боже мой! "--- воскликнул обрадованный Иоанн. "--- Здесь живут благочестивые люди, которые стыдятся только душевной наготы; моя телесная нагота для них не будет в соблазн». Он смело вошел в мирную обитель и был принят с таким человеколюбием, какого можно ожидать только от святого места. Там, рассуждая о суетности мира и о бедствиях человеческих, через некоторое время он постригся и начал подвизаться в молитве и посте; сердечное спокойствие, удел благочестивой невинности, было в нем нарушаемо только сокрушением о любезном Аркадии, которого Иоанн считал погибшим.

Но та же десница извлекла из бездн морских и Аркадия. Очувствовавшись на берегу морском, он увидел себя среди народа, который старался подать ему помощь. Тут он выпросил себе одежду, вошел в близлежащую церковь и, помолившись о брате своем Иоанне, от усталости и изнеможения уснул. Вдруг является ему во сне Иоанн и говорит: «Не скорби, дражайший брат! Я по благодати Божией жив». Воспрянув ото сна, Аркадий принес благодарение Богу и хотел возвратиться в Царьград. Но мысль, что ему первому придется поразить родителей плачевной вестью о брате, его остановила. Горько восплакав, он решился искать его и прежде всего пошел в Иерусалим, чтобы в стране, освященной стопами Иисуса Христа, испросить на то помощь Небесную. На пути он встретился с одним, украшенным сединой, иноком и как изумился, когда старец, благословляя его, сказал: «Бог да помилует тебя молитвами твоего брата, который богоугодно подвизается в образе ангельском!» Аркадий припал к ногам старца и умолял открыть ему местопребывание возлюбленного Иоанна; но прозорливый инок отвечал: «Будет время, когда Сам Бог покажет тебе его; а ныне ожидай спокойно». "--- «Хочу же и я, "--- воскликнул тогда Аркадий, "--- облечься в иночество, чтобы в единодушии с братом моим послужить мне Господу». Старец благословил его доброе намерение, привел в одну из святых обителей, постриг и, наставив равноангельскому житию, удалился в свою пустынную келью.

Между тем после крушения корабля прошло два года. Ксенофонт, не получая известия от детей своих и сомневаясь, живы ли они, послал одного из рабов в Финикию, чтобы тот разузнал об Иоанне и Аркадии. Но так как их там не было, то верный слуга подумал, не в Афинах ли они, и отправился туда, но и в Афинах не нашел Иоанна и Аркадия. Не зная, какой ответ принести своим господам, он пошел обратно в Царьград и на одном ночлеге встретился с иноком, лицо которого показалось ему знакомо. В самом деле, это был один из рабов Ксенофонтовых, потерпевший с детьми его кораблекрушение. Но радость свидания была кратковременна и вскоре обратилась в горесть. Один рассказывая, а другой, слушая об ужасном происшествии на море, оба плакали о своих добрых господах.

Какой удар был для родительского сердца, когда Ксенофонт и Мария услышали о гибели детей своих! Но в следующую ночь они увидели в ночном видении Иоанна и Аркадия, сияющих ангельской славой, и положились во всем на Господа. Припомнив о рабе, принявшем иночество, они заключили тоже и о своих детях. «В волнах ли погибли наши чада, "--- говорили они друг другу, "--- или подвизаются где"=нибудь в мирной обители, но они умерли для света. Не время ли и нам умереть для сует мирских?» В самом деле, они вскоре отправились в Иерусалим, чтобы посетить святые места и там постричься.

Уже благочестивые путешественники были близ святого града, как однажды подошел к ним старец, тот самый, который облек в иноческий образ Аркадия, и, посмотрев на них, сказал: «Что побудило к столь дальнему путешествию Ксенофонта и Марию? Вероятно, сетование о чадах? Утешьтесь: они живы, и вы их увидите». Восхищенные родители начали опрашивать его об Иоанне и Аркадии. Но старец сказал им: «Они сами придут к вам, когда, по вашему обещанию, обойдете святые места». Ксенофонт и Мария, в несомненной надежде на Господа, приняли от него благословение и пошли на Иордан.

В тот же день прозорливый инок, будучи на Голгофе у церкви Воскресения Господня, встретил Иоанна, который из своего монастыря пришел туда поклониться Гробу Господню. «Чадо мое, любезный Иоанн! "--- сказал чудный старец. "--- Где ты был доселе? Родители твои давно тебя ищут, как и ты ищешь брата твоего Аркадия». Удивленный Иоанн не знал, что подумать о его приветствии, но, видя в нем дар прозорливости, припал к стопам его со словами: «Поведай мне, святой авва, где находится брат мой? Без него я страшусь увидеть и моих родителей». "--- «Сядь подле меня, "--- отвечал старец, "--- и увидишь Аркадия». В это мгновение показался издали юный инок. Приблизившись к ним, он принял благословение от старца и вдруг бросился в объятия Иоанна. Это был Аркадий, брат его. Они плакали, лобызались и прославляли Бога, сподобившего их увидеть друг друга.

Через несколько дней возвратились в Иерусалим и Ксенофонт с Марией. Увидев прозорливого старца при Живоносном Гробе, они опять начали умолять его, чтобы сказал, где находятся их дети, между тем как Иоанн и Аркадий были тут же. Узнав своих родителей, они едва могли удержать свою радость и, только повинуясь старцу, не бросились в их объятия. Они стояли, потупив глаза, чтобы не изменить себе самим; иноческая одежда и лица, увядшие от воздержания, тому способствовали. Наконец святой старец сказал Ксенофонту и Марии: «Идите в вашу гостиницу и уготовьте трапезу, я с учениками моими приму у вас пищу, а потом возвещу, где ваши дети». Обрадованные родители исполнили волю старца, и он с Иоанном и Аркадием вскоре пришел к ним.

Сидя за трапезой, Ксенофонт и Мария вели беседу и в то же время взглядывали на юных иноков. Наконец Ксенофонт, вздохнув, сказал: «Как мне любезны ученики твои, святой авва! Как только я посмотрел на них, душа моя привязалась к ним, и возвеселилось сердце, как будто при свидании с Иоанном и Аркадием. О, если бы таковы были дети наши!» Ксенофонт умолк, а у Марии навернулись на очах слезы. Тогда старец, обратившись к Аркадию, сказал: «Поведай, чадо, где ты родился, где воспитан и откуда пришел сюда?» Но едва Аркадий начал свою повесть, как оба юноши очутились в объятиях родительских. Кто может изобразить радость столь внезапного свидания! Сам прозорливый старец прослезился от удовольствия, что Господь избрал его Своим орудием и чрез него возвратил родителям чад и чадам родителей.

После этого Ксенофонт поручил ближайшему из родственников продать свой дом, рабам даровать свободу, имение раздать убогим и от руки прозорливого старца принял образ иноческий. Мария вступила в лик святых жен. Иоанн и Аркадий удалились в пустыню с мудрым наставником. Все они просияли верой, благочестием, добродетелями и даром чудотворения.

\section{Незлобие великого старца}

Святой Ефрем\footnote{Прп. Ефрем Сирин (†~373~-- 379), уроженец Эдесский, жил в царствование Феодосия Великого. Память его празднуется 28~января (10~февраля).}, любя заниматься богомыслием в уединении, не ходил на общую трапезу, но ученик его в определенное время приносил ему пищу. Однажды, идя из поварни, он нечаянно уронил сосуд и разбил. Боясь гнева старца, он не знал, что делать. Но незлобивый Ефрем, догадавшись об этом, с кротостью сказал ему: «Не печалься, чадо мое! Если не хотела к нам прийти пища, то нам должно идти к ней». Он и в самом деле пошел, сел подле разбившегося сосуда и начал собирать пищу.

\section{Свидание преподобного Ефрема с Василием Великим}

Преподобный Ефрем, стоя ночью на молитве, узрел в видении огненный столп, простиравшийся от земли до неба, и услышал глас: «Ефрем! Ефрем! Таков есть Василий Великий!» Старец душевно возжелал увидеть великого архипастыря и на другой же день пошел в Кесарию Каппадокийскую. Он застал его в церкви, проповедующего Слово Божие, и начал восклицать громко: «Воистину, велик Василий! Воистину, столп огненный Василий! Воистину, Дух Святой глаголет устами Василия!» Весь народ обратил внимание на Ефрема, а некоторые с коварной улыбкой сказали: «Видно, этот странник хочет польстить архиепископу и похвалой выманить у него хорошее место». Таковы люди вообще: свои страсти и пороки обычно приписывают и другим.

По окончании Божественной службы Василий пригласил к себе старца и, между прочим, спросил, почему он в церкви и посреди народа так прославлял его. «Я видел, "--- отвечал Ефрем, "--- белого голубя, сидящего у тебя на правом плече и нечто к уху твоему глаголющего, видел также огонь, исходящий из уст твоих, и, этим восхищенный, не мог удержать языка моего». Святой Василий, удивляясь прозорливости старца, посвятил его в пресвитера. Но преподобный Ефрем хотя из послушания к архипастырю и принял священство, однако во всю жизнь не хотел совершать литургии, почитая себя недостойным служить Страшным Божественным Тайнам, и занимался только проповеданием Слова Господня.

\section{Степени подвижничества и коль великий подвиг есть послушание}

Один великий старец, находясь в ангелоподобном восхищении, видел на небе четыре ступени, знаменующие совершенство подвижников. На первой ступени стоял удрученный недугами, но благословляющий имя Господне; на второй "--- бескорыстный странноприимец; на третьей "--- безмолвный пустынножитель; наконец, четвертую, и самую, высшую ступень занимал послушный своему наставнику и всем сердцем ему преданный ради Господа. Этот человек имел на себе багряную ризу, знаменуя Того, Кто \textit{послушлив быв даже до смерти, смерти же крестныя} (Флп. 2, 8), и более всех блистал славой. «Почему этот по-видимому меньший делами возвеличен более прочих?» "--- подумал удостоенный небесного видения старец. «Потому, "--- вдруг ответствовал ему таинственный голос, "--- что странноприимец упражняется в добродетели, столь любезной его сердцу, по своей воле, равно и пустынник удалился от света по своему благорассуждению и живет свободно. Что касается до удрученного недугами, он бы с радостью переменил их на здравие. Но этот, принявший на себя труднейшее дело послушания, оставив все свои желания, зависит от Бога и своего наставника».

\section{Совет инока отрекающемуся от света}

Патриций Мануил, находясь в тяжкой болезни и отчаявшись в жизни своей, восхотел грехи прошедшего времени прикрыть полным покаянием и отречься от света, пока смерть не разлучила с оным. Для этого он призвал святого Николая, игумена Студийского\footnote{Прп. Николай исповедник, игумен Студийский (†~868), жил в IX~в., в царствование Феофила Иконоборца, сына его, Михаила, и Василия Македонянина. Память его празднуется 4~(17) февраля.}, и слезно просил, чтобы немедленно облек его в образ ангельский. Но прозорливый старец ответствовал: «Чадо! Эта жертва не будет на пользу душе твоей. Останься в мире и жди посещения Господня. Именем Спасителя нашего уверяю тебя, что ты вскоре будешь здрав и получишь знатный сан. Прейди это поприще во благо Церкви и отечества. Тогда я постригу тебя, да пойдешь в другой мир с добрыми делами».

Пророчество святого Николая сбылось. Мануил вскоре получил здравие и, будучи взыскан милостью императора, до глубокой старости исправлял важные государственные должности: был неусыпен, человеколюбив и правосуден. Наконец, когда приспела его блаженная кончина, преподобным Николаем был пострижен и в мире отошел к Господу.

\section{О том, сколь ужасно оскорблять родителей}

Однажды к преподобному Парфению, епископу Лампсакийскому\footnote{Прп. Парфений жил в IV~в. в царствование Константина Великого. Память его празднуется 7~(20) февраля.}, приведен был юноша, жестоко мучимый нечистым духом. Человеколюбивый Парфений при первом взгляде на страждущих даже без их просьбы всегда помогал им своими молитвами, но, посмотрев на юношу, обнаружил неудовольствие. Родители, припадая к стопам человека Божия, слезно умоляли его, дабы умилосердился над их сыном и избавил от лютого недуга. Но человек Божий отвечал им: «Ваш сын недостоин исцеления. Дух"~мучитель дан ему в наказание за то, что он "--- почти отцеубийца». Родители ужаснулись. Парфений спросил у них: «Сын ваш часто оскорблял вас?» "--- «Так!» "--- отвечал отец. «Вы молились в горести души вашей, "--- продолжал Парфений, "--- чтобы Господь наказал его?» "--- «Согрешили пред Господом», "--- отвечали со вздохом отец и мать. Тогда человек Божий сказал: «Пусть же он страждет как заслуживший это наказание». Но чадолюбивые родители, болезнуя сердцем своим о сыне, не переставали проливать слезы и умоляли святителя Христова, да испросит ему у Бога прощение. Только их просьбой преклоненный, Парфений благословил юношу и, помолившись Господу, исцелил его.

Дети! Ужасайтесь оскорблять своих родителей: их клятва низводит на вас клятву Небесную.

\section{Помощь ближнему Господь принимает как истинное богослужение}

Преподобный Досифей\footnote{Память прп. Досифея (†~VII~в.), ученика прп. аввы Дорофея, празднуется 19~февраля (4~марта).}, будучи одержим тяжкою болезнью, сказал одному великому старцу "--- Варсонофию: «Владыка мой! Уже не могу в живых быть». На это Варсонофий отвечал: «Чадо! Иди с миром предстать Святей Троице и молись о нас».

Братия, услышав этот ответ великого старца, начала негодовать и говорить между собой: «Что великое сделал этот юный монах, что удостоился принять такой ответ от святого отца? Мы не видели, чтобы он удручал себя постом или бдением более других. Напротив того, часто на всенощную службу он приходил после всех нас, а иногда оставался и дома».

Когда эти разговоры достигли слуха Варсонофия, святой старец за общей трапезой спросил у братии: «Когда колокол призывает в храм Господень, а у меня есть на руках больной брат, тогда что надлежит мне делать: оставить брата и идти в церковь или оставить церковь и утешать брата?» "--- «Помощь, оказываемую ближнему, без сомнения Господь примет в этом случае за истинное богослужение», "--- отвечали иноки. «А если силы мои от постничества ослабеют настолько, что я буду не в состоянии надлежащим образом служить болящей братии, должен ли я укрепить их пищей, чтобы ходить за больным неусыпно, или должен продолжать свой пост, хотя бы чрез это нечто отнималось у болящих, которые обыкновенно бывают очень прихотливы и взыскательны?» "--- «Для этого не надобно оставлять поста, предписанного уставом обители, "--- отвечали иноки, "--- но самопроизвольный и чрезмерный пост, кажется, в этом случае не столько будет приятен Богу, сколько попечение обо всех нуждах больного брата». "--- «Вы рассуждаете справедливо, "--- сказал тогда святой Варсонофий гласом учительским, "--- для чего же языку вашему даете свободу говорить иначе? Богобоязненный Досифей, имея попечение о больных наших собратьях, не столь долговременно постился, как некоторые из вас. Но вы сами были свидетелями, сколь усердно и неусыпно служил он болящей братии, с какой любовью предупреждал строптивые их требования и прихоти! Мы не видели, чтобы он возроптал когда"=нибудь на свои труды и усталость. Напротив того, сколь часто видели его плачущим как о величайшем согрешении, если когда, будучи не в силах снести излишних требований болящего, что"=нибудь скажет ему с сердцем».

Этот старец говорил истину. \textit{Егда же приидет Сын Человеческий в славе Своей и вси святии Ангели с Ним}, скажет всем, \textit{сущим одесную Его: понеже сотвористе единому сих братий Моих меньших, Мне сотвористе} (Мф. 25, 31, 34, 40).

\section{Беспристрастие к себе и справедливость к другим, или о том, что общую пользу должно предпочитать своей пользе\footnote{Из жития свт. Тарасия, архиепископа Константинопольского (†~806), память которого празднуется 25~февраля (10~марта).}}

После смерти иконоборствующего царя Льва IV Хазара патриарх Царьградский Павел, муж добродетельный и благочестивый, хотел восстановить поклонение святым иконам, но, будучи боязлив\footnote{Эта боязливость принудила патриарха Павла IV утвердить своим рукоприкладством еретические правила иконоборного собора. Впоследствии прославлен в лике святых. Память его празднуется 30~августа (12~сентября).} и нерешителен и не имея необходимо нужного для высших начальников искусства избирать достойных людей себе в помощники, не мог начать столь великого дела, тем более что иконоборство весьма укрепилось и имело своими поборниками первых сановников государства. Не принимая никаких мер, он только сетовал и сокрушался и, наконец, не ожидая успеха, тайно удалился в монастырь и там принял на себя схиму.

Царствовавшая тогда Ирина, мать малолетнего императора Константина, весьма удивилась, что Павел переменил патриарший престол на тесную келью инока. И поскольку душу ее занимала одна великая мысль, чтобы немедленно восстановить Православие, то императрица весьма беспокоилась, что этому делу надлежало остановиться по случаю избрания другого патриарха. Благочестивая государыня сама с сыном своим прибыла в монастырь, где находился патриарх, и уговаривала его возвратиться на оставленный престол. Но Павел на все ее просьбы с тяжким воздыханием ответствовал: «Государыня! Церковное смущение принудило меня оставить престол патриарший. Сама ведаешь, сколько болезнует Церковь ересью иконоборной, от долговременного лжемудрствования она получила неисцельную язву. И я, злополучный, я сам увязнул языком и рукой в сетях зловерия, и это уязвляет душу мою безмерной печалью. Ныне скипетр самодержавия дан от Бога в руки благочестивые, но я, не имея сил и способности быть споспешником твоих благих намерений, лучше хочу сойти во гроб, нежели возвратиться на степень священноначальника. Благочестивая государыня! Ты имеешь искусного и неустрашимого помощника в твоих царских палатах. Тарасий, первый ваш советодатель, достойнее меня престола патриаршего. Разум его силен рассеять еретические заблуждения. Облеки его в сан святительский и вместе с ним избавь от скорби матерь нашу, Церковь Христову. Из этой пустыни я буду видеть и радоваться, что мое место заступил достойный подвижник».

Благочестивая царица и сын ее Константин, увидев твердое намерение Павла не принимать престола патриаршего, положили в совете возвести на оный Тарасия, первым делом которого был Седьмой Вселенский Собор.

\section{Наказанный оскорбитель святыни}

Юлиан, дядя Юлиана Отступника, нарочно отправленный им в Антиохию, чтобы восстановить идолослужение, начал исполнять волю своего государя разграблением соборной церкви, которая по своему великолепию и богатству называлась «Златой». Все церковнослужители от страха разбежались, кроме святого Феодорита\footnote{Память св. священномученика Феодорита, пресвитера Антиохийского (†~361~-- 363), празднуется 8~(21) марта.}, бывшего церковным сосудохранителем, который, не успев сберечь сокровищ, принадлежащих Богу, не хотел беречь и сам себя. Этого"=то Юлиану и хотелось. Он отнял у Феодорита ключи, а самого его связал и бросил в темницу и тогда пошел с царским сокровищехранителем Феликсом и с воинским отрядом в храм Господень. Как тать и разбойник, он бросал в груду священные сосуды и оклады с ликов Господних. Потом сел на эту груду сокровищ, с ругательством повторяя несколько раз: «Пойди сюда, галилеянин, защити Твое богатство!» Один из бывших тут христиан, по имени Евзой, напомнил этому безумцу, чтобы он, если решился быть святотатцем, по крайней мере, воздержался от столь ужасного бесчестия святыне Господней и хулы Сыну Божию. Злодей, ударив его по голове, отвечал: «Христиане не имеют о себе Промысла Божия, они изверги!» Поругавшись над святыней и осквернив алтарь Господень, Юлиан начал мучить служителя Божия Феодорита, чтобы он или возвратил имущества языческим богам (ибо Феодорит при равноапостольном царе Константине разорил несколько идольских капищ и на их месте соорудил церкви), или заплатил бы им поклонением и жертвоприношением. Но поскольку человек Божий отвечал на все требования Юлиана только обличением богоотступничества его и царя, то мучитель повелел отрубить ему голову.

Но Всевышний изрек праведный Свой Суд. Вскоре узрели на этом изверге руку Небесную, которая два месяца удручала его отчаянием, бешеным сумасшествием и нестерпимыми болезнями и, наконец, поразила бедственной смертью. Издыхающий отправил к Юлиану нарочного и заклинал его прекратить гонение. «Это неправедное дело, "--- писал он, "--- которое я, из угождения тебе, на себя принял, низвергло меня в столь лютое состояние». Равным образом и Феликс не избег казни Божией: вскоре изверг он душу свою в мучительных болезнях.

\section{Каков подвиг, такова и награда\footnote{Повесть прп. Иоанна Прозорливого, пустынника и затворника Египетского (†~ок. 395), память которого празднуется 27~марта (9~апреля).}}

Один старец благочестивыми подвигами столько угодил Богу, что каждый день ему предлагаема была пища невидимой рукой. Входя по вечеру в один из своих вертепов, он обретал мягкий и чистый хлеб и, когда ощущал, что плоть его требует подкрепления, поклонившись Богу, вкушал и опять начинал утешать душу свою песнопением и славословием Бога. Таким образом, пустынножитель день от дня возрастал в совершенстве человека по образу Божию.

Уже чудный старец не сомневался в будущем воздаянии и как бы в руках имел оное; а это и было причиной его падения. Ему начало приходить на мысль, будто он имеет у Бога благодати более всех отшельников и более всех других, а что всего опаснее, он возомнил, что отнюдь не может поскользнуться от столь высокого жития в добродетелях. От этого чрез некоторое время родилось в нем какое"=то уныние; от уныния возросла леность: старец начал ранее ложиться и позже вставать на псалмопение, и молитвы его были короче прежних. Он смущался, волновался мыслями и тайно от самого себя, часто помышлял нечто нелепое; хотя прежний навык иногда исторгал его от лености и смущения, но он опять низвергался в оные.

Однажды, войдя в вертеп после молитв, при закате солнца, он по обыкновению нашел посылаемый ему от Бога хлеб, но уже не такой чистый, как прежде; и хотя испугался этой перемены, однако, подкрепив свои силы, не отверг нечистых помыслов и даже находил в них услаждение. На другой день, также после обыкновенных, хотя и нарушаемых помыслами молитв, он обрел хлеб свой, но черный и черствый. Взалкавший\footnote{\textit{Взалкавший} "--- проголодавшийся.} старец, вкушая, тужил духом, но вместо того, чтобы в изменяющемся хлебе узреть образ своей изменяющейся души, он умножил пагубные помышления и сам, кажется, старался согревать их, как змею в недрах. Однако же и на третий день, хотя с трудом, он совершил «правило» и вечером пошел в вертеп. Но как ужаснулся он, когда увидел одни остатки хлеба, раздробленные и по земле разбросанные, зеленью покрытые, и пылью оскверненные!

Старец вздохнул и прослезился, но не столько сокрушился в сердце своем, сколько бы довлело к обузданию страстей, на него воюющих. Он собрал валяющиеся крохи и, немного вкусив, лег уснуть. Вдруг нашло на него облако гнуснейших помышлений, влекущих его из уединения в мир, тягота их давила его и не давала ему прийти в чувство. Наконец, не помня сам себя, он встал и ночью пошел глухой пустыней, желая достигнуть жилищ человеческих, чтобы свергнуть там с себя одежду ангельскую...

(Продолжение в следующей статье).

\section{Спасительная встреча}

Погиб бы сей, столько прежде Богом возлюбленный старец, если бы Божие милосердие, которое и прежде, безмолвием изменяющегося хлеба, велегласнее грома вещало ему: «Познай себя!», другим способом не возвратило его на путь спасения. Эту перемену произвел следующий случай.

Старец всю ночь шел по глубоким пескам и не чувствовал усталости, но когда наступил день и солнце начало обливать его зноем, он утомился и изнемог: смотрел туда и сюда в поисках какого"=нибудь монастыря, чтобы, уклонившись с пути, найти в нем отдых. По смотрению Божию вскоре так и случилось. Он поспешно вошел в монастырь и встречен был братией с почтением и любовью. Как великому отцу, они умыли ему ноги и предложили трапезу; когда же старец укрепился пищей, просили его, дабы сказал им «слово спасения». Идущий к погибели путник не отрекся и начал отечески увещевать их, да будут крепки и постоянны в подвигах, яко трудники, вскоре надеющиеся получить от Христа успокоение. Он говорил им о постничестве, о молитве, о смиренномудрии, о великой науке знать самих себя, словом "--- обо всех добродетелях, от которых отрекся сам. Спасительная беседа продолжалась далее полуночи, и братия, отходя ко сну, благословляла его, как бы одного из древних святых учителей, и благодарила Бога за этот дар, внезапно им ниспосланный.

Когда путешественник, оставшись один, лег уснуть, вдруг объял его некоторый страх. Он начал размышлять "--- как, других научая, о себе небрежет, других наставляет на путь спасения, а сам идет в бездну погибели. Размышляя так, он воспрянул с одра и, не простившись с гостеприимной братией, в ту же минуту скрылся. Он не шел уже, но бежал на прежнее свое место, рыдая о своем падении. «\textit{Аще не Господь помогл бы ми}, "--- задыхаясь и препинаясь от усталости, говорил он, "--- \textit{вмале вселилася бы во ад душа моя}» (Пс. 93, 17).

Ни дальний путь, ни топкие пески, ни рыкающие в темноте ночной звери, ни зной от дневного светила, ни сама слабость сил не могли преклонить его к отдохновению. Он возвратился в свою келью, пал на землю и, посыпая голову свою перстью, плакал и рыдал несколько дней, пока голос Небесный не известил его, что Господь принял его покаяние. Но хлеб, ниспосылаемый ему прежде от Бога, уже отнят был у него навсегда, и старец с того времени искал себе пропитания от труда рук своих.

Христиане! Если бы милосердый Бог не указал несчастному старцу святой обители, то вихрь мирской жизни увлек бы его невозвратно, если бы Бог не вложил в сердце гостеприимной братии требовать от него «слова спасения», то не умилилась бы душа его. Да веруем, итак, что Нехотящий смерти грешников употребляет все возможные средства, \textit{еже обратитися} нам. Мы сами виновны, что не употребляем надлежащего внимания, чтобы их увидеть и воспользоваться ими. Все, что мы по неразумию или по одной привычке называем случаем, есть не что иное, как действие благодеющего Промысла.

\section{Свыше вдохновенное любопытство}

Восемнадцатилетний отрок, по имени Астион\footnote{Память преподобномучеников Астиона монаха и Епиктета пресвитера празднуется 7~(20) июля.}, сын знатных родителей, однажды с приятелями вышел для прогулки за город.

Наслаждаясь там благорастворением воздуха, они отходили далее и далее от города и, наконец, незаметно очутились в лесу пред низкой и ветхой хижиной. Вдруг Астиона объяло какое"=то тайное желание узнать, кто тут обитает. И хотя товарищи говорили, что это "--- жилище дровосека или лесного сторожа, но Астион (так сердце его возбуждала благодать Божия!) непременно хотел войти в хижину. Спутники рассмеялись над его любопытством и сели на траву. Астион постучался в дверь и немедленно был встречен достопочтенным старцем, который принял его ласково, посадил подле себя и начал спрашивать, откуда он и чей сын. Астион, объявив о себе и о своем роде, примолвил, что родители чрезвычайно любят его. Тогда старец сказал: «Они и должны любить тебя больше, нежели родительской любовью, ибо блаженная душа твоя любезна и Самому Христу, Спасителю нашему, Который избрал ее, как вижу, Себе в служение». Астион, хотя был воспитан в идолопоклонстве, но при имени Иисуса Христа оказал благоговение. Тогда обрадованный Епиктет (имя старца) отверз свои уста и начал вкратце объяснять: какой был предвечный совет Триипостасного Бога о человеке; как Сын Божий воплотился и пострадал; какие жизнь и блаженство ожидают истинных богопочитателей; как, любя своих родителей по плоти, больше всего должно любить Отца Небесного; как Христова Церковь крещением рождает в бессмертие чад своих. Благоразумный юноша слушал с великим усердием, всему веровал, дал клятву быть почитателем Христовым и расстался с мудрым Епиктетом не прежде, как спутники начали его кликать.

На другой день Астион, уже один, опять посетил старца и с того времени каждое утро и каждый вечер ходил в лес, будто для прогулки. Так мудрая пчела любит летать в те места, где надеется обрести сладкий мед. Когда же научился всем истинам веры Евангельской, тайно принял Святое Крещение. А чрез несколько лет блаженная душа Астиона, уже ликуя с Ангелами, имела радость видеть рабами Христовыми и своих родителей.

\section{Христианское общество, или разность между христианами и нехристианами в отправлении обязанностей, возлагаемых отечеством}

Когда Константин Великий воевал против Максентия, один из чиновников, которым поручено было образовать или пополнить легионы в Египте, между прочими молодыми людьми записал в военную службу также и двадцатилетнего Пахомия\footnote{Память прп. Пахомия (†~ок. 348), после нареченного Великим, празднуется 15~(28) мая.}. Новоизбранные воины, будучи отправлены морем, пристали к одному христианскому городу в Фиваиде, называемому Оксиринх, и поскольку во времена тогдашних императоров, беспрестанно оспаривавших владычество друг у друга, любовь к отечеству иссякла в римлянах (а Пахомий, сверх того, был иноземец), то молодые воины для предупреждения побегов содержались тут под крепкой стражей. Сам Пахомий, имея те же чувствования, имел ту же и участь. Но это послужило к его спасению.

Граждане оксиринхские, едва узнали о прибытии воинов из Египта, начали стекаться толпами, стараясь один пред другим оказать им всевозможную помощь: приносили пищу, белье, деньги. Между этими благодетелями Пахомий увидел подобных себе воинов, только что взятых на службу. Будучи записан в войска против воли, он долго удивлялся, почему обременяющие его узы не есть участь начинающих военную службу граждан оксиринхских, и, к вящему изумлению, услышал, что они, как христиане, почитают за величайшее преступление исполнять по принуждению те обязанности, которые возлагает на них отечество. Имя христиан, о которых Пахомий прежде слышал столь много чудесного, их готовность проливать кровь свою за власть предержащую "--- все это побудило Пахомия войти в подробные расспросы о жителях оксиринхских, ибо благодать Божия начала приуготовлять его к совершеннейшему познанию истины. Пахомий узнал, что в Оксиринхе еще до Константина Великого, во времена гонений на христианство, все идольские капища были обращены в храмы Христовы, которые составляли первое украшение города. Стены и башни наполнены были живущими в них святыми отшельниками, которые неусыпно молились о благосостоянии отечества и поучали народ вере и добродетели. Не было в нем ни одного еретика, ни одного идолопоклонника. Преступления были столь редки, что наказания знали только по слуху. Готовность служить отечеству, в каком бы звании ни случилось, была столь пламенна, что оксиринхцы крайне удивились, видя молодых воинов, охраняемых, подобно невольникам, и несущих свои обязанности \textit{не за совесть}, но за единый \textit{страх}. Начальство имело в городских воротах нарочную стражу, которой поручено было принимать всех пришельцев и сопровождать в общественные гостиницы, где питали их и упокоивали (предоставляли им покой). Граждане, ревнуя странноприимному правительству, спешили один пред другим оказать ту же добродетель и тот день почитали праздником, в который введут какого"=либо пришельца в дом свой. Общее братолюбие было столь велико, что весь город казался одним великим семейством. Но при этом множестве святых иноков и святых дев, при непрестанном упражнении в христианских добродетелях город был богат и цветущ, давал лучших воинов отечеству и более прочих вспомоществовал в нуждах государственных. Таковы плоды истинного христианства!

Пахомий возлюбил веру Евангельскую, в сердце его воспламенился страх Божий, душа возрадовалась об имени Христовом. Уединившись от своих товарищей, он простер руки свои к небу и дал нерушимый обет всегда ходить по заповедям Христовым.

С того времени Пахомий соблюдал себя от всех пороков, которым иногда подвергаются военные люди, и храбро и усердно служил на поле брани во всю войну с Максентием. Когда же Константин восторжествовал над этим мятежником и распустил некоторую часть войска, Пахомий возвратился в свое отечество и принял Святое Крещение.

\section{Сила молитвы}

В одно время в Мелитине Арменской была чрезвычайная засуха, которая угрожала неминуемым голодом, и сетование жителей день ото дня умножалось. Наконец, все прибегли к епископу своему, преподобному Акакию\footnote{Прп. Акакий, епископ Мелитинский (†~ок. 435), жил в царствование Феодосия II, в начале V~в. Память его празднуется 17~(30) апреля.}, чтобы он умолил Бога явить к ним милость Свою. Святитель, собрав народ, уже начавший алкать, пошел к церкви святого великомученика Евстафия, бывшей вне города, умоляя страдальца Христова участвовать в их молитве и вместе с ними испросить у Бога дождь иссохшей земле. Избрав близ церкви лучшее местоположение, он велел принести и поставить Божественный престол и на открытом поле, без крова, начал совершать бескровную жертву, возводя к небу слезные очи. Уповая на всемогущество и милосердие Божие, Акакий не растворил вино водой, но, ум свой вперив в Бога, прилежно молился, да Он Сам свыше растворит дождем Святую чашу и вкупе напоит иссохшую землю. Эта молитва столь сильна была у Бога, что немедленно пролился великий дождь и растворил вино водой, земную стихию "--- стихией водной, а сердце народа "--- радостью.

\section{Раскаяние одного и незлобие другого}

Когда Ирод осудил за проповедь Евангельскую на смерть святого апостола Иакова Зеведеева\footnote{Св. апостол Иаков, брат св. апостола и евангелиста Иоанна Богослова, был вторым по св. архидиаконе Стефане мучеником.}, тогда иудеянин, по имени Иосия, один из оклеветавших его, видя мужество и дерзновение по Господе святого Иакова, познал истину его учения, уверовал во Иисуса Христа и всенародно исповедал воплощение, вольную страсть и воскресение Мессии, за что и был осужден на смерть вместе с Апостолом. Когда привели их на место казни, Иосия, повергшись к стопам святого Иакова, умолял его, дабы простил ему грех, содеянный в неведении. Кроткий проповедник человеколюбия Евангельского обнял его и, облобызав, сказал: «Господь да благословит тебя!» После этого оба преклонили под меч главы свои.

\section{Молитва и смерть подвижника Христова}

Преподобный Марк Фраческий\footnote{Память прп. Марка Афинского (†~ок. 400), подвизавшегося в горе Фраческой, что в Ефиопии, празднуется 5~(18) апреля.}, живший девяносто пять лет в пустынном вертепе единственно для прославления имени Божия, узнав время своей кончины, просил Господа, дабы для его погребения послал он авву Серапиона. Молитва была услышана. Серапион, руководимый Ангелом, достиг Фраческой горы и был встречен преподобным Марком с радостью.

По взаимном целовании\footnote{После взаимного приветствия.} чудный пустынножитель рассказал святому Серапиону все происшествия своей жизни, борьбу души против страстей, на нее воюющих, и свое торжество над бесплотными губителями. Серапион ничего не видел, кроме Марка, ничего не слышал, кроме повести о его равноангельской жизни, и два дня показались ему одним часом. На другую ночь они пребыли без сна: один готовился к смерти, другой благодарил Бога, что сподобился узреть столь великого старца.

Совершив молитву, преподобный Марк сказал ему: «Велик для меня день этот! Велик паче всех дней живота моего! Днесь разрешается душа моя от плотских страданий и грядет упокоиться в обителях небесных, днесь почиет тело мое от трудов и болезней. Отхожу от временной жизни и всем остающимся на земле желаю спастись. Да спасутся постники, по горам и пустыням для Бога скитающиеся! да спасутся узники Христовы, за истину гонимые! да спасутся обители святые и церкви Господни! да спасутся священники, ходатаи к Богу о людях Его! да спасутся человеколюбцы, приемлющие странных, яко Самого Христа! да спасутся богатые, богатеющие о Господе, и бедные, Господа ради обнищавшие! да спасутся благочестивые цари, утешение народов своих! да спасутся градодержцы, правду милосердием растворяющие! да спасутся христолюбивые воины, за веру, царя и отечество кровь проливающие! да спасутся смиренномудрые и трудолюбивые подвижники! да спасутся все друг друга о Христе любящие! да спасутся чада Царствия Небесного, усыновленные Христу Святым Крещением! Спасена буди, вся земля и все, на тебе живущие!» Потом, взяв Серапиона за руку и облобызав, присовокупил: «Спасись и ты, возлюбленный о Христе брат!» Святой Серапион возрыдал и внезапно услышал глас Небесный: «Гряди, Марк! Гряди, верный раб Мой! Почий во свете и радости духовной жизни». Он устремил к небу взор свой и узрел душу Марка, возносимую руками Ангелов. Сердце его летело вслед за ней, пока чудное видение не сокрылось от очей его.

Так преставился к Богу святой Марк. Уже бездыханный, он еще стоял на коленах, простерши руки свои к небу. Святому Серапиону казалось, что угодник Божий еще молится, но охладевшее тело требовало погребения.

\section{Святое дерзновение}

Святой Василий Анкирский\footnote{Память священномученика Василия, пресвитера Анкирского (†~362~-- 363), празднуется 22~марта (4~апреля).}, который в царствование Константина мужественно боролся против арианской ереси, показал себя истинным подвижником веры и при Юлиане. Он ходил всюду, где только надеялся найти христиан, и с великим дерзновением всенародно и втайне ободрял их крепко держаться учения и благодати Иисуса Христа, соблюдать себя беспорочным от всех скверн языческих, а особенно остерегаться соблазнов, которые богоотступник предлагал им в богатстве и почестях. В одно время, нечаянно явившись на площади, где тогда язычники приносили торжественную жертву идолам, святой Василий при всем народе пал на колена и, воздев руки свои к небу, с сокрушением сердца молил Бога, да ни одного из христиан, кроме этих заблудших людей, не \textit{предаде} бы \textit{их Бог в неискусен ум, творити неподобная} (Рим. 1, 27), но паче и этих несчастных да приведет ко свету истины. «О Спасителю мира Христе, Свете незаходимый, "--- вопиял он, "--- верных сокровище, волею Отчею прогоняяй тьму и Его Духом вся наставляяй! Призри святым и страшным Твоим оком и виждь, яко вера Твоя потребляется паче всех времен от лица земли». Здесь его схватывают, влекут на мучения, но непобедимый воин Христов с веселым постоянством вытерпел ужаснейшую пытку. Когда в Анкиру прибыл по пути на восток император Юлиан, святой Василий был представлен ему и был осыпан от него сначала ласками, а потом ругательствами. На все это праведник, погрозив ему перстом, спокойно сказал: «\textit{Вскоре отъято будет у тебя царство, яко не помянул еси воздаяний Христовых, ни устыдился еси алтаря, имже спасен был еси от убийственные смерти, егда тя, осмолетна отрока суща, священное место сокры}»\footnote{Слова Четии"=Минеи. Юлиан во время убиения Констанцием его отца и дядей был сохранен в алтаре, по свидетельству некоторых, Марком Арефусийским.}. После этого воин Христов с радостным лицом пошел в темницу, где, пронзенный раскаленным железом, предал Господу чистую душу.

Таковой же пример дерзновения о Господе представляет святая Поплия (Публия)\footnote{Память св. Поплии (Публии) исповедницы, диаконисы Антиохийской (†~ок. 361~-- 363), празднуется 9~(22) октября. Эта благочестивая жена, пожив краткое время в супружестве, родила св. Иоанна Пресвитера, а овдовев, посвятила себя на служение Богу.}, диаконисса Антиохийской Церкви, которая, собрав нескольких христианских девиц, поучала их вере и благочестию и вместе с ними молилась об утолении свирепеющего идолопоклонства. Сколько раз ни случалось Юлиану проходить мимо ее дома, он всегда слышал пение: \textit{Идоли язык дела рук человеческих: подобни им да будут творящии я и вси надеющиися на ня} (Пс. 113, 12, 16). Раздраженный Юлиан, наконец, потребовал святую Публию пред себя и посреди народа приказал одному из оруженосцев дать ей несколько заушений\footnote{\textit{Заушение} "--- пощечина.}. Это бесславие она приняла за славу и на каждый удар отвечала молитвой: \textit{Да воскреснет Бог, и расточатся врази Его!} (Пс. 67, 2). Если богоотступник не повелел предать ее мучительной смерти, то это потому только, что постыдился излить гнев свой на слабую, как говорил он, женщину.

Святой Афанасий\footnote{Память свт. Афанасия, архиепископа Александрийского (†~373), празднуется 18~(31) января.}, архиепископ Александрийский, которого злоба Юлианова преследовала и в Египте, вынужден был уйти из Александрии. Восходя на корабль, он в утешение сопровождающих его друзей сказал: «Не страшитесь, чада мои! Юлиан есть только небольшое облако, которое пройдет очень скоро». А когда приметил, что преследователи, посланные за ним, уже приближаются, обратившись к устрашенным спутникам, сказал: «Возвратимся назад и пойдем смело навстречу нашим гонителям: вы увидите, сколь сильнее Тот, Кто за нас, нежели тот, кто на нас». Кормчий, ободренный дерзновением святого, поворотил корабль, и враги близ них проплыли мимо, как будто видя не видели или не имели до них никакого дела. Святой Афанасий возвратился в Александрию и жил сокровенно до смерти богоотступника.

Святой Феодор\footnote{Память мученика Феодора Антиохийского (†~IV~в.) празднуется 23~ноября (6~декабря).}, антиохийский юноша, был обвинен за благочестивую ревность. Когда христиане в безмолвной печали поднимали чудотворные мощи святого Вавилы, чтобы от соседства Аполлонова, как выражался Юлиан, унести их обратно в соборную церковь, то Феодор в самой Дафне\footnote{См. в IV части, под 4"~м числом сентября.} пел священные песни во славу Иисуса Христа и Его страстотерпцев. Это святое дерзновение сочтено было оскорблением величества языческих богов. От утра до вечера юный христианин лежал распростертый под ужаснейшими пытками. Его тело было истерзано, но подвижник веры не изменился даже в лице и с веселым духом и пламенной ревностью продолжал святое песнопение. Это беспримерное мужество изумило мучителей, и Саллюотий, градоначальник Антиохийский, сделал убедительное представление Юлиану, дабы перестал преследовать христиан, ибо их вера непобедима. После этого святой юноша был отпущен.

\section{Порок безобразит не только душу, но и самую наружность человека}

Один молодой инок, живший в келье преподобного Пафнутия\footnote{Прп. Пафнутий Боровский был внуком одного из татарских чиновников, называемых баскаками, который пришел в Россию в Батыевом войске и, тут оставшись, крестился. Прп. Пафнутий родился в селе Кудинове недалеко от Боровска, постригся в 20~лет, 20~лет исправлял разные монастырские службы, игуменом был 13~лет, в схимонашестве подвизался 27~лет. Умер в 1477~г. Память его празднуется 1~(14) мая.}, будучи по монастырскому делу в городе, увидел на гулянье разного состояния и пола людей. Еще не отвыкнув от мирских пристрастий, он остановился и, услаждаясь недозволенным ему зрелищем, с сожалением вспоминал дни, проведенные им в мире.

Таким образом, умедлив несколько времени, инок возвратился к святому старцу и, застав его за книгой, доложил о своем деле. Но Пафнутий, едва воззрел на него, вдруг познал нечистые помыслы, смущающие душу его. «Се человек! Уже не таков, как прежде», "--- сказал он и отвратил от него лицо свое. Изобличенный инок затрепетал и, видя, что старец не хочет с ним более говорить, вышел вон. Уже наутро, по совету одного престарелого брата, испросив у святого настоятеля прощение и приняв спасительные уроки воздержания и целомудрия, этот инок успокоился.

\section{Небесная помощь}

Благоверный князь Александр Невский, вступив с новгородским войском в Ингерманландию\footnote{Ингрия, Ижора, Ижорская земля (швед. \textit{Ingermanland}; фин. \textit{Inkeri, Inkerinmaa}; эст. \textit{Ingeri, Ingerimaa}; др. "--- рус. \textit{Ижера, Ижерская земля)} "--- этнокультурная и историческая область на северо"=западе современной России. Располагается по берегам Невы, ограничивается Финским заливом, рекой Нарвой, Чудским озером на западе и Ладожским озером с прилегающими к нему равнинами на востоке. Ее границей с Карелией считаются реки Сестра и Смородинка.}, чтобы очистить ее от шведов, в следующее утро намеревался дать битву. Один из воевод его, Филипп, муж богобоязненный, имея поручение ночью наблюдать за движением неприятелей, на заре утренней внезапно увидел плывущий по Неве корабль, посреди которого стояли святые мученики Борис и Глеб\footnote{Память благоверных князей"=страстотерпцев Бориса и Глеба (†~1015) празднуется 2~(15) мая.}, а гребцы сидели \textit{как бы одеяны мглою}. Тогда святой Борис сказал святому Глебу: «Любезный брат! Ускорим шествие, да поможем нашему сроднику Александру против неистовых шведов». Устрашенный и вместе обрадованный воевода немедленно возвестил о своем видении князя. Одушевленный Небесною помощью, Александр дал сражение, разбил врагов и, преследуя их по их же трупам до кораблей, ранил в лицо самого короля. После этого место битвы он назвал «победой» (виктория), а сам от современников и потомства был наименован Невским. Здесь"=то Петр Великий, во славу Бога, в Троице славимого, и во имя Александра Невского соорудил знаменитую обитель, где и по сей день почивают чудотворные мощи святого князя.

\section{Совершенство безмолвия}

Преподобный Арсений Великий ничего так не желал, как уединения, и без крайней нужды не показывался даже пустынножителям, почему авва Марк однажды спросил у него: «Отче! Почему от нас удаляешься?» "--- «Богу известно, что люблю вас, "--- отвечал Арсений. "--- Но не могу обращение с людьми предпочесть обращению с Богом, ибо горних сил тысяча тысяч обитают на Небеси, но все имеют одну волю и единодушно славят Бога, а на земле человеческие воли бесчисленны и помышления различны. Как же могу оставить Бога и жить с людьми?»

Установив такой образ жития, Арсений Великий по большей части оставался безмолвен, когда кто"=нибудь из братии или из мирян его посещали. Следующее происшествие покажет, как он поступал иногда с посетителями, как в мире о нем думали и сколько уединение и безмолвие прославили его у Бога. Некоторый брат пришел издалека в скит единственно для того, чтобы увидеть Арсения, и упросил одного из церковнослужителей, чтобы представил его к этому великому старцу. Когда они постучались у дверей его кельи, Арсений впустил их и сел безмолвно, потупив глаза в землю. Пришельцы также сидели и молчали. Прошло какое"=то время, и ни один из трех не отверз уст своих. Наконец, церковнослужитель, отозвавшись недосугом, пошел вон. Странник, посовестившись остаться один с безмолвным Арсением, также поклонился ему и вышел. Сожалея, что решился на столь дальний путь и потерял столько времени, а не мог услышать ни одного слова от Арсения, странник попросил церковнослужителя проводить его к преподобному Моисею. Какая разность в принятии и обхождении! Моисей встретил их с радостью, упокоил, угостил; подавал им наставления, сам требовал от них совета и, явив всевозможную любовь, напутствовал их благословением. Идя дорогой, церковнослужитель сказал страннику: «Ты видел ныне обоих, Арсения и Моисея; скажи же, который из них показался тебе лучшим?» "--- «Без сомнения, тот, который принял нас с братскою любовью», "--- отвечал странник. Этот разговор между ними услышал один старец и, восхотев узнать: безмолвие ли Арсения или собеседничество Моисея более угодно Богу, начал молиться и просить Небесного откровения. Через некоторое время он узрел в видении два корабля, плавающие по реке. В одном из них был Арсений, и Дух Божий в глубоком безмолвии управлял кораблем, в другом находился Моисей, и Ангелы Божии, управляя кораблем, в уста его влагали мед. Это откровение старец объявил прочим мудрым старцам, и все они рассудили, что безмолвствующий Арсений совершеннее, нежели Моисей, всех приемлющий, ибо с Арсением Сам Бог, а с Моисеем Ангелы Божии пребывают.

\section{Единочасное покаяние}

Одна богатая девица, родом египтянка, оставшись сиротой, какое"=то время жила богоугодно, но потом, познакомившись с развратными обоего пола людьми, несчастная впала в пороки, как голубица, запутавшаяся в сетях. Скитские отцы, которые знали добродетельных ее родителей, душевно сожалея о погибающей отроковице, положили твердое намерение спасти ее и поручили преподобному Иоанну Колову, наставнику Арсения Великого, идти к бедной девице и преклонить ее к покаянию. А сами, возложив на себя пост, начали молиться, да Господь поможет Иоанну. Великий старец при первом взгляде на отроковицу тяжко вздохнул и, преклонив главу, горько заплакал. Этот внезапный поступок, как стрела огненная, пронзил ее сердце. Ужаснувшись сама себя, она спросила: «Авва! Есть ли грешникам покаяние?» "--- «Есть! Есть! "--- воскликнул Иоанн. "--- Спаситель готов принять тебя в отеческие объятия, и я ручаюсь в том, что днесь же "--- только искренно обратись к Богу "--- возрадуются о тебе Ангелы, как о невесте Христовой». Выслушав это, отроковица сказала: «Буди воля Господня! Возьми меня, авва, с собой и отведи туда, где знаешь место, удобное к покаянию». Иоанн назначил безмолвнейшую из женских обителей, и девица тогда же оставила дом свой, не сделав никакого распоряжения о своем имуществе. Старец шел впереди; отроковица следовала за ним издалека. Когда они достигли пустыни, уже наступила ночь. Иоанн сделал из песку возглавие и сказал ей: «Почий здесь и спи, покрываемая самой благодатью Божиею». Он оградил ее крестным знамением и ушел на близслучившееся (рядом находившееся) возвышение, где, совершив свои обыкновенные молитвы, возлег на землю и крепко заснул. Пробудившись в полночь, Иоанн узрел блистание свыше и, возведши очи, увидел светлый, радужный путь, простирающийся от небес к спящей отроковице. Смотря на нее пристально, он узрел Ангелов Божиих, возносящих этим путем душу ее к небу. Иоанн смотрел дотоле, пока дивное видение не скрылось от очей его. Восстав, он пошел к ней "--- и обрел ее умершей. Старец повергся лицом на землю пред Богом, ибо его объял страх и трепет. Тогда был к нему глас свыше: «Покаяние ее, единым часом содеянное, приятнее Богу, чем покаяние тех, которые каются целые десятилетия, но такой горячности к Богу не являют». Иоанн пробыл в молитве до света утреннего и похоронил тело отроковицы. Вся Египетская пустыня и самый мир прославили милосердие Божие и заслуги Господа нашего Иисуса Христа.

\section{Мщение святого человека}

Однажды преподобный Епифаний, бывший после архиепископом Кипрским, странствуя по пустыне, встретился с толпой сарацинских разбойников. Варвары, вообще ненавидя иноков, начали над ним ругаться, а один из них (одним глазом слепой) даже обнажил меч; но едва взмахнул им, как вдруг открылось у него невидевшее око. Варвар поверг на землю меч и в изумлении показывал своим товарищам исцелевший глаз. Видя внезапное чудо, разбойники переменили свои ругательства на благоговение. «Будь нашим хранителем, "--- сказали они, "--- ходи с нами и защищай нас от случающихся бедствий». Сколько Епифаний ни отказывался, они, при всем к нему уважении, увели его с собой насильно. Человек Божий, блуждая с ними три месяца, употребил это время во славу Божию и беспрестанно напоминал им о будущих наградах и наказаниях, так что наскучил сарацинам своими речами. Не в силах более терпеть его властных советов, они опять начали умолять его, чтобы оставил их и возвратился на свое место. Варвары проводили его туда, откуда взяли, и не прежде его оставили, как соорудив ему без всяких его просьб келью и дав клятву впредь жить добропорядочно. А получивший исцеление ока уверовал во Христа и, оставшись при Епифании, сделался учеником его. Нареченный Иоанном, этот свирепый варвар после был первым проповедником чудотворений Епифаниевых и писателем его жизни. А что исправились в житии своем и прочие, это подтверждается тем, что они неоднократно приходили к святому Епифанию для благословения и, когда умножились его ученики, построили ему небольшой монастырь.

\section{Ужасный долг, исполняемый с милосердием}

Когда святой мученик Александр\footnote{Память св. мученика Александра Римского (†~начало IV~в.) празднуется 13~(26) мая.} приведен был на место усекновения, то исполнитель неправедного приговора, Целестин, сквозь слезы сказал ему: «Страдалец Христов! Помолись Богу твоему, да не поставит мне в грех смерть твою, я исполняю повеление начальства». Святой Александр отвечал на это: «Ты делаешь это не от твоего произволения, поэтому весь грех будет на душе повелевшего, а ты исполняй, что тебе предписывает твоя должность\footnote{Эту должность тогда исполнял кто"=либо из воинского отряда, на этот случай назначаемого.}. Я спешу к моему Господу». Тогда Целестин завязал ему глаза чистым убрусцем; но едва поднял меч свой, как узрел святых Ангелов, внезапно приступивших к мученику, дабы принять душу его; он ужаснулся и стоял неподвижно. Страдалец, долго не получая удара, сказал: «Твори, брате, еже имаши творити». "--- «Раб Божий! "--- отвечал воин. "--- Я вижу чудных мужей, стоящих близ тебя». Тогда святой Александр воззвал к небу: «Господи Иисусе Христе, даждь мне в этот час скончатися». Ангелы несколько отступили. Целестин, приблизившись, исполнил ужасную обязанность, и святая душа мученика, в гласе хваления Божия, на руках Ангельских вознеслась на Небо.

\section{О том, как должно поступать в случае противоборства одной обязанности с другой}

Римские легионы при императорах составляемы были из разноплеменных народов, и если нужно было дополнять оные, то посылались нарочно воинские чиновники с указами, где кому набирать воинов. Таковое повеление некто Нумериан привез на остров Хийский. Едва узнали об этом жители, все молодые язычники скрылись или разбежались; одни христиане спокойно ожидали своей участи, и те, которые записаны были в воинство, без прекословия и роптания последовали за Нумерианом через пространство морей.

В этом числе был и святой Исидор\footnote{Память св. мученика Исидора (†~251) празднуется 14~(27) мая.}, уроженец александрийский. Не взирая на то, что военная жизнь всегда шумна и окружена соблазнами, этот богобоязненный юноша насколько был крепок и мужествен, столь же усердно облекся в оружие и, не оставляя поста, молитв и добрых дел, равно не оставлял своих обязанностей по службе. Не щадить последней капли своей крови за царя и отечество "--- был обет христолюбивого юноши, и пролить оную Исидор считал за единственную славу воина.

Но эта геройская кровь должна была пролиться за другого Подвигоположника. Едва святой Исидор начал отличаться усердием и ревностью по службе военной, объявили указ императора Деция, чтобы воины, как стражи и защитники отечества, непременно поклонялись богам, хранителям римского могущества. Святой Исидор, как ревностнейший из христиан, немедленно был оклеветан пред Нумерианом. Когда брали его под стражу, то чиновник, для этого отправленный, с укоризной сказал ему: «Безумно делаешь ты, что для какого"=то Христа пренебрегаешь служением царю и отечеству». Святой Исидор ответствовал на это: «Я хотел воздавать Божия Богови и кесарева кесареви, но когда воля правительства требует службы одному кесарю, то для меня гораздо лучше служить одному Богу. Я дал клятву служить в воинстве не иначе, как защищая отечество земное и Небесное. Теперь вижу, что моя кровь не нужна царю земному, пусть же она прольется за Царя Небесного». С этими чувствованиями святой Исидор предстал суду, ответствовал с дерзновением воина Христова и спокойно пошел на место казни.

\section{О том, сколько святые люди старались истребить чуть зарождавшиеся в сердце их пагубные страсти}

Преподобный Пахомий\footnote{Прп. Пахомий Великий (†~ок. 348) первый установил образ большого пострижения. Родом он был из Египта, где в царствование Константина Великого, в 315~г., на месте, именуемом Тавенисия, создал монастырь и начальствовал над 7~тысячами монахов. Память его празднуется 15~(28) мая.}, чрез Ангела приняв повеление Божие основать Тавенисиотскую обитель для имеющих впредь собраться пустынников, начал с братом своим Иоанном строить кельи, но вскоре возникло между ними некоторое разногласие. Пахомий хотел занять под свой монастырь много больше пространства, а Иоанн, любя безмолвие, желал заключиться в тесной ограде и после дружеских с той и другой стороны пререканий сказал Пахомию: «Перестань величаться и расширять себя. Таковые поступки непотребны в очах Бога». Пахомий, почувствовав всю горечь этой укоризны (впрочем, не столь благоразумной, если вспомнить повеление свыше), прогневался на Иоанна, однако, по своей кротости и из почтения к старшему брату, не стал возражать: он только оставил его, неся негодование в сердце. Но в следующую ночь почувствовал и в этом поступке столько неправды, что затворился в своей келье и начал плакать, в молитве исповедуясь Богу: «Горе мне, что я, вопреки Твоих заповедей, о Агнче, \textit{прямо стригущего} Тебя \textit{безгласный} (ср. Ис. 53, 7), возымел гнев на моего брата, возвестившего мне истину. Еще есть в сердце моем лжемудрие плоти! Столько времени обучаясь жить духовно, еще терзаюсь досадой! Помилуй меня, Господи, да не погибну до конца. Если благодать Твоя не утвердит меня, враг обрящет в сердце моем несколько приятных ему деяний и поработит воле своей меня, преступника Твоих законов. Горе мне! Хочу научить тех, которых Ты, Господь и Бог мой, обещал чрез меня призвать в иноческое житие. А сам не могу научиться побеждать свои страсти». Вопия таким образом к Богу, блаженный Пахомий пребыл в молитве до другого дня; слезами и потом он облил землю, на которой стоял.

\section{Молитва мученика в час смертный}

Святой мученик Феодот\footnote{Св. мученик Феодот (†~303) был родом из Анкиры Галатийской; пострадал в царствование Диоклетиана. Память его празднуется 18~(31) мая.} пришел на место казни, повергся на колена пред народом и сопровождавшими его сродниками и излил к Богу следующую молитву: «Господи Иисусе Христе!

Благодарю Тебя, что удостоил раба Твоего быть гражданином Небесным и участником Твоего Царствия, даровав мне силу победить змия и стереть главу его. Но умоляю Тебя, сотвори верным чадам Твоея Церкви облегчение от належащей скорби, да скончается на мне гонение, воздвигаемое на святую веру от нечестивых. Дай мир царству благодати». Потом, обратившись к рыдающим своим братьям, продолжал: «Не плачьте обо мне, братия мои, но паче прославьте Господа, укрепившего меня на этот священный подвиг. Совершив оный, я буду иметь на Небе дерзновение молить о вас Бога». Сказав это, святой мученик Феодот преклонил под меч главу свою.

\section{Злой совет обращается на главу советодателя}

Когда святой мученик Фалалей\footnote{Св. мученик Фалалей (†~ок. 284) умер в царствование Нумериана. Память его празднуется 20~мая (2~июня).} мужественно вытерпел все мучения, какие могла вымыслить злость человеческая, и мучитель Феодор был в недоумении, какой бы казнью утомить мужество страдальца, тогда один из стоявших тут жрецов, по имени Урвикий, дал совет, чтобы предать его на съедение зверям. Феодор согласился и тогда же приказал начальнику зверинца сделать нужное для этого распоряжение, а святому Фалалею, отсылая его на смертное позорище, сказал: «Или принеси жертву хваления богам, или плоть твою пожрут звери». Но мученик отвечал: «Пророчески объявляю тебе: \textit{не умру, но жив буду и повем дела Господня} (Пс. 117, 17)» "--- и пошел спокойно.

Немедленно выпущена была лютая медведица, но, к общему изумлению, она легла у ног его и начала лизать оные. Скрежеща зубами от досады, Феодор приказал выпустить льва и львицу, но и они стали пред мучеником, как смиренные агнцы. Тогда весь народ воззвал велегласно: «Велик Бог христианский!» "--- и, схватив Урвикия, поверг его посреди этих зверей, которые мгновенно и растерзали его. Устрашенный Феодор безмолвно удалился в дом свой, чтобы не подвергнуться той же участи.

\section{О том, сколь иногда опасно дело благотворения поручать другим}

Один бедный человек принес преподобному Феодору\footnote{Прп. Феодор Освященный (†~368), ученик Пахомия Великого. Память его празднуется 16~(29) мая.} немного луку, за который просил его заплатить чем"=нибудь из съестных припасов. Феодор велел ученику своему заплатить мукой. Но так как мука лежала в двух сосудах, в одном чистая, а в другом неочищенная, то ученик насыпал меру последней. Святой старец, заметив это, воззрел на него печально и укоризненно, чего молодой инок так испугался, что выпустил из рук и рассыпал муку. Потом, повергшись перед ним на колена, просил отпустить ему вину. «Встань, "--- сказал преподобный Феодор, "--- ты не столько виноват, не исполнив мою волю, сколько я, препоручив тебе то, что бы должно сделать самому». После этого наполнил полу свою лучшим хлебом и отдал бедному продавцу вместе с луком.

\section{Средство извлекать из обыкновенных разговоров пользу для души}

Преподобный Иоанн (Колов) не любил разговаривать о том, что происходит в мире. Некоторые из братии, желая испытать такую его решимость, однажды пришли к нему и всемерно старались вовлечь его в разговор. Но, о чем говорить ни начинали, авва молчал и не позволял мыслям своим веять над предметами, для него посторонними. Наконец, иноки сказали: «Благодарим Бога, что в нынешний год довольно было дождя, пальмы были увлажняемы надлежащим образом и принесли плод. Будет чем заняться братьям осенью». "--- «Так и Дух Святой, "--- сказал им старец, "--- если снидет на сердце человеческое, оно обновляется, возрастает в страхе Божием и приносит плод истинных добродетелей».

\section{Пост и трапеза страннолюбия}

Преподобный Кассиан\footnote{Прп. Кассиан Римлянин (†~435) жил в царствование Феодосия II. Память его празднуется 28~февраля (13~марта) (переносится с 29~февраля и совершается в этот день через три года в четвертый).}, будучи весьма гостеприимно принят одним египетским старцем, спросил у него: «Почему вы, принимая к себе странников, не наблюдаете поста, как делают у нас в Палестине?» "--- «Пост всегда при мне, "--- отвечал старец, "--- а вы всегда со мною быть не можете. Сверх того, хотя пост полезен и необходим, но он зависит от нашей воли, а исполнение любви возлагается на нас Законом Божиим и непременно требуется от нас. Итак, приемля в лице странников Христа, я должен с усердием и попечительностью ходить за ними. Когда же отпущу их, могу вознаградить потерю усилением поста».

Преподобный Макарий Египетский в дружеских беседах с иноками держался следующего правила: «Когда есть вино, пей "--- для братии и за одну чашу "--- один день не пей воды». Старцы иногда угощали Макария вином, и Макарий всегда принимал оное, чтобы после удручить себя жаждой. Но ученик его, зная поведение Макария, обыкновенно говорил им: «Ради Бога, не подносите моему наставнику вина, в противном случае он в келье своей удручит себя до смерти». Братия, узнав это, перестали угощать Макария. Но, когда приходили к нему, Макарий не изменял своего правила: угощал братию и, дабы угощение было чистосердечнее и для них приятнее, вкушал и сам с ними.

\section{Инок XIV века}

Преподобный Сергий, Радонежский чудотворец, будучи двадцати лет от рождения, в 1337~году удалился в непроходимый лес, как человек вовсе незнаемый. Там почти три десятилетия протекли в постничестве, молитвах, богомыслии. Служа Единому Богу, святой Сергий не думал, что чрез это приготовляет себя на служение отечеству. Но вдруг глас государя и первосвятителя вызывает его из уединения. В 1365~году великий старец идет от Димитрия Иоанновича Донского к Борису Константиновичу Нижегородскому, чтобы преклонить его к миролюбию, и за упорство этого честолюбца, по власти, данной ему от митрополита Алексия, возлагает запрещение на все княжество. В 1380~году, при всеобщем смущении и ужасе, старец своим напутствием и предсказанием победы одушевляет великого князя отражать Мамая оружием, а не дарами, как советовали прочие. И этим дерзновенным упованием на Бога столько устрашает Олега Рязанского, что этот предатель отечества, равно как и его союзник Ягелло Литовский, не смели соединить свои войска с полчищами Мамаевыми. В 1385~году как посол великого князя и богодухновенный учитель он так низлагает гордый и враждебный дух того же Олега, более всех нарушавшего общее спокойствие отечества, что этот из непримиримого врага становится искренним другом и союзником. Все эти подвиги святой Сергий увенчивает отречением своим от митрополитского сана, который предлагал ему святой Алексий.

\section{Урок родителям и детям}

Известно, что по страдальческом преставлении святого Иоанна Златоуста Церковь сопричислила его к лику угодников Божиих, и сами неприятели Иоанна, вскоре признав свою несправедливость, начали ублажать \textit{память праведного с похвалами} (Притч. 10, 7). Один святой Кирилл, архиепископ Александрийский\footnote{Память свт. Кирилла, архиепископа Александрийского (†~444), празднуется 9~(22) июня.}, племянник патриарха Феофила, первого врага и гонителя Иоаннова\footnote{Златоуст сильно вооружался против сребролюбия, тогда как Феофил был обладаем этим демоном не менее супруги Аркадиевой. Вот причина ненависти.}, не переставал гневаться на усопшего святителя. Грех подумать, чтобы сердцем человека Божия обладала злоба; но некоторое наследственное неудовольствие и боязнь, дабы не посрамить память дяди, уничижив бывший под его председательством собор против Иоанна, удерживали Кирилла отрясти порок от своей святости. Столько"=то справедливо, что и святые люди подлежат слабостям человеческим: Един Бог совершен, а из смертных "--- никто.

Святой Исидор Пилусиотский, сродник Кирилла, душевно сокрушаясь, что святой враждует со святым и тем заслуживает гнев Божий и негодование всей Церкви, весьма часто писал к нему, увещевая не уничижать славы Иоанна, невинно пострадавшего от его дяди и уже не обретающегося посреди живых. В одном из этих писем святой Исидор изъясняется следующим образом: «Устрашают меня события, изображенные в Божественном Писании, и принуждают писать к тебе всегда о едином. Если я, недостойный Исидор, "--- твой отец, как ты меня нарицаешь за старость лет, то боюсь осуждения, какое приял Илий, ветхозаконный священник, за то, что не наказывал согрешающих сыновей своих. Если же я, как и сам думаю, твой сын, то страшусь, да не постигнет меня казнь, постигшая Иоанафана, сына Саулова, за то, что отца своего, ищущего волшебства, мог отвести от греха и не отвел. Итак, не гневайся, что пишу к тебе и всегда буду писать о том, что служит на пользу души твоей. Я должен делать это, да не буду осужден, и ты, владыка, чтобы не быть тебе также осужденным от праведного Судии, послушай меня, отец ли я твой по старости лет моих или сын по твоему архипастырскому сану! Отложи гнев твой на усопшего праведника, да не смутишь живую Церковь и да не соделаешь в ней раздора».

Таковыми и тому подобными убеждениями святой Исидор Пилусиот довел Кирилла до того, что он, как бы опомнившись от греха своего, собрал всех египетских епископов и составил духовный праздник в честь Иоанна Златоуста.

\section{Видение о святых людях\footnote{Из жития св. мученика Евтропия чтеца (†~ок. 404), память которого празднуется 16~(29) июня.}}

Епископ по имени Сисиний в сонном видении узрел некоего мужа ангелоподобного, осияваемого Небесным светом, с удивлением и сетованием вещавшего: «\textit{Весь град нарочно обыдох ища добродетельных мужей и посреде толикаго множества человеков возмогох обрести единого доброго мужа Евтропия}». Сисиний, воспрянув ото сна, немедленно призвал бывшего при нем пресвитера и, рассказав ему видение, повелел искать и спрашивать по всему Царьграду, кто таков Евтропий. Пресвитер исполнил волю епископа и, к изумлению своему, услышал, что некоторый чтец, по имени Евтропий, быв мучен, как злодей, пред всем народом, ввержен в общественную темницу. Он пошел туда, увидел мученика, припал к нему со слезами и, беседуя с ним, узнал, что он из числа тех невинно истязуемых, которым, как единомышленникам святого Иоанна Златоуста, приписывают пожар, ниспосланный гневом Божиим за его изгнание\footnote{В день изгнания Иоанна в Царьградской соборной церкви внезапно разлился пламень и, прорвавшись сквозь верхние окна, окружил весь храм и все пожег. Потом вдруг поднявшимся ветром огонь переброшен был через торжище на дом Верховного Совета и с ним истребил множество зданий. Достойно замечания, что этому бедствию более всего подверглись дома врагов Иоанновых.}. Пресвитер, возвратившись к епископу, объявил ему, кто таков Евтропий, и удивленный Сисиний, пролив слезы сострадания о святом мученике, возблагодарил Бога, дарующего крепость рабам Своим.

\section{Средство усмирять жесточайших врагов}

В обители преподобного Кирилла Белоезерского\footnote{Память прп. Кирилла, игумена Белоезерского (†~1427), празднуется 9~(22) июня.} был инок, по имени Феодот, который, сам не зная за что, возненавидел святого настоятеля и в этой пагубной страсти так ожесточился, что не мог равнодушно не только видеть, но и слышать его голоса. Сколько другие иноки ни увещевали его, сколько ни доказывали, что святой Кирилл достоин всей их любви и даже благоговения, но Феодот не мог или, лучше сказать, не хотел исцелить от лютого недуга свою душу и, наконец, решился оставить монастырь. Однако же, послушав одного уважаемого им старца, прежде пошел к преподобному Кириллу, чтобы исповедать перед ним смущение своих мыслей.

Простой упрек настоятеля, может быть, произвел бы то, что ослепленный старец излил бы пред ним всю свою ненависть и пошел бы из обители на погибель души своей. Но святой Кирилл, приняв его с отеческой любовью, совершенно низложил ею своего неприятеля. Устыдившись седин напрасно оскорбленного и столь кроткого начальника, Феодот не знал, что сказать, и хотел выйти вон. Тогда прозорливый Кирилл, взяв его за руку, сказал: «Возлюбленный о Христе брат! Все обманулись и погрешили, почитая меня добрым человеком; ты один истинно судил, познав мои грехи и злобу. Но уповаю на милость Господа своего, что Он поможет мне исправиться, а ты прости мне досады и оскорбления и помолись обо мне к Тому, \textit{Иже не хощет смерти грешника} (ср. Иез. 33, 11)». Инок, глубоко пораженный прозорливостью и смирением человека по сердцу Божию, пал к ногам его и со слезами признался, что напрасно ненавидел его. С того времени Феодот обрел покой своему сердцу и начал более всех любить святого наставника.

\section{Добрым намерениям содействует Бог}

Святой Вассиан\footnote{Память свт. Вассиана, епископа Лавдийского (†~409), празднуется 10~(23) июня.}, друг великого в святителях Амвросия Медиоланского, будучи сыном идолопоклонника и воспитанным посреди таковых же родственников в доме, исполненном всеми обрядами кумирослужения, принял христианскую веру в такое время, когда этого, по"=видимому, нельзя было ожидать от юноши. Ибо Вассиан крестился в Риме, куда отправлен был единственно для того, чтобы усовершенствовать себя в науках и лучше узнать большой свет. Но благими намерениями управляет Сам Бог: человеку только стоит приложить к ним свое сердце. Эту истину доказывает пример Вассиана, ибо когда он вошел во святую купель, то увидел прекраснейшего юношу, сиявшего, как солнце, державшего белую одежду, приуготовленную для новокрещаемого. Блаженный Вассиан ужаснулся, но, собрав силы, спросил, кто он и откуда. «Аз давно с небесе послан есм к тебе твое святое намерение управити, и яже суть противная, та отгнати от тебе», "--- отвечал Ангел Божий и стал невидим. Благоухание наполнило храмину крещения, так что все тут бывшие думали, что они не на земле, но посреди жилища райского.

\section{Цель путешествия}

Старец Пафнутий, путешествуя по дальней пустыне с тем намерением, чтобы посетить всех обитавших там отшельников, наконец пришел к преподобному Онуфрию\footnote{Этот великий старец жил в IV~в. Память прп. Онуфрия Великого (†~IV~в.) празднуется 12~(25) июня.}. Любуясь здесь веселым местоположением вертепа и садом, который изобиловал разными плодами и орошаем был чистым источником, Пафнутий возжелал остаться при чудном пустынножителе, чтобы наслаждаться его беседами, а после быть наследником столь приятного уединения. Но мудрый старец сказал ему: «Не для того послал тебя Бог в эту пустыню, чтобы навсегда в ней безмолвствовать, но для того, чтобы, увидев рабов Его, возвратиться в мир и поведать их добродетельное житие в пользу слушающих и во славу Христа Бога нашего. Продолжай же твое путешествие не для себя одного, но и для ближних, подобно пчеле, которая, собирая с разных цветов мед, питается им сама и услаждает других».

Преподобный Пафнутий в точности исполнил совет великого пустынножителя и, созерцая дивное величие Божие на святых Его, все, что видел и слышал, старался всю жизнь употреблять во спасение других.

\section{О том, до какой степени злочестие иногда ожесточается против добродетели и не видит истины}

Греческий царь Феофил, упорный иконоборец, возвратил из заточения святейшего патриарха Мефодия, великого поборника Православия\footnote{Память святителя Христова Мефодия, патриарха Константинопольского (†~847), празднуется 14~(27) июня.}, единственно ради того, чтобы иметь ученого и мудрого собеседника. Ибо Феофил, любя науки, намеревался восстановить в Царьграде опустевшие училища\footnote{В продолжение иконоборствующего столетия науки в Греции почти истребились. Еще Лев Исаврянин положил первое и ужаснейшее начало этому запустению. При церкви св. Софии было знаменитое многочисленное училище с огромной библиотекой, которая содержала до 300~тысяч книг. Но так как попечитель и профессора всей академии были противного с иконоборствующим царем мнения, то Лев решился на поступок, неслыханный в истории: во время полного собрания всех ученых, учащих и учащихся он приказал войску окружить это здание и со всех сторон поджег оное, так что никто не мог спастись оттуда.}, а святой Мефодий был в то время муж просвещеннейший и сколь ревностный защитник христианской веры, столь же мудрый советник в делах государственных.

Наемник не преминул бы пожертвовать истиной для этой царской милости. Но Мефодий, как добрый пастырь, хотел воздать \textit{кесарева кесареви} обращением его на путь истины. Он опять начал пререкаться с иконоборцами и проповедовать поклонение святым иконам, отнюдь невзирая в праведном деле на гнев императора. Феофил, со своей стороны, хотя видел в этом первосвященнике преткновение своим злочестивым намерениям, но, имея в нем нужду, терпел до времени и, подавляя лютую досаду, наружно оказывал ему свое благоволение.

Между тем у греков загорелась война с сарацинами. Феофил, сам предводительствуя войсками, взял с собою и святого Мефодия, будто для молитвы и собеседования, но в самом деле по одному только подозрению, чтобы любимый народом архипастырь в его отсутствие не воздвиг возмущения против иконоборного правительства. Преданный \textit{в неискусен ум, творити неподобная} (Рим. 1, 27), воображал, что святые люди всего более способны быть мятежниками! Но это было только началом оскорбления избранных Божиих: вскоре обстоятельства, вместо обращения императора на путь истинный, умножили в нем ожесточение.

Агаряне разбили греческое войско, так что сам Феофил едва с малым остатком избавился от плена. Тогда"=то иконоборец обнаружил весь гнев на святого Мефодия и, обратив всю вину поражения единственно на него, с ругательством утверждал: «Из"=за того Бог отнял у нас победу, что имеем в воинстве идолопоклонство и поборника идолопоклонствующих» (разумея под этим поклонение иконам и святого Мефодия). Первосвященник с кротостью представлял, что Господь прогневался за Свое поругание на святых иконах и потому даровал врагам победу над ними. Что царь должен приписывать свое спасение малому числу избранных, находящихся в его воинстве, а если не могли они отвратить совершенного поражения греков, тому причиной заблуждение царя как первой особы, как главы их отечества. Святой Мефодий \textit{возвещал истину Духом Святым}; но ожесточенный Феофил наложил ему раны и осудил на ужаснейшее заточение.

Вот до какой степени ожесточается и бывает слеп нераскаянный грешник! Он приписывает гнев Божий таким делам, на которые Бог ниспосылает Свое благословение, и ждет благословения Божия на те дела, на которые давно излилась полная чаша Небесного гнева.

\section{Устав преподобного Паисия}

Святой старец Паисий\footnote{Память прп. Паисия Великого (†~V~в.) празднуется 19~июня (2~июля).} твердо определил и от юных лет до глубокой старости строго наблюдал каждому делу свое время. У него было время безмолвствовать, время говорить, время уединяться в своей келье, время выходить к братии и с ними беседовать. В безмолвии он богомысленным восхождением присвоялся Богу, а в беседах искал спасения ближних.

Неудивительно, что к этому великому наставнику благочестия и учителю добрых дел стекалось множество иноков и мирян, старцев и юношей, вельмож и простолюдинов. Как рой трудолюбивых пчел, привитающий цветоносному полю, все желали насытиться его духовным медом, а преподобный Паисий, радуясь, что каждый день умножается его пустыннолюбное общество, поступал не иначе, как истинный отец и начальник. Судя по духовному дарованию приходящих боголюбцев, Паисий иных отсылал к безмолвному пребыванию, чтобы наедине в молитвах беседовали с Богом, иным повелевал славословить в соборном храме. Иным "--- быть в разных послушаниях и вообще трудиться, иным повелевал учиться какому"=либо рукоделию, чтобы не только сами питались от трудов своих, но питали алчущих и дружелюбно упокоевали странников. Но главным и общим для всех правилом было то, дабы ни один из братии не творил волю свою, но все бы делал с ведома настоятеля или по рассуждению опытных и искусных старцев.

Надобно думать, что каждый человек увидит, сколько этот пример полезен для всякого общества и для всякого семейства.

\section{Мудрый совет о милостыне}

Один боголюбивый князь, благоговея к имени преподобного Паисия, пожелал оказать ему всевозможную щедрость: для этого, взяв с собой довольно серебра и золота, пошел сам в пустыню, где этот великий старец подвизался. В то время Паисий был на пути к другому великому старцу и встретился с князем. Князь спросил у него: «Где живет Паисий?» "--- «А для чего тебе надобен Паисий?» "--- возразил святой пустынножитель. «Я несу к нему серебро и золото, "--- сказал князь, "--- чтобы человек Божий сам имел все потребное и наделял других иноков». Тогда Паисий простер руки свои к небу и возблагодарил Бога, что в мире посреди всех сует еще находятся люди, не забывающие рабов Господних. Потом, обратясь к князю, отвечал: «Человеколюбивый христолюбец! Именем Божиим уверяю тебя, что пустынникам золото и серебро не нужны. Ты можешь скоро найти Паисия, но ни он, ни другие отшельники ничего от тебя не примут. Итак, возвратись без всякого прискорбия к жилищам человеческим. Здесь Господь только благословляет твое даяние, а там прострет Свою руку, чтобы принять оное. Там много найдешь убогих и нищих, недужных и разоренных, сирот и вдовиц: попекись о них, согрей, напитай, одень их "--- и приимешь воздаяние от Господа». Щедролюбивый князь, не зная, что это был Паисий, последовал его совету, возвратился в город и все раздал бедным согражданам и тем несчастным, которых мог отыскать в окрестностях города.

\section{Всякое звание должно избрать с рассуждением и молитвами}

Преподобный Паисий Великий, возжелав безмолвнейшего жития, намеревался оставить скитскую братию. Преподобный Иоанн Колов, спостник и друг Паисия, узнав его мысли, почувствовал всю тягость разлуки с великим и любезным старцем и хотел употребить все средства, чтобы удержать его в ските, но, уважая более пользу своего друга, нежели потребность своего сердца, сказал ему: «Возлюбленный о Христе брат! Я имею в сердце то же, что и ты; столь же усердно желаю уединения, но сомневаюсь, от Бога ли этот помысел. Пагубно будет для нас, если он "--- от человеческого самохотения! Помолимся убо Господу, да устроит Он по Своей воле житие наше и подаст нам известие: здесь ли нам пребывать? Уйти ли в отдаленнейшую пустыню или разлучиться друг с другом?» Паисий охотно принял совет Иоанна, и оба они, став на молитве, всю ночь испрашивали у Бога откровения.

Господь не отверг моления рабов Своих. На рассвете следующего дня явился им Ангел и, благословляя святых друзей, сказал: «Каждому из вас Бог определяет свое житие. Ты, Иоанн, пребывая здесь, будь наставником во спасение других, а ты, Паисий, прейди отсюда к западной стороне пустыни, да прославится там тобой имя Божие». Сказав это, Ангел скрылся. Преподобные отцы возблагодарили человеколюбивого Бога и, дав друг другу целование, разлучились.

Христиане! Преподобный Паисий, переменив свое место только после усердной молитвы, принял в напутствие благословение Божие и соделался столь велик, что Сам Спаситель мира являлся ему и утешал его. Столь же был знаменит и друг его Иоанн: в последующем времени из его иноческого училища вышли величайшие отцы, каков, например, был Арсений Великий. Вы иногда одно звание переменяете на другое: подражайте этим старцам, и Бог намерения ваши управит во благое.

\section{Неустрашимость за правду}

Святой Мелетий, бывший прежде епископом Севастийским, вступил на престол Антиохийской Церкви с общего согласия православных и ариан, и избирательный об этом лист поручен был на сохранение Евсевию, епископу Самосатскому\footnote{Память священномученика Евсевия, епископа Самосатского (†~380), празднуется 22~июня (5~июля).}. Но когда Мелетий, как ревностный поборник Православия, был изгнан и осужден на заточение, то ариане, опасаясь, чтобы избрание, утвержденное общим рукоприкладством, не послужило когда"=нибудь в обвинение их непостоянства и злобы, потребовали его от Евсевия назад. Святой Евсевий отказал. Тогда арианские архиереи упросили императора Констанция употребить в этом деле свою власть, но неустрашимый архипастырь отвечал: «Общего суда, мне вверенного, не отдам, разве когда все, вверившие мне его, соберутся воедино». Раздраженный император отправил к нему вторично одного из царедворцев и, чтобы принудить Евсевия к покорности, приказал употребить угрозы. Царедворец, подавая ему указ, сказал, что, в случае непослушания, он имеет повеление отрубить у него руку. «Итак, не нужно читать указ», "--- отвечал на это Евсевий и, положив бумагу на стол, простер обе руки. Увидев это, царедворец изумился и безмолвно оставил неустрашимого святителя. Сам Констанций, сколько ни был озлоблен, отказал арианам в просьбе.

\section{Урок испытующим тайны}

Однажды иноки, обитавшие в ските, собрались, чтобы решить вопрос: кто был Мельхиседек\footnote{Таинственное лицо, прообразовавшее Христа (см.: Быт. 14, 18; Пс. 109, 4; Евр. 5, 6).}? Всякий сказал, что думал. Когда же дошла очередь до аввы Коприя, он, трижды ударив себя по устам, произнес: «Горе тебе, Коприй! Горе тебе, Коприй! Горе тебе, Коприй! Что оставляешь дела, которые повелел человеку творить Бог, и испытуешь то, чего Он от тебя не требует?» Услышав это, братия с поспешностью разошлись по своим кельям.

\section{Военачальствующий пустынник}

Преподобный Афанасий, пустынножитель Афонской горы\footnote{Память прп. Афанасия Великого (†~1000) празднуется 5~(18) июля.}, столь прославился благочестием, добрыми делами и даром чудотворения, что в лавру его не только стекались отовсюду молодые иноки, но даже архиереи, оставляя пастырство\footnote{Таковы были: патриарх Николай, известный под именем и Харитона, Андрей Хризополит и Акакий.}, а вельможи свой сан, приходили к нему учиться равноангельской жизни.

В числе последних был грузинский князь, по имени Торникий, который, проведя несколько лет в греческой службе с отличной славой, наконец, восхотел с другими тремя грузинскими же князьями "--- Иоанном, Евфимием и Георгием, провести остаток жизни в обители преподобного Афанасия и от рук его принял монашеский образ.

Вскоре персияне, ободренные смертью греческого императора Романа II, напали на соседние области и, опустошая оные, грозили всему государству. Вдовствующая императрица Зоя, рассуждая, кому лучше в столь опасных обстоятельствах препоручить войска для отвращения персидских набегов, наконец решилась призвать Торникия. Но удалившийся от всех сует мира пустынножитель отказался. Тогда преподобный Афанасий при старейших и более уважаемых иноках с отеческой властью сказал ему: «Мы все "--- дети одного отечества и потому все должны защищать оное. Неизменная обязанность пустынножителей "--- противопоставлять нападениям врагов свои молитвы к сокрушающему брани Богу, но если предержащая власть почитает необходимым употребить руку нашу и грудь, да повинуемся беспрекословно и облечемся в оружие. Возлюбленный о Христе брат!

Кто думает и поступает иначе, тот раздражает Бога. И ты, при всех подвигах твоего иночества, подвергнешься этой же страшной участи, если не послушаешь царя, его же устами вещает Сам Господь. Ты будешь отвечать за кровь избиенных, как соотечественник, который мог, но не хотел спасти их; будешь отвечать за разорение храмов Божиих. Гряди убо с миром и, защищая отечество, защити Святую Церковь. Не бойся чрез это утратить сладостные для нас часы богомыслия: Моисей предводительствовал воинством и беседовал с Богом. В любви к ближнему заключается и любовь к Богу. Дерзну сказать, что любовь к ближнему приятнее Богу, нежели крепкое тщание о спасении своей только души, ибо \textit{никтоже бо нас себе живет, и никтоже себе умирает} (Рим. 14, 7)».

Торникий повиновался святому Афанасию и, сложив на время монашеское одеяние, принял военачальство. Поход его был благополучен. Многократно поразив персов, он принудил их заключить выгодный для Греции мир. Возвратившись в Царьград, он сдал начальство и, вместо всяких наград за столь знаменитое дело, ему предлагаемых, испросил только у императрицы немного денег на устроение обители в Афонской Горе, под названием Иверского монастыря\footnote{То есть Грузинского монастыря, ибо грузины у греков называются иверами.}, который существует и доныне. Благодарная государыня с радостью исполнила желание Торникия и повелела поставить его архимандритом новосозидаемой обители.

В ризнице Иверского монастыря, в память этому герою и своему основателю, доныне сохраняются удивительные по тяжести и драгоценному украшению военачальнические доспехи его\footnote{См.: Пешеходца \textit{Василия Григоровича"=Барского"=Плаки"=Албова}, уроженца киевского, монаха антиохийского, Путешествие к святым местам, в Европе, Азии и Африке находящимся, предпринятое в 1723~и оконченное в 1747~г. М, 1847. С. 580.}. Вот памятник, взирая на который каждый путешественник может научиться любви к Богу и отечеству, служению Царю Небесному и вместе царю земному!

\section{Христолюбивое воинство}

Богоотступник Юлиан, вознамерившись истребить христианскую веру, сделал тому начало с воинства, чтобы с помощью сильнейшего из государственных сословий после принудить к язычеству и всю империю. «Если отвлеку от Христа мои легионы, "--- говорил он своим друзьям, "--- то одержу половину победы над всем государством». Юлиан думал по"=человечески, и, казалось, должно было ожидать совершенного успеха, поскольку, во"=первых, воинство весьма любило Юлиана за его воинские добродетели, с величайшим удовольствием исполняло все его приказания и под его предводительством почитало себя непобедимым. Во"=вторых, военные люди, проводя шумную жизнь посреди своевольств, сопряженных с войной, не могли, кажется, полюбить кротости Евангельской. И не имели ни времени, ни способов исполнять все христианские обязанности, а через то нечувствительно забывали оные и прилеплялись к язычеству, которое нередко помещало счастливых воинов в число богов. Эти"=то причины абсолютно утвердили Юлиана в безбожном намерении. Но, чтобы внезапным и скорым переломом их совести не испортить дела, он начал действовать исподволь.

У римских воинов было обыкновение воздавать божескую честь императорским истуканам и портретам, и христиане исполняли это как верноподданническую обязанность. Они преклоняли колена пред изображениями императора не потому, яко бы его обоготворяли, а только потому, что Бог поставил его на земле как Свое подобие. Это обыкновение богоотступник хотел употребить в пользу идолопоклонства и приказал везде изображать себя не одного, но посреди языческих богов, чтобы солдаты, поклоняясь ему, вместе с тем поклонялись и идолам и таким образом мало"=помалу привыкали к кумирослужению. Он пошел еще далее. Известно, что военное знамя почитается за вещь священную, поэтому Юлиан уничтожил на нем имя Иисуса, которое равноапостольный Константин приказал изображать в знак своей веры, и начал употреблять в походах своих знамена с изображением Юпитера и прочих богов язычества. Такими знаменами Юлиан обычно обставлял и свой трон в те дни, когда солдатам должно было раздавать жалованье или награды, и ввел в обыкновение, чтобы каждый воин, прежде нежели получить деньги, положил несколько зерен ладану на горящие угли, откуда курение восходило к изображениям идолов. Он был уверен, что солдаты, от радости и торопясь получить деньги, на тот раз не будут рассуждать, правильно ли они поступают, "--- что, поступая этим образом сначала по простоте или по нужде, вскоре не будут почитать идолопоклонство не только преступлением, но и странностью. И наконец, будут служить его богам с удовольствием и от чистого сердца. Но, сколько Юлиан ни был хитер, он обманулся в этом предположении.

Солдаты имели смелость презирать милость государя и его деньги, чтобы не изменить своей совести и вере. Некоторые в его очах объявляли дерзновенно, что они не хотят курить фимиама бездушным идолам. Другие, хотя по простоте своей и делали это, но, размыслив после о своем поступке, в сердечном сокрушении рвали на себе волосы и, выбежав на общенародную площадь, свидетельствовали пред Богом и людьми, что они учинили преступление не с намерения, но из одной оплошности и торопливости. Многие возвращались к самому императору, бросали пред ним деньги и неотступно просили его предать их смерти, чтобы преступление их омылось их же кровью.

Но не одни солдаты противостали искушениям и насилию Юлиана. Воинские начальники, которые должны были ожидать от него всех милостей, лучше хотели страдать, нежели хотя наружно сделаться идолопоклонниками. Воздавая \textit{кесарева кесареви} беспрекословным повиновением, усердием к службе, храбростью на войне, нежалением своей крови за честь государя, они хотели воздавать и \textit{Божия Богови} внутренним и наружным служением Иисусу Христу, защищением невинно гонимых христиан и дерзновением обличать заблуждение и жестокость тирана. Из многочисленных примеров приведем некоторые.

Валентиниан, полковник императорской гвардии и сам после бывший императором, однажды по должности своей сопровождал Юлиана в языческий храм. Когда они приближались к капищу, то встретил их жрец и для очищения начал окроплять водой, якобы освященной именем идолов. Император и все за ним следовавшие приняли это благословение с великим благоговением. Но Валентиниан, почувствовав у себя на левой руке несколько капель и увидев, что одна пола его одежды обрызгана, на глазах у Юлиана ударил по щеке жреца и, обтерши руку, отодрал окропленную часть платья и бросил с гневом. Неустрашимый христианин не лишился жизни только потому, что Юлиан в то время еще не явно истреблял христианство, однако заключен был в оковы и сослан в Армению. Это происшествие доказывает, что Валентиниан готов был пострадать даже до смерти.

Святой Артемий, воевода египетских легионов\footnote{Этот на брани поседевший воин имел важный чин еще при Константине Великом и был свидетелем являвшегося на небеси крестного знамения, звездами изображенного. При Констанции ему поручено было перенести в Царьград тела святых апостолов: Андрея из Патр и Фомы из Фив. После этого Артемий был сделан военным губернатором в Египте. И хотя, в угождение двору, все чиновники тогда держались арианского вероисповедания, он пребывал непоколебим в Православии, а мудрость в градоправительстве и оказанные отечеству услуги на полях брани заставили Констанция забыть, что Артемий не арианин, между тем как прочие за то же неминуемо подвергались царскому гневу, отнятию чинов, заточению, а иногда и смерти. Один Юлиан не пощадил его!.. Память св. великомученика Артемия (†~362) празднуется 20~октября (2~ноября).}, получив повеление идти с порученным ему войском в Персию, немедленно прибыл в Антиохию, где находился тогда богоотступник. Но вместо подвигов за славу царя земного предлежал ему страдальческий подвиг за славу Царя Небесного. Это происходило следующим образом.

Юлиан по своему обыкновению захотел посмеяться над христианскими священнослужителями и, призвав к себе Евгения и Макария\footnote{Память преподобных Евгения и Макария исповедников, пресвитеров Антиохийских (†~363), празднуется 19~февраля (4~марта).} "--- пастырей, искусных в Священном Писании, жития непорочного и всеми уважаемых, "--- начал предлагать им ругательные вопросы о таинствах Евангельской веры, с благоговением вспоминая между тем богомерзкие таинства идолослужения. Но поскольку Евгений и Макарий имели премудрость, \textit{ейже не возмогут противитися или отвещати вси противляющиися} ей (Лк. 21, 15), то Юлиан, изобличенный в своем злочестии, не стерпев стыда своего, сбросил личину мудролюбия и приказал, якобы за оскорбление величества, Евгению дать пятьсот ран, а Макарию без числа.

В то время у Юлиана по своей должности был святой Артемий. Взирая на их страдания, он страдал в своем сердце, но уважение к царской особе долго удерживало язык его. Наконец жесточайшие хулы его на Иисуса Христа подвигли его сказать дерзновенно: «За что, государь, так мучишь неповинных и Богу посвященных мужей? Не они пришли обличать лжебогов, коим ты поклоняешься: ты призвал их, чтобы посрамить веру в Единого Истинного Бога. О, государь! Ты заставляешь меня говорить правду, горестную, может быть, не столько для тебя, сколько для нас, верных твоих подданных: вместо того, чтобы в лице твоем почитать власть Божию, народы видят в тебе врага, который некогда испросил у Бога позволение причинить напасти Иову. Но поверь, что суетны твои начинания. С того времени, как Христос нисшел на землю, пала идольская гордыня. Государь! Иисусова сила непобедима, и владычество Его бесконечно».

Богоотступник воспылал яростью и воскликнул грозно: «Как! Уже и мои военачальники делаются мятежниками и в лице дерзают меня злословить?» Но, дабы сокрыть зверство против христиан под завесой правосудия, он начал укорять святого старца в убиении брата своего Галла, умерщвленного Констанцием. «Я, по сродной мне кротости и милосердию, "--- наконец сказал он, "--- простил всех злодеев и истребителей моего рода. Но враги не перестают быть врагами! Благодарю бессмертных богов, что они сами предают в руки мои одного из этих убийц». Немедленно сняты были с Артемия все знаки чиноначалия, возложены оковы и принесены орудия казни. «Ведает небо, земля и весь лик святых Ангелов, "--- предаваясь в руки мучителей, сказал воин Христов, "--- ведает и Господь мой, Которому служу, что чист есмь от крови брата твоего. И не только не соизволял убийству, любя его и почитая за веру и благочестие, но, будучи доныне в Египте, не знал и намерения царева. Разве моя верность Констанцию делает меня в глазах твоих убийцей Галла?\footnote{Галл был военачальником на границах персидских. За некоторые своевольства и притязание излишней власти Констанций повелел лишить его начальства, но посланный для этого чиновник лишил его жизни.} Но христиане исполняют только те веления царевы, которые имеют целью славу Божию, честь государя и пользу отечества. Впрочем, на что тебе изыскивать преступления на главу мою, чтобы оправдать свой поступок? Я знаю, за что стражду, знает Господь мой, Иисус Христос, и узнает вселенная. Твори, яже хощеши».

Несколько дней продолжались страдания святого Артемия, и каждый раз Юлиан предлагал ему прощение и свою милость, если оставит Христа и обратится к идолам, а тем самым противоречил собственному приговору, по которому он якобы мстил за истребление своего рода. Наконец, непоколебимый страдалец, возвестив богоотступнику неизбежную и скорую погибель, усечен был в главу.

Ювентин и Максим\footnote{Память святых мучеников Ювентина и Максима воинов (†~361~-- 363) празднуется 5~(18) сентября.} служили в Юлиановой гвардии капитанами и, будучи ревностными христолюбцами, сохраняли при злочестивом дворе свою веру и чистую совесть. В безмолвии сетуя о несчастии христиан, они, как верные подданные, весьма усердно служили своему государю. Но когда Юлиан велел окропить в Антиохии идоложертвенной водой общественные колодези и съестные припасы, чтобы чрез то насильно навязать своим подданным языческую веру, то этот поступок показался Ювентину и Максиму страшным соблазном. Благочестивые воины начали жаловаться открыто и выражались словами израильтян, когда они были под игом Вавилонским: \textit{Предал еси нас, Господи, в руки врагов беззаконных, мерзких отступников, и царю неправедну и лукавнейшему паче всея земли} (Дан. 3, 32). Раздраженный Юлиан приказал наказать их телесно, заключить в темницу и описать в казну их имения. Он надеялся чрез это поколебать их веру и преклонить к отступничеству, но так как добродетельные воины остались последователями Евангельских истин, то и приказано обоим отрубить головы.

Святой Иоанн, один из воинских чиновников\footnote{Память мученика Иоанна Воина (†~IV~в.) празднуется 30~июля (12~августа).}, имея от Юлиана повеление оскорблять, притеснять, брать под стражу и даже убивать христиан, принял меры, удобнейшие к их защищению. Он наружно казался гонителем, но втайне им благодетельствовал: объявляя им указы императорские, доставлял случаи скрываться от погибели; прежде взятым под стражу, когда мог, тайно давал свободу; под видом строжайшего наблюдения посещая ночью темницы, приносил им пищу, белье и другие потребности. И когда друзья советовали ему быть осторожнее, святой воин отвечал: «Если мне и страшен гнев Юлиана, то потому только, что, подвергшись оному, не буду в состоянии помогать людям Христовым. Это одно запрещает мне излить пред отступником мои чувствования и пострадать за Иисуса Христа».

Долго для пользы страждущих христиан Господь покрывал тайной поступки Иоанна. Наконец, в бытность Юлиана в Персии, он был оклеветан и заключен в ужаснейшую темницу, где, томимый голодом и каждодневно удручаемый муками, присужден был ждать тирана для примерной казни. Но Бог определил иначе: отступник погиб, а воцарившийся после него Иовиан объявил свободу и милость всем узникам Христовым. Святой Иоанн, уважаемый государем и благословляемый народом, жил мирно, доколе Господь не воззвал его в обитель Небесную.

\section{Отроческая любовь к учению}

Святой Феодор, бывший после епископом Эдесской Церкви\footnote{Свт. Феодор, епископ Эдесский, жил в IX~в. Память его празднуется 9~(22) июля.}, в отроческих летах имел слабую память, не очень счастливые дарования и в науках был весьма медлителен, за что терпел часто насмешки от соучеников и строгие выговоры от учителей. Но для благонравного отрока всего чувствительнее была печаль отца и матери, которые, казалось, не ожидали от него ничего доброго.

Однако святой Феодор не был из числа тех, которые предаются отчаянию по причине неуспеха, и прибегнул к Помощнику, Который никогда не восхощет посрамить уповающих на Него. В одно время, наставляемый Небесным вдохновением, рано поутру он пришел в церковь и, видя, что за ним не следят, скрылся под святым престолом. Тут, проливая слезные молитвы к Богу, да просветит его разум в познание истины, Феодор уснул. Между тем собрались священнослужители, пришел сам архиерей, и началась Божественная служба.

Оглашаемый святым песнопением, спящий Феодор узрел равнолетнего себе отрока, подобного Ангелу Божию, который питал его медом. Обрадованный и вместе с тем устрашенный, Феодор проснулся и, не успев опомниться, вдруг вышел из"=под престола. Архиерей и все тут бывшие удивились внезапному явлению и начали спрашивать: кто он и что его побудило тут спрятаться? Феодор рассказал все подробно и открыл сновидение. Святитель не оставил без внимания столь редкой любви к учению и видимого просвещения свыше и причислил святого Феодора к церковному клиру.

С того времени дарования святого Феодора открылись в полном свете. Остроумием и глубокомыслием он вскоре превзошел всех сверстников, оставив нам пример, с какою ревностью должно стараться о просвещении своего ума и у кого в этом случае должно просить помощи.

\section{Достойный человек не должен удаляться от общественных обязанностей}

Преподобный Феодор против своей воли был возведен из обители святого Саввы на епископство Эдесской Церкви. Проливая слезы о своей безмолвной келье и простившись с любезной ему братией, Феодор с некоторыми из знаменитых эдесских граждан отправился в путь.

Но, когда достигли реки Евфрата и поставили кущи\footnote{\textit{Кущи} "--- шатры, палатки, шалаши.} свои для ночлега, святой пустыннолюбец, рассетовавшись наедине о священном граде и святой братии, воспел: «\textit{На реках Вавилонских, тамо седохом и плакахом, внегда помянути нам Сиона}» (Пс. 136, 1). Потом облился слезами и умыслил бежать, но внезапно был объят неким дреманием и услышал неизвестный глас: «\textit{Страшись участи раба ленивого, талант Господень в землю сокрывшего: тяжко согрешает, кто, имея силы, не хочет носить ига Христова»}. После этого святой Феодор воспрянул и, поразмыслив, сказал: «Да будет воля Господня!» Он отложил сетование и охотно пошел далее, к священной обязанности.

\section{Удаляйся и невинных случаев воспользоваться чужим достоянием\footnote{См. в житии прп. Феодора, епископа Эдесского, под 9"~м числом июля.}}

Два родные брата, Иоанн и Феодосий, несколько лет обитали в пустыне, в разных кельях, стараясь единственно о спасении своих душ, и, наконец, сделались столь совершенны, что часто являлся им Ангел Господень и укреплял их на подвиги, Богу угодные. Но что есть совершенство человека, пока плоть удерживает его в мире?

Однажды Иоанн и Феодосий, ходя по пустыне в некотором расстоянии друг от друга, собирали зелье. Вдруг Иоанн остановился с изумлением, но в ту же минуту, оградившись крестным знамением, перескочил место и быстро побежал к своей келье. Феодосий, издали увидев это, удивился и, думая, не змей ли так испугался брат его, из любопытства пошел туда "--- и узрел кучу золота. Сотворив клятву, он снял мантию, собрал в нее сокровище и с трудом принес его в келью, всемерно скрываясь от брата. На другой день, не сказавшись ему, он пошел в Едес, купил пространный дом с плодоносным садом, с водоемами и устроил в нем гостиницу и больницу в успокоение странным и недугующим. Потом нашел хорошего домоуправителя и, вручив ему из оставшегося богатства тысячу златниц для потреб благотворительного заведения, а другую раздав нищим, опять пошел в пустыню к брату своему Иоанну.

На пути размышлял он: «Брат мой мог, но не хотел или не умел сделать добра, а я совершил великое и богоугодное дело». Предав свое сердце этим высокомерным помыслам, Феодосий приблизился к тому месту, где жил Иоанн. Но тут встретил его тот самый Ангел, который прежде посещал его, и, воззрев грозно, сказал ему: «Зачем тщеславишься и возносишься пред твоим братом? Познай, что весь твой долговременный труд, весь твой странноприимный и врачебный дом не могут сравниться с Иоанновым перескочением чрез золото, ему не принадлежавшее. Перешагнув чрез чужое сокровище, он легкими крылами перелетел пропасть, утвердившуюся между богачом и Лазарем, и лоно Авраамово ждет его. Посему не узришь лица его во всю жизнь твою, не узришь и меня, доколе не умилостивишь Бога». Сказав это, Ангел сокрылся. Устрашенный Феодосий едва достиг вертепа братнего и, не найдя его, восплакал и возрыдал.

Сетование Феодосия продолжалось семь дней. После этого он удалился в Эдес и там при церкви святого Георгия пребыл в покаянии сорок девять лет. Но и здесь некие помыслы, как темное облако, не переставали посещать душу его; часто неутешная скорбь наполняла его сердце. Столь ужасно любостяжание, ибо с ним неразлучна обида, а часто и разорение ближнего! Посему уже в пятидесятый год Господь помянул Феодосия и снова повелел быть при нем видимо Ангелу"=хранителю.

\section{Неосторожность в осуждении, человеколюбиво обличенная}

Некоторый старец из Рима пришел в скит и поселился в одной из лучших келий подле церкви. Здесь, провождая жизнь богоугодную, он не наблюдал, однако же, всей строгости пустынножителей: имел раба, одевался чисто, ел только умеренно, а не один хлеб с водой и имел хотя не мягкую, но покойную постель. Этим образом он прожил двадцать пять лет и сделался так славен сколько благочестием, столько и духом прозорливости, что отовсюду приходили посетители, чтобы у него чему"=нибудь научиться.

Все уходили от него с удовлетворением и пользой. Но один, также великий, египетский пустынник, посмотрев на его житие, соблазнился. Прозорливый старец тотчас узнал мысли египтянина; однако же, невзирая на то, приказал рабу своему уготовить праздничную вечерю. Хозяин и гость вкусили чистого хлеба, плодов, елея и вина и, прочитав несколько псалмов, пошли ко сну.

Наутро египетский пустынник простился с хозяином и отправился в путь свой, не получив для души своей пользы. Но старец, сокрушаясь о заблуждении брата, приказал слуге возвратить его и, опять приняв с радостью, спросил: «Где твое отечество, авва?»

\textit{Гость}. В Египте.

\textit{Хозяин}. Из какого ты города?

\textit{Гость}. Я из села.

\textit{Хозяин}. Чем там занимался?

\textit{Гость}. Пас стада.

\textit{Хозяин}. Имел ли у себя дом?

\textit{Гость}. Нет; я день и ночь проводил на поле.

\textit{Хозяин}. Была ли у тебя постель?

\textit{Гость}. Трава "--- обыкновенное ложе для пастухов.

\textit{Хозяин}. Какую употреблял пищу и какое пил вино?

\textit{Гость}. Пищей был сухой хлеб, иногда с солью, а вино и на мысль не приходило: вода заменяла оное.

\textit{Хозяин}. Жизнь довольно неудобная, но для омовения была ли у тебя баня?

\textit{Гость}. Возможно ли? Когда было нужно, я мылся в реке.

После этого старец, вздохнув, сказал: «Ты довольно потрудился и потерпел, любезный о Христе брат! Благодарю Бога, что Он в старости лет даровал тебе жизнь лучшую. А я, бедный, дни моей юности провел посреди блеска и обилия. Отечество мое "--- древняя столица вселенной; служба моя была у престола царского и в Сенате». При этих словах египтянин познал свое заблуждение и, сравнивая прежнее состояние старца и свое с настоящим его и своим состоянием, увидел, кто из них более отрекся от мира и переменил удовольствия на Крест Христов. Между тем старец продолжал: «Но теперь, слава Богу, устраивающему спасение наше, я оставил столицу и двор и пришел в пустыню; оставил богатые дома и роскошные сады и вступил в тесную келью; оставил одры, изваянные из слоновой кости и золота и устланные драгоценными коврами, и принял от Бога это ложе; совлек с себя пурпур и багряницу и ношу, видишь, какие одежды; даровал свободу множеству рабов, и вот, вместо них, внушил Господь этому старцу, чтобы не оставил меня без помощи. Сверх того, вместо роскошных бань я возливаю несколько воды на мои ноги; ношу сандалии по слабости, а вместо мусикийских орудий и пения читаю Псалтирь. Итак, умоляю тебя, авва, прости мне, если имею что"=нибудь излишнее против других пустынножителей. Дух мой бодр, но плоть немощна». Выслушав это, египетский инок едва мог прийти в себя от удивления. «Увы мне! "--- воскликнул он. "--- Я от бедствий мира пришел в место покойное, а ты после стольких удовольствий принял удручение плоти, после богатств и славы "--- нищету и уничижение».

После этого с душеспасительным уроком он отправился в путь свой, всем рассказывал о мудром и человеколюбивом наставлении истинного крестоносца, убогого паче всех трудников, как говорил он (он иначе и не называл его). Когда же какой"=либо вредный помысел возмущал его душу, он всегда ходил к мудрому наставнику, чтобы явственнее узреть, что один приобрел с пустынной жизнью, а другой "--- чего лишился.

\section{Кто осуждает ближнего, тот святотатственно предвосхищает право Господне}

Некогда авва Исаак Фивский, будучи в соседнем монастыре, осудил другого авву за нерадивое житие. Эта мысль, отчасти самолюбивая, настолько заняла сердце Исаака, что он, и возвращаясь домой, не переставал разбирать и осуждать поступки брата. Но когда он подошел к своей келье, то в самых дверях предстал ему Ангел и угрюмо сказал: «Спаситель повелел мне узнать твои мысли, что должно сделать с аввой, который осужден тобой?». Исаак почувствовал вину свою и, повергшись пред Ангелом, воскликнул: «Согреших! Беззаконновах! Неправдовах! Помолись обо мне Спасителю мира». "--- «Восстань, "--- сказал Ангел, "--- на этот раз Господь прощает тебя, но впредь страшись осуждать твоего брата прежде, нежели Господь осудит его, иначе и в эту келью впущен не будешь».

\section{Важная наука "--- не согрешать языком}

Преподобный Памва\footnote{Прп. Памва жил в IV~в. Память его празднуется 18~(31) июля.}, который впоследствии столь основательно знал Священное Писание, в начале своего иночествования не знал грамоты и потому ходил к одному из братий, чтобы под его руководством выучить наизусть псалмы Давидовы. Но, услышав в одно время стих: \textit{Рех: сохраню пути моя, еже не согрешати ми языком моим} (Пс. 38, 2), вдруг оставил своего наставника и перестал ходить к нему. Когда же через шесть месяцев после того учитель, увидев его, спросил, почему он так долго у него не был, святой Памва отвечал: «Я еще не научился самым делом исполнять слова Давидовы: \textit{Рех: сохраню пути моя, еже не согрешати ми языком моим»}. Только через девятнадцать лет этот старец едва осмелился сказать, что научился самым делом исполнять то, чему этот стих научает.

Так преподобный Памва, воздерживая свой язык, \textit{еже не согрешати} им, не только во всю жизнь свою не осудил ближнего, не произнес праздного слова, но и вообще в разговорах был весьма осторожен и рассудителен в ответах. А когда спрашивали его о чем"=либо из Священного Писания или о другом важном деле, никогда своего мнения вдруг не объявлял, но прежде в безмолвии рассуждал сам с собою и часто говорил: «Не обретаю, что отвечать на этот вопрос». Редко через три дня, чаще через три недели, а иногда через три месяца он принимал решение. По этой причине ответ его был всегда истинен и полезен, как даруемый его разуму от благодати Божией. Все слушали старца и благодарили Бога, глаголющего чрез него.

Нет лучше молчания к месту. С ним человек, храня заповедь Божию о любви к ближнему, не делает язык свой \textit{огнем, лепотою неправды, злом неудержимым}. Не обличая этим образом своего сердца во злобе, он не подвергает и разум свой осуждению в невежестве. Жалки и смешны те люди, которые не могут понять или не успеют сообразить дела, а языку своему дают полную власть сделать приговор.

\section{Воспитание и жизнь преподобной Макрины\footnote{Память прп. Макрины, сестры свт. Василия Великого (†~380), празднуется 19~июля (1~августа).}}

Преподобная Макрина, сестра Василия Великого, родилась в одном из тех знаменитых и просвещенных домов, которые не щадят ни трудов, ни издержек на воспитание и образование своих детей. Добродетельная Емилия, мать ее, отнюдь не хотела уступить другим в образовании ума и сердца своей дочери. Но, что важнее всего, она не была из числа тех матерей, которые извиняются своими недосугами и поручают детей в чужие руки, иногда вовсе не умеющие приняться за этот тонкий и нежный цвет. Эта осторожность родителей была почти общим обыкновением первых и святейших веков Церкви.

Как только младенческий язык Макрины начал образовывать несколько стройные звуки, то первым ее выражением был Тот, Кто дарует нам бытие и жизнь. Емилия неизреченно восхищалась, слушая, как дочь ее слабым голосом и шепелеватым языком произносит сладчайшее имя Иисуса Христа, Которому она сочеталась при Святом Крещении. Почитая себя первой хранительницей этих священных обетов, она обыкновенно брала ее на свои колени и между родительскими ласками рассказывала сообразнейшие с детским возрастом истории из Ветхого Завета и события Евангельские. Таким образом юная Макрина научалась христианскому закону от одной своей родительницы, без пособия наставников.

Благочестивая Емилия сама учила ее грамоте, сама преподавала ей первые уроки нравственности, но, не следуя общему обыкновению, избрала другой путь. Греки почти всегда начинали учить детей с басней и некоторых поэтических сочинений, но Емилия увидела, что в них находится много таких мыслей и изображений, которые оскорбительны для чистого девического слуха, и предпочла избрать для нее учебной книгой песни Давида и премудрость Соломона. Избирая из этих священных сокровищ лучшие стихи, в которых содержатся или молитвы, или славословие Бога, или похвала какой"=либо добродетели, она приказывала читать их и по большей части учить наизусть. Одаренная умом отроковица, будучи семи лет, уже знала из Ветхого и Нового Завета все места, как в разное время дня, в разных обстоятельствах жизни благодарить Бога и предаваться святой Его воле. И на вопросы своих родственников отвечала всегда удовлетворительно о своих обязанностях к Богу, к себе самой, к разным лицам и в разных состояниях.

Емилия учила юную Макрину и пению. Но этим нежным голосом не были исполняемы любострастные песни Анакреона и Сафо. \textit{Свете тихий святыя славы} и сему подобные песнопения приводили ее слушателей в сладкое восхищение. Не только известная страсть, господствующая в песенных сочинениях, но и резкая шутка или слишком острый оборот делали их недостойными ее голоса и слуха. Емилия правильно рассуждала, что первые понятия и слова дитяти должны быть посвящаемы Богу, поскольку Ему Единому принадлежат начатки всех плодов. Благоразумная мать определяла ее жизнь в свете, но предостерегала от суетности света; она хотела, чтобы дочь ее с разумом острым и основательным соединяла просвещенную набожность. Для этого удаляла от нее только то, что не соответствовало этому намерению.

Часто матери водят сами своих детей в театр и в другие увеселительные места, которые отвращают их от трудолюбия, и между тем эти же матери выражают желание приучить их к трудолюбию. Емилия поняла несообразность такового плана в воспитании и не хотела мешать яда в здоровую пищу. Ходить в храм Божий "--- была единственная прогулка Макрины, утренняя и вечерняя. Там слушала она поучения духовных пастырей, которыми четвертый христианский век был столько украшен и прославлен и, чему научилась, обязана была повторять дома пред своею матерью.

Таким образом, не приучая юной Макрины к излишеству светских манер и не вдыхая в нее охоты к пустым удовольствиям, Емилия обучала ее всему, что прилично знать девице, которая должна жить в свете и стать хозяйкой. Женское рукоделие и наблюдение по дому были ее занятием, а чтение Священного Писания или отеческих сочинений служило ей отдохновением. В этих упражнениях Макрине минуло двенадцать лет.

Тогда многие из знаменитых каппадокийских граждан начали искать руки ее для своих сыновей, и благоразумные родители, наконец, избрали одного юношу, превосходившего других не только родом, но и разумом и благонравием, с которым святая Макрина и была обручена. Но судьбы Божии установили иначе. Когда родители с обеих сторон питались надеждой вскоре устроить счастье своих детей, жених ее умер. Святая девица, оплакав его, помыслила, что Бог требует от нее не супружеских добродетелей, и положила в сердце своем не выходить за другого. После этого, сколько родители и сродники ни напоминали ей о браке, она обыкновенно отвечала: «Несправедливо девице, обрученной единому мужу, отдавать свое сердце другому, ибо по закону природы должно быть супружество едино, как едино рождение и смерть едина. Вы говорите, что мой жених умер; но я в надежде воскресения верую, что он жив Богу. Грех и стыд супруге, в отсутствие своего супруга, не сохранить ему верности».

Не думая более о жизни в свете, святая Макрина облегчала домашние заботы своей матери, пособляла воспитывать младших сестер и братьев, сама наставляла их в науках, благоповедении и добронравии. Святой Василий, впоследствии нареченный Великим, возвратившись из своего путешествия, где собирал духовные сокровища, как юный мудрец, обнаруживал некоторую гордость о своем разуме: святая сестра кроткими и боговдохновенными разговорами внушала ему совершенное смиренномудрие. Другого ее брата, Григория, советы рабы Христовой соделали достойнейшим архипастырем Нисской Церкви. Брат Навклир всем мирским удовольствиям предпочел служение престарелым пустынножителям. Младший, Петр, возросший единственно на ее руках, также был святителем, не меньшим в угодниках Божиих. Сама Емилия, по кончине супруга, вняв совету Макрины, оставила все суеты мира и вместе с ней удалилась в девичий монастырь.

Там мать и дочь, уневестившись Христу, приняли на себя иноческий образ и жили вместе со своими рабынями, которые из одной любви захотели последовать за ними и разделить с ними благочестивое уединение. Все у них было общее: одна келья, одна трапеза, одни одежды. Все единодушно работали Господу в молитвах, постничестве, смирении и любви; святая же Макрина была им предшественницей в этих добродетелях. Следующее чудесное происшествие покажет нам, сколько она была целомудренна и приятна Богу. Однажды сделался у нее на груди столь опасный веред\footnote{\textit{Веред} "--- чирей, болячка.}, что Емилия начала бояться, чтобы болезнь не разлилась по всему телу и не коснулась сердца, и потому убеждала Макрину показать веред врачу и просить помощи. Но святая девица никак не могла согласиться, чтобы обнажить пред мужскими очами свои перси, и лучше хотела страдать, нежели допустить прикосновение к ним рук мужчины. В один из этих болезненных вечеров, просидев по своему обыкновению у ложа матери до тех пор, пока она не заснула, Макрина пошла в молитвенную храмину и, заключившись там, пребыла всю ночь на молитве. Преклоняя долу колена и лицо и окропляя слезами землю, она от Единого Бога просила исцеления. Потом, взяв персть\footnote{\textit{Персть} "--- горсть земли, праха; пыль.}, окропленную ее слезами, приложила к болящему месту "--- и в ту же минуту почувствовала облегчение, а наутро болезнь ее прошла совершенно. Емилия, при первом пробуждении, опять напомнила ей о необходимости отдаться в руки врача, но святая Макрина, не желая себе и той славы, что Господь столь чудесно посетил ее в болезни, ласкаясь к матери, отвечала: «О любезная родительница! Довольно будет для моего исцеления одной твоей руки: прикоснись только к болящему месту и положи на нем крестное знамение». Нежная мать исполнила ее просьбу и тогда же, к удивлению своему, узнала, что Небесный Врач душ и телес уже исцелил ее.

\begin{center}\small\textsc{Конец пятой части.}\end{center}

\chapter{ЧАСТЬ ШЕСТАЯ}
\section{Черта от образа Божия}

Какое сладкое, сколь истинное удовольствие ощущает душа, когда воображает невинное состояние первозданного человека. Жизнь блаженная! Дни радости неизреченной! Прекраснейшее создание рук Божиих, \textit{умаленное малым чим пред Ангелами} (ср. Пс. 8, 6), первый человек находится посреди прелестного Рая, окруженный всеми животными, созданными на службу ему. Разум его ведает Бога "--- и в этом боговедении, как в чистейшем зеркале, зрит всю природу и свое блаженство. Непорочная воля его имеет все способности и средства быть таковой. Невинный владыка находит невинным и все вокруг себя: пернатые, летая над ним, не страшатся ловитвы; звери возлежат при его стопах. Лев с агнцем, орел с голубем вкушают одну пищу. Все повинуются человеку, как человек повинуется Богу. Этот малый мир обретает в себе мир и тишину, и все разнородные существа неизмеримого мира подражают миру и тишине его сердца. Он дышит радостью, и это дыхание растворяет радостью все, сущее под небесами. Мудрый, благий, бессмертный, повелитель всех тварей, он отображает в себе подобие Создателя, как утренняя капля росы отображает в себе лучезарное солнце.

Но сколь горестная воспоследовала перемена, когда этот любимец Бога нарушил Его заповедь! Сделавшись преслушником, он отовсюду узрел преслушание и к себе. Вся природа восстала на человека, поскольку человек восстал на Бога и, как Его образ, восстал сам на себя. Эта внутренняя война была столь жестока, что ни в телесном, ни в духовном мире не обреталось средства потушить оную и возвратить человеку прежнее владычество над собой и над всеми творениями. Это средство мог изобрести един Премудрый, Всемогущий, Премилосердый Бог. Через единородного Своего Сына даровав нам веру, повелел Он соединить с ней святую жизнь. Верой человек оставляет ветхого человека и облекается в нового; святой жизнью "--- в этом достоинстве утверждается. Все совершенства, с которыми первенец человечества вышел из рук Божиих, ему возвращаются.

Он знает истины Евангелия, следовательно "--- премудр. Почерпает из него побудительные причины удаляться от зла и творить благо, следовательно "--- свят. Возрождается и оправдывается для вечного блаженства, следовательно "--- бессмертен. Премудрость, святость и бессмертие "--- вот Божественные совершенства, которые возвращают человеку образ Божий.

Остается одна черта, по"=видимому, для человека невозвратная, "--- первоначальное владычество над прочими тварями. Здесь не имеется в виду преимущество разума человеческого над крепким составом животных: им обладал и падший человек; здесь говорится о том повиновении, которое оказывали человеку самые лютые звери. Кто же не видит из жития истинно благочестивых людей, что святая жизнь возвращала им и это первоначальное владычество? История первых христианских времен представляет множество разительных тому примеров. Страстотерпцы и преподобные, столь же невинные, как обитатель земного рая, столько же, как он [Адам до грехопадения], послушные Господу, обретали и сами удивительное им повиновение у тех бессловесных, один взор которых поражает ужасом. Приведем несколько примеров.

1. Преподобный Савва Освященный, уступив мятежному духу некоторых братий, удалился в скифопольские пределы. Найдя там некую пещеру, он безбоязненно вошел в нее и, поскольку был вечер, помолившись, лег уснуть. Но в полночь пришел лев и, увидев на ложе своем почивающего человека, взял зубами за его одежду и повлек из вертепа. Святой старец, пробудившись, не ужаснулся, но, восстав спокойно, начал читать полунощные молитвы, а лев, несколько посторонившись, как будто ждал, доколе Савва совершит «правило». После этого старец сел, и лев опять начал влечь его из пещеры. Тогда человек по образу Божию сказал ему: «Послушай! Пещера пространна; мы оба можем жить вместе, ибо один Творец нас создал. Если же тебе неугодно со мною жить, то приличнее тебе уйти отсюда, ибо я, как созданный рукой Господа и возвеличенный Его образом, несравненно лучше тебя». Выслушав это, грозный зверь вышел из пещеры и уже туда не возвращался.

2. Преподобные отцы Иоанн Евират и Софроний Софист повествуют в Лимонаре следующее. Однажды в пустыне Иорданской с преподобным Герасимом\footnote{Память прп. Герасима, иже на Иордане (†~475), празднуется 4~(17) марта.} встретился лев и, болезнуя, показывал свою ногу, глубоко пронзенную терном, который, оставшись там, причинил чрезвычайную опухоль. Бессловесный зверь, умиленными очами взирая на старца, смиренно просил у него помощи. Преподобный Герасим, сжалившись над страждущим животным, велел ему лечь на землю, сел сам и, небоязненно взяв его лапу, вынул терн, очистил рану и, обвязав убрусцем, дал знак, чтобы шел он в свое место. Но лев не хотел оставить своего избавителя, пошел за ним в самую обитель и уже не возвращался в пустыню, но везде ходил за ним, так что все с изумлением смотрели на привязанность зверя. Святой старец из своих рук кормил его хлебом и давал другие яства, которыми сам питался; он назвал его Иорданом, и лев, откликаясь на свое имя, всегда прибегал к нему и как бы ожидал повеления. Это послушание царя всех зверей простиралось до того, что он пас монастырский скот и даже исправлял некоторые работы.

Но следующее обстоятельство еще достопримечательнее. По собственному ли побуждению или по смотрению Божию этот чудный зверь на какое"=то время отлучился из лавры. Между тем преподобный Герасим скончался и был погребен. Вскоре явился лев и начал искать своего благодетеля. Все старцы вновь поражены были горестью о своем наставнике, увидев, с каким нетерпением ищет его несмысленный зверь, а один из учеников святого Герасима, Савватий, сказал ему: «Иордан! Старец наш оставил нас сирых и отошел к Господу». После этого лев, озираясь всюду, скорбно зарыкал. Савватий принес ему хлеба, но лев, не приняв оного, усугубил вопль свой, как бы требуя, чтобы святой Герасим вышел к нему. Тогда Савватий, не в силах будучи и сам удержать слез, взял его за гриву и повел ко гробу усопшего отца. «Здесь сокрыта плоть нашего старца, а там душа его», "--- указав на гроб и на небо, сказал он и, преклонив колена, начал молиться. Зверь, услышав это и увидев плачущего Савватия, ударился головой о землю и, рыкнув ужасно, испустил дух на могиле старца.

Всякому известно, что этот лев не имел разумной и словесной души. Но, читая эту утвержденную временем и верой повесть, должно исповедать, что Бог, даровав святому Герасиму такую власть над лютым зверем, чрез то хотел показать нам, \textit{коликое имели послушание звери к Адаму в раю прежде его преслушания и от рая отпадения}; и до какой степени истинно верующему возвращается образ Божий, погубленный преступлением первозданного человека для всех его потомков.

3. Преподобный Иоанникий Великий\footnote{Прп. Иоанникий Великий (†~846) родился в 750~г., при царе Льве Исавре; после 94"=летней жизни скончался в царствование Михаила и Феодоры. Память его празднуется 4~(17) ноября.}, обитая в горах Контурийских, где было множество зверей и змей, не видел от них никакой опасности. Однажды при наступлении зимы, ходя по угрюмым высотам, он встретил глубокую пещеру и, поскольку не имел у себя кельи, почел эту пещеру способной защитить его от холода: немедленно вошел в нее и весьма обрадовался, узрев что"=то горящее. Иоанникий поднял с земли сухую ветвь и бросил на потухающий, по его мнению, огонь. Но вдруг этот огонь сотрясся, и хворост опять упал на землю. Святой старец подошел ближе и увидел, что это был огромный змий, глаза которого блистали, как горящие угли. Однако имеющий власть от Бога на змия невидимого не убоялся видимого змия, но, уклонившись к правой стороне вертепа, пребыл тут вместе с чудовищем, пока миновала зима.

4. Преподобный Сергий, Радонежский чудотворец, сокрывшись в уединенное и удаленное от человеческих жилищ место, имел странного посетителя. Когда благорассуждением, молитвой и трудом обращал он в ничто все страхи и мечтания, тогда приходил к нему большой и свирепый медведь. Святой пустынножитель кормил его и часто отдавал зверю последний кусок хлеба, сам оставаясь без пищи. Столь милостива была душа Сергия! За то лютого медведя он претворил яко в агнца, играющего пред ним.

Христиане! Чего не сильна произвести вера? Да не дивимся убо, что Адаму в его невинном состоянии повиновались звери и были кротки. Можно думать, что ныне от жестокости наших нравов они рассвирепели и сделались столь ужасны; человеколюбие же и добродетель могут свирепство и ярость претворить в кротость и тихость. Для этого необходимо, чтобы наш разум и воля обновились и освятились, тогда и наша душа облечется в ту святость, в каковой был создан первый Адам. Да не думаем, яко бы это было невозможно. Сам Бог \textit{предустави нас сообразных быти образу Сына Своего} (Рим. 8, 29). Христиане, облеченные в нового человека, способны быть \textit{Божественнаго причастницы естества}, если отбегнут, \textit{яже в міре, похотныя тли} (2~Пет. 1, 4). \textit{Откровенным лицем славу Господню взирающе, в тойже образ преобразуемся от славы в славу, якоже от Господня Духа} (2~Кор. 3, 18). Это освящение есть воля Отца Небесного, ибо Он хочет, да будем \textit{святи, якоже и Он свят есть} (ср. 1~Пет. 1, 16), да будем \textit{совершени, якоже и Он совершен есть} (Мф. 5, 48). Да будем \textit{милосерды, якоже и Он милосерд есть} (Лк. 6, 36). Все это святые совершили и в делах своих некоторым образом уподобились делам Божественным. Итак, безумен тот, кто не верит чудесам, совершаемым избранными людьми! \textit{Тщание все привнесше, подадим в вере нашей добродетель, в добродетели же разум, в разуме же воздержание, в воздержании же терпение, в терпении же благочестие, во благочестии же братолюбие, в братолюбии же любовь} (2~Пет. 1, 5~-- 7); тогда увидим, что возможно и для человека \textit{заграждать уста львов} (Евр. 11, 33).

\section{Семь эфесских отроков\footnote{Память святых семи отроков, иже во Ефесе (ок. 250), празднуется 4~(17) августа.}}

В бытность императора Деция в Эфесе взяты были под стражу, в числе прочих исповедников имени Христова, семь юношей: Максимилиан, Ямвлих, Мартиниан, Иоанн, Дионисий, Екзакустодиан и Антонин, из которых первый был сын Эфесского градоначальника, а прочие "--- других гражданских чиновников, и все служили в римском войске офицерами. Деций, раздраженный более тем, что посреди его оруженосцев и телохранителей находятся презрители богов, которым он покланялся, лишил их чинов и приказал или принудить к идольскому служению, или погубить мучительною смертью. Но, когда увидел их воинскую красоту, почувствовал сожаление и, дав им время на размышление, освободил из"=под стражи, а сам между тем отправился для обозрения других городов области Эфесской.

Святые юноши употребили отсутствие Деция единственно на молитвы и милостыню, а чтобы лучше приуготовить себя к мученичеству, они удалились в одну пещеру. Младший из всех, Ямвлих, переменив одежды, изредка ходил в город, чтобы подать милостыню и для себя купить хлеба. Таким образом, протекло несколько времени, и святых отроков, казалось, забыл весь свет. Но в один день Ямвлих возвратился из Эфеса весьма скоро и с трепетом объявил своим товарищам, что Деций прибыл и их велено к нему представить. Юноши пали на землю и, молясь Богу с плачем и стенаниями, поручали себя Его милосердию. При закате солнца они немного вкусили и, беседуя между собой и взаимно укрепляя себя к мужественному страданию за Христа, вдруг воздремали и уснули странным и дивным успением\footnote{Успение "--- смерть, кончина.}.

Наутро, по приказанию Деция, в самом деле начали искать святых отроков и, наконец, открыли их убежище. Сами их родители к тому способствовали "--- или из страха, чтобы не подвергнуться той же участи, или по общему изуверству язычников. Однако Деций, при всей ярости, еще сожалел о них и, дабы не видеть смерти столь упрямых, по его мнению, суеверов и вместе с тем удовлетворить гневу богов, приказал заградить камнями устье пещеры, где находились святые отроки. К славе христианской Церкви, Феодор и Руфин, царские постельники, будучи потаенными христианами, успели вложить между этими камнями оловянные дощечки, где изображены были имена и страдания святых семи отроков.

Вскоре после этого Деций погиб. Следовавшие за ним гонители христианства также с шумом миновали чреду свою. Настало царство Евангелия и посреди свирепеющих ересей торжествовало до Феодосия Второго. Тогда водворилось горшее из всех зол: под руководством Феодора, епископа Эгинского, повсюду рассеялись соблазнители, отрицавшие воскресение мертвых, и заразили все сословия народа, а наипаче укоренилось то мнение, что восстание изгнивших и обратившихся в прах наших тел при Втором Пришествии Христа невозможно. Безумные не рассуждали, что Бог, создавший из ничего весь мир и человека, тем более может испортившееся восстановить в прежний порядок! Благочестивый царь молился Сыну Божиему, Его же \textit{глас услышат мертвии, и услышавше оживут} (Ин. 5, 25), и Господь, вняв его молению, открыл чрез семь святых отроков тайну чаемого восстания из гробов следующим образом. Некто Адолий, владелец окрестностей пещеры, заключавшей телеса страдальцев, начал (по смотрению Божию) строить каменную ограду для своих стад и, не зная событий минувшего времени, приказал брать те самые камни, которыми завалено было устье вертепа. Вскоре сделалось отверстие, так что можно было пройти человеку. В это время Господь наш, обладающий жизнью и смертью, благоволил воззвать семь оных отроков, спавших почти два столетия. Воскресли святые исповедники, души которых хранимы были в деснице Божией, а тела, яко спящие, возлежали в пещере нетленными и неизменными. Они думали, что пробудились от обыкновенного ночного сна, ибо не было на них ни единого знака мертвенности. Одежды были целые, лица красотой цветущие, тот же возраст юности. Воздав благодарение Богу, они беседовали между собой о настоящем кумирослужении, о гонении на христиан и о поисках Деция. Потом поручили Ямвлиху опять сходить в город и, разведав, что в нем происходит, купить немного хлеба. После этого сказали они в один голос: «Будем готовы предстать Децию».

Ямвлих, едва вышел из пещеры, изумился, увидев камни, лежащие у входа в оную. Спустившись с горы, он шел боязливо; несколько раз останавливался, опасаясь идти в Эфес и там быть узнанным. Но изумление его еще более усилилось, когда, приближаясь к городским воротам, узрел над ними Крест Господень. Шествуя далее по улицам Эфеса, он видит храмы, крестами же увенчанные. «Что это? "--- думал Ямвлих. "--- Вчера нигде крестного знамения не зрелось, а ныне повсюду». Он слышит клянущихся именем Христовым; видит другие здания, другие одежды и все это почитает за сонное мечтание, но, осязая сам себя, уверяется, что зрит и шествует наяву. Думая, что он каким"=либо чудом перенесен в другое место, он спросил у одного из встретившихся об имени города и, услышав имя Эфеса, не поверил. Наконец, он решился купить хлеба и как можно скорее возвратиться в пещеру, сомневаясь, найдет ли ее.

Но какой ужас объял Ямвлиха, когда хлебопродавец, получив от него сребреник, начал всем показывать его как некоторую редкость! Увидев же, что все тут находящиеся перешептываются друг с другом, святой юноша подумал, что его узнали, и, по своей невинности, хотел бежать, но вдруг был схвачен. Множество людей стеклось на шум и, подозревая, что Ямвлих нашел клад, повели его к градоначальнику. Святой отрок не знал, что с ним делается: искал глазами кого"=нибудь из знакомых, но никого не видел; удивлялся, как и сам он в одну ночь сделался для всех незнакомым. В этом непонятном состоянии он был представлен к градоначальнику, у которого был на этот раз и епископ Эфесский.

Сначала Ямвлих думал, что его ведут к Децию, но, увидев в комнатах градоначальника святые образа, опять пришел в недоумение; а представ пред ним, не мог не оградиться крестным знамением, увидев крест на персях епископа. Градоначальник, рассматривая сребреник, спросил: «Где нашел ты сокровище?» "--- «Я не находил никакого сокровища, "--- отвечал святой Ямвлих, "--- а этот сребреник взял у моих родителей». "--- «Или неправду говоришь, "--- возразил градоначальник, "--- или твои родители нашли клад, ибо такие деньги находятся только в собрании царских редкостей, а у частного человека быть не могут». И показал ему те деньги, которые тогда употреблялись в Греции. После этого спросил, откуда он и как именуется. Когда Ямвлих объявил имя отца, матери и некоторых родственников, то одни из присутствующих почли его юродивым, а другие "--- притворяющимся. Святой юноша стоял в полном недоумении, безмолвствуя и потупив голову. Наконец градоначальник потребовал с гневом, чтобы Ямвлих непременно объявил, где находится сокровище, им обретенное, и, в случае запирательства угрожал жестоким наказанием. Тогда устрашенный отрок сквозь слезы сказал: «Заклинаю вас Богом, скажите мне, "--- а я открою вам всю истину, "--- здесь ли государь наш Деций?» "--- «Чадо! "--- отвечал на это удивленный епископ. "--- Деций царствовал в древних летах, а ныне владычествует Феодосий, царь благочестивый». Потом, обратившись к градоначальнику, продолжал: «Этот юноша, кажется, потерял разум, а потому и должно с ним поступать человеколюбивее». Тогда Ямвлих ободрился. «Умоляю вас, идите за мною, "--- сказал он, "--- я покажу вам в пещере, которая находится в горе Охлон, друзей моих; от них узнаете вы, что все мной сказанное есть сущая правда. Свидетельствуюсь Богом, что мы, за несколько дней перед этим, бежали от лица Деция и скрылись в вертепе. Я сам моими очами вчера его видел; а что делается сегодня, не знаю». Епископ задумался, потом сказал градоначальнику: «Вижу, что Бог хочет открыть нечто дивное; идем с ним, да узрим волю Божию». И все немедленно пошли к горе Охлон, сопровождаемые Ямвлихом. Когда достигли пещеры, Ямвлих первый вошел в нее. Епископ и прочие, следуя за ним, нечаянно увидели в устье пещерном выставившийся между двумя камнями ковчег. Они с любопытством открыли его и нашли две оловянные дощечки, на коих было изображено, что семь отроков: Максимилиан, сын Эфесского градоначальника, Ямвлих, Мартиниан, Иоанн, Дионисий, Екзакустодиан и Антонин, "--- убежав от Деция, сокрылись в этой пещере и, по распоряжению мучителя загражденные камнями, скончались за Христа страдальчески. Все удивились и прославили Бога.

Войдя в пещеру, они обрели святых отроков, спокойно беседующих между собой. Лица их цвели прежней юностью и сияли светом благодати. Епископ, градоначальник и прочие облобызали их с благоговением. Святые отроки сначала ужаснулись, почитая в них служителей Деция, но, видя прославляемым имя Христово, обрадовались и рассказали все происшествия минувших, как им казалось, дней. Немедленно к царю Феодосию отправлено было донесение, в котором епископ и градоначальник просили его, чтобы он прислал благочестивых и мудрых людей, заслуживших общую доверенность, освидетельствовать семь чудных отроков, в коих показался образ будущего воскресения. Но возрадовавшийся небесной радостью Феодосий тогда же со всем двором сам отправился в Эфес.

Сопровождаемый эфесским духовенством, чиновниками и народом, благочестивый государь, войдя в пещеру, узрел их как Ангелов Божиих, обнял и восплакал на их шеях. Он долго беседовал с ними, прославляя Бога. Но посреди самих разговоров, когда все присутствовавшие прилежно смотрели на них, внезапно святые отроки преклонили главы свои к земле и, по Божию велению, уснули сном смерти.

\section{Подвиги преподобного Симеона\footnote{Этот подвижник жил при Феодосии I, Маркиане, Феодосии II и Льве Великом. Память прп. Симеона Столпника (†~459) празднуется 1~(14) сентября.}}

1. Преподобный Симеон, нарицаемый Столпник, будучи восемнадцати лет, удалился в пустыню. Положив на сердце упражняться в едином богомыслии, он обрел некоторый вертеп и, повергшись крестообразно на земле, с плачем молился Богу, да покажет ему путь спасения. Утружденный земным поклонением, здесь юный Симеон уснул и узрел следующее видение: ему казалось, что копает землю, чтобы положить основание для какого"=то прочного здания, и слышит голос: «Копай глубже». Потрудившись некоторое время и думая, что этой глубины довольно для твердого основания, Симеон остановился, но опять услышал: «Копай глубже». Трудник начал снова рыть землю, но едва от утомления выпустил из рук заступ, как услышал в третий раз тот же голос: «Копай глубже». Тогда Симеон собрал все свои силы и трудился дотоле, пока невидимый голос не сказал ему: «Перестань; основание будет твердо; в уповании на Бога созидай». Пробудившись от сна, Симеон возблагодарил Бога за столь спасительный урок и всегда помнил, что основание добродетельного жития есть труд.

2. Преподобный Симеон восхотел трудиться по"=своему. Взойдя на один высокий холм, он сам себя приковал к камню "--- в том намерении, чтобы не мог оставить места духовных подвигов, хотя бы в сердце и возродилось на то поползновение. В этом удивительном самопринуждении застал его Мелетий, архиепископ Антиохийский, пришедший посетить его как величайшего в подвижниках Христовых.

Но, восхваляя труды Симеона, мудрый архипастырь не похвалил этих вещественных оков. «Человек и без оков может владеть собой, "--- отечески сказал он, "--- и ты, чадо мое, к единому месту можешь привязать себя не железом, но волей и разумом». Услышав великую истину, Симеон немедленно снял с себя оковы и, не оставляя намерения подвизаться на одном месте, с того времени помышления свои обуздывал своею волей и \textit{пленял свой разум в послушание Христово} (2~Кор. 10, 5), да будет узником Креста Его.

3. Несколько лет преподобный Симеон подвизался на этом холме, но, будучи со всех сторон, как волнами, окружаем пришельцами, восхотел избыть молвы человеческой и для этого избрал дотоле не слыханный образ уединения: соорудил столп и на нем тесную хижину, отчего и был назван Столпником. Это новое и странное жилище сначала было высотой в шесть локтей. Через несколько времени благочестивые посетители воздвигли ему другой столп, в двенадцать локтей, потом в двадцать два и, наконец, в сорок локтей. Так святой Симеон различными столпами, как лестницами, восходил к стране Небесной, к ближайшему лицезрению Бога. Весь свет и самые строгие пустыннолюбцы удивлялись необыкновенному житию святого Симеона, а некоторые из них, желая \textit{искусить сущего в нем духа}, послали трех достопочтенных старцев, чтобы они именем собора укорили его за новый вымысел и приказали жить по примеру других отцов. Едва посланники объявили ему волю святого братства, Симеон ступил ногой на лестницу, чтобы сойти вниз. Тогда три старца в один голос сказали ему: «Не сходи, святой юноша, но останься в подвиге, который начал. Теперь знаем, что твое намерение внушил тебе Сам Бог». Но как святой Симеон изъявил страх, чтобы преслушанием не прогневать собора отцов, то старцы объявили ему, что они имели поручение свести его со столпа насильно, если будет противиться приказанию, и, напротив того, оставить на столпе, если захочет сойти сам. После этого чудный Симеон остался на своем месте, а старцы пошли от него, благословляя Бога, укрепляющего Своих избранных в деле спасения.

4. Хотя преподобный Симеон более всего желал уединения и безмолвия, но может ли укрыться светильник, стоящий наверху горы? Слава действующей в нем благодати была столь велика, что не было на Востоке народа, который бы не знал Столпника. Грузины, армяне, персы, измаилиты стекались к нему тысячами и принимали при его столпе Святое Крещение. Удивительная сила не только его словес, но и самих взоров мгновенно поражала и умягчала ожесточенных зловерием варваров, и они с воплем проклинали заблуждения своих предков. Симеон, стоя на одном месте, был столько же полезен для Церкви, как если бы кто проходил вселенную, проповедуя и благодетельствуя. Посреди этих удивительных подвигов и чудодеяний он был так кроток и смирен, что всегда и пред всеми показывался с одинаковым весельем в очах, с одинаковою любовью в разговорах. Государи и земледельцы, вельможи и рабы, богатые и убогие, добродетельные и порочные, даже общественные злодеи видели в Симеоне наставника и отца, всем благовествующего спасение. Одни возвращались от него совершенными пустыннолюбцами, другие "--- великими поборниками Православия, все "--- усердными гражданами. А некоторые, желая всегда наслаждаться святой беседой Симеона, при столпе его основали свои жилища, так что чудный дом его казался стоящим посреди города, населенного множеством разноплеменных народов.

5. Повседневный устав жития Симеона был не менее достоин удивления. Всю ночь и день, даже до девятого часа, он стоял на молитве. После этого поучал обитающий вокруг столпа народ и пришельцев, проповедовал Евангелие, подавал советы, исцелял молитвой недугующих. Наконец, начинал разбирать свары и прения и водворять между всеми мир. После заката солнца опять обращался на молитву. Но, посвящая все часы своей чудной жизни богомыслию и человеколюбию, он не переставал помышлять о мире церковном, разоряя языческое безбожие, препирая иудейские хулы, истребляя еретическое учение. Царей и всех сильных земли он поучал своими мудрыми посланиями страху Божию и взывал к милосердию, любви и защите Церкви Христовой. Три патриарха, один после другого управлявшие Антиохийской Церковью, считали своим непременным долгом посетить Столпника. Святой Мелетий, наставляя Симеона в его юности на путь спасения, вместе с тем благословлял его неимоверные подвиги. Блаженный Домн приходил к нему единственно для того, чтобы совершить рядом с ним литургию и приобщиться Животворящих Таин Христовых. Святейший Мартирий называл его защитником Антиохии от врагов видимых и невидимых, могущим утолять гнев Божий.

6. Посреди этого торжества веры и благочестия святой Симеон на сто третьем году своей жизни достиг блаженной кончины, которая была столько же торжественна. В один день, а именно в Великую пятницу, Столпник не показался собравшемуся народу, как обыкновенно делал это. Также в субботу и воскресенье все пришельцы тщетно ожидали от него спасительных уроков и благословения. «Наконец я, "--- говорит Антоний, ученик преподобного Симеона, "--- ужаснулся и взошел на столп. Святой старец стоял, имея голову поникшую и руки, на персях крестообразно согбенные. Думая, что Симеон безмолвно молится, и я стал позади него в безмолвии, но вскоре узрел, что святая душа оставила чистое тело. Восплакав горько, я положил его на землю и, целуя очи, уста и руки, спрятал его мощи. Недоумевая, что сказать народу, а наипаче больным, нуждавшимся от него в исцелении, я не смел сойти вниз и, облившись опять слезами над бездыханным праведником, от горести уснул. Потом, будучи утешен явившимся мне в сновидении Симеоном, тайно послал одного верного брата в Антиохию к святейшему Мартирию "--- возвестить об успении великого старца. Но плачевная весть в один час, как вихрем, разнеслась повсюду; голос плача разливался на семь стадий. Патриарх прибыл с тремя епископами и всем священным собором. Сам Антиохийский градоначальник, сопровождаемый своим воинством, пришел воздать честь усопшему праведнику. В таком торжестве эти святые мощи перенесены в Антиохию». Георгий Кедрин и прочие историки свидетельствуют, что император Лев Великий хотел иметь нетленные останки Симеона в своем царствующем граде, но антиохияне в этом случае готовы были восстать на посланников царских, если не послушают они моления и слез. «Город наш не имеет каменных стен, "--- в один голос вопияли они, "--- потому да почивает у нас святое тело Симеона. Оно будет Антиохии стеной и защищением».

\section{О том, сколь неугодна Богу жертва, приносимая не от чистого сердца\footnote{Из жития св. мученика Маманта. См. в Четии"=Минее под 2"~м числом сентября.}}

Юлиан Отступник еще в молодых летах, находясь в Кесарии Каппадокийской, уговорился с братом своим Галлом построить великолепный храм над гробом святого мученика Маманта. Не благочестие (о чем свидетельствует время его царствования) побудило его к этому, но или тщеславие и зависть к Галлу, которого все любили за христианскую ревность, или желание подать хорошие мысли о своей вере императору Констанцию, который, может быть, уже подозревал его в приверженности к идолам.

Как бы то ни было, Галл и Юлиан разделили между собой, кому какие должно начать и совершить здания, и каждый из них с возможной ревностью старался возвести свою половину. Но Бог видимо здесь показал, сколь Ему неугодна жертва, приносимая не от чистого сердца. В то время как художники Галла счастливо продолжали свою работу, невидимая сила противоборствовала строителям Юлиановым. Сначала они не могли найти твердой почвы под основание, потом от времени до времени показывалось землетрясение, которое вырывало камни даже из земли и ниспровергало в одно мгновение то, что с великим трудом было укрепляемо. Иногда гром ударял в мраморные столпы и разрушал их на части. Иногда вихрь срывал крыши, под которыми хранилась известь, и развевал ее по воздуху. Не видно ли из этого, что Господь еще тогда изобличал будущее злочестие Юлиана и видимыми знаками предупреждал его от отпадения и гибели?

\section{Священномученик Вавила\footnote{Память священномученика Вавилы, епископа Великой Антиохии (†~251), празднуется 4~(17) сентября.}}

Этот пастырь Антиохийской Церкви, живший в третьем веке по~Р.Х., был для духовного стада совершенным примером всех добродетелей, наставлял детей своих словом, житием, любовью, верою и кротостью. Столь святой и славный человек не мог при гонителях христианства избегнуть страдальчества.

Однажды цезарь Нумериан, будучи в Антиохии, приносил торжественную жертву идолам, а святой архиерей между тем совершал бескровную жертву во храме Христа Спасителя. Возвращаясь домой по окончании всех обрядов, Нумериан услышал пение христиан и захотел войти в их церковь, отчасти из любопытства, а более для того, чтобы разогнать молящихся, но святой Вавила воспротивился и, употребив власть архипастыря, удержал Нумериана на пороге храма. Император мог бы с помощью стражи исполнить свою волю, но, видя множество православных, убоялся кровопролития и, скрыв досаду, оставил на тот раз святого человека в покое.

На другой день, по повелению императора, христианская церковь была сожжена, и святой Вавила в оковах представлен был к нему для ответа. «Подлый раб! "--- воскликнул раздраженный Нумериан. "--- Какое безумие понудило тебя оскорбить твоего государя?» "--- «Я не оскорблял государя, "--- отвечал Вавила, "--- ибо знаю, что преступление против величества предержащей власти превыше всякого неистовства, я только не допустил идолопоклонника, обрызганного кровью богомерзких жертв, оскорбить святыню Господню». Нумериан, вне себя от гнева, велел водить праведника по всей Антиохии, а народу "--- над ним ругаться.

Но какое торжество для человека Божия! Трое юных учеников\footnote{Имена этих юных страдальцев: Урван 12"~ти лет, Прелидиан 9"~ти лет и Епполоний 7"~ми лет.}, воспитанных святым Вавилой, следуют за ним, пред всем народом свидетельствуют, что они с великим наставником неразлучны в жизни и смерти и, презирая ласки и обещания Нумериана, не хотят оставить святого Вавилы. В юном возрасте совершенные мужи, они не устрашились ни мучений, ни смерти.

Вытерпев все неистовства народа и все мучения, учитель и ученики осуждены были, наконец, на смертную казнь. Когда пришли на лобное место, святой Вавила сказал: «\textit{Обратися, душе моя, в покой твой, яко Господь благодействова тя}» (Пс. 114, 6), "--- и, поставив пред собой отроков, повелел их \textit{предпослать к Богу}, в тех видах, чтобы они, лишившись наставника, не лишились твердости против ужасов смерти. Дети приняли венец страдальческий, а праведник воскликнул: «Се \textit{аз и дети, яже ми дал ecи, Боже}» (ср. Ис. 8, 18) "--- и сам преклонил главу свою под орудие смерти.

\section{Гнев Божий за оскорбление святых мощей}

Вблизи города Антиохии было приятное и веселое место, от природы осененное кипарисными и лавровыми деревьями и орошаемое быстрым и прозрачным источником. Оно называлось Дафной, от имени языческой девицы Дафны, за которой будто некогда гнался языческий бог Аполлон и тут настиг ее. Но как язычники в лице своих идолов поклонялись порокам, то и это происшествие заслужило новую честь Аполлону, и на том месте сооружен был ему великолепный храм, в котором лжебог на все вопросы давал ответы. Галл, брат Юлиана, соцарствовавший Констанцию, стараясь распространить христианство, соорудил тут же небольшую церковь и перенес в нее из Антиохийского собора мощи священномученика Вавилы и трех отроков, пострадавших с ним, дабы непрестанно стекавшийся народ, видя чудеса от гроба его, мало"=помалу обращался к вере Евангельской. С того времени словоохотный идол онемел и, сколько ни курили пред ним ладану, сколько ни закалывали тельцов, "--- не давал ответа.

Вскоре воцарился злочестивый Юлиан. Имея пламенную приверженность к богам язычества, он особенно благоговел пред Аполлоном как пред богом "--- покровителем просвещения, ибо Юлиан любил называть себя философом. Зная по слуху, что его любимого бога поразил недуг немоты, он уверился в том и лично, будучи в Антиохии и удивляясь столь необыкновенному, по его мнению, чуду, неотступно вопрошал идола о том и о другом. Но вместо всех ответов едва, наконец, получил один, что Аполлону препятствует говорить злой сосед, который лежит близ его капища. Это относилось к мощам священномученика Вавилы. Огорченный Юлиан сначала велел было разметать по улицам или предать огню святые останки человека Господня, но его объял священный ужас, от которого не могут защититься и самые буйные головы. Итак, по некотором размышлении, он приказал Антиохийским христианам унести гроб святого Вавилы и трех отроков обратно в соборную церковь, что и было исполнено.

После этого христоненавистник привел в прежнее состояние храм Аполлона, уставил всю рощу изображениями языческих сюжетов, пред которыми вера и благочестие должны были руками закрывать лицо свое, и намеревался сделать туда торжественный ход, чтобы снова освятить или, вернее, осквернить место, освященное христианскими молитвами. Но жив Господь, наказуя неправду и нечестие! В ночь на 22~октября упало с неба на капище пламя и мгновенно объяло все здания. Этот огненный вихрь в присутствии Юлиана пожрал храм, жертвенник и самого идола так, что остались одни растрескавшиеся камни. Богоотступник, вместо того, чтобы в столь чудном происшествии познать десницу Господню, приписал этот пожар мщению христиан и, чтобы это подозрение сделать неопровержимым, захотел выслушать пред судилищем языческих жрецов, которые, чередуясь день и ночь, охраняли капище. Но Господь разрешил тогда язык их на прославление истины. Идолослужители не сказали того, что хотел слышать Юлиан, за что и преданы были пытке. Однако и посреди мучений они не переставали говорить, что пламя упало с неба. То же подтвердили с клятвой и поселяне, которые на тот раз шли мимо Дафны, и уверяли Юлиана пред всем народом, что ночью, при ясном небе, вдруг низверглось множество огня прямо на капище.

Таким образом Господь отмстил за оскорбление святых останков Своего угодника и посрамил злостное намерение язычников. Но Юлиан, столь видимо вразумляемый от Того, \textit{Иже не хощет смерти грешника, но еже обратитися и живу быти ему} (Иез. 33, 11), насильно стремился к своей погибели. Он жестоко отомстил за свое посрамление на христианах антиохийских: закрыл их церкви, отнял священные сосуды, осквернил их употреблением при идольских жертвоприношениях и обнародовал указ, чтобы христиан впредь не называли христианами, но галилеянами, "--- а это на его языке означало презреннейшего человека.

\section{О том, насколько милостыня споспешествует нашему спасению}

У одного африканского князя был придворный чиновник, по имени Петр, который имел обязанность доставлять съестные припасы для стола княжеского, отчего нажил великое богатство. Но при всем том был столько скуп, что ни разу, как начал помнить себя, не подавал милостыни ни одному нищему.

Казалось, что человек столь жестокосердый погиб безвозвратно. Но милосердый Бог привел Петра на путь покаяния следующим образом. Однажды, отправляя во дворец печеные хлебы, он увидел пред собой нищего, который просил у него милостыню. Петр закричал на него с ругательством, но нищий не переставал его умолять и просить ради Христа. Тогда Петр начал озираться на все стороны, не попадется ли ему камень, чтобы ударить столь наглого, как думал он, бродягу, но так как камня вблизи не нашлось, то он в крайней досаде схватил хлеб и бросил в лицо нищему. Нищий подхватил хлеб и пошел, благословляя подавшего милостыню.

Чрез два дня после этого Петр разболелся так тяжко, что находился при смерти. В этом состоянии он узрел себя в видении истязуемым на некоем судилище. Перед ним стояли весы, на которые с одной стороны черные, злобные страшилища возлагали его деяния, а с другой стороны Ангелы Господни стояли в недоумении. Первые рукоплескали и хохотали, видя тяготу грехов, вторые же с унынием говорили между собой: «А мы что положим сюда? Нет ничего, кроме одного хлеба, который Петр бросил в лицо нищему». Потом эту милостыню возложили они на весы, но хлеб едва"=едва приподнял чашу, обремененную грехами. Тогда святые Ангелы сказали ему: «Иди, убогий Петр, и приложи к этому хлебу еще несколько хлебов, чтобы не похитили тебя наши соперники и не увлекли в геенну огненную». В это мгновение Петр пришел в себя и, размышляя о видении, познал, что оно было не мечтой расстроенного недугом воображения, но самой истиной, уроком милосердия свыше, ибо увидел на весах все грехи, содеянные им от юности. «Если один хлеб, с лютой досадой брошенный, столько мне помог, "--- обливаясь слезами, сказал он, "--- то сколько ублажает человека милостыня, во всю жизнь с благосердием подаваемая!» И с того времени поклялся пред Богом быть милосердым, а Бог, видя сердечное покаяние, вскоре восставил его от одра болезни.

И действительно, Петр так твердо соблюдал данный Богу обет, что положил на сердце не щадить и самого себя. Однажды встретился с ним незнакомец, обнищавший вследствие кораблекрушения, и, припадая к ногам его, просил прикрыть чем"=нибудь наготу его. Немедленно Петр снял с себя верхнюю одежду и отдал несчастному. Но так как незнакомцу нечем было не только одеться, но и насытиться, то он, желая сделать оборот, отдал одежду Петра купцу для продажи. По случаю Петр шел мимо и, увидев оную на торжище, весьма оскорбился. От печали в тот день не вкусил пищи и, заключившись в уединенной храмине, плакал и рыдал. «Господь не принял моей милостыни, "--- думал он, "--- видно, я недостоин того, чтобы пред Ним поминали мое имя убогие, возлюбленные рабы Его!» Скорбя и воздыхая, он уснул "--- и вдруг видит пред собой благообразного Мужа, паче солнца сияющего, крест на главе имеющего и в его одежду облеченного. «Видишь ли эту одежду?» "--- сказал ему явившийся. «Ей, Владыка! "--- отвечал Петр. "--- Она прежде была моя». "--- «Итак, не скорби, "--- продолжал явившийся. "--- Я от тебя принял ее. Я ношу ее и благословляю тебя за то, что Меня, погибающего от холода, ты одел оною». Удивился воспрянувший Петр и сказал сам себе: «Когда Христос заступает место убогих, то не умру и я, если соделаюсь един из них». После этого все свое имение он раздал нищим, освободил рабов и ушел в Иерусалим. Там, поклонившись Животворящему Гробу, упросил он именем Господним одного старца, чтобы продал его в рабство, а плату за него разделил нищим.

Долго святой Петр служил своему господину и остался бы тут до смерти, если бы не узнали его приехавшие в Иерусалим единоземцы. Тогда раб Божий, убегая славы человеческой, скрылся в неизвестное место и там пребыл до блаженной кончины. Един Бог ведает, где почивает многострадальное тело его, но Святая Церковь по всей вселенной восхваляет имя его\footnote{Память праведного Петра, бывшего мытаря (†~VI~в.), празднуется 22~сентября (5~октября).}.

\section{Прилежание и труд, а более молитва, все преодолевают}

Преподобный Сергий\footnote{Его память празднуется 25~сентября (8~октября).}, Радонежский чудотворец, в отроческих летах обучался чтению и пению вместе с братьями своими, старшим Стефаном и младшим Петром. Но сколько счастливо они успевали, столько Сергий был непонятлив и никак не мог поспевать за своими сверстниками, за лень его родители сильно укоряли и наказывали учителя. Нестерпимо горько бывает благонравному отроку, когда он всем сердцем и душой желает снискать просвещение, но встречает к тому препятствие в каких"=либо естественных затруднениях. Но чего не преодолеет прилежание и труд, а особенно самая искренняя молитва, от усердной души восходящая к Отцу светов?

Возбуждаемый нетерпением получить истинную мудрость, юный Сергий изыскивает благонадежное средство. Прибегает в молитве к Тому, \textit{Иже просвещает} всякого \textit{человека, грядущего в мир}, и просит, проливая слезы, да отверзет его разум разуметь писания и да расторгнет стесняющие его узы. \textit{Близ Господь всем призывающим Его}: Он скоро открывает Сергию благоприятный случай.

В один день сетующий отрок, будучи послан отцом своим для отыскания коней и ходя по разным местам, увидел некоего старца, почтенного саном пресвитерским, благообразного и сияющего душевной добротой, который, стоя под дубом, молился Богу Вездесущему. Сергий приблизился к нему и ожидал конца молитвы, не дерзая нарушить его собеседование с Богом. Но старец, когда престал от молитвы, сам вопросил святого отрока: «Чего ищешь или чего хочешь, чадо?» Сергий хотя послан был искать коней, но, имея всю душу, объятую желанием учения, отвечал старцу: «Более всего желает душа моя научиться грамоте, и я весьма скорблю, что не имею к тому понятия». Старец, услышав доброе намерение отрока и увидев сквозь завесу его лица доброту его души, воздвиг очи и руки свои к небу и произнес прилежную молитву, испрашивая юному Сергию просвещения свыше. После того дал ему часть просфоры, повелев оную съесть, и сказал: «Это тебе дается в знамение благодати Божией и разумения Святого Писания». С того времени дар Божий воздействовал в юном Сергии, и светом слова Божия столько просветилась мысль его, что вскоре сделался он первым наставником и просветителем своего времени.

\section{Дух смиренномудрия}

В одно время преподобный Сергий своими молитвами к Владыке живота и смерти воскресил умершего младенца. Отец, узрев дитя свое живым и будучи объят страхом и радостью, припал к ногам человека Божия и воздавал ему боголепное благодарение. Святой это свойственное Единому Богу дело не только себе не присвоил, но и самое мнение о том почел для себя тягостным. «Нет, чадо мое! "--- сказал он. "--- Сын твой не был мертв, но от стужи изнемог и ослаб, только тебе показалось, что дитя умерло. Будь уверен, что, будучи принесено в теплую келью, оно согрелось, и отнюдь не думай, что ожило, ибо прежде общего воскресения нельзя ожидать никому воскресения».

\section{Откровение преподобному Сергию о духовном его потомстве}

Некогда преподобный Сергий, стоя, по своему обыкновению, в полуночи на правиле и прилежно молясь Богу о своих учениках, услышал глас: «Сергий!» Удивленный старец перекрестился, открыл окно, чтобы увидеть, кто зовет его, "--- и узрел свет великий, простирающийся от небес к его лавре, так что ночь сделалась светлее дня. Тогда вторично он услышал тот же глас: «Сергий! Ты молишься о своих чадах, и восприято моление твое; зри и виждь число братий, во имя Святой Троицы собирающихся в обитель твою». В это мгновение святой Сергий узрел множество прекрасных птиц не только в лавре, но и окрест ее сидящих и сладкогласно поющих. Преподобный Сергий забыл свое земное бытие и думал, что он в раю посреди лика Ангелов, но вдруг услышал в третий раз: «Сергий! Сколько видишь здесь птиц, столько умножатся ученики твои; сколь разнообразна красота этих пернатых, столь чудно и различно они украшены будут добродетелями». С этим гласом видение прекратилось.

Глагол небесный оправдался на деле. Лавра преподобного Сергия всегда процветала великими по вере и благочестию мужами. Его духовные чада и потомки были наставниками всей России: сооружали святые обители; украшали архипастырские престолы; млеком духовным воспитывали юношество; гласом Евангелия созывали воинства на защиту отечества; питали тысячи граждан и ратоборцев\footnote{Все это было в страшную годину междуцарствия.}; сокрывали царей от врагов и злодеев\footnote{Например, Петра Великого.}; на глас страждущего отечества метали гром брани из тех рук, которые в мирное время простирали к небу в молитвах, и, не переставая служить Богу мира, единоборствовали с Голиафом\footnote{Таковы были Пересвет и Ослябя.}.

\section{Христова невеста}

Святая мученица Харитина\footnote{Память мученицы Харитины (†~304) празднуется 5~(18) октября.}, в нежном отрочестве лишившись родителей, была воспитана одним благочестивым старцем, по имени Клавдий, и поскольку она была благоразумна, кротка и послушна, то Клавдий любил ее, как родную дочь, и, радуясь ее добродетелям, назначил ей особые комнаты, куда собирались усерднейшие христианки того времени. Старшие из них поучали ее Закону Господню, изображая чудесную славу Его человеколюбия, долготерпения и будущее блаженство Его последователей, а младшие этому учились уже от святой Харитины.

В это время Диоклетиан воздвиг гонение на людей Христовых и повелел везде искать их на мучения, а особенно тех, которых разум, ревность по вере и добродетельная жизнь не могли укрыться от взоров идолопоклонников. Удивительно ли, что Харитина первая была оклеветана? За ней пришли стражи и, объявив повеление судилища, где старались обесславить поступки христиан, повлекли невинную девицу за собой. Возрыдал чадолюбивый Клавдий и, крепко держа ее в объятиях, дерзнул сопротивляться воинам. Тогда святая девица сказала своему благодетелю: «Отпусти меня, второй мой отец; отпусти вослед Сладчайшего Иисуса и не скорби, но паче радуйся, что буду благоприятная Богу жертва за свои и твои грехи». После этого она отдалась в руки жестокосердой стражи. Клавдий сопровождал ее, сквозь слезы и рыдания произнося только: «Помяни меня у Небесного Царя, когда предстанешь Ему в лике святых мучениц».

Домициан (имя судьи), при первом взгляде на святую Харитину, назвал ее упрямой христианкой, которая развращает других. На это неробкая девица отвечала: «Исповедую пред Богом и людьми имя Иисуса Христа и упрямство в этом случае почитаю высочайшей добродетелью; но совершенная ложь, будто я развращаю других: нет, я не развращаю их, но от разврата отвращаю». Домициан, удерживая гнев свой, советовал святой Харитине умилостивить богов, оскорбленных, как говорил он, ее отступничеством; предлагал милость царскую и, выхваляя молодость и красоту ее, увещевал пощадить оные. Но мученица возвела очи свои к небу и, оградившись крестом, сказала: «\textit{Тебя Единого, Иисусе, женише мой, люблю: Тебе только сраспинаюся, да и живу с Тобой}». Тогда"=то неправедный судья начал над ней ругаться: он велел обрезать ей волосы, но волосы в то же мгновение выросли пред очами всех. Пламень и железо были орудиями ее страдальчества, но Ангел укреплял рабу Господню. В волнах морских назначен был гроб ее, но влажная стихия огустела под ее стопами. Наконец, святая девица, прешед сквозь огонь и воду, помолилась Господу, да приимет душу ее, и уснула блаженною смертью.

\section{Любопытство святого путешественника}

Преподобный Иларион\footnote{Память прп. Илариона Великого (†~371~-- 372) празднуется 21~октября (3~ноября).}, имея от роду шестьдесят три года, по внушению Божию предпринял путешествие не для того, чтобы обозреть редкости света и чудеса рук человеческих, но чтобы обогатить себя сокровищами духовной мудрости. Оставив Палестину, он проходил высокие горы и знойные степи посреди рыкающих зверей и скитающихся разбойников. Посещал пустынных старцев и юных отшельников, чтобы от одних научиться, а других научить своим примером. Беседовал с пастырями и святителями, страждущими в заточении за исповедание Евангельских истин. Своими молитвами избавлял встречающиеся города и веси от гнева Небесного; всюду проповедовал благочестие и братолюбие. Наконец, пройдя три дня страшной пустыней, святой Иларион достиг высокой горы, где некогда обитал величайший из пустынножителей, преподобный Антоний.

Здесь, сопровождаемый двумя учениками в Бозе почивающего Антония, преподобный путешественник несколько времени молился над святыми его останками и потом восхотел обойти все места, запечатленные следами великого старца. С того времени, как Иларион удалился в пустыню, казалось, в первый раз любопытство объяло всю его душу. Исаак и Пелусион (имена Антониевых учеников) показывали ему все, что напоминало об их учителе. «На этом месте, "--- говорили они, "--- великий отец пел священные гимны, там безмолвствовал. Здесь молился, а там, сидя, плел кошницы; здесь отдыхал от трудов, а там вкушал краткий, но сладостный сон. Здесь принимал благочестивых посетителей, а там поучал братий житию равноангельскому. Эти древа насадил и это гумно устроил своими руками. Этот пруд ископал с великим трудом и потом; вот заступ, который употреблял святой старец во все время своего пустынножительства». Они пришли на то место, где стояло убогое ложе Антония: Иларион с благоговением и радостью возлег на него. Наверху горы были две каменные кельи, куда Антоний иногда удалялся от докук приходящего народа; туда по ступеням, иссеченным его рукой, возвели Илариона и, показывая разные деревья, обремененные плодами, рассказывали, когда какое дерево насаждал преподобный. Здесь святой Иларион отдохнул, воспоминая великого трудника, вкусил от плодов его и опять возвратился к его гробу. Прочитав тут с коленопреклонением молитву, он простился с Исааком и Пелусионом и пошел далее, продолжая благочестивое и душеполезное путешествие.

Похвально любопытство молодых людей, которые, путешествуя по знаменитым городам, посещают святилища наук и другие общеполезные учреждения, обозревают редкости искусств и чудеса природы и удивляются памятникам минувшего времени. Но если между этими занятиями они забывают беседовать с мужами, прославившимися благочестием и любовью к роду человеческому, то теряют из виду главную цель, для которой должно предпринимать путешествие.

\section{Улыбка праведника}

Святой мученик Карп\footnote{Память мученика Карпа, епископа Фиатирского (†~ок. 251), празднуется 13~(26) октября.}, перенося жестокие мучения, какие только зверонравное изуверство могло выдумать, весь облитый кровью, раздробляемый ударами и опаляемый огнем, вдруг улыбнулся, и эта небесная улыбка долго сияла на лице праведника. Все зрители, а особенно грубый и свирепый Валерий, главный над исполнителями казни, удивились и наперерыв спрашивали о причине столь неуместной, по их мнению, радости. «Я вижу благодать Христа моего, "--- отвечал святой Карп, "--- как же могу удержать восхищение моего сердца?»

И, поистине, можно ли не улыбаться победителю всех адских соблазнов и ужасов, когда пред ним отверсты небеса, и Господь, сидящий на престоле, окруженный Ангелами, призывает его в недра блаженной вечности? Меньшая скорбь уничтожается большей радостью. Все скорби земные сравнительно с наслаждениями будущей жизни "--- это не что иное, как мгновенная боль от укуса малого насекомого.

\section{Юные исповедники имени Христова}

Максимиан Галерий, убийца святого Анфима Никомидийского\footnote{Память священномученика Анфима, епископа Никомидийского (†~302), празднуется 3~(16) сентября.}, святого Петра Александрийского\footnote{Память священномученика Петра, архиепископа Александрийского (†~311), празднуется 25~ноября (8~декабря).} и святого Лукиана Пресвитера\footnote{Память преподобномученика Лукиана, пресвитера Антиохийского (†~312), празднуется 15~(28) октября.}, так ненавидел христиан, что не хотел оставить в покое и их детей, предписав законом, чтобы они или были поклонниками идолов, или погибли. Отовсюду собирая в Антиохию эти невинные жертвы, Максимиан старался обесславить пред ними имя Христово то воспитанием в языческих училищах, то блеском придворного кумирослужения, то ласками и угрозами.

Посреди этих ухищрений соблазна и тиранства донесено было гонителю христианства Максимиану о двух отроках, родных братьях, детях знаменитых родителей, как об упорных христопоклонниках. Император употреблял все средства преклонить их к нечестию: соблазнял их сладкой пищей от идольских жертв и не однажды приказывал осквернять их ею насильно. Но когда благочестивые дети сказали дерзновенно, что ни за какое благополучие в свете, даже за страх смерти, они не согласятся поступить вопреки наставлениям родительским, тогда приказано было нещадно бить их розгами. Потом император отдал их одному из придворных софистов\footnote{\textit{Софист} "--- лжемудрствователь; знаток искусства ведения спора, платный учитель философии и красноречия, могущий доказать в равной мере любую, в том числе заведомо неверную, точку зрения и мало интересующийся истиной.}, повелев испытать последнее усилие "--- обратить их в язычество убеждениями.

Философ"=софист, не полагаясь на свои доказательства, устроил некоторый едкий состав и, намазав им их головы, запер в горячей бане. Здесь, как огнем палимые или молнией поражаемые, умерли юные исповедники имени Христова. Посреди лютейших страданий они утешали друг друга и, прославляя Сладчайшего Иисуса, пали бездыханными "--- сначала младший, а потом старший, лобызая тело своего брата.

\section{Награды должно ожидать не в начале, но в конце подвига}

Авва Иоанн Фивский двенадцать лет служил одному удрученному летами и недугами старцу и почти не отступал от одра его. Невзирая на это, старец, по"=видимому, не уважал неусыпных трудов Иоанна и никогда не высказал ему даже обыкновенной благодарности. Но когда приблизился час его смерти, то, в присутствии собравшихся иноков взяв Иоанна за руку и возвышая, сколько можно, предсмертный голос, старец сказал ему: «Благодарю тебя, чадо мое! благодарю тебя, чадо мое! благодарю тебя, чадо мое!» Потом, поручив его покровительству старших отцов, промолвил: «Это Ангел, а не человек».

\section{Красота Евангельского слога}

Святой Александр, епископ Команский\footnote{Память священномученика Александра, епископа Команского (†~III~в.), празднуется 12~(25) августа.}, поучая людей Христовых, не старался о красоте слога и изящности выражения, но имел попечение о единой истинной пользе душ и для простого народа проповедовал хотя сильно, но весьма просто. Однажды случилось быть в Команах молодому афинскому софисту. Услышав проповедь Александра, он начал смеяться, что нет в ней афинской остроты и красноречия. Но Бог вывел его из заблуждения следующим образом. В одну ночь философ увидел в сновидении стадо голубей, белых, весьма благолепных, имеющих, по словам Псалмопевца, \textit{криле посребрене, и междорамия их в блещании злата} (Пс. 67, 14), и услышал глас: «Се глаголы святителя Александра, которым ты, легкомысленный человек, столько смеешься!» Софист, воспрянув от видения, так устыдился своего неразумия и суетности, что немедленно пошел к святому Александру и, сердечно раскаиваясь, просил отпустить ему оскорбление, учиненное Христову проповеднику.

\section{Истинное человеколюбие}

О преподобном Исидоре, пресвитере скитском, повествуют, что он, видя брата слабого, или нерадивого, или малодушного, или буйного, изгоняемого от старцев за эти пороки, всегда приглашал его к себе и с помощью то терпения, то дружелюбных замечаний наставлял его на путь спасения. Сколь редко такое усердие и искусство!

\section{Безопасное место для сбережения сокровищ}

Одна знаменитая римлянка, по имени Маркелла, не имея у себя наследников, раздала все свои богатства бедным семействам и святым обителям. Вскоре готы опустошили Рим. Услышав о том, праведная Маркелла воздела руки свои к небу и радостно воскликнула: «Благодарю Тебя, Господи Боже мой, что Ты, милосердуя о мне, недостойной, заранее привел наследство мое в безопасность и сохранил от расхищения варварского».

Какое величие души в этих словах рабы Христовой! Исполнив Евангельскую заповедь: \textit{Скрывайте же себе сокровище на небеси, идеже ни червь, ни тля тлит, и идеже татие не подкопывают, ни крадут} (Мф. 6, 20), она достойно радовалась, что несчастие Рима застало, а не сделало ее неимущей.

\section{Бестрепетный пастырь стада Христова}

Святой Аверкий\footnote{Память равноапостольного Аверкия, епископа Иерапольского, чудотворца (†~ок. 167), празднуется 22~октября (4~ноября).}, епископ Иерапольский, ночью сокрушил всех идолов в капище Аполлоновом и затем спокойно возвратился в свой дом, как победитель, совершивший волю Небесного Подвигоположника Иисуса Христа.

Поутру узнали, кто был виновник истребления идолов, и снарядили стражу, чтобы взять святого Аверкия.

Между тем некоторые из благочестивых соседей известили о том святителя и просили его где"=нибудь укрыться. Но бестрепетный пастырь отвечал: «Иисус Христос повелел Своим апостолам проповедовать слово спасения дерзновенно: я ли осмелюсь бояться восстающих на Него? Братия! Мы идем под знаменем Того же Господа; Он наш Помощник и Покровитель: да не отчаиваемся убо, да не страшимся». После этого, окруженный христианами, вышел он на средину города и начал всенародно прославлять Бога, в Трех Лицах поклоняемого.

Городские власти и народ, услышав о дерзновении святого, еще более рассвирепели и с яростью устремились на него. Но в то самое время, когда хотели его схватить, Аверкий именем Христовым исцелил двух беснующихся "--- и гонители его изменили свою ярость. «Един есть Бог Истинный, Которого проповедует Аверкий!» "--- воскликнули они и припали к стопам его, умоляя, дабы научил их веровать в Иисуса Христа. Только некоторые из начальствующих удалились безмолвно, чтобы не подвергнуться той участи, какую готовили исповеднику Христову. Аверкий продолжил проповедь Евангелия даже до вечера, благословил народ и повелел готовиться к Святому Крещению.

С того времени святой Аверкий беспрепятственно отправлял великую должность архиерейства. Когда же для отвращения распространившихся болезней извел он своими молитвами источник теплых и целительных вод, то весь Иераполь присоединился к царству Иисусову. Но совершенное торжество христианской веры в его пастве наступило тогда, когда чудотворец Божий был призван в Рим и, против чаяния всех придворных врачей, восставил от смертного одра дочь императора Марка Аврелия. Благодарный государь немедленно повелел остановить гонение на христиан по всей вселенной и, невзирая на общую ненависть к ним язычников, определил, по просьбе святителя, отпускать из государственной казны бедным иерапольским христианам ежегодно по три тысячи мер пшеницы\footnote{Эта милостыня продолжалась до Юлиана Отступника, который прекратил ее указом.} и построить для них бани при теплых водах.

Возвратившись в отечество, этот великий святитель жил в преподобии, правде и подвигах благовествования до глубокой старости. И с миром предал святую душу свою в руце Господни, оставив пример того, что вера и добродетель хотя имеют определенное им торжество только на небеси, однако часто торжествуют и на земле вопреки всем усилиям нечестия.

\section{Инок, ходатайствующий за свою родину}

В царствование Юстиниана Великого взбунтовались самаряне и под предводительством одного из богатейших сограждан, по имени Юлиан, напали на палестинских христиан, выжгли города и села, разорили до основания церкви и вообще произвели ужасное кровопролитие. Император наказал их строго, но после, поверив клевете некоего Арсения, родом самарянина же, человека при дворе случайного, будто сами палестинские христиане были причиной возмущения, обратил на них весь свой гнев. Вся Палестина ожидала неприятных последствий.

Тогда патриарх Иерусалимский Петр предложил Савве Освященному отправиться в Царьград, чтобы разубедить императора Юстиниана. Девяностолетний старец, предпочитая благоденствие отечества собственному спокойствию, с радостью принял на себя тягость путешествия и, будучи принят от царя с уважением, какого достойны святые люди, исполнил свое дело так, как желали христиане. Юстиниан не только был выведен из заблуждения, но и осудил на смерть клеветника, и это бы совершилось, если бы Савва же не ходатайствовал за него пред царем. Отпуская святого старца в обратный путь, император хотел оказать ему всевозможную милость: он приказал выдать из государственной казны большие деньги для расширения, украшения и продовольствия его обители. Но мудрый инок желал не богатств себе и своей братии, а общей пользы христианам и соотечественникам. Благодаря государя за щедрость, он с покорностью просил отменить на несколько лет собираемые с Палестины подати, чтобы разоренные от самарянских мятежей жители могли поправиться; просил также соорудить в Иерусалиме странноприимный дом для христиан, издалека приходящих ко Гробу Господню, учредить больницу для убогих и дать искусных врачей, построить крепости от стороны варварских народов и снабдить оберегательным войском. «За эти благодеяния, "--- пророчески сказал он Юстиниану, "--- Сам Бог излиет на тебя Свои благодеяния и поможет присовокупить к державе твоей те страны, которых лишились прежние государи». Император беспрекословно на все согласился. Когда давал он повеление государственному казначею и прочим чиновникам, чтобы сделали распоряжение по просьбе Саввы, человек Божий, несколько отступив, начал усердно молиться. Удивляясь этому, один из его спутников с некоторой укоризной сказал: «Отче! Государь столько заботится о твоем прошении, а ты, как посторонний человек, занимаешься другим». "--- «Чадо! "--- отвечал Савва. "--- Они занимаются своим делом, чтобы оказать милость людям, а мы должны заниматься своим делом, чтобы испросить им милость Небесную».

Молитвы преподобного старца не остались у Бога тщетны. Благодетельный Юстиниан вскоре имел удовольствие видеть исполнение пророчества. Его военачальники возвратили достояние империи: Велизарий разрушил царство вандалов в Африке, а Нарзес "--- царство готов в Испании.

\section{Хвала богохульных уст для любящих Бога есть хула\footnote{Из жития св. мученика Арефы (†~523), память которого празднуется 24~октября (6~ноября).}}

Когда злочестивый Дунаан, князь Омиритский, подверг всем ужасам мучений святителя Христова Арефу и его христолюбивое стадо, тогда между прочими страдальцами представлена была одна благородная и добродетельная вдовица с двумя юными дочерьми. Мучитель с удивлением посмотрел на красоту этих христианок и, невольно умягчив свирепое свое сердце, сказал кротко: «Я всегда слышал с удовольствием о твоем разуме; теперь и самое лицо твое свидетельствует истину носящейся о тебе славы. Итак, советую тебе, как царь и друг, поступить сообразно с твоими достоинствами: оставь Назарея, поносно жившего и умершего, Которому ты поклоняешься, и исповедуй единого с нами Саваофа\footnote{Этот Дунаан был иудей.}. Тогда будешь подругой царицы, моей супруги, и твои дочери будут воспитаны, как дети царские. Все призывает вас на эту высокую степень, только христианское суеверие, зараза низкой и невежественной черни, преграждает вам путь к благополучию».

Благочестивая женщина, которая при каждом богохульном слове с ужасом возводила очи свои к небу, с дерзновением Христовой подвижницы отвечала Дунаану: «Государь! Тебе бы должно было почтить Того, Кто даровал тебе власть, порфиру и диадему, Кто даровал тебе самое бытие и жизнь: тебе надлежало бы первому почтить Иисуса Христа, Сына Божия. А ты, неблагодарный, столь дерзко злословишь твоего благодетеля! Бойся, чтобы земля не поглотила тебя, как это было с Дафаном и Авироном. Но ты, погибая сам, хочешь и всех погубить: тем ближе над тобой мстящая рука Иисусова. Ты обещаешься почтить меня всеми отличиями, но всуе трудишься! Эта честь для меня ужаснейшее бесчестие. Государь! Я даже не хочу, чтобы меня хвалил твой язык: несносна хвала хулителя Божия. Прославь прежде имя Иисуса Христа, тогда малейшая твоя хвала будет для меня истинно царскою милостью».

Злочестивый Дунаан исполнился гнева и ярости, изрек зверский приговор "--- и укрепляемая Христом вдовица и ее дочери спокойно пошли на мучения и смерть.

\section{Примерная супруга}

Флакилла, супруга греко"=римского императора Феодосия Великого, происходила от древнего дома Елианов и в числе своих предков считала римского императора Елия Адриана. Это преимущество еще более возвеличилось тем, что она сама сделалась императрицей всего известного тогда мира. Но для Флакиллы эти преимущества были бы совершенно ничто, если бы они, окружая светом только ее одну, не могли расширить пределов для подвигов ее благочестия и любви христианской. Благотворительность составляла истинную радость ее души. Не было в Царьграде бедного семейства, которому бы она несколько раз в году не подала руку помощи. Она имела у себя нарочных людей, честных и попечительных, которые должны были наведываться, не приключилось ли с кем какое"=либо несчастие. Правители отдаленнейших областей обязаны были доносить ей обо всех случаях, в которых требовалась ее материнская помощь. Вдовицы, сироты, жены, дети и матери убитых на войне, недужные и разорившиеся, даже незлонамеренные преступники составляли как бы семейство Флакиллы, которому она благодетельствовала разными способами, какие только для нее были возможны.

Часто она сама посещала убежища недугующих, утешала их и из своих рук подавала пищу и лекарства. Часто присутствовала, когда перевязывали их раны. И когда представляли ей, что есть дела не менее благочестивые, но более приличные ее сану, что ей не нужно и непристойно нисходить до самых крайних проявлений человеколюбия и что она может поручить это другим, "--- тогда Флакилла отвечала: «Попечение о делах Церкви и отечества, равно как и награждение заслуг, я оставляю императору. Пусть он во славу благочестия и для блага подданных употребляет свою царскую славу, а для меня довольно чести, если принесу в жертву Богу малое попечение и смиренное служение моих рук. Я не могу иначе засвидетельствовать Ему моей благодарности, как нисходя с престола, на который Он меня возвел, для служения Ему в лице нищих».

Чрез это смирение она получала всегда большое почтение и власть над сердцем своего супруга. Но эту власть употребляла она только для подавания ему советов, разговаривая с ним о Законе Божием, который знала в совершенстве, и вдыхая в него такую же ревность по вере, какою сама пылала. Часто между разговорами о прошедшем времени приводила ему на память, кто он был, и чрез то побуждала его не употреблять во зло настоящего величия. Желание видеть супруга своего благочестивым было в ней сильнее, нежели радость, что он был обладателем света.

Насколько все эти добродетели делали ее любезной народу, показала смерть, которая постигла ее в цветущих летах. Как скоро разнесся слух о ее кончине, весь город облекся в печаль, и народ бежал толпами ко дворцу. Сетующий Феодосий не мог другим средством снискать себе утешение, как оказав ей всевозможные почести при погребении. Святой Григорий Нисский говорил надгробную речь, в которой называл ее «утверждением Церкви, сокровищем убогих и прибежищем несчастных».

\section{Дом земной и дом Небесный}

Когда святого Акепсима, епископа Персидского\footnote{Мученик Акепсим епископ жил в IV~в. по~Р.Х. Память его празднуется 3~(16) ноября.}, взяли под стражу за исповедание имени Христова и повели из дому, то некто из знакомых, подступив к нему, сказал на ухо: «Сделай какое"=нибудь завещание о твоем доме». Но святой Акепсим отвечал: «Друг мой! Странное и безумное дело "--- заботиться об этой убогой храмине, когда Царь Небесный повелевает мне переселиться в дом славы, \textit{иже есть на небесех»}.

\section{Нелицеприятие при раздаянии милостыни}

Однажды церковный строитель донес Иоанну Милостивому\footnote{Память свт. Иоанна Милостивого, патриарха Александрийского (†~620), празднуется 12~(25) ноября.}, что между убогими людьми к ним приходят за милостыней иногда люди хорошо одетые, и требовал разрешения: надобно ли им подавать, как и другим нищим? Иоанн отвечал на это: «Если ты "--- раб Христов и послушник смиренного Иоанна, то подавай милостыню так, как повелел Христос, невзирая на лица требующих от тебя помощи. Мало ли есть сребролюбцев, одетых в разодранные рубища? И мало ли есть бедных, достойных всякого сожаления, всякой помощи, одетых опрятно? Встречаются тысячи случаев, что тот, кто вчера, при всех недостатках, еще мог обойтись без помощи других, сегодня принужден просить оной! Или ты, может быть, опасаешься, что на беспрестанные милостыни недостанет церковного имения? Я не хочу разделять с тобою твое маловерие. Мы даем не свое, мы даем Христово: а кто может быть богаче Иисуса?»

\section{Каково подаяние, таково и воздаяние}

Однажды, когда святой Иоанн Милостивый шел в церковь, приступил к нему один гражданин, прежде бывший богатым, но по некоторым обстоятельствам обнищавший, и просил помощи. Патриарх, сжалившись, что этот честный и благородный человек так разорился, приказал строителям дать ему пятнадцать литр золота, но они, опасаясь, чтобы не оскудела церковная сокровищница, выдали только пять литр. Когда же патриарх возвращался из церкви, остановила его одна богатая вдова и вручила ему бумагу, в которой было написано, что она дает в церковь пятьсот литр золота. Иоанн, читая хартию, познал от благодати Духа Святого, что она отдает не все, что прежде положила в уме своем. Придя домой, он призывает церковных строителей и спрашивает, сколько они дали обнищавшему гражданину. «Пятнадцать литр золота», "--- отвечали строители. «От вас потребует Господь то, что удержала у себя благочестивая вдовица», "--- обличая их ложь, сказал святитель. Потом призывает к себе вдовицу и спрашивает: «Скажи нам истину, сколько, из любви твоей к Богу, ты намерена была принести в церковь золота?» Вдовица, видя, что ее намерение не утаилось от святого, отвечала: «Свидетельствуюсь Богом, владыка святой, что за несколько дней в этой бумаге у меня было написано тысяча пятьсот литр золота, которые я желала передать в твои святые руки; но сегодня, разогнув хартию, я увидела, что тысяча литр, не знаю, как и кем, изглажена. Я подумала, что не благоволит Господь отдать твоему святейшеству более пятисот литр, и так поступила». Экономы, услышав это, устыдились и ужаснулись своей неправды, а святитель, напутствовав благословением вдовицу, сказал им: «Вы не послушали меня и удержали десять литр золота: за это Господь удержал от нас в руках этой благотворительницы тысячу литр золота. Итак, помните: какова милостыня, таково от Бога и воздаяние, "--- помните и страшитесь Суда Божия, ибо сколько бедных от вашей неправды теперь лишились пропитания!»

\section{Слушай бедных, если хочешь, чтобы тебя слушал Бог}

Однажды, когда Иоанн Милостивый шел в церковь, остановила его бедная женщина и, горько жалуясь на обиды, причиненные ей одним родственником, просила защиты у святого патриарха. Но так как было уже время начать богослужение, то сопровождавшие Иоанна советовали ему выслушать вдовицу тогда, когда он возвратится домой. Но человек Божий отвечал: «А меня послушает ли Бог, если я не восхощу послушать ее?» "--- и не двинулся с места, пока не разобрал дело и не приказал обидевшему удовлетворить бедную женщину.

\section{О том, как трудно исправить сребролюбца}

Иоанн Милостивый, узнав, что один епископ, по имени Троил, был весьма сребролюбив и скуп, пригласил его в больницу "--- посетить убогих и недугующих. Приметив там, что Троил имеет с собой деньги, святой сказал ему: «Владыка! Вот время и случай утешить тебе бедную братию, подав им милостыню». Троил, опасаясь обнаружить свою скупость, начал раздавать золото и, оделив всех, издержал тридцать литр, но, возвратившись домой, раскаялся о добром деле и почувствовал такую скорбь, что лег в постель. Через какое"=то время Иоанн позвал его к себе на обед, но Троил отговорился болезнью. Иоанн догадался, что и отчего с ним сделалось, и, взяв с собой тридцать литр золота, пошел навестить его. «Я принес тебе деньги, которые ты, по моей просьбе, раздал в больнице: возьми их и своей рукой напиши, да будет от Господа мне та награда, которая следовала тебе». Троил, увидев золото, затрепетал от радости и немедленно написал: «Боже милосердый! Даждь мзду господину моему Иоанну, патриарху Александрийскому, за золото, которое я, грешный Троил, раздал в больнице, ибо он мое возвратил мне». Получив деньги, Троил восстал с одра, как будто никогда не был болен, и пошел к патриарху обедать.

Иоанн, угощая его, в сердце своем молился Богу, да исцелит его от столь ужасного сребролюбия, и Бог услышал молитву его. В следующую ночь Троил узрел в сновидении великолепный дом, неизреченно украшенный, у которого над воротами было написано золотыми буквами: «Обитель и покой вечный Троила епископа». Взыграло от радости сердце Троила. Но внезапно явился некий благообразный и грозный муж и сказал бывшим тут слугам: «Господь всего мира повелел мне изгладить имя Троила и вместо него начертать имя Иоанна, патриарха Александрийского, который купил у него этот дом за тридцать литр золота. Итак, перемените надпись». Троил вскричал от ужаса и жалости "--- и воспрянул ото сна. Познав всю цену дома, который потерял на Небе, он начал плакать и укорять себя в златолюбии. Едва дождавшись утра, пошел он к блаженному Иоанну, объявил ему о своем сновидении, дал клятву исправиться "--- и сдержал свой обет, ибо с того времени был щедр и для всех милостив.

\section{Не только сами подающие милостыню, но и дети их не оставляются Богом}

Святой Иоанн Милостивый услышал, что один молодой человек, сирота богатых родителей, живет в крайней бедности. Услышал, что эта бедность постигла его потому, что родители щедрой рукой расточали свое имение на убогих, а оставшуюся часть расхитили опекуны; услышал, что юноша добронравен, благочестив и не ропщет на свою участь. Человек Божий положил на сердце подать ему помощь, но так, чтобы юноша не мог считать это милостыней. Он призывает к себе эконома и, взяв с него клятву хранить тайну, приказывает изготовить будто бы давно сделанное духовное завещание от имени некоего Феопемпта, в котором патриарх и убогий сирота, как его родственники, назначены наследниками имения. С этой бумагой он посылает эконома к юноше.

Эконом исполнил волю патриарха следующим образом. Будто сам от себя призвав юношу, он сказал ему: «Ужели не знаешь ты, что патриарх тебе родственник, и страдаешь в такой нищете?» Юноша изумился и почел слова его за насмешку. Но эконом, показывая духовную, будто бы в тот день нечаянно между прочими бумагами им найденную, сказал: «Если ты, друг мой, стыдишься сам открыться перед святителем, то я доложу о тебе». Убежденный сирота обрадовался и начал просить, чтобы это родство объяснил он патриарху.

Когда эконом объявил Иоанну, что его приказание исполнено, тогда патриарх повелел привести к себе убогого юношу и, приняв ласково, сказал: «Прости меня, добрый юноша, что я так долго, по неведению, оставлял тебя в бедности. Я помню, что дядя мой имел сына, но я не знал, где он находится, а ты, по своей кротости, не хотел ко мне явиться. Благодарю тебя, "--- обратившись к эконому, продолжал он, "--- что ты, отыскав бумагу, снял с души моей этот грех и нашел мне родственника».

Святой Иоанн не только отдал доброму юноше часть, назначенную ему в мнимом завещании, но и купил ему дом и все потребное, сочетал с благородной девицей и дал почетную должность. Эта тайна между экономом и патриархом хранилась до самой кончины святителя. Тогда только сын благочестивых родителей узнал, что своим счастьем обязан он не родству с патриархом, но Тому, Кто не хочет видеть \textit{праведника оставлена, ниже семене его просяща хлебы} (Пс. 36, 25).

\section{Защитник Церкви и отечества}

У императора Аркадия был военачальник, родом из варваров, по имени Гаина, который, хотя и принял христианскую веру, но держался Ариевой ереси. Будучи храбр на войне и любим наемными войсками, он имел великую дерзость и часто заставлял "--- даже императора "--- себя опасаться. Этот"=то сильный, на все решительный человек непрестанно просил Аркадия, чтобы позволил арианам иметь в Царьграде церковь. Отказать ему было опасно для престола, а позволить "--- опасно для Православия. Аркадий, откладывая под разными предлогами удовлетворение его просьбы, наконец, прибегнул к Иоанну Златоусту\footnote{Память свт. Иоанна Златоуста празднуется 13~(26) ноября.} и требовал совета. Решили, чтобы патриарх был у царя, когда Гаина будет просить церкви, что и случилось вскоре.

Дерзкий и нетерпеливый варвар начал требовать милости арианам, как должного воздаяния за воинские подвиги и свою храбрость. Но великий святитель немедленно остановил его. «Если царь хочет быть богобоязнен, "--- гласом веры сказал он, "--- то не имеет права над Церковью, в которой от Бога поставлены духовные власти. Если же тебе нужна церковь, то войди в какую хочешь и молись. Они все отверсты пред тобой». "--- «Но я другого исповедания, "--- возразил Гаина, "--- и потому хочу иметь в столице с моими единоверцами особенный храм; надеюсь, что государь исполнит мое прошение. Владыка святой! Ты сам знаешь, сколько я подъял трудов, сражаясь за ваше отечество, проливая мою кровь и полагая за царя душу». "--- «Никто хвалы твоей упразднить не может и не хочет, "--- отвечал ему Иоанн, "--- но нельзя умолчать и о том, что ты за все подвиги уже принял в воздаяние честь, славу, сан и дары». После того, видя, что варвар начал разгорячаться, продолжал: «Кто имеет совесть, тот должен размыслить, что он был прежде и чем сделался ныне, какие имел богатства по ту сторону Дуная и какие сокровища имеет в нашем отечестве. Не хлеб ли там ел с водой и не всего ли света плодами, птицами и рыбами ныне услаждает свой вкус? Не был ли там простым человеком и не учинился ли здесь вельможей? Гаина! Вот воздаяние за труды твои! Будь же благодарен царю и служи верно его державе, а церковных даяний за мирское служение не проси. Дерзнув на это, ты подвергаешься опасности быть обвиненным в оскорблении величества, ибо хочешь соделать твоего государя виновным в святотатстве». Выслушав, хотя и с великим принуждением, этот урок истины, Гаина имел столько духа, что обратил разговор на дела, касающиеся воинства, и с того времени уже не просил для ариан церкви. Аркадий, удивляясь убедительной мудрости Иоанна, благодарил его от всего сердца.

Но меньше нежели через год Гаина или в отмщение за этот отказ, или по общей ненависти варваров к Греции и по их хищному и мятежному духу оказался вероломцем. Стоя на границах государства с войсками, составленными по большей части из варваров же, он вдруг поднял оружие и пошел прямо к Царьграду. Аркадий, не имея готового войска, не знал, что начать, и опять прибегнул к Иоанну Златоусту с просьбой, чтобы употребил против варвара то же оружие слова, которым недавно защитил Православие. Иоанн, хотя и знал, что прогневал Гаину, хотя представлял все, что могла произвести его мстительность, но, готовый положить душу свою за царя и отечество, безбоязненно пошел навстречу мятежнику. Бог содействовал рабу Своему. Златоуст укротил богомудрыми речами гордого варвара, из волка претворил его в агнца и был миротворцем между царем и Гаиной. Аркадий по необходимости простил мятежника, а Гаина дал клятву с того времени служить верно.

\section{Мученик"=младенец}

Святой мученик Роман\footnote{Св. мученик Роман (†~303), диакон Кесарийской церкви, замучен в Антиохии, куда пришел проповедовать Евангелие, в царствование Галерия и Максимиана. Память его празднуется 18~ноября (1~декабря).} терпел все мучения с удивительной твердостью духа и только возводил в безмолвии очи свои к небу. Но, когда мучитель начал укорять его безумием, называя христианскую веру изуверством, мученик вступился за честь имени Иисусова и, показав на одного младенца, стоявшего вблизи, спокойно сказал Асклипиаду (имя мучителя): «Это юное отроча без сравнения разумнее тебя, старого безумца. Оно хотя и малолетно, но знает Истинного Бога; ты же исполнен лет, а Бога не ведаешь». Удивленный столь внезапной укоризной, мучитель подозвал к себе отрока и спросил: «Какого, друг мой, ты почитаешь Бога?» "--- «Иисуса Христа», "--- отвечал отрок. «Да чем же лучше ваш Христос всех богов наших?» "--- опять спросил Асклипиад. «Тем, что Он есть Истинный Бог и всех нас создал и искупил, "--- сказал младенец, "--- а ваши боги есть не что иное, как живописный холст или изваянная медь». Так в юном исповеднике имени Христова действовал Дух Святой, \textit{совершая из уст его хвалу}, да посрамит нечестивого Асклипиада и всех идолопоклонников. Мучитель не знал, что более отвечать, и прибег к обыкновенному средству: приказал нещадно бить его розгами. Изнемогшее под ударами дитя наконец запросило пить. Но вдруг из толпы вышла мать его и начала его увещевать, чтобы мужественно терпел все мучения. Когда же еще более раздраженный этим мучитель приказал отсечь ему голову, вторая Соломония\footnote{Имя этой мужественной матери неизвестно. Юный мученик, ее сын, именовался Варул. А Соломония, известная также по мужеству, была мать семи мучеников Маккавеев: Авима, Антонина, Гурия, Елеазара, Евсевона, Алима и Маркелла, которые вместе с ней и с учителем своим Елеазаром пострадали от Селевка, сына Антиоха Великого, за 173~года до~Р.Х.}, взяв его на руки, понесла на место казни. Обнимая и лобызая, она утешала его и укрепляла, чтобы не страшился, видя меч над своей головой. «Не бойся, сладчайшее чадо мое, "--- говорила она, "--- не бойся смерти; ты не умрешь, но будешь вечно жив: из"=под секиры приимут тебя Ангелы и взлетят с тобой в райский вертоград, в жилище вечного блаженства. Там обымет тебя Христос; там будешь другом всех святых, и я туда же приду к тебе "--- скоро, весьма скоро». Уговаривая так свое дитя, мужественная мать принесла его на место усекновения. Юный страдалец умер, не испустив вопля, не оказав страха; и она, взяв тело его, омыла сладкими слезами и, погребая, ликовала, что младенец пролил за Христа Спасителя кровь, от нее принятую.

\section{Духовное братолюбие}

Святой мученик Валериан\footnote{См. под 22"~м числом ноября.}, обращенный в христианскую веру своей невестой, с таким усердием предал свою душу и сердце Спасителю мира, что удостоился узреть Ангела Божия, который вещал ему: «Ты не упорствовал против истин, проповеданных тебе от девицы, потому Бог послал меня к тебе, да приимешь от Него, чего восхощет душа твоя». Восхищенный Валериан, поклонившись Ангелу, отвечал: «Нет в мире этом ничего для меня любезнее брата моего Тивуртия. Итак, молю Господа, да избавит его от гибельного идолопоклонства, да обратит к Себе и да соделает обоих нас столь же совершенными в исповедании Его имени, сколь совершенна обрученная мне невеста». Небесный посланник порадовался желанию Валериана и сказал: «Благоугодно Богу твое прошение: как тебя Он спас чрез девицу, так чрез тебя спасет и брата твоего, и всех вас сподобит венцов Царствия Небесного».

Где укрепляет слабого человека Небесная благодать, там не имеют никакой тягости все труды, для других неодолимые, "--- и святой Валериан в тот же день обратил на путь истины брата своего Тивуртия, да трое единодушно славят Единого Бога, в Троице поклоняемого.

\section{Место свято там, где человек живет богоугодно}

Святой Алипий\footnote{Память преп. Алипия Столпника (†~640) празднуется 26~ноября (9~декабря).}, диакон Адрианопольской Церкви, за благочестие и добродетели был любим епископом и всеми христианами. Духовное начальство возлагало на Алипия все важные и касающиеся богоугодных заведений дела, а граждане адрианопольские имели к нему такое уважение, которое можно было назвать благоговением. Но при всех преимуществах, сердцу Алипия недоставало важнейшего преимущества "--- совершенного уединения.

Побуждаемый этим святым желанием, вскоре Алипий, тайно от всех, попросив только благословение у своей матери, ушел из Адрианополя, чтобы водвориться в какой"=нибудь из отдаленных пустынь. Но епископ, узнав о его отшествии, чрезвычайно опечалился; церковнослужители сетовали, что лишились сотрудника, украшавшего их братство. Все граждане почитали потерю его невознаградимой и послали искать его. Вскоре Бог открыл убежище святого Алипия, и они то просьбами, то угрозами уговорили его возвратиться в отечество. Алипий сетовал сердечно, что не совершилось его желание, и помышлял опять скрыться.

Но Бог остановил его намерение следующим образом. В одну ночь Алипий до того восскорбел, что начал плакать и в этих слезах заснул. Вдруг явился ему Ангел Божий и сказал: «Не скорби, Алипий, что ты возвращен сюда от желанного для тебя пути; Бог устроил это для пользы твоего отечества, а любить отечество и служить ему есть непременный долг каждого христианина. Кроме этого, да будет тебе известно, что \textit{там только свято есть место, где человек, любящий Бога, благочестиво и богоугодно жити начнет»}. Будучи утешен этим видением, Алипий перестал скорбеть, и провел всю жизнь на своей родине, прилежно подвизаясь и работая для Бога и людей.

\section{Дерзновение за веру}

Греческий император Константин, по прозванию Копроним, стараясь истребить Православную веру и на развалинах ее утвердить иконоборную ересь, надеялся не иначе успеть в своем намерении, как преклонив на свою сторону преподобного Стефана Нового\footnote{Преподобный и исповедник Стефан Новый пострадал в 767~г. по~Р.Х. Память его празднуется 28~ноября (11~декабря).}, которого греки и римляне почитали, как единого из древних великих Отцов. Для этого он послал к нему своего тайного советника, именем Патрикия, и искусного витию Каллиста и с ними отправил в дар не серебро и не золото, "--- ибо знал, что человек Божий не имеет нужды в сокровищах, "--- но различные овощи и плоды. Стефан принял дары царские, желая, чтобы благодать Божия просветила душу Константина. Но, когда Патрикий и Каллист объявили царскую волю, чтобы святой подписал деяния собора, состоявшего из иконоборцев, Стефан сказал с апостольским дерзновением: «Собора вашего, исполненного празднословия и лжи, я не подпишу, иначе забвенна буди десница моя! И никогда его не похвалю, в противном случае да прилипнет язык мой к гортани моей! Скажите государю, что отнюдь не нареку горькое сладким, тьму "--- светом, да не навлеку Божией клятвы на главу мою. За святые же иконы готов умереть и небрегу о царском прещении». Потом простер руку свою и, согнув длань, продолжал: «Если бы я имел в себе крови одну только эту горсть, и ту не пожалел бы пролить за икону Христову: скажите об этом государю. А эти дары отнесите обратно; не нуждаясь в дарах его, ответствую словами Божественного Писания: \textit{Елей же грешнаго да не намастит главы моея} (Пс. 140, 5)». Сказав это, преподобный Стефан ушел в другую келью, а посрамленные Патрикий и Каллист возвратились к Константину без всякого успеха.

\section{Священный ужас пред Святыней Тела и Крови Господней}

В одном кипрском монастыре, называемом «обитель Филоксенова», был старец, родом из Митилены, по имени Исидор, который беспрестанно плакал; и когда прочие иноки просили его, чтобы он хоть немного перестал плакать, тот, не внимая никаким увещаниям, обыкновенно говорил: «Я столь ужасный грешник, какого не бывало от Адама и до этого дня». Однажды пришел туда авва Иулиан с некоторыми из своих учеников и, желая утешить рыдающего старца, между прочим сказал ему: «Будь уверен, любезный о Христе брат, что, кроме Бога, все грешники». "--- «Увы мне! "--- отвечал старец. "--- Ни летописи мира, ни устные предания не представляют такого греха, какой сделан мной: внемлите, что открою вам, и помолитесь за меня, окаянного. Живя в мире, я имел жену, и оба мы следовали учению Севира\footnote{\textit{Севир Антиохийский} "--- один из основателей еретического монофизитского движения (жил в VI~в.), учивший, что во Христе из двух естеств (Божественного и человеческого) при Воплощении составилось одно смешанное естество.}. Однажды, возвратясь домой, я не застал жены моей и, узнав, что она ушла к подруге, чтобы с ней приобщиться Святых и Животворящих Таин в Православной Церкви, побежал туда; увидев, что жена моя только что приняла святую часть, схватил ее за гортань и теснил дотоле, что святыня "--- о, злодеяние! "--- была извержена. В бешенстве я схватил оную и повергнул в грязь. В это мгновение блеснула молния и восхитила с собой Божественную часть. Через два дня я узрел злобного человека, который, потрепав меня по плечу, сказал: «Мы друзья; навек будем неразлучны». "--- «А кто ты?» "--- спросил я. «Я тот, "--- отвечал он, "--- который во время страсти Иисуса Христа ударил Его в ланиту». "--- «Могу ли после этого, "--- облившись слезами и задыхаясь от рыдания, присовокупил пустынник, "--- могу ли сколько"=нибудь утешиться?»

\textit{Боготворящую кровь ужаснися, человече, зря: огнь бо есть, недостойныя попаляяй}\footnote{Из Последования ко Святому Причащению.}. Ужасно согрешил последователь Севира, повергнув святыню в грязь, но столь же ужасно согрешают и те, которые приемлют оную, имея сердце, исполненное всякой скверны.

\section{О том, сколь ужасно возвращаться к прежним грехам}

Когда преподобный Феодор Студит\footnote{Из жития прп. Феодора Студита, Исповедника (†~826), память которого празднуется 11 (24) ноября.} за поклонение святым иконам претерпевал от иконоборцев в области Смирнской все бедствия заточения, тогда воевода тамошних войск, племянник греческого царя Льва Армянина и столь же лютый иконоборец, жестоко заболел и был при последнем уже издыхании. Один из православных напомнил ему о преподобном Феодоре, что он имеет благодать от Бога исцелять всякие недуги. Опасение за свою жизнь принудило иконоборца обратиться к человеку Божию и просить у него помощи.

Незлобивый Феодор не отверг просьбы умирающего гонителя. «Скажите моим именем цареву племяннику, "--- отвечал он, "--- скажите этими словами: помысли, что будешь отвечать в день исхода твоего, сделав столько зла Православной Церкви, умертвив столько святых мужей! Они радуются ныне в Царствии Божием, а тебя кто избавит от муки вечной? Однако, хотя при кончине покайся в твоих злодеяниях». Болящий воевода принял Феодорово обличение с трепетом и, размыслив обо всем, что сделал, вторично послал к преподобному, умоляя о прощении и обещаясь возвратиться в недра Православной веры. Тогда святой Феодор прислал к нему икону Пресвятой Богородицы, повелев иметь ее при себе во всю жизнь и молиться с глубоким благоговением. Приняв святой образ, воевода вместе с ним получил облегчение болезни и начал выздоравливать. К несчастию, об этом узнал еретичествующий епископ Смирнский и, желая присвоить себе честь его исцеления, послал к нему елей, благословленный его молитвами. Но, едва воевода помазался им, опять прежний недуг возвратился к нему, и он вскоре умер.

\section{Патриотизм инока}

Михаил Травлий, воцарившийся после Льва Армянина, хотя сам был иконоборец, но позволял каждому веровать, как хочет, и возвратил всех святых отцов, бывших в заточении. Он запретил только иметь иконы в Царьграде. Преподобный Феодор Студит, не успев обратить императора на путь истины, удалился в пустыню.

Вскоре один из вельмож, по имени Фома, склонив на свою сторону часть войска и народа, восстал на Михаила. Возгорелась междоусобная война, и каждый день страшились, что Царьград обагрится кровью, ибо мятежники приближались к столице. Услышав об этом, преподобный Феодор Студит с братией немедленно пришел в Царьград и, проповедуя там верность престолу и любовь к соотечественникам, не допустил воспылать бунту в самом сердце государства, а между тем царские военачальники успели рассеять мятежников.

Так воздав \textit{кесарева кесареви}, святой старец возвратился в свое уединение "--- воздавать \textit{Божия Богови}.

\section{Не осуждай других, и тем более в том, где кроется для тебя что"=нибудь непонятное}

В обители преподобного Саввы Освященного был богобоязненный инок Иоанн, который служил братии сначала в поварне, потом в странноприимнице и, наконец, управлял экономией. Когда же скончался пресвитер обители, святой Савва восхотел произвести его в иерейский сан. Не сказав о том Иоанну, он пошел с ним в Иерусалим и, рассказав патриарху Илии все его добродетели, просил рукоположить Иоанна в пресвитера.

Святой Иоанн узнал об этом уже в церкви, когда наступило время рукоположения, и, к общему удивлению, обнаружил чрезвычайный ужас. «Владыка святый! "--- сказал он патриарху. "--- Я имею открыть тебе нечто тайное и прошу выслушать меня без свидетелей. Если и после этого почтешь возможным для меня рукоположение, то не отрекусь принять оное». Патриарх согласился на его просьбу и после краткого разговора в ризнице, выйдя оттуда с изумлением на лице, сказал преподобному Савве: «Иоанн открыл мне некоторые сокровенные дела, по которым он не может быть пресвитером; повелеваю тебе впредь не предлагать ему даже диаконского сана». Огорченный Савва, не зная, что подумать об Иоанне, отправился из Иерусалима и всю дорогу не только не говорил с ним, но не мог равнодушно и воззреть на него.

По прибытии в Лавру преподобный Савва немедленно удалился за тридцать стадий в некоторую пещеру. Там, повергшись на землю, он со слезами воскликнул к Богу: «Почто, Господи, презрел меня, раба Твоего, и утаил от меня житие Иоанна? Ах! Я жестоко обманулся, почитая его достойным пресвитерства. Я думал, что Иоанн есть сосуд святой и избранный для благодати Духа Святого, а этот сосуд оказался непотребным пред Твоим величеством. Об этом прискорбна душа моя до смерти». Посреди этих слезных молитв, растворяемых гневом и скорбью, вдруг явился ему Ангел Господень и сказал: «Не осуждай дела Иоанновы и не скорби о нем; ибо епископ не может быть рукоположен в пресвитера». Услышав это, преподобный Савва начал понимать значение тайного разговора Иоанна с патриархом и с нетерпеливой радостью пошел в келью Иоанна. Там, повергшись к его стопам, воскликнул: «Святитель Христов! Прости грешного раба твоего, который, как слабая тварь, имел предосудительные о тебе мысли. Но благодарение Богу! Он открыл мне свыше твое дарование». "--- «Это прискорбно мне, "--- отвечал Иоанн, "--- ибо я, кроме Бога, никому открыться не хотел. Но патриарх узнал от меня по необходимости, а тебя известил Сам Бог. Теперь признаюсь, что я, грешный Иоанн, есмь епископ, но оставил престол свой ради грехов моих и, будучи крепок телом, осудил себя служить братии, да помогают молитвы их слабой душе моей, ибо умножились без числа грехи мои»\footnote{Память прп. Иоанна (†~558), нарицаемого Молчаливым, потому что он по открытии своей тайны заключился в безмолвную келью, празднуется 3~(16) декабря.}.

\section{Мятежный дух братии, укрощенный любовью и смирением}

Часто посреди пшеницы рождаются плевелы, а в винограде "--- терны. У Елисея был ученик Гиезий, и в лике апостольском "--- предатель Иуда. Равным образом и в обители святого Саввы нашлись злонравные братия, которые всемерно оскорбляли святого отца и вскоре злоумышление свое простерли до того, что скрытно ушли в Иерусалим к патриарху Саллюстию с просьбой, чтобы тот дал им игумена. Зная, сколько знаменито везде имя Саввы, мятежники сначала старались скрыть оное и сказали только то, что незадолго пред этим они поселились при некотором пустынном источнике. Однако же наконец принуждены были открыть истину и, когда патриарх спросил: «Где же Савва?», отвечая не на вопрос, начали клеветать, что Савва невежда и грубиян, не знает, как управлять братией, и, сверх того, не хочет сам и не позволяет другим принять священство. Выслушав это, патриарх спросил: «Вы ли приняли Савву или Савва вас принял?» И когда они сказали, что Савва принял их, Саллюстий отвечал: «Если этот старец в пустынном месте мог вас собрать, то не думаю, чтобы он не мог управлять вами». Потом, оставив их без всякого решения, послал за святым Саввой и, не поминая о жалобах, немедленно, хотя и против воли, посвятил его в пресвитера и, довольно поучив беспокойных братий, приказал им возвратиться с преподобным отцом в обитель.

Но злоба не погасла и только таилась под пеплом: мятежники опять восстали на человека Божия. К большому огорчению, он узнал, что некоторые из братий, родом армяне, заражены ересью и не хотят внимать его наставлениям. Тогда"=то Савва, уступая гневу и ожесточению, удалился в безмолвную пустыню, а злобствующие иноки, обрадовавшись тому, распустили слух и даже донесли патриарху Илии, будто преподобный Савва съеден зверями, и просили себе другого настоятеля. Но патриарх хладнокровно сказал им только это: «Я знаю, что Бог праведен, не презрит добрых дел отца вашего и не попустит ему погибнуть от зверей; ждите убо его, пока возвратится». Узнав об этом, святой старец отчаялся в их раскаянии и сам начал просить патриарха не считать его более настоятелем лавры. Но Илия не хотел и слушать того. «Не могу стерпеть, "--- сказал он гласом святительским, "--- чтобы твоими трудами обладали другие». И послал его опять в обитель с грамотой, в которой, укоряя мятежников, повелевал принять его беспрекословно, а в случае непослушания грозил гневом Божиим.

Казалось, этот глас первосвятителя должен был, подобно грому, поразить мятежные души; \textit{но обуяша лютии взыскателе}! Когда патриаршая грамота была прочтена посреди церкви, мятежники подняли крик: укоряли, досаждали, злословили невинного старца. И тогда же, собрав все, что имели собственного, пошли в другую обитель. Но худая слава носилась пред ними, и мятежники отовсюду изгоняемы были с бесчестием. Наконец, удалились они к так называемому «Фекутийскому потоку» и там начали жить, отверженные Богом и людьми.

Преподобный Савва радовался, что дух раздора удалился из его лавры, и болезновал о погибели мятежных братий. Вскоре он услышал, что они живут в крайнем оскудении, тесноте и расстройстве, как овцы, не имеющие пастыря. Отеческое сердце забыло обиды и чувствовало одни их несчастия. Он взял с собою довольно пищи и питья и пошел к ним, желая утолить их гнев и оказать помощь. Встреченный ругательством, святой старец благословлял их и, уговаривая как чад и братий, предложил угощение. После этого он отправил одного из своих учеников к патриарху Иерусалимскому, умоляя его принять их под свое покровительство, и брал на себя их устроение. Получив согласие святителя и семьдесят златниц в пособие, он пробыл у них пять месяцев. Создал церковь, устроил кельи, дал из своей лавры настоятеля и, против их воли заставив себя возлюбить, возвратился спокойным.

\section{Благодарность варваров}

Преподобный Савва Освященный, обитая в пустыне, увидел четырех сарацинов, изнемогающих от голода, и хотя знал, что эти варвары всегда оскорбляли иноков, но, сжалившись над их состоянием, подошел к ним, велел сесть и, отдав последнюю пищу, угостил, как братий. Сарацины, укрепившись, пошли с благодарностью и, к удивлению Саввы, с великим вниманием рассматривали место, пересчитывали изредка стоящие кусты и клали на земле некоторые знаки. Но этот их поступок вскоре объяснился. Сарацины через несколько дней возвратились и принесли с избытком хлеба, сыра и разных плодов, вторично благодаря Савву за его благодеяние. Возрадовался преподобный, что из врагов мог сделать друзей, но, поучая их, чтобы и впредь не обижали беззащитных пустынников, вдруг задумался и, прослезившись, сказал сам себе: «Горе, душа моя! Горе тебе! Эти люди за малое наше благодеяние, однажды им оказанное, видишь, сколь благодарны, а мы каждый час приемлем неизреченные дары от Бога, и что делаем, неблагодарные? Живем в лености и бесчувствии, не сохраняя святых Его заповедей!» Таким образом, святой старец не оставлял ни одного случая, не пропускал ни одной встречи без того, чтобы не сделать добра, и из всякого происшествия, как в естественном, так и в нравственном мире, извлекал для себя какое"=нибудь спасительное правило.

\section{Чтобы быть хорошим начальником, должно быть прежде хорошим подчиненным}

Один черноризец, по имени Иаков, человек беспокойный и гордый, уговорившись с единонравными себе иноками, оставил обитель преподобного Саввы и, желая быть ему равным, начал в другом месте строить свой монастырь. Святой старец, возвратившись из пустыни, где по обычаю проводил в безмолвии великопостное время, пошел к нему и отечески советовал оставить предприятие и возвратиться в обитель. Но так как Иаков не хотел его слушать, то святой Савва с горестью сказал ему: «Начинания, порожденные дерзостью и высокомерием, всегда были пагубны: блюдись и ты, да не будешь наказан».

Святой Савва ушел в свою лавру, а с Иаковом и его братией случилось подобное тому, что некогда постигло строителей Вавилонской башни. Язык их остался тот же, но смешались мысли: один хотел того, другой иного; сам Иаков беспрестанно переменял свои намерения, и строительство монастыря совершенно остановилось.

Что же касается общежительного устава, в таком неустройстве его и ожидать было невозможно. Но этого мало: Иаков при каждом начинании начал робеть и ужасаться, сам не зная чего. А за этим постигла его столь лютая болезнь, что он более полугода ничего почти не говорил, и, уже отчаявшись в жизни своей, велел нести себя к преподобному Савве, чтобы при кончине своей испросить у него прощение. Святой старец сделал ему отеческое наставление и именем Господним исцелил от болезни.

Иаков, познав всю важность преступления, уже не возвратился на свое новое место, но остался в гостинице служить странникам. Однако, будучи нерадив, готовил пищи иногда много, иногда мало, и когда однажды осталось от трапезы несколько вареных стручьев, то Иаков бросил их за окно в поток. Преподобный Савва, давно замечая его небрежение, как скоро узнал о том, собрал стручья, перемыл их и, высушив в келье своей на солнце, через некоторое время сварил их. И, позвав к себе на обед Иакова, извинился, что, не умея готовить снеди, не может так угостить его, как бы хотел. Когда же Иаков похвалил пищу и признался, что давно не едал с таким вкусом, тогда Савва сказал: «Поверь же мне, чадо, что это те стручья, которые ты, незадолго пред этим, высыпал в поток; и знай, что тот, кто не умеет устроить в меру горшка какой"=нибудь травы, не может устроить монастыря, а кольми паче не может управлять братией! \textit{Аще же кто своего дому не умеет правити}, "--- говорит Апостол, "--- \textit{како о Церкви Божией прилежати возможет?} (1~Тим. 3, 5)». Этим простым замечанием мудрый старец заставил Иакова устыдиться и в небрежении к своей должности, и в прежнем своем любоначалии.

\section{Когда нужно служить Богу втайне и когда "--- открыто}

Святитель Христов Николай до принятия им архиерейского сана всемерно старался, убегая славы человеческой, скрывать свои добрые дела. И если не погреб себя в пустынном безмолвии для беседы с Единым Богом, то единственно из повиновения гласу Небесному, который повелел ему обратиться к людям и служить их счастью и спасению. Однако, обитая в многолюдном городе, он жил никому не знаемый, яко един от пришельцев, и показывался только в дому Господнем, имея одно пристанище "--- Бога. Все дела его частной жизни свидетельствуют, что святой Николай желал, чтобы не ведала шуйца, что творит десница его. Но когда он был возведен на престол Мирликийской Церкви, тогда сказал сам себе: «О, Николай! Этот сан и это место требуют от тебя иных обычаев; ты отселе должен жить не для себя, но для других». С того времени, ревнуя научить порученное ему Христово стадо благочестию и добродетели, святитель Господень уже не скрывал своего благочестия и добродетелей. Прежде один только Бог ведал его житие; теперь же просветился \textit{свет его пред человеки, яко да видят добрая его дела} (Мф. 5, 16). Он служил Богу открыто, чтобы и другие так же Ему служили; благодетельствовал ближним, чтобы и другие так же им благодетельствовали. Совершая много чудес для спасения невинно гонимых, для избавления погибающих посреди неприятелей, в пламени и в волнах моря, он объявлял свое имя, и это славное имя заставляло царей царствовать благодетельно, пастырей "--- править Церковью богоугодно, народ "--- жить богобоязненно. Святой Николай стоял пред лицом вселенной, как зеркало добрых дел, и был \textit{образ верным словом, житием, любовию, духом, верою и чистотою} (1~Тим. 4, 12).

\section{Наказанная жестокость}

Когда жестокосердый Диоскор, отец святой великомученицы Варвары, бежал за ней с обнаженным мечом, тогда исповедуемый ею Бог, защитник невинности, повелел раздвинуться каменной скале и чрез расселину свободно провел ее на верх горы. Рассвирепевший отец, не видя пред собой Варвары, весьма удивился и, обходя гору, всюду искал ее. Наконец, встретившись с двумя пастухами, пасущими овец своих, спросил у них, не видели ли они бегущей девицы? Один из них, приметив, что Диоскор весьма разгневан, сказал: «Не знаю». Но другой молча показал перстом на место, где скрылась святая Варвара. Диоскор нашел дочь свою в пещере, трепещущую, как голубицу, и, попирая ногами, повлек ее за волосы по жестким камням. Но Бог, защищавший невинность, тогда же наказал злодейскую услужливость пастуха: умертвил его смертью неслыханной "--- окаменением.

\section{Желание слепца и употребление чувства зрения}

Некогда к преподобному Патапию\footnote{Память прп. Патапия (†~VII~в.) празднуется 8~(21) декабря.} пришел слепой от рождения юноша и умолял святого старца, чтобы испросил ему у Бога прозрение очей. «Но для чего тебе более нужно прозрение?» "--- приблизившись к Распятию Иисуса, спросил у молодого человека Патапий. «Чтобы видеть тварь, "--- отвечал юноша, "--- и от твари прийти в совершеннейшее познание Творца и Его прославить». Старец, умиленный разумом и благочестием слепца, возопил к Богу: «Иисусе Христе, дарующий слепым свет и мертвым жизнь! Отверзи очи жаждущему просветиться светом Твоего богопознания». И в то же мгновение слепой юноша прозрел. Тогда святой Патапий сказал ему: «Иди с миром в дом твой, но опасайся, чтобы с отверзением очей телесных не помрачились у тебя очи душевные. Ни на какое чувство не могут так действовать соблазны мира, как на чувство зрения. Горе, если восслепотствует в тебе внутренний человек. Тогда и при этих очах ты не узришь Бога там, где все слепцы узрят Его уже не верой, но лицом к лицу».

\section{Кончина святого Амвросия Медиоланского}

Когда святой Амвросий Медиоланский\footnote{Память свт. Амвросия, епископа Медиоланского (†~397), празднуется 7~(20) декабря.} приблизился к блаженной кончине, то Стилихон, первый воевода и опекун римского императора Гонория, услышав о том, с болезненным вздохом произнес: «Погибнет Италия, если умрет этот святитель!» Потом отправил к нему знатных людей, которых любил Амвросий, умоляя чрез них, чтобы угодник Божий испросил себе у Господа еще несколько лет жития для пользы государства. Человек Божий возвел очи свои к небу и умирающим голосом сказал: «Благодарю Тебя, Господи Боже мой, что я, хотя и грешный человек, жил так, что не стыжусь долее жить и не боюсь смерти».

\section{Сам Бог не велит без нужды подвергаться опасностям}

Преподобный Даниил\footnote{Память прп. Даниила (†~493), также Столпника, празднуется 11~(24) декабря.}, обитая в «ограде» святого Симеона Столпника, вознамерился сходить в Иерусалим для поклонения богошественным местам и оттуда уйти на безмолвие в пустыню. Симеон, любя его, советовал остаться с ним, но не мог преодолеть желания Даниилова "--- и Даниил с душевным рвением отправился в путь.

Дорогою услышал он, что в Палестине свирепствует междоусобие, ибо тогда самаряне взбунтовались и, истребляя христиан, не щадили ни возраста, ни состояния. Однако нетерпение Даниила было столь велико, что он, презрев все опасности и самую смерть, продолжал свой путь. Но вскоре он встретил престарелого инока, который, облобызав его, спросил по"=сирийски: «Куда идешь, любезный брат?» "--- «Если Бог благословит мой путь, "--- отвечал Даниил, "--- то хочу быть в Иерусалиме». "--- «Справедливо говоришь: если Бог благословит, "--- возразил старец, но теперь путь твой не от Божия благословения. Ужели не слышал ты, что в Палестине мятеж и кровопролитие?» "--- «Я слышал об этом, "--- сказал Даниил, "--- но полагаюсь на Бога, что этот Помощник и Покровитель не попустит прийти на меня какому"=либо злу, а если и случится это, не боюсь: \textit{ибо аще живем, аще умираем, Господни есмы} (Рим. 14, 8)». На это старец с прещением сказал ему: «\textit{Не даждь во смятение ноги твоея}; тогда \textit{не воздремлет и Храняй тя}» (Пс. 120, 3). Однако Даниил не переставал упорствовать и отозвался с решимостью, что он на этом пути готов умереть за Христа. Старец, с негодованием отвратив лицо свое, еще строже произнес: «Не повелел Бог безвременно подвергаться смерти, не повелел искать случаев быть убиенными. Он вещает напротив того: \textit{Егда же гонят вы во граде сем, бегайте в другий} (Мф. 10, 23)». После этого Даниил задумался и, начав склоняться к совету старца, сказал: «Если так благоугодно тебе, я возвращусь». Тогда старец, переменив голос, сказал ласково: «Сын мой! Я не советую тебе оставить твое намерение навсегда, в таком случае я был бы безумен, но советую не идти только в это злое время. Итак, возвратись в Царьград, который по своим святыням есть второй Иерусалим: там можешь ты насытить свои очи и сердце видением священных редкостей. Если же хочешь безмолвствовать, то Бог укажет на пользу души твоей место или в вышней Фракии, или в самом устье Понта. Впрочем, не должно помышлять, будто только в Иерусалиме обрящешь Бога, а в Царьграде нет. О, возлюбленный! Бог местом не объемлется».

В то время как этот разговор продолжался, закатилось солнце, и они направили путь свой к близстоящему монастырю. Но вдруг старца не стало. Даниил, рассуждая об этом видении, уверился, что это был или Ангел Божий, или сам Симеон, "--- и на другой же день пошел в Царьград.

\section{Наказанный сребролюбец}

В архипастырство святителя Спиридона Тримифунтского\footnote{Память свт. Спиридона, епископа Тримифунтского, чудотворца (†~ок. 348), празднуется 12~(25) декабря.} некий богач, в голодное время привезший на кораблях из других стран множество хлеба, не хотел продавать за ту цену, какую застал в Тримифунте, но, насыпав житницы, ждал, пока цена еще повысится. Сколько бедные ни умоляли ненасытного хлебопродавца, он оставался глух, как аспид. Несчастные прибегли к святителю Спиридону и жаловались на его жестокость и свое бедствие. Человек Божий, ободряя унывающих, сказал им: «Не плачьте, но идите и утешьте ваши семейства упованием на Бога. Тако бо глаголет Господь: яко заутра наполнятся храмины ваши жита; богатого же узрите молящего вас и дающего вам пищу без цены». Тихий луч надежды озарил сердца алчущих, однако не осушил слез их.

Но в первый сумрак ночи по повелению Божию вдруг пролился великий дождь, сопровождаемый столь сильным ветром, что опроверглись житницы немилосердного богача, и стремящаяся потоками вода разнесла весь хлеб. Хлебопродавец, бросаясь с домашними своими туда и сюда, не знал, что делать, и просил помощи у собравшегося народа. А неимущие между тем, видя разнесенное по пути жито, собирали его и, унося домой, опять возвращались на то же дело. Вспомнив пророчество святого Спиридона, они на этот раз не считали за грех брать, что дарует Сам Бог, мужья, жены и дети старались наполнить свои одежды и все, что имели. Сам лихоимец, почувствовав отяготевшую над ним руку общего Отца, \textit{Иже есть на небесех}, ободрял бедных и просил брать, кто сколько может и хочет.

А святой Спиридон, услышав о его усердии, благословил Бога и за то, что Он дал пищу алчущим, и за то, что внезапным несчастием умягчил жестокое сердце сребролюбца.

\section{Поступок святого Спиридона с хищниками}

Однажды в архиерейский дом святого Спиридона закрались воры и, захватив несколько овец, хотели уйти. Но Бог, любя угодника Своего и соблюдая убогое его имение, связал их невидимой силой, так что они не могли двинуться с места и остались так до утра. Святитель Спиридон, выйдя на рассвете из кельи и увидев их в этом жалком состоянии, дал наставление, чтобы они не похищали чужого, но питались трудами своих рук, потом молитвой разрешил невидимые их узы и, отпуская, подарил им одного овна, примолвив: «Возьмите его, да не будет всуе ваш труд и всенощное бдение».

\section{Хищничество не обогащает, но еще более разоряет}

Один купец для оборота в торговле часто занимал у святого Спиридона деньги; когда же, возвращаясь от купли, приносил ему долг свой, святой старец обыкновенно приказывал ему, чтобы сам положил его в ковчег, из которого взял. Не любя стяжания, он не любопытствовал, все ли деньги должник отдает ему.

Многократно купец, с позволения святого, брал и отдавал золото, и Господь благословлял его торговлю. Но однажды, соблазнившись, купец не положил принесенных денег в ковчег и, обманув святого Спиридона, удержал у себя. Что же вышло? Купец вскоре обнищал, ибо утаенное золото не только не принесло ему прибытка, но, как огонь, поело и его имение. В этой крайности купец опять пришел к святителю Спиридону и просил у него взаймы золота. Святой по обыкновению послал его в свою спальню, чтобы он там взял сам, но так как купец ничего не положил туда, то там ничего и не было. Возвратившись с пустыми руками, объявил он святому Спиридону, что в ковчеге ничего нет. «Кроме твоей руки, доселе ничьей в ковчеге не было, "--- спокойно отвечал ему святой старец, "--- если бы ты положил тогда деньги, то теперь опять взял бы их; роптать не на кого, сам виноват». Пораженный стыдом, купец пал к ногам Спиридона. Святой немедленно простил его, но отпустил в дом уже не с деньгами, ибо не имел их, а только с убедительным наставлением впредь не осквернять совести своей обманом.

\section{Горький плод роптания}

Преподобный Ираклий имел у себя ученика, более других украшенного добродетелью послушания. В одно время этот трудник повергся к ногам старца и просил, да облечет его в образ иноческий. Ираклий согласился и, когда для инока соорудили особливую келью, сказал ученику своему, как учитель: «Чадо, исполняй мою заповедь: когда взалчешь "--- вкушай, когда возжаждешь "--- пей, когда воздремлешь "--- усни.

Только не выходи из кельи даже до субботы, а тогда посети меня». Юный инок два дня с охотою исполнял приказание старца, но на третий день почувствовал скуку и начал жаловаться на строгое запрещение учителя. Он пропел, хотя с некоторым смущением, вечернее правило и по закату солнца пошел спать. Что же увидел? На его одре возлежит некто черный и злобный и скрежещет на него зубами. Ужаснувшийся инок без памяти ринулся из кельи, прибежал к старцу и застучал в дверь. «Сжалься надо мной, авва! "--- вопиял он. "--- Отопри, авва!» Но старец, провидя духом его роптание, которое было, без сомнения, первым шагом к преступлению заповеди, не отверз кельи до рассвета. Когда наконец он впустил его, трепещущий ученик воскликнул: «Отче! Отходя спать, я увидел на ложе моем чудовище, до смерти страшусь его». "--- «Это сделалось оттого, "--- сказал Ираклий, "--- что ты, чадо мое, возроптал и тем поползнулся к преступлению заповеди. Этот черный, злобный дух тебя ужасает, но от него есть защита "--- крест и молитва. Горе, если овладеют сердцем твоим дела законопреступные! Тогда ты сам внутри себя, в твоей совести, обретешь еще более ужасное чудовище, скрежещущее на тебя: оно поглотит тебя, пожрет тебя, и ты погибнешь». Этим и тому подобным образом поучив юного пустынника, преподобный Ираклий чрез некоторое время постриг его, и он вскоре из ученика сделался учителем.

\section{В каждом состоянии возможно спастись}

Один городской житель сказал преподобному Нифонту\footnote{Память прп. Нифонта, епископа Кипрского (†~IV~в.), празднуется 23~декабря (5~января).}: «Обращаясь в мире, невозможно спастись; если человек сам по себе и благонравен, то другие приводят его в соблазн. Сверх того, сколько пересудов: одним кажется, что боголюбцы слишком умствуют, другим не нравится отменный от прочих образ жизни; те осуждают за то, что они не наблюдают благоприличий света, другие называют их нелюдимами. А это для непостоянного сердца весьма опасно. Кто хочет быть совершен, непременно должен жить в монастыре или в пустыне». Выслушав это, преподобный Нифонт отвечал: «Чадо! Место не спасет человека и не погубит. Одни дела спасают и погубляют. Нет помощи ни от святого сана, ни от святого места тому, кто не исполняет заповедей Господних. Саул жил посреди великолепия царского "--- и погиб; Давид жил посреди того же великолепия "--- и принял венец; Лот жил посреди беззаконных содомлян "--- и спасся; Иуда находился в лике апостолов "--- и наследовал геенну. Кто говорит, что невозможно спастись в мире с супругой и детьми, тот льстит своему безумию и порокам. Авраам имел супругу и детей, триста восемьдесят рабов, столько же рабынь и множество золота и серебра, однако это не помешало ему приобрести имя друга Божия. Сколько спаслось служителей Церкви и пустыннолюбцев! Сколько вельмож и воинов! Сколько ремесленников и земледельцев! Сколько "--- посреди шумных столиц и посреди безмолвных пустынь! Прочитай жития святых "--- и узришь имена угодников Божиих. С другой стороны, в этих же санах и сословиях, в этих же местах и в то же время бесчисленное множество людей погибло. От царей до рабов есть чада Небесного Царствия, и от царей до рабов есть чада погибели. Сын Церкви Христовой! Не соблазняй ума твоего: на всяком месте обретешь от Бога спасение, если исполняешь волю Его. Господь приемлет душу праведную в объятия Свои "--- одинаково с престола и от сохи, из алтаря и с поля брани. Итак, живет ли кто в мире, да не отчаивается. Согрешит ли "--- покаянием может опять приблизиться к Богу. Каждый исполняй все добродетели звания, которое Бог возложил на него; сверх того, будь благочестив и человеколюбив: тогда каждый спасется. Напротив того, если кто и удалится в безмолвную пустыню, но не оставит далече за собой злых дел или злого произволения, тот погибнет неминуемо».

\section{Твердость исповедания веры}

Когда святой преподобномученик Феодор с братом своим преподобным Феофаном\footnote{Прп. Феодор Начертанный, исповедник, скончался мученически в заточении в царствование Феофила иконоборца в 840~г. Память его празднуется 27~декабря (9~января). А прп. Феофан по смерти Феофила был возвращен и скончался митрополитом Никейским ок. 850~г. Память его празднуется 11~(24) октября.} после разных страданий за поклонение святым иконам были осуждены на новую и неслыханную в то время казнь, чтобы на их лицах начертаны были раскаленным железом некоторые ругательные слова, тогда чиновник, которому поручено было это исполнить, сказал святым страдальцам: «Только однажды причаститесь с нами\footnote{То есть с еретиками, в еретической церкви, и дарами, лжеосвященными еретичествующим священником.}, и я отпущу вас идти, куда пожелаете». Услышав это, блаженный Феодор рассмеялся и отвечал: «Ты говоришь то же, как если бы кто сказал мне: я ничего от тебя не требую, кроме того, что однажды отсеку твою голову, а потом иди куда хочешь. Великое благодеяние обещаешь нам! Будь уверен, что отвратить нас от правоверия столь же трудно, сколь невозможно переставить между собой небо и землю». После этого преподобные отцы спокойно пошли на мучения.

\section{Торжество христианского смирения}

Святая Мелания Старшая\footnote{Эта праведная Мелания называется Старшей для различия от Младшей Мелании Римляныни (†~439), своей внучки, память которой празднуется 31~декабря (13~января).}, происходившая из знаменитейшего в Риме дома, дочь и вдова сенатора, оставив свое отечество, странствовала всюду единственно для украшения святых храмов и утешения страждущего человечества. Посещала святых отцов, обитавших в Нитрийской горе, благодетельствовала церквам и обителям, наполняла благотворениями темницы и, наконец, остановившись в Иерусалиме, в течение тридцати семи лет была совершеннейшим образом страннолюбия Авраамова: питала, успокаивала, утешала и напутствовала всех, приходивших туда с востока и запада, с севера и юга.

Подвиг поистине достойный, чтобы сопровождать на него с большим почетом, нежели с каким сопровождают военачальников, идущих на поражение супостата! И святой Мелании была воздана честь эта. Павлин, епископ Ноланский, которого она в своем путешествии посетила первого, описывает это зрелище следующим образом: «Все родственники праведной Мелании, то есть все особы, бывшие тогда знаменитейшими в Римской империи, вышли к ней навстречу и сопровождали ее с великолепием, приличным столь великой особе. Дорога Аппиева покрыта была золотыми и блестящими каретами, богато убранными лошадьми и великим числом колесниц всякого рода. Посреди такой пышной обстановки ехала одна госпожа, достопочтенная по ее летам, а еще более по ее важному и кроткому виду, сидящая в простой коляске и одетая в простое шерстяное платье. Однако же взоры всех обращены были на смиренную Меланию. Никто не смотрел ни на золото, ни на шелк, ни на пурпуровые одежды, которые блистали по всем сторонам: простая одежда помрачала весь этот тщеславный блеск. Можно было, на этот раз, видеть в детях ее и родственниках то же презрение к блестящей суетности, которое показывала их мать, оставив и поправ оное, дабы принести в жертву Богу \textit{сердце чисто и дух сокрушен}. Вельможи и госпожи, которые составляли великолепную свиту, не только не стыдились этого низкого и в очах света презренного состояния, в котором видели святую вдовицу, но еще вменяли себе в честь подходить к ней и прикасаться к ее одеянию, думая, что чрез смиренное и почтительное уничижение они очищали гордость их богатого и блестящего состояния. Таким образом, пышность римского величия воздала в этом случае честь убожеству Евангельскому».

\section{Святость невольно возбуждает к себе уважение}

Мятежник Максим, убийца императора Грациана, совершив внезапное нападение на Италию, где царствовал Валентиниан Второй, всюду распространял опустошение и смерть. Разлившись вдруг, как быстрая река, его войско разорило почти до основания Пиаченцу, Модену, Реджию и Болонью и истребило мечом и пламенем все, встретившееся ему на пути. Не было свирепства, хищения, бесстыдства, насилия и святотатства, которого не учинило бы войско Максима. Половина жителей погибла, другая принуждена была воздыхать в жестоком плену. Медиолан ожидал той же участи и более прочих ужасался опустошения; ибо известно было, что тиран ненавидел святого Амвросия, архиепископа Медиоланского, приписывая ему свою остановку в завоевании Италии и почитая хитрецом, отвлекшим его при первом посольстве от завоеваний. Но Максим, сверх всякого чаяния, оставил Медиолан в покое посреди окружавших его пожаров и не захотел препятствовать святому архиепископу проповедовать покаяние и верность законному государю. В столь высоком почтении бывает святость у самих тиранов!

\section{Жертва вере, принесенная вельможей}

Юстина, мать западного римского императора Валентиниана Второго, жестоко огорченная поступками Амвросия, епископа Медиоланского, который привел Ариеву ересь в крайнее бессилие, решилась, наконец, погубить его.

Чтобы исполнить это, она вознамерилась от имени сына своего, Валентиниана, издать указ, которым позволяла арианам отправлять богослужение открыто, и всех тех, кто осмелился бы делать им какое"=либо препятствие, объявляла мятежниками, нарушителями церковного мира, оскорбителями величества и преступниками, достойными казни. Для этого она призвала Беневола, первого государственного секретаря и приказала ему изготовить этот указ. Но этот муж, почитая имя православного выше всех достоинств, отрекся исполнить волю ее. Императрица принуждала его, угрожала своим гневом, обещала возвести на высшую степень, но благочестивый вельможа отвечал великодушно: «Я такой ценой не хочу покупать ваших достоинств; отберите у меня и то, что я имею, но оставьте мне мою совесть и мою веру». С этими словами он снял с себя пояс, который был знаком его достоинства, и положил к ногам Юстины. Вскоре потом удалился он в Брессу, где провел остальное время своей жизни в строгом исполнении христианских добродетелей.

\begin{center}\small\textsc{Конец шестой части.}\end{center}
