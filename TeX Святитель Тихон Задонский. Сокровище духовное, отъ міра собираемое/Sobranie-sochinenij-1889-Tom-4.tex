\begin{quotation}
«Сокровище духовное» писано святителемъ Тихономъ въ Задонскомъ монастырѣ; святитель трудился надъ этимъ сочиненіемъ въ теченіе 3"~хъ почти лѣтъ, именно съ 1777~г. по 19~ноября 1779~г. Напечатано было это сочиненіе въ первый разъ въ 1784~г. въ Спб., и потомъ издавалось много разъ, какъ въ полномъ видѣ, такъ и по частямъ. Во всѣхъ этихъ изданіяхъ «Сокровище духовное» было подраздѣляемо на 4~части, съ отдѣльнымъ притомъ, особымъ для каждой части, счетомъ статей. Въ подлинникѣ такого дѣленія не видимъ: вся книга, по подлинной рукописи, состоитъ изъ 157~отдѣльныхъ статей, связанныхъ непрерывно идущею чрезъ все сочиненіе нумераціей. Такое же распредѣленіе разсматриваемаго творенія принято и въ настоящемъ изданіи.

«Сокровище духовное» напечатано въ настоящемъ изданіи по подлиннымъ рукописямъ, хранящимся въ Усм. Соф. м"=рѣ (№, по тип. счету,~5--8). Подлинникъ писанъ рукою келейника св. Тихона, Ивана Еѳимова, и, какъ и другіе подлинники, писанные не собственною рукою святителя и предназначенные имъ для представленія въ Св. Синодъ, тщательно пересмотрѣнъ былъ самимъ авторомъ: на поляхъ и въ текстѣ (особенно съ № 57~по 107)~встрѣчаются поправки и вписки, мѣстами на полстраницы, собственной руки святителя; даже исправлены очевидныя, буквенныя ошибки писца; въ концѣ каждаго тома его же, святителя, собственноручная подпись: «Тѵхонъ N\textit{(едостойный)} Епіскопъ», и помѣта: подъ т. 2"~мъ «1778~г. декабря 12~дня»; подъ 3"~мъ: «марта 29~дня 1779~г.» и подъ 4"~мъ: «1779~года ноября 19~дня». Подлинная рукопись раздѣлена на 4~переплета, изъ коихъ въ 1"~мъ, по современной помѣтѣ,~251~л., во 2"~мъ 195~л., въ 3"~мъ 212 и въ 4"~мъ 161~л., всѣ въ 4"=ку.
\end{quotation}

\chapter*{Сокровище духовное,\\отъ міра собираемое.}

\begin{quotation}
Какъ купецъ отъ различныхъ странъ собираетъ различныя товары, и въ домъ свой привозитъ, и сокрываетъ ихъ; такъ христіанину можно отъ міра сего собирать душеполезныя мысли, и слагать ихъ въ клѣти сердца своего, и тѣми душу свою созидать.
\end{quotation}

\section{1. Міръ.}

Ничто само отъ себе не бываетъ. Всякій градъ не отъ себе, но отъ инаго; всякій домъ не отъ себе, но отъ инаго созидается; всякое письмо не отъ себе, но отъ инаго пишется; всякая книга не отъ себе, но отъ инаго сочиняется; словомъ, всякая вещь не отъ себе, но отъ инаго дѣлается. Тако міръ сей не отъ себе, но отъ Создателя своего сотворенъ. \textit{Той рече, и быша; Той повелѣ, и создашася}\footnote{Пс.~148,~5.}. Который Создатель есть Богъ нашъ въ единомъ естествѣ, но въ тріехъ лицахъ вѣруемый, исповѣдуемый и покланяемый "--- Отецъ и Сынъ и Святый Духъ. Міръ не такъ отъ Создателя созданъ есть, какъ человѣки созидаютъ. Человѣки созидаютъ одно отъ другаго, то есть, изъ какой нибудь матеріи дѣло дѣлаютъ, и то съ трудомъ; но Богъ міръ сей, то есть, небо и землю съ исполненіемъ ихъ, изъ ничего и безъ всякаго труда, единымъ хотѣніемъ и словомъ создалъ. \textit{Рече, и быша: повелѣ, и создашася}, "--- которое дѣло всемогущей Божіей силѣ приписуется. Разумъ нашъ, который говоритъ, что ничего изъ ничего не бываетъ, не можетъ понять того, како такъ великій міра сего составъ изъ ничего произведенъ; но вѣра недостатокъ разума дополняетъ, и убѣждаетъ его признавать тое, что Богу, яко всемогущему, все возможно. \textit{Яко не изнеможетъ у Бога всякъ глаголъ}\footnote{Лук.~1,~37.}. \textit{Вѣрою разумѣваемъ совершитися вѣкомъ глаголомъ Божіимъ, во еже отъ неявляемыхъ видимымъ быти}\footnote{Евр.~11,~3.}. Она слышитъ отъ Духа Святаго свидѣтельство: \textit{въ началѣ Богъ сотвори небо и землю}\footnote{Быт.~1,~1.}, и тако безъ сумнѣнія держитъ. Якоже бо сочинитель книги изъ разума своего износитъ слова, и написуетъ ихъ на хартіи, и тако сочиняетъ книгу, и какъ бы изъ ничего нѣчто дѣлаетъ: тако премудрый и всемогущій Создатель, что въ Божественномъ своемъ разумѣ имѣлъ, и что восхотѣлъ, все сотворилъ, и какъ бы книгу изъ двухъ листовъ, то"=есть, неба и земли состоящую, сочинилъ. Въ которой книгѣ видимъ Божіе всемогущество, премудрость и благость. Всемогущество, яко все изъ ничего хотѣніемъ и словомъ сотворилъ. Премудрость, яко все премудро сотворилъ: \textit{вся премудростію сотворилъ еси}\footnote{Пс.~103,~24.}. Благость, яко все не ради Себе, но ради насъ сотворилъ. Добро бо есть сообщительно самого себе. Ибо Богъ Самъ ради Себе ничего не требуетъ. Какъ прежде вѣкъ, такъ и нынѣ, и во вѣки вѣковъ, во всесовершенномъ блаженствѣ пребываетъ.

\section{2. Солнце.}

Прежде восхожденія солнца тьма и нощь пребываетъ, но какъ солнце взойдетъ, тьма отступаетъ, и свѣтъ возсіяваетъ. Тако до пришествія Хрістова, иже есть Солнце праведное, тьма всю вселенную покрывала и нощь глубокая была; но какъ сіе свѣтлѣйшее возсіяло Солнце, и свои теплѣйшіе лучи на всю вселенную испустило, благопріятнѣйшій и сладчайшій душамъ нашимъ возсіялъ день. Тогда исполнилося пророческое слово: \textit{людіе ходящіи во тьмѣ, видѣша свѣтъ велій: и сидящимъ въ странѣ и сѣни смертнѣй, свѣтъ возсія имъ}\footnote{Ис.~9,~2; Мѳ.~4,~16.}. \textit{Нощь убо прейде, а день приближися. Отложимъ убо дѣла темная, и облечемся во оружіе свѣта. Яко во дни, благообразно да ходимъ, не козлогласованіи и піянствы, не любодѣяніи и студодѣяніи, не рвеніемъ и завистію: но облецытеся Господемъ нашимъ Іисусъ Хрістомъ, и плоти угодія не творите въ похоти}, глаголетъ Апостолъ\footnote{Римл.~13,~12--14.}. Сотворимъ убо и мы по увѣщанію апостольскому, да будемъ сынове свѣта и дне.

Предъ солнцемъ сіяющимъ на небеси, вси люди ходятъ, и солнце на всѣхъ и на всякаго смотритъ: тако предъ Богомъ вездѣсущимъ и вся назирающимъ люди ходятъ, и что кто ни дѣлаетъ, помышляетъ, зачинаетъ, намѣреваетъ, очи Господни видятъ. \textit{Съ небесе призрѣ Господь, видѣ вся сыны человѣческія: отъ готоваго жилища Своего призрѣ на вся живущія на земли: создавый на единѣ сердца ихъ, разумѣваяй на вся дѣла ихъ}\footnote{Пс.~32,~13--15.}. И паки: \textit{очи Твои}, Господи, \textit{отверсты на вся пути сыновъ человѣческихъ, дати комуждо по пути его, и по плоду начинаній его}\footnote{Іер.~32,~19.}. И еще: \textit{очи Господни тьмами темъ кратъ свѣтлѣйшіи солнца суть, прозирающіи вся пути человѣческія, и разсмотряющіи въ тайныхъ мѣстѣхъ: прежде неже создана быша, вся увѣдѣна Ему, такожде и по скончаніи}\footnote{Сир.~23,~27--29.}. Должно убо намъ предъ Богомъ, яко вся назирающимъ и воздающимъ комуждо по пути его, съ опасеніемъ, страхомъ и благоговѣніемъ ходить, и, что волѣ Его святой угодно, творить, да не прогнѣваемъ величество Его. \textit{Блюдите, како опасно ходите, не якоже немудри, но якоже премудри}, глаголетъ намъ Апостолъ\footnote{Еф.~5,~15.}. О человѣче! Богъ на тебе смотритъ, и что ни дѣлаешь, помышляешь, начинаешь, намѣреваешь, что любиши, чего ненавидиши, чимъ утѣшаешься, и чимъ оскорбляешися, чего ищеши, отъ чего убегаеши, каковъ къ нему, Создателю твоему, имѣешися, како съ ближнимъ твоимъ поступаеши, видитъ; како и что говориши, како и что вопрошаеши и отвѣтствуеши, слышитъ, и воздастъ тебѣ по пути твоему, и по плоду начинаній твоихъ. \textit{Блюдися} убо, возлюбленне, \textit{како опасно ходиши, не якоже немудръ, но якоже премудръ!}

Когда солнце сіяетъ на небеси, все ясно бываетъ; всякъ видитъ путь, по которому надобно итить, и куда итить, что дѣлать и чего уклоняться; видитъ, распознаетъ одну вещь отъ другой, и полезное отъ вреднаго, и прочая. И тако \textit{всякъ исходитъ на дѣло свое, и на дѣланіе свое до вечера}\footnote{Пс.~103,~23.}. Тако въ душѣ бываетъ, которую Хрістосъ "--- Солнце праведное просвѣтитъ. Таковая душа все ясно видитъ, познаетъ прелесть и суету міра сего, познаетъ добро и зло, порокъ и добродѣтель, вредъ и пользу, путь къ погибели и путь къ вѣчному животу ведущій, исходитъ на дѣла Богу угодная и себѣ полезная. Таковой душѣ сладко слово Божіе, якоже Самъ Богъ. Таковая душа все въ мірѣ семъ пріятное и дорогое за ничто имѣетъ, поминая Спасителево слово: \textit{кая польза человѣку, аще міръ весь пріобрящетъ, душу же свою отщетитъ}\footnote{Мѳ.~16,~26.}? И всегда къ блаженной вѣчности стремится: у ней то и на умѣ, какъ бы Богу, Создателю своему, угодить, и въ числѣ спасаемыхъ быть. Блаженна есть таковая душа, которую Божественный озаритъ Свѣтъ!

Когда солнце зайдетъ, нощь и тьма бываетъ; тогда люди не видятъ ничего, не распознаютъ одной вещи отъ другой, и ходятъ, какъ слѣпіи, въ ровъ падаютъ, и отъ вреда уклониться не знаютъ. Таково состояніе тѣхъ душъ, въ которыхъ свѣтъ Хрістовъ не возсіялъ! Осязаютъ и они, какъ слѣпіи, не раздѣляютъ добра отъ зла, не познаютъ пользы и вреда, падаютъ отъ грѣха въ грѣхъ; хватаются за то, что ничто въ себѣ; оставляютъ тое, что велико въ себѣ; думаютъ, что они идутъ прямымъ путемъ, но не познаютъ, что онъ въ ровъ погибели ихъ ведетъ. Таковыя суть, которыи къ суетѣ міра сего прилѣпилися, и въ томъ только поучаются, какъ бы собрать великое богатство, на честь высокую взойти, прославиться въ мірѣ семъ, и проч.; а о вѣчномъ спасенія сокровищѣ мало"=что пекутся, и какъ послѣднѣйшее дѣло имѣютъ. Удивленія, или паче сожалѣнія достойно есть! Видятъ, что умирающіи и отъ міра сего отходящіи все въ мірѣ оставляютъ, и наги отъ міра исходятъ, какъ и вошли въ міръ; однакожъ такъ тщатся богатѣть, какъ бы имъ вѣчно въ мірѣ жить. И не дивно, что язычники, не имѣющіи упованія, въ такой суетѣ запутались; а то христіане, позванные къ вѣчному животу и вѣчнымъ благимъ, о которыхъ всегда слышатъ во Евангеліи, дѣлаютъ тое. О бѣдный и окаянный человѣче! како не познаешь прелесть міра сего, гоняешися за тѣмъ, что въ малѣ является, и скоро, какъ дымъ, исчезаетъ; оставляеши тое, что истинно есть, и во вѣки пребываетъ! Все отъ насъ отходитъ, кромѣ единой добродѣтели, что ни имѣемъ въ мірѣ семъ; но никогда не отлучится, что въ вѣчности получимъ. "--- Но чтобы отъ таковой слѣпоты и тьмы человѣку свободитися, и свѣтомъ Хрістовымъ просвѣтитися, надобно ко Хрісту, Свѣту истинному, приступить, и со слѣпцами просить Его: \textit{помилуй ны Господи, Сыне Давидовъ}\footnote{Мѳ.~20,~30.}, и смотрѣть всегда на примѣръ святаго житія Его, и послѣдовать святымъ стопамъ Его: тогда Свѣтъ истинный, Хрістосъ, просвѣтитъ таковаго человѣка. Ибо когда имѣемъ предъ собою свѣтъ, и смотримъ на него, просвѣщаемся: тако, когда душа приступаетъ ко Хрісту, и взираетъ на образъ пресвятаго житія Его, и тому послѣдуетъ, свѣтомъ Его просвѣщается. Чимъ ближае мы къ свѣту, тѣмъ болѣе просвѣщаемся. Свѣтъ есть Хрістосъ: чимъ убо кто ближае къ сему Свѣту приходитъ, тѣмъ болѣе просвѣщается, и уже въ тьмѣ не будетъ ходить. \textit{Азъ есмь Свѣтъ міру: ходяй по Мнѣ, не имать ходити во тьмѣ, но имать свѣтъ животный}\footnote{Іоан.~8,~12.}. Аще убо ходящій по Хрістѣ не имать ходити во тьмѣ: неотмѣнно во тьмѣ ходитъ, кто отъ сего Свѣта удаляется.

Солнце всю поднебесную согрѣваетъ, и какъ бы оживляетъ: тако Богъ, Солнце вѣчное, теплотою любве Своея все созданіе, а наипаче родъ человѣческій согрѣваетъ и оживляетъ. \textit{О немъ бо живемъ и движемся и есмы}\footnote{Дѣян.~17,~28.}. А наипаче всю теплоту любве Своея изліялъ на насъ въ посланіи единороднаго Сына Своего къ намъ, "--- къ намъ, отступившимъ отъ Него, и погибшимъ. \textit{Тако бо возлюби Богъ міръ, яко и Сына Своего единороднаго далъ есть, да всякъ вѣруяй въ Онь не погибнетъ, но имать животъ вѣчный. Не посла бо Богъ Сына Своего въ міръ, да судитъ мірови, но да спасется Имъ міръ}\footnote{Іоан.~3,~16--17.}. \textit{Хвалите Господа вси языцы, похвалите Его вси людіе: яко утвердися милость Его на насъ, и истина Господня пребываетъ во вѣкъ}\footnote{Пс.~116,~1--2.}. Хвали и ты, душе моя, Господа! буди имя Господне благословенно отъ нынѣ и до вѣка! Помышляй, христіанине, великое сіе дѣло, и благодари Бога!

На солнце вси смотрятъ, и хотящіи теплотою его согрѣтися, къ нему исходятъ: такъ къ Богу вси вѣрніи взираютъ, и теплотою милости Его согрѣваются. \textit{Очи всѣхъ на Тя уповаютъ, и Ты даеши имъ пищу во благовременіи; отверзаеши Ты руку Твою, и исполняеши всякое животно благоволенія}\footnote{Пс.~144,~15--16; Іоан.~15,~5.}. \textit{Услыша ны, Боже Спасителю нашъ, упованіе всѣхъ концевъ земли, и сущихъ въ мори далече}\footnote{64,~6.}. Возводи и ты, христіанине, очи твои къ Богу, да теплотою милости Его согрѣешися, и почасту глаголи со Пророкомъ изъ глубины сердца твоего: \textit{къ Тебѣ возведохъ очи мои, живущему на небеси. Се яко очи рабъ въ руку господій своихъ, яко очи рабыни въ руку госпожи своея: тако очи наши ко Господу Богу нашему, дондеже ущедритъ ны}\footnote{122,~1--2.}.

Безъ солнца никакой плодъ не растетъ и не совершается: тако безъ Бога, вѣчнаго Солнца, никакое доброе дѣло ни начинается, ни дѣлается, ни совершается. Доброе хотѣніе наше, доброе начало наше, средина и конецъ "--- Его дѣло есть; Онъ въ насъ начинаетъ, дѣлаетъ и совершаетъ. Безъ Него не можемъ ничего творити: \textit{безъ Мене не можете творити ничесоже}\footnote{Іоан.~15,~5.}. "--- Отсюду учись, хрістіанине,~1)~познавать немощь свою и окаянство и ничтожество, что ты самъ въ себѣ не иное что еси, какъ изсохшее древо, которое никакого плода не можетъ творити. 2)~Отъ сего научайся смиренію. 3)~Всякое доброе дѣло, какое содѣлалъ ты, или дѣлаешь, Богу единому приписывай, да не Его дѣло себѣ присвоиши, и славу Его похитиши, и тако тяжко согрѣшиши. 4)~На всякое время воздыхай къ Богу, да не отыметъ отъ тебе всемогущія руки Своей, и тако, безъ помощи Божіей, отъ грѣха въ грѣхъ падати будеши. Повторяй часто Псаломника молитву: \textit{не остави мене, Господи Боже мой, не отступи отъ мене; вонми въ помощь мою, Господи спасенія моего}\footnote{Пс.~37,~33--33.}. 5)~Когда посылаетъ тебѣ Богъ напасть, скорбь и печаль, то хощетъ исправити тя, и древомъ плодовитымъ сотворити тя: потерпи убо Господу твоему, якоже терпиши врачу, горькимъ лѣкарствомъ пользующему тебе. Горькое лѣкарство "--- плоти скорбь и печаль: но симъ душа болящая исцѣляется. Убо \textit{потерпи Господа, мужайся, и да крѣпится сердце твое, и потерпи Господа}\footnote{26,~14.}.

Солнце не престаетъ никогда отъ теченія своего, но всегда отъ востока къ западу идетъ, и свѣтъ и теплоту свою на поднебесную низпущаетъ: тако Богъ никогда не престаетъ отъ благотворенія, но всегда благотворитъ намъ, естество бо Его такое есть. Богъ бо есть существенно благъ, такъ, что \textit{никтоже есть благъ, токмо единъ Богъ}, по свидѣтельству Спасителя\footnote{Мѳ.~19,~17.}, и потому не можетъ не благотворити.

Солнце сіяетъ, и теплоту свою испущаетъ на злыхъ и добрыхъ: тако Солнце вѣчное "--- Богъ, и добрымъ и злымъ, благочестивымъ и нечестивымъ благотворитъ. \textit{Благослови душе моя Господа}\footnote{Пс. 102.}. Да подражаемъ и мы въ семъ дѣлѣ Создателю нашему, и сотворимъ доброе добрымъ и злымъ, любящимъ и ненавидящимъ насъ, по увѣщанію Апостола: \textit{бывайте подражатели Богу, якоже чада возлюбленная}\footnote{Еф.~5,~1.}. \textit{Яко солнце Свое сіяетъ на злыя и благія, и дождитъ на праведныя и на неправедныя}\footnote{Мѳ.~5,~45.}.

Солнце на всѣ вещи равно теплоту низпущаетъ; но иныя вещи теплотою его растаеваютъ, какъ"=то воскъ и проч.; а иныя ожесточаются, какъ"=то глина, и проч.: тако Богъ равно всѣмъ благотворитъ, и теплоту благости Своея низпосылаетъ; но люди иныи благостію Его умягчаются и творятъ покаяніе; иныи ожесточаются и погибаютъ, какъ"=то Фараонъ ожесточился и погиблъ, "--- что и нынѣ въ мірѣ дѣлается. \textit{О человѣче! или о богатствѣ благости Его и кротости и долготерпѣніи нерадиши, невѣдый, яко благость Божія на покаяніе тя ведетъ}\footnote{Римл.~2,~4.}. Берегись убо, человѣче, благостію Божіею ожесточитися; но паче подвигнися къ покаянію, и долготерпѣніе Божіе, во спасеніе твое, помышляй!

Солнце въ чистой и тихой водѣ ясно видится, и подобіе его изображается: тако Богъ, вѣчное Солнце, въ тихой, непорочной и чистой душѣ показуется, и образъ Свой въ ней изображаетъ. \textit{Очистимъ убо себе, о возлюбленніи, отъ всякія скверны плоти и духа, творяще святыню въ страсѣ Божіи}, увѣщеваетъ насъ Апостолъ\footnote{2~Кор.~7,~1.}, да и въ насъ вселится Богъ, вѣчное Солнце, и тако образъ его святый въ насъ изобразится.

Солнечнаго свѣта не вмѣщаютъ дебелыя и густыя вещи, какъ"=то: земля, стѣны каменныя и древянныя, и проч.; напротивъ того, стекло, чистая вода, хрусталь и прочее вмѣщаютъ: тако умъ, грѣхами и похотьми міра сего помраченный, не можетъ вмѣстити Божія просвѣщенія. Подобное бо въ подобное вмѣщается. \textit{Сего ради глаголетъ: востани спяй, и воскресни отъ мертвыхъ, и освѣтитъ тя Хрістосъ}\footnote{Еф.~5,~14.}. Покайся, и очисти душу твою покаяніемъ и слезами, и отжени облакъ суетныхъ помышленій твоихъ, и тогда просвѣтитъ тя Хрістосъ!

Солнце чимъ болѣе приближается, тѣмъ меньшая тѣнь бываетъ; чимъ болѣе удаляется, тѣмъ большая тѣнь бываетъ; какъ зайдетъ солнце, то и тѣнь изчезаетъ: тако, чимъ болѣе Богъ къ человѣку приближается, тѣмъ меньшимъ самъ въ себѣ человѣкъ дѣлается, болѣе уничтожается и смиряется; видитъ бо свое недостоинство и ничтожество, и Божіе величество, и потому смиряется. Напротивъ того, чимъ болѣе Богъ отъ человѣка удаляется, тѣмъ болѣе человѣкъ возносится, величается, гордится. А какъ совсѣмъ удалится Богъ отъ человѣка, погибаетъ человѣкъ, якоже тѣнь исчезаетъ, когда солнце зайдетъ. Берегись, человѣче, высокоумія, да не яко діаволъ падеши; не высокоумствуй, но бойся. \textit{Яко всякъ возносяйся смирится, смиряяй же себе, вознесется}\footnote{Лук.~18,~14.}.

Кто хощетъ отъ солнца просвѣтиться и согрѣться, тому должно изъ темнаго мѣста выйтить, и предъ солнцемъ стать: тако, кто хощетъ отъ Бога, вѣчнаго Солнца, просвѣтиться, и теплотою сладчайшія любве Его согрѣться, тому должно тьму грѣховную оставить, беззаконное и нечистое житіе возненавидѣть, къ Нему обратиться и приступить съ покаяніемъ и сокрушеніемъ сердца, и молиться; и тогда свѣтомъ Его Божественнымъ просвѣтится, и теплотою любве Его Божественныя согрѣется сердце его. \textit{Приступите къ Нему и просвѣтитеся, и лица ваша не постыдятся}\footnote{Пс.~33,~6.}. Обратимся убо, хрістіанине, къ Богу всѣмъ сердцемъ, да свѣтомъ Его просвѣтимся и любовію согрѣемся.

Больнымъ глазамъ не возможно на солнце смотрѣть и видѣть его: тако душѣ, болящей лукавствомъ, тщеславіемъ, гордостію, самомнѣніемъ, самолюбіемъ, любовію міра сего, не возможно видѣть вѣчное Божество. Богъ видится простымъ и здравымъ окомъ душевнымъ. Бога, безъ Самого Бога, познать и видѣть не возможно. А открываетъ Онъ Себе простымъ и незлобивымъ младенцамъ. \textit{Утаилъ еси сія отъ премудрыхъ и разумныхъ, и открылъ еси та младенцемъ}\footnote{Мѳ.~11,~25.}. Да будемъ убо, хрістіанине, простіи и чистіи сердцемъ, яко да Бога узримъ.

Кто много на солнце смотритъ, и прилѣжно внимаетъ тому, у того глаза помрачаются и повреждаются: тако кто Бога "--- вѣчное Солнце любопытно разсматриваетъ, и непостижимыя Его тайны тщится испытать, у того умъ помрачается, и въ великое заблужденіе впадаетъ. Берегись, человѣче, испытывать, что тебѣ не надо знать. Довлѣетъ намъ тое только о Бозѣ разсуждать, что святое Его Слово открыло.

Кто къ солнцу идетъ, за тѣмъ тѣнь послѣдуетъ; кто отъ солнца отходитъ, отъ того тѣнь отступаетъ: тако кто ко Хрісту, праведному Солнцу, приближается, и вѣрою и любовію Ему послѣдуетъ, за тѣмъ ненависть, злоба и гоненіе міра сего неотлучно послѣдуетъ. Міръ бо, лжею діавольскою ослѣпленный, истины Божія не любитъ; и потому, какъ Хріста, самую Истину, ненавидѣлъ, гналъ, такъ и хрістіанина, держащагося истины, ненавидитъ и гонитъ; и сіе то есть, что написалъ Апостолъ: \textit{и вси же, хотящіи благочестно жити о Хрістѣ Іисусѣ, гоними будутъ}\footnote{2~Тим.~3,~12.}. Однакоже о семъ, возлюбленный хрістіанине, не скорби, что міръ тебе ненавидитъ. Хрістосъ большій и лучшій несравненно всего міра, и любовь Его сладчайшая есть паче всего міра. Кого Хрістосъ любитъ и милуетъ, тому ненависть и злоба міра сего ничего не вредитъ. Только смотри, чтобы ты Хрістовъ былъ: Хрістосъ Своего не оставитъ. Толико Онъ за тебе претерпѣлъ и пострадалъ: оставитъ ли тебе въ нуждѣ твоей? Слыши утѣшительное Его обѣщаніе Своимъ рабамъ: \textit{съ нимъ есмь въ скорби, и изму его}\footnote{Пс.~90,~15.}. А откуду познается, кто Хрістовъ есть, примѣту Апостолъ показуетъ: \textit{иже Хрістовы суть, плоть распяша со страстьми и похотьми}\footnote{Гал.~5,~24.}. И паки: \textit{да отступитъ отъ неправды всякъ именуяй имя Господне}\footnote{2~Тим.~2,~19.}. Буди убо Хрістовъ, и весь свѣтъ и вси діаволы не повредятъ тебе, яко Онъ несравненно сильнѣйшій всѣхъ, и земныи и преисподніи всѣ силы Его Божественныя трепещутъ.

Когда небо густыми и темными облаками покрывается и великая буря находитъ, "--- кажется, что будто солнце оставляетъ насъ: однакожъ оно всегда и въ такое время тончайшіи свои лучи низпущаетъ къ намъ. Тако, когда темный облакъ и буря искушеній находитъ на насъ, и покрываетъ насъ, "--- тогда кажется, что аки бы Богъ оставилъ насъ: однакожъ Богъ и въ такой тьмѣ вѣрнаго Своего не оставляетъ, но сокровенною силою Своею сохраняетъ его. И яко \textit{вѣренъ, не оставитъ его искуситися паче, еже можетъ; но сотворитъ со искушеніемъ и избытіе, яко возмощи ему понести}\footnote{1~Кор.~10,~13.}. "--- Не отчаявайся убо, возлюбленный христіанине, когда искушеніе какое постигнетъ тя, но молись и призывай Его; тогда и ожидай помощи Его, по неложному обѣщанію Его: \textit{призови Мя въ день скорби твоея, и изму тя, и прославиши Мя}\footnote{Пс.~49,~15.}. Богъ ожидаетъ во искушеніи терпѣнія и подвига твоего, а не унынія и негодованія. \textit{Потерпи Господа, мужайся, и да крѣпится сердце твое, и потерпи Господа}\footnote{26,~14.}, безъ Котораго совѣта и изволенія ничто не бываетъ.

Какъ непогода и буря пройдетъ, солнце пріятнѣйше намъ сіяетъ: тако когда буря искушеній минуется, благопріятность и любезность благости Божія въ сердцѣ человѣческомъ чувствуется. Тогда сладчайшее ведро въ душѣ сіяетъ; тогда душа утѣшается, мира и покоя, какъ сладчайшія вечери, насыщается; тогда въ душѣ таковой \textit{милость и истина срѣтаются, правда и миръ облобызаются}\footnote{Пс.~84,~11.}. Тогда вкушаетъ и видитъ душа, \textit{яко благъ Господь}\footnote{33,~9.}. Таковая сладость, по претерпѣнномъ искушеніи, въ сердцѣ человѣческомъ послѣдуетъ. Терпи убо нашедшее искушеніе, хрістіанине, да потомъ почувствуеши благопріятное и сладчайшее ведро въ душѣ твоей.

Сверхъ того, когда смотримъ на солнце, хрістіанине, воспомянемъ о томъ, что Хрістосъ Господь сказалъ: \textit{тогда праведницы просвѣтятся, яко солнце, въ царствіи Отца ихъ}\footnote{Мѳ.~13,~43.}. Въ такъ великую и чудную славу облекутся избранніи Божіи, что какъ солнце будутъ сіять. Да взираемъ убо душевными очами на оную славу, и презримъ прелестную міра сего славу и честь.

\section{3. Отецъ и дѣти.}

Дѣти отъ отца раждаются: тако хрістіане, люди обновленія и сынове Божіи по благодати, раждаются отъ Бога. \textit{Елицы пріяша Его} (Хріста Сына Божія), \textit{даде имъ область чадомъ Божіимъ быти, вѣрующимъ во имя Его. Иже не отъ крове, ни отъ похоти плотскія, ни отъ похоти мужескія, но отъ Бога родишася}\footnote{Іоан.~1,~12 и 13.}. Дѣти раждаются отъ сѣмене отчаго: хрістіане раждаются водою и Духомъ, словомъ Божіимъ и вѣрою. \textit{Восхотѣвъ породи насъ словомъ истины}, и проч.\footnote{Іак 1,~18.} \textit{Порождена не отъ сѣмене истлѣнна, но неистлѣнна, словомъ живаго Бога и пребывающаго во вѣки}\footnote{1~Петр.~1,~23.}.

Дѣти отъ отца по плоти раждаются: хрістіане раждаются отъ Бога по Духу. \textit{Рожденное отъ плоти, плоть есть: и рожденное отъ Духа, духъ есть}\footnote{Іоан.~3,~6.}.

Въ дѣтяхъ примѣчаются свойство и подобіе отчее: тако въ христіанѣхъ должны быть свойство и подобіе Отца небеснаго; должны и они быть святы, благи, милосердни, кротки, терпѣливы, и проч. Откуду глаголется имъ: \textit{по звавшему вы Святому, и сами святи во всемъ житіи будите. Зане писано есть: святи будите, яко Азъ святъ есмь}\footnote{1~Петр.~1,~16.}. \textit{Будите милосерди, якоже и Отецъ вашъ милосердъ есть}, и проч.\footnote{Лук.~6,~36.}

Дѣти что видятъ въ отцѣ своемъ, то и сами тщатся дѣлать: тако и хрістіане должны подражать Отцу своему небесному, въ чемъ можно. \textit{Бывайте подражатели Богу, якоже чада возлюбленная}\footnote{Ефес.~6,~1.}. Богъ никого не обидитъ: тако и они должны никого не обижать. Богъ всѣмъ благотворитъ: тако и они должны всѣмъ благое творить. Богъ всѣмъ кающимся отпущаетъ грѣхи: тако и хрістіане отпущать должны человѣкомъ согрѣшенія ихъ. Богъ грѣха ненавидитъ: тако и хрістіане должны грѣха ненавидѣть и отъ него уклоняться. \textit{Богъ всѣмъ хощетъ спастися, и въ разумъ истины пріити}\footnote{1~Тим.~2,~4.}: тако и хрістіане сердечно должны желать всѣмъ спасенія, и проч.

Отецъ любитъ дѣтей своихъ, и дѣти любятъ отца своего: тако Богъ любитъ хрістіанъ, и хрістіане должны Бога, яко Отца своего, сердечно любить.

Дѣти отцу своему угождаютъ: тако и хрістіане должны Богу, Отцу своему, угождать. Дѣти всего того берегутся, чимъ отецъ ихъ оскорбляется: тако и хрістіанамъ должно всегда того удаляться, чимъ Отецъ ихъ небесный оскорбляется; оскорбляется же всякимъ грѣхомъ и презрѣніемъ добродѣтели.

Отецъ съ дѣтьми, и дѣти съ отцемъ любовно разглагольствуютъ: тако Богъ съ вѣрными душами любовно въ Святомъ Писаніи бесѣдуетъ, и вѣрныя души съ Богомъ любезно въ молитвѣ и въ славословіи бесѣдуютъ. О любезная и сладкая бесѣда, которая бываетъ между величествомъ Божіимъ и подлымъ человѣкомъ, иже есть земля и пепелъ! \textit{Господи! что есть человѣкъ, яко познался еси ему? или сынъ человѣчь, яко вмѣняеши его}\footnote{Пс.~142,~3.}?

Отецъ о дѣтяхъ радуется, и дѣти о отцѣ своемъ взаимно утѣшаются: тако небесный Отецъ о вѣрныхъ душахъ радуется, и вѣрныя души о Немъ радуются. \textit{Сердце мое и плоть моя возрадовастася о Бозѣ живѣ}\footnote{83,~3.}.

Дѣти отца своего отъ любви называютъ "--- \textit{отче} или \textit{батюшко}: тако хрістіане Бога отъ любви называютъ Отцемъ, и вопіютъ къ нему: \textit{Авва Отче}\footnote{Римл.~8,~15.}! \textit{Да яко есте сынове, посла Богъ Духа Сына Своего въ сердца ваша, вопіюща: Авва Отче}\footnote{Гал.~4,~6.}! \textit{Отче нашъ, Иже еси на небесѣхъ!}

Дѣти всего у отца просятъ, но отецъ не все имъ даетъ, а только тое, что нужно и полезно имъ: тако хрістіане всего у небеснаго своего Отца просятъ, но не все имъ подаетъ, а только нужное и полезное. \textit{Кто есть отъ васъ человѣкъ, егоже аще воспроситъ сынъ его хлѣба, еда камень подастъ ему? Или аще рыбы проситъ, еда змію подастъ ему? Аще убо вы лукави суще, умѣете даянія блага даяти чадомъ вашимъ: кольми паче Отецъ вашъ небесный дастъ блага просящимъ у Него}\footnote{Мѳ.~7,~9--11.}.

Дѣти предъ отцемъ своимъ поступаютъ благоговѣйно, ничего не дѣлаютъ и не говорятъ непристойнаго, и всякое почтеніе ему доказываютъ: тако хрістіанамъ предъ Богомъ вездѣсущимъ и вся назирающимъ должно ходить со страхомъ и благоговѣніемъ, ничего ни дѣлать, ни говорить, ни мыслить непристойнаго.

Отецъ о дѣтяхъ промышляетъ, печется и воспитываетъ ихъ: тако Богъ о хрістіанахъ промышляетъ, печется и воспитываетъ ихъ словомъ Своимъ и животворящими тайнами. Дѣти отъ навѣтовъ злыхъ людей подъ защищеніе отца прибѣгаютъ: тако хрістіане, навѣтуемы отъ діавола и злыхъ его служителей, подъ покровъ и защищеніе Отца небеснаго прибѣгаютъ. \textit{Отче, избави насъ отъ лукаваго}\footnote{6,~13.}. Они дерзаютъ и поютъ: \textit{Богъ намъ прибѣжище и сила, помощникъ въ скорбехъ, обрѣтшихъ ны зѣло}\footnote{45,~2.}.

Отецъ дѣтей, за погрѣшности наказуетъ, но съ любовію: тако Богъ, Отецъ небесный, наказуетъ хрістіанъ не отъ гнѣва, но отъ любви, за согрѣшеніе ихъ. \textit{Егоже любитъ Господь, наказуетъ; біетъ же всякаго сына, егоже пріемлетъ. Аще наказаніе терпите, якоже сыновомъ обрѣтается вамъ Богъ: который бо есть сынъ, егоже не наказуетъ отецъ}\footnote{Евр.~12,~6--7.}?

Дѣти добрыя смиряются, погрѣшности свои исповѣдуютъ, и признаютъ себе виноватыми предъ отцемъ своимъ: тако должно хрістіаномъ смиряться, отъ сердца исповѣдывать согрѣшенія своя, и неправду свою признавать и исповѣдаться: \textit{благо мнѣ, яко смирилъ мя еси}\footnote{Пс.~118,~71.}. Отецъ наказуетъ дѣтей своихъ, чтобы честными были: Богъ наказуетъ насъ, \textit{да причастимся святыни Его}\footnote{Евр.~12,~10.}.

Отецъ дѣтямъ своимъ готовитъ наслѣдіе: Богъ тако хрістіанамъ готовитъ наслѣдіе вѣчнаго живота и небеснаго царствія. Отецъ въ свое время даетъ наслѣдіе дѣтямъ своимъ: тако Богъ, когда пріидетъ время, дастъ наслѣдіе хрістіанамъ, и утѣшительно возглаголетъ имъ: \textit{пріидите благословенніи Отца Моего, наслѣдуйте уготованное вамъ царствіе отъ сложенія міра}\footnote{Мѳ.~25,~34.}.

Дѣти добріи вездѣ поступать честно тщатся, чтобы не обезславить имя отца своего: тако должно и хрістіанамъ предъ людьми жить, дабы имя Божіе не похулилось. Славится и похваляется отецъ, когда дѣти, и постоянно, добрѣ живутъ: тако славится имя Отца небеснаго, когда хрістіане благочестиво и достойно званія жительствуютъ. Къ сему увѣщеваетъ насъ Господь: \textit{тако да просвѣтится свѣтъ вашъ предъ человѣки, яко да видятъ ваша добрая дѣла, и прославятъ Отца вашего, Иже на небесахъ}\footnote{5,~16.}. Дѣти, какое ни дѣлаютъ почтеніе и угожденіе отцу своему, должное ему, яко родителю, воспитателю и промышлителю, воздаютъ, иначе бы не благодарны были: тако хрістіане, какъ ни тщатся угождать небесному Отцу, должное отдаютъ, и тѣмъ заслужить ничего не могутъ; но что отъ Бога получаютъ, туне получаютъ. Богу бо, за благодѣяніе Его къ намъ показанное и показуемое, никакъ и ничимъ не можемъ воздать, но всегда предъ Нимъ должники остаемся. Дѣти, когда кто ихъ при отцѣ безчеститъ, или како обижаетъ, не отмщеваютъ сами обидящему, но на отца своего взираютъ, и ему обиду свою поручаютъ: тако должно хрістіанамъ, когда ихъ кто обижаетъ, не самимъ за себе отмщевать, но къ небесному своему Отцу возводить умныя очи, и Тому, яко судящему праведно, отмщеніе предавать, Который глаголетъ: \textit{Мнѣ отмщеніе, Азъ воздамъ}\footnote{Римл.~12,~19.}.

Отецъ сына непостояннаго и по довольномъ наказаніи неисправнаго отрекается, лишаетъ его наслѣдія, и пущаетъ его по своей волѣ жить: тако Отецъ небесный неисправныхъ хрістіанъ, который не хотятъ Его слушать, отрекается, исключаетъ ихъ отъ наслѣдія вѣчнаго живота, и пущаетъ ихъ уже по своей волѣ жить. \textit{И не послушаша людіе Мои гласа Моего, и Израиль не внятъ Ми. И отпустихъ я по начинаніямъ сердецъ ихъ, пойдутъ въ начинаніихъ своихъ}\footnote{Пс.~80,~12--13.}.

Отецъ чужихъ дѣтей не наказуетъ, хотя и видитъ ихъ неисправныхъ: тако Отецъ небесный тѣхъ людей, которыи не Его присніи и не домашніи, оставляетъ безъ наказанія, яко не своихъ: \textit{аще безъ наказанія есте, емуже причастницы быша вси, убо прелюбодѣйчищи есте, а не сынове}\footnote{Евр.~12,~8.}.

Отецъ дѣтей своихъ наказуетъ, но и утѣшаетъ: тако Отецъ небесный наказуетъ чадъ Своихъ, истинныхъ хрістіанъ, но и утѣшаетъ ихъ. \textit{Смиривый помилуетъ по множеству милости Своея}\footnote{Пл. Іер.~3,~32.}. \textit{По множеству болѣзней моихъ въ сердцѣ моемъ, утѣшенія Твоя возвеселиша душу мою, Боже}\footnote{Пс.~93,~19.}.

Разсуждай, хрістіанине, и примѣчай, каковъ Богъ къ хрістіаномъ, и каковы хрістіане къ Богу должны быть.

Отсюду утѣшеніе и радость хрістіанская: 1)~Какъ тѣсный союзъ и общеніе истинныи хрістіане съ Богомъ имѣютъ! Такъ, какъ дѣти со отцемъ: \textit{общеніе наше со Отцемъ и Сыномъ Его Іисусомъ Хрістомъ}, глаголетъ Апостолъ\footnote{1~Іоан.~1,~3.}. Коль великое сіе и преславное дѣло, постигнуть умомъ невозможно. 2)~Коль великое есть и высокое достоинство хрістіанъ! Велико есть быть чадомъ нѣкоего господина, болѣе вельможи, далеко болѣе быти царскимъ чадомъ: коль несравненно болѣе быти чадомъ Божіимъ! Сему удивляется Апостолъ святый, и глаголетъ: \textit{видите, какову любовь далъ есть Отецъ намъ, да чада Божія наречемся, и есмы: сего ради міръ не знаетъ насъ, зане не позна Его}\footnote{3,~1.}. 3)~Аще хрістіане суть чада Божія: то въ какую славу облекутся, когда открыются чада Божія! Царь земный одѣваетъ дѣтей своихъ въ красныя и свѣтлыя одежды: въ коль преславныя и свѣтлыя одежды облекутся чада Божія! \textit{Преобразитъ тѣло смиренія нашего, яко быти сему сообразну тѣлу славы Его}\footnote{Фил.~3,~21.}. \textit{Возлюбленніи, нынѣ чада Божія есмы, и не у явися, что будемъ: вѣмы же, яко, егда явится, подобни Ему будемъ, и узримъ Его, якоже есть}\footnote{Іоан.~3,~2.}. \textit{Помяни насъ Господи во благоволеніи людей Твоихъ, посѣти насъ спасеніемъ Твоимъ: видѣти во благости избранныя Твоя, возвеселитися въ веселіи языка Твоего, хвалитися съ достояніемъ Твоимъ}\footnote{Пс. 105.~4,~5.}.

\section{4. Господинъ и рабъ.}

Рабъ называется именемъ того господина, котораго есть рабъ: тако хрістіанинъ называется хрістіаниномъ, понеже есть рабъ Хрістовъ. Рабъ того ради называется рабомъ нѣкоего господина, яко работаетъ ему, волю его творитъ и угождаетъ ему: тако хрістіанинъ того ради называется рабомъ Хрістовымъ, яко работаетъ Ему, волю Его творитъ и угождаетъ Ему. Господинъ не признаетъ за вѣрнаго раба своего того раба, который воли его не творитъ: тако Хрістосъ не признаетъ того хрістіанина за вѣрнаго Своего раба, который заповѣдей Его не исполняетъ. \textit{Что мя зовете Господи, Господи, и не творите, яже глаголю?}\footnote{Лук.~6,~46.}

Рабъ предъ господиномъ со страхомъ и благоговѣніемъ ходитъ, дабы не прогнѣвать его: тако хрістіанамъ предъ Хрістомъ Господемъ своимъ ходить должно со страхомъ и благоговѣніемъ, дабы Его не прогнѣвать. Хотя и не видятъ Его предъ собою, но Онъ видитъ всѣхъ и смотритъ на всѣхъ.

Рабъ не знаетъ, когда его господинъ его позоветъ къ себѣ; и потому добрый рабъ на всякое время готовъ есть, и ожидаетъ его: тако хрістіанинъ не знаетъ, когда его Хрістосъ Господь къ Себѣ позоветъ; зоветъ же всякаго Себѣ Хрістосъ чрезъ смерть, и потому добрый хрістіанинъ всегда къ исходу готовится. Господинъ когда куда отъѣдетъ, рабъ добрый всегда ожидаетъ приходу его: тако хрістіанамъ всегда должно ожидать пришествія Хрістова съ небеси, и къ срѣтенію Его готовиться \textit{Да будутъ чресла ваша препоясана, и свѣтильницы горящіи: и вы подобни человѣкомъ, чающимъ Господа своего, когда возвратится отъ брака: да пришедшу и толкнувшу, абіе отверзутъ Ему}\footnote{Лук.~12,~35 и 36.}.

Рабъ, когда явится предъ господиномъ неисправенъ, наказуется: тако хрістіанинъ, когда явится предъ Хрістомъ Господемъ неисправенъ, наказаніе пріиметъ.

Рабъ, какъ вѣрно ни работаетъ господину своему, должное показываетъ, творитъ тое, что должно творить: тако хрістіанинъ, хотя и вѣрно работаетъ Хрісту Господу, однакожъ долженъ себе признавать недостойнымъ. \textit{Егда сотворите вся повелѣнная вамъ, глаголите, яко раби неключими есмы: яко, еже должни бѣхомъ сотворити, сотворихомъ}\footnote{17,~10.}. "--- Однакожъ рабъ за вѣрную службу награжденіе и мзду получаетъ отъ добраго господина своего, по единой милости его: тако хрістіане, раби Хрістовы, наградятся отъ Хріста Господа въ будущемъ вѣцѣ, но по единой Его милости, а не по заслугамъ ихъ.

Раби непостоянныи и злыи, за глазами господина своего, по волѣ своей живутъ, и своевольствуютъ; дѣлаютъ, что хотятъ, помышляя, что не видитъ ихъ господинъ ихъ: тако многіи безсовѣстныи хрістіане поступаютъ противъ совѣсти и закона Божія, такъ, какъ бы ихъ Хрістосъ не видѣлъ, и только что не говорятъ: «Хрістосъ есть на небеси, далеко отъ насъ, не видитъ насъ и дѣлъ нашихъ». А можетъ быть, что многіе думаютъ, и говорятъ такъ. Но Псаломникъ поетъ всѣмъ: \textit{съ небесе призрѣ Господь, видѣ вся сыны человѣческія: отъ готоваго жилища Своего призрѣ на вся живущія на земли: создавый на единѣ сердца ихъ, разумѣваяй на вся дѣла ихъ}\footnote{Псал.~32,~13--15.}. И пророкъ Іеремія глаголетъ ко Господу: \textit{очи Твои, Господи, отверсты на вся пути сыновъ человѣческихъ, дати комуждо по пути его и по плоду начинаній его}\footnote{Іер.~32,~19.}. И Давидъ святый обличаетъ ихъ: \textit{разумѣйте безумніи въ людехъ, и буіи нѣкогда умудритеся: Насаждей ухо не слышитъ ли? или Создавый око, не сматряетъ ли? Наказуяй языки, не обличитъ ли, учай человѣка разуму}\footnote{Пс.~93,~8--10.}? И самъ Господь грѣшнику глаголетъ: \textit{обличу тя, и представлю предъ лицемъ твоимъ грѣхи твоя}\footnote{49,~21.}.

Аще бы рабъ въ такое безуміе и неистовство пришелъ, что предъ господиномъ своимъ дерзнулъ бы кричать, шумѣть, безмѣрно смѣяться, скакать, плясать и прочее безчиніе дѣлать, "--- крайнее непочтеніе и досажденіе показалъ бы господину своему, и на великій бы гнѣвъ его подвигнулъ: таковое безуміе и неистовство показываютъ хрістіане, которыи предъ Хрістомъ Господомъ, вся назирающимъ, беззаконнуютъ и своевольствуютъ. Сюды надлежатъ безчинные кличи, кулачные бои, плясанія, танцы, скверныя и неподобныя пѣсни, конскія ристанія, ссоры и брани, ругательства, клеветы и осужденія ближняго, безчинныя собранія, пиршества, піянства, любодѣянія, прелюбодѣянія, хищенія, воровства, и прочая подобная симъ. Все сіе есть безуміе и неистовство хрістіанское, которое предъ очами Вседержителя Іисуса Хріста Господа совершается отъ слѣпыхъ хрістіанъ. Таковыя вси великое непочтеніе и досажденіе дѣлаютъ Хрісту Господу, и его на гнѣвъ праведный подвигаютъ. Отсюду страшныи бываютъ казни, какъ"=то: великіе пожары, моровыя язвы, нашествіе иноплеменниковъ, трясеніе земли, и проч. И воистину великая есть слѣпота человѣческая и безуміе! Рабъ предъ господиномъ "--- человѣкомъ не дерзаетъ безчинствовать: но хрістіане предъ Хрістомъ "--- Богомъ не ужасаются дѣлать того, что человѣкъ предъ человѣкомъ не смѣетъ дѣлать, хотя и слышатъ изъ Святаго Писанія, что Богъ все видитъ. Тако ослѣпляетъ сатана разумы людей, чтобы погибели своей не видали! "--- Берегись, человѣче, предъ Богомъ дѣлать тое, что дѣлать предъ властію земною боишися, и предъ честнымъ человѣкомъ стыдишися. Терпѣлъ тебѣ доселѣ Богъ, но впредь стерпитъ ли, неизвѣстно. Богъ нашъ огнь есть поядаяй: страшно впасти въ руцѣ Божіи. Разсуждай сіе.

\section{5. Царь и почтенный отъ него подданный преступникъ.}

Бываетъ, что царь подданнаго своего жалуетъ, подаетъ ему высокій рангъ, обогащаетъ его, и прочую высокую свою милость являетъ ему; но онъ, царскою высокою милостію возгордѣвся, и царю вмѣсто благодарности неблагодарность показываетъ, законы его нарушаетъ, и отечеству недоброхотъ является. Таковаго неблагодарнаго и безсовѣстнаго человѣка хотячи отъ милости своей отвергнуть, царь жалится на него всему отечеству, и злодѣяніе его и неблагодарность его публикуетъ: «я"=де такому"=то человѣку, тому"=то подданному моему, такія и такія милости показалъ; но онъ, все тое пренебрегши, и мнѣ и отечеству своему зло воздалъ». И тако таковый законопреступникъ всей царской милости лишается, и казни предается. Тако Богъ, Царь небесный, жалится предъ всѣмъ свѣтомъ на людей, подданныхъ Своихъ, которыи имя Его знаютъ, но Его не почитаютъ; многоразличными отъ Него благодѣяніями обогащены, но тое все забываютъ, и неблагодарны къ Нему являются, и законъ Его святый и праведный нарушаютъ, и тако преогорчеваютъ и раздражаютъ Его, якоже читаемъ въ книгахъ пророческихъ и во псалмѣхъ о Іудеяхъ неблагодарныхъ: \textit{и забыша благодѣянія Его, и чудеса Его, яже показа имъ}\footnote{Пс.~77,~11.}. Чего имъ человѣколюбивый Господь не дѣлалъ? какихъ милостей не показывалъ? какихъ благодѣяній не являлъ? Избралъ ихъ въ людей Своихъ, и явилъ имъ имя Свое; сшедшихъ во Египетъ и озлобленныхъ отъ мучителя стенаніе услышалъ, вступился за нихъ и избавилъ ихъ, поразилъ Египта съ первенцы его, и извелъ Израиля отъ среды ихъ рукою крѣпкою и мышцею высокою; раздѣлилъ Чермное море и провелъ Израиля посредѣ его; истряслъ Фараона и силу его въ море Чермное; привелъ ихъ пустынею и всякими благими въ пустыни удовольствовалъ ихъ; поразилъ предъ лицемъ ихъ цари великія, и убилъ цари крѣпкія; и далъ землю, ихъ достояніе, достояніе Израилю, рабу Своему: и тако \textit{введе я въ гору святыни Своея, гору сію, юже стяжа десница Его. И изгна отъ лица ихъ языки, и по жребію даде имъ (землю) ужемъ жребодаянія, и всели въ селеніяхъ ихъ колѣна Израилева}\footnote{Псал.~77,~54 и 55.}. Но \textit{коль краты преогорчиша Его въ пустынѣ, прогнѣваша Его въ земли безводнѣй}\footnote{Тамъ же,~40.}, якоже о томъ пишется въ книгахъ Моѵсеовыхъ; такожде и вшедшіи въ землю обѣтованную \textit{коль краты преогорчиша Его, и искусиша и преогорчиша Бога вышняго, и свидѣній Его не сохраниша, и отвратишася и отвергошася, якоже и отцы ихъ: превратишася въ лукъ развращенъ. И прогнѣваша Его въ холмѣхъ своихъ, и во истуканныхъ своихъ раздражиша Его. Слыша Богъ, и презрѣ, и уничижи зѣло Израиля}\footnote{Тамъ же,~56--59.}. И предъ небомъ и землею свидѣтельствуетъ ихъ неблагодарность: \textit{слыши небо, и внуши земле, яко Господь возглагола; сыны родихъ и возвысихъ, тіи же отвергошася Мене. Позна волъ стяжавшаго и, и оселъ ясли господина своего: Израиль же Мене не позна, и людіе Мои не разумѣша}\footnote{Ис.~1,~2 и 3.}. Посылалъ къ нимъ рабовъ Своихъ пророковъ, дабы ихъ къ покаянію привести, и къ Себѣ обратить: но Израиль не только не принялъ ихъ, но и побилъ ихъ. Потомъ Самъ пришелъ къ нимъ во образѣ человѣческомъ; но Евангелистъ святый съ жалостію свидѣтельствуетъ міру, что \textit{во своя пріиде, и свои Его не пріяша}\footnote{Іоан.~1,~11.}. И не только не приняли, но и похулили, озлобили, поругались и умертвили Того, Который ихъ пришелъ спасти. Чего ни дѣлалъ имъ Хрістосъ Господь, какъ ни старался обратить ихъ къ покаянію; но неисцѣльны и ожесточенны остались. \textit{Іерусалиме, Іерусалиме, избивый пророки, и каменіемъ побиваяй посланныя къ тебѣ! коль краты восхотѣхъ собрати чада твоя, якоже собираетъ кокошъ птенцы своя подъ крилѣ, и не восхотѣсте}. Наконецъ услышали страшное Божіе слово: \textit{се оставляется вамъ домъ вашъ пустъ}\footnote{Мѳ.~23,~37--38.}. \textit{Постиже на нихъ гнѣвъ до конца}\footnote{1~Сол.~2,~16.}! И такъ видитъ весь свѣтъ праведныя судьбы Божіи, видитъ и признаетъ, что праведно они отвержены отъ Бога; и исполнилось слово Хрістово, къ нимъ сказанное: \textit{яко отымется отъ васъ царствіе Божіе, и дастся языку, творящему плоды его}\footnote{Мѳ.~21,~43.}. Языковъ, которыми они гнушалися, обратилъ и призвалъ къ Себѣ Богъ; и вступили вмѣсто ихъ, и, обратившеся отъ идоловъ къ Богу живому, сдѣлались людьми Божіими. Но которыи лицемѣрно Бога почитаютъ, устами имя Божіе исповѣдуютъ, но дѣлами отмещутся Его, "--- обѣщались при крещеніи Богу работать, но работаютъ діаволу (чію бо кто волю исполняетъ, тотъ тому и работаетъ): таковыи хрістіане послѣдуютъ лицемѣрнымъ и строптивымъ жидамъ; и которое обличеніе отъ пророковъ и отъ Самого Господа Іудеомъ было за неправое и неблагодарное сердце ихъ, тоежде касается и лживыхъ хрістіанъ. Хрістіане бо вступили въ мѣсто Іудеовъ. Слова Божія книги въ руки себѣ взяли, но по слову Божію жить не хотятъ; Христа приняли, но дѣлами своими противятся Ему; слышатъ проповѣдниковъ всегда, яко пророковъ, отъ Бога посланныхъ, но ученію ихъ повиноваться не хотятъ, и что горше того, хулятъ ихъ, злословятъ и поносятъ имъ. Сего ради, какъ жиды нынѣ въ книгахъ святыхъ предъ всѣмъ свѣтомъ обличаются: тако неблагодарные и ложные хрістіане, въ послѣдній день на второмъ пришествіи Христовомъ, обличатся предъ святыми ангелами и избранными Божіими, и отымется отъ нихъ царствіе Божіе, и тогда вси святіи увидятъ и признаютъ праведный судъ Божій. Ибо, что нынѣ въ мірѣ семъ человѣкъ дѣлалъ, говорилъ, замышлялъ, начиналъ, желалъ и искалъ, тое все за нимъ на оный вѣкъ послѣдуетъ, и съ нимъ явится на судѣ Хрістовомъ. Добрая, злая, тайная дѣла, слова, помышленія человѣческая открыются тогда всѣмъ ангеломъ и человѣкомъ, глаголетъ Василій великій въ посланіи къ дѣвѣ падшей. "--- Убоимся убо, хрістіанине, и сотворимъ достойны покаянія плоды, да не и съ нами явятся на судѣ Хрістовомъ грѣхи наши, яко соперники наши, обличающіи насъ; загладимъ ихъ нынѣ покаяніемъ и слезами, да не и намъ скажется тогда: се человѣкъ и дѣла его! Горе человѣку тому, съ которымъ грѣхи его на всемірномъ позорищи явятся! По дѣламъ бо его воздастся ему.

\section{6. Плѣнники и свободитель ихъ.}

Бываетъ, что люди изъ отечества плѣняются на чужую сторону, и тамо отъ мучителя, плѣнившаго ихъ, всякое зло страждутъ. Царь добрый, милосердуя и жалѣя о людехъ своихъ, въ такое злостраданіе впадшихъ, посылаетъ къ нимъ свободителя, который ихъ или сребромъ искупляетъ, или инымъ какимъ образомъ свобождаетъ: тако Царь вѣчный, небесный Отецъ, жалѣя о насъ плѣненныхъ, послалъ къ намъ Единороднаго Сына Своего избавить насъ отъ плѣни горькой, и свидѣтельствуетъ намъ съ небеси: \textit{Сей есть Сынъ Мой возлюбленный, о Немже благоволихъ; Того послушайте}\footnote{Мѳ.~17,~5.}. И Самъ Сынъ Божій глаголетъ: \textit{Духъ Господень на Мнѣ, Егоже ради помаза Мя благовѣстити нищимъ, посла Мя исцѣлити сокрушенныя сердцемъ, проповѣдати плѣненнымъ отпущеніе и слѣпымъ прозрѣніе, отпустити сокрушенныя во отраду, проповѣдати лѣто Господне пріятно}\footnote{Лук.~4,~18--19.}. Понеже когда прародители наши въ раи послушали совѣта зміина, и Божію заповѣдь преступили: сдѣлалися плѣнниками діавола со всѣмъ своимъ потомствомъ, то"=есть, со всѣмъ родомъ человѣческимъ; а онъ торжествовалъ и ярился надъ ними, яко злый исполинъ и мучитель, и дѣлалъ съ ними, что хотѣлъ. Мало было людемъ солнце и прочая небесная свѣтила вмѣсто Бога почитать: почитали звѣрей, скотовъ, зеліе, и почти не было той твари, которой бы не боготворили. О прочихъ дѣлахъ, въ которыхъ волю плѣнившаго ихъ врага исполняли, срамно есть и глаголати. Въ такъ горькую плѣнь попался бѣдный человѣкъ! \textit{Вѣдомъ былъ во Іудеи Богъ: во Израили веліе имя Его}\footnote{Пс.~75,~2.}. Но и въ семъ малѣйшемъ свѣта углѣ, сколько тщался многокозненный врагъ истинное Богопочитаніе истребить! сколько разъ Израиля приводилъ во идолослуженіе! Сколько въ прочія тягчайшія беззаконія ввергалъ ихъ, такъ что \textit{пожроша сыны своя и дщери своя бѣсовомъ; и пролияша кровь неповинную, кровь сыновъ своихъ и дщерей, яже пожроша истуканнымъ ханаанскимъ: и убіена бысть земля ихъ кровми!}\footnote{105,~37--38.} Не сыскалось никого въ людехъ, кто бы отъ мучительскихъ рукъ его свободилъ насъ. Надобно было сильнѣйшему и крѣпчайшему паче его пріитить, и свободить насъ. \textit{Како можетъ кто внити въ домъ крѣпкаго, и сосуды его расхитити, аще не первѣе свяжетъ крѣпкаго и тогда домъ его расхититъ?}\footnote{Мѳ.~12,~29.} Хрістосъ Сынъ Божій, видя, что намъ никакого не было помощника, благоволеніемъ Отца Своего небеснаго, вступился за насъ, якоже Самъ чрезъ пророка глаголетъ: \textit{и воззрѣхъ, и не бѣ помощника, и помыслихъ, и никтоже заступи: и избави я мышца Моя, и ярость Моя наста}\footnote{Ис.~63,~5.}. Сей милостивѣйшій нашъ заступникъ, Господь крѣпокъ и силенъ, Господь силенъ въ брани, вступилъ въ подвигъ за насъ, и побѣдилъ врага мучителя нашего; и якоже Давидъ Голіаѳа не оружіемъ, но пращею поразилъ: тако Хрістосъ поразилъ діавола врага нашего не оружіемъ, но силою Креста Своего, страданіемъ, терпѣніемъ, и сокрушилъ враждебную его главу, и поверглъ его на землю, яко мертвеца; и который нами ругался и попиралъ насъ, того отдалъ въ попраніе намъ, рабомъ Своимъ, и извелъ насъ на свободу. О семъ заступникѣ нашемъ, смерти и ада побѣдителѣ, Іисусѣ, дерзаемъ и глаголемъ: аще Онъ за насъ, кто на насъ? Его силою укрѣплени, наступаемъ на змію и скорпію и на всю силу вражію, и торжествуемъ, восклицая: \textit{гдѣ ти, смерте, жало? гдѣ ти, аде, побѣда?.. Богу благодареніе, давшему намъ побѣду Господемъ нашимъ Іисусъ Хрістомъ}\footnote{1~Кор.~15,~55--57.}! "--- Что же намъ, свободившимся отъ такого мучителя, дѣлать, хрістіане? "--- Отвѣтъ: 1)~Отъ чиста сердца такъ великому нашему благодѣтелю благодарить, Который свободилъ насъ отъ горькаго плѣненія такого врага, отъ котораго силою нашею никакъ не могли мы свободитися, и вѣчно бы были въ темной власти, аще не бы руку помощи подалъ намъ милостивый Іисусъ, Сынъ Божій. О семъ буди Ему слава со Отцемъ Его и Святымъ Духомъ! 2)~Во всегдашней памяти содержать великое сіе Его дѣло, и пѣть Ему побѣдную пѣснь, яко побѣдителю супостата нашего. \textit{Помощникъ и покровитель бысть мнѣ во спасеніе: Сей мой Богъ, и прославлю Его}\footnote{Исх.~15,~2.}. Врагъ нашъ палъ: мы же востали и исправилися. Онъ постыдился: мы же прославилися. Онъ, плѣнившій насъ, плѣненъ: мы плѣненныи свободилися. Онъ вознесшійся смирился: мы смирившіяся вознеслися; \textit{яко во смиреніи нашемъ помянулъ насъ Господь}. 3)~Отъ чистаго сердца любить такъ великаго и высокаго благодѣтеля, Котораго нѣтъ и не можетъ быть большій. Аще бо человѣка благодѣтеля, который малое какое намъ добро сдѣлалъ, любимъ: кольми паче Хріста, Который такъ насъ обдолжилъ, что ничимъ и никогда воздать Ему не можемъ, любить должно. 4)~Усердное послушаніе показывать Ему и Святыя Его заповѣди соблюдать, да не оскорбимъ Его. Отсюду бо и любовь наша къ Нему познается, якоже Самъ глаголетъ: \textit{имѣяй заповѣди Моя и соблюдаяй ихъ, той есть любяй Мя}\footnote{Іоан.~14,~21.}. 5)~Берещися, чтобы паки не попасться въ плѣненіе злокозненному врагу, да не како вѣчными плѣнниками его сдѣлаемся. На свободу насъ вывелъ Хрістосъ: потщимся сохранить ее.

\section{7. Господинъ и рабъ его купленный.}

Видимъ, что рабъ тому господину работаетъ, который его у другаго купилъ: Хрістосъ Сынъ Божій искупилъ насъ Себѣ отъ діавола и темной его власти; и искупилъ не сребромъ и златомъ или иною какою тлѣнною цѣною, но Своею Кровію; Крове ради насъ Своея не пощадѣлъ, пострадать и умереть не отреклся, чтобы насъ искупить и себѣ присвоить. Ибо сатана лестно завладѣлъ"=было нами и поработилъ себѣ; но Хрістосъ Господь нашъ вступился за насъ, и избавилъ насъ отъ горькой работы его, и присвоилъ Себѣ, и сотворилъ насъ рабами Своими: и мы на крещеніи святомъ, когда вступали въ хрістіанство, отреклись врага онаго, и всея гордыни его, и всѣхъ злыхъ дѣлъ его; и обѣщались Искупителю нашему Іисусу работать. \textit{Нѣсмы убо свои: куплени бо есмы цѣною}\footnote{1~Кор.~6,~20.}. "--- Какая цѣна дана за насъ? сребро или злато? Нѣтъ! Что же? Кровь честная изліяна! Чія она? простаго ли человѣка? Никакъ! Ктожъ Онъ такой, Который такъ милосердно съ нами поступилъ, что крове Своея, ради насъ порабощенныхъ врагу и мучителю, не пощадѣлъ? Господь и Богъ нашъ есть! \textit{Онъ стяжалъ насъ Кровію Своею}\footnote{Дѣян.~2,~20.}. О воистину великая цѣна, безцѣнная цѣна дана за насъ! Воистину нѣсмы свои, но купленніи раби Хрістовы и Хрісту убо, яко Господу нашему, купившему насъ, работать должны, волю Его творить, и заповѣди Его исполнять. \textit{Хрістосъ за вся умре, да живущіи не ктому себѣ живутъ, но умершему за нихъ и воскресшему}\footnote{2~Кор.~5,~15.}. Господинъ повелѣваетъ рабу, и слушаетъ его рабъ; глаголетъ ему: дѣлай тое, и дѣлаетъ; иди туды, и идетъ; набирай на столъ, и набираетъ. Хрістосъ Господь нашъ повелѣваетъ намъ: хрістіанине, люби ближняго, какъ себе; буди милосердъ, смиренъ, кротокъ, податливъ; берегись всякаго грѣха, и проч. Должно убо и намъ, яко рабамъ Его купленнымъ, дѣлать тое, что Онъ повелѣваетъ намъ; а паче, что мы и обѣщались своею волею служить Ему, и служеніе тое намъ, а не Ему полезно. Почтилъ Онъ насъ такъ дорого, что и Себе ради насъ не пощадѣлъ. \textit{Господи! что есть человѣкъ, яко помниши его? или сынъ человѣчь, яко посѣщаеши его?} Почтимъ и мы Его, такъ почетшаго насъ; поработаемъ Ему со всякимъ смиренномудріемъ, яко Господу нашему, да и Онъ насъ познаетъ и признаетъ за рабовъ Своихъ въ день втораго Своего пришествія, и речетъ каждому изъ насъ: \textit{благій рабе и вѣрный! вниди въ радость Господа твоего}\footnote{Мѳ.~25,~21.}.

\section{8. Плѣнникъ и свободитель.}

Аще бы кто въ плѣненіи былъ, и тамо всякое мученіе отъ плѣнившаго мучителя терпѣлъ, и ничего не ожидалъ бы отъ него, кромѣ горькія смерти; а пришелъ бы къ нему какой добрый и сильный человѣкъ, и сказалъ бы ему: «вотъ я тебе свобождаю отъ горькія сей работы и всего бѣдствія, и приведу тя въ отечество твое, только иди за мною»: о, коль съ великою радостію восхотѣлъ бы бѣдный той плѣнникъ итить и послѣдовать милостивому тому свободителю! "--- Хрістосъ, Сынъ Божій, всесильный и преблагій, къ намъ плѣненнымъ пришелъ съ небеси, отъ Отца своего небеснаго посланный, и глаголетъ намъ: «Я къ вамъ отъ Отца Моего небеснаго посланъ свободить васъ, и привести васъ къ Нему въ Небесное Его царствіе. \textit{Азъ есмь путь и истина и животъ: никтоже пріидетъ ко Отцу, токмо Мною»}\footnote{Іоан.~14,~6.}. Аще убо хощете въ вѣчное оное блаженство пріити, и небесное царство наслѣдовать: идите за Мною, и приведу васъ туда. \textit{Аще же кто хощетъ по Мнѣ ити, да отвержется себе, и возметъ крестъ свой, и по Мнѣ грядетъ}\footnote{Матѳ.~16,~24.}. "--- Пойдемъ, хрістіане, за Хрістомъ, Избавителемъ нашимъ, Который къ намъ пришелъ, и милостивно посѣтилъ насъ, и зоветъ насъ въ небесное Свое царствіе; и приведетъ насъ во оный блаженнѣйшій покой и присносущную радость, идѣже есть всѣхъ веселящихся жилище. Истинный и вѣрный есть вождь Іисусъ, и безъ Него никто въ царствіе Божіе не внидетъ: поручимъ убо и мы себе Ему въ предводительство, и неотлучно послѣдуемъ Ему, да приведетъ и насъ въ мѣсто вѣчнаго упокоенія. Хрісту послѣдуютъ избранніи Его не ногами, но сердцемъ и вѣрою, любовію, смиреніемъ, терпѣніемъ и кротостію. Возлюбимъ и мы сей Хрістовъ путь, да послѣдуемъ Ему. Низкій, смиренный и отъ многихъ презрѣнный сей путь; но единъ въ высокое отечество, небо, ведетъ\footnote{7,~14.}. Нѣтъ бо инаго пути къ вѣчному животу, кромѣ пути крестнаго: \textit{иже не пріиметъ креста своего, и въ слѣдъ Мене грядетъ, нѣсть Мене достоинъ}, глаголетъ Господь\footnote{10,~38.}. Аще же Хріста не достоинъ: то кого достоинъ будетъ, развѣ противника Его? Какъ ни думай, человѣче, и куды ни обращай мыслей своихъ, надобно нести крестъ свой и ити за Хрістомъ, то"=есть, послѣдовать Его смиренію, любви, терпѣнію и кротости, то"=есть, что ни приключится тебѣ скорбное, терпѣть безъ роптанія, ради того, что Хрістосъ, Избавитель твой, тако терпѣлъ. Претерпи убо все безъ роптанія, что ни послетъ тебѣ святая Божія десница; испій чашу, какую подастъ тебѣ небесный Отецъ, да будеши истинный удъ духовнаго тѣла Хрістова, и тако будеши послѣдовати главѣ твоей, Хрісту. И какъ съ Нимъ страждеши, тако съ Нимъ и прославишися, яко удъ съ главою твоею, тамже писано есть: \textit{съ Нимъ страждемъ, да и съ Нимъ прославимся}\footnote{Римл.~8,~17.}. И паки: \textit{аще кто Мнѣ служитъ, Мнѣ да послѣдствуетъ: и идѣже есмь Азъ, ту и слуга Мой будетъ}\footnote{Іоан.~12,~26.}. Убо \textit{потерпи Господа, мужайся, и да крѣпится сердце твое, и потерпи Господа}\footnote{Псал.~26,~14.}.

\section{9. Преступники и радостная имъ вѣсть.}

Аще бы какіе люди впали въ знатное предъ царемъ согрѣшеніе, и отъ него бы посланы были въ заточеніе или въ ссылку, и пришелъ бы какой вѣстникъ отъ царя къ нимъ, и объявилъ бы имъ, что царь ихъ прощаетъ, и преступленіе ихъ оставляетъ имъ, и въ отечество и домъ ихъ паки возвращаетъ, и въ мирѣ и покоѣ даетъ имъ жить: о, коль радостна была бы вѣсть бѣднымъ тѣмъ осужденникамъ! Мы съ праотцомъ нашимъ Адамомъ вси предъ Богомъ, Царемъ небеснымъ, согрѣшили, и изъ рая въ міръ сей, аки въ заточеніе и ссылку, посланы. Хрістосъ, Сынъ Божій, великаго совѣта Ангелъ, пришелъ къ намъ изгнаннымъ и осужденнымъ, и принеслъ пресладкую отъ небеснаго Своего Отца вѣсть, и объявилъ намъ, что Богъ прощаетъ насъ, и паки въ милость Свою пріемлетъ. \textit{И пришедъ благовѣсти миръ вамъ дальнимъ и ближнимъ}, глаголетъ Апостолъ\footnote{Еф.~2,~17.}. И Самъ Хрістосъ о Себѣ благовѣствуетъ намъ, \textit{благовѣстити нищимъ посла Мя} (Отецъ), \textit{исцѣлити сокрушенныя сердцемъ, проповѣдати плѣненнымъ отпущеніе и слѣпымъ прозрѣніе, отпустити сокрушенныя во отраду, проповѣдати лѣто Господне пріятно}\footnote{Лук.~4,~8--19.}. Какъ бы сказалъ: не бойтеся, ни ужасайтеся, бѣдніи и отверженніи человѣцы! Я вамъ добрую отъ Отца Моего небеснаго вѣсть принеслъ. Согрѣшили вы: Онъ вамъ грѣхи прощаетъ. Отвратились вы отъ Него: Онъ васъ къ Себѣ паки обращаетъ. Удалились и изгнались вы отъ лица Его: Онъ васъ паки къ Себѣ возвращаетъ. Погибли вы: Онъ васъ чрезъ Мене спасаетъ. Выгнались вы изъ рая чрезъ грѣхъ свой: Онъ вамъ благодатію Своею, вмѣсто рая, небо подаетъ. Пріимите убо Мене, яко посланника Его. \textit{Да не смущается сердце ваше! Вѣруйте въ Бога, и въ Мя вѣруйте. Въ дому Отца Моего обители многи суть}\footnote{Іоан.~14,~1--2.}. Въ тыя обители васъ, изгнанныхъ, призываю и возвращаю. "--- О вѣсти, всему міру дражайшія, сладчайшія и благопріятнѣйшія! Ничего воистину пріятнѣе не можетъ быть намъ, бѣднымъ грѣшникомъ, паче вѣсти сея! Не такъ алчущимъ пища, жаждущимъ холодная вода, плѣненнымъ свобода, сѣдящимъ во тьмѣ свѣтъ, утружденнымъ покой, немощнымъ здравіе, какъ намъ, бѣднымъ, удаленнымъ отъ Бога, отверженнымъ, осужденнымъ и погибшимъ грѣшникамъ, вѣсть сія пріятна. Не бойся, не отчаявайся, бѣдный грѣшниче! Хрістосъ Сынъ Бога живаго, Царь и Царя небеснаго Сынъ возлюбленный и единородный, отъ Отца Своего небеснаго посланный, пришелъ тебе спасти, и въ небесное Свое царствіе привести. \textit{Пріиде Сынъ человѣческій взыскати и спасти погибшаго}\footnote{Лук.~19,~10.}. \textit{Вѣрно слово и всякаго пріятія достойно, яко Хрістосъ Іисусъ пріиде въ міръ грѣшники спасти}\footnote{1~Тим.~1,~15.}. \textit{Тако бо возлюби Богъ міръ, яко и Сына Своего единороднаго далъ есть, да всякъ, вѣруяй въ Онь, не погибнетъ, но имать животъ вѣчный. Не посла бо Богъ Сына Своего въ міръ, да судитъ мірови, но да спасется Имъ міръ}\footnote{Іоан.~3,~16--17.}. Благодаримъ Тя, человѣколюбче, единородный Сыне и Слове Божій, что Ты намъ такъ сладкую и утѣшительную вѣсть отъ Отца Своего небеснаго принеслъ, и тою сердца наша, ядомъ древняго змія огорченная, усладилъ! Вѣруемъ въ Тя, Сына Бога живаго, и пославшаго Тя небеснаго Твоего Отца; и Тобою, Иже еси путь, истина и животъ, къ Нему пріити надѣемся, и славити Тя на вѣки безконечныя со Отцемъ Твоимъ и Святымъ Духомъ! \textit{Буди} убо, \textit{Господи, милость Твоя на насъ, якоже уповахомъ на Тя}\footnote{Пс.~32,~22.}! На Тя, Господи, уповахомъ, да не постыдимся во вѣки, аминь! "--- Сею пресладкою вѣстію, или евангеліемъ Господа нашего Іисуса Хріста утѣшайся, печальная душа, согрѣшившая и кающаяся! Сія и въ житіи твоемъ, во время зноя духовнаго да будетъ прохлажденіе, и въ часъ смертнаго подвига живое утѣшеніе. Пріиде Сынъ Божій грѣшники спасти, не такіе и такіе, но всякіе, какіе бы они ни были; только бы покаялися, и вѣровали въ пришедшаго грѣшники спасти Іисуса Хріста: и рай имъ отверзется рукою на крестѣ распятаго Хріста. "--- Покайся убо и ты, грѣшниче, и вѣруй въ святое и сладчайшее евангеліе сіе, и безъ сумнѣнія внидеши съ разбойникомъ въ рай. \textit{Благословенъ Господь Богъ Израилевъ, яко посѣти и сотвори избавленіе людемъ Своимъ и воздвиже рогъ спасенія намъ въ дому Давида отрока Своего: якоже глагола усты святыхъ сущихъ отъ вѣка пророкъ Его}\footnote{Лук.~1,~69--70.}.

\section{10. Бѣдствующіи люди, и царь ихъ посѣтитель и имъ состраждущій.}

Аще бы къ людемъ, въ плѣненіи находящимся и всякое злостраданіе терпящимъ и ничего болѣе, только горькой смерти, ожидающимъ, самъ царь ихъ пришелъ къ нимъ и посѣтилъ ихъ, и съ ними бѣдствовать и для ихъ страдать восхотѣлъ, чтобы ихъ отъ горькой той работы избавить и въ свое отечество возвратить: такъ бы сіе велико и дивно дѣло было, что всѣ бы вѣки тое съ великимъ удивленіемъ славили. И воистину не можетъ большая любовь и милосердіе и снисходительство быть человѣку подданному отъ царя своего! Но царь, хотя и высокое лице, однакожъ такой же человѣкъ смертный, какъ и подданный его, такожде земля и пепелъ есть, такомужде тлѣнію подлежитъ, какъ и прочіи человѣцы. То чудно, и всякое удивленіе превосходитъ, что Богъ, Творецъ неба и земли, великій и непостижимый, къ намъ, бѣднымъ, пришелъ и посѣтилъ насъ, и пришелъ въ нашемъ образѣ къ намъ. Духъ невещественный, плотію одушевленною и намъ подобострастною одѣятися, прежде вѣкъ отъ Отца, Богъ отъ Бога, Свѣтъ отъ Свѣта нетлѣнно возсіявый, отъ Дѣвы неискусобрачныя и Пресвятыя родитися, и Ее Матерію называти, и Ея Сынъ называтися "--- о чудо чудесъ! Богъ рабу Свою Матерію Своею нарицати, и Сынъ Божій Сынъ Дѣвы нарицатися благоизволилъ, "--- и тако безначальный начатися, невидимый видѣтися, неосязаемый осязатися, неприступный херувимамъ и серафимамъ, грѣшникамъ приступнымъ быть; Царь небесный на землю снити; Богъ, намъ, человѣкомъ, уподобился, съ нами на земли пожити, и тако съ нами и насъ ради смиритися; всемогущій немощнымъ, всесильный безсильнымъ быти. И что чуднѣе того: Господь отъ рабовъ Своихъ злыхъ хуленіе, безчестіе, поношеніе, вязаніе, наруганіе, посмѣяніе, укореніе, судъ и осужденіе, оплеваніе, заушеніе, оболганіе, оклеветаніе, біеніе, раны, на смерть веденіе, ко кресту пригвожденіе, и со беззаконными вмѣненіе, кротко и волею претерпѣти восхотѣлъ! "--- Ради кого? "--- Ради тебе и мене, сущихъ враговъ Своихъ!... Сіе такъ великое и чудное дѣло, что тое издалеча провидя Пророкъ, со ужасомъ воскликнулъ: \textit{Господи! услышахъ слухъ Твой и убояхся. Господи! разумѣхъ дѣла Твоя, и ужасохся}\footnote{Аввак.~3,~1--2.}. "--- Удивляйся жъ и ты, хрістіанине, великому сему Господа твоего дѣлу, котораго нѣтъ и не можетъ быть больше; пой и прославляй человѣколюбіе Его, котораго и перомъ описать, и умомъ понять не возможно. Здѣ воистину съ пророкомъ воскликнути можемъ: \textit{Господи! что есть человѣкъ, яко познался еси ему, или сынъ человѣчь, яко вмѣняеши его}\footnote{Пс.~43,~3.}? Ради мене сошелъ еси съ небесъ, да мене на небо возведеши; ради мене смирился еси, да мене вознесеши; ради мене, Богъ сый, вочеловѣчился еси, да мене Божественнаго естества Твоего причастникомъ сотвориши\footnote{1~Петр.~1,~4.}. Ради мене плотію родился еси, дабы мнѣ духомъ родитися; ради мене гоненіе претерпѣлъ еси, да мене, изгнаннаго изъ рая, паки въ рай введеши. Ради мене странствовати изволилъ еси, да мене плѣненнаго въ отечество возвратиши. Ради мене похуленъ былъ еси, да врагу моему, клеветнику діаволу, заградиши уста. Ради мене связанъ былъ еси, да мене отъ узъ грѣховныхъ разрѣшиши. Ради мене тужилъ, скорбѣлъ, печалился и плакалъ, да мене отъ скорби, туги, печали и слезъ вѣчныхъ избавиши. Ради мене обезчещенъ и поруганъ былъ еси, да мене прославиши. Ради мене со беззаконными вмѣненъ былъ еси, да мене оправдаеши. Ради мене, животе мой, смерть вкусилъ еси, да мене оживиши. Удивляюсь человѣколюбію Твоему! пою милосердіе Твое! покланяюся глубочайшему смиренію и снисхожденію Твоему! Что воздамъ Господеви о всѣхъ, яже воздаде ми, нищій и убогій рабъ Твой? "--- Представляй, хрістіанине, предъ умныя очеса великое сіе дѣло часто, и всегда будешь видѣти чудо, большее паче всѣхъ чудесъ. Хощеши на всякъ день видѣти чудо: на всякъ день о воплощеніи и страданіи Хрістовомъ размышляй. А отъ сего душеспасительнаго размышленія послѣдуетъ: 1)~\textit{вѣра} живая. Аще бо Хрістосъ за всѣхъ пострадалъ и умеръ, то и за мене и за тебе. Аще Онъ всѣхъ Спаситель есть, истинно въ Него вѣрующихъ, то мой и твой есть Спаситель. Симъ размышленіемъ возжигается свѣтильникъ вѣры. 2)~Послѣдуетъ \textit{надежда}. Аще бо Онъ толикая за тебе пострадалъ, то оставитъ ли тебе въ нуждѣ твоей? Аще Онъ умеръ за тебе, то отречется ли помощи тебѣ въ бѣдствіи твоемъ? Можетъ и хощетъ помощи тебѣ: можетъ, понеже премудръ и всесиленъ; хощетъ, понеже благъ и любитель твой есть. Отъ благости бо и любви Своей за тебе пострадалъ. Аще же медлитъ, ждетъ терпѣнія твоего: потерпи убо и ожидай Его. Непремѣнно пріидетъ и не закоснитъ: вѣренъ бо есть въ словесѣхъ Своихъ. 3)~Послѣдуетъ \textit{любовь къ Нему}. Онъ мене и тебе, врага сущаго, возлюбилъ такъ, что и умеръ за насъ: нѣтъ болѣе сея любви! Какъ убо намъ такого любителя не любить, и любовь за любовь взаимно не воздавать? Человѣка, малое какое намъ добро сдѣлавшаго, любимъ: Хріста ли, Сына Божія, такъ высокаго и великаго благодѣтеля, не любить? 4)~Послѣдуетъ \textit{послушаніе}. Ибо, чтобъ любителю любимаго не оскорбить, неотмѣнно должно ему послушаніе показывать. Отъ послушанія бо и любовь показуется, какъ Онъ Самъ научаетъ: \textit{имѣяй заповѣди Моя, и соблюдаяй ихъ, той есть любяй Мя}\footnote{Іоан.~14,~21.}. 5)~Послѣдуетъ \textit{благодарность}. Ибо Онъ не златомъ и сребромъ и не иною какою цѣною, но честною Своею кровію искупилъ тебе отъ смерти вѣчныя: Что разсуждая, неотмѣнно убѣдишися благодарить Его отъ чистаго сердца, и отъ благодарности угождать Ему, яко своему великому благодѣтелю. Нѣтъ и не можетъ быть большаго паче сего благодѣянія. Сіе Хрістово благодѣяніе, намъ недостойнымъ показанное, такъ велико есть, что за него никакъ и ничимъ и никогда воздать не возможно, такъ что хотя бы ты сто или два ста лѣтъ жилъ, и на всякій день жесточайшее мученіе страдалъ и терпѣлъ ради имене и чести Хрістовой, то ничто бы было противу Его благодѣянія. Понеже я и ты и вси люди ничто противъ такъ высокаго и великаго лица Хрістова. Ибо мы "--- человѣцы есмы, земля и пепелъ: Онъ Господь и Богъ нашъ есть, Который толикое къ намъ показалъ снисхожденіе. Требуется убо и отъ насъ всеусердное за то благодареніе. Чимъ бо большее благодѣяніе, тѣмъ большая за то должна быть благодарность. 6)~Послѣдуетъ \textit{страхъ}. Яко кому Хрістово страданіе не будетъ въ пользу и спасеніе вѣчное, тому въ большее будетъ осужденіе, а паче хрістіанамъ, познавшимъ Хріста, но не слушающимъ Хріста. Бойся убо грѣшить, да не лишишися благодати Божія, да не и самая кровь Хрістова изліянная, но отъ тебе пренебреженная, возопіетъ на тя къ Богу.

7)~Послѣдуетъ \textit{утѣшеніе} согрѣшившему и кающемуся. Аще бо Хрістосъ за тебе пострадалъ и умеръ, то како согрѣшившаго и кающагося не пріиметъ? Тяжко ли Ему отпустить тебѣ грѣхъ твой, тебѣ, съ сокрушеніемъ и жалѣніемъ сердца къ Нему приходящему и просящему прощенія, Которому не тяжко было за тебе пострадать и умереть? Съ радостію отпуститъ Онъ тебѣ грѣхи твои кающемуся, Который за грѣхи твоя умеръ. 8)~Отсюду познавай, хрістіанине, благородіе, честь, достоинство и преимущество души человѣческія. Ради избавленія и искупленія ея, не ангелъ, не иной какой ходатай отъ Бога посланный, но Самъ Богъ и Господь пришелъ; Самъ ее Своимъ пришествіемъ почтилъ и посѣтилъ; и искупилъ ее отъ діавола, смерти, ада и прочихъ враговъ, не сребромъ или златомъ, или иною какою цѣною тлѣнною, но честною кровію Своею. Почтилъ Онъ насъ въ созданіи нашемъ, когда насъ по образу Своему и по подобію сотворилъ; но болѣе почтилъ, когда къ намъ, падшимъ и погибшимъ, Самъ во образѣ нашемъ пришелъ, и пострадалъ и умеръ за насъ. Такъ дорого душу человѣческую постановилъ Господь, что Самъ ради взысканія ея пришелъ, и ради стяжанія ея кровь Свою изліялъ. Чимъ дражайшая вещь, тѣмъ большая цѣна за нее дается; а какъ нѣтъ и не можетъ быть ничего дороже паче Хріста и крове Его: то отсюду и драгость души человѣческія познается, что ради ея Себе и крове Своея Хрістосъ Сынъ Божій не пощадѣлъ. Познавай убо, хрістіанине, драгость души твоея, и не прельщайся прелестными міра сего чинами и титулами, которыми сынове вѣка сего прельщаются и обезумляются. Честь, слава и достоинство души твоея большее есть паче славы всего міра: душа твоя есть невѣста, Хрісту Сыну Божію на вѣки обрученная и святою кровію Его искупленная. Разумѣй убо ея благородіе. Благородный женихъ, благородная и невѣста, Ему сопряженная. Не видишь сего благородія и достоинства нынѣ; но увидишь тогда, когда открыются чада Божія. \textit{Не у явися, что будемъ: вѣмы же, яко егда явится, подобни Ему будемъ, и узримъ Его, якоже есть}\footnote{1~Іоан.~3,~2.}. Все убо благородіе, достоинство и преимущество міра сего, во едино собранное, есть какъ гной предъ честію и благородіемъ души человѣческія. 9)~Отсюду послѣдуетъ \textit{смиреніе}. "--- Какъ? "--- Вотъ какимъ образомъ: аще бы царь земный пришелъ къ тебѣ, подлому человѣку, и посѣтилъ тебе, и пришелъ бы, не позванъ отъ тебе, но самъ своею волею: не ужаснулся ли бы ты, и, падши предъ нимъ, не сказалъ ли бы сего: «кто я такой, что мене подлаго и недостойнаго великій государь мой посѣтилъ?» "--- Къ намъ, бѣднымъ и подлымъ рабамъ Своимъ, не земный, но небесный Царь, Царь царствующихъ и Господь господствующихъ, пришелъ и посѣтилъ, не по нашему умоленію, но по своему милостивому обѣщанію; и не только посѣтилъ, но и страданіе и смерть претерпѣлъ за насъ. Смиряйся же, хрістіанине, толикое видя Царя твоего, Бога и Господа твоего къ тебѣ снисхожденіе, покланяйся, падай предъ Нимъ и говори къ Нему со страхомъ и смиреніемъ: «кто я такой, земля и пепелъ сый, грѣшникъ и рабъ неключимый, что ко мнѣ Господь мой и Богъ мой пришелъ? Но тако безприкладной Его благости изволися! Слава благости Его, слава человѣколюбію и безмѣрнымъ щедротамъ!» 10)~Видимъ, какъ смирился для насъ Господь нашъ: смиримся и мы, хрістіанине, ради Его и ради себе. Ради Его: яко Ему ничто такъ не пріятно и не любезно, какъ смиреніе наше. Ради себе: яко намъ ничто такъ не полезно, какъ смиреніе; ибо \textit{гордымъ Богъ противится, смиреннымъ же даетъ благодать}\footnote{1~Петр.~5,~5.}. Не возгнушался Онъ нами подлыми и грѣшными рабами: не возгнушаемся и мы подобною себѣ братіею, хотя и низшею отъ насъ и подлѣйшею, взирая на живый той смиренія образъ; то"=есть, господа съ рабами своими, высокіи съ низкими, богатіи съ нищими, славніи съ подлыми, мудріи съ неразумными, ученіи съ неучеными, яко съ братіею своею да обходятся, и да никто другаго не презираетъ. Сего бо сродное естество наше требуетъ. Не презрѣлъ Онъ насъ, подлыхъ и уничиженныхъ: не презримъ и мы, подлыи подлыхъ и уничиженныи уничиженныхъ. Вси бо до единаго человѣка, по естеству и природѣ нашей, нищіи, убогіи, бѣдніи и окаянніи есмы и отъ князя и вельможи и до простаго мужика. Почто жъ убо бѣдному бѣднаго презирать, когда Господь и Царь славы насъ, подлыхъ и бѣдныхъ, не презрѣлъ? 11)~Понеже мы Хрісту, за Его высочайшее къ намъ благодѣяніе, не можемъ здѣ возблагодарить достойно, и Онъ ничего не хочетъ такъ, какъ нашего спасенія (на сіе бо пришелъ въ міръ и пострадалъ и умеръ): то потщимся благодатію Его спастися, и тако угодить Ему, давъ будущей жизни, чрезъ всю некончаемую вѣчность, будемъ Ему пѣть со Отцемъ Его и Святымъ Духомъ пѣснь благодаренія, хвалы и славословія. Аминь.

\section{11. Осужденникъ и казнь его смертная.}

Бываетъ, что человѣкъ по заслугамъ своимъ и по законамъ осуждается на смертную казнь; и ведутъ его за градъ, и за нимъ послѣдуетъ множество народа, и тако на мѣстѣ опредѣленномъ пріемлетъ казнь. Отъ сего позорища обрати, хрістіанине, умъ твой и размышленіе къ страшному и ужасному спасительнаго Хрістова страданія позорищу. Тако Хрістосъ Сынъ Божій сужденъ и осужденъ былъ на смерть; но неправедно и неповинно веденъ былъ на смерть, яко овча на заколеніе, и \textit{идяше во слѣдъ Его народъ многъ людей}\footnote{Лук.~23,~27.}. И внѣ града изведенъ бысть на мѣсто смерти, и тамо между двумя злодѣями крестную и смертную казнь претерпѣлъ. Ужасное и умилительное позорище! Сынъ Божій и Господь славы, яко злодѣй, сужденъ и веденъ былъ на смерть, и смертную казнь принялъ! Отъ кого?.. Отъ злыхъ рабовъ Своихъ! За кого?.. За непотребныхъ рабовъ Своихъ! Раби согрѣшили, и наказаніе претерпѣлъ Владыка! \textit{Той язвенъ бысть за грѣхи наша, и мученъ бысть за беззаконія наша}\footnote{Ис.~53,~5.}. Гдѣ?.. Внѣ вратъ Іерусалимскихъ! Предъ кимъ?.. Предъ очами Божіими, ангельскими и человѣческими. "--- Видишь, хрістіанине, страшное и ужасное позорище: смотри жъ и разсуждай причину того! Причина тому грѣхи наши. Тако мои и твои и всего міра грѣхи очищались. Тако правдѣ Божіей удовлетворялось, которая грѣхами міра раздражена была! Тако Богъ міру примирялся, и міръ Богу! Тако оправданіе намъ, грѣшнымъ, заслуживалось, и спасеніе вѣчное содѣловалось! Тако отъ плѣненія и власти діавольскія, адскаго мученія и вѣчныя смерти избавлялися мы! Тако восходъ на небо уготовлялся намъ, и небесное царствіе, грѣхами нашими затворенное, отверзалося, и все, что во Адамѣ мы потеряли, возвращалось! Во Адамѣ мы діаволу и власти его темной подверглися: но чрезъ Хрістово страданіе отъ того избавилися. Во Адамѣ согрѣшили мы и клятвѣ подвержены сотворилися: но чрезъ Хрістово страданіе, оправданіе и благословеніе Божіе возвратилося намъ. Во Адамѣ мы умерли и погибли: но Хрістовымъ страданіемъ ожили и спаслися. Во Адамѣ мы отъ Бога удалились: но Хрістовымъ страданіемъ къ Богу привелися. Во Адамѣ мы изъ рая выгнались: но Хрістовымъ страданіемъ въ рай не земный, но небесный возвратилися. Во Адамѣ обезчестилися и посрамилися мы: но чрезъ Хрістово страданіе почтилися и прославилися. Во Адамѣ всего добра лишилися мы, и всякому бѣдствію и окаянству подпали: но Хрістовымъ страданіемъ всякое добро получили, и отъ всякаго злополучія избавилися. И тако отъ Хрістова страданія, какъ отъ источника спасенія, все наше благополучіе проистекаетъ. Его страданіе нашей отрады виновно; Его смерть нашего живота виновна; Его судомъ и осужденіемъ мы отъ суда и осужденія вѣчнаго избавилися; Его ранами мы исцѣлѣли; Его узы, которыми Онъ связанъ былъ, насъ отъ узъ грѣховныхъ разрѣшили, за что поемъ Ему со пророкомъ: \textit{растерзалъ еси узы моя. Тебѣ пожру жертву хвалы}\footnote{Пс.~115,~7--8.}. Отъ Его печали, скорби и туги миръ и радость вѣчная намъ проистекла; Его милосердными слезами отираются слезы очей нашихъ, и дѣйствительны предъ Отцемъ небеснымъ бываютъ; Божественнаго Его тѣла обнаженіе ризу намъ спасенія и одежду веселія содѣлало; смиреніемъ и уничиженіемъ Его мы вознеслись; отъ поруганія и безчестія Его вѣчная честь и слава послѣдовала намъ. Словомъ: все блаженства духовное и вѣчное отъ сего спасительнаго Источника проистекло намъ. О семъ Ему, начальнику жизни нашея, со Отцемъ и Святымъ Духомъ, возсылаемъ вѣчное благодареніе. "--- Поминай убо всегда, хрістіанине, оное время, въ которое сіе наше блаженство устроялося; поминай оное время, въ которое Богъ въ образѣ человѣческомъ и рабіемъ зракѣ насъ ради на земли явился и пожилъ, между грѣшниками обращался, съ грѣшниками ялъ, пилъ и бесѣдовалъ; оное время, въ которое хуленіе и гоненіе отъ злыхъ рабовъ Своихъ претерпѣлъ; оное время, въ которое грѣхъ ради нашихъ скорбѣлъ, тужилъ, ужасался и кровавымъ потомъ обливался, "--- въ которое отъ льстиваго и неблагодарнаго ученика проданъ, и преданъ и отъ прочихъ учениковъ оставленъ былъ, "--- въ которое связанъ, веденъ на судъ, сужденъ и осужденъ былъ на смерть, поруганъ, оплеванъ и заушенъ былъ, "--- въ которое вопили на Него: \textit{возми, возми, распни Его}, "--- въ которое со беззаконными вмѣнился, и спасенія ради нашего на крестѣ умеръ; и тщись Искупителю твоему вѣрою, любовію, смиреніемъ, кротостію и терпѣніемъ послѣдовать и угодить, да и съ Нимъ вѣчно будешь царствовати.

\section{12. Царь, ожидаемый отъ гражданъ, и пришествіе его во градъ, и его отъ гражданъ пріятіе.}

Царь, хотячи въ какій градъ пріити, посылаетъ напередъ въ тотъ градъ вѣстника объявить гражданамъ имѣющее быть пришествіе его. Почему граждане ожидаютъ его, и къ срѣтенію и пріятію его пріуготовляются, и пришедшаго съ подобающею честію срѣтаютъ и пріемлютъ. Тако Царь небесный, имѣя пріити въ міръ, яко преславный градъ, руками Его сотворенный, послалъ напередъ вѣстниковъ, рабовъ Своихъ, пророковъ, дабы возвѣстили міру имѣющее быть пришествіе Его въ міръ, которое многократно и многоразлично они и предвозвѣстили; и вси ветхозаконніи вѣрніи и святіи съ великимъ желаніемъ ожидали того. \textit{Мнози пророцы и праведницы вожделѣша видѣти и слышати} Хріста Сына Божія во плоти\footnote{Мѳ.~13,~17.}. И когда время пришло, пришелъ Царь небесный въ міръ, Іисусъ Хрістосъ, и срѣтили Его и приняли Его вси, которыи возлюбили Божественное явленіе Его. Пожилъ Царь небесный на земли, и содѣлалъ спасеніе посредѣ земли, ради чего и пришелъ; и паки возшелъ на небо, и сѣлъ на престолъ славы Своея. Проповѣдали Его спасительное тое смотрѣніе раби Его, святіи апостоли во всѣхъ концахъ земли, и спасаются вси, который ни пріемлютъ спасительную ихъ проповѣдь. И такъ пророки и апостоли единаго Спасителя міру, Сына Божія проповѣдали намъ. Пророки, имѣющаго въ міръ пріити, "--- апостоли, уже въ міръ пришедшаго и спасеніе содѣлавшаго проповѣдали. Ветхозаконніи вѣрніи въ грядущаго Хріста вѣровали, и спаслись: новыя благодати вѣрніи въ пришедшаго Хріста вѣруютъ, и спасаются. И тако единъ Хрістосъ, и въ Ветхомъ и Новомъ Завѣтѣ, вѣрныхъ есть Спаситель. Намъ, хрістіанамъ, ничего болѣе нынѣ дѣлать не должно, какъ угождать Господу и Спасителю нашему Іисусу, небесному Царю, и о полученіи вѣчнаго спасенія, трудами и болѣзньми Его содѣланнаго, пещися и ожидать втораго Его съ небесе пришествія. Было первое Его пришествіе, будетъ и второе. Первое пришествіе проповѣдывали пророки, глаголя: грядетъ Хрістосъ въ міръ, грядетъ Хрістосъ въ міръ. И пришелъ Хрістосъ въ міръ тихо и нечаянно: \textit{сниде яко дождь на руно, и яко капля каплющая на землю}\footnote{Пс.~71,~6.}. О второмъ пришествіи пастыри и учители церковніи глаголютъ: пріидетъ Хрістосъ въ міръ, пріидетъ Хрістосъ въ міръ, паки явится нечаянно, но страшно и славно. Якоже убо граждане ожидаютъ пришествія царя своего и къ срѣтенію его пріуготовляются: тако и намъ, хрістіанине, пришествія Царя нашего небеснаго ожидать и къ достойному Его срѣтенію пріуготовляться должно; а паче, что не знаемъ дне и часа, въ который Онъ пріидетъ, и пріидетъ не жити, учити и страдати (было уже тое), но судити и воздати всякому по дѣломъ его. Потщимся убо, хрістіанине, бдѣть и готовыми быть къ срѣтенію Царя нашего, поминая увѣщательное слово Его: \textit{бдите, яко не вѣсте дне, ни часа, въ оньже Сынъ человѣческій пріидетъ}\footnote{Мѳ.~25,~13.}, "--- да сподобимся одесную Его въ пришествіи Его стати.

\section{13. Подлый человѣкъ, высокому лицу усыновленный.}

Бываетъ, что высокое лице пріемлетъ подлаго человѣка въ сына себѣ, совлекаетъ съ него рубища, омываетъ, облекаетъ въ одежду, чести своей пристойную, и тако дѣлаетъ его сыномъ и наслѣдникомъ своимъ. Тако Богъ, небесный Царь, поступаетъ съ бѣднымъ и подлымъ человѣкомъ. Сіе бываетъ и совершается на святомъ крещеніи. Не видитъ сего человѣкъ тѣлесными очами; но сіе таинство вѣрою постигается. Тогда бѣднаго человѣка Богъ въ высочайшую Свою милость пріемлетъ, не поминаетъ его прежнихъ грѣховъ и беззаконій, оставляетъ ему всякое законопреступленіе; праведный гнѣвъ, который на него, яко чадо гнѣва, имѣлъ "--- отлагаетъ; совлекаетъ съ него смрадное грѣховное рубище, омываетъ водою чистою, одѣваетъ прекрасною оправданія Хрістова порфирою, дѣлаетъ его сыномъ Своимъ и наслѣдникомъ вѣчнаго небеснаго царствія, якоже ко крещенному іерей глаголетъ изъ Божіяго слова: \textit{оправдился еси, просвѣтился еси, освятился еси, омылся еси, именемъ Господа нашего Іисуса Хріста, и Духомъ Бога нашего}\footnote{1~Кор.~6,~11.}. О, неизреченнаго Божія человѣколюбія къ намъ, бѣдніи и отверженніи человѣки! О непостижимой благости! Человѣка "--- преступника и отступника Своего, врага Своего, въ такъ великую милость и высочайшее достоинство, котораго выше быть не можетъ, пріемлетъ Богъ! Сей великой и непостижимой любви достойно удивляется Апостолъ святый: \textit{видите, какову любовь далъ есть Отецъ намъ, да чада Божія наречемся, и есмы: сего ради міръ не знаетъ насъ, зане не позна Его. Возлюбленніи! нынѣ чада Божія есмы, и не у явися, что будемъ; вѣмы же, яко, егда явится, подобни Ему будемъ, и узримъ Его, якоже есть}\footnote{1~Іоан.~3,~1--2.}. "--- Намъ, хрістіанине, не иное что остается дѣлать, какъ слѣдующее: 1)~За сію высочайшую Божію къ намъ милость и человѣколюбіе всегда отъ чиста сердца благодарить Ему. 2)~Любить Его, яко чадомъ отца своего. 3)~Сыновнее послушаніе и угожденіе нелицемѣрно показывать Ему. 4)~Отъ всякаго грѣха, которымъ Онъ оскорбляется, берещись, да не оскорбимъ Его. 5)~Подражать Ему, яко отцу чада подражаютъ, по увѣщанію апостольскому: \textit{бывайте подражатели Богу, яко чада возлюбленная}\footnote{Еф.~5,~1.}. Принялъ Онъ насъ въ высочайшую Свою милость: да помнимъ и мы сію Его милость. Сотворилъ Онъ насъ чадами Своими: да будемъ и мы истинніи, а не лицемѣрніи чада Его, то"=есть, да творимъ дѣла, чадамъ Божіимъ пристойная. Святъ Онъ: да подражаемъ и мы святости Его, творяще святыню въ страсѣ Божіи. Благъ Онъ и милосердъ: \textit{да будемъ и мы другъ ко другу блази, милосерди, прощающе другъ другу, якоже и Богъ во Хрістѣ простилъ есть намъ}\footnote{Еф.~4,~32.}. \textit{Сіяетъ Онъ солнце} Свое \textit{на злыя и благія, и дождитъ на праведныя и на неправедныя}\footnote{Мѳ.~5,~45.}: да творимъ и мы добро всѣмъ, знаемымъ и незнаемымъ, своимъ и чужимъ, единовѣрнымъ и иновѣрнымъ, другамъ и врагамъ, добротворящимъ и злотворящимъ намъ. Долго терпитъ Онъ согрѣшенія всего міра, ожидая всѣхъ на покаяніе: да будемъ и мы терпѣливы къ согрѣшающимъ намъ. Сотворимъ сія вся и прочее, не въ нашу, но въ Его славу, по увѣщанію Спасителя: \textit{тако да просвѣтится свѣтъ вашъ предъ человѣки, яко да видятъ ваша добрая дѣла и прославятъ Отца вашего, Иже на небесѣхъ}\footnote{5,~16.}.

\subsection{О томжде.}

Аще бы помянутый подлый человѣкъ, отъ своего благодѣтеля почтенный, вдался въ безчестныя дѣла, и тако чести и приличнаго одѣянія совлеклся бы, и одѣлся бы паки въ подлое и срамное рубище: всѣмъ людямъ дѣло сіе удивленія, сожалѣнія и смѣха достойно было бы. И воистину, не малое бы оскорбленіе было почетшему его благодѣтелю! И люди, иніи бы сожалѣли, иніи бы удивлялись, иніи бы смѣялись безумію онаго глупаго человѣка. Сіе приключается хрістіанину, который въ такъ высокую милость Божію принятъ, и отъ Бога почтенъ во святомъ крещеніи; хрістіанину, говорю, такому, который въ беззаконныя дѣла вдается, законъ Божій безстрашно нарушаетъ; такому приличествуетъ истинная притча: \textit{песъ возвращся на свою блевотину, и свинія омывшися въ калъ тинный}\footnote{2~Петр.~2,~22.}. Совлекается и онъ прекрасныя и священныя одежды, которою душа одѣяна была; лишается высочайшія усыновленія Божія чести и наслѣдія вѣчнаго; облекается въ безчестное и срамное грѣховное рубище, и бываетъ подлѣйшимъ рабомъ; дѣлается печаль и плачъ ангеламъ святымъ и всѣмъ избраннымъ Божіимъ жителямъ небеснымъ, котораго блаженству прежде радовалися вси они; дѣлается смѣхъ и поруганіе и радость злымъ духамъ, которыи прежде ему завидовали. Въ толикое и толь бѣдное состояніе впадаетъ всякъ хрістіанинъ, по святомъ крещеніи беззаконнующій! Паки въ тоежде бѣдствіе и страшнѣйшее состояніе попадается таковая душа, какъ и прежде крещенія была, паче же горшее. О таковыхъ Апостолъ написалъ: \textit{быша имъ послѣдняя горша первыхъ. Лучше бо бѣ имъ не познати пути правды, нежели познавшимъ возвратитися вспять отъ преданныя имъ святыя заповѣди}\footnote{2,~20--21.}. Не видитъ, къ бѣдствію своему, таковый бѣдный человѣкъ какъ блаженства, котораго лишается, такъ и бѣдствія, въ которое падаетъ, потому что и тое и другое не на тѣлѣ, но внутрь, въ душѣ имѣется. Истинное бо какъ блаженство, такъ и бѣдствіе хрістіанское едино, то"=есть, которое въ душѣ находится. О, когда бы увидѣлъ человѣкъ, какого онъ блаженства по крещеніи чрезъ грѣхъ лишается, и въ коликое бѣдствіе впадаетъ непрестанно бы и неутѣшно бы рыдалъ и плакалъ! Увидятъ тое бѣдніи грѣшники тогда, когда откроется слава сыновъ Божіихъ, и ихъ самихъ явится безобразіе; увидятъ, но къ большему своему бѣдствію. Тогда они восплачутся и возрыдаютъ, но уже безполезно. "--- Сего же бѣдственнаго по крещеніи состоянія причиною бываетъ: 1)~Немощь и растлѣніе, которое внутрь, въ сердцѣ человѣческомъ, крыется и съ нимъ раждается. 2)~Недоброе воспитаніе. Сего же причиною бываютъ родители, которыи какъ о себѣ, такъ и о дѣтяхъ своихъ нерадятъ, и тако сами идутъ и ихъ ведутъ въ погибель. 3)~Пастырское нерадѣніе, которыи о врученномъ себѣ стадѣ Хрістовомъ небрегутъ. 4)~Соблазны недобрыхъ родителей и прочіихъ злыхъ людей, на которыхъ смотря, юные соблазняются и развращаются. 5)~Козни общаго христіанамъ врага, діавола, который, \textit{какъ левъ рыкая, ходитъ, искій кого поглотити}\footnote{1~Петр.~5,~8.}. 6)~Самого человѣка о себѣ небреженіе и нерадѣніе, который, слушая въ храмѣхъ чтомое Божіе слово, небрежетъ и нерадитъ о томъ, и идетъ, какъ слѣпый въ яму, въ вѣчную погибель. А отсюду слѣдуетъ: 7)~Какъ нужно есть доброе юнымъ дѣтямъ воспитаніе и наставленіе! Не по"=французски, не по"=нѣмецки, не по"=италіански говорить, не танцовать, о чемъ нынѣшніе люди наипаче тщатся, но по"=хрістіански жить, обдолжаются закономъ Божіимъ учить дѣтей своихъ хрістіане. Хотя никакого языка не будутъ знать дѣти, но будутъ по"=хрістіански жить: истиннаго блаженства не лишатся.

8)~Какъ тщательно юнымъ должно берещися отъ соблазновъ, которыми зло, въ сердцѣ ихъ крыющееся, возбуждается! 9)~Грѣшнику, совратившемуся по крещеніи съ пути истиннаго едина остается надежда "--- истинное покаяніе. Сего ради обратися, грѣшная душа, къ небесному Отцу, какъ блудный сынъ сотворилъ, и пади предъ милосердными Его очами, и съ сокрушеніемъ и стенаніемъ сердечнымъ воззови къ Нему: \textit{Отче! согрѣшихъ на небо и предъ Тобою: и уже нѣсмь достоинъ нарещися сынъ Твой; сотвори мя, яко единаго отъ наемникъ Твоихъ}\footnote{Лук.~15,~18--19.}. И безъ сумнѣнія вѣруй, что Онъ пріиметъ тебе, и въ первую одежду облечетъ тебе, и тако и о тебѣ радость будетъ предъ ангелами Божіими.

\section{14. Благодѣтель, и отъ него получающій благодѣяніе къ нему неблагодаренъ.}

Аще бы кто у нѣкоего господина добраго и милостиваго въ дому жилъ, и отъ него пищу, питаніе, одѣяніе, защищеніе и прочая благая получалъ, но ему, какъ благодѣтелю своему, не токмо достойныя благодарности не показывалъ бы, но и всякую бы грубость, непочтеніе и досажденіе являлъ, поносилъ бы ему и ругалъ бы его: сожалѣнія бы достойно тое было, и вси бы на такъ безумнаго человѣка негодовали. И подлинно достоинъ бы таковый былъ всякаго негодованія и поношенія! Великое бо зло, безуміе и слѣпота есть неблагодарность, и всякому человѣку мерзко есть. Въ семъ порокѣ находятся дѣти, которыя родителей своихъ не почитаютъ; люди, которые пастыря своего поносятъ и злословятъ; ученики, которые учителямъ своимъ досажденіе показываютъ; подвластные, которые властямъ, о общей пользѣ пекущимся достойныя чести и любви не показываютъ; нищіи, которые подающихъ имъ милостыню не любятъ, и не помнятъ ихъ благодѣянія, и проч. Такую, или паче несравненно горшую, хрістіане беззаконнующіе показываютъ Богу неблагодарность. Ибо человѣкъ человѣку какое ни дѣлаетъ добро, не свое, но Божіе дѣлаетъ добро. Богъ бо всего, какое на свѣтѣ ни есть, добра источникъ и вина. Но Богъ Своимъ истымъ добромъ снабдѣваетъ насъ. Какого добра и благодѣянія не показалъ намъ Богъ, о хрістіане! Создалъ насъ, и создалъ насъ не скотами, но человѣками, разумомъ одаренными; создалъ особливѣйшимъ Своимъ совѣтомъ: \textit{сотворимъ человѣка}; создалъ по образу Своему и по подобію. Какая, честь можетъ быть больше человѣку, какъ быть по образу Божію! За сіе едино ничимъ и никогда не можемъ Богу возблагодарить. Но когда согрѣшили мы, пали и погибли: не оставилъ насъ и тогда Создатель нашъ преблагій въ погибели нашей. И какого посредствія не изобрѣлъ, чтобы насъ возставити и привести къ Себѣ! Послалъ къ намъ пророковъ Своихъ, которые отступившихъ насъ отъ Него обращали къ Нему; далъ намъ слово Свое святое, какъ посланіе, въ которомъ открылъ и объявилъ волю Свою святую. И понеже мы такъ поражены и уязвлены были отъ лютаго разбойника, врага нашего, діавола, что ничимъ не могли исцѣлитися: то умилосердился надъ нами, бѣдными, милосердый Создатель нашъ, такъ что и Сына Своего единороднаго не пощадѣлъ ради насъ, но послалъ Его къ намъ, взыскати заблудшихъ насъ, исцѣлити уязвленныхъ и сокрушенныхъ, спасти погибшихъ. О чемъ Самъ Сынъ Божій единородный благовѣствуетъ намъ: \textit{тако возлюби Богъ міръ, яко и Сына Своего единороднаго далъ есть, да всякъ вѣруяй въ Онь не погибнетъ, но имать животъ вѣчный. Не посла бо Богъ Сына Своего въ міръ, да судитъ мірови, но да спасется Имъ міръ}\footnote{Іоан.~3,~16--17.}. Который, живучи или паче странствуя на земли, чего ни дѣлалъ! какихъ благодѣяній ни показывалъ! чего ради насъ ни пострадалъ! какихъ поношеній, хуленій, какого озлобленія отъ неблагодарныхъ людей Своихъ ни претерпѣлъ! Наконецъ крестною смертію нашего ради спасенія умеръ! Сіе же все сотворилъ преблагій Владыка нашъ благоволеніемъ Отца Своего небеснаго и Своимъ вольнымъ хотѣніемъ. Такъ чудный показавъ о насъ промыслъ, многомилостивый Богъ нашъ позвалъ насъ къ вѣрѣ Своей святой и омылъ насъ, освятилъ и оправдалъ именемъ Господа нашего Іисуса Хріста и Духомъ Своимъ Святымъ, якоже Апостолъ глаголетъ хрістіанамъ: \textit{но омыстеся, но освятистеся, но оправдистеся именемъ Господа нашего Іисуса Хріста и Духомъ Бога нашего}\footnote{1~Кор. 6--11.}. Сія и прочая благодѣянія Божія къ душѣ нашей и вѣчному животу надлежитъ. "--- Какое тѣлу нашему и временному житію показуетъ добро, весь міръ о томъ свидѣтельствуетъ. Обрати, человѣче, умъ и очи твои ко всему созданію Его, и разсуди: кому оно служитъ? не намъ ли? Кому солнце, луна и звѣзды? не намъ ли? Кому воздухъ и облака? не намъ ли? Кому земля съ плодами? не намъ ли? Кому скоты, звѣри и птицы? не намъ ли? Кому вода съ рыбами и прочими живущими въ ней? не намъ ли, о хрістіане? Намъ вся тварь повелѣніемъ Божіимъ служитъ, яко безъ той и малѣйшаго времени не можемъ жить. Кто безъ хлѣба и воды, кто безъ одѣянія прожить можетъ? Воздухъ такъ нуженъ намъ, что и минуты безъ него прожить не можемъ. Какое бы житіе наше было, ежели бы Богъ отнялъ свѣтъ отъ насъ? не всѣ ли бы блудили, какъ слѣпіи? Кто бы могъ цѣлъ быти отъ врага невидимаго діавола, когда бы Богъ всемогущею рукою Своею не хранилъ насъ отъ врага, который презѣльною злобою ярится на родъ нашъ, \textit{ходитъ, яко левъ рыкая, искій кого поглотити}\footnote{1~Петр. 5--8.}. Но почто много исчислять? Надобно признаться, что нѣтъ числа Божіихъ благодѣяній, которыхъ всякъ отъ насъ сподобляется. Въ любви и благодѣяніяхъ Божіихъ заключаемся вси. На что ни посмотришь, къ чему ни обратишь умъ твой: все тое показуетъ любовь Божію къ намъ и благодѣяніе. Самая геенна, объявленная намъ отъ Него, благодѣтельствуетъ намъ: ибо устрашаетъ насъ, да каемся и плачемся предъ Господемъ, сотворшимъ насъ, да помилуетъ насъ и избавитъ насъ отъ нея. Самаго діавола попущаетъ Создатель нашъ въ пользу нашу; и столько попущаетъ, сколько пользуетъ намъ; и столько сей врагъ искушаетъ насъ, сколько полезно намъ. Какъ пользуетъ? Возбуждаетъ насъ отъ лѣности, побуждаетъ къ молитвѣ, да, чувствуя присутствіе сего врага, прибѣгаемъ къ Богу и отъ Него помощи и защищенія просимъ. И якоже воинъ, чимъ частѣе на брани бываетъ, тѣмъ искуснѣйшій дѣлается; тако хрістіанинъ, чимъ болѣе искушеній отъ сего супостата мужественно претерпитъ, тѣмъ лучшій и искуснѣйшій въ званіи хрістіанскомъ бываетъ; а чимъ болѣе побѣдъ надъ симъ врагомъ одержитъ, тѣмъ краснѣйшаго вѣнца сподобится отъ Подвигоположника Іисуса Хріста. «Попеченіе имѣя», глаголетъ Златоустъ, «о нашемъ спасеніи Богъ, оставилъ діавола, дабы насъ отъ лѣности возбуждалъ, и случай къ полученію вѣнца намъ пріуготовлялъ»\footnote{На Быт.~23,~2.}. И Павелъ святый глаголетъ: \textit{дадеся ми пакостникъ плоти, аггелъ сатанинъ, да ми пакости дѣетъ, да не превозношуся}\footnote{2~Кор.~12,~7.}. Видишь, какъ пользуетъ намъ враждебный духъ; видишь, какъ гордый къ смиренію приводитъ? Пакоститъ намъ, да не превозносимся. И тако злый и злобный духъ содѣйствуетъ нашему добру намѣреніемъ недобрымъ. Но когда хрістіанинъ беззаконнуетъ: вся сія благодѣянія Божія забываетъ, и Бога, Благодѣтеля своего, не почитаетъ, и такъ весьма неблагодаренъ къ Нему бываетъ, и подобнымъ тому дѣлается, который благодѣтеля своего, человѣка поноситъ и ругаетъ и доказуемое себѣ благодѣяніе ни во что вмѣняетъ. Не почитаетъ же и презираетъ христіанинъ Бога потому, что Его не хощетъ слушать. Страшно сіе слово, о хрістіане, но подлинно истина есть! Какъ? "--- Слушай и внимай. Богъ повелѣваетъ: не клянися именемъ Моимъ всуе; но беззаконный хрістіанинъ клянется. Богъ повелѣваетъ: почитай родителей твоихъ и властей; но беззаконный хрістіанинъ не почитаетъ. Богъ повелѣваетъ: не гнѣвайся, не злобися на ближняго твоего, ни ненавиди его; но беззаконный хрістіанинъ держитъ гнѣвъ, памятозлобствуетъ и ненавидитъ ближняго своего, по образу Божію сотвореннаго. Богъ повелѣваетъ: не любодѣйствуй и не прелюбодѣйствуй; но беззаконный хрістіанинъ или прелюбодѣйствуетъ, или любодѣйствуетъ, и тако оскверняетъ свою и другаго душу и тѣло, оскверняетъ, говорю, душу, по образу Божію созданную. Богъ повелѣваетъ: не укради, не похищай чуждаго; но беззаконный хрістіанинъ крадетъ и похищаетъ. Богъ повелѣваетъ: не лжесвидѣтельствуй, не злослови, не ругай ближняго твоего, не лги, не обманывай, не осуждай и не оклеветай; но беззаконный хрістіанинъ не смотритъ на то Божіе повелѣніе, лжесвидѣтельствуетъ, лжетъ, обманываетъ, поноситъ, ругаетъ, судитъ и осуждаетъ подобнаго себѣ человѣка, и когда бы не лучшаго. Видишь, какъ не слушаетъ и презираетъ беззаконникъ Создателя своего! Какое почитаніе отцу отъ сына и господину отъ раба, когда сынъ отца и рабъ господина не слушаетъ? не явное ли презрѣніе и уничиженіе? О, колико на свѣтѣ презирателей и уничижителей Божіихъ! колико Божіихъ враговъ и отъ тѣхъ, который мнятся Его почитать! О, въ какое, какъ въ бѣдное состояніе пришло хрістіанство! Хрістіанинъ, толико милостей и благодѣяній отъ Бога сподобившійся и сподобляющійся, врагомъ Богу, высочайшему своему Благодѣтелю, дѣлается, "--- тотъ, который знаетъ, что какъ ради всѣхъ, такъ и ради его Сынъ Божій въ міръ пришелъ, пострадалъ и умеръ, "--- тотъ, который позванъ словомъ Божіимъ къ вѣчному животу, "--- тотъ, который банею крещенія отъ Бога омытъ, освященъ, оправданъ. О, како заразилъ бѣднаго человѣка душегубецъ діаволъ! Какъ самъ есть богопротивникъ, такимъ же и человѣка сдѣлалъ. Видишь, хрістіанине, къ чему грѣхъ твой тебе приводитъ! Врагомъ Божіимъ, который долженъ быть боголюбецъ и богочтецъ, чрезъ грѣхъ дѣлаешися. Сладокъ онъ тебѣ кажется; но горьки плоды его! Не видишь сего нынѣ; но тогда увидишь, когда вся дѣла и помышленія человѣческая открыются. Тяжко тебѣ, человѣку, терпѣть, когда подобный тебѣ человѣкъ, малое нѣкое благодѣяніе отъ тебе получившій, не показуетъ тебѣ благодарности: кольми паче Богу, всѣхъ благихъ Источнику, тяжка есть неблагодарность. Откуду видимъ, что вездѣ Богъ жалится на неблагодарность: \textit{и забыша благодѣянія Его и чудеса Его}\footnote{Пс.~77,~11.}. И паки: \textit{слыши небо, и внуши земле, яко Господь возглагола: сыны родихъ и возвысихъ, тіи же отвергошася Мене}\footnote{Ис.~1,~2.}. И паки: \textit{согрѣшиша, не Того чада порочная: роде строптивый и развращенный! сія ли Господеви воздаете}\footnote{Второз.~32,~5--6.}? И на прочіихъ Писанія мѣстахъ. "--- Очувствуйся убо, хрістіанине, и, признавъ свою неблагодарность и согрѣшенія, пади со смиреніемъ предъ милосердными Божіими очами, и изъ глубины сердца взывай Ему: согрѣшихъ, Господи, помилуй мя! Пріими мя, заблуждшую овцу и сопричти избранному Твоему стаду! Дай мнѣ сердце, сердцевѣдче Боже, Тебе почитающее, Тебе боящееся, Тебе любящее, волѣ Твоей послѣдующее! Настави мя на путь твой, и пойду во истинѣ Твоей! Призри на мя и помилуй мя, по суду любящихъ имя Твое! "--- И когда, обратившися, начнеши каятися, и новое хрістіанское житіе проходити: не повредитъ тебѣ прежнее твое беззаконное житіе, и тогда самъ ты узнаешь, каковъ ты предъ очесами Божіими былъ. \textit{Егда возвратився воздохнеши: тогда спасешися, и уразумѣеши, гдѣ еси былъ}\footnote{Лук.~15,~24.}. Уразумѣеши, какъ въ бѣдственномъ состояніи былъ ты, какъ далеко ты блудилъ отъ пути истины; уразумѣеши, что ты имени хрістіанскаго недостоинъ былъ, который называлъ себе хрістіаниномъ, что мертвъ былъ подлинно, который мечталъ себе живымъ. Всякъ бо живущій грѣху, мертвъ есть Богу, какъ живущій Богу, есть мертвъ грѣху. Тогда сбудется и о тебѣ реченное: \textit{яко сынъ мой сей мертвъ бѣ, и оживе; изгиблъ бѣ, и обрѣтеся}\footnote{Ис.~30,~15.}. И тогда радость будетъ на небеси о тебѣ предъ ангелами Божіими. О, да сбудется дѣло сіе! Господи Боже силъ! обрати насъ и просвѣти лице Твое, и спасемся.

\subsection{О томжде.}

Когда неблагодарный своему благодѣтелю человѣкъ пріидетъ въ разсужденіе и раскаяніе о своемъ неблагодарствѣ, начнетъ мыслити и говорити тако въ себѣ: «что это я окаянный дѣлаю? то ли мнѣ должно воздавать своему благодѣтелю? ненависть ли за любовь, и зло за добро? воистину обезумился я!» И такъ жалѣетъ и сокрушается о беззаконныхъ своихъ поступкахъ, и стыдится смотрѣть на своего благодѣтеля; падаетъ предъ нимъ со смиреніемъ и проситъ отъ него прощенія, не бояся чего отъ него, но только жалѣя о томъ, что котораго долженъ былъ любить и почитать, какъ своего любителя, того поносилъ, ругалъ и оскорблялъ. Сіе есть истинное раскаяніе! "--- Видишь, хрістіанине, что человѣкъ человѣку, благодѣтелю своему, дѣлаетъ, когда покажетъ ему неблагодарность и въ раскаяніе пріидетъ! Тако хрістіанинъ, когда пріидетъ въ разсужденіе о Божіихъ благодѣяніяхъ, толико какъ прочіимъ, такъ и ему показанныхъ и показуемыхъ, и о Богѣ, Благодѣтелѣ своемъ, отъ него забвенномъ и презрѣнномъ, весьма сокрушается и снѣдается; и печалію по Бозѣ, какъ стрѣлою, уязвляется, и на себе самого, какъ врага, гнѣвается, судитъ себе недостойнаго ни неба, ни земли, укруха хлѣба и одѣянія рубищнаго; словомъ, ничего себе достойнымъ не почитаетъ, но только единаго наказанія достойнымъ. Видитъ бо, что Его, какъ Создателя своего и Бога долженъ былъ почитать, но не почиталъ; долженъ былъ слушать, но не слушалъ; долженъ былъ любить, но не любилъ; долженъ былъ смиряться предъ Нимъ, но не смирялся; долженъ былъ покоряться Ему, но не покорялся: и тако оскорблялъ Того, Котораго въ любви и благодѣяніяхъ заключенъ; Того, Который есть едина любовь и благостыня; Того, у Котораго въ руцѣ вси концы земли, и онъ самъ; Которому ангели со страхомъ и любовію покланяются, почитаютъ и поютъ. Сія есть печаль по Бозѣ, которая покаяніе нераскаянное во спасеніе содѣловаетъ! Таковою печалію сокрушенное сердце не можетъ не испущать слезъ. Пожелаетъ и онъ со пророкомъ источника слезъ. Таковому тяжко уже впредь согрѣшать. Лучше онъ пожелаетъ умереть, нежели согрѣшить. Сіе есть истинное хрістіанское покаяніе, которое происходитъ отъ размышленія о презрѣнномъ Бозѣ благодѣтелѣ и благодѣяніяхъ Его! О, когда бы всякъ хрістіанинъ разсуждалъ, какое ему Богъ добро дѣлалъ и дѣлаетъ, и какъ тяжко грѣхомъ оскорбляетъ Его, Того, отъ Кого всякое благодѣяніе получаетъ, и взиралъ бы на Божіе величество и свое недостоинство: никогда бы не захотѣлъ согрѣшить! Но то бѣда, что закрытыя душевныя очи имѣетъ, и тако не видитъ, что самъ дѣлаетъ. "--- \textit{Востани спяй, и воскресни отъ мертвыхъ, и освѣтитъ тя Хрістосъ}\footnote{Еф.~5,~14.}. Пробудись, о бѣдная душа! Се сатана хощетъ тебе вѣчнаго сокровища на вѣки лишить.

\section{15. Царь, и подданный его, отъ него просящій милости.}

Когда человѣкъ предъ царемъ своимъ стоитъ и отъ него проститъ милости, со смиреніемъ и благоговѣніемъ на него взираетъ, колѣна преклоняетъ и падаетъ предъ нимъ, весь умъ и мысли тутъ собраны имѣетъ, о томъ тщится и старается, какъ бы царя на милость преклонить и желаемое получить. Видишь, хрістіанине, какъ человѣкъ у человѣка милости проситъ! какое вниманіе смиреніе и благоговѣніе показуетъ, желая просимое получить! "--- Отъ сего примѣра учимся мы, како намъ должно въ молитвѣ стоять предъ Господемъ Богомъ и Царемъ нашимъ, со умиленіемъ взирать на пресвятое лице Его, колѣна со смиреніемъ преклонять и падать предъ Его величествомъ и желаемое просить. О человѣче! помяни въ молитвѣ твоей, что ты предъ Богомъ стоишь, Богу бесѣдуешь, Богу говоришь: Господи, помилуй! Господи, ущедри! Господи, услыши! Господи, спаси! и проч. И псалмы ли читаешь, или пѣсни церковныя поешь, Бога поешь, къ Богу простираешь рѣчь твою: Ты, Господи! Тебѣ, Господи! и прочее. "--- Кто ты? земля и пепелъ и грѣшникъ. Предъ кѣмъ стоишь и бесѣдуешь? предъ Богомъ святымъ и страшнымъ, предъ Которымъ вся созданія, видимая и невидимая, какъ ничто. Ты малый и бѣдный червячокъ, ктомуже и грѣшникъ, Богу вѣчному, непостижимо"=всемогущему, предстоишь и молишися. Самъ убо разсуди, какое здѣ смиреніе и благоговѣинство потребно тебѣ! Царю земному со смиреніемъ предстоимъ и просимъ Его: кольми паче Царю небесному! "--- Отсюду послѣдуетъ: 1)~Чтеніе молитвъ со скоростію и какъ можетъ языкъ исправиться, ничего не пользуетъ. 2)~Чтомые стихи и поемыя пѣсни вдругъ шумъ только одинъ дѣлаютъ; а читающимъ и поющимъ не токмо не пользуютъ, но и въ грѣхъ обращаются. 3)~Отсюду бываетъ, что попы, клирики и прочіи люди, тако читающіи и поющіи, не токмо не исправляются, но и въ горшее успѣваютъ. Ибо никогда не молятся, хотя и часто въ церковь ходятъ и молятся. Молитва бо есть всѣхъ благихъ ходатаица намъ; и потому безъ молитвы невозможно исправить себе, жить по"=хрістіански и добро творить. Всякаго бо добра надобно у Бога просить. Безъ Бога добрыми быть не можемъ, зли суще. 4)~Понеже молитва всякихъ благихъ виною намъ бываетъ; а сатана, врагъ нашъ, вѣдая великую сію пользу, отъ молитвы намъ происходящую, всякимъ образомъ препятствуетъ намъ, то мысли о мірскихъ вещахъ предлагая, то злыми помышленіями скучая намъ, то уныніе влагая: то и благочестивымъ надобно въ молитвѣ осторожными быть, врагу противиться, мыслямъ не допущать разсѣяваться, и Богу единому внимать, дабы какъ тѣломъ, такъ и духомъ предъ Богомъ стояли, и какъ тѣломъ предъ Нимъ падаемъ, такъ и духомъ предъ Нимъ падали, и что языкъ говоритъ, о томъ бы умъ и сердце не молчало; словомъ, чтобы внутренность молящагося съ наружностію согласна была.

\subsection{О томжде.}

Къ доброму и милостивому человѣку, хотя и высокаго званія будетъ, удобно вси приступаемъ. Благость бо его и милость всякаго къ себѣ влечетъ, и надежда милости его удобный приступъ къ нему содѣловаетъ. Всякъ бо въ себѣ тако помышляетъ о немъ: хотя онъ великъ и высокъ есть по своему сану; однако всѣмъ бѣднымъ, приступающимъ къ нему и просящимъ милости у него, удобно приступенъ есть и отверзаетъ утробы милосердія. "--- Тако, и наипаче, да разсуждаемъ о Господѣ Бозѣ нашемъ, хрістіанине! Хотя Онъ и великъ есть и страшенъ, и величество Его непостижимо есть: но сколько великъ есть, столько благъ и милостивъ; и якоже величество Его, тако и милость Его. \textit{Щедръ и милостивъ Господь, долготерпѣливъ и многомилостивъ; благъ Господь всяческимъ, и щедроты Его на всѣхъ дѣлѣхъ Его}\footnote{Пс.~144,~8--9.}. "--- И все святое слово Его ничто такъ намъ не проповѣдуетъ, какъ благость и милость Его, къ надеждѣ и утѣшенію нашему, да не усомнѣваемся къ Нему приступать и толкать въ двери милосердія Его. Да и Самъ Онъ велитъ намъ приступать къ Себѣ съ молитвою и прошеніемъ въ нуждахъ нашихъ. \textit{Просите, и дастся вамъ; ищите и обрящете; толцыте, и отверзется вамъ. Всякъ бо просяй пріемлетъ, и ищай обрѣтаетъ, и толкущему отверзется}\footnote{Мѳ.~7,~7--8.}. Что утѣшительнѣе можетъ быть намъ, бѣднымъ грѣшникамъ, паче сего словесе? Къ какому убо человѣку, такъ доброму, такъ милостивому, такъ щедрому и кроткому, такъ удобенъ приступъ бываетъ, какъ къ Богу, Который и въ сердцѣ человѣческомъ доброту, милость, щедроты и кротость насадилъ? У какого отца дѣти такъ удобно и скоро испрашиваютъ желаемое, какъ у Бога сынове человѣчестіи? Богъ нашъ не только смотритъ на прошеніе и исполняетъ, но и желаніе слушаетъ убогихъ людей Своихъ. \textit{Желаніе убогихъ услышалъ еси, Господи, уготованію сердца ихъ внятъ ухо Твое}\footnote{Пс.~9,~38.}. "--- Милосердія двери и приступъ къ Богу отворилъ намъ сладчайшій Спаситель нашъ Іисусъ предрагими Своими заслугами. Да не устрашаетъ убо насъ величество Его приступать къ Нему съ молитвою, но да ободряетъ благость и человѣколюбіе. \textit{Да приступаемъ убо съ дерзновеніемъ къ престолу благодати, яко да пріимемъ милость, и благодать обрящемъ, во благовременну помощь}\footnote{Евр.~4,~16.}.

\subsection{О томжде.}

Къ царю, или высокому какому лицу, не на всякое время и не на всякомъ мѣстѣ съ прошеніемъ приступать можно, какъ видимъ. Къ Богу нашему, хрістіанине, не тако. Къ Нему вездѣ и всегда удобенъ приступъ бываетъ. Вездѣ и всегда готовъ есть слушати нашу молитву. Нощь и день, вечеръ и утро и всякое время и часъ удобенъ есть къ молитвѣ. Такожде на всякомъ мѣстѣ удобно къ Нему приступать можно. Ибо вездѣ и всегда со всѣми нами обрѣтается. Можемъ убо къ Нему приступать съ прошеніемъ нашимъ въ церкви, въ дому, въ собраніи, на пути, на дѣлѣ, на ложи, ходя и сидя, трудяся и почивая, и во всякомъ времени. Ибо вездѣ и всегда можемъ умъ и сердце къ нему возводить и прошеніе сердца Ему предлагать, и съ колѣнопреклоненіемъ сердечнымъ падать предъ милосердными Его очами, и милости отъ Него ожидать. Къ Богу бо не ногами, но сердцами приступаемъ. "--- Приступай убо, хрістіанине, вездѣ и всегда къ милосердному Создателю твоему, исповѣдуй и призирай нищету твою, бѣдность и окаянство, и получиши богатство благости отъ Него.

\section{16. Лоза и розги.}

Что лоза и розги между собою суть: тое Хрістосъ и хрістіане суть между собою. Розги съ лозою суть соединены: тое хрістіане со Хрістомъ суть духовно соединены. Розги отъ лозы сокъ пріемлютъ, и плодъ творятъ: тако хрістіане отъ Хріста силу живительную добронравія и добродѣтелей пріемлютъ, и плоды добрыхъ дѣлъ раждаютъ. На розгахъ хотя и видится плодъ, лозѣ приписуется; тое хотя хрістіане и приносятъ плодъ добродѣтелей, однакожъ Хрісту Сыну Божію приписуются онѣ. Розги сами въ себѣ безъ лозы не могутъ плода творить: тако хрістіане безъ Христа ничего не могутъ творить. Розги отребляются и очищаются отъ дѣлателя, чтобы лучшій и большій плодъ приносили: тако хрістіане отъ небеснаго Отца наказуются, да множайшій и пріятнѣйшій плодъ добродѣтелей принесутъ. Розги внѣ не красны, но внутрь добрый и пріятный сокъ и плодъ содержатъ: тако хрістіане внѣ не красны, презрѣнны, но внутрь добры; не красно говорятъ, но красно живутъ. Розги, чимъ болѣе плодами обременяются, тѣмъ болѣе къ землѣ приклоняются и низпущаются: тако хрістіане, чимъ болѣе добрыхъ дѣлъ творятъ, тѣмъ болѣе смиряются. Розги плодъ творятъ ради дѣлателя: тако хрістіане добрыя дѣла творятъ по слову небеснаго Отца, отъ Котораго всякое добро происходитъ. Розга, не творящая плода, отсѣкается отъ лозы: тако хрістіанинъ, который не творитъ плода добра, отторгается отъ Хріста. Розга, отсѣченная отъ лозы, ссыхаетъ: тако хрістіанинъ, отторженный отъ Хріста, всю духовную свою живность погубляетъ и духовно изсыхаетъ. Розга иссохшая ни къ чему иному не угодна, какъ къ сожженію: тако хрістіанинъ, отторгнувшійся отъ Хріста и изсохшій, вѣчному предается огню\footnote{Іоан.~15,~4--6.}. "--- Отсюду видишь, хрістіанине: 1)~Коль тѣсный союзъ и общеніе истинныхъ хрістіанъ со Хрістомъ! Сей есть лоза: тіи же суть розги. 2)~Коль великое и высокое достоинство есть! Что бо преславнѣе, какъ со Хрістомъ Царемъ небеснымъ имѣть общеніе? Преславно съ царемъ земнымъ или княземъ имѣть общеніе; но несравненно славнѣе со Хрістомъ, Царемъ царствующихъ. 3)~Коль великое ихъ блаженство! Аще бо Хрістосъ съ хрістіанами: кто противу ихъ? Весь свѣтъ и весь адъ ничего не можетъ противу хрістіанина: ибо Хрістосъ его крѣпость и сила. 4)~Безъ Хріста быть добродѣтельны и добрыя дѣла творить не можемъ такъ, какъ розги плодотворить безъ лозы не могутъ. 5)~Отсюду послѣдуетъ, что первѣе должно во Хрістѣ насадиться, и тако плоды добрыхъ дѣлъ творити: надобно первѣе сдѣлаться добрымъ, и тогда добро творити. Злое бо древо не можетъ плода добра творити\footnote{Мѳ.~7,~18.}. 6)~Коль бѣдное состояніе тѣхъ хрістіанъ, которыи беззаконнымъ житіемъ отъ Хріста отторгнулись! Ибо суть яко розги изсохшія. 7)~Всякому, кто хощетъ спастися, должно чистымъ сердцемъ обратитися ко Господу, и покаяніемъ и слезами омыти себе, и тако Хрісту "--- лозѣ истинной прицѣпитися. Ибо кромѣ Хріста и внѣ Хріста нѣтъ спасенія. Хрістосъ есть животъ и свѣтъ: надобно убо тому быть въ смерти и во тьмѣ, кто отъ живота и свѣта отлучился. Разсуждай сіе, хрістіанине, и покаяніемъ и слезами омый грѣхи твоя, да паки животу "--- Хрісту соединишися.

\section{17. Глава и тѣло.}

Что глава и тѣло между собою: тое Хрістосъ и хрістіане. Тѣло съ главою соединено: тако хрістіане со Хрістомъ духовно соединены. Глава прочее тѣло управляетъ: тако Хрістосъ тѣло Свое духовное или церковь Свою святую управляетъ. Глава о пользѣ и цѣлости всего тѣла промышляетъ: тако Хрістосъ о пользѣ и спасеніи церкви Своея или вѣрныхъ промышляетъ. Что глава замышляетъ и хощетъ, тое уды тѣлесные и дѣлаютъ; руки дѣлаютъ, что глава хощетъ; ноги идутъ, куда глава хощетъ; безъ изволенія главы никакій удъ не дѣлаетъ ничего: тако по волѣ Хрістовой, то"=есть, что Онъ хощетъ, все должно дѣлать хрістіанамъ. Главѣ вси уды тѣлесные повинуются: тако Хрісту должно повиноватися хрістіанамъ. Когда глава страждетъ, то и всѣ тѣлесные уды ей состраждутъ: тако со Хрістомъ пострадавшимъ должно и хрістіанамъ въ мірѣ здѣ страдать, должно съ посмѣяннымъ Хрістомъ посмѣянными быть, съ поруганнымъ поруганными быть; словомъ, со Хрістомъ понесшимъ крестъ, крестъ свой нести и Ему послѣдовать. Когда глава прославляется, то и уды тѣлесные тояжде славы пріобщаются: тако съ прославленнымъ Хрістомъ прославятся и духовные уды Его "--- хрістіане во царствіи Его. Уды тѣлесные суть орудія главы, глава бо все дѣлаетъ, но чрезъ уды своя: тако хрістіане суть орудія главы "--- Хріста, Который чрезъ нихъ добрыя дѣла дѣлаетъ. Что уды добрѣ дѣлаютъ, тое главѣ приписуется: тако должно все Хрісту приписывать, что хрістіане добрѣ дѣлаютъ. Когда глава отъ кого біется, то руки всякимъ образомъ защищаютъ ее и охраняютъ отъ біющаго, дабы не повредилась: тако, когда Хрістосъ хулится и безчестится, должно хрістіанамъ стоять и славу имене Его защищать, хотя бы и самимъ слѣдовало страдать. Отъ главы все тѣлесное блаженство зависитъ: тако отъ Хріста все хрістіанъ блаженство зависитъ. Когда тѣло или удъ какій страждетъ, то состраждетъ и глава: тако когда хрістіане страждутъ, состраждетъ имъ и Хрістосъ. Что уду какому тѣлесному дѣлается, тое вмѣняетъ себѣ и глава: тако, что хрістіанамъ дѣлается, тое вмѣняетъ Себѣ и Хрістосъ. Обида ли убо или зло какое дѣлается хрістіанину? тое все Самому Хрісту дѣлается. Гонится ли, ругается, поносится, укоряется хрістіанинъ? все то Самого Хріста касается. Добро ли и благодѣяніе какое дѣлается хрістіанину? тое вмѣняетъ Себѣ Хрістосъ. И сіе то есть, что Хрістосъ глаголетъ: \textit{понеже сотвористе единому сихъ братій Моихъ меншихъ, Мнѣ сотвористе}. И паки: \textit{понеже не сотвористе единому сихъ меншихъ, ни Мнѣ сотвористе}\footnote{Мѳ.~25,~40 и 45.}. "--- Отсюду видишь, хрістіанине: 1)~Коликій союзъ и общеніе имѣютъ хрістіане со Хрістомъ, яко уды съ главою. 2)~Свята глава "--- Хрістосъ: святынѣ Его надобно пріобщаться и удамъ Его "--- хрістіанамъ, и творити святыню въ страсѣ Божіи: иначе, \textit{кое причастіе правдѣ къ беззаконію, или кое общеніе свѣту ко тьмѣ}\footnote{2~Кор.~6,~14.}? Надобно убо очиститься истиннымъ покаяніемъ и творить плоды достойны покаянія, чтобы быть истиннымъ хрістіаниномъ, живымъ удомъ Хрістовымъ, и тако пріобщаться святынѣ Его. 3)~Коль великое преимущество и блаженство хрістіанъ! Что бо болѣе можетъ быть человѣку, какъ быть удомъ пренебесныя главы "--- Хріста Царя? 4)~Коль тяжкій грѣхъ дѣлаютъ тіи хрістіане, который хрістіанъ обиждаютъ! Понеже тая обида Самаго Хріста касается, что и говорить страшно, но въ самомъ дѣлѣ истина есть. 5)~Отсюду утѣшеніе послѣдуетъ хрістіанамъ, что Хрістосъ "--- глава ихъ состраждетъ имъ въ страданіи, и все, что имъ ни дѣлается, тое Себѣ вмѣняетъ. 6)~Потщимся убо и мы, хрістіанине, истинными быть хрістіанами, да пріобщимся святыни Хрістовой и блаженства истинныхъ хрістіанъ.

\section{18. Уды, или члены, между собою.}

Что члены въ тѣлѣ: тое хрістіане между собою суть. Члены тѣлесные составляютъ тѣло: тако хрістіане составляютъ тѣло духовное. Члены въ тѣлѣ суть соединены, и суть едино тѣло: тако хрістіане суть соединены духовно, и суть едино духовное тѣло. Члены въ тѣлѣ соединяются и связуются жилами: тако хрістіане соединяются и связуются вѣрою, духомъ и любовію. Члены въ тѣлѣ не дѣлаютъ другъ другу вреда: тако хрістіане не должны другъ другу обиды и вреда дѣлать. Члены въ тѣлѣ суть мирны и согласны между собою: тако въ хрістіанахъ долженъ быть миръ и согласіе. Члены въ тѣлѣ другъ, другу помогаютъ, другъ друга защищаютъ: тако хрістіане должны другъ другу помогать, другъ друга защищать. Когда одинъ членъ страждетъ въ тѣлѣ, то и прочіе члены ему состраждутъ: тако и хрістіане, когда хрістіанинъ какій страждетъ, должны ему сострадать, должны плакать съ плачущими. Члены въ тѣлѣ другъ друга не презираютъ, другъ на друга не гордятся: тако хрістіане не должны другъ друга презирать, другъ на друга гордиться. Члены въ тѣлѣ вси суть въ равномъ достоинствѣ: тако и хрістіане о себѣ думать должны. Члены другъ друга предостерегаютъ отъ находящаго зла, какъ"=то: очи видѣніемъ, уши слухомъ, ноги бѣгствомъ, и проч.: тако хрістіане должны другъ друга предостерегать, иной словомъ, иной совѣтомъ, иной страхомъ, иной молитвою. О всемъ составѣ тѣлесномъ глава печется и промышляетъ: тако о всемъ духовномъ Своемъ тѣлѣ глава "--- Хрістосъ печется и промышляетъ. Тому слава во вѣки, аминь! "--- Видишь, хрістіанине, что суть хрістіане, и какая между ими должна быть любовь, миръ, согласіе, милосердіе, состраданіе и проч. Разсуждай убо, надлежиши ли ты до святаго сего общества. О первыхъ хрістіанахъ пишется: \textit{народу вѣровавшему бѣ сердце и душа едина}\footnote{Дѣян.~4,~32.}. О когда бы и нынѣ тое было! Но противное съ болѣзнію и воздыханіемъ видимъ\footnote{Римл.~12,~4 и 5; 1~Кор.~10,~17 и 12, гл.~12,~13 и проч.}.

\section{19. Овцы.}

Хрістіане въ святомъ Писаніи овцами называются потому, что между хрістіанами и овцами немалое имѣется сходство. Овцы никакому скоту не дѣлаютъ обиды: тако хрістіане никого не обиждаютъ. Въ овцахъ примѣчается простота; когда едину овцу волкъ рѣжетъ, не бѣгутъ прочь, но вси на тое смотрятъ: тако хрістіане суть простосердечны, \textit{просты во злое, но мудры во благое}\footnote{Римл.~16,~19.}. Овцы великую пользу приносятъ хозяевамъ, то"=есть, подаютъ имъ волну или шерсть, молоко, кожу, мясо: тако хрістіане всѣмъ добры и полезны суть, и Богу Господу своему приносятъ жертву хваленія и исповѣданія. Въ овцахъ примѣчается миръ и согласіе; ибо и въ малой хлѣвинѣ много ихъ помѣщается: тако хрістіане мирны суть и согласны между собою. Въ овцахъ примѣчается кротость и терпѣніе; когда стригутъ ихъ, молчатъ, и когда бьютъ, молчатъ: тако хрістіане суть смиренніи, кротки и терпѣливы. Въ овцахъ не примѣчается зависти; ибо когда ѣдятъ, не дерутся между собою: тако хрістіане не завистливы суть. Овцы весьма алчны; ибо много ѣдятъ: тако хрістіане хотя и благочестивы суть, всегда алчутъ и жаждутъ благочестія. Овцы пастухамъ своимъ послушливы: тако хрістіане Пастырю и Господу своему Іисусу Хрісту показуютъ послушаніе. \textit{Овцы Моя гласа Моего слушаютъ}, глаголетъ Хрістосъ\footnote{1~Іоан.~10,~27.}. Видишь, хрістіанине, свойства овецъ Хрістовыхъ, истинныхъ хрістіанъ! Разсуждай себе, надлежиши ли ты до благословеннаго стада сего. Когда хощеши во оградѣ небесной быть: надобно тебѣ быть неотмѣнно овцею Хрістовою.

\section{20. Козлища.}

Козлищамъ злыи люди уподобляются. Ибо немалое сходство между козлищами и злыми людьми. Козлища почти всякой скотъ рогами своими бодутъ и обиждаютъ: тако злыи люди всякому человѣку, доброму и злому, или дѣломъ, или словомъ, дѣлаютъ обиду. Въ козлищахъ примѣчается гордость; ибо ходятъ наипаче по высокимъ мѣстамъ, по стремнинамъ, а часто и на зданія всходятъ: тако злыи люди гордятся и высоко возносятся. Козлища всегда тщатся въ стадѣ быть напереду: тако злыи люди всегда хотятъ надъ другими начальствовать или первое мѣсто имѣть. Козлища, когда ихъ влекутъ куда или біютъ, не молчатъ, но кричатъ: такъ злыи люди, когда наказуются, не терпятъ, но ропщутъ, а часто и хулятъ: какая"=де моя вина? что я согрѣшилъ? и проч. Козлища особливый нѣкій издаютъ смрадъ: тако злыи люди смердятъ злонравіемъ и беззаконнымъ житіемъ. Примѣчай, хрістіанине, сходства козлищъ и злыхъ людей, и помни, что написано о овцахъ и козлищахъ: \textit{и соберутся предъ Нимъ} (Хрістомъ Царемъ) \textit{вси языцы, и разлучитъ ихъ другъ отъ друга, якоже пастырь разлучаетъ овцы отъ козлищъ. И поставитъ овцы одесную Себе, а козлища ошуюю}\footnote{Мѳ.~25,~32--33.}. "--- Исправляй убо себе покаяніемъ и молитвою, да будеши овца Хрістова, да не явишися тогда яко козлище, когда разлучатся добрый отъ злыхъ, яко овцы отъ козлищъ.

\section{21. Пастухъ и стадо.}

Видимъ въ мірѣ семъ, что скотъ отъ людей поручается пастухамъ ради пастьбы и охраненія: тако овцы Хрістовы, хрістіане, поручаются отъ Хріста Господа пастырямъ, епископамъ и іереямъ, ради пастьбы и охраненія. Пастухи выбираются отъ всего села или деревни, чтобы пасли скотъ ихъ: тако пастыри, епископы и пресвитеры, избираются отъ церкви, чтобы пасли души хрістіанскія. Пастухи выбираются добрый и разумный, чтобы не погубили скота: тако пастырей должно избирать добрыхъ и разумныхъ, чтобы не погубили словесныхъ Хрістовыхъ овецъ, дабы и на нихъ не сбылося писанное: \textit{слѣпецъ слѣпца аще водитъ, оба въ яму впадутъ}\footnote{Мѳ.~15,~14.}. Пастухи выгоняютъ скотъ на поле, на траву и на добрую пажить, и тамо пасутъ его: пастырямъ должно питать овецъ Хрістовыхъ Словомъ Божіимъ и Тайнами святыми. Пастухи стадо свое берегутъ и защищаютъ отъ волковъ и прочихъ звѣрей: тако пастырямъ должно стадо овецъ Хрістовыхъ берещи и защищать отъ діавола, демоновъ и еретиковъ, и прочихъ душевныхъ звѣрей. Пастухи скотину, отъ стада отлучившуюся и заблудившую, ищутъ и загоняютъ въ стадо: тако пастырямъ должно заблуждшаго хрістіанина искать и обращать къ стаду Хрістову; всякій же хрістіанинъ заблуждаетъ, который не по"=хрістіански живетъ. Пастухи пригоняютъ стадо свое съ поля въ домы хозяевамъ: тако пастырямъ должно стадо овецъ Хрістовыхъ отъ міра сего предпосылать во \textit{ограду небесную}\footnote{Во ограду небесную "--- слова эти внесены въ подлинникъ собственною рукою Святителя. "--- Одно изъ тѣхъ мѣстъ, на которыя мы указали въ предисловіи.}, да и сами потомъ явившися скажутъ: се азъ и овцы Твоя, яже далъ ми еси, Господи! Пастухи за труды и пастьбу скота получаютъ мзду отъ хозяевъ: тако пастыри отъ Хріста Господа за труды и пастьбу стада Его пріимутъ мзду, \textit{неувядаемый славы вѣнецъ}\footnote{1~Петр.~5,~4.}. Отъ пастуховъ, ежели какой скотины не пригонятъ въ домъ, хозяева спрашиваютъ: гдѣ моя такая"=то и такая скотина? Тако отъ пастырей, ежели какій хрістіанинъ погибнетъ и не явится въ небесной оградѣ, спроситъ Хрістосъ Господь: гдѣ Моя тая"=то овца? гдѣ овца, которую Я не сребромъ и златомъ, но Своею Кровію стяжалъ? гдѣ овца, которую ты принялъ пасти и хранити? Пастухъ не можетъ сказать хозяину: не знаю гдѣ; потому что былъ сторожъ стада: тако пастырь не можетъ сказать Хрісту Господу: не знаю, гдѣ овца; потому что сторожъ былъ стада Хрістова\footnote{Іез.~34,~8 и проч.}. Пастухъ лишается отъ хозяина мзды за потеряніе скотины; тако пастырь лишается отъ Хріста мзды за погубленіе словесной овцы, и отдается въ наказаніе. Пастухъ не виновенъ бываетъ, ежели не его нерадѣніемъ, но какимъ"=либо другимъ образомъ скотина погибнетъ: тако пастырь невиновенъ будетъ, когда хрістіанинъ не его нерадѣніемъ и несмотрѣніемъ погибнетъ, но своимъ небреженіемъ и своевольствомъ; то есть, когда пастырь его училъ, наставлялъ, увѣщавалъ и образъ добрыхъ дѣлъ ему показывалъ, но его, яко пастыря своего, не слушалъ. Пастухамъ стадо скотское повинуется; ибо туда идетъ, куда гонятъ его: тако хрістіане должны пастырямъ своимъ повиноваться, и дѣлать тое, чего они научаютъ, по увѣщанію апостольскому: \textit{повинуйтеся наставникомъ вашимъ и покоряйтеся: тіи бо бдятъ о душахъ вашихъ, яко слово воздати хотяще: да съ радостію сіе творятъ, а не воздыхающе}\footnote{Евр.~13,~17.}. "--- Пастырь! помни, что ты сторожъ стада овецъ Хрістовыхъ. Люди! да внимаютъ, чтó Хрістосъ глаголетъ о пастыряхъ, которые учатъ, но не творятъ: \textit{на Моѵсеовѣ сѣдалищи сѣдоша книжницы и фарисее: вся убо, елика аще рекутъ вамъ блюсти, соблюдайте и творите: по дѣломъ же ихъ не творите; глаголютъ бо и не творятъ}\footnote{Мѳ.~23,~2--3.}.

\section{22. Женихъ и невѣста.}

Что между собою женихъ и невѣста: тое Хрістосъ и хрістіанская душа. Невѣста жениху обручается: тако душа человѣческая вѣрою обручается Хрісту Сыну Божію, и омывается банею Крещенія. Невѣста оставляетъ домъ и родителей, и прилѣпляется единому жениху своему: тако хрістіанская душа, обручившаяся Хрісту Сыну Божію, должна оставить міръ и мірскія прихоти, и прилѣпиться единому Жениху своему Іисусу Хрісту; къ чему Духъ Святый чрезъ пророка увѣщаваетъ ее: \textit{слыши дщи и виждь и приклони ухо твое, и забуди люди твоя и домъ отца, твоего. И возжелаетъ Царь доброты твоея}\footnote{Пс.~44,~11--12.}. Невѣста убирается въ платье цвѣтное и украшается, чтобы понравилася жениху своему: тако душа хрістіанская должна убираться въ пристойное себѣ одѣяніе и украшать себе внутрь, чтобы понравиться Жениху своему, Іисусу Хрісту. Одѣяніе души указуетъ Духъ Святый чрезъ Апостола сіе: \textit{облецытеся убо (якоже избранніи Божіи, святи и возлюбленни) во утробы щедротъ, благодать, смиренномудріе, кротость и долготерпѣніе}\footnote{Кол.~3,~12.}. "--- Добрая невѣста хранитъ вѣрность жениху своему: тако душа хрістіанская должна быть вѣрна Іисусу Хрісту до смерти; о чемъ Самъ Хрістосъ ей говоритъ: \textit{буди вѣренъ до смерти, и дамъ ти вѣнецъ живота}\footnote{Апок.~2,~10.}.

Добрая невѣста не любитъ никого равно, а паче болѣе жениха своего: тако душа хрістіанская не должна любить никого равно, а паче болѣе, Хріста Жениха своего. Не нравится невѣста жениху, когда равную, яко ему, а паче когда большую, любовь другому отдаетъ: тако душа хрістіанская не угодна есть Хрісту, когда другое что равно Хрісту, или, что горше того, болѣе Его любитъ. \textit{Иже любитъ отца, или матерь паче Мене, нѣсть Мене достоинъ; и иже любитъ сына, или дщерь, паче Мене, нѣсть Мене достоинъ}, глаголетъ Хрістосъ\footnote{Мѳ.~10,~37.}. "--- Благородіе, честь и достоинство невѣсты отъ жениха зависитъ: тако души хрістіанской честь, благородіе и достоинство отъ Хріста зависитъ. Чимъ благороднѣйшій женихъ, тѣмъ благороднѣйшая бываетъ и невѣста, ему обрученная: но какъ нѣтъ честнѣйшаго, благороднѣйшаго и достойнѣйшаго паче Хріста, Сына Божія: то и хрістіанской души не можетъ быть большая честь и достоинство, какъ быть обрученною невѣстою Хрісту Сыну Божію. Честь, благородіе и достоинство невѣсты не видно, пока жениху своему бракомъ не соединится: тако души хрістіанскія честь и слава не видна, пока бракомъ Іисусу Хрісту не соединится въ будущемъ вѣкѣ. \textit{Возлюбленніи}, глаголетъ Апостолъ, \textit{не у явися, что будемъ}\footnote{1~Іоан.~3,~2.}. Покажется слава и достоинство невѣсты, когда жениху своему благороднѣйшему бракомъ соединится: тако открыется превеликая слава хрістіанскія души, когда въ пришествіи Его второмъ бракомъ Ему соединится; тогда она подобна будетъ краснѣйшему своему Жениху. \textit{Вѣмы, яко, егда явится, подобни Ему будемъ, и узримъ Его, якоже есть}\footnote{1~Іоан.~3,~2.}. По бракѣ невѣста въ домъ жениховъ приводится съ веселіемъ: тако душа хрістіанская \textit{введется} въ домъ и палату небеснаго своего Жениха, Іисуса Хріста, съ радостію и веселіемъ\footnote{Пс.~44,~16.}. \textit{Господемъ возвратятся, и пріидутъ въ Сіонъ съ радостію и веселіемъ вѣчнымъ: на главѣ бо ихъ веселіе и хвала, и радость пріиметъ я: отбѣже болѣзнь, и печаль, и воздыханіе}\footnote{Ис.~51,~11.}. "--- По бракѣ бываетъ пиръ и веселіе: тако на бракѣ ономъ будетъ великая вечеря, радость и веселіе; тогда ясти будутъ, пити будутъ, радоватися будутъ, веселитися въ веселіи сердца будутъ\footnote{65,~13--14.}. Сіе открылося святому Апостолу Іоанну: \textit{и слышахъ яко гласъ народа многа, и яко гласъ водъ многихъ, и яко гласъ громовъ крѣпкихъ, глаголющихъ: аллилуія! яко воцарися Господь Богъ Вседержитель. Радуимся и веселимся и дадимъ славу Ему: яко пріиде бракъ Агнчій: и жена Его уготовила есть себе. И дано бысть ей облещися въ виссонъ чистъ и свѣтелъ: виссонъ бо оправданія святыхъ есть}\footnote{Апок.~19,~6--8.}. "--- Женихъ радуется о невѣстѣ своей, и невѣста о женихѣ своемъ: тако возрадуется Хрістосъ, Женихъ небесный, о душахъ спасшихся и прославленныхъ. \textit{И будетъ, якоже радуется женихъ о невѣстѣ, тако возрадуется Господь о тебѣ}\footnote{Ис.~62,~5.}. И души возрадуются о пресладкомъ и прекрасномъ Женихѣ своемъ во вѣки вѣковъ. \textit{Да возрадуется душа моя о Господѣ! облече бо мя въ ризу спасенія, и одеждею веселія одѣя мя; яко на жениха возложи на мя вѣнецъ, и яко невѣсту украси мя красотою}\footnote{61,~10.}. "--- О семъ таинственномъ обрученіи и бракѣ глаголетъ Господь чрезъ пророка: \textit{обручу тя Себѣ во вѣкъ; и обручу тя Себѣ въ правдѣ, и въ судѣ, и въ милости, и въ щедротахъ: и обручу тя Себѣ въ вѣрѣ и увѣси Господа}\footnote{Ос.~2,~19--20.}. И Апостолъ глаголетъ хрістіанамъ: \textit{обручихъ васъ единому мужу дѣву чисту представити Хрістови}\footnote{2~Кор.~11,~2.}. "--- Потщимся, возлюбленный хрістіанине, очистить себе покаяніемъ, и души наши украсить одеждою добродѣтелей, да угодимъ небесному нашему Жениху, Іисусу Хрісту, и пріиметъ насъ въ небесный Свой и вѣчный чертогъ.

\section{23. Брань.}

Видимъ въ мірѣ семъ, что когда возстаетъ царство на царство, то между ими бываетъ брань: тако и хрістіане имѣютъ своихъ враговъ, возстающихъ на нихъ, и бываетъ съ ними брань. На брани міра сего люди на людей возстаютъ: тако въ брани хрістіанской сатана и злые его аггелы противу хрістіанъ возстаютъ. Люди, на людей возстающіи, видимы суть: но сатана и аггелы его, враги наши, невидимы намъ суть. На брани міра сего противная противную сторону видитъ, и такъ другъ друга опасаются: но въ хрістіанской брани хрістіане видимы отъ враговъ своихъ демоновъ, но сами ихъ не видятъ. Люты и тяжки намъ враги видимыи: но лютѣйшіи и тяжчайшіи намъ суть сіи враги невидимыи, то"=есть, демоны. На брани видимой, чимъ злѣйшій и хитрѣйшій врагъ, тѣмъ опаснѣйшій бываетъ: но какъ нѣтъ злѣйшаго и хитрѣйшаго врага, какъ сатана и демоны его; того ради и брань съ ними весьма опасная намъ. Когда люди противу людей воюютъ, то временемъ и почиваютъ отъ брани: но сатана и аггелы его злые никогда не спятъ, но всегда бодрствуютъ и тщатся насъ низложить. Брань, которая между людьми бываетъ, хотя и продолжается, однако престаетъ и миръ заключается: но у хрістіанъ непрестанная брань, даже до смерти противу враговъ своихъ, и смертію кончится. Бываетъ на брани, что непріятель отступаетъ и будто престаетъ отъ брани, но тогда опаснѣйшій бываетъ; ибо тако хитрость свою употребляетъ и замышляетъ противную сторону искуснѣе поразить: тако врагъ нашъ сатана дѣлаетъ, часто аки отступаетъ отъ насъ и престаетъ отъ брани; но сіе хитрость и коварство его есть: ибо тако замышляетъ въ неосторожность и безопасность привести насъ и удобнѣе насъ низложить. На брани люди другъ противу друга съ оружіемъ выходятъ, другъ друга оружіемъ уязвляютъ и поражаютъ: тако и на брани хрістіанской есть оружіе. Демоны имѣютъ свое оружіе, и хрістіане имѣютъ свое оружіе. Демоны борютъ насъ оружіемъ страстей и удовъ нашихъ; и столько у нихъ оружія, сколько въ плоти нашей страстей. Хрістіанское оружіе есть слово Божіе и молитва; Симъ оружіемъ хрістіане противу демоновъ ополчаются и защищаютъ себе, и недѣйствительными творятъ враговъ своихъ стрѣлы. Что у воина мечь и прочее оружіе, тое у хрістіанъ молитва и глаголъ Божій. Хрістіанинъ убо безъ молитвы и глагола Божія, какъ воинъ безъ меча и ружья. Воины на брани всегда при себѣ имѣютъ Мечь и оружіе: тако хрістіане всегда должны быть вооружены духовнымъ мечемъ глагола Божія и оружіемъ молитвы. Ибо непрестанная у нихъ брань противу враговъ своихъ. Откуду повелѣвается имъ: \textit{непрестанно молитеся}\footnote{1~Сол.~5,~17.}. Воины на брани бодрствуютъ и весьма осторожно поступаютъ ради окружающихъ враговъ: тако хрістіанамъ на брани своей должно бодрствовать и осторожно поступать всегда; ибо всегда окружаютъ враги ихъ. Откуду глаголется имъ: \textit{трезвитеся, бодрствуйте, зане супостатъ вашъ, діаволъ, яко левъ рыкая, ходитъ, искій кого поглотити: емуже противитеся тверди вѣрою}\footnote{1~Петр.~5,~8--9.}. Воины на брани вси единодушно стоятъ противу непріятеля и подвизаются: тако хрістіанамъ должно всѣмъ единодушно стоять противу врага своего діавола и подвизаться. Воины на брани другъ другу помогаютъ: тако хрістіанамъ должно другъ другу помогать противу врага своего, то совѣтомъ, то увѣщаніемъ, то молитвою. Непріятель, когда видитъ свое изнеможеніе, другихъ союзниковъ своихъ на помощь себѣ призываетъ: тако врагъ нашъ діаволъ, когда самъ собою не успѣетъ противу хрістіанина, то ищетъ злыхъ людей и боретъ чрезъ нихъ хрістіанина. Сего ради всякъ хрістіанинъ, когда ближняго своего прельщаетъ, гонитъ, озлобляетъ и какимъ либо образомъ ко грѣху приводитъ, съ діаволомъ купно противу его вооружается. Смотри, хрістіанине! что еси, и съ кѣмъ въ согласіи имѣешися. На брани имѣются начальники и полководцы, которыи воиновъ научаютъ, наставляютъ и поощряютъ къ доброму подвигу противу врага: тако на брани хрістіанской начальники суть пастыри и учители, которыи хрістіанъ вооружаютъ словомъ Божіимъ противу врага діавола, и научаютъ и наставляютъ, како противу его стоять и подвизаться. Непріятель на брани тщится наипаче начальниковъ и полководцевъ противныя стороны поразить и побить, дабы въ непорядокъ и замѣшательство все воинство привести и тако погубить или плѣнить его; отъ начальниковъ бо и полководцевъ, наипаче мудрыхъ, вся цѣлость и благополучіе воинства зависитъ: тако на брани хрістіанской врагъ діаволъ о семъ наипаче тщится, какъ бы пастырей и учителей низложить, дабы тако удобѣе могъ и прочихъ хрістіанъ плѣнить и погубить; отъ пастырей бо и учителей вся цѣлость и спасеніе хрістіанскаго собранія зависитъ. Безъ добраго и разумнаго пастыря хрістіане суть какъ овцы заблуждшія. И когда врагъ не можетъ пастыря низложить, то возставляетъ на него людей, которыи волю его творятъ, дабы слухъ золъ о немъ проносили, и тако бы люди ученію его не вѣрили. Отсюду то бываетъ, что пастыри и учители много клеветы, поношенія, злословія, гоненія и изгнанія претерпѣваютъ: о чемъ вси вѣки свидѣтельствуютъ. Никто бо такъ врагу сему не досаждаетъ, какъ пастыри и учители добрыи. Ибо они темное его царство и власть словомъ Божіимъ и силою Духа Святаго разрушаютъ и изъ рукъ его души хрістіанскія, любимую его корысть, восхищаютъ. Сего ради ни на кого болѣе, какъ на пастырей и учителей, злобный духъ не ярится и свирѣпѣетъ. Берегись убо, хрістіанине, хотя всякаго человѣка, а паче пастыря и учителя злословить, да не съ діаволомъ едино будешь мудрствовать. "--- На брани міра сего тщатся люди рубежи и грады своя защищать, или у противной стороны отбирать: но на брани хрістіанской не тако. Врагъ діаволъ не о городахъ, не о рубежахъ, не о богатствѣ нашемъ, но о душахъ нашихъ тщится, не тѣло, но душу нашу плѣнить и вѣчно погубить, не земное и временное блаженство, но небесное и вѣчное отнять. О семъ онъ все тщаніе и стараніе полагаетъ, какъ бы насъ съ собою въ погибель привлещи. Смотри, хрістіанине, злобу, вражду и лютость противу тебе врага твоего, и берегись его. Не злато, не сребро и прочее вещество тлѣнное, но вѣчное и нетлѣнное сокровище, спасеніе твое тщится у тебе отнять врагъ твой. Береги убо сіе не токмо паче имѣнія, но и паче живота твоего. "--- На брани міра сего случается, что, которымъ съ начала неблагополучна война бываетъ, при концѣ благополучно отправляется, и тако благополучно окончавшимъ побѣда приписуется; которая сторона сначала побѣждена бываетъ, тая при окончаніи брани побѣждаетъ, и тако благополучно кончаетъ брань: тако бываетъ и на хрістіанской брани. Многіе хрістіане съ начала житія своего побѣждены бываютъ отъ супостата своего діавола; но потомъ благодатію Божіею вставше и силою Хрістовою укрѣпившеся, крѣпко ополчаются противу врага и побѣждаютъ его, и тако благополучно оканчиваютъ. Благополучіе же и неблагополучіе отъ конца зависитъ. Не тотъ благополученъ, кто добрѣ началъ, но тотъ, кто добрѣ скончалъ. "--- Воинъ, чимъ болѣе на брани и сраженіи бываетъ, тѣмъ искуснѣйшимъ и храбрѣйшимъ дѣлается: тако хрістіанинъ, чимъ частѣе впадаетъ во искушенія, бѣды и напасти, тѣмъ искуснѣйшимъ бываетъ въ дѣлѣ хрістіанскомъ. Откуду Апостолъ глаголетъ: \textit{всяку радость имѣйте, братіе моя, егда во искушенія впадаете различна, вѣдяще, яко искушеніе вашея вѣры содѣловаетъ терпѣніе; терпѣніе же дѣло совершенно да имать, яко да будете совершена и всецѣли, ни въ чемже лишена}\footnote{Іак.~1,~2--4.}. По окончаніи брани побѣдители съ торжествомъ и радостію во отечество возвращаются и отъ царя своего почесть пріемлютъ: тако хрістіане, благополучно окончивше брань свою и блаженно скончавше житіе временное, съ торжествомъ и веселіемъ идутъ во отечество небесное и пріемлютъ вѣнецъ правды отъ Царя небеснаго, Іисуса Хріста. \textit{Подвигомъ добрымъ подвизахся, теченіе скончахъ, вѣру соблюдохъ: прочее убо соблюдается мнѣ вѣнецъ правды, егоже воздастъ ми Господь въ день онъ, Праведный Судія, не токмо же мнѣ, но и всѣмъ возлюбльшимъ явленіе Его}\footnote{2~Тим. 4--7.}. Сію хрістіанскую брань представляетъ намъ Апостолъ и вооружаетъ насъ: \textit{прочее же, братіе моя, возмогайте о Господѣ, и въ державѣ крѣпости Его; облецытеся во вся оружія Божія, яко возмощи вамъ стати противу кознемъ діавольскимъ: яко нѣсть наша брань противу крове и плоти, но къ началомъ, и ко властемъ, и къ міродержителемъ тьмы вѣка сего, къ духовомъ злобы поднебеснымъ. Сего ради пріимите вся оружія Божія, да возможете противитися въ день лютъ, и вся содѣявше стати. Станите убо препоясана чресла ваша истиною, и оболкшеся въ броня правды, и обуете нозѣ во уготованіе благовѣствованія мира; надъ всѣми же воспріимше щитъ вѣры, въ немже возможете вся стрѣлы лукаваго разженныя угасити; и шлемъ спасенія воспріимите, и мечь духовный, иже есть глаголъ Божій}\footnote{Еф.~6,~10--17.}. "--- Отсюду видишь, хрістіанине: 1)~Хрістіанское житіе въ мірѣ семъ не иное что есть, какъ непрестанная брань. 2)~Брань не противу плоти и крови, то есть, не противу человѣковъ, но противу діавола и демоновъ. 3)~Враги сіи суть невидими, весьма злобныи, враждебныи, хитрыи, которыи противу насъ востаютъ и борютъ насъ. 4)~Не временную корысть, но вѣчное спасеніе тщатся отнять у насъ, и съ собою въ погибель привести. 5)~Отсюду видишь, коль лютая и тяжкая брань противу сихъ враговъ. Сего ради не должно намъ дремать, но бодрствовать и осторожно поступать. Здѣ спасеніе вѣчное или пріобрѣтается, или погубляется. 6)~Подвигъ сей и побѣда безъ помощи Хрістовой не бываетъ: здѣ сила человѣческая ничего не можетъ. Сего ради должно намъ непрестанно молитися и воздыхать ко Хрісту подвигоположнику, да поможетъ намъ и укрѣпитъ. Непрестанно требуемъ помощи Его: ибо непрестанно насъ враги наши окружаютъ. 7)~Понеже на пастырей и учителей нашихъ болѣе враги наши устремляются; то всѣмъ хрістіанамъ болѣе о нихъ должно молитися, да поможетъ имъ Господь. 8)~Понеже слово Божіе и молитва суть оружіе хрістіанское; то кто слово Божіе и молитву оставляетъ, тотъ дѣлаетъ такъ, какъ воинъ, который, будучи на брани, мечь и оружіе съ себе бросаетъ, и знакъ, что врагу же уступилъ. 9)~Вступивши въ подвигъ не унывай, не отчаивайся, видя такъ лютую брань, но стой неподвижно. Хрістосъ насъ ободряетъ, хрістіанине: \textit{дерзайте, яко Азъ побѣдихъ міръ}\footnote{Іоан.~16,~33.}. Дерзай убо, хрістіанине! Богъ за насъ стоитъ. \textit{Аще Богъ за насъ, кто на насъ}\footnote{Рим.~8,~32.}. \textit{О Бозѣ сотворимъ силу, и Той уничижитъ враги наша}\footnote{Пс.~107,~14.}. 10)~Падшій и лежащій востани, какъ воинамъ на брани обычно. Они падаютъ и востаютъ, уязвляются и уязвляютъ. Тако твори и ты, хрістіанине! Стани на ноги твои и ободрись, и призвавши Господа силъ, Іисуса Хріста, въ помощь, стой и подвизайся. Неотмѣнно поможетъ тебѣ, видя твое тщаніе и усердіе; поможетъ тебѣ, Который сдѣлался подобнымъ тебѣ ради тебе, и пострадалъ и умеръ за тебе. Воинамъ, необыкшимъ къ брани, сначала страшна бываетъ брань, но потомъ не страшна, и съ дерзновеніемъ исходятъ на брань: тако и тебѣ страшенъ сначала подвигъ сей, но потомъ легокъ и удобенъ будетъ. Дерзай убо возлюбленне, и изыди и подвизайся противу супостата твоего. Хрістосъ царь нашъ, Хрістосъ заступникъ нашъ, Хрістосъ помощникъ нашъ, Хрістосъ крѣпокъ и силенъ въ брани; Той зоветъ насъ на брань и подвигъ противу врага нашего, который и самаго имени Его трепещетъ; зоветъ, и обѣщаетъ намъ помощь Свою подать: \textit{дерзайте, яко Азъ побѣдихъ міръ}. Той убо въ насъ побѣдитъ его: только востанемъ и станемъ добрѣ, и Онъ востанетъ въ помощь нашу и станетъ за насъ, и будетъ столпъ крѣпости отъ лица вражія. \textit{Воскресни Господи, помози намъ, и избави насъ имене Твоего ради}\footnote{Пс.~43,~27.}. 11)~Пастырь, видишь, какъ разумнымъ и мудрымъ подобаетъ тебѣ быть, который начальникомъ еси на брани сей духовной! Аще бо на видимой брани требуются искусный и мудрый начальники, гдѣ люди съ людьми и плоть съ плотію сражаются: кольми паче на брани духовной должны быть разумны и мудры начальники, гдѣ брань бываетъ у людей противу демоновъ, и у плоти и крови противу духовъ невидимыхъ и хитрыхъ. О пастырь! тебѣ наипаче подобаетъ быть умудреннымъ въ словѣ Божіи, который и прочихъ наставлять и умудрять долженъ; тебѣ наипаче препоясаться мечемъ глагола Божія, и облещися во всеоружіе и бодрствовать, и опасно поступать на брани сей, на которой супостатъ тебе наипаче уязвить и низложить тщится, и тако себе и прочіихъ порученныхъ тебѣ хранить. Отъ руки твоей вси хрістіанскія души взыщутся. Помни сіе, возлюбленне, и вложи въ сердце твое, и не дремли, стоя на стражи Господни!

12)~Воины на брани надеждою побѣды и высшаго ранга себе ободряютъ, и тако храбро подвизаются. Возлюбленный хрістіанине! твоя побѣда не надъ плотію и кровію, но надъ духомъ злобы, и потому преславна будетъ, и вѣнецъ тебѣ, и почесть, и слава, и похвала, и миръ, и торжество не временное и земное, но небесное и вѣчное, не отъ человѣка, но отъ Бога, солгать не могущаго, обѣщано. \textit{Буди вѣренъ до смерти, и дамъ ти вѣнецъ живота}\footnote{Апок.~2,~10.}. Ради временнаго и тлѣннаго сынове вѣка сего трудятся и подвизаются: намъ ли ради вѣчнаго и неизреченнаго добра не подвизаться? 13)~Демоновъ орудія, которыми насъ борютъ и ко злу приводятъ, суть помышленія злая, отъ нихъ въ сердцѣ нашемъ возбуждаемая: сія они на насъ стрѣляютъ. Сколько разъ имъ соизволяемъ, и мыслимъ, или говоримъ, или дѣлаемъ зло, столько разъ отъ нихъ побѣждени бываемъ; сколько разъ противимся имъ, столько разъ ихъ побѣждаемъ, и Бога нашего почитаемъ, Который милостивно посѣщаетъ и поощряетъ насъ, дабы мы съ ними боролись; помогаетъ намъ, дабы ихъ побѣждали; укрѣпляетъ насъ, дабы не ослабѣвали въ брани; обѣщаетъ намъ вѣнецъ нетлѣнный, да до конца пребудемъ въ подвигѣ. \textit{Буди вѣренъ до смерти, и дамъ ти вѣнецъ живота}.

\subsection{О томжде.}

На брани міра сего тогда люди побѣждаютъ, когда противную сторону гонятъ; тогда побѣждаются, когда противникамъ уступаютъ: на брани хрістіанской не тако. Тогда хрістіане побѣждаютъ, когда врагамъ своимъ уступаютъ; когда отъ людей озлобляются, но сами ихъ не озлобляютъ; отъ людей ругаеми и посмѣваемы бываютъ, но сами не ругаютъ, ни посмѣваются; отъ людей біются, но сами ихъ не біютъ и проч. Ибо, когда людемъ уступаютъ, тогда діаволу не уступаютъ. Се есть побѣда хрістіанская, которая въ терпѣніи, а не во отмщеніи состоитъ. Большая язва діаволу и славнѣйшая хрістіанъ побѣда, когда они \textit{любятъ враговъ своихъ, благословляютъ кленущія ихъ, добро творятъ ненавидящимъ ихъ, и молятся за творящихъ имъ напасть и изгонящія ихъ}\footnote{Мѳ.~5,~44.}. Се есть преславная хрістіанская побѣда "--- любовію и благостынею побѣждать злобу враговъ своихъ! Въ семъ во образъ подалъ намъ себе Хрістосъ, и показалъ, како намъ враговъ своихъ побѣждать, Который за враговъ Своихъ молился: \textit{Отче отпусти имъ}\footnote{Лук.~23,~34.}. Вотъ тебѣ, хрістіанине, хрістіанская побѣда "--- не отмщевать врагамъ, но молиться за враговъ! Убо \textit{не побѣжденъ бывай отъ зла, но побѣждай благимъ злое}\footnote{Римл.~12,~21.}.

\subsection{О томжде.}

Како слово Божіе и молитва есть оружіе хрістіанское, слыши и внимай. Діаволъ поощряетъ тебе ко грѣху, но ты въ сердцѣ твоемъ отвѣтствуй ему: не хощу; ибо Богу противно, Богъ то запретилъ. Діаволъ возбуждаетъ въ тебѣ скверную и блудную мысль; ты отвѣщай ему: Богъ мой запретилъ сіе мнѣ. Діаволъ возбуждаетъ въ тебѣ гнѣвъ и злобу ко отмщенію; ты пресѣкай сію мысль мечемъ глагола Божія, глаголя въ сердцѣ твоемъ: Богъ того не повелѣлъ. Указуетъ тебѣ діаволъ на чужую вещь, и подстрекаетъ сердце твое къ хищенію тоя; говори въ сердцѣ твоемъ: Богъ тое запретилъ: \textit{не укради, не пожелай}. Тако и въ прочихъ мысляхъ, противныхъ закону Божію, возстающихъ въ сердцѣ твоемъ, поступай и смотри, согласная ли, или противная закону Божію мысль твоя? Согласную пріемли, и въ дѣло производи: противную закону отражай, да не укрѣпившися побѣдитъ тебе. Въ семъ бо образъ подалъ намъ Себе Хрістосъ Спаситель нашъ, Который на всякое діавольское искушеніе отвѣтствовалъ искусителю: \textit{писано есть, писано есть}\footnote{Мѳ.~4,~4,~7,~10.}. "--- Приводитъ тебе сатана ко отчаянію, и глаголетъ тебѣ въ сердцѣ твоемъ: нѣсть тебѣ спасенія, ты много согрѣшилъ, ты толико и толико золъ содѣлалъ. Ты отвѣщай ему: ты осужденный, а не судія; тебѣ нѣсть спасенія, но уготованъ вѣчный огнь. Моя надежда и спасеніе Хрістосъ Богъ, Который пришелъ въ міръ грѣшники спасти. Спасетъ убо и мене: понеже и я единъ отъ грѣшниковъ, которыхъ Онъ пришелъ спасти. Но вездѣ нужно есть другое оружіе, то есть, молитва, безъ которой все наше тщаніе и сопротивленіе безсильно. Во всякомъ убо искушеніи должно возводить очи свои ко Хрісту и молиться Ему: \textit{Господи, помози мнѣ}; или иначе какъ просить отъ Него помощи. \textit{Возмогай} убо, хрістіанине, \textit{во Господѣ, и въ державѣ крѣпости Его}\footnote{Еф.~6,~10.}. И какъ садовникъ отрѣзываетъ сучки и отрасли, древу вредные, дабы возрастше древо не повредили: тако ты возникающіи злые помыслы тотчасъ пресѣкай мечемъ глагола Божія и молитвою, дабы укрѣпившеся не повредили и не умертвили внутренняго человѣка. Угашай искру, пока въ пламень не возросла, и убивай врага, пока малъ есть. Аще бо искру не угасиши, то великой огнь будетъ, и аще врага не убіеши, пока малъ есть, то возрастши укрѣпится, одолѣетъ тя и низложитъ тя. \textit{Блаженъ иже иметъ и разбіетъ младенцы} сія о \textit{камень}\footnote{Пс.136,~9.}.

\section{24. Путникъ.}

Человѣкъ, который отъ мѣста къ мѣсту намѣренному идетъ, есть путникъ: тако мы, хрістіане, отъ дня рожденія нашего до дня смерти, путники есмы, по пути житія сего шествуемъ. Путнику иному должайшій путь, иному кратчайшій: тако намъ путь житія нашего не равный; иной бо скорѣе оканчиваетъ путь свой, иной болѣе идетъ. Путникъ, по пути идучи, всякаго злополучія опасается: тако намъ, по пути міра сего идущимъ, надобно всего опасаться. \textit{Блюдите, како опасно ходите; не якоже немудри, но якоже премудри, искупующе время, яко дніе лукави суть}\footnote{Еф.~15,~15--16.}. Путникъ подлежитъ всякой непогодѣ и бури: тако намъ, путь свой идущимъ, приключается всякая буря бѣдъ, напастей и искушеній. Путникъ о всемъ томъ небрежетъ, но, все претерпѣвая, идетъ къ намѣренному мѣсту: тако и намъ должно о всемъ неблагополучіи, намъ приключающемся, пренебрегать, терпѣть и итить къ желаемому своему отечеству. На путника часто нападаютъ разбойники и обнажаютъ, а часто и убиваютъ, и тако до намѣреннаго мѣста не допущаютъ: тоежде приключается и намъ; ибо и на насъ демоны, какъ разбойники, нападаютъ и неосторожныхъ уязвляютъ и убиваютъ, и тако до желаемаго отечества не допущаютъ. Путники защищаютъ себе отъ разбойниковъ оружіемъ: тако намъ должно защищать себе отъ демоновъ молитвою и глаголомъ Божіимъ. У путника всегда на умѣ мѣсто, къ которому идетъ, и тщится, какъ бы до него дойти: тако у насъ всегда должно быть отечество небесное на умѣ, и о томъ тщаніе, какъ бы до него достигнуть. \textit{Тако тецыте, да постигнете}, глаголетъ намъ Апостолъ святый\footnote{1~Кор.~9,~24.}. Путникъ чимъ далѣе по пути идетъ, тѣмъ болѣе сокращается путь; тако, чимъ болѣе мы продолжаемъ путь житія нашего, тѣмъ болѣе къ концу приближаемся. Путникъ, пришедши въ мѣсто, куда шелъ, безопасенъ бываетъ и упокоевается, и радуется о благополучномъ окончаніи пути своего: тако и мы, хрістіанине, когда благополучно окончаемъ путь нашъ, и пріидемъ въ небесное отечество, безопасны будемъ, всѣхъ бѣдъ и напастей свободимся, упокоеваться и о вѣчномъ своемъ блаженствѣ радоватися будемъ, которое буди намъ получить благодатію и человѣколюбіемъ Господа нашего Іисуса Хріста, аминь! \textit{Къ Тебѣ возведохъ очи мои, живущему на небеси. Се яко очи рабъ въ руку господій своихъ, яко очи рабыни въ руку госпожи своея: тако очи наши ко Господу Богу нашему, дондеже ущедритъ ны. Помилуй насъ, Господи, помилуй насъ, яко помногу исполнихомся уничиженія: наипаче наполнися душа наша поношенія гобзующихъ, и уничиженія гордыхъ}\footnote{Пс.~122,~1--4.}.

\section{25. Путникъ и вождь.}

Видимъ въ мірѣ семъ, что путникъ, которому путь къ намѣренному мѣсту незнаемъ и неизвѣстенъ, требуетъ и ищетъ вождя себѣ; а безъ вождя удобно заблуждаетъ и не доходитъ до намѣреннаго мѣста. Возлюбленный хрістіанине! Мы въ мірѣ семъ путники есмы, и хощемъ дойти до желаемаго намъ небеснаго отечества, какъ выше сказано, "--- отечества, котораго чрезъ грѣхъ лишились. Вожделѣнное, любезное, пресладкое, прекрасное, покойное, мирное и всякихъ благъ, \textit{ихже око не видѣ, и ухо не слыша, и на сердце человѣку не взыдоша}, исполненное отечество; отечество, на которое съ юдоли сей плачевной слезными смотримъ очами, "--- предрагое, говорю, есть отечество: но путь къ нему неизвѣстенъ и страшенъ намъ. Неизвѣстенъ: ибо не знаемъ, какъ туда пріити. Страшенъ: ибо безчисленные демоны, какъ разбойники, окружаютъ его, и идущимъ препятствуютъ и тщатся не допустить до онаго. Требуемъ убо непремѣнно искуснаго и премудраго вождя въ такъ важномъ дѣлѣ. Гдѣ жъ намъ его искать? Въ премудрыхъ ли и разумныхъ вѣка сего? Нѣтъ! вси бо не токмо простыи, но и премудрыи заблудили отъ пути сего; \textit{вси яко овцы заблудихомъ: человѣкъ отъ пути своего заблудилъ}\footnote{Ис.~53,~6.}. Потому и самыи премудрыи вѣка сего требуютъ вождя къ отечеству оному. Гдѣжъ убо намъ искать его? Пророки, апостоли и учители вси указуютъ на Іисуса Хріста, Сына Божія. Онъ есть единъ вѣрный и премудрый Вождь къ небесному отечеству. Никто не пріидетъ туда, аще Онъ не приведетъ, якоже Самъ свидѣтельствуетъ, глаголя: \textit{никтоже пріидетъ ко Отцу, токмо Мною}\footnote{Іоан.~14,~6.}. Богъ, хотя извести Израиля отъ работы Египетскія и привести въ землю обѣтованную, послалъ Моѵсеа, вѣрнаго раба Своего, на тое великое дѣло, который и извелъ Израиля отъ горькой оной работы и велъ пустынею въ землю обѣтованную; по скончаніи Моѵсеа, воздвиглъ Господь Іисуса Навина вождя имъ, который и ввелъ людей Божіихъ въ землю, кипящую медомъ и млекомъ. Хрістіанине! Моѵсей, Іисусъ Навинъ образовали Іисуса Хріста Сына Божія: ветхій Израиль "--- новаго Израиля, хрістіанъ; работа египетская "--- тяжкую работу діавольскую; прехожденіе чрезъ море "--- святое крещеніе; хожденіе по пустыни Израиля "--- житіе хрістіанъ въ мірѣ семъ; земля обѣтованная "--- небесное отечество значитъ. Богъ послалъ къ намъ Единороднаго Сына Своего, Іисуса Хріста, чтобы насъ, изведши изъ работы діавольскія, привелъ ко Отцу Своему небесному. О немъ Отецъ небесный свидѣтельствуетъ съ небесе: \textit{Сей есть Сынъ мой возлюбленный, о Немже благоволихъ; Того послушайте}\footnote{Мѳ.~17,~5.}. О немъ вси пророки, апостоли и учители церковныи свидѣтельствуютъ, и научаютъ насъ послѣдовать Ему. \textit{Хрістосъ пострада по насъ, намъ оставль образъ, да послѣдуемъ стопамъ Его}\footnote{1~Петр.2,~21.}. И Самъ глаголетъ: \textit{образъ дахъ вамъ, да якоже Азъ сотворилъ вамъ, и вы творите}\footnote{Іоан.~13,~15.}. И паки: \textit{аще кто Мнѣ служитъ, Мнѣ да послѣдствуетъ; и идѣже есмь Азъ, ту и слуга Мой будетъ}\footnote{12,~26.}. Ему послѣдовали вси святіи: и привелъ ихъ въ землю, кипящую медомъ и млекомъ небесныя сладости. Ему послѣдуютъ и нынѣ вѣрныи рабы Его: послѣдуютъ вѣрою, любовію, смиреніемъ, терпѣніемъ, кротостію; и ведетъ ихъ во отечество оное, и, яко Пастырь добрый, вземши ихъ на рамена Своя, приноситъ къ Отцу Своему небесному. О хрістіанине! прилѣпимся и мы къ благословенной и святой дружинѣ сей, и пойдемъ за Іисусомъ вѣрою, любовію, смиреніемъ: и приведетъ насъ не въ землю ханаанскую, но въ вѣчное Божіе царство, и подастъ намъ всякому свой жребій. О Іисусе, Сило спасенія нашего! настави мя на путь Твой, влецы мене за Собою, побѣжимъ.

\section{26. Пришлецъ или странникъ.}

Кто отъ дому и отечества своего отлучился и на чужой сторонѣ жительствуетъ, тотъ пришлецъ тамо и странникъ есть; какъ"=то Россіанинъ, находящійся въ Италіи или въ иной какой земли, есть тамо пришлецъ и странникъ: тако хрістіанинъ, отъ небеснаго отечества удаленный и въ многобѣдственномъ мірѣ семъ живущій, есть пришлецъ и странникъ. О семъ Апостолъ святый съ вѣрными глаголетъ: \textit{не имамы здѣ пребывающаго града, но грядущаго взыскуемъ}\footnote{Евр.~13,~14.}. И Давидъ святый исповѣдуетъ тое: \textit{яко пресельникъ азъ есмь у Тебе и пришлецъ, якоже вси отцы мои}\footnote{Пс.~38,~13.}. И паки молится: \textit{пришлецъ азъ есмь на земли; не скрый отъ мене заповѣдей Твоихъ}\footnote{118,~19.}. Странный человѣкъ, на чужой земли живучи, все тщаніе полагаетъ о томъ, чтобы ему тое сдѣлать и совершить, ради чего на чужую землю прибылъ: тако хрістіанинъ, словомъ Божіимъ позванный и святымъ крещеніемъ обновленный къ вѣчному животу, о томъ тщится, какъ бы вѣчнаго живота не лишиться, который здѣ въ мірѣ семъ или пріобрѣтается, или погубляется. Странный человѣкъ, на чужой земли живучи, съ немалымъ опасеніемъ обращается, потому что между незнаемыми людьми находится: тако хрістіанинъ, въ мірѣ семъ живучи, яко на чужой земли, всего опасается и бережется, то"=есть, духовъ злобы демоновъ, грѣха, прелести міра, злыхъ и безбожныхъ людей. Страннаго человѣка вси чуждаются и удаляются отъ него, яко не своего и чужеземца: тако истиннаго хрістіанина вси міролюбцы и сынове вѣка сего чуждаются, удаляются и ненавидятъ, яко не своего и имъ противнаго. О семъ глаголетъ Господь: \textit{аще бы отъ міра бысте были, міръ убо свое любилъ бы: якоже отъ міра нѣсте, но Азъ избралъ вы отъ міра; сего ради ненавидитъ васъ міръ}\footnote{Іоан.~15,~19.}. Море мертваго тѣла, какъ сказываютъ, не держитъ въ себѣ, но извергаетъ вонъ: тако міръ непостоянный, какъ море, благочестивую душу, яко міру умершую, изгоняетъ. Міролюбивый человѣкъ есть любезное чадо міру, якоже презиратель міра и похотей его прелестныхъ есть какъ врагъ. Странный человѣкъ ничего недвижимаго, то"=есть, ни домовъ, ни садовъ, ни инаго чего подобнаго симъ не заводитъ на чужой земли, кромѣ нужнаго, безъ чего прожить невозможно: тако истинному хрістіанину все въ мірѣ семъ недвижимо есть; все бо оставитъ въ мірѣ семъ, и самое тѣло. О семъ глаголетъ Апостолъ святый: \textit{ничтоже внесохомъ въ міръ сей, явѣ, яко ниже изнести что можемъ}\footnote{1~Тим.~6,~7.}. Сего ради истинный хрістіанинъ ничего въ мірѣ семъ не ищетъ, кромѣ нужнаго, со Апостоломъ глаголя: \textit{имѣюще пищу и одѣяніе, сими довольни будемъ}\footnote{6,~8.}. Странный человѣкъ движимыя вещи, какъ"=то деньги и товаръ, во отечество свое предпосылаетъ или везетъ: тако истинному хрістіанину движимыя вещи въ мірѣ семъ, которыи онъ съ собою взять можетъ и на оный вѣкъ перенести, суть добрыя дѣла; сія онъ тщится собрать здѣ въ мірѣ живучи, какъ духовный купецъ духовный товаръ, и въ небесное свое отечество принести и съ тѣмъ явиться и предстать небесному Отцу. О семъ увѣщаваетъ насъ Господь, хрістіанине: \textit{скрывайте себѣ сокровище на небеси, идѣже ни червь, ни тля тлитъ, идѣже татіе ни подкопываютъ, ни крадутъ}\footnote{Мѳ.~6,~20.}. Сынове вѣка сего пекутся о тѣлѣ смертномъ: но благочестивыя души пекутся о безсмертной душѣ. Сынове вѣка сего ищутъ временныхъ и земныхъ своихъ сокровищъ: но благочестивыя души къ вѣчнымъ и небеснымъ стремятся, и желаютъ тѣхъ благъ, \textit{ихже око не видѣ, и ухо не слыша, и на сердце человѣку не взыдоша}\footnote{1~Кор.~2,~9.}. На сіе сокровище невидимое и непостижимое вѣрою они взирая, все земное пренебрегаютъ. Сынове вѣка сего тщатся на земли прославитися: но истинныи хрістіане ищутъ славы на небеси, гдѣ отечество ихъ. Сынове вѣка сего украшаютъ тѣло различнымъ одѣяніемъ: но сынове царствія Божія украшаютъ безсмертную душу и облекаются, по увѣщанію Апостола, \textit{во утробы щедротъ, благость, смиренномудріе, кротость и долготерпѣніе}\footnote{Кол.~3,~12.}. И потому сынове вѣка сего суть несмысленни и безумни, яко ищутъ того, что въ себѣ ничто: сынове царствія Божія суть разумны и мудры, яко пекутся о томъ, что все блаженство вѣчное въ себѣ заключаетъ. Странному человѣку на чужой земли жить скучно: тако истинному хрістіанину въ мірѣ семъ жить скучно и скорбно; ему въ мірѣ семъ вездѣ ссылка, тюрьма и мѣсто изгнанія, яко удаленному отъ небеснаго отечества. \textit{Увы мнѣ}, глаголетъ Давидъ святый, \textit{яко пришельствіе мое продолжися}\footnote{Пс.~119,~5.}. Тако и прочіи святіи о семъ сѣтуютъ и воздыхаютъ. Странному человѣку хотя и скучно жить на чужой земли, однако живетъ ради нужды, ради которой отъ отечества своего отлучился: тако истинному хрістіанину хотя и скорбно жить въ мірѣ семъ, однакожъ, пока Богъ велитъ, живетъ и терпитъ тое странствованіе. У страннаго человѣка всегда во умѣ и памяти есть отечество и домъ свой, и желаетъ возвратитися во отечество свое; Іудеи, будучи въ Вавилонѣ, всегда имѣли въ мысли и памяти отечество свое Іерусалимъ, и всеусердно желали возвратитися въ свое отечество: тако истинныи хрістіане въ мірѣ семъ, какъ при рѣкахъ вавилонскихъ, сѣдятъ и плачутъ, поминая горній Іерусалимъ "--- отечество небесное, и къ нему съ воздыханіемъ плачевныя своя возводятъ очи, и желаютъ пріити туда. \textit{О семъ воздыхаемъ, въ жилище наше небесное облещися желающе}, воздыхаетъ Павелъ святый съ вѣрными\footnote{2~Кор.~5,~2.}. Сынове вѣка сего, пристрастившеся къ міру, міръ, какъ отечество и рай, себѣ имѣютъ, и потому не хотятъ отъ него отлучитися: но сынове царствія Божія, сердцемъ отъ міра отлучившеся и всякія скорби въ мірѣ терпяще, желаютъ къ оному отечеству пріити; хрістіанину бо истинному въ мірѣ семъ житіе не иное что есть, какъ всегдашнее страданіе и крестъ. Когда странный человѣкъ въ отечество и домъ свой возвратится, то домашніи, сосѣды и други его радуются о немъ, и привѣтствуютъ его благополучнымъ прибытіемъ: тако, когда хрістіанинъ, окончавъ свое въ мірѣ странствованіе, пріидетъ во отечество небесное, радуются о немъ вси ангели и вси святіи жители небесныи. Странный человѣкъ, пришедши во отечество и домъ свой, безопасно живетъ и упокоевается: тако хрістіанинъ, вшедши въ небесное отечество, упокоевается, живетъ безопасно и ничего не боится, радуется и веселится о блаженствѣ своемъ. Отсюду видишь, хрістіанине: 1)~Житіе наше въ мірѣ семъ не иное что есть, какъ странствованіе и пришельствіе, якоже Господь глаголетъ: \textit{яко пришельцы и пресельницы вы есте предо Мною}\footnote{Лев.~25,~33.}. 2)~Отечество наше истинное не здѣ, но на небеси, къ которому мы созданы, крещеніемъ обновлены и Словомъ Божіимъ позваны. 3)~Не должно намъ, яко позваннымъ къ небеснымъ благимъ, земныхъ искать и имъ прилѣпляться, кромѣ нужныхъ, какъ"=то: пищи, одежды, дому и прочаго. 4)~Человѣку хрістіанину, въ мірѣ живущему, ничего болѣе не желать, какъ вѣчнаго живота, да \textit{идѣже сокровище его будетъ, тамо и сердце его будетъ}\footnote{Мѳ.~6,~21.}. 5)~Кто хощетъ спастися, надобно тому сердцемъ отъ міра отлучиться, пока душею отъ міра не отъидетъ. 6)~Кто въ мірѣ семъ ищетъ богатѣть и прославиться, знакъ есть, что міръ, а не небо, за отечество свое имѣетъ, и потому заблуждаетъ, что въ день смерти своея уразумѣетъ.

\section{27. Гражданинъ.}

Видимъ въ мірѣ семъ, что человѣкъ, гдѣ ни живетъ и обращается, называется того града житель или гражданинъ, въ которомъ домъ свой имѣетъ, напр., московскій житель \textit{Москвичъ}, новгородскій житель \textit{Новгородецъ} и проч.: тако истинныи хрістіане, хотя въ мірѣ семъ и обращаются, однако же имѣютъ градъ въ небесномъ отечествѣ, \textit{его же художникъ и содѣтель Богъ}\footnote{Евр.~11,~10.}. И того града нарицаются граждане. Градъ сей есть Іерусалимъ небесный, который видѣлъ святый Іоаннъ апостолъ въ откровеніи своемъ: \textit{градъ} отъ \textit{злата чиста, подобенъ стклу чисту, и стогны града злато чисто, яко сткло пресвѣтло: градъ не требуяй солнца и луны, да свѣтятъ въ немъ, слава бо Божія просвѣти его, и свѣтильникъ ему Агнецъ}\footnote{Апок.~21,~18,~21 и 23.}. На стогнахъ его непрестанно поется пресладкая пѣснь: \textit{Аллилуіа}\footnote{19,~1,~3,~4,~6.}. Во градъ оный \textit{не имать внити всяко скверно, и творяй мерзость и лжу, но токмо написанные въ книгахъ животныхъ Агнца}\footnote{21,~27.}. Ибо \textit{внѣ псы и чародѣи, и любодѣи, и убійцы, и идолослужители, и всякъ любяй и творяй лжу}\footnote{22,~15.}. Сего прекраснаго и пресвѣтлаго града истинныи хрістіане называются и суть граждане, хотя и на земли странствуютъ; они тамо обители своя имѣютъ, уготованныя имъ отъ Іисуса Хріста, Искупителя своего; туда они душевныя свои очи и воздыханія возводятъ отъ странствованія своего. Понеже во градъ оный не внидетъ ничто скверно, какъ выше видѣли мы: \textit{очистимъ себе}, возлюбленный хрістіанине, \textit{отъ всякія скверны плоти и духа, творяще святыню въ страсѣ Божіи}, по увѣщанію апостольскому\footnote{2~Кор.~7,~1.}, да и мы пресвѣтлаго онаго града граждане будемъ, и, изшедше отъ міра сего, внити въ него сподобимся, благодатію Спаса нашего Іисуса Хріста, Ему же слава со Отцемъ и Святымъ Духомъ во вѣки, аминь.

\section{28. Обѣдъ или вечеря.}

Бываетъ въ мірѣ семъ, какъ видимъ, что знатный и богатый какой человѣкъ дѣлаетъ богатую и знатную вечерю или обѣдъ, и зоветъ многихъ людей на той обѣдъ: тако Богъ, Царь небесный, сотворилъ великую вѣчнаго блаженства вечерю, всякаго неизреченнаго утѣшенія, наслажденія, радости и веселія исполненную, и звалъ и зоветъ всѣхъ по Своему человѣколюбію "--- всѣхъ, говорю, славныхъ и безславныхъ, благородныхъ и худородныхъ, богатыхъ и нищихъ, мудрыхъ и неразумныхъ, и всякаго полу, званія и чина людей. Звалъ чрезъ пророковъ, звалъ чрезъ Единороднаго Сына Своего, звалъ чрезъ апостоловъ святыхъ; зоветъ и нынѣ чрезъ проповѣдниковъ слова Своего, пастырей и учителей: \textit{грядите, яко уже готова суть вся}\footnote{Лук.~14,~17.}. "--- Возлюбленный хрістіанине! Богу благодареніе, и мы недостойныи позваны отъ человѣколюбца Бога къ великой оной вечери. Пріимемъ убо благодарно сію Царя небеснаго къ намъ милость, и поспѣшимъ съ радостію на святую и пресладкую оную вечерю. Тамо мы увидимъ Царя небеснаго \textit{лицемъ къ лицу}\footnote{1~Кор.~12,~13.}. Увидимъ пламеноносныхъ слугъ Его, Ангелъ и Архангелъ; увидимъ ликъ патріарховъ, ликъ пророковъ, ликъ апостоловъ, святителей, ликъ мучениковъ, ликъ преподобныхъ отцевъ, и всѣхъ Богу угодившихъ, и со всѣми ими любезное возъимемъ дружество. Видимъ, что многіи, какъ прежде отреклися, такъ и нынѣ отрицаются на преславную оную вечерю ити, но, вмѣсто вѣчныхъ и небесныхъ благъ, временная и земная избираютъ, и сія онымъ, какъ тѣнь истинѣ, предпочитаютъ; и такъ сами себя недостойными творятъ вѣчнаго блаженства, и неблагодарными къ Царю небесному, такъ великому своему Благодѣтелю, показуются; почему и на праведный гнѣвъ Его подвизаютъ. \textit{Глаголю вамъ}, рече Господь, \textit{яко ни единъ мужей тѣхъ званныхъ вкуситъ Моея вечери: мнози бо суть звани, мало же избранныхъ}\footnote{Лук.~14,~24.}. Хрістіанине! вложимъ во уши наша и затвердимъ въ памяти нашей слово сіе Божіе: ни единъ \textit{мужей тѣхъ званныхъ вкуситъ Моея вечери}. Видимъ здѣ, какъ гнѣвается Господь на тѣхъ неблагодарныхъ людей, которыи къ вѣчному блаженству, какъ преславной и пресладкой вечери, словомъ Его позваны, но, презрѣвше великую сію Его благодать, обратилися къ снисканію временныхъ, чести, славы, богатства, угодія и прочихъ міра сего сокровищъ. \textit{Ни единъ мужей тѣхъ званныхъ вкуситъ Моея вечери}. Читаемъ въ книгахъ Моѵсеовыхъ Числъ, что когда сыны израилевы, изшедшіи изъ Египта, роптали: \textit{кто ны напитаетъ мясы? Помянухомъ рыбы, яже ядохомъ въ земли Египетстѣй туне, и огурцы, и дыни, лукъ, и червленый лукъ и чеснокъ. Услыша, Господь роптаніе ихъ, и разгнѣвася гнѣвомъ, и разгорѣся въ нихъ огнь отъ Господа}\footnote{Числ.~11,~4,~10 и 3.}. Хрістіане! ропщущіи израильтяне знаменуютъ тѣхъ хрістіанъ, которыи временная благая вѣчнымъ и небеснымъ, какъ мяса египетская и прочая снѣди израильтяне небесной маннѣ, предпочитаютъ; почему и гнѣвъ Божій на себе воздвигаютъ, и огнь не временный, но вѣчный, на нихъ возгорится. \textit{Глаголю вамъ: ни единъ мужей тѣхъ званныхъ вкуситъ Моея вечери}. Страшно слово сіе, хрістіанине, но истинно. Будетъ тое презирателямъ званія Божія. Убоимся убо мы, хрістіанине, и, послушавше зовущаго насъ Господа, \textit{горняя да мудрствуемъ, а не земная}\footnote{Кол.~3,~2.}. Безстыдно и страшно, когда царь земный зоветъ, не итить на вечерю его, какъ самъ знаешь: кольми паче не итить на вечерю, которую Богъ Царь небесный уготовалъ, и непрестанно чрезъ рабовъ Своихъ зоветъ. Послушаемъ убо спасительнаго голоса Его, и пойдемъ съ благодареніемъ и поспѣшностію на пресладкій пиръ оный. Вечеря уготована, вечеря преславная, вѣчная и пресладкая; раби Божіи, посланныи отъ Бога къ намъ, зовутъ насъ: \textit{грядите, яко уже готова суть вся}; двери отверсты; Царь небесный, человѣколюбивый и преблагій, зоветъ всѣхъ и ждетъ; входятъ съ благодареніемъ послушающіи гласа Его и пріобщаются небесныя тоя сладости. Хрістіанине! прилѣпимся и мы святой сей дружинѣ и послѣдуемъ имъ, и введетъ насъ Царь въ чертогъ Свой. Аминь!

\subsection{О томжде.}

Въ мірѣ семъ бываетъ, что славныи и богатыи на вечери свои зовутъ по большей части славныхъ и богатыхъ, подобныхъ себѣ. Царь небесный, человѣколюбивый Богъ, не тако. Онъ, по Своему человѣколюбію, никѣмъ не гнушается, но на небесную Свою вечерю зоветъ, не взирая на лица, всѣхъ равно, богатыхъ и нищихъ, славныхъ и подлыхъ, "--- словомъ, всякаго человѣка, мужеска пола и женска, и всѣмъ отверзаетъ двери. Однакожъ по большой части спасительнаго того званія слушаютъ и идутъ на преславную вечерю Его убогіи, нищія, бѣдныи и подлыи; а богатыи и славныи по большей части отрицаются. Ибо не хотятъ оставить гордости, пышности и роскошей своихъ; они хотятъ здѣ въ мірѣ семъ веселиться и царствовать, и потому великую оную вечерю презираютъ; благая міра сего видятъ, но вѣчныхъ благъ не видятъ. Пишется о израильтянахъ: \textit{уничижиша землю желанную, не яша вѣры словеси Его}\footnote{Пс.~105,~24.}. Тако богатыи и славныи, прилѣпившеся къ благимъ міра сего, уничижаютъ великую оную вечерю и не вѣруютъ Господу, зовущему ихъ на вечерю оную. Почему и написано отъ Апостола: \textit{видите званіе ваше, братіе, яко не мнози ли премудра по плоти? не мнози ли сильны? не мнози ли благородни? Но буяя міра избра Богъ, да премудрыя посрамитъ; и немощная міра избра Богъ, да посрамитъ крѣпкая; и худородная міра и уничиженная избра Богъ, и не сущая, да сущая упразднитъ. Яко да не похвалится всяка плоть предъ Богомъ}\footnote{1~Кор.~1,~26--29.}. Охотнѣе слушаютъ нищіи и бѣдныи, и худородныи и презрѣнныи міра сего, зовущаго гласа Господня, и идутъ на вечерю оную, яко благихъ міра сего мало что имѣютъ. Любовь чести, славы, богатства, сластей и роскошей міра сего ослѣпляетъ душевныя очи и не допущаетъ видѣти вѣчныхъ благъ; а человѣкъ, чего не видитъ, того не желаетъ. О, когда бы человѣкъ увидѣлъ будущая благая стремился бы къ нимъ не иначе, какъ елень на источники водные! Но у всѣхъ любителей міра сего тма, аки покрывало нѣкакое, лежитъ на очахъ душевныхъ: и потому не видятъ оныхъ благъ. А кто чистымъ сердцемъ ко Господу \textit{обратится}: тогда \textit{покрывало} оное \textit{отнимается}, и окомъ вѣры видитъ оная благая и со усердіемъ ищетъ ихъ\footnote{2~Кор.~3,~16 и 17.}. "--- Хрістіанине, богатый и нищій, славный и убогій! обратимся отъ міра ко Господу, и отверзутся очи наши, и спѣшно пойдемъ на вечерю оную; повѣримъ слову Божію, которымъ зоветъ насъ. Къ великимъ, небеснымъ, вѣчнымъ и неизреченнымъ благимъ позваны мы отъ Бога: почто къ тѣмъ прилагаемъ сердце наше, которыя какъ ничто суть въ себѣ, и вскорѣ ихъ оставимъ? Не напрасно Апостолъ увѣщеваетъ насъ: \textit{не любите міра, ни яже въ мірѣ}, и проч.\footnote{1~Іоан.~2,~15.} И паки: \textit{аще воскреснусте со Хрістомъ: вышнихъ ищите, идѣже есть Хрістосъ, одесную Бога сѣдя; горняя мудрствуйте, а не земная}\footnote{Кол.~3,~1--2.}. Понеже и любовь міра сего запинаетъ ноги наша и не попускаетъ ити на преславную оную вечерю.—Хрістіанине! оставь благая міра сего, и тамо вся и несравненно лучшая получимъ благая, \textit{ихже око не видѣ и ухо не слыша, и на сердце человѣку не взыдоша}\footnote{2~Кор. 2--9.}. Буди твердо увѣренъ о благихъ оныхъ: и неотмѣнно, истину тебѣ говорю, презрѣвши міръ, со всякимъ усердіемъ оныхъ будешь искать.

\subsection{О томжде.}

Бываетъ въ мірѣ семъ, что когда званныи на вечерю вси соберутся; тогда звавшій ихъ хозяинъ вводитъ ихъ въ палату украшенную, для вечери уготованную, и тако садятся вси на своихъ мѣстахъ: тако будетъ и званнымъ на вечерю небесную и пришедшимъ къ той. Тогда Царь небесный, послушавшимъ званія Его и пришедшимъ къ Нему на вечерю, речетъ: \textit{пріидите благословенніи Отца Моего, наслѣдуйте уготованное вамъ царствіе отъ сложенія міра}\footnote{Мѳ.~25,~34.}. И пойдутъ съ торжествомъ и веселіемъ въ преславную, небесную, нерукотворенную, вѣчную и царскую Его палату, \textit{и возлягутъ со Авраамомъ и Исаакомъ и Іаковомъ во царствіи небесномъ}\footnote{8,~11.}. Тогда исполнится пророческое слово: \textit{собранніи Господемъ обратятся, и пріидутъ въ Сіонъ съ радостію, и радость вѣчная надъ главою ихъ; надъ главою бо ихъ хвала и веселіе, и радость пріиметъ я: отбѣже болѣзнь, печаль и воздыханіе}\footnote{Ис.~35,~10.}. "--- Хрістіанине! послушаемъ нынѣ гласа Господня и обратимся къ Господу всѣмъ сердцемъ и покаемся, да и тогда услышимъ пресладкій гласъ Его: \textit{пріидите благословенніи}, и проч.

\subsection{О томжде.}

На знатной міра сего вечери сѣдящіи и пирующіи одѣяны бываютъ въ свѣтлыя и красныя одежды: тако будетъ и на преславной оной великолѣпной небесной вечери. Тамо возлежащіи одѣяны будутъ въ ризу спасенія и одеждою веселія, облечени будутъ \textit{въ ризы позлащенныя}\footnote{Пс.~44,~10 и 14.}, облечени въ вѵссонъ чистъ и свѣтелъ. \textit{И дано бысть ей}, то"=есть церкви, \textit{облещися въ вѵссонъ чистъ и свѣтелъ: вѵссонъ бо оправданія святыхъ есть}, глаголетъ Іоаннъ святый во Откровеніи своемъ\footnote{19,~8.}. Сіе показалъ и Хрістосъ Спаситель нашъ прославленіемъ святаго тѣла Своего на горѣ Ѳаворстѣй, гдѣ \textit{просвѣтилося лице Его яко солнце; ризы же Его были бѣлы яко свѣтъ}\footnote{Мѳ.~17,~2.}. Прославленному Хрістову тѣлу сообразны будутъ тамо и святыхъ Божіихъ тѣлеса; о семъ увѣряетъ насъ святый Апостолъ: \textit{наше житіе на небесѣхъ есть, отонудуже и Спасителя ждемъ, Господа нашего Іисуса Хріста, Иже преобразитъ тѣло смиренія нашего, яко быти сему сообразну тѣлу славы Его, по дѣйству еже возмогати Ему и покорити Себѣ всяческая}\footnote{Филип.~3,~20--21.}. И Самъ Хрістосъ глаголетъ: \textit{тогда праведницы просвѣтятся, яко солнце, во царствіи Отца ихъ}\footnote{Мѳ.~13,~43.}. Хрістіанине! помолимся и мы свѣтодавцу Хрісту, да и намъ подастъ свѣтлую оправданія ризу, и возляжемъ со избранными Его на вечери оной. \textit{Чертогъ Твой вижду, Спасе мой, украшенный, и одежды не имамъ, да вниду въ онь: просвѣти одѣяніе души моея, Свѣтодавче, и спаси мя}.

\subsection{О томжде.}

На знатной міра сего вечери всякое добро бываетъ къ утѣшенію и услажденію сѣдящихъ и пирующихъ. Подобно будетъ и въ небесной вечери. Тамо всякое добро, но несравненно лучшее паче земнаго, будетъ. Тамо люди Божіи святіи насытятся всѣхъ благихъ небесныхъ, которыя уготовалъ имъ благій и человѣколюбивый Господь. \textit{Не взалчутъ ктому, ниже вжаждутъ, не имать же пасти на нихъ солнце, ниже всякъ зной; яко Агнецъ, Иже посредѣ престола, упасетъ я, и наставитъ ихъ на животныя источники водъ, и отъиметъ Богъ всяку слезу отъ очію ихъ}\footnote{Апок.~7,~16--17.}. \textit{Работающіи Господу яти будутъ, пити будутъ, радоватися будутъ, веселитися будутъ, въ веселіи сердца}\footnote{Ис.~65,~13--14.}. \textit{Упіются отъ тука дому Твоего; и потокомъ сладости Твоея напоиши я. Яко у Тебе источникъ живота, во свѣтѣ Твоемъ узримъ свѣтъ}\footnote{Пс.~35,~9--10.}. \textit{Сія глаголетъ Господь: се Азъ уклоняю на ня, аки рѣку мира и аки потокъ наводняемый, славу языковъ; дѣти ихъ на рамена взяты будутъ, и на колѣну утѣшатся. Якоже аще кого мати утѣшаетъ, тако и Азъ утѣшу вы, и во Іерусалимѣ утѣшитеся. И узрите, и возрадуется сердце ваше}\footnote{Ис.~66,~12--14.}. \textit{Видимъ нынѣ якоже зерцаломъ въ гаданіи, тогда же лицемъ къ лицу}\footnote{1~Кор.~13,~12.}. \textit{Ихже око не видѣ, и ухо не слыша, и на сердце человѣку не взыдоша, яже уготова Богъ любящимъ Его}, и проч.\footnote{2,~9.} "--- Блаженство убо тамо будетъ сіе: 1)~Увидимъ Бога лицемъ къ лицу; отъ чего неизреченное утѣшеніе, веселіе, радость и восклицаніе сердечное произыдетъ. Увидимъ лицо Божіе и отъ радости будемъ восклицати. 2)~Увидимъ Господа и Искупителя нашего Іисуса Хріста въ Божественной Его славѣ, Того, Который за насъ такъ страшная пострадалъ и поносною смертію умеръ, и тако насъ отъ смерти избавилъ. 3)~Насладимся тогда всѣхъ даровъ Святаго Духа, яко Источника животворящаго. 4)~Будемъ имѣть любезное дружество со святыми ангелами и всѣми святыми, отъ созданія міра угодившими Богу. 5)~Будемъ торжествовать надъ всѣми врагами нашими. \textit{Гдѣ ти, смерте, жало? гдѣ ти, аде, побѣда}\footnote{15,~55.}? \textit{Богу благодареніе, давшему намъ побѣду Господемъ нашимъ Іисусъ Хрістомъ}\footnote{Тамъ же.}. "--- Хрістіанине! Богъ милосердый еще зоветъ всѣхъ: полно уже гоняться за временными и земными; поспѣшай скорѣе на небесную оную вечерю, гдѣ Богъ видится лицемъ въ лицу, и отъ Того, какъ отъ Источника животворящаго и приснотекущаго, проистекаетъ всякое утѣшеніе и наслажденіе и всерадостное восклицаніе; поспѣшай, поспѣшай, пока еще двери не затворены, да не поздно пришедши, \textit{внѣ стояти} будеши \textit{и} безполезно \textit{ударяти въ двери глаголя: Господи, Господи, отверзи намъ}\footnote{Лук.~13,~25.}.

\subsection{О томжде.}

Бываетъ на вечери міра сего, что хозяинъ пришедшимъ гостямъ говоритъ: вотъ вамъ благая, предложенная на трапезѣ моей! идите, пійте и веселитеся! Тако человѣколюбивый Господь возглаголетъ послушавшимъ званія Его и пришедшимъ на небесную Его вечерю: вотъ вамъ благая, которая уготовахъ вамъ, которая обѣщахъ вамъ, которыхъ вѣрою и надеждою ожидали вы отъ Мене, которыхъ око не видѣло и ухо не слышало, и на сердце человѣку не взыдоша! Вотъ вамъ благая Моя, обѣщанная вамъ! Обѣщалъ Я вамъ воскресеніе мертвыхъ вашихъ тѣлесъ: вотъ видите тое! воскреснусте изъ мертвыхъ. Обѣщалъ Я вамъ тѣло духовное, нетлѣнное и безсмертное: вотъ имѣете тое! Обѣщалъ Я вамъ тѣло прославленное, чистое, свѣтлое и сіяющее: вотъ сіяете, яко солнце и яко звѣзды небесныя\footnote{1~Кор.~1,~15,~42--46.}! Обѣщалъ Я вамъ вѣчный животъ: вотъ имѣете животъ вѣчный! живите во вѣки! Обѣщалъ Я вамъ царство небесное: вотъ даю тое вамъ! наслѣдуйте тое и царствуйте во вѣки! Обѣщалъ Я вамъ явить Лице Мое: вотъ вамъ Лице Мое! зрите и веселитеся! Обѣщалъ Я вамъ вѣнецъ живота, вѣнецъ неувядаемый: вотъ вамъ вѣнецъ той! Обѣщалъ Я вамъ честь и славу: вотъ вамъ слава, яко чадамъ Моимъ! Обѣщалъ Я утѣшить васъ: вотъ утѣшаю васъ, яко мати утѣшаетъ чада своя! Обѣщалъ Я вамъ дать миръ непрестанный въ сердцахъ вашихъ, аки рѣку: \textit{миръ Мой даю вамъ}. Обѣщалъ Я отнять отъ васъ печаль, болѣзнь и воздыханіе: вотъ \textit{отбѣже болѣзнь}, печаль и воздыханіе отъ сердецъ вашихъ! Обѣщалъ Я вамъ пищу и питіе, сладости и радости вѣчныя: вотъ вамъ тое утѣшеніе и наслажденіе; \textit{ядите ближніи, и пійте, и упійтеся}\footnote{Пѣсн. пѣсн.~5,~1.}. \textit{Въ путь узкій хождшіи прискорбный, вси въ житіи крестъ, яко яремъ, вземшіи и Мнѣ послѣдовавшіи вѣрою! пріидите, насладитеся, ихже уготовахъ вамъ почестей и вѣнцевъ небесныхъ}, глаголетъ Господь. Тогда исполнится написанное: \textit{якоже слышахомъ, тако и видѣхомъ во градѣ Господа силъ, во градѣ Бога нашего: Богъ основа его въ вѣкъ. Пріяхомъ, Боже милость Твою посредѣ людей Твоихъ}\footnote{Пс.~47,~9--10.}. Хрістіанине, упившійся любовію міра сего! истрезвись, пока еще время не ушло, и поищи небесныхъ благъ, которыя обѣщалъ Богъ любящимъ Его, да не потомъ пожелаеши, но не получиши.

\subsection{О томжде.}

На знатной міра сего вечери бываетъ музыка и пѣніе ради увеселенія пирующихъ гостей. Пѣніе будетъ и на небесной оной вечери, но несравненно лучшее. Тамо услышится пресладкое пѣніе святыхъ ангелъ, хвалящихъ и поющихъ Господа Создателя своего: \textit{святъ, святъ, святъ Господь Саваоѳъ: исполнь вся земля славы Его}\footnote{Ис.~6,~3.}! И самыи святіи Божіи человѣцы, живущіи въ дому Божіемъ, въ вѣки вѣковъ восхвалятъ Господа и совокупно съ небесными силами воспоютъ Святую Троицу. О прекрасная музыка! о пресладкія пѣсни! о дивное согласіе ангеловъ и человѣковъ, подобныхъ ангеламъ, поющихъ Господа, Создателя своего! \textit{Благословенъ Господь Богъ Израилевъ, яко посѣти, и сотвори избавленіе людемъ Своимъ; и воздвиже рогъ спасенія намъ въ дому Давида отрока Своего}\footnote{Лук.~1,~68--69.}! \textit{Благослови душе моя Господа}\footnote{Пс.~102,~1--2.}! \textit{Благословенъ еси Господи, Боже отецъ нашихъ, и препѣтый и превозносимый во вся вѣки; и благословенно имя славы Твоея святое, препѣтое и превозносимое во вѣки}\footnote{Дан.~3,~52.}. Поспѣшимъ, возлюбленный хрістіанине, поспѣшимъ на великолѣпную и дивную оную вечерю, и услышимъ тамо пресладкую небесную музыку, и сами не вѣрою уже, якоже нынѣ, но лицемъ къ лицу восхвалимъ Господа, Создателя и Искупителя нашего. \textit{Господи Боже силъ! услыши молитву мою: внуши, Боже Іаковль! Защитниче нашъ, виждь, Боже, и призри на лице Хріста Твоего! Яко лучше день единъ во дворѣхъ Твоихъ паче тысящъ: изволихъ приметатися въ дому Бога моего паче, неже жити ми въ селеніихъ грѣшничихъ}\footnote{Пс.~83,~9--11.}.

\subsection{О томжде.}

Въ мірѣ семъ на вечери гости пришедшіи, насытившеся, болѣе уже не хотятъ ясти и не ядятъ. На вечери оной небесной не тако будетъ. Будутъ ясти и пити, но безъ сытости; будутъ вечеряти, но всегда съ желаніемъ. Непрестанная нѣкая, но пресладкая алчба и жажда въ нихъ будетъ. Будутъ насыщатися небесныя пищи, но всегда ея будутъ хотѣть. Будутъ видѣти Бога лицемъ къ лицу и Того сладчайшимъ лицезрѣніемъ насыщатися, но никогда не насытятся, но всегда всерадостнаго и пресладкаго того насыщенія будутъ желать; и чимъ болѣе того будутъ лицезрѣнія Божія насыщатися, тѣмъ болѣе и болѣе пожелаютъ того. Божіе бо лице видѣти есть радость паче всякія радости, утѣшеніе паче всякаго утѣшенія, сладость паче всякія сладости. Се есть пища и питіе святыхъ ангелъ и избранныхъ Божіихъ! Отсюду, яко отъ источника приснотекущаго и животворящаго, рѣка будетъ "--- непрестанное желаніе къ зрѣнію лица Божія, и отъ того непрестанная радость, веселіе, утѣшеніе, наслажденіе, восклицаніе и сердечное нѣкое играніе. И какъ рѣка течетъ безпрерывно и непрестанно: тако непрестанно и непрерывно небесныя вечери блаженство потечетъ. Се есть, хрістіанине, велія небесная вечеря, къ которой отъ Бога позваны мы! Се есть рай Божій исполненный радостей, и сладостей, имущій посредѣ древо живота, отъ котораго ядущіи никогда не умрутъ, но во вѣки живи будутъ! Се есть животъ вѣчный и блаженная вѣчность! "--- Хрістіанине! поищемъ и мы нынѣ утѣшенія онаго покаяніемъ и вѣрою; поищемъ, пока ищется и обрѣтается. Здѣ оно только ищется и обрѣтается.

\subsection{О томжде.}

На богатой вечери въ мірѣ семъ бываетъ, что поставляются свѣтильники и свѣщи горящія. На небесной оной вечери не потребенъ будетъ таковый свѣтъ, ниже отъ солнца, ниже отъ луны, ибо и нощи не будетъ тамо: слава бо Господня просвѣтитъ вся. \textit{Градъ не требуяй солнца и луны, да свѣтятъ въ немъ: слава бо Божія просвѣти его, и свѣтильникъ ему Агнецъ}\footnote{Апок.~21,~23.}. \textit{И нощи не будетъ тамо, и не потребуютъ свѣта отъ свѣтильника, ни свѣта солнечнаго: яко Господь Богъ просвѣщаетъ я}\footnote{22,~5.}. \textit{И не будетъ тебѣ} (Іерусалиме небесный) \textit{ктому солнце во свѣтъ дне, ниже восходъ луны просвѣтитъ твою нощь; но будетъ тебѣ Господь свѣтъ вѣчный, и Богъ слава твоя. Не зайдетъ бо солнце тебѣ, и луна не оскудѣетъ тебѣ: будетъ бо Господь тебѣ Свѣтъ вѣчный}\footnote{Ис.~60,~19--20.}. "--- Господи, во свѣтѣ Твоемъ узримъ свѣтъ! Пробави милость Твою вѣдущимъ Тя. Буди, Господи, милость Твоя на насъ, якоже уповахомъ на Тя.

\subsection{О томжде.}

Въ мірѣ семъ вечеря временно бываетъ; гости бо, пришедшіи на вечерю, удовольствовавшеся отъ хозяина, въ домы своя расходятся, и тако какъ вечеря, такъ и утѣшеніе ея престаетъ: всякое бо временное утѣшеніе и услажденіе, какъ дымъ, вскорѣ исчезаетъ. Не тако будетъ на великой оной вечери. Она, единожды наченшися, никогда не скончается, и потому какъ непрестанно, такъ и безъ конца будетъ. На оной вечери возлежащіи безъ конца будутъ видѣти лице Божіе, и сладчайшимъ Его лицезрѣніемъ безъ сытости во вѣки вѣковъ насыщатися; безъ конца и безъ сытости сію небесную пищу ясти и сіе пресладкое питіе пити будутъ; и якоже царствію Хрістову не будетъ конца, тако и святіи Божіи человѣцы съ Нимъ, яко съ главою уды, царствовать будутъ во вѣки вѣковъ. \textit{Блажени живущіи въ дому Твоемъ: въ вѣки вѣковъ восхвалятъ Тя, Царю мой и Боже мой}\footnote{Пс.~83,~5 и 4.}. \textit{Помяни насъ Господи во благоволеніи людей Твоихъ, посѣти насъ спасеніемъ Твоимъ, видѣти во благости избранныя Твоя, возвеселитися въ веселіи языка Твоего, хвалитися съ достояніемъ Твоимъ}\footnote{105,~4--5.}. \textit{Обратися душе моя въ покой твой, яко Господь благодѣйствова тя: яко изъятъ душу мою отъ смерти, очи мои отъ слезъ, и нозѣ мои отъ поползновенія. Благоугожду предъ Господемъ во странѣ живыхъ}\footnote{114,~6--8.}. Господи Іисусе! \textit{рцы души моей: спасеніе твое есмь Азъ}\footnote{34,~3.}. \textit{Аминь глаголю тебѣ: со Мною будеши въ раи}\footnote{Лук.~23,~43.}. Буди, буди! аминь.

\section{29. Завѣса или покрывало.}

Что завѣсою завѣшено, или покрываломъ покрыто, того не видимъ. Тако не видимъ солнца, облаками закрытаго; не видимъ лица, платомъ или одеждою закрытаго; не видимъ вещей, за завѣсою лежащихъ, и проч. Тако у всякаго грѣшника нераскаяннаго, у всякаго блудника и прелюбодѣя, и всякаго хищника, татя и мздоимца, у всякаго лживаго и обманщика, словомъ, у всякаго безсовѣстнаго человѣка предъ глазами завѣса нѣкая виситъ, и на душевныхъ очахъ, какъ покрывало нѣкое, лежитъ и не допущаетъ его видѣти, какихъ онъ благъ лишается, и какъ въ горькую погибель идетъ. Идетъ въ ровъ, и не видитъ рва; и впадетъ въ него, аще не осмотрится. Сію завѣсу и покрывало сатана, врагъ человѣческій, содѣловаетъ тѣмъ людемъ, которыи неосторожно живутъ, не внимаютъ Слову Божію, яко свѣтильнику ногамъ нашимъ, и оставляютъ молитву, безъ чего всякъ человѣкъ слѣпъ есть; содѣловаетъ, да не усмотрятъ слѣдующія имъ погибели. Отнимается сія завѣса или покрывало, когда человѣкъ истиннымъ сердцемъ обратится ко Господу. Тогда онъ увидитъ бѣдствіе и погибель свою; тогда уразумѣетъ, гдѣ былъ, како заблуждалъ; тогда начнетъ воздыхать, плакать и рыдать; тогда онъ не иначе, какъ бы отъ сна воставши, начнетъ ходить, какъ прежде ходилъ. \textit{Сего ради глаголетъ: востани спяй, и воскресни отъ мертвыхъ, и освѣтитъ тя Хрістосъ}\footnote{Еф.~5,~14.}. "--- \textit{Господи Боже силъ, обрати насъ, и просвѣти Лице Твое, и спасемся}\footnote{Пс.~79,~4.}.

\section{30. Глухій.}

Имѣетъ тѣло слухъ, имѣетъ и душа свой слухъ. Не всякое тѣло имѣетъ отверстый слухъ: не всякая и душа. Богъ душѣ глаголетъ: \textit{не убій, не укради, не прелюбодѣйствуй, уклонися отъ зла и сотвори благо}, и проч. И которая душа отверстый слухъ свой имѣетъ; слышитъ и слушаетъ Бога глаголющаго и творитъ, что повелѣваетъ Богъ. Невозможно бо душѣ, воистину невозможно не послушать Бога и не творить, что Онъ повелѣваетъ, аще уши свои отверстыя имѣетъ. Царя земнаго и низшую власть слушаетъ человѣкъ и творитъ, что велитъ: Бога ли не послушаетъ душа глаголющаго, когда отверстый слухъ имѣетъ? Ей! со всякимъ усердіемъ и сладостію послушаетъ и возглаголетъ ему: \textit{готово сердце мое, Боже, г}о\textit{тово сердце мое}\footnote{Пс.~107,~2.}. "--- Отверстый слухъ души своея имѣлъ святый патріархъ Авраамъ, которому отъ Бога сказано: \textit{изыди отъ земли твоея, и отъ рода твоего, и отъ дому отца твоего, и иди въ землю, юже ти покажу}, и проч. \textit{И иде Авраамъ, якоже глагола ему Господь}\footnote{Быт.~12,~1 и 4.}. Отверстый слухъ имѣли вси тіи, которымъ сказано отъ Хріста: \textit{грядите по Мнѣ; и по Немъ идоша}\footnote{Мѳ.~4,~19--22.}. Отверстый слухъ души и всякъ тотъ имѣетъ, кто со усердіемъ тщится заповѣди Божіи исполнять. Не напрасно Хрістосъ Господь глаголетъ во Евангеліи: \textit{имѣяй уши слышати, да слышитъ}. Вси имѣютъ уши у себе, но не вси имѣютъ уши слышати. Чрезъ уши убо \textit{слышати}, разумѣются уши души отверстыи. Никакой пользы не дѣлаютъ намъ уши тѣлесныя, когда уши душевныя затворены. Тѣло имѣетъ свою глухоту, когда уши его заключены: тако и душа имѣетъ свою глухоту, когда слухъ у нея заключенъ. Тѣлесно глухій не слышитъ, что ему говоритъ, приказуетъ и обѣщаетъ человѣкъ, ибо не можетъ слышати: тако душевно глухій, то"=есть тотъ, который слухъ души своея заключенъ имѣетъ, не слышитъ, что ему Богъ говоритъ, приказуетъ и обѣщаетъ; како бо можетъ слышати безъ открытаго слуха? У всякаго блудника и прелюбодѣя, у всякаго хищника, татя, мздоимца, лихоимца, у всякаго сребролюбца, сластолюбца, словомъ, у всякаго беззаконнующаго заключены уши душевныя, и потому душа глуха. Сколько бо таковые слышатъ слово Божіе; но не слушаютъ Бога, и не исправляются. Слышатъ они, и не слышатъ; слышатъ тѣлесными, но не слышатъ душевными ушами. И како могутъ ими слышати, когда ихъ затворенныхъ имѣютъ? Сколько люди слышатъ Хрістово слово: \textit{пріидите ко Мнѣ вси труждающіися и обремененніи, и Азъ упокою вы}\footnote{Мѳ.~11,~28.}: но не хотятъ ко Хрісту пріити и покой имѣть; держатся за міръ, какъ любимое свое сокровище, хотя и безпокойствуетъ и погубляетъ ихъ. Се есть глухота душевная! \textit{Се стою при дверехъ и толку: аще кто услышитъ гласъ Мой, и отверзетъ двери, вниду къ нему, и вечеряю съ нимъ, и той со Мною}, глаголетъ Хрістосъ Господь\footnote{Апок.~3,~20.}. Истинно и неложно слово сіе! За всѣхъ кровь Свою изліялъ милосердый Іисусъ, всѣмъ хощетъ спастися, всѣхъ и всякаго посѣщаетъ, у всякаго при домѣ сердечномъ стоитъ и толкаетъ въ двери, и хощетъ внити и сладчайшую сотворити вечерю; но глухая душа не слышитъ пресладкаго гласа сего. Какая бы душа не отверзла дверей дома своего, и такъ великаго, такъ дорогаго, такъ вожделѣннаго гостя съ охотою и радостію не приняла, аще бы услышала гласъ Его? Гость сей нашей пищи и питія не требуетъ; Свою Онъ намъ поставляетъ трапезу, Своею небесною пищею и Своимъ сладкимъ питіемъ учреждаетъ насъ; но то бѣда, что бѣдная душа не слышитъ гласа Іисусова, потому и дверей не отверзаетъ Ему, и посѣщенія Его и пресладкаго учрежденія лишается. Бѣдственна есть глухота тѣлесная, какъ самъ знаешь, человѣче: но далеко бѣдственнѣйшая есть душевная глухота! Никакого бо Божія слова не допуститъ до души дойти, безъ котораго все злополучно. "--- О, милосердый Іисусе! отверзи уши душъ нашихъ, да услышимъ святое слово Твое! Призри съ небесе святаго Своего, и рцы намъ: \textit{Еффаѳа, еже есть, разверзися}\footnote{Марк.~7,~34.}!

\subsection{О томжде.}

Тѣлесно глухій ко всякому слову, худому и доброму, заключенный слухъ имѣетъ: но душевно глухій не тако. Онъ ко всему худому, къ клеветамъ, къ негоднымъ баснямъ, къ сквернымъ и соблазнительнымъ пѣснямъ, къ шептамъ діавольскимъ, къ повѣстямъ о прихотяхъ плотскихъ и мірскихъ, отверстый слухъ имѣетъ: напротивъ того, къ слышанію Божія слова, чудныхъ дѣлъ Его и похвалы Его заключенный слухъ имѣетъ. Какъ кто скажетъ ему: вотъ такой"=то человѣкъ такимъ и такимъ способомъ довольно имѣнія себѣ нажилъ; такой"=то воеводами, такой"=то секретарь столько тысящъ денегъ собралъ, и столько душъ крестьянъ купилъ: тотчасъ и самъ внутрь себе начинаетъ думать, какъ бы и ему тое достать. Когда повѣствуется, что такія"=то матеріи и такія, такія"=то кареты и такія, такія"=то вина и такія, и прочая, прихотямъ человѣческимъ служащая, продаются: тотчасъ думаетъ, какъ бы тыя видѣть и у себя въ домѣ имѣть. А когда проповѣдуется Божіе слово, и глаголется ему: покайся, исправь себе, начни новое хрістіанское житіе, бойся Бога и праведнаго суда Его, берегись грѣха, люби добродѣтель и тщись творить ее, и проч.: то такъ тому онъ внимаетъ, какъ глухій пѣсни. Воистину великая есть душевная глухота! Что страшнѣе, какъ вѣчная погибель? и что вожделѣннѣе, какъ вѣчный животъ? Богъ въ Писаніи Своемъ говоритъ и чрезъ проповѣдниковъ, рабовъ Своихъ, аки гремитъ: бѣдная душа! погибнеши, погибнеши во вѣки, аще не покаешися. \textit{Одождитъ на грѣшники сѣти: огнь и жупелъ и духъ буренъ, часть чаши ихъ. Яко праведенъ Господь, и правды возлюби, правоты видѣ лице Его}\footnote{Пс.~10,~6 и 7.}. \textit{Страшливымъ и невѣрнымъ, и сквернымъ и убійцамъ, и блудъ творящимъ, и чары творящимъ, идоложерцемъ и всѣмъ лживымъ, часть имъ въ езерѣ горящемъ огнемъ и жупеломъ, еже есть смерть вторая}\footnote{Апок.~21,~8.}. Но бѣдная душа сильнаго того Божія гласа не слышитъ и пребываетъ безъ покаянія, какъ пребывала. Богъ зоветъ: обратися ко Мнѣ, душа, и спасешися, и вѣчно жива будеши; но не слышитъ душа Божія званія, и идетъ и стремится въ погибель, какъ стремилась; не слышитъ, понеже отверстаго слуха не имѣетъ. Страшенъ гласъ: во вѣки погибнеши, аще не покаешися. Любезный и вожделѣнный гласъ: во вѣки спасена будеши, аще покаешися. О, когда бы услышала душа гласъ сей: сильно бы ужаснулася и вострепетала бы, и избрала бы истинное покаяніе, дабы убѣжать отъ погибели и получить вѣчное спасеніе! Сего ради всякому хрістіанину должно тщиться и молиться о томъ, дабы внутренній души слухъ отверстъ былъ. Отсюду бо начало какъ обращенія и покаянія истиннаго, такъ и спасенія бываетъ. О, бѣдный хрістіанине, хрістіанине глухій! пробудись, пожалуй пробудись и понуди себе, падай предъ Іисусомъ, воздыхай и молись Ему со усердіемъ, да силою своею Божественною отверзетъ слухъ души твоея, и тогда съ охотою и радостію послушаеши гласа Господня. О, Іисусе, сило и надеждо нашего спасенія, отверзаяй очи слѣпымъ и слухъ глухимъ! отверзи слухъ душъ нашихъ, да услышимъ Божественное слово Твое; скажи и нашей грѣшной душѣ: \textit{Еффаѳа, еже есть, разверзися}; и разверзется слухъ ея. Знаки душевныя глухоты: 1)~Когда хрістіане слышатъ слово Божіе, и не исправляются. 2)~Любовь міра сего, то есть, чести, славы, богатства, роскошей и сластей. Сію глухоту сатана шептаніемъ своимъ содѣловаетъ, дабы люди не могли слышать слово Божіе, и такъ бы погибли.

\subsection{О томжде.}

Когда въ храминѣ различный гласъ, шумъ, крикъ и мятежъ бываетъ, тогда, что ни говорится человѣку, не слышитъ; ибо шумъ тотъ препятствуетъ ему и заключаетъ слухъ его: тако и въ душѣ бываетъ. Когда ее различный мірскихъ похотей шумъ и мятежъ безпокойствуетъ, тогда она не можетъ слышати слова Божія; ибо мірскія похоти, которыя \textit{воюютъ на душу}\footnote{1~Петр.~2,~11.}, ее окружающія, не допущаютъ до нее пріити Божію слову, и отражаютъ и отгоняютъ его отъ нея. Что въ такомъ шумѣ гласъ слова Божія успѣетъ? Какъ глухому музыка: тако таковымъ шумомъ объятой душѣ Божіе слово. Слышитъ таковый часто проповѣдь Божія слова и похваляетъ ее, но пользы оттуду никакой не получаетъ; отходитъ отъ проповѣди таковымъ же, каковымъ и пришелъ на проповѣдь, и когда бы не горшимъ. Ибо слышанное Божіе слово въ большій вредъ обращается тому, кто слышитъ его, но не исправляется отъ него. Тако книжники и фарисеи часто слышали слово Божіе отъ Хріста и не исправлялись, чего ради въ горшее успѣвали. Сего ради сатана, врагъ душъ человѣческихъ, всѣми силами старается запутать душу человѣческую въ похоти мірскія, какъ рыбу въ сѣти, дабы слова Божія до себе не допустила, и тако бы погибла. Хрістіанине! совѣтую тебѣ и молю тя, ради твоего спасенія, успокойся хотя на малое время отъ сего пагубнаго шума, и тогда истину сію признаешь и увидишь, что мятежъ мірскихъ похотей слово Божіе отгоняетъ отъ души; тогда почувствуешь нѣкое движеніе къ вѣчности; аки тончайшій гласъ, тогда помалу будетъ къ тебѣ приходить мысль, кто ты и къ какому концу идеши, что тебя по смерти срящетъ, и проч. Сіе"=то есть знакъ приходящаго слова Божія къ душѣ! Ибо Божіе слово, какъ сѣмя, подобный себѣ и плодъ раждаетъ.

\subsection{О томжде.}

Кто отверстый тѣлесный слухъ имѣетъ и ничимъ не препятствуется, тотъ все слышитъ, что ни говорится; слышитъ что страшное, и ужасается; слышитъ что пріятное ему, и утѣшается; слышитъ что смѣшное, и смѣется; слышитъ что нужное ему, и желаетъ и ищетъ тое. Тако дѣлаетъ и тотъ хрістіанинъ, который имѣетъ отверстый слухъ души своея. Онъ всякое Божіе слово слышитъ и допущаетъ тое до внутренности души своея, и силою того, какъ конь шпорами, подвигается и возбуждается. Божіе бо слово силу нѣкую Божественную имѣетъ, которая душу человѣческую подвигаетъ, какъ крѣпкая водка или спиртъ чувство уханія; и какъ слово Божіе есть духовное, то къ духовному дѣлу и побуждаетъ ее. Таковый хрістіанинъ слышитъ слово о праведномъ судѣ Божіи, и боится того; слышитъ слово о вѣчной мукѣ, и ужасается, и тщится, какъ отъ той избавиться; слышитъ о вѣчномъ животѣ и блаженствѣ его, и сердцемъ восхищается къ тому и печется, какъ бы не лишиться того; слышитъ, коль великое и пагубное зло есть грѣхъ, и всѣми силами бережется отъ него; слышитъ, коль красна и благопріятна есть добродѣтель, и тщится стяжать ее; слышитъ слово о покаяніи, и тотчасъ чувствуетъ внутрь себе желаніе и побужденіе нѣкое къ покаянію, и полагаетъ въ себѣ, какъ бы тое дѣломъ самымъ исполнить; слышитъ обличительное слово за грѣхъ, и той видя въ совѣсти своей, печалію ради грѣха того, печалію по Бозѣ сокрушается, и аки стрѣлою пронзается, и негодуетъ и гнѣвается на себе, а часто и слезы отъ сокрушеннаго сердца проливаетъ; слышитъ, коль дивная благость Божія показалась къ роду человѣческому въ воплощеніи Сына Божія, и весьма тому спасительному и человѣколюбивому Божію промыслу удивляется; слышитъ, что и онъ единъ отъ тѣхъ, ради которыхъ Сынъ Божій и Богъ великій во плоти явился и пришелъ взыскати и спасти погибшихъ, и благодаритъ отъ сердца и со смиреніемъ покланяется за тое человѣколюбивому Богу; слышитъ утѣшительное Евангелія слово, и чувствуетъ духовное нѣкое внутрь живое утѣшеніе. Евангеліе бо, яко радостная Божія вѣсть, съ небесъ посланная къ человѣкамъ, душѣ слышащей тое безъ утѣшенія не бываетъ, и проч. Сіе есть дѣйствіе Божіяго слова, до внутренности души доходящаго! Тако слово Божіе, всельшееся въ душу хрістіанскую, хрістіанина обновляетъ и дѣлаетъ его богомудрымъ, богобоящимся, благочестивымъ, святымъ, словомъ, \textit{новою тварію} о Хрістѣ. Якоже бо бальсамъ, внесенный въ домъ, исполняетъ домъ благоуханія: тако Божіе слово, какъ духовный и благопріятный бальсамъ, вшедши въ душу человѣческую, дѣлаетъ въ ней благоуханіе. Бальсама сего благоуханіе есть страхъ Божій, любовь Божія и ближняго, покаяніе, печаль по Бозѣ, духъ сокрушенъ, воздыханіе, умиленіе, слезы, утѣшеніе, духовная радость. Сіе небесное благовоніе Духъ Святый содѣловаетъ въ душѣ чрезъ слово Свое святое. Хрістіанине! почувствуешь и ты въ душѣ твоей сію добрую воню небеснаго бальсама, когда, отлучивши душу твою отъ шума и мятежа мірскихъ прихотей, отвориши слухъ души твоея слову Божію. Како убо можетъ тотъ почувствовать силу и дѣйствіе Божія слова, который не о спасеніи, но о прихотяхъ міра сего думаетъ? Отжени убо вредный и пагубный шумъ сей отъ души твоей, и почувствуетъ душа въ себѣ небесный тотъ бальсамъ. Престани думать о томъ, какъ бы сыскать честь, славу, богатства въ мірѣ семъ, какъ бы богатый столъ набрать, какъ бы гостей принять и въ гости съѣздить, какъ бы богатый домъ построить и украсить, какъ бы лучшимъ паче прочіихъ платьемъ одѣться, какъ бы мудрѣйшимъ и славнѣйшимъ паче другихъ показаться, какъ бы на добрыхъ коняхъ и богатыхъ каретахъ проѣзжаться, какъ бы болѣе земли и крестьянъ достать, какъ угоднѣйшіи сады построить, какъ бы красныя въ нихъ галдареи сдѣлать и выгодныи пруды выкопать, и проч.: но думай, какъ бы вѣчное спасеніе получить. Сіе первѣйшее дѣло да будетъ тебѣ во всѣхъ твоихъ замыслахъ, начинаніяхъ и дѣлахъ: тогда, истину тебѣ говорю, услышитъ душа твоя слово Божіе, и подобный тому плодъ сотворитъ. Иначе, хотя всю святую Библію и прочія хрістіанскія книги наизусть будешь знать, никакой пользы души твоей отъ нихъ не получиши, аще прихотей, яко оглушающихъ душу и слова Божіи до нея не допущающихъ, не оставиши. Остави убо ихъ, да внидетъ въ душу твою слово Божіе.

\section{31. Сѣмя.}

Видимъ въ мірѣ семъ, что каково сѣмя сѣется, такой и плодъ отъ него родится, какъ"=то: отъ пшеницы пшеница, отъ ржи рожь, отъ овса овесъ, и проч.; \textit{рожденное отъ плоти, плоть есть}\footnote{Іоан.~3,~6.}. Тако Божіе слово, яко сѣмя на земли сердецъ человѣческихъ сѣемое, подобный себѣ плодъ раждаетъ. Божіе слово есть сѣмя духовное, доброе, небесное, святое, животворящее: убо посѣянное на сердцѣ человѣческомъ, раждаетъ человѣка подобнымъ себѣ, то есть, дѣлаетъ его духовнымъ, добрымъ, небеснымъ, святымъ, живымъ. Разсуждай убо, человѣче, како слышишь слово Божіе: чувствуеши ли отъ слышанія силу и дѣйствіе, подобное слову Божію? чувствуеши ли въ душѣ твоей мысли и желанія духовныя, добрыя, небесныя? Аще не чувствуеши, то знакъ есть, что сѣмя слова Божія погибаетъ въ тебѣ. \textit{Имѣяй уши слышати, да слышитъ}, глаголетъ Господь\footnote{Лук.~8,~8 и 15.}.

\subsection{О томжде.}

Сѣмя равно сѣется, но не на равную землю падаетъ: тако слово Божіе равно всѣмъ проповѣдуется, но не во всѣхъ сила и плодъ его бываетъ. Сѣмя иное падаетъ на \textit{пути}, и отъ \textit{птицъ небесныхъ} восхищается: тако слово Божіе, проповѣданное сердцамъ человѣческимъ, по которымъ, какъ по пути, различныя мысли переходятъ, погибаетъ и безплодно бываетъ; ибо \textit{дуси лукавіи}, яко птицы, восхищаютъ тое, \textit{да не} человѣцы \textit{вѣровавши спасутся}. Сѣмя иное падаетъ на \textit{каменной} земли, и понеже не имѣетъ довольной земли, гдѣ бы углубить и утвердить корень свой, скоро прозябаетъ, но отъ зноя солнечнаго прозябшее, яко корене не имущее, изсыхаетъ: тако слово Божіе, падшее на сердцахъ человѣческихъ, но не углубленное и не вкоренившееся, хотя и съ радостію пріемлется, однакожъ отъ нашедшаго искушенія, печали и гоненія, лишается сладкаго своего плода, и тако безплодно бываетъ. Сѣмя иное падаетъ между \textit{терніемъ}, и отъ тернія возрастшаго подавляется, сего ради и тое безплодно бываетъ: тако слово Божіе слышатъ многіи люди, но \textit{печалію вѣка сего и лестію богатства}, какъ терніемъ, объятыя сердца имѣя, никакого плода не творятъ; и тако и въ сихъ слово Божіе \textit{безплодно} бываетъ. Иное сѣмя падаетъ на \textit{доброй земли}, и сіе едино плодъ приноситъ: тако слово Божіе, когда слышатъ тое \textit{добрыя сердца}, плодъ свой получаетъ, и отъ того плодотворятъ \textit{ово сто, ово же шестьдесятъ, ово же тридесять}\footnote{Мѳ.~13,~3--23; Марк.~4,~3--20; Лук.~8,~5--15.}. "--- Отъ сего видишь, хрістіанине: 1)~Слово Божіе проповѣдуемое, то"=есть, чтомое Евангеліе или Апостолъ, или иное какое отъ Ветхаго Завѣта, или проповѣдь отъ проповѣдника глаголемую должно слушать со всякимъ прилѣжаніемъ и вниманіемъ, да не, яко птицы сѣмя на пути посѣянное восхищаютъ, и слово Божіе, тебѣ проповѣданное, духи лукавіи отъ тебе восхитятъ, и тако безплоденъ будеши. 2)~Слово Божіе, слышанное или прочтенное, должно углубить и вкоренить во глубины сердца, и въ томъ день и нощь поучаться, якоже Псаломникъ глаголетъ о себѣ, и намъ во образъ себя представляетъ: \textit{въ сердцѣ моемъ скрыхъ словеса Твоя, Боже, яко да не согрѣшу Тебѣ}\footnote{Пс.~118,~11.}. И паки: \textit{рабъ Твой глумляшеся во оправданіихъ Твоихъ; ибо свѣдѣнія Твоя поученіе мое есть, и совѣти мои оправданія Твоя}\footnote{стих. 23 и 24.}. И паки: \textit{и поучахся въ заповѣдехъ Твоихъ, яже возлюбихъ зѣло}\footnote{47.}. И паки: \textit{и глумляхся во оправданіихъ Твоихъ}\footnote{48.}. И паки: \textit{коль возлюбихъ законъ Твой, Господи! весь день поученіе мое есть}\footnote{97.}. И паки: \textit{скорби и нужды обрѣтоша мя: заповѣди Твоя поученіе мое}\footnote{143.}. Почему и пишется о праведникѣ: \textit{законъ Бога его въ сердцѣ его, и не запнутся стопы его}\footnote{Пс.~36,~31.}, и \textit{блаженнымъ} называется тотъ человѣкъ, \textit{котораго въ законѣ Господни воля, и въ законѣ Его поучается день и нощь}: сего ради и уподобляется таковый \textit{древу, насажденному при исходищихъ водъ, еже плодъ свой дастъ во время свое, и листъ его не отпадетъ, и вся, елика аще творитъ, успѣетъ}\footnote{1,~1--3.}. 3)~Понеже прихоти міра сего, какъ терніе сѣмя, слово Божіе, въ сердцѣ посѣянное, подавляютъ; то надобно вредное сіе терніе отъ сердца искоренить, и \textit{не любить міра, ни яже въ мірѣ}, по слову апостольскому\footnote{1~Іоан.~2,~15.}. Иначе сѣмя Божія слова безплодно будетъ. 4)~Хотящаго работати Богу и плодъ слова Его принести Ему, различное срѣтаетъ искушеніе, по писанному: \textit{чадо! аще приступаеши работати Господеви Богу, уготови душу твою во искушеніе}\footnote{Сир.~2,~1.}. Сего ради надобно все тое искушеніе терпѣніемъ побѣждать; откуду глаголетъ Евангелистъ святый, что \textit{добрымъ сердцемъ и благимъ слышавше, слово держатъ, и плодъ творятъ въ терпѣніи}\footnote{Лук.~8,~15.}. Сего ради хотящимъ плодъ слова Божія творить, нужно есть терпѣніе. 5)~Тщаніе человѣческое безъ Божія помощи не сильно. Земляное поле, "--- сердце человѣческое съ небеснымъ сѣменемъ совокупитися и плодъ сотворити не можетъ: надобно тутъ небеснаго дѣлателя, Іисуса Хріста, силѣ и помощи дѣйствовать, якоже Самъ глаголетъ: \textit{безъ Мене не можете творити ничесоже}\footnote{Іоан.~5,~15.}! Сего ради хотящему и тщащемуся плодъ слова Божія творити, надобно усердно молитися, чтобы Самъ Господь плодъ слова Своего въ сердцахъ нашихъ творилъ и совершалъ. \textit{Просите, и дастся вамъ; ищите и, обрящете; толцыте, и отверзется вамъ; всякъ бо просяй пріемлетъ, и ищай обрѣтаетъ, и толкущему отверзется}\footnote{Мѳ.~7,~7 и 8.}. Откуду такъ съ великимъ усердіемъ Псаломникъ, чрезъ весь псаломъ \textit{118"~й}, молится, чтобы Самъ Богъ велъ его по пути заповѣдей Своихъ и слова Своего плодъ совершалъ. Господи, Іисусе Хрісте, Слове безначальнаго Твоего Отца, помози намъ!

\subsection{О томжде.}

Хотя и добрая земля будетъ, на которой сѣмя посѣяно, однакожъ плода не можетъ сотворить, ежели теплотою солнечною не будетъ согрѣваться и дождемъ орошаться. Тако хотя и на доброе сердце падетъ сѣмя слова Божія; однакожъ никакого плода не принесетъ, аще теплотою благодати Божія не будетъ согрѣваться и росою свыше орошаться. Сего ради, какъ посѣяннымъ сѣменамъ сіяніе солнечное и дождь къ плодотворенію, такъ слышанному слову Божію, чтобы не безъ плода было, нужна есть Божія благодать, согрѣвающая и орошающая. Почему и Давидъ святый глаголетъ: \textit{путь заповѣдей Твоихъ текохъ, егда разширилъ еси сердце мое}\footnote{Пс.~118,~32.}. Сего ради всегда должно ко Господу воздыхать, чтобы благодатію Своею сердца наша согрѣвалъ и орошалъ. На всякій часъ и минуту требуемъ Божія благодати: безъ той бо, какъ младенцы ходить еще незнающіе, падемъ и лежать будемъ. \textit{Милость Твоя поженетъ мя вся дни живота моего}\footnote{22,~6.}! "--- Господи, постави насъ всемогущею Твоею рукою на пути заповѣдей Твоихъ, и укрѣпи, и веди!

\section{32. Сѣятва и жатва.}

У земледѣльцевъ есть сѣятва и жатва: тако и у хрістіанъ есть своя сѣятва и жатва. У земледѣльцевъ суть сѣмена: тако и у хрістіанъ суть сѣмена своя. У земледѣльцевъ сѣмена отъ сѣменъ раждаются: тако и у хрістіанъ отъ сѣмене слова Божія духовныя сѣмена раждаются. У земледѣльцевъ сѣмена суть: рожь, пшеница, ячмень и проч.: у хрістіанъ сѣмена суть: покаяніе, печаль по Бозѣ, воздыханіе, слезы, молитва, благодареніе, пѣніе, дѣла милости, терпѣніе и проч. Сіи духовныя сѣмена раждаются отъ сѣмене слова Божія, павшаго на сердцахъ человѣческихъ. У земледѣльцевъ есть приличное время сѣятвы и жатвы: тако и у хрістіанъ есть такоежде время. Земледѣльцы знаютъ время, въ которомъ надобно сѣять: тако и у хрістіанъ есть приличное время, въ которомъ должно имъ сѣять сѣмена своя, "--- которое время есть время настоящаго житія. \textit{Се нынѣ время благопріятно! се нынѣ день спасенія}\footnote{2~Кор.~6,~2.}! Земледѣльцы разумныи и мудрыи не пропускаютъ времени, въ которое должно сѣять, но тогда труждаются и сѣютъ: тако разумныи и мудрыи хрістіане не пропущаютъ настоящаго житія времени; но труждаются и мещутъ сѣмена своя, каются, творятъ дѣла покаянія, творятъ дѣла милости, и проч. Нынѣшнее бо время есть только время сѣятвы: а въ будущемъ вѣкѣ того не будетъ. Нынѣ время благопріятно есть каятися, плакати за грѣхи, молитися, добро творити всѣмъ; но въ будущемъ вѣкѣ все то престанетъ. Земледѣльцы трудятся и сѣютъ съ надеждою плода, отъ сѣмене имущаго родитися: тако хрістіане трудятся и сѣютъ съ надеждою духовнаго плода, милостиваго отъ Бога воздаянія. Земледѣльцы доспѣлые трудовъ своихъ плоды во время жатвы пожинаютъ: тако хрістіане трудовъ и сѣменъ своихъ плоды пожнутъ въ кончинѣ вѣка и въ воскресеніи изъ мертвыхъ. Земледѣльцы, собирая плоды трудовъ своихъ, радуются: тако и хрістіане, видя и собирая плоды трудовъ и сѣменъ своихъ, возрадуются. \textit{Сѣющіи слезами радостію пожнутъ. Ходящіи хождаху и плакахуся, метающе сѣмена своя: грядущіи же пріидутъ радостію, вземлюще рукояти своя}\footnote{Пс.~125,~5 и 6.}. Земледѣльцы лѣнивые и нерадивые, которыи въ приличное время не сѣютъ, и плодовъ во время жатвы не сбираютъ: тако и хрістіане несмысленныи, нерадивыи, небрежливыи, которыи нынѣ духовныхъ сѣменъ не сѣютъ, въ кончину вѣка и въ воскресеніе изъ мертвыхъ никакихъ плодовъ не пожнутъ, но безплодны явятся предъ Господемъ. Како бо соберутъ тогда, когда нынѣ не расточали? како пожнутъ тогда, когда нынѣ не сѣяли? "--- Хрістіанине! потрудимся и посѣемъ нынѣ сѣмена наша, да тогда съ радостію пожнемъ плоды трудовъ нашихъ.

\subsection{О томжде.}

Земледѣльцы, которыи болѣе сѣменъ сѣютъ, болѣе и собираютъ во время жатвы; а которыи менѣе сѣютъ, тыи менѣе и пожнутъ: тако и хрістіане, которыи болѣе духовныхъ сѣменъ въ настоящемъ житіи сѣютъ, тыи въ послѣдній день, въ день суда Хрістова, въ который всякаго и вси дѣла явятся, болѣе плодовъ пожнутъ; а которыи скудно сѣютъ, скудно и пожнутъ, якоже Апостолъ глаголетъ: \textit{сѣяй скудостію, скудостію и пожнетъ}\footnote{2~Кор.~9,~6.}. Праведенъ бо Господь нашъ есть: всѣмъ \textit{воздастъ по дѣломъ ихъ}\footnote{Рим.~2,~6.}. Сего ради увѣщаваетъ насъ Апостолъ: \textit{доброе творяще, да не стужаемъ си; во время бо свое пожнемъ, не ослабѣюще. Тѣмже убо, дондеже время имамы, да дѣлаимъ благое ко всѣмъ, паче же къ приснымъ въ вѣрѣ}\footnote{Гал.~6,~9 и 10.}. Хрістіанине! напишемъ слово сіе апостольское: \textit{дондеже время имамы}, на сердцахъ нашихъ. Сіе время настоящаго житія нашего время есть; нынѣ только можемъ благотворити, милостыню подавати; въ будущемъ вѣкѣ все тое престанетъ. Потщимся убо нынѣ сѣять, да тогда съ радостію пожнемъ; нынѣ трудиться въ доброе, да тогда упокоимся; здѣ имѣнія наша расточимъ, да тамо далеко большее соберемъ; подадимъ въ руки убогихъ злато и сребро наше, да тогда съ лихвою отъ Хріста Господа пріимемъ. Не бойся, хрістіанине! \textit{вѣренъ есть обѣщавый}\footnote{Евр.~10,~23.}. Нищимъ въ руки даешь сребро твое; но Хрістосъ обѣщался за нихъ отдать тебѣ, и отдастъ съ великимъ прибыткомъ. Все тое погибнетъ, что ни иждивается на прихоти и роскоши, на богатые и красные домы, на богатые столы, на богатыя одѣянія, на богатыя кареты и кони и прочая; все сіе иждивеніе суетно есть и погибаетъ. Познаешь тое въ день смерти твоея, когда все оставишь здѣ. Но что въ руки нищихъ влагается, тое не погибнетъ, но сторичный плодъ принесетъ, какъ сѣмя падшее на доброй земли не погибаетъ, но сѣявшему плодъ приноситъ. Повѣримъ убо имѣніе наше въ руки нищихъ во имя Хрістово, да отъ Хріста сторицею воспріимемъ.

\subsection{О томжде.}

Когда земледѣлецъ сѣетъ и мещетъ сѣмена своя въ землю, тогда глупый человѣкъ, который, не зная плода отъ сѣмене происходящаго, смѣется тому, думая, что земледѣлецъ напрасно погубляетъ сѣмена своя, пометая ихъ въ землю; но земледѣлецъ, имѣя надежду, что сѣмя его, пометаемое въ землю, множайшій плодъ принесетъ ему, нежели сѣется, не престаетъ сѣять. Тако люди глупыи и несмысленныи дѣлаютъ. Они, видя, что хрістіанинъ расточаетъ имѣніе и даетъ убогимъ, смѣются тому дѣлу, думая, что погубляетъ онъ свое имѣніе; сіи люди думаютъ, что тое только имѣніе не погибаетъ, которое иждивается на тое, что видимо есть, то"=есть, на суету и прихоти міра сего; почему и сами не на милостыню, но на похоти плотскія иждиваютъ имѣнія своя; а надъ тѣми, которыи расточаютъ и даютъ убогимъ, смѣются. Смѣйся, человѣче, смѣйся; но послѣ горько восплачешися и возрыдаеши, ты, который не жалѣешь тысящей или сотенъ на приданое дочери твоей, на строеніе и украшеніе богатыхъ домовъ, на драгія вина и богатые столы, на шелковыя и вѵссонныя одѣянія, на кареты и кони, на собачью охоту и прочія прихоти твоя изнурять; а ради Хріста, Который за тебе пострадалъ и умеръ, Который такъ тебе врага сущаго возлюбилъ, что и Самого Себе ради тебе не пощадѣлъ, "--- ради Хріста, глаголю, такъ великаго и высокаго твоего благодѣтеля, и рубля или полтины жалѣешь подать, да и еще отъ Его добра, тебѣ даннаго. Все бо Его добро есть, какое ни имѣемъ. \textit{Господня бо земля, и исполненіе ея}\footnote{Пс.~23,~1.}. Разсуждай сія, и неотмѣнно будешь жалѣть и плакать; а когда нынѣ не будешь жалѣть и плакать, то во второе пришествіе Хрістово и въ день воздаянія будешь жалѣть и плакать, но поздно и безполезно. Якоже бо мужики лѣнивые и гуляки, видя земледѣльцевъ, сбирающихъ плоды посѣянныхъ сѣменъ своихъ и о нихъ радующихся, сами сѣтуютъ и скорбятъ, ничего не видя въ рукахъ своихъ, яко ни трудилися, ни сѣяли, почему и плодовъ не собираютъ, и тако остаются безъ надежды пропитанія: тако хрістіане роскошные и гуляки въ кончину вѣка и воскресеніе изъ мертвыхъ (тогда бо хрістіанская жатва будетъ)\footnote{Мѳ.~13,~30.}, постыдятся и посрамятся; восплачутся и возрыдаютъ тогда, когда увидятъ добрыхъ хрістіанъ пожинающихъ плоды сѣменъ своихъ, себя же безплодныхъ и въ крайней нищетѣ. Тогда они подлинно познаютъ, что богатство, на прихоти и роскоши изнуренное, погибаетъ и изнурившаго въ погибель ввергаетъ, а въ руки нищихъ и убогихъ, ради имени Хрістова вложенное, сохраняется, и далеко лучшее и съ великимъ прибыткомъ возвращается. \textit{Блажени милостивы: яко тіи помиловани будутъ}\footnote{5,~7.}. Окаянніи немилостивіи, яко милости лишатся: \textit{судъ бо безъ милости не сотворшему милости}\footnote{Іак.~2,~13.}. Тогда милостивіи, сотворшіи дѣла милости, услышатъ отъ праведнаго Судіи: \textit{пріидите благословенніи Отца Моего, наслѣдуйте уготованное вамъ царствіе отъ сложенія міра. Взалкахся бо, и дасте Ми ясти; возжадахся, и напоисте Мя; страненъ бѣхъ, и введосте Мене; нагъ, и одѣясте Мя; боленъ, и посѣтисте Мене; въ темницѣ бѣхъ, и пріидосте ко Мнѣ}. А немилостивіи, роскошныи и гуляки, расточившіи имѣнія благая Господня не на убогихъ, но на прихоти своя, яко раби неключимыи и строители невѣрныи, услышатъ: \textit{идите отъ Мене проклятіи въ огнь вѣчный, уготованный діаволу и аггеломъ его. Взалкахся бо, и не дасте Ми ясти; возжадахся и не напоисте Мене; страненъ бѣхъ, и не введосте Мене; нагъ, и не одѣясте Мене; боленъ и въ темницѣ, и не посѣтисте Мене}\footnote{Мѳ.~25,~34--43.}. Хрістосъ не требуетъ благихъ нашихъ, но Самъ все намъ подаетъ; не требуетъ, глаголю, Хрістосъ, но требуютъ хрістіане, братія наша. И что дается хрістіанамъ, тое, по человѣколюбію Своему, Хрістосъ Себѣ вмѣняетъ, якоже глаголетъ: \textit{понеже сотвористе единому сихъ братій Моихъ меньшихъ, Мнѣ сотвористе}\footnote{Ст. 40.}. А что не дѣлается хрістіанамъ, тое Самому Хрісту не дѣлается, по реченному: \textit{понеже не сотвористе единому сихъ меньшихъ, ни Мнѣ сотвористе}\footnote{45.}. Что бо дѣлается членамъ, тое и главѣ дѣлается; и чего не дѣлается членамъ, тое и главѣ не дѣлается. Дѣлается добро членамъ, тое вмѣняетъ и глава себѣ; не дѣлается добра членамъ, не дѣлается и главѣ самой. Разсуждай сіе, человѣче, скупый на подаяніе милостыни, щедрый на прихоти и роскоши своя! Благословенный хрістіанине! Хрістолюбивая душа! не смотри на сыновъ вѣка сего, которыи всячески стараются міру сему угодить, а не Хрісту, умершему за нихъ; но доброе творить не преставай: \textit{во время бо свое пожнешь, не ослабѣюще}\footnote{Гал.~6,~9.}. Возлюбленне, не уподобляйся злому, но благому: \textit{благотворяй отъ Бога есть; а зло творяй не видѣ Бога}\footnote{3~Іоан.~3,~11.}. Подражай земледѣльцу, въ приличное время сѣющему съ надеждою плода; сѣй и ты сѣмена своя въ нынѣшнее время, да во время жатвы съ радостію пожнеши. \textit{Еже бо аще сѣетъ человѣкъ, тожде и пожнетъ: яко сѣяй въ плоть свою, отъ плоти пожнетъ истлѣніе; а сѣяй въ духъ, отъ духа пожнетъ животъ вѣчный}\footnote{Гал.~6,~7 и 8.}. Вожделѣнно и сладко будетъ слышати: \textit{пріидите благословенніи!} Страшно и ужасно: \textit{идите отъ Мене проклятіи!} Хрістіанине, да звенятъ всегда въ ушахъ нашихъ два сіи слова: \textit{пріидите, отъидите}. Помилуемъ братію нашу, да и сами помиловали будемъ; подадимъ, да и намъ подастся. Аминь.

\subsection{О томжде.}

Земледѣльцы не просто мещутъ сѣмена своя въ землю, но прежде во умѣ своемъ полагаютъ жатву, и смотрятъ на плоды отъ сѣменъ посѣянныхъ раждаемые, и тѣхъ со усердіемъ желаютъ, и тако, взирая на будущіе прибытки, трудятся и сѣютъ. Тако и купцы не просто по чужимъ сторонамъ ѣздятъ, но прежде взираютъ на богатство, отъ купечества собираемое, и тако странствуютъ и купечествуютъ. Тако и путники не просто вдаются въ путь, но прежде смотрятъ на мѣсто, къ которому хотятъ итить, и тако воспріемлютъ путь, и трудятся и идутъ по пути къ намѣренному мѣсту. Тако и воины не просто идутъ на брань, но прежде умными очами взираютъ на побѣду и славу и честь, отъ побѣды послѣдующую, и тако идутъ на брань и подвизаются противу врага. Тако и всякъ въ мірѣ семъ трудится не просто, но прежде взираетъ на конецъ, ради котораго трудъ подъемлетъ, и отъ труда послѣдующій плодъ, и тако начинаетъ трудиться. Никто бо ничего не начинаетъ и не дѣлаетъ безъ предложенія и намѣренія. Возлюбленный хрістіанине, подражай и ты трудникамъ таковымъ, которые трудятся ради временныхъ благъ. Они трудятся, взирая на добро тлѣнное и скоро преходящее: ты воззри на вѣчная благая окомъ вѣры, благая, \textit{ихже око не видѣ, и ухо не слыша, и на сердце человѣку не взыдоша, яже уготова Богъ любящимъ Его}\footnote{1~Кор.~2,~9.}. И трудись, и сѣй съ надеждою полученія оныхъ благъ. Воззри на жатву оную, которая будетъ въ послѣдній день, жатву, въ которой не земные, но небесные, не тлѣнные, но нетлѣнные, не временные, но вѣчные собираются плоды; и тако нынѣ на земли, донелѣже время имѣеши, донелѣже сѣятва есть и приличное къ сѣянію время не прошло, сѣй сѣмена твоя, да во время жатвы оныя пожнеши плоды. Земледѣльцы и прочіи трудящіеся ради земныхъ благъ часто и обманываются въ надеждѣ своей; не всегда бо доброплодіе бываетъ, и не всегда получаютъ люди, чего желаютъ, и ради чего трудятся: но кто съ вѣрою, и надеждою и усердіемъ ищетъ вѣчныхъ благъ, не обманывается, но получаетъ ихъ; ибо \textit{вѣренъ есть обѣщавый}\footnote{Евр.~10,~23.}. Трудись убо и ты и сѣй нынѣ, сѣй съ вѣрою и надеждою. Не бойся: сѣмя не погибнетъ. Въ руки нищихъ даешь, но въ рукахъ Хрістовыхъ даемое обрѣтается, и съ великимъ прибыткомъ тебѣ возвратится. Положи во умѣ твоемъ и затверди въ памяти твоей сіе, что отечество и домъ твой есть на небеси: и тако \textit{сокрывай сокровище} твое \textit{не на земли, идѣже червь и тля тлитъ, и идѣже татіе подкопываютъ и крадутъ; но скрывай себѣ сокровище на небеси, идѣже ни червь, ни тля тлитъ, идѣже татіе не подкопываютъ, ни крадутъ}\footnote{Мѳ.~6,~19--20.}. Не опасайся: тамо сокровенное сокровище все цѣло будетъ и въ послѣдній день предъ всѣмъ свѣтомъ произнесется и объявится, и отдастся тебѣ, и будетъ неотлучно съ тобою во вѣки вѣковъ. Отсюду послѣдуетъ: 1)~Покажеши о себѣ, что ты душею и умомъ и сердцемъ отъ міра сего изшелъ, и гражданство свое имѣеши на небеси, хотя тѣломъ и въ мірѣ обращаешися; и можеши со Апостоломъ глаголати: \textit{наше житіе на небесѣхъ есть, отонудуже и Спасителя ждемъ, Господа нашего Іисуса Хріста}\footnote{Филип.~3,~20.}. \textit{Идѣже бо сокровище твое, ту будетъ и сердце твое}\footnote{Мѳ.~6,~21.}. Неотмѣнно тамо духомъ твоимъ, сердцемъ и мыслію обращаться будеши, гдѣ сокровище твое сокровенно есть. 2)~Получиши оное во свое время. Услышиши отъ Хріста во онъ день: возлюбленне, вотъ твое сокровище, которое сокрывалъ ты не на земли, но на небеси, по слову Моему! вотъ оно цѣло отдается тебѣ! пріими сіе и утѣшайся тѣмъ, и буди богатъ не тлѣннымъ, но нетлѣннымъ богатствомъ. 3)~Сокровище, сокрываемое на земли, дѣлаетъ страхъ и печаль сокрывающему, чтобы не окрадено было. Ты, сокрывая сокровище свое на небеси, отъ сего страха и печали свободишися, и тако свободенъ, покоенъ и миренъ душею будеши, "--- что есть превеликое добро душевное. Хрістіанине! куда ни обращай мысль твою, однако не будешь имѣть покоя, ежели во оное отечество и домъ не предпослешь твоего сокровища. Хотя въ земли сокрывай его, хотя въ клѣти и сундукахъ заключай, хотя сторожамъ повѣряй, хотя на прихоти и роскоши расточай: однакожъ не убѣжишь мучительныя о немъ заботы; всегда, и гдѣ ни будешь, будетъ съ тобою попеченіе, печаль и страхъ, и, какъ червь внутрь дерево, злое сіе внутрь сердце твое и душу твою будетъ снѣдать и мучить; и такъ сугубое зло постраждешь "--- грѣшить и мучиться будешь. А когда въ небо чрезъ руки нищихъ и убогихъ людей послеши, то всего сего зла свободишися. Сего ради повѣрь слову Хрістову: \textit{идѣже есть сокровище ваше, ту будетъ и сердце ваше}, и перенеси сокровище твое отъ земли на небо, и тамо съ сокровищемъ твоимъ сердце твое обращаться будетъ. Что же есть любезнѣе и сладчае, какъ тѣломъ на земли жить, а душею и сердцемъ на небеси обращаться. "--- О Іисусе, свѣте душевныхъ нашихъ очесъ! отверзи душевныя наши очи, да увидимъ блага оная, \textit{яже Ты уготовалъ любящимъ Тя}, и съ желаніемъ и усердіемъ поищемъ ихъ. Возлюбленный хрістіанине! ежели бы ты хотя малую частицу оныхъ благъ увидѣлъ, то бы ты, все въ мірѣ семъ презрѣвши, съ великимъ стремленіемъ къ онымъ спѣшилъ. Но повѣрь, пожалуй, неложному Божію слову, которое достовѣрнѣйшее есть паче всякаго видѣнія: и поищеши ихъ.

\subsection{О томжде.}

Земледѣльцы убогіи, которыи мало собранныхъ плодовъ имѣютъ, хотя и нужду терпятъ, однакожъ берегутъ ихъ, чтобы было что сѣять, и тако плоды собрать. Убогій хрістіанине! твори и ты такожде, какъ оные убогіи и разумныи трудники творятъ. Убогое имѣніе у себе имѣешь, но не все расточай на домашніе твои расходы; береги, чтобы было что посѣять, и во время жатвы плоды собрать. Удѣли часть и отъ убогаго Хрісту, Который сторицею въ свое время тебѣ отдастъ; сѣй хотя мало сѣменъ въ руки нищихъ, да благодатію Божіею умножится плодъ твой. Давай просящему у тебе со усердіемъ и малое: отъ доброхотнаго дателя поданное за великое почтетъ и пріиметъ человѣколюбивый Іисусъ, якоже отъ убогія вдовицы два лепта принялъ. \textit{Аминь глаголю вамъ, яко вдовица сія убогая множае всѣхъ вверже вметающихъ въ сокровищное хранилище. Вси бо отъ избытка своего ввергоша: сія же отъ лишенія своего вся, елика имѣяше, вверже все житіе свое}\footnote{Марк.~12,~43 и 44.}. Вметай убо и ты хотя лепты въ руки нищихъ, какъ сѣмена въ землю: и въ свое время съ радостію пожнешь. И \textit{чаша} бо студеныя \textit{воды, во имя Хрістово} поданная, \textit{незабвенна} предъ Нимъ бываетъ\footnote{9,~41.}.

\section{33. Десница.}

Видимъ, что десница или десная рука ничего не держитъ въ себѣ, но что пріемлетъ, тое паки отдаетъ. Хрістіанине! человѣкъ ничего своего не имѣетъ, кромѣ немощей, грѣховъ, бѣдности и окаянства; но все отъ Создателя своего пріемлетъ: разумъ, здравіе, крѣпость, богатство, сребро и злато, домъ, одѣяніе, хлѣбъ, пищу, питіе, скотъ и проч.: есть убо десницею пріемлющею; но долженъ быть и десницею отдающею. Многіе бываютъ десницею пріемлющею, но не вси десницею отдающею. Многіи пріемлютъ, но не вси отдаютъ. Буди убо, хрістіанине, не токмо десницею пріемлющею, но и отдающею. Принялъ ты отъ Бога добро: не держи тое у себе, но отдавай въ славу Божію и въ пользу ближняго. Принялъ отъ Бога разумъ: не сокрывай его, но давай неразумнымъ и несмысленнымъ, да и твой умножится талантъ. Принялъ здравіе и крѣпость: не сокрывай ихъ, но употребляй на благословенные труды. Принялъ богатство: не сокрывай его въ земли, въ клѣти и сундукахъ, ни изнуряй на прихоти и роскоши Божіяго добра, но раздѣляй нищимъ и убогимъ людемъ "--- братіи твоей. Пріемлешь хлѣбъ: раздробляй алчущимъ хлѣбъ твой. Имѣешь одѣянія: не держи ихъ въ клѣти твоей, но одѣвай ими нагую братію твою. Имѣешь скотъ: да служитъ онъ не токмо тебѣ, но и прочіимъ людямъ. Имѣешь домъ: не затворяй его, но отворяй приходящимъ и странствующимъ, и проч. Вотъ тебѣ, хрістіанине, десница! Буди убо не токмо пріемлющая, но и отдающая десница. Отъ Бога пріемлешь всякое добро: отдавай тое во славу добра "--- Дателя и созиданіе требующаго добра твоего. Тако будеши вѣрный строитель дарованій Божіихъ; и что отъ Бога получишь, тое паки къ Богу обратиши, то есть, во славу Божію; и за тое Богъ воздастъ тебѣ, яко вѣрному строителю, уже не земными, но небесными, не временными, но вѣчными благими. Аще ли же не твориши тако; то, яко злый и невѣрный рабъ, отъ Господа твоего истяжешися и услышиши: \textit{неключимаго раба вверзите во тму кромѣшнюю: ту будетъ плачъ и скрежетъ зубомъ}\footnote{Мѳ.~25,~30.}.

\section{34. Царь датель, и даянія его похититель.}

Аще бы царь подданному своему далъ какой подарокъ въ руки его, и тотъ бы часъ какой нахальникъ изъ рукъ его предъ глазами царскими похитилъ тотъ подарокъ: что бы тогда отъ царя послѣдовало похитившему? Неотмѣнно бы сильный на хищника возгорѣлся гнѣвъ. И праведно: ибо и царю давшему досада великая, и рабу его пріемшему обида была бы. Богъ, Царь небесный, по благости и человѣколюбію Своему, всѣмъ людямъ благая Своя даетъ, и какъ въ руки имъ влагаетъ, дабы вси довольствовалися благими Его, и тако бы Его, яко своего благодѣтеля, знали и благодарили: но люди безстрашные дарованія Божія у людей похищаютъ и какъ бы изъ рукъ вырываютъ; и творятъ то беззаконное дѣло предъ всевидящими Божіими очами. Богъ даетъ Свое добро рабамъ Своимъ, какъ подарокъ въ руки ихъ, и они поданное добро Божіе похищаютъ; и кого Богъ удостояетъ, они аки недостойнымъ дѣлаютъ; и кого Богъ снабдѣваетъ, они лишаютъ; и кого Богъ богатитъ, они убожатъ. И Богъ беззаконное ихъ тое дѣло видитъ. Разсуди убо, человѣче, какая досада бываетъ Богу, и людямъ обида отъ таковыхъ похитителей; а отсюду, какъ сильный гнѣвъ Божій возгорится на нихъ, сказать невозможно! Тѣхъ, которыи братіи своей "--- нищимъ и убогимъ людемъ "--- не помоществовали, отсылаетъ во огнь вѣчный, какъ читаемъ у Матѳеа, въ главѣ 25"~й: что уже будетъ тѣмъ, которыи не токмо не помоществуютъ, но и похищаютъ и отнимаютъ? Разсуждай сіе, лакомый похититель и чужимъ добромъ насыщающійся! "--- Сіе беззаконное дѣло творятъ: 1)~Разбойники, которые на путниковъ нападаютъ и у нихъ имѣнія ихъ отнимаютъ. 2)~Сильныи, но безстрашныи лица, которыи у бѣдныхъ вдовицъ и прочіихъ беззаступныхъ людей отнимаютъ земли, рощи и прочія угодія, къ содержанію живота ихъ нужныя. 3)~Начальники, который не даютъ подначальнымъ жалованья, отъ Государя имъ опредѣленнаго, или даютъ, но не сполна, безъ всякой правильной причины, и при себѣ, ради лакомства своего, тое удерживаютъ. 4)~Судіи и приказныи люди, которыи отъ приходящихъ къ суду мзду берутъ и безъ мзды дѣла не дѣлаютъ. У нихъ обыкновенная въ томъ хитрость есть: \textit{завтра пріиди}; а сіе слово значитъ тое: \textit{принеси, и будетъ сдѣлано дѣло}. 5)~Господа помѣщики, которыи или великими оброками, или многими работами крестьянъ своихъ отягчаютъ. Сіе хищеніе тяжко есть, хотя ослѣпленный господинъ и не видитъ. Бѣдный крестьянинъ лѣто цѣлое трудится и потѣетъ, но питаться и одѣваться съ домашними нечѣмъ; оретъ, пашетъ и сѣетъ, но почти все одинъ помѣщикъ собираетъ. Много стонетъ и вздыхаетъ отечество наше отъ сихъ \textit{благородныхъ} душъ, какъ и отъ \textit{справедливыхъ} судей!... Господи, пощади созданіе Твое! 6)~Тайныи тати и воры, которыи лавки, клѣти, хлѣбъ окрадываютъ, и прочее чужое добро тайно похищаютъ. 7)~Беззаконныи купцы, которыи худую вещь за добрую и дешевую за дорогую продаютъ, и въ какой нибудь продажѣ ближняго своего обманываютъ. 8)~Всякъ, кто какимъ нибудь способомъ чужую вещь неправедно себѣ присвояетъ. Вси таковыи добро Божіе, отъ Бога въ руки человѣческія данное, похищаютъ, и какъ бы изъ рукъ ихъ предъ Богомъ вырываютъ. Человѣче! Богъ ближнему твоему даетъ добро Свое во владѣніе его, какъ въ руки, а ты добро тое изъ рукъ его похищаешь; и Богъ твое сіе беззаконное дѣло видитъ, видитъ и въ книгѣ Своей записываетъ, и представитъ тое предъ лицомъ твоимъ на судѣ Своемъ: тогда увидишь, какъ ты худо дѣлалъ! Но покайся и возврати похищенное, и разсыпай добрѣ, собранное злѣ, да не явится съ тобою на судѣ ономъ, и скажется о тебѣ: се \textit{человѣкъ, и дѣла его!}

\subsection{О томжде.}

Похититель царскаго подарка болѣе обижаетъ себе, нежели того человѣка, у котораго похищаетъ подарокъ тотъ: тако всякъ хищникъ, воръ и тать болѣе вредитъ себѣ, нежели тому, у кого похищаетъ. Похититель царскаго дара лишается милости царскія и подпадаетъ царскому гнѣву: тако всякъ хищникъ и тать, который чужое добро похищаетъ, лишается милости Божія и подлежитъ праведному Его гнѣву. Похититель царскаго дара, яко царскій противникъ и его царскаго величества досадитель, выключается изъ числа добрыхъ подданныхъ и предается наказанію: тако всякій хищникъ и воръ выключается изъ числа добрыхъ хрістіанъ и предается наказанію временному и вѣчному. Что временному наказанію подлежитъ, тое видимъ; видимъ, что какъ собранное хищниковъ неправдою имѣніе разсыпается и погибаетъ, такъ сами они безчестіе, грызеніе и мученіе совѣсти, и прочая злая въ мірѣ семъ страждутъ; и часто бываетъ, что таковыи на нѣсколько времени обогащаются и, какъ цвѣтъ на полѣ, нѣсколько блистаютъ, но вдругъ, какъ цвѣтъ, увядаютъ и низпадаютъ и дѣлаются, какъ единъ отъ нищихъ и убогихъ. Похищенное бо чужое добро, какъ огнь, въ домъ вшедшее, все прочее имѣніе поядаетъ. Ибо гдѣ неправда, тамо клятва Божія вселяется; гдѣ клятва Божія, тамо никакого добра не будетъ, но всякое злополучіе послѣдуетъ. Хищники и грабители подобни суть человѣку, черплющему рѣшетомъ воду: тако и изъ ихъ рукъ все чрезъ различные ручьи вытекаетъ, что ни похищаютъ и собираютъ, какъ рѣшетомъ черплемая вода. Похищай и хватай, человѣче, какъ хощешь и что хощешь: однако все изъ рукъ твоихъ вытечетъ, и собственное твое все неправда, какъ огнь, потребитъ. А что вѣчному наказанію подпадаетъ хищникъ и тать, то явно есть оттуду, что Апостолъ написалъ: \textit{ни лихоимцы, ни хищницы, ни татіе царствія Божія не наслѣдятъ}\footnote{1~Кор.~6,~10.}. Не сотворшіи милости ближнему своему изгоняются и отсылаются во огнь вѣчный\footnote{Мѳ.~25,~41.}: что уже постраждутъ чуждая похищающіи? Тако хищникъ и тать болѣе себе обиждаетъ, нежели ближняго своего, у котораго похищаетъ; болѣе себѣ вредитъ, нежели ему. О человѣче! временно корыстуешися чужимъ, но вѣчнымъ страданіемъ за тое платить будеши. Страшно есть и слышать о вѣчномъ страданіи, а не токмо страдать; но не избѣжиши того, воистину не избѣжиши, аще не покаешися и не примиришися съ братомъ твоимъ. \textit{Буди} убо, глаголетъ тебѣ Господь, \textit{увѣщаваяся съ соперникомъ твоимъ скоро, дондеже еси на пути съ нимъ, да не предаетъ тебе соперникъ судіи, и судія тя предастъ слузѣ, и въ темницу вверженъ будеши. Аминь глаголю тебѣ: не изыдеши оттуду, дондеже воздаси послѣдній кодрантъ}\footnote{5,~25 и 26.}. "--- Примиряйся, человѣче, примиряйся скоро со всѣми, кого ты ни обидѣлъ; примиряйся, пока находишься на пути житія твоего, пока праведный Судія не зоветъ тебе къ Себѣ на судъ. Зоветъ же всякаго чрезъ смерть. А каковъ позванъ будеши, таковъ и суду Его предстанеши.

\section{35. Воротись! не туда пошелъ.}

Бываетъ въ мірѣ семъ, что человѣкъ человѣка идущаго возвращаетъ въ задъ, и кричитъ въ слѣдъ его: слышиши ли? \textit{воротись! не туда пошелъ}. Тако Богъ къ совѣсти человѣку глаголетъ и вопіетъ ему, когда хощетъ зло содѣлать: возвратися, человѣче! \textit{не туда идешь ты. Уклонися отъ зла}\footnote{Пс.~33,~15.}. Хрістіанине! хощеши ли ты отмстить, повредить, или, что горше того, убить ближняго твоего? Богъ въ слѣдъ тебѣ вопіетъ: человѣче, воротись! Хощеши ли блудъ сотворить и осквернить тѣло твое? Богъ вопіетъ тебѣ: человѣче, воротись! Хощеши ли украсть, похитить и отнять у ближняго твоего добро? Богъ гремитъ въ совѣсти твоей: человѣче, возвратись! Хощеши ли ближняго твоего прельстить и обмануть? Богъ тебѣ вопіетъ: человѣче, воротись! Хощеши ли оклеветать, осудить, обругать и опорочить брата твоего? Богъ возвращаетъ тебе отъ того и глаголетъ: человѣче, воротись! \textit{Удержи языкъ твой отъ зла}\footnote{14.}. Тако и отъ прочіихъ грѣховъ отвращаетъ тебе Богъ и въ совѣсти твоей зоветъ тебе: человѣче, воротись! \textit{Уклонися отъ зла и сотвори благо}. А что въ совѣсти тебѣ говоритъ, тое и въ словѣ Своемъ святомъ говоритъ. Совѣсть непогрѣшительная и слово Божіе суть согласны. Что совѣсть говоритъ, тое и слово Божіе; отъ чего совѣсть удерживаетъ и отвращаетъ, отъ того и Божіе слово; за что совѣсть обличаетъ, за тое и Божіе слово; и за что совѣсть похваляетъ, за тое и Божіе слово. Напримѣръ: обличаетъ тебе за воровство совѣсть, "--- обличаетъ за то и Божіе слово; похваляетъ тебе совѣсть за милость, сотворенную ближнему твоему, "--- похваляетъ и Божіе слово. \textit{Блажени милостивы}. Сего ради когда совѣсть насъ отъ чего отвращаетъ и удерживаетъ: се есть гласъ Божій, вопіющій внутрь насъ, отвращающій и удерживающій насъ отъ зла! Хрістіанине! послушаемъ гласа Божія, и отвратимся отъ зла, по увѣщанію Святаго Духа: \textit{днесь аще гласъ Его услышите, не ожесточите сердецъ вашихъ}\footnote{Пс.~94,~7 и 8.}. Да и мы воззовемъ, и услышитъ насъ Господь.

\section{36. Не касайся сего! здѣ ядъ.}

Бываетъ, что человѣкъ, хотячи другаго человѣка остерещи, чтобы не повредился, удерживаетъ его и говоритъ ему: \textit{не касайся сего! здѣ ядъ} сокровенъ есть, какъ много таковыхъ случаевъ въ мірѣ видимъ. Тако Божіе слово остерегаетъ насъ отъ грѣха и глаголетъ всякому изъ насъ: человѣче, берегись грѣха! Въ грѣхѣ ядъ сокровенъ есть; когда коснешися его, умертвитъ тебе: \textit{оброцы грѣха смерть}\footnote{Римл.~6,~23.}. Ядъ въ ненависти, злопомнѣніи и убійствѣ: берегись того! ядъ въ блудѣ и во всякой нечистотѣ: берегись того! Ядъ въ піянствѣ, роскоши и сластолюбіи: берегись того! Ядъ во враждѣ и ссорѣ: берегись того! Ядъ въ воровствѣ, хищеніи, насиліи, лихоиманіи и всякомъ неправедномъ дѣлѣ: берегись того! Ядъ въ праздности и лѣности: берегись того! Ядъ въ клеветѣ, осужденіи, злорѣчіи, ругательствѣ, сквернословіи, буесловіи, кощунствѣ и всякомъ неполезномъ и гниломъ словѣ: человѣче, берегись того! Ядъ въ гордости, славолюбіи, тщеславіи и лицемѣріи: берегись того! Ядъ во всякомъ дѣлѣ, словѣ, помышленіи и начинаніи богопротивномъ: берегись того! Человѣче, берегись всякаго грѣха, да не умертвить тебе! "--- Ядъ тѣло умерщвляетъ, грѣхъ душу. Бережешися смерти тѣлесныя: кольми паче должно берещися смерти душевныя, которую грѣхъ содѣлываетъ; чимъ бо честнѣйшая и дражайшая душа паче тѣла, тѣмъ болѣе должно ее и берещи. Берегись, или не берегись тѣлесной смерти, не убѣжишь ея; всѣмъ сіе необходимо "--- и тебѣ. А когда душу будешь берещи отъ смертоноснаго яда грѣха; то во вѣки будешь жить, хотя и умрешь; ибо благодатію Божіею возстанеши не въ смерть, но въ животъ вѣчный. Берегись и ты, душе моя, всякаго грѣха, яко яда смертоноснаго, да не умертвитъ тя. \textit{Пріидите чада}, глаголетъ къ намъ Духъ Святый, \textit{послушайте Мене, страху Господню научу васъ. Кто есть человѣкъ хотяй животъ, любяй дни видѣти благи? Удержи языкъ твой отъ зла, и устнѣ твои еже не глаголати льсти; уклонися отъ зла, и сотвори благо; взыщи мира, и пожени и. Очи Господни на праведныя, и уши Его въ молитву ихъ: лице же Господне на творящія злая, еже потребити отъ земли память ихъ}\footnote{Пс.~33,~12--18.}. Блаженъ человѣкъ, который слушаетъ увѣщанія и предостереганія Божія слова, и бережется всякаго грѣха! Окаяненъ, кто душеспасительному сему \textit{гласу} Божію не внимаетъ!

\subsection{О томжде.}

Видимъ, что когда человѣкъ, по неосторожности, яда или инаго чего вреднаго вкуситъ и почувствуетъ въ желудкѣ болѣзнь, "--- не медлитъ, но тотчасъ прибѣгаетъ къ лекарю и ищетъ исцѣленія. Хрістіанине! не медли и ты, когда, по неосторожности и дѣйству діавольскому, вкусишь яда грѣховнаго; не медли, говорю, но тотчасъ ищи исцѣленія; прибѣгай вѣрою ко Хрісту, Врачу душъ и тѣлесъ, проси отъ Него исцѣленія души твоей; объявляй Ему язву, которою уязвилъ тебе врагъ твой; объявляй, хотя Онъ и знаетъ ее; падай предъ Нимъ и съ сокрушеніемъ сердца признавай свою болѣзнь; не стыдись и не бойся! Онъ прежде признанія твоего знаетъ твою болѣзнь; но твоего требуетъ самохотнаго и добровольнаго признанія, дабы ты самъ себе предъ Нимъ обвинилъ и грѣха своего не покрылъ. Говори убо Ему и исповѣдуй на тя беззаконіе твое: \textit{согрѣшихъ, Господи, помилуй мя! Исцѣли мя Господи, яко согрѣшихъ Тебѣ}. Сынове Израилевы, когда вышли изъ Египта и шли въ землю обѣтованную, въ нѣкоторой пустынѣ угрызаемы были отъ зміевъ. Богъ, милосердуя о нихъ, повелѣлъ Моѵсею на высокомъ древѣ вознести змію мѣдяную, дабы люди угрызенніи на тую змію взирали ради исцѣленія своего, и тако взирая исцѣлялись\footnote{Числ.~21,~6--9.}. Хрістіанине! мы идемъ въ землю обѣтованную "--- небесное отечество, намъ обѣщанное, и проходя пустыню міра сего, много страждемъ отъ адскаго змія, діавола. Аще убо, по неосторожности нашей, сей лютый змій угрызнетъ насъ, и почувствуемъ душевредный ядъ его въ душѣ нашей: тотчасъ возведемъ очи наши ко Хрісту Сыну Божію, сѣдящему одесную Бога Отца, за наши грѣхи нѣкогда вознесенному на древо крестное, распятому и умершему. \textit{Якоже Моѵсей вознесе змію въ пустыни: тако подобаетъ вознестися Сыну человѣческому, да всякъ вѣруяй въ Онь не погибнетъ, но имать животъ вѣчный}\footnote{Іоан.~8,~14.}. "--- О Іисусе, сило и надеждо спасенія нашего, сѣдяй одесную Отца во славѣ Отчей! помилуй насъ, которыхъ честною Кровію искупилъ еси.

\section{37. Плачъ.}

Видимъ въ мірѣ, что люди плачутъ: раждаются съ плачемъ, живутъ съ плачемъ умираютъ съ плачемъ. Плачутъ люди: ибо живутъ въ мірѣ, мѣстѣ плача, юдоли плачевной. Многія причины суть, ради которыхъ люди плачутъ; и у всякаго плачущаго своя причина есть плача. Плачи и ты, хрістіанине! ибо и ты живешь въ юдоли плачевной; имѣешь и ты много причинъ, ради которыхъ должно плакать. Плачи, пока время не ушло, пока полезны слезы; плачи, да не во вѣки восплачешися; плачи, да утѣшишися: \textit{блаженны бо плачущіи, яко тіи утѣшатся}\footnote{Мѳ.~5,~4.}. Плачутъ люди, что несчастливы суть: плачи и ты, хрістіанине, что ты грѣшенъ еси, что ты Господу твоему согрѣшилъ; великое бо несчастіе есть грѣхъ. Плачутъ люди, что не имѣютъ здравія тѣлеснаго: плачи и ты, что не имѣешь здравія душевнаго. Плачутъ люди, что находятся въ недугѣ и болѣзни: плачи и ты, что душа твоя недугуетъ, болѣзнуетъ и немоществуетъ; тяжкій бо недугъ есть гордость, зависть, гнѣвъ, нечистота, сластолюбіе, славолюбіе, сребролюбіе; и столько мучащихъ болѣзней имѣетъ, сколько страстей и похотей. \textit{Исцѣли мя, Господи, и исцѣлюся}\footnote{Іер.~17,~14.}: яко Ты еси Богъ, Спасъ мой. Плачутъ люди, что потеряли богатство: плачи и ты, хрістіанине, что потерялъ богатство, во святомъ крещеніи данное тебѣ отъ небеснаго твоего Отца. Плачутъ люди, что находятся въ нищетѣ и скудости: плачи и ты, что нищъ и убогъ, бѣденъ и окаяненъ; несносна бо нищета есть грѣхъ. \textit{Приклони, Господи, ухо Твое и услыша мя; яко нищъ и убогъ есмь азъ}\footnote{Пс.~85,~1.}. Плачутъ люди, что не имѣютъ пищи и питія: плачи и ты, что душа твоя истаеваетъ гладомъ, лишается слышанія слова Божія; великій бо и весьма тяжкій гладъ "--- не слышать Божія слова. Плачутъ люди, что наги суть, и не имѣютъ чимъ прикрыти наготы своея: плачи и ты, что наготствуетъ душа твоя: грѣхъ ее обнажилъ. Срамна нагота тѣлесная, но срамнѣйшая нагота душевная. Тѣлесную наготу человѣцы видятъ; но душевную наготу Богъ и святіи ангели Его видятъ. \textit{Блаженъ бдяй и блюдый ризы своя, да не нагъ ходитъ и узрятъ срамоту его}\footnote{Апок.~16,~15.}. Окаяненъ и бѣденъ не бдяй и не блюдый ризъ своихъ, яко нагъ ходитъ и зрятъ срамоту его! \textit{Ризу мнѣ подаждь свѣтлу, одѣяйся свѣтомъ, яко ризою, многомилостивый Хрісте Боже мой}. Плачутъ люди, что падаютъ и разбиваются: плачи и ты, хрістіанине, что падаешь въ грѣхъ, и разбивается душа твоя и изнемогаетъ; тяжкое бо и лютое паденіе есть грѣхъ. Лучше есть далеко пасти тѣломъ, нежели душею, и ногами, нежели грѣхомъ. \textit{Мняйся стояти, да блюдется, да не падетъ}\footnote{1~Кор.~10,~12.}. Плачутъ люди по умершихъ родителяхъ, братіи, сродникахъ и другахъ: плачи и ты, хрістіанине, по умершей душѣ твоей, яко Марѳа и Марія по Лазарѣ\footnote{Іоан.~11,~31--33.}. Да скажетъ небесный Отецъ и о тебѣ: \textit{яко сынъ Мой сей мертвъ бѣ, и оживе; и изгиблъ бѣ, и обрѣтеся}\footnote{Лук.~15,~24.}. О Іисусе, воскресителю мертвыхъ! воскреси грѣхами умерщвленную мою душу, якоже воскресилъ еси умершаго сына вдовицы\footnote{7,~12--15.}! Плачутъ люди, что обиду и насиліе терпятъ отъ враговъ своихъ: плачи и ты, хрістіанине, предъ Богомъ на враговъ твоихъ душевныхъ, которые тщатся отнять у тебе вѣчное спасеніе. Они суть діаволъ и зліи аггели его. \textit{Суди Господи, обидящія мя, побори борющія мя. Пріими оружіе и щитъ, и востани въ помощь мою. Изсуни Мечь, и заключи сопротивъ гонящихъ мя; рцы души моей: спасеніе твое есмь Азъ}\footnote{Пс.~34,~1--3.}. Плачутъ люди зовомы и ведомы на судъ, боячись, дабы не посрамиться на судѣ и не быть осужденными, и не попасть наказанію: плачи и ты, хрістіанине, ибо и ты на судъ позовешися, не человѣческій, но Божій, на которомъ Судія не требуетъ свидѣтелей, но Самъ вся дѣла, слова и помышленія наша знаетъ; плачи нынѣ, пока не позовешися; плачи, да умилостивиши Судію слезами твоими; плачи, да не осудишися и вверженъ будеши во тму кромѣшную: \textit{тамо будетъ плачъ и скрежетъ зубомъ}\footnote{Мѳ.~25,~30.}. \textit{Не вниди, Господи, въ судъ съ рабомъ Твоимъ; яко не оправдится предъ Тобою всякъ живый}\footnote{Пс.~142,~2.}. Плачутъ люди, что много долгу на себѣ имѣютъ, и не имѣя чимъ отдать, боятся, да не ввержены будутъ въ темницу: плачи и ты, хрістіанине, что ты много одолжился небесному Царю грѣхами твоими, и воистину не имѣеши чимъ заплатить; плачи и умилостивляй слезами Его, да оставитъ тебѣ долгъ твой, да не въ вѣчную темницу вверженъ будеши. \textit{Отче, остави намъ долги наша}\footnote{Мѳ.~6,~12.}! Плачутъ люди, сѣдящіи въ темницѣ, что не видятъ свѣта: плачи и ты, хрістіанине, что душа твоя тмою страстей объята, не видитъ свѣта Божественнаго. \textit{Изведи изъ темницы душу мою}, Господи, \textit{исповѣдатися имени Твоему}\footnote{Пс.~141,~8.}. Плачутъ люди окованныи узами и кандалами, что не имѣютъ свободы: плачи и ты, хрістіанине, что душа твоя грѣхами, аки узами, связана есть и не имѣетъ своея свободы. \textit{Аще сынъ Божій свободитъ насъ, воистинну свободни будемъ}\footnote{Іоан.~8,~36.}. О Іисусе, искупителю плѣнныхъ душъ нашихъ! растерзай узы наши, да пожремъ тебѣ жертву хвалы. Плачутъ люди, что біеніе и раны терпятъ: плачи и ты, хрістіанине, что совѣсть твоя злая душу твою уязвляетъ и мучитъ паче всякаго мучителя. \textit{Помилуй мя, Боже, по велицѣй милости Твоей, и по множеству щедротъ Твоихъ, очисти беззаконіе мое}\footnote{Пс.~50,~3.}! Плачутъ люди, живущіи на чужой земли, что не видятъ дома и любезнаго отечества своего. Тако плакали Іудеи, будучи въ плѣненіи вавилонскомъ, якоже сами исповѣдуютъ: \textit{на рѣкахъ вавилонскихъ, тамо сѣдохомъ и плакахомъ, внегда помянути намъ Сіона}\footnote{136,~1.}. Плачи и ты, хрістіанине, что ты въ мірѣ семъ, какъ на земли чужой, живеши и не видиши отечества небеснаго и прекраснаго "--- горняго Іерусалима. \textit{Увы мнѣ, яко пришельствіе мое продолжися}\footnote{119,~5.}. Плачи и ты, душе моя, да и здѣ и тамо утѣшишися. \textit{Кто дастъ главѣ моей воду, и очесемъ моимъ источникъ слезъ, и плачуся день и нощь}\footnote{Іер.~9,~1.}? \textit{Услыши молитву мою, Господи, и моленіе мое внуши, слезъ моихъ не премолчи: яко пресельникъ азъ есмь у Тебе и пришлецъ, якоже вси отцы мои. Ослаби ми, да почію, прежде даже не отъиду, и ктому не буду}\footnote{Пс.~38,~13 и 14.}. \textit{Пріидите поклонимся, и припадемъ Ему, и восплачемся предъ Господемъ сотворшимъ насъ: яко Той есть Богъ нашъ, и мы людіе пажити Его, и овцы руки Его}\footnote{94,~6 и 7.}. \textit{Услыши ны, Боже Спасителю нашъ, упованіе всѣхъ концевъ земли, и сущихъ въ мори далече}\footnote{64,~6.}. \textit{Согрѣшихомъ, беззаконновахомъ, неправдовахомъ предъ Тобою, ниже сотворихомъ, ниже соблюдохомъ, якоже заповѣдалъ еси намъ; но не предаждь насъ до конца отцевъ Боже!}

\section{38. Долгъ.}

Бываетъ въ мірѣ, что единъ у другаго заимообразно беретъ нѣсколько денегъ или другихъ какихъ вещей, и взятое называется \textit{долгъ}, а взявшій \textit{должникъ}. И чимъ кто болѣе беретъ, тѣмъ болѣе умножаетъ долгъ свой, и заимодавцу должнѣйшій бываетъ. Тако всякъ человѣкъ, когда заповѣдь Божію разоряетъ и согрѣшаетъ предъ Богомъ, въ долгъ предъ Богомъ впадаетъ, и Ему должникомъ дѣлается; и чимъ болѣе заповѣдей Божіихъ разоряетъ, и Создателю своему согрѣшаетъ, тѣмъ большее бремя долговъ собираетъ и величеству Божію должнѣйшимъ дѣлается. Отсюду грѣхи человѣческія долги предъ Богомъ называются, а грѣшники должники, якоже читаемъ въ молитвѣ Господней: \textit{остави намъ долги наша, якоже и мы оставляемъ должникомъ нашимъ}\footnote{Мѳ.~6,~12.}. Законъ обдолжаетъ должника ко отданію долга заимодавцу: тако всякъ грѣшникъ закономъ Божіимъ обдолжается ко отданію за грѣхи Богу. Должникъ, когда не отдастъ долга заимодавцу, въ темницу ввергается: тако грѣшникъ, когда не отдастъ долга своего Создателю Своему, ввержется въ темницу вѣчную. Но чимъ грѣшнику отдать Богу, который столько разъ оскорбилъ безконечное величество Его? За единъ грѣхъ нечѣмъ заплатить: а чимъ уже за многіе и тяжкіе грѣхи? Ибо и единъ грѣхъ, яко оскорбленіе \textit{безконечнаго} величества Божія, вѣчныя казни и смерти достоинъ: тако научаетъ хрістіанская Богословія. Хрістіанине! тяжко есть обдолжиться человѣку: но несравненно тягчае обдолжиться Богу грѣхами. Люто есть и страшно ввержену быть во временную темницу: но несравненно лютѣе и страшнѣе предану быть въ темницу вѣчную. Сладокъ людемъ грѣхъ: но горьки плоды его. \textit{Оброцы грѣха смерть}, глаголетъ Павелъ\footnote{Римл.~6,~23.}. Удобно грѣхъ сотворить: но не удобно отъ грѣха свободиться. Надобно неотмѣнно къ свобожденію Тому придти, который глаголетъ: \textit{аще Сынъ вы свободитъ, воистинну свободни будете}\footnote{Іоан.~8,~36.}. Бываетъ, что когда должникъ не можетъ отдать долга заимодавцу, то заимодавецъ по единой милости своей оставляетъ ему долгъ, и тако должникъ свободенъ бываетъ. Хрістіанине! мы должники Божіи: мы не могли, и не можемъ долговъ нашихъ отдать Богу, Которому и были и есмы должны, но, по безприкладной милости Своей, оставляетъ намъ долги грѣховъ нашихъ. Мы есмы оный должникъ, тмою талантъ обдолжившійся царю своему, надъ которымъ, яко не имѣлъ чимъ отдать долга своего, \textit{умилосердился царь, и отпустилъ ему} весь \textit{долгъ}\footnote{Мѳ.~18,~24--28.}. Намъ должникамъ Своимъ Царь небесный оставляетъ многіи и тяжкіи долги грѣховъ, какъ тму талантъ, яко не имѣемъ чимъ отдать: но оставляетъ чистымъ сердцемъ къ Нему обращающимся и за сотворенные грѣхи кающимся, и падающимъ предъ Нимъ и милости отъ Него просящимъ; оставляетъ благодатію и человѣколюбіемъ единороднаго Сына Своего, Господа нашего Іисуса Хріста, \textit{Иже преданъ бысть за прегрѣшенія наша, и воста за оправданіе наше; Иже язвенъ бысть за грѣхи наша и мученъ бысть за беззаконія наша}\footnote{Римл.~4,~25; Ис.~53,~5.}. Оставляетъ, говорю, кающимся: ибо грѣшникъ не кающійся и отъ грѣховъ не отстающій, какъ былъ, такъ и есть въ долгѣ грѣховномъ предъ Богомъ, небеснымъ Царемъ; и сколько разъ безстрашно разоряетъ заповѣдь Божію, столько разъ обдолжается предъ Богомъ; и сколько грѣховъ творитъ, столько умножаетъ долгъ свой. Блудникъ, прелюбодѣй и всякъ нечистый сколько разъ беззаконнуетъ, столько къ долгу своему прилагаетъ. Хищникъ, тать и лихоимецъ, чимъ болѣе похищаетъ, тѣмъ болѣе долгомъ грѣховнымъ обременяется. Клеветникъ и сквернословецъ, буесловецъ и кощунникъ и всякъ ругатель, чимъ болѣе языкъ свой гнилымъ и злымъ словомъ оскверняетъ, тѣмъ большимъ должникомъ Богу бываетъ. Словомъ: всякъ законопреступникъ, чимъ болѣе законъ Божій безстрашно нарушаетъ, тѣмъ большее бремя долга своего собираетъ и большему подвергаетъ себе наказанію. О человѣче! сладокъ тебѣ грѣхъ, которымъ безконечное величество Божіе оскорбляешь и праведный Его гнѣвъ на себе разжигаешь! Сладокъ тебѣ грѣхъ, за который горчайшую страданія и смерти чашу Хрістосъ, Сынъ Божій, испилъ! Сладокъ тебѣ грѣхъ, за который горесть вѣчныя смерти вкушать будеши, аще покаешися! Видишь, сколь горьки плоды грѣха, хотя онъ тебѣ и кажется сладокъ! Но полно уже обременяться долгомъ; пора и тое бремя съ плечъ своихъ свергать, да не явишися съ нимъ на судѣ Хрістовомъ. Горе человѣку, который съ симъ тяжкимъ бременемъ явится тамо! Неотмѣнно погрузитъ его во дно адово. Падай убо, бѣдный человѣче, предъ Создателемъ твоимъ, и со смиреннымъ и сокрушеннымъ сердцемъ взывай Ему, подражая мытарю: \textit{Боже, милостивъ буди мнѣ грѣшному}\footnote{Лук.~18,~3.}, да и на тебе призритъ Господь благоутробнымъ окомъ, и оставитъ тебѣ вси долги твои. Но впредь всячески берегися Его оскорблять, и предъ Нимъ обдолжаться. А дознавая на себѣ милость Божію, показывай милость и ближнему твоему, и отпущай ему согрѣшенія его, да не паки дознаеши на себѣ гнѣвъ Божій, и возвратится оставленный долгъ твой на тебе, якоже лукавому оному рабу, въ притчѣ поминаемому сотворилося\footnote{Мѳ.~18,~28--35.}. Злопомнѣніе бо заключаетъ дверь къ полученію милости Божіей и оставленію грѣховъ. \textit{Аще отпущаете человѣкомъ согрѣшенія ихъ, отпуститъ и вамъ Отецъ вашъ небесный. Аще ли не отпущаете человѣкомъ согрѣшенія ихъ, ни Отецъ вашъ отпуститъ вамъ согрѣшеній вашихъ}\footnote{6,~14 и 15.}. \textit{Отче! остави намъ долги наша, якоже и мы оставляемъ должникомъ нашимъ}\footnote{ст. 12.}.

\subsection{О томжде.}

Должникъ, получивши отъ заимодавца милость и видя долгъ свой отъ него оставленный, радуется: тако грѣшникъ, получивши отъ Бога высочайшую милость и грѣховъ оставленіе, чувствуетъ въ сердцѣ своемъ живое утѣшеніе и радуется о милости Его, на грѣшники изливающейся. Истинно бо обратившееся сердце къ Богу, кающееся и сокрушенное, безъ утѣшенія не оставляется отъ милосердаго Создателя своего. Ибо Богъ, яко человѣколюбецъ и милосердый, любезно и милосердо зритъ на сердце обратившееся къ Нему, сокрушенное и падающее предъ Нимъ, такъ что \textit{сердце сокрушенно и смиренно Богъ не уничижитъ}\footnote{Пс.~50,~19.}; и на таковое сердце изливаетъ елей милости Своея, отъ котораго уязвленное печалію сердце, какъ рана отъ живительнаго пластыря, исцѣляется и живность нѣкую получаетъ. \textit{Исцѣляяй сокрушенныя сердцемъ, и обязуяй сокрушенія ихъ Господь}\footnote{Пс.~146,~3.}. Се есть Евангельское живое утѣшеніе, котораго кающееся сердце сподобляется отъ милосердаго Создателя своего! Сіе есть оное благовѣстіе, о которомъ Хрістосъ Искупитель міра глаголетъ: \textit{благовѣстити нищимъ посла Мя, исцѣлити сокрушенныя сердцемъ, проповѣдати плѣненнымъ отпущеніе и слѣпымъ прозрѣніе, отпустити сокрушенныя во отраду, проповѣдати лѣто Господне пріятно}\footnote{Лук.~4,~18--19.}. Должникъ заимодавца, яко милостиваго своего благодѣтеля, любитъ и благодаренъ ему бываетъ чрезъ все житіе свое, и помнитъ его благодѣяніе даже до смерти: тако грѣшникъ, чувствуя великую на себѣ Божію милость, сердечно любитъ Его и отъ сердца, благодаритъ Ему, и помнитъ тую благодать Его даже до смерти. Благодарность бо есть память благодѣяній. Гдѣ память благодѣяній, тамо и благодарность: гдѣ забывается благодѣяніе, тамо вступаетъ и неблагодарность. Должникъ, видя милость, сотворенную себѣ отъ заимодавца и памятуя тую милость, и самъ должнику своему, ежели его имѣетъ, являетъ милость и отпущаетъ ему долгъ, "--- и сіе знакъ добронравнаго человѣка и благодарнаго: тако хрістіанинъ благодарный и добросердечный, видя являеемую себѣ Божію милость и чувствуя грѣховъ своихъ оставленіе, и самъ являетъ ближнему своему милость и отпущаетъ ему согрѣшенія его. Бываетъ, что лукавый должникъ, получивши отъ заимодавца оставленіе долга, самъ оставить не хощетъ долга своему должнику; чего ради заимодавца своего праведно на гнѣвъ подвигаетъ, и долгъ свой паки на себѣ отъ него возвращенный видитъ, и ему должникомъ, какъ и прежде былъ, дѣлается: тако неблагодарный и лукавый хрістіанинъ, чувствуя великую на себѣ Божію милость, и получа оставленіе тяжкихъ и великихъ грѣховъ отъ Бога, самъ не хощетъ и малыхъ ближнему своему согрѣшеній оставить. \textit{Малыхъ}, говорю, согрѣшеній: ибо когда человѣкъ человѣку согрѣшаетъ, какъ ни согрѣшаетъ, весьма тое мало есть предъ тѣмъ, что Богу согрѣшаемъ. И тако на праведный и большій гнѣвъ Бога Создателя своего подвигаетъ, и какъ прежде былъ, такъ паки должникомъ Ему дѣлается, еще большимъ, нежели былъ. Читай о семъ, хрістіанине, притчу о царѣ и должникѣ его, и увидишь, какъ тяжко и опасно злобиться и мстить ближнему своему\footnote{Мѳ.~18,~23--35.}. Нѣтъ ничего безопаснѣе, какъ простить, и нѣтъ ничего опаснѣе, какъ не простить и мстить согрѣшенія ближнему. \textit{Судъ бо безъ милости не сотворшему милости}\footnote{Іак.~2,~13.}. Богъ по благости Своей всѣмъ намъ являетъ милость, чувствуемъ тую на всякій не токмо день, но и часъ; но когда человѣкъ, сподобляяся милости Божія, самъ не хощетъ явить милости подобному себѣ человѣку; тогда Богъ и Свою милость отъ него, яко неблагодарнаго и лукаваго раба, отнимаетъ; и тако человѣкъ, вмѣсто милости, праведному Божію суду подлежитъ и за всѣ свои грѣхи, какіи ни дѣлалъ въ житіи своемъ, судимъ будетъ. Видимъ, хрістіанине, какъ страшно и опасно не простить и мстить ближнему. \textit{Рабе лукавый!} (глагола царь должнику своему, которому долгъ отпустилъ"=было) \textit{весь долгъ оный отпустихъ тебѣ, понеже умолилъ мя еси: не подобаше ли и тебѣ помиловати клеврета твоего, якоже и азъ тя помиловахъ?} Тоежде и небесный Царь хрістіанину, не милующему ближняго своего, глаголетъ. \textit{И прогнѣвався господь его, предаде его мучителемъ, дондеже воздастъ весь долгъ свой; тако}, заключаетъ Хрістосъ Господь притчу, \textit{и Отецъ Мой небесный сотворитъ вамъ, аще не отпустите кійждо брату своему отъ сердецъ вашихъ согрѣшенія ихъ}\footnote{Мѳ.~18,~32--35.}.

\section{39. Господинъ, зовущій раба.}

Бываетъ, что господинъ, хотячи позвать нѣкоего своего слугу, ему извѣстнаго, глаголетъ прочіимъ рабамъ: пошлите мнѣ того"=то слугу; и зовутъ его: иди, господинъ тебя требуетъ; и идетъ. Тако Хрістосъ, Господь всѣхъ, зоветъ человѣка отъ міра сего; и отходитъ отъ міра. Господинъ слугу своего зоветъ нечаянно: тако Хрістосъ Господь всякаго человѣка зоветъ нечаянно. Смерть всякому человѣку "--- званіе его. Блаженъ рабъ, который позовется отъ господина своего исправнымъ: окаяненъ, который явится неисправнымъ. Блаженъ человѣкъ, который, позванный отъ Хріста Господа, явится предъ Нимъ исправнымъ: окаяненъ, который явится неисправнымъ. Хрістіанине!

исправность наша есть истинное покаяніе. Неисправность наша есть небреженіе истиннаго покаянія. Званіе наше ко Господу есть кончина наша. Видимъ приближившуюся кончину нашу: тутъ видимъ и званіе наше. Извѣстна намъ кончина наша и неизвѣстна. Извѣстна: яко скончаемся. Неизвѣстна: яко не знаемъ, когда скончаемся. Тако и званіе Господне: не знаемъ, когда позоветъ насъ Господь. Хощеши ли убо, чтобы тебѣ позванному исправнымъ предъ Господемъ явиться? Буди всегда въ истинномъ покаяніи, буди всегда таковымъ, каковымъ хощешь быть при кончинѣ твоей, каковымъ хощешь позванъ быть. Хощеши ли блаженно скончаться? Поминай сей часъ и живи благочестиво. Мудрыи и разумныи раби бдятъ и всегда ожидаютъ, когда господинъ позоветъ ихъ. Хрістіанине! подражай и ты въ томъ сынамъ вѣка сего; бди и ожидай званія Господа твоего. Буди въ семъ важномъ дѣлѣ разуменъ и мудръ, да и блаженнымъ будеши. Отъ сего часа зависитъ вѣчное или благополучіе, или неблагополучіе; тутъ отворится дверь къ вѣчности или блаженной, или неблагополучной; отсюду пойдешь или въ вѣчную жизнь и утѣшеніе, или вѣчную смерть и мученіе. Видишь, коль страшенъ часъ сей. Блаженъ бдяй и ожидаяй часа сего! Бди и ты, душе моя, и ожидай часа сего, да тогда съ радостію возможеши воспѣть: \textit{нынѣ отпущаеши раба твоего, Владыко, по глаголу твоему съ миромъ}, и прочая\footnote{Лук.~2,~29--32.}. \textit{Скажи ми, Господи, кончину мою, и число дней моихъ, кое есть, да разумѣю, что лишаюся азъ? Се пяди положилъ еси дни моя и составъ мой яко ничтоже предъ Тобою: обаче всяческая суета, всякъ человѣкъ живый. Убо образомъ ходитъ человѣкъ, обаче всуе мятется: сокровиществуетъ, и не вѣсть, кому соберетъ я}\footnote{Пс.~38,~5--7.}. О Іисусе, Искупителю мой! очисти мя прежде, даже не возмеши отсюду.

\subsection{О томжде.}

Бываетъ, что слуга, нерадя о господинѣ своемъ и не боячись его, тако размышляетъ въ себѣ: господинъ мой не позоветъ мене сего дня къ себѣ, или еще не скоро позоветъ; пойду и съ другами моими погуляю и повеселюсь: и тако, отшедши, начнетъ пьянствовать. Тутъ нечаянно прибѣгаетъ къ нему вѣстникъ и зоветъ его, глаголя: поди, господинъ тебе требуетъ. Тогда обращается веселіе его въ печаль, и дерзновеніе въ страхъ и трепетъ; идетъ къ господину своему трепеща, мятется мыслями своими, думаетъ, какій отвѣтъ дать господину своему, безъ котораго воли отлучился и своевольствовалъ и безчинно поступалъ. Тако не богобоящійся и нерадящій о своемъ спасеніи хрістіанинъ думаетъ и въ мысли своей располагаетъ: я еще молодъ и здоровъ, не видна еще кончина моя, еще не скоро умру. Погуляю и повеселюсь на свѣтѣ, наслаждуся міра сего благими. Достану себѣ честь, чтобы люди мене знали; тамо и мнѣ не малый доходъ будетъ, соберу себѣ довольно и исполню сундуки мои. Нынѣ люди то и дѣлаютъ, что собираютъ; что и мнѣ иное дѣлать? Собравши, прикуплю себѣ поболѣе деревень и крестьянъ. Мало у мене земли: прибавлю еще. Отыму у той"=то вдовицы, и у того"=то подлаго человѣка. Ихъ заступить и мнѣ воспрепятствовать некому. До царя далеко; а судей руки исполню: и все благополучно будетъ. Потомъ богатый домъ себѣ построю и украшу его; куплю себѣ аглицкую карету, и цугъ добрыхъ лошадей; насажду красный садъ съ аллеями и галдареями; сдѣлаю въ немъ и пруды хорошіи, чтобы было гдѣ съ гостьми прохаживаться, гулять и веселиться. Накуплю себѣ вина и буду поставлять трапезу, какъ и прочіи дѣлаютъ, и знаться съ людьми; я къ нимъ, а они ко мнѣ будутъ въ гости ѣздить. Соберу довольно слугъ, и уберу ихъ въ пристойное платье, чтобы было кому мнѣ служить, и пріѣзжающимъ гостямъ моимъ. Постараюсь имѣть музыку вокальную и инструментальную, ради лучшаго увеселенія мене и гостей моихъ. Сія и подобная симъ, осуетившися въ помышленіяхъ своихъ, бѣдный замышляетъ человѣкъ; и что суетно мыслитъ, то и въ дѣло производитъ. Иной иное замышляетъ и хощетъ дѣлать. Видимъ сіе въ бѣдныхъ нынѣшняго вѣка хрістіанахъ; видимъ, что они не о спасеніи и жизни вѣчной, къ которой словомъ Божіимъ позваны и святымъ крещеніемъ обновлены, но о суетѣ все тщаніе полагаютъ; видимъ и соболѣзнуемъ. Но въ такихъ замыслахъ и дѣлахъ нечаянно смерть, какъ скорый вѣстникъ, пришедши ко всякому безгласно, говоритъ: иди, человѣче! Господь Вседержитель зоветъ тебе. Тутъ бѣдную душу страхъ и трепетъ объемлетъ; совѣсть обличаетъ и мучитъ; судъ Божій воображается: душа стѣснена отвсюду, не знаетъ, что дѣлать? Ахъ! зоветъ мене Господь, а я неисправенъ. Къ вѣчности дверь мнѣ отворилась, о которой я не думалъ никогда. О міръ, міръ!... \textit{Суета суетствій и всяческая суета!} И тако \textit{изыдетъ духъ его, и возвратится въ землю свою; въ той день погибнутъ вся помышленія его}\footnote{Пс.~145,~4.}. Тако кончается, который долгое себѣ обѣщалъ житіе; вкушаетъ горести смертной, который хотѣлъ гулять и веселиться; остается единъ, который хотѣлъ имѣть множество слугъ и крестьянъ; лежитъ на одрѣ смертномъ, который думалъ проѣзжаться каретою и цугомъ; полагается во гробѣ, который мыслилъ въ богатомъ и красномъ домѣ жить; зарывается въ треаршинной ямѣ, который думалъ землю свою разширять; со слезами и плачемъ провождается въ путь всея земли, который желалъ увеселять себе музыкою! Тако кончина нечаянно приходитъ, и Господь зоветъ насъ тогда, когда мы не чаемъ. \textit{Отврати очи мои}, Господи, \textit{еже не видѣти суеты: въ пути Твоемъ живи мя}\footnote{Пс.~118,~37.}! Хрістіанине несмысленный, котораго сердце прилѣпилося къ суетѣ и о своемъ спасеніи нерадишь! видишь, что дѣлается людямъ: ожидай того и себѣ. Постигаетъ ихъ кончина: постигнетъ и тебе. Зоветъ ихъ Господь и Судія къ Себѣ: позоветъ и тебе; и позоветъ, когда не чаешь, и гдѣ не чаешь, и какъ не чаешь. Трепещутъ они суда Божія при кончинѣ и мятутся: вострепещешь и ты и возмятешися. Готовься же заранѣе, чтобы не такъ страшенъ былъ тебѣ часъ тотъ, всѣмъ страшный. Блаженъ будеши, когда, отвратившися отъ суеты заранѣе, будеши разсуждать о часѣ томъ, въ которомъ дверь въ вѣчность всякому отворяется!

\section{40. Раби, ожидающіе господина своего.}

Бываетъ, что господинъ отходитъ изъ дому своего ради извѣстнаго ему дѣла. Тако Хрістосъ Господь, совершивши великое спасенія нашего дѣло отшелъ отъ міра сего и, вознесшися на небо, сѣлъ одесную Бога Отца. Господинъ, исходя изъ дому своего, глаголетъ рабамъ своимъ: сѣдите дома, и всякъ свое дѣло дѣлайте; я скоро возвращусь. Тако Хрістосъ Господь, отходя отъ міра сего, приказалъ намъ, рабамъ Своимъ, глаголя: \textit{Я пріиду паки} къ вамъ, и пріиду скоро. \textit{И се гряду скоро, и мзда Моя со Мною, воздати комуждо по дѣломъ его}\footnote{Іоан.~14,~3; Апок.~22,~12.}. Будите готови: \textit{да будутъ чресла ваша препоясана, и свѣтильницы горящіи: и вы подобни человѣкомъ чающимъ Господа своего, когда возвратится отъ брака, да пришедшу и толкнувшу абіе отверзутъ ему}\footnote{Лук.~12,~35--36.}. Вѣрніи и добріи раби дѣлаютъ, что господинъ имъ приказалъ, и ожидаютъ прихода его: тако вѣрніи и добріи раби Хрістовы дѣлаютъ по слову Хріста Господа своего, ожидаютъ пришествія Его и готовятся къ срѣтенію Его. \textit{Наше житіе на небесѣхъ есть, отонудуже и Спасителя ждемъ, Господа нашего Іисуса Хріста}\footnote{Филип.~3,~20.}. Бываетъ, что раби, не чая скораго пришествія господина своего, такъ говорятъ: не скоро еще пріидетъ господинъ нашъ. И начнутъ гулять и безчинствовать. Но тутъ нечаянно слышится вопль: се, господинъ приходитъ! Тако хрістіане, недобрыи и нерадящіи о спасеніи своемъ, помышляя въ сердцѣ своемъ: \textit{коснитъ Господь нашъ пріити}, начнутъ гулять и нечествовать. Но тутъ внезапу услышится вопль и гласъ трубный: \textit{се Господь грядетъ! исходите въ срѣтеніе Ему}. Блаженни раби тіи, которыхъ господинъ ихъ пришедъ обрящетъ бдящихъ и исходящихъ въ срѣтеніе ему: тако \textit{блажени} суть хрістіане тіи, которыхъ Хрістосъ Господь пришедъ \textit{обрящетъ бдящихъ и готовыхъ} къ срѣтенію Его. Окаянніи раби, которыхъ господинъ пришедъ увидитъ неисправныхъ: тако окаянніи тіи хрістіане, которыхъ Хрістосъ Господь пришедъ обрящетъ невѣрныхъ Себѣ и по своимъ прихотямъ ходящихъ. Неисправныи раби предаются отъ господина въ наказаніе: тако хрістіане, не творящіи воли Господа своего Іисуса Хріста, предадутся въ вѣчное наказаніе. \textit{И неключимаго раба вверзите во тму кромѣшнюю, ту будетъ плачъ и скрежетъ зубомъ, бдите убо, яко не вѣсте, въ кій часъ Господь вашъ пріидетъ!} Якоже бо неизвѣстну кончину опредѣлилъ намъ Хрістосъ Господь, да всегда ее ожидаемъ: тако и неизвѣстно пришествіе Свое въ міръ положилъ, да на всякій день Его чаемъ. И якоже не знаемъ, когда насъ позоветъ Господь нашъ къ Себѣ, да всегда готовы будемъ: тако не знаемъ дне и часа, когда Онъ къ намъ пріидетъ, да на всякій день и часъ готовы будемъ въ срѣтенію Его. Хрістіанине! якоже видишь, что человѣкъ кончается тогда, когда не чаетъ, и отходитъ на оный вѣкъ: тако разумѣй и помышляй, что и Господь пріидетъ къ намъ нечаянно. Тогда нечаянно услышится вопль: \textit{се Женихъ грядетъ! исходите въ срѣтеніе Ему}\footnote{Мѳ.~24,~27,~51; 25,~1--13,~30.}. Услышится глазъ трубный: возставайте мертвіи! идите на судъ! \textit{Тогда явится знаменіе Сына человѣческаго на небеси; и тогда восплачутся вся колѣна земная, и узрятъ Сына человѣческаго, грядуща на облацѣхъ небесныхъ съ силою и славою многою. Якоже бо молнія исходитъ отъ востокъ и является до западъ, тако будетъ пришествіе Сына человѣческаго}. Якоже убо ожидаешь кончины твоея и готовишися къ ней, да блаженно скончаешися: тако ожидай и пришествія Господа твоего и готовися къ срѣтенію Его, да сподобившися въ пришествіи Его одесную Его стати, и услышати съ слышащими сладчайшій гласъ Его: \textit{пріидите благословенніи Отца Моего, наслѣдуйте уготованное вамъ царствіе отъ сложенія міра}\footnote{Мѳ.~25,~34.}.

\section{41. Зовомый на судъ человѣкъ.}

Видимъ, что когда человѣкъ на судъ позывается, о томъ и думаетъ и все тщаніе свое полагаетъ, и съ другами своими совѣтуетъ, дабы на судѣ не осужденнымъ быть и не посрамиться. Хрістіанине! ты позовешися на судъ не человѣческій, но Божій; \textit{всѣмъ бо явитися намъ подобаетъ предъ, судищемъ Хрістовымъ, да пріиметъ кійждо, яже съ тѣломъ содѣла, или блага, или зла}\footnote{2~Кор.~5,~10.}: кольми убо паче должно тебѣ готовитися и все тщаніе полагать, чтобы на судѣ ономъ не посрамиться и не осужденнымъ быть. Часто бываетъ, что человѣкъ отъ суда человѣческаго и убѣгаетъ: отъ суда онаго никто избыть не можетъ; ибо неотмѣнно будетъ, и будетъ \textit{всѣмъ}. Тако сіе истинно есть, какъ истинно есть, что вчерашній день былъ и сегоднешній есть. На судѣ человѣческомъ требуются свидѣтели ради изысканія истины, которыи часто и виноватаго оправдаютъ; на судѣ ономъ не тако: Судія оный не требуетъ свидѣтелей, но Самъ все знаетъ. На судѣ человѣческомъ часто помогаетъ сребро и злато виноватому, какъ то бываетъ на беззаконныхъ судахъ; на судѣ ономъ не тако: ибо Судія оный даровъ не требуетъ и не пріемлетъ. На судѣ человѣческомъ часто защищаютъ виноватаго высокія и сильныя лица и прочіи помощники: на ономъ судѣ ничего они не могутъ; они и сами тогда скрыются и будутъ, какъ подлыи. На судѣ человѣческомъ часто лица пріемлются; часто богатый убогому, почтенный подлому, благородный худородному предпочитается: на судѣ ономъ не тако; ибо Судія оный лицъ не пріемлетъ, и смотритъ не на лица, но на совѣсть и дѣла; тогда рядомъ предъ Нимъ вси, раби и господа, цари и подданныи ихъ, богатыи и нищія станутъ. На судѣ человѣческомъ часто хитрость оправдаетъ виноватаго: на ономъ судѣ она умолкнетъ и онѣмѣетъ; тамо ей мѣста не будетъ; ибо Судія, яко всевѣдущій, все знаетъ. На судѣ человѣческомъ судится человѣкъ предъ немногими: на судѣ ономъ будетъ судитися предъ всѣмъ свѣтомъ, предъ ангелами и человѣками. На судѣ человѣческомъ пороки судимаго немногіе знаютъ: на судѣ ономъ судимыхъ грѣшниковъ дѣла, слова и помышленія злая предъ всѣмъ свѣтомъ обличатся. На судѣ человѣческомъ предъ немногими стыдъ и срамъ судимому бываетъ: на судѣ ономъ предъ всѣмъ міромъ постыдится и посрамится бѣдный грѣшникъ. Отъ суда человѣческаго не ино что послѣдуетъ виноватому, только временная смерть, или иная какая казнь временная: отъ суда онаго не временной, но вѣчной смерти предастся осужденный грѣшникъ. "--- Видишь, хрістіанине, нѣкоторыи обстоятельства человѣческаго и Хрістова онаго суда; видишь и тое, какъ человѣкъ со всякимъ опасеніемъ готовится къ суду человѣческому, который предъ судомъ онымъ есть какъ ничто. Однакожъ со всякимъ прилѣжаніемъ готовятся люди къ суду сему, да не посрамятся. Хрістіанине неисправный, который прилѣпился къ суетѣ міра сего, который не престаешь безстрашно законъ Божій нарушать, который за мало, или, что горше того, за ничто грѣхъ поставляешь! пощади душу твою бѣдную; возведи очи твои мысленныя къ страшному оному всемірному позорищу, на которомъ и за \textit{праздное слово воздадятъ} люди \textit{отвѣтъ}\footnote{Мѳ.~12,~36.}. И готовься со всякимъ опасеніемъ ко дню оному, отъ котораго слѣдуетъ всякому итить или въ муку вѣчную, или въ животъ вѣчный. Подражай въ семъ важномъ дѣлѣ хотя сыномъ вѣка сего, которыи, зовомы на судъ человѣческій, со всякимъ тщаніемъ къ Тому себя приготовляютъ, да не постыдятся. Ты предстанешь судищу Божію, а не человѣческому, на которомъ не будетъ лицепріятія: помышляй убо о томъ заранѣе, и готови себе къ великому оному дню!

\subsection{О томжде.}

Бываетъ, что люди, сдѣлавше какое преступленіе и узнавше свой грѣхъ, пока не позовутся на судъ, приходятъ къ царю или иной какой своей власти, со смиреніемъ грѣхъ свой исповѣдуютъ, падаютъ предъ властію и просятъ прощенія, и получаютъ. Хрістіанине! знаешь ты, что позовешися на судъ Божій, и за всѣ твои грѣхи, дѣломъ, словомъ и помышленіемъ содѣланныя, истязанъ будеши: сотвори убо и ты, какъ разумныи сыны вѣка сего дѣлаютъ; обратися отъ грѣховъ твоихъ и пріиди ко Хрісту Судіи, Царю небесному, Которому ты согрѣшилъ, и исповѣдуй предъ Нимъ грѣхи твои со смиреніемъ; падай предъ Нимъ, и осуждай самъ себе нынѣ предъ Нимъ, да не тогда отъ Него осудишися; зови къ Нему мытаревымъ гласомъ: \textit{Боже, милостивъ буди мнѣ грѣшному! И твори плоды достойны покаянія}\footnote{Лук.~18,~13; 3,~8.}. Тогда всѣ твои грѣхи и беззаконія загладятся и не помянутся; ибо Онъ будетъ судить не несогрѣшившихъ, но согрѣшившихъ и непокаявшихся; на то бо и покаяніе проповѣдалося, да покаемся согрѣшившіи, и получимъ отъ Него милость. Кайся убо, и заглаждай воздыханіемъ и слезами грѣхи твоя, въ совѣсти твоей написанныя, да и въ Божіей книзѣ загладятся и не явятся на судѣ ономъ. Се есть приготовленіе къ страшному оному суду! Инаго кромѣ сего нѣтъ. Буди убо всегда въ истинномъ покаяніи: и будешь готовиться къ суду оному, на которомъ неотмѣнно явишися и ты; и тако, наченши новое хрістіанское житіе, ожидай отъ Него милости. Кающемуся и въ новомъ житіи хрістіанскомъ живущему, уже прежнее его беззаконное житіе не повредитъ; едино бо только отъ насъ требуется, чтобы мы исправилися и премѣнили себе въ лучшее. Помышляй всегда о дни томъ и твердо держи, что ты, какъ и прочія, на оный судъ, судъ Божій, а не человѣческій, позовешися, и позовешися нечаянно, когда вострубитъ архангельская труба. Сіе самое размышленіе подвигнетъ тебе къ истинному покаянію и содержать будетъ во смиреніи и сокрушенія сердца. Память и размышленіе онаго суда не попуститъ тебѣ грѣхъ творить, ближнему отмщать; подвигнетъ къ прилѣжной и усердной молитвѣ. Памятуя судъ оный, не будешь искать веселыхъ дней въ мірѣ, но паче будешь желать слезъ, плача и воздыханія. Что люди веселятся, и плоти угождаютъ и беззаконнуютъ, то бываетъ отъ забвенія и неразмышленія о судѣ томъ. Помни убо судъ, и будеши истинно каяться и по всякій день обновляться и въ лучшее измѣняться. Тако не тотъ будеши, какъ прежде былъ. Якоже бо огнь всякій смрадъ: тако страхъ суда онаго всякую гнилость и смрадъ отъ души будетъ изгонять и ее часъ отъ часу очищать.

\subsection{О томжде.}

Видимъ въ мірѣ, что на судъ неправедный къ царю или иной какой власти аппелляція или доносъ бываетъ, и судившіе сами къ суду позываются и судятся судіи неправедныи. Тако неправости, насилія, озлобленія, обиды, которыя бѣднымъ людямъ показуются, и суды неправедные царей, князей, господъ, пастырей и прочихъ властей на небо вопіютъ, и вопіенія ихъ во уши Господа Вседержителя, Царя царствующихъ и Господа господствующихъ входятъ, и суда отъ Него просятъ. Бѣдныи люди, не имѣя гдѣ сыскать, кто бы ихъ плачь услышалъ, и обиды и насилія отвратилъ, и слезы отерлъ, на небо къ Богу, Отцу сирыхъ и Судіи вдовицъ, воздыхаютъ и вопіютъ: Боже! Ты нашъ Создатель и Господь, Ты нашъ Защитникъ и Помощникъ. Мы на земли бѣдности и слезамъ нашимъ не находимъ удовольствія: Ты услыши и помилуй насъ бѣдныхъ, и отри слезы озлобленнаго созданія Твоего! Сильныи сильнымъ, благородныи благороднымъ, господа господамъ, богатыя богатымъ, судіи судіямъ помогаютъ: мы нигдѣ не сыщемъ помощи себѣ. У Тебе убо просимъ! \textit{Тебѣ оставленъ есть нищій, сиру Ты буди Помощникъ}\footnote{Пс.~9,~35.}. Видитъ Господь беззаконныхъ властей и судей беззаконія, и вопль ихъ, какъ содомскій, входитъ ко Господу, и воздыханія убогихъ людей Своихъ слышитъ, и судитъ \textit{судяй всей земли}. Отсюду"=то видимъ страшныя казни, отъ праведнаго суда Божія посылаемыи на государства и грады; видимъ страшныя войны и кровопролитія, видимъ моровыя язвы; видимъ прелестниковъ и пагубныхъ сыновъ, и возмутителей отечества, которые высокихъ лицъ себѣ имена приписуютъ, и мятежъ и вредъ отечеству дѣлаютъ; видимъ страшные пожары и глады и прочія казни. Сія вся Богъ, Царь царствующихъ и Господь господствующихъ, за неправедные суды, озлобленія и насилія бѣдныхъ людей, какъ Судія праведный и судяй всей земли, насылаетъ. Но позовутся еще на всемірный судъ вси судіи и власти; тогда станутъ сами предъ Судіею Богомъ судившіи людямъ Его, и воздадятъ отвѣтъ; станутъ цари и воздадятъ слово; станутъ князи и вельможи и воздадятъ слово; станутъ господа и воздадятъ слово; станутъ пастыри и воздадятъ слово; станутъ вси власти, отъ высшихъ до низшихъ, и воздадятъ слово за неправедныя суды и безсовѣстные и беззаконные съ людьми имъ порученными поступки. Горе тогда будетъ беззаконнымъ властямъ и людей Божіихъ неправдою и мучительствомъ озлобившимъ! \textit{Иже} (Богъ) \textit{истяжетъ дѣла ваша, и помышленія испытаетъ: яко слузи суще царства Его, не судисте право, ни сохранисте закона, ниже по воли Божіей ходите. Страшно и скоро явится вамъ: яко судъ жесточайшій преимущимъ бываетъ. Ибо малый достоинъ есть милости: сильніи же сильнѣ истязани будутъ}\footnote{Прем.~6,~3--6.}. Власти, господа и судіи! дѣла ваша неправая на небо вопіютъ, и Богу какъ бы доносъ чинятъ на васъ.; слышитъ Богъ вопль ихъ, и позоветъ васъ къ Себѣ на судъ: покайтеся убо, возлюбленніи, пока еще не зоветъ васъ! Онъ вси ваши дѣла и неправости видитъ; но \textit{долготерпитъ} на всѣхъ, \textit{не хотя, да кто погибнетъ, но да вси въ покаяніе пріидутъ}\footnote{2~Петр.~3,~9.}. А когда нынѣ не покаетеся, то уже, когда позоветъ васъ къ Себѣ на судъ, будете каятися, но поздно и безполезно! Разсуждайте приведенное Соломоново слово, и сами узнаете, коль страшенъ судъ Божій вамъ будетъ. \textit{Господи Боже силъ! обрати ны, и просвѣти лице Твое, и, спасемся}\footnote{Пс.~79,~4.}.

\subsection{О томжде.}

Бываетъ, что когда преступники и злодѣи на судъ ведутся, то приводятся и съ кандалами тѣми, которыми окованы. Тако бѣдныи грѣшники приведутся на судъ Божій съ грѣхами своими, тѣми, какъ узами, связаны, и съ ними явятся предъ праведнымъ Судіею и предъ всѣмъ свѣтомъ; и что въ тайномъ мѣстѣ дѣлали, или въ сердцѣ помышляли худое, тое все въ познаніе пріидетъ всѣмъ ангеломъ и человѣкомъ Тогда съ блудникомъ и прелюбодѣемъ блуды и прелюбодѣянія, съ воромъ и хищникомъ всякое хищеніе и воровство, съ лестцемъ, лживымъ и обманщикомъ лесть и ложь, съ хитрецомъ, лукавымъ и лицемѣромъ вси хитрые замыслы, и лицемѣрныя дѣла, съ клеветникомъ, ругателемъ и злорѣчивымъ вси его поносныя и пагубныя слова явятся на судѣ ономъ. Такожде со всякимъ виновнымъ человѣкомъ явятся злая дѣла его тако; со всякою властію нерадѣніе о подвластныхъ и прочія беззаконія его: съ пастыремъ нерадѣніе о паствѣ и прочіе грѣхи его; съ судіею неправедные суды и прочія беззаконія его; съ господиномъ обиды и озлобленія, которыя показывалъ слугамъ и крестьянамъ своимъ, и прочія беззаконія его будутъ съ нимъ тамо. Словомъ: со всякимъ грѣшникомъ вси дѣла его, слова и помышленія богопротивныя, яко соперники и обличители, на страшномъ ономъ испытаніи явятся, аще отъ нихъ здѣ въ мірѣ семъ истиннымъ покаяніемъ, сокрушеніемъ сердца и слезами не избавится. Тогда о всякомъ грѣшникѣ скажется: \textit{се человѣкъ, и дѣла, его!} Се человѣкъ, который \textit{христіаниномъ} назывался, но хрістіанскихъ дѣлъ не творилъ; рабомъ Божіимъ назывался, но діаволу работалъ. \textit{Се человѣкъ и дѣла его!} О бѣдный грѣшникъ! горе тебѣ будетъ тамо, когда отъ сихъ лютыхъ соперниковъ и клеветниковъ твоихъ здѣ въ мірѣ семъ не избудешь! Потщися убо, возлюбленне, нынѣ заранѣе отъ нихъ избыть, да не съ тобою на оный вѣкъ пойдутъ; свободи душу твою отъ нихъ, да не тогда обыдутъ тебе, и посрамятъ тя; остави ихъ, да и они оставятъ тя; отстани отъ нихъ и отступи, да и они отстанутъ отъ тебе и отступятъ, да тако свободенъ будеши отъ нихъ. А за содѣланныя кайся и повергай себя предъ милосердыми очами Іисусовыми, да Своею кровію, тебе ради изліянною, очиститъ тебя отъ нихъ: тогда воистину чистъ и свободенъ будеши! \textit{Аще Сынъ вы свободитъ, воистину свободни будете}\footnote{Іоан.~8,~36.}. \textit{Образъ есмь неизреченныя Твоея славы, аще и язвы ношу прегрѣшеній; ущедри Твое созданіе, Владыко, и очисти Твоимъ благоутробіемъ, и вожделѣнное отечество подаждь ми, рая паки жителя мя сотворяя!}

\subsection{О томжде.}

Позванный на судъ отъ засѣдающихъ на судѣ судится, отъ свидѣтелей обличается, отъ судей выговоръ терпитъ: ты"=де тое и тое сдѣлалъ, такой"=то законъ и указъ нарушилъ, и проч. Тако грѣшникъ бѣдный, здѣ покаяніемъ отъ грѣховъ не очистившійся и на судъ оный позванный, будетъ судитися отъ Бога, обличатися, и увидитъ грѣхи свои, предъ собою поставленные, и услышитъ отъ праведнаго Судіи выговоръ: \textit{ты возненавидѣлъ еси наказаніе; и отверглъ еси словеса Моя вспять. Аще видѣлъ татя, теклъ еси съ нимъ, и съ прелюбодѣемъ участіе твое полагалъ еси. Уста твоя умножиша злобу, и языкъ твой сплетеніе льщенія; сѣдя на брата, твоего клеветалъ еси, и на сына матери твоея полагалъ еси соблазнъ}\footnote{Пс.~49,~17--20.}. И паки; \textit{взалкахся, и не дасте Ми ясти; возжадахся, и не напоисте Мене; страненъ бѣхъ, и не введосте Мене; нагъ, и не одѣясте Мене; боленъ, и въ темницѣ, и не посѣтисте Мене}\footnote{Мѳ.~25,~42 и 43.}. Тогда сбудется слово Божіе, грѣшнику въ Писаніи Святомъ реченное: \textit{обличу тя, и представлю предъ лицемъ твоимъ грѣхи твоя}\footnote{Пс.~49,~21.}. Стоящаго на судѣ и обличаемаго объемлетъ стыдъ и страхъ: тако грѣшниковъ на ономъ судѣ стыдъ безмѣрный, страхъ, трепетъ и ужасъ обыметъ; обличатся бо отъ Самого Бога, Котораго не хотѣли почитать; обличатся за неблагодарность, которую Ему доказывали; обличатся предъ всѣмъ міромъ. Обличенный на судѣ законопреступникъ выключается изъ числа добрыхъ гражданъ и означается злодѣемъ, а не сыномъ отечества: тако хрістіане законопреступники и не покаявшіеся вычислятся отъ добрыхъ хрістіанъ на судѣ ономъ, и, якоже козлища, разлучатся отъ овецъ, и означатся невѣрными, а не вѣрными, ложными, а не истинными хрістіанами. Законопреступникъ на судѣ осуждается на смерть или иную казнь, по законамъ опредѣленную: тако на ономъ судѣ грѣшники осудятся на вѣчную казнь. \textit{Идите отъ Мене проклятіи во огнь вѣчный, уготованный діаволу и аггеломъ его}\footnote{Мѳ.~25,~41.}. Осужденный на законную казнь преступникъ отлучается отъ домашнихъ своихъ сродниковъ и друговъ, и идетъ съ неутѣшнымъ плачемъ на определѣнную ему казнь: тако грѣшники, на судѣ ономъ осужденныи на вѣчную казнь, отлучатся отъ самого Бога, отъ святыхъ ангелъ Его и избранныхъ Божіихъ, отлучатся на вѣки безконечные и пойдутъ съ неутѣшнымъ плачемъ, трепетомъ, ужасомъ и отчаяніемъ всеконечнымъ въ вѣчную муку. \textit{И идутъ сіи въ муку вѣчную}\footnote{Мѳ.~25,~46.}. Бѣдный грѣшникъ! покайся, и очищай грѣхи свои сокрушеніемъ сердца и слезами, да не всего сего постраждешъ тамо. Пощади, Господи, пощади созданіе Твое! \textit{Помилуй мя, Боже, по велицѣй милости Твоей, и по множеству щедротъ Твоихъ очисти беззаконіе мое! Отврати лице Твое отъ грѣхъ моихъ, и вся беззаконія моя очисти}\footnote{Пс.~50,~3 и 11.}. \textit{Аще бо беззаконія назриши, Господи, Господи! кто постоитъ}\footnote{129,~3.}? Отврати лице Твое, Владыко, отъ грѣхъ моихъ и обрати лице на Хріста, Господа моего, Іисуса Хріста, единороднаго Сына Твоего, Который за мене недостойнаго и бѣднаго грѣшника пострадалъ и умеръ, и Самого Себе за грѣхи моя въ жертву и въ воню благоуханія Тебѣ, небесному Своему Отцу и моему Создателю, принеслъ; и Того благодатію и человѣколюбіемъ излей на мя милость Твою, и вся беззаконія и грѣхи моя заглади кровію Его, мене ради изліянною! Аминь.

\section{42. Человѣкъ въ ранахъ.}

Видимъ въ мірѣ, что многія и различныя суть болѣзни въ людехъ, между которыми и тое видимъ, что человѣкъ весь въ ранахъ и язвахъ бываетъ. Что человѣку раны и язвы: тое душѣ грѣшной грѣхи и беззаконія. Тѣло уязвляется и покрывается ранами: душа грѣшнаго человѣка уязвлена и ранена грѣхами. Бываетъ, что язвы и раны тѣлесныя смердятъ и гніютъ: тако язвы и раны на душѣ грѣховныя смердятъ и гніютъ; о семъ глаголетъ Псаломникъ: \textit{возсмердѣша и согниша раны моя отъ лица безумія моего}\footnote{Пс.~37,~6.}. Бываетъ, что человѣкъ отъ ногъ до главы весь въ язвахъ и ранахъ: тако и грѣшная душа, во грѣхахъ пребывающая и грѣхи ко грѣхамъ прилагающая, вся въ язвахъ и ранахъ; о семъ глаголется во пророкѣ: \textit{отъ ногъ даже до главы нѣсть въ немъ цѣлости}\footnote{Ис.~1,~6.}. Раны и язвы тѣлесныя чимъ болѣе гніютъ, тѣмъ большій смрадъ издаютъ: тако грѣшная душа, чимъ болѣе грѣшитъ, тѣмъ болѣе уязвляется и ранится и струны своя грѣховныя раздражаетъ и отъ тѣхъ большій смрадъ издаетъ. Раны и язвы тѣлесныя видятъ люди и смрадъ ихъ чувствуютъ: раны и язвы душевныя видитъ Богъ. Тяжекъ и несносенъ смрадъ ранъ тѣлесныхъ людямъ: тяжекъ и несносенъ смрадъ ранъ душевныхъ Богу. Отъ ранъ тѣлесныхъ и смрада ихъ отвращаются люди: отъ ранъ душевныхъ гніющихъ и смердящихъ отвращается Богъ. Люто есть, возлюбленный хрістіанине, быть человѣку всему въ ранахъ, и ранахъ гнойныхъ и смердящихъ; но далеко \textit{лютѣе}\footnote{Лютѣе "--- въ подлинникѣ стояло болѣе, но, какъ очевидно ошибочное, исправлено, собственною рукою святителя. О значеніи подобныхъ исправленій см. въ Предисл.} есть душѣ, "--- быть въ своихъ ранахъ грѣховныхъ и смердящихъ. Тѣло бо смертно и тлѣнно: но душа безсмертна и нетлѣнна; когда нынѣ отъ ранъ своихъ не исцѣлится, въ тѣхъ своихъ ранахъ предъ Судіею на судѣ ономъ станетъ, и вовѣки такова пребудетъ. Умиленія и сожалѣнія достоинъ предъ глазами твоими обращается, человѣче, человѣкъ въ ранахъ; созданіе Божіе любезное, по образу Божію созданное, въ такую гнусность пришедшее. Грѣхъ тому виновенъ. \textit{Тебѣ Господи, правда, намъ же стыдѣніе лица}\footnote{Дан.~9,~7.}. Обрати очи твои отъ сего позорища на душу, которая грѣхами, какъ ранами, уязвлена, и увидишь большаго сожалѣнія и умиленія достойный позоръ! Раны и язвы ея суть гордость, зависть, злоба, нечистота, сребролюбіе и проч. Чимъ болѣе она въ сихъ пребываетъ, тѣмъ болѣе уязвляется; чимъ болѣе уязвляется, тѣмъ болѣе гніютъ и смердятъ раны ея. Бѣдный грѣшникъ! полно уже уязвляться: и такъ смердятъ и гніютъ раны души твоея. Пора уже лѣчиться, пора пластыри покаянія къ язвамъ и ранамъ прилагать. Тѣло больное лѣчишь: душа вся отъ ранъ и язвъ изнемогаетъ, и небрежешь! О бѣдный грѣшники! прибѣгнемъ съ вѣрою ко Іисусу Хрісту, врачу душъ и тѣлесъ, и, не смѣя къ Нему приближитися ради гніющихъ и смердящихъ ранъ, хотя издалеча станемъ, и гласъ десяти прокаженныхъ изъ глубины сердца вознесемъ къ Нему: \textit{Іисусе, наставниче, помилуй ны}\footnote{Лук.~17,~12--13.}. Нѣтъ такъ лютыхъ ранъ и язвъ, которыхъ бы Онъ не хотѣлъ и не могъ \textit{словомъ единымъ} исцѣлить; ибо есть всесиленъ, милосердъ и человѣколюбивъ. Тѣлеса смертная и тлѣнная исцѣлялъ, какъ повѣствуетъ Евангеліе: душъ ли безсмертныхъ, которыхъ пришелъ спасти, не восхощетъ исцѣлить? Исцѣлитъ воистину, аще съ вѣрою пріидемъ къ Нему и со усердіемъ будемъ просить отъ Него исцѣленія. Онъ раны и язвы наши видитъ и хощетъ исцѣлить, на сіе бо и пришелъ въ міръ; но хощетъ, чтобы мы признали предъ Нимъ неисцѣльныя нами язвы наши, которыя единъ Онъ только можетъ исцѣлить. Исцѣли мя, Господи, яко согрѣшилъ Тебѣ! \textit{Исцѣли мя, и исцѣлюся: яко Ты еси Богъ Спасъ мой}\footnote{Іер.~17,~14.}! \textit{Господи! нѣсмь достоинъ, да подъ кровъ мой внидеши; но токмо рцы слово, и исцѣлѣетъ отрокъ мой} (душа моя)\footnote{Мѳ.~8,~8.}. \textit{Образъ есмь неизреченныя Твоея славы, аще и язвы ношу прегрѣшеній: ущедри Твое созданіе, Владыко, и очисти Твоимъ благоутробіемъ, и вожделѣнное отечество подаждь ми, рая паки жителя мя сотворяя!}

\subsection{О томжде.}

Когда видимъ человѣка въ ранахъ, или во иной какой лютой болѣзни, сожалѣемъ ему и соболѣзнуемъ: тако наипаче, когда видимъ человѣка во грѣхахъ находящагося, должно о немъ сожалѣть и соболѣзновать; ибо общее всѣмъ намъ естество, общее бѣдствіе и окаянство. Аще убо тѣлесному бѣдствію человѣческому соболѣзнуемъ и состраждемъ: кольми паче душевному бѣдствію ближняго соболѣзновать и сострадать должно. Аще лютая болѣзнь тѣлесная ближняго нашего подвигаетъ насъ на милосердіе: кольми паче душевная болѣзнь его должна подвигнуть на милосердіе и состраданіе. Душа бо честнѣйшая и дражайшая есть, нежели тѣло: потому и бѣдствіе и окаянство ея лютѣйшее и опаснѣйшее есть, нежели тѣлесное. Тѣло смертно и тлѣнно и согніетъ въ землѣ, хотя и здраво будетъ: душа безсмертна, и во вѣки неисцѣльна пребудетъ, аще нынѣ не исцѣлѣетъ. Важна убо причина есть соболѣзновать и сострадать въ ранахъ и язвахъ душевныхъ находящемуся и язвы грѣховныя къ язвамъ прилагающему. Почто жъ убо, человѣче, видя брата твоего согрѣшающаго, осуждаешь его, въ крайней опасности находящемуся не сожалѣешь? Надъ тѣломъ страждущимъ умилостивляешися; хорошо то и похвально: почто жъ надъ душею страждущею не умилостивляешься, но паче судишь и осуждаешь? Лучше въ такомъ случаѣ о братѣ ко Господу воздохнуть, и сказать: Господи, помилуй и избави его отъ того окаянства, а мене удержи всесильною десницею Твоею отъ того. Но обрати и на себе очи твои и посмотри въ совѣсть твою, и увидишь, что и ты въ подобныхъ ранахъ находишься, а можетъ быть, что и въ большихъ и гнуснѣйшихъ. Сіе бо самое, что ближняго твоего судишь и осуждаешь, весьма ранитъ и уязвляетъ душу твою и гнусною и смердящею предъ Богомъ дѣлаетъ: \textit{судить бо есть дѣло Сына Божія}, которое дѣло ты себѣ дерзновенно похищаешь, да еще грѣшникъ будучи\footnote{Іоан.~5,~22.}. Почто жъ убо грѣшникъ грѣшника, и уязвленный уязвленнаго презираешь, уничижаешь и судишь? Ни чимъ ты лучшій отъ него. Я, ты и онъ и прочіи вси равны въ себѣ: \textit{вси бо согрѣшиша, и лишени суть славы Божія}\footnote{Римл.~3,~23.}. Такъ насъ уязвилъ врагъ нашъ сатана, что безъ помощи Божіей не ино что дѣлаемъ, какъ только падаемъ и уязвляемся! "--- \textit{Помяни насъ, Господи, во благоволеніи людей Твоихъ, посѣти насъ спасеніемъ Твоимъ, видѣти во благости избранныя Твоя, возвеселитися въ веселіи языка Твоего, хвалитися съ достояніемъ Твоимъ. Согрѣшихомъ со отцы нашими, беззаконновахомъ, неправдовахомъ. Боже! ущедри ны и благослови ны; просвѣти лице Твое на ны, и помилуй ны "--- познати на земли путь Твой, во всѣхъ языцѣхъ спасеніе Твое}\footnote{Пс.~105,~4--6; 66,~2 и 3.}.

\section{43. Слѣпый удерживаемый.}

Бываетъ, что когда слѣпый идетъ самъ о себѣ и приходитъ ко рву, или иному какому вредному мѣсту, гдѣ ему великая можетъ приключиться бѣда, другой человѣкъ, видя то и хотя его остерещи отъ того бѣдствія, въ слѣдъ его кричитъ: \textit{возвратись! не туды пошелъ ты}; или, вземши за руку его, отводитъ его отъ того опаснаго мѣста. Возлюбленный хрістианине! братъ твой и ближній твой, едино съ тобою созданіе Божіе, по образу Божію и по подобію сотворенное, идетъ въ ровъ погибели, какъ слѣпый, когда беззаконнуетъ. Блудникъ и прелюбодѣй идетъ въ ровъ погибели; гнѣвливый и памятозлобивый и убійца идетъ въ ровъ погибели; воръ и хищникъ и лихоимецъ идетъ въ ровъ погибели; сквернословецъ, кощунникъ и буесловецъ идетъ въ ровъ погибели; клеветникъ, злорѣчивый и ругатель идетъ въ ровъ погибели; всякъ законопреступникъ идетъ въ ровъ погибели, и погибели вѣчныя. Ахъ, идетъ какъ слѣпый, и имѣетъ впасти, и не изыдетъ оттуду! Закричи убо въ слѣдъ его, возлюбленне, когда видишь его идуща, закричи; \textit{брате! не туды идешь}; тамо ровъ предъ тобою ископанъ, въ который имѣешь впасти, и никогда оттуду не изыти. Удержи, возлюбленне, удержи его, пока еще не впалъ! Братъ твой онъ, ближній твой онъ, сродное тебѣ естество; человѣкъ, какъ и ты, душею разумною почтенный, идетъ въ ровъ погибели. Умилосердися убо надъ нимъ, и возврати его, какъ можешь! Онъ не знаетъ самъ, куды идетъ; врагъ ему помрачилъ глаза своею прелестію. Жалѣешь о слѣпцѣ, идущемъ въ ровъ видимый, которому временное только бѣдствіе приключится: кольми паче жалѣть должно о такомъ слѣпцѣ, который идетъ и стремится въ ровъ вѣчный погибели. "--- Сіе дѣло, хотя и не всякаго хрістіанина надлежитъ, яко того хрістіанская любовь требуетъ, однакожъ наипаче надлежитъ до пастырей церковныхъ, епископовъ и іереевъ, до родителей, до властей, до господъ. Всякъ бо властелинъ, аще хрістіанинъ есть и надъ хрістіанами поставленъ, есть пастырь имъ; а долженъ, сколько возможно, людей, порученныхъ себѣ, стремящихся къ погибели, удерживать отъ той, и жезломъ, даннымъ ему отъ Бога, возвращать на путь спасенія. Тако писано о Давидѣ царѣ Израилевѣ: \textit{и избра Давида раба Своего, пріятъ его отъ стадъ овчихъ; отъ доилицъ поятъ его, пасти Іакова раба Своего, и Израиля достояніе Свое: и упасе я въ незлобіи сердца своего, и въ разумѣхъ руку своею наставилъ я есть}\footnote{Пс.~77,~70--72.}. Возлюбленніи пастыри! вамъ поручилъ Господь стадо Свое: пасите его, и отъ волковъ духовныхъ сохраняйте. Родители! вамъ далъ Господь дѣтей: воспитывайте ихъ въ страсѣ Божіи и наказаніи; и юность склонную ко злу, возвращайте отъ зла. Власти! васъ поставилъ Господь надъ людьми Своими, и далъ вамъ мечь: отсѣкайте тѣмъ отъ подчиненныхъ вашихъ всякое злое. Господа! вы не токмо стрищи овечекъ Хрістовыхъ должны, но и пасти ихъ. Но что, когда самъ пастырь и властелинъ, какъ слѣпый, идетъ въ ровъ погибели, и за собою порученныхъ себѣ людей влечетъ? \textit{Аще свѣтъ тьма, то тма кольми паче}\footnote{Мѳ.~6,~23.}. О, Боже преблагій и милосердый! отврати злое сіе отъ Израиля. \textit{Пасый Израиля вонми! не помяни нашихъ беззаконій первыхъ; скоро да предварятъ ны щедроты Твоя, Господи, яко обнищахомъ зѣло. Помози намъ, Боже Спасителю нашъ, славы ради имене Твоего. Господи! избави ны, и очисти грѣхи наша, имене ради Твоего, да не когда рекутъ языцы; гдѣ есть Богъ ихъ? Господи Боже силъ, обрати ны, и просвѣти лице Твое, и спасемся}\footnote{Пс.~79,~2 и 4; 78,~8--10.}.

\section{44. Древо высокое посредѣ малаго лѣса.}

Какъ древо высокое посредѣ низкаго лѣса: такъ человѣкъ, на чести пребывающій, посредѣ подлыхъ людей. Древо высокое отъ всѣхъ издалека видно: тако на чести сѣдящаго человѣка вси видятъ. Чимъ высшее древо, тѣмъ далѣе и отъ множайшихъ видится: тако чимъ высшій властелинъ, тѣмъ множайшіи люди его знаютъ, и что дѣлаетъ, говоритъ, вѣдаютъ. Высокое древо всякому вѣтру и бурѣ подлежитъ, и отъ всякой непогоды обуревается: тако и высокій властелинъ всякому \textit{искушенію}\footnote{Искушенію "--- въ подлинникѣ стояло утѣшенію и, какъ очевидно ошибочное, исправлено собственною рукою святителя. О знач. подоб. исправленій см. въ Предисловіи.} и бѣдствію подлежитъ; и чимъ высочайшій властелинъ, тѣмъ большему и подверженъ искушенію. На малое древо не всякій находитъ вѣтръ; а на высокое всякій и отвсюду вѣетъ и колеблетъ его. О, возлюбленне, который сидишь на высокомъ мѣстѣ! буди какъ древо великое, которое корень свой глубоко въ землю пущаетъ, и тѣмъ содержится и не падаетъ отъ вѣтра и бури; углубляй и ты вѣру и надежду твою въ благости и всемогуществѣ Божіи, какъ якорь въ глубинѣ морской; и такъ укрѣпляй сердце твое, отъ козней вражіихъ какъ отъ бурнаго вѣтра колеблемое. Не бойся, хотя и великая буря то отсюду, то оттуду востаетъ и вѣетъ на тебе. Буди только вѣренъ Господу твоему; а Онъ всесильною Своею рукою удержитъ и укрѣпитъ тебе. \textit{Не даждь во смятеніе ноги твоея, ниже воздремлетъ храняй тя. Се не воздремлетъ, ниже уснетъ храняй Израиля. Господь сохранитъ тя, Господь покровъ твой на руку десную твою. Во дни солнце не ожжетъ тебе, ниже луна нощію. Господь сохранитъ тя отъ всякаго зла, сохранитъ душу твою Господь, Господь сохранитъ вхожденіе твое и исхожденіе твое отъ нынѣ и до вѣка}\footnote{Пс.~120,~3--8.}.

\subsection{О томжде.}

Когда высокое древо, посредѣ малаго лѣса стоящее, падетъ, то далеко слышно паденіе его, и много малаго лѣса, близъ себе стоящаго, сокрушитъ: тако когда въ великій грѣхъ падетъ пастырь, или иный какій властелинъ, далеко паденіе его прослышится, и многихъ соблазнитъ. Не возможно бо, воистину не возможно, паденіямъ пастырскимъ и властелинскимъ утаитися, какъ они ни сокрываютъ. Далеко послышится паденіе ихъ, какъ высокихъ древесъ падшихъ. Тое"=то, или тое сдѣлалъ онъ. Кто? Епископъ или священникъ, губернаторъ или воевода, князь или царь, или иный какій властелинъ, или хозяинъ въ домѣ. Падетъ отецъ, и знаютъ дѣти его; падетъ господинъ, и знаютъ раби его; падетъ пастырь, и знаютъ люди его; падетъ судія, и знаютъ вси; падетъ царь, и вездѣ слышится паденіе его. Тако палъ царь Израилевъ Давидъ, и вси узнали паденіе его, и хотя всякимъ образомъ тщался сокрыть паденіе свое, однако вездѣ пронеслось такъ, что весь свѣтъ узналъ. Видимъ тое и нынѣ; падаютъ и нынѣ власти, и вси узнаютъ паденіе ихъ. О, возлюбленне, котораго Богъ посадилъ на высокомъ мѣстѣ! всею силою берегись паденія, мужайся и крѣпися, да не падеши, да не падши многихъ за собою повлечеши въ паденіе, какъ высокое древо малый лѣсъ. \textit{Мняйся стояти да блюдется, да не падетъ}\footnote{1~Кор.~10,~12.}. Аще же и падеши, то не медли, но скоро востани и исправися, и тако паденіе востаніемъ и исправленіемъ наградится. Палъ Давидъ, но и восталъ, и паденіе свое покаяніемъ и горячими слезами загладилъ, какъ видимъ во псалмахъ его. Подражай и ты, возлюбленне, сему царю Израилеву; востани добрѣ, и покрыется паденіе твое. Къ нему посланъ былъ отъ Бога пророкъ Наѳанъ: къ тебѣ посланъ онъ самъ, падшій и воставшій, который востаніемъ своимъ и исправленіемъ научаетъ тебе востать и исправиться. О вы, въ простотѣ и смиреніи живущіи, овечки Хрістовы! не изнемогайте душею вашею, возлюбленніи, когда услышите паденіе пастыря или инаго какого властелина. Онъ такуюжде немощь имѣетъ, какъ и прочіи люди; но большему и множайшему подлежитъ искушенію, нежели прочіи, какъ выше сказано. Врагъ на видимой брани никого болѣе уязвить и низложить не тщится, какъ начальниковъ воинства: тако врагъ душъ хрістіанскихъ, діаволъ, никого болѣе не тщится въ паденіе вринуть, какъ пастырей и прочихъ хрістіанскихъ властей; о томъ у него все тщаніе, какъ бы повредить и заразить пастыря и начальника хрістіанскаго, который можетъ другихъ пользовать, "--- повредить, говорю, и заразить, дабы не моглъ пользовать, но чтобы и самъ шелъ и прочихъ велъ за собою въ погибель. Берегись убо, возлюбленный хрістіанине, паденіе начальника осуждать, хотя и подлинно знаешь. Много паче берегись паденіе его другимъ открывать, и клеветою соблазнъ разсѣвать, да не уподобишися Хаму, сыну Ноеву, который стыдѣніе отца своего другимъ объявилъ; но паче прикрой молчаніемъ твоимъ, якоже сотворили Симъ и Іафетъ, тогожде Ноя сыны, которыи срамъ отца своего, зря вспять, покрыли. Притомъ знай, что много ложнаго слуха проносится о пастыряхъ и властяхъ хрістіанскихъ; и сіе бываетъ дѣйствомъ общаго всѣхъ врага діавола, дабы соблазнъ разсѣвался, и тако бы всякое нестроеніе и замѣшательство было въ обществѣ хрістіанскомъ.

\section{45. Разумная голова.}

Что въ тѣлѣ голова, тое въ обществѣ хрістіанскомъ пастырь и начальникъ. Когда въ тѣлѣ разумная есть голова, то все тѣло въ добромъ состояніи находится: тако когда въ обществѣ хрістіанскомъ разумный пастырь и начальникъ имѣется, то въ добромъ состояніи общество тое имѣется. Счастливое то есть тѣло, въ которомъ разумная голова: тако блаженное то есть стадо хрістіанское, въ которомъ разумный, добрый и бодрый пастырь. Блаженное тое царство, въ которомъ разумный и добрый царь, царствуетъ; блаженный тотъ градъ и страна, которыми разумный и добрый властелинъ управляетъ; блаженніи тіи раби, надъ которыми разумный господинъ господствуетъ; блаженніи тіи дѣти, у которыхъ разумный и добрый отецъ; блаженный тотъ всякъ домъ, въ которомъ разумный и добрый хозяинъ. Все бо, какъ счастіе, такъ и несчастіе, во всякомъ обществѣ отъ властелина зависитъ, какъ цѣлость тѣла отъ главы. О, возлюбленніи, пастырь и всякій начальникъ! Вы свѣтъ есте обществу хрістіанскому; на васъ смотрятъ вси люди, что дѣлаете и что говорите, какъ вси смотримъ на свѣчу горящую въ нощи: будите убо свѣтъ, да смотрящіи на васъ просвѣтятся; будите какъ зеркало, въ которое смотря, люди отираютъ на своемъ лицѣ пороки; будите какъ разумная голова на тѣлѣ. Аще вы разумны и добры будете, то блаженно и все вамъ подчиненное хрістіанское общество будетъ, и вы блаженнѣйшіи будете: \textit{аще же свѣтъ тма, тма кольми паче?} и аще свѣтильникъ угаснетъ, то чимъ сущіи во тмѣ просвѣтятся? и аще зерцало помрачится, то въ чемъ люди усмотрятъ свои пороки? Будите убо, возлюбленніи, разумны и добры, да и себѣ самимъ и людямъ Божіимъ блаженство пріобрящете! \textit{Тако}, глаголетъ вамъ Хрістосъ Господь, \textit{да просвѣтится свѣтъ вашъ предъ человѣки, яко да видятъ ваша добрая дѣла, и прославятъ Отца вашего, Иже на небесѣхъ}\footnote{Мѳ.~5,~16.}. А понеже все блаженство хрістіанскаго общества отъ разумныхъ и добрыхъ пастырей и начальниковъ зависитъ; то вы, хрістоименитыи люди, не преставайте молить общаго всѣхъ Владыку Хріста, да вразумитъ и наставитъ пастырей и прочіихъ начальниковъ хрістіанскихъ: се бо есть блаженство ваше! \textit{Молю убо прежде всѣхъ творити молитвы, моленія, прошенія, благодаренія за вся человѣки, за царя, и за вся, иже во власти суть: да тихое и безмолвное житіе поживемъ во всякомъ благочестіи и чистотѣ}\footnote{1~Тим.~2,~1--2.}.

\section{46. Желаніе.}

Видимъ, что люди много кое"=чего желаютъ въ мірѣ, и чего желаютъ, того ищутъ. Но пустъ желаютъ и ищутъ люди, чего хотятъ. Хрістіанине! ты желай и ищи своего, то есть, что тебѣ прилично и нужно. Желаютъ люди счастіе и благополучіе въ мірѣ семъ имѣть: ты желай и тщись истинное благочестіе въ сердцѣ своемъ имѣть; се есть хрістіанское желаніе! Желаютъ люди свободитися отъ работы, и быть свободными: ты желай свободитися отъ работы грѣховныя, и быть свободнымъ отъ грѣховъ; се есть желаніе хрістіанское! се есть истинная свобода! \textit{Всякъ творяй грѣхъ, рабъ есть грѣха. Аще Сынъ вы свободитъ, воистинну свободни будете}\footnote{Іоан.~8,~34 и 36.}. Желаютъ люди побѣдить людей: ты желай побѣдить самого себе; се есть желаніе хрістіанское! се преславная побѣда! Желаютъ люди господами быть, и другими повелѣвать: ты желай и тщись надъ плотію твоею господствовать, и той повелѣвать; се есть желаніе хрістіанское! се есть истинное господство! Желаютъ люди надъ людьми царствовать: ты желай и тщись надъ своими страстьми и похотьми царствовать; се есть желаніе хрістіанское! Истинный царь есть, кто собою и страстьми своими владѣетъ. \textit{Иже Хрістовы суть, плоть распяша со страстьми и похотьми}\footnote{Гал.~5,~24.}. То"=то господа, владѣтели и цари! Господь помоги намъ! "--- Желаютъ люди врагамъ своимъ отмстить, и обиду обидою отвратить: ты, хрістіанине, желай и тщись врагамъ твоимъ уступить и обиду ихъ добротвореніемъ наградить; се есть желаніе хрістіанское! се есть мщеніе истинныхъ хрістіанъ! \textit{Аще убо алчетъ врагъ твой, ухлѣби его; аще ли жаждетъ, напой его: сіе бо творя, угліе огненное собираеши на главу его. Не побѣжденъ бывай отъ зла, но побѣждай благимъ злое}\footnote{Римл.~12,~20 и 21.}. Желаютъ люди здравое тѣло имѣть: ты желай и тщись, чтобы исцѣлилася и здрава была душа твоя; се есть желаніе хрістіанское! се есть желаемое здравіе! Желаютъ люди говорить и съ людьми разглагольствовать: ты желай и тщись болѣе молчать, нежели говорить, да избудеши отъ грѣха; се есть желаніе хрістианское! Желаютъ люди красно и гладко говорить: ты желай красно и по"=хрістіански жить; се есть желаніе хрістіанское! Желаютъ люди отъ людей принимать: ты, хрістіанине, желай и тщись людямъ давать, поминая слово Господа Іисуса, яко Самъ рече: \textit{блаженнѣе есть паче даяти, нежели пріимати}\footnote{Дѣян.~20,~35.}. Желаютъ люди въ домѣ красномъ и богатомъ жить: ты желай и тщись \textit{въ дому Божіемъ жить, яже есть церковь Бога жива, и быть сожителемъ святымъ и приснымъ Богу}\footnote{1~Тим.~3,~15; Еф.~2,~19.}. Се есть желаніе хрістіанское! се есть прекрасный и пребогатый домъ! \textit{Господи! кто обитаетъ въ жилищи Твоемъ? или кто вселится во святую гору твою? Ходяй непороченъ, и дѣлаяй правду, глаголяй истину въ сердце своемъ, иже не ульсти языкомъ своимъ, и не сотвори искреннему своему зла, и поношенія не пріятъ на ближнія своя; уничиженъ есть предъ нимъ лукавнуяй, боящіяжеся Господа славитъ; кленыйся искреннему своему, и не отметаяйся: сребра своего не даде въ лихву, мзды на неповинныхъ не пріятъ. Творяй сія не подвижится во вѣкъ}\footnote{Пс.~14,~1--5.}. \textit{Едино просихъ отъ Господа, то взыщу: еже жити ми въ дому Господни вся дни живота моего. Господи! возлюбихъ благолѣпіе дому Твоего, и мѣсто селенія славы Твоея. Да не погубиши съ нечестивыми душу мою, и съ мужи кровей животъ мой, ихже въ руку беззаконія, десница ихъ исполнися мзды. Азъ же незлобою моею ходихъ: избави мя, Господи, и помилуй мя. Изволихъ приметатися въ дому Бога моего паче, неже жити ми въ селеніихъ грѣшничихъ}\footnote{26,~4; 25,~8--11; 83,~11.}. Желаютъ люди въ красномъ и богатомъ одѣяніи ходить: ты желай и тщись \textit{облещися во утробы щедротъ, благость, смиреномудріе, кротость и долготерпѣніе}, и проч.\footnote{Кол.~3,~12.} Се есть желаніе хрістіанское! се есть прекрасная хрістіанская одежда! Сею утварію хрістіанская душа, какъ царская дщерь, украшается. Желаютъ люди насытить чрево свое: ты желай и тщись насытить душу твою словомъ Божіимъ и добрыми и святыми мыслями; се есть желаніе хрістіанское! се есть пресладкая хрістіанскія души пища! Желаютъ люди сладкаго и дорогаго вина: ты желай слезъ и плача духовнаго, да возвеселится сердце твое о Бозѣ Спасѣ твоемъ; се есть желаніе хрістіанское! се есть сладкое вино святыхъ! Надобно тому прежде довольно плакать, кто хощетъ истинную имѣть радость. Желаютъ люди дружество имѣть съ знатными и славными міра сего: ты желай и тщись дружество имѣть со Хрістомъ и святыми Его; се есть желаніе хрістіанское! се есть преславный и пресладкій союзъ, \textit{Прилѣпляяйся Господеви, единъ духъ есть съ Господемъ. Вы друзи Мои есте}, глаголетъ Господь, \textit{аще творите, елика Азъ заповѣдаю вамъ}\footnote{1~Кор.~6,~17; Іоан.~15,~14.}. Дружество есть взаимная любовь. Хрістосъ тебе возлюбилъ; люби и ты Его: и будетъ дружество. \textit{Мнѣ же прилѣплятися Богови благо есть}\footnote{Пс.~72,~28.}. Желаютъ люди въ домъ свой принять царя: ты желай и тщись въ домъ сердца твоего принять Царя небеснаго "--- Іисуса Хріста. \textit{Аще кто любитъ Мя, слово Мое соблюдетъ: и Отецъ Мой возлюбитъ его, и къ нему пріидемъ, и обитель у него сотворимъ}\footnote{Іоан.~14,~23.}. О, блаженное тое сердце, въ которомъ обитаетъ тріѵпостасный Богъ! Царствіе Божіе тамо есть; рай сладости и радости тамо есть. \textit{Се стою при дверехъ, и толку: аще кто услышитъ гласъ Мой, и отверзетъ двери, вниду къ нему и вечеряю съ нимъ, и той со Мною}\footnote{Апок.~3,~20.}. О Іисусе, пастырю и посѣтителю душъ нашихъ! не мини и мене, грѣшнаго раба Твоего, и услышанъ мнѣ сотвори гласъ Твой: яко гласъ Твой сладокъ, и приходъ Твой спасителенъ и миренъ. \textit{Готово сердце мое, Боже, готово сердце мое}\footnote{Пс.~107,~2.}. Желаютъ люди отъ сильныхъ людей защитниковъ и помощниковъ себѣ имѣть: ты желай и тщись Бога помощника и защитника себѣ имѣть; се есть желаніе хрістіанское! се помощникъ и защитникъ вѣрный! \textit{Боже, въ помощь мою вонми! Господи, помощи ми потщися! помощникъ мой еси, Тебѣ пою: яко Богъ заступникъ мой еси, Боже мой, милость моя}\footnote{69, ст. 2; 58, ст. 18.}. Желаютъ люди богатство въ мірѣ семъ имѣть: ты желай и тщись добрыми дѣлами богатѣть; се есть истинное хрістіанское богатство! Желаютъ люди славу и похвалу въ мірѣ семъ имѣть: ты желай и тщись отъ того убѣгать, и будеши истинную славу имѣть; се есть хрістіанское желаніе! се есть истинная слава и похвала "--- презирать славу и похвалу міра! Желаютъ люди много знать, и мудрыми показаться; ты желай и тщись паче всего самого себе познать; се есть мудрость хрістіанская! Желаютъ люди вѣдать, что тамо и тамо дѣлается: ты желай и тщись разсматривать, что въ душѣ твоей дѣлается, въ какомъ она состояніи находится, какія мысли, желанія и начинанія производитъ, къ чему она стремится "--- къ вѣчности, или времени? гдѣ она свое сокровище сокрываетъ "--- на небеси, или на земли? къ чему прилѣпляется "--- къ Богу, или къ міру? Се есть знаніе, необходимо нужное! \textit{Идѣже есть сокровище ваше, ту будетъ и сердце ваше. Свѣтильникъ тѣлу есть око: аще убо будетъ око твое просто, все тѣло твое свѣтло будетъ, аще ли око твое лукаво будетъ, все тѣло твое темно будетъ. Аще убо свѣтъ, иже въ тебѣ, тма есть, то тма кольми}\footnote{Мѳ.~6,~21--23.}. Желаютъ люди человѣкамъ угождать: ты желай и тщись Богу Господу твоему угождать. \textit{Аще быхъ еще человѣкомъ угождалъ Хрістовъ рабъ не быхъ убо былъ}\footnote{Гал. 1--10.}. Но и то писано: \textit{кійждо васъ ближнему да угождаетъ во благое къ созиданію}\footnote{Римл.~15,~2.}. А когда тако угождаемъ человѣкамъ, то Богу угождаемъ: ибо тако угождать человѣкамъ повелѣлъ Богъ. Желаютъ люди упокоить тѣло: ты желай и тщись упокоить и умирить совѣсть твою и умъ твой отъ злыхъ и суетныхъ помысловъ; се есть желаніе хрістіанское! се есть богоугодный покой и хрістіанское субботство! Желаютъ люди силы и крѣпости тѣлесной: ты желай силы и крѣпости душевной; се есть желаніе хрістіанское! се есть сила и крѣпость истинная "--- быть неподвижнымъ во искушеніяхъ, въ бѣдахъ и напастяхъ! \textit{Господи! волею Твоею подаждь добротѣ моей силу}\footnote{Пс.~29,~8.}. \textit{Господь крѣпость людемъ Своимъ дастъ, Господь благословитъ люди Своя миромъ}\footnote{28,~11.}. \textit{Крѣпость моя и пѣніе мое Господь, и бысть мнѣ во спасеніе}\footnote{117,~4.}. \textit{Сладцѣ убо похвалюся паче въ немощехъ моихъ, да вселится въ мя сила Хрістова}\footnote{2~Кор.~12,~9.}. Желаютъ люди долго жить: ты, хрістіанине, желай блаженно умереть. \textit{Помяни мя, Господи, во царствіи Твоемъ}\footnote{Лук.~23,~42.}. \textit{Имене ради Твоего наставиши мя, и препитаеши мя; изведеши мя отъ сѣти сея, юже скрыша ми: яко Ты еси защититель мой, Господи! Въ руцѣ Твои предложу духъ мой}\footnote{Пс.~30,~4--6.}. \textit{Жива будетъ душа моя, и восхвалитъ Тя}\footnote{Пс.~118,~175.}. \textit{Благословенъ Господь, Иже не даде насъ въ ловитву зубомъ ихъ! душа наша, яко птица, избавися отъ сѣти ловящихъ: сѣть сокрушися, и мы избавлени быхомъ}\footnote{123,~6 и 7.}. Желаютъ люди въ милости быть у Царей, князей и вельможъ: ты желай, хрістіанине, и тщись быть въ милости у Бога, Вседержителя, небеснаго Царя. \textit{Не надѣйтеся на князи, на сыны человѣческія, въ нихже нѣсть спасенія}\footnote{145,~3.}. \textit{Вы же яко человѣцы умираете, и яко единъ отъ князей падаете}\footnote{81,~8.}. \textit{Господи! милость твоя во вѣкъ: дѣлъ руку Твоею не презри}\footnote{137,~8.}! \textit{Буди, Господи, милость Твоя на насъ, якоже уповахомъ на Тя}\footnote{32,~22.}! \textit{Благо есть надѣятися на Господа, нежели надѣятися на человѣка! благо есть уповати на Господа, нежели уповати на князи}\footnote{117,~8 и 9.}! Желаютъ люди видѣть царя или высокое какое лице: ты, хрістіанине, паче всего желай и тщись видѣть Бога; се есть всѣхъ желаній конецъ! \textit{Что ми есть на небеси? и отъ Тебе что восхотѣхъ на земли? Исчезе сердце мое и плоть моя: Боже сердца моего, и часть моя Боже во вѣкъ}\footnote{Пс.~72,~25 и 26.}. \textit{Имже образомъ желаетъ елень на источники водныя: сице желаетъ душа моя къ Тебѣ, Боже! Возжада душа моя къ Богу крѣпкому живому: когда пріиду и явлюся лицу Божію}\footnote{41,~2 и 3.}? \textit{Явится Богъ боговъ въ Сіонѣ}\footnote{83,~8.}. \textit{Видимъ бо нынѣ якоже зерцаломъ въ гаданіи, тогда же лицемъ къ лицу}\footnote{1~Кор.~13,~12.}. \textit{И узримъ Его, якоже есть}\footnote{1~Іоан.~3,~2.}. \textit{Миръ имѣйте и святыню со всѣми, ихже кромѣ никтоже узритъ Господа}\footnote{Евр.~12,~2.}. \textit{Блажени чистіи сердцемъ: яко тіи Бога узрятъ}\footnote{Мѳ.~5,~8.}. Сія суть, возлюбленный хрістіанине, желанія хрістіанскія души! сія суть воздыханія, и движенія внутренняя хрістіанина, пекущагося о своемъ спасеніи! Кто чего желаетъ усердно, того усердно и ищетъ и обрѣтаетъ. \textit{Просите, и дастся вамъ; ищите, и обрящете; толцыте, и отверзется вамъ. Всякъ бо просяй пріемлетъ, и ищай обрѣтаетъ, и толкущему отверзется}\footnote{7,~7 и 8.}. Желай и ты, душе моя, и ищи добра твоего, здравія твоего, богатства твоего, покоя и мира твоего и проч. \textit{Желаніе убогихъ услышалъ еси, Господи! упованію сердца ихъ внятъ ухо Твое}\footnote{Пс.~9,~38.}. \textit{Господи! предъ Тобою все желаніе мое и воздыханіе мое отъ Тебе не утаися. Сердце мое смятеся во мнѣ, остави мя сила моя, и свѣтъ очію моею, и той нѣсть со мною}, и проч. \textit{Не остави мене, Господи Боже мой, не отступи отъ мене: вонми въ помощь мою, Господи спасенія моего}\footnote{37,~10,~11,~22 и 23.}.

\section{47. Счастіе.}

Люди, когда по желанію своему что получаютъ, счастливыми называются: тако хрістіанинъ счастливъ есть, когда имѣетъ тое, чего желаетъ и ищетъ. Счастливы люди тіи, которыи отъ работы, или отъ темницы, или отъ ссылки, или отъ плѣна свободилися. Хрістіанинъ счастливъ есть, когда отъ грѣховъ, которыми, какъ узами, связанъ былъ, и отъ работы діавольскія свободился. Счастіе у людей есть, что здравое тѣло имѣютъ: хрістіанское счастіе есть имѣти здравую душу. Счастливыми называются тіи воины, который одерживаютъ побѣду надъ врагами своими: воистину счастливъ тотъ хрістіанинъ, который себе самого побѣждаетъ. За счастіе себѣ почитаютъ многіи, что имъ удалось врагамъ своимъ отмстить: истинное хрістіанина счастіе есть, что онъ врагамъ своимъ уступаетъ и не воздаетъ зла за зло; большее еще счастіе есть, когда \textit{ихъ любитъ и благотворитъ имъ}; тогда"=то онъ подлинно сынъ есть небеснаго Отца, \textit{Иже солнце Свое сіяетъ на злыя и благія, и дождитъ на праведныя и на неправедныя}\footnote{Мѳ.~5,~44--45.}. Счастіе у людей есть, что много отъ людей принимаютъ и собираютъ: счастливъ воистину тотъ хрістіанинъ, который отъ всего, что имѣетъ у себе, ближнимъ своимъ удѣляетъ. \textit{Блажени милостивіи: яко тіи помиловани будутъ}\footnote{7.}. \textit{Благъ мужъ щедря и дая: устроитъ словеса своя на судѣ, яко въ вѣкъ не подвижится}\footnote{Пс.~111,~5.}. Счастіе у людей есть быть богатымъ и славнымъ въ мірѣ семъ: хрістіанина счастіе есть добрыми дѣлами богатѣть, и славу міра сего презирать. Счастіе у людей есть надъ другими людьми господствовать и царствовать: хрістіанинъ счастливъ есть, когда надъ собою самимъ, надъ плотію своею, надъ страстьми и похотьми своими господствуетъ и царствуетъ, и ими повелѣваетъ. Сей есть воистину господинъ, царь и повелитель! Многіи надъ людьми господствуютъ и царствуютъ; но сами рабы суть грѣха и плѣнники страстей своихъ. Воистину преславный царь есть и господинъ, кто страстьми и похотьми своими владѣетъ! Счастіе у людей есть богатую поставлять трапезу и тою насыщаться: хрістіанина счастіе есть внутрь сердца своего слышать слово Божіе, и тѣмъ насыщать душу свою, утѣшать и увеселять; се, есть богатая трапеза, преславная и пресладкая хрістіанскія души вечеря! Счастливъ у людей есть, кто въ радости и веселіи міра сего живетъ: христіанина счастіе есть, ежели онъ печаль по Богѣ имѣетъ, и плачется своихъ ради грѣховъ. \textit{Блажени плачущіи, яко тіи утѣшатся}\footnote{Мѳ.~5,~4.}. Счастіе у людей есть, когда человѣкъ въ богатомъ и красномъ домѣ живетъ: христіанина счастіе есть, ежели онъ истинно внутрь церкви Божіей обрѣтается, есть истинный сынъ церкви, сожитель святымъ, присный Богу, и духовный удъ Хрістовъ. \textit{Блажени живущіи въ дому Твоемъ! въ вѣки вѣковъ восхвалятъ Тя}\footnote{Пс.~83,~5.}. Счастіе у людей есть "--- имѣть себѣ защитниковъ и помощниковъ отъ сильныхъ людей:, хрістіанина счастіе есть "--- имѣть себѣ Бога защитника и помощника. \textit{Блаженъ мужъ, емуже есть заступленіе его у Тебе}, Боже! \textit{Блаженъ, емуже Богъ Іаковль помощникъ его! упованіе его на Господа Бога своего, сотворшаго небо и землю, море и вся, яже въ нихъ; хранящаго истину въ вѣкъ, творящаго судъ обидимымъ; дающаго пищу алчущимъ}\footnote{6; 114,~5--7.}. Счастіе у людей есть "--- быть крѣпкимъ тѣломъ и храбрымъ на брани: христіанина счастіе есть, ежели онъ крѣпокъ душою, неподвижимый и неодолимый на брани, отъ діавола бываемой, и во искушеніяхъ бѣдахъ, и напастяхъ, отъ него наносимыхъ, непреклоненъ есть. Се есть крѣпость и великодушіе хрістіанское! се есть храбрый воинъ Хрістовъ! \textit{Ты моя крѣпость, Господи, Ты моя и сила!} и проч. Съ Тобою все могу. Безъ Тебе ничего не могу. \textit{Вся могу, о укрѣпляющемъ мя Іисусе Хрістѣ}, и проч.\footnote{Филип.~4,~13.} Возлюбленный хрістіанине! пусть люди въ счастіи міра, сего процвѣтаютъ: ты довольно счастливъ будеши, когда внутрь себе счастіе будеши имѣть; то"=то есть истинное, неподвижимое и неотъемлемое счастіе! Пусть люди много знаютъ внѣ себе: ты довольно счастливъ, когда знаешь самого себе и внутрь себе, хотя и ничего инаго не знаешь. Пусть люди красуются здравіемъ, тѣлеснымъ: ты довольно счастливъ, когда здравіе душевное имѣешь, хотя тѣломъ и недугуешь. Пусть люди изобилуютъ тлѣннымъ богатствомъ: ты довольно счастливъ, когда внутрь себе благочестіе и сокровище добродѣтелей носишь. Пусть люди хвалятся защитниками и помощниками своими: я"=де того и того имѣю князя и вельможу защитника своего, тотъ"=де и тотъ мнѣ мужъ помогаетъ въ бѣдахъ моихъ и напастяхъ: ты довольно счастливъ, когда Богъ помощникъ и защитникъ твой есть. Пусть люди свободу и благородіе свое оказываютъ: ты довольно счастливъ, когда хрістіанское благородіе и душу свободну отъ грѣховъ имѣешь, хотя и работаешь людямъ, или въ темницѣ заключенъ имѣешися, или въ ссылку посланъ страждеши, или въ плѣну находишися. Пусть люди хвалятся, что у царя, или вельможи, или князя имѣются въ милости: ты весьма счастливъ будеши, ежели у Бога, небеснаго Царя, въ милости быть сподобишися, хотя и никто отъ человѣкъ не будетъ тебѣ милостивъ; единаго Бога милость будетъ тебѣ паче милости всего міра. Пусть люди живутъ въ красныхъ палатахъ и чертогахъ: ты довольно счастливъ, когда живеши въ дому Божіи, яже есть церковь Бога жива, "--- и сожитель святымъ, и присный Богу, и домашній Хрістовъ имѣешися, хотя и въ хижинѣ, или пещерѣ живеши, или не имѣеши гдѣ главы подклонити. Пусть люди питаются и насыщаются богатою и сладкою трапезою: ты довольно счастливъ, когда душа твоя питается и насыщается гласомъ Божіимъ, и утѣшается небесными, святыми и духовными мыслями, хотя и хлѣба единаго съ водою, ради подкрѣпленія немощи тѣлесныя, вкушаеши. Пусть люди хвалятся крѣпостію силъ своихъ: ты довольно счастливъ, когда душа твоя стоитъ неподвижима противу козней и искушеній діавольскихъ, хотя и никакой тѣлесной крѣпости не имѣеши. Пусть люди ходятъ въ шелковыхъ и красныхъ одеждахъ: ты довольно счастливъ, когда душа твоя благодатію Хрістовою, вѣрою, надеждою, любовію и прочею духовною утварію украшается, хотя и рубищемъ одѣваешися. Пусть люди славятся, что съ сильными и славными міра сего дружество имѣютъ: ты довольно счастливъ и славенъ, когда вошелъ въ дружество со Хрістомъ, Сыномъ Божіимъ, имѣешь \textit{общеніе со Отцемъ и съ Сыномъ Его Іисусомъ Хрістомъ}\footnote{1~Іоан.~1,~3.}, хотя и никого отъ человѣкъ друга себѣ не имѣеши, и прочая. Се есть истинное счастіе и подлинное блаженство, котораго ни огнь, ни вода, ни темница, ни ссылка, ни плѣненіе, ни хитрость, ни злоба человѣческая, ни смерть отнять не можетъ. \textit{Аще Богъ по насъ; кто на насъ}\footnote{Римл.~8,~32.}? Сіе блаженство вездѣ и всегда носитъ съ собою человѣкъ, ибо внутрь себе имѣетъ. \textit{Се бо царствіе Божіе внутрь васъ есть}\footnote{Лук.~17,~21.}. \textit{Нѣсть бо царство Божіе брашно и питіе, но правда, и миръ, и радость о Дусѣ Святѣ}\footnote{Римл.~14,~17.}. Съ симъ блаженствомъ хрістіанинъ и на оный вѣкъ отходитъ, и съ собою износитъ его отъ міра сего, и приноситъ въ небесное свое отечество, и отъ блаженства въ вѣчное и совершеннѣйшее блаженство приходитъ, гдѣ будетъ душею и тѣломъ совершенно блаженъ. О, воистину истинное есть хрістіанское блаженство, воистину дражайшее счастіе! Что убо пользуетъ человѣку по міру счастливымъ быть, но хрістіанскаго счастія не имѣть? Что пользуетъ тѣломъ здравствовать, но душу разслабленную имѣть? Разслабленныя души знакъ есть злонравіе. Что пользуетъ благородіемъ и свободою тѣлесною хвалиться, но рабомъ грѣха и страстей быть? \textit{Всякъ творяй грѣхъ рабъ есть грѣха}\footnote{Іоан.~8,~34.}. \textit{Имже кто побѣжденъ бываетъ, сему и работенъ есть}\footnote{Петр.~2,~19.}. Что пользуетъ внѣ себя много знать, но себе самого не знать? Бѣдный человѣче! старайся прежде познать себе самого, и тогда много будешь знать. Что пользуетъ тлѣннымъ веществомъ богатѣть, но душею нищетствовать? Нищій тотъ богачъ, кто тлѣннымъ сребромъ и златомъ богатѣетъ, но добродѣтелію нищетствуетъ. Что пользуетъ въ богатомъ и красномъ домѣ жить, но внѣ дома Божія, церкве святыя, какъ извергу, считаться? Что пользуетъ людьми владѣть и управлять тому, кто самъ собою владѣть не можетъ? Человѣче! научись прежде собою владѣть, и тогда счастливо людьми владѣть будеши. Что пользуетъ тлѣнное украшать тѣло, а душа безсмертная наготуетъ и скареднымъ грѣховнымъ рубищемъ покрывается? Человѣче несмысленный! оставь смертное тѣло "--- прахъ, землю и пепелъ, а украшай безсмертную душу; и тогда будеши внутрь себе имѣть красоту, но истинную и нетлѣнную, и проч. Хрістіанине! міра сего счастіе есть общее нечестивымъ и благочестивымъ, паче же нечестивымъ. Кто бо болѣе въ мірскомъ счастіи живетъ, какъ нечестивый? О нихъ"=то написано: \textit{нѣсть восклоненія въ смерти ихъ и утвержденія въ ранѣ ихъ: въ трудѣхъ человѣческихъ не суть, и съ человѣки не пріимутъ ранъ}\footnote{Пс.~72,~4--5.}. Но духовное счастіе есть свойственное единымъ истиннымъ хрістіанамъ. Имѣй истинное благочестіе внутрь себе: и тогда будешь имѣть истинное счастіе и внутрь тебе, но неотъемлемое, и несравненно лучшее паче мірскаго. \textit{Елицы правиломъ симъ жительствуютъ, миръ на нихъ и милость и на Израили Божіи}\footnote{Гал.~6,~16.}. Отсюду почерпать могутъ утѣшеніе люди благочестивыи, но по міру сему несчастливыи: 1)~\textit{Немощный}. Вы, страдальцы любезныи, не изнемогайте душею вашею! Хотя тѣло ваше и тлѣетъ; но святыя души ваши исцѣляются, и здравіе свое получаютъ. \textit{Аще и внѣшній вашъ человѣкъ тлѣетъ, но внутренній обновляется по вся дни}\footnote{2~Кор.~4,~16.}. Вы съ нищимъ Лазаремъ страждете здѣ, но съ нимъ и лоно Авраамле наслѣдите\footnote{Лук.~16,~22.}. 2)~\textit{Раби и крестьяне господамъ подчиненныи}. Вы, смиренніи и кроткіи овечки Хрістовы, отсюду утѣшеніе свое почерпайте, что хотя и называетеся рабами человѣческими, но есте свободники Хрістовы; тѣломъ работаете людямъ, но души ваши свободны отъ работы грѣховныя; не имѣете свободы и благородія, якоже господа ваши, но души ваши свободою и благородіемъ хрістіанскимъ украшаются. Радуйтеся убо и веселитеся, яко и васъ ждетъ наслѣдіе сыновъ Божіихъ. 3)~\textit{Нищіи и убогіи}. Вы, въ скудости и убожествѣ живущіи раби Хрістовы, не унывайте, видя другихъ въ богатствѣ живущихъ. Оставятъ и они все, что ни имѣютъ. \textit{Не убойся, егда разбогатѣетъ человѣкъ, или егда умножится слава дому его: яко внегда умрети ему, не возметъ вся, ниже снидетъ съ нимъ слава его}\footnote{Пс.~42,~17--18.}. Вы не имѣете тлѣннаго злата и сребра и прочаго вещества; но внутрь себе носите нетлѣнное сокровище: тѣмъ довольни будите. Не живете въ богатыхъ домахъ; но есте присніи Богу, домашніи Хрістовы и сожители святымъ: тѣмъ довольни будите. Носите на тѣлѣ вашемъ рубище; но души ваши святыя одеждою спасенія одѣваются. Не имѣете благихъ міра сего; но сокровенна вамъ суть благая на небесѣхъ, \textit{ихже око не видѣ, и ухо не слыша, и на сердце человѣку не взыдоша, яже уготова Богъ любящимъ Его}\footnote{1~Кор.~2,~9.}. Поминайте Хрістову нищету: \textit{вѣсте бо благодать Господа нашего Іисуса Хріста, яко васъ ради обнища богатъ сый, да вы нищетою Его обогатитеся}\footnote{2~Кор.~8,~9.}. 4)~\textit{Посмѣянные и поруганные отъ міра}. Вы, которые, какъ сметье, отъ міра попираетеся, воспримите въ сердце ваше слово утѣшенія. Пусть міръ дѣлаетъ вамъ, что хощетъ: вы Хрістовы есте. Се есть велика слава и похвала, быть Хрістовымъ, Иже есть Господь славы! \textit{О Господѣ похвалится душа моя}\footnote{Пс.~33,~3.}. Вамъ посмѣвается, васъ уничижаетъ и ругаетъ міръ, но души ваши честни суть у Бога; васъ осуждаетъ міръ, но оправдаетъ васъ Богъ; васъ проклинаетъ міръ, но благословляетъ васъ Богъ. \textit{Прокленутъ тіи, и Ты благословиши}. Васъ извергаетъ міръ, какъ непотребныхъ, но пріемлетъ Богъ. \textit{Боже, хвалы моея не премолчи: яко уста грѣшнича и уста льстиваго на мя отверзошася, глаголаша на мя языкомъ льстивымъ: и словесы ненавистными обыдоша мя, и брашася со мною туне; вмѣсто еже любити мя, оболгаху мя, азъ же моляхся}\footnote{Пс.~108,~1--4.}. Поминайте, что Хрістосъ, глава наша и Господь славы, презрѣнъ, уничиженъ, посмѣянъ и поруганъ былъ отъ злаго міра. \textit{Нѣсть рабъ болій Господа своего}\footnote{Іоан.~13,~16.}. \textit{Аще господина дому веельзевула нарекоша, кольми паче домашнія его}\footnote{Мѳ.~10,~25.}. \textit{Блажени есте, егда поносятъ вамъ, и изженутъ, и рекутъ всякъ золъ глаголъ, на вы лжуще Мене ради. Радуйтеся и веселитеся, яко мзда ваша многа на небесѣхъ}\footnote{5,~11 и 12.}. 5)~\textit{Сѣдящіи въ темницѣ}. Вы, возлюбленніи страдальцы Хрістовы, претерпите, что страждете! Тѣло ваше заключено, но духъ вашъ свободенъ; руки и ноги ваши связаны, но души ваши святыя разрѣшены благодатію Хрістовою; свѣта солнечнаго не видите, но свѣтъ Божественный внутрь просвѣщаетъ васъ; въ темницѣ заключены вы, но небо отверсто вамъ; не имѣете утѣшенія міра сего, но благодать Божія внутрь утѣшаетъ васъ. Поминайте, о сострадальцы и сопричастники мученикамъ и исповѣдникамъ Хрістовымъ, что Хрістосъ Господь нашъ связанъ былъ грѣхъ ради нашихъ. По сей темницѣ воцаритеся со Хрістомъ, съ которымъ страждете. \textit{Съ Нимъ страждемъ, да и съ Нимъ прославимся. Непщую бо, яко недостойны страсти нынѣшняго времене къ хотящей славѣ явитися въ насъ}\footnote{Римл.~8,~17--18.}. 6)~\textit{Находящіеся въ ссылкѣ}. Вы, изгнанники и странники, поминайте слово Хрістово: \textit{всякъ, еже оставитъ домъ, или братію, или сестры, или отца, или матерь, или жену, или чада, или села, имене Моего ради, сторицею пріиметъ, и животъ вѣчный наслѣдитъ}\footnote{Мѳ.~19,~29.}. Отлучились вы отъ дому своего, отца и прочихъ своихъ любезныхъ сродниковъ и друговъ, но отъ Бога и дому Его святаго не отлучилися; оставили они васъ, но Богъ васъ не оставилъ. Онъ и во изгнаніи вашемъ съ вами есть, и есть вамъ все. Поминайте и тое, что благочестивымъ душамъ, въ мірѣ семъ живущимъ, всякое мѣсто есть ссылка. Они въ мірѣ семъ имѣются такъ, какъ нѣкогда Іудеи въ плѣненіи вавилонскомъ, которые, поминая Сіонъ "--- свое отечество, \textit{тамо сидѣли и плакали}\footnote{Пс.~136,~1.}. Тако всякая благочестивая душа въ мірѣ семъ, какъ при рѣкахъ вавилонскихъ, сѣдитъ и плачетъ, поминая любезное свое небесное отечество и горній Іерусалимъ. Нѣтъ благочестивой душѣ въ мірѣ семъ истиннаго дома и отечества, но вездѣ изгнаніе и ссылка. Потерпите убо, возлюбленніи, и пождите, пока Господь васъ отъ міра сего, какъ изъ Вавилона, въ горній Іерусалимъ позоветъ. Тогда узрите домъ вашъ и отечество ваше, но уже небесное, вѣчное, блаженное; \textit{и возрадуется сердце ваше, и радости вашея никтоже возметъ отъ васъ}\footnote{Іоан.~16,~22.}. \textit{Возврати Господи плѣненіе наше, яко потоки югомъ}\footnote{Пс.~125,~4.}. 7)~\textit{Вси} благочестивіи, какъ нибудь \textit{страждущіи}, возведите сердечныя очи ваши во обители небеснаго Отца, и разсмотрите жителей тѣхъ святыхъ; никого тамо не увидите, кто бы туда скорбнымъ путемъ не пришелъ; и услышите оттуда гласъ, о нихъ свидѣтельствующій; \textit{сіи суть, иже пріидоша отъ скорби великія; и испраша ризы своя, и убѣлиша ризы своя въ крови Агнчи. Сего ради суть предъ престоломъ Божіимъ, и служатъ Ему день и нощь въ церкви Его, и сѣдяй на престолѣ вселится въ нихъ. Не взалчутъ ктому, ниже вжаждутъ, не имать же пасти на нихъ солнце, ниже всякъ зной: яко Агнецъ, Иже посреди престола, упасетъ я, и наставитъ ихъ на животныя источники водъ, и отыметъ Богъ всяку слезу отъ очію ихъ}\footnote{Апок.~7,~14--17.}. \textit{Да не кто убо отъ васъ постраждетъ, яко убійца, или яко тать, или яко злодѣй, или яко чуждо"=посѣтитель} (чуждаго желатель): \textit{аще ли же яко хрістіанинъ, да не стыдится, да прославляетъ же Бога въ части сей}, и проч.\footnote{1~Петр.~14,~15 и 16.}

\section{48. Свѣща горящая.}

Человѣче! Что свѣща горящая предъ тобою, тое предъ тобою житіе твое. Чимъ свѣща болѣе горитъ, тѣмъ болѣе умаляется: тако ты, чимъ болѣе живешь, тѣмъ болѣе сокращается житіе твое. Догораетъ свѣща, и погасаетъ: тако оканчиваетъ человѣкъ житіе свое, и умираетъ. Какъ погаснетъ свѣща, то кажется, какъ бы ея не было; тако, когда умретъ человѣкъ и погребется, кажется, какъ бы и не былъ. Видишь, человѣче, что есть человѣкъ, и что есть житіе его. Поминай убо, что тако и житіе твое погаснетъ, какъ видишь погасшую свѣщу, и заранѣе готовися къ кончинѣ Твоей, да блаженно скончаешися. \textit{Скажи ми, Господи, кончину мою, и число дней моихъ кое есть, да разумѣю, что лишаюся азъ? Се пяди положилъ еси дни моя, и составъ мой яко ничтоже предъ Тобою; обаче всяческая суета всякъ человѣкъ живый. Убо образомъ ходитъ человѣкъ, обаче всуе мятется: сокровиществуетъ, и не вѣсть, кому соберетъ я}\footnote{Пс.~38,~5 и 7.}.

\section{49. Мечь надъ главою висящій.}

Аще бы кто на какомъ мѣстѣ сидѣлъ или стоялъ, а надъ главою бы его мечь острый на тонкой ниткѣ висѣлъ: всякъ бы о немъ судилъ, что онъ въ великой опасности находится; ибо въ такомъ случаѣ очевидная бы смерть надъ нимъ висѣла. И всякъ бы, видя человѣка въ такой опасности, звалъ бы его съ того мѣста. Тако надъ всякимъ нераскаяннымъ грѣшникомъ мечь правосудія и гнѣва Божія виситъ; виситъ надъ блудникомъ и прелюбодѣемъ; виситъ надъ воромъ и хищникомъ; виситъ надъ клеветникомъ и ругателемъ; виситъ надъ властелиномъ, который своей скверной корысти, а не общей пользы ищетъ; виситъ надъ пастыремъ, который о словесныхъ овцахъ, кровію Хрістовою искупленныхъ, погибающихъ, нерадитъ; виситъ надъ судіею мздоимцомъ, который по мздѣ и по лицамъ, а не по дѣламъ судитъ; виситъ надъ господиномъ, который крестьянъ своихъ или какъ звѣрь мучительски терзаетъ, или излишними оброками, или работами отягчаетъ; виситъ надъ родителями, которыи дѣтей своихъ не страху Божію учатъ, но беззаконнымъ своимъ житіемъ соблазняютъ и развращаютъ, и имъ, банею пакибытія омытымъ и спасеннымъ, путь къ погибели стелютъ; виситъ наконецъ надъ всякимъ беззаконникомъ, который произволительно и безстрашно законъ Божій дерзаетъ нарушать. \textit{Открывается бо гнѣвъ Божій съ небесе на всякое нечестіе и неправду человѣческую}\footnote{Римл.~1,~18.}. Виситъ, говорю, и какъ только падетъ, то не иное что, какъ вѣчная смерть бѣдному тому грѣшнику послѣдуетъ. Тако палъ гнѣвъ Божій на Содомлянъ, и погибли; палъ на Фараона ожесточеннаго, и въ морѣ погрязнулъ, яко олово, со всѣмъ своимъ воинствомъ; палъ на беззаконныхъ Израильтянъ въ пустыни, и умертвилъ ихъ. \textit{И гнѣвъ Божій взыде на ня, и уби множайшая ихъ}\footnote{Пс.~77,~31.}. Ахъ, бѣдный грѣшникъ! мечь острый надъ тобою виситъ, и не инымъ чимъ тебѣ, какъ смертію грозитъ; и ежели не сойдеши съ мѣста, падетъ онъ на тебе. Сойди убо, пожалуй сойди, да не падетъ и поразитъ тя и \textit{протешетъ} тя \textit{полма, и часть} твою \textit{съ невѣрными положитъ}\footnote{Мѳ.~24,~51.}. Еще Богъ тебѣ долготерпитъ Своею благостію: исправься убо, пока долготерпитъ! Терпѣлъ до сего времени: а впредь потерпитъ ли, не знаю. Соскочи убо, возлюбленне, пожалуй поскорѣе соскочи съ пути нечестивыхъ, надъ которыми мечь гнѣва Божія виситъ, пока и тебе съ ними не поразитъ! \textit{О человѣче! или о богатствѣ благости Его и кротости и долготерпѣніи нерадиши, невѣдый, яко благость Божія на покаяніе тя ведетъ}? и проч.\footnote{Римл.~2,~31.} \textit{Блаженъ мужъ, иже не иде на совѣтъ нечестивыхъ, и на пути грѣшныхъ не ста, и на сѣдалищи губителей не сѣде!} и проч.\footnote{Пс.~1,~1--2.}

\subsection{О томжде.}

Когда человѣкъ вышепоказанной опасности уклонится, тогда безопасно и свободно ходитъ. Тако грѣшникъ, когда къ Богу всѣмъ сердцемъ обратится, и съ пути нечестивыхъ сойдетъ и начнетъ въ покаяніи истинномъ жить, то уже свободенъ бываетъ отъ належащаго гнѣва Божія и уже, вмѣсто гнѣва Божія, милости Божія надѣется и ожидаетъ. Судъ бо и гнѣвъ Божій не падаетъ на грѣшниковъ кающихся; ибо того и хощетъ и ожидаетъ отъ насъ Богъ, чтобы мы къ Нему обратилися и каялися. Тако висѣлъ гнѣвъ Божій надъ Ниневитянами; но когда обратилися и покаялися, не палъ на нихъ гнѣвъ Божій, но вмѣсто гнѣва милость Божію дознали. \textit{И видѣ Богъ дѣла ихъ, яко обратишася отъ путей своихъ лукавыхъ; и раскаяся Богъ о злѣ, еже глаголаше сотворити имъ, и не сотвори}\footnote{Іоан.~3,~10.}. \textit{Услышу, что речетъ о мнѣ Господь Богъ: яко речетъ миръ на люди Своя, и на преподобныя Своя, и на обращающія сердца къ Нему}\footnote{Пс.~84,~9.}. Сей миренъ отвѣтъ отъ благости Божія и всякое кающееся сердце получитъ. О, воистину блаженъ тотъ человѣкъ, который, съ пути нечестивыхъ сошедши, находится въ истинномъ покаяніи. \textit{О таковомъ радуются ангели} Божіи на небеси\footnote{Лук.~15,~10.}. Аще убо благодатію Божіею подвигнешися къ сему доброму дѣлу, хрістіанине: стой и крѣпися, возлюбленне, въ томъ до конца, да и во ономъ вѣкѣ со ангелами вселишися, и съ ними славити будешь преблагаго Бога въ безконечные вѣки. Не начало, но конецъ похваляется. Начни добрѣ и скончай добрѣ: и блаженъ будеши. \textit{Обаче близъ боящихся Его} (Бога) \textit{спасеніе Его}\footnote{Пс.~84,~10.}. \textit{Буди вѣренъ даже до смерти, и дамъ ти вѣнецъ живота}, глаголетъ тебѣ Господь\footnote{Апок.~2,~10.}.

\section{50. Сѣть.}

Видимъ въ мірѣ семъ, какъ"=то рыболовы мещутъ сѣти своя въ воду и распростираютъ ихъ, дабы рыбу уловить. Тако діаволъ съ бѣсами своими вездѣ и различныя сѣти ради насъ, о хрістіанине, распростерлъ: распростерлъ въ дому, распростерлъ на пути, распростерлъ во градѣхъ и селѣхъ, распростерлъ въ пустыни, распростерлъ на земли и на морѣ; скрылъ сѣти въ пищѣ, въ питіи, въ сластолюбіи, скрылъ сѣти въ чести и въ богатствѣ, скрылъ сѣти въ бесѣдахъ и молчаніи, скрылъ сѣти въ видѣніи и слухѣ; скрылъ ради насъ, дабы насъ уловить въ свою погибель. Сколько сѣтей его пагубныхъ, столько козней и хитростей ради насъ; сколько хитростей и козней его, столько опасностей, бѣдъ и напастей нашихъ. О, кто сіи скрытыя сѣти узнать и отъ нихъ избавитися можетъ? Сѣти многіи, сѣти различныи, сѣти вездѣ, сѣти скрытыи и невидимыи, сѣти ради нашего бѣдствія и погибели! Кто ихъ усмотрѣть и отъ нихъ уклониться можетъ, аще не Ты, Господи, покажеши ихъ и Твоею всесильною десницею сохраниши отъ нихъ? \textit{Трезвитеся, бодрствуйте! зане супостатъ вашъ діаволъ яко левъ рыкая ходитъ, искій, кого поглотити}\footnote{1~Петр.~5,~8.}. \textit{Живый въ помощи Вышняго, въ кровѣ Бога небеснаго водворится. Речетъ Господеви: заступникъ мой еси и прибѣжище мое, Богъ мой, и уповаю на Него: яко Той избавитъ тя отъ сѣти ловчи}\footnote{Пс.~90,~1--3.}. \textit{Къ тебѣ, Господи, Господи, очи мои; на Тя уповахъ, не отъими душу мою; сохрани мя отъ сѣти, юже составиша ми, и отъ соблазнъ дѣлающихъ беззаконіе}\footnote{Пс.~140,~8--9.}. \textit{На пути семъ, по немуже хождахъ, скрыта сѣть мнѣ. Смотряхъ одесную и возглядахъ, и не бѣ знаяй мене: погибе бѣгство отъ мене, и нѣсть взыскаяй душу мою. Воззвахъ къ Тебѣ, Господи, рѣхъ: Ты еси упованіе мое, часть моя еси на земли живыхъ. Вонми моленію моему, яко смирился зѣло; избави мя отъ гонящихъ мя, яко укрѣпишася паче мене}\footnote{141,~5--7.}. \textit{Сохрани мя, Господи, изъ руки грѣшничи, отъ человѣкъ неправедныхъ изми мя, иже помыслиша запяти стопы моя. Скрыша гордіи сѣть мнѣ, и ужы препяша сѣть ногама моима: при стези соблазны положиша ми. Рѣхъ Господеви: Богъ мой еси Ты: внуши, Господи, гласъ моленія моего. Господи, Господи, сило спасенія моего! осѣнилъ еси надъ главою моею въ день брани. Не предаждь мене, Господи, отъ желанія моего грѣшнику: помыслиша на мя, не остави мене, да не когда вознесутся}\footnote{139,~4--9.}. \textit{Не предаждь звѣремъ душу исповѣдающуюся Тебѣ}\footnote{73,~19.}, да воспою Тебѣ благодарственную пѣснь: \textit{Благословенъ Господь, Иже не даде насъ въ ловитву зубомъ ихъ. Душа наша, яко птица, избавися отъ сѣти ловящихъ: сѣть сокрушися, и мы избавлени быхомъ. Помощь наша во имя Господа, сотворшаго небо и землю}\footnote{13,~6--8.}.

\subsection{О томжде.}

Рыба, которая не минуетъ сѣти рыболовца, попадаетъ въ сѣть и увязаетъ въ сѣти. Тако, кто не избѣжитъ діавольской сѣти, увязаетъ въ ней. Блудникъ и прелюбодѣйца увязнулъ въ сѣти его; клеветникъ и злорѣчивый увязнулъ въ сѣти его; піяница увязнулъ въ сѣти его; пастырь, нерадящій о своемъ и стада Хрістова спасеніи, увязнулъ въ сѣти его; властелинъ, не общей пользы, но своей скверной корысти ищущій, увязнулъ въ сѣти его; судія, мздою и дарами растлѣнный, увязнулъ въ сѣти его; господинъ, мучащій, или излишними оброками обременяющій крестьянъ своихъ, увязнулъ въ сѣти его; словомъ, всякъ грѣшникъ, некающійся и нерадящій о своемъ спасеніи, увязнулъ въ сѣти его. Вси таковыи \textit{живи уловлени отъ него въ свою его волю}\footnote{Тим.~2,~26.}. Рыболовъ увязшую въ сѣти своей рыбу влечетъ къ себѣ: тако діаволъ грѣшника, увязшаго въ сѣть его пагубную, влечетъ за собою въ погибель, въ которой самъ находится. Ахъ! влечетъ, и привлечетъ, аще не исторгнется отъ сѣти его. О, бѣдный грѣшникъ! возстенай и возопій изъ глубины сердечной ко Іисусу, Свободителю душъ нашихъ; воззови изъ сей погибели, якоже Іона изъ чрева китова, или якоже Манассія изъ средины узъ, ко Всесильному Іисусу, да послетъ тебѣ помощь Свою, да, тою укрѣпленъ, исторгнешися отъ пагубныя сея сѣти; и потщися о семъ, возлюбленне, заранѣе, пока еще не привлеклъ тебя въ погибель свою. Положи только начало доброе, и взывай къ Нему со всякимъ усердіемъ. Онъ, яко милосердъ и человѣколюбивъ, простретъ Свою тебѣ всесильную десницу, и исторгнетъ тя отъ сѣти сея вражія, и свободенъ будеши отъ нея. О, Іисусе Человѣколюбче! пощади созданіе рукъ Твоихъ, егоже ради въ міръ пришелъ еси и пострадалъ, да не порадуется врагъ нашъ о погибели нашей. \textit{Аще Сынъ вы свободитъ, воистинну свободни будете}\footnote{Іоан.~8,~36.}. А когда сію милость отъ Него получиши, тогда радостнымъ духомъ воспоеши Ему: \textit{Растерзалъ еси узы моя. Тебѣ пожру жертвы хвалы}\footnote{Пс.~115,~7--8.}. И паки: \textit{Исповѣмся Тебѣ Господи Боже мой, всѣмъ сердцемъ моимъ, и прославлю имя Твое въ вѣкъ: яко милость Твоя велія на мнѣ, и избавилъ еси душу мою отъ ада преисподнѣйшаго}\footnote{85,~12--13.}. Такую милость получивши отъ Него, впредь опасно поступай, и берегись сѣтей вражіихъ, призывая на помощь себѣ Господа твоего. \textit{Заступникъ души моей буди, Боже, яко посредѣ хожду сѣтей многихъ; избави мя отъ нихъ, и спаси мя, Блаже, яко человѣколюбецъ}.

\section{51. Піянство.}

Есть піянство отъ вина и сикеры; есть піянство и не отъ вина, якоже глаголется: \textit{піяніи безъ вина}\footnote{Ис.~28,~1.}. Піянство отъ вина есть, когда человѣкъ выше мѣры употребляетъ вина; піянство не отъ вина, когда человѣкъ любовію міра сего, суетными мыслями и беззаконными начинаніями упивается. Упившійся отъ вина часто не знаетъ, что говоритъ и дѣлаетъ, и ни стыда, ни страха не имѣетъ, и что ни дѣлаетъ, почти все смѣха достойное дѣлаетъ: тако упившійся любовію міра сего и прочими беззаконными мыслями, не знаетъ, что дѣлаетъ; то за то, то за другое хватается, но все начинаніе и дѣло его противу его есть. Видитъ, что вси умираютъ, и никто съ собою ничего не относитъ: однакожъ такъ старается о умноженіи богатства, о разширеніи земли, о созиданіи домовъ и прочіихъ прихотей своихъ, о пріобрѣтеніи чести и славы суетныя, "--- такъ, говорю, старается, какъ бы онъ одинъ въ мірѣ семъ имѣлъ вѣчно жить. \textit{Рече же ему Богъ: безумне! въ сію нощь душу твою истяжутъ отъ тебе: а яже уготовалъ еси, кому будутъ}\footnote{Лук.~12,~20.}? Воистину безуменъ, и смѣха, или паче сожалѣнія, достоинъ таковый! Аще бы кто на чуждой сторонѣ былъ, и имѣлъ бы оттуду скоро возвратитися въ отечество и домъ свой, а много бы тамо запасалъ недвижимыхъ вещей, не былъ ли бы онъ смѣха достоинъ? Неотмѣнно бы всякъ, видя дѣла его и начинанія, достойно выговаривалъ ему: вѣдь ты все сіе здѣ оставишь; почтожъ убо такъ много заготовляешь? Тако безуменъ есть и смѣха достоинъ, который въ мірѣ семъ много запасаетъ, но вѣдаетъ, что все сіе надобно ему оставить въ мірѣ, какъ на чужой сторонѣ, и оставить вскорѣ. Дѣлаетъ то въ немъ піянство, не отъ вина и сикеры, но отъ любви міра сего сотворенное, которое такъ умъ его помрачило, что бѣдный самъ не знаетъ, что дѣлаетъ. Люто есть піянство отъ вина и сикеры; но сіе лютѣйшее есть. Упившійся отъ вина и сикеры удобно истрезвляется; но упившійся любовію міра сего съ великою трудностію. Похоть, которая въ немъ живетъ и обезумляетъ его, много причинъ вымышляетъ, и не допущаетъ его истрезвитися. Мнѣ"=де ради жены, ради дѣтей, ради старости моей не мало надобно. А честь и слава, которой не менѣе, какъ богатства желаешь, къ чему тебѣ надобна? къ чему надобенъ богатый уборъ платья, домовъ, трапезъ, слугъ, коней, каретъ, алмазовъ, брилліантовъ, и прочія суеты? О, упившійся не виномъ и сикерою, но похотію суетнаго міра! продери глаза твои и посмотри на тѣхъ, который были такіежде, какъ и ты: гдѣ ихъ богатство? гдѣ честь и слава? гдѣ прихоти и гордыня? гдѣ суетный уборъ? Оставило ихъ все, когда міръ они оставили; или паче міръ ихъ оставилъ и нехотящихъ. И случилося имъ тое, что тѣмъ людямъ, которыи во снѣ много имѣютъ и много пьютъ, но пробудившися ничего въ рукахъ не видятъ и чувствуютъ великую жажду. Тако упившіеся похотію суетнаго міра, пока въ мірѣ жили, какъ во снѣ, много имѣли и пили; но при смерти, а паче по смерти, увидѣли себе нищихъ и почувствовали въ себѣ великую жажду; и просятъ капли воды, но не дается имъ. Слышатъ отвѣтъ: \textit{чадо! помяни, яко воспріялъ еси благая твоя въ животѣ твоемъ}\footnote{Лук.~16,~25.}. И сбывается на нихъ, но поздное, безполезное оное раскаяніе нечестивыхъ: \textit{что пользова намъ гордыня? и богатство съ величаніемъ что воздаде намъ? Преидоша вся она яко сѣнь, и яко вѣсть претекающая; яко корабль преходяй волнующуюся воду, якоже проходу нѣсть стопы обрѣсти, ниже стези шествія его въ волнахъ; или яко птицы прелетающія по аеру, ни едино обрѣтается знаменіе пути, язвою же смущаяй біемый духъ легкій, и разсѣцаемый силою шумящею движеніемъ крилъ прелетѣ, и по семъ ни едино знаменіе обрѣтеся проходу въ немъ; или яко стрѣлою испущеною на намѣренное мѣсто разсѣченный аеръ внезапу въ себѣ самомъ заключенъ бысть, яко не познатися проходу ея. Тако и мы рождени оскудѣхомъ, и добродѣтели убо ни единою знаменія можемъ показати, во злобѣ же нашей скончахомся}\footnote{Пр. Солом.~5,~8--13.}. Берегись, чтобы и тебѣ въ сіе поздное и безполезное раскаяніе не пріити. Мудръ и блаженъ, кто отъ бѣдствія иныхъ научается самъ бѣдствія убѣгать.

\subsection{О томжде.}

Часто бываетъ, что упившійся виномъ много вреда людямъ дѣлаетъ. Ибо тогда умъ у него помраченъ, и потому не имѣетъ здраваго разсужденія, а дѣйствуетъ только горячесть, отъ вина ему прибывшая, и такъ подобенъ есть бѣсноватому, который самъ не знаетъ, что дѣлаетъ. Тако упившійся злою суетнаго міра похотію, много вреда людямъ дѣлаетъ; и далеко болѣе, нежели піяный отъ вина. Отъ кого бѣдныи и беззаступныи насилія и обиды терпятъ? Отъ кого вдовицы и сироты плачутъ и кровавыми слезами умываются, какъ отъ сильныхъ, любовію міра сего упоенныхъ? Гдѣ болѣе нищихъ, убогихъ, полунагихъ, и всякій недостатокъ къ пропитанію и житію имѣющихъ, какъ въ крестьянахъ, которыми господа міролюбивые владѣютъ? Въ которой странѣ болѣе хищеній, воровства, насилій, разбоевъ, убійствъ и прочіихъ беззаконныхъ дѣлъ, какъ въ той, въ которой властелинъ вредною міра сего любовію упоенъ? Вся сія и большая сихъ злая ненасытное суетнаго міра піянство дѣлаетъ. Видимъ сіе беззаконное піянство, вездѣ разливающееся; видимъ и упоенныхъ, тѣмъ свирѣпѣющихъ; видимъ, и воздыхаемъ. Тако упившемуся что на умъ не приходитъ? чего не замышляетъ, чтобы жажду, крыющуюся въ сердцѣ своемъ, угасить? Хочется въ богатомъ и красномъ домѣ жить, богатую и изобильную трапезу поставлять, въ богатомъ и красномъ одѣяніи ходить, себѣ, женѣ и дѣтямъ слугъ въ пристойномъ уборѣ предстоящихъ имѣть, на дорогихъ коняхъ и каретахъ проѣзжаться, пруды, сады и галдареи увеселительныи имѣть, и прочая симъ подобная дѣлать. Но откуду взять? гдѣ сыскать на то все сумму? Господину надобно собирать тую съ крестьянъ, надобно болѣе налагать на нихъ оброковъ!.. Судіи надобно собирать съ приходящихъ къ суду; вмѣсто правды неправду дѣлать, о законѣ Божіи и о Бозѣ нерадѣть; не смотрѣть на праваго и виноватаго, оправдать нечестиваго и осуждать праведнаго!.. Беззаконному купцу надобно лгать, обманывать, льстить, и худую вещь за добрую, и дешевую за дорогую продавать!.. Тако и въ прочемъ чину и званіи піянство не отъ вина, но отъ міра, много зла и вреда дѣлаетъ. О человѣче, не виномъ, но похотію міра сего упившійся! разсуждай сіе и смотри, сколько ты обиды и вреда людямъ дѣлаешь: воистину болѣе, нежели піяный отъ вина! Упившійся отъ вина, истрезвившися, часто жалѣетъ, что въ шумствѣ своемъ того"=то и того обидѣлъ: но упоенный отъ похоти міра не жалѣетъ, что толь много дѣлалъ и дѣлаетъ людямъ подобнымъ себѣ обиды: не жалѣетъ, понеже помраченный умъ, яко піяный, имѣетъ. О лютое и пагубное піянство! окаяненъ, кто тѣмъ плѣненъ! блаженъ, кто тѣмъ не окалялся! О хрістіанине, упившійся не отъ вина, но отъ похоти суетнаго міра! оставь сіе душепагубное піянство и истрезвися, да не съ нимъ явишися предъ страшнымъ онымъ Судіею и Господемъ твоимъ; истрезвись пожалуй, и самъ увидишь, какъ твой помраченъ умъ былъ, и какъ ты худо дѣлалъ, что ни дѣлалъ, и неотмѣнно будешь самъ жалѣть и каяться о дѣлахъ твоихъ! \textit{Тако глаголетъ Господь, Господь Святый Израилевъ: егда возвратився воздохнеши, тогда спасешися, и уразумѣеши, гдѣ еси былъ}\footnote{Ис.~30,~15.}. \textit{Свѣтильникъ тѣлу есть око: аще убо будетъ око твое просто, все тѣло твое свѣтло будетъ; аще ли око твое лукаво будетъ, все тѣло твое темно будетъ. Аще убо свѣтъ, иже въ тебѣ, тма есть, то тма кольми}\footnote{Мѳ.~6,~22--23.}. \textit{Не любите міра, ни яже въ мірѣ. Аще кто любитъ міръ, нѣсть любве Отчи въ немъ: яко все, еже въ мірѣ, похоть плотская, и похоть очесъ, и гордость житейская, нѣсть отъ Отца, но отъ міра сего есть. И міръ преходитъ, и похоть его: а творяй волю Божію пребываетъ вовѣки}\footnote{1~Іоан.~2,~15--17.}. \textit{Горняя мудрствуйте, а не земная}\footnote{Кол.~3,~2.}. \textit{Отврати очи мои, Господи, еже не видѣти суеты}\footnote{Пс.~118,~37.}.

\subsection{О томжде.}

Упивающіеся отъ вина учатся другъ отъ друга беззаконнаго того дѣла, какъ видимъ. Похоть бо, крыющаяся въ сердцѣ человѣческомъ, возбуждается и разжигается видѣніемъ и слухомъ. Юному человѣку, да и всякому, весьма трудно не научиться піянства, и тако не развратиться, когда съ піяницею будетъ водиться. Тако піянства не отъ вина, но отъ похоти міра, люди другъ отъ друга научаются. Видимъ сію беззаконную вездѣ ревность. Зло сіе предъ глазами всѣхъ обращается. Сколько видимъ перемѣнъ въ созиданіи домовъ, въ сочиненіи одеждъ, въ пріуготовленіи трапезъ, въ убраніи каретъ и коней, и прочія суеты, красоты и пышности міра сего; сколько, говорю, видимъ перемѣнъ; но всѣ онѣ въ горшее бываютъ. Благочестіе всегда одинаково; какъ отъ начала міра было, такъ и нынѣ тоежде есть и всегда тоежде будетъ. Истина бо всегда истина есть, всегда непремѣнна и была и есть и будетъ. Суета и прелесть непостоянна, но всегда премѣняется. Смотри на суету! Единъ построилъ такія"=то и такія хоромы, единъ началъ носить такую"=то и такую одежду, единъ поставилъ такія"=то и такія зеркала въ домѣ своемъ, единъ началъ въ такой"=то и въ такой каретѣ ѣздить, столько и столько коней избранныхъ имѣть, такую и такую трапезу поставлять, въ такомъ"=то и въ такомъ убранствѣ слугъ предстоящихъ имѣть, и прочая. Видитъ тое другій, и подражаетъ ему; видятъ тое вси, и дѣлаютъ тое, что одинъ. Итакъ разливается вездѣ и умножается роскошь, и часъ отъ часу болѣе и болѣе усиливается; и съ роскошію умножается всякое зло, и поядаетъ души человѣческія не иначе, какъ пожаръ, который, въ одномъ домѣ наченшися, весь градъ или село пожигаетъ, "--- или какъ моровая язва, въ одномъ человѣкѣ наченшаяся, многихъ близъ находящихся заражаетъ и умерщвляетъ. Видимъ сію во отечествѣ нашемъ всепагубную язву, которая не тѣлеса, но души хрістіанскія заразила. Какъ посмотрѣть на роскошь людей: то уже и подлое благородство, и купцы, которыи прежде какъ люди простыя ходили и жили, вси князьями и вельможами сдѣлались, не хотятъ уже жить, какъ только въ богатыхъ и красныхъ домахъ; не хотятъ сидѣть, какъ только за богатою и различныхъ снѣдей исполненною трапезою; не хотятъ вкушать, какъ только избраннаго и дорогаго вина; не хотятъ ходить, какъ только въ шелковыхъ и красныхъ одеждахъ, въ лисьихъ, куньихъ и собольихъ шубахъ; не хотятъ проѣзжаться, какъ только аглицкою каретою. И тако толикая суета, гордость и пышность міра сего вошла въ хрістіанъ и день отъ дня умножается, что ежели бы предки наши востали изъ мертвыхъ, то не узнали бы своего отечества. Предки"=де наши не знали, какъ на свѣтѣ жить; они не умѣли, какъ добра употреблять. Осмотрись, друже, кому сіе слово паче приличествуетъ вамъ ли, или предкамъ вашимъ? Предки ваши жили въ простотѣ и смиреніи, и потому по"=хрістіански и разумно жили: вы въ гордости и пышности живете, и потому далеко отъ хрістіанскаго житія отстоите. У предковъ вашихъ менѣе было роскошей, такъ болѣе благочестія; и менѣе нищихъ и убогихъ людей было: ибо менѣе они брали съ людей, и болѣе давали убогимъ и нищимъ; у васъ умножилось роскошей, такъ умножилось нищихъ и убогихъ, плачущихъ и кровавыя слезы проливающихъ. Вы начали въ богатыхъ домахъ жить: такъ много находится такихъ, которые хижинъ не имѣютъ, гдѣ главу подклонить. Вы начали богатую трапезу поставлять, и драгія вина пить: такъ многіе начали не имѣть дневнаго пропитанія. Вы стали въ богатомъ одѣяніи ходить: такъ видимъ, что многіе въ рубищахъ, многіе полунаги ходятъ. Вы вздумали и захотѣли въ каретахъ и на коняхъ ѣздить богатыхъ: такъ многіе плачутъ и жалуются, что не имѣютъ чимъ землю пахать, и прочая. Сами убо разсудите "--- вы ли, или предки ваши разумнѣйшіи? О хрістіане, не виномъ, но похотію міра сего упившіеся! на сіе ли Хрістосъ нашъ позвалъ насъ въ вѣру Свою святую? Онъ намъ не роскоши, но кресты и скорби предлагаетъ въ мірѣ семъ. \textit{Внидите узкими враты}, и пр. \textit{Аще кто хощетъ по Мнѣ ити, да отвержется себе, и возметъ крестъ свой, и по Мнѣ грядетъ. Въ мірѣ скорбни будете}\footnote{Мѳ.~7,~13; 16,~24; Іоан.~16,~33.}. У васъ здѣ обычайный отвѣтъ: «не всѣмъ"=де въ монастыри и пустыни итить». О возлюбленніи! тогда еще и монастырей не было, когда сіи слова сказаны: онѣ предлагаются не токмо пустыннымъ и монастырскимъ жителямъ, но всѣмъ хрістіанамъ, по городамъ и селамъ живущимъ, и всякаго званія, чина и обоего пола людямъ, кто только ни хощетъ Хрістовымъ быть и спастися. \textit{Иже Хрістовы суть, плоть распяша со страстьми и похотьми}\footnote{Гал.~5,~24.}. Послѣдовательно не Хрістовы суть, иже не распяша плоти со страстьми и похотьми. Видите, что неправильный вашъ отвѣтъ. Но царствуйте, царствуйте здѣ съ міромъ, когда хощете; веселитеся и утѣшайтеся роскошами вашими; ѣздите другъ къ другу въ гости, пируйте, банкетуйте и танцы свои производите! Какъ"=то тамо будете ликовать и танцовать!.. Читаемъ въ святомъ Евангеліи, что \textit{человѣкъ нѣкій былъ богатъ и облачался въ порфѵру и вѵссонъ, веселяся на вся дни свѣтло}. Но видимъ тамжде, что по смерти ужасная ему перемѣна учинилась; по роскошахъ своихъ пошелъ въ пламенное мученіе; и по дорогихъ винахъ проситъ капли воды: и не дается ему; слышитъ отвѣтъ: \textit{чадо! помяни, яко воспріялъ еси благая твоя въ животѣ твоемъ}\footnote{Лук.~16,~19--26.}. Берегитесь убо, возлюбленніи, чтобы и вамъ не пріити по роскошахъ вашихъ на оное мѣсто! Богъ лица не пріемлетъ. Богъ подалъ благая міра сего намъ и позволилъ ихъ употреблять, но на нужды наши, а не на роскоши. Сердцемъ же и любовію нашею къ Нему \textit{единому} прилѣплятися повелѣлъ. \textit{Что ми есть на небеси и отъ Тебе что восхотѣхъ на земли}, и проч.\footnote{Пс.~72,~25.} \textit{Спаси мя, Господи, яко оскудѣ преподобный: яко умалишася истины отъ сыновъ человѣческихъ}\footnote{11,~2.}. О ты, благочестивая душа, которая душепагубнаго сего піянства не имѣешь и живешь въ мірѣ, какъ Лотъ въ Содомѣ! стой утверждайся, и буди непоступна; \textit{не ревнуй лукавнующимъ, ниже завиди творящимъ беззаконіе: зане яко трава скоро изсшутъ, и яко зеліе злака скоро отпадутъ}\footnote{Пс.~36,~1--2.}. Буди и ты, какъ Даніилъ въ Вавилонѣ, который дверцами отверстыми смотрѣлъ ко Іерусалиму, отечеству своему, и тако колѣна своя преклонялъ и молился\footnote{Дан.~6,~10.}. Смотри и ты вѣрою и душевнымъ окомъ къ горнему Іерусалиму отъ міра сего, какъ изъ Вавилона, и преклоняяся возводи очи твои ко Господу. \textit{Къ Тебѣ возведохъ очи мои, живущему на небеси}\footnote{Пс.~122,~1.} и проч.

\subsection{О томжде.}

Піяный отъ вина не чувствуетъ, коль вредно есть піянство, пока упивается: тако упившійся отъ похотей міра сего не знаетъ, коль вредны суть похоти тыи, пока въ нихъ пребываетъ. Ибо какъ у того, такъ и у сего умъ помраченъ бываетъ. Упившійся отъ вина, какъ начнетъ истрезвлятися, познаетъ, коль вредно есть піянство: тако упившійся отъ похоти міра, когда начнетъ въ чувство приходить, познаетъ, коль вредны суть похоти міра сего. Истрезвляющійся отъ піянства чувствуетъ въ тѣлѣ немалую слабость: тако отъ прихотей міра сего отстающій познаетъ души разслабленіе. Какъ бо піянство отъ вина тѣло, тако прихоти міра сего душу разслабляютъ. Человѣкъ, истрезвившися отъ піянства, жалѣетъ и стыдится, что безъ мѣры употреблялъ вино, и въ безчувствіи былъ и безчинно поступалъ, и тако самъ себѣ вредилъ и людямъ смѣхъ былъ: тако хрістіанинъ, пришедшій въ чувство, жалѣетъ и стыдится, кается и сокрушается, что за прихотями міра сего гонялся и оставлялъ истинное добро, и вси прежніе дни, какъ погибшіе, оплакиваетъ; тогда онъ познаетъ, въ какой суетѣ жилъ и прелести. Тако Соломонъ, пришедши въ чувство и позная дѣла свои суетныи, признался, глаголя: \textit{возвеличихъ твореніе мое, создалъ ми домы, насадилъ ми вінограды, сотворилъ ми вертограды и сады, насадилъ въ нихъ древесъ всякаго плода; сотворилъ ми купѣли водныя, еже напаяти отъ нихъ прозябеніе древесъ. Притяжахъ рабы и рабыни, и домочадцы быша ми, и стяжаніе скота и стадъ много ми бысть, паче всѣхъ бывшихъ прежде мене во Іерусалимѣ. Собралъ ми злато и сребро, и имѣнія царей и странъ: сотворилъ ми поющихъ и поющія, и услажденія сыновъ человѣческихъ, віночерпцы и виночерпицы. И возвеличился, и приложихся мудрости, паче всѣхъ бывшихъ прежде мене во Іерусалимѣ; и мудрость моя пребысть со мною. И все, егоже просиста очи мои, не отъяхъ отъ нихъ, и не возбранилъ сердцу моему отъ всякаго веселія моего, яко сердце мое возвеселися во всякомъ трудѣ моемъ, и сіе бысть часть моя отъ всего труда моего. И призрѣлъ азъ на вся творенія моя, яже сотвориста руцѣ мои, и на трудъ, имже трудился творити: и се вся суета}\footnote{Еккл.~2,~4--11.}. Тако и ты, христіанине, когда истрезвишися отъ сего душевреднаго піянства, то воистину познаешь, что все, что ты ни дѣлалъ, думалъ, замышлялъ и начиналъ, есть суета и прелесть; что тебѣ казалось красно, тое въ себѣ есть безобразно; и что видѣлось тебѣ добро, тое внутрь есть зло; и признаешь самъ, что все сіе есть \textit{суета}. Прелесть бо всегда кажется нѣчто быти; но какъ разсмотришь ее, то внутрь себе ничто есть; и кажется нѣчто сладкое быти, но внутрь есть горько. Упившимся отъ вина обычай есть много воды пить, чтобы водою прибывшую отъ вина горячесть удобнѣе выгнать: ты, хрістіанине, возми въ разсужденіе \textit{послѣдняя четыре}: разсуждай почаще о \textit{смерти}, при которой всѣ прихоти твои оставишь; и о страшномъ \textit{судѣ Хрістовомъ}, предъ которымъ надобно и тебѣ явиться, какъ и всѣмъ; и о блаженной \textit{вѣчности}, въ которую пойдутъ боголюбцы; и о несчастливой и \textit{мучительной}, которой не избѣжатъ міролюбцы и вси грѣшники. Таковымъ разсужденіемъ, какъ водою, піянство твое, въ сердцѣ твоемъ крыющееся, угашай. Воистину, говорю тебѣ, скоро выйдетъ изъ головы твоей вся міра сего суета, когда почаще будешь о тѣхъ пунктахъ разсуждать! Болѣе будешь желать плача и слезъ, нежели веселыхъ дней. Доволенъ будеши укрухомъ хлѣба со щами, и вертепомъ и хижиною вмѣсто богатаго и краснаго дома; и всю красоту міра сего, какъ мертвечину, вмѣнишь. Что люди за суетою гоняются, и въ прихотяхъ міра сего, какъ піяныи, свирѣпѣютъ; то оттуду бываетъ, что забываютъ о вѣчности и обстоятельствахъ ея. Сіе забвеніе дѣлаетъ врагъ душъ человѣческихъ, сатана, чтобы люди, не памятуя о вѣчности, не памятовали и не старались о спасеніи своемъ вѣчномъ, которое Хрістосъ Сынъ Божій страданіемъ и смертію Своею содѣлалъ. Помни убо, возлюбленный хрістіанине, пожалуй помни и разсуждай о вѣчности: и истрезвишися, и здравъ разумъ будешь имѣть, и всю красоту и прихоти міра сего за ничто вмѣниши. Се есть хрістіанская мудрость! Душа твоя, ради которой Хрістосъ Сынъ Божій пострадалъ и умеръ, паче всего міра дражайшая есть; о той, чтобы спасена была, пекися и старайся прилѣжно. Образъ Божій, по которому созданы были мы, есть краснѣйшая и великолѣпнѣйшая души доброта: сея доброты и красоты истиннымъ покаяніемъ и вѣрою ищи. Когда она здѣ сыщется, то во вѣки пребудетъ: когда здѣ не сыщется, то уже никогда не сыщется. \textit{Міръ преходитъ, и похоть его: а творяй волю Божію пребываетъ во вѣки}\footnote{1~Іоан.~2,~17.}.

\section{52. Зеркало.}

Видимъ, что тое въ зеркалѣ изображается, къ чему оно обращается: тако имѣется и душа человѣческая; къ чему она любовію обращается и прилѣпляется, тое въ ней и изображается. Зеркало, когда къ небу обращается, небо, "--- когда къ землѣ обратится, земля въ немъ видится: тако и душа человѣческая, когда къ небу обращается, небесный образъ въ ней и изображается: когда къ земнымъ прилѣпляется любовію своею, земляный и скотскій образъ въ ней изображается, и тако погубляетъ благородіе свое и небесную свою доброту. Сего ради увѣщаваетъ насъ Духъ Святый чрезъ апостола: \textit{не любите міра, ни яже въ мірѣ: аще кто любитъ міръ, нѣсть любве Отчи въ немъ; яко все, еже въ мірѣ, похоть плотская, и похоть очесъ, и гордость житейская, нѣсть отъ Отца, но отъ міра сего есть}\footnote{2,~15--16.}. А что въ душѣ нынѣ изображается, съ тѣмъ она и на судѣ Хрістовомъ явится, когда нынѣ истинно не покается; и тако всякъ человѣкъ самъ въ себѣ тогда обличеніе будетъ имѣть, и увидитъ въ себѣ самомъ, что онъ мудрствовалъ въ мірѣ "--- горняя, или земная? Хрістіане словомъ Божіимъ позваны и святымъ крещеніемъ обновлены къ вѣчной жизни и небеснымъ благимъ: сихъ усердно и искать должны. Но когда къ земнымъ прилѣпляются: Бога прогнѣвляютъ. Читаемъ, что Израильтяне, изшедши изъ Египта руководствомъ Моѵсеа раба Божія, когда сердцами обратилися во Египетъ, и \textit{похотѣли снѣдей его}, весьма \textit{прогнѣвали Бога}\footnote{Числ.~11,~1--11.}: тако хрістіане, въ святомъ крещеніи изведенные изъ работы діавольскія, и отъ міра къ небу, какъ къ земли обѣтованной позванные, когда къ міру сердцами обращаются, праведный на себя разжигаютъ гнѣвъ Божій, и случается имъ тое, что позваннымъ на велію вечерю, но отрекшимся ити, о которыхъ сказано: \textit{яко ни единъ мужей тѣхъ званыхъ вкуситъ Моея вечери. Мнози бо суть звани, мало же избранныхъ}\footnote{Лук.~14,~24.}. Вси хрістіане позваны на великую вечерю вѣчныя жизни: но хотя не словомъ, однакожъ дѣломъ, многіи отрицаются ити; изволяютъ здѣ въ мірѣ царствовати, нежели во оно царство итить: и тако \textit{ни единъ отъ нихъ вкуситъ вечери} оныя. Хрістово житіе научаетъ насъ, како въ мірѣ семъ обращаться и жительствовать намъ должно. Хрістосъ все могъ въ мірѣ семъ имѣть, яко Господь всѣхъ, но не хотѣлъ; въ великой нищетѣ и смиреніи, и скорби и поруганіи пожилъ, \textit{образъ подая намъ}. Кто сему Вождю послѣдуетъ, тотъ вкуситъ вечери оныя, и съ Нимъ, яко удъ съ главою, будетъ тамо царствовать. \textit{Аще кто Мнѣ служитъ, Мнѣ да послѣдствуетъ: и идѣже есмь Азъ, ту и слуга Мой будетъ; и аще кто Мнѣ служитъ, почтитъ его Отецъ Мой}\footnote{Іоан.~12,~26.}. \textit{Вы есте пребывше со Мною въ напастехъ Моихъ: и Азъ завѣщаю вамъ, якоже завѣща Мнѣ Отецъ Мой, царство, да ясте и піете на трапезѣ Моей во царствіи Моемъ}\footnote{Лук.~22,~28--30.}. \textit{Иже не пріиметъ креста своего, и въ слѣдъ Мене грядетъ, нѣсть Мене достоинъ}\footnote{Мѳ.~10,~38.}. \textit{Господи Боже силъ! обрати ны и просвѣти лице Твое, и спасемся}\footnote{Пс.~79,~20.}. \textit{Возврати, Господи, плѣненіе наше, яко потоки югомъ}\footnote{125,~4.}.

\section{53. Изшедшій изъ тьмы, изъ темницы, изъ плѣна, и проч.}

Якоже изшедшій изъ тьмы на свѣтъ: тако обратившійся отъ беззаконнаго житія къ Богу. Ибо и сей въ беззаконномъ житіи, какъ во тьмѣ, обращался и блудилъ и осязалъ, какъ во тьмѣ ходящія блудятъ и осязаютъ какъ слѣпіи. Якоже изшедшій изъ темницы: тако оставившій нечестивое и безбожное житіе и находящійся въ истинномъ покаяніи. Ибо и сей въ безбожномъ житіи, какъ въ темницѣ, былъ; и грѣхами и беззаконіями, какъ узами, связанъ былъ и не моглъ свободно ходить. Якоже свободившійся отъ плѣна: тако избѣгшій и свободившійся отъ грѣховнаго и нечистаго житія и свободнымъ духомъ работающій Господеви. Ибо и сей въ плѣну у мучителя, не человѣка, но діавола, былъ, и ему работалъ и злую его волю исполнялъ. Якоже изъ рва излѣзшій и ходящій безопаснымъ путемъ: тако отъ нераскаяннаго житія въ покаяніе пришедшій и въ немъ пребывающій. Ибо и сей въ ровѣ былъ не тѣломъ, но душею. Якоже свободившійся отъ потопленія и изшедшій на сушу: тако преставшій отъ грѣховъ и наченшій новое христіанское житіе. Ибо и сей во грѣхахъ, какъ во глубинѣ водной, тонулъ; и не иное что послѣдовало ему, какъ вѣчная смерть. Якоже шедшій къ ямѣ, и ее усмотрѣвшій, и взадъ возвратившійся: тако ходившій въ беззаконномъ житіи, и узнавшій свою бѣду душевную, и обратившійся къ истинному покаянію, и въ томъ пребывающій. Ибо и сей шелъ въ яму вѣчныя погибели; но возвратился отъ нея, и проч. Счастливъ, кто выходитъ изъ тьмы на свѣтъ, видитъ бо свѣтъ міра сего: далеко счастливѣйшій, кто выходитъ изъ тьмы грѣховныя; ибо сего озаряетъ не свѣтъ міра сего, но свѣтъ Божественный. \textit{Ходяй по Мнѣ, не имать ходити во тьмѣ, но имать свѣтъ животный}\footnote{Іоан.~8,~12.}. \textit{Сего ради глаголетъ: востани спяй, и воскресни отъ мертвыхъ, и освѣтитъ тя Хрістосъ}\footnote{Еф.~5,~14.}. Счастливъ, кто вышелъ изъ темницы и свободился отъ узъ: далеко счастливѣйшій, кто изъ темницы нераскаяннаго житія и узъ грѣховныхъ свободился; счастливъ, кто свободился отъ плѣна тѣлеснаго: но много счастливѣйшій, кто свободился душею отъ плѣненія діавольскаго; счастливъ, кто изъ рва видимаго избавился: но далеко счастливѣйшій, кто изъ рва грѣховнаго избавился. Счастливъ, кто спаслся отъ потопленія воднаго: но далеко болѣе счастливъ, кто спаслся отъ потопленія грѣховнаго и глубины золъ. Счастливъ, кто шелъ къ ямѣ, и имѣлъ въ нее впасти, и, усмотрѣвъ ее, возвратился вспять: несравненно счастливѣйшій, кто шелъ къ ямѣ вѣчныя погибели, и, увидѣвши ее и отвратившися, взадъ спѣшитъ. Аще кто таковый есть, и счастія таковаго сподобился: поздравляю его тѣмъ, и возвѣщаю отъ Божія слова, что \textit{ангели на небеси о немъ радуются}\footnote{Лук.~15,~10.}. Но, о возлюбленне! стой въ томъ, что началъ. Поминай жену Лотову, и не озирайся вспять на Содомъ и Гоморръ. \textit{Никтоже бо возложь руку свою на рало, и зря вспять, управленъ есть въ царствіи Божіи}\footnote{Лук.~9,~62.}. И притомъ помни писанное: \textit{чадо! аще приступаеши работати Господеви Богу, уготови душу твою во искушеніе}\footnote{Сир.~2,~1.}. Поженетъ въ слѣдъ тебе, какъ Фараонъ въ слѣдъ Израиля, діаволъ съ козньми, отъ котораго работы ты свободился, и будетъ тщаться паки тебе покорить себѣ въ работу: ты стой, не бойся, не унывай, призывая всесильнаго избавителя душъ нашихъ Іисуса: и увидишь спасеніе Его, и радостнымъ духомъ воспоешь Ему: \textit{Помощникъ и покровитель бысть мнѣ во спасеніе; Сей мой Богъ, и прославлю Его}, и проч.\footnote{Исх.~15,~2.} И когда будетъ ввергать тебе тойжде врагъ въ разженную искушенія пещь; молися съ тремя святыми отроками: и тутъ тебѣ приспѣетъ Іисусъ Сынъ Божій на помощь, якоже тѣмъ святымъ отрокамъ приспѣлъ во \textit{образѣ Ангела}; и прохладитъ тебе, якоже и оныхъ \textit{прохладилъ}\footnote{Дан.~3,~49--50.}. Совершай убо дѣло съ помощію Хрістовою, которое началъ. Счастливъ ты, что таковое начало положилъ; но совершенно счастливъ будеши, когда счастливо и окончишь: многіе бо начинаютъ, но не многіе совершаютъ; \textit{мнози бо суть звани, мало же избранныхъ}\footnote{Лук.~14,~24.}. Того ради глаголетъ тебѣ начальникъ спасенія нашего и совершитель Іисусъ: \textit{буди вѣренъ до смерти, и дамъ ти вѣнецъ живота}\footnote{Апок.~2,~10.}. \textit{Не остави мене, Господи Боже мой, не отступи отъ мене; вонми въ помощь мою, Господи спасенія моего}\footnote{Пс.~37,~22--23.}. \textit{Не даждь во смятеніе ноги твоея, ниже воздремлетъ храняй тя. Се не воздремлетъ, ниже уснетъ храняй Израиля! Господь сохранитъ тя, Господь покровъ твой на руку десную твою. Во дни солнце не ожжетъ тебе, ниже луна нощію. Господь сохранитъ тя отъ всякаго зла, сохранитъ душу твою Господь: Господь сохранитъ вхожденіе твое и исхожденіе твое отъ нынѣ и до вѣка}\footnote{Пс.~120,~4--8.}.

\section{54. Стыдъ.}

Видимъ, что подданные предъ монархомъ своимъ, подвластныи предъ властію своею, раби предъ господиномъ своимъ, дѣти предъ отцемъ своимъ стыдятся дѣлать то, что не прилично; стыдятся, да и боятся. Хрістіанине! Богъ на всякомъ мѣстѣ есть, и вездѣ присутствуетъ, и на насъ, и на дѣла наша призираетъ; что мы дѣлаемъ, мыслимъ и начинаемъ, видитъ, и что говоримъ, слышитъ. О семъ на многихъ святаго Писанія мѣстахъ свидѣтельствуется\footnote{Пс.~32,~13--15; 138,~1--16; Сир.~23,~27--29; Іер.~32,~17,~19,~27, и проч.}. О семъ премудрый Августинъ къ Богу говоритъ: «Признаюсь, Господь, яко что я ни дѣлаю и какое дѣло ни дѣлаю, предъ Тобою дѣлаю; и все тое, что дѣлаю, лучше Ты видишь, нежели я, который дѣлаю. Что бо всегда ни дѣлаю, Ты равно всегда присутствуешь, яко всегдашній назиратель всѣхъ помышленій, намѣреній, услажденій и дѣяній моихъ»\footnote{Бес. съ самимъ собою (soliloquia),~14,~41.}. Что о себѣ Августинъ признаетъ, то и всякъ о себѣ долженъ признавать. Должно убо намъ, "--- намъ, говорю, которые Бога признаемъ и исповѣдуемъ, и въ молитвѣ призываемъ и поемъ, и чаемъ и ожидаемъ отъ Него милости, должно, говорю, стыдиться дѣлать тое, что не прилично, и Его величеству противно, и намъ вредно; должно стыдиться, да и бояться: ибо \textit{Богъ нашъ огнь есть}, нечествующихъ къ Нему \textit{поядаяй}\footnote{Евр.~12,~22.}. Человѣче! предъ человѣкомъ стыдишися худое дѣлать: кольми паче предъ Богомъ должно стыдитися! Не видишь ты Его, и видѣть не можешь; но Онъ видитъ тебе, и на дѣло твое, начинаніе и помышленіе смотритъ, и всякое слово твое слышитъ. \textit{Насаждей ухо не слышитъ ли? или создавый око не сматряетъ ли?}\footnote{Пс.~93,~9, и проч.} Оскорбляется царь, когда предъ нимъ подданный его безчинствуетъ; и господинъ, когда предъ нимъ рабъ его неприличное дѣлаетъ; и отецъ, когда дѣти его предъ нимъ нечестно поступаютъ: видимъ, тое вси мы. Хрістіанине! думай и держи, что тако Богъ оскорбляется, когда предъ святѣйшими очами Его люди, а паче хрістіане, знающіе и исповѣдающіе имя Его, беззаконнуютъ. Правда Его неправдою твоею, истина Его лжею твоею, святость Его нечистотою твоею, благость Его злостію твоею, кротость и долготерпѣніе Его гнѣвомъ и злобою твоею, любовь Его ненавистію твоею оскорбляется. О, коль тяжко Бога великаго и благаго и человѣколюбиваго оскорбить! Кому? тебѣ и мнѣ, малымъ червячкамъ, землѣ и персти, оскорблять Создателя своего, искупителя своего, хранителя, питателя, словомъ, высочайшаго благодѣтеля, котораго благодѣяніями окружены мы, и толико ихъ на всякій день пріемлемъ, что безъ нихъ и минуты жить не можемъ; тяжко Его оскорблять, да и страшно! Ибо оскорбляющій дознаетъ на себѣ праведный судъ Его, когда не очувствуется и не покается. Богъ бо, яко благъ, долготерпѣливъ и многомилостивъ, ожидаетъ грѣшнича покаянія, и потому долготерпитъ ему; а когда не покается, то уже узнаетъ на себѣ гнѣвъ Его праведный. Стыдишися и боишися предъ царемъ твоимъ, предъ господиномъ твоимъ предъ отцемъ твоимъ, не токмо блудодѣйствовать и другое беззаконное дѣло дѣлать, но и праздное слово сказать и смѣятися: но предъ Богомъ \textit{Всевидящимъ} и Царемъ \textit{страшнымъ} не стыдишися и не боишися не токмо празднословить и смѣяться, но тяжкое беззаконіе дѣлать, блудодѣйствовать, прелюбодѣйствовать, похищать, грабить, воровать, лгать, клеветать, осуждать, злословить, ругать и прочія безстыдныя дѣла, дѣлать! О хрістіанине! покрый глаза твои, который творишь таковая; постыдись, который исповѣдуешь Бога вездѣсущаго, истиннаго, всемогущаго, праведнаго, пекущагося о тебѣ и о спасеніи твоемъ, и воздающаго всѣмъ по дѣламъ ихъ. Слѣпый и бѣдный той хрістіанинъ, который предъ людьми, подобными себѣ, хотя и высокими и почтенными, стыдится безчинія показывать, но предъ Богомъ, яко вездѣсущимъ и все назирающимъ, беззаконновать не стыдится и не боится. О! когда бы человѣкъ хотя мало страшныя Божія славы увидѣлъ: палъ бы яко мертвъ, и разсуждалъ бы, какъ онъ худо и безстыдно дѣлаетъ, что оскорбляетъ Того, отъ Котораго животъ свой и всякое добро имѣетъ, какое ни имѣетъ; и неутѣшно бы плакалъ и рыдалъ. Написано о нечестивыхъ: \textit{не предложиша Бога предъ собою}\footnote{Пс.~53,~5.}, и потому не стыдятся беззаконновать предъ Нимъ. Хрістіанине! ты предложи во умѣ своемъ Бога предъ Собою, и вѣруй и помни, что гдѣ ты ни находишися, и что ни дѣлаеши, и что ни говориши, Богъ тебѣ присутствуетъ; и смотри на Него умнымъ окомъ всегда, якоже пророкъ о себѣ глаголетъ: \textit{предзрѣхъ Господа предо мною выну}\footnote{15,~8.}. Тако видя Его и смотря на Него, падай предъ Нимъ со смиреніемъ, кайся и жалѣй за преждебывшіе твои грѣхи, за безстыдныя дѣла и слова, и проси прощенія, повторяя часто: \textit{согрѣшихъ Господи, помилуй мя!} А впредь стыдись и бойся предъ Нимъ безчинствовать и святый законъ Его разорять, да не и здѣ дознавши на себѣ мстительную руку Его, и на всемірномъ позорищи постыдишися. \textit{Призри, услыши мя, Господи Боже мой! просвѣти очи мои, да не когда усну въ смерть, да не когда речетъ врагъ мой: укрѣпихся на него}\footnote{12,~4--5.}. \textit{Открый очи мои, и уразумѣю чудеса отъ закона Твоего. Пришлецъ азъ есмь на земли: не скрый отъ мене заповѣди Твоя! Возлюби душа моя возжелати судьбы Твоя на всякое время}\footnote{118,~18--20.}. \textit{Господи! искусилъ мя еси, и позналъ мя еси. Ты позналъ еси сѣданіе мое и востаніе мое; Ты разумѣлъ еси помышленія моя издалеча; стезю мою и уже мое Ты еси изслѣдовалъ, и вся пути моя провидѣлъ еси}, и проч.\footnote{138,~1--3.}

\section{55. Нигдѣ я отъ тебе уйти не могу.}

Бываетъ, что человѣкъ человѣку, ради нѣкоторыхъ причинъ, говоритъ: \textit{нигдѣ я отъ тебе уйти не могу}. Слышимъ слово сіе въ людехъ; но оно, какъ разсудить, всякому человѣку, какой бы онъ ни былъ, неправильно приписуется. Ибо не токмо отъ простаго человѣка, но и отъ самыхъ царей, хотя и долгія руки имѣютъ, уйти и сокрыться возможно, какіе сіе видимъ. Почему слово сіе: \textit{нигдѣ я отъ тебе уйти не могу}, единому Богу приличествуетъ говорить. Отъ Него мы уйти нигдѣ не можемъ, куда ни побѣжимъ; и нигдѣ сокрыться не можемъ, гдѣ ни вздумаемъ скрываться. Вездѣ Онъ насъ предваряетъ, куда ни хощемъ бѣжать; и тамо присутствуетъ, гдѣ ни хощемъ скрытися. Въ домѣ твоемъ имѣешися? тутъ Богъ. Путемъ идеши? не оставляетъ тебе Богъ. На квартиру пришелъ? тутъ Богъ. Во градѣ или селѣ пребываеши? тамо присутствуетъ Богъ. Въ пустыню, или въ конецъ земли побѣжиши? тамо прежде тебе Богъ. Въ землѣ или глубинѣ морской хощешь скрытися? и тамо Онъ съ тобою. Думаешь: не скрыюся ли во тьмѣ и нощи отъ Него? нѣтъ, ненадежное сіе убѣжище! У насъ день и нощь, свѣтъ и тьма: но у Бога нѣтъ тьмы, нѣтъ нощи, но все свѣтъ и день. Очи Божіи несравненно \textit{свѣтлѣйшіи суть солнца, презирающіи вся пути человѣческія, и разсмотряющіи въ тайныхъ мѣстѣхъ}\footnote{Сир.~23,~28.}. Узналъ сіе Псаломникъ, что отъ Бога нигдѣ уйти и сокрыться не возможно, и воззвалъ къ Нему: \textit{камо пойду отъ Духа Твоего, и отъ лица Твоего камо бѣжу? Аще взыду на небо, Ты тамо еси; аще сниду во адъ, тамо еси, аще возму крилѣ мои рано, и вселюся въ послѣднихъ моря, и тамо бо рука Твоя наставитъ мя, и удержитъ мя десница Твоя. И рѣхъ: еда тьма поперетъ мя, и нощь просвѣщеніе въ сладости моей? Яко тьма не помрачится отъ Тебе, и нощь яко день просвѣтится: яко тьма ея, тако и свѣтъ ея}\footnote{Пс.~138,~7--12.}. Бѣжалъ было нѣкогда отъ лица Господня Іона пророкъ: но десницею Божіей удержанъ, и увидѣлъ себя во чревѣ китовѣ; и оттуду тоюжде десницею Божіею изверженъ на землю\footnote{Іоны 1,~3; 2,~1,~11.}. Нигдѣ убо отъ Бога уйтить и скрыться не возможно. О бѣдный грѣшникъ! куда ты отъ Того убѣжишь, Который вездѣ есть? и гдѣ ты отъ Того скрыешися, Который все видитъ? Знаеши ли, куда бѣжать? Отъ правды Его къ благости Его, и отъ суда его къ милости Его бѣжи. \textit{Яко бо величество Его, тако и милость Его}. И сокрывайся вѣрою въ пресвятыхъ язвахъ Хрістовыхъ. Се есть градъ убѣжища, къ которому хрістіане прибѣгаютъ и безопасно сокрываются отъ гнѣва Божія! О Іисусе! \textit{сынове человѣчестіи въ кровѣ крилу Твоею надѣятися имутъ}\footnote{Пс.~35,~8.}.

Якоже птенцы подъ крилами кокоша, тако грѣшники подъ кровомъ благости и человѣколюбія Твоего сокрываются. \textit{Господи! прибѣжище былъ еси намъ въ родъ и родъ}\footnote{89,~2.}. \textit{Пріидите ко Мнѣ еси труждающіися и обремененніи, и Азъ упокою вы}\footnote{Мѳ.~11,~28.}.

\section{56. Самъ здѣ.}

Бываетъ, что, когда господинъ близъ имѣется, и раби его не видятъ его или не знаютъ, что близъ есть, рабъ, знающій о немъ, другихъ рабовъ остерегая, чтобы шумомъ своимъ или инымъ какимъ безчиніемъ не прогнѣвали его, и тако бы наказанію не подпали, говоритъ имъ тихо: тише, или молчите; \textit{самъ здѣ}. Тако раби Божіи, пророки и апостоли, Духомъ Божіимъ просвѣщенніи, остерегаютъ насъ, о хрістіане! и глаголютъ намъ: слышите; \textit{Самъ здѣ}, Который есть вездѣ и вся исполняетъ. \textit{Кротость ваша разумна да будетъ всѣмъ человѣкомъ. Господь близъ}\footnote{Филип.~4,~5.}. Хрістіане! берегитеся: \textit{Господь близъ}. Берегитеся грѣха въ словѣ и въ дѣлѣ и помышленіи: \textit{Господь близъ}, Который грѣха ненавидитъ и тѣмъ оскорбляется и прогнѣвляется, и согрѣшающихъ предаетъ наказанію. Вы, которыи собралися въ храмъ Господень на молитву и славословіе Божіе, слышите: \textit{Господь близъ}; Самъ здѣ Господь, предъ Которымъ стоите молящеся, и смотритъ, како предъ Нимъ стоите вы, какимъ духомъ и сердцемъ Ему молитеся, какъ поете и славословите Его. Преклоняя колѣна и главу Ему, преклоняете ли и сердце ваше? Прося милости отъ Него, дѣлаете ли и сами милость ближнимъ вашимъ? Прося отпущенія грѣховъ, отпущаете ли и сами человѣкомъ согрѣшенія ихъ? Стоя предъ Богомъ тѣломъ, стоите ли предъ Нимъ духомъ и сердцемъ вашимъ? Моляся Ему языкомъ, и глаголя: \textit{Господи помилуй}, молитеся ли умомъ и сердцемъ вашимъ? Поете и славословите Его устами, поете ли и сердцами? Внимайте сему, возлюбленніи: \textit{Господь близъ}; Самъ Господь здѣ, Которому предстоите, Который смотритъ на васъ и слышитъ, како молитеся и поете Его. Берегитеся, любезніи, да не молитва и славословіе ваше въ грѣхъ вамъ будетъ. Предъ царемъ "--- человѣкомъ стоимъ со страхомъ и смиреніемъ, когда чего просимъ у него, и всѣмъ умомъ внимаемъ ему: кольми паче должно дѣлать тое, когда предъ Богомъ стоимъ, и молимся Ему. Близъ \textit{Господь всѣмъ призывающимъ Его, всѣмъ призывающимъ Его во истинѣ: волю боящихся Его сотворитъ, и молитву ихъ услышитъ, и спасетъ я}\footnote{Пс.~144,~18--19.}. Вы, которыи засѣдаете на мѣстѣ суда, и судите и разсуждаете между правымъ и виноватымъ, между обижденнымъ и обидѣвшимъ, слышите: \textit{Господь близъ}; Самъ здѣ, Судія судей и \textit{судяй всей земли} праведно, Царь царствующихъ и Господь господствующихъ; Самъ Онъ здѣ невидимо присутствуетъ, и видитъ ваши дѣла, начинанія и помышленія, и слышитъ ваши слова, и смотритъ на васъ, како вы судите людей Его, како и кого оправдаете, и кого обвиняете, и за что. Берегитеся убо, возлюбленніи! \textit{Господь близъ. Богъ ста въ сонмѣ боговъ, посреди же боги разсудитъ. Доколѣ судите неправду, и лица грѣшниковъ пріемлете? Судите сиру и убогу, смирена и нища оправдайте: измите нища и убога изъ руки грѣшничи избавити его}\footnote{81,~34.}. Поступайте убо, возлюбленніи, въ судѣ вашемъ, какъ Богъ повелѣлъ вамъ; и клятва ваша, которую вы учинили предъ Нимъ, обдолжаетъ васъ, и дѣло правосудія требуетъ, на которое вы позваны, да не неправдою вашею раздраженный Господь изліетъ на васъ праведный гнѣвъ Свой. Вы, которыи въ мірѣ семъ называетеся господами и имѣете подъ собою рабовъ и крестьянъ, слышите: \textit{Господь близъ}; Самъ Господь здѣ и видитъ ваши дѣла, и смотритъ, како вы съ людьми Его, вамъ подчиненными, поступаете. Онъ не яко человѣкъ, лицепріятія не имѣетъ, но у Него равны суть вси господа и раби ихъ. Берегитесь убо, возлюбленніи, да не разгнѣваете Господа вашего, Который самъ здѣ присутствуетъ. Вы, которыи называетеся родителями, имѣете у себе дѣтей, слышите: \textit{Господь близъ}; Самъ Господь здѣ, Который видитъ ваши дѣла и слышитъ ваши слова. Како съ дѣтьми вашими поступаете? како ихъ учите и наставляете? не учите ли вмѣсто истины суетѣ, и вмѣсто страха безстрашію? и вмѣсто наставленія и полезнаго ученія, не подаете ли яда соблазновъ юнымъ сердцамъ? Блюдитеся, возлюбленніи, како опасно ходите предъ Богомъ, смотрящимъ на васъ, и предъ юными дѣтьми вашими, которыя примѣчаютъ, что вы дѣлаете и что вы говорите, и сами то учатся дѣлать и говорить. Вы, которые проповѣдуете слово Божіе, слышите: \textit{Господь близъ}; Самъ Господь здѣ, и видитъ, какими вы руками и какимъ языкомъ священнаго его сокровища касаетеся; къ славѣ ли Его и пользѣ ближняго великій тотъ даръ Божій употребляете, или къ своей похвалѣ; Самъ Господь здѣ, \textit{испытуяй сердца и утробы. Грѣшнику же рече Богъ: вскую ты повѣдаеши оправданія Моя, и воспріемлеши завѣтъ Мой усты твоими? Ты же возненавидѣлъ еси наказаніе, и отверглъ еси словеса Моя вспять. Аще видѣлъ еси татя, теклъ еси съ нимъ; и съ прелюбодѣемъ участіе твое полагалъ еси. Уста твоя умножиша злобу, и языкъ твой сплеташе льщенія; сѣдя на брата твоего клеветалъ еси, и на сына матере твоея полагалъ еси соблазнъ}\footnote{Пс.~49,~16--20.}. Вы, которыи купуете и продаете, слышите: \textit{Господь близъ}; Самъ Господь здѣ, Который видитъ ваши дѣла, смотритъ, како купуете и како продаете. Не продаете ли убо гнилую вещь за здоровую и худую за добрую, и дешевую за дорогую? не лжете ли? не обманываете ли въ товарахъ вашихъ ближнихъ вашихъ? и, что горше того, не кленетеся ли именемъ Божіимъ во лжу, сквернаго ради прибытка? Устыдитеся и убойтеся предъ очами Божіими лгать, и страшное имя Его во лжи призывать, и людей Его обманывать. "--- Вы, которыи за трапезою сѣдите и вкушаете дарованія Божія, слышите: \textit{Господь близъ}; Самъ Господь здѣ, и смотритъ на васъ. Како убо вкушаете Божіяго добра? съ молитвою ли и благодареніемъ? Вкушаете Божіяго благодѣянія: но памятуете ли благодѣтеля? Все Божіе добро, а не наше. \textit{Господня бо земля, и исполненіе ея}\footnote{Пс.~23,~1.}. "--- Вы, которыи, собравшеся во едино, разговариваете, слышите: \textit{Господь близъ}; Самъ Господь здѣ, и смотритъ на васъ, и слышитъ, что и о комъ разговариваете. Берегитеся убо, да не разговоръ вашъ въ грѣхъ вамъ будетъ. Опасно говоримъ предъ царемъ присутствующимъ: кольми паче предъ Богомъ, Который здѣ и тамо, и на всякомъ мѣстѣ присутствуетъ. Блюдите убо, како и что и о комъ разговариваете. "--- Вы, которыи хощете душу и тѣло свое осквернить нечистотою, слышите: \textit{Господь близъ}; Самъ Господь здѣ, Который видитъ беззаконное зачинаніе ваше. Устыдитеся и убойтеся беззаконновать предъ очами Божіими. Вы, которыи содержите гнѣвъ на ближняго вашего и мыслите ему отмстить и повредить его, слышите: Господь близъ, Самъ Господь здѣ, и видитъ злое умышленіе ваше. Устыдитеся убо и убойтеся присутствующаго Господа, и оставьте злое начинаніе ваше. "--- Вы, которыи думаете чужое добро украсть и похитить, слышите: \textit{Господь близъ}, и видитъ ваше зачатое дѣло. Устыдитеся убо и ужаснитеся предъ очами Божіими совершать беззаконное дѣло ваше. Вы, которыи, или отъ злобы своей, или отъ злой привычки, оклеветаете и осуждаете ближняго своего, слышите: \textit{Господь близъ}, и слышитъ вашу клевету, которою разсѣваете соблазны людей, и судъ вашъ, которымъ подобныхъ себѣ грѣшниковъ судите. Устыдитеся убо и убойтеся Господа, предъ Которымъ оклеветаете и осуждаете людей Его, и удерживайте языкъ свой отъ беззаконнаго того дѣла. \textit{Что же видиши сучецъ, иже во оцѣ брата твоего, бервна же, еже есть во оцѣ твоемъ, не чуеши? Или како речеши брату твоему: остави, да изму сучецъ изъ очесе твоего; и се бервно во оцѣ твоемъ? Лицемѣре! изми первѣе бервно изъ очесе твоего, и тогда узриши изъяти сучецъ изъ очесе брата твоего}\footnote{Мѳ.~7,~3--5.}. \textit{Не оклеветайте другъ друга, братіе; оклеветаяй бо брата, или осуждаяй брата своего, оклеветаетъ законъ, и осуждаетъ законъ}\footnote{Іак.~4,~11.}. "--- Всякъ хрістіанинъ слыши: \textit{Господь близъ}. Гдѣ ты ни имѣешися, что ты ни дѣлаеши, что ни говориши и мыслиши, Онъ присутствуетъ тебѣ, и всякое дѣло твое видитъ, и слово слышитъ. Устыдись убо, возлюбленне, и убойся предъ Господемъ твоимъ согрѣшать, якоже стыдишися и боишися предъ царемъ земнымъ и иною властію законъ нарушать. Не уподобляйся, любезный, нечестивымъ, которыи не \textit{предложиша Бога предъ собою}\footnote{Пс.~53,~5.}, но уподобляйся святому оному мужу, который написалъ о себѣ: \textit{предзрѣхъ Господа предо мною выну}\footnote{Пс.~15,~8.}. Зри убо и ты предъ собою Господа твоего вездѣ и всегда. И аще дѣлаеши что, или говориши, или мыслиши, или яси и піеши, или дома сѣдиши, или путемъ идеши, или продаеши, или купуеши, или молчиши, или съ кѣмъ бесѣдуеши, или молишися, или поеши Господа твоего: да обращается Онъ всегда предъ умными очами твоими, хотя и не видиши Его; почитай Его сердцемъ, устами и дѣлами твоими, якоже почитаешь присутствующаго царя. Аще же въ чемъ проступишися и погрѣшиши предъ Нимъ, яко человѣкъ: не медли, но тотчасъ падай предъ Нимъ со смиреніемъ, и проси прощенія: \textit{согрѣшихъ, Господи, помилуй мя!} и отпустится тебѣ согрѣшеніе твое: милостивъ бо Онъ и человѣколюбивъ есть, и знаетъ нашу немощь. Слыши и ты, душе моя, и стыдися и бойся вездѣ и всегда Господа твоего! Слыши и ты, о богобоящаяся и благочестивая душа, нелицемѣрно Господа твоего почитающая, слыши: \textit{Господь близъ}, Который говоритъ тебѣ: \textit{Мой еси ты. И аще преходиши сквозѣ воду, съ тобою есмь, и рѣки не покрыютъ тебе; и аще сквозѣ огнь прейдеши, не сожжешися, и пламень не опалитъ тебе}\footnote{Ис.~43,~1--2.}. О возлюбленне! \textit{не даждь во смятеніе ноги твоея, ниже воздремлетъ храняй тя. Се не воздремлетъ, ниже уснетъ храняй Израиля! Господь сохранитъ тя. Господь покровъ твой на руку десную твою. Во дни солнце не ожжетъ тебе, ниже луна нощію. Господь сохранитъ тя отъ всякаго зла, сохранитъ душу твою Господь. Господь сохранитъ вхожденіе твое и исхожденіе твое, отъ нынѣ и до вѣка}\footnote{Пс.~120,~3--8.}.

\section{57. Царь, входящій во градъ или домъ.}

Бываетъ, что когда царь во градъ или домъ какій хощетъ внити, слуги и посланники его говорятъ прочіимъ людямъ, во градѣ или домѣ живущимъ: отверзите врата, царь идетъ: и тако входитъ царь во градъ или домъ, и пріемлется съ честію отъ гражданъ или домашнихъ. Тако Хрістосъ Господь Царь небесный, пришедши на землю и содѣлавши спасеніе вѣчное посредѣ земли, послалъ во вся страны и грады всея земли святыхъ Своихъ учениковъ и апостоловъ возвѣстити пришествіе Свое и спасительный Свой къ нимъ входъ. \textit{Шедше въ міръ весь, проповѣдите Евангеліе всей твари}\footnote{Марк.~16,~15.}. Сіи посланники Божіи и вѣрніи раби Хрістовы, \textit{изшедше, проповѣдаху всюду, Господу поспѣшствующу, и слово утверждающу послѣдствующими знаменьми}\footnote{ст. 20.}. Они во всякомъ градѣ и странѣ живущимъ людямъ говорили: отверзите врата, и внидетъ къ вамъ Царь небесный, Царь вашъ, но вами незнаемый. Онъ хощетъ въ васъ царствовати, и въ вѣчное Свое царствіе васъ привести: отверзите убо врата, и внидетъ къ вамъ. \textit{Возмите врата князи ваша, и возмитеся врата вѣчная: и внидетъ Царь славы}\footnote{Пс.~23,~7.}. И тако во грады и страны языческія вошелъ Царь славы съ Своею благодатію и спасеніемъ вѣчнымъ. Не приняли Его, яко истиннаго своего Царя, Іудеи, къ которымъ и посланъ былъ, по писанному: \textit{во Своя пріиде, и Свои Его не пріяша}\footnote{Іоан.~1,~11.}. Сего ради и отнялось отъ нихъ царствіе Божіе. Языки, не знавшіе и не ожидавшіе Его, съ радостію и вѣрою отворили Ему грады своя, и приняли Его, и поклонилися Ему, и приняли на себе сладкое иго Его. Сего ради и царствіе Божіе обратилося къ нимъ: ибо гдѣ Хрістосъ Господь и Царь славы, тамо и царствіе Божіе. Богу благодареніе, что и въ нашу страну, въ наше отечество и грады Царь славы и мира, Іисусъ Хрістосъ вошелъ, и съ Нимъ царствіе Божіе! Но видимъ, что отъ многихъ странъ и градовъ за неблагодарность и презрѣніе слова Его святаго паки отшелъ, а съ Нимъ и царствіе Божіе отъ нихъ отнялось. А понеже видимъ, что и въ нашемъ уже отечествѣ всякая умножилась неправда, любовь уже совсѣмъ почти изсякла, безмѣрныя роскоши умножились, плотское и безстрашное житіе почти вездѣ усматривается; и тако слово Божіе въ крайнемъ находится презрѣніи, и вси почти прихотямъ, и плотскимъ и мірскимъ, а не устамъ Божіимъ внимаютъ; вси почти плоти и міру, а не Богу угождаютъ; временныхъ и мірскихъ, а не вѣчныхъ благъ ищутъ: опасно, хрістіанине, и весьма опасно, чтобы и отъ нашей страны и градовъ не отшелъ Хрістосъ Господь съ Своимъ царствіемъ. Не ино бо что оттуду послѣдуетъ, когда люди мнятся знать Его, но не почитаютъ Его, но вмѣсто Его почитается мамона неправды, Евангеліе Его святое пренебрегается, законъ святый попирается, міръ и суета, а не царствіе Его отъ всѣхъ ищется: вси таковыи люди, послѣдуя Іудеямъ, выгоняютъ отъ себе Хріста и тако убѣждается отъ нихъ отъити; и что Іудеямъ, тое и симъ людямъ говоритъ: \textit{се оставляется вамъ домъ вашъ пустъ}\footnote{Мѳ.~23,~38.}. Страшно слово сіе, и весьма страшно! что бо безъ Хріста, какъ не явная погибель? Обратимся убо, хрістіанине, ко Хрісту всѣмъ сердцемъ, и покаемся и восплачемся предъ Нимъ, да не оставитъ насъ, да не отнимется отъ насъ царствіе Божіе, якоже отнялося отъ ожесточенныхъ Іудеевъ. \textit{Не остави мене, Господи Боже мой, не отступи отъ мене! Вонми въ помощь мою, Господи спасенія моего}\footnote{Пс.~37,~22--23.}.

\section{58. Горе.}

Горе мѣсту тому, въ которомъ нѣтъ свѣта: тамо люди блудятъ, какъ слѣпіи; тамо не познается вредное отъ полезнаго, и частое то преткновеніе, то паденіе бываетъ; словомъ: всякая бѣда, напасть и вредъ тамо приключается. Наипаче горе тѣмъ душамъ, въ которыхъ нѣтъ Хріста, истиннаго свѣта! Нѣтъ тамо ничего кромѣ единыя тьмы, сѣни смертной, бѣдствія, окаянства и погибели! \textit{Азъ есмь Свѣтъ міру: ходяй по Мнѣ, не имать ходити во тьмѣ, но имать свѣтъ животный}\footnote{Іоан.~8,~12.}. Но \textit{востани спяй, и воскресни отъ мертвыхъ, и освѣтитъ тя Хрістосъ}\footnote{Еф.~5,~14.}. Горе дому тому, въ которомъ нѣтъ благоразумнаго хозяина; всякое бо бываетъ тамо нестроеніе: наипаче горе душамъ, въ которыхъ дому владыка Хрістосъ не обитаетъ! Тамо всякое истинное бываетъ нестроеніе и смятеніе; тамо жилище всякихъ духовъ злыхъ. Горе кораблю тому, въ которомъ нѣтъ добраго кормчія; ибо близъ потопленія есть: горе наипаче душѣ, душѣ плавающей по морю міра сего, въ которой не имѣется премудраго Правителя Хріста; ибо близъ потопленія есть. \textit{Аще кто Духа Хрістова не имать, сей нѣсть Еговъ}\footnote{Рим.~8,~9.}. Горе людемъ тѣмъ, которыи хлѣба и воды не имѣютъ; ибо отъ глада и жажды истаеваютъ и умираютъ: наипаче горе тѣмъ душамъ, которыи лишаются хлѣба животнаго, Іисуса Хріста; ибо отъ глада истаеваютъ и имутъ умрети. \textit{Азъ есмь хлѣбъ животный}, и проч. \textit{хлѣбъ бо Божій есть сходяй съ небесе, и даяй животъ міру}\footnote{Іоан.~6,~33 и 35.}. \textit{Иже піетъ отъ воды, юже Азъ дамъ ему, не вжаждется во вѣки: но вода, юже Азъ дамъ ему, будетъ въ немъ источникъ воды, текущія въ животъ вѣчный}\footnote{Іоан.~4,~14.}. Хлѣбомъ насыщается тѣло, и водою напаяется; безъ хлѣба и воды изнемогаетъ и умираетъ. Души истинное брашно и питіе есть Хрістосъ. Симъ брашномъ и питіемъ она оживляется и живетъ: безъ сего брашна и питія изнемогаетъ и умираетъ, и проч. Аще кто Духа Хрістова не имѣетъ, сей и Хріста не имѣетъ: въ комъ Хрістосъ живетъ, въ томъ показуется и житіе Хрістово, въ томъ является смиреніе Его, любовь, терпѣніе и кротость. Обитаніе бо Хрістово въ человѣкѣ праздно быть не можетъ; неотмѣнно покажется чрезъ дѣла такія, какихъ Хрістосъ хощетъ, какъ возсіявшее солнце показуется чрезъ лучи, или какъ согрѣтая отъ огня пещь чрезъ теплоту, или какъ доброе древо черезъ сладкіе плоды. Въ комъ Хрістосъ живетъ, \textit{Свѣтъ животный}, въ томъ и свѣтъ Его является, и тьма прогоняется, и темныя дѣла не являются. Въ комъ Хрістосъ живетъ, \textit{Огнь чистительный, просвѣщающій}, въ томъ теплота любве и милосердія чувствуется. Въ комъ Хрістосъ живетъ, \textit{Агнецъ Божій} смиренный и кротчайшій, въ томъ смиреніе Его, кротость и терпѣніе является. Въ комъ Хрістосъ живетъ, \textit{Древо животное} и мысленное, въ томъ и сладкіе плоды Его являются: какое бо древо, таковыи и плоды его. Кто Хрісту соединенъ, какъ удъ главѣ, или какъ розга лозѣ; тотъ подобный Ему и плодъ творитъ. Тако обитаніе Хрістово въ человѣцѣ праздно быть не можетъ; но неотмѣнно и чрезъ внутреннія духовныя движенія чувствуется, и чрезъ внѣшнія подобныя дѣла является. Обратимся убо, хрістіанине, всѣмъ сердцемъ ко Хрісту, и воздохнемъ и восплачемся предъ Нимъ, и сокрушеннымъ сердцемъ покаемся, и помолимся Ему усердно, да и къ намъ грѣшнымъ пріидетъ, и \textit{вѣрою вселится въ сердца наша}, и возобладаетъ нами; и дотолѣ просити и искати, и толкати въ двери милосердія Его не престанемъ, \textit{дондеже вообразится Хрістосъ въ насъ}\footnote{Гал.~4,~19.}.

\section{59. Древо доброе.}

Что доброе древо есть, тое хрістіанинъ добрый есть, духовно въ себѣ разсуждаемый. Древо доброе отъ добраго сѣмене бываетъ: добрый хрістіанинъ отъ Божія слова, небеснаго и добраго сѣмене раждается, по Писанію: \textit{порождени не отъ сѣмене истлѣнна, но неистлѣнна, словомъ живаго Бога и пребывающа во вѣки}\footnote{1~Петр. 1~,25.}. Древо доброе познается отъ добрыхъ его плодовъ: тако добрый хрістіанинъ познается отъ добрыхъ дѣлъ. Древа добраго добрые плоды извнутрь его раждаются: тако добрый хрістіанинъ извнутрь отъ сердца и нелицемѣрно добрыя творитъ дѣла. Древо доброе показуютъ добрые плоды, а не дѣлаютъ добрымъ: тако хрістіанина добраго показуютъ добрыя дѣла, а не дѣлаютъ добрымъ. Древу надобно прежде добрымъ быть, и тогда добрые плоды творить; древо бо злое не можетъ добрыхъ плодовъ творити\footnote{Мѳ.~7,~17--18; Лук.~6,~43.}: тако хрістіанину надобно прежде сдѣлаться добрымъ, и тогда добрыя дѣла творити; злый бо не можетъ добрѣ творити. Древо доброе приноситъ плоды хозяину и прочимъ людямъ: тако добрый хрістіанинъ творитъ добрыя дѣла для славы имене Божія и пользы ближнихъ своихъ. Добрые плоды добраго древа хотя и висятъ и показуются быть на древѣ, однакожъ не древу самому, но сѣмени доброму, отъ котораго древо произрасло, приписуются: тако и добрыя дѣла, хотя и творитъ ихъ добрый хрістіанинъ, благому и человѣколюбивому Богу должно восписывать, Который доброе сѣмя на сердцѣ человѣческомъ посѣялъ, и добрымъ его сдѣлалъ, и помогаетъ ему и укрѣпляетъ его добро творити; безъ Бога бо и добрымъ быть и добро творить невозможно, по словеси Господню: \textit{безъ Мене не можете творити ничесоже}\footnote{Іоан.~5,~15.}. Древо доброе чѣмъ болѣе отребляется и очищается отъ дѣлателя, тѣмъ болѣе плодовъ приноситъ: тако добрый хрістіанинъ, чимъ болѣе наказуется и исправляется отъ небеснаго Дѣлателя Бога, тѣмъ лучшій бываетъ, и болѣе добрыхъ дѣлъ творитъ\footnote{Іоан.~15,~1 и 2.}. Древо доброе всѣмъ безъ разбора, домашнимъ и чужимъ, плоды своя сообщаетъ: тако добрый хрістіанинъ всѣмъ, другамъ и врагамъ, знаемымъ и незнаемымъ, добро творитъ, подражая въ томъ небесному Отцу, Который \textit{солнце Свое сіяетъ на злыя и благія, и дождитъ на праведныя и на неправедныя}\footnote{Мѳ.~5,~45.}. Домъ его всякому странному отверстъ хлѣбомъ его всякъ алчущій питается; нагаго не отпуститъ не одѣвши, когда имѣетъ чимъ одѣть; требующему помощи не откажетъ, когда можетъ; со страждущимъ и плачущимъ прослезится и восплачетъ; къ скорбящему съ утѣшеніемъ первый, и тщится, какъ можетъ, уязвленное сердце уврачевать: съ радующимся радуется о его благополучіи, аки о своемъ; всякому милости у Бога ищетъ и проситъ, какъ и себѣ; всеусердно радъ бы былъ, дабы всѣхъ видѣть спасенныхъ. Аще бы возможно было, и имѣлъ бы чимъ, всѣмъ бы бѣдствующимъ помоглъ. Видя брата согрѣшающаго, не судитъ ему, но духомъ милосердія соболѣзнуетъ. Сія суть и прочія \textit{добраго древа плоды!} Хрістіанине! сдѣлайся добрымъ, и тогда будеши добрыя дѣла творить. А чтобы добрымъ сдѣлаться, надобно учинить: 1)~Обратиться всѣмъ сердцемъ къ Богу, и быть въ истинномъ покаяніи: безъ того бо добрымъ человѣкъ быть не можетъ. 2)~Читать со вниманіемъ и разсуждать съ прилѣжаніемъ Божіе слово и прочія хрістіанскія книги. \textit{Всяко бо писаніе богодухновенно и полезно есть ко ученію, ко обличенію, ко исправленію, къ наказанію, еже въ правдѣ: да совершенъ будетъ Божій человѣкъ, на всякое дѣло благое уготованъ}\footnote{2~Тим.~3,~6 и 7.}. 3)~Тщаться познавать растлѣніе и окаянство сердца своего. Отъ познанія бо себе самого послѣдуетъ смиреніе и тщаніе о исправленіи себе самого: ибо надобно прежде познать немощь немощному, и тогда исцѣленія искать. Начало здравія "--- познаніе немощи, и начало исправленія "--- познаніе бѣдствія и окаянства своего. Кто бо, видя ядъ въ себѣ смертоносный крыющійся, не со усердіемъ пожелаетъ отъ того свободиться? И кто, видя погибель свою, не будетъ тщаться отъ тоя избыть? Познать убо себе, начало спасенія есть. Познай убо себе, человѣче, и нелицемѣрно признай свою бѣдность и окаянство предъ Богомъ; и отдай себе въ руцѣ Хрістовы, какъ немощный отдаетъ себе лѣкарю: и тогда Хрістосъ исцѣлитъ тебе! 4)~Отъ всякаго грѣха берещися, и нудить себе ко всякому добру, хотя и не хощетъ сердце. Таковое тщаніе и усердіе къ добру видя Господь, будетъ дѣлать милость со тщащимся, и день отъ дня въ лучшее его исправлять. Немощный всего бережется, что лѣкарь ему ни запрещаетъ, "--- хотя желаніе его и порѣваетъ его къ тому, "--- когда хощетъ исцѣлиться: тако и хрістіанинъ всего уклоняться долженъ, что небесный Врачъ Хрістосъ запрещаетъ, хотя и не хощетъ сердце; иначе исцѣлиться не можетъ. 5)~Усердно молиться и просить исправленія у Самого Хріста, Который прокаженныхъ очистилъ, слѣпыхъ просвѣтилъ, немощныхъ исцѣлилъ. Той и души наша безсмертныя исцѣлитъ, аще усердно Ему будемъ молитися, и сами будемъ тщаніе къ тому прилагать, по неложному Его обѣщанію: \textit{просите, и дастся вамъ, ищите, и обрящете; толцыте, и отверзется вамъ. Всякъ бо просяй пріемлетъ, и ищай обрѣтаетъ, и толкущему отверзется}\footnote{Мѳ.~7,~7 и 8.}.

\textit{Обрати мя, Господи, и обращуся. Исцѣли мя, Господи, и исцѣлюся}\footnote{Іер.~31,~18; 17,~14.}, безъ Котораго обращеніе и исцѣленіе не бываетъ.

\section{60. Тина или грязь на днѣ ключа.}

Видимъ, что хотя и чистая вода имѣется въ ключѣ, однакожъ бываетъ на днѣ его тина или грязь: тако во глубинѣ сердца человѣческаго имѣется всякая нечистота; какъ смердящая тина и зловоніе, тамо кроется гордость и высокоуміе, тамо сребролюбіе, тамо гнѣвъ, злоба и зависть, тамо скотская нечистота и всякая мерзость. Въ ключѣ познается на днѣ его нечистота имѣющаяся тогда, когда жезломъ или инымъ какимъ орудіемъ во дно его ударяется; тогда отъ тины или грязи, на днѣ его лежащія, вся вода въ ключѣ возмущается и бываетъ мутна: тако нечистота страстей и скотскій злый нравъ, во глубинѣ сердца человѣческаго лежащій, во время искушенія и соблазновъ познается. Кто бы позналъ, что на днѣ ключа имѣется тина или грязь, аще бы она оттуду, въ случаѣ ударенія, не исходила и себе не оказывала? Тако откуду бы мы знали, что въ глубинѣ сердца человѣческаго толикая крыется мерзость и нечистота, аще бы она оттуду не выходила и себе внѣшними дѣлами не оказывала? Видимъ, какъ человѣкъ въ случаѣ ярится, кричитъ, хулитъ и прочія безчинныя дѣйствія показуетъ: дѣйствуетъ то въ немъ и къ такимъ безчиніямъ возбуждаетъ его гнѣвъ, въ сердцѣ его, какъ ядъ, сокровенный и въ случаѣ искушенія и обиды оказуемый. Видимъ, сколько человѣкъ собираетъ ради убогаго и смертнаго тѣла своего, которое малымъ хлѣба укрухомъ и какимъ нибудь одѣяніемъ довольствуется; сколько, говорю, собираетъ, хотя и знаетъ, что все при смерти оставитъ: дѣйствуетъ тое въ немъ сребролюбіе и лютая богатства похоть, въ сердцѣ гнѣздящаяся. Видимъ, какъ человѣкъ возносится, какихъ способовъ не изобрѣтаетъ, чтобы его люди знали, хвалили, славили и почитали; какъ подобныхъ себѣ людей презираетъ и за подножіе имѣетъ; какъ судитъ и пересуждаетъ дѣла ихъ, хотя и самъ такойжде; какъ вездѣ первенствовать и надъ другими начальствовать тщится и ищетъ, и проч.: дѣйствуетъ тое въ немъ гордость, гнѣздящаяся въ сердцѣ его. Блудная нечистота, внутрь человѣка крыющаяся, чрезъ какія, коль скаредныя, коль мерзкія, коль смрадныя и безстудныя дѣла оказываетъ себе: срамно есть и глаголати! Приникни, человѣче, во глубину сердца твоего, и разсматривай и познавай, коль смрадная страстей тина въ немъ лежитъ! Какое видишь зло въ ближнемъ твоемъ, тое и въ сердцѣ твоемъ имѣется; и въ чемъ судишь и осуждаешь ближняго твоего, тое и въ тебѣ есть, хотя внѣ и не является. Божіе же око не токмо внѣшнее дѣло, но и глубину сердца видитъ. Многіи мнятъ о себѣ, что они смиренніи и кроткіи суть; но въ случаѣ искушенія иное показуется. Многіе называютъ себе грѣшными и многогрѣшными (безъ сумнѣнія, \textit{всякъ} человѣкъ \textit{грѣшникъ}, по Писанію)\footnote{1~Іоан.~1,~8.}, но отъ людей того не терпятъ. Кто истинно и нелицемѣрно и въ сердцѣ своемъ называетъ себе грѣшникомъ, тотъ всякое поносное слово удобно претерпитъ и знака гнѣва не покажетъ: яко смиренъ\footnote{Начиная со словъ: примѣчай убо, человѣче, и кончая: признаютъ свою немощь и отъ него просятъ исцѣленія, "--- писано собственною рукою святителя. См. ссылку въ предисл.}. Примѣчай убо, человѣче, и разсматривай и искушай, какое зло въ сердцѣ твоемъ сокровенно лежитъ, какая гордость, славолюбіе, самолюбіе, гнѣвъ, зависть, сребролюбіе, нечистота и прочая мерзость. Како отъ злато сѣмене добрый плодъ, и какъ отъ злато сердца доброе дѣло можетъ быть? Како злый можетъ добро творить? И како кто можетъ добрымъ быть, когда его Хрістова благодать не исправитъ? Исправляетъ Хрістосъ того, кто свое окаянство и бѣдность познаетъ и предъ Нимъ признаетъ, и помощи отъ Него и исправленія проситъ. Читаемъ въ Евангеліи, что тѣхъ Онъ исцѣлялъ, которые бѣдность свою предъ Нимъ признавали, и отъ Него просили исцѣленія: тако и нынѣ тыя души исцѣляетъ, которыя признаютъ свою немощь и отъ Него просятъ исцѣленія. Посматривай убо, человѣче, чаще въ сердце твое, да познаешь его: отъ того бо зависитъ начало исправленія. Чимъ чаще будешь въ него приникать и разсматривать, тѣмъ болѣе его будешь познавать; чимъ болѣе сердце будешь познавать, тѣмъ болѣе будешь познавать зло, въ немъ крыющееся; чимъ болѣе зло въ себѣ будешь познавать, тѣмъ болѣе будешь познавать бѣдность свою и окаянство. Познаніе бѣдности и окаянства убѣдитъ тебе смиряться, и искать помощи и избавленія у Хріста, Который все можетъ, и изъ злаго доброе сдѣлать. \textit{Смиреннымъ Богъ даетъ благодать}\footnote{Іак.~4,~6.}. \textit{Сердце чисто созижди во мнѣ, Боже, и духъ правъ обнови во утробѣ моей! Не отвержи мене отъ лица Твоего, и Духа Твоего Святаго не отъими отъ мене! Воздаждь ми радость спасенія Твоего, и духомъ Владычнымъ утверди мя}\footnote{Пс.~50,~12--14.}.

\section{61. Человѣкъ, впадшій въ разбойники и отъ нихъ уязвленный.}

Бываетъ, что человѣкъ впадаетъ въ руцѣ разбойническія, и отъ нихъ обнажается и сильно уязвляется; видимъ въ мірѣ зло сіе. Тако весь родъ человѣческій, по отступленіи отъ Бога и преступленіи святыя заповѣди Его, впалъ въ руки діавольскія, аки разбойническія, и отъ него обнажился святыя и боготканныя одежды, и несказанно уязвился, и лежалъ на пути міра сего уязвленный, аки единъ человѣкъ. Толь сильныя язвы его были, что ихъ никто не могъ исцѣлить. Посланъ былъ къ нему съ небесе Божій \textit{Законъ}: но той обличалъ его только, и язвы его ему показывалъ, а не исцѣлялъ; смертію ему грозилъ, а не помогалъ. Посланы были \textit{Пророки}: но и тіи ничего не успѣли. Лежалъ бѣдный человѣческій родъ неисцѣленный! лежалъ ураненный, уязвленный, умученный, обезчещенный, обруганный отъ разбойника діавола! лежалъ \textit{еле живъ}\footnote{Лук.~10,~30--32.}. Ахъ, любезнѣйшее Божіе созданіе, \textit{человѣкъ}! въ какое ты состояніе, въ какое бѣдствіе, въ какое безчестіе хитростію зміиною пришелъ! Гдѣ твоя краснѣйшая оная доброта, которою Создатель твой тебе удобрилъ? гдѣ твоя честь, которою въ началѣ тебе почтилъ? гдѣ образъ Божій и подобіе Божіе, по которому ты созданъ? гдѣ тое блаженство, къ которому ты сотворенъ? Позавидѣлъ лукавый змій блаженству нашему, и хитро запнулъ насъ, и неисцѣльно уязвилъ насъ, и лишилъ насъ нашего блаженства: и \textit{познали} мы, \textit{добро и зло} самымъ искусомъ; лишилися добра, и познали добро; впали во зло, и познали зло; восхотѣли чести Божескія, и лишилися Божіяго образа. \textit{И человѣкъ въ чести сый не разумѣ, приложися скотомъ несмысленнымъ, и уподобися имъ}\footnote{Пс.~48,~13.}! Язвы и раны, которыми насъ неисцѣльно уязвилъ врагъ нашъ, суть: гордость, высокоуміе, безмѣрное самолюбіе, безчинное похотѣніе, тщеславіе, невѣдѣніе Бога и о Немъ нерадѣніе, ненависть, зависть, гнѣвъ, злоба, нечистота, и проч. Что бо человѣкъ, благодатію Божіею необновленный, замышляетъ и хощетъ, какъ только едино зло и суету? Какъ отъ источника смердящаго едина только злая воня исходитъ: тако отъ сердца человѣческаго, силою Божіею необновленнаго, ничто не происходитъ, кромѣ зла. Зло сіе тѣмъ болѣе опаснѣйшее есть, что глубоко въ человѣкѣ сокровенно есть; и отъ тѣхъ только познается, которыи со всякимъ прилѣжаніемъ разсматриваютъ внутреннее свое состояніе, и въ различныхъ находятся искушеніяхъ. Такъ сильно уязвилъ и заразилъ насъ врагъ нашъ, человѣче, что никакой силѣ не возможно было исцѣлити насъ. Требуется къ тому сила Божія, которая изъ ничего создала насъ. Надобно было пріитить Самому Создателю къ Своему созданію, неисцѣльно отъ врага уязвленному, и на пути міра сего поверженному, котораго ни законъ, ни пророки не могли исцѣлить, и Самому его исцѣлить, никакою силою другою неисцѣлимаго. Пришелъ Создатель и \textit{умилосердился} надъ нимъ; пришелъ во образѣ человѣческомъ къ человѣку уязвленному \textit{и полумертвому}, да тако удобнѣе его исцѣлитъ, и въ первое приведетъ блаженство, котораго хитростію зміиною лишился\footnote{Лук.~10,~33--35.}. "--- Пою Тя, Боже и Создателю мой! Слухомъ бо, Господи, услышахъ, и ужасохся. До мене бо пришелъ еси, мене ища заблуждшаго. Тѣмъ многое Твое снисхожденіе, еже для мя, прославляю Многомилостиве! "--- Познавай убо, человѣче, язвы души твоея, когда хощешь отъ нихъ исцѣлитися, и со всякимъ усердіемъ молись и воздыхай ко Хрісту, да Онъ исцѣлитъ тебе, безъ Котораго исцѣленіе твое быть не можетъ: понеже никакой созданной силѣ не возможно есть. Изъ тьмы свѣтъ, и изъ злаго доброе \textit{единъ} Богъ можетъ сотворить, Который изъ ничего все творитъ.

\section{62. Госпиталь или лазаретъ.}

Что госпиталь есть больнымъ: тое церковь святая есть хрістіаномъ, духовно болящимъ. Въ госпиталь больные входятъ дверьми: въ церковь святую болящіе духовно входятъ вѣрою и крещеніемъ святымъ. Въ госпиталь того ради входятъ больные, чтобы исцѣлиться отъ болѣзни, и тако здравіе получить: въ церковь святую того ради входятъ болящіи духовно, чтобы исцѣлиться отъ болѣзней душевныхъ и тако спасенными быть. Въ госпитали имѣется лѣкарь, который больныхъ посѣщаетъ, смотритъ и лѣчитъ: въ церкви святой Врачь есть Хрістосъ, Который хрістіанъ, духовно болящихъ, посѣщаетъ и врачуетъ. Въ госпитали болящимъ лѣкарь запрещаетъ все тое, что лѣкарству его, больнымъ подаваемому, препятствіе чинитъ: въ церкви святой находящимся хрістіанамъ отъ всего того воздерживаться повелѣваетъ Хрістосъ, что душевному ихъ исцѣленію и полученію вѣчнаго спасенія препятствуетъ. Въ госпитали болящіе, которые хотятъ исцѣлиться, слушаютъ лѣкаря, и исцѣляются: тако и хрістіанамъ, когда хотятъ исцѣлитися и тако спасенными быть, должно Хріста"=Врача слушать, и отъ всего того удаляться, что Онъ запрещаетъ. Въ госпитали не исцѣляются и здравія не получаютъ тіи больніи, который лѣкаря не слушаютъ, и отъ того не воздерживаются, что имъ запрещаетъ: тако и хрістіане неисцѣльными пребываютъ и спасенія лишаются, которыи Хріста не слушаютъ и по своимъ прихотямъ, а не по правилу ученія Его ходятъ. Въ госпитали болящіи лѣкарю объявляютъ свои болѣзни, «я"=де тѣмъ"=то немогу»: тако хрістіанамъ должно объявлять свои болѣзни Хрісту "--- небесному Врачу, и отъ Него исцѣленія просить. \textit{Не требуютъ бо здравіи врача, но болящіи}\footnote{Мѳ.~9,~12.}. Въ госпитали бываетъ, что не вси болящіи исцѣляются: бываютъ бо болѣзни неисцѣльныи, и человѣкъ не все можетъ, что хощетъ: въ церкви святой не тако. Нѣтъ такой душевной болѣзни, которой бы Хрістосъ не хотѣлъ и не могъ исцѣлить, только бы самъ больный хотѣлъ того, и усердно желалъ и просилъ и искалъ у Хріста. Все бо Тому возможно, Котораго слову и манію все повинуется, и Котораго хотѣніе и слово есть дѣло, Котораго повелѣніемъ \textit{прокаженніи очищаются, слѣпіи прозираютъ, глухіи слышатъ, нѣміи глаголютъ, разслабленніи и лежащіи на одрѣ востаютъ, мертвіи воскресаютъ}. Познай только, хрістіанине, и признай немощь твою, и проси со смиреніемъ отъ Врача Сего исцѣленія, и безъ сумнѣнія ожидай: и неотмѣнно получишь; только берегись того, что спасительному Его врачевству препятствуетъ. Въ госпитали болящіи, хотящіи исцѣлитися, отдаютъ себе въ волю лѣкаря, да, якоже хощетъ, съ ними поступаетъ: тако хрістіанамъ, когда истинно хотятъ исцѣлитися, должно поручить и ввѣрить себе премудрому и вѣрному Врачу, Іисусу Хрісту, да творитъ съ ними, якоже хощетъ; горькое ли, или сладкое подаетъ имъ врачевство, все за благо принимать. Мудрый бо есть и вѣрный и человѣколюбивый Врачъ Іисусъ Хрістосъ, и весьма хощетъ души наша исцѣлить, и тако спасти: на сіе бо и въ міръ пришелъ, и пострадалъ, и умеръ за насъ. Тѣлеса тлѣнныя и смертныя исцѣлялъ, какъ Евангеліе Его объявляетъ намъ: души ли безсмертныя не исцѣлитъ? Повѣримъ только Ему себе, да исцѣляетъ насъ, якоже хощетъ: а мы потщимся отъ всего того воздерживаться, что Онъ намъ запрещаетъ, и спасительному Его врачеванію препятствіе чинитъ. Что бо больному пользуетъ лѣкарство, который по своей волѣ, а не по лѣкарской, поступаетъ? Разсуждай сіе хрістіанине, когда хощешь исцѣлитися и тако спастися. Вси мы болящіи есмы и требуемъ Врача и отъ него исцѣленія: но не вси познаютъ и признаютъ болѣзней своихъ. Начало же исцѣленія "--- познать немощь свою. Многіе не исцѣляются, яко не познаютъ немощи своея; и не познаютъ, яко не тщатся познавать; и тако, не познавая и не признавая немощи своея, не ищутъ и исцѣленія. Видитъ человѣкъ немощь тѣлесную, и ищетъ исцѣленія. О, съ коликимъ бы усердіемъ искалъ исцѣленія души своей, когда бы увидѣлъ тяжкую болѣзнь ея! Но то бѣда, что какъ сама душа, такъ и немощь ея не видна, и только отъ тѣхъ усматривается, которыи со всякимъ прилѣжаніемъ разсматриваютъ состояніе ея. Разсматривай убо, человѣче, и познавай многоразличную немощь души твоея, да поищеши исцѣленія ей отъ небеснаго Врача"=Хріста. Чимъ болѣе будешь разсматривать и познавать немощь ея, тѣмъ съ большимъ усердіемъ и желаніемъ поищешь исцѣленія ей. Тѣлесная болѣзнь неисцѣльная временною смертію грозитъ: душевная болѣзнь неисцѣленная вѣчною смертію грозитъ. Оставь убо смотрѣть и искать, что внѣ тебе есть и по большей части вредитъ тебѣ, а не пользуетъ: разсматривай и познавай, что въ душѣ твоей дѣлается, сколько и какія болѣзни ее мучатъ и къ смерти ведутъ. Вшелъ ты въ священную церкви святыя врачебницу, исцѣленія отъ Хріста Спасителя искать души твоей, и тако вѣчно спасеннымъ быть, а не чести, славы и богатства искать, не банкеты строить, не въ гости ѣздить и гостей принимать, и за прочею суетою гоняться. Лѣчиться въ лазаретъ приходятъ, а не прихоти своя исполнять. Бѣдный хрістіанине! познай заблужденіе свое, и отъ суеты обратись ко Хрісту, и тщись познавать немощь души твоея, и признавай ее предъ Хрістомъ, да и исцѣлишися отъ Него. \textit{Не требуютъ здравіи врача, но болящіи}\footnote{Мѳ.~9,~12.}. \textit{Азъ рѣхъ: Господи, помилуй мя, исцѣли душу мою, яко согрѣшихъ Ти! Возсмердѣша и согниша раны моя отъ лица безумія моего}\footnote{Пс.~40,~5; 37,~6.}.

\subsection{О томжде.}

Бываетъ, что немощный отъ многихъ лѣкарей ищетъ исцѣленія, но не получаетъ, якоже читаемъ о Евангельской оной кровоточивой женѣ\footnote{Марк.~3,~25 и 26.}: тако бѣдный хрістіанинъ, видя свою лютую немощь душевную, то оттуду, то отсюду ищетъ помощи себѣ, но не получаетъ; многія книги съ немалымъ трудомъ прочитываетъ, но отъ нихъ исцѣлитися не можетъ. Любезный хрістіанине! какъ ни тщись, и куды ни обращай себе, но, кромѣ Хріста, нигдѣ и ни отъ чего не получишь исцѣленія. Безъ сумнѣнія полезно и нужно прочитывать книги хрістіанскія, и ихъ разсуждать, и въ нихъ поучаться; просвѣщаютъ умъ, и даютъ разумъ, и къ покаянію подвигаютъ, и къ молитвѣ возбуждаютъ: но онѣ вси, какъ видимъ, на Хріста указуютъ и ко Хрісту насъ отсылаютъ, и руководствуютъ, да отъ Него ищемъ исцѣленія и спасенія. Онъ \textit{единъ}, яко Свѣтъ присносущный, и слѣпыхъ просвѣщаетъ; и, яко Врачъ, немощныхъ исцѣляетъ; и, яко Животъ, мертвыхъ оживотворяетъ; и, яко Всесильный, разслабленныхъ возставляетъ и хромыхъ исправляетъ. Бѣдная оная кровоточивая жена, когда, ни отъ какихъ врачевъ не возмогши исцѣлитися, но паче въ горшее пришедши, послышала о Іисусѣ, о Которомъ слухъ тогда всюду прошелъ, что всѣхъ къ Себѣ допущаетъ, и всякія болѣзни не быліемъ, но словомъ и силою Своею исцѣляетъ; подумала сама съ собою тако: пойду и я къ великому тому Чудотворцу и Врачу, и за стыдъ немощи моея не явлюся предъ лицемъ Его, ни буду объявлять скаредной немощи моей, но приступлю созади и прикоснуся ризамъ Его, и тако исцѣлюся. И съ такою вѣрою, \textit{пришедши въ народъ, созади прикоснуся ризѣ Его: глаголаше бо, яко аще прикоснуся ризамъ Его, спасена буду. И абіе изсякну источникъ крове ея; и ощути тѣломъ, яко исцѣлѣ отъ раны}\footnote{Марк.~5,~27--29.}. Видишь, хрістіанине, вѣру бѣдныя жены! видишь силу, дѣйствіе и плодъ вѣры ея! Сія жена научаетъ насъ, гдѣ намъ и како искать исцѣленія немощамъ нашимъ душевнымъ, то"=есть, \textit{у Хріста вѣрою}. Она говорила въ себѣ: \textit{аще прикоснуся ризамъ Его, спасена буду}. Се есть вѣры свойство! И получила тое, чего надѣялася. Идѣже бо истинная вѣра, тамо и надежда. Возмемъ убо въ примѣръ жену сію, и мы приступимъ ко Хрісту вѣрою и надеждою, и хотя молча, падемъ предъ Всевидящимъ окомъ Его; и когда молимся, съ вѣрою и надеждою да молимся; и когда къ святымъ и животворящимъ Тайнамъ Его приступаемъ, съ вѣрою и надеждою исцѣленія и обновленія да приступаемъ, поминая и подражая оную кровоточивую жену. Тако душевная наша исцѣлится немощь, и \textit{изсякнетъ источникъ} пагубныхъ страстей нашихъ, мучащихъ душу нашу. Знакъ есть, что душевная немощь начала исцѣлятися, когда страсти душевредныя начнутъ усмирятися и утихать, "--- когда менѣе гордости, самолюбія, сребролюбія, ненависти, зависти, гнѣва, ярости, скупости, славолюбія, нечистоты и прочаго зла показуется въ человѣкѣ: якоже знаменіе есть тѣла наченшаго исцѣлятися, когда вредныхъ соковъ начнетъ свобождатися. Что бо вредные соки суть въ тѣлѣ, тое грѣховныя страсти суть въ душѣ. Вредные соки мучатъ тѣло и умерщвляютъ: грѣховныя страсти мучатъ душу, и къ вѣчной ведутъ смерти. Въ здравіе тѣло приходитъ, когда отъ злыхъ и вредныхъ соковъ исцѣляется: тако здравіе свое получаетъ душа, когда отъ злыхъ и грѣховныхъ страстей свобождается. \textit{Помилуй мя, Господи, Сыне Давидовъ! дщи моя} (душа моя) \textit{злѣ бѣснуется. Господи, помози ми}\footnote{Мѳ.~15,~22 и 25.}! Великъ бѣсъ есть грѣхъ, который въ душѣ живетъ и душу мучитъ; онъ не даетъ ей покоя; но то къ тому, то къ другому, то къ третьему богопротивному и пагубному дѣлу порѣваетъ и убѣждаетъ. \textit{Господи, помози ми! Ей, Господи! ибо и пси ядятъ отъ крупицъ падающихъ отъ трапезы господей своихъ}. Да услышу и я недостойный: \textit{буди тебѣ, якоже хощеши!} якоже услышала жена хананейская\footnote{ст. 27 и 28.}.

\section{63. Ядъ, сокровенный въ человѣкѣ.}

Бываетъ, что человѣку отъ злыхъ людей подается ядъ: тако отъ древняго змія, врага нашего діавола, вліялся ядъ грѣховный и смертоносный въ естество наше. Человѣкъ, хотя и здоровъ бываетъ, однакожъ, когда въ себе пріемлетъ какимъ либо случаемъ ядъ, отъ того сильно немоществуетъ: тако естество наше было чистое, непорочное, святое, доброе; но, когда ядомъ хитраго и лукаваго онаго змія заразилося, тогда въ неисцѣльную немощь и бѣду впало. Ядъ, имѣющійся въ человѣкѣ, все тѣло его заражаетъ: тако смертоносный оный ядъ зміинъ всѣ силы душевныя и тѣлесныя наши заразилъ. Отъ сего яда бываетъ, что \textit{извнутрь уду, отъ сердца, человѣческа помышленія злая исходятъ, прелюбодѣянія, любодѣянія, убійства, татьбы, лихоимства, (обиды), лукавствія, лесть, студодѣянія, око лукаво, хула, гордыня, безумство. Вся сія злая извнутрь исходятъ, и сквернятъ человѣка}\footnote{Марк.~7,~21--23.}. Отсюду происходятъ: \textit{прелюбодѣяніе, блудъ, нечистота, студодѣяніе, идолослуженіе, чародѣяніе, вражды, рвенія, завиды, ярости, разжженія, распри, соблазны, ереси, зависти, убійства, піянства, безчинны кличи, и подобная симъ}\footnote{Гал.~5,~19--21.}. Отсюду бываетъ гордость, высокоуміе, презрѣніе ближняго, осужденіе, оклеветаніе, злословіе, руганіе, дѣломъ и словомъ отмщеніе, желаніе и исканіе собственныя своея чести, славы и похвалы, отсюду лесть, лукавство, хитрость, ложь и лицемѣріе; отсюду студное дѣло, срамословіе; отсюду излишнее о пищи и питіи и о трапезахъ попеченіе; отсюду столько вымышляютъ люди перемѣнъ въ одеждѣ и платьѣ, въ строеніи и украшеніи домовъ, въ пріуготовленіи каретъ и коней, и прочія суеты. Все сіе и прочее, подобное сему, отъ плотскаго мудрованія и смертоноснаго яда зміина, въ сердце человѣческое всѣяннаго, происходитъ. Ядъ, имѣющійся внутрь человѣка, мучитъ человѣка, и временемъ нестерпимую ему содѣловаетъ болѣзнь: тако ядъ оный зміинъ, сокровенный въ душѣ, весьма мучитъ душу и различную ей содѣловаетъ болѣзнь. Смотри, что дѣлаетъ \textit{гордость} въ человѣкѣ! какъ его мучитъ! сколько онъ вымышляетъ способовъ, какъ бы достать честь, славу и похвалу въ мірѣ семъ! Доставши, съ какимъ трудомъ и попеченіемъ бережетъ сокровище свое сіе! Какъ негодуетъ, когда отъ кого презирается! какъ болѣзнуетъ, смущается, ропщетъ и хулитъ, когда чести лишится, такъ что многіи себе умерщвляютъ! Что \textit{сребролюбіе} дѣлаетъ, какое попеченіе и болѣзнь бѣдной человѣческой душѣ, сказать не возможно! Видимъ, сколь неусыпнымъ тщаніемъ и непрестанными трудами бѣдныи сіи люди ищутъ богатства! Сысканное съ коликимъ страхомъ и боязнію стерегутъ и берегутъ! Лишившись того, какъ печалятся, скорбятъ, сѣтуютъ, тоскуютъ и мучатся, которой несносной болѣзни не терпя, часто и смерти себе предаютъ! \textit{Блудная похоть} какое мучительство, какое жженіе въ сердцѣ и во всемъ тѣлѣ человѣческомъ возставляетъ, всякому, а паче безбрачно живущимъ, извѣстно. Она, какъ лютая горячка, во всемъ составѣ человѣческомъ жженіе и движеніе дѣлаетъ. \textit{Завистію} какъ мучится бѣдная душа, изобразить не можно! Самое тѣло блѣднѣетъ и изсыхаетъ \textit{отъ зависти}. Посмотри еще на \textit{гнѣвъ}, какіе онъ знаки мучительства своего производитъ! Смотри, что во гнѣвѣ человѣкъ дѣлаетъ: какъ негодуетъ и шумитъ, клянетъ и ругаетъ, самъ себе терзаетъ и біетъ, ударяетъ въ главу и лице свое и, какъ въ лихорадкѣ, весь трясется; словомъ, подобенъ тогда является бѣсноватому. Аще внѣ такъ скареденъ видъ является: что уже дѣлается въ бѣдной душѣ! какъ ее бѣсъ сей мучитъ! Видишь, человѣче, какъ лютый сокровенъ зміинъ ядъ въ душѣ, и какъ горько ее мучитъ! Познавай убо со всякимъ прилѣжаніемъ и испытаніемъ смертоносный ядъ сей, въ сердцѣ и душѣ твоей крыющійся. Дѣйствіе его лютое и тлетворное самаго его показуетъ. Аще бы его не было въ сердцѣ твоемъ и въ душѣ твоей; не бы толь лютую и горькую болѣзнь содѣловалъ. Всякая бо вещь отъ дѣйствія ея познается: тако сокровенный ядъ зміинъ въ сердцѣ и душѣ человѣческой отъ лютыхъ и мучительныхъ дѣйствій познается. Познающему великое сіе и пагубное зло надобно искать отъ него исцѣленія, да не во вѣки отъ него умретъ душею и тѣломъ. Ядъ, въ тѣлѣ человѣческомъ имѣющійся, и неисцѣленный, смерть человѣку содѣловаетъ: тако ядъ сей всепагубный, когда силою Хрістовою не исцѣлится, ничѣмъ инымъ, какъ вѣчною смертію грозитъ душѣ. Когда тѣло немоществуетъ, ищешь лѣкаря и тщишися исцѣлить его, хотя и знаешь, что храмина сія разрушится и обратится въ прахъ и землю: не много ли болѣе должно пещися о душѣ безсмертной и нетлѣнной, которыя драгость и цѣна весь міръ, небо и землю превосходитъ, и исцѣленія ей искать? Толь дорога душа есть, яко по образу Божію созданная и безцѣнною Сына Божія кровію искупленная: и нерадиши о ней!... Нерадиши: яко не видиши и не познаеши лютыя и смертоносныя болѣзни ея и яда, въ ней сокровеннаго. О, когда бы возможно было человѣку видѣть лютую заразу ея! воистину ужаснулся бы, и неутѣшно бы плакалъ и рыдалъ о бѣдствіи и погибели ея; и все бы оставивши позади себе, искалъ ее исцѣлить, и въ первое благообразіе и доброту привести! Кто бо, видя крайнюю бѣду свою, отъ ней свободиться не поищетъ? Якоже видимъ на болящихъ тѣломъ, которыи, какую нибудь почувствовавше на себѣ болѣзнь, ищутъ исцѣленія: души, познавши и почувствовавши болѣзнь лютую и пагубную, не будешь ли тщаться исцѣлить? Не возможно сіе! Познанная бо бѣда убѣждаетъ человѣка искать способа и помощи избавиться отъ бѣды. Познавай убо и разсматривай, человѣче, лютую немощь души твоея, и будешь тщаться о исцѣленіи ея; безъ познанія немощи не бываетъ исцѣленія ея. Познаемъ убо, возлюбленніи, лютую немощь, въ сердцѣ и душѣ нашей крыющуюся, и обратимся ко Хрісту, Врачу душъ и тѣлесъ, и отъ Него со всякимъ тщаніемъ и усердіемъ поищемъ исцѣленія, часто и изъ глубины сердца вознося въ Нему десяти прокаженныхъ гласъ: \textit{Іисусе, Наставниче, помилуй насъ}\footnote{Лук.~17,~13.}! Но и сами о себѣ да не нерадимъ, уклоняяся отъ всякаго грѣха, яко тѣмъ Онъ оскорбляется. Кто бо прогнѣвляетъ врача, отъ котораго ищетъ исцѣлитися? не паче ли тщится угождать ему? Заразилъ насъ змій древній лютымъ ядомъ своимъ: отъ Хріста Сына Божія поищемъ исцѣленія. Онъ единъ можетъ отъ того исцѣлить насъ; и исцѣляетъ вѣрою къ Нему воздыхающихъ и молящихся. Израильтяне, угрызаемые въ пустыни отъ зміевъ, взирали, по повелѣнію Божію, на вознесенную \textit{змію мѣдяну}, и тако исцѣлялися\footnote{Числ.~21,~6--9.}: тако должно намъ вѣрою взирать на Хріста Сына Божія, за грѣхи наши вознесеннаго на древо, и крестомъ Своимъ сокрушившаго онаго змія главу, да отъ смертоноснаго его яда исцѣлимся. \textit{Якоже Моѵсей вознесе змію въ пустыни, тако подобаетъ вознестися Сыну человѣческому. Да всякъ вѣруяй въ Онь не погибнетъ, но имать животъ вѣчный}\footnote{Іоан.~3,~14 и 15.}.

\section{64. Брань.}

Видимъ въ мірѣ, что когда одно государство противъ другаго востаетъ, то и другое государство убѣждается себе защищать, и противу востающаго воставать и брань съ нимъ творить: тако, понеже плоть наша со страстьми и похотьми востаетъ на насъ, хрістіанине; то и мы не должны дремать, но защищать себе и ей противитися. Когда одно государство другому противится, и тщится одно другое побѣдить; то между ими брань бываетъ. Тако, когда хрістіанинъ плоти со страстьми и похотьми противится и не соизволяетъ ей, что хощетъ, но паче тщится ее духу покорить; тогда между хрістіаниномъ и плотію брань бываетъ. И сіе"=то есть, что Апостолъ глаголетъ: \textit{плоть похотствуетъ на духа, духъ же на плоть: сія другъ другу противятся}\footnote{Гал.~5,~17.}. На брани видимой, когда одна сторона другой не хощетъ покоритися, и тако въ работу ей предатися, всѣми силами противу той стоитъ и подвизается: тако хрістіанину со всякимъ тщаніемъ и прилѣжаніемъ должно противу плоти стоять и подвизаться, когда не хощетъ отъ нея побѣжденнымъ быть и ей покориться, и плѣнникомъ ея быть, и тако ей работать. На брани воины, како противу врага своего подвизаться, отъ военачальниковъ наставленіе и другъ отъ друга помощь получаютъ: тако хрістіанину, противу плоти подвизающемуся, должно отъ Хріста"=Подвигоположника наставленія и помощи просить и ожидать, и Того силою востаніе плоти своея усмирять. Плоть хощетъ гордиться и возноситься: но хрістіанину должно ее \textit{смиреніемъ} Хрістовымъ усмирять. Плоть хощетъ въ мірѣ семъ богатѣть и много собирать: но хрістіанину должно \textit{нищетою} Хрістовою хотѣніе ея пресѣкать. Плоть хощетъ на человѣка гнѣватися, и за обиду ему отмщевать: но хрістіанину должно движеніе ея \textit{кротостію} и тихостію Хрістовою укрощать. Плоть въ несчастіи волнуется, мятется, смущается, и хощетъ роптать и не терпѣть: но хрістіанину должно \textit{силою} и \textit{терпѣніемъ} Хрістовымъ безчинное ея движеніе успокоивать. Плоть хощетъ враждующихъ ей ненавидѣть и злобиться: но хрістіанину должно \textit{благостію} и \textit{любовію} Хрістовою ее побѣждать. Тако и въ прочемъ хрістіанину должно силою и \textit{примѣромъ} Хрістовымъ противу плоти стоять и подвизаться, и ее побѣждать. "--- Бываетъ на видимой брани, что сторона побѣжденная паки востаетъ и исправляется, и противу побѣдившей стороны сильнѣе подвизается: тако хрістіанину, побѣжденному отъ плоти, должно востать, и, призвавши въ помощь \textit{Всесильнаго} Іисуса, укрѣпиться силою Его и паки противу плоти стоять и подвизаться, и не попущать ей, дабы она владѣла и господствовала надъ нимъ. На видимой брани, чимъ болѣе одна сторона побѣждается, а другая побѣждаетъ, тѣмъ немощнѣйшая дѣлается побѣжденная, а побѣждающая сильнѣйшая и крѣпчайшая бываетъ: тако на брани между плотію и хрістіаниномъ бываетъ; чимъ болѣе хрістіанинъ побѣждаетъ плоть, тѣмъ болѣе плоть изнемогаетъ, страсти ея и похоти усмиряются и утихаютъ, а хрістіанинъ болѣе усиливается и укрѣпляется и лучшимъ бываетъ. На видимой брани воинъ не противу одного врага, но противу всякаго стоитъ и подвизается: тако должно хрістіанину не противу одной только страсти, но и противу всѣхъ стоять и подвизаться. Что бо пользуетъ воину противу одного врага стоять и подвизаться, а другимъ не противиться, но отъ нихъ побѣжденнымъ и умерщвленнымъ быть? Воинъ, когда хощетъ животъ свой сохранить и побѣдителемъ быть, всѣмъ врагамъ востающимъ противитися долженъ. Что пользуетъ и хрістіанину противу одной нѣкоей страсти стоять и подвизаться, а другимъ покоряться и работать? Многіи подвизаются противу блудной похоти, что похвально и славно, но гнѣвомъ и яростію побѣждаются; иніи щедры и милостивы ближнимъ своимъ, но языкомъ своимъ вредятъ человѣка, оклеветая и осуждая его; многіи удерживаютъ чрево свое отъ объяденія и піянства, но отъ злопомнѣнія и мало поститься не хотятъ; тако и въ прочемъ. Якоже убо вооружаемся и стоимъ противу единой страсти, тако должно и противу прочихъ вооружитися и съ ними брань творить. О семъ глаголетъ Апостолъ: \textit{возлюбленніи! молю яко пришельцевъ и странниковъ, огребатися отъ плотскихъ похотей, яже воюютъ на душу}\footnote{1~Петр.~2,~11.}. Хрістіанине! якоже вооружился еси и брань творишь противу единой нѣкоей страсти: тако и противу прочіихъ вооружайся и не попущай себя отъ нихъ побѣжденнымъ быть. Якоже борешися съ блудною похотію, и не попущаеши ей одолѣть тебе: тако борись и брань твори съ гордостію, борись съ высокоуміемъ, борись съ тщеславіемъ, борись съ гнѣвомъ, яростію и памятозлобіемъ, борись со сребролюбіемъ и скупостію, борись съ ненавистію и завистію, и проч. Якоже воздерживаеши чрево твое отъ объяденія и піянства: тако воздерживай языкъ твой отъ клеветы и осужденія, празднословія, сквернословія и буесловія. Якоже удерживаешь руки твои отъ убійства, воровства, хищенія и грабленія: тако удерживай отъ ударенія и біенія. Якоже постишися отъ пищи и питія, тако постися отъ всякаго зла. Се есть хрістіанскій постъ! се есть истинное воздержаніе! Трудный сей подвигъ; но хрістіанская должность сего требуетъ. Многіи побѣждаютъ людей, государства и грады; но себе побѣждать не хотятъ. Се есть хрістіанская побѣда "--- себе самого, то"=есть, плоть свою побѣдить! Воинъ чимъ чаще на брани бываетъ и противу враговъ подвизается, тѣмъ искуснѣйшій и храбрѣйшій бываетъ: тако хрістіанинъ, чимъ болѣе противу плоти, страстей и похотей ея подвизается, и благодатію Хрістовою ихъ побѣждаетъ, тѣмъ искуснѣйшимъ въ званіи хрістіанскомъ и часъ отъ часу лучшимъ дѣлается. Ибо Господь, видя тщаніе его, трудъ и подвигъ, дѣлаетъ съ нимъ милость, и исправляетъ его, и свобождаетъ отъ страстныя работы, и дѣлаетъ его добрымъ древомъ, которое само отъ себе добрые плоды творитъ. \textit{Аще по плоти живете, имате умрети: аще ли духомъ дѣянія плотская умерщвляете, живи будете}\footnote{Рим.~8,~13.}. \textit{А иже Хрістовы суть, плоть распяша со страстьми и похотьми}\footnote{Гал.~5,~24.}. \textit{Ни едино убо нынѣ осужденіе сущимъ о Хрістѣ Іисусѣ, не по плоти ходящимъ, но по духу}\footnote{Рим.~8,~1.}. \textit{Неистовствующееся бурею душетлѣнною, Владыко Хрісте, страстей море укроти, и отъ тли возведи мя, яко благоутробенъ}\footnote{Пѣснь 6~глас. 5"~го. Приписана собственною рукою святителя. См. ссылку въ Предисловіи.}!

\section{65. Званіе.}

Видимъ въ мірѣ, что въ различная званія люди позываются; иныи бываютъ судіями, иныи воинами, иныи полководцами, иныи скотъ пасутъ, и проч: тако всякій хрістіанинъ позванъ отъ Хріста Господа на покаяніе, якоже глаголетъ: \textit{не пріидохъ призвати праведники, но грѣшники на покаяніе}\footnote{Мѳ.~9,~13.}. А какъ вси грѣшники: \textit{(вси бо согрѣшиша, и лишени суть славы Божія)}\footnote{Рим.~3,~23.}: то вси и на покаяніе призваны, которыи ни призваны. Званіе убо хрістіанское, къ которому они отъ Хріста призваны, есть истинное покаяніе: на сіе бо ихъ Хрістосъ призвалъ. Всякому человѣку званіе его есть первѣйшее дѣло, пока онъ въ званіи своемъ находится. Судей первѣйшее дѣло есть истину изыскивать и по законамъ судить, доколѣ они въ семъ званіи находятся: тако хрістіанъ первѣйшее дѣло есть быть въ покаяніи; ибо къ тому ихъ Хрістосъ позвалъ. Смотри, хрістіанине, къ чему ты позванъ? Не богатства, ни чести, ни славы въ мірѣ искать, не банкеты строить, не въ гости ѣздить и гостей принимать и проч., но, отъ всѣхъ сихъ отвратившись, быть въ покаяніи, и тако вѣчное спасеніе, къ которому позванъ ты, о Хрістѣ Іисусѣ получить. Не въ семъ бо только покаяніе состоитъ, чтобы отъ внѣшнихъ великихъ грѣховъ отстать, но въ премѣненіи ума и сердца, и обновленіи внутренняго состоянія; то"=есть, отвратиться отъ всѣхъ міра сего суетствій, яко онѣ всякому хотящему спастися запинаютъ; разсматривать различную душевную немощь, то"=есть, гордость, гнѣвъ зависть, нечистоту, сребролюбіе и проч., и жалѣть и сокрушаться, что такое зло вошло въ душу нашу чрезъ ядъ зміинъ, которая чиста и непорочна была создана, и съ такимъ жалѣніемъ и сокрушеніемъ молиться ко Хрісту, чтобы силою Своею насъ исправилъ и исцѣлилъ. А когда сердце или внутреннее состояніе перемѣнится и исправится, тогда и внѣшнее житіе и внѣшнія дѣла добрыя будутъ. Ноги не пойдутъ на зло, когда душа не захощетъ; руки не будутъ дѣлать зла, когда воля не захощетъ; языкъ не будетъ говорить зла, когда сердце не захощетъ\footnote{Начиная со словъ: тѣло не будетъ блудодѣйствовать, и до словъ: въ сердцѣ живетъ убійца, прелюбодѣй, тать, писано собственною рукою святителя. См. ссылку въ Предисловіи.}; тѣло не будетъ блудодѣйствовать, когда сердце не захощетъ. Что пользуетъ внѣ не дѣлать зла, но внутрь быть злымъ? Что пользуетъ не красть, но внутрь лихоимцемъ быть? Что пользуетъ руками не убивать, но внутрь злобою и убійствомъ дышать? Въ сердцѣ живетъ убійца, прелюбодѣй, тать. Нѣтъ никакой пользы въ сосудѣ, который внѣ является чистъ, но внутрь всякаго смрада и нечистоты исполненъ: тако нѣтъ никакой пользы и человѣку, который внѣ является добръ, но внутрь золъ; внѣ является смиренъ и тихъ, но внутрь гордости, зависти и злобы исполненъ; ласково и гладко говоритъ, но лесть и лукавство на сердцѣ имѣетъ; отъ хищенія и грабленія руки удерживаетъ, но и своего никому не даетъ, и проч. Сего ради люди таковыи, который внѣ являются добрыми, но внутрь злы, уподобляются \textit{гробамъ повапленнымъ, которыи внѣ являются красны, но внутрь полны суть костей мертвыхъ, и всякія нечистоты}\footnote{Мѳ.~23,~27.}. \textit{Богъ же нашъ}, хрістіанине, \textit{сердца и утробы испытуяй, смотритъ на сердце наше}, а не на внѣшность и наружность\footnote{Пс.~10,~7; 1~Цар.~19,~7.}. Онъ на то смотритъ, что внутрь въ сердцѣ крыется, а не на то, что внѣ является; и по тому судитъ человѣку. Являешися внѣ смиренъ, "--- хорошо: но не возносишися ли сердцемъ? Не показуеши внѣ гнѣва и не яришися, "--- хорошо: но не снѣдаешися ли внутрь злобою на ближняго твоего? Не убиваеши брата твоего руками, "--- не худо: но смотри, не ненавидиши ли его въ сердцѣ твоемъ? \textit{Всякъ бо ненавидяй брата своего, человѣкоубійца есть}\footnote{1~Іоан.~3,~15.}. Не смѣшаешися съ женою тѣломъ твоимъ, "--- не худо: но похотствуеши сердцемъ, блудникъ еси. \textit{Всякъ, иже воззритъ на жену, ко еже вожделѣти ея, уже любодѣйствова съ нею въ сердцѣ своемъ}\footnote{Мѳ.~5,~28.}. Не похищаеши чуждаго имѣнія, но внутрь въ сердцѣ твоемъ желаеши того: истинный тать еси. Не крадеши чуждаго, но отъ своего не даеши отъ скупости бѣдствующему: истинный тать еси; яко добро Божіе, которое дано тебѣ не для тебе единаго, но и для прочіихъ, при себѣ единомъ удерживаеши; и столько людей обиждаешь, сколько можешь пользовать добромъ твоимъ, которое у себе удерживаешь, и проч. Сердце человѣческое сосудъ есть, нечистотою и смрадомъ похотей исполненный. Возлюбленный хрістіанине! потщимся сей сосудъ внутрь насъ очистить; и тогда внѣшняя наша дѣла Богу угодна будутъ. Велико обѣщалъ намъ Богъ; Онъ обѣщалъ Самъ въ сердцахъ нашихъ жить: \textit{яко вселюся въ нихъ, и похожду; и буду имъ Богъ, и тіи будутъ Мнѣ людіе}\footnote{2~Кор.~6,~16; Лев.~26,~12.}. Нѣтъ и не можетъ быть большаго, дражайшаго, пріятнѣйшаго и любезнѣйшаго сокровища, какъ Бога внутрь себе живущаго имѣть; и нѣтъ большія чести человѣку, какъ храмомъ Божіимъ быть! Богъ въ чистой душѣ, какъ въ благопріятнѣйшемъ храмѣ Своемъ, живетъ; и любезнѣе Ему въ чистой душѣ обитать, нежели въ рукотворенныхъ храмахъ: понеже въ душѣ образъ Божій есть. Очистимъ убо и мы сердца наша, о хрістіанине, да и въ насъ явится образъ Божій, прекрасная душъ нашихъ доброта, и такъ будемъ храмомъ Бога нашего. \textit{Сицева убо имуще обѣтованія, о возлюбленніи, очистимъ себе отъ всякія скверны плоти и духа, творяще святыню въ страсѣ Божіи}\footnote{2~Кор.~7,~1.}. "--- \textit{Пророка Іону подражая, вопію: животъ мой, Блаже, свободи изъ тли, и спаси мя, Спасе міра, зовуща: слава Тебѣ}\footnote{Пѣснь 6"~я глас. 1"~го. Приписана собственною рукою святителя. См. ссылку въ Предисловіи.}.

\section{66. Отецъ твой давно тебе ждетъ: не медли!}

Бываетъ, что человѣкъ другому человѣку, отъ отца своего отлучившемуся и медлящему, говоритъ тако: отецъ твой давно тебе ждетъ: что ты здѣ медлишь? Тако пророки, апостоли, пастыри и учители церковныи всякому грѣшнику говорятъ, который на крещеніи святомъ къ небесному Отцу присталъ"=было и къ домашнимъ Его святымъ Божіимъ причислился было, но потомъ беззаконнымъ и развращеннымъ житіемъ отлучился отъ Него, и аки \textit{въ далекую страну отшелъ}; къ такому говорятъ и увѣщаваютъ его: Отецъ небесный давно ожидаетъ тебе; что ты, бѣдный грѣшникъ, медлишь на беззаконной странѣ сей и пагубномъ удаленіи? \textit{Имѣніе}, данное тебѣ, все ты потерялъ; \textit{лишаешися} дневнаго пропитанія: \textit{гладъ} пагубный постигнулъ тебе. Оставивши милосердаго и благоутробнаго Отца своего, работавши злому и немилостиву мучителю, \textit{пасешь} безсловесный и нечистый скотъ его, \textit{свиней}: и въ такой тяжкой и мерзкой работѣ не имѣешь чимъ чрево свое насытити; и уже до того пришелъ ты, что желаешь и ищешь \textit{рожцевъ}, которыми питаются свиніи; но и таковой, такъ худой и послѣдней пищи, никто не даетъ тебѣ. Видишь самъ, въ какомъ злополучіи и бѣдности и окаянствѣ находишися ты! въ какую нищету съ коликаго богатства упалъ ты! въ какую пропасть съ такъ высокой чести низринулся ты! коль великаго блаженства лишился ты, и какъ въ великое злополучіе попалъ ты! Все сіе тебѣ сдѣлалось оттуду, что ты самовольно отлучился отъ благаго и благоутробнаго Отца твоего, дому Его, святой фамиліи. Въ дому Его живущимъ всѣмъ всякое изобиліе проистекаетъ и подается: ты всего лишаешися! Тамо вси во всякомъ благополучіи живутъ благодатію милостиваго Отца своего: ты во всякомъ бѣдствіи находишися! Тамо вси довольною пищею насыщаются и питіемъ прохлаждаются: ты же \textit{гладомъ гибнеши!} Тамо у благоутробнаго Отца и сладкія пищи и благопріятнаго питія довольно всѣмъ домашнимъ Его: ты до такой скудости дошелъ, что \textit{желаетъ чрево свое насытити отъ рожецъ, которыя ядятъ свиніи}, но и тѣхъ \textit{никто не даетъ тебѣ!} Тамо вси въ дому упокоеваются подъ защищеніемъ Отца своего милосердаго: ты здѣ между нечистою скотиною, \textit{свиніями} обращаешися! Тамо вси доброю одеждою одѣваются: ты раздраннымъ и смердящимъ рубищемъ, какъ какій лишенникъ, покрываешися! Тамо вси утѣшаются: ты сѣтуешь; тамо вси радуются: ты плачешь. Тамо вси веселятся: ты скорбишь и печалишися. Тамо вси ядятъ и піютъ изобильно: ты всего \textit{лишаешися}. О, въ коликое, въ коль тяжкое бѣдствіе ты пришелъ, ты, который былъ въ чести, въ славѣ, богатствѣ, изобиліи, мирѣ, покоѣ и во всякомъ блаженствѣ, живучи при благоутробномъ и милосерднѣйшемъ Отцѣ своемъ! Таковыя слова и увѣщанія слышитъ часто заблуждшій сынъ, однакожъ медлитъ во отлученіи своемъ бѣдственномъ. Ждетъ благоутробный Отецъ, и часто посматриваетъ, когда заблуждшій сынъ Его къ Нему пріидетъ; но у сына нѣтъ такой мысли и сердца, чтобы къ Отцу своему паки возвратиться. Отецъ хощетъ милость ему явить; но сынъ о милости Его нерадитъ. Отецъ хощетъ его принять паки въ сына и подать наслѣдіе; но сынъ пріити не хощетъ ко Отцу своему. Отецъ хощетъ блаженства Своего участникомъ его сотворить; но сынъ того не разумѣетъ. Отецъ съ фамиліею Своею ожидаетъ въ домъ свой; но сынъ медлитъ, какъ медлилъ въ беззаконномъ отлученіи и въ пагубной странѣ; изволяетъ лучше быть при зломъ мучителѣ, нежели при благомъ и благоутробномъ Отцѣ своемъ, "--- смраднымъ и гнуснымъ рубищемъ покрываться, нежели добрую у Отца своего одежду носить, "--- свинскою пищею, рожцами насыщать чрево свое, нежели благихъ дому Отца своего наслаждаться, "--- въ бѣднѣйшемъ состояніи находиться, нежели въ блаженствѣ быть. О благости, о благоутробія Отча, Который сына, самовольно отлучившагося, и Его злымъ нравомъ прогнѣвившаго, не презираетъ, не забываетъ, не отвергаетъ, но съ великимъ желаніемъ ждетъ, когда заблуждшій сынъ Его къ Нему возвратится! О слѣпоты, о безчувствія сыновня, который изволяетъ лучше на чуждой странѣ и въ крайнемъ злостраданіи находиться, нежели ко Отцу своему возвратиться, въ дому Его и во всякомъ благополучіи быть! О, бѣдный сынъ, очувствуйся, оставь беззаконную и пагубную страну сію! Помяни благоутробіе Отца твоего, помяни многія щедроты Его! Прогнѣвалъ ты Его: но Онъ милости Своея не отымаетъ отъ тебе. Удалился ты отъ Него: но Онъ ожидаетъ тебе къ Себѣ, и хощетъ принять тебе. Помяни, коликое изобиліе благихъ въ дому Отца твоего, \textit{колико наемникомъ Отца твоего избываютъ хлѣбы: ты же гладомъ погибаеши!} Воставши убо, возвратись къ Отцу твоему. \textit{Расточилъ} ты \textit{богатство Его}: не бойся! Онъ благоутробенъ, долготерпѣливъ и многомилостивъ; не помянетъ твоего своевольства, не будетъ тебѣ выговаривать твоихъ противныхъ Ему поступковъ. Щедръ и богатъ: паки обогатитъ тебе. \textit{Воставъ} убо, \textit{иди} смѣло \textit{къ Отцу твоему}, и, падши предъ милосердыми очами Его, признай свою винность, и скажи Ему съ сокрушеніемъ и жалѣніемъ сердца: \textit{Отче, согрѣшихъ на небо и предъ тобою; и уже нѣсмь достоинъ нарещися сынъ твой: сотвори мя яко единаго отъ наемникъ Твоихъ!} Онъ съ великимъ желаніемъ ожидаетъ тебе, и съ радостію пріиметъ тебе. Какъ только увидитъ тебе возвращающагося, хотя \textit{издалеча}, "--- милосердо и любезно воззритъ на тебе, и \textit{милъ Ему будеши}; смятется милосердая Его утроба тебе ради, и милуя помилуетъ тебе. Ахъ! чадо Мое любезное, чадо заблуждшее, возвращается ко Мнѣ; сынъ Мой, отлучившійся отъ Мене, паки идетъ ко Мнѣ: \textit{сынъ Мой мертвъ бѣ, и оживе: и изгиблъ бѣ, и обрѣтеся. Милуя помилую его}\footnote{Іер.~31,~20.}. Изыду въ срѣтеніе ему, объиму его руками Моими, и облобызаю его. Воистину, \textit{еще далече тебѣ сущу, узритъ тебе Отецъ твой, и милъ Ему будеши, и текъ нападетъ на выю твою, и облобызаетъ тебе}. И введетъ тебе въ домъ Свой и подастъ тебѣ \textit{одежду первую}, и въ тую \textit{облечетъ} тебе, и \textit{перстень на руку твою, и сапоги на ноги} твои подастъ; и повелитъ всей святой Своей фамиліи \textit{радоватися и веселитися о тебѣ}, глаголя: \textit{яко сынъ Мой сей мертвъ бѣ, и оживе: и изгиблъ бѣ, и обрѣтеся}; и начнутъ веселитися о тебѣ\footnote{Лук.~15,~13--24.}. \textit{Радость бо бываетъ предъ ангелы Божіими о единомъ грѣшникѣ кающемся}\footnote{10.}. Чимъ же усерднѣе будешь каятися и исправлятися, и въ лучшее премѣнятися, тѣмъ поспѣшнѣе къ Отцу небесному будешь ити и приближатися: ибо къ Нему не ногами, но сердцами, не премѣненіемъ мѣста, но премѣненіемъ воли и нравовъ приходимъ. Како бо грѣхами отлучаемся отъ Него и удаляемся: тако истиннымъ покаяніемъ, исправленіемъ и премѣненіемъ воли нашея и нравовъ въ лучшее, къ Нему приближаемся. Обратися и ты, душе моя, къ Отцу твоему небесному, и получиши у Него покой и блаженство свое. \textit{Колико наемникомъ Отца моего избываютъ хлѣбы: азъ же гладомъ гиблю! Воставъ иду ко Отцу моему, и реку Ему: Отче, согрѣшихъ на небо и предъ Тобою, и уже нѣсмь достоинъ нарещися сынъ Твой: сотвори мя, яко единаго отъ наемникъ Твоихъ!}

\section{67. И мы туды пойдемъ.}

Бываетъ, что когда люди въ какое мѣсто или градъ, или на какое дѣло пойдутъ, другій домашнимъ ихъ, или сосѣдамъ говорятъ тако: ваши"=де туды"=то пошли, или на такое"=то дѣло; тіи отвѣщаютъ имъ тако: вотъ"=де \textit{и мы туды пойдемъ}. Хрістіанине! и намъ сіе слово: \textit{и мы туды пойдемъ}, въ разсужденіи отшествія нашего отъ міра, приличествуетъ. Братія и предки наши отшли отъ міра сего, и на оный вѣкъ пошли: и мы туды пойдемъ. Оставили они міръ: оставимъ и мы. Оставили они утѣхи свои: оставимъ и мы. Оставили они сродниковъ и друговъ своихъ: оставимъ и мы. Оставили они домы и весь драгой уборъ свой: оставимъ и мы. Оставили они вотчины и деревни своя: оставимъ и мы. Оставили они злато, сребро и все свое богатство: оставимъ и мы. Оставили они саны свои, титулы, имена и всю славу свою: оставимъ и мы. Оставили они всѣ роскоши свои: оставимъ и мы. Словомъ, все они оставили: и мы оставимъ все. Ничего они съ собою не вынесли: не вынесемъ и мы ничего. Наги они отошли отъ міра сего: наги отыдемъ и мы, и отыдемъ вскорѣ. "--- Почто жъ много труждаемся и собираемъ? Пошли они къ праведному Судіи приняти по дѣломъ своимъ: пойдемъ и мы. Ахъ! пойдемъ къ Судіи всѣхъ Богу, Который смотритъ не на лица, но на совѣсть и на дѣла, у Котораго цари и подданныи ихъ, господа и раби ихъ, славныи и подлыи, богатыи и убогіи, князи и земледѣльцы, равны суть. Къ такому Судіи на оный вѣкъ пошли предки и братія наша, хрістіанине: \textit{пойдемъ и мы туды}. Высокопочтенныи цари! предки и братія ваша пошли туды пріяти жребій свой: пойдете и вы туды, и станете предъ Судіею онымъ съ подданными вашими, и отдадите отвѣтъ Ему, како людей Его, вамъ порученныхъ, управляли. Вельможи, князи и господа! братія и предки ваши пошли туды воспріяти мзду свою: пойдете и вы туды, и станете рядомъ съ слугами и крестьянами вашими предъ Судіею онымъ, и воздадите слово, како вы съ ними поступали. Судіи! предки и братія ваша пошли туды, праведному Судіи явитися, отвѣтъ дати: пойдете и вы, и будете истязаны отъ Него, како людей Его судили вы. Пастыри (епископы и іереи)! братія и предки ваши пошли на оный вѣкъ, пріяти по дѣламъ и трудамъ своимъ: пойдете, возлюбленніи, и вы, и спроситъ у васъ Хрістосъ, пастырей Начальникъ, душъ хрістіанскихъ, которыхъ стяжалъ кровію Своею. Богатыи! братія ваша пошли на оный вѣкъ дати отвѣтъ, како и на что богатство имъ данное держали: пойдете и вы, и о всемъ вашемъ имѣніи воздадите отвѣтъ. Хрістіанине! вси туды идутъ, и никто оттуду не возвращается: пойдемъ и мы, и не возвратимся. Отшедшіи туды вси въ своихъ мѣстахъ находятся, и ждутъ общаго воскресенія и послѣдняго суда: пойдемъ и мы, и будемъ всякъ въ своемъ мѣстѣ. Покаемся убо, возлюбленне, и исправимъ себе, и заранѣе приготовимъ себе къ часу смертному, да съ доброю надеждою отыдемъ отсюду, и у Хріста Господа получимъ милость, и тако будемъ на мѣстѣ покоя, гдѣ праведніи и святіи Его упокоеваются. \textit{Буди, Господи, милость Твоя на насъ, якоже уповахомъ на Тя}\footnote{Пс.~32,~22.}.

\section{68. Весна.}

Весна есть образъ и знаменіе воскресенія мертвыхъ. Что во время весны дѣлается, тое будетъ и въ воскресеніе мертвыхъ, какъ видимъ въ Святомъ Писаніи, и вѣруемъ. Во время весны вся поднебесная тварь обновляется: тако во время воскресенія все обновится, по неложному Божію обѣщанію. \textit{Се нова вся творю}, глаголетъ Господь\footnote{Апок.~21,~5.}. \textit{Будетъ небо ново и земля нова}, и проч.\footnote{Ис.~65,~17.} \textit{Нова небесе и новы земли по обѣтованію Его чаемъ, въ нихже правда живетъ}\footnote{2~Петр.~3,~13.}. Во время весны вся тварь, какъ видимъ, оживляется: тако въ воскресеніе мертвыхъ все человѣческое естество оживится. Во время весны всякая трава и зеліе исходитъ изъ нѣдръ земли, и является въ своемъ видѣ: тако въ послѣдній день умершій человѣки изыдутъ изъ гробовъ своихъ, и явится всякъ въ своемъ образѣ. \textit{Яко грядетъ часъ, въ оньже вси сущіи во гробѣхъ услышатъ гласъ Сына Божія, и изыдутъ сотворшіи благая, въ воскрешеніе живота, а сотворшіи злая, въ воскрешеніе суда}\footnote{Іоан.~5,~28 и 29.}. Въ зимѣ древеса и травы показуются аки изсохшіи; но во время весны живность ихъ покажется: тако умершіи невѣдущимъ воскресенія мертвыхъ показуются какъ бы погибшіи; но насъ вѣра святая научаетъ, что животъ ихъ явится въ послѣдній день. Во время зимы и сухія и суровыя древа одинаковы являются; но во время весны различіе ихъ познается: тако въ нынѣшнемъ вѣкѣ и въ самой смерти праведніи и нечестивіи равенъ внѣшній видъ имѣютъ: но во время воскресенія мертвыхъ великое различіе между ими покажется. Суровыя древеса и травы во время весны одѣваются листвіемъ и различными цвѣтами: тако въ воскресеніи мертвыхъ, благочестивыхъ и праведныхъ тѣлеса преобразятся въ новый, свѣтлый, благопріятный и прекрасный видъ, по неложному обѣщанію Божію. \textit{Наше житіе на небесѣхъ есть, отонудуже и Спасителя ждемъ, Господа нашего Іисуса Хріста, Иже преобразитъ тѣло смиренія нашего, яко быти сему сообразну тѣлу славы Его}\footnote{Филип.~3,~20 и 21.}. \textit{Возлюбленніи! нынѣ чада Божія есмы, и не у явися что будемъ: вѣмы же, яко, егда явится, подобни Ему будемъ, и узримъ Его якоже есть}\footnote{1~Іоан.~3,~2.}. \textit{Тогда праведницы просвѣтятся яко солнце во царствіи Отца ихъ}\footnote{Мѳ.~13,~43.}. Сухія древа и травы, какъ во время зимы, такъ и во время весны, непремѣнный и равный видъ имѣютъ, то"=есть, непріятный и черный: тако нечестивіи и безбожніи люди какъ нынѣ, такъ и тогда непріятный, паче же скареднѣйшій видъ возъимѣютъ. "--- Хрістіанине! видишь въ веснѣ образъ и знаменіе воскресенія мертвыхъ; и какое праведныхъ, и какое грѣшныхъ воскресеніе будетъ, видишь (тогда бо вси по дѣломъ своимъ воспріимутъ): живи нынѣ тако, какъ хощешь тогда воскреснуть. \textit{Чаешь воскресенія мертвыхъ, и жизни будущаго вѣка}: живи не противно, но достойно того. Хощешь съ благочестивыми, праведными и святыми тогда участіе имѣть: живи нынѣ благочестно, свято и праведно. Что нынѣ внутрь у человѣка имѣется, тое тогда внѣ явится. Имѣется ли благочестіе? тогда оно покажется. Имѣется ли нечестіе и лицемѣріе? тогда оно въ явленіе пріидетъ. Тогда \textit{явлена будутъ чада Божія, и чада діаволя}: тогда \textit{раздѣлятся овцы отъ козлищъ}\footnote{1~Іоан.~3,~10; Мѳ.~25,~32 и 33.}. Аще убо хощемъ, возлюбленне, постигнути въ воскресеніе праведныхъ, должно намъ нынѣ духовно воскреснуть со Хрістомъ воскресшимъ, \textit{да якоже воста Хрістосъ отъ мертвыхъ славою Отчею, тако и мы во обновленіи жизни ходити начнемъ}\footnote{Рим.~6,~4.}. \textit{Востани спяй, и воскресни отъ мертвыхъ, и освѣтитъ тя Хрістосъ}\footnote{Еф.~5,~14.}. \textit{Блаженъ и святъ, иже имать часть въ воскресеніи первомъ}\footnote{Апок.~2,~6.}.

\section{69. Востани.}

Бываетъ, что человѣкъ человѣка лѣниваго возбуждаетъ и говоритъ ему тако: \textit{востани}, пора итить на дѣло! Тако спасительная Божія благодать возбуждаетъ хрістіанина лѣниваго и нерадящаго о своемъ спасеніи: \textit{востани}, и иди на дѣло, на которое призвалъ тебе Господь. Всякъ бо грѣшникъ нерадивый и некающійся лежитъ, и \textit{со страхомъ и трепетомъ не содѣловаетъ своего спасенія: не трудится въ віноградѣ Господнемъ}\footnote{Филип.~3,~12; Мѳ.~20,~1, и проч.; 21,~33.}. Лежитъ блудникъ и прелюбодѣй; лежитъ тать, хищникъ и грабитель; лежитъ злобный и человѣконенавистникъ; лежитъ клеветникъ, ругатель, поноситель и уничижающій ближняго своего, и языкомъ своимъ, какъ оружіемъ, его біющій; лежитъ піяница и сластолюбецъ; словомъ, лежитъ всякъ грѣшникъ неисправный и некающійся. Къ таковому всякому благодать Божія приходитъ, и ударяетъ его, и глаголетъ ему: \textit{востани!} пора итить на дѣло Господне, пора каяться, пора трудиться въ дѣлѣ спасенія. \textit{Востани!} время уходитъ, и не возвратится; житіе твое сокращается; смерть нечаянно приходитъ; сатана, какъ хитрый тать, спасеніе твое крадетъ. Востани убо, и отряси сонъ отъ очей твоихъ. \textit{Востани спяй и воскресни отъ мертвыхъ, и освѣтитъ тя Хрістосъ}\footnote{Еф.~5,~14.}. \textit{Востани!} Богъ, яко благоутробный и милосердый Отецъ, какъ блуднаго сына, ожидаетъ тебе къ Себѣ. \textit{Егда возвратишися воздохнеши, тогда спасешися, и уразумѣеши, гдѣ еси былъ}\footnote{Ис.~30,~15.}. \textit{Востани!} Хрістосъ Господь ради тебе пострадалъ и умеръ; Кровь Хрістова ради тебе изліяна; не сребромъ или златомъ, ни иною какою цѣною, но дражайшею Кровію Хрістовою спасеніе тебѣ сыскано. Хощеши ли такъ дорогое сокровище нерадѣніемъ потерять, и вринуть себе въ вѣчную власть діавольскую, и въ вѣчное мученіе, отъ котораго Хрістосъ кровію Своею избавилъ тебе? Помяни, что смерть и кровь Хрістова грѣшнику некающемуся не пользуетъ, якоже больному лѣкарство, который самъ о себѣ нерадитъ. \textit{Востани!} Се Господь грозитъ казнію на нечестивыхъ и беззаконныхъ: да не и тебе постигнетъ казнь. \textit{Аще сокрыются во адѣ, то и оттуду рука Моя исторгнетъ я; и аще взыдутъ на небо, то и оттуду свергу я. И аще скрыются на версѣ Кармила, то и оттуду взыщу, и возму я; и аще погрузятся отъ очію Моею во глубинахъ морскихъ, то и тамо повелю зміеви, и угрызетъ я; и аще пойдутъ въ плѣнъ предъ лицемъ врагъ своихъ, и тамо повелю оружію, да избіетъ ихъ: и утвержу очи Мои на нихъ во злая, а не во благая}\footnote{Амос.~9,~2--4.}. \textit{Аще не обратитеся, оружіе свое очиститъ, лукъ свой напряже, и уготова и; и въ немъ уготова сосуды смертныя, стрѣлы своя сгораемымъ содѣла}\footnote{Пс.~7,~13 и 14.}. \textit{Изліется кровь ихъ яко персть, и плоти ихъ яко лайна; и сребро и злато ихъ не возможетъ изъяти ихъ въ день гнѣва Господня, и огнемъ рвенія Его поядена будетъ вся земля}\footnote{Соф.~1,~17 и 18.}. Человѣче бѣдный! покайся, да избѣжиши казни Божія. Прещеніе Божіе не суетно есть; сбудется, когда грѣшники не покаются. \textit{Востани!} Се житіе твое сокращается, и на всякій день нѣкая часть его отнимается; а смерть наступаетъ, и, какъ скорый вѣстникъ, къ тебѣ пріидетъ; и уже ближае сего дня она къ тебѣ, нежели вчера и третьяго дня была; и въ чемъ тебе застанетъ, въ томъ на оный вѣкъ пошлетъ. Берегись же, да не застанетъ тебе въ нераскаяніи и въ неисправности! Она нечаянно, какъ тать, приходитъ, и тогда восхищаетъ людей, когда не чаютъ, и тамо, гдѣ не чаютъ, и такъ, какъ не чаютъ. Берегись же, и на всякій день и часъ ее ожидай; и воставши трезвися, бодрствуй и буди таковъ, каковъ въ часъ ея хощешь быть. Помяни, гдѣ пожившій во грѣхахъ, въ роскошахъ, въ сластолюбіи, въ веселостяхъ міра, въ чести, славѣ и богатствѣ? Умерли они; отошло отъ нихъ и угодіе ихъ; и имѣются нынѣ на своихъ мѣстахъ и ожидаютъ всемірнаго суда Хрістова. \textit{Востани!} Се Хрістосъ Господь пріидетъ на судъ, и всѣхъ на судъ Свой позоветъ, \textit{да пріиметъ кійждо яже съ тѣломъ содѣла, или блага, или зла}\footnote{2~Кор.~5,~10.}. Позовешися и ты на судъ оный, и воздаси слово за всякое дѣло, слово и помышленіе богопротивное. Тамо на двѣ части раздѣлятся вси люди, и станутъ одни по правую сторону Судіи, а другіи по лѣвую. Тіи услышатъ отъ праведнаго Судіи: \textit{пріидите благословенніи Отца Моего, наслѣдуйте уготованное вамъ царствіе отъ сложенія міра}. Сіи послышатъ: \textit{идите отъ Мене проклятіи во огнь вѣчный, уготованный діаволу и аггеломъ его}\footnote{Мѳ.~25,~34 и 41.}. Блаженніи, который услышатъ: \textit{пріидите}! Окаянніи и бѣдніи, которымъ скажется: \textit{идите отъ Мене! "--- Востани} убо, бѣдный грѣшникъ, и со \textit{страхомъ и трепетомъ содѣловай спасеніе свое}, да сподобишися тогда быти въ числѣ блаженныхъ. \textit{Востани!} Мука вѣчная небрежливыхъ и нераскаянныхъ грѣшниковъ ожидаетъ! Чего и тебѣ ожидать, аще не востанешь, и съ сокрушеніемъ сердца не будешь каятися? Нынѣ всякому кающемуся милосердіе Божіе обѣщается и дѣлается: тогда всякое милосердіе престанетъ; тогда услышится реченное слово: \textit{чадо! помяни, яко воспріялъ еси благая въ животѣ твоемъ}\footnote{Лук.~16,~25.}. Будешь слышать слово сіе и ты; будешь страдать безъ конца; будешь всегда умирать, и никогда не умрешь; будешь жить и умирать; будешь горѣть, и никогда не сгоришь; будешь вкушать горести вѣчныя смерти. Огнь мучащій тя не угаснетъ, червь снѣдающій тя не умретъ, аще не покаешися. Воставъ убо покайся, и съ плачемъ и слезами обратися ко Господу Богу твоему, да помилуетъ тебе, да не впадеши въ безконечное оное зло. \textit{Востани!} Вотъ праведнымъ, святымъ, и всѣмъ грѣшникамъ сокрушеннымъ сердцемъ кающимся, обѣщается отъ Бога вѣчная жизнь, вѣчная радость и вѣчное блаженство! Будутъ съ Богомъ во вѣки; будутъ Бога лицемъ къ лицу видѣть, и тѣмъ, какъ пресладкимъ питіемъ, наслаждаться во вѣки; будутъ утѣшаться во вѣки, будутъ радоваться и ликовати во вѣки; будутъ, яко царіе и господіе, царствовать во вѣки. Въ вѣчную оную жизнь никакій грѣшникъ не внидетъ, аще сокрушеніемъ сердца и горячими слезами не омыется и не очистится, аще вѣрою \textit{ризъ своихъ не убѣлитъ въ крови Агнчей}\footnote{Апок.~7,~14.}. Ибо \textit{внѣ псы и чародѣи, и любодѣи, и убійцы, и идолослужители, и всякъ любяй и творяй лжу}\footnote{22,~15.}. \textit{Востани} убо, и кайся, и трудись въ дѣлѣ спасенія своего, да не лишишися блаженства онаго! \textit{Днесь аще гласъ Его услышите, не ожесточите сердецъ вашихъ, яко въ прогнѣваніи, по дни искушенія въ пустыни: въ оньже искусиша Мя отцы ваши, искусиша Мя, и видѣша дѣла Моя. Четыредесять лѣтъ негодовахъ рода того, и рѣхъ: присно заблуждаютъ сердцемъ, тіи же не познаша путей Моихъ: яко кляхся во гнѣвѣ Моемъ, аще внидутъ въ покой мой}\footnote{Пс.~94,~8--11.}. \textit{Споспѣшествующе же и молимъ, не вотще благодать Божію пріяти вамъ. Глаголетъ бо: во время пріятно послушахъ тебе, и въ день спасенія помогохъ ти. Се нынѣ время благопріятно! се нынѣ день спасенія}\footnote{2~Кор.~6,~1 и 2.}.

\section{70. Скотъ.}

Какое въ скотѣ и въ звѣрѣ примѣчается злонравіе, таковое имѣется и въ человѣкѣ, благодатію Божіею не отрожденномъ и не обновленномъ. Въ скотѣ видимъ самолюбіе; видимъ, како самъ единъ пищу хощетъ пожирать, со скоростію хватаетъ ее и пожираетъ, прочій скотъ не допущаетъ и отгоняетъ, и проч: тоежде есть и въ человѣкѣ. Самъ обиды не терпитъ, но прочихъ обижаетъ; самъ презрѣнія не терпитъ, но прочихъ презираетъ; самъ о себѣ клеветы слышать не хощетъ, но на другихъ клевещетъ; не хощетъ, чтобы имѣніе его похищено было, но самъ чужое похищаетъ; хощетъ, чтобы кто въ нуждѣ ему помогъ, но самъ другимъ въ нуждѣ не помогаетъ; хощетъ, чтобы алченъ напитанъ былъ, нагъ одѣянъ былъ, страненъ въ домъ введенъ былъ, и проч., но самъ другимъ не дѣлаетъ того. Не хощетъ себѣ никакого зла, но желаетъ всякаго добра, а другимъ дѣлаетъ зло, и никакого не дѣлаетъ добра. Хощетъ у себе все имѣть; но ближній его хотя и ничего не имѣетъ, о томъ нерадитъ. Словомъ: хощетъ самъ во всякомъ благополучіи быть, и злополучія убѣгаетъ; но о другихъ, подобныхъ себѣ человѣкахъ, небрежетъ. Се есть скотское и премерзкое самолюбіе! "--- Въ скотѣ примѣчается гордость: таяжде видится и въ человѣкѣ. Видимъ, како бѣдный человѣкъ другихъ уничижаетъ, себе возноситъ; другихъ презираетъ, себе прославляетъ; другихъ обвиняетъ, себе извиняетъ; другихъ осуждаетъ, себе оправдаетъ; другихъ злословитъ и хулитъ, себе хвалитъ; видимъ, како вездѣ первенства ищетъ, другими владѣть, надъ другими господствовать, другимъ повелѣвать, отъ всякаго почтеніе имѣть. Что значитъ вымыслъ красныхъ и златотканныхъ одѣяній, богатыхъ домовъ, высокихъ каретъ и дорогихъ коней, богатыхъ трапезъ, многоразличными снѣдьми и питіемъ наполненныхъ, и прочія суеты и пышности, "--- что, говорю, сіе значитъ, аще не гордость, въ сердцѣ человѣческомъ крыющуюся, которая во всемъ и во всякой вещи ищетъ себѣ прославленія? Гордость вездѣ и всѣмъ ищетъ себе показать: смиреніе Богу и человѣкамъ любезное, сокрыться. Видишь, человѣче, гордость человѣческую: но знай, что, чимъ мерзостнѣйшій есть предъ Богомъ порокъ, тѣмъ сокровеннѣйшій есть, и мало отъ кого познается, но только отъ тѣхъ, которыи со всякимъ прилѣжаніемъ себе разсматриваютъ, и въ различныхъ находятся искушеніяхъ. Почто земля и пепелъ и сѣнь смертная гордится? "--- Видимъ въ скотѣ гнѣвъ и ярость: тоежде имѣется и въ человѣкѣ. Видимъ, какъ гнѣвается, какъ негодуетъ, ропщетъ, ярится, какъ себе терзаетъ, весь отъ гнѣва, какъ листъ отъ вѣтра, трясется, какъ себе заклинаетъ, какъ злословитъ, хулитъ и часто страшныя хулы извергаетъ, какъ обидѣвшему грозитъ, како на него злобится и ищетъ всякимъ способомъ ему отмстить. Се есть дѣйствіе богомерзкаго и пагубнаго гнѣва, въ сердцѣ у человѣка крыющагося! Видимъ, что скоты дерутся: видимъ тое и въ человѣкахъ; видимъ, сколько ссоръ и кровопролитій различныхъ между бѣдными людьми имѣется. Въ скотѣ примѣчается хитрость и лукавство: тоежде пагубное зло имѣетъ и человѣкъ въ себѣ; видимъ, како притворяетъ себе, како показуется быть добрымъ внутрь золъ; како хитритъ, лукавствуетъ, льститъ, лжетъ, обманываетъ, лицемѣритъ. Въ скотахъ примѣчается хищеніе: тоежде видится и въ человѣкѣ; и сколько явныхъ, тайныхъ и лестныхъ хищенія способовъ вымышляетъ человѣкъ, исчислить не возможно, такъ что мало кто отъ того свободенъ, а паче купцы, мастеровые люди, приказные служители, господа, помѣщики, которыи крестьянъ имѣютъ, казначеи, богатыи, славныи, и прочіи, міръ и богатство, славу и честь любящіи. Въ скотахъ примѣчается обжирство: тоежде есть и въ человѣкѣ. Въ скотахъ видится нечистота: тоежде видимъ, и въ человѣкѣ; видимъ какъ бѣдный человѣкъ, якоже свинья въ грязи, въ сластолюбіи валяется. Въ скотахъ примѣчается лѣность: видимъ тое и въ человѣкѣ; видимъ како нерадитъ и небрежетъ о себѣ самомъ, и о своемъ спасеніи, и прочее. Но что горше того и болѣе умножаетъ бѣдность и окаянство человѣческое, "--- которые злые нравы въ различныхъ звѣряхъ и скотахъ находятся, тѣ вси въ единомъ человѣкѣ необновленномъ имѣются. Безсловесные "--- иніи горды, но иніи смиренныи; иніи хитры и лукавы, но иніи просты; иніи сердиты и гнѣвливы, но иніи кротки; иніи лѣнивы, но иніи бодры, и прочая. А въ грѣшникѣ бѣдномъ всякое злонравіе скотское имѣется: гордъ, гнѣвливъ, хищный, лѣнивъ, хитръ, лукавъ, нечистъ, и прочая. О, какъ смертоносный ядъ, какъ великое и пагубное зло, въ сердцѣ человѣческомъ крыется! какъ бѣденъ и окаяненъ человѣкъ! Внѣшній видъ человѣка, какъ созданъ, показуетъ: но внутрь истинный есть скотъ и звѣрь, паче же горшій скота и звѣря; ибо не единаго скота какого и звѣря, но всѣхъ скотовъ и звѣрей злонравіе имѣетъ въ себѣ. О бѣдный человѣкъ! въ какое ты состояніе пришелъ, согрѣшивши Господу и Создателю твоему, послушавши совѣта лукаваго врага твоего! О, созданіе Божіе высокопочтенное, созданіе, образомъ Божіимъ и подобіемъ удобренное, въ какую подлость и гнусность впало! Гдѣ твоя первосозданная доброта и благолѣпіе? Гдѣ твоя святость, чистота, правда и непорочность? Гдѣ твой Божественный свѣтъ, который душу и разумъ твой просвѣщалъ? Гдѣ твое блаженство богоподобное, которое Создатель твой въ началѣ тебѣ даровалъ? Все тое отошло отъ тебе! Всего того лишилъ тебе грѣхъ твой: лишилъ блаженства, и вринулъ во всякое бѣдствіе и окаянство; лишилъ свѣта прелюбезнаго, и вринулъ въ пагубную тьму; лишилъ Божественнаго образа и подобія, и изобразилъ скотское безобразіе; былъ сообразенъ и подобенъ Богу, но сдѣлался безобразенъ и подобенъ скотомъ. \textit{И человѣкъ въ чести сый не разумѣ, приложися скотомъ несмысленнымъ и уподобися имъ}\footnote{Пс.~48,~13 и 21.}. Былъ свободенъ, но сдѣлался плѣнникомъ; былъ святъ и чистъ, но сдѣлался оскверненъ и мерзокъ; былъ доброобразенъ и свѣтелъ, но остался безобразенъ и теменъ; былъ храмомъ Святаго Духа, но стался жилищемъ нечистыхъ духовъ. О лютаго паденія! о пренесчастливыя перемѣны! Лютое то время, день и часъ, въ который человѣкъ лукаваго зміина совѣта послушалъ, и заповѣдь Создателя своего преступилъ! \textit{Помяни, Господи, что бысть намъ: призри и виждь укоризну нашу. Разсыпася радость сердецъ нашихъ, обратися въ плачь ликъ нашъ, спаде вѣнецъ съ главы нашея: горе намъ, яко согрѣшихомъ! О семъ смутися сердце наше, о семъ померкнуша очи наши}\footnote{Пл. Іер.~5,~1,~15--17.}. Человѣче бѣдный, ты, который крестился во имя Святыя Троицы, и имя свое записалъ Іисусу Хрісту Сыну Божію! смотри, какая мерзость въ сердцѣ человѣческомъ крыется! Что тебѣ пользуетъ, что ты имя Божіе исповѣдуешь и часто взываешь: \textit{Господи! Господи!} а такой злый и скотскій нравъ въ себѣ имѣешь? Како можетъ внити въ царствіе Божіе толикую нечистоту имѣющій, аще отъ всего сердца къ Богу не обратится, и истиннымъ покаяніемъ, сокрушеніемъ сердца, теплыми слезами и благодатію Святаго Духа не обновится? \textit{Ибо внѣ псы и чародѣи, и любодѣи, и убійцы, и идолослужители и всякъ любяй и творяй лжу}\footnote{Апок.~22,~15.}. \textit{Не имать въ онь внити всякое скверно}\footnote{21,~27.}. \textit{Помяни насъ, Господи, во благоволеніи людей Твоихъ; посѣти насъ спасеніемъ Твоимъ, видѣти во благости избранныя Твоя, возвеселитися въ веселіи языка Твоего, хвалитися Съ достояніемъ Твоимъ. Согрѣшихомъ со отцы нашими, беззаконновахомъ, неправдовахомъ}\footnote{Пс.~105,~4--6.}.

\section{71. Обращеніе.}

Когда рабъ отъ господина своего отвратится и отходитъ отъ него; уже къ нему обращаетъ хребетъ, а не лице; и что говоритъ господинъ его, не слушаетъ и не внимаетъ тому: тако грѣшникъ бѣдный, когда отвратится отъ Бога любовію и послушаніемъ и обратится къ злымъ дѣламъ, какъ бы хребетъ Ему обращаетъ, и отходитъ отъ Него; и уже что Богъ говоритъ въ Словѣ Своемъ святомъ, не внимаетъ тому и небрежетъ о томъ. О семъ пророкъ написалъ Божію о беззаконныхъ жалобу: \textit{обратиша, ко Мнѣ хребты, а не лица своя}\footnote{Іер.~2,~17.}. И паки пророкъ глаголетъ на нечестивыхъ: \textit{и не предложиша Бога предъ собою}\footnote{Пс.~53,~5.}. Се есть премерзкое отвращеніе и отступленіе отъ Бога! Не токмо бо люди отступаютъ отъ Бога, когда имени Его святаго отрицаются и приступаютъ къ богамъ чуждымъ; но и тогда, когда воли Его не хотятъ творить и безстрашно беззаконнуютъ, якоже о таковыхъ написалъ Апостолъ: \textit{Бога исповѣдуютъ вѣдѣти, а дѣлы отмещутся Его, мерзцы суще и непокориви, и на всякое дѣло благое неискусни}\footnote{Тит.~1,~16.}. Разсуждай сіе, хрістіанине, который имя Хрістово исповѣдуешь, но не по"=хрістіански живешь! Знай точно, что подъ именемъ хрістіанскимъ много имѣется язычниковъ, а паче въ нынѣшнее лютое время, въ которое болѣе почитается отъ хрістіанъ похоть плотская, похоть очесъ и гордость житейская, какъ троякій богъ, нежели Богъ, Творецъ неба и земли, всякаго человѣческаго почитанія и прославленія высшій. Когда рабъ обратится къ господину, и станетъ предъ нимъ; уже лицемъ къ нему стоитъ, и что господинъ ему говоритъ, слушаетъ и внимаетъ тому: тако грѣшникъ, когда очувствуется, и отъ грѣховъ къ Богу обращается, уже какъ бы лицемъ къ Нему стоитъ, и отдаетъ Ему почитаніе, яко Богу и Создателю своему, и внимаетъ святымъ Его устамъ; и что въ Писаніи Святомъ говоритъ, все тое исполнять тщится, и вездѣ, гдѣ ни обращается, душевными очами и вѣрою зритъ Бога предъ собою, и всякаго грѣха въ дѣлѣ, словѣ и помышленіи бережется, да не прогнѣваетъ Его. О семъ глаголетъ Пророкъ: \textit{предзрѣхъ Господа предо мною выну}\footnote{Пс.~15,~8.}. Се есть истинное обращеніе, истинное востаніе отъ грѣховнаго сна, истинное покаяніе, истинное воскресеніе душевное! Блаженъ человѣкъ, кто тако воскреснетъ и пребудетъ въ томъ до конца! Воистину таковый воскреснетъ въ послѣдній день не въ смерть, но въ животъ и блаженство вѣчное! \textit{Блаженъ и святъ, иже имать часть въ воскресеніи первомъ: на нихже смерть вторая не иметъ области}\footnote{Апок.~20,~6.}. Хрістіанине! потщися тако обратится ко Господу Богу твоему: и непремѣнно благодатію Хрістовою спасешися. \textit{Господи Боже силъ, обрати насъ, и просвѣти лице Твое, и спасемся}\footnote{Пс.~79,~8.}. \textit{Не помяни нашихъ беззаконій первыхъ; скоро да предварятъ ны щедроты Твоя, Господи, яко обнищахомъ зѣло. Помози намъ, Боже Спасителю нашъ, славы ради имене Твоего! Господи, избави ны, и очисти грѣхи наша имене ради Твоего}\footnote{78,~8--10.}.

\subsection{О томжде.}

Когда человѣкъ къ человѣку обращается и приходитъ предъ него, то тѣломъ обращается и приходитъ, хотя часто бываетъ, что сердце и духъ отвращенный имѣетъ. Къ Богу обращеніе не тако бываетъ. Къ Богу обращаемся и приходимъ не тѣломъ и ногами. Какъ бо и куды обратимся къ Тому, Который вездѣ и на всякомъ мѣстѣ есть, и куды хощемъ идти, прежде пришествія нашего тамо есть? Къ Богу убо, яко вездѣсущему, не мѣстомъ, тѣломъ и ногами, но сердцемъ и духомъ обращеніе бываетъ. Къ Богу обращаемся, когда познаемъ Его, что Онъ есть высочайшее наше добро и блаженство; и, оставивше прочія вещи созданныя, міръ и суетствіе его, и всякіе грѣхи, яко ему противные, Его единаго желаемъ и любимъ и ищемъ, и паче всѣхъ созданныхъ вещей почитаемъ; и всегда Его въ памяти имѣемъ, и опасаемся, чтобы Его чимъ не оскорбить. Се есть истинное къ Богу обращеніе! Какъ бо къ міру обращеніе бываетъ тогда, когда человѣкъ сердцемъ къ мірскимъ вещамъ, то"=есть, къ чести, славѣ, богатству, роскоши и всякой суетѣ обращается, прилѣпляется и ищетъ ихъ, яко любимаго своего сокровища; тако къ Богу обращеніе бываетъ, когда человѣкъ, все тое оставивши, единаго Бога любитъ, желаетъ, сердцемъ прилѣпляется Ему и ищетъ Его, яко высочайшаго добра. Что бо человѣкъ познаетъ и признаетъ за свое добро и блаженство, тое и любитъ; что любитъ, того и желаетъ; чего желаетъ, о томъ и мыслитъ всегда; о чемъ мыслитъ, того со усердіемъ и ищетъ. Ищеши ли чести, славы, богатства и прочей суеты въ мірѣ семъ? Знаменіе есть, что ты тое за добро свое и блаженство почитаешь, и любишь тое и желаешь; и подлинный знакъ есть, что ты сердцемъ отъ Бога и Создателя твоего отвратился, и обратился къ созданію Его, и почитаешь тое болѣе, нежели Создателя твоего. А когда все тое презрѣвши и оставивши, ищешь единаго Бога, и Его единаго желаешь пріобрѣсти и имѣть: знаменіе есть, что ты Его паче всего созданія почитаешь, и въ Немъ свое удовольствіе находишь, и крайнее свое добро и блаженство въ Немъ полагаешь. Тако, по словеси Хрістову, \textit{идѣже сокровище} человѣческое \textit{есть, тамо и сердце} его, тамо любовь его, тамо мысль его, тамо желанія его; о томъ думаетъ, тщится, ищетъ и разговариваетъ\footnote{Мѳ.~6,~21.}. Кто честь, богатство и славу міра сего, и все въ немъ содержащееся за сокровище имѣетъ: въ томъ у него и сердце съ своимъ желаніемъ и любовію. Кому Богъ единъ есть сокровище, тотъ Ему единому и прилѣпляется. О вѣчное сокровище! коль Ты драгое, неоцѣненное, ни съ чимъ несравненное, паче всего міра вожделѣннѣйшее, любезнѣйшее, сладчайшее любящимъ Тя душамъ! Въ Тебѣ они все находятъ, чего любители міра въ мірѣ ищутъ; все находятъ, но несравненно лучшее. Въ Тебѣ находятъ богатство, честь, славу, покой, миръ, утѣшеніе, радость, сладость, и несравненно большее блаженство, нежели весь міръ въ себѣ имѣетъ. Такъ дорогое и высокое добро есть Богъ, яко несозданное, присносущное и вѣчное, отъ Котораго, яко отъ источника, всякое добро проистекаетъ, какое ни есть и можетъ быть, но отъ любителей міра сего презирается и пренебрегается!... О бѣдный хрістіанине! поищешь сего добра нѣкогда, но не сыщешь, когда нынѣ его сердечно не поищешь. Обратися убо къ Нему нынѣ, и поищи Его, да сыщеши Его! \textit{Взыщите Господа, и утвердитеся, взыщите лица Его выну}\footnote{Пс.~104,~4.}. \textit{Услышу, что речетъ о мнѣ Господь Богъ: яко речетъ миръ на люди Своя и на преподобныя Своя, и на обращающія сердца къ Нему. Обаче близъ боящихся Его спасеніе Его}\footnote{84,~9 и 10.}. \textit{Что ми есть на небеси? и отъ Тебе что восхотѣхъ на земли? Исчезе сердце мое и плоть моя: Боже сердца моего, и часть моя Боже во вѣкъ! Яко се удаляющіи себе отъ Тебе погибнутъ: потребилъ еси всякаго любодѣющаго отъ Тебе. Мнѣ же прилѣплятися Богови благо есть}\footnote{72,~25--28.}. \textit{Возлюблю Тя, Господи, крѣпосте моя! Господь утвержденіе мое и прибѣжище мое, и избавитель мой, Богъ мой, помощникъ мой, и уповаю на Него, защититель мой и рогъ спасенія моего, и заступникъ мой}\footnote{Пс.~17,~2 и 3.}. Какъ"=де мнѣ, и гдѣ искать Бога? Отвѣтъ: 1)~Не желай и не ищи ничего, кромѣ Бога и воли Его святой, послѣдуя пророку, который ни на небеси, ни на земли, кромѣ Бога, ничего не хотѣлъ: \textit{что ми есть на небеси? и отъ Тебе что восхотѣхъ на земли?} и будешь искать Бога. 2)~Вездѣ и на всякомъ мѣстѣ ищется и обрѣтается тотъ, Который вездѣ есть; гдѣ ни имѣешься и живешь, близъ тебе Богъ есть, тамо ищи Его. Во градѣ живешь? тутъ Богъ есть; тутъ ищи его. На селѣ живешь? тутъ Богъ есть; тутъ ищи Его. Въ монастырѣ или въ пустыни живешь? Богъ съ тобою есть; тамо обрѣтается Богъ; тамо ищи Его. Словомъ: гдѣ ни пребываешь, близъ тебе есть Богъ. Ищи Его близъ тебе, какъ надлежитъ: и сыщешь Его. \textit{Се стою при дверехъ и толку: аще кто услышитъ гласъ Мой, и отверзетъ двери, вниду къ нему, и вечеряю съ нимъ, и той со Мною}\footnote{Апок.~3,~20.}. Вотъ самъ Богъ хощетъ къ намъ пріити, и Себе въ познаніе намъ подать! Онъ у всякаго при дверехъ стоитъ, и всякому хощетъ познатися: но мало кто слышитъ Его, толкающаго въ двери! Понеже у всякаго слухъ заглушенъ похотьми грѣховными и любовію міра. И тако потолкавши въ двери, и ничего не успѣвши, отходитъ празденъ отъ человѣка. Успокой убо и утиши умъ и сердце твое отъ похотей плотскихъ и шума мірскихъ вожделѣній; отъ всего того отвратись, и внимай Ему единому. Тогда воистину познаешь, что Онъ близъ тебе стоитъ и толкаетъ въ двери сердца твоего, и услышишь пресладкій гласъ Его, и отверзеши Ему двери. Тогда внидетъ въ домъ твой и будетъ вечерять съ тобою и ты съ Нимъ. Тогда \textit{вкусишь и увидишь, коль благъ Господь}\footnote{Пс.~33,~9.}. Тогда воззовеши и ты съ любовію и радостію: \textit{щедръ и милостивъ Господь, долготерпѣливъ и многомилостивъ и истиненъ}\footnote{Исх.~34,~6.}. И паки: \textit{возлюблю Тя, Господи, крѣпосте моя}, и проч. И паки: \textit{что ми есть на небеси? и отъ Тебе что восхотѣхъ на земли?} и проч. Ищи убо того вездѣ, Который вездѣ есть; и, оставивши все, \textit{единаго} Его ищи: и тако неотмѣнно сыщеши.

\section{72. Подражаніе.}

Видимъ въ мірѣ, что подданные царю, раби господину, дѣти отцу, ученики учителю, и прочіе люди другимъ людямъ подражаютъ. Что бо одинъ дѣлаетъ, то и другій, то и третій, и вси дѣлаютъ. Одинъ сдѣлалъ такіе"=то покои, и другій и третій дѣлаетъ такіежде; одинъ началъ въ такомъ"=то платьѣ ходить, и прочіе ему послѣдуютъ въ томъ. Тако и въ прочемъ. Видишь, хрістіанине, что сыны вѣка сего дѣлаютъ, и какъ другъ другу подражаютъ. Намъ что дѣлать? кому подражать? Мы, хрістіане, отъ Хріста: кому же и подражать, какъ не Хрісту, отъ Котораго имя свое имѣемъ; Онъ нашъ Царь: мы подданныи Его. Онъ нашъ Господь: мы недостойныи раби Его. Онъ нашъ Отецъ, отъ Котораго вновь родилися мы: мы недостойныи чада Его. Онъ нашъ Учитель: мы недостойныи ученики Его. Комужъ убо подражать намъ, какъ не Ему? Пусть сыны вѣка сего другъ другу подражаютъ: мы положимъ предъ собою Хріста и святое житіе Его, и подражаемъ Тому, якоже Самъ намъ Себе во образъ и подражаніе представляетъ: \textit{образъ дахъ вамъ, да якоже Азъ сотворихъ вамъ, и вы творите}\footnote{Іоан.~13,~5.}. Былъ Онъ смиренъ, кротокъ, тихъ, терпѣливъ, милостивъ, милосердъ, нищъ такъ, что не имѣлъ гдѣ главы подклонити, якоже Самъ о Себѣ свидѣтельствуетъ: \textit{лиси язвины имутъ, и птицы небесныя гнѣзда: Сынъ же человѣческій не имать гдѣ главы подклонити}\footnote{Мѳ.~8,~20.}. Предложимъ сіи и прочія прекрасныя Его добродѣтели предъ умными нашими очами, и потщимся по силѣ подражать имъ. Былъ онъ смиренъ: будемъ и мы смиренни предъ Богомъ и человѣки. Любилъ Онъ всѣхъ такъ, что и душу Свою за всѣхъ положилъ: возлюбимъ и мы другъ друга, и Его Самого во первыхъ. Былъ Онъ кротокъ ко всѣмъ хулящимъ Его: будемъ и мы кротки къ поношающимъ и укоряющимъ насъ. Былъ Онъ долготерпѣливъ во всѣхъ страданіяхъ Своихъ: претерпимъ и мы все, что намъ ни приключится противное. Презрѣлъ Онъ всю міра сего славу, честь и богатство, Господь сый всѣхъ, Егоже \textit{земля и исполненіе ея}\footnote{Пс.~23,~1.}: презримъ и мы, и возлюбимъ вѣчная благая. Видимъ, что люди подражаютъ другимъ ради различныхъ причинъ: иніи подражаютъ во угодность тѣмъ, коимъ поражаютъ, иніи въ свое удовольствіе и угожденіе, иніи ради своей пользы. Намъ, хрістіанине, для всѣхъ сихъ причинъ должно Хрісту Господу нашему подражать. Угодно сіе есть Хрісту, когда Ему подражаемъ: ибо Онъ хощетъ сего отъ насъ, не такъ ради Себе, какъ ради насъ. Угодно сіе и намъ, и благопріятно: что бо любезнѣе, благопріятнѣе и сладчае, какъ небесному Царю смиренному, насъ ради обнищавшему, кроткому, долготерпѣливому, многомилостивому, преблагому, щедрому, Агнцу Божію незлобивому, непорочному, пречистому послѣдовать? Весьма намъ и полезно сіе. Самъ разсуди, что полезнѣе, какъ Хрісту послѣдовать: Онъ есть Свѣтъ міру; когда въ слѣдъ Его будемъ ходить, не пребудемъ во тьмѣ, но во свѣтѣ. \textit{Азъ есмь свѣтъ міру: ходяй по Мнѣ, не имать ходити во тмѣ, но имать свѣтъ животный}\footnote{Іоан.~8,~12.}. \textit{Онъ есть путь, и истина, и животъ}\footnote{14,~6.}. Когда пути сего будемъ держаться, подлинно не заблудимъ; когда истинѣ сей будемъ послѣдовать, не прельстимся отъ злаго и прелестнаго міра; когда при животѣ семъ пребудемъ, не умремъ, но живы будемъ во вѣки. Видишь, хрістіанине, что не токмо полезно, но и нужно Хрісту послѣдовать. Мы, бѣдныи, какъ овцы заблуждшіи въ мірѣ семъ есмы; не знаемъ пути къ небесному отечеству. Поручимъ себе въ предводительство Спасителю нашему Іисусу Хрісту, и пойдемъ въ слѣдъ Его, и приведетъ насъ къ небесному Своему Отцу, въ вѣчное Его царство, якоже глаголетъ: \textit{никтоже пріидетъ ко Отцу, токмо Мною}\footnote{Іоан.~14,~6.}. Хрістіанине! аще боишися вѣчно погибнуть, и сердечно желаешь вѣчно спастися, вѣруй во Хріста, и смиренно живи на земли, какъ и Онъ жилъ: и спасешися. Чего и себѣ и тебѣ желаю.

\section{73. Путь безопасный.}

Видимъ въ мірѣ семъ, что люди съ мѣста на мѣсто переходятъ, и бываетъ путь иный опасный, иный безопасный. Хрістіанине! мы отъ рожденія нашего до смерти путники есмы, и тщимся дойти къ вѣчному животу и царствію Божію, къ которому крещеніемъ святымъ обновлены и словомъ Божіимъ позваны. Изберемъ убо, возлюбленне, безопасный путь ко оному блаженству. Какій есть безопасный путь намъ? Сердечно вѣровать во Хріста, и въ единомъ страданіи и смерти Его полагать свое спасеніе; на Его единаго, яко Искупителя и Спасителя, надѣяться; кромѣ Его, къ полученію вѣчнаго спасенія, посредствія не знать и по примѣру Его, смиренно, кротко и терпѣливо жить на земли. Сей есть безопасный путь. \textit{Настави мя, Господи, на путь сей!} Аще же ищешь въ мірѣ семъ чести, славы, богатства, плотоугодія: знай точно, что ты заблудилъ отъ безопаснаго сего пути и идешь путемъ опаснымъ. Аще не вѣруешь сему, прочитай святое Божіе слово и прочія хрістіанскія книги съ прилѣжаніемъ, и самъ узнаешь истину сію. Хрістіане въ мірѣ семъ позваны суть къ крестоношенію, а не къ чести, славѣ, богатству, плотоугодію. Имѣютъ они свою честь, славу, богатство, радость и сладость, но внутрь себе, а не внѣ, "--- духовное все, а не плотское. Внимай сему, возлюбленне, что вѣчная Истина, Хрістосъ Господь нашъ, говорить: \textit{внидите узкими враты, яко пространная врата и широкій путь вводяй въ пагубу, и мнози суть входящіи имъ. Что узкая врата, и тѣсный путь вводяй въ животъ, и мало ихъ есть, иже обрѣтаютъ его}\footnote{Мѳ.~7,~13 и 14.}. Видишь, что есть безопасный путь. Сей путь низокъ и смиренъ есть; но въ высокое отечество, небо ведетъ. Аще хощеши во отечество пріити, симъ иди путемъ. \textit{Настави мя Господи на путь Твой, и пойду во истинѣ Твоей: да возвеселится сердце мое боятися имене Твоего}\footnote{Пс.~85,~11.}.

\section{74. Зеркало.}

Видимъ, что люди зеркала имѣютъ въ своихъ покояхъ ради потребы своей. Хрістіанине! что сынамъ вѣка сего зеркало, тое да будетъ намъ Евангеліе и непорочное житіе Хрістово. Они посматриваютъ въ зеркала, и исправляютъ тѣло свое и пороки на лицѣ очищаютъ. Сотворимъ и мы тако. Хрістосъ Сынъ Божій Себе и святое житіе Свое въ образъ намъ подалъ, какъ выше видѣли мы: \textit{образъ дахъ вамъ, да якоже Азъ сотворихъ вамъ, и вы творите}\footnote{Іоан.~13,~15.}. Предложимъ убо и мы предъ душевными нашими очами чистое сіе зеркало, и посмотримъ въ тое: сообразно ли наше житіе житію Хрістову? Надобно же неотмѣнно сообразнымъ быть. Якоже бо ветхому Адаму сотворилися мы сообразными: тако должны сотворитися сообразными и Новому, Іисусу Хрісту, аще хощемъ внити въ животъ вѣчный. И якоже уподобилися злонравіемъ первому оному человѣку: тако должно уподобитися добронравіемъ второму Человѣку, Господу съ небесе, Іисусу Хрісту, да тако будемъ вѣрою \textit{нова тварь}\footnote{2~Кор.~5,~17.}. \textit{Яковъ перстный, такови и перстніи: и яковъ Небесный, тацы же и небесніи. И якоже облекохомся во образъ перстнаго, да облечемся и во образъ Небеснаго}\footnote{1~Кор.~15,~48 и 49.}. Посматривай убо часто въ чистое непорочнаго житія Хрістова зеркало, и пороки на душѣ твоей прилѣпшія покаяніемъ и сокрушеніемъ сердца стирай, и, сколько силъ есть, Тому сообразуйся; сообразуйся Хрісту Сыну Божію нынѣ въ смиреніи, кротости, терпѣніи, любви и прочіихъ добродѣтеляхъ, да и тамо сообразенъ будеши, а нынѣшнимъ хрістіанамъ не сообразуйся. У нынѣшнихъ хрістіанъ по большей части житіе есть противное Хрісту, а не сообразное. Хрістосъ жилъ въ смиреніи: нынѣшніи хрістіане любятъ жить въ гордости и пышности. Хрістосъ искренно и просто со всѣми обходился: нынѣшніи хрістіане хитро и лукаво со всѣми обходятся. Хрістосъ жилъ въ нищетѣ великой такъ, что не имѣлъ гдѣ и главы подклонити, хотя и все моглъ имѣть, яко Господь всѣхъ: нынѣшніи хрістіане о томъ думаютъ и тщатся, какъ бы великое богатство собрать, и ежели бы можно было, всѣ бы сокровища міра у себе имѣли. Хрістосъ чести и славы убѣгалъ, всякаго почитанія и прославленія достоинъ: нынѣшніи хрістіане въ томъ всю силу и тщаніе свое полагаютъ, чтобы въ мірѣ семъ почтенными и прославленными быть. Хрістосъ жилъ въ любви и милости: нынѣшніи хрістіане другъ друга ненавидятъ, другъ на друга клевещутъ, другъ друга злословятъ, хулятъ, другъ у друга крадутъ, похищаютъ, отнимаютъ, другъ другу козни и сѣти соплетаютъ. Хрістосъ за обиду никому не отмщевалъ, хотя и моглъ всѣхъ враговъ Своихъ маніемъ единымъ поразить: нынѣшніи хрістіане и словомъ и дѣломъ другъ другу отмщеваютъ. Хрістосъ былъ хулителямъ Своимъ кротокъ, укоряемъ противу не укорялъ: нынѣшніи хрістіане другъ друга укоряютъ, и другъ друга языкомъ, какъ мечемъ острымъ, сѣкутъ. Хрістосъ въ великомъ терпѣніи жилъ: нынѣшніи хрістіане, хотя и малая какая напасть и бѣда приключится, ропщутъ и хулятъ. У Хріста едино сіе тщаніе было, чтобы волю небеснаго Своего Отца исполнить, и дѣло спасенія нашего совершить, "--- исполнилъ и совершилъ: у нынѣшнихъ хрістіанъ все иное; они о волѣ Божіей небрегутъ, но все тщаніе полагаютъ въ томъ, какъ бы міру угодить, и прихоти свои исполнить. Видишь, хрістіанине, что Хрістосъ въ мірѣ семъ дѣлалъ; видишь, что дѣлаютъ и хрістіане. Самъ убо познавай, какое есть нынѣшнихъ хрістіанъ хрістіанство! Видишь самъ, что оно ложное есть. Не сообразуйся убо нынѣшнимъ хрістіанамъ, яко всѣ ихъ замыслы, начинанія, тщанія и дѣла противны Хрісту: но якоже въ зеркало, ради исправленія лица твоего, посматриваешь, тако посматривай часто въ зеркало непорочнаго житія Хрістова, и нравъ свой исправляй по тому, и душу свою очищай. \textit{Сіе да мудрствуется въ насъ, еже и во Хрістѣ Іисусѣ}\footnote{Филип.~2,~5.}. \textit{Молю убо вы, братіе, щедротами Божіими, представите тѣлеса ваша жертву живу, святу, благоугодну Богови, словесное служеніе ваше, и не сообразуйтеся вѣку сему, но преобразуйтеся обновленіемъ ума вашего, во еже искушати вамъ, что есть воля Божія благая и угодная и совершенная}\footnote{Римл.~12,~1 и 2.}.

\section{75. Моровая язва.}

Что моровая язва есть тѣлу, тое есть душѣ соблазнъ. Язва моровая заражаетъ и умерщвляетъ тѣло: соблазнъ заражаетъ и умерщвляетъ душу человѣческую. Язва моровая въ единомъ человѣкѣ прежде зачнется, потомъ весь домъ, а отъ того весь градъ или село, а далѣе и вся страна заражается и погибаетъ, какъ видимъ: тако соблазнъ въ единомъ человѣкѣ начинается, а потомъ и къ многимъ переходитъ; похоть бо, въ сердцѣ человѣческомъ крыющаяся, видѣніемъ и слухомъ, какъ огнь вѣтромъ, возбуждается и разжигается ко злу. Видимъ сіе въ мірѣ, видимъ, какъ другъ отъ друга заражаются соблазномъ и погибаютъ. Что глаза видятъ и уши слышатъ, тое и въ сердце человѣческое ударяетъ. Единъ началъ такой"=то домъ созидать себѣ, въ такомъ"=то платьѣ ходить, на такой"=то каретѣ и коняхъ ѣздить, такой"=то уборъ и украшеніе въ домѣ своемъ имѣть, и прочая: видитъ тое другій, и третій, и прочіе, и вси; и вси дѣлаютъ тое. Единъ помѣщикъ тако съ крестьянами своими поступаетъ, такіе"=то съ нихъ оброки и столько беретъ, или столько ему дней въ седмицѣ работаютъ они: видятъ тое другіи, и такожде поступаютъ со своими крестьянами. Единъ судья заразился мздоимствомъ, и столько"=то тысящей собралъ себѣ отъ беззаконнаго того дѣла: слышитъ то другій, третій, и прочій, и замышляютъ въ себѣ тоежде беззаконное дѣло; онъ"=де столько и столько собралъ себѣ, "--- соберу и я; и собираетъ. И тако переходитъ лютое сіе зло отъ единаго къ другому, и отъ того къ другимъ и ко всѣмъ. Видимъ, какъ заразилося бѣдное наше отечество лютою сею язвою! Нѣтъ суда нигдѣ безъ денегъ. Не судятъ, но только"=что купуютъ, да продаютъ. Единъ началъ банкеты строить и гостей къ себѣ звать и принимать: дѣлаютъ тое и другій, и уже то "--- и знаютъ, что другъ къ другу въ гости ѣздятъ, и проч. Тако соблазнъ, какъ моровая язва, не тѣла, но души человѣческія заражаетъ, и отнимаетъ у нихъ не временный, но вѣчный животъ! Откуду видимъ, что иной градъ тѣмъ, иной другимъ беззаконіемъ изобилуетъ. Сіе не отъинуду бываетъ, какъ отъ соблазна. О лютое зло соблазнъ!.. Сего ради столь сильно запретилъ Хрістосъ Господь подавать соблазнъ\footnote{Мѳ.~18,~6--9.}. Хрістіанине! берегись подать соблазнъ и отъ соблазна, какъ моровой язвы, берегись. Люби сердечно законъ Божій, и не будетъ тебѣ соблазна. \textit{Миръ многъ любящимъ законъ Твой}, Господи, \textit{и нѣсть имъ соблазна}\footnote{Пс.~118,~165.}. Презри міръ сей съ прелестію своею, и возлюби единаго Бога и вѣчный животъ: и будеши жить въ мірѣ, какъ Лотъ въ Содомѣ, невредимъ.

\subsection{О томжде.}

Что моровою язвою зараженный человѣкъ, тое есть клеветникъ. Моровою язвою зараженный повреждаетъ того, кто съ нимъ сообщается и неопасно поступаетъ: клеветникъ повреждаетъ того, кто его слушаетъ клевету. Отъ зараженнаго человѣка язва заразительная приходитъ къ другому, отъ другаго къ третьему, отъ третьяго къ четвертому, и тако ко всѣмъ людямъ, аще не остерегутся; и тако бываетъ, что многіи тысящи людей отъ единаго зараженнаго заражаются и погибаютъ: тако отъ клеветника единъ услышитъ клевету, и другому скажетъ, другій третьему, третій четвертому, и тако вси слышатъ и повреждаются клеветою, и бываетъ, что вся страна и все государство слышитъ и повреждается. Ибо иніи слышачи осуждаютъ того, о комъ клевета носится: худо"=де онъ сдѣлалъ или дѣлаетъ; и тако тяжко грѣшатъ, похищая себѣ тое, что единому Хрісту, "--- праведному Судіи, приличествуетъ: Онъ бо есть единъ Судія всѣхъ. Иніи къ тому же беззаконному дѣлу возбуждаются, чтó слышатъ, и тоежде дѣлаютъ. Всему сему пагубному злу клеветникъ, разсѣватель зла, виновенъ. И тако видишь, хрістіанине, коль пагубное зло есть клевета, хотя нынѣшніи хрістіане ничимъ такъ, какъ клеветою утѣшаются и услаждаются. Что бо нынѣ у людей во устахъ, собравшихся во едино, какъ то того, то другаго имя носится? О чемъ болѣе разговоровъ у нихъ, какъ то о томъ, то о другомъ бѣдномъ грѣшникѣ? Тотъ скажетъ о томъ, другій о другомъ, третій о третьемъ, и тако столько грѣховъ и беззаконій бываетъ въ собраніи, сколько словъ. Всему тому злу клеветникъ, отъ котораго клевета началась, причиною бываетъ. Клеветникъ вредитъ того, на кого клевещетъ: ибо языкомъ своимъ уязвляетъ его, какъ мечемъ, и славу его, какъ песъ зубами одежду, терзаетъ: онъ"=де тое и тое дѣлаетъ. Вредитъ себе: ибо тяжко грѣшитъ. Вредитъ тѣхъ, который слушаютъ клевету его: ибо подаетъ причину имъ къ клеветѣ и осужденію, и тако ихъ къ тому же беззаконному дѣлу, въ которомъ самъ находится, приводитъ. И тако, какъ отъ единаго зараженнаго человѣка многіи люди тѣломъ заражаются и погибаютъ: тако отъ единаго клеветника, начальника клеветы, многія хрістіанскія души заражаются и погибаютъ. О необузданный языкъ! сколько ты міру дѣлаешь зла! Какъ вѣтръ пожаръ по всему граду или селу, тако необузданный языкъ всякое зло по всему государству и по всему міру разноситъ. Единъ клеветникъ узналъ, и вси уже знаютъ; единъ сказалъ, и вси говорятъ. О! воистину лучше сто кратъ пасть ногами, нежели языкомъ. Малый удъ языкъ, но великая и многая зла дѣлаетъ; двоякою оградою загражденъ, то"=есть, зубами и губами, но весьма удобно вырывается и выскакиваетъ. Хрістіанине! берегись клеветника, какъ моровою язвою зараженнаго человѣка бережешися, да не и самъ отъ него заразишися и погибнешь. У клеветниковъ обыкновенно есть испытывать дѣла людскія: что"=де тотъ"=то и тотъ дѣлаетъ; и тако испытавше, разносятъ клевету. Они другъ друга знаютъ и во едино сходятся, и то о томъ, то о другомъ другъ у друга спрашиваютъ: и тако, узнавше, разсѣваютъ клеветы. Таковыхъ людей, какъ прокаженныхъ, берегись. Берегись и самъ испытывать людскихъ грѣховъ, да не будеши судить и клеветать на ближняго твоего. Испытаніе бо грѣховъ чуждыхъ и познаніе есть начало осужденія и клеветы. Отврати глаза твои отъ ближняго твоего, и обрати на себе самого, и испытывай и познавай свои грѣхи, и очищай ихъ истиннымъ покаяніемъ и вѣрою. Сіе бо есть хрістіанское дѣло, къ которому ты позванъ отъ Хріста. Всякимъ образомъ тщись хранить языкъ: и многихъ грѣховъ избѣжишь.

\section{76. Скотское послѣдованіе.}

Видимъ, что куды едина скотина пойдетъ, за нею и прочій скотъ идетъ и той послѣдуетъ, хотя и вредъ оттуду ему будетъ. Тако многіи люди безразсудно дѣлаютъ, и по чувствамъ, а не по разуму поступаютъ, и другъ другу по примѣру скотскому послѣдуютъ, не внимая, полезно ли имъ тое будетъ, или вредно. Видимъ сію беззаконную и пагубную въ мірѣ ревность. Единъ началъ частые банкеты строить, пить, гулять и упиваться, и прочихъ упоевать: тоеже дѣлаетъ и другій и прочій. Единъ началъ щеголять: послѣдуютъ ему и прочіи. Единъ много собралъ богатства: и прочіи о томжде тщатся. Едина безстудница намазываетъ лице свое бѣлилами и красками: послѣдуютъ ей и прочіи, и дѣлаютъ пагубно вси, что едина. Единъ беззаконный купецъ проситъ у торговца цѣны высшія, нежели товаръ стоитъ, и клятву къ тому придаетъ, и святое и страшное имя Божіе въ томъ беззаконномъ дѣлѣ призываетъ: дѣлаютъ тое и прочіи, подобніи ему. Такъ и въ прочемъ бѣдныи люди другъ другу подражаютъ; подражаютъ, но на свой вредъ и погибель. Се есть безразсудное дѣло! се есть скотское послѣдованіе! се есть моровая язва, которою души хрістіанскіи, пресвятою кровію Хрістовою искупленніи, заражаются и погибаютъ! се есть пожаръ душепагубный, который въ единомъ наченшися, прочіе душевные домы пожигаетъ! Увы бѣда! увы горе! Зло начинается въ единомъ, и вси то дѣлаютъ, и въ обычай входитъ, и сильно утверждается такъ, что и искоренить его не возможно, какъ застарѣлую болѣзнь. "--- Мнѣ"=де люди будутъ смѣяться, когда мнѣ не дѣлать, какъ они дѣлаютъ? Отвѣтъ: 1)~Нечестіе всегда благочестію смѣется и ругается ему. Пусть какой богатый человѣкъ продаетъ богатый домъ свой и вси принадлежности, и все имѣніе свое расточитъ и раздастъ убогимъ, и возлюбитъ нищету Хрістову: самъ увидитъ, что ему будетъ отъ злаго міра. Вотъ"=де дуракъ, что сдѣлалъ! что люди собираютъ, тое онъ расточилъ. Тако сынамъ вѣка сего благочестіе "--- буйство показуется! 2)~Хрістосъ Спаситель нашъ, Господь и Царь славы, посмѣянъ и поруганъ былъ отъ любителей міра. Пусть міръ и тебѣ смѣется и ругается: ты свое знай, и дѣлай, что Ему угодно, а не міру; Ему, яко Начальнику твоему, сообразуйся, а не вѣку сему. 3)~Неужели все намъ дѣлать тое, что міръ дѣлаетъ? Многіи убиваютъ, прелюбодѣйствуютъ и любодѣйствуютъ, крадутъ и похищаютъ, и проч.: убо и намъ тое дѣлать? да не будетъ! Они идутъ въ погибель: и намъ итить? да сохранитъ насъ Господь отъ того! 4)~Мы не только чувство, но и разумъ имѣемъ; и можемъ разсудить, что добро и что зло, что полезно и что вредно. И такъ, когда видимъ, что люди дѣлаютъ, то не чувствамъ, по примѣру скотовъ, но разуму послѣдовать должно. Скоти суть, а не люди, которыи единымъ чувствамъ, а не разуму послѣдуютъ.

5)~Мы хрістіане; мы имѣемъ святое Божіе слово; мы видимъ въ немъ, что добро и что зло, что полезно и что вредно. Тако Богъ запрещаетъ, что зло и намъ вредно; повелѣваетъ, что добро, и намъ полезно. Что убо люди дѣлаютъ, приложимъ тое къ Святому Писанію, какъ къ чистому зеркалу, и посмотримъ въ тое: сходно ли, или противно тому дѣлаютъ люди? Когда сходно дѣлаютъ, "--- хорошо, и намъ полезно тое дѣлать: когда противно, то отвратимъ очи наши отъ того, и послушаемъ, чего священная сія научаетъ книга. 6)~Пусть хотя весь свѣтъ дѣлаетъ тое, что Богу противно и себѣ пагубно: ты дѣлай, что Богу угодно и тебѣ душеполезно. Весь свѣтъ не заступитъ тебе предъ судомъ Божіимъ. Тамо не скажешь: вотъ"=де тотъ и тотъ дѣлалъ тое. Едино услышишь отъ Судіи: ради чего ты не дѣлалъ того, что Я приказалъ тебѣ? Единаго Бога, паче всего свѣта, несравненно почитать должно. Буди убо въ мірѣ, какъ Лотъ въ Содомѣ, гдѣ вси беззаконновали, но онъ имъ не подражалъ, и дѣлалъ, что святой волѣ Божіей угодно было: дѣлай и ты тако, не подражай тому, что злый міръ дѣлаетъ. Что худо дѣлаютъ люди, и видишь или слышишь: буди какъ не видяй, и якоже глухій, не слышай. Обращай всегда къ вѣчности сердечныя твои очи, и будутъ вси мірскія дѣла какъ позади тебе, и жить будешь въ мірѣ, какъ единъ, вѣдая Создателя твоего и святую волю Его. \textit{Аще убо воскреснусте со Хрістомъ, вышнихъ ищите, идѣже есть Хрістосъ одесную Бога сѣдя. Горняя мудрствуйте, а не земная}\footnote{Кол.~3,~1--2.}.

\section{77. Свѣтильникъ.}

Что свѣтильникъ въ домѣ, то есть вѣра живая въ сердцѣ человѣческомъ. Свѣтильникъ зажигается отъ человѣка: свѣтильникъ вѣры зажигается отъ Духа Святаго чрезъ слышанное слово Божіе, по писанному: \textit{вѣра отъ слуха, слухъ же глаголомъ Божіимъ}\footnote{Рим.~10,~17.}. Когда свѣтильникъ въ домѣ горитъ и сіяетъ; все ясно въ домѣ, и живущіи въ немъ все видятъ, и ходящіи не претыкаются, и всякъ свое дѣло дѣлаетъ, яко отъ свѣтильника всякъ освѣщается: тако, когда свѣтильникъ вѣры въ сердцѣ сіяетъ, человѣкъ все духовное ясно видитъ; Бога невидимаго, яко видимаго, и прочая невидимая, яко видимая, видитъ, и дѣла хрістіанскому званію приличная дѣлаетъ. Когда свѣтильника въ домѣ не имѣется, тьма въ домѣ есть: тако, когда свѣтильника вѣры въ сердцѣ человѣческомъ не имѣется, не ино что тамо есть, какъ только тьма и всякое заблужденіе. Чтобы свѣтильникъ въ домѣ горящій не угаснулъ, должно приливать елей: тако чтобы свѣтильникъ вѣры въ сердцѣ не угаснулъ, должно чинить: 1)~Читать, или слушать и разсуждать слово Божіе и прочія хрістіанскія книги. 2)~Молиться прилѣжно Богу.

3)~Причащатися Святыхъ и Животворящихъ Таинъ тѣла и крове Хрістовы. 4)~Дѣла милости творить. \textit{Блажени бо милостивіи, яко тіи помиловани будутъ}\footnote{Мѳ.~5,~7.}. Знаки убо свѣтильника вѣры, въ сердцѣ человѣческомъ горящаго, въ Святомъ Писаніи сіи примѣчаются: 1)~Таковый человѣкъ слово Божіе читаетъ, или слушаетъ и въ немъ прилѣжно поучается. 2)~Бога сердечно призываетъ, молится и благодаритъ Ему за вся Его благодѣянія.

3)~Со всякимъ усердіемъ тщится жить достойно званія хрістіанскаго и Евангелія. 4)~Вѣру свою къ Богу оказываетъ добрыми дѣлами, якоже требуетъ Апостолъ: \textit{покажи ми вѣру твою отъ дѣлъ твоихъ}\footnote{Іак.~2,~18.}. 5)~Всякаго грѣха бережется, и противу всякаго грѣха подвизается и не допущаетъ, чтобы онъ владѣлъ имъ. 6)~Въ мірѣ семъ живетъ, какъ странникъ и пришлецъ, и всегда къ небесному отечеству сердечныя очи возводитъ и воздыхаетъ; и потому къ мірскимъ вещамъ не прилѣпляется сердцемъ, но все со страхомъ и къ нуждѣ употребляетъ. Таковаго ничто въ мірѣ семъ не веселитъ, кромѣ единаго Бога и надежды вѣчнаго живота. 7)~Явный знакъ свѣтильника вѣры, горящаго въ сердцѣ человѣческомъ, есть радость духовная, въ сердцѣ ощущаемая, радость о Дусѣ Святѣ, радость о Бозѣ, каковая радость во псалмахъ изображается: \textit{сердце мое и плоть моя возрадовастася о Бозѣ живѣ}\footnote{Пс.~83,~3.}. Ибо гдѣ вѣра къ Богу, тамо и любовь къ Богу; гдѣ любовь къ Богу, тамо и радость о Бозѣ. Что бо любимъ, о томъ и радуемся. "--- Отъ сего видно, какіе суть знаки вѣры, въ сердцѣ человѣческомъ угасшіе; а именно: 1)~Беззаконное и слову Божію противное житіе. У блудника, прелюбодѣя и всякаго нечистаго угасла вѣра; у вора, татя, лихоимца, насильника и грабителя угасла вѣра; у злобнаго, ненавистливаго и отмщеніемъ дышущаго угасла вѣра; у клеветника, досадителя, укорителя, поносителя и ругателя угасла вѣра; у лицемѣра, лживаго и льстиваго угасла вѣра; словомъ, у всякаго беззаконника, который совѣсти своея не хранитъ, но противу ея поступаетъ, угасла вѣра, но вмѣсто свѣтильника вѣры тьма въ нихъ невѣдѣнія Божія. Кое бо участіе свѣту со тьмою? Свѣтъ "--- вѣра: тьма "--- беззаконное житіе. Кое убо общеніе вѣрѣ со тьмою беззаконнаго житія? Таковыи хотя Богу и молятся, но лицемѣрно: яко сердца ихъ далеко отъ Бога отстоятъ. Имъ приличествуетъ Божіе оное слово: \textit{приближаются Мнѣ людіе сіи усты своими, и устнами чтутъ Мя; сердце же ихъ далече отстоитъ отъ Мене}\footnote{Мѳ.~15,~8; Ис.~29,~13.}. 2)~Презрѣніе Божія слова и отъ Него отвращеніе. 3)~Оставленіе молитвы. 4)~Удаленіе отъ причастія Святыхъ Таинъ Хрістовыхъ: безъ сихъ бо вѣра быть и сохраниться въ сердцѣ человѣческомъ не можетъ. 5)~Пристрастіе къ временнымъ вещамъ, богатству, чести, славѣ и роскоши: вѣра бо есть даръ небесный; потому сердце человѣческое отъ земнаго мудрованія отвлекаетъ, и подвигаетъ къ мудрованію небесному. Таковый человѣкъ грѣшитъ противу апостольскаго слова: \textit{горняя мудрствуйте, а не земная}, и проч.\footnote{Кол.~3,~2.} Хрістіане, которыи имѣютъ свѣтильникъ вѣры горящій въ сердцахъ своихъ, и того до конца будутъ имѣть, изыдутъ въ срѣтеніе Жениху Хрісту, пришедшему судити живыхъ и мертвыхъ, и съ Нимъ внидутъ въ чертогъ небесный, яко мудрыя дѣвы: \textit{буди} убо \textit{вѣренъ даже до смерти} (имѣющій сей свѣтильникъ въ себѣ), \textit{и дамъ ти вѣнецъ живота}, глаголетъ тебѣ Хрістосъ Господь\footnote{Апок.~2,~10.}. А которыи свѣтильника сего не будутъ имѣть въ сердцахъ своихъ, таковыи не сподобятся срѣсти Хріста Царя, и до чертога онаго не допущены будутъ, но внѣ его останутся, и услышатъ отъ Хріста Царя: \textit{не вѣмъ васъ}\footnote{Мѳ.~25,~12.}. Хрістіанине, страшно слово сіе слышать отъ Хріста хрістіанамъ: \textit{не вѣмъ васъ!} Обратимся убо всѣмъ сердцемъ къ Богу, да и наши свѣтильники благодатію Его зажгутся. \textit{Востани спяй, и воскресни отъ мертвыхъ, и освѣтитъ тя Хрістосъ}\footnote{Еф.~25,~14.}. \textit{Ты просвѣтиши свѣтильникъ мой, Господи: Боже мой, просвѣтиши тьму мою}\footnote{Пс.~17,~29.}.

\section{78. Слѣпый и видящій при немъ.}

Бываетъ, что при слѣпомъ имѣется человѣкъ видящій; но слѣпый его не знаетъ, что онъ при немъ имѣется: тако при всякомъ человѣкѣ имѣется Богъ, яко вездѣсущій; но человѣкъ, который вѣрою не просвѣщенъ, понеже не видитъ Его, то и не познаетъ, что Господь и Судія его при немъ есть невидимо. Часто бываетъ, что слѣпый, человѣка, при себѣ находящагося, не видя, дѣлаетъ дѣла непристойная и срамная, помышляя въ себѣ, что какъ онъ никого не видитъ, такъ и его никто не видитъ: тако грѣшникъ, какъ слѣпый, не видя людей прочихъ при себѣ, думаетъ и мечтаетъ, что никто его не видитъ, какъ онъ никого не видитъ, и такъ дерзаетъ на беззаконная дѣла. Но весьма обманывается бѣдный: ибо Богъ, Который присутствуетъ ему, предъ Которымъ все онъ дѣлаетъ и мыслитъ, видитъ его; и всякое дѣло и помышленіе его видитъ и въ книгѣ Своей записываетъ, и въ послѣдній день обличитъ его. Грѣшнику бо рече Богъ: \textit{обличу тя, и представлю предъ лицемъ твоимъ грѣхи твоя}\footnote{Пс.~49,~21.}. Думалъ ты, что никто тебе не видалъ, когда ты беззаконныя дѣла дѣлалъ; но Я видѣлъ тебе, какъ ты законъ Мой нарушалъ, и представлю предъ тобою беззаконія твоя. Ты творилъ ихъ тайно; но Я тя \textit{обличу} предъ всѣмъ міромъ, ангелами и человѣками. Тогда ты узнаешь, грѣшникъ, что Я все видѣлъ, что ты ни дѣлалъ, зачиналъ и замышлялъ въ сердцѣ своемъ. Укрываешься ты отъ людей; но отъ Мене нигдѣ укрыться не можешь. Таковый слѣпецъ есть всякъ блудникъ, прелюбодѣй, сквернитель и нечистоты любитель; всякъ воръ, тать, хищникъ, грабитель и мздоимецъ "--- судія, всякъ клеветникъ и ближняго славу терзающій; всякъ, кто съ другимъ лестно, коварно, хитро, лукаво, лживо и лицемѣрно обходится, и проч. Вси таковыи слѣпотствуютъ, не тѣлесными, но душевными очами; и тщатся укрытися отъ людей, но отъ Бога всевидящаго скрытися не могутъ; утаеваютъ своя беззаконная дѣла отъ человѣковъ, но Богъ, паче всего міра людей, видитъ ихъ. \textit{Разумѣйте же безумніи въ людехъ, и буіи нѣкогда умудритеся. Насаждей ухо, не слышитъ ли? или Создавый око, не сматряетъ ли}\footnote{Пс.~93,~8 и 9.}? \textit{Съ небесе призрѣ Господь, видѣ вся сыны человѣческія. Отъ готоваго жилища Своего призрѣ на вся живущія на земли, создавый на единѣ сердца ихъ, разумѣваяй на вся дѣла ихъ}\footnote{32,~13--15.}. \textit{Очи Твои, Господи, отверсты на вся пути сыновъ человѣческихъ, дати комуждо по пути его, и по плоду начинаній его}\footnote{Іер.~32,~19.}. \textit{Камо пойду отъ Духа Твоего? и отъ лица Твоего камо бѣжу? Аще взыду на небо, Ты тамо еси; аще сниду во адъ, тамо еси; еще возму крилѣ мои рано, и вселюся въ послѣднихъ моря, и тамо бо рука Твоя наставитъ мя, и удержитъ мя десница Твоя. И рѣхъ: еда тма поперетъ мя и нощь просвѣщеніе въ сладости моей? Яко тма не помрачится отъ Тебе, и нощь яко день просвѣтится: яко тма ея, тако и свѣтъ ея}\footnote{Пс.~138,~7--12.}. Видишь, человѣче, что хотя нѣтъ никого при тебѣ отъ человѣковъ, но Богъ Судія присутствуетъ тебѣ; и хотя никто отъ людей не видитъ тебе, но Богъ видитъ дѣла твоя, Котораго тебѣ должно боятися и стыдитися паче всего міра, паче всѣхъ ангеловъ и человѣковъ: яко предъ Нимъ все созданіе, какъ ничто. Убойся убо и покайся, и впредь берегись предъ очами Божіими беззаконновать, да не дознаеши на себѣ мстительную руку Его!

\subsection{О томжде.}

Когда слѣпый хотя и не видитъ человѣка при себѣ имѣющагося, но точно знаетъ, что при немъ находится человѣкъ, бережется дѣлать непристойное: тако хрістіанинъ, вѣрою просвѣщенный, хотя и не видитъ Бога присутствующаго (и видѣти бо Его не возможно), однакожъ видитъ Его внутреннимъ вѣры окомъ, и Невидимаго, яко видимаго, зритъ предъ собою. Тако глаголетъ Псаломникъ о себѣ, вѣрою и Духомъ Святымъ просвѣщенный: \textit{предзрѣхъ Господа предо мною выну}\footnote{Пс.~15,~8.}. Тако пишется о Моисеѣ пророкѣ: \textit{Невидимаго яко видя терпяше}\footnote{Евр.~11,~27.}. Вѣры бо сіе есть свойство, что она невидимаго Бога, яко видимаго, душевнымъ очамъ представляетъ. Таковою вѣрою просвѣщенная душа вездѣ и на всякомъ мѣстѣ ходитъ предъ Богомъ со страхомъ и благоговѣинствомъ; и всего бережется, что Ему противно, и поступаетъ предъ Нимъ тако, какъ ученики предъ учителемъ своимъ, раби предъ господиномъ своимъ, дѣти предъ отцемъ своимъ, и подданныи предъ монархомъ своимъ. Съ таковою душею Богъ человѣколюбивый во всѣхъ бѣдахъ, напастяхъ и искушеніяхъ Защитникъ, Помощникъ и Заступникъ. \textit{Мой еси ты. И аще преходиши сквозѣ воду, съ тобою есмь, и рѣки не покрыютъ тебе: и аще сквозѣ огнь пройдеши, не сожжешися и пламень не опалитъ тебе. Яко Азъ Господь Богъ твой, Святый Израилевъ, спасаяй тя}, глаголетъ Господь\footnote{Ис.~43,~2 и 3.}. О семъ дерзаетъ вѣрная душа, и возглашаетъ: \textit{Господь пасетъ мя, и ничтоже мя лишитъ. На мѣстѣ злачнѣ, тамо всели мя: на водѣ покойнѣ воспита мя. Душу мою обрати: настави мя на стезю правды имене ради Своего. Аще бо и пойду посредѣ сѣни смертныя, не убоюся зла, яко Ты со мною еси}\footnote{Пс.~22,~1--4.}. \textit{Господь просвѣщеніе мое и Спаситель мой: кого убоюся? Господь Защититель живота моего: отъ кого устрашуся? Внегда приближатися на мя злобующимъ, еже снѣсти плоти моя, оскорбляющіи мя и врази мои, тіи изнемогоша и падоша. Аще ополчится на мя полкъ, не убоится сердце мое: аще востанетъ на мя брань, на Него азъ уповаю}\footnote{26,~1--4.}.

Сего ради, вѣрная душе, \textit{не даждь во смятеніе ноги твоея, ниже воздремлетъ храняй тя. Се не воздремлетъ, ниже уснетъ храняй Израиля. Господь сохранитъ тя, Господь покровъ твой на руку десную твою. Во дни солнце не ожжетъ тебе, ниже луна нощію. Господь сохранитъ тя отъ всякаго зла, сохранитъ душу твою Господь; Господь сохранитъ вхожденіе твое и исхожденіе твое, отъ нынѣ и до вѣка}\footnote{120,~3--8.}.

\section{79. Подданный предъ царемъ своимъ законъ его нарушающій.}

Аще бы кто въ такое дерзновеніе пришелъ, что предъ царемъ своимъ законъ его началъ нарушать: скажи пожалуй, человѣче, скажи, не дѣлалъ ли бы великаго оскорбленія и досажденія монарху своему, и не подвигнулъ ли бы его на праведный гнѣвъ таковый безчинникъ? Воистину за великое безчестіе имѣлъ бы себѣ таковое дѣло монархъ, и праведно возъярился бы къ гнѣву на законопреступника того. "--- Хрістіанине! мы всѣ предъ Богомъ, Царемъ небесе и земли, вездѣсущимъ и вся назирающимъ, ходимъ, живемъ и обращаемся; и что ни дѣлаемъ, мыслимъ и говоримъ, предъ святыми очами Его все дѣлаемъ, мыслимъ и говоримъ, хотя и не видимъ Его. Тако слово Его святое насъ увѣряетъ. Всякъ убо беззаконникъ законъ Его святый предъ Нимъ разоряетъ; всякъ хульникъ святое и страшное имя Его предъ Нимъ хулитъ; всякъ лживый святымъ и страшнымъ именемъ Его во лжи предъ Нимъ кленется, во лжи великое имя Его предъ Нимъ призываетъ; всякъ злобникъ и убійца предъ Нимъ человѣка убиваетъ; всякъ блудникъ и прелюбодѣй и нечистоты любитель предъ Нимъ беззаконнуетъ; всякъ воръ, тать, хищникъ и грабитель предъ Нимъ законъ Его разоряетъ "--- \textit{не укради}; всякъ клеветникъ, ругатель, досадитель, укоритель, сквернословецъ и кощунникъ предъ Нимъ беззаконнуетъ. Всякъ, кто ни замышляетъ и мыслитъ лукавое и злое въ сердцѣ своемъ, предъ Нимъ замышляетъ: яко Богъ и сердце наше и помышленіе его видитъ. Какое убо досажденіе и оскорбленіе Создателю нашему можетъ быть паче сего? О, колико враговъ Божіихъ на свѣтѣ есть, не токмо отъ тѣхъ, которыи не знаютъ Его, но и отъ тѣхъ, которыи мнятся Его почитать и вѣрою Его святою хвалятся! Воистину бо врагъ Божій есть не токмо тотъ, который явныя хулы отрыгаетъ на имя Его святое, но и тотъ, который безстрашно дерзаетъ грѣшить. Вси таковыи глаголютъ въ сердцѣ своемъ: \textit{не узритъ Господь, ниже уразумѣетъ Богъ Іаковль} (дѣлъ нашихъ). \textit{Разумѣйте же безумніи въ людехъ, и буіи нѣкогда умудритеся. Насаждей ухо, не слышитъ ли? или Создавый око, не сматряетъ ли?}\footnote{Пс.~93,~7--9.} О, человѣче, который хрістіанскимъ именемъ называешися, но съ идолопоклонниками едино дѣлаешь; Бога исповѣдуешь, но дѣлами отмещешися Его; имя Божіе почитаешь, но преступленіемъ закона Божія безчестишь тое! Осмотрись пожалуй, каковъ ты предъ Богомъ имѣешися, и къ чему, къ какому концу злонравіе ведетъ тебе; осмотрись, пока кончина не постигнетъ тебе, и отъидеши воспріяти по дѣломъ твоимъ. Богъ тебѣ долготерпѣлъ и долготерпитъ, ожидая твоего покаянія, а впредь стерпитъ ли, неизвѣстно. Благъ и милостивъ Господь, но и праведенъ. Милость Его дознаютъ обращающіися и кающіися: но правду и судъ Его праведный нераскаянныи грѣшники на себѣ дознаютъ. Обратися убо и покайся, да не и ты дознавши на себѣ судъ Его праведный, но паче да сподобишися милости Его. \textit{Помилуй мя, Боже, по велицѣй милости Твоей, и по множеству щедротъ Твоихъ, очисти беззаконіе мое}, и проч.\footnote{50,~3.}

\section{80. Рабъ, обиждаемый предъ господиномъ его.}

Бываетъ, что грубый какій человѣкъ обиду дѣлаетъ рабу предъ господиномъ его: тако грѣшникъ, когда какую нибудь обиду дѣлаетъ хрістіанину, дѣлаетъ тую предъ Господемъ Богомъ, яко вездѣсущимъ и вся видящимъ. Сюды надлежатъ: 1)~Прелюбодѣи, который ложа чужаго беззаконно касаются. 2)~Повреждающіе какимъ нибудь образомъ здравіе людей. 3)~Біющіи и убивающіи человѣка убійцы. 4)~Поносящіи и укоряющіи ближнихъ своихъ. 5)~Тайно или явно оклеветающіи ближнихъ своихъ. 6)~Воры, тати, хищники и грабители, которыи чужое добро къ себѣ неправедно привлекаютъ. 7)~Купцы, въ товарахъ продаваемыхъ людей обманывающіи, и худую вещь за добрую и дешевую за дорогую продающіи. 8)~Господа помѣщики, которыи или звѣрски мучатъ крестьянъ своихъ, или излишними оброками ихъ отягчаютъ, или паче мѣры работать ради себе убѣждаютъ ихъ. 9)~Судіи, который по мздѣ, а не по правдѣ судятъ. 10)~Тіи, которыи не отдаютъ мзды наемникамъ, или отдаютъ, но не сполна. 11)~Наемники, который взяли мзду довольную за труды, но не хотятъ трудиться за тую, или трудятся, но лукаво и лицемѣрно. 12)~Начальники, которыи не отдаютъ жалованья опредѣленнаго подначальнымъ, или отдаютъ, но не сполна. 13)~Всякъ, кто съ ближнимъ хитро, лукаво и лицемѣрно поступаетъ и его обманываетъ, и какъ нибудь его обиждаетъ, къ тому же числу принадлежитъ. Вси таковыи предъ Господомъ Богомъ людей Его обиждаютъ. Обида, творимая рабу предъ господиномъ его, касается и господина его, и досажденіе ему не малое содѣловаетъ: тако обида, хрістіанамъ показываемая, и Самого Хріста Господа касается, и Ему не малое досажденіе дѣлаетъ. Какъ бо добро, творимое хрістіанамъ, Себѣ вмѣняетъ Хрістосъ Господь, по реченному: \textit{понеже сотвористе единому сихъ братій Моихъ меньшихъ, Мнѣ сотвористе}\footnote{Мѳ.~25,~40.}: тако и обиду, творимую хрістіанамъ, Себѣ вмѣняетъ Господь. Рабъ, обиду терпящій при господинѣ, взираетъ на господина своего, и ему свою обиду препоручаетъ: тако хрістіанинъ, яко предъ Господомъ обиждаемый, долженъ на Него взирать, и Ему свою обиду, яко судящему праведно, предавать, Который и глаголетъ: \textit{Мнѣ отмщеніе, Азъ воздамъ}\footnote{Римл.~12,~19.}. Отъ сего видно: 1)~Кто противу человѣка грѣшить, тотъ и противу Бога грѣшитъ; кто человѣку обиду творитъ, тотъ и Господу творитъ; кто человѣку досаждаетъ, тотъ и Господу досаждаетъ. Ибо, обиждая человѣка, заповѣдь Божію разоряетъ, и обиждаетъ человѣка раба Божія предъ Господемъ его. 2)~Коль тяжко грѣшатъ таковыи, видишь, хрістіанине! Досадно тебѣ, и весьма досадно, когда кто твоего раба предъ тобою обиждаетъ: ты же дерзаешь хрістіанина, раба Хрістова, предъ Господемъ его обиждать!.. Развѣ то думаешь, что Господь всѣхъ не видитъ тебе, какъ и ты не видишь Его? нѣтъ! нѣтъ! Никто и ничто, зло дѣлающій и злое дѣло его, отъ всевидящаго ока Его не утаится. На пути ли или въ пустыни, въ домѣ или на торжищи, на единѣ или на публикѣ, тайно или явно дѣлаешь зло, или во внутренности сердца твоего замышляешь неправду, "--- Онъ вездѣ присутствуетъ, и все, что и гдѣ дѣлаешь и замышляешь, видитъ и въ книгѣ Своей записываетъ, и въ послѣдній день во всемъ томъ предъ всѣмъ міромъ обличитъ тебе, по реченному: \textit{обличу тя, и представлю предъ лицемъ твоимъ грѣхи твоя}\footnote{Пс.~49,~21.}. Укоряеши ли убо, злословиши, ругаеши и поносиши ближняго твоего? тутъ Онъ присутствуетъ, и слышитъ злое слово твое, и видитъ обиду его. Клевещеши ли на брата твоего? тутъ Онъ есть, и слышитъ клевету твою. Біеши ли или убиваеши ближняго твоего? не далеко Онъ отъ тебе есть, и видитъ злое дѣло твое. Похищаеши ли и крадеши чужое добро? предъ святыми очами Его крадеши. Лжеши ли и обманываеши ближняго твоего? Видятъ очи Его ложь твою. Словомъ: все, что ни дѣлаешь ближнему твоему, все предъ Господемъ всѣхъ дѣлаешь. А отъ сего видишь самъ, какое досажденіе величеству Его дѣлаешь, и коль тяжко грѣшишь. О, лучше, воистину лучше неоднократно умереть, нежели Хрісту Господу досадить, Хрісту, Который такъ насъ возлюбилъ непотребныхъ рабовъ, что и умереть за насъ не отреклся. О, человѣче, который какимъ нибудь образомъ ближняго твоего обиждаешь! осмотрись, пожалуй осмотрись, что ты и кому и предъ кѣмъ дѣлаешь; осмотрись, пока время не ушло! Называешися ты хрістіаниномъ, знаменуешися крестомъ, ходишь въ церковь, покланяешися и молишися Богу, поеши "--- \textit{аллилуіа}, и, что болѣе всего того, приступаеши ко олтарю и причащаешися Животворящихъ Таинъ Хрістовыхъ, слушаеши и Слово Божіе, и проч.: знаки то суть подлинно хрістіанства; но когда обиды дѣлать ближнему и грѣшить не престаешь: берегись, чтобы, вмѣсто хрістіанина, врагомъ Хрістовымъ не быть. Испытай убо себе, и разсуждай, како ты съ ближнимъ своимъ обходишися, и покайся и исправь себе, пока время есть покаянія. \textit{Или о богатствѣ благости Его и кротости и долготерпѣніи нерадиши, не вѣдый, яко благость Божія на покаяніе тя ведетъ? По жестокости же твоей и непокаянному сердцу, собираеши себѣ гнѣвъ въ день гнѣва и откровенія праведнаго суда Божія. Иже воздастъ коемуждо по дѣломъ его: овымъ убо по терпѣнію дѣла благаго, славу и честь и нетлѣніе ищущимъ животъ вѣчный, а иже по рвенію противляющимся убо истинѣ, повинующимся же неправдѣ, ярость и гнѣвъ, скорбь и тѣснота на всяку душу человѣка творящаго злое, Іудеа же прежде и Еллина}\footnote{Рим.~2,~4--9.}.

\section{81. Милость царская, законопреступникамъ обѣщанная и публикованная, и обращающимся являемая.}

Бываетъ, что царь, хотячи помиловать и исправить преступниковъ закона и развращенныхъ людей, которыи и сами погибаютъ, и отечеству великій вредъ дѣлаютъ, обѣщанною милостію отвлекаетъ ихъ отъ злодѣйскихъ дѣлъ, и, къ постоянному житію приводя ихъ, указомъ вездѣ публикуетъ, что ежели они, самовольно признавъ свое законопреступленіе, исправятся и впредь порядочно будутъ жить, всѣ ихъ злодѣйскія дѣла оставятся имъ и не помянутся, но будутъ въ покоѣ и мирѣ жить, какъ и прочіи сыны отечества. И сподобляются таковыи милости отъ царя, которыи, почувствовавъ царскую къ себѣ милость, дѣлаютъ по публикованному указу. Тако Богъ, небесный Царь, Царь царствующихъ и Господь господствующихъ, хотячи грѣшниковъ помиловать, чрезъ Слово Свое святое публикуетъ имъ покаяніе, дабы отъ грѣховъ отстали и къ Нему обратилися съ истиннымъ жалѣніемъ и сокрушеніемъ сердца, и исправили бы себе. \textit{Покайтеся!} А когда то сотворятъ, то благій и человѣколюбивый Богъ вси имъ грѣхи, какіе ни сотворили, обѣщается отпустить, и не помянуть ихъ. Отпущу имъ беззаконія ихъ, и грѣховъ ихъ не имамъ помянути: \textit{аще беззаконникъ обратится отъ всѣхъ беззаконій своихъ, жизнію поживетъ и не умретъ; вся согрѣшенія, елика согрѣшилъ, не помянутся ему}\footnote{Іер.~31,~34; Іез.~18,~21 и 23.}. Сіе милостивое Свое обѣщаніе милосердый Богъ и \textit{клятвою Своею} утвердилъ, дабы бѣдныи грѣшники не сумнѣвались о обѣщаніи Его, и обращаться и съ покаяніемъ къ Нему приходить не боялись. \textit{Живу Азъ}, глаголетъ Господь, \textit{яко не хощу смерти грѣшника, но еже обратитися, и живу быти ему}\footnote{Іез.~31,~11; Евр.~6,~13,~17 и 18; Іез.~45,~22 и 23.}. Сію великую и безприкладную Божію милость, всему міру публикованную, послышали многіи грѣшники, многіи разбойники, блудники и блудницы и прочіи тяжкіи грѣшники, и съ любовію облобызали тую, и, бросивши мерзкія свои дѣла, покаяніемъ поспѣшили къ милосердому Создателю своему, и милость получили отъ Него и, покаяніемъ очистившися, сдѣлалися другами Его и вошли въ вѣчное Его царство. Грѣшникъ! ты что медлишь, слышачи такъ милостивую отъ небеснаго Отца публикацію? Се мытари и любодѣйцы кающіися восхищаютъ царство небесное: ты что пребываеши во грѣхахъ твоихъ и медлишь въ нихъ, какъ на чужой сторонѣ? Се Богъ, яко Отецъ чадолюбивый, ожидаетъ тебе! ну жъ, пріиди и ты въ себе и поспѣши покаяніемъ къ небесному Отцу. Воистину, говорю тебѣ, съ радостію пріиметъ тебе, и милосердыми очами любезно воззритъ на тебе, и \textit{милъ Ему будеши, и текъ нападетъ на выю твою, и облобызаетъ тебе}. Не бойся! не будетъ никакого выговора чинить тебѣ за неисправность и отлученіе твое; но повелитъ въ \textit{первую одежду} облещи тебе, и дать \textit{перстень} на руку твою, и \textit{сапоги} на ноги твои; тогда будетъ радость предъ ангелами Божіими и о тебѣ; тогда и о тебѣ слово оное скажется: \textit{сынъ Мой сей мертвъ бѣ, и оживе; и изгиблъ бѣ, и обрѣтеся}\footnote{Лук.~15,~20,~22 и 24.}. Поспѣшимъ убо, поспѣшимъ, бѣдный грѣшникъ, къ небесному Отцу, пока ждетъ и отверсты двери царствія Его!

\subsection{О томжде.}

Законопреступники, которыи публикованную царскую милость презрятъ, и въ своемъ злодѣйствіи ожесточатся, и отъ злыхъ и пагубныхъ дѣлъ отстать не восхотятъ, достойно царскую милость на великій гнѣвъ обращаютъ; и уже, судомъ обличенныи, по всей строгости закона казнены будутъ. Не хотѣли публикованной царской милости въ исправленіе и пользу свою принять: такъ дознаютъ правосудіе его и праведно мстительную руку его; тогда бо сильно возгорится царскій гнѣвъ на такихъ нечувственныхъ и неблагодарныхъ людей. Тако грѣшники, которыи нынѣ слышатъ проповѣдуемое покаяніе и кающимся являемую Божію милость, но не хотятъ чистосердечно обратиться и отъ грѣховъ отстать и покаяніе творити, вмѣсто милости Божія уже дознаютъ на себѣ правду Его и мстительную руку. \textit{Одождитъ на грѣшники сѣти: огнь и жупелъ, и духъ буренъ, часть чаши ихъ}\footnote{Пс.~10,~6.}. \textit{Страшливымъ и невѣрнымъ, и сквернымъ и убійцамъ, и блудъ творящимъ, и чары творящимъ, идоложерцемъ и всѣмъ лживымъ, часть имъ въ езерѣ горящемъ огнемъ и жупеломъ, еже есть смерть вторая}\footnote{Апок.~21,~8.}. Тогда они самымъ дѣломъ на себѣ дознаютъ страшныя оныя прещенія, которыя нынѣ въ святомъ Божіемъ словѣ слышатъ. Не хотятъ являемыя Божія милости на себѣ дознать: такъ уже дознаютъ мстительную правду Его. Нынѣ Богъ зоветъ и обѣщаетъ милость: тогда отъ Себе отошлетъ и изліетъ праведный гнѣвъ Свой на нихъ. Нынѣ глаголетъ: \textit{покайтеся}; тогда возглаголетъ: отвѣщайте. Нынѣ глаголетъ: \textit{пріидите ко Мнѣ}; тогда возглаголетъ: \textit{идите отъ Мене} "--- куды? \textit{во огнь вѣчный, уготованный діаволу и аггеломъ его}\footnote{Мѳ.~25,~41.}. Нынѣ слушаетъ кающихся и молящихся Ему; тогда грѣшники услышатъ: \textit{не вѣмъ васъ}; вы Мене не узнали, и Я васъ не узнаю; вы Мене не слушали, и Я васъ не слушаю. \textit{Понеже звахъ, и не послушасте; и простирахъ словеса, и не внимасте, но отметасте Мои совѣты, и Моимъ обличеніемъ не внимасте: убо и Азъ вашей погибели посмѣюся, порадуюся же, егда пріидетъ вамъ пагуба}\footnote{Притч. Сол.~1,~24--26.}. И паки: \textit{Звахъ васъ, и не послушасте; глаголахъ и преслушасте, и сотвористе лукавое предо Мною, и, яже не хотѣхъ, избрасте. Сего ради тако глаголетъ Господь: се работающіи Ми ясти будутъ, вы же взалчете; се работающіи Ми пити будутъ, вы же возжаждете; се работающіи Ми возрадуются, вы же посрамитеся; се работающіи Ми возвеселятся въ веселіи сердца, вы же возопіете въ болѣзни сердца вашего, и отъ сокрушенія сердца восплачетеся}\footnote{Ис.~65,~12--14.}. Грѣшники! убоимся прещеній Божіихъ; покаемся, пока проповѣдуется покаяніе; пріимемъ благодарно Божію милость, пока является; обратимся къ Богу, пока зоветъ и ожидаетъ; помолимся Ему, пока молящихся слушаетъ. \textit{По Хрістѣ убо молимъ, яко Богу молящу нами: молимъ} убо \textit{по Хрістѣ, примиритися съ Богомъ. Не вѣдѣвшаго бо грѣха по насъ грѣхъ сотвори, да мы будемъ правда Божія о немъ. Споспѣшествующе же и молимъ, не вотще благодать Божію пріяти вамъ. Глаголетъ бо: во время пріятно послушахъ тебе, и въ день спасенія помогохъ ти. Се нынѣ время благопріятно! се нынѣ день спасенія}\footnote{2~Кор.~5,~20 и 21; 6,~1 и 2.}! \textit{Пріидите ко Мнѣ вси труждающіися и обремененніи, и Азъ упокою вы. Возмите иго Мое на себе, и научитеся отъ Мене, яко кротокъ есмь и смиренъ сердцемъ: и обрящете покой душамъ вашимъ. Иго бо Мое благо, и бремя Мое легко есть}, глаголетъ Господь\footnote{Мѳ.~11,~28,~30.}. \textit{Или о богатствѣ благости Его и кротости и долготерпѣніи нерадиши, невѣдый, яко благость Божія на покаяніе тя ведетъ? По жестокости же твоей и непокаянному сердцу, собираеши себѣ гнѣвъ въ день гнѣва и откровенія праведнаго суда Божія, Иже воздастъ коемуждо по дѣломъ его: овымъ убо по терпѣнію дѣла благаго, славу и честь и нетлѣніе ищущимъ животъ вѣчный: а иже по рвенію противляющимся убо истинѣ, повинующимся же неправдѣ, ярость и гнѣвъ. Скорбь и тѣснота на всяку душу человѣка творящаго злое, Іудеа же прежде и Еллина}\footnote{Рим.~2,~4--9.}.

\section{82. Плевелы между пшеницею.}

Бываетъ, что плевелы растутъ между пшеницею. Что пшеница и плевелы: то благочестивый и нечестивый. Плевелы растутъ между пшеницею: тако нечестивыи имѣются и обращаются между благочестивыми. Плевелы сначала не распознаются отъ пшеницы; но потомъ, когда пшеница начнетъ созрѣвать и плодъ творить, "--- оказуются, что они не пшеница суть, но плевелы: тако нечестивыи не скоро познаются нечестивыми, но потомъ мало"=по"=малу оказывается ихъ нечестіе и злонравіе: \textit{отъ плодъ ихъ познаете ихъ}, глаголетъ Господь\footnote{Мѳ.~7,~16.}. Между плевелами и пшеницею великое разнствіе есть: тако между нечестивыми и благочестивыми. Плевелы, какъ видимъ, ни къ чему негодны; пшеница ко всему полезна: тако нечестивыи ни къ какому дѣлу не годятся, благочестивыи на все полезны. Плевелы, яко негодныи, во время жатвы собираются въ снопы и сожигаются; но пшеница собирается въ житницу: тако нечестивыи, яко безплодныи, при кончинѣ вѣка соберутся и свяжутся въ снопы, и предадутся вѣчному огню. Блудники и прелюбодѣи сквернители и студоложники составятъ свой снопъ; воры, хищники и грабители, чуждое добро похищающіи, составятъ свой снопъ; лживыи, хитрецы и лукавцы составятъ свой снопъ; клеветники, злорѣчивыи и ругатели, составятъ свой снопъ; лицемѣры, которыи оказываютъ себе внѣ святыми, но внутрь суть злы, свой снопъ составятъ: сіи и прочіи беззаконники, яко Господу своему безплодныи, ввергнутся въ вѣчный огнь на сожженіе. Но благочестивыи, праведныи, святіи, добрыи, яко плодъ слова Божія, ово сто, ово шестьдесятъ, ово тридесять въ терпѣніи творящіи, соберутся въ небесное царствіе, яко пшеница въ житницу. \textit{Сѣявый доброе сѣмя, есть Сынъ человѣческій: а село есть міръ: доброе же сѣмя, сіи суть сынове царствія: а плевелы, суть сынове непріязненніи: а врагъ, всѣявый ихъ, есть діаволъ: а жатва, кончина вѣка есть: а жатели, ангели суть. Якоже убо собираютъ плевелы, и огнемъ сожигаютъ: тако будетъ въ скончаніе вѣка сего. Послетъ Сынъ человѣческій ангелы своя, и соберутъ отъ царствія Его вся соблазны, и творящія беззаконіе. И ввергнутъ ихъ въ пещь огненну: ту будетъ плачь и скрежетъ зубомъ. Тогда праведницы просвѣтятся, яко солнце, въ царствіи Отца ихъ. Имѣяй уши слышати, да слышитъ}\footnote{Мѳ.~13,~37--43.}. Грѣшники! покаемся, да не, яко плевелы, вѣчнымъ огнемъ сгоримъ: и сотворимъ достойны плоды покаянія, да будемъ благодатію Божіею пшеница, которая въ небесную собирается житницу. \textit{Боже силъ! обрати ны, и просвѣти лице Твое, и спасемся}.

\section{83. За чѣмъ ты здѣ?}

Бываетъ, что человѣкъ, видя знакомаго, въ село или градъ или иное какое мѣсто пришедшаго, спрашиваетъ его: \textit{за чимъ ты здѣ?} Тако можно вопросить христіанина, который въ церковь святую или въ хрістіанство вошелъ ради спасенія, но о спасеніи нерадитъ; можно вопросить его: скажи, \textit{за чимъ ты здѣ? за чимъ} ты въ хрістіанство вошелъ, но не по"=хрістіански живеши? въ виноградъ Хрістовъ вошелъ, но не дѣлаеши? въ святое общество вступилъ, но не свято живеши? \textit{За чимъ ты здѣ?} Вѣдь ты пришелъ сюды не богатства, злата и сребра собирать, не чести и славы искать, не плоти и міру угождать, не грѣху и страстемъ работать, но Хрісту Господу работать и угождать, и тако спасеніе получить. \textit{Хрістосъ за всѣхъ умре, да живущіи не ктому себѣ живутъ, но умершему за нихъ и воскресшему}\footnote{2~Кор.~5,~15.}. Помяни, какіе ты обѣты чинилъ Богу, вступая въ хрістіанство, како отрицался сатаны и всѣхъ злыхъ дѣлъ его, како отрицался міра и всѣхъ прихотей его, како обѣщался работать Хрісту, Сыну Божію, умершему за тя и воскресшему: гдѣ жъ нынѣ твои обѣты? Помяни, како служитель Божій тогда привѣтствовалъ тебе великою милостію Божіею: \textit{омылся еси, оправдался еси, освятился еси именемъ Господа нашего Іисуса Хріста, и Духомъ Бога нашего}: гдѣ жъ убо обѣты тіи, которыи ты Богу и предъ Богомъ и святою церковію Его учинилъ? гдѣ тая правда и святость, которую ты туне, безъ всякихъ заслугъ твоихъ, и отъ единой Божіей милости получилъ? Покажи ми тую отъ дѣлъ твоихъ. Получилъ ты святость: гдѣ она? почто не свято живеши? Получилъ ты правду: гдѣ правда твоя? почто не твориши правды? Омылся ты святою оною банею отъ сквернъ грѣховныхъ: почто жъ паки оскверняешися грѣхами? \textit{Случися} убо и тебѣ истинная притча: \textit{песъ возвращся на свою блевотину, и свинія омывшися въ калъ тинный}\footnote{2~Петр.~2,~22.}. Якоже бо земледѣлецъ исходитъ на поле землю дѣлать, купецъ въ лавку приходитъ торговать, ученикъ въ школу приходитъ учитися, судья въ судебное мѣсто приходитъ судить и правду изыскивать, воинъ на брань исходитъ подвизаться противу враговъ и отечество свое защищать: тако хрістіанинъ входитъ въ хрістіанство свято жить, Хрісту Господу угождать, и вѣрою въ Него, отъ Него спасенія вѣчнаго искать. \textit{По звавшему вы Святому, и сами святи во всемъ житіи будите. Зане писано есть: святи будите, яко Азъ святъ есмь}\footnote{1~Петр.~1,~15 и 16.}. За чѣмъ убо ты здѣ, хрістіанине, ты, который оскверняешь тѣло и душу свою нечистотою; ты, который духомъ злобы и мщенія на ближняго твоего дышешь; ты, который простираешь руку свою на похищеніе чуждаго добра; ты, который проливаешь слезы вдовъ, сиротъ и прочихъ бѣдныхъ людей; ты, который на брата твоего клевещешь, и языкомъ своимъ, какъ мечемъ, біешь его; ты, который лжешь, обманываешь и хитростію своею прельщаешь ближняго твоего, и проч.? Все сіе, какъ тьма отъ свѣта, удалено отъ хрістіанъ: за чимъ убо ты здѣ? за чимъ со тьмою свѣтъ примѣшаешися? кое бо участіе свѣту ко тьмѣ? \textit{Не призва бо насъ Богъ на нечистоту, но во святость}\footnote{1~Сол.~4,~7.}. Хрістіанине! помяни обѣты твоя, бывшія при крещеніи святомъ, которыи ты не человѣку, но Богу учинилъ. Страшно есть Богу солгать! Богъ поруганъ не бываетъ. Взыщетъ Онъ отъ тебе обѣтовъ твоихъ, когда позоветъ тебе на судъ къ себѣ. Тамо ты увидишь, кому ты солгалъ и не сохранилъ обѣтовъ своихъ. Помяни убо оныя нынѣ и покайся и твори достойныя дѣла обѣтовъ твоихъ, да не со лжею на судѣ ономъ явишися и часть возъимѣеши со лживыми. \textit{Часть же имъ въ езерѣ, горящемъ огнемъ и жупеломъ, еже есть смерть вторая}\footnote{Апок.~21,~8.}. \textit{Востани спяй, и воскресни отъ мертвыхъ, и освѣтитъ тя Хрістосъ}\footnote{Еф.~5,~14.}.

\section{84. Ложь.}

Видимъ, что различна ложь въ мірѣ бываетъ. Лжетъ купецъ, когда сказуетъ, что товаръ его такой"=то цѣны стоитъ, но иначе есть. Лжетъ свидѣтель на судѣ, когда говоритъ тое, чего не видѣлъ и не слышалъ; или не говоритъ того, что видѣлъ и слышалъ, и черное называетъ бѣлымъ, и горькое сладкимъ. Лжетъ рабъ, когда господину говоритъ тое, чего не было. Лжетъ судія, который обѣщался и присягалъ по совѣсти на судѣ поступать, и хранить и изыскивать правду, но не дѣлаетъ того. Лжетъ работникъ, который, взявши достойную цѣну, обѣщался работать усердно наемшему его, но лѣниво работаетъ, или совсѣмъ не работаетъ. Лжетъ должникъ, который пріемлетъ деньги отъ заимодавца, и обѣщается ему отдать, но не отдаетъ. Лжетъ всякъ тотъ, кто нанимаетъ другихъ работать себѣ, и обѣщается имъ достойную дать цѣну за труды ихъ, но не даетъ. Лжетъ воинъ, который присягалъ Государю и обществу служить, но не дѣлаетъ того. Лжетъ пастырь, который обѣщается и присягаетъ стадо овецъ Хрістовыхъ пасти, но не пасетъ, или нерадиво пасетъ, и проч. Тако лжетъ хрістіанинъ, который въ крещеніи святомъ обѣщается Хрісту Господу работать, но не работаетъ. Таковый всякъ есть, который по крещеніи святомъ беззаконнуетъ и къ суетѣ міра сего прилѣпляется. Лжетъ Хрісту Господу блудникъ, прелюбодѣй и всякій сквернитель; лжетъ хищникъ, воръ и грабитель; лжетъ кощунникъ, сквернословецъ и буесловецъ; лжетъ злорѣчивый, клеветникъ и ругатель; словомъ: лжетъ всякій законопреступникъ. Ибо беззаконное и нехрістіанское житіе противно есть обѣтамъ тѣмъ, которые въ крещеніи отъ хрістіанъ бываютъ Богу. Всякъ бо хрістіанинъ въ крещеніи обѣщается свято, непорочно и благочестно о Хрістѣ Іисусѣ жить, и тако Ему работать: безъ святаго бо и благочестиваго житія не возможно Хрісту работать. Но когда не благочестиво, но беззаконно живетъ: Хрісту не работаетъ и обѣтовъ своихъ не хранитъ, и тако лжетъ Ему; и дотоль во лжи будетъ пребывать, доколѣ въ нераскаянномъ житіи будетъ жить. Хрістіанине! тяжко есть человѣку солгать, человѣку простому, тягчае чиновному, далеко тягчае царю: коль несравненно тягчае и страшнѣе Богу солгать! Помяни убо, кому ты обѣты своя чинилъ въ крещеніи, и, не сохранивши ихъ, кому солгалъ? Богу ты солгалъ, а не человѣку. Покайся убо, пока время не ушло, и исполни обѣты твоя, да и Господь пріиметъ тя въ милость Свою. \textit{Помилуй мя Боже, по велицѣй милости Твоей, и по множеству щедротъ Твоихъ очисти беззаконіе мое. Наипаче омый мя отъ беззаконія моего, и отъ грѣха моего очисти мя. Яко беззаконіе мое азъ знаю, и грѣхъ мой предо мною есть выну. Тебѣ единому согрѣшихъ, и лукавое предъ Тобою сотворихъ}, и проч.\footnote{Пс.~50,~3--6.}

\section{85. Кокошъ или курица.}

Видимъ, что кокошъ малыхъ птенцовъ своихъ собираетъ подъ крилы своя, и согрѣваетъ ихъ: тако Хрістосъ Господь нашъ хощетъ собрать грѣшниковъ подъ крилѣ благости Своея и согрѣть ихъ, якоже глаголетъ Іерусалиму: \textit{Іерусалиме, Іерусалиме}, и прочая, \textit{коль краты восхотѣлъ собрати чада твоя, якоже собираетъ кокошъ птенцы своя подъ крилѣ, и не восхотѣсте}\footnote{Мѳ.~23,~37.}. Хощетъ, говорю, собрати грѣшниковъ Хрістосъ Господь. И како не хощетъ Тотъ, Который по образу Своему и по подобію насъ сотворилъ? Како не хощетъ Тотъ, Который бѣдствію нашему состраждетъ и соболѣзнуетъ? Како не хощетъ Тотъ, Который пророковъ посылалъ къ намъ, призвати насъ къ Себѣ? Како не хощетъ Тотъ, Который и Самъ къ намъ ради насъ пришелъ, пожилъ на земли, трудился, страдалъ и умеръ за насъ? Како таковый любитель нашъ не хощетъ насъ собрати подъ крилы Своя? О грѣшники! хощетъ, хощетъ, и, простерши крилы благости Своея, ожидаетъ насъ. \textit{Пріидите ко Мнѣ вси труждающіися и обремененніи, и Азъ упокою вы}\footnote{11,~28.}. Собрались подъ крилы Его вси святіи, и упокоеваются подъ кровомъ крилъ Его. Грѣшники! что мы стоимъ? что разсѣявшеся блудимъ и скитаемся, яко птенцы, отлучившіися отъ матери своей? Ждетъ насъ Хрістосъ Господь, и, якоже кокошъ, хощетъ собирать подъ крилы Своя; слышимъ и гласъ Его зовущій: \textit{пріидите ко Мнѣ}; ну жъ, на гласъ Его побѣжимъ, якоже птенцы бѣгутъ на гласъ матере своея, и прибѣгнемъ къ Нему, и сокрыемся подъ кровомъ благости Его, по увѣщанію Псаломника: \textit{днесь, аще гласъ Его услышите, не ожесточите сердецъ вашихъ}\footnote{Пс.~94,~8.}; прибѣгнемъ съ покаяніемъ и жалѣніемъ, да не и намъ слово оное приличествуетъ, которое ожесточенному Іерусалиму отъ Него сказано: \textit{коль краты восхотѣлъ собрати чада твоя, якоже собираетъ кокошъ птенцы своя подъ крилѣ, и не восхотѣсте! Се оставляется вамъ домъ вашъ пустъ}. Страшное слово есть: \textit{се оставляется вамъ домъ вашъ пустъ!} Убоимся убо того, и возвратимся къ Пастырю и Посѣтителю душъ нашихъ: кромѣ Его бо нигдѣ не сыщемъ себѣ покоя; безъ Него нѣтъ покоя, безъ Него нѣтъ спасенія, но явная бѣда и погибель. Обратимся убо къ Нему, пока ждетъ и зоветъ и хощетъ приняти подъ крилы благости Своея. \textit{Грядущаго ко Мнѣ не иждену вонъ}\footnote{Іоан.~6,~37.}. \textit{Пріидите ко Мнѣ вси труждающіися и обремененніи, и Азъ упокою вы}.

\section{86. Пластырь живительный.}

Что пластырь живительный язвѣ или ранѣ тѣлесной, тое душѣ грѣшной, страхомъ суда Божія и печалію за грѣхи уязвившейся, есть Евангеліе. Язва тѣлесная ослабленіе болѣзни и исцѣленіе отъ живительнаго пластыря получаетъ: тако душа грѣшная, трепещущая суда Божія, страхомъ того и печалію сокрушенная, ослабленіе и живое утѣшеніе и исцѣленіе отъ Евангелія получаетъ. И сіе"=то есть, что Псаломникъ воспѣлъ о человѣколюбіи Божіи: \textit{исцѣляяй сокрушенныя сердцемъ, и обязуяй сокрушенія ихъ}\footnote{Пс.~146,~3.}. Не бойся, грѣшная, но кающаяся и болѣзнующая за грѣхи душа! вѣруй только во Евангеліе, и почувствуеши въ сердцѣ твоемъ живность того. Разсуди самъ: ради кого Хрістосъ Сынъ Божій въ міръ пришелъ? ради грѣшниковъ. Ради кого на земли жилъ и трудился? ради грѣшниковъ. Ради кого пострадалъ, умеръ и воскресъ? ради грѣшниковъ, якоже Апостолъ къ утѣшенію нашему глаголетъ: \textit{вѣрно слово, и всякаго пріятія достойно, яко Хрістосъ Іисусъ пріиде въ міръ грѣшники спасти}\footnote{1~Тим.~1,~15.}. О безмѣрнаго человѣколюбія Божія! О великаго утѣшенія грѣшникамъ! Самъ Богъ и Господь ради ихъ въ міръ пришелъ, взыскати и спасти ихъ погибшихъ. Ты единъ отъ грѣшныхъ и погибшихъ еси: убо и тебе пришелъ взыскати и спасти. Держись только вѣрою Пришедшаго взыскати и спасти грѣшниковъ, и утверждай надежду свою на Немъ: то неотмѣнно взыщетъ и спасетъ и тебе. И не сказано: ради таковыхъ и такихъ"=то грѣшниковъ пришелъ Хрістосъ, но ради всѣхъ и всякихъ, коль великіи и тяжкіи ни были бы. Едино только отъ нихъ требуется, чтобы отстали отъ грѣховъ и каялись и сокрушенное сердце имѣли; а духовный и живительный Евангелія пластырь ранамъ ихъ приложится, и почувствуютъ въ сердцахъ своихъ радостное о милости Божіей къ нимъ извѣстіе. Уже бо они начинаютъ приносить жертву Богу на олтарѣ сердца своего "--- духъ сокрушенъ. \textit{Жертва Богу духъ, сокрушенъ: сердце сокрушенно и смиренно Богъ не уничижитъ}\footnote{Пс.~50,~19.}. Грѣшная душа, страхомъ и печалію сокрушенная! представь внутреннимъ твоимъ очамъ, како благоутробный оный отецъ, въ притчѣ Евангельской упоминаемый, юнѣйшаго своего сына, отъ заблужденія возвратившагося, пріемлетъ! како любезно еще издалеча смотритъ на него, како срѣтаетъ его, како нападаетъ на выю его, како объемлетъ и облобызаетъ его! како повелѣваетъ рабамъ своимъ изнести одежду первую и облещи его, и дати перстень на руку его и сапоги на ноги, и велитъ всей своей фамиліи радоватися и веселитися о немъ! Ахъ, сынъ мой возвратился ко мнѣ! сынъ мой цѣлъ и здравъ пришелъ ко мнѣ! \textit{Сынъ мой, сей мертвъ бѣ, и оживе; и изгиблъ бѣ, и обрѣтеся}\footnote{Лук.~15,~20--25.}. Тако Богъ, небесный Отецъ, любезно пріемлетъ грѣшника кающагося, и съ сокрушеніемъ сердца приходящаго къ Нему, и грѣховъ и беззаконій первыхъ не поминаетъ ему, и велитъ всей небесной Своей фамиліи радоватися о немъ. \textit{Тако}, глаголетъ Господь, \textit{радость бываетъ предъ ангелы Божіими о единомъ грѣшницѣ кающемся}\footnote{ст. 10.}. И воистину тако есть. Аще бо, ради грѣшника, Сына Своего не пощадѣлъ, но на смерть предалъ: како грѣшника кающагося не пріиметъ? \textit{Иже Своего Сына не пощадѣ, но за насъ всѣхъ предалъ есть Его: како убо не и съ Нимъ вся намъ дарствуетъ}\footnote{Римл.~8,~32.}? Како не помилуетъ насъ, просящихъ у Него милости? како не прииметъ насъ, обращающихся къ Нему? како не отпуститъ намъ согрѣшеній нашихъ? како не спасетъ насъ? \textit{Аще Богъ по насъ, кто на насъ}\footnote{31.}? \textit{Богъ спасеній нашихъ, Богъ нашъ, Богъ еже спасати}\footnote{Пс.~67,~20 и 21.}. Не тако отецъ, не тако матерь надъ больнымъ чадомъ своимъ умилостивляется, якоже милосердый Богъ умилостивляется надъ грѣшникомъ, въ сокрушеніи и въ болѣзни сердца къ Нему воздыхающимъ. Тогда мятется утроба Его надъ нимъ; тогда милуя помилуетъ его: якоже бо величество Его, тако и милость Его. Слава щедротамъ Его! Слава благости Его! Слава человѣколюбію Его! \textit{Буди имя Господне благословенно отъ нынѣ и до вѣка! Хвалите Господа вси языцы, похвалите Его вси людіе! Яко утвердися милость Его на насъ, и истина Господня пребываетъ во вѣкъ. Пріидите}, грѣшники, \textit{и припадемъ и восплачемся предъ Господемъ, сотворшимъ насъ: яко Той есть Богъ нашъ, и мы людіе пажити Его, и овцы руки Его. Яко у Господа милость, и многое у Нею избавленіе: и Той избавитъ Израиля отъ всѣхъ беззаконій его}\footnote{Пс.~116,~1 и 2; 94,~6 и 7; 129,~6.}. Отсюду видишь, хрістіанине, кому Евангеліе съ утѣшеніемъ своимъ приличествуетъ; а именно, не блудникамъ, не прелюбодѣямъ, не ворамъ, хищникамъ и грабителямъ, и прочіимъ людямъ, беззаконно живущимъ; имъ глаголется: \textit{покайтеся; постраждите и слезите и плачитеся; смѣхъ вашъ въ плачь да обратится, и радость въ сѣтованіе}\footnote{Іак.~4,~9.}. Когда сіе сотворятъ, тогда и они Евангеліемъ, какъ по постѣ пищею, утѣшаться будутъ; тогда и ихъ уязвленному сердцу Евангеліе, какъ живительный пластырь ранѣ, приложится: Кому жъ убо Евангеліе приличествуетъ? Отвѣтъ: грѣшникамъ, отъ грѣховъ обратившимся къ Богу, за содѣянные грѣхи болѣзнующимъ, суда Божія боящимся, печалію сокрушеннымъ, милости Божія ищущимъ, и со смиреніемъ Ему припадающимъ; откуду писано есть: \textit{покайтеся и вѣруйте во Евангеліе}\footnote{Марк.~1,~15.}. Видимъ, что прежде покаяніе предлагается, а потомъ Евангеліе. Безъ покаянія бо Евангеліе ничего не пользуетъ. Евангеліе бо утѣшеніе намъ приноситъ. Но на что тому утѣшеніе, который печальнаго и сокрушеннаго сердца не имѣетъ? И сіе"=то есть, что Хрістосъ Господь глаголетъ о Себѣ: \textit{благовѣстити нищимъ посла Мя, исцѣлити сокрушенныя сердцемъ}, и проч.\footnote{Лук.~4,~19.} Видишь, кому Хрістосъ благовѣствуетъ. Покайся убо, и обратись отъ грѣховъ къ Богу, и имѣй сокрушенное сердце: тогда и тебѣ Евангеліе святое будетъ приличествовать, какъ язвѣ тѣлесной живительный пластырь.

\section{87. Куды ты идешь? тамо тебѣ бѣда будетъ.}

Бываетъ, что человѣкъ человѣка, къ нѣкоей опасности идущаго, остерегаетъ, глаголя: \textit{куды ты идешь? тамо тебѣ будетъ бѣда}: тако святіи хрістіанскія книги грѣшника неисправнаго, стремящагося въ погибель, остерегаютъ и удерживаютъ его, глаголя: бѣдный грѣшникъ! \textit{куды ты идешь? тамо тебѣ бѣда будетъ}. Куды идешь? Какая бо большая можетъ быть бѣда, какъ вѣчная погибель и вѣчная смерть? Какъ бо едино истинное добро есть вѣчное добро: тако едино истинное зло есть вѣчное зло. Куды убо ты, бѣдный грѣшникъ, идешь? въ погибель! Блудникъ, прелюбодѣй и нечистоты любитель! куды ты идешь? въ погибель! Злобникъ, и отмщеніемъ на брата своего дышущій и кровію человѣческою руки своя оскверняющій! куды ты идешь? въ погибель! Тать, хищникъ и грабитель! куды ты идешь? въ погибель! Клеветникъ, ругатель, злословецъ и сквернословецъ! Куды ты идешь? въ погибель! Піяница и всякій безчинникъ! куды ты идешь? въ погибель! Хитрецъ, льстецъ и обманщикъ! куды ты идешь? въ погибель! Пастырь нерадивый и соблазняющій! куды ты идешь? въ погибель! Въ погибель и съ собою многихъ влечешь, многихъ, за которыхъ кровь Свою Хрістосъ изліялъ. Судія клятвопреступникъ и мздоимецъ! куды ты идешь? въ погибель; и своимъ примѣромъ подкоманднымъ своимъ служителямъ туда же путь стелешь! Властелинъ, прихотямъ своимъ, а не обществу служащій и своей корысти, а не общей пользы ищущій! куды ты идешь? въ погибель Помѣщикъ, крестьянъ своихъ мучащій или обременяющій! куды ты идешь? въ погибель! Купецъ лживый, дешевую вещь за дорогую, и худую за добрую, и гнилую за здоровую продающій! куды ты идешь? въ погибель! Наемникъ, достойную цѣну пріемлющій, и не работающій, или лицемѣрно работающій! куды ты идешь? въ погибель! Сребролюбецъ, удерживающій мзду наемнику! куды ты идешь? въ погибель! Чародѣй и призывающій его въ помощь себѣ! куды ты идешь? въ погибель! Грѣшникъ всякъ неисправный и нераскаянный! куды ты идешь? въ погибель! Всякаго бо человѣка житіе, пока въ мірѣ семъ живетъ, путь ему есть. Каково житіе, таковъ и путь его; и како живетъ, тако и идетъ; и каковое житіе имѣетъ, такимъ путемъ и идетъ, и идетъ или къ вѣчному блаженству, или къ вѣчной погибели. Благочестиво ли о Хрістѣ Іисусѣ живетъ, "--- путемъ благочестивымъ идетъ, и идетъ въ вѣчную жизнь. Беззаконно ли живетъ, "--- путемъ беззаконія идетъ, и идетъ въ вѣчную погибель. Ахъ, идетъ въ погибель, идетъ тотъ, за котораго Хрістосъ умеръ; идетъ и впадаетъ, аще не возвратится! Бѣдный грѣшникъ! удержись, пока еще на пути находишися: возвратись, пока еще не впалъ. Какъ впадешь, то уже и не изыдешь; будешь во вѣки, яко мертвецъ, въ погибели и въ смерти вѣчной. Возвратись взадъ! Се Хрістосъ Господь, Который за тебе пострадалъ и умеръ, отзываетъ тебе и призываетъ къ покаянію. \textit{Не пріидохъ приземли праведники, но грѣшники на покаяніе}\footnote{Мѳ.~9,~13.}. Обратися убо, и взойди на путь покаянія, путь ведущій въ вѣчный животъ, да, идучи по тому, блаженно скончаешися и внидеши въ вѣчную жизнь. \textit{Не льстите себе: ни блудницы, ни идолослужители, ни прелюбодѣи, ни сквернители, ни малакіи, ни мужеложницы, ни лихоимцы, ни татіе, ни піяницы, ни досадители, ни хищницы, царствія Божія не наслѣдятъ}\footnote{1~Кор.~6,~9 и 10.}. \textit{Страшливымъ и невѣрнымъ, и сквернымъ и убійцамъ, и блудъ творящимъ, и чары творящимъ, идоложерцемъ и всѣмъ лживымъ, часть имъ въ езерѣ горящемъ огнемъ и жупеломъ, еже есть смерть вторая}\footnote{Апок.~21,~8.}. \textit{Пріидите ко Мнѣ вси труждающіися и обремененніи, и Азъ упокою вы}, и проч.\footnote{Мѳ.~11,~28.}

\section{88. Ты чего сталъ?}

Бываетъ, что, когда люди собравшіися на какое дѣло работаютъ и трудятся, и единъ отъ нихъ разлѣнившися стоитъ; тогда другій ему говоритъ: \textit{ты чего сталъ?} Тако наченшему, но ослабѣвающему въ подвигѣ благочестія, можно къ поощренію сказать: прочіи хрістіане въ дѣлѣ своемъ трудятся, а \textit{ты чего сталъ?} Прочіи со усердіемъ каются, и воздыхаютъ за грѣхи своя: \textit{ты чего сталъ?} Прочіи подвизаются противу грѣха, діавола, міра и страстей: \textit{ты чего сталъ?} Прочіи прилѣжно дѣлаютъ въ виноградѣ Господни: \textit{ты чего сталъ?} Прочіи со усердіемъ работаютъ Господеви: \textit{ты чего сталъ?} Прочіи со тщаніемъ молятся, просятъ и толкутъ въ двери милосердія Божія: \textit{ты чего сталъ?} Прочіи сѣютъ сѣмена своя духовная: \textit{ты чего сталъ?} Прочіи щедрятъ и даютъ милостыню: \textit{ты чего сталъ?} Прочіи благотворятъ ближнимъ своимъ: \textit{ты чего сталъ?} Прочіи богатство, честь, славу и всякую суету міра сего презирая, на вѣчное блаженство взираютъ и текутъ къ тому: \textit{ты чего сталъ?} Прочіи смиряются предъ Богомъ и человѣками: \textit{ты чего сталъ?} Прочіи съ любовію работаютъ ближнимъ своимъ: \textit{ты чего сталъ?} Прочіи къ почести вышняго званія идутъ и текутъ: ты чего смотришь, \textit{чего сталъ}, и не идешь? Прочіи емлются за вѣчную жизнь: \textit{ты чего сталъ}, и не емлешися? Прочіи, отрекшися себе и вземше крестъ свой, идутъ за Хрістомъ, и послѣдуютъ Ему терпѣніемъ, любовію, кротостію и смиреніемъ, и идутъ за Нимъ въ вѣчную жизнь, жизнь, идѣже есть всѣхъ веселящихся жилище; идѣже есть радость, веселіе, сладость, честь, слава и блаженство вѣчное; идѣже видится Богъ лицемъ къ лицу; откуду отбѣже болѣзнь, печаль и воздыханіе; гдѣ слышится гласъ радости и веселія, слышится гласъ и шумъ празднующихъ и веселящихся, гдѣ поютъ непрестанно и сладко пѣснь: \textit{аллилуія}. Гдѣ сладкое и блаженное дружество со ангелами, туды идутъ послѣдующіи Хрісту, яко овцы пастырю: \textit{овцы Моя гласа Моего слушаютъ, и Азъ знаю ихъ, и по Мнѣ грядутъ. И Азъ животъ вѣчный дамъ имъ, и не погибнутъ во вѣки, и не восхититъ ихъ никтоже отъ руки Моея}, глаголетъ Господь\footnote{Іоан.~10,~27 и 28.}. \textit{Ты чего сталъ?} для чего туды же въ покойное, мирное и сладкое мѣсто не спѣшишь? Туды пошли и вошли вси святіи; вошли патріархи, пророки, апостоли, святители, мученицы; преподобныи и вси, отъ вѣка вѣрою угодившіи Богу, и водворяются тамо, и радуются и веселятся, и ожидаютъ общаго воскресенія и совершеннѣйшаго блаженства, по неложному Божію обѣщанію: \textit{ты чего сталъ?} чего для туды же не поспѣшаешь? Они ожидаютъ насъ съ великимъ желаніемъ, чтобы туды шли и пришли мы. Ждетъ и небесный Отецъ; ждетъ, Который Сына Своего ради насъ не пощадѣлъ, но на смерть предалъ Его; ждетъ, которыхъ по образу Своему и по подобію сотворилъ; ждетъ, да ублажитъ насъ вѣчнымъ блаженствомъ во царствіи Своемъ. \textit{Чего} убо \textit{стоимъ мы?} чего ради дремлемъ? чего ради не идемъ? чего ради туды не спѣшимъ? какое блаженство видимъ въ мірѣ семъ? какой покой, миръ и добро? какое веселіе и радость? \textit{Суета суетствій, и всяческая суета}\footnote{Еккл.~1,~2.}. Бѣды во градѣхъ, бѣды въ селѣхъ, бѣды въ пустыняхъ, бѣды на земли, бѣды на морѣ, бѣды отъ явныхъ, бѣды отъ тайныхъ враговъ, бѣды отъ языка, бѣды отъ рукъ человѣческихъ, бѣды отъ плоти и страстей воюющихъ на душу, бѣды отъ діавола, который, \textit{яко левъ рыкая, ходитъ, искій кого поглотити}; бѣды отъ лживыхъ прелестниковъ и обманщиковъ, бѣды отъ сродниковъ и отъ лжебратіи, бѣды отъ міра всего. \textit{Міръ весь во злѣ лежитъ}\footnote{1~Іоан.~5,~19.}. Что въ мірѣ не горестно? Богатство съ нищетою сопряжено, слава съ безславіемъ, честь съ безчестіемъ, покой не безъ безпокойствія, миръ не безъ смущенія и мятежа, любовь лицемѣрна, дружество не безъ подозрѣнія, сладость растворенная горестію. Отвсюду страхъ, отвсюду боязнь, подозрѣніе, опасность и печаль. Нѣтъ истины, нѣтъ вѣрности. Умалишася истины отъ сыновъ человѣческихъ. Воистину \textit{весь міръ во злѣ лежитъ}. Чего убо чаемъ въ мірѣ семъ добраго? \textit{Суета суетствій, и всяческая суета}. Суетная глагола кійждо ко искреннему своему; устнѣ льстивыя въ сердцѣ, и въ сердцѣ глаголаша злая. Увы мнѣ! яко пришельствіе мое продолжися. О Іисусе, сладосте жизненная, радосте и утѣхо любящихъ Тя! Къ Тебѣ, живущему на небеси, плачевныя моя очи возвожу отъ горькой и плачевной сей юдоли: влецы мене за Собою!.. Побѣжимъ, идѣже есть всѣхъ веселящихся жилище. \textit{Къ Тебѣ возведохъ очи мои, живущему на небеси. Се яко очи рабъ въ руку господій своихъ, яко очи рабыни въ руку госпожи своея: тако очи наши ко Господу Богу нашему, дондеже ущедритъ ны. Помилуй насъ, Господи, помилуй насъ, яко помногу исполнихомся уничиженія: наипаче наполнися душа наша поношенія гобзующихъ и уничиженія гордыхъ}\footnote{Пс.~122,~1--4.}.

\section{89. Постъ.}

Есть постъ тѣлесный, какъ видимъ; есть постъ и душевный. Тѣлесный постъ есть, когда чрево постится отъ пищи и питія: душевный постъ есть, когда душа воздерживается отъ злыхъ помысловъ, дѣлъ и словъ. Изрядный постникъ есть, кто удерживаетъ себе отъ блуда, прелюбодѣянія и всякія нечистоты. Изрядный постникъ есть, кто воздерживаетъ себе отъ гнѣва, ярости, злобы и мщенія. Изрядный постникъ есть, кто наложилъ языку своему воздержаніе, и удерживаетъ его отъ празднословія, сквернословія, буесловія, клеветы, осужденія, льсти, лжи и всякаго злорѣчія. Изрядный постникъ есть, кто руки свои удерживаетъ отъ воровства, хищенія, грабленія, и сердце свое отъ желанія чужихъ вещей. Словомъ, добрый постникъ есть, кто отъ всякаго удаляется зла. Видишь, хрістіанине, постъ душевный. Полезенъ намъ постъ тѣлесный, яко служитъ намъ ко умерщвленію страстей; но постъ душевный неотмѣнно нуженъ такъ, что и тѣлесный постъ безъ него ничтоже есть. Многіи постятся тѣломъ, но не постятся душею; многіи постятся отъ пищи и питія, но не постятся отъ злыхъ помысловъ, дѣлъ и словъ, и какая имъ отъ того польза? Многіи постятся чрезъ день и два и болѣе, но отъ гнѣва, злопомнѣнія и мщенія поститься не хотятъ; многіи воздерживаются отъ вина, мяса, рыбы, но языкомъ своимъ людей, подобныхъ себѣ, кусаютъ, и какая имъ отъ того польза? Суть такіи, которыи часто не касаются руками снѣдей, но тыя простираютъ на мздоимство, хищеніе и грабленіе чуждаго добра, и какая имъ отъ того польза? Истинный бо и прямый постъ есть воздержаніе отъ всякаго зла. Аще убо хощеши, хрістіанине, чтобы тебѣ постъ полезенъ былъ: то, постяся тѣлесно, постися и душевно, и постися всегда. Якоже убо налагаешь постъ чреву твоему: наложи злымъ мыслямъ и прихотямъ твоимъ. Да постится умъ твой отъ суетныхъ помышленій; да постится память отъ злопомнѣнія; да постится воля твоя отъ злаго хотѣнія; да постятся очи твои отъ худаго видѣнія: \textit{отврати очи твои, еже не видѣти суеты}; да постятся уши твои отъ скверныхъ пѣсней и шептаній клеветническихъ; да постится языкъ твой отъ клеветы, осужденія, кощунства, лжи, лести, сквернословія, и всякаго празднаго и гнилаго слова; да постятся руки твои отъ біенія и хищенія чуждаго добра; да постятся ноги твои отъ хожденія на злое дѣло. \textit{Уклонися отъ зла, и сотвори благо}\footnote{Пс.~33,~15; 1~Петр.~3,~11.}. Се есть хрістіанскій постъ, каковаго Богъ нашъ отъ насъ требуетъ. Покайся убо, и, воздерживая себе отъ всякаго злато слова, дѣла и помышленія, поучайся всякой добродѣтели, и будеши всегда предъ Богомъ поститися. \textit{Аще въ судѣхъ и сварѣхъ поститеся, и біете пястми смиреннаго: вскую Мнѣ поститеся, якоже днесь, еже услышану быти съ воплемъ гласу вашему? Не сицеваго поста Азъ избрахъ, и дне, еже смирити человѣку душу свою, ниже аще слячеши яко серпъ выю твою, и вретище и пепелъ постелеши, ниже тако наречете постъ пріятенъ. Не таковаго бо поста Азъ избрахъ, глаголетъ Господь, но разрѣшай всякъ соузъ неправды, разрушай обдолженія насильныхъ писаній, отпусти сокрушенныя въ свободу, и всякое писаніе неправедное раздери}.

\textit{Раздробляй алчущимъ хлѣбъ твой, и нищія безкровныя введи въ домъ твой; аще видиши нага, одѣй, и отъ свойственныхъ племене твоего не презри. Тогда разверзется рано свѣтъ твой, и исцѣленія твоя скоро возсіяютъ: и предъидетъ предъ тобою правда твоя, и слава Божія объиметъ тя. Тогда воззовеши, и Богъ услышитъ тя, и еще глаголющу ти, речетъ: се пріидохъ. Аще отъимеши отъ себе соузъ, и рукобіеніе, и глаголъ роптанія, и даси алчущему хлѣбъ отъ души твоея, и душу смиренную насытиши; тогда возсіяетъ во тмѣ свѣтъ твой, и тма твоя будетъ яко полудне}, и проч.\footnote{Ис.~58,~4--10.}

\section{90. Отверженіе Хріста.}

Люди то только отверженіе Хріста разумѣютъ, когда хрістіанинъ устами и словомъ Хріста отвергается. И подлинно, сіе есть отверженіе Хріста, и есть тяжкое и пагубное дѣло, ибо отрещися Хріста есть отрещися живота, и повергнуть себе въ явную смерть. Но есть и другое подобное тому отверженіе Хріста. А оно тое есть, которое бываетъ дѣлами и беззаконнымъ житіемъ. О семъ отверженіи глаголетъ Апостолъ: \textit{Бога исповѣдуютъ вѣдѣти, а дѣлы отмещутся Его, мерзцы суще и непокориви, и на всяко дѣло благое неискусни}\footnote{Тит.~1,~16.}. Есть убо отверженіе Хріста языкомъ, есть отверженіе и дѣломъ. Не ласкай убо, хрістіанине, себе, что ты хрістіанинъ, когда не благочестиво, но беззаконно живеши. Кто презираетъ повелѣніе, тотъ презираетъ и оставляетъ повелѣвающаго. Невѣрный рабъ господину своему есть, который послушанія ему не показуетъ. Какій есть рабъ Хрістовъ тотъ хрістіанинъ, который Хріста не слушаетъ? \textit{Что Мя зовете}, глаголетъ Господь, \textit{Господи, Господи, и не творите, яже глаголю}\footnote{Лук.~6,~46.}? И паки глаголетъ: \textit{иже нѣсть со Мною, на Мя есть}\footnote{Мѳ.~12,~30.}. Всякъ беззаконнующій хрістіанинъ не со Хрістомъ есть, убо есть противу Хріста, убо есть противникъ Хрістовъ, убо отреклся Хріста, хотя и глаголетъ Ему: \textit{Господи, Господи}. Блудодѣйствуеши ли, хрістіанине, прелюбодѣйствуеши и инымъ образомъ душу и тѣло твое оскверняеши: отвергаешися Хріста. Держиши ли на ближняго твоего гнѣвъ, и умышляеши, какъ его повредить: отвергаешися Хріста. Простираеши руку твою на похищеніе и грабленіе чуждаго добра: отвергаешися Хріста. Льстиши ли, лукавнуеши и обманываеши ближняго твоего: отвергаешися Хріста. Клевещеши ли, ругаеши и поносиши ближняго твоего: отвергаешися Хріста. Словомъ, всякое беззаконное дѣло противу совѣсти и отъ умысла сотворяя, отвергаешися Хріста; и столько разъ страстной твоей похоти жрешь, какъ идолу, сколько разъ ее слушаешь. Отрекись убо похоти, убѣждающей тя на злое дѣло, и слушай Хріста, и не будешь отрицаться Его. Златыхъ, сребряныхъ, мѣдяныхъ и древянныхъ идоловъ удобно храниться; но отъ тѣхъ, которыи внутрь насъ суть, то"=есть въ сердцѣ нашемъ, весьма неудобно. \textit{Иже Хрістовы суть, плоть распяша со страстьми и похотьми}\footnote{Гал.~5,~24.}. Послѣдовательно не Хрістовы суть, который не распяша плоти со страстьми и похотьми. А когда не Хрістовы суть: то уже разсуди, что они суть предъ Богомъ? какая ихъ молитва? какая ихъ жертва и приношеніе? О, колико есть ложныхъ хрістіанъ! колико подъ именемъ хрістіанскимъ идолопоклонниковъ крыется! Отъ плода древо познается: и хрістіанинъ отъ вѣры и добрыхъ дѣлъ. \textit{Покажи ми вѣру твою отъ дѣлъ твоихъ}\footnote{Іак.~2,~18.}, требуетъ отъ тебе, хрістіанине, Апостолъ. Обратися убо, христіанине, и покайся, умерщвляй плоть твою со страстьми и похотьми, да будеши истинный хрістіанинъ, и добрую надежду возъимѣеши о Хрістѣ Іисусѣ.

\section{91. Рабъ, вѣдущій и невѣдущій волю господина своего.}

Бываетъ въ мірѣ, что раби иныи знаютъ волю господина своего, и не творятъ ея; иныи не знаютъ и не творятъ. И за то наказаніе отъ господина своего пріемлютъ: знающій волю господина своего и не творящій ея, наказуется жестоко; не знающій и не творящій ея, наказуется менѣе, якоже глаголетъ Господь: \textit{той рабъ вѣдѣвый волю господина своего, и не уготовавъ, ни сотворивъ по воли его, біенъ будетъ мною. Не вѣдѣвый же, сотворивъ же достойная ранамъ, біенъ будетъ мало}\footnote{Лук.~12,~47 и 48.}. Тако имѣются предъ Хрістомъ Господемъ язычники и хрістіане. Язычники, незнающіи Хріста Господа, не знаютъ и воли Его, и не творятъ ея, и наказаны будутъ въ будущемъ вѣкѣ менѣе, нежели хрістіане злыи. Хрістіане знаютъ Хріста, пришедшаго въ міръ, и всегда слышатъ проповѣдуемое слово Божіе, и въ немъ волю Хріста Господа своего; но ее творить не хотятъ, и потому въ будущемъ вѣкѣ мучены будутъ жестоко, и болѣе, нежели язычники, идолопоклонники и Турки. Горе тогда будетъ хрістіанамъ, познавшимъ Хріста, но не творящимъ волю Хрістову! Тогда они услышатъ отъ Хріста: \textit{глаголю вамъ: не вѣмъ васъ, откуду есте; отступите отъ Мене вси дѣлателіе неправды}\footnote{13,~27.}. И сіе"=то есть, что сказано: \textit{яко Содомляномъ въ день той отраднѣе будетъ, неже граду тому}\footnote{10,~12.}; то"=есть, незнающимъ Бога и не слышащимъ Божія слова, но беззаконнующимъ, отраднѣе тогда будетъ, нежели беззаконнымъ хрістіанамъ, которыи слышатъ Божіе слово, и Бога исповѣдуютъ вѣдѣти, но дѣлы отмещутся Его. \textit{Не всякъ глаголяй ми: Господи Господи, внидетъ въ царствіе небесное: но творяй волю Отца Моего, Иже есть на небесѣхъ}\footnote{Мѳ.~7,~21.}. Покайся убо, хрістіанине, и сотвори волю Хрістову, Который тебе ради пострадалъ и умеръ, да не болѣе язычниковъ въ день судный осудишися.

\section{92. Осмотрись.}

Слышимъ часто, что люди людямъ въ различныхъ случаяхъ говорятъ: \textit{осмотрись!} Хрістіанине, который Бога исповѣдуешь, и слышишь слово Его святое, и знаешь, что будетъ грѣшникамъ вѣчная мука, и праведнымъ вѣчная жизнь! тебѣ наипаче прилично сказать слово сіе: \textit{осмотрись}. Отреклся ты сатаны и всѣхъ злыхъ дѣлъ его на крещеніи: не обратился ли паки къ нему злымъ и развращеннымъ житіемъ своимъ? \textit{Осмотрись!} Обѣщался ты и присягалъ тогда работать вѣрою и правдою Хрісту Господу, умершему за тя и воскресшему, но вмѣсто того не работаеши ли грѣху и міру? \textit{Осмотрись! Всякъ творяй грѣхъ, рабъ есть грѣха}\footnote{Іоан.~8,~34.}. \textit{Не возможно Богу работати и мамонѣ}\footnote{Мѳ.~6,~24.}. Омылся ты тогда, освятился, оправдался именемъ Господа нашего Іисуса Хріста, и Духомъ Бога нашего; но не осквернился ли и не оскверняешися ли паки беззаконными дѣлами? \textit{Осмотрись!} Таковымъ приличествуетъ истинная притча: \textit{песъ возвращся на свою блевотину, и свинія омывшися въ калъ тинный}\footnote{2~Петр.~2,~22.}. Словомъ Божіимъ позванъ и святымъ крещеніемъ обновленъ ты къ вѣчной жизни и небеснымъ благимъ: не мудрствуеши ли земная? не ищеши ли въ мірѣ семъ прославитися, обогатитися, въ честь произыти? не прилѣпляешися ли сердцемъ своимъ къ суетѣ міра сего? не идеши ли путемъ нечестивыхъ въ погибель? \textit{Осмотрись! Пространная врата и широкій путь вводяй въ пагубу, и мнози суть входящіи имъ. Что узкая врата, и тѣсный путь вводяй въ животъ, и мало ихъ есть, иже обрѣтаютъ его}, глаголетъ Господь\footnote{Мѳ.~7,~13 и 14.}. Богъ на всякомъ мѣстѣ есть, и на всѣхъ насъ призираетъ, и всякія наши слова слышитъ, и дѣла и помышленія наша видитъ, и сердца и утробы испытуетъ, якоже святое слово Его проповѣдуетъ, и святая вѣра наша научаетъ насъ: ты како предъ Богомъ вездѣсущимъ и вся назирающимъ обращаешися? како говориши предъ Тѣмъ, Который всякое слово твое слышитъ? како дѣлаеши и мыслиши предъ Тѣмъ, Который всякое твое дѣло и помышленіе видитъ, и все въ книзѣ Своей записываетъ, и въ день суда Своего объявитъ тебѣ? какія мысли о Немъ Самомъ въ сердцѣ своемъ питаеши, согласныя ли слову Божію и вѣрѣ святой? \textit{Осмотрись!} Страшное и святое имя Божіе исповѣдуеши, и въ молитвѣ призываеши и поеши, но не оскверняеши ли тѣхъ устъ, которыми Бога исповѣдуеши и поеши, не оскверняеши ли, говорю, сквернословіемъ, клеветою, злорѣчіемъ, буесловіемъ, кощунствомъ, осужденіемъ, лестію, лжею и всякимъ гнилымъ и празднымъ словомъ? \textit{Осмотрись!} Воздѣваеши руки твоя къ Богу великому и святому, но не порочиши ли ихъ хищеніемъ, грабленіемъ и всякою неправдою? не имѣеши ли на ближняго твоего ненависти и злобы? не желаеши ли ему отмстить и повредить какъ нибудь? \textit{Осмотрись!} Приступаеши къ святымъ и животворящимъ Тайнамъ Хрістовымъ: о, коль велико дѣло сіе! коль велика благость Божія! коль велико и почтеніе человѣку! Человѣкъ, земля и пепелъ, Тѣла и Крове Божественныя касается! \textit{Благослови душе моя Господа!} Возлюбленный хрістіанине! ты къ великому сему и страшному таинству како приступаеши? съ какою вѣрою? съ каковымъ страхомъ? съ каковымъ сердцемъ? съ каковыми устами? съ каковыми руками? \textit{Осмотрись!} Огнь бо есть недостойныя попаляяй. \textit{Да искушаетъ человѣкъ себе, и тако отъ хлѣба да ястъ, и отъ чаши да піетъ. Ядый бо и піяй недостойнѣ, судъ себѣ ястъ и піетъ, не разсуждая Тѣла Господня}\footnote{1~Кор.~11,~28 и 29.}. Богъ есть высочайшій твой Благодѣтель, каковаго не было, нѣтъ и не можетъ быть больше. Онъ какъ создалъ тебе изъ ничего, такъ и все и всякое добро подаетъ тебѣ. Безъ Его добра и минуты жить не можемъ. Солнце, луна и звѣзды, которыи тебѣ сіяютъ, суть Его добро. Воздухъ, который сохраняетъ жизнь твою, Его добро есть. Земля, на которой ты живеши, которая плоды тебѣ и скотамъ твоимъ раждаетъ, Его добро есть. Скоты, которыи тебѣ служатъ, Его добро есть. Вода, которая тебе и скотовъ твоихъ напаяетъ, и рыбы живущія въ ней, Его добро есть. Домъ, въ которомъ упокоеваешися, и отъ бури и непогоды воздушной сохраняешися, Его добро есть. Пища и питіе, которыми укрѣпляется и прохлаждается немощное тѣло твое, Его добро суть. Огнь и дрова, которыми варится пища твоя и согрѣвается домъ твой, суть Его добро. Одѣяніе, которымъ покрывается и согрѣвается нагое тѣло твое, Его добро есть. Все сіе и прочее Онъ подаетъ тебѣ невидимою и всесильною и щедрою рукою Своею, и подаетъ отъ единой благости и любви къ тебѣ. Естество бо Его такое есть, которое не можетъ не благотворить. Видишь Его къ тебѣ любовь, но чувствуеши ли въ себѣ взаимную къ Нему любовь? Любовь ничимъ инымъ, какъ только любовію награждается. Получаеши благодѣяніе отъ Него, но помниши ли Благодѣтеля, и благодариши ли Благодѣтеля? \textit{Осмотрись!} Тяжка неблагодарность человѣку, отъ котораго малое какое добро получаемъ и не благодаримъ ему; несравненно тяжчайшая неблагодарность Богу, когда отъ Него все и всякое добро пріемлемъ, но неблагодарны Ему являемся. Безъ любви благодарность Ему быть не можетъ; и неблагодаренъ Ему тотъ, кто волѣ Его святой не угождаетъ. Высочайшая Его и непостижимая любовь къ намъ открылася въ томъ, что единородный Сынъ Его, благоволеніемъ Его, къ намъ пришелъ взыскати и спасти насъ погибшихъ. О сей любви Самъ Хрістосъ Господь нашъ глаголетъ: \textit{тако возлюби Богъ міръ, яко Сына Своего единороднаго далъ есть, да всякъ вѣруяй въ Онь не погибнетъ, но имать животъ вѣчный}, и проч.\footnote{Іоан.~3,~16.} Бѣдный грѣшникъ! слѣдовало"=было тебѣ за грѣхъ твой, которымъ ты праведнаго и \textit{безконечнаго} Бога обезчестилъ и прогнѣвалъ, аки мертвецу, поверженному въ горахъ, во адѣ вѣчною умирать смертію, и безъ конца горькую гнѣва Божія чашу пить, и съ сатаною и аггелами его за едино считаться, и въ негасимомъ огнѣ страдать. Сіи плоды суть грѣха: \textit{оброцы бо грѣха смерть}\footnote{Рим.~6,~23.}. Богъ человѣколюбивый, по Своей благости и любви къ намъ, не допустилъ тебе до того. Слава благости Его, слава человѣколюбію Его, слава щедротамъ и милосердію Его! Не допустилъ, говорю, тебе до того, "--- кого? отступника и врага Своего. Помиловалъ тебе, умилосердился надъ тобою, возлюбилъ тебе, послалъ единороднаго Сына Своего къ тебѣ спасти тебе. Пришелъ Онъ къ тебѣ, Создатель къ созданію Своему, Господь къ непотребному рабу Своему, Богъ святый и великій ко грѣшнику; Царь небесный явился на земли, и херувимамъ и серафимамъ неприступный, бѣднымъ и отверженнымъ человѣкомъ и грѣшникомъ приступнымъ содѣлался; вообразился въ подобострастное тебѣ тѣло, жилъ на земли, обращался между грѣшниками, яко свѣтъ между тьмою, \textit{и тьма Его не объятъ}; трудился, болѣзновалъ, плакалъ, страдалъ и умеръ за тебе. Видишь, грѣшникъ, что Богъ тебе ради сотворилъ. \textit{Благословенъ Господь Богъ Израилевъ, яко посѣти и сотвори избавленіе людемъ Своимъ, и воздвиже рогъ спасенія намъ, въ дому Давида отрока Своего}\footnote{Лук.~1,~68 и 69.}. Чего болѣе уже отъ Бога ожидать тебѣ, грѣшникъ, когда и Сына Своего единороднаго ради тебе не пощадѣлъ Онъ? Какая большая любовь, какое большее благодѣяніе паче сего можетъ быть! Помниши ли убо великое сіе дѣло Божіе, котораго болѣе не можетъ быть? Помниши ли дѣло сіе, которымъ отъ вѣчной смерти и ада избавленъ ты, въ которомъ вѣчный животъ твой и все вѣчное блаженство состоитъ? Помниши ли тое, и отъ чиста сердца благодариши ли Благодѣтелю и Искупителю Богу твоему? \textit{Осмотрись!} Почитаеши ли истинно, а не лицемѣрно, такъ почетшаго тя Господа? \textit{Осмотрись!} Соблюдаеши ли святыя заповѣди Его, безъ чего любовь къ Нему быть не можетъ? \textit{Осмотрись! Имѣяй заповѣди Моя и соблюдаяй ихъ, той есть любяй Мя}, и проч. \textit{Не любяй Мя, словесъ Моихъ не соблюдаетъ}, глаголетъ Господь\footnote{Іоан.~14,~21 и 24.}. \textit{Хрістосъ за всѣхъ умре, да живущіи не ктому себѣ живутъ, но умершему за нихъ и воскресшему}\footnote{2~Кор.~5,~15.}. Богъ, который какъ всякаго человѣка, такъ и тебе создалъ и любитъ, и отъ любви Своей какъ всѣмъ, такъ и тебѣ благая Своя подаетъ, и какъ о всѣхъ, такъ и о тебѣ такъ милостивый и человѣколюбивый чудный промыслъ въ дѣлѣ спасенія показалъ; Богъ, говорю, толикій любитель твой и благодѣтель, хощетъ и велитъ, чтобы ты любилъ ближняго твоего, то"=есть, всякаго человѣка, якоже себе любишь: удовлетворяеши ли убо святому хотѣнію толикаго Любителя твоего и Благодѣтеля? угождаеши ли святой волѣ Его? показуеши ли усердное послушаніе Ему? любиши ли любимаго Богомъ ближняго твоего? не дѣлаеши ли ему того, чего себѣ не хощешь? подаеши ли ему въ нуждахъ его руку помощи, чего себѣ хощешь? \textit{Осмотрись!} Время житія нашего краткое, и проходитъ, какъ вода мимотекущая, и потерянное не возвращается, якоже слово сказанное: ты како его провождаешь и въ чемъ? не въ праздности ли? не въ суетѣ ли? не въ гуляніяхъ ли и пиршествахъ? не въ снисканіи ли чести, славы, богатства? \textit{Осмотрись!} Дано оно тебѣ отъ Бога къ покаянію, а не къ плотоугодію: тако ли убо провождаеши его, какъ Богъ отъ тебе хощетъ? \textit{Осмотрись!} Все человѣкъ въ мірѣ семъ можетъ сыскать, но потеряннаго времени сыскать не можетъ. Смерть невидимою дорогою за человѣкомъ ходитъ, и восхищаетъ его тогда, когда не чаетъ, и тамо, гдѣ не чаетъ, и такимъ образомъ, какимъ не чаетъ. Извѣстна намъ кончина наша и неизвѣстна: извѣстна, что скончаемся; неизвѣстна, что не знаемъ, когда, гдѣ и како скончаемся. Чимъ болѣе живемъ, тѣмъ болѣе умаляется дней нашихъ, и приближаемся къ концу житія нашего. Хрістіанине! помниши ли ты часъ сей страшный, на который вси святіи взирали и плакали, "--- часъ, въ который слѣдуетъ всякому итить или въ блаженную или мучительную вѣчность, и готовишися ли къ тому истиннымъ покаяніемъ? \textit{Осмотрись!} Въ чемъ застанетъ тебе часъ тотъ, въ томъ и суду Божію предстанеши. Блаженъ, кто помнитъ часъ сей. \textit{Скажи мнѣ, Господи, кончину мою, и число дней моихъ, кое есть, да разумѣю, что лишаюся азъ. Се пяди положилъ еси дни моя, и составъ мой яко ничтоже предъ Тобою. Обаче всяческая суета, всякъ человѣкъ живый. Убо образомъ ходитъ человѣкъ, обаче всуе мятется; сокровиществуетъ, и не вѣсть, кому соберетъ я}\footnote{Пс.~38,~5--7.}. Всѣмъ намъ явитися подобаетъ предъ судищемъ Хрістовымъ; тому суду, суду Божію, а не человѣческому, предстанеши и ты. Тамо будетъ правое и строгое дѣлъ, словъ и помышленій нашихъ испытаніе; тамо будетъ все явное, что мы во всемъ житіи нашемъ ни сдѣлали, и предъ всѣмъ свѣтомъ открыется, и всякъ отъ совѣсти обличится. Тамо рядомъ станутъ цари и подданныи ихъ, господа и раби ихъ, родители и дѣти ихъ, пастыри и люди ихъ, благородныи и подлыи, богатыи и убогіи, и всякъ за себе слово воздастъ. Грѣхи, когда ихъ грѣшникъ здѣ покаяніемъ, сокрушеніемъ сердца и слезами загладитъ, уже и тамо не явятся; грѣшника же непокаявшагося вси, какіе онъ ни сдѣлалъ, грѣхи словомъ или дѣломъ или помышленіемъ, явны тамо будутъ, и всему міру въ явленіе пріидутъ. Тогда о немъ скажется: \textit{се человѣкъ, и дѣла его!} Тамо на двѣ части вси собравшіися люди раздѣлятся: одни станутъ по правую сторону праведнаго Судіи, и призоветъ ихъ въ вѣчное Свое царство Царь славы Господь; другіе станутъ по лѣвую сторону онаго Судіи, и отошлетъ ихъ въ вѣчный огнь, уготованный діаволу и аггеломъ его. \textit{И идутъ сіи въ муку вѣчную: праведницы же въ животъ вѣчный}\footnote{Мѳ.~25,~46.}. Хрістіанине! поминаеши ли ты страшный день оный, и имѣеши ли свѣтильникъ горящій, и елей въ сосудѣ, и готовишися ли къ срѣтенію грядущаго онаго Судіи? \textit{Осмотрись! Осмотрись} и ты, возлюбленный пастырь, како пасеши стадо Хрістово? Поручилъ тебѣ Хрістосъ Господь пасти овецъ Своихъ, овецъ не безсловесныхъ, но словесныхъ, души хрістіанскія, Кровію Его святою искупленныя: како убо ихъ пасеши? питаеши ли ихъ пищею слова Божія? предходиши ли имъ образомъ добрыхъ дѣлъ? показуеши ли имъ путь спасенія? храниши ли ихъ отъ волковъ душевныхъ, демоновъ? \textit{Осмотрись!} Начальникъ и властитель, како поступаеши съ людьми Божіими, порученными тебѣ? печешися ли о нихъ и промышляеши ли о общей пользѣ, чего должность твоя отъ тебе и сила присяги требуетъ? не ищеши ли своей скверной корысти? \textit{Осмотрись!} Судія, како судиши, како и кого оправдаеши, и кого обвиняеши? \textit{Осмотрись!} Господинъ, како съ рабами и крестьянами, людьми подобными тебѣ, поступаеши? \textit{Осмотрись!} Рабъ, како и какою совѣстію господину твоему работаеши? \textit{Осмотрись!} Отецъ, како дѣтей своихъ воспитываеши? чему ихъ учиши? не вливаеши ли яда соблазновъ въ юныя сердца ихъ, вмѣсто здраваго ученія и наставленія? \textit{Осмотрись!} Чадо, како отца рождшаго тя почитаеши? отдаеши ли должное родителю и питателю твоему? \textit{Осмотрись!} Мужъ, храниши ли вѣрность къ женѣ своей, и жена, храниши ли вѣрность къ мужу своему? \textit{Осмотритесь!} Купецъ, како купуеши, и како продаеши? \textit{Осмотрись! Осмотрись}, пожалуй, всякъ хрістіанинъ, что и како нынѣ дѣлаеши, говориши и мыслиши? Все, что ни дѣлаеши, говориши и мыслиши нынѣ, на ономъ всенародномъ позорищи явится предъ тобою. Не неради убо о себѣ, возлюбленне; Богъ хощетъ тебе спасти: да будетъ хотѣніе твое, и спасешися. Кто чего усердно хощетъ, тотъ того усердно и ищетъ. \textit{Осмотрись} и ты, душе моя, како живеши, како обращаешися, како дѣлаеши, говориши и мыслиши, что любиши и что ненавидиши, въ чемъ упокоеваешися, чимъ утѣшаешися и услаждаешися, чимъ огорчаешися и опечаляешися, и въ чемъ надежду твою полагаеши. \textit{Призри, услыши мя, Господи Боже мой! просвѣти очи мои, да не когда усну въ смерть, да не когда речетъ врагъ мой: укрѣпихся на него}\footnote{Пс.~12,~4--5.}. \textit{Блюдите, како опасно ходите, не якоже немудри, но якоже премудри, искупующе время, яко дніе лукави суть. Сего ради не бывайте несмысленни, но разумѣвайте, что есть воля Божія}\footnote{Еф.~5,~15--17.}.

\section{93. Щитъ.}

Видимъ въ мірѣ, что люди на брани щитомъ себе отъ стрѣлъ вражіихъ охраняютъ. Хрістіанине! наше житіе не ино что есть, какъ брань, и брань противу различныхъ враговъ; брань противу діавола и аггеловъ его; брань противу плоти со страстьми и похотьми; брань противу злаго и враждебнаго міра, брань противу лжебратіи, брань противу тайныхъ и явныхъ враговъ; брань есть непрестанная и лютая: какимъ убо щитомъ намъ отъ стрѣлъ враговъ нашихъ защищаться въ день лютъ? Святое Божіе слово указуетъ намъ на вѣру и имя Господне, и всѣхъ святыхъ примѣры показуютъ тое; и вси истинные хрістіане въ день скорби своея къ сему щиту прибѣгаютъ, и тѣмъ защищаются отъ лица вражія. \textit{Богъ намъ прибѣжище и сила, помощникъ въ скорбехъ, обрѣтшихъ ны зло. Сего ради не убоимся, внегда смущается земля, и прелагаются горы въ сердца морская}, и проч.\footnote{Пс.~45,~2--3.} \textit{На Тя уповаша, хвало Израилева, отцы наши; уповаша, и избавилъ еси я; къ Тебѣ воззваша, и спасошася; на Тя уповаша, и не постыдѣшася}\footnote{21,~5--6.}. \textit{Господи, прибѣжище былъ еси намъ въ родъ и родъ}\footnote{39,~2.}. Симъ щитомъ, аки крѣпкою стѣною, защищались праотцы наши, Авраамъ и Ісаакъ и Іаковъ, якоже писано о нихъ: \textit{не остави человѣка обидѣти ихъ, и обличи о нихъ цари}. Подъ симъ щитомъ сохраняемъ былъ Іосифъ, душею и тѣломъ прекрасный, и метаемыя на него стрѣлы искушеній и напастей ничего не успѣли. Симъ щитомъ защищался Израиль, изшедшій изъ Египта, защищался отъ врага и мучителя своего гонящаго Фараона, и отъ срѣтающихъ на пути и востающихъ враговъ; подъ симъ щитомъ сохранялся отъ многоразличныхъ супостатовъ, живучи въ землѣ обѣтованной. \textit{Яко аще не Господь бы былъ въ насъ, да речетъ убо Израиль: яко аще не Господь бы былъ въ насъ, внегда востати человѣкомъ на ны, убо живыхъ пожерли быша насъ; внегда прогнѣватися ярости ихъ на ны, убо вода потопила бы насъ}\footnote{123,~1--3.}. Тѣмъ противлялся и Давидъ святый, царь Израилевъ, всѣмъ врагамъ своимъ, якоже самъ о себѣ глаголетъ: \textit{вси языцы обыдоша мя, и именемъ Господнимъ противляхся имъ; обышедше обыдоша мя и именемъ Господнимъ противляхся имъ; обыдоша мя, яко пчелы сотъ, и разгорѣшася, яко огнь въ терніи, и именемъ Господнимъ противляхся имъ}\footnote{117,~10--12.}. Подъ симъ сохранялися пророки, апостоли, святители, мученики, преподобныи, пустынники и вси святіи, отъ вѣка угодившіи Богу; сохранилися во многоразличныхъ искушеніяхъ, бѣдахъ, напастяхъ, мученіяхъ и страданіяхъ. Подъ сей щитъ безопасный и непобѣдимый и нынѣ истинныя хрістіане прибѣгаютъ, и тѣмъ защищаяся отъ лица вражія, дерзаютъ и поютъ прекрасную пѣснь: \textit{Богъ намъ прибѣжище и сила, помощникъ въ скорбехъ, обрѣтшихъ ны зѣло}, и проч. \textit{Господь силъ съ нами, заступникъ нашъ Богъ Іаковль}\footnote{Пс.~45,~2 и 8.}. \textit{Аще Богъ по насъ: кто на ны? Иже убо Сына Своего не пощадѣ, но за насъ всѣхъ предалъ есть Его: како убо не и съ Нимъ вся намъ дарствуетъ}\footnote{Рим.~8,~32.}? Дерзаетъ и торжествуетъ надъ всѣмъ адомъ и силою вражіею Павелъ святый, боголюбивая душа. Возлюбленный хрістіанине! сатана и вси злые духи его неукротимую ярость на человѣка, а паче на истиннаго христіанина имѣютъ, и, яко звѣри въ нощи, обходятъ его невидимо и многоразлично возстаютъ на него, и тщатся погубить его, то чрезъ себя, то чрезъ ложныхъ хрістіанъ и прочіихъ злыхъ людей, служителей своихъ. Кто бы отъ навѣтовъ и стрѣлъ ихъ могъ сохраниться, аще бы не Господь щитомъ благости Своея и кровомъ благодатныхъ крилъ Своихъ защищалъ его? Воистину въ едину минуту погиблъ бы. Воистину, что Давидъ святый воспѣлъ о ветхомъ Израили и о защищеніи и покровеніи его Божіи: \textit{яко аще не Господь бы былъ въ насъ, да речетъ убо Израиль: яко аще не Господь бы былъ въ насъ, внегда востати человѣкомъ на ны, убо живыхъ пожерли быша насъ; внегда прогнѣватися ярости ихъ на ны, убо вода потопила бы насъ}, "--- тое приличествуетъ новому Израилю, хрістіанамъ. Воистину живыхъ пожерли бы насъ враги наши, аще не бы былъ Богъ въ насъ и не хранилъ насъ. \textit{Благословенъ Господь, Иже не даде насъ въ ловитву зубомъ ихъ}\footnote{Пс.~123,~6.}. Отсюду видимъ, коль суетна надежда на все, кромѣ Бога; коль гнилой подпоръ на всякаго человѣка и ложный щитъ на всякое созданіе. Сего ради, все оставивше, къ сему надежному и непреоборимому да прибѣгаемъ щиту. \textit{Благо есть надѣятися на Господа, нежели на человѣка. Благо есть уповати на Господа, нежели уповати на князи}\footnote{117,~8 и 9.}. Пусть иныи защищаютъ себе хитростію и хитросплетеннымъ словомъ, иныи златомъ, сребромъ и прочимъ своимъ сокровищемъ, иныи саномъ и высокими своими титулами, иныи ходатайствомъ князей и вельможъ, иныи инымъ, чимъ хотятъ: но все, чимъ ни защищаютъ себе, все ложный и гнилый щитъ; обманетъ ихъ надежда ихъ. Мы именемъ Господнимъ защитимъ себе, и не посрамимся. Смотри, хрістіанине, буди только Божій, и имя Господне защититъ тебе отъ всѣхъ навѣтъ вражіихъ. \textit{Возлюблю Тя, Господи, крѣпосте моя. Господь утвержденіе мое, и прибѣжище мое, и избавитель мой, Богъ мой, помощникъ мой, и уповаю на Него: защититель мой, и рогъ спасенія моего, и заступникъ мой. Хваля призову Господа, и отъ врагъ моихъ спасуся}\footnote{17,~2--4.}. \textit{Аще и пойду посредѣ сѣни смертныя, не убоюся зла, яко Ты со мною еси}\footnote{22,~4.}. \textit{Ты еси прибѣжище мое отъ скорби обдержащія мя. Радосте моя, избави мя отъ обышедшихъ мя}\footnote{31,~7.}. \textit{Помощникъ мой еси, Тебѣ пою: яко Богъ заступникъ мой еси, Боже мой, милость моя}\footnote{Пс.~58,~18.}. Когда христіанинъ въ благодати Божіей и въ помощи и защищеніи Вышняго находится: то, хотя вси злыи люди, весь свѣтъ, весь міръ, вси діаволы и адъ подвигнутся и востанутъ на него, и вси бѣды и напасти, въ мірѣ случающіися, окружатъ его, ничего не успѣютъ. Ибо Богъ сильнѣйшій всѣхъ, Который во всемогущей Своей десницѣ содержитъ его, защищаетъ и сохраняетъ его. Не бойся убо, благочестивая и боголюбивая душа. \textit{Яко Той избавитъ тя отъ сѣти ловчи, и отъ словесе мятежна; плещма Своима осѣнитъ тя, и подъ крилѣ Его надѣешися; оружіемъ обыдетъ тя истина Его. Не убоишися отъ страха нощнаго, отъ стрѣлы летящія во дни, отъ вещи во тмѣ преходящія, отъ сряща и бѣса полуденнаго. Падетъ отъ страны твоея тысяща, и тма одесную тебе, къ тебѣ же не приближится}\footnote{90,~3--7.}. О чемъ и весь псаломъ сей поучаетъ.

\section{94. Миръ.}

Видимъ въ мірѣ, что когда едино царство съ другимъ, городъ съ другимъ городомъ, село съ другимъ селомъ, сосѣдъ съ сосѣдомъ, жена съ мужемъ, и прочіи живущіи въ единомъ домѣ другъ съ другомъ миръ имѣютъ; тогда все благополучно бываетъ, и никакого зла другъ отъ друга не опасаются, только бы миръ истинный, а не лицемѣрный былъ, каковый часто между людьми бываетъ. Тако, когда хрістіанинъ миръ съ Богомъ имѣетъ, тогда все блаженство въ себѣ заключаетъ, и всякое бѣдствіе и злополучіе отгоняетъ. Тогда человѣкъ въ милости у всесильнаго Бога, и покрывающаго Своихъ рабовъ, находится; гнѣва и суда Божія не боится; ада и геенны не ужасается; діаволовъ и всѣхъ навѣтовъ вражіихъ не устрашается; вѣчныя жизни и тоя обѣщанныхъ благихъ и всякаго благословенія Божія безъ сумнѣнія надѣется. Единаго того опасается, чтобы предъ Богомъ не согрѣшить и Его не прогнѣвать, и тако пресладкаго мира не лишиться. Сей миръ не можетъ, какъ только въ чистой и непорочной совѣсти быть, и есть высочайшій даръ Святаго Духа. Сего мира Апостолъ святый въ началѣ почти всѣхъ своихъ посланій хрістіанамъ желаетъ: \textit{благодать вамъ и миръ}, и проч. Сего мира и служители Божіи на всѣхъ публичныхъ Богослуженіяхъ намъ желаютъ: \textit{миръ всѣмъ!} Сей миръ Хрістосъ Господь, воставъ отъ мертвыхъ, Апостоламъ благовѣствовалъ: \textit{миръ вамъ}\footnote{Іоан.~20,~20.}. Миръ сей, миръ съ Богомъ, потеряли"=было мы во Адамѣ, но во Хрістѣ Господѣ нашемъ паки пріобрѣли. Онъ учинился Ходатаемъ между Богомъ и человѣками, якоже церковь поетъ Ему: \textit{Ходатай Богу и человѣкомъ былъ еси, Хрісте Боже}, и проч.\footnote{Пѣснь 5"~я 2"~го гласа.} Онъ страданіемъ и смертію Своею за грѣхи наши удовлетворилъ, грѣхи отнялъ, и вѣчную правду ввелъ, и Бога умилостивилъ, и тако насъ Богу и Бога намъ примирилъ; ангеловъ съ человѣками, и человѣковъ съ ангелами, небесная съ земными, и земная съ небесными совокупилъ; смерть, адъ и все бѣсовское полчище, и всѣхъ враговъ нашихъ, яко сильный въ крѣпости Господь, Заступникъ и помощникъ нашъ, побѣдилъ и попралъ; и добрую надежду вѣчнаго живота подалъ намъ, такъ что надѣемся, по неложному Его обѣщанію, въ блаженной вѣчности торжествовать надъ всѣми врагами нашими. \textit{Гдѣ ти, смерте, жало? гдѣ ти, аде, побѣда}\footnote{1~Кор.~15,~55.}? О семъ смерти и ада и всѣхъ нашихъ враговъ Побѣдителѣ дерзаемъ! О семъ Заступникѣ и Помощникѣ нашемъ хвалимся! На сего Защитителя во всѣхъ нашихъ бѣдахъ, напастяхъ, искушеніяхъ и козняхъ вражіихъ, намъ наносимыхъ, уповаемъ! Онъ за насъ, яко Заступникъ и Помощникъ нашъ, стоитъ. Онъ всесиленъ. Аще убо Онъ за насъ: кто на насъ? \textit{Ты моя крѣпость, Господи, Ты моя и сила, Ты мой Богъ, Ты мое радованіе: не оставль нѣдра Отча, и нашу тщету посѣтивъ}, и проч. Богу же благодареніе, давшему намъ побѣду Господомъ нашимъ Іисусомъ Хрістомъ. Сего ради жертву хваленія и благодареніе Ему единому, яко Начальнику мира нашего и Совершителю, Отцу будущаго вѣка, Пастырю и Посѣтителю душъ нашихъ, Начальнику вѣры, новаго завѣта Ходатаю и Великому Архіерею, Начальнику и Совершителю спасенія нашего и всего вѣчнаго нашего блаженства, Іисусу Хрісту, Господу нашему со Отцемъ и Святымъ Его Духомъ приносимъ. \textit{Буди имя Господне благословенно отъ нынѣ и до вѣка. Оправдившеся убо вѣрою, миръ имамы къ Богу Господемъ нашимъ Іисусъ Хрістомъ, Имже и приведеніе обрѣтохомъ вѣрою во благодать сію, въ ней же стоимъ}\footnote{Римл.~5,~1 и 2.}. И \textit{пришедъ} (Хрістосъ) \textit{благовѣсти миръ вамъ дальнимъ и ближнимъ: зане Тѣмъ имамы приведеніе обои во единомъ Дусѣ ко Отцу}\footnote{Еф.~2,~17--18.}. \textit{Нѣсть царство Божіе брашно и питіе, но правда и миръ и радость о Дусѣ Святѣ. Иже бо сими служитъ Хрістови, благоугоденъ Богови, и искусенъ человѣкомъ}\footnote{Римл.~14,~17 и 18.}. \textit{Господи Боже нашъ, миръ даждь намъ, вся бо воздалъ еси намъ. Господи Боже нашъ, стяжи ны: Господи, развѣ Тебе инаго не вѣмы: имя Твое именуемъ}\footnote{Ис.~26,~12 и 13.}.

\section{95. Прямое и кривое древо.}

Что прямое и кривое дерево, то есть правое и неправое сердце. Какъ прямое и кривое дерево между собою имѣются, такъ правое и неправое сердце. Прямое съ кривымъ не ладится; тако между правымъ и неправымъ сердцемъ нѣтъ никакого сходства и согласія. Но посмотримъ, какое правое и неправое сердце, и увидимъ, какое между ими различіе есть, и можетъ ли между ими какое быть согласіе. Правое сердце есть богобоящееся, но неправое страха Божія не имѣетъ. Правое сердце Богу и волѣ Его святой послѣдуетъ, но неправое своей злой волѣ и своимъ прихотямъ. Правое сердце всякаго бережется грѣха, но неправое сердце о томъ нерадитъ. Правое сердце смиряется и смиренная мудрствуетъ, но неправое возносится и высокомудрствуетъ. Правое сердце небесная мудрствуетъ, и стремится ко онымъ благимъ, но неправое сердце земная мудрствуетъ и къ суетѣ міра сего прилѣпляется. Правое сердце всякаго человѣка любитъ, и любви своей и отъ ненавидящихъ его не отъемлетъ, но неправое сердце только любящихъ его любитъ, и часто и тѣхъ ненавидитъ. Правое сердце просто, нелестно и нелицемѣрно съ ближнимъ своимъ обходится; но неправое сердце коварно, лукаво и лицемѣрно поступаетъ со всякимъ человѣкомъ. Правое сердце, что на языкѣ имѣетъ, и словомъ объявляетъ, тое и внутрь себе имѣетъ, тое и мыслитъ; но неправое сердце иное говоритъ, иное мыслитъ; иное на языкѣ, иное внутрь себе имѣетъ. Правое сердце какъ не желаетъ, такъ и не касается чужаго добра; но неправое сердце и хощетъ того, и руки свои къ тому простираетъ. Правое сердце всѣмъ хощетъ и тщится благотворить; но неправое сердце небрежетъ о томъ. Правое сердце, чего не хощетъ себѣ, того и ближнему не творитъ; но неправое сердце ближнему тое творитъ, чего себѣ не хощетъ; ближняго осуждаетъ, оклеветаетъ, злословитъ, хулитъ и ругаетъ, хотя и само крайне того не хощетъ себѣ. Правое сердце обиду не отмщеваетъ, терпитъ и прощаетъ; но неправое сердце никакой обиды терпѣть не хощетъ, хотя само и всякаго обижаетъ. Правое сердце, хотя воткнется и падетъ, тотчасъ покаяніемъ востаетъ; но неправое сердце, единожды падши, всегда лежитъ. Правое сердце за все, что ни получаетъ отъ Бога, благодаритъ Ему; но неправое всякое Божіе благодѣяніе забываетъ. Правое сердце въ несчастіи и злостраданіи приключающемся терпитъ; но неправое негодуетъ, ропщетъ, а часто и хулитъ. Правое сердце истиною и духомъ и усерднымъ послушаніемъ Бога почитаетъ; но неправое едиными только устами и наружностію; о таковыхъ глаголетъ Господь: \textit{приближаются Мнѣ людіе сіи усты своими, и устнами чтутъ Мя: сердце же ихъ далече отстоитъ отъ Мене}\footnote{Мѳ.~15,~8; Ис.~29,~13.}. Правое сердце всякое добро, какое ни творитъ, Богу, яко Источнику всякаго добра, приписуетъ; но неправое сердце себѣ и своему тщанію. Правое сердце дѣла своя намѣреваетъ къ славѣ Божіей и пользѣ ближняго; но неправое сердце все творитъ для своей похвалы и тщеславія, и проч. Видишь изображеніе праваго и неправаго сердца. Самъ убо разсуждай, какое сходство и согласіе между ими можетъ быть. Какое бо согласіе свѣту со тьмою, сладкому съ горькимъ, бѣлому съ чернымъ, прямому съ кривымъ? Не ищи убо, не ищи мира и согласія между злою женою и добрымъ мужемъ, и доброю женою и злымъ мужемъ, между злою свекровію и доброю невѣсткою; и злою невѣсткою и доброю свекровію, между злымъ и добрымъ братомъ, между злою и доброю сестрою, между злымъ и добрымъ сосѣдомъ, и между всякимъ злымъ и добрымъ человѣкомъ. Нѣтъ, не было и не можетъ быть мира и согласія между ими. И сіе"=то есть, что глаголетъ Господь: \textit{мните ли, яко миръ пріидохъ дати на землю? ни, глаголю вамъ, но раздѣленіе. Будутъ бо отселѣ пять во единомъ дому раздѣлены, тріе на два, и два на три. Раздѣлится отецъ на сына, и сынъ на отца; мати на дщерь, и дщи на матерь; свекры на невѣстку свою, и невѣстка на свекровь свою}\footnote{Лук.~12,~51--53.}. Раздѣлятся неотмѣнно другъ на друга, когда одни Хрісту "--- свѣту, а другіе тьмѣ будутъ послѣдовать; одни небесная, а другіе земная будутъ мудрствовать. Прямое древо ко всякому дѣлу годится, все изъ него можно дѣлать; но кривое дерево ни къ чему не годится: тако правое сердце ко всякому чину и званію и общей пользѣ годится и потребно; но неправое сердце ни къ какому званію и чину не годится; оно гдѣ ни будетъ, во всемъ волѣ своей, а не Божіей будетъ послѣдовать; вездѣ скверной своей корысти, а не общей пользы, будетъ искать. Обратися убо, человѣче, къ Богу, и исправь себе, и послѣдуй волѣ Божіей, и будеши прямое дерево, и дѣла твоя будутъ права. \textit{Елицы правиломъ симъ жительствуютъ, миръ на нихъ и милость, и на Израили Божіи}\footnote{Гал.~6,~16.}. Но неправое сердце всего того лишается, \textit{и гнѣвъ Божій пребываетъ на немъ}\footnote{Іоан.~3,~36.}. \textit{Сердце чисто созижди во мнѣ, Боже, и духъ правъ обнови во утробѣ моей. Не отвержи мене отъ лица Твоего, и Духа Твоего Святаго не отъими отъ мене. Воздаждь ми радость спасенія Твоего, и духомъ владычнимъ утверди мя}\footnote{Пс.~50,~12--14.}. Молись, хрістіанине, тако со всякимъ усердіемъ, чтобы Самъ Богъ исправилъ и исправлялъ сердце твое Своею благодатію; безъ той бо сердце наше правое быть не можетъ. Когда благодать Божія въ сердце твое вселится, и поживетъ; то день отъ дне обновляться, перемѣняться и исправляться будеши.

\section{96. Вода съ высокихъ горъ на низкіи мѣста течетъ.}

Видимъ, что вода съ горъ на мѣста низкія стекаетъ: тако благодать Божія отъ небеснаго Отца на смиренныя сердца изливается. Но посмотримъ свойствъ смиреннаго сердца, и увидимъ истину сію. 1)~Смиренное сердце видитъ въ себѣ грѣхи свои, окаянство, бѣдствіе, подлость и ничтожество, и отъ того всего познаетъ свое недостоинство. Таковое сердце не смѣетъ очесъ своихъ на небо возвести и Богу много глаголати; но біетъ въ перси своя, и, воздыхая, падаетъ предъ Богомъ и молится: \textit{Боже, милостивъ буди мнѣ грѣшнику!} якоже мытарь сотворилъ\footnote{Лук.~18,~13.}. Судитъ бо себе недостойнаго, чтобы къ Богу взирати, и Ему приближитися. Таковое смиреніе показалъ Петръ, когда сказалъ Хрісту: \textit{изыди отъ мене, яко мужъ грѣшенъ есмь, Господи}\footnote{5,~8.}. 2)~Смиренное сердце всякаго сана, чести и славы убѣгаетъ; и ежели необходимо ему нужно быть въ какой чести и санѣ, съ крайнимъ нехотѣніемъ и ради послушанія пріемлетъ тое, яко видитъ свое невѣжество и недостоинство. 3)~Смиренное сердце высшимъ послушаніе показуетъ, равныхъ и низшихъ себе не презираетъ, но со всѣми ими обходится, яко съ братіею, хотя бы и честнѣйшій ихъ былъ, и большая бы паче ихъ имѣлъ дарованія. Ибо оно смотритъ не на дарованія, но на подлость свою; и познаетъ, что дарованія не его, но чужія, и есть едино влагалище ихъ, а не господинъ ихъ, а подлость и ничтожество его собственное есть, якоже всѣхъ человѣковъ. Всякъ бо человѣкъ самъ въ себѣ бѣденъ и окаяненъ есть. И потому съ подлыми обходится, яко единъ отъ подлыхъ.

4)~Ежели кого чимъ оскорбитъ отъ невѣдѣнія и неосторожности, какъ то всякому человѣку случается, не стыдится падать предъ нимъ и просить прощенія, хотя бы оскорбленный и низшій его или подвластный его былъ. О любезное позорище, когда высшее лице къ низшему склоняется и проситъ прощенія! 5)~Смиренное сердце никакого благодѣянія и всякаго злостраданія достойнымъ себе судитъ; потому, когда въ несчастіи находится, не ропщетъ, не негодуетъ, но великодушно терпитъ, судитъ бо себе достойнымъ того. 6)~Когда отъ кого словомъ или дѣломъ обиду пріемлетъ, не гнѣвается на обидящаго, много паче не отмщеваетъ ему, судитъ бо себе достойнымъ того. 7)~Когда видитъ или слышитъ кого согрѣшающаго, не осуждаетъ его; видитъ бо, что и самъ онъ такой же грѣшникъ. 8)~Какое добро и благодѣяніе отъ Бога ни получаетъ, видитъ, что туне, безъ всякой заслуги своей и по единой Его милости получаетъ. Ибо видитъ свое недостоинство, что всего того недостоинъ, что получаетъ. И потому сердечно за все Богу благодаритъ, что ни получаетъ. 9)~Какое добро дѣлаетъ самъ, все тое Богу восписуетъ. Видитъ бо свою немощь и ничтожество, что самъ собою безъ помощи Божіей, какъ древо сухое плода, никакого добра сотворить не можетъ. 10)~Въ собраніи, на обѣдѣ и на вечери \textit{на послѣднемъ мѣстѣ} сидѣть избираетъ\footnote{Лук.~15,~10.}. 11)~Лучшаго дома, лучшаго одѣянія, лучшія пищи и прочаго не ищетъ, но довольствуется тѣмъ, что имѣетъ; ибо и того, что имѣетъ, судитъ себе быть недостойнымъ. И како можетъ лучшаго желать и искать, когда и самаго того, что имѣетъ, за недостойнаго себе быть признаетъ? Видишь, хрістіанине, нѣкоторыя смиреннаго сердца изображенія. На таковое сердце струя благодати Божія течетъ. На таковое сердце милостивно Богъ призираетъ. Дай мнѣ, Господи, познать себе, и увидѣть свое окаянство, бѣдность и ничтожество, и буду имѣть смиренное сердце. \textit{Призри и услыши мя, Господи Боже мой! Просвѣти очи мои, да не когда усну въ смерть, да не когда речетъ врагъ мой: укрѣпихся на него}\footnote{Пс.~12,~4 и 5.}. Бѣденъ и окаяненъ человѣкъ, но тѣмъ бѣднѣйшій и окаяннѣйшій есть, что не видитъ своей бѣдности и окаянства. Думаетъ, что онъ бѣлъ, но онъ, какъ воронъ, черный. Думаетъ, что онъ видитъ и знаетъ все, но онъ слѣпъ и ничего не знаетъ. Думаетъ, что онъ богатъ, но онъ подлинно нищь и убогъ. Думаетъ, что онъ честенъ, но онъ безчестенъ, приложися скотомъ и уподобися имъ. Думаетъ, что онъ добръ, но есть подлинно золъ. Думаетъ, что онъ здоровъ, но онъ подлинно разслабленъ. Думаетъ, что онъ счастливъ, но онъ бѣднѣйшій и окаяннѣйшій паче всякаго созданія. Грѣхъ его таковымъ сдѣлалъ. \textit{Господи, что есть человѣкъ, яко познался еси ему, или сынъ человѣчь, яко вмѣняеши его}\footnote{143,~3.}? Но чимъ бѣднѣйшій есть человѣкъ, тѣмъ большая и удивительнѣйшая явилась благодать Божія на немъ. Самъ Богъ вообразился въ человѣка и ради человѣка. \textit{Хвали, душе моя, Господа}\footnote{Пс.~145,~1.}. Начало благополучія "--- познать свое неблагополучіе, якоже начало здравія "--- познать свою немощь. Хрістіанине! познай свою бѣдность и признай, и будеши благополученъ, яко будеши смиренъ. \textit{Богъ гордымъ противится: смиреннымъ же даетъ благодать. Всякъ возносяйся смирится: смиряяй же себе вознесется}\footnote{1~Петр.~5,~5; Лук.~18,~14.}.

\section{97. Вода мимотекущая.}

Что вода мимотекущая, тое житіе наше и все въ житіи случающееся. Видимъ, что вода въ рѣкѣ непрестанно течетъ и проходитъ, и все вверху воды пловущее, какъ"=то: лѣсъ, соръ и прочее проходитъ. Хрістіанине! тако житіе наше, и съ житіемъ все благополучіе и неблагополучіе, мимо идетъ. Не было мене прежде нѣсколькихъ лѣтъ, и се есмь въ мірѣ, какъ и прочіи твари. \textit{Руцѣ Твои сотвористѣ мя, и создастѣ мя}, Господи\footnote{Пс.~118,~73.}. Былъ я младенецъ, и миновало то. Былъ я отрокъ, и то прошло. Былъ я юноша, и то отошло отъ мене. Былъ я мужъ совершенный и крѣпкій, минуло и тое. Нынѣ сѣдѣютъ власы мои, и отъ старости изнемогаю; но и то проходитъ, и къ концу приближаюся, и пойду въ путь всея земли. Родился я на то, чтобы мнѣ умереть. Умираю ради того, чтобы мнѣ жить. \textit{Помяни мя, Господи, во царствіи Твоемъ!} Что мнѣ случилося, то и всякому человѣку. Былъ я здоровъ и боленъ, и паки здоровъ и паки боленъ, и прошло то. Былъ въ благополучіи и неблагополучіи: прошло время, со временемъ все миновало. Былъ я въ чести: прошло то время, и честь отъ мене отступила. Люди мене почитали и покланялись: минуло то время, и не вижу того. Былъ я веселъ, былъ и печаленъ, радовался я и плакалъ; и нынѣ тоежде мнѣ случается: проходятъ дни, проходятъ съ ними печаль и веселіе, радость и плачь. Хвалили мене и славили люди, хулили и поносили; и которыи хвалили, тѣ и проклинали; и который хулили, тѣ и хвалили: прошло время, прошло и все, миновалась хвала и хула, слава и безславіе. Тоежде слышу и нынѣ: то хвалятъ, то хулятъ, то прославляютъ, то безславятъ; знаю, что и то минуется: пройдетъ время, пройдетъ хула и хвала, слава и безславіе. Что мнѣ случилося и случается, то и всякому человѣку, живущему въ мірѣ семъ. Таковъ бо есть міръ, такое и постоянство его, такое и житіе наше въ мірѣ. Бѣдный человѣкъ отъ утробы матере своея даже до гроба: раждается съ плачемъ, живетъ въ мірѣ, какъ корабль въ морѣ плаваетъ, и то восходитъ, то нисходитъ, то подымается, то ниспускается, и умираетъ съ плачемъ. Жилъ я въ богатомъ дому, жилъ и въ хижинѣ; прошло время, и таковый покой отступилъ отъ мене. Живу нынѣ въ хижинѣ и тѣснотѣ, и того не будетъ. Сидѣлъ я за богатою трапезою, сидѣлъ и за скудною: прошли дни тые, прошло и все, что въ нихъ было. Тоежде и нынѣ случается, и тое пройдетъ. Вкушалъ я сладости и горести: отступилъ вкусъ, отступила сладость и горесть отъ мене. Слышалъ сладкую музыку: прошло время слышанія, и престало мое веселіе. Одѣвался я доброю одеждою: прошло то время, отошло тое и одѣяніе отъ мене. Ѣздилъ я на колесницѣ: минулось тое время, и нѣтъ того. Предстояли мнѣ слуги и раби: минуло тое время, и отступили они отъ мене. Имѣлъ я у себе друговъ и пріятелей: прошли тыи дни, и обратились они то во враговъ моихъ, то во лжебратію. Что мнѣ, то и всякому случается. Гдѣ тое время, въ которое счастливъ я былъ, въ которое здоровъ, веселъ, радостенъ, славимъ, хвалимъ, почитаемъ; въ которое богатою трапезою и музыкою утѣшался, ѣздилъ колесницею и цугомъ? Прошло время, прошло и все съ нимъ счастіе мое и утѣшеніе мое. Гдѣ тое время, въ которое я былъ несчастливъ, былъ боленъ, печаленъ, скорбенъ, хулимъ и поносимъ, укоряемъ и ругаемъ, и проч? Прошли тыи дни, прошло и тое все несчастіе мое. Пройдетъ и все, что нынѣ въ семъ времени случается, яко все съ преходящимъ временемъ проходитъ. Что единъ человѣкъ на себѣ дознаетъ, тое и всякому случается. Нѣтъ бо человѣка такого, который бы отъ рожденія до смерти въ непремѣнномъ пребылъ благополучіи или неблагополучіи. Какъ подъ небомъ то ведро, то пасмурно, то непогода и буря, то ясно и тишина бываетъ: тако и всякому человѣку случается то въ благополучіи быть, то въ неблагополучіи, то въ страхѣ, то въ покоѣ, то въ печали, то въ радости быть. Но какъ дни и часы преходятъ, тако всякое счастіе и несчастіе съ ними проходитъ. Видишь, хрістіанине, что какъ вода мимо текущая и все по ней пловущее, тако время житія нашего и все со временемъ благополучіе и неблагополучіе наше проходитъ. А тако и все житіе наше пройдетъ. Какъ нынѣ у насъ прешедшіе дни наши, и все въ нихъ случившееся, суть счастіе и несчастіе наше: тако и все прочее житіе наше, и все, имѣющее въ немъ случитися, будетъ. И какъ прешедшіе дни наши, аки во снѣ, видимъ, и съ ними все благополучіе и неблагополучіе: тако и прочее время при концѣ житія, какъ во снѣ, будемъ видѣть; и только помнить будемъ тогда и мечтать, что тое и тое съ нами было. Такое"=то житіе наше въ мірѣ семъ! Не тако житіе будущаго вѣка будетъ, якоже слово Божіе и вѣра наша насъ увѣряетъ. Тамо житіе наше единожды начнется, но никогда не скончается, будетъ непрестанное и непремѣнное. Тѣло наше не будетъ имѣть немощи, дряхлости, старости, смерти и тлѣнія; но будетъ тѣло духовное, нетлѣнное, безсмертное, здравое, сильное, легкое и благоцвѣтущее. \textit{Сѣется въ тлѣніе, востаетъ въ нетлѣніи; сѣется не въ честь, востаетъ въ славѣ; сѣется въ немощи, востаетъ въ силѣ; сѣется тѣло душевное, востаетъ тѣло духовное. Подобаетъ бо тлѣнному сему облещися въ нетлѣніе, и мертвенному сему облещися въ безсмертіе. Егда же тлѣнное сіе облечется въ нетлѣніе, и смертное сіе облечется въ безсмертіе, тогда будетъ слово написанное: пожерта бысть смерть побѣдою. Гдѣ ти, смерте, жало? гдѣ ти, аде, побѣда}\footnote{1~Кор.~13,~42--44; 53 и 55.}? Такожде слава, честь, покой, миръ, утѣшеніе, радость, веселіе и все блаженство тамо непрестанно будетъ; непрестанно будутъ видѣть Бога лицемъ къ лицу избранные Божіи, и отъ того непрестанно утѣшаться, радоваться, веселиться; непрестанно будутъ со Хрістомъ, яко уды съ главою, царствовать. \textit{Понеже съ Нимъ страждемъ, да и съ Нимъ прославимся}\footnote{Римл.~8,~17.}. Такожде и въ неблагополучной вѣчности непрестанная будетъ смерть. Единожды умерше, осужденныи непрестанно будутъ умирать и желать смерти; непрестанно будутъ мучиться и страдать, и никакого не будутъ чувствовать утѣшенія, прохлажденія, облегченія. Се есть смерть вторая и смерть вѣчная. Отсюду послѣдуетъ: 1)~Понеже временное наше житіе непостоянное, и все въ немъ перемѣняется и проходить, то не должно намъ къ временнымъ и мірскимъ вещамъ прилѣпляться, но со всякимъ усердіемъ вѣчнаго живота и оныхъ благихъ искать, \textit{горняя мудрствовать, а не земная}\footnote{Кол.~3,~2.}. 2)~Благополучіемъ міра сего, то есть, богатствомъ, славою, честію, похвалою и проч. не возноситься. Ибо всякое благополучіе временное какъ приходитъ, такъ и отходитъ отъ насъ, и бываемъ какъ единъ отъ несчастливыхъ, нищихъ, бѣдныхъ и презрѣнныхъ. 3)~Въ неблагополучіи не унывать, ибо и тое проходитъ. И какъ, по прошедшей нощи, бываетъ день, и послѣ бури и непогоды возсіяваетъ ведро: тако по скорби и печали радость, и по неблагополучіи благополучіе приходитъ. 4)~Клевету, хуленіе, укореніе и поношеніе своевольныхъ и беззаконныхъ людей великодушно терпѣть и не огорчаться тѣмъ. Ибо и тое все проходитъ. Силенъ же Богъ и клятву ихъ во благословеніе намъ обратить, по реченному: \textit{прокленутъ тіи, и Ты благословиши}\footnote{Пс.~108,~28.}.

5)~Кто въ терпѣніи, въ благополучіи и неблагополучіи равно постояненъ будетъ, тотъ въ тишинѣ, покоѣ и мирѣ всегда будетъ пребывать; и благопріятное, сладкое, Хрістову житію подобное и сообразное, будетъ имѣть житіе; и на земли небесную будетъ чувствовать радость, и во временной жизни вѣчнаго живота сладости вкушать. \textit{Господи, помози мнѣ! И да возвеселятся вси уповающіи на Тя, во вѣкъ возрадуются, и вселишися въ нихъ, и похвалятся о Тебѣ любящіи имя Твое}\footnote{5,~12.}.

\section{98. Нынѣ ты всѣхъ бѣдъ свобождаешися.}

Бываетъ, что люди въ различныхъ случаяхъ людемъ говорятъ: \textit{нынѣ ты всѣхъ бѣдъ свобождаешися}. Какъ"=то, говорится оно тому, который въ полону былъ, но отъ того исходитъ. Хотя тое слово и говорится людемъ, въ мірѣ семъ живущимъ, однакожъ не совсѣмъ оно приличествуетъ имъ. Невозможно бо человѣку въ мірѣ семъ живущему, отъ всѣхъ бѣдъ свободитися. Бываетъ, что единой свободился бѣды: такъ другая срѣтаетъ его. Какъ морскіе плаватели всегда непогоды и бури ожидаютъ: тако и живущему въ мірѣ семъ всегда бѣдъ и напастей ожидать надобно. Одна бѣда прошла, тутъ нечаянно и другая наступаетъ; міръ бо безъ того не бываетъ. Почему слово сіе: \textit{нынѣ ты отъ всѣхъ бѣдъ свобождаешися}, приличествуетъ единому человѣку вѣрному и благочестивому, при кончинѣ своей находящемуся. Онъ подлинно тогда всѣхъ бѣдъ свобождается. Возлюбленный хрістіанине, который оканчиваешь уже многобѣдное житіе сіе, \textit{ты нынѣ всѣхъ бѣдъ свобождаешися}. Родился ты, чтобы тебѣ умереть: умираеши нынѣ, чтобы жить. Родился ты на бѣды и скорби: кончаеши нынѣ житіе твое; кончаются бѣды и скорби твои. Былъ ты въ многоразличныхъ бѣдахъ и напастехъ, уже отселѣ не увидишь ихъ. Плакалъ ты и болѣзновалъ, уже отселѣ тебѣ не будетъ того. Терпѣлъ ты болѣзнь душевную и тѣлесную, уже отселѣ не будешь чувствовать того. Терпѣлъ ты скуку и скорбь, уже отселѣ не будетъ тебѣ она скучать. Терпѣлъ ты всякій недостатокъ и тѣсноту, уже она отступаетъ отъ тебе. Боролся ты не безъ скорби и труда противу плоти со страстьми и похотьми, "--- уже она отселѣ тебѣ не будетъ скучать. Терпѣлъ ты нападеніе діавола и злыхъ его духовъ, и подвизался противу ихъ, "--- уже отселѣ оставятъ они тебе. Терпѣлъ ты обиду и озлобленіе отъ злыхъ людей, "--- уже отселѣ не будутъ обиждать и озлоблять тебе они. Трудился ты и не имѣлъ истиннаго покоя, "--- уже отселѣ въ истинномъ покоѣ будеши. Опасался ты всего смертнаго и вредительнаго, "--- уже отселѣ безъ страха и въ безопасности будеши. Требовалъ ты пищи, питія и одежды ради немощнаго тѣла твоего, "--- уже отселѣ не потребуеши. Видѣлъ ты и нехотя соблазны міра сего, "--- уже отселѣ не увидиши. Слышалъ ты и нехотя богопротивныя и пагубныя слова своевольныхъ людей, "--- уже отселѣ не услышиши. Терпѣлъ ты укореніе и хуленіе враговъ своихъ, "--- уже отселѣ не будетъ тебѣ того. Разлучаешися ты отъ отца и матери; но къ Богу Отцу вѣчному идеши. Оставляеши и друговъ и братію твою; но ангелы и лики святыхъ будутъ друзья твои. Не увидишь солнца сего; но Хрістосъ Господь солнце праведное возсіяетъ тебѣ, Который и умеръ за тебе, и смертію Своею оживилъ тебе. Оставляешь міръ сей, но съ міромъ и вся бѣды и напасти оставляешь, и въ царствіе Божіе внидеши. Кончаешь временную жизнь, но вѣчную и блаженную начинаеши. Вкушаеши временно горести смертной, но вѣчныя жизни сладости вкушать будеши. Нощи и дня не увидиши, но всегда и непрестанно день будетъ тебѣ сіять. Тѣло твое, яко персть, персти предается; но гласомъ Сына Божія возбудится и востанетъ, и тлѣнное облечется въ нетлѣніе, и смертное облечется въ безсмертіе, и душевное облечется въ духовное, и немощное востанетъ въ силѣ, и безславное востанетъ въ славѣ. \textit{Богу благодареніе, давшему намъ побѣду Господемъ нашимъ Іисусъ Хрістомъ}\footnote{1~Кор.~15,~57.}. \textit{Блаженъ убо путь, въ оньже пойдеши, душе: яко уготовася тебѣ мѣсто упокоенія}. Туды пошли и пришли и упокоеваются тамо вси святіи, патріархи, пророки, апостоли, святители, мученики, преподобніи и вси, отъ вѣка вѣрою угодившіи Богу. Туды идеши и ты, туды, гдѣ вси святіи упокоеваются. Узриши, и возрадуется сердце твое, и \textit{радости} твоея \textit{никтоже возметъ} отъ тебе. Узриши славу Божію, и возрадуется сердце твое. Узриши Хріста Сына Божія, Который за тебе пострадалъ и умеръ; узриши во славѣ, и возрадуется сердце твое. Узриши лики ангелъ и архангелъ, и избранныхъ Божіихъ, и возрадуется сердце твое. Узриши прекрасный горній Іерусалимъ, и неизреченную доброту его, и возрадуется сердце твое. Узриши домъ небеснаго Отца и Того многія обители, и возрадуется сердце твое. Узриши тамо все иное, несравненно лучшее и краснѣйшее, нежели здѣ видиши, и возрадуется сердце твое. Узриши благая, \textit{ихже око не видѣ, и ухо не слыша, и на сердце человѣку не взыдоша, яже уготова Богъ любящимъ Его}; узриши, и возрадуется сердце твое. Узриши дивная, преславная и ужасная, ихже нѣсть числа, и возрадуется сердце твое. "--- \textit{Я"=де недостоинъ небеснаго царствія?} Отвѣтъ: подлинно, никто отъ человѣкъ не достоинъ того. Но Христосъ Господь нашъ достоинъ, Который безцѣнною крове Своея цѣною заслужилъ намъ тое. Отъ Его достоинства и наше достоинство зависитъ. Онъ своимъ достоинствомъ и насъ недостойныхъ достойными сотворилъ, и Его благодатію благій и человѣколюбивый Отецъ небесный, по безприкладной Своей милости, удостояетъ. Онъ, какъ всѣмъ вѣрнымъ, такъ и тебѣ, есть Отецъ, и отверзаетъ двери царствія Своего, и подаетъ наслѣдіе вѣчнаго живота. "--- \textit{Я"=де грѣшникъ?} Отвѣтъ: \textit{Хрістосъ Іисусъ пріиде въ міръ грѣшники спасти}\footnote{1~Тим.~1,~15.}. Той есть оправданіе наше, \textit{Иже бысть намъ премудрость отъ Бога, правда же и освященіе и избавленіе}\footnote{1~Кор.~1,~30.}. "--- \textit{Я"=де многіи грѣхи содѣлалъ?} Отвѣтъ: \textit{идѣже умножися грѣхъ, преизбыточества благодать}\footnote{Рим.~5,~20.}. Многіи грѣхи по множеству щедротъ Божіихъ очищаются. "--- \textit{Я"=де боюсь суда?} Отвѣтъ глаголетъ Христосъ: \textit{аминь, аминь глаголю вамъ: яко слушаяй словесе Моего, и вѣруяй пославшему Мя, имать животъ вѣчный, и на судъ не пріидетъ, но прейдетъ отъ смерти въ животъ}\footnote{Іоан.~5,~21.}. "--- \textit{Боюсь"=де погибнутъ?} Отвѣтъ глаголетъ Псаломникъ: \textit{Богъ спасеній нашихъ, Богъ нашъ, Богъ еже спасати}\footnote{Пс.~67,~20 и 21.}. "--- \textit{Боюсь"=де ада и вѣчнаго мученія?} Отвѣтъ: Христосъ Господь смертію и кровію Своею искупилъ насъ отъ того. И церковь поетъ: \textit{изъ чрева адова избави насъ}. "--- \textit{Какъ"=де мнѣ со святыми веселиться, которые въ толикихъ добродѣтеляхъ просіяли?} Отвѣтъ: и святіи благодатію Хрістовою спаслися. Ты молись съ разбойникомъ быть: \textit{помяни мя, Господи, во царствіи Твоемъ}\footnote{Лук.~23,~42.}. Каковый гласъ и вся церковь приноситъ Ему. \textit{Мужайся убо, и да крѣпится сердце твое}\footnote{Пс.~30,~25.}. Господь помощникъ твой, Господь заступникъ твой, Господь защититель твой, Господь искупитель твой, Господь избавитель твой, Господь Спаситель твой, Той, который возлюбилъ тебе и предалъ Себе за тебе, и пострадалъ и умеръ за тебе, и всѣхъ нашихъ враговъ побѣдилъ, "--- Той всесильною рукою Своею сохранитъ тебе. Ангели Божіи сопутники твои. \textit{Господь просвѣщеніе мое и Спаситель мой: кого убоюся? Господь защититель живота моего: отъ кого устрашуся}\footnote{26,~1.}? \textit{Обратися душе моя въ покой твой, яко Господь благодѣйствова тя: яко изъятъ душу мою отъ смерти, очи мои отъ слезъ, и нозѣ мои отъ поползновенія. Благоугожду предъ Господемъ во странѣ живыхъ}\footnote{Пс.~114,~6--8.}. \textit{Вѣмы, яко, аще земная наша храмина тѣла разорится, созданіе отъ Бога имамы, храмину нерукотворенну, вѣчну на небесѣхъ}\footnote{2~Кор.~5,~1.}.

\section{99. Малое деревцо.}

Видимъ, что малое деревцо ко всякой сторонѣ удобно преклоняется, куды ни преклоняется, и куды преклонится, тако и ростетъ. Тако имѣется юное и малое отроча: чего научается, того и навыкаетъ, и чего навыкаетъ, то и въ прочее время творить будетъ. Научится ли добра въ юности своей, "--- добръ и чрезъ все житіе будетъ. Научится ли зла, "--- и золъ во всемъ житіи будетъ, и изъ малаго отрока можетъ быть и ангелъ, можетъ быть и діаволъ. Какое воспитаніе и наставленіе будетъ имѣть, таковъ и будетъ; отъ воспитанія бо, какъ отъ сѣмене плоды, все прочее житія время зависитъ. Сего ради увѣщаваетъ Божіе слово родителей: \textit{да воспитываютъ чадъ своихъ въ наказаніи и ученіи Господни}\footnote{Еф.~6,~4.}. И въ Притчахъ Соломоновыхъ написано: \textit{иже щадитъ жезлъ свой, ненавидитъ сына своего; любяй же наказуетъ прилѣжно}\footnote{Притч. Сол.~16,~25.}. Многіи учатъ дѣтей своихъ свѣтской политики; иныи научаютъ иностранныхъ языковъ, по"=французски, по"=нѣмецки, по"=италіански говорить, и на то не мало суммы иждиваютъ; иныи тщатся обучить купечества, и другихъ художествъ; но по"=хрістіански жить дѣтей едва кто научаетъ. А безъ сего всякая наука ничтоже есть, и всякая мудрость буйство есть. Что бо пользуетъ христіанину по"=италіански, по"=французски, по"=нѣмецки говорить, но безбожно жить? Что пользуетъ въ купечествѣ и художествахъ искуснымъ быть, но страха Божія не имѣть? Таковыи художники и мудрецы злѣйшими бываютъ паче неискусныхъ простяковъ. Юное сердце, яко ко злу склонное, на всякое зло стремится, когда страхомъ Божіимъ и уздою наказанія не воздерживается; а паче таковыя, которыи науками и художествами умъ свой острятъ, но воли не исправляютъ, на вся злая искусны бываютъ, какъ и самые примѣры тое показуютъ. Сего ради, христіанине, первое тщаніе да будетъ твое о томъ, чтобы дѣтей своихъ научить по"=хрістіански жить. Все бо наученіе и наставленіе твое безъ того ничтоже есть. Богъ не взыщетъ отъ тебе, училъ ли ты дѣтей своихъ свѣтской политики и иностранныхъ языковъ; но взыщетъ, училъ ли ты по"=хрістіански жить, наставлялъ ли ихъ благочестію. Горе юнымъ дѣтямъ, у которыхъ злыи отцы! Чего они отъ нихъ могутъ научиться, кромѣ зла? Како бо злый можетъ научить добра? Хотя бываетъ, что злый отецъ наказуетъ дѣтей своихъ за своевольство, но соблазнами своими научаетъ того. Юныи бо люди болѣе научаются отъ дѣлъ, нежели отъ словъ и наказанія; оттуду бываетъ, что отцы злыи, дѣти злѣйшіи, а внуки и того злѣйшіи бываютъ. И тако бѣдный человѣкъ, научившися зла и привыкши, нечестіемъ, какъ хлѣбомъ, насыщается, и отъ зла во зло и отъ беззаконія въ беззаконіе падаетъ, и стремится въ погибель, якоже камень, съ верху горы пущенный внизъ. И хотя бываетъ, что нѣкоторыи отъ таковыхъ узнаютъ свою бѣду и гибель, и содрогаются и ужасаются, и начинаютъ каятися; но привычкою, яко веревкою влекомы, на нечестіе обращаются. Сіе зло отъ злаго воспитанія происходитъ. Горе убо юнымъ дѣтямъ; но сугубое горе отцамъ, которыи не токмо не научаютъ дѣтей добра, но соблазнами своими подаютъ поводъ ко всякому злу! Таковыи отцы не тѣлеса, но души хрістіанскія убиваютъ, за которыхъ умеръ Хрістосъ, и лишаютъ ихъ не временнаго, но вѣчнаго живота. Узнаютъ они страшную свою сію бѣду тамо, гдѣ все явлено будетъ, и всякому предъ глаза представятся дѣла его. Сего ради отцы, въ которыхъ искра благочестія есть, всячески стараться должны, чтобы дѣтей своихъ научить жить по"=хрістіански, и въ юныя сердца ихъ вливать млеко благочестія, \textit{да о немъ возрастутъ во спасеніе}\footnote{1~Петр.~2,~2.}. Полезно учить наукамъ и художествамъ, но учить жить по"=хрістіански нужно; ибо всякая наука и художество безъ хрістіанскаго житія ничтоже есть. Внимайте сему, родители, чтобы тѣмъ дѣтямъ, которыхъ вы на свѣтъ родили, не быть убійцами. Истинный отецъ не тотъ, который родилъ, но тотъ, который добрѣ воспиталъ и научилъ. Родившій подалъ только жить, а добрѣ воспитавшій и научившій далъ добрѣ жить. Живутъ на свѣтѣ и язычники, во тьмѣ и идолопоклонствѣ находящіися, паче же и самыи скоты; но добрѣ живутъ единыи хрістіане. Сего ради отцамъ, родившимъ насъ, должники мы находимся; но отцамъ, добрѣ воспитавшимъ и наставившимъ насъ ко благочестію, далеко большіи должники. Ибо отцы, родившіи насъ, родили къ временной жизни; а отцы, добрѣ воспитавшіи и научившіи насъ къ благочестію, раждаютъ къ вѣчной жизни. \textit{Блаженъ, иже сотворитъ добрѣ, и научитъ добрѣ}\footnote{Мѳ.~5,~19.}. Блаженъ родитель, который и къ временной жизни родилъ, и къ вѣчной жизни отродилъ дѣтей своихъ. Окаяненъ родитель, который къ временной жизни родилъ дѣтей, но къ вѣчной жизни двери имъ заключилъ, "--- заключилъ то небреженіемъ добраго воспитанія, то соблазнами своими. Лучше человѣку не родиться, нежели родиться, и въ вѣчной погибели быть.

\subsection{О томжде.}

Малое деревцо, когда исторгается изъ земли, удобно исторгается: тако всякая злая страсть, пока еще не вкоренилась въ сердце и не застарѣла, безъ труда отъ человѣка изъимается. Великое дерево неудобно и съ великою трудностію исторгается изъ земли; понеже глубоко корень свой въ земли утвердило: тако застарѣлая страсть, въ сердцѣ человѣческомъ глубоко вкоренившаяся, съ великою неудобностію и трудностію исторгается. Много труда и подвига требуется привыкшимъ къ піянству, чтобы отъ піянства свободиться; привыкшимъ къ любодѣянію, чтобы любодѣянія свободиться; привыкшимъ къ сребролюбію, чтобы сребролюбія свободиться; привыкшимъ къ воровству и хищенію, чтобы хищенія свободиться; привыкшимъ къ клеветѣ, осужденію и злорѣчію, чтобы языкъ свой отъ того удерживать; привыкшимъ къ божбѣ, чтобы не божиться; привыкшимъ ко лжи, чтобы не лгать; привыкшимъ къ пышности, гордости и суетѣ міра сего, чтобы отъ того свободиться, и проч. Многіи стараются, чтобы отъ страсти, въ которой находятся, свободиться, и окаеваютъ себе и плачутъ, видя свою бѣду; и временемъ отъ ней отстаютъ, но паки обращаются. Страсть бо, какъ магнитъ, къ себѣ привлекаетъ ихъ. Сего ради, хрістіанине, вооружись противу плоти твоея сначала, пока еще молодъ еси; вооружись противу плоти, которая со страстьми и похотьми воюетъ на душу твою, и хощетъ ее умертвить. Исторгай страсти и похоти изъ сердца твоего, пока еще молоды и не возрасли, да не возрастши умертвятъ душу твою. Сіи суть первѣйшіи твои враги и домашніи, и всегда съ тобою суть, гдѣ ты ни обращаешися; ибо внутрь тебе суть. Они окружаютъ душу твою, и непрестанно угнетаютъ ее и отнимаютъ благородіе ея, и лишаютъ пресладкія свободы ея. Отвращаютъ отъ Бога и святой любви Его; не допущаютъ волю Его творити, Ему работати и угождати, чистую и богоугодную молитву творити, небесная мудрствовати, слово Божіе слышати и того плодъ творити, небесными, святыми и добрыми мыслями питатися и утѣшатися, къ вѣчной и блаженной жизни стремитися и тещи, яко истыи врази ея возбраняютъ. Сими врагами душа окружается, и во тьмѣ сей, какъ въ темницѣ, заключена сидитъ и лишается пресладкія свободы и благородія своего. \textit{Изведи изъ темницы душу мою, Господи, исповѣдатися имени Твоему: мене ждутъ праведницы, дондеже воздаси мнѣ}\footnote{Пс.~141,~8.}. Помяни Господи, что бысть намъ; призри и виждь укоризну нашу. "--- Умерщвляй убо сихъ враждебныхъ и беззаконныхъ младенцевъ, которыхъ страстная плоть твоя раждаетъ; убивай враговъ, пока малы суть, да не, возрастше и укрѣпившеся, убіютъ тебе. Пусть они умираютъ, да живетъ душа твоя. Буди борецъ и запятникъ, а не лежень; буди храбрый воинъ противу сихъ враговъ, а не подданникъ ихъ, повелитель ихъ, а не слушатель, господинъ, а не рабъ ихъ. Стой противу ихъ, а не уступай имъ, да не одолѣютъ тебе и плѣнятъ тебе, и яко плѣнникомъ обладати будутъ. Горе душѣ, которою сіи супостаты обладаютъ! Повинуйся убо Господеви, а не тѣмъ, да рабъ Господень будеши, а не тѣхъ. Не возможно бо и Богу работати и страстямъ, и Богу угождати и плоти страстной, и волю Божію творити, и волю плоти своея въ похотехъ ея, ибо одно другому противно. Чего Богъ хощетъ, того страстная плоть не хощетъ; и чего Богъ не хощетъ, того страстная плоть хощетъ. Отрекись убо страстной плоти, \textit{и распинай ее со страстьми и похотьми, да будеши рабъ Божій и человѣкъ Божій. Иже бо Хрістовы суть, плоть распяша со страстьми и похотьми}\footnote{Гал.~5,~24.}. \textit{Жива будетъ душа моя, и восхвалитъ Тя: и судьбы Твоя помогутъ мнѣ. Заблудихъ яко овча погибшее: взыщи раба Твоего, яко заповѣдей Твоихъ не забыхъ}\footnote{Пс.~118,~175 и 176.}. \textit{Боже, въ помощь мою вонми, Господи, помощи ми потщися. Да постыдятся и посрамятся ищущіи душу мою}, и проч.\footnote{69,~2 и 3.}

\section{100. Покой.}

Есть покой тѣлесный, какъ видишь; есть покой и душевный. Тѣло упокоевается, когда по пути, или по трудахъ почиваетъ: тако душа упокоевается, когда въ чистой и непорочной совѣсти почиваетъ, и отъ ней ни въ чемъ не смущается. Се есть пресладкій душевный покой. Сладко упокоевается тѣло, когда по трудахъ почиваетъ: въ пресладкомъ покоѣ находится и душа, когда чистую и непорочную имѣетъ совѣсть. Имѣетъ свое безпокойствіе тѣло, имѣетъ и душа. Тѣло лишается покоя, когда находится въ трудахъ: душа лишается покоя, когда ее злая безпокойствуетъ совѣсть. Душевный покой отнимается: 1)~всякимъ \textit{грѣхомъ} и беззаконіемъ. У блудника и прелюбодѣя и у всякаго нечистоты любителя не можетъ быть покой душевный; у злобнаго, мстителя и убійцы не можетъ быть покой душевный; у вора, хищника, грабителя и мздоимца "--- судіи не можетъ быть покой душевный; у клеветника, ругателя и злорѣчиваго не можетъ быть покой душевный; у лживаго и обманщика не можетъ быть покой душевный. Словомъ, у всякаго грѣшника, который законъ Божій разоряетъ, не можетъ быть покой душевный. Ибо разоряя законъ Божій, разоряетъ и покой совѣсти, и раздражаетъ ее, которая обличаетъ и терзаетъ душу согрѣшившую. Что бо законъ Божій говоритъ, тое говоритъ и совѣсть; и въ чемъ законъ Божій обличаетъ человѣка, въ томъ обличаетъ и совѣсть. Грѣшитъ ли убо человѣкъ противу закона Божія, "--- грѣшитъ и противу совѣсти, и раздражаетъ ее. Дѣлается ли что противу совѣсти, "--- дѣлается и противу закона Божія. Откуду бываетъ, что когда богобоящійся человѣкъ и совѣсть свою чистую хранить тщащійся въ чемъ проступится и согрѣшитъ, тотчасъ совѣсть его ударитъ и душу его возмутитъ, и ее безпокойствуетъ дотолѣ, доколѣ покаяніемъ, воздыханіемъ и сокрушеніемъ сердца не очиститъ грѣхъ, и тако покой душевный паки возвратится. И хотя и у людей беззаконно живущихъ совѣсть до времени молчитъ и аки спитъ; но когда пробудится и грѣхи и беззаконія своя содѣланная въ ней увидитъ: тогда сильно на нихъ востанетъ, и начнетъ мучить ихъ и терзать, и такъ смущать, что къ отчаянію склоняются, а часто сами себе и умерщвляютъ, не стерпя мучительства совѣсти, какъ"=то случилось Іудѣ, предателю Господню. \textit{И повергъ сребреники въ церкви, отъиде, и шедъ удавися}\footnote{Мѳ.~27,~5.}. Сего ради людемъ беззаконнующимъ, чтобы покой душевный имѣть, должно отъ грѣховъ обратиться къ Богу, и совѣсть свою грѣхами замаранную покаяніемъ очистить, и раздраженную усмирить, и тако покой и тишина душевная возвратится къ нимъ. Безъ того бо покоя имѣть никакъ не могутъ, какъ бы совѣсть свою ни утѣшали; и опасно, чтобы отъ временнаго сего безпокойствія къ вѣчному безпокойствію не пріитить, гдѣ совѣсть души согрѣшившей всегда будетъ грѣхи представлять, и воспоминаніемъ ихъ мучить ее и терзать. Сей есть плодъ грѣха. Сладокъ грѣхъ людямъ, но плодъ его горекъ: \textit{оброцы бо грѣха смерть}\footnote{Римл.~6,~23.}. 2)~Тщится отнять покой душевный у людей благочестивыхъ сатана всякимъ и различнымъ \textit{искушеніемъ} и злыми помыслами, когда ихъ старается въ грѣхъ привести, и тако совѣсть ихъ раздражить и обезпокоить. О возлюбленне! берегись сего злаго шепотника, да не потеряеши покой душевный. Сколько разъ чувствуеши злое какое помышленіе, востающее въ сердцѣ твоемъ, знай точно, что тогда приступилъ врагъ къ душѣ твоей, и даетъ ей злый совѣтъ свой, яко Евѣ змій. Духъ отъ запаха познается. Злый духъ злый и запахъ издаетъ. Чувствуешь злый запахъ, знай точно, что тутъ и злый духъ, который мерзкую свою и смрадную воню изрыгаетъ. Отгоняй убо отъ себе злый сей запахъ, да и злаго духа отженеши. \textit{Повинитеся убо Богу, противитеся же діаволу, и бѣжитъ отъ васъ}\footnote{Іак.~4,~7.}. Не попущай душѣ твоей слушати долго шептанія того, да не, яко змій Еву, прельститъ тебе; но тотчасъ, какъ только послышишь злый совѣтъ его, отвращайся его, и обращайся ко \textit{всесильному} Іисусу, Заступнику и Спасителю твоему, и молись Ему, да поможетъ тебѣ въ подвигѣ твоемъ. Скученъ и труденъ подвигъ сей; но преславная тому послѣдуетъ побѣда. Подвизаемся мы, и не уступаемъ врагу; но помогаетъ намъ и побѣждаетъ въ насъ Іисусъ Хрістосъ, смерти и ада Побѣдитель. Тому благодареніе и слава. \textit{Благослови душе моя Господа}\footnote{Пс.~102,~1.}!

\textit{Помощникъ и покровитель бысть мнѣ во спасеніе: Сей мой Богъ, и прославлю Его}\footnote{Исх.~15,~2.}. Сколько разъ противимся врагу нашему и отражаемъ злые совѣты его, столько разъ побѣждаемъ и посрамляемъ его; сколько разъ побѣждаемъ его, столько разъ возвеселяемъ ангеловъ святыхъ и прославляемъ Бога и Отца нашего, Иже на небесѣхъ. Аще убо злый врагъ приступаетъ къ тебѣ, и слышиши злый совѣтъ его: тотчасъ ободрись и стань, яко воинъ Хрістовъ, и помяни, яко Богъ видитъ все, и смотритъ на тебе, и ожидаетъ подвига твоего, и хощетъ помощи тебѣ. И сколько разъ вступаешь въ подвигъ противу врага своего и побѣждаешь его, столько разъ вѣнчаетъ тебе Подвигоположникъ Іисусъ, и радуются святіи ангели Его о тебѣ. \textit{Радость бо бываетъ предъ ангелами Божіими о грѣшникѣ кающемся}\footnote{Лук.~15,~10.}. Ибо и подвигъ сей, подвигъ противу діавола и грѣха къ истинному покаянію надлежитъ. Истинно бо кающійся противу всякаго грѣха и того изобрѣтателя"=діавола подвизается. Сего ради стой въ подвигѣ твоемъ, и не уступай врагу твоему, да не посмѣется тебѣ, но паче да посрамится. Хрістосъ Господь съ престола славы Своея смотритъ на подвигъ твой, и готовитъ тебѣ вѣчной славы вѣнецъ; смотрятъ и святіи ангели на тебе, и радуются о подвигѣ твоемъ. Хрістосъ Господь глаголетъ тебѣ: \textit{буди вѣренъ до смерти, и дамъ ти вѣнецъ живота}\footnote{Апок.~2,~10.}. 3)~Тщится отнять покой душевный той же злый супостатъ, когда человѣка благочестиваго и пекущагося о своемъ спасеніи смущаетъ страхомъ и ужасомъ вѣчныя смерти и ада, и \textit{отчаяніемъ} Божія милости. Таковыи помыслы суть разжженныя стрѣлы лукаваго, которыя пущаетъ и мещетъ на вѣрную душу и тѣми весьма ее возмущаетъ. Таковымъ безпокойствіемъ искушается вѣра и надежда наша, твердо ли стоитъ человѣкъ въ вѣрѣ и надеждѣ, неотступно ли ожидаетъ милости Божія, которую вѣрнымъ Своимъ обѣщалъ. Тако безпокойствуемой и смущаемой душѣ должно съ помощію Божіею въ вѣрѣ и надеждѣ утвердиться, взирая на неложное Божіе обѣщаніе и за тое крѣпко и неуклонно держася, яко обѣщавый спасти вѣрующихъ въ Него непремѣнно спасетъ, и тако \textit{упованіе не посрамитъ}\footnote{Римл.~5,~5.}. \textit{Вѣрую видѣти благая Господня на земли живыхъ. Потерпи Господа, мужайся, и да крѣпится сердце твое, и потерпи Господа}\footnote{Пс.~26,~13 и 14.}. Вѣрующему убо должно и терпѣть и ожидать. \textit{Вѣренъ бо есть обѣщавый}\footnote{Евр.~10,~23.}. \textit{Аще Богъ по насъ: кто на ны? Иже убо Сына Своего не пощадѣ, но за насъ всѣхъ предалъ есть Его: како убо не и съ Нимъ вся намъ дарствуетъ}\footnote{Римл.~8,~23.}? 4)~Безпокойствуетъ благочестивую душу тойжде врагъ чрезъ \textit{злыхъ людей}. Отсюду бываетъ, что благочестивая душа претерпѣваетъ всякую ненависть, озлобленіе, гоненіе, поношеніе, клевету и прочее злостраданіе. И сіе есть кознь злаго духа. Первѣе самъ искушаетъ и смущаетъ благочестивую душу; но когда не удастся ему злое свое хотѣніе на ней исполнить и ее совратить, то другій изобрѣтаетъ способъ къ озлобленію ея, насылаетъ на нее злыхъ людей, которыи волю его злую творятъ. Тогда на человѣка то оттуду, то отсюду буря гоненій, поношеній, клеветъ и озлобленій востаетъ. Душѣ благочестивой въ такомъ бѣдственномъ состояніи не должно унывать и волноваться, но къ тихому терпѣнія пристанищу прибѣгать; терпѣніе бо душѣ, на морѣ міра сего волнующейся и бѣдствующей, есть подлинно тихое и безопасное пристанище, "--- и поминать утѣшительныя Божія слова предложенія: \textit{въ мірѣ скорбни будете}\footnote{Іоан.~16,~33.}. \textit{Аще господина дому веельзевула нарекоша, кольми паче домашнія Его}\footnote{Мѳ.~10,~25.}. \textit{Многими скорбьми подобаетъ намъ внити въ царствіе Божіе}\footnote{Дѣян.~14,~22.}. О чемъ и гласъ съ небесе слышанъ бысть, свидѣтельствующій о спасенныхъ: \textit{сіи суть, иже пріидоша отъ скорби великія}, и проч.\footnote{Апок.~7,~14.} \textit{Егоже любитъ Господь, наказуетъ: біетъ же всякаго сына, егоже пріемлетъ. Аще наказаніе терпите, якоже сыновомъ обрѣтается вамъ Богъ: который бо есть сынъ, егоже не наказуетъ отецъ}\footnote{Евр.~12,~6 и 7.}. Тако бо души благочестивыя сообразными дѣлаются единородному Сыну Божію Іисусу Хрісту, Который отъ злаго міра обезчещенъ, поруганъ, озлобленъ и крестною смертію умерщвленъ. Скажи, пожалуй, не утѣшительно ли человѣку быть сообразнымъ Господу славы, и Ему, яко Агнцу Божію, послѣдовать, "--- сообразнымъ здѣ въ мірѣ семъ въ страданіи, въ вѣчной же жизни "--- въ славѣ? \textit{Съ Нимъ страждемъ, да и съ Нимъ прославимся}\footnote{Римл.~8,~17.}. Воистинну не можетъ быть большее утѣшеніе хрістолюбивой душѣ. Воистинну возрадуется и возвеселится душа во всякомъ злостраданіи, когда сіе разсудитъ. Разсуди, возлюбленне, кто ты, и кто Хрістосъ, Которому дѣлаешися сообразнымъ. Ты "--- земля и пепелъ, да еще и грѣшникъ: Хрістосъ "--- Господь славы, единородный Сынъ Божій, Царь вѣчный, и вѣчнаго Царя Сынъ, Богъ великій и крѣпкій, Создатель твой, Любитель твой, Искупитель твой, Спаситель твой, Который такъ тебе возлюбилъ, что и предалъ Себе за тебе, и пострадалъ и умеръ за твою душу, и тако избавилъ тебе отъ смерти и ада. Сія сотворилъ Онъ ради тебе; понеже не терпѣлъ въ погибели видѣти тебе. \textit{Хвали, душе моя, Господа, возлюбившаго тебе, и предавшаго Себе по тебѣ}\footnote{Пс.~145,~1; Еф.~5,~2.}. Хрістіанине! Тому ты славы Господу, ты земля и персть, сообразнымъ дѣлаешися. О, воистинну великая честь, великая слава, великое блаженство, великая похвала, великое утѣшеніе, радость и веселіе, быть сообразнымъ единородному Сыну Божію и здѣ, и въ вѣчномъ Его царствіи! О Іисусе, радосте и сладосте ангеловъ святыхъ! влецы мене за Собою. "--- Побѣжимъ... \textit{Смѵрна и стакти и кассія отъ ризъ Твоихъ}\footnote{Пс.~44,~96.}. Духъ бодръ, но плоть немощна; духъ желаетъ, но плоть отвращается; духъ стремится, но плоть мятется; духъ восхищается, но плоть обременяетъ; духъ бѣжитъ, но плоть удерживаетъ. Влецы убо немощнаго крѣпкій, и яко сильный исполинъ слабое отроча, влецы мене. "--- Побѣжимъ, "--- чимъ? прекраснымъ и спасительнымъ путемъ Твоимъ, которымъ Ты отъ рожденія до смерти, мене ради бѣднаго, шествовалъ. \textit{Видѣна быша шествія твоя, Боже}\footnote{67,~25.}. Куды? въ вѣчное Твое и преславное царствіе. Туды побѣжимъ, да и здѣ, и тамо сообразенъ Тебѣ буду. Аминь. \textit{Аще кто Мнѣ служитъ, Мнѣ да послѣдствуетъ; и идѣже есмь Азъ, ту и слуга Мой будетъ; и аще кто Мнѣ служитъ, почтитъ его Отецъ Мой}\footnote{Іоан.~12,~16.}.

\textit{Трезвитеся, бодрствуйте: зане супостатъ вашъ діаволъ, яко левъ рыкая, ходитъ, искій кого поглотити. Емуже противитеся тверди вѣрою}\footnote{1~Петр.~56,~8 и 9.}.

\section{101. Узда.}

Что коню свирѣпѣющему и бѣснующемуся узда, тое плоти страстной и похотливой воздержаніе. Конь уздою воздерживается и повинуется хотѣнію всадника правящаго: тако хрістіанину должно плоть похотствующую воздерживать и покорять духу или уму. Но повидимъ, чего хощетъ плоть, и чего духъ, и какая между ими брань. Плоть хощетъ гордиться, возноситься: духъ того не хощетъ. Плоть хощетъ другаго презирать и уничтожать: духъ того не хощетъ. Плоть хощетъ человѣка оклеветать, осудить, опорочить: духъ того не хощетъ. Плоть хощетъ въ гордости и пышности міра сего жить: духъ не хощетъ. Плоть хощетъ богатство, славу и честь въ мірѣ семъ искать и имѣть: духъ не хощетъ. Плоть хощетъ всякія веселости въ мірѣ семъ имѣть: духъ тѣхъ отвращается. Плоть хощетъ банкетовать и пиршествовать и веселиться: духъ того не хощетъ. Плоть хощетъ красною и щегольскою одеждою одѣваться: духъ не хощетъ. Плоть хощетъ за обиду на ближняго гнѣваться, злобиться, отмщевать, зло за зло и досажденіе за досажденіе воздавать, укоряющаго укорять, поношающему поносить, злословящаго злословить, обидящаго обидѣть и съ нимъ ссориться: духъ всего того отвращается. Плоть хощетъ блудодѣйствовать, прелюбодѣйствовать и въ нечистотѣ валяться: духъ того отвращается. Плоть хощетъ упиваться и піянствовать: духъ того не хощетъ. Плоть хощетъ лгать, льстить, обманывать: духъ того не хощетъ. Плоть хощетъ въ праздности жить: духъ того не хощетъ. Плоть хощетъ празднословить, буесловить, кощунствовать: духъ того не хощетъ. Плоть хощетъ чуждое добро похищать, красть, отнимать и всякимъ образомъ себѣ присвоять тое: духъ не хощетъ, и прочая. Се есть плотское хотѣніе и мудрованіе. Напротивъ того, чего хощетъ духъ, того не хощетъ плоть. Духъ хощетъ волѣ Божіей послѣдовать и ее творить: плоть не хощетъ. Духъ хощетъ предъ Богомъ и человѣками смиряться: плоть не хощетъ. Духъ хощетъ Богу послушаніе показывать и заповѣди Его святыя творить: плоть не хощетъ. Духъ хощетъ любовно, мирно, искренно, простосердечно съ ближнимъ поступать: плоть того не хощетъ. Духъ хощетъ воздержно и цѣломудренно жить: плоть не хощетъ. Духъ хощетъ ближняго миловать и его въ нуждахъ снабдѣвать и тому помогать: плоть не хощетъ. Духъ хощетъ обиды терпѣть, и согрѣшенія человѣкамъ прощать: плоть того не хощетъ. Духъ хощетъ враговъ любить, добро творить ненавидящимъ, благословить кленущихъ, и молиться за творящихъ напасть: плоть того всего отвращается. Духъ хощетъ въ бѣдахъ, напастяхъ, искушеніяхъ и всякомъ злостраданіи молчать, терпѣть, и кротко сносить: плоть того не хощетъ. Духъ хощетъ во всемъ надежду свою полагать на Бога: плоть того не хощетъ. Духъ хощетъ всякую правду творить: плоть не хощетъ. Духъ хощетъ во всемъ Богу угождать: плоть не хощетъ, но себѣ угождать, и проч. Се есть хотѣніе и мудрованіе духовное. Видишь, какая брань и несогласіе между плотію и духомъ. И сіе"=то есть, что Апостолъ глаголетъ: \textit{плоть похотствуетъ на духа, духъ же на плоть: сія же другъ другу противятся}\footnote{Гал.~5,~17.}. Видишь мудрованіе плотское, и мудрованіе духовное; видишь, чего хощетъ плоть, и чего хощетъ духъ. Якоже убо конь обуздывается и уздою покоряется хотѣнію всадника: тако намъ, хрістіанине, должно покорять плоть духу, и плотское мудрованіе духомъ умерщвлять, и дѣлать не тое, что страстная и похотливая плоть хощетъ, но тое, что духъ хощетъ. \textit{Не будите яко конь и мескъ, имже нѣсть разума}\footnote{Пс.~31,~9.}. \textit{Сущіи по плоти, плотская мудрствуютъ: а иже по духу, духовная. Мудрованіе бо плотское, смерть есть: а мудрованіе духовное, животъ и миръ. Зане мудрованіе плотское, вражда на Бога: закону бо Божію не покоряется, ниже бо можетъ. Сущіи же во плоти, Богу угодити не могутъ. Тѣмже убо, братіе, должни есмы не плоти, еже по плоти жити. Аще бо по плоти живете, имате умрети: аще ли духомъ дѣянія плотская умерщвляете, живи будете. Плоти угодія не творите въ похоти}\footnote{Римл.~8,~5--8,~12 и 13; 13,~14.}. \textit{Духомъ ходите, и похоти плотскія не совершайте. Плоть бо похотствуетъ на духа, духъ же на плоть: сія же другъ другу противятся, да не яже хощете, сія творите. А иже Хрістовы суть, плоть распяша со страстьми и похотьми}\footnote{Гал.~5,~16~,17 и 24.}. \textit{Иже въ девятый часъ насъ ради плотію смерть вкусивый, умертви плоти нашея мудрованія, Хрісте Боже, и спаси насъ}.

\section{102. Иди за мною.}

Бываетъ, что единъ другому въ различномъ случаѣ говоритъ: \textit{иди за мною}. Хрістіанине! тако Хрістосъ Господь всякому отъ насъ говоритъ: \textit{иди за Мною}. Читай святое Евангеліе, и внимай тому, и услышишь сладчайшій Его гласъ сей: \textit{иди за Мною}. Я Создатель твой, ты созданіе Мое: \textit{иди за Мною}. Я Господь твой, ты рабъ Мой: \textit{иди за Мною}. Я Царь твой, ты подданный мой: \textit{иди за Мною}. Я Богъ твой, въ подобострастную тебѣ плоть облекшійся: \textit{иди за Мною}. Я въ міръ ради тебе пришелъ: \textit{иди за Мною}, Который къ тебѣ и ради тебе пришелъ. Я, невидимый, на земли явился ради тебе и всѣхъ: \textit{иди за Мною}. Я, неприступный херувимамъ и серафимамъ, приступенъ грѣшникамъ учинился, и тебѣ: \textit{иди за Мною}. Я, Царь небесный, на земли пожилъ ради тебе: \textit{иди за Мною}. Я, всесильный и всемогущій, немощнымъ учинился ради тебе: \textit{иди за Мною}. Я, богатъ сый, тебе ради обнищалъ, да ты нищетою Моею обогатишися\footnote{2~Кор.~8,~9.}: \textit{иди за Мною}. Я, Господь славы, поруганъ и обезчещенъ ради тебе: \textit{иди за Мною}. Я, животъ вѣчный и присносущный, вкусилъ смерти ради тебе, смерти же крестныя: \textit{иди за Мною}. Я, на престолѣ славы сѣдяй, и отъ ангеловъ покланяемый и хвалимый и славимый, отъ грѣшниковъ похуленъ былъ ради тебе: \textit{иди за Мною}. Я, единъ имѣяй безсмертіе и во свѣтѣ живый неприступномъ, въ мертвыхъ вмѣненъ и въ темномъ гробѣ положенъ былъ тебе ради: \textit{иди за Мною}. Я твой Искупитель, твой Избавитель, твой Спаситель, Который тебе не сребромъ и златомъ, но Своею кровію отъ діавола, смерти и ада искупилъ: \textit{иди за Мною}. Видишь Мою къ тебѣ любовь, покажи и ты Любителю твоему свою любовь, и любовію \textit{иди за Мною}. Любовь твоя тебѣ пользуетъ, а не Мнѣ, якоже нелюбовь тебѣ вредитъ, а не Мнѣ: \textit{иди за Мною. Господи! что есть человѣкъ, яко познался еси ему, или сынъ, человѣчь, яко вмѣняеши его? Человѣкъ суетѣ уподобился}\footnote{Пс.~43,~3 и 4.}. Пою Тя, слухомъ бо, Господи, услышахъ и ужасохся. До мене бо пришелъ еси, мене ища заблуждшаго. Тѣмъ многое Твое снисхожденіе, еже для мя, прославляю, многомилостиво. Готово сердце мое, Боже, готово сердце мое. Помози мнѣ, Господи Боже мой, и спаси мя по милости Твоей. "--- Что, хрістіанине, хощеши ли отрещися толикаго Господа твоего и Благодѣтеля, Который за Собою зоветъ тебе, и не итить за Нимъ? Страшно и безстыдно! Страшно, чтобы не подвигнуть Его на праведный гнѣвъ, и не подпасть суду Его праведному, и тако не погибнуть. Безстыдно, ибо Онъ высочайшій твой Любитель и Благодѣтель. Скажи пожалуй: ежели бы царь земный позвалъ тебе за собою, не все ли бросивши, съ радостію поспѣшилъ бы за нимъ, ради малыя и временныя чести и корысти? Царь царствующихъ и Господь господствующихъ, Царь небесный зоветъ тебе за Собою: \textit{иди за Мною}; и зоветъ не къ временной чести и славѣ и корысти, но къ вѣчному животу, царствію, чести и славѣ. \textit{Иди за Мною, иди за Мною}, и приведу тебе въ вѣчное Мое царство, къ вѣчному и небесному Моему Отцу: \textit{никтоже бо пріидетъ ко Отцу токмо Мною}\footnote{Іоан.~14,~16.}. Блудникъ, прелюбодѣй и нечистоты любитель! слыши гласъ Господень: \textit{покайся, и иди за Мною}. Злобный мститель и убійца! слыши гласъ Господень: \textit{покайся, и иди за Мною}. Тать, хищникъ, грабитель и лихоимецъ! слыши гласъ Господень: \textit{покайся, и иди за Мною}. Укоритель, ругатель, клеветникъ и всякъ злорѣчивый! слыши гласъ Господень: \textit{покайся, и иди за Мною}. Лживый, обманщикъ, прелестникъ и лицемѣръ! слыши гласъ Господень: \textit{покайся, и иди за Мною}. Всякъ грѣшникъ, въ нераскаяніи живущій! слыши гласъ Господень: \textit{покайся, и иди за Мною}. Слыши гласъ Господень, зовущій тебе; слыши гласъ Того, Который такъ тебе возлюбилъ, и такъ милостивый, человѣколюбивый и чудный промыслъ о тебѣ показалъ; слыши гласъ Того: \textit{покайся, и иди за Мною}. Человѣче, любимое Мое созданіе! Я снидохъ съ небесе, да тебе на небо возведу: \textit{иди за Мною}. Я на земли пожилъ, да тебе жителя небеснаго сотворю: \textit{иди за Мною}. Я не имѣлъ, гдѣ главы подклонить, да тебе въ домъ Отца Моего небеснаго приведу: \textit{иди за Мною}. Я трудился, да тебе въ вѣчный покой введу: \textit{иди за Мною}. Я обнищалъ, да тебе обогащу: \textit{иди за Мною}. Я плакалъ, болѣзновалъ, скорбѣлъ и тужилъ, да тебѣ истинное подамъ утѣшеніе, радость и веселіе: \textit{иди за Мною}. Я похуленъ, поруганъ и обезчещенъ былъ, да тебе обезчещенное Мое созданіе почту и прославлю: \textit{иди за Мною}. Я связанъ былъ, да тебе отъ узъ грѣховныхъ разрѣшу: \textit{иди за Мною}. Я, Судія живыхъ и мертвыхъ, судимъ былъ и осужденъ, да тебе вѣчнаго суда избавлю: \textit{иди за Мною}. Я со беззаконными вмѣненъ былъ, да тебе оправдаю: \textit{иди за Мною}. Я вкусилъ смерти, смерти же крестныя, да тебе, созданіе Мое, ядомъ зміинымъ умерщвленное, оживлю: \textit{иди за Мною}. Я восталъ изъ мертвыхъ, да ты душею и тѣломъ востанеши: \textit{иди за Мною}. Я вознеслся на небо, да и ты вознесешися: \textit{иди за Мною}. Я сѣлъ одесную Бога и Отца, да и ты прославишися: \textit{иди за Мною}. Я во всемъ уподобился тебѣ, кромѣ грѣха, да и ты подобенъ Мнѣ будеши: \textit{иди за Мною}. Образъ твой на Себе воспріялъ, да и ты Мнѣ сообразенъ будеши: \textit{иди за Мною}. Я пришелъ къ тебѣ, да привлеку тебе къ Себѣ, тебе, любезное Мое созданіе отпадшее: \textit{иди за Мною}. Видишь, человѣче, любовь Мою, къ тебѣ показанную; видишь и промыслъ Мой о тебѣ. За сіе ничего отъ тебе не требую, какъ только того, чтобы ты Мнѣ благодаренъ былъ и шелъ за Мною, и тако бы спасеніе со славою вѣчною получилъ. Сего отъ тебе хощу; да будетъ убо хотѣніе и твое. \textit{Иди убо за Мною}, Который алчу и жажду спасенія твоего, и подамъ тебѣ спасеніе Мое. Возлюби Мою нищету, и будеши истинно богатъ. Возлюби Мое смиреніе, и будеши истинно славенъ и великъ. Возлюби Мою кротость и терпѣніе, и будеши истинно тихъ, покоенъ и миренъ. Остави землю, и будеши имѣть небо. Остави міръ, и будеши имѣть внутрь тебе Бога. Отрекись себе, и будеши истинно обладать собою. Остави утѣху плоти и міра, и будеши имѣть истинное утѣшеніе. Остави богатство тлѣнное, и будеши имѣть нетлѣнное. Остави честь и славу земную, и будеши имѣть небесную. \textit{Иди за Мною}, и будеши имѣть все, чего желаетъ душа твоя, но истинное, несравненно лучшее паче всего, что ни оставишь. Слыши убо и внимай, душа грѣшная, гласъ любителя и Искупителя твоего Іисуса, и иди за Нимъ, да постигнеши. Поспѣшай, поспѣшай, возлюбленне, пока зоветъ, и двери отверсты. Небо отверсто, и входятъ въ тое мытари, прелюбодѣи и всякіе грѣшники кающіися; поспѣшай и ты, бѣдный грѣшникъ, тудыже, да и ты восхитиши царство небесное. Туды вошли апостоли, мученики, святители, преподобные и вси святіи Божіи и водворяются нынѣ во обителяхъ небеснаго Отца; идутъ туды и нынѣ многіи, и приходятъ, но идущіи за пресладкимъ своимъ Іисусомъ, и благодатію Его вселяются въ покоѣ вѣчномъ. Душе моя, слыши и ты гласъ Господень, и прилѣпися къ святой дружинѣ той, и съ ними иди за Іисусомъ, вѣрнымъ вождемъ, да и въ часъ исхода твоего услышиши гласъ Его: \textit{въ путь узкій хождшіи прискорбный, вси въ житіи крестъ яко яремъ вземшіи, и Мнѣ послѣдовавшіи вѣрою, пріидите, насладитеся, ихже уготовахъ вамъ почестей и вѣнцевъ небесныхъ. "--- Хрістосъ пострада по насъ, намъ оставль образъ, да послѣдуемъ стопамъ Его. Иже грѣха не сотвори, ни обретѣся лесть во устѣхъ Его; Иже укоряемъ противу не укоряше, стражда не прещаше, предаяше же судящему праведно}\footnote{1~Петр.~2,~21--23.}. \textit{Аще кто Мнѣ служитъ, Мнѣ да послѣдствуетъ: и идѣже есмь Азъ, ту и слуга Мой будетъ}\footnote{Іоан.~12,~26.}. \textit{Иже не пріиметъ креста своего, и въ слѣдъ Мене грядетъ, нѣсть Мене достоинъ}\footnote{Мѳ.~10,~38.}. \textit{Азъ есмь свѣтъ міру: ходяй по Мнѣ, не имать ходити во тьмѣ, но имать свѣтъ животный}, глаголетъ Господь\footnote{Іоан.~8,~12.}.

\subsection{О томжде.}

Бываетъ, что единъ зоветъ человѣка, глаголя: \textit{иди за мною}; но другій говоритъ ему, отзывая къ себѣ: \textit{иди за мною}. Тако Хрістосъ Господь всякому хрістіанину глаголетъ: \textit{иди за Мною}, какъ выше сказано; но сатана, врагъ рода человѣческаго, шепчетъ во уши человѣка, и отзываетъ къ себѣ, глаголя: \textit{иди за мною}. Слышитъ человѣкъ злыи шептанія его столько разъ, сколько разъ чувствуетъ злые и богопротивные помыслы, въ сердцѣ его востающіе. Сіе все есть мерзкое шептаніе его. О человѣче! кого тебѣ лучше послушать? Хріста ли Господа твоего, Который такъ чудный о тебѣ показалъ промыслъ, и хощетъ тебе спасти, и въ вѣчное Свое привести царствіе, какъ выше видѣлъ ты, "--- или злаго шепотника діавола, который хощетъ тебе отъ Хріста Спасителя твоего отвести, и въ вѣчную погибель вринуть? Хрістосъ есть свѣтъ: діаволъ есть тьма. Хрістосъ есть животъ: діаволъ есть смерть. Хрістосъ есть истина: діаволъ есть ложь, и отецъ лжи. Хрістосъ есть любитель твой: діаволъ есть врагъ твой. Хрістосъ есть благодѣтель твой: діаволъ есть злотворитель твой. Хрістосъ есть истинное и высочайшее добро; діаволъ крайнее зло. Хрістосъ есть Спаситель твой: діаволъ губитель твой. Хрістосъ хощетъ вѣчно спасти тебе; на сіе бо и въ міръ пришелъ ради тебе: діаволъ хощетъ вѣчно погубить тебе. Хрістосъ хощетъ вѣчный животъ подать тебѣ: діаволъ хощетъ вѣчно умертвить тебе. Хрістосъ хощетъ въ вѣчное Свое царство ввести тебе: діаволъ хощетъ въ вѣчное мученіе за собою привести тебе. Хрістосъ хощетъ вѣчно обогатить тебе: діаволъ хощетъ вѣчно нищимъ сотворить тебе. Хрістосъ хощетъ вѣчно прославить тебе: діаволъ хощетъ посрамить тебе. Хрістосъ хощетъ вѣчно почтить тебе: діаволъ хощетъ вѣчно обезчестить тебе. Видишь, что есть Хрістосъ, и что есть діаволъ, и ради чего шепчетъ онъ тебѣ и отзываетъ отъ Хріста Господа твоего въ слѣдъ себе. Хощетъ онъ тебе вѣчно погубить, якоже и самъ въ погибели находится. Сія есть хитрость его, сей есть злой умыслъ его на насъ. Отрекись убо, отрекись, возлюбленне, злаго шепотника того, и плюнь на него, якоже въ святомъ крещеніи сотворилъ ты, и, отвратившись прелестника того, обратись ко Хрісту Господу твоему, и послѣдуй Ему вѣрою и правдою, якоже обѣщался Ему въ крещеніи. Богъ твой Онъ есть, Господь твой, любитель твой, благодѣтель твой, вѣчный животъ твой, присносущный свѣтъ твой, истинное и вѣчное блаженство твое, безъ котораго ни въ семъ вѣкѣ, ни въ будущемъ блаженными быть не можемъ. Пристань убо и прилѣпись Господу Богу твоему: \textit{мнѣ же прилѣплятися Богови благо есть}. Но не смотря на то, что діаволъ губитель есть, многіи слушаютъ его, и за нимъ великими толпами идутъ. Идутъ блудники, прелюбодѣи, и вси нечистоты любители. Идутъ памятозлобивыи, ненавистники человѣческіи, убійцы и проливающіи кровь человѣческую. Идутъ противящіися родителямъ, злословящіи ихъ и непочитающіи. Идутъ тати, хищники, грабители, мздоимцы "--- судіи, удерживающіи мзду наемничу, и всякою неправдою и лестію чуждое добро похищающіи. Идутъ клеветники, ругатели и вси, ближняго своего языкомъ, какъ мечемъ, біющіи и уязвляющіи. Идутъ піяницы и сластолюбцы. Идутъ хитрецы, лукавцы, прелестники и вси ближняго своего прельщающіи. Идутъ всякіи беззаконники, развращеннымъ житіемъ своимъ слову Божію противящіися. Идутъ наконецъ возлюбившіи вѣкъ сей, земная, а не горняя мудрствующіи, въ гордости и пышности міра сего любящіи жить. Словомъ, вси, которыи о словѣ Божіи и о Бозѣ небрегутъ, къ земнымъ сердцемъ прилѣпляются, и о вѣчномъ спасеніи своемъ, кровію Хрістовою сысканномъ, нерадятъ; вси таковыи идутъ въ слѣдъ сатаны. Всѣмъ таковымъ приличествуетъ оное апостольское слово: \textit{се нѣкія развратишася въ слѣдъ сатаны}\footnote{1~Тим.~5,~15.}. Вси бо таковыи срамляются словесъ Хрістовыхъ и Самого Хріста; ужасаются Его смиреннаго, поруганнаго и обезчещеннаго, и удаляются отъ Него; однакожъ хотятъ быть съ прославленнымъ. Не хотятъ здѣ въ мірѣ семъ съ Нимъ быть, и Ему послѣдовать; однакожъ хотятъ царствія Его участниками быть, что невозможное дѣло есть. Съ кѣмъ кто въ мірѣ семъ, съ тѣмъ и въ будущемъ вѣкѣ будетъ. О человѣче, который Хрістово имя на себѣ носиши, но сатанѣ послѣдуеши! помяни, какія отрицанія и обѣты чинилъ ты въ крещеніи, како отрицался сатаны и всѣхъ дѣлъ его и гордыни его, како плевалъ на него, како отходилъ отъ него, како присталъ ко Хрісту, како обѣщался и присягалъ вѣрою и правдою работать Ему, смиреніемъ и любовію послѣдовать Ему, якоже послѣдуетъ невѣста жениху. Гдѣ нынѣ тыя твоя отрицанія, гдѣ обѣты, гдѣ присяга, гдѣ работа Хрісту, гдѣ Тому послѣдованіе? \textit{Богу солгалъ ты, а не человѣку}. Бога оставилъ ты, а не человѣка; свѣтъ оставилъ ты, и возлюбилъ тьму; животъ оставилъ ты, и возлюбилъ смерть. Помяни убо сія, и обратися ко Хрісту Спасителю твоему. Бѣги отъ прелестника того, якоже Израиль отъ Фараона; и хотя въ слѣдъ тебе поженетъ онъ, яко лишился своея корысти, однакожъ дерзай, и изъ глубины сердца воздыхай ко всесильному Іисусу, да не помянетъ беззаконій твоихъ, и поможетъ тебѣ. Онъ съ радостію ждетъ тебе, Который умеръ за тебе, и милосердыми объятіями, \textit{яко блуднаго сына, обыметъ тебе}\footnote{Лук.~15,~20.}. А когда отъ тяжкія работы мучителя того свободишися, тогда съ радостію воспоеши побѣдную пѣснь Помощнику твоему: \textit{Помощникъ и Покровитель бысть мнѣ во спасеніе: Сей мой Богъ, и прославлю Его: Богъ отца моего, и вознесу Его}\footnote{Исх.~15,~2.}. \textit{Возрадуются на небеси и ангели о тебѣ. Радость бо бываетъ предъ ангелами Божіими о единомъ грѣшницѣ кающемся}\footnote{Лук.~15,~10.}.

\section{103. Ученіе.}

Видимъ, что люди различнымъ художествамъ и наукамъ обучаются. Иный учится платье шить, иный купечествовать, иный домы созидать, иный иностранными языками говорить, иный красно и сладко говорить и писать, иный философствовать и въ естественныхъ вещахъ разумнымъ и искуснымъ быть, и проч. Тако хрістіанамъ должно обучаться по"=хрістіански жить. Безъ того бо всякое художество и наука ничтоже есть. Хрістіанине! внѣшняя наука и художество общее есть язычникамъ и хрістіанамъ; ибо и язычники, какъ видимъ, обучаются художествамъ и наукамъ; но благочестиво и по"=хрістіански жить ученіе и художество единымъ хрістіанамъ свойственно есть. Познается хрістіанинъ не отъ того, что красно говоритъ и пишетъ, но отъ того, что красно и богоугодно живетъ, "--- не отъ внѣшняго любомудрія, но отъ евангельской и хрістіанской философіи обученія. Многіи красно говорятъ и пишутъ, но грубо живутъ; многіи изрядно въ естественныхъ вещахъ философствуютъ, но хрістіанскаго и алѳавита не знаютъ. Истинный мудрецъ есть, кто міру юродъ, но Хрісту мудръ. Подлинно несмысленъ и безуменъ, кто хотя и все знаетъ, но Бога и Хріста Сына Божія не знаетъ. Учись убо, христіанине, хрістіанской евангельской философіи, и будеши истинно мудръ, хотя міръ и за юрода тебе будетъ имѣть. Обученіе хрістіанское состоитъ въ сихъ пунктахъ: 1)~Въ \textit{познаніи Бога}. Когда Бога будешь познавать, то будешь умудряться, но внутрь, а не внѣ; и чимъ болѣе будешь познавать, тѣмъ мудрѣйшій будеши и въ званіи хрістіанскомъ искуснѣйшій. Когда познаешь истинно, что Богъ на всякомъ мѣстѣ есть, и все твое дѣло, помышленіе, начинаніе и намѣреніе видитъ, и всякое слово твое слышитъ, и въ Его всемогущей рукѣ состоитъ животъ твой и смерть: сіе познаніе научитъ тебе неотмѣнно бояться Его и трепетать, и вездѣ опасно поступать, опасно говорить, опасно дѣлать, опасно мыслить и начинать. Когда познаешь, что Богъ правосудливъ, и воздаетъ всякому по дѣломъ его, то сіе познаніе подвигнетъ тебе къ усердному покаянію за преждебывшіи грѣхи; будешь и ты воздыхать и вопить къ Нему изъ глубины сердца: \textit{не вниди въ судъ съ рабомъ Твоимъ, яко не оправдится предъ Тобою всякъ живый}\footnote{Пс.~142,~2.}. И паки: \textit{аще беззаконія назриши Господи, Господи кто постоитъ}\footnote{129,~3.}? Когда познаешь, что Богъ есть истинная, присносущная и непремѣняемая благость и любовь, то неотмѣнно будешь любить Его. Кто бо, познавши добро, добра не любитъ? и проч. А отъ таковаго познанія неотмѣнно послѣдуетъ усердное Ему послушаніе, покореніе и повиновеніе, отъ послушанія тщательное заповѣдей Его святыхъ исполненіе.

2)~Въ воздержаніи и \textit{умерщвленіи страстей} и похотей, которыя воюютъ на душу. Но и здѣ нужно познаніе. Познавай убо, что въ сердцѣ твоемъ крыется. Посматривай туды часто, и увидишь, что въ немъ есть гордость, высокоуміе, самолюбіе, сребролюбіе, гнѣвъ, зависть, славолюбіе, нечистота, желаніе мщенія, и всякая грѣховная мерзость. \textit{Извнутрь бо отъ сердца человѣческа помышленія злая исходятъ, прелюбодѣянія, любодѣянія, убійства, татьбы, лихоимства, обиды, лукавствія, лесть, студодѣянія, око лукаво, хула, гордыня, безумство}\footnote{Марк.~7,~21--23; Мѳ.~15,~18 и 19.}. А когда исходитъ, то и имѣются тамо. И что въ единомъ человѣкѣ имѣется, то и въ другомъ, вси бо симъ ядомъ заражены мы. Сіе все зло не къ чему иному ведетъ человѣка, какъ только къ погибели. Учись убо познавать пагубное зло, крыющееся внутрь тебе, и тогда будешь учиться удерживать и умерщвлять тое, которое ведетъ тебе къ погибели, и тщаться побѣждать и убивать тѣхъ враговъ, которыи вѣчно хотятъ тебе умертвить. Надобно неотмѣнно плоть распять со страстьми и похотьми, когда хощешь по"=хрістіански жить и быть Хрістовымъ. Едини бо тіи суть Хрістовы, \textit{иже плоть распяша со страстьми и похотьми}, по ученію апостольскому\footnote{Гал.~5,~24.}. Учись убо познавать себе, да познаеши, коль великое бѣдствіе крыется въ тебѣ, и будешь тщаться отъ того избавитися; учись познавать враговъ, которыи внутрь, а не внѣ тебе суть, и, познавая, умерщвляй ихъ, да не вѣчно умертвятъ тебе. Когда они мертвы будутъ, то жива будетъ душа твоя; а когда они живы будутъ и обладаютъ тобою, то мертва будетъ душа твоя. Побѣждай убо и умерщвляй ихъ, да живетъ душа твоя. Возлюбленне! побѣди себе тако, и будешь тобою обладать, будешь истинно благороденъ, господинъ, царь и повелитель. 3)~Въ \textit{правдѣ и любви} къ ближнему. И здѣ познаніе нужно. Познавай, что ближній твой такойжде человѣкъ, какъ и ты, и все то имѣетъ, что и ты; и всего того не хощетъ, чего и ты; хощетъ всякаго добра, не хощетъ никакого зла, какъ и ты. Не дѣлай убо ему, чего себѣ не хощешь, и дѣлай, чего себѣ хощешь. Въ семъ состоитъ правда и любовь хрістіанская къ ближнему. Сего хощетъ и Господь нашъ отъ насъ: \textit{возлюбиши искренняго твоего, яко себе самого}\footnote{Мѳ.~19,~19.}. Любовь зла не творитъ ближнему, но хощетъ ему всякаго добра. Люби убо ближняго твоего, и не будешь ему дѣлать, чего себѣ не хощешь, и восхощешь дѣлать все, чего себѣ хощешь. Сія суть, въ которыхъ хрістіанину отъ младенчества до кончины живота нужно поучаться. Сія есть хрістіанская наука. Сіе есть изрядное художество. Сей философіи обучающійся есть истинный мудрости любитель. Начинающіи обучатися художеству и наукамъ признаютъ свое невѣжество, и тако обучаются тому, чего не знаютъ, ради того бо и вдаются въ наученіе, понеже не знаютъ: тако хрістіанамъ, когда хотятъ хрістіанской учитися и научитися философіи, должно признать свое невѣжество и безуміе. Безъ того суетно и безполезно будетъ обученіе. Учатся бо невѣжи, а не мудрецы. Обучающіися внѣшнимъ наукамъ и художествамъ, чимъ болѣе и тщательнѣе обучаются, тѣмъ искуснѣйшими бываютъ, какъ видимъ: тако, чимъ кто усерднѣе и тщательнѣе обучается хрістіанской философіи и по правиламъ ея себе исправляетъ, тѣмъ искуснѣйшій хрістіанинъ бываетъ. Чимъ болѣе Бога познаетъ, тѣмъ болѣе боится Его и любитъ; чимъ болѣе разсматриваетъ себе, тѣмъ болѣе смиряется и исправляется; чимъ болѣе разсуждаетъ о ближнемъ своемъ, тѣмъ и справедливѣе и любовнѣе съ нимъ обходится и поступаетъ; и тако къ совершенству, сколько возможно во плоти живущему, приходитъ. 4)~Всякое человѣческое тщаніе и усердіе, безъ помощи Божіей, не сильно. Ибо плоть всегда хощетъ обладати человѣкомъ; и разумъ нашъ безъ просвѣщенія Божія слѣпъ; діаволъ всегда и вездѣ ногамъ нашимъ полагаетъ сѣти, и тщится запнуть, а міръ съ соблазнами своими окружаетъ насъ, и хощетъ разслабить насъ и развратить. Сего ради неотмѣнно нужно со всякимъ усердіемъ и часто воздыхать и молиться ко Господу, да Самъ Онъ руководствуетъ насъ въ наукѣ нашей, и съ успѣхомъ обучаться поможетъ намъ. Онъ милостивно обѣщалъ намъ подать благодать Свою. А какъ обѣщалъ, такъ и подастъ просящимъ. \textit{Просите, и дастся вамъ; ищите и обрящете; толцыте, и отверзется вамъ. Всякъ бо просяй пріемлетъ, и ищай обрѣтаетъ, и толкущему отверзется}\footnote{Мѳ.~7,~7 и 8.}. Обучайся убо, хрістіанине, хрістіанской наукѣ, да будеши истинный хрістіанинъ и Хрістовъ ученикъ. Хрістосъ Господь не спроситъ насъ на второмъ Своемъ пришествіи, искусны ли мы были въ купечествѣ, краснорѣчіи, звѣздочетствѣ, въ географіи и прочіихъ наукахъ; но спроситъ, дѣлали ли и обучалися ли тому, чему научаетъ Его святое Евангеліе, и что въ крещеніи святомъ дѣлать обѣщались и присягали Ему. Сей науки и художества спроситъ у насъ тогда. "--- \textit{Боже, въ помощь мою вонми. Господи, помощи ми потщися}\footnote{Пс.~69,~2.}. \textit{Призри на мя и помилуй мя, по суду любящихъ имя Твое. Стопы моя направи по словеси Твоему, и да не обладаетъ мною всякое беззаконіе}\footnote{113,~132 и 133.}.

\section{104. Алѳавитъ.}

Что алѳавитъ или азбука начинающимъ учитися чтенію книжному, тое хрістіанамъ, хотящимъ учитися хрістіанскаго житія, есть смиреніе. Обучающимся книжному чтенію надобно прежде учитися алѳавита: тако хрістіанамъ, когда хотятъ съ успѣхомъ учитися добродѣтельнаго житія, надобно вопервыхъ учитися смиренія. Алѳавитъ есть начало и основаніе всего книжнаго ученія: тако смиреніе есть начало и основаніе хрістіанскихъ добродѣтелей. Безъ того бо всякое духовное созиданіе нарушается и падаетъ. Истинная добродѣтель безъ смиренія быть не можетъ, и хотя начнется, то паки разорится. Безъ Бога бо человѣкъ и начать и дѣлать не можетъ, по реченному: \textit{безъ Мене не можете творити ничесоже}\footnote{Іоан.~15,~5.}. А, по свидѣтельству Писанія, \textit{Богъ гордымъ противится, смиреннымъ же даетъ благодать}\footnote{1~Петр.~5,~5.}. Учись убо, хрістіанине, хрістіанскаго сего алѳавита, и будеши учитися съ успѣхомъ хрістіанской философіи. А чтобы не безъ успѣха учитися смиренія, то надобно учиться познавать себе и свою бѣдность. Отъ познанія бо себе и своея бѣдности неотмѣнно послѣдуетъ смиреніе. Кто бо, видя свою бѣдность, не смирится? Подлинно бѣденъ человѣкъ, но не всякъ свою бѣдность познаетъ. Раждается человѣкъ на бѣды, и тыя аки предчувствуя плачетъ, ибо всякій младенецъ съ плачемъ раждается. Живетъ и воспитывается въ бѣдахъ; чимъ болѣе живетъ, тѣмъ болѣе умножается ему бѣдъ. Умираетъ со страхомъ и на большія бѣды, кромѣ того, кто о Господѣ умираетъ. Бѣденъ внѣ человѣкъ, но и внутрь бѣденъ; различныя немощи и болѣзни удручаютъ и мучатъ его. Плоть со страстьми и похотьми востаетъ на него, и хощетъ обладать имъ и вѣчной смерти предать. Безпрестанно востаніе тоя и мучительство чувствуетъ человѣкъ; и чимъ болѣе противится ей, тѣмъ болѣе востаетъ она на него. И сіе внутреннее его бѣдствіе и зло, сей ядъ, которымъ зараженъ человѣкъ, не временно, но вѣчно умерщвляетъ его. Внѣ діаволъ непрестанно подлагаетъ сѣти ему, и тщится уловить его въ свою погибель. Различные міра соблазны, какъ терніе, бодутъ его и хотятъ душу его умертвить. Въ такомъ бѣдствіи живетъ человѣкъ всякъ, но тѣмъ бѣднѣйшій, когда не видитъ того. Чимъ убо человѣку возноситься и хвалиться? развѣ бѣдностію, тлѣніемъ и грѣхами? Какая се похвала? Сіе не похвала, но хула, не слава, но безславіе, не честь, но безчестіе. Чимъ убо похвалишися, человѣче? Развѣ тѣмъ, что богатство, славу, честь и премудрость вѣка сего имѣешь? Какая и се похвала? Благополучіе не внѣшнее, но внутреннее дѣлаетъ похвальнымъ и блаженнымъ человѣка. Всякое бо внѣшнее благополучіе ничто есть. Древо называемъ добрымъ не тое, которое внѣ красно, и многими вѣтвьми и листвіемъ изобилуетъ; но тое, которое внутрь добро и добры плоды творитъ. Бываетъ, что человѣкъ внѣ богатъ, но внутрь нищь и убогъ; внѣ славенъ и честенъ, но внутрь безславенъ и безчестенъ; внѣ мудръ, но внутрь юродъ, внѣ благополученъ, но внутрь окаяннѣйшій паче всѣхъ. И сіе по большей части случается. Ублажаютъ люди тѣхъ, которыи тако блистаютъ; но Божіе слово ублажаетъ тѣхъ, имже Господь Богъ ихъ. \textit{Блажени людіе, имже Господь Богъ ихъ}\footnote{Пс.~143,~15.}. Человѣче, не смотри, каковъ внѣ человѣкъ (сіе бо и самимъ нечестивѣйшимъ и Туркамъ общее есть), но каковъ внутрь; не каковъ предъ людьми, но каковъ предъ очами Божіими. Часто, и по большей части бываетъ, что человѣкъ предъ людьми мудръ, но предъ Богомъ юродъ; предъ людьми славенъ, но предъ Богомъ безславенъ; предъ людьми богатъ, но предъ Богомъ весьма нищъ; предъ людьми великъ, но предъ Богомъ ничто. И сей благополучія цвѣтокъ, который предъ людьми только является и сіяетъ, скоро увядаетъ; какъ найдетъ буря напасти, то изсохши и падаетъ, а конечно кончина его у всякаго отнимаетъ. Князь и вельможа до гроба только князь и вельможа; славный до гроба только славенъ; богатый до гроба только богатъ; мудрый до гроба только мудръ. Тако и все міра сего сокровище до гроба только человѣку служитъ; тогда бо все отъ него отступаетъ; тогда вси познаютъ себе нищими, убогими, бѣдными и окаянными, какъ и были; тогда князи, вельможи и господа дѣлаются какъ подлѣйшіи раби ихъ; тогда и богатіи познаютъ свою нищету и бываютъ убогшими нищихъ. Возносись убо и хвались симъ сокровищемъ, человѣче, когда хощешь. Я тебѣ истину говорю, что ты подлинно бѣденъ, какъ бы ты въ мірѣ славенъ ни былъ. Посмотри во гробы мертвыхъ, и увѣришися. Гдѣ князи, вельможи и господа, и гдѣ раби ихъ лежатъ, ей, познать не возможно! Гдѣ богатіи и нищіи, гдѣ мудрецы и простяки, гдѣ благородныи и худородныи, гдѣ славныи и безчестныи лежатъ, воистину не узнаешь! Тамо вси сравнялися и въ землю обратилися, какъ и были земля. На сіе тлѣніе смотря, хвалися счастіемъ, которое имѣешь, исчисляй свои имена и титулы, возносися богатствомъ и мудростію. Видишь тутъ, что вси равны суть. Какъ раждаются равными вси, такъ равными и умираютъ, кромѣ того, что одинъ спасается и въ вѣчную жизнь отходитъ, а другій погибаетъ и вѣчной предается смерти. \textit{Блажени умирающіи о Господѣ}\footnote{Апок.~14,~13.}. \textit{Помяни мя, Господи, во царствіи Твоемъ}\footnote{Лук.~23,~42.}. Былъ человѣкъ подлинно блаженъ, но потерялъ свое блаженство. Былъ подобенъ Богу, по образу Божію и по подобію сотворенъ, но \textit{приложися скотомъ несмысленнымъ, и уподобися имъ}\footnote{Пс.~48,~21.}. Былъ нетлѣненъ, но сотворился тлѣннымъ; былъ безсмертенъ, но сотворился смертнымъ: \textit{оброцы бо грѣха смерть}\footnote{Римл.~6,~23.}. Былъ премудръ, но сдѣлался безуменъ. Былъ богатъ, но сдѣлался нищь. Былъ красенъ, но сдѣлался безобразенъ. Былъ святъ и чистъ, но сдѣлался скверенъ и нечистъ. Былъ храмомъ и жилищемъ Святаго Духа, но сдѣлался обиталищемъ нечистыхъ духовъ. Былъ господинъ и обладатель, но сдѣлался рабомъ и плѣнникомъ. Былъ жителемъ пресладкаго рая, яко царь прекраснаго дворца, но изгнанъ въ міръ сей, яко въ ссылку. Толикаго блаженства лишился, и толикому неблагополучію подпалъ человѣкъ! Подпали и мы вси, сыны его. \textit{Тебѣ, Господи, правда, намъ стыдѣніе лица}\footnote{Дан.~9,~7.}. \textit{Согрѣшихомъ со отцы нашими, беззаконновахомъ, неправдовахомъ}\footnote{Пс.~105,~6.}. Что убо человѣку въ такомъ бѣдствіи думать и говорить, какъ только едино тое: \textit{согрѣшилъ, Господи, помилуй мя?} Какъ начало здравія есть познаніе и признаніе немощи: тако начало хрістіанскаго блаженства, познать и признать свою бѣдность и окаянство. И сіе бываетъ не отъ самаго познанія и признанія, но отъ того, что Богъ, яко милосердый и человѣколюбивый, на таковое познаніе и признаніе милосердо смотритъ и тому ниспосылаетъ Свою благодать, по писанному: \textit{Богъ смиреннымъ даетъ благодать}\footnote{1~Петр.~5,~5.}. Богъ хощетъ отъ насъ единаго того, чтобы мы познали и признали свою бѣдность и предъ Нимъ смирилися, и въ смиреніи ходили и жили, и тогда Самъ поведетъ насъ къ нашему блаженству. И сіе"=то есть, что Апостолъ написалъ: \textit{смиритеся убо подъ крѣпкую руку Божію, да вы вознесетъ во время: всю печаль вашу возведите нань, яко той печется о васъ}\footnote{1~Петр.~5,~6 и 7.}. Богъ нашъ есть Богъ всемогущій, все \textit{изъ ничего дѣлаетъ}. Видимъ, коль преславныя дѣла, небо и землю со исполненіемъ ихъ сотворилъ. Тако, кто познаетъ и признаетъ себе ничто, изъ того дѣлаетъ нѣчто, но доброе, то"=есть, мудраго, благочестиваго, добродѣтельнаго и святаго человѣка; Богъ бо не дѣлаетъ ничего, какъ только доброе, ибо и Самъ добръ. Но то наибольшая бѣдность и окаянство человѣческое, что высоко о себѣ мечтаетъ, хотя и подлинно ничто есть; думаетъ, что онъ все знаетъ, но ничего не знаетъ; думаетъ, что онъ разуменъ, но есть несмысленъ и безуменъ думаетъ, что онъ блаженъ, но есть подлинно окаяненъ и бѣденъ; думаетъ, что онъ богатъ, но онъ подлинно нищь и убогъ; думаетъ, что онъ добръ, но онъ подлинно золъ; думаетъ, что онъ святъ и праведенъ, но онъ подлинно грѣшенъ и оскверненъ; думаетъ, что онъ нѣчто есть, но онъ подлинно ничто. Что жъ убо съ такимъ веществомъ Богу всемогущему дѣлать, Который все изъ ничего дѣлаетъ? Оставляетъ его въ мнимомъ его блаженствѣ, и дѣлаетъ дѣло Свое на томъ, который себе уничтожаетъ, и Самъ того дѣлаетъ блаженнымъ, который познаетъ и признаетъ себе бѣднымъ и окаяннымъ. Сего ради глаголетъ Господь къ небесному Отцу: \textit{утаилъ еси сія} (таинства Евангелія) \textit{отъ премудрыхъ и разумныхъ, и открылъ еси та младенцемъ}\footnote{Мѳ.~11,~25.}. Книжники, фарисеи и законоучители думали о себѣ, что они мудры и разумны: \textit{еда и мы слѣпи есмы}\footnote{Іоан.~9,~40.}? Сего ради скрылся отъ нихъ свѣтъ и премудрость Божія. Апостоли, какъ младенцы, простыи и смиренныи были; но просвѣтились и умудрились отъ Самого Бога. Хрістіанине! Великая слѣпота и безуміе есть гордость; она не допущаетъ, узнать и признать свое окаянство, и тако остается человѣкъ и пребываетъ въ бѣдности своей и окаянствѣ. Великое просвѣщеніе и мудрость есть смиреніе, которое бываетъ отъ познанія своея бѣдности и окаянства. Смиренный человѣкъ неотмѣнно получить блаженство хрістіанское. Ибо Богъ не оставитъ его, яко смиренныхъ сердецъ любитель. \textit{Блажени нищіи духомъ, яко тѣхъ есть царство небесное}\footnote{Мѳ.~5,~3.}. Вотъ куды ведетъ смиреніе человѣка! Всякъ бо смиренный есть нищій духомъ. Познавай убо, хрістіанине, самого себе и свою бѣдность, и будеши смиренъ; а когда смиренъ будеши, то и блаженъ будеши. Познавай и признавай свою нищету предъ Богомъ, и будеши богатъ. Познавай и признавай свою слѣпоту предъ Богомъ, и открыются очи твои сердечныи. Познавай и признавай свое безуміе предъ Богомъ, и будеши мудръ. Познавай и признавай свои грѣхи предъ Богомъ, и будеши отъ Того оправданъ. Познавай и признавай свою нечистоту предъ Богомъ, и будеши чистъ и святъ. Познавай и признавай свою неисправность предъ Богомъ, и исправишися. Познавай и признавай свою немощь предъ Богомъ, и будеши въ здравіе приходить. Познавай и признавай свое заблужденіе предъ Богомъ, и будеши взысканъ. \textit{Заблудихъ, яко овча погибшее: взыщи раба Твоего}\footnote{Пс.~118,~176.}. Познавай наконецъ и признавай, что ты бѣденъ и окаяненъ, и будеши блаженъ. Все въ тебѣ благодать Божія будетъ исправлять, которая будетъ съ тобою, когда будеши смиренъ. Се есть, хрістіанине, алфавитъ хрістіанскаго ученія. Учись убо того, да съ успѣхомъ будешь учитися хрістіанской философіи. \textit{Всякъ возносяйся смирится: смиряяй же себе вознесется}\footnote{Лук.~18,~1.}.

\section{105. Учитель и ученики.}

Видимъ, что обучающіися художествамъ и наукамъ учатся тѣмъ отъ учителей. Тако хрістіанамъ, обучающимся хрістіанскому житію, должно учитися отъ Хріста. Ибо и хрістіане не иное что, какъ ученики Хрістовы. Въ началѣ бо хрістіане называлися \textit{учениками}, какъ видимъ въ Дѣяніяхъ святыхъ Апостолъ, а потомъ уже названы \textit{хрістіанами}\footnote{Дѣян.~11,~26.}. Учители, научая учениковъ своихъ художествамъ и наукамъ, показуютъ имъ примѣръ, и чему научаютъ, тое и сами дѣлаютъ, и такимъ образомъ руководствуютъ ихъ къ успѣху во ученіи ихъ. Сей бо добрый учитель есть, который и учитъ, и примѣръ показуетъ ученія своего. Тако Хрістосъ Господь, живучи на земли, училъ и творилъ; училъ добру и творилъ добро; и чему училъ, тое и творилъ, и тако Своимъ примѣромъ научалъ насъ святому и хрістіанскому житію. Откуду святый Лука Евангелистъ глаголетъ о Немъ: \textit{Іисусъ начатъ творити и учити}\footnote{1,~1.}. Хощеши ли убо, хрістіанине, учитися свято и по"=хрістіански жить, "--- а неотмѣнно нужно, когда хощеши Хрістовымъ быть, быть истиннымъ хрістіаниномъ, а не ложнымъ: то предложи предъ собою святое Евангеліе и непорочное житіе Хрістово, и учись отъ того. Якоже лице твое представляешь предъ зеркаломъ, и, усмотрѣвши на немъ пороки, стираешь и обмываешь ихъ, да не, въ народъ вышедши, посмѣянъ будеши: тако душѣ твоей да будетъ зеркало Евангеліе и святое житіе Хрістово, въ немъ написанное; и, изъ того усматривая пороки души твоея, заглаждай ихъ покаяніемъ, жалѣніемъ и слезами, да не съ ними явишися на второмъ Хрістовомъ пришествіи, и тако предъ праведнымъ Судіею Богомъ и святыми ангелами Его и человѣками постыдишися. Надобно бо неотмѣнно знаменіе имѣть, и съ тѣмъ тамо явитися и показать, что мы какъ имя Хрістово здѣ исповѣдывали, такъ и самымъ дѣломъ и житіемъ Хрістовы были. Надобно опасаться, чтобы и намъ не услышать: \textit{николиже знахъ васъ}\footnote{Мѳ.~7,~23.}. Все, что видишь въ тебѣ противно Евангелію и святому житію Хрістову, есть порокъ. Все убо заглаждай, что есть противно тому. Не послѣдуй волѣ плоти твоея, хотя она и ласкаетъ тебе; но послѣдуй волѣ Хрістовой, которая ведетъ тебе къ добру и блаженству твоему. "--- Тяжко"=де сіе и горько. \textit{Отвѣтъ}. 1)~Не тяжко, но легко носить иго Хрістово, по свидѣтельству Самого Хріста: \textit{иго Мое благо, и бремя Мое легко есть}\footnote{11,~30.}. Самъ разсуди, что тяжчае есть, и что легчае: мстить, или простить; гнѣваться или не гнѣваться; ненавидѣть, или любить; въ гордости, или смиреніи жить; богатства, чести и славы искать, или небрещи о томъ; нетерпѣливымъ или терпѣливымъ быть, и проч. Самая совѣсть убѣждаетъ признать, что далеко легчае простить, нежели мстить; кроткимъ быть, нежели гнѣваться; терпѣть, нежели не терпѣть; любить, нежели ненавидѣть; нерадѣть о земныхъ, нежели искать ихъ; горняя мудрствовать, нежели земная. Мстить немалаго труда требуетъ, а простить нѣтъ никакого труда. Любить легко и сладко, ненавидѣть тяжко и горько. Ненависть связуетъ и отягчаетъ сердце; но любовь развязуетъ, распространяетъ и облегчаетъ сердце, и тако человѣка радостнымъ и веселымъ дѣлаетъ. Смиреніе не боится падежа, яко на земли лежитъ, и по земли ходитъ: куды бо тому пасти, кто по земли ходитъ? Гордость высоко подымается и возносится, но всегда въ страхѣ и трепетѣ находится, чтобы не пасть; и хотя мятется и всѣми силами бережется падежа, однако падаетъ и сокрушается. Душа терпѣливая всегда въ покоѣ и тишинѣ: нетерпѣливая въ смущеніи, безпокойствѣ и мятежѣ; и чимъ въ большемъ нетерпѣніи находится, тѣмъ большее безпокойствіе и мятежъ имѣетъ: якоже терпѣливая душа, чимъ болѣе терпитъ, тѣмъ большій покой и тишину внутрь себе чувствуетъ. О! когда бы возможно тебѣ видѣть сердце того, кто иго Евангельское, иго Хрістово носитъ: увидѣлъ бы ты, что въ немъ рай радости и сладости есть, и царствіе Божіе въ немъ есть, хотя извнѣ и различно безпокойствуется, и, какъ благовонная роза терніемъ, бѣдами и напастьми окружается. Не можетъ бо въ томъ сердцѣ не быть утѣшеніе и радость истинная, въ которомъ царствіе Божіе чувствуется. Бѣдная душе моя, воздыхай, молись и тщись благое иго Хрістово носить, и будеши на земли сообразно небесному житію жить. Даждь мнѣ, Господи, иго твое доброе и бремя легкое носить, и всегда буду покоенъ, миренъ, радостенъ и веселъ, и на земли крупицы, отъ небесныя Твоея трапезы падающія, яко псы отъ трапезы господей своихъ, вкушати буду. \textit{Ей, Господи, помози мнѣ}\footnote{Мѳ.~15,~25 и 27.}! \textit{Услыши мя, Господи, яко блага милость Твоя!} 2)~Когда любишь Хріста, то долженъ любити святое Евангеліе Его и святое житіе Его; невозможно бо любящему лице не любить нравы его. Аще убо любишь Хріста, какъ неотмѣнно долженъ любить, яко благаго Господа и Любителя твоего, и Искупителя и Спасителя твоего, Который искупилъ тебе не златомъ и сребромъ, но Своею кровію отъ страшныя бѣды и погибели, "--- аще Его любишь, то люби и пресвятые и преблагіе нравы Его; аще нравы Его любишь, то ревнуй имъ и послѣдуй, и тщись ихъ, сколько возможно человѣку немощному, на душѣ своей изображать, да тако Ему сообразенъ будеши и здѣ, въ житіи семъ, и въ славѣ будущія жизни. Любителю съ любимымъ купно быть и ему послѣдовать сладко, якоже отъ него отлучитися горько. Якоже магнитъ желѣзо, тако хрістолюбивую душу нравы Хрістовы привлекаютъ къ себѣ; и якоже чувствующіи благовоніе болѣе и болѣе хотятъ обонять того: тако познающіи Хріста и святые и пресладкіе нравы Его болѣе и болѣе тщатся подражать имъ. \textit{Смѵрна и стакти и кассіа отъ ризъ Твоихъ}, Слове Божій и Дѣвы Сыне, Іисусе Хрісте\footnote{Пс.~44,~9.}. Возлюбленный хрістіанине! чимъ болѣе будемъ познавать Хріста, тѣмъ болѣе будемъ любить Его; чимъ болѣе будемъ любить Его, тѣмъ прилѣжнѣе будемъ послѣдовать пресвятымъ нравамъ Его. Чувствующему благовоніе невозможно не желать и не стараться болѣе и болѣе обонять того: тако ощущающему Хрістову любовь и смиреніе въ сердцѣ своемъ не возможно въ слѣдъ Его и святымъ нравамъ Его не слѣдовать. Познавай убо Хріста, и будешь любить Хріста и святое житіе Его, и будешь послѣдовать преблагимъ нравамъ Его. 3)~Видно"=де, что живущіи по правилу святаго Евангелія отъ всѣхъ ненавидимы суть: тяжко"=де сіе. \textit{Отвѣтъ}. Подлинно ненавидимы суть, какъ сіе и Самъ Хрістосъ предсказалъ къ утвержденію и утѣшенію нашему: \textit{будете ненавидими всѣми имене Моего ради}\footnote{Мѳ. 10~,22.}. И подлинно тяжко, и немалый крестъ есть отъ всѣхъ быть ненавидимымъ; но утѣшительно, что сіе ради Хріста бываетъ, ради Котораго все мы должны съ радостію терпѣть. Слыши утѣшительное слово Хрістово: \textit{аще міръ васъ ненавидитъ, вѣдите, яко Мене прежде васъ возненавидѣ. Аще отъ міра бысте были, міръ убо свое любилъ бы: якоже отъ міра нѣсте, но Азъ избрахъ вы отъ міра, сего ради ненавидитъ васъ міръ}. И паки: \textit{имѣяй заповѣди Моя, и соблюдаяй ихъ, той есть любяй Мя, а любяй Мя, возлюбленъ будетъ Отцемъ Моимъ; и Азъ возлюблю его, и явлюся ему Самъ}\footnote{Іоан.~15,~18 и 19; 14,~21.}. Что сихъ словесъ утѣшительнѣе можетъ быть хрістіанской душѣ? Ненавидитъ ее міръ, но Богъ любить ее; чуждается ея міръ, но Богъ избираетъ ее и пріемлетъ. Скажи, скажи пожалуй, что лучше "--- отъ Бога любимымъ быть, или отъ міра? Весь міръ въ соравненіи съ Богомъ ничто есть. Горька есть міра сего любовь, и къ большей горести ведетъ; кажется нѣчто извнѣ, какъ красное яблоко, но внутрь гнилости и горести исполнено. Сладка есть Божія любовь, утѣшительна и радостотворна, и къ вѣчной радости и сладости ведетъ. Лучше убо, и несравненно лучше, отъ единаго Бога любимымъ быть, нежели отъ всего міра. Я изволяю и избираю сіе. Пусть мене весь міръ ненавидитъ, когда хощетъ, и дѣлаетъ мнѣ, что хощетъ: только бы Богъ единъ любилъ и въ милости Своей содержалъ. Любовь Его и милость паче всего міра любви и милости мнѣ. Благо мнѣ, Господи, милость Твоя. \textit{Мнѣ же Богови прилѣплятися благо есть}. Не хощу я ничего ни на земли, ни на небеси, кромѣ Тебе единаго и любви Твоей. Съ Тобою и въ сѣни смертнѣй живъ буду и блаженъ: безъ Тебе и небо ничтоже есть. \textit{Призри на мя и помилуй мя, по суду любящихъ имя Твое}\footnote{Пс.~118,~132.}. 4)~Живущіи"=де по Евангелію въ презрѣніи находятся; стыдно"=де презрѣннымъ быть. \textit{Отвѣтъ}. Истинно и то, что благочестивыя души презрѣнны суть отъ міра. Но сказано и то отъ Хріста въ предосторожность нашу: \textit{иже аще постыдится Мене и Моихъ словесъ въ родѣ семъ прелюбодѣйнѣмъ и грѣшнѣмъ, и Сынъ человѣческій постыдится его, егда пріидетъ во славѣ Отца Своего со ангелы святыми}\footnote{Марк.~8,~38.}. Стыдно убо и слышать тогда отъ Хріста: \textit{не вѣмъ васъ, откуду есте}\footnote{Лук.~13,~27.}. Вы Мене не знали, и Я васъ не знаю; Мене и Моего смиренія стыдились, и Я васъ нынѣ стыжусь. Стыдно будетъ тогда слышать, хрістіанине, слово сіе, слышать предъ всѣмъ свѣтомъ, слышать отъ Хріста; стыдно, но и страшно, ибо оттуду вѣчная послѣдуетъ пагуба. Души благочестивыя подлинно презрѣнны отъ міра, но почтенны отъ Бога. \textit{Аще кто Мнѣ служитъ, почтитъ его Отецъ Мой}, глаголетъ Господь\footnote{Іоан.~12,~26.}. Презрѣнны отъ міра, но домашніи Хрістовы, \textit{и сожители святымъ и присніи Богу}\footnote{Еф.~2,~19.}. Презрѣнны отъ міра, но суть друзи Хріста Господа и Царя славы. \textit{Вы друзи Мои есте, аще творите, елика Азъ заповѣдаю вамъ}\footnote{Іоан.~15,~14.}. Презрѣнны отъ міра, но суть живіи пренебесныя Главы \textit{Хріста уды}\footnote{Еф.~5,~30.}. Презрѣнны отъ міра, но суть \textit{жилище} и домъ \textit{Святаго Духа}\footnote{Римл.~8,~9; 1~Кор.~3,~16; 6,~19.}. Презрѣнны отъ міра, но суть \textit{сынове Бога} небеснаго\footnote{Римл.~8,~14; Гал.~3,~26.}. Презрѣнны отъ міра, но общеніе имѣютъ со Отцемъ и Сыномъ Его Іисусомъ Хрістомъ. \textit{Общеніе, наше со Отцемъ и съ Сыномъ Его Іисусомъ Хрістомъ}\footnote{1~Іоан.~1,~3.}. Лишаются славы и чести міра сего, но сподобляются \textit{славы} Божія\footnote{2~Кор.~4,~17; Кол.~3,~4.}. Что, человѣче, стыдно ли бы тебѣ отъ міра презрѣннымъ быть, ежели бы ты сіе достоинство и славу отъ Бога имѣлъ? Что сего достоинства достойнѣе, что сего благородія благороднѣе, что сего великолѣпія великолѣпнѣе, что сея славы славнѣе, какъ быть человѣку домашнимъ Божіимъ, быть другомъ Хрістовымъ, быть домомъ и жилищемъ Божіимъ, быть сыномъ Бога вышняго, быть живымъ удомъ Хрістовымъ, общеніе имѣть со Отцемъ и съ Сыномъ Его Іисусомъ Хрістомъ? Что сего славнѣе и ужаснѣе можетъ быть? Сіе всю славу міра сего несравненно превосходитъ и есть предъ тѣмъ, какъ мертвечина и гной. Человѣче! не смотри, что внѣ человѣкъ, но что внутрь; не каковъ предъ міромъ, но каковъ предъ Богомъ. Истинныи хрістіане презрѣнны предъ міромъ, но внутрь великую славу и сокровище имѣютъ. Сіе сокровище рѣдко и весьма трудно сыскивается подъ коронами, подъ титулами и именами міра сего, подъ порфирою и вѵссономъ, между богатыми и красными стѣнами, но по большей части подъ рубищемъ и въ хижинахъ и вертепахъ; тутъ оно болѣе мѣсто свое имѣетъ. Истинныи хрістіане суть, какъ злато закопченное, но отъ людей незнаемое и попираемое. Слава человѣколюбію Божію, слава благости Его, слава милости и щедротамъ Его, что толикія чести и славы бѣднаго человѣка удостоилъ! Сіе все Спаситель нашъ и Господь заслужилъ намъ. Что убо, хрістіанине, хощеши ли учитися Евангельской философіи, и быть ученикомъ Хрістовымъ, или не хощешь отъ міра презрѣннымъ быть, и не учитися? 5)~Хощу"=де, но труденъ подвигъ есть распинать плоть со страстьми и похотьми, чего требуетъ евангельская и хрістіанская философія. "--- И то истинно есть, что труденъ, но нуженъ и славенъ. Съ немалымъ трудомъ и ученики учатся въ школахъ міра сего; они учатся по большей части для тѣлесной и временной корысти: мы учимся ради душевной и вѣчной пользы, и не отъ человѣка, но отъ Хріста Господа славы, отъ Котораго учитися и пресладко и преславно. Съ немалымъ трудомъ и воины на брани міра сего подвизаются; они подвизаются для земной и временной чести: мы ради небесной и вѣчной и несравненно лучшей подвизаемся. Съ немалымъ трудомъ и купцы по различнымъ странамъ и городамъ странствуютъ и скитаются; они труждаются ради тлѣннаго сокровища: мы же ради нетлѣннаго и духовнаго. Съ немалымъ трудомъ и земледѣльцы дѣлаютъ землю и сѣютъ; они трудятся, да тѣлесные и временные плоды соберутъ: мы трудимся, да сотворимъ плодъ духовный. \textit{Плодъ же духовный есть любы, радость, миръ, долготерпѣніе, благость, милосердіе, вѣра, кротость, воздержаніе}\footnote{Гал.~5,~22.}. Ради сихъ пресладкихъ плодовъ должно намъ, хрістіанине, учитися, трудитися и подвизатися. 6)~Не возможно"=де самого себе побѣждать, что требуется отъ ученика евангельской философіи. "--- И то истинно есть, но съ помощію Хрістовою возможно. Онъ какъ повелѣлъ намъ учитися отъ Себе: \textit{научитеся отъ Мене, яко кротокъ есмь и смиренъ сердцемъ}\footnote{Мѳ.~11,~29.}: такъ и обѣщалъ намъ подать помощь просящимъ. \textit{Просите, и дастся вамъ; ищите и обрящете; толцыте, и отверзется вамъ. Всякъ бо просяй пріемлетъ, и ищай обрѣтаетъ, и толкущему отверзется}\footnote{7,~7 и 8.}. Ты только тщись, пекись, трудись, подвизайся, невѣжество и немощь свою предъ Нимъ признавай, и со усердіемъ молись Ему, и подастся тебѣ помощь. Тако съ помощію Его возможно тебѣ будетъ, что тебѣ самому и безъ Его помощи не возможно. \textit{Все возможешь} и ты \textit{о укрѣпляющемъ тя Іисусѣ Хрістѣ}\footnote{Филип.~4,~13.}. \textit{Сила бо Его въ немощехъ} нашихъ \textit{совершается}, когда немощи наши признаемъ\footnote{2~Кор.~12,~9.}. \textit{Хрістосъ пострада по насъ, намъ оставль образъ, да послѣдуемъ стопамъ Его}, и проч.\footnote{1~Петр.~2,~21.} "--- \textit{Образъ дахъ вамъ, да якоже Азъ сотворилъ вамъ, и вы творите такожде}\footnote{Іоан.~13,~15.}. \textit{Иже Мнѣ служитъ, Мнѣ да послѣдствуетъ: и идѣже есмь Азъ, ту и слуга Мой будетъ}\footnote{12,~26.}. \textit{И иже не пріиметъ креста своего, и въ слѣдъ Мене грядетъ, нѣсть Мене достоинъ}\footnote{Мѳ.~10,~38.}. \textit{Побѣждающему дамъ ясти отъ манны сокровенныя, и дамъ ему камень бѣлъ, и на камени имя ново написано, егоже никтоже вѣсть, токмо пріемляй}\footnote{Апок.~2,~17.}. \textit{Настави мя, Господи, на путь Твой, и пойду во истинѣ Твоей: да возвеселится сердце мое боятися имене Твоего}\footnote{Пс.~85,~11.}.

\subsection{О томжде.}

Когда отроки и юноши вдаются въ наученіе художествъ и наукъ, то боятся и ужасаются къ тому приступать, и сперва немалую чувствуютъ горесть; но когда пріобыкнутъ, и нѣсколько пользы отъ ученія художествъ и наукъ происходящія познаютъ, то немалую чувствуютъ сладость внутрь себе отъ ученія, и чимъ болѣе учатся и въ разумѣ просвѣщаются наукою, тѣмъ болѣе радуются и веселятся. Откуду обще отъ всѣхъ говорится: \textit{корень ученія горекъ, но плоды его сладки}. Къ тому бо всякое художество и наука ведетъ. Тако приступающимъ къ наукѣ евангельской философіи и той учитися начинающимъ, сперва страшно кажется и немалую чувствуютъ горесть; плоти бо духовное хрістіанское житіе горько, и иго Хрістово тяжко; но когда начнутъ учитися, и повидятъ въ своемъ ученіи духовную пользу, тогда радуются, и болѣе и болѣе стараются учитися, и чимъ болѣе обучаются, тѣмъ большую внутрь себе чувствуютъ сладость. Тако всякому плотскому человѣку сначала Хрістосъ страшенъ и горекъ, но потомъ любезенъ и сладокъ. Иго сатанинское сначала пріятно и сладко, но потомъ гнусно и горько, какъ и подлинно есть: иго Хрістово сначала страшно и горько, но потомъ любезно и сладко. Пріобучившемуся хрістіанской философіи добро творить радостно и сладко, а не учившемуся трудно и скучно. Представь себѣ, хрістіанине, двухъ человѣковъ, обучившагося и необучившагося хрістіанской философіи, и увидишь великое между ими различіе. Тотъ и мудрствуетъ и говоритъ духовная, сей плотская. У того и мысли и рѣчи о Бозѣ, словѣ Божіи и вѣчномъ животѣ: у сего о суетѣ міра и роскоши. Тотъ думаетъ и печется, чтобы вѣчный животъ получить: сей, какъ бы богатство собрать, честь и славу въ мірѣ семъ сыскать. Тому равно жить, какъ въ богатыхъ и красныхъ покояхъ, такъ и въ худой хижинѣ: сему не хочется, какъ только въ добромъ и красномъ домѣ жить. Тому равно какъ за богатою, такъ и за скудною трапезою сидѣть: сей о лучшей всегда замышляетъ. Того равно какъ пригожее, такъ смиренное платье одѣваетъ: сей не хочетъ, какъ только въ пригожемъ одѣяніи ходить, и съ богачемъ евангельскимъ въ порфиру и вѵссонъ одѣваться. Тотъ, имѣетъ ли богатство міра сего, или не имѣетъ, нерадитъ о томъ, и тѣмъ, что имѣетъ, довольствуется: сей печется о немъ, и когда не получаетъ, скорбитъ и сѣтуетъ. Тотъ, когда имѣетъ богатство, на нищихъ и странныхъ и прочія богоугодныя дѣла раздѣляетъ: сей на прихоти и роскоши своя. Тотъ бѣдному человѣку, какъ своему уду страждущему, состраждетъ: сей думаетъ и говоритъ: что мнѣ до его нужды? вѣдь"=де онъ мнѣ не свой, и проч. Тотъ укоренъ или инако какъ обижденъ молчитъ, или кротко обидѣвшему отвѣщаетъ: сей обижденный гнѣвомъ распаляется, и обиду за обиду, и досажденіе за досажденіе, и зло за зло воздаетъ, и прочимъ образомъ тщится отмстить, и прочая. Видишь, хрістіанине, изображеніе обучающагося и неучащагося хрістіанской философіи, и сколько единъ отъ другаго разнствуетъ, видишь. Пристань убо къ симъ ученикамъ, которыи евангельской и хрістіанской философіи обучаются, и учись тоя съ ними отъ Хріста, да воистинну хрістіанинъ и Хрістовъ ученикъ будеши; и хотя страшно и горестно тебѣ кажется начало того ученія, убѣждай и понуждай себе, какъ осла лѣниваго. Сіе тщаніе твое видя, Господь будетъ помогать тебѣ, и день отъ дне святое иго Его легчайшимъ и сладчайшимъ будетъ казаться.

\section{106. Много я тебѣ долженъ.}

Слышимъ, что люди людямъ говорятъ: \textit{много я тебѣ долженъ}. И сіе слово показуетъ благодарность сердца. Говорятъ тое люди тѣмъ, отъ которыхъ не мало добра, благодѣянія и помощи туне получили. Правильно оно и отъ благодарнаго сердца говорится, когда не лицемѣрно, но отъ искренности сердечной говорится. Ибо и люди людемъ, своимъ благодѣтелямъ, должны быть благодарны. Хрістіанине! намъ наипаче прилично слово сіе со смиреніемъ и искренностію сердца къ Богу нашему говорить и исповѣдываться: \textit{много я Тебѣ, Господи, долженъ}; и исповѣдываться на всякъ день, часъ и минуту должно, яко непрестанно и туне отъ руки Господни благая получаемъ. \textit{Благословлю Господа на всякое время, выну хвала Его во устѣхъ моихъ}\footnote{Пс.~33,~2.}. Не было мене, и се есмь и живу, какъ и прочіи люди. Ты, Господи, благоволилъ мнѣ быть, и между дѣлами рукъ Твоихъ считаться, и видѣть дѣла рукъ Твоихъ "--- небо и землю съ исполненіемъ ихъ, и отъ великихъ и преславныхъ дѣлъ Твоихъ Тебе великаго и преславнаго Создателя познавать, и благими Твоими довольствоваться и утѣшаться. \textit{Руцѣ твои сотвористѣ мя, и создастѣ мя}\footnote{118,~73.}. \textit{Много я Тебѣ за сіе долженъ}, и не могу ничимъ воздать. Сотворилъ Ты мене не бездушною тварію, не скотомъ, не птицею, не рыбою, ни инымъ какимъ безсловеснымъ животнымъ, которыя всѣ суть и живутъ, но блаженства своего не разумѣютъ; но сотворилъ разумнымъ человѣкомъ, который могу познать и разумѣть, что я отъ Тебе начало бытія и житія своего имѣю; я Твое созданіе, Ты мой Создатель: \textit{много я Тебѣ за сіе долженъ}, и никакъ не могу воздать. Тако сотворенный отъ Тебе, быть и жить безъ Тебе и благихъ Твоихъ не могу. Рука Твоя, Господи, содержитъ мене, и благая Твоя подаетъ мнѣ. Не могу я безъ свѣта жить и обращаться: свѣтила Твоя, солнце, луна и звѣзды свѣтятъ мнѣ. Не могу я безъ огня жить: огнь Твой согрѣваетъ мене, и варитъ пищу мою. Не могу я безъ воздуха цѣлъ и живъ быть: воздухъ Твой оживляетъ мене, и сохраняетъ жизнь мою. Не могу я безъ пищи быть: щедрая Твоя рука, Господи, подаетъ мнѣ пищу. Не могу я безъ питія быть: благость Твоя, Господи, произвела ради мене источники, рѣки и озера, и тѣми прохлаждаюся и омываюся. Не могу я, обнаженный одежды боготканныя, безъ одѣянія быть: отъ щедрой Твоей десницы и тое получаю. Не могу я безъ дома быть: подаешь мнѣ и домъ ко упокоенію немощнаго и многобѣднаго тѣла моего. Не могу я быть безъ скотовъ: скоты Твои служатъ и работаютъ мнѣ. Сія и прочая безчисленная благая, какъ прочіи человѣцы братія моя, такъ и я получаю отъ Тебе: \textit{много я Тебѣ за сіе долженъ}. Но \textit{что воздамъ Господеви о всѣхъ, яже воздаде ми}\footnote{Пс.~115,~3.}? Что бы я былъ, когда бы Ты хотя едино отъ сихъ благъ отнялъ отъ мене? Что бы мнѣ пользовали очи мои, когда бы свѣтъ Твой мнѣ не свѣтилъ? Неотмѣнно блудилъ бы, какъ слѣпый. Какъ бы я могъ жить, когда бы не подавалъ мнѣ пищу Твою? Безъ воздуха и малѣйшей минуты жить не могу. Тако всякое созданіе Твое весьма добро, и мнѣ бѣдному служитъ. Но сія благая получая отъ Тебе, надѣюсь будущихъ и лучшихъ благъ, по неложному обѣщанію Твоему, и отъ видимыхъ къ невидимымъ, и отъ настоящихъ къ будущимъ, и отъ временныхъ къ вѣчнымъ, и отъ земныхъ къ небеснымъ стремится духъ мой. Аще бо во изгнаніи и ссылкѣ толикая благая подаеши намъ, коликая подаси во отечествѣ и дому Твоемъ! И аще знающимъ Тебе и незнающимъ, почитающимъ и непочитающимъ, другамъ и врагамъ твоимъ тако щедро отверзаеши руку Твою: коликихъ уже благъ въ вѣчной жизни сподобятся любящіи Тя! \textit{Ихже око не видѣ, и ухо не слыша, и на сердце человѣку не взыдоша, яже уготова Богъ любящимъ Его}\footnote{1~Кор.~2,~9.}. \textit{Буди убо, Господи, милость Твоя на насъ, якоже уповахомъ на Тя}\footnote{Пс.~32,~22.}. Слѣпъ я, какъ и прочіи, безъ благодати и просвѣщенія Твоего; грѣхъ мой мене ослѣпилъ. Сего ради и въ семъ благость Твоя, какъ о всѣхъ, такъ и о мнѣ недостойномъ человѣколюбно промыслила. Къ просвѣщенію какъ всѣхъ, такъ и мене низпослалъ Ты намъ слово Твое святое чрезъ избранныхъ рабовъ и служителей Твоихъ. Сіе мнѣ, какъ свѣща во тьмѣ сѣдящимъ, сіяетъ и показуетъ вредъ и пользу, зло и добро, грѣхъ и добродѣтель, ложь и истину, неугодное Тебѣ и угодное, невѣріе и вѣру, и тако прогоняетъ мою слѣпоту и просвѣщаетъ разумъ мой. \textit{Свѣтильникъ ногама моима законъ Твой, и свѣтъ стезямъ моимъ}\footnote{118,~105.}. \textit{Много я за сіе Тебѣ, Господи, долженъ}, и ничимъ не могу воздать. Отъ того познаю Тебе Бога и Создателя моего. Въ томъ я вижу, что Ты преславная и чудная дѣла сотворилъ и твориши, каковыхъ рука человѣческая и никакое созданіе сотворить не можетъ; и отъ тѣхъ познаю, что Ты тойжде еси, дивно и славно творяй чудеса въ созданномъ отъ Тебе мірѣ, который міръ изъ ничего создалъ, и отъ всего того познаю всемогущую силу Твою. Какъ бо вещи изъ ничего создать, такъ созданныя перемѣнить всемогущія Твоея силы есть. Рѣки преложить въ кровь, воду изъ камене источить, и огнь въ росу преложить, слѣпа отъ рожденія просвѣтить, мертваго воскресить, и прочая симъ подобная дѣла дѣлать, кому иному, какъ только всесильной десницѣ Твоей свойственно есть? Сію всемогущую Твою силу разсуждая, убѣждаюсь трепетать и смиряться предъ Тобою, \textit{смиряться подъ крѣпкую Твою руку}\footnote{1~Петр.~5,~6.}. Како бо не могу трепетать Того и смиряться предъ Тѣмъ, у Котораго въ руцѣ вси концы земли и я? И смерть и животъ мой въ руцѣ Его. Боже преблагій и милосердый! пощади мене бѣднаго грѣшника. "--- Въ словѣ Твоемъ вижу я, что всемогущая десница Твоя защищаетъ, сохраняетъ и спасаетъ боящихся, любящихъ и почитающихъ Тебе; своевольныхъ же, небоящихся и нечествующихъ къ Тебѣ смиряетъ, наказуетъ и казнитъ; награждаетъ добродѣтель, казнитъ грѣхъ. И отъ того познаю правду Твою, которая всѣмъ воздастъ по дѣламъ ихъ, и боюсь суда Твоего; и тако убѣждаюсь къ покаянію за прежде бывшіе грѣхи, и понуждаюсь вопить къ Тебѣ изъ глубины сердца моего: \textit{Боже, милостивъ буди мнѣ грѣшнику}\footnote{Лук.~18,~13.}. \textit{Не вниди въ судъ съ рабомъ Твоимъ, яко не оправдится предъ Тобою всякъ живый}\footnote{Пс.~142,~2.}. \textit{Аще бо беззаконія назриши, Господи, Господи, кто постоитъ?}\footnote{129,~3.} Сіе предостерегаетъ мене и отъ прочихъ грѣховъ, къ которымъ слабость моя мене клонитъ. "--- Въ словѣ Твоемъ вижду я, что Ты всѣхъ, отвратившихся отъ Тебе и нечествующихъ призываеши къ Себѣ; и обратившихся и съ покаяніемъ приходящихъ, и милости отъ тебе ищущихъ и просящихъ, яко Отецъ чадолюбивый заблудшихъ сыновъ, человѣколюбно пріемлеши, и милость Свою имъ являеши, и грѣховъ ихъ и беззаконій не поминаеши, и къ числу любящихъ и почитающихъ Тебе присовокупляеши. Отъ сего познаю благость и безприкладное Твое милосердіе къ бѣднымъ грѣшникамъ, а тако не отчаяваюся и самъ за грѣхи мои, но съ надеждою къ Тебѣ, благому и милосердому Богу, прибѣгаю, и милости прошу. \textit{Помилуй мя, Боже, по велицѣй милости Твоей, и по множеству щедротъ Твоихъ очисти беззаконіе мое}\footnote{Пс.~50,~3.}. Благость Твою показуютъ вси созданія Твоя, вездѣ проповѣдуетъ слово Твое, дознаютъ вся животная Твоя, удивляются ангели и человѣцы, лобызаютъ грѣшники кающіися, чувствуютъ праведники, не лишаются и нечувственныи грѣшники, яко благость Твоя ихъ \textit{долготерпитъ, и на покаяніе ведетъ}\footnote{Римл.~2,~4.}. Чувствовалъ, чувствую и лобызаю и я, бѣдный грѣшникъ, благость Твою, и съ пророкомъ Твоимъ взываю: \textit{вкусите и видите, яко благъ Господь}\footnote{Пс.~33,~9.}. И сколько разъ ощущаю благость Твою въ сердцѣ моемъ, столько разъ возбуждается и возжигается сердце мое къ любви Твоей. Слава благости Твоей, слава щедротамъ Твоимъ, слава милосердію Твоему, слава долготерпѣнію Твоему, слава человѣколюбію Твоему! \textit{Возлюблю Тя, Господи, крѣпосте моя. Господь утвержденіе мое, и прибѣжище мое, и избавитель мой, Богъ мой, помощникъ мой, и уповаю на Него; защититель мой, и рогъ спасенія моего, и заступникъ мой}\footnote{17,~2 и 3.}. "--- Въ словѣ Твоемъ вижу я, что все, что ни открываеши намъ, точно тако есть; и что предсказуеши, тое сбывается; и что обѣщалъ и обѣщаеши, тое все исполняеши, яко вѣренъ Господь во всѣхъ словесѣхъ Своихъ; и вижу, что небо и земля мимо идутъ, словеса же Твоя не имутъ прейти. И оттуду познаю истину Твою, и утверждаюся въ вѣрѣ моей, и безъ сомнѣнія вѣрую святому и милостивому обѣщанію Твоему, и въ надеждѣ моей непоколебимъ бываю. Милость Твоя, Господи, и истина Твоя оживляетъ мя, содержитъ мя, сохраняетъ и укрѣпляетъ мя, и ведетъ мя къ будущимъ и вѣчнымъ обѣщаннымъ благимъ Твоимъ. Знаю я, что храмина тѣла моего разорится, и тѣло мое, яко земля, землѣ предастся; но милость Твоя увѣряетъ мене, что тоежде самое тѣло, въ которомъ нынѣ живетъ душа моя, силою и всемогущимъ гласомъ Твоимъ востанетъ, и благодатію Твоею въ лучшій и краснѣйшій видъ преобразится. \textit{Сѣется бо не въ честь, востаетъ въ славѣ; сѣется въ немощи, востаетъ въ силѣ; сѣется тѣло душевное, востаетъ тѣло духовное... Подобаетъ бо тлѣнному сему облещися въ нетлѣніе, и мертвенному сему облещися въ безсмертіе}\footnote{1~Кор.~15,~43,~44 и 53.}. И тако съ Церковію Твоею святою \textit{чаю воскресенія мертвыхъ, и жизни будущаго вѣка, аминь}. Вижу я въ словѣ Твоемъ святомъ, что Ты, какъ начала, такъ и конца не имѣешь. Вси вещи, созданныя отъ Тебе, какъ начались, такъ могутъ и скончаться, и какъ не были, такъ могутъ и не быть. Что пребываетъ, то зависитъ отъ святаго хотѣнія и всемогущія силы Твоея. Ты же, какъ безъ начала пребываеши, такъ безъ конца будеши и не можешь не быть. \textit{Прежде даже горамъ не быти и создатися земли и вселеннѣй, и отъ вѣка и до вѣка Ты еси}\footnote{Пс.~89,~3.}. \textit{Въ началѣ Ты, Господи, землю основалъ еси, и дѣла руку Твоею суть небеса. Та погибнутъ, Ты же пребываеши; и вся, яко риза, обетшаютъ, и яко одежду свіеши я, и измѣнятся. Ты же Тойжде еси, и лѣта Твоя не оскудѣютъ}\footnote{101,~26--28.}. И отъ сего вижу, что Ты единъ имѣешь безсмертіе, во свѣтѣ живый неприступнѣмъ; единъ имѣешь присносущіе, единъ имѣешь житіе непремѣняемое, и еси вѣчность присносущная, и присносущіе вѣчное, вѣчный животъ, безъ Котораго жить и живы быть не можемъ, у Котораго источникъ живота. Съ Тобою быть есть животъ: безъ Тебе быть есть явная смерть. Съ Тобою быть есть блаженство: безъ Тебе быть есть окаянство. Съ Тобою быть есть утѣшеніе, радость и сладость: безъ Тебе быть есть скорбь и горесть. Съ Тобою живущія истинно живутъ: безъ Тебе живущіи живи умерли, предъ людьми живутъ, но предъ Твоими очами мертвы суть, въ смерти пребываютъ. Надобно бо мертвымъ быть, кто отъ живота удаляется, какъ надобно во тьмѣ быть, кто отъ свѣта удаляется. Сіе разсуждая, трепещу и боюся отъ Тебе, живота моего, удалитися; тщуся и стараюся при Тебѣ быть, недостойное Твое созданіе. \textit{Не отвержи мене}, Господи, \textit{отъ лица Твоего, и Духа Твоего Святаго не отъими отъ мене. Воздаждь ми радость спасенія Твоего, и духомъ владычнимъ утверди мя}\footnote{50,~14.}. Ты мой Создатель, Ты мой и животъ, Ты моя крѣпость, Господи, Ты моя и сила, Ты мой Богъ, Ты мое радованіе. \textit{Что бо ми есть на небеси? и отъ Тебе что восхотѣхъ на земли? Исчезе сердце мое, и плоть моя; Боже сердца моего, и часть моя Боже во вѣкъ. Яко се удаляющіи себе отъ Тебе погибнутъ: потребилъ еси всякаго любодѣющаго отъ Тебе. Мнѣ же прилѣплятися Богови благо есть, полагати на Господа упованіе мое}\footnote{Пс.~72,~25--28.}. Прилѣжно, весьма прилѣжно внимай, душе моя, слову тому; се \textit{удаляющіи себе отъ Тебе погибнутъ}. Берегись удалиться живота, да не въ смерти будеши; берегись удалиться свѣта, да не во тьмѣ пребудеши, грѣхъ отъ Бога удаляетъ и разлучаетъ. Согрѣшили ангели, и удалились и погибли. Согрѣшили прародители въ раи, и удалились отъ Бога и умерли. Согрѣшаютъ и нынѣ люди и удаляются отъ Бога, и въ смерти пребываютъ. Берегись душе моя, да не тоежде постраждеши и ты. Прилѣпляйся убо животу, да въ Немъ и съ Нимъ живеши. Держись свѣта, да имаши свѣтъ животный. Прилѣпляйся блаженству, да блаженна будеши. Прилѣпляйся благому Богу, да блага будеши. Держись радости и сладости существенныя, да въ утѣшеніи, радости и веселіи пребудеши. Умри грѣху и суетѣ міра сего, да Богу живеши. \textit{Мнѣ же прилѣплятися Богови благо есть}. "--- Вижу я въ словѣ Твоемъ святомъ, что Ты имѣешь святость безприкладную и непостижимую, которой избранніи и святіи Твои ангели удивляются и ужасаются, такъ что съ трепетомъ и ужасомъ убѣждаются взывать: \textit{святъ, святъ, святъ Господь Саваоѳъ}\footnote{Ис.~6,~3.}. И знаю, что съ Тобою быть есть животъ, безъ Тебе быть есть смерть. Страшно отъ Тебе отлучиться, но невозможно опороченному и оскверненному быть съ Тобою. Надобно и тому во свѣтѣ быть и ходить, кто съ Тобою "--- Свѣтомъ животнымъ, въ Которомъ \textit{тьмы нѣтъ ни единыя, общеніе} хощетъ \textit{имѣть}\footnote{Іоан.~1,~5 и 6.}. Сіе разсужденіе учитъ мене бояться всякаго грѣха и удаляться отъ него, и быть въ покаяніи и сокрушеніи за преждебывшіе грѣхи, и со смиреніемъ и покаяніемъ молиться Тебѣ. \textit{Отврати лице Твое отъ грѣхъ моихъ и вся беззаконія моя очисти}\footnote{Пс.~50,~11.}. Отъими, Господи, беззаконія моя отъ мене, и буду Твой, и буду съ Тобою. \textit{Мнѣ бо прилѣплятися Богови благо есть}. "--- Вижу я въ словѣ Твоемъ святомъ, что Ты вездѣ и на всякомъ мѣстѣ еси, и нѣтъ таковаго мѣста, гдѣ бы Ты существенно не присутствовалъ, и гдѣ я ни хожу и обращаюся, предъ Тобою хожу и обращаюся; и что ни дѣлаю, говорю, мыслю, начинаю, предъ Тобою дѣлаю, говорю, мыслю и начинаю; и Ты все видишь и знаешь, и далеко лучше видишь и знаешь, нежели я самъ знаю; и все, что ни дѣлаю, говорю, помышляю и начинаю, въ книзѣ Своей записываешь, и воздаси по дѣломъ, мыслямъ, словамъ и начинаніямъ моимъ. Какъ убо вездѣ еси, такъ и все видишь и знаешь. Почему нигдѣ и ни въ чемъ сокрыться отъ Тебе не могу. \textit{Вси путіе мои предъ Тобою, Господи}\footnote{118,~168.}. \textit{Камо пойду отъ Духа Твоего? и отъ лица Твоего камо бѣжу? Аще взыду на небо, Ты тамо еси, аще сниду во адъ, тамо еси; аще возму крилѣ мои рано, и вселюся въ послѣднихъ моря, и тамо бо рука Твоя наставитъ мя, и удержитъ мя десница Твоя. И рѣхъ; еда тьма поперетъ мя, и нощь просвѣщеніе въ сладости моей? Яко тьма не помрачится отъ Тебе, и нощь яко день просвѣтится: яко тьма ея, тако и свѣтъ ея}\footnote{138,~7--12.}. Сіе поученіе и размышленіе научаетъ мене всегда и вездѣ бояться Тебе и трепетать, со страхомъ и опасеніемъ жить и обращаться, дѣлать, говорить, мыслить и начинать такъ, какъ дѣти предъ отцемъ своимъ, раби предъ господиномъ своимъ, подданныи предъ царемъ своимъ ходятъ и обращаются, яко все предъ Тобою бываетъ, и все предъ всевидящимъ Твоимъ окомъ явно и откровенно есть. Отъ сего научаюсь вездѣ и на всякомъ мѣстѣ колѣна моя преклонять предъ Тобою, и со смиреніемъ призывать Тебе, и милости искать у Тебе, и надежду мою полагать на Тебе: яко \textit{аще и пойду посредѣ сѣни смертныя, не убоюся зла, яко Ты со мною еси}\footnote{Пс.~22,~4.}. \textit{Помощникъ мой еси, Тебе пою: яко Богъ заступникъ мой еси, Боже мой, милость моя}\footnote{58,~18.}. "--- Вижу я въ словѣ Твоемъ святомъ, что Ты Духъ невещественный, Котораго ни окомъ видѣть, ни ухомъ слышать, ни руками осязать и никакимъ чувствомъ чувствовать невозможно, но единымъ окомъ вѣры и умомъ видишися и познаешися. \textit{Духъ есть Богъ}\footnote{Іоан.~4,~24.}. Отъ сего научаюся почитать Тебе не веществомъ, но духомъ (Духъ бо духовно почитается), и \textit{кланяться Тебѣ духомъ и истиною}. Господи, научи мя почитать Тебе и покланяться Тебѣ духомъ и истиною. "--- Вижу я въ словѣ Твоемъ, что Ты тамо находишь способъ спасенія намъ, гдѣ не видится, и все порядочно и къ доброму концу ведеши; и отъ сего познаваю непостижимую премудрость Твою и удивляюся той; и вижу, что Ты, какъ вся премудростію сотворилъ еси, такъ вся премудро и управлявши. И отъ сего научаюся во всемъ, въ благополучіи и неблагополучіи моемъ, сдаваться на премудрый и дивный Твой промыслъ. Все, что ни посылается мнѣ отъ Тебе, на добро мое и блаженство мое посылается, яко отъ источника всѣхъ благъ посылается. Возносиши ли мене "--- благо мнѣ. Смиряеши ли мене "--- благо мнѣ. Радостотвориши ли мене "--- благо мнѣ. Опечаляеши ли мене "--- благо мнѣ. Слава Тебѣ, Боже, о всемъ! Все бо твориши, да блаженнымъ мене содѣлаеши. \textit{Благословлю Господа на всякое время} (печальное и радостное). "--- Тако отъ слова Твоего святаго научаюсь познавать Тебе, Создателя и Бога моего, и познавая почитать, и отъ того великую и неизреченную душѣ моей получаю пользу. Что бы я былъ, аще бы слово Твое святое не руководствовало мене? Неотмѣнно бы ходилъ и блудилъ, какъ слѣпый и какъ ходящіи во тьмѣ, яко вси языцы, не имѣющія свѣтильника Твоего сего и тому не внимающіи, ходятъ и заблуждаютъ, и Тебе, Создателя своего, не знаютъ, и Твоими благими, которыя имъ подаеши, довольствуяся, Тебе Благодѣтеля не разумѣютъ, но почитаютъ тварь вмѣсто творца, и тую честь, которую Тебѣ, Создателю и Благодѣтелю своему, должны воздавать, твари Твоей отдаютъ. Слово Твое святое отъ сего душепагубнаго заблужденія отвращаетъ мене, и на Тебе, Создателя моего, указуетъ, и къ Тебѣ ведетъ и Тебе познавать, и единаго почитать яко Бога и Творца моего, и Тебѣ единому покланяться и служить, и Твою Тебѣ честь отдавать научаетъ. \textit{Много я тебѣ за сей небесный Твой даръ долженъ, Господи}, и никакъ и ничимъ не могу воздать. Благодарю Тя, человѣколюбче, яко сей божественный свѣтильникъ Твой возжеглъ еси, и какъ предъ всѣми, такъ и предо мною бѣднымъ поставилъ еси его. На сей взирая, озаряюсь и просвѣщаюсь и Тебе, Создателя и Бога моего, понимаю и познаю, и на путь истинный наставляюсь. "--- Въ словѣ Твоемъ вижу я, чего Создатель отъ разумнаго созданія Своего, Господь отъ раба своего, высочайшій Благодѣтель отъ того, которому дѣлаетъ благодѣяніе, и Богъ отъ человѣка требуетъ, то есть, такого почитанія, покоренія, поклоненія, послушанія и любви, каковаго почитанія никакой твари отдавать не должно. И отъ того научаюсь Тебе тако почитать, какъ вѣра моя и слово Твое святое научаетъ. Сіе и совѣсть моя мнѣ показуетъ, и къ таковому почитанію мене убѣждаетъ. "--- Въ словѣ твоемъ вижу я, кто я и кто ближній мой, и что я ему долженъ, то есть, какъ себе люблю, такъ и его долженъ любить, яко человѣкъ человѣка и сродное себѣ естество, и отъ Тебе возлюбленное и почтенное; и ничего ему не дѣлать, чего себѣ не хощу, и все ему дѣлать, чего себѣ хощу. Сіе ему долженъ я, и сего отъ слова Твоего святаго научаюсь я. "--- Въ словѣ Твоемъ вижу я, что Ты, яко Создатель, о всемъ созданіи Своемъ промышляешь такъ, что и птичка малая \textit{не падаетъ безъ воли Твоей}\footnote{Мѳ.~10,~29.}; но наипаче о человѣкѣ, котораго по образу и по подобію Твоему сотворилъ еси. Въ томъ вижу я, что Ты дивный и уму нашему непостижимый промыслъ человѣколюбно показалъ о бѣдномъ человѣкѣ. Подлинно человѣкъ въ созданіи своемъ почтенъ отъ Тебе, Создателя своего, почтенъ паче всея твари Твоея особливымъ Твоимъ совѣтомъ: \textit{сотворимъ человѣка}. Сотворенъ по образу Твоему и по подобію Твоему, сотворенъ святъ, чистъ, непороченъ, премудръ, и къ вѣчному блаженству сотворенъ. Но лестнымъ лукаваго змія совѣтомъ прельщенъ, всея своея чести и блаженства лишился, и тако отвратившися отъ Тебе, Создателя своего, погибнулъ. О пагубнаго совѣта! о лютаго паденія! Палъ человѣкъ, и погибъ тотъ, который по образу Божію и по подобію Божію сотворенъ, и къ вѣчному блаженству сотворенъ. \textit{Адаме! гдѣ еси?} Ахъ, любезное Мое созданіе, руками Моими сотворенное, по образу Моему и по подобію Моему сотворенное, \textit{гдѣ еси}? Съ какой высоты въ какую подлость, съ какой славы въ какое безславіе, съ какой чести въ какое безчестіе, съ какого блаженства въ какое окаянство низпалъ ты! Змій лукавый прельстилъ созданіе Мое почтенное; разбойникъ похитилъ и отвелъ въ плѣнъ человѣка, руками Моими сотвореннаго; звѣрь лютый растерзалъ чадо Мое любезное. \textit{Человѣкъ въ чести сый не разумѣ, приложися скотомъ несмысленнымъ, и уподобися имъ}\footnote{Пс.~48,~13 и 21.}. Тако погиблъ человѣкъ первозданный; погибли и мы, сыны его. Ты же, благій, милосердый и человѣколюбивый Боже, не терпѣлъ видѣть бѣднаго человѣка въ погибели, но умилосердился надъ нимъ. Подалъ"=было ему законъ Твой, дабы его отпадшаго отъ Тебе обратилъ и привелъ къ Тебѣ; но онъ только показывалъ немощь его ему, а не исцѣлялъ его; обличалъ его и устрашалъ, а не утѣшалъ; показывалъ погибель его, а не спасалъ его, и тако такъ сильно уязвленнаго и немощнаго не моглъ спасти, яко не моглъ его исполнити. Послалъ Ты къ падшему человѣку избранныхъ рабовъ Твоихъ пророковъ; и тіи только обличали человѣка, а не помогли человѣку, но указывали на Грядущаго спасти человѣка, и Того возвѣщали, и тѣмъ отраду нѣкую дѣлали бѣдному человѣку: грядетъ, грядетъ спасти насъ; \textit{Грядый пріидетъ и не закоснитъ}. Наконецъ пришло время, и святѣйшимъ благоволеніемъ Твоимъ послалъ Ты къ намъ Самого единороднаго Сына Твоего, Тебѣ соестественнаго и соприсносущнаго и собезначальнаго. Той въ подобострастную намъ плоть облеклся, и отъ пренепорочныя Дѣвы Матери родился, и, Богъ сый безначальный, младенчествовалъ, и Царь небесный, на земли жилъ, о Которомъ Ты съ небесе святаго Своего божественнѣйшимъ Своимъ засвидѣтельствовалъ гласомъ: \textit{Сей есть Сынъ Мой возлюбленный, о немже благоволихъ; Того послушайте}\footnote{Мѳ.~17,~5.}. Той отъ Тебе, Бога и Создателя нашего, къ намъ бѣднымъ посланъ взыскати и спасти насъ, погибшее Твое созданіе, якоже самъ Онъ о Твоей къ намъ недостойнымъ любви утѣшительно проповѣдалъ: \textit{тако возлюби Богъ міръ, яко и Сына Своего единороднаго далъ есть, да всякъ вѣруяй въ Онь не погибнетъ, но имать животъ вѣчный. Не посла бо Богъ Сына Своего въ міръ, да судитъ мірови, но да спасется Имъ міръ}\footnote{Іоан.~3,~16 и 17.}. Онъ святѣйшее Твое благоволеніе и высочайшую Твою милость, пришедши къ намъ, объявилъ намъ; и святымъ Твоимъ Евангеліемъ, которое отъ Тебе принеслъ намъ, сердца наша, ядомъ зміинымъ огорченная, утѣшилъ и возвеселилъ; проповѣдалъ намъ, что Ты грѣхи наши, которыми Тебе, Создателя и Бога нашего, прогнѣвали, оставляешь намъ, и насъ къ Себѣ чрезъ Него призываешь, и вѣрующимъ въ Него и послушающимъ Его двери вѣчнаго Твоего царствія отверзаеши, якоже Самъ Онъ о томъ засвидѣтельствовалъ къ нашему утѣшенію: \textit{Духъ Господень на Мнѣ: Егоже ради помаза Мя благовѣстити нищимъ, посла Мя исцѣлити сокрушенныя сердцемъ, проповѣдати плѣненнымъ отпущеніе, слѣпымъ прозрѣніе, отпустити сокрушенныя во отраду, проповѣдати лѣто Господне пріятно}\footnote{Лук.~4,~18 и 19.}. Сію всепріятную и всерадостную вѣсть отъ Тебе принесши намъ бѣднымъ грѣшникамъ, съ человѣками, яко человѣкъ, на земли пожилъ, и вѣровать въ Тебе, и угождать Тебѣ и волю Твою творить, чрезъ Себе Самого научилъ насъ, и къ вѣчному Твоему царствію путь показалъ намъ и призвалъ насъ, и пострадалъ и умеръ отъ непріемшихъ Его, и изъ мертвыхъ восталъ, и возшелъ на небо, откуду и пришелъ, и сѣлъ одесную Тебе Бога и Отца. Тако страданіемъ и смертію Своею и кровію, насъ ради изліянною, очистивши насъ вѣрующихъ въ Него отъ грѣховъ нашихъ, влечетъ къ Тебѣ, небесному Своему Отцу, яко пастырь добрый, заблудшія овцы Твоя. За сей человѣколюбивый, спасительный, чудный и умомъ нашимъ непостижимый промыслъ Твой, какъ вси человѣки, такъ и я, бѣдный грѣшникъ, \textit{много Тебѣ, Господи, долженъ}. Но \textit{что воздамъ Господеви о всѣхъ, яже воздаде ми?} Исповѣдаюся Тебѣ, Отче, съ единороднымъ Твоимъ Сыномъ и Пресвятымъ Твоимъ Духомъ! Пою человѣколюбіе Твое! Прославляю благость и милосердіе Твое, недостойный и бѣдный грѣшникъ! \textit{Благословенъ Господь Богъ Израилевъ, яко посѣти и сотвори избавленіе людемъ Своимъ, и воздвиже рогъ спасенія намъ, въ дому Давида отрока Своего}\footnote{Лук.~1,~68 и 69.}. Гдѣ бы я былъ, грѣшникъ и законопреступникъ, аще не въ погибели и въ вѣчной смерти, яко мертвецъ поверженный, лежалъ бы? Вкушалъ бы, какъ и демоны, вкушалъ бы вѣчно горькіе грѣховъ моихъ плоды. \textit{Оброцы бо грѣха смерть}\footnote{Римл.~6,~23.}. Благость Твоя, милосердіе и человѣколюбіе Твое не допустило мене до того, но такъ чудно спаслъ мя еси. \textit{Много я Тебѣ, Господи, за сіе долженъ. Благослови душе моя Господа, и вся внутренняя моя имя святое Его. Благослови душе моя Господа, и не забывай всѣхъ воздаяній Его}, и проч.\footnote{Пс.~102,~1 и 2.} "--- Вижу я въ словѣ Твоемъ святомъ, что діаволъ, злый оный змій древній, который и прародителей моихъ въ раи прельстилъ, тойжде и нынѣ, завидя блаженству нашему, къ которому благодатію единороднаго Сына Твоего ведеши насъ, \textit{яко левъ рыкая, ходитъ, искій кого поглотити}\footnote{1~Петр.~5,~8.}. Отъ сего врага нашего, какъ всѣхъ знающихъ Тебе, такъ и мене недостойнаго предостерегаетъ благость Твоя, и сохраняетъ всесильная десница Твоя. О, како бы цѣлъ моглъ я быть, како бы не поглотилъ мене жива, аще бы не всемогущая Твоя хранила мене десница! \textit{Много я Тебѣ, Господи, и за сіе долженъ. Благословенъ Господь, Иже не даде насъ въ ловитву зубомъ ихъ}\footnote{Пс.~123,~6.}. Но не остави мене, Господи, и до конца, но избави отъ лукаваго того. \textit{Воскресни Господи Боже мой, да вознесется рука Твоя, не забуди убогихъ Твоихъ до конца}\footnote{9,~33.}. \textit{Не предаждь звѣремъ душу исповѣдающуюся Тебѣ: душъ убогихъ Твоихъ не забуди до конца}\footnote{73,~19.}. "--- Вижу я въ словѣ Твоемъ святомъ, что ангели Твои святіи, предстоящіи престолу славы Твоея, и видящіи святѣйшее лице Твое, \textit{ополчаются окрестъ} знающихъ и боящихся Тебе, и сохраняютъ ихъ повелѣніемъ Твоимъ отъ лукавыхъ демоновъ\footnote{33,~8.}, \textit{и въ служеніе посылаеми бываютъ за хотящихъ наслѣдовати спасеніе}\footnote{Евр.~1,~14.}. Не оставляеши бо тѣхъ, ихъ же возлюбилъ еси, и честною кровію единороднаго Сына Твоего искупилъ еси; но чрезъ служителей Твоихъ святыхъ сохраняеши ихъ, и къ вѣчному покою приводиши. Тоя благодати Твоея и я, недостойный и непотребный рабъ Твой, по милости Твоей, сподобляюся. \textit{Много я Тебѣ, Господи, и за сіе долженъ}. Знаю я и исповѣдуюсь со смиреніемъ, что я много Тебѣ, Создателю и Богу моему, согрѣшилъ, и жалѣю о томъ, и иные грѣхи вижу въ совѣсти моей, а иныхъ и не вижу, и болѣе не вижу, нежели вижу: \textit{грѣхопаденія бо кто разумѣетъ}\footnote{Пс.~18,~13.}? Жалѣю и сокрушаюся, что ими Тебе, благаго и человѣколюбиваго Бога моего, Создателя моего, Искупителя моего, высочайшаго Благодѣтеля моего, Того, Котораго ангели святіи со страхомъ и трепетомъ почитаютъ, поютъ и покланяются, "--- Тебе, такого и толикаго Господа, безумно я, послѣднѣйшій червячекъ, много оскорбилъ. И сколько разъ согрѣшилъ Тебѣ, столько разъ видѣлъ Ты согрѣшающаго мене; и сколько разъ видѣлъ, столько разъ терпѣлъ Ты мнѣ по благости Твоей; и сколько разъ терпѣлъ мнѣ, столько разъ миловалъ мене. И аще бы по святѣйшей правдѣ Твоей поступилъ Ты со мною, уже бы давно сошла душа моя во адъ; но благость Твоя и человѣколюбіе Твое и долготерпѣніе Твое удержало Тебе, и не допустило мене, бѣднаго грѣшника, до погибели моей. \textit{Много я Тебѣ, Господи, и за сію великую Твою ко мнѣ милость долженъ. Исповѣмся Тебѣ, Господи Боже мой, всѣмъ сердцемъ моимъ, и прославлю имя Твое въ вѣкъ: яко милость Твоя велія на мнѣ, и избавилъ еси душу мою отъ ада преисподнѣйшаго}\footnote{85,~12 и 13.}. Отъ сихъ и прочіихъ, которыхъ и не знаю, благодѣяній Твоихъ ко мнѣ вижу я, что въ любви Твоей ко мнѣ и благости, человѣколюбіи и долготерпѣніи и щедротахъ весь заключаюся. Благость Твоя, Господи, есть, что я еще не погиблъ, еще живу. Вижу я, что Ты къ спасенію вѣчному мене ведешь, которое обѣщалъ Ты знающимъ Тебе и почитающимъ. Помилуй убо мене, бѣднаго грѣшника, до конца, и, благодатію единороднаго Сына Твоего, заглади вся согрѣшенія моя, и спаси мя въ вѣчную жизнь, да тамо со всѣми избранными Твоими за вся Твоя благодѣянія буду благодарить, хвалить и пѣть Тебе со Единороднымъ Твоимъ Сыномъ и Пресвятымъ Духомъ, не вѣрою, но лицемъ къ лицу въ некончаемыя вѣки, аминь. "--- Отъ сихъ разсужденій видишь, хрістіанине: 1)~Что я и ты Богу долженъ; то есть, какъ я, такъ и ты всего себе долженъ: что живешь, что движешися, Божія то благость есть, и хотя, что ни дѣлаешь, какъ ни угождаешь Богу, такъ что хотя бы ты чрезъ все житіе свое страдалъ ради имене Его, и то бы ничто было, во всегда должникомъ остаешися предъ Богомъ. Тако мы обдолжены и обдолжаемся отъ Него! За сіе едино, что Онъ создалъ насъ, никакъ и ничимъ воздать не можемъ, а что уже о прочіихъ безчисленныхъ и высочайшихъ Его благодѣяніяхъ воздадимъ? \textit{Что бо воздадимъ Господеви о всѣхъ, яже воздаде намъ}? Едино сіе "--- \textit{ничто}. 2)~Отсюду научаемся со смиреніемъ Ему благодарить отъ искренности сердца, и признавать свой недостатокъ и нищету въ томъ, что отъ Него, яко Благодѣтеля, получаемъ благодѣянія на всякій день и часъ, но Благодѣтелю воздать ничимъ не можемъ. 3)~За вся благая должны мы Ему благодарить, но паче за слово Его, яко высочайшій Его даръ, къ намъ посланный, и за спасительное единороднаго Его Сына къ намъ пришествіе, и ради Того обѣщанный намъ вѣчный животъ и блаженство. Въ семъ бо неизреченная и непостижимая Божія къ намъ благость и любовь открылась. 4)~Отсюду видишь, хрістіанине, какъ ты убогъ и бѣденъ, что безъ благихъ Божіихъ и минуты жить не можешь. Убогъ есть и бѣденъ тотъ, который пищи не имѣетъ, тотъ, который одѣянія не имѣетъ, тотъ, который дому не имѣетъ и прочая: ты всего того и прочаго не имѣешь своего, но все отъ Бога, богатаго въ милости, получаешь. И аще бы не подалъ тебѣ Господь всего того, то бѣднѣйшій и окаяннѣйшій былъ бы ты паче всея твари. Видишь убо, какъ ты нищъ и убогъ. Познавай убо и признавай свою бѣдность и нищету, да обогатитъ тебе Господь богатый милостію. 5)~Отсюду видишь, что мы ничего у Бога чрезъ себе заслужить не можемъ, но все туне отъ щедрой Его десницы получаемъ. Онъ, яко благій и богатый въ милости, видя нашу бѣдность и нищету, сокровище благости Своея отверзаетъ, и отъ того все туне намъ подаетъ. Естество бо Его такое есть, которое не можетъ не благотворить. 6)~Отъ познанія Божія послѣдуетъ обновленіе, исправленіе, благочестіе и святое хрістіанское житіе, какъ выше видѣлъ ты. Бога истинно знать и не благочестиво жить противна суть себѣ. Не возможно бо не исправиться и не обновиться тому, кто Бога познаетъ, и чимъ болѣе познаетъ, тѣмъ болѣе исправляется, обновляется и лучшимъ бываетъ. 7)~Отсюду послѣдуетъ, что хрістіане, беззаконно живущіи, Бога не знаютъ, хотя имя святое Его исповѣдаютъ, и молятся Ему, и въ церковь ходятъ, и Таинъ Хрістовыхъ пріобщаются, и проч. И сіе"=то есть, что апостолъ написалъ: \textit{и о семъ разумѣемъ, яко познахомъ Его, аще заповѣди Его соблюдаемъ. Глаголяй, яко познахъ Его, и заповѣди Его не соблюдаетъ, ложь есть, и въ семъ истины нѣсть}\footnote{1~Іоан.~2,~3 и 4.}. Таковыи вси въ смерти пребываютъ, яко отъ живота удаляющіися; тѣломъ живутъ, но душами мертви суть; предъ міромъ живутъ, но предъ Богомъ умерли. Не возможно бо оскверненному грѣхами съ Богомъ святымъ общеніе имѣть, а отлучившемуся отъ Него, яко живота, надобно мертвымъ быть, по реченному: \textit{се удаляющіи себе отъ Тебе погибнутъ}. Сюда надлежатъ блудники, прелюбодѣи, всякіи сквернители, злобники и мстители, хищники, воры и грабители, клеветники и ругатели, прелестники, обманщики и лицемѣры, и прочіи подобныи симъ. Все таковыя отъ Бога удалились злонравіемъ своимъ и потому въ смерти пребываютъ. 9)~Когда беззаконникъ къ Богу всѣмъ сердцемъ обратится, и будетъ въ истинномъ покаяніи пребывать, то благодатію Божіею паки оживетъ. Аще убо, грѣшникъ, хощеши жить душею, а не тѣломъ единымъ, то обратись всѣмъ сердцемъ къ Богу "--- Животу, всѣхъ оживляющему, и держись Того вѣрою и правдою, и будеши живъ истинно и нынѣ и въ будущемъ вѣкѣ, и хотя умрешь, паки оживешь, и изыдешь въ воскресеніе вѣчнаго живота. \textit{Хвалите Господа вси языцы, похвалите Его вси людіе: яко утвердися милость Его на насъ и истина Господня пребываетъ во вѣкъ}\footnote{Пс.~116,~1 и 2.}. \textit{Хвали, душе моя Господа: восхвалю Господа въ животѣ моемъ: пою Богу моему дондеже есмь}\footnote{145,~1 и 2.}.

\section{107. Память благодѣтеля отсутствующаго.}

Бываетъ, что человѣкъ отъ человѣка другаго много благодѣянія пріемлетъ, и, поминая благодѣянія его, всегда помнитъ и благодѣтеля, и отъ любви и благодарности къ нему и другамъ своимъ о немъ напоминаетъ. Благодарнаго бо сердца есть всегда помнить благодѣтеля и благодѣянія его, якоже неблагодарнаго "--- забывать и благодѣянія и благодѣтеля, якоже пишется о Израильтянахъ неблагодарныхъ: \textit{и забыша благодѣянія Его, и чудеса Его, яже показа имъ}, и проч.\footnote{77,~11.} И понеже благодарный человѣкъ помнитъ благодѣтеля и благодѣянія его, то часто и предъ другами своими о немъ хвалится, и имъ показуетъ благодѣянія благодѣтеля своего, и говоритъ: онъ"=де мнѣ тое и тое сдѣлалъ добро; онъ мнѣ сей домъ построилъ, сію одежду онъ мнѣ далъ, отъ такой"=то бѣды онъ мене свободилъ, и проч. Сіи знаки суть благодарнаго сердца къ человѣку благодѣтелю своему. Тако хрістіанину должно Бога и благодѣянія Его безчисленная во всегдашней памяти имѣть, и внутрь себе съ душею своею говорить: Богъ мой мене создалъ и изъ небытія въ бытіе привелъ, и тое и тое добро мнѣ подаетъ. Свѣтила сіи, которыя я вижу, солнце, луна и звѣзды повелѣніемъ Его мене недостойнаго освѣщаютъ, и показуютъ мнѣ путь во дни и въ нощи, и ихъ помощію уклоняюся я отъ зла и всякаго вреда. О, како бы я бѣденъ былъ, и ничимъ бы разнствовалъ отъ слѣпаго, ежели бы онѣ скрылись отъ мене! Воздухъ, который привлекаю въ себе и испущаю изъ себе, Его повелѣніемъ мнѣ служитъ и сохраняетъ животъ мой. Тотчасъ бы я исчезнулъ, ежели бы хотя на едину минуту отнялся онъ отъ мене. Облака, которыя, какъ мѣхи, внутрь себе содержатъ воду, и съ мѣста на мѣсто переносятъ, и воздухъ и землю орошаютъ, Его благоволеніемъ служатъ мнѣ. Что бы мнѣ пользовала земля плодовитая и посѣянныя въ ней сѣмена, аще бы ее не орошали свыше облака? Была бы безплодная земля, и я бы, и скоты мои исчезли безъ пищи, какъ бываетъ во время бездождія и глада. Домъ сей, въ которомъ я живу и упокоеваюся, и отъ воздушной непогоды и бури сохраняюся, всемогущая десница Его устроила мнѣ. Какъ бы горькое житіе было, когда бы я того не имѣлъ! Благость Его и той мнѣ промыслила. Пищу сію, которую я вкушаю и укрѣпляю немощное тѣло мое и утѣшаюсь; питіе сіе, которымъ прохлаждаюсь, и прогоняю жажду мою, щедрая Его рука подаетъ мнѣ. Како бы я моглъ жить, аще бы Онъ не отверзалъ сокровища Своего, и не подавалъ мнѣ добра Своего того? Сію одежду, которою покрываю и согрѣваю нагое и многобѣдное тѣло мое, посылаетъ мнѣ святая и всесильная десница Его. Како бы я моглъ ходить и обращаться нагъ, и цѣлъ быти отъ холода и мраза? Сего ради и въ семъ человѣколюбіе Его промыслило о мнѣ. Его огнь согрѣваетъ домъ мой, и варитъ пищу мою мнѣ, и въ нощи освѣщаетъ мене. Его всесильная и щедрая десница проліяла озера, рѣки и источники ради мене и скотовъ моихъ, и въ нихъ многоразличные рыбъ и всякихъ животныхъ роды произвела ради потребы моей. Вижу я на поляхъ и степяхъ различные травъ и зелій роды. Его повелѣніемъ земля отверзаетъ сокровища своя. Сія и тая подаетъ то мнѣ, то скотамъ моимъ: отъ тѣхъ немощному тѣлу моему врачевство, скотамъ моимъ пища бываетъ. Онъ насадилъ различныя древеса, и тыя различнымъ моимъ нуждамъ служатъ. Его повелѣніемъ плодятся различныя птицы, скоты и звѣри: отъ тѣхъ я то пищу, то одежду, то иную многоразличную потребу получаю. Онъ изводитъ вѣтры отъ сокровищъ Своихъ: и тіи мене и скотовъ моихъ прохлаждаютъ во время зноя, очищаютъ воздухъ, прогоняютъ мглу и дѣлаютъ ясное ведро. Его всесильнымъ словомъ бываетъ день и нощь, восходитъ солнце и заходитъ: восходитъ, и бываетъ день; заходитъ, и оставляетъ нощь. День мнѣ служитъ, служитъ и нощь: во дни дѣлаю и исхожду на дѣло свое и на дѣланіе свое до вечера; въ нощи упокоеваю немощное тѣло мое, и тѣмъ собираю новыя силы къ поднятію трудовъ слѣдующаго дня, и проч. Онъ послалъ мнѣ слово Свое святое, яко царь милостивый письмо къ рабу своему, послалъ чрезъ рабовъ Своихъ, избранныхъ на тое и освященныхъ мужей, которое освѣщаетъ мене, во тьмѣ невѣдѣнія живущаго, и, яко свѣтильникъ, свѣтитъ мнѣ и показуетъ путь правый мнѣ, и научаетъ раздѣлять тьму отъ свѣта, зло отъ добра, лесть отъ истины, вредъ отъ пользы, грѣхъ отъ добродѣтели, суевѣріе и невѣріе отъ вѣры, прелестное блаженство отъ истиннаго, безуміе отъ мудрости, нечестіе отъ благочестія, и тако вразумляетъ мене и умудряетъ. Въ томъ я вижу, кто я, и кто Тотъ, Который о мнѣ такъ милостивно и человѣколюбно промышляетъ; въ томъ я вижу, что я созданіе, Онъ Создатель мой. Я начало бытія имѣю отъ Него, и буду имѣть конецъ: Онъ какъ начала, такъ и конца не имѣетъ, и какъ не начался, такъ и скончаться не можетъ, и единъ имѣетъ безсмертіе. Въ томъ вижу я, что какъ вси концы земли, такъ и я во всемогущей руцѣ Его содержуся. Въ томъ вижу я, что Онъ какъ о всѣхъ созданіяхъ, а наипаче о человѣкѣ, такъ и о мнѣ премудро промышляетъ. Въ томъ вижу я, что небо и земля съ исполненіемъ ихъ "--- Его рукъ дѣло суть. Солнце, луна и звѣзды, которыя мнѣ свѣтятъ, Его дѣло суть. Воздухъ, который оживляетъ мене; облака, которыи орошаютъ мене; земля, которая раждаетъ плоды мнѣ, и на которой живу я; воды, которыя прохлаждаютъ и служатъ мнѣ; огнь, который согрѣваетъ мене и варитъ пищу мнѣ; скоты, которыи работаютъ мнѣ; древеса, которыи служатъ нуждамъ моимъ, "--- дѣла рукъ Его суть. Домъ, въ которомъ упокоеваюсь я; пища и питіе, которыми укрѣпляю немощное тѣло мое; одежда, которою покрываю и согрѣваю нагое тѣло мое, и проч., "--- Его добро суть. Онъ мнѣ вся сія подаетъ. Слово Его святое научаетъ мене, како жить и обращаться, что творить и чего уклоняться, какъ и чимъ Ему угождать; и показуетъ немощь мою мнѣ, и обличаетъ грѣхи мои, и угрожаетъ мнѣ за тые судомъ Его, и тако въ страхъ и смиреніе приводитъ мене, и убѣждаетъ каятися, жалѣти за грѣхи, и съ сокрушеніемъ сердца къ Нему прибѣгати, милости искати и молитися: \textit{Боже, милостивъ буди мнѣ грѣшному}. Но тако трепещущаго мене и сокрушенное сердце имѣющаго, и суда Его боящагося, уже ободряетъ, оживляетъ и утѣшаетъ, да не пожертъ буду печалію, и всеконечно отчаюся милости Его. Ибо въ словѣ Его вижу я, что Онъ всѣмъ бѣднымъ грѣшникамъ, обращающимся къ Нему и за грѣхи кающимся, и со смиреніемъ Ему припадающимъ, и милости просящимъ, отверзаетъ двери милосердія Своего и пріемлетъ ихъ. Аще всѣмъ являетъ милость, то и мнѣ, понеже и я единъ отъ всѣхъ. Аще Онъ всѣмъ благъ, то и мнѣ благъ; аще Онъ всѣмъ человѣколюбивый, то и мнѣ; аще Онъ всѣмъ милосердый, то и мнѣ; аще Онъ всѣхъ пріемлетъ, то и мене. Аще Онъ всѣхъ вѣрующихъ въ Него спасаетъ, то и мене, ибо и я вѣрую въ Него. Гдѣ жъ, о возлюбленне, гдѣ Тотъ, Который столько тебѣ благодѣтельствуетъ? Гдѣ толикій Любитель твой, Который такую любовь являетъ тебѣ? Гдѣ тотъ великій Благодѣтель твой, Который столько благодѣяній показуетъ тебѣ? куды отшелъ? гдѣ живетъ, пребываетъ и обращается? Скажи, скажи, возлюбленне! "--- Онъ вездѣ, и отъ мене не далече, но близъ мене есть, но я не вижу Его, и тѣмъ болѣзную, и уязвляюся, что не вижу Любителя моего, Благодѣтеля моего. \textit{Ибо живущіи въ тѣлѣ отходимъ отъ Господа (вѣрою бо ходимъ, а не видѣніемъ})\footnote{2~Кор.~5,~6 и 7.}. Вижу я Его благая, но благихъ Дателя не вижу. Получаю отъ щедрой Его десницы благодѣянія, но Благодѣтеля не вижу. Вижу небо и землю съ исполненіемъ ихъ, которая вся сотворила рука Его, но Самого Его Сотворшаго не вижу. Его солнце, луна и звѣзды мнѣ свѣтятъ; Его воздухъ, облака, рѣки, озера и источники мнѣ служатъ; Его земля различныя древеса, травы, зелія и плоды мнѣ раждаетъ, но Его Самого не вижу. Онъ невидимою рукою Своею всякое мнѣ добро посылаетъ, но Самого Его Посылающаго не вижу. Онъ сокровище Свое отверзаетъ, и подаетъ мнѣ пищу, питіе и одежду, но Подающаго не вижу: \textit{вѣрою бо хожу, а не видѣніемъ}. Его всесильная десница сохраняетъ мене отъ козней вражіихъ, но Хранителя моего не вижу. Онъ мое прибѣжище, мой Заступникъ, мой Помощникъ, мое утвержденіе въ различныхъ искушеніяхъ, бѣдахъ, напастяхъ и скорбяхъ, но Самого Его не вижу. Чувствую помощь и заступленіе Его, но Помощника и Заступника не вижу. Онъ мене во дни бдящаго хранитъ, хранитъ и въ нощи спящаго, но Хранителя моего не вижу. Въ домѣ ли я сижу "--- Онъ со мною. Изъ дому ли изыду "--- Онъ не оставляетъ мене. По пути ли иду "--- со мною есть. Во градѣ ли, въ селѣ ли, въ пустыни ли, съ людьми ли, или безъ людей имѣюсь "--- не отступаетъ отъ мене Онъ, но не вижу Его: \textit{вѣрою бо хожу, а не видѣніемъ}. Дѣлаю ли что, или говорю, или мыслю, "--- предъ Нимъ все дѣлаю, говорю и мыслю, но не вижу Его: \textit{вѣрою бо хожу, а не видѣніемъ}. Въ церкви ли, или индѣ на молитвѣ стою, "--- предъ Нимъ стою; но Его не вижу: \textit{вѣрою бо хожу, а не видѣніемъ}. Онъ на мене смотритъ, и видитъ мене, и видитъ сѣданіе мое и востаніе мое, и разумѣетъ помышленія моя, но я Его не вижу: \textit{вѣрою бо хожу, а не видѣніемъ}. Предъ Нимъ я колѣна своя преклоняю "--- и припадаю, и покланяюсь, и воздыхаю, и молюсь Ему, и прошу и ищу у Него милости себѣ, но Его не вижу: \textit{вѣрою бо хожу, а не видѣніемъ}. Къ Нему простираю руки мои, и очи мои возвожду, но Его Самого не вижу: \textit{вѣрою бо хожу, а не видѣніемъ}. Ему благодарю, яко Благодѣтелю моему, Его пою, Его хвалю, Его благословлю и превозношу, но Самого Его не вижу: \textit{вѣрою бо хожу, а не видѣніемъ}. Чувствую и святую десницу Его, мене недостойнаго касающуюся, но Самого Его не вижу: \textit{вѣрою бо хожу, а не видѣніемъ}. Онъ ударяетъ въ сердце мое страхомъ Своимъ святымъ, и тако боюсь и трепещу Его, но Самого Его не вижу: \textit{вѣрою бо хожу, а не видѣніемъ}. Онъ касается и святою любовію сердца моего, и тако радость, веселіе и восклицаніе ощущаетъ сердце мое, но Самого Любителя моего не вижу: \textit{вѣрою бо хожу, а не видѣніемъ}. Онъ мене наказуетъ и милуетъ, яко чадолюбивый Отецъ; Онъ опечаляетъ и утѣшаетъ, но Его Самого не вижу: \textit{вѣрою бо хожу, а не видѣніемъ}. Слышу я и слово Его святое, и въ томъ слышу, что хотя храмина тѣла моего и разорится, однакожъ востанетъ, паки созиждется и во вѣки не разрушится, но Самого Глаголющаго не вижу: \textit{вѣрою бо хожу, а не видѣніемъ}. О, когда пріиду и явлюся лицу Божію? Когда увижу Того, Котораго желаетъ душа моя? Когда увижу Того, на Котораго ангели не смѣютъ взирати, Котораго съ удивленіемъ и ужасомъ поютъ херувимы и серафимы: \textit{святъ, святъ, святъ Господь Саваоѳъ?} Когда увижу Того, Котораго видѣть есть вѣчная жизнь; на Котораго смотрѣть есть едино утѣшеніе, радость, веселіе, сладость и восклицаніе; Которому предстоять не вѣрою, но лицемъ къ лицу, есть неизреченная слава, честь и блаженство; съ Которымъ быть есть быть въ непрестанномъ блаженствѣ, жизни сладостной, радости, веселіи и непрестанномъ восклицаніи? Когда пріиду и увижу Создателя моего, Промыслителя моего, Благодѣтеля моего, Любителя моего, Искупителя моего, Помощника и Заступника моего, Избавителя и Спасителя моего, славу, честь, радость, веселіе, утѣшеніе и блаженство мое вѣчное? \textit{Имже образомъ желаетъ елень на источники водныя: сице желаетъ душа моя къ Тебѣ, Боже. Возжада душа моя къ Богу крѣпкому живому: когда пріиду, и явлюся лицу Божію}\footnote{Пс.~41,~2 и 3.}? \textit{Помяни мя Господи во царствіи твоемъ}\footnote{Лук.~23,~42.}.

\section{108. Онъ сдѣлалъ дѣло свое, и отшелъ.}

Слышимъ, что въ различныхъ случаяхъ единъ о другомъ говоритъ слово сіе: \textit{онъ сдѣлалъ дѣло свое, и отшелъ}. Говорится тое слово тогда наипаче, когда единъ у другаго спрашиваетъ: гдѣ той"=то человѣкъ? тогда отвѣтствуетъ ему другой тако: \textit{сдѣлалъ дѣло свое и отшелъ}. Хрістіанине! мы слово сіе о Хрістѣ Спасителѣ нашемъ можемъ и должны говорить; а наипаче тогда, когда насъ вопрошаютъ безбожніи и нечестивіи и незнающіи Его и непріемлющіи; тогда мы со дерзновеніемъ отвѣщать должны: \textit{сдѣлалъ дѣло свое, и отшелъ}, "--- хотя Той и невидимо и неотступно съ нами пребываетъ, и до скончанія вѣка пребудетъ, по неложному Своему обѣщанію: \textit{се Азъ съ вами есмь во вся дни до скончанія вѣка, аминь}\footnote{Мѳ.~28,~20.}. Дѣло Его есть дѣло спасенія нашего, которое Онъ сдѣлалъ посредѣ земли, и отшелъ. О семъ дѣлѣ глаголетъ Онъ къ небесному Своему Отцу: \textit{дѣло совершихъ, еже далъ еси Мнѣ, да сотворю}\footnote{Іоан.~17,~4.}. Хрістіанине! дѣло сіе велико есть, и умомъ нашимъ непостижимо. Сіе дѣло пророки издалеча предвидѣли, и удивлялись и ужасались тому, и со удивленіемъ и ужасомъ проповѣдовали, и возвѣщали міру: \textit{грядетъ, грядетъ Той, Иже есть прежде вѣкъ Богъ нашъ; скоро идетъ, и не закоснитъ; Самъ Господь пріидетъ, и спасетъ насъ}. Сіе дѣло Его самовидцы апостоли видѣли и всему міру проповѣдали, и тѣмъ, яко сокровищемъ неоцѣненнымъ, всю вселенную обогатили. \textit{Еже бѣ исперва, еже слышахомъ, еже видѣхомъ очима нашима, еже узрѣхомъ, и руки наша осязаша, о словеси животнѣмъ; и животъ явися, и видѣхомъ, и свидѣтельствуемъ, и возвѣщаемъ вамъ животъ вѣчный, иже бѣ у Отца, и явися намъ: еже видѣхомъ и слышахомъ, повѣдаемъ вамъ, да и вы общеніе имате съ нами}\footnote{1~Іоан.~1,~1--3.}. Дѣло сіе Его великое и преславное объявляетъ намъ и душевнымъ нашимъ очесамъ представляетъ святое Евангеліе Его, которое Онъ намъ оставилъ къ нашему просвѣщенію, вѣрѣ, утвержденію и утѣшенію, оставилъ чрезъ избранныхъ рабовъ Своихъ, апостоловъ. Тамо видимъ великое сіе дѣло Его, которое Онъ Самъ Собою нашего ради спасенія совершилъ, и отшелъ къ небесному Своему Отцу. Тамо видимъ спасительное Его пришествіе къ намъ, труды и подвигъ, нашего ради спасенія подъятый, и отшествіе. Хотя и имѣя родитися прежде всѣхъ вѣковъ, Богъ нашъ, нашего ради спасенія, послалъ ангела Своего къ пресвятой и преблагословеннѣй Дѣвѣ Маріи благовѣстити Ей, что Онъ отъ дѣвическихъ Ея кровей составитъ и пріиметъ Себѣ плоть одушевленную. И бысть тако. Видимъ дѣло сіе, намъ непонятное. Родился Сынъ Божій на земли отъ Дѣвы"=Матере безъ Отца, прежде вѣкъ отъ Отца рожденный. Святое рождество Его воспѣли ангели: \textit{слава въ вышнихъ Богу, и на земли миръ, во человѣцѣхъ благоволеніе!} Видѣли рождшагося Младенца пастыри, и поклонилися и прославили Его. Пришли отъ Востокъ волхвы, чудесною звѣздою наставлены, и рожденному Царю принесли дары, \textit{и падше поклонишася Ему}. Срѣтилъ Его, во храмъ принесеннаго, съ великою радостію святый старецъ Сѵмеонъ, и на руки своя пріялъ предвѣчнаго Младенца\footnote{Лук. 1 и 2; Мѳ. 1 и 2.}. О блаженное чрево, носившее Господа славы! О блаженныя руки, подъемшія и объемшія носящаго всю тварь Господа! О блаженніи очи, видѣвшіи плотію младенствующаго Бога! Того видѣть желали цари, пророки, праведники, и не видѣли\footnote{Лук.~10,~23 и 24.}. \textit{Николиже бо тако явися во Израили}. Ибо Сей явивыйся есть Богъ Авраамовъ, Исааковъ и Іаковль. Сей есть, Который явился Моисею въ купинѣ, огнемъ горящей и несгараемой, и глаголалъ съ нимъ. Сей есть, Который поразилъ Египта язвами, и извелъ изъ него люди Своя. Сей есть, Который раздѣлилъ Чермное море, и провелъ Израиля сквозѣ его. Сей есть, Который глаголалъ съ Моисеомъ на горѣ Синайской, и законъ Свой далъ ему. Сей есть, Который провелъ люди Своя Израиля въ пустыни, и питалъ ихъ манною четыредесять лѣтъ, и поразилъ предъ лицемъ ихъ всѣхъ враговъ ихъ. Сей есть, Который ввелъ люди Своя Израиля въ землю обѣтованную, въ землю, кипящую медомъ и млекомъ, и изгналъ отъ лица ихъ всѣхъ языковъ, и по жребію далъ имъ землю тую. Сей есть, Который послалъ въ міръ пророки Своя. Той Самъ нынѣ пришелъ къ намъ, пришелъ во образѣ подобномъ намъ. \textit{Богъ во плоти явися}\footnote{1~Тим.~3,~16.}. Видишь, хрістіанине, Божіе снисхожденіе, и честь человѣка. Сколько въ созданіи почтенъ отъ Бога человѣкъ, столько падшій превознесенъ въ возстановленіи: 1)~Самъ Богъ пришелъ къ нему, взыскати и спасти его. Въ созданіи образомъ Своимъ почтилъ человѣка; а падшаго и погибшаго Самъ пришелъ спасти. 2)~Отъ рода человѣческаго воспріялъ и соединилъ Себѣ одушевленную плоть, и бысть человѣкъ. И тако уподобился намъ, и учинился Господь нашъ братомъ нашимъ, якоже Самъ глаголетъ: \textit{возвѣщу имя Твое братіи Моей}\footnote{Пс.~21,~23.}. И къ Маріи Магдалинѣ глаголетъ: \textit{иди ко братіи Моей, и рцы имъ: восхожду ко Отцу Моему и Отцу вашему, и Богу Моему и Богу вашему}\footnote{Іоан.~20,~17.}. О непостижимаго Божія снисхожденія! О несказанныя человѣческія чести! \textit{Не отъ ангелъ убо когда пріемлетъ, но отъ сѣмене Авраамова пріемлетъ}\footnote{Евр.~2,~16.}. \textit{Услышите сія вси языцы, внушите вси живущіи по вселеннѣй, земнородніи же и сынове человѣчестіи, вкупѣ богатъ и убогъ}\footnote{Пс.~48,~2 и 3.}. Слышите, Самъ Богъ къ намъ пришелъ, пришелъ въ нашемъ образѣ, уподобися человѣкамъ, и милостивно посѣтилъ насъ. Пріидите, срѣтимъ Его и поклонимся Ему, и дадимъ славу Богу, пришедшему къ намъ. \textit{Благословенъ Господь Богъ Израилевъ, яко посѣти и сотвори избавленіе людемъ Своимъ; и воздвиже рогъ спасенія намъ, въ дому Давида отрока Своего}\footnote{Лук.~1,~68 и 69.}. \textit{Хвали, душе моя, Господа: восхвалю Господа въ животѣ моемъ. Пою Богу моему, дондеже есмь}\footnote{Пс.~145,~1 и 2.}. Пою Тя: слухомъ бо, Господи, услышахъ и ужасохся! Ко мнѣ бо пришелъ еси, мене ища заблуждшаго. Тѣмъ многое Твое снисхожденіе, еже для мене, прославляю, Многомилостиве!

Видимъ въ святомъ Евангеліи, что младенствующій Сынъ Божій искомъ былъ на убіеніе, и со святою Матерію Своею, уклонялся злодѣйскихъ рукъ, бѣжалъ во Египетъ, и тамо нѣсколько времени отъ нихъ сокрывался. Видимъ, что беззаконный царь Иродъ, хотя и ища неповиннаго Іисуса, пришедшаго спасти міръ, убить, устремился гнѣвомъ своимъ на убіеніе неповинныхъ младенцевъ тоя страны, и убилъ ихъ четыренадесять тысящь, какъ Церковь глаголетъ, замышляя, дабы тако между младенцами прочіими и Іисуса младенца рождшагося, царя Іудейскаго, убить. Но не удалось учинить то беззаконному убійцѣ. Пострадали младенцы отъ убійцы, и вѣнчались яко мученики. Іисусъ Божіимъ совѣтомъ сохраненъ: отнесенъ святою Матерію Своею во Египетъ, и еще младенецъ былъ пришлецъ на земли чуждей, и пребылъ тамо, дондеже паки возвращенъ въ землю Израилеву\footnote{Мѳ.~2,~13--18.}. Примѣчай, хрістіанине: 1)Что дѣлаетъ властолюбіе. Иродъ, бояся лишитися царскія чести, устремляется на невиннаго рождшагося Царя Іудейскаго Іисуса, и ради сего столько тысящей неповинныхъ младенцевъ убиваетъ. О, люто и велико зло есть властолюбіе!

Властолюбивый и беззаконный царь толико неповинной крови пролить не ужаснулся!? Ради чего? чтобы чести не лишиться. Видимъ зло сіе и нынѣ въ мірѣ. Властолюбіе не ужасается подымать меча на монарховъ и порфиры ихъ обагрять кровію ихъ, и инымъ образомъ умерщвлять помазанниковъ Божіихъ, и великія нестроенія и замѣшательства во обществѣ дѣлать. Столько царю враговъ есть, сколько есть окружающихъ его властолюбивыхъ сердецъ, хотя они и ласкаютъ ему. 2)~Сынъ Божій, уклоняяся отъ напасти и смерти, подаетъ и намъ образъ не вдаваться самовольно въ напасть, но уклоняться отъ той. Онъ моглъ и тамо сохранится, гдѣ искали Его на убіеніе; но бѣжалъ во Египетъ, въ чужую землю, научая насъ, да и мы, аще гонятъ насъ въ градѣ семъ, \textit{бѣгаемъ въ другій}\footnote{Мѳ.~10,~23.}. 3)~Хрістосъ, Сынъ Божій, какъ только родился на земли, началъ гоненіе терпѣть. Отсюду видимъ, что Онъ отъ самаго рожденія началъ, ради нашего спасенія, крестъ нести. Родился въ вертепѣ и въ скотскихъ яслехъ возлегъ Той, Котораго вся земля и исполненіе ея; и искомъ былъ на убіеніе отъ убійцы и бѣгалъ, у Котораго въ руцѣ животъ и смерть всѣхъ. Да постыдится гордость человѣческая, которая хощетъ и ищетъ въ богатыхъ и красныхъ палатахъ упокоеваться, въ богатствѣ и славѣ міра сего жить, и не терпитъ слышати противнаго себѣ слова; да постыдится, говорю, смотря на небеснаго Царя, рождшагося въ вертепѣ и въ яслехъ почивающаго, и дающаго мѣсто гнѣву беззаконнаго царя, и тому уступающаго. Скажи, скажи пожалуй, хрістіанине, не моглъ ли Хрістосъ Царь небесный имѣть Себѣ мѣста славнаго и великолѣпнаго ради плотскаго Своего рожденія и упокоенія? Како не моглъ? Онъ бо Господь всѣхъ. Его небо и земля вся, и исполненіе ея. Но вертепъ, яко прекрасную палату, и ясли скотскія, яко многоцѣнный одръ, избралъ, да научитъ насъ Своимъ примѣромъ не искать въ мірѣ семъ богатства и славы, но быть яко странниками и пришельцами на земли чуждей, и небеснаго отечества искать, къ которому и созданы мы. Бѣдный тотъ хрістіанинъ, который много въ мірѣ семъ ищетъ и запасаетъ. Несумнѣнный знакъ, что онъ того только желаетъ и ищетъ, что видитъ, а чего не видитъ, того не желаетъ и не ищетъ. Моглъ Хрістосъ не токмо сохранить Себе отъ беззаконныхъ убійцъ, но силою божественною и праведнымъ судомъ Своимъ поразить ихъ, но не хотѣлъ того. Ибо пришелъ въ міръ не погубити, но \textit{спасти души человѣческія}\footnote{Лук.~9,~56.}. Сего ради уклонился и бѣжалъ отъ враговъ Своихъ, и тако подалъ намъ образъ давать мѣсто гнѣву, и ненавидящимъ и обидящимъ насъ уступать. Царь славы и всесильный Господь уступалъ: что уже мы будемъ дѣлать, мы, смиренніи, бѣдніи и немощніи? Будемъ ли отмщевать, всякаго наказанія достойніи? И отмщеніе не наше, но Божіе дѣло есть. \textit{Мнѣ отмщеніе, Азъ воздамъ, глаголетъ Господь}\footnote{Римл.~12,~19.}. 4)~Отсюду видимъ, что міръ, прелестію ослѣпленный, какъ Хріста истины не любилъ, такъ и истинныхъ хрістіанъ ненавидитъ. Хрістосъ Господь съ небесе какъ только явился на земли, то и гоненіе началъ терпѣть отъ злаго міра: тако кто начнетъ приходить ко Хрісту и прилѣпляться Ему вѣрою и любовію, то міръ за нимъ съ ненавистію и гоненіемъ, и уже какъ не своего изгоняетъ. Душа благочестивая глазамъ нечестивыхъ какъ врагъ и злодѣй кажется, хотя имъ никакого не дѣлаетъ и не мыслитъ зла. То имъ едино не любо, что онъ прихотей ихъ уклоняется, и своимъ святымъ житіемъ злобу ихъ, какъ свѣтильникъ, показуетъ и обличаетъ тьму. И сіе"=то есть, что глаголетъ Господь: \textit{будете ненавидими всѣми имене Моего ради}\footnote{Мѳ.~10,~22.}. И Апостолъ Павелъ написалъ: \textit{вси хотящіи благочестно жити о Хрістѣ Іисусѣ, гоними будутъ}\footnote{2~Тим.~3,~12.}. Однакожъ, душа благочестивая, о семъ не смущайся, не унывай и не бойся. Безъ Божія совѣта ничего намъ не приключается. Богъ попущаетъ скорби на благочестивыхъ, и попущаетъ столько, сколько могутъ понести, и попущаетъ на великую пользу ихъ, но и не оставляетъ ихъ, яко Своихъ, безъ утѣшенія. И сіе имъ бываетъ, какъ прохлажденіе во время зноя. Истинно тое есть, что \textit{многи скорби праведнымъ}\footnote{Пс.~33,~20.}, но нечестивымъ далеко большія и множайшія ради вѣчнаго во адѣ мученія ихъ. Хотя они и свирѣпѣютъ здѣ, но тамо гордости и нечестія своего вкушать будутъ плоды. Но и здѣ во всякомъ безпокойствѣ и мятежѣ живутъ: другъ у друга отнимаютъ, другъ друга лишаютъ, другъ на друга жалятся, поносятъ и ругаютъ, и хитросплетенными и язвительными судебныя мѣста наполняютъ челобитными. Чего въ такихъ случаяхъ и обстоятельствахъ ожидать, кромѣ смущенія и безпокойствія? А когда въ совѣсть ихъ посмотрѣть, что иное тамо дѣлается, какъ только непрестанный судъ, осужденіе, мятежъ и мученіе? Злая бо совѣсть паче всякія скорби оскорбляетъ, и паче всякаго мучителя мучитъ человѣка. Цвѣтутъ убо внѣ нечестивіи въ своемъ, благополучіи, но внутрь какъ цвѣты: червивіи. Не тако благочестивіи: они хотя внѣ и обезпокоиваются, но внутрь суть покойны, мирны и тихи; и здѣ утѣшаются, и въ будущемъ вѣкѣ преизобильно и вѣчно утѣшатся. 5)~Видимъ въ святомъ Евангеліи, что о тѣхъ, который искали младенствующаго Іисуса на убіеніе, написано: \textit{изомроша ищущіи души Отрочате}\footnote{М3.~2,~20.}. Тако судъ Божій постигаетъ всѣхъ тѣхъ, которыи благочестивыхъ озлобляютъ и гонятъ. Души благочестивыя, какъ овцы, смиренны, кротки и безоружны; но Богъ за нихъ стоитъ и защищаетъ ихъ, и своевольныхъ и нечестивыхъ смиряетъ. Откуду видимъ, что дивныя Божіи судьбы ихъ постигаютъ, и тамо падаютъ въ ровъ, гдѣ не чаютъ, и тую яму пріуготовляютъ себѣ, которую ради благочестивыхъ ископываютъ. Тако Авессаломъ, сынъ Давидовъ, искалъ святаго своего отца убить, и царствомъ Израилевымъ завладѣть; но, вмѣсто того, погибель себѣ сыскалъ. Аманъ сготовилъ было висѣлицу неповинному Мардохею; но на той висѣлицѣ самъ повѣшенъ, которую Мардохею уготовалъ. Фараонъ, царь египетскій, гналъ въ слѣдъ Израиля, и хотѣлъ паки мучительствомъ своимъ озлобить его; но дозналъ на себѣ праведный судъ Божій, и въ морѣ со всѣмъ воинствомъ своимъ потонулъ. Подобный судъ Божій и нынѣ постигаетъ всѣхъ своевольныхъ, которыи озлобляютъ и гонятъ благочестивыхъ.

Видимъ въ святомъ Евангеліи, что Іисусъ Сынъ Божій \textit{плотію}, которую ради нашего спасенія воспріялъ отъ пречистыя Дѣвы Богородицы, \textit{возрасталъ}, какъ и прочіи человѣцы, и, въ совершенный возрастъ пришедши, \textit{крестился} въ струяхъ іорданскихъ, освящая воды Своимъ прикосновеніемъ, и установляя спасительное крещенія нашего Таинство и таинственную пакибытія нашего баню, которою мы оскверненніи омываемся, и умершіи оживляемся, и погибшія спасаемся, и обетшавшіи и истлѣвшіи обновляемся, и отлучившіися отъ Бога къ Богу паки возвращаемся и пристаемъ, и Ему примиряемся, дѣлаемся вѣчнаго Его царствія наслѣдниками.

Видимъ въ святомъ Евангеліи Его, что Іисусъ Хрістосъ по крещеніи Своемъ возведенъ былъ духомъ въ пустыню, и \textit{искушаемъ} былъ \textit{сатаною}, и побѣдилъ искусителя. Отсюду видишь, хрістіанине, что всякому, кто родится водою и Духомъ, то есть, всякому хрістіанину слѣдуетъ искушеніе отъ діавола, и съ нимъ всегдашняя брань. Аще бо Хріста Сына Божія дерзнулъ искушать злохитрый духъ, оставитъ ли уже хрістіанъ? Хрістосъ бо Господь во всемъ образъ намъ Себе подалъ. Онъ искушаемъ былъ сатаною: надобно и намъ искушенными быть. Онъ побѣдилъ силою Своею искусителя: надобно и намъ Того силою его побѣдить. Не на оперы и банкеты и роскоши позваны хрістіане, но на брань и подвигъ духовный, и подвигъ не противу плоти и крови, но противу діавола и злыхъ его духовъ. \textit{Нѣсть наша брань ко крови и плоти, но къ началомъ, и къ властемъ, и къ міродержителемъ тьмы вѣка сего, къ духовомъ злобы поднебеснымъ}\footnote{Еф.~6,~12.}. Сія брань отправляется благополучно недремлющимъ окомъ. Тяжка брань людямъ съ людьми, но далеко тяжчайшая людямъ съ духами злобными. Они насъ видятъ и, что дѣлаемъ и говоримъ, примѣчаютъ, но мы ихъ не видимъ. Они тщатся у насъ отнять не города, не рубежи, не тлѣнное сокровище, но спасеніе вѣчное, которое Хрістосъ Сынъ Божій честною Кровію и смертію Своею намъ пріобрѣлъ. Сіе неоцѣненное сокровище восхитить у насъ они подвизаются, и подвизаются день и нощь. Борютъ они насъ бодрствующихъ, борютъ и спящихъ; навѣтуютъ во дни, навѣтуютъ и въ нощи. Хрістіанине! надобно и намъ не дремать, когда хощемъ себе спасти и ихъ плѣнниками не быть. Оружіе они свое находятъ въ насъ самихъ. Страстьми и похотьми нашими борютъ насъ. Недосугъ тому богатства, чести и славы въ мірѣ семъ искать, кто противу сихъ враговъ хощетъ благополучно подвизаться. И знакъ есть, что плѣнникъ ихъ есть, кто, оставльши небесная, земная мудрствуетъ; кто не Богу, но міру и прихотямъ своимъ угождаетъ. Таковый отъ нихъ побѣжденъ и плѣненъ. Плѣненъ всякъ блудникъ и прелюбодѣй и нечистоты любитель; плѣненъ тать, хищникъ и грабитель; плѣненъ чародѣй и призывающій его къ себѣ; плѣненъ лживый, обманщикъ и прелестникъ; плѣненъ клеветникъ, ругатель и злорѣчивый; плѣненъ всякъ законопреступникъ, который безстрашно законъ Божій разоряетъ. Ахъ! плѣненъ, и отведетъ его врагъ въ вѣчную за собою погибель, аще отъ мучительства его не избавится и не прибѣгнетъ съ покаяніемъ и жалѣніемъ ко Хрісту. "--- Бѣдный хрістіанине, который міру и грѣху работаешь! Осмотрись, и познавай прелесть душегубца врага и твою погибель. Увеселяютъ тебе прихоти міра сего, но плѣняютъ душу твою. Сіи суть сѣти діавольскія, которыми не тѣлеса, но души хрістіанскія уловляются. Сладокъ тебѣ кажется грѣхъ, но горькій его плодъ "--- плѣненіе и смерть души: \textit{оброцы бо грѣха смерть}\footnote{Римл.~6,~23.}. Бережишися, чтобы тебе какій мучитель и варваръ не плѣнилъ, который только тѣло, а не душу, плѣняетъ; кольми паче должно берещися, чтобы отъ сего мучителя плѣненнымъ не быть, который и душу и тѣло плѣняетъ. Сего ради увѣщеваетъ насъ Апостолъ: \textit{трезвитеся, бодрствуйте, зане супостатъ вашъ діаволъ, яко левъ рыкая, ходитъ, искій кого поглотити: емуже противитеся тверди вѣрою}\footnote{1~Петр.~5,~8--9.}. И паки: \textit{повинитеся убо Богу, противитеся же діаволу, и бѣжитъ отъ васъ}\footnote{Іак.~4,~7.}. И паки: \textit{возмогайте во Господѣ, и въ державѣ крѣпости Его: облецытеся во вся оружія Божія, яко возмощи вамъ стати противу кознемъ діавольскимъ}, и проч.\footnote{Еф.~6,~10 и 11.} Тако вооружаютъ насъ, хрістіанине, противу врага того и злыхъ его духовъ апостоли и посланники Спасителя нашего Іисуса Хріста. Берегись убо, возлюбленный хрістіанине, врага сего и злыхъ его духовъ. Они невидимо около насъ ходятъ, и то тіи, то другія стрѣлы на насъ мещутъ, и ногамъ нашимъ сѣти своя полагаютъ, и тщатся насъ низложить и плѣнить. Берегись убо и стой твердо, но и не унывай, понеже Богъ за насъ стоитъ, Котораго они и \textit{имени трепещутъ}\footnote{Марк.~16,~17.}. Ты только стой вѣрою вооруженъ, берегись, крѣпись, тщись, молись и призывай имя Господне въ нужномъ случаѣ, и приспѣетъ тебѣ помощь Его. \textit{Аще Богъ по насъ: кто на ны}\footnote{Римл.~8,~32.}? \textit{Боже! заступникъ мой еси Ты, и милость Твоя предваритъ мя. Помощникъ мой еси, Тебѣ пою: яко Богъ заступникъ мой еси, Боже мой, милость моя}\footnote{Пс.~58,~10,~11 и 18.}.

Видимъ въ святомъ Евангеліи, что Господь нашъ Іисусъ Хрістосъ, по искушеніи отъ злаго духа, \textit{изшелъ на проповѣдь святаго Евангелія}, которое отъ небеснаго Своего Отца на землю принеслъ, и собравши дванадесять учениковъ, которыхъ апостолами назвалъ, съ ними съ мѣста на мѣсто и отъ града во градъ проходилъ и проповѣдывалъ Евангеліе царствія Божія, и на земли сердецъ человѣческихъ сѣялъ спасительное слова Божія сѣмя, и научалъ всѣхъ познавать небеснаго Своего Отца, и Тому вѣрою и правдою угождать, волю Его святую творить, и Себе во образъ всѣмъ подавалъ, и тако всѣхъ научалъ небеснаго и святаго житія. Не видѣлось въ Немъ ничего, какъ только глубочайшее смиреніе, горячайшая любовь къ небесному Отцу и всѣмъ человѣкамъ, сострадательное милосердіе къ бѣднымъ, неисповѣдимая кротость и долготерпѣніе къ хулителямъ и врагамъ, и прочіи пресладкіи и божественныи нравы, которыи явно показуютъ Его не проста человѣка, но Сына Божія и Бога воплотившагося, и во образѣ человѣческомъ явившагося, и по земли ходившаго. Благость и человѣколюбіе Божіе, которое во всемъ Священномъ Писаніи изображается, все въ Немъ изображенное видимъ, яко въ живомъ образѣ Бога невидимаго. Хощеши ли, хрістіанине, Божіе сердце и преблагій Его нравъ видѣть? Смотри на Хріста, единороднаго Сына Его. Хощеши ли Хрістовъ нравъ и сердце Его видѣти? Читай святое Евангеліе Его, и внимай тому, и представитъ тебѣ. Сей Богъ нашъ, и не приложится инъ къ Нему. Посемъ явися на земли, и съ человѣки поживе. \textit{Видѣна быша шествія Твоя, Боже!} Видѣно было вхожденіе Твое и исхожденіе Твое. Видѣно было сѣданіе Твое и востаніе Твое. Видѣно было обращеніе Твое между грѣшниками, которыхъ взыскати и спасти пришелъ еси. Видѣнъ былъ еси ядый и піяй, яко человѣкъ, даяй пищу всякой плоти. Видѣнъ былъ еси приступенъ грѣшникамъ, неприступный херувимамъ и серафимамъ. Видѣли Тебе очи человѣческія, на Котораго не смѣютъ чини ангельскіи взирати. Слышанъ былъ гласъ Твой, яко гласъ Сына Божія, гласъ яко единороднаго отъ Отца, и воплотившагося Бога нашего. Изліяся благодать въ устнахъ Твоихъ, краснѣйшій добротою паче сыновъ человѣческихъ, Сыне Божій. Видѣна быша преславная дѣла Твоя, яже сотворилъ еси на земли. Видѣны были божественніи и прекрасніи добродѣтели Твои, которыя намъ бѣднымъ грѣшникамъ къ подражанію показалъ и оставилъ еси. О, блаженное тое время, въ которое Солнце праведное на земли обращалося! \textit{Людіе сѣдящіи во тмѣ видѣша свѣтъ велій, и сѣдящимъ въ странѣ и сѣни смертнѣй, свѣтъ возсія имъ}\footnote{Мѳ.~4,~16.}. Блаженніи очи, видѣвшіи Бога во образѣ человѣческомъ! Блаженніи уши, слышавшіи гласъ Сына Божія и небеснаго Царя, явившагося на земли! Того желали пророки, праведники и цари видѣти, и не видѣли, и слышати, и не слышали. Хрістіанине! блаженніи и мы, что видимъ образъ Его, изображенный во Евангеліи, и слышимъ гласъ Его въ томъ же Евангеліи, и исповѣдуемъ и призываемъ имя Его, и таинственно причащаемся пречистаго Тѣла и животворящія Крове Его. Да поревнуемъ убо и прекраснымъ добродѣтелямъ Его, которыя къ подражанію нашему оставилъ Онъ намъ; да послѣдуемъ Богу, насъ ради во образѣ человѣческомъ явившемуся и ходившему по земли. Возлюбимъ Его смиреніе, любовь, милосердіе, милость, кротость, терпѣніе и пресладкій нравъ. \textit{Азъ есмь Свѣтъ міру: ходяй по Мнѣ не имать ходити во тьмѣ, но имать свѣтъ животный}\footnote{Іоан.~8,~12.}. Тако Онъ въ слѣдъ Себе зоветъ насъ. Аще убо въ слѣдъ Его пойдемъ, и возлюбимъ преблагій нравъ Его, то не будемъ во тьмѣ ходить, но свѣтомъ Его просвѣтимся; а когда за Нимъ не пойдемъ, но прихотямъ нашимъ будемъ послѣдовать, то непремѣнно во тьмѣ пребудемъ. Надобно бо тому во тьмѣ быть, кто отъ свѣта удаляется, и въ смерти пребывать, кто отъ живота отлучается. \textit{Яко се удаляющіи себе отъ Тебе, погибнутъ: потребилъ еси всякаго любодѣющаго отъ Тебе. Мнѣ же прилѣплятися Богови благо есть}\footnote{Пс.~72,~27 и 28.}.

Видимъ въ святомъ Евангеліи, что Господь нашъ трудился и утруждался нашего ради спасенія, и, проповѣдуя Евангеліе царствія и тако дѣло нашего спасенія совершая, пѣшъ отъ града во градъ и въ различныя мѣста ходилъ. О великое чудо! Всесильный и всемогущій трудился и утруждался, и сѣдяй на престолѣ славы Своея пѣшъ по земли ходилъ. Бѣдный грѣшникъ! тебе ради сіе снисхожденіе сотворилъ Господь. \textit{Благослови душе моя Господа!} Поемъ и покланяемся неизреченному человѣколюбію и снисхожденію Твоему, Господи, непотребніи раби Твои, грѣшники. Преизбыточествова грѣхъ нашъ къ Тебѣ, но превозмогла благость и человѣколюбіе Твое къ намъ, согрѣшившимъ Тебѣ. Отсюду учимся, хрістіанине: 1)~Не въ праздности, но \textit{въ трудахъ добрыхъ жить}. Многіи хрістіане, понеже хлѣбъ готовый имѣютъ, ни за какое дѣло приняться не хотятъ, но въ праздности живутъ и въ безполезныхъ и бездѣльныхъ разговорахъ упражняются, или то и знаютъ что прогуливаются и изъ гостей въ гости проѣзживаются. Таковая праздность всему злу ихъ научаетъ. Таковыи празднолюбцы и тунеядцы непрестанно грѣшатъ противу Бога, глаголющаго: \textit{въ потѣ лица твоего снѣси хлѣбъ твой}\footnote{Быт.~3,~19.}. Стыдно грѣшникамъ лежать, когда Самъ Господь ради ихъ трудился. 2)~Не въ пышности и гордости міра сего, но \textit{въ смиреніи жить}. Многіи хрістіане не хотятъ ѣздить, какъ только цугомъ и на высокихъ, богатыхъ, позлащенныхъ аглицкихъ каретахъ. Сія ихъ пышность и гордость есть. Сами они пусть видятъ и разсуждаютъ, сколь далеко они отстоятъ отъ Хріста, Который пѣшъ отъ страны въ страну переходилъ. Да посрамится гордость человѣческая, и рогъ свой вознесенный да низпуститъ, когда Царь небесный и Господь славы въ таковомъ смиреніи на земли пожилъ. Бѣдный человѣкъ! смотри на сіе смиреніе, и отложши гордость свою, смирись, когда хощеши за Нимъ слѣдовать въ вѣчную жизнь. \textit{Тѣсный путь и узкая врата въ вѣчную жизнь}. Цугами и широкими колесницами туды войти невозможно. Надобно всю сію пышность и гордость отложить, и за Хрістомъ смиреніемъ слѣдовать. Не роскошами, но скорбьми, и \textit{скорбьми многими въ царствіе Божіе входятъ}\footnote{Дѣян.~14,~22.}.

Видимъ въ святомъ Евангеліи, что Хрістосъ Господь, живучи на земли, не имѣлъ, гдѣ главы подклонить, якоже Самъ свидѣтельствуетъ: \textit{лиси язвины имутъ, и птицы небесныя гнѣзда: Сынъ же человѣческій не имать гдѣ главы подклонити}\footnote{Мѳ.~8,~20.}. Отсюду видишь, хрістіанине: 1)~Каково житіе Хрістово на земли было, какъ великая \textit{нищета}. Той, Котораго земля и исполненіе ея, не имѣлъ, гдѣ главы подклонити. Онъ все моглъ имѣть, яко Господь всего; но и дома Своего не имѣлъ, не имѣлъ, гдѣ главы подклонить. Тако Онъ смирился, тако и обнищалъ насъ ради богатый въ милости. Чудно смиреніе, чудна и нищета Его, Иже есть надъ всѣми Богъ сый благословенъ во вѣки! 2)~Отсюду учимся и мы не желать и не искать въ мірѣ семъ богатства и богатыхъ домовъ, но \textit{довольными быть} тѣмъ, что имѣемъ, и Богъ отъ милости Своей подалъ намъ. Многіи хрістіане нынѣшняго вѣка тако стараются собирать, что рады бы были, ежели бы все сокровище міра сего въ руку ихъ пришло; не довольствуются многими тысящами, но болѣе и болѣе собираютъ и умножаютъ своя сокровища; не довольствуются тою землею, которая къ нимъ принадлежитъ, но болѣе и болѣе разширяютъ ее, и нарицаютъ имена своя на земляхъ; не довольствуются домами предковъ своихъ, но то и знаютъ, что новые изобрѣтаютъ и разширяютъ и возвышаютъ и великолѣпно украшаютъ. Сія ихъ пышность и гордость есть, и всему злу виновное лихоимство. Видятъ они сами, что житіе ихъ есть противное Хрісту, Который нищъ былъ, и не имѣлъ, гдѣ главы подклонити. Бѣдный человѣкъ! что \textit{много печешися и молвиши}? Все, что ни собираешь и сколько ни собираешь, оставишь въ мірѣ. Почто разширяешь земли своя? Посмотри впередъ себе, и увидишь, что тебѣ яма въ три аршина изрывается. Почто широко разширяешь домъ твой? Вотъ скоро переселишися въ треаршинный гробъ, который тебѣ готовится. Посмотри на тотъ, и увидишь, что вси замыслы и начинанія твоя суетны суть. Помяни и тое, что ты странникъ и путникъ еси на земли: почто жъ страннику и путнику много запасать и себе безполезно обременять? Хрістосъ Господь Своимъ примѣромъ научаетъ тебе того. Смотри на Его, и уцѣломудрись. Хрістосъ не прельщаетъ насъ, "--- да не будетъ, "--- но истинѣ научаетъ. Посмотри на тѣхъ, которыи прежде насъ жили и много собирали: все здѣ оставили и отошли на оный вѣкъ безъ всего; пребываютъ нынѣ въ своихъ мѣстахъ, и ожидаютъ общаго воскресенія и суда Божія; они одни тамо, а сокровище ихъ все здѣ осталось; нынѣ они видятъ, что все, что ни имѣли здѣ, не ихъ и было, и познаютъ, что не за инымъ чимъ они гонялися, какъ за тѣнію и вѣтрами. Тоежде приключится и тебѣ: и ты все оставишь, и за ними въ слѣдъ пойдешь, и будешь видѣть, что все, о чемъ нынѣ ни стараешися и собираеши, не твое есть. Блаженъ и мудръ, кто отъ бѣдствія чуждаго научается опасно поступать. Смотри убо на умершихъ и кончину твою и гробъ твой, который уже тебѣ готовится, и не знаешь, когда положитъ тебе смерть твоя въ немъ, и остави суету, пока она тебе не оставитъ. Естеству нашему не много надобно: пищею и одѣяніемъ и хижиною довольствуется. Похоть и роскошь много желаетъ и ищетъ; ей и самое государство не довлѣетъ; она никогда насытиться не можетъ, якоже жаръ, крыющійся въ сердцѣ, утолитися, сколько ни піетъ больный, не можетъ. Познай убо и похоть и нужду естественную, и поступай по требованію естества, а не по желанію похоти. Возлюбленный хрістіанине! помяни и осмотрись, что ты хрістіанинъ, а не идолопоклонникъ. Хрістіане чаютъ воскресенія мертвыхъ и жизни будущаго вѣка: возведи убо и ты туды сердце свое. Надобно воскреснуть здѣ душею тому, кто тогда хощетъ воскреснуть въ вѣчную жизнь. Многіи хрістіане на всякъ день говорятъ: \textit{чаю воскресенія мертвыхъ}; но такъ прилѣпилися къ суетѣ міра сего, и такъ живутъ, какъ бы не было воскресенія мертвыхъ. А можетъ быть что и мечтаютъ такъ, и баснословятъ!... Но пусть, какъ хотятъ, мечтаютъ, и что хотятъ, о немъ говорятъ, оно неотмѣнно будетъ, только имъ не въ пользу, но въ стыдъ и поношеніе вѣчное; тогда они узнаютъ, и самымъ искусомъ дознаютъ, что истинно есть слово Божіе, которое проповѣдуетъ воскресеніе мертвыхъ: и \textit{изыдутъ сотворшіи благая въ воскрешеніе живота, а сотворшіи злая въ воскрешеніе суда}\footnote{Іоан. 5~,29.}. Таковыхъ епикуровъ, какъ моровой язвы, ты удаляйся, и подлинно держи, что наслѣдіе хрістіанское, богатство, честь, слава и все блаженство не здѣ въ мірѣ семъ, но въ жизни будущаго вѣка, какъ и отечество и домъ ихъ тамо. Тамо имъ уготовалъ небесный Отецъ вся благая, \textit{ихже око не видѣ, и ухо не слыша и на сердце человѣку не взыдоша}\footnote{1~Кор.~2,~9.}. Истинныи хрістіане тѣхъ благихъ получить стараются, а о житейскихъ небрегутъ, но тѣмъ довольствуются, что имѣютъ, по ученію апостольскому: \textit{ничтоже внесохомъ въ міръ сей, явѣ, яко ниже изнести что можемъ: имѣюще же пищу и одѣяніе, сими довольни да будемъ}\footnote{1~Тим.~6,~7.}. Поревнуй убо святой дружинѣ сей, и послѣдуй имъ, а паче Самому Хрісту, Который въ мірѣ семъ нищъ былъ и не имѣлъ, гдѣ главы подклонить, хотя и весь міръ, небо и землю съ исполненіемъ ихъ въ руцѣ Своей имѣлъ; и углуби въ памяти своей апостольское увѣщательное слово: \textit{горняя мудрствуйте, а не земная}\footnote{Кол.~3,~2.}. Сіе слово всѣмъ хрістіанамъ сказано, которыи хотятъ истинными, а не ложными быть хрістіанами. А на тѣхъ не смотри, которыи то и знаютъ, что собираютъ и прихоти своя исполняютъ, и во умноженіи богатства, и въ пріисканіи чести и славы, и въ разширеніи и украшеніи домовъ, и въ пріуготовленіи коней и каретъ, и въ изобрѣтеніи новыхъ одѣяній, и въ представленіи различныхъ трапезъ, плотоугодію служащихъ, и въ прочемъ тыя своя прихоти оказываютъ. Таковыи далеко отъ Хріста, и слова Его святаго отступили. И видно, что они того желаютъ и ищутъ по подобію скотовъ, что видятъ, а чего не видятъ, того и не ищутъ. На сердцахъ ихъ лежитъ покрывало, какъ мгла, которымъ закрылся отъ нихъ вѣчный животъ: сего ради за тѣнію вмѣсто истины гоняются.

Видимъ въ святомъ Евангеліи, что за Хрістомъ, Богомъ, явившимся на земли и во образѣ человѣческомъ ходившимъ, люди обоего пола въ великомъ множествѣ ходили, то слышати слово Его святое, то исцѣлитися отъ недугъ своихъ желая. И тѣхъ, когда не имѣли что ясти, въ пустыни чудесно питалъ такъ, что \textit{многія тысящи пятьми хлѣбами насыщалися, и собираемо было болѣе останковъ, нежели хлѣбовъ было}. Тутъ присутствовалъ Самъ во образѣ человѣческомъ, и хлѣбы умножилъ, Который небо и землю и все изъ ничего сотворилъ. Пастырь добрый явился на земли. \textit{Николиже тако явися во Израили}. Познали Пастыря своего заблуждшія овцы, и услышали гласъ Его зовущій ихъ, и прибѣжали къ Нему. Сила Его божественная привлекла ихъ къ Нему. Оставляли домы и хозяйство и дѣла своя, и прибѣгали къ явившемуся новому и неслыханному проповѣднику и чудотворцу. \textit{Николиже бо тако явися во Израили. Николиже тако глаголалъ человѣкъ, якоже Іисусъ}. Какъ Самъ съ небесе, такъ и слово Его небесное было, и глаголы Его "--- глаголы вѣчнаго живота. Хрістіанине! пойдемъ и мы за Хрістомъ, не ногами, но сердцами и премѣненіемъ нравовъ, и услышимъ глаголы Его внутрь насъ, и напитаетъ насъ алчущихъ въ пустыни міра сего, не хлѣбомъ тлѣннымъ, но нетлѣннымъ, отъ небесныя Его трапезы посылаемымъ. Симъ хлѣбомъ не тѣлеса, но души наши укрѣпляться и утѣшаться будутъ.

Видимъ въ святомъ Евангеліи, что вси бѣдніи, которыи къ Нему ни приступали съ прошеніемъ и вѣрою, свое желаніе получали. Приступали прокаженніи, и очищались; приступали слѣпіи, и прозирали; приступали хроміи, и исправлялись. Онъ и разслабленныхъ исцѣлилъ, и мертвыхъ воскресилъ, и бѣсноватыхъ отъ мучительства бѣсовскаго избавилъ, и глухимъ слухъ, и нѣмымъ языкъ разрѣшилъ, и прочая преславная и вышеестественная чудеса сотворилъ. Отъ сихъ видимъ: 1)~Что онъ тотъ Мессія и Избавитель міра явился на земли, Который отъ Бога обѣщанъ, и отъ всѣхъ пророковъ проповѣданъ, и отъ всѣхъ вѣрныхъ ожидаемый былъ. Явился Онъ на земли: показалися чудныя дѣла Божія. Увидѣли люди дѣла, каковыхъ рука человѣческая творити не можетъ. Начали слѣпіи прозирать и хроміи ходить, прокаженніи очищаться и глухіи слышать, мертвыи воставать и нищіи проповѣданіе благовѣствованія съ утѣшеніемъ и радостію слышать. И сими чудесами, какъ перстомъ, показывалось, что сей есть Тотъ преславный Чудотворецъ, Котораго пророки міру предвозвѣстили. Тотъ явился тогда, и чудодѣйствовалъ. Откуду народы, видѣвше преславная и ужасная дѣла Его, съ великимъ удивленіемъ возглашали: \textit{преславная видѣхомъ днесь: николиже тако явися во Израили}. Подлинно николиже тако явися во Израили! Понеже николиже тако явися Богъ во Израили. Прежде являлся Богъ во Израили различно, тутъ явился Богъ во образѣ человѣческомъ. Прежде чудодѣйствовалъ Богъ чрезъ пророковъ и избранныхъ рабовъ Своихъ, тутъ Самъ Собою чудодѣйствовалъ. Прежде глаголалъ чрезъ пророковъ, тутъ Самъ въ Своемъ лицѣ глаголалъ. Воистинну \textit{николиже тако явися во Израили!} 2)~Хрістосъ Господь всѣхъ приступающихъ къ Нему съ вѣрою исцѣлялъ. Хрістіанине! да приступаемъ и мы съ вѣрою къ небесному врачу, Іисусу. Онъ нашъ истинный и вѣрный врачъ; поручимъ убо себе Ему, да исцѣляетъ насъ, якоже хощетъ. Ибо и мы бѣдніи и немощніи, и немощи ни отъ кого, кромѣ Его, исцѣлитися не могутъ. \textit{Слѣпы мы}. Умъ слѣпотствуетъ въ познаніи Бога и святой воли Его, въ познаніи истины и лжи, вѣры и суевѣрія, добродѣтели и порока, добра и зла, пользы и вреда и прочая. Да взываемъ и мы къ Нему изъ глубины сердецъ нашихъ, якоже звали Ему слѣпцы: \textit{помилуй насъ, Господи, Сыне Давидовъ}\footnote{Мѳ.~20,~30.}! \textit{Глухи мы}. Грѣхъ нашъ насъ оглушилъ; не можемъ слышати пресладкихъ глаголовъ Божіихъ. Іисусе, Сыне Божій! \textit{разверзи слухъ} душъ нашихъ, да услышимъ глаголы Твоя, глаголы вѣчнаго живота. \textit{Прокаженны мы}, не тѣломъ, но душею. Люта есть проказа грѣхъ, и гнусна предъ очами Божіими. Возвысимъ убо и мы гласъ къ Нему съ прокаженными: \textit{Іисусе, наставниче, помилуй насъ}\footnote{Лук.~17,~13.}. \textit{Души наши злѣ бѣснуются}. Великъ и лютъ бѣсъ есть грѣхъ. Лютъ бѣсъ есть гордость, лютъ бѣсъ есть гнѣвъ и злоба, лютъ бѣсъ есть сребролюбіе, лютъ бѣсъ есть зависть, лютъ бѣсъ есть нечистота, лютъ бѣсъ есть жестокость и немилосердіе, лютъ бѣсъ есть ненависть, и прочая. Сіи бѣси не тѣлеса, но души наши мучатъ. Да послѣдуемъ убо женѣ хананейской, и да взываемъ неотступно ко Іисусу: \textit{помилуй мя, Господи, Сыне Давидовъ}; и хотя медлитъ подать намъ прошеніе наше, однакожъ не престанемъ толкать въ двери милосердія Его, глаголя: \textit{Господи, помози мнѣ}\footnote{Мѳ.~15,~21--28.}. Получили отъ Него милость вси бѣдніи, который приступали къ Нему, какъ видимъ во Евангеліи: получимъ и мы. Исцѣлилъ Онъ немощныя тѣлеса, много паче исцѣлитъ души безсмертныя. Умилосердился Онъ надъ бѣдностію тѣлесъ, кольми паче умилосердится надъ бѣдностію душъ. Не такъ Онъ жалѣетъ о бѣдствіи тѣлесъ, какъ о бѣдствіи душъ нашихъ. Далеко честнѣйшая и дражайшая душа Ему, нежели тѣло. О семъ всемъ просить насъ увѣщаваетъ Онъ Самъ: \textit{просите, и дастся вамъ; ищите, и обрящете; толцыте, и отверзется вамъ}\footnote{Мѳ.~7,~7.}. \textit{Приступите къ Нему, и просвѣтитеся, и лица ваша не постыдятся}, увѣщаваетъ насъ пророкъ Его\footnote{Пс.~33,~6.}. 3)~Не отказывалъ Хрістосъ никому бѣдному. Не откажемъ и мы никому просящему у насъ, по словеси Его: \textit{просящему у тебе дай}\footnote{Мѳ.~5,~42.}. Не откажемъ, что можемъ, подать, да тако Ему въ семъ дѣлѣ послѣдуемъ. Проситъ кто хлѣба у насъ, подадимъ хлѣбъ. Проситъ питія, подадимъ питіе. Проситъ одежды, подадимъ одежду. Проситъ денегъ, подадимъ. Проситъ квартеры и упокоенія въ домѣ нашемъ, подадимъ. Проситъ помощи и защищенія, подадимъ. Проситъ совѣта и наставленія, подадимъ. Проситъ утѣшенія, подадимъ. Проситъ избавленія, подадимъ. Проситъ послуженія, подадимъ. Проситъ и инаго чего, по волѣ Божіей и возможности нашей, подадимъ. Сего бо и Хрістосъ нашъ отъ насъ требуетъ, хрістіанине! Не требуетъ Онъ отъ насъ, чтобы мы чудеса творили, "--- сіе дѣло Божіе есть, а не наше, и намъ не возможное, ибо выше силъ нашихъ есть, "--- но требуетъ, чтобы мы другъ друга любили, и другъ друга миловали, и другъ другу руку помощи подавали. Сего и союзъ хрістіанства, и воля Его святая хощетъ, и намъ оттуду великая польза будетъ. \textit{Милуяй нищаго, взаимъ даетъ Богови: по даянію же его воздастся ему}. Что преславнѣе, какъ \textit{Богу взаимъ давать}, и что полезнѣе, въ руки давать Тому, у Котораго въ руцѣ все, и съ великимъ прибыткомъ воздаетъ дающему? О, благости и человѣколюбія Божія! Хрістіанине, все, что ни имѣемъ, или можемъ имѣть, Его есть. Но святое слово Его глаголетъ: \textit{милуяй нищаго, взаимъ даетъ Богови}. Тако Онъ поощряетъ насъ къ любви и милости, да и бѣдніи, которымъ Онъ, яко Отецъ чадолюбивый, состраждетъ, въ скорбехъ своихъ утѣшеніе получатъ; и милующіи ихъ и подающіи имъ руку помощи, мздовоздаянія своего, не лишатся. Можетъ Богъ, яко всесильный, отверзти сокровища Своя, и невидимою десницею Своею подать требованія бѣднымъ, и тако ихъ удовольствовать; но представляетъ ихъ очесамъ нашимъ, и указываетъ на нихъ, и велитъ намъ миловать ихъ, и Самъ за нихъ обѣщается воздати намъ, да и бѣдніи, о которыхъ Онъ жалѣетъ, яко милосердый, безъ удовольствія не останутся, и мы награжденіе сторицею отъ Него пріимемъ. Сотворимъ убо, возлюбленный хрістіанине, ближнимъ нашимъ милость, и всякому прошеніе его подадимъ, да услышимъ отъ Хріста Господа предъ всѣмъ міромъ пресладкій гласъ: \textit{пріидите благословенніи Отца Моего, наслѣдуйте уготованное вамъ царствіе отъ сложенія міра. Взалкахся бо, и дасте Ми ясти; возжадахся, и напоисте Мя; страненъ бѣхъ, и введосте Мене; нагъ, и одѣясте Мя; боленъ, и посѣтисте Мене; въ темницѣ бѣхъ, и пріидосте ко Мнѣ}\footnote{Мѳ.~25,~34--36.}. 4)~Сердца и нрава хрістіанскаго не имѣютъ, и отъ Хріста далеко отстоятъ тіи хрістіане, которыи милосердія и состраданія къ бѣднымъ не имѣютъ, и руки помощи въ нуждахъ не подаютъ имъ, но имѣніе свое на прихоти и роскоши иждиваютъ. Таковыи хрістіане жалѣютъ Хріста ради одѣть нагаго, или хижину состроить неимущему дома, или нѣсколько денегъ подать требующему на великую нужду, и прочая; но не жалѣютъ на прихоти и роскоши, на приданое дочерямъ и наслѣдіе сынамъ, и на снисканіе чести, и прочая, не жалѣютъ, говорю, многихъ сотенъ, или тысящей. Такая"=то вѣра ихъ, такое и хрістіанство! Ради Хріста подать жалѣютъ, но ради прихоти и суеты міра, и ради плоти и крови ничего изнурить не жалѣютъ. Кто что или кого любитъ, ради того ничего не жалѣетъ, и все ради его, что хощетъ, дѣлаетъ. Знаменіе есть несумнительное, что и сіи хрістіане міръ, плоть и кровь свою любятъ, а не Хріста, хотя Онъ и умеръ за нихъ. Чего ради и услышатъ предъ всѣмъ свѣтомъ страшный гласъ Его: \textit{идите отъ Мене проклятіи во огнь вѣчный, уготованный діаволу и аггеломъ его. Взалкахся бо, и не дасте Ми ясти; возжадахся, и не напоисте Мене; страненъ бѣхъ, и не введосте Мене; нагъ, и не одѣясте Мене; боленъ и въ темницѣ, и не посѣтисте Мене}\footnote{Мѳ.~25,~41--43.}. 5)~Горшій и развращеннѣйшій нравъ имѣютъ хрістіане тіи, которыи у хрістіанъ отнимаютъ и похищаютъ ихъ добро. Сюды надлежатъ беззаконніи судіи, который по мздѣ, а не по правдѣ судятъ, и бѣдныхъ проливаютъ слезы; надлежатъ господа помѣщики, который или оброками или работами несносными обременяютъ крестьянъ своихъ, и нужная къ пропитанію и содержанію ихъ отнимаютъ у нихъ; надлежатъ господа власти, которыи подчиненнымъ своимъ не даютъ, или не исполна даютъ опредѣленнаго имъ отъ Государя жалованья; надлежатъ тіи безсовѣстніи люди, которыи съ пожара похищаютъ, и хозяину оскорбленному большую придаютъ скорбь; надлежатъ купцы, которыи купующихъ обманываютъ, и вмѣсто надлежащей цѣны большую требуютъ цѣну; надлежатъ наемники, которыи, взявши достойную цѣну, не работаютъ за тую, или лѣниво работаютъ; надлежатъ хозяева, которыи нанимаютъ работать, но мзды наемникамъ не отдаютъ, или отдаютъ, но не исполна; надлежатъ тати, воры, хищники, грабители, которыи какимъ нибудь способомъ чуждое добро себѣ присвояютъ, или тайно похищаютъ, и проч. Вси таковыя развращенный нравъ имѣютъ и далеко отъ Хріста отстоятъ, и, какъ сказать истину, противники Хрістовы суть, и противу Хріста суть. \textit{Иже нѣсть со Мною, на Мя есть; и иже не собираетъ со Мною, расточаетъ}, глаголетъ Господь\footnote{Мѳ.~12,~30.}. Таковыи не со Хрістомъ суть: \textit{кое бо причастіе правдѣ къ беззаконію? или кое общеніе свѣту ко тьмѣ}\footnote{2~Кор.~6,~14.}? Убо противу Хріста суть. Страшно слово сіе, хрістіанине, но истинно. Страшенъ и судъ имъ будетъ. Аще бо недающіи отсылаются во огнь геенскій, что уже будетъ похищающимъ? Бѣдный хрістіанине! покайся, да не, яко вещество, во огнѣ геенскомъ будешь горѣть, но никогда не сгарать.

Видимъ въ святомъ Евангеліи, что Хрістосъ Господь въ домы входилъ тѣхъ людей, которыи Его къ себѣ звали, и ялъ у нихъ. О блаженніи тіи домы, которыи Царь небесный посѣщалъ, и, \textit{даяй пищу всякой плоти}, пищи въ нихъ вкушалъ! За счастіе люди имѣютъ царя земнаго въ домъ принять: коль несравненно большее счастіе есть принять въ домъ Царя небеснаго! Пріемшіи тогда Хріста въ домы своя не ино что о Немъ думали, какъ что Онъ былъ пророкъ, славный учитель и чудотворецъ; но не знали того, что Онъ есть пророковъ Господь, хотя свѣтъ божества Его изъ разныхъ чудесъ и преславныхъ дѣлъ Его, и всевѣдѣнія Его, и власти Его, и прочаго, яко солнце изъ лучей его, показывался. О, когда бы знали, что сей Гость, Котораго пріемлютъ, есть истинный Богъ, ходящій во образѣ человѣческомъ по земли: съ великимъ страхомъ и неизреченною радостію принимали бы Его! Но скрылся отъ нихъ тогда Сей Божественный и вѣчный свѣтъ. Хрістіанине! намъ благодатію Божіею блеснулъ свѣтъ Тотъ; мы нынѣ вѣруемъ и знаемъ, что Хрістосъ Господь и Богъ нашъ есть, Который пришелъ въ міръ грѣшники спасти: покаемся убо, и обратимся къ Нему; отверземъ домы сердецъ нашихъ, и со смиреніемъ попросимъ Его къ себѣ, да и насъ смиренныхъ и грѣшныхъ посѣтитъ. Хотя бо и не видимъ Его, однакожъ Онъ и нынѣ между вѣрными Своими обращается и человѣколюбно посѣщаетъ ихъ, по обѣщанію Своему: \textit{се Азъ съ вами во вся дни до скончанія вѣка}\footnote{Мѳ.~28,~20.}. Онъ, какъ видимъ во Евангеліи, никѣмъ не гнушался, кто Его ни звалъ къ себѣ; не возгнушается и нами, яко человѣколюбецъ, аще покаяніемъ очистимъ домы наша, и со смиреніемъ попросимъ Его. Но Онъ и Самъ готовъ и радъ къ намъ пріити, аще только увидитъ угодное Себѣ мѣсто въ насъ, глаголетъ бо Самъ: \textit{се стою при дверехъ, и толку: аще кто услышитъ гласъ Мой, и отверзетъ двери, вниду къ нему, и вечеряю съ нимъ, и той со Мною}\footnote{Апок.~3,~20.}. Гость великъ есть и славенъ, и любезенъ, и радостотворенъ, яко Богъ. Учрежденія, пищи и питія отъ насъ не потребуетъ. Самъ предложитъ трапезу намъ, не трапезу, каковую люди представляютъ, но трапезу укрѣпляющую, утѣшающую, увеселяющую души наша, и сладость вѣчнаго живота имѣющую. На сей пресладкой вечери отчасти вкусимъ сладости небесныя трапезы, которой ангели Его святіи вкушаютъ. Очистимъ убо домы сердецъ нашихъ истиннымъ покаяніемъ, и изженемъ шумъ различныхъ попеченій и прихотей міра сего, да услышимъ пресладкій гласъ Его при дверехъ нашихъ, и отверземъ двери Ему, и тако пріидетъ преславный Гость Сей къ намъ. Господи, сладчайшій Іисусе, Пастырю и Посѣтителю душъ нашихъ! не мини и мене нищаго и убогаго раба Твоего, но человѣколюбіемъ Твоимъ посѣти смиренную мою душу; посѣти того грѣшника, ради котораго честную кровь Твою проліялъ еси и крестную смерть претерпѣлъ еси. \textit{Готово сердце мое, Боже, готово сердце мое}.

Видимъ въ святомъ Евангеліи, что Іисусъ и грѣшникамъ приступенъ былъ. \textit{Бяху приближающеся Ему вси мытаріе и грѣшницы, послушати Его}\footnote{Лук.~15,~1.}. И не гнушался съ ними возлежать и хлѣба вкушать. \textit{И бысть Ему возлежащу въ дому, и се мнози мытари и грѣшницы пришедше возлежаху со Іисусомъ и со ученики Его}\footnote{Мѳ.~9,~10.}. Видишь отсюду, хрістіанине: 1)~Херувимамъ и серафимамъ неприступный, и отъ нихъ со страхомъ и благоговѣніемъ покланяемый и воспѣваемый, и во свѣтѣ живый неприступнѣмъ, явнымъ грѣшникамъ приступенъ учинился. О благости и человѣколюбія Твоего, Іисусе Боже нашъ! Слава благости Твоей, слава человѣколюбію Твоему, слава щедротамъ Твоимъ, слава снисхожденію Твоему! Здѣ воистину благовременно со пророкомъ удивиться и воскликнуть: \textit{Господи! что есть человѣкъ, яко познался еси ему, или сынъ человѣчь, яко вмѣняеши его}\footnote{Пс.~143,~3.}. Слышите, грѣшники, и разумѣйте! Богъ великій и непостижимый, по Своему человѣколюбію, приступенъ намъ сдѣлался: \textit{бяху приближающеся Ему вси мытаріе и грѣшницы}. Познали грѣшники Спасителя своего; познали немощніи Врача своего; познали заблуждшія овцы добраго Пастыря своего; познали сѣдящіи во тьмѣ свѣтъ свой; познали погибшіи животъ свой; познали бѣдніи блаженство свое, и \textit{приближалися Ему}. Сила Его божественная, и человѣколюбивый зракъ лица Его святаго, и небесное и пресладкое слово Его святое привлекало ихъ.

Ничего бо не видѣлось во Іисусѣ, какъ только смиреніе, кротость, долготерпѣніе, милосердіе, состраданіе и человѣколюбіе. Грѣшники, что и мы дремлемъ, что медлимъ, почто и мы не приступаемъ ко Іисусу? почто во тьмѣ пребываемъ? почто къ свѣту Тому не приближаемся? почто неисцѣльны пребываемъ, и не ищемъ исцѣленія отъ небеснаго Врача? Ну"=жъ, съ дерзновеніемъ и вѣрою приступимъ къ Нему, и лица наша не постыдятся. 2)~Отсюду самымъ дѣломъ показалося, что \textit{вѣрно слово и всякаго пріятія достойно, яко Хрістосъ Іисусъ пріиде въ міръ грѣшники спасти, и пріиде Сынъ человѣческій взыскати и спасти погибшаго}\footnote{1~Тим.~1,~15; Лук.~19,~10.}. Грѣшниковъ спасти пришелъ Іисусъ Хрістосъ, грѣшникамъ и приступенъ учинился. Но надобно грѣшникамъ и тому внимать, что глаголетъ Онъ: \textit{покайтеся}, и паки: \textit{не пріидохъ призвати праведники, но грѣшники на покаяніе}\footnote{Мѳ.~4,~17; 9,~13.}. Должно убо грѣшникамъ каятися и исправить себе, и тако спасетъ ихъ Хрістосъ. 3)~Отсюду грѣшникамъ, въ истинномъ покаяніи находящимся, утѣшеніе проистекаетъ. Аще бо грѣшниковъ пришелъ Онъ спасти, то и тебе, понеже и ты единъ отъ грѣшниковъ. Аще погибшихъ пришелъ взыскати, то и тебе взыщетъ, понеже и ты единъ отъ погибшихъ. 4)~Видимъ, что истинная святость никакими грѣшниками не гнушается. Истинно святый грѣха ненавидитъ, но не грѣшниковъ; грѣхами гнушается, но грѣшниками не гнушается. Книжники и фарисеи, мнимою святостію надменныи, гнушалися грѣшниками; откуду и укоряли апостоловъ: \textit{почто со грѣшники Учитель вашъ ястъ и піетъ}\footnote{Мѳ.~9,~11.}? Но Хрістосъ, святыхъ Святѣйшій и Источникъ святыни, никакими грѣшниками не гнушался. Тому послѣдуютъ и святіи раби Его, которыи грѣховъ отвращаются, но не грѣшниковъ; грѣховъ ненавидятъ, но грѣшникамъ соболѣзнуютъ и состраждутъ. Да постыдится убо надменная фарисейская гордость, которая подобными себѣ грѣшниками гнушается! 5)~Отсюду учатся пастыри со грѣшниками обращаться и всякимъ образомъ приводить ихъ къ покаянію, но грѣхамъ ихъ не сообщаться. Пастырское дѣло есть искать заблуждшія овцы. И добрый лѣкари тамъ наипаче обращаются, гдѣ болящіе лежатъ. \textit{Не требуютъ бо здравіи врача, но болящіи}\footnote{12.}. Отсюду учатся господа, князи, вельможи, власти, и прочіи по міру сему славніи, и вси хрістіане, никакимъ человѣкомъ, и самымъ подлѣйшимъ, не гнушаться и не презирать его, и не уничтожать. Хрістосъ, Царь царей и Господь господствующихъ, никѣмъ не гнушался и никого не презиралъ: кольми паче человѣкамъ, какъ бы они въ мірѣ славны ни были, подобными себѣ людьми гнушаться не должно. Вси бо человѣцы есмы, вси сродное естество имѣемъ, вси славныи такую же душу и тѣло имѣютъ, какъ и подлыи. И часто бываетъ, и по большей части, что подлый "--- честнѣйшій у Бога, нежели славный, и лучшій предъ очами Божіими "--- рабъ и крестьянинъ, нежели господинъ его, князь и вельможа. Драгость бо и честность человѣческая судится не отъ блистанія злата и сребра, не отъ именъ и титуловъ міра сего, но отъ честности души и драгости добродѣтели. И Богъ судитъ по внутренности, а не по внѣшности. Лучшій и дражайшій у Бога единъ благочестивый и богобоящійся человѣкъ, нежели тысящи нечестивыхъ, хотя бы и какъ боги въ мірѣ семъ они имѣлись и почитались. Да взираютъ убо вси хрістіане на Хріста, Господа своего, живый смиренія и снисхожденія образъ, и Тому да послѣдуютъ, и никого въ презрѣніи да не имѣютъ. 7)~Отсюду обличается бѣсовская гордость тѣхъ господъ, князей и вельможъ, которыи слугъ своихъ и крестьянъ за подножіе имѣютъ, и когда бы не нижше псовъ своихъ, "--- и богатыхъ людей, которыи нищихъ и убогихъ презираютъ. \textit{Почто гордится земля и пепелъ?} почто грѣшникъ грѣшника презираетъ? почто бѣдный бѣднымъ гнушается? Вси человѣцы равны суть, хотя всѣмъ не равное счастіе въ мірѣ семъ. Князи, вельможи, господа и славныи! посмотрите во гробы предковъ вашихъ, и сами признаете истину сію, что вси человѣки, какъ славныи, такъ и подлыи, равны суть. Души благородныя свойство есть бѣдному и грѣшному соболѣзновать и сострадать, а не презирать и гнушаться имъ.

Видимъ въ святомъ Евангеліи, что Хрістосъ Господь нашъ много поношенія и хуленія, живучи на земли, неповинно претерпѣлъ, и претерпѣлъ отъ Своихъ людей, со всякою кротостію претерпѣлъ. Видишь здѣ хрістіанине: 1)~Богъ, на престолѣ славы сѣдяй и отъ всѣхъ небесныхъ Силъ покланяемый и почитаемый, какъ во плоти явился, насъ ради, такъ насъ ради хуленіе и поношеніе отъ устъ грѣшныхъ претерпѣлъ. Грѣхи наши виновны тому. Грѣховъ бо нашихъ ради, которыми мы Богу нашему досадили, милостивый Іисусъ, Избавитель нашъ, всякую хулу претерпѣлъ. О благости и человѣколюбія! Будемъ убо Ему за сію любовь Его къ намъ благодарны. 2)~Видимъ, что міръ ослѣпленный правды и истины не любитъ. Хрістосъ Господь, вѣчная Истина, ничего не дѣлалъ, какъ только правду и истину, и все къ созиданію человѣческому дѣлалъ, и ничего не училъ, какъ только правды и истины. Сего ради отъ міра, прелестію ослѣпленнаго, ненавидимъ и хулимъ былъ. Больныя глаза свѣта не любятъ, и неправда правду и ложь истину ненавидитъ. 3)~Отсюду послѣдуетъ, что пастыри и учители Истины и небеснаго слова проповѣдники, ненависти и хуленію злыхъ людей подлежатъ, понеже злобу и прелесть и темныя дѣла ихъ свѣтомъ небеснаго ученія обличаютъ. \textit{Обличеніе бо нечестивому раны ему}\footnote{Притч.~9,~7.}. Обличеніемъ нечестивый, какъ рожномъ, ударяется. О семъ примѣры пророковъ, апостоловъ и всѣхъ вѣрныхъ святителей, въ древности пожившихъ, свидѣтельствуютъ, которыи вси за Слово Божіе и истину гоненіе и поношеніе претерпѣли. Чего и нынѣ истины проповѣдникамъ ожидать отъ злаго міра? Аще кто богобоящійся хрістіанинъ есть, сему совѣтую, хулѣ, на пастырей и учителей разсѣиваемой, не вѣрить, но паче хулящимъ уста заграждать. Здѣ хитрость діавольская кроется. Онъ научаетъ своихъ служителей разсѣвать хулы на проповѣдниковъ истины, чтобы люди о нихъ злѣ думали и слову ихъ не вѣрили, и тако бы погибли. Сіе намѣреваетъ злый духъ, и тако хулы и всякія клеветы злымъ его совѣтомъ разсѣваются на пастырей. Берегись сего, хрістіанине! Аще убо кого услышиши хулящаго и клевету разсѣвающаго на пастыря, знай точно, что устами его, яко орудіемъ своимъ, злый духъ смрадъ свой извергаетъ. 4)~Такому же поношенію и ненависти подлежатъ и вси истинныи хрістіане. Видимъ въ древней исторіи, что язычники страшныя и различныя клеветы на хрістіанъ вымышляли и изобрѣтали, и всякими поносными именами ихъ называли. Сіе, какъ можно видѣть, дѣйствомъ діавольскимъ дѣлалось, чтобы хрістіане отъ хрістіанской вѣры отступали, а идолопоклонники къ ней приступать боялися. Но премудростію и силою Божіею разсыпался злый злаго духа совѣтъ. Чимъ болѣе клеветъ, поношеній и гоненій возставлялъ онъ на хрістіанъ, тѣмъ болѣе умножалось хрістіанъ. Подобныя клеветы и поношенія и нынѣ вымышляются на хрістіанъ не отъ идолопоклонниковъ уже, но отъ ложныхъ хрістіанъ и любителей міра сего и прелести его, и пышности и гордости его. Отъ сей лжебратіи много всякихъ клеветъ и поношеній страждутъ хрістіане; то въ тыя, то въ другія поруганія одежды одѣваютъ ихъ. И кромѣ обычныхъ и нечестивыхъ поношеній, различныя на благочестивыя души изобрѣтаютъ хулы и посмѣянія. Ежели примѣтятъ они, что хрістіанинъ отъ роскошей и темныхъ дѣлъ ихъ удаляется, то у нихъ \textit{раскольникъ}. Когда смиренно живетъ и обращается, и щегольства ненавидитъ, то у нихъ \textit{ханжа}. Когда уединяется ради покаянія и убѣжанія отъ грѣха (удобнѣе бо во уединеніи каяться и грѣха берещися, нежели въ народѣ), то у нихъ \textit{святоша}. Когда за грѣхи кается, сокрушается, сѣтуетъ и воздыхаетъ, то у нихъ \textit{меланхоликъ}. Кто милостыню даетъ, тотъ у нихъ \textit{тщеславецъ и лицемѣръ}. Когда узнаютъ, что хрістіанинъ часто Богу молится, тутъ они уста своя разверзаютъ: вотъ"=де \textit{богомолъ}. Когда хрістіанинъ, по регулѣ евангельской, обидящему не отмщеваетъ, тутъ они на падаютъ на него: вотъ"=де какой \textit{дуракъ}, не умѣетъ за себе стоять. Когда, по регулѣ тогожде Евангелія, расточаетъ имѣніе и даетъ убогимъ, тутъ они подымаются на того, и укоряютъ: обезумился"=де онъ; что предки его собрали, то онъ расточаетъ. Сія и прочая подобная симъ хуленія и посмѣянія изрыгаютъ нечестивыя уста на душу благочестивыхъ. И сіе"=то есть, что Хрістосъ Господь нашъ сказалъ къ нашему утѣшенію: \textit{аще господина дому веельзевула нарекоша, кольми паче домашнія Его}\footnote{Мѳ.~10,~25.}. Возненавидѣли Хріста, Господина дому: ненавидятъ и хрістіанъ, домашнихъ Его. Похулили Хріста и поругались Хрісту, Господину дома: хулятъ и хрістіанъ, домашнихъ Его, и ругаются имъ. Изгнали Хріста, Господина дому: изгонятъ и хрістіанъ, домашнихъ Его. \textit{Нѣсть бо рабъ болій господина своего}\footnote{Іоан.~15,~20.}. Таковое состояніе хрістіанъ въ нынѣшнемъ вѣкѣ! Хрістіане не суть отъ міра сего, якоже о нихъ глаголетъ Самъ Хрістосъ: \textit{не суть отъ міра, яко же и Азъ отъ міра нѣсмь}\footnote{17,~14.}. Сего ради изгоняетъ и ругаетъ ихъ міръ, яко не своихъ. И сіе такожде дѣйствуется хитростію діавольскою, дабы душу благочестивую съ добраго пути совратить. Сей есть всехитрый лукаваго духа вымыслъ. \textit{Не даждь во смятеніе ноги твоея}, возлюбленне, \textit{ниже воздремлетъ храняй тя. Се не воздремлетъ, ниже уснетъ храняй Израиля. Господь сохранитъ тя, Господь покровъ твой на руку десную твою. Во дни солнце не ожжетъ тебе, ниже луна нощію. Господь сохранитъ тя отъ всякаго зла, сохранитъ душу твою Господь; Господь сохранитъ вхожденіе твое и исхожденіе твое отъ нынѣ и до вѣка}\footnote{Пс.~120,~3--8.}. 5)~Хрістіанамъ не прилично хулящихъ хулить, и укоряющихъ укорять, и вѣрѣ хрістіанской противно. Сего ради смотрѣть имъ должно на живый кротости примѣръ Господа своего, Іисуса, \textit{Иже укоряемъ противу не укоряше, и, яко агнецъ предъ стригущимъ его, безгласенъ былъ}. Зло зломъ и хула хулою не побѣждается, но паче раздражается и свирѣпѣетъ. И хрістіанская побѣда состоитъ не во отмщеніи, но въ кротости и терпѣніи. Того ради хрістіане, яко овцы Хрістовы, смиреніемъ и кротостію да побѣждаютъ хулителей своихъ, и да не воздаютъ имъ зла за зло и досажденія за досажденіе; и не гнѣваться на нихъ, но паче сожалѣть имъ должны, яко таковыми хулителями и поносителями діаволъ обладаетъ, "--- и молиться за нихъ, да не вѣчными его плѣнниками будутъ. 6)~Хрістіане въ семъ злостраданіи да внимаютъ утѣшительному слову Хрістову: \textit{аще міръ васъ ненавидитъ, вѣдите, яко Мене прежде васъ возненавидѣ. Аще отъ міра бысте были, міръ убо свое любилъ бы: якоже отъ міра нѣсте, но Азъ избрахъ вы отъ міра, сего ради ненавидитъ васъ міръ. Поминайте слово, еже Азъ рѣхъ вамъ: нѣсть рабъ болій господа своего. Аще Мене изгнаша, и васъ изженутъ}\footnote{Іоан.~15,~18--20.}. Утвердите убо сердца ваша, возлюбленніи! Пусть любители міра, что хотятъ, вамъ дѣлаютъ: вы тѣмъ довольни будите, что вы \textit{Хрістовы}. Пусть они ругаютъ васъ, вы \textit{домашніи Хрістовы}. Велика слава и честь быть Хрістовымъ, хотя сіе сокровище предъ міромъ и сокровенно есть. Пусть ненавидитъ васъ міръ, Богъ любитъ васъ. Пусть проклинаетъ міръ, Богъ васъ благословляетъ. \textit{Прокленутъ тіи, и Ты благословиши}\footnote{Пс.~113,~28.}. Симъ довольни будите и утѣшайтеся. Міръ лукавый на зло вамъ замышляетъ, но Богъ обращаетъ вамъ тое въ добро ваше. Фараонъ мучитель озлоблялъ Израиля, но Израиль съ сребромъ и златомъ и съ сокровищемъ изъ Египта вышелъ. Терпите и вы все, что вамъ ни дѣлаетъ міръ. И вы изыдите отъ міра съ сокровищемъ духовнымъ, сокровищемъ не міра сего, и съ тѣмъ явитеся небесному своему Отцу. 7)~Ругатели и хулители пріимутъ жребій свой. Они шумятъ яко вода во время весны, но и уничижаются, яко вода мимотекущая; возносятся яко дымъ, но и исчезаютъ яко дымъ. Сіе ихъ свойство, сей и жребій ихъ есть. Богъ праведный слышитъ хулы ихъ и поношенія, которыми души благочестивыхъ уязвляютъ; слышитъ, и въ книгѣ своей записываетъ, и представитъ ихъ предъ лицемъ ихъ, и укажетъ имъ поруганнаго и посмѣяннаго ими не въ такомъ уже образѣ, въ какомъ отъ нихъ посмѣваемъ и поношаемъ былъ, но въ славѣ избранныхъ Своихъ и въ числѣ сыновъ Божіихъ: вотъ"=де вашъ раскольникъ, вашъ ханжа, вашъ меланхоликъ, вашъ плутъ и мотъ! Смотрите на него, и примѣчайте, тотъ ли, надъ которымъ вы смѣялись, котораго проклинали, котораго хулили, которому ругались. О, каковый срамъ и страхъ обыметъ ругателей тогда! каковый стыдъ покрыетъ лица ихъ! \textit{Тогда станетъ въ дерзновеніи мнозѣ праведникъ предъ лицемъ оскорбившихъ его, и отметающихъ труды его. Видящіи смятутся страхомъ тяжкимъ, и ужаснутся о преславномъ спасеніи его. И рекутъ въ себѣ кающеся, и въ тѣснотѣ духа воздыхающе: сей бѣ, егоже имѣхомъ нѣкогда въ посмѣхъ, и въ притчу поношенія. Безумніи житіе его вмѣнихомъ неистово, и кончину его безчестну. Како вмѣнися въ сынѣхъ Божіихъ, и въ святыхъ жребій его есть}\footnote{Прем. Сол.~5,~1--5.}? Бѣдный человѣкъ! Богъ хощетъ, чтобы и ты въ смиреніи и покаяніи жилъ, и тако бы спаслся, а не смиреннымъ и благочестивымъ ругался ты. Покайся убо, и не падетъ гнѣвъ Божій на тебе.

Видимъ въ святомъ Евангеліи, что Хрістосъ Господь нашъ, проповѣдуя Евангеліе царствія Божія словомъ, показалъ тое самымъ дѣломъ и вещію, ко утвержденію нашея вѣры и утѣшенію: \textit{поемши} трехъ учениковъ Своихъ, \textit{Петра, Іакова и Іоанна, возвелъ на гору высокую, и тамо показалъ} имъ нѣкую \textit{часть славы царствія Своего}. Возлюбленный хрістіанине! Взыдемъ и мы нынѣ умомъ нашимъ на святую оную гору, на которой стояли пречистыя ноги Спасителя нашего Бога, и увидимъ тамо преславное и отъ начала міра неслыханное позорище; увидимъ, что тамо воплотившагося нашего Бога, Который по земли ходилъ таковымъ образомъ, какъ и прочіи человѣцы, сіяетъ лице яко солнце, и ризы блистаютъ яко свѣтъ; видимъ, что тамо предстоятъ Ему два славнѣйшіе пророка, Моисей и Илія, и съ Нимъ бесѣдуютъ; увидимъ тамо, что Петръ съ своими товарищами видѣніемъ славы тоя такъ обрадовался, что съ горы той и сойти не хотѣлъ, но въ радости духа своего восклицаетъ ко Хрісту: \textit{Господи, добро есть намъ здѣ быти!} Тамо услышимъ гласъ Божій изъ облака, о Сынѣ Своемъ свидѣтельствующій: \textit{Сей есть Сынъ Мой возлюбленный, о Немже благоволихъ: Того послушайте}\footnote{Мѳ.~17,~1--7.}. Смотря на сіе преславное позорище умнымъ нашимъ окомъ, помедлимъ тутъ, и въ пользу нашу размыслимъ тое, да и наша сердца видѣніемъ и размышленіемъ преславнаго того позорища обрадуются. 1)~Видя боголѣпную славу Хріста Бога нашего на Ѳаворѣ, видимъ и тое, какъ Онъ насъ ради смирилъ Себе. Тойже Хрістосъ и по земли ходилъ, и страдалъ, и во образѣ человѣческомъ, подобнымъ намъ, между человѣками обращался, Который на горѣ святѣй былъ. Таяжде и слава Его была всегда съ Нимъ, но отъ очей человѣческихъ сокровенна. Кто бы моглъ и дерзнулъ къ Нему приступить, ежели бы Онъ хотя въ такой славѣ, каковую на горѣ святѣй показалъ, обращался и ходилъ по земли? Но, сокрывши славу Свою подъ плотію, ходилъ въ смиреніи, яко единъ отъ простыхъ человѣкъ, и тако всѣмъ бѣднымъ и грѣшникамъ приступенъ былъ. Видишь, что Онъ, \textit{сіяніе славы Отчія и образъ ѵпостаси Его и во свѣтѣ живый неприступнѣмъ}, въ смиренномъ человѣческомъ зракѣ обращался и ходилъ по земли. Слава смиренію Его, слава снисхожденію Его! 2)~Предстали Ему тамо два великіе пророка, Моисей и Илія: Моисей, законодавецъ и вождь во исходѣ Израилевѣ изъ Египта, Илія, ревнитель по Господѣ Саваоѳѣ, и взятый огненною колесницею яко на небо. Сихъ пророковъ и Іудеи и весь Израиль за славнѣйшихъ пророковъ имѣли. Тіи явилися и предстали Хрісту на горѣ святѣй. Видишь здѣ, что Хрістосъ есть Господь пророковъ, Который во образѣ человѣческомъ явился на земли, и есть истинный Мессія, пророками проповѣданный. Слава явившемуся во плоти Господу, и пришедшему спасти грѣшниковъ Богу! 3)~Прославленная на горѣ святѣй плоть Хрістова увѣряетъ и обнадеживаетъ насъ, что подобная Хрісту слава будетъ избранныхъ Божіихъ въ вѣчной жизни, якоже Самъ Хрістосъ словомъ объявилъ: \textit{тогда праведницы просвѣтятся яко солнце во царствіи Отца ихъ}\footnote{Мѳ.~13,~43.}. И Апостолъ писалъ: \textit{Иже преобразитъ тѣло смиренія нашего, яко быти сему сообразну тѣлу славы Его}\footnote{Филип.~3,~24.}. И паки другій Апостолъ глаголетъ: \textit{возлюбленніи! нынѣ чада Божія есмы, и не у явися, что будемъ: вѣмы же, яко егда явится, подобни Ему будемъ}\footnote{1~Іоан.~3,~2.}. Хрістіанине! увѣрися въ семъ, и взирай на тую славу. 4)~Петръ Апостолъ, съ сущими съ нимъ, такъ сильно обрадовался, хотя нѣкую часть увидѣлъ славы оныя: какая уже радость въ будущей жизни изліется въ сердца избранныхъ Божіихъ, когда не часть славы, но вся она открыется! Како сильно возрадуются, и радоватися будутъ во вѣки, когда Бога лицемъ къ лицу и славу Хрістову увидятъ, и безъ конца и сытости будутъ видѣть! Сея радости и сладости малую частицу и нынѣ святіи Божіи на земли чувствуютъ, и аки малѣйшія крупицы, отъ небесныя трапезы падающія, вкушаютъ, и гортани душъ своихъ услаждаютъ. 5)~Оная слава и радость поощреніемъ должна быть людемъ благочестивымъ къ подвигу доброму, и пребыванію въ томъ до конца, по словеси Хрістову: \textit{буди вѣренъ даже до смерти, и дамъ ти вѣнецъ живота}\footnote{Апок.~2,~10.}. Тако воиновъ надежда побѣды и славы поощряетъ къ подвигу противъ враговъ; тако земледѣльцевъ надежда плодовъ поощряетъ къ земледѣльству и трудамъ; тако купцовъ поощряетъ надежда богатства странствовать и скитаться по незнаемымъ и опаснымъ мѣстамъ; тако учениковъ надежда разума поощряетъ трудиться въ наукахъ; тако и хрістіанина благочестиваго надежда будущія славы и радости должна поощрять къ непремѣнному за благочестіе подвигу и теченію къ вышнему званію. Хрістіанине! въ скукѣ и нуждѣ твоей возводи умъ твой туды, гдѣ Хрістово лице просвѣтилося яко солнце, и ризы были бѣлы яко свѣтъ; и туды, гдѣ праведницы сіяютъ яко свѣтила, и Бога видятъ лицемъ къ лицу, и человѣки любезное дружество имѣютъ со ангелами, и въ радости духа непрестанно поютъ: \textit{аллилуіа}. Туды возводи умъ твой, и утвердишися въ вѣрѣ и подвигѣ твоемъ.

\textit{Блажени живущіи въ дому Твоемъ: въ вѣки вѣковъ восхвалятъ Тя, Господи}\footnote{Пс.~83,~5.}. 6)~Ко оной славѣ и радости возведите очи свои вы, которыи земная, а не горняя мудрствуете; которыи желаете и ищете въ мірѣ семъ обогатитися, прославитися и въ честь произойти; которыи любите въ богатыхъ и красныхъ домахъ жить, цугами и высокими каретами проѣзжаться, въ порфиры и виссоны облачаться, и тако въ мірѣ семь царствовать. Не широкимъ, но \textit{тѣснымъ путемъ}, и не пространными, но \textit{узкими вратами входятъ туды}\footnote{Мѳ.~7,~13 и 14.}. Не роскошами, но скорбьми, и скорбьми многими въ царствіе Божіе входятъ, якоже Апостолъ сказалъ: \textit{многими скорбьми подобаетъ намъ внити въ царствіе Божіе}\footnote{Дѣян.~14,~22.}. И святый Іоаннъ слышалъ гласъ съ небесе, о избранныхъ Божіихъ свидѣтельствующій тако: \textit{сіи суть, иже пріидоша отъ скорби великія}\footnote{Апок.~7,~14.}. Видите, что отъ скорби, а не отъ роскоши туды приходятъ. Роскоши и веселости міра сего не въ царство Божіе, но во адъ ведутъ, якоже примѣръ евангельскаго богача свидѣтельствуетъ, который \textit{облачашеся въ порфиру и виссонъ, веселяся на вся дни свѣтло}\footnote{Лук.~16,~19.}. Осмотритеся убо, какимъ путемъ и куды идете вы. Возлюбленніи! вы хрістіане, вы позваны словомъ Божіимъ къ вѣчнымъ благимъ, а не къ временнымъ; въ вѣчной жизни уготована намъ вся благая отъ небеснаго Отца: тѣхъ поищемъ со усердіемъ, а мірскихъ благихъ, яко странники, со страхомъ и опасностію, и къ нуждѣ нашей, а не къ роскоши, да употребляемъ. Кто великаго желаетъ, тотъ о маломъ нерадитъ, и кто вѣчнаго блаженства ищетъ, тотъ о всемъ временномъ небрежетъ. \textit{Ничтоже внесохомъ въ міръ сей, явѣ, яко ниже изнести что можемъ. Имѣюще же пищу и одѣяніе, сими довольни да будемъ}\footnote{1~Тим.~6,~7 и 8.}. 7)~Тудыже возведите очи и вы, которыи любодѣяніемъ и прелюбодѣяніемъ души и тѣлеса ваши оскверняете; вы, которыи чужое добро похищаете, грабите и крадете; вы, которыи языкомъ своимъ льстите и лукавнуете; вы, которыи ближняго своего оклеветаете, и хульными словами яко стрѣлами, его уязвляете; и прочіи, законъ Божій безстрашно нарушающіи, "--- возведите очи ваши туды и осмотритеся, какой славы вы лишаетеся ради нераскаяннаго вашего житія. \textit{Не имать} туда \textit{внити всяко скверно, и творяй мерзость и лжу... Внѣ псы и чародѣи, и любодѣи и убійцы, и идолослужители и всякъ любяй и творяй лжу}\footnote{Апок.~2,~27; 22,~15.}. 8)~Вси, хрістіане, возведемъ очи наши и мы туды, и осмотримся, како мы обращаемся въ мірѣ семъ. Позваны мы словомъ Божіимъ, и святымъ крещеніемъ обновлены къ великой оной славѣ: достойно ли убо званія онаго живемъ, осмотримся. Хрістіанина не исповѣданіе едино, но и вѣра и нравъ, вѣрѣ сообразный, дѣлаетъ хрістіаниномъ истиннымъ. Исповѣдуемъ Хріста Сына Божія; но слушаемъ ли Хріста, Котораго исповѣдуемъ? Знаемъ и вѣруемъ, что Богъ послалъ Сына Своего къ намъ спасти насъ, но слышимъ, что тойжде Богъ съ небесе Святаго Своего глаголетъ намъ о Немъ: \textit{Того послушайте}\footnote{Мѳ.~17,~5.}. Слушаемъ ли убо Его, когда хощемъ Его благодатію спастися? "--- осмотримся. Видимъ, какъ выше сказано, что сообразна будетъ слава избранныхъ Божіихъ славѣ Хрістовой; сего ради, когда хощемъ славу оную получить, то должны мы и здѣ, въ житіи семъ, сообразны Ему быть. Когда тамо хощемъ съ Нимъ быть, то и здѣ не должны отъ Него отлучаться, но неотлучно съ Нимъ здѣ пребывать. Когда тамо хощемъ подобными Ему быть, то должны и здѣ, въ житіи семъ, подобными Ему быть. Когда хощемъ въ вѣчный животъ за Нимъ и чрезъ Него пріити, то должны за Нимъ слѣдовать Его путемъ, и подражать Его смиренію, любви, терпѣнію и кротости. Когда хощемъ вѣчнаго Его царствія участниками быть, то должны быть и страданія и терпѣнія Его участниками. Когда хощемъ съ Нимъ прославитися, то должны и страдать съ Нимъ, и поношеніе Его на себѣ носить, по Писанію: \textit{съ Нимъ страждемъ, да и съ Нимъ прославимся}\footnote{Римл.~8,~17.}. Когда хощемъ \textit{ясти и пити на трапезѣ Его во царствіи Его}, то должны и здѣ, въ мірѣ семъ, съ Нимъ \textit{вкушать желчи} горести. Тако, когда здѣ сообразны Ему будемъ, то будемъ сообразны и тамо. Видѣли мы душевными нашими очами на Ѳаворѣ славу Хрістову, и съ Петромъ Апостоломъ нѣсколько обрадовались; но когда хощемъ тую славу получить, да послѣдуемъ Хрісту и до Голгоѳы: \textit{да исходимъ къ Нему внѣ стана, поношеніе Его носяще}\footnote{Евр.~13,~13.}. Вси хотятъ со Хрістомъ царствовать, но не вси хотятъ со Хрістомъ скорби терпѣть. Видимъ о Хрістѣ написанное: \textit{не сія ли подобаше пострадати Хрісту, и внити въ славу Свою}\footnote{Лук.~24,~26.}? Но видимъ написанное и о хрістіанѣхъ: \textit{многими скорбьми подобаетъ намъ внити въ царствіе Божіе}\footnote{Дѣян.~14,~22.}. Хрістіане не на оперы, не на танцы, не на банкеты, не на веселости міра сего позваны, но на покаяніе и скорби и кресты. Надобно всякому свой крестъ нести и Хрісту, понесшему крестъ, послѣдовать, и тако за Нимъ и чрезъ Него въ царствіе Божіе внити. Будетъ хрістіанамъ слава, честь, богатство, наслѣдіе, блаженство, утѣшеніе, радость, веселіе и великая вечеря, услаждающая ихъ, но въ будущемъ вѣкѣ, хотя и понынѣ они безъ утѣшенія не оставляются. И самыя бо слезы ихъ, которыя отъ сокрушеннаго духа источаютъ, вмѣсто прохлажденія и утѣшенія бываютъ имъ. Они скорбятъ и радуются, печалятся и утѣшаются, плачутъ и веселятся. Имѣютъ бо присутствующаго Утѣшителя, Который, яко благоутробенъ, въ скорбехъ ихъ человѣколюбно утѣшаетъ ихъ. \textit{Молю васъ: подобни мнѣ бывайте, якоже азъ Хрісту}, глаголетъ Апостолъ\footnote{1~Кор.~4,~16.}.

Видимъ въ святомъ Евангеліи, что Хрістосъ Господь нашъ, хотячи показать намъ совершеннѣйшій \textit{смиренія образъ}, умылъ ноги ученикамъ Своимъ и апостоламъ. Возлюбленный хрістіанине! взыдемъ нынѣ умомъ нашимъ въ горницу оную, въ которой тайная вечеря совершалася, и увидимъ тамо преславное чудо. Она покажетъ намъ, како Господь славы востаетъ съ вечери, како слагаетъ ризы Своя, како пріемлетъ лентіонъ и препоясуется, како вливаетъ воду во умывальницу, како преклоняетъ колѣна и приступаетъ ко умовенію ногъ ихъ, и склоняется и умываетъ ноги ихъ и отираетъ лентіемъ, имже бѣ препоясанъ, и отъ одного къ другому, и отъ другаго къ третіему, и тако послѣдовательно ко всѣмъ приступаетъ и склоняется, и каждаго ноги въ святыя руки Своя пріемлетъ, и водою омываетъ, и полотномъ отираетъ, и подлѣйшее рабское дѣло совершаетъ Богъ, во образѣ человѣческомъ явившійся\footnote{Іоан.~15,~4--15.}.

Видимъ здѣ, хрістіанине; 1)~Сколько великъ и высокъ Хрістосъ Господь нашъ, столько велико и глубоко смиреніе его. Высочество и величество Его непостижимо, непостижимо убо и смиреніе Его. Николиже тако видѣно было на земли. Нигдѣ не слышимъ, чтобы человѣкъ "--- царь подданнымъ своимъ рабамъ ноги умылъ. Отъ вѣка не слыхано то, ниже видѣно. Богъ, Царь небесней, явился во образѣ человѣческомъ, и сотворилъ дѣло тое на земли, да всѣмъ \textit{образъ смиренія} покажетъ. Услышите сія вси языцы, внушите вси живущія по вселеннѣй! Царь небесный и Господь славы ноги рабамъ Своимъ умылъ… О славное и чудное позорище! Господь во образѣ раба служитъ. Вси чудны и дивны дѣла Твои, Господи Іисусе Боже нашъ! Чудно небо и земля съ исполненіемъ ихъ; чудна дѣла Твоя, которыя видимъ въ словѣ Твоемъ святомъ; но чуднѣйшее дѣло сіе, которое насъ ради сотворилъ еси! Тамо видимъ всемогущество Твое, здѣ удивляемся снисхожденію Твоему. Непостижимо всемогущество Твое, непостижимо и смиреніе Твое. Слава человѣколюбію Твоему, слава снисхожденію Твоему, слава смиренію Твоему, слава благости Твоей! Насъ ради чудное сіе сотворилъ Ты дѣло. Господи, смиренныхъ и кроткихъ Любителю! напиши на сердцѣ моемъ дѣло Твое сіе, да на тое всегда смотрю, и послѣдую тому. 2)~Высокопочтенные цари, князи, вельможи, господа и вси славніи въ мірѣ семъ! смотрите въ сіе чистѣйшее смиренія зерцало, и противный тому порокъ, усмотрѣвши на душахъ вашихъ, отирайте. Сынъ Божій и Царь небесный рабамъ Своимъ умылъ ноги: вы, земныя, поступайте убо подобно Тому съ земными рабами и подданными вашими. Не бойтеся: отъ того вы не умалитеся, но паче возвеличитеся; не уничтожитеся, но паче вознесетеся; не обезчеститеся, но паче почтетеся; не посрамитеся, но паче прославитеся. \textit{Всякъ смиряяй себе вознесется}\footnote{Лук.~14,~11.}! Мало и низко предъ міромъ дѣло сіе кажется, но предъ Богомъ велико и высоко. Кто истинно смиренъ, тотъ у Бога, смиренныхъ Любителя, великъ. Но и люди разумныи удивятся вашему сему дѣлу. Хрістову смиренію ради того удивляемся вси, хрістіане, что Онъ тако великъ есть, но тако смирилъ Себе. Воистину удивятся вси, знающіи силу добродѣтели, когда и васъ въ подобномъ смиреніи, увидятъ. 3)~На сій живый смиренія образъ смотрите пастыри, и примѣръ смиренія вашего подавайте людемъ вашимъ. Пусть люди смотрятъ на васъ, и видятъ въ васъ смиреніе Хрістово и познаютъ, что вы отъ Пастыреначальника Іисуса Хріста посланы къ нимъ въ пастыри. 4)~Смотрите вси хрістіане, и подражайте Господу своему. Не устыдился Онъ послужить человѣкамъ, не стыдитесь и вы другъ другу служить и ноги умыть. Тѣмъ, покажите, что вы Хрістовы есте. Послѣдуйте Хрістову смиренію, и будете Хрістовы. Хрістовъ рабъ ни отъ чего такъ, какъ отъ смиренія, познается. Внимай сему и ты, душе моя, и люби смиреніе Хрістово паче всего. 5)~Общество подобно есть составу тѣла человѣческаго. Уды въ тѣлѣ человѣческомъ не презираютъ другъ друга, но другъ другу служатъ, и тако цѣлость всего тѣла сохраняютъ. Голова разумомъ, очи видѣніемъ, уши слышаніемъ, ноздри обоняніемъ, уста и языкъ вкусомъ и словомъ, руки дѣланіемъ, ноги хожденіемъ и бѣгомъ, чрево вареніемъ пищи, всему тѣлу и другъ другу взаимно служатъ. Тако и во обществѣ, начальники служатъ подначальнымъ промысломъ, наставленіемъ и попеченіемъ о нихъ; но подначальныи служатъ начальникамъ работою, услуженіемъ въ нуждахъ ихъ, пищею, одѣяніемъ и прочіими нужными потребами. Тако взаимною должностію и взаимнымъ служеніемъ начальники и подначальныи другъ съ другомъ связаны, какъ члены въ составѣ тѣла человѣческаго. Худое и неблагополучное бываетъ состояніе подначальныхъ безъ начальниковъ; но что можетъ и начальникъ безъ службы подначальныхъ? Голова въ тѣлѣ человѣческомъ мнится быть честнѣйшая паче прочихъ членовъ; но что она можетъ безъ очей, безъ ушей, безъ рукъ и безъ ногъ? Тако и начальникъ во обществѣ показуется честнѣйшій быть паче прочіихъ; но что сдѣлаетъ безъ службы подначальныхъ? Едино ничто. Чтобы голова, и разумная, управляла добрѣ составъ тѣла, потребны ей очи видѣть, уши слышать, уста и языкъ говорить, руки дѣлать, ноги ходить, и проч. Тако начальнику, и самому разумному, чтобы съ успѣхомъ моглъ промышлять о обществѣ, потребны ему другіи люди, иныи какъ очи, иныи какъ уши, иныи какъ уста, иныи какъ руки, иныи какъ ноги, и проч. Тако благополучно правленіе его будетъ. Что голова безъ прочихъ удовъ? Ничто. Что и начальникъ безъ подначальныхъ? Ничто. Что военачальникъ безъ офицеровъ и солдатъ? Какъ простой солдатъ. Что князь, вельможа и господинъ безъ слугъ и крестьянъ? Той же простый человѣкъ. Защищаетъ воинство отечества своего цѣлость; но отечество снабдѣваетъ его оружіемъ, провіантомъ и прочіими нужными потребами, и упалыя того мѣста новыми наполняетъ воинами. Что можетъ воинство, когда отечество не будетъ подавать ему потребныхъ? Возносятся города надъ селами и деревнями; но что города безъ селъ и деревень? Ничто. Села и деревни служатъ городамъ хлѣбомъ и прочіими припасами; исчезнутъ города, когда села и деревни не подадутъ имъ хлѣба своего. Тако мнящіися быть подлѣйшіи, честны и нужны суть. Сіе представляется не ради того, дабы начальники отъ подначальныхъ были презираемы; да не будетъ! Должно имъ воздавать достойную имъ честь, по Писанію: \textit{воздадите всѣмъ должная: емуже убо урокъ, урокъ; а емуже дань, дань; а емуже страхъ, страхъ, и емуже честь, честь}\footnote{Римл.~13,~7.}, "--- но дабы мнящіися быть честнѣйшіи, низшихъ не презирали, и своими титулами и именами не гордились. Едино требуется отъ хрістіанъ начальниковъ, дабы они и сами со страхомъ Божіимъ жили, и обществу хрістіанскому, по силѣ присяги своей и правилу святаго Писанія, служили. Тогда они подлинно велики будутъ, велики и здѣ, и въ царствіи небесномъ велики. А когда по прихотямъ своимъ живутъ здѣ, и Тому, Который власть имъ далъ, не повинуются, то и въ вѣчномъ неблагополучіи велики будутъ. Столько имъ умножится страданія, сколько здѣ свирѣпѣли. \textit{Ибо малый достоинъ есть милости: сильніи же сильнѣ мучены будутъ}\footnote{Прем. Сол.~6,~6.}. Возлюбленне, который на высокомъ мѣстѣ сидишь! смотри почаще на смиреніе Хрістово, и не допуститъ тебе возноситься, и братію свою подлѣйшую презирать, и безумно въ мірѣ семъ свирѣпѣть, и тако въ бѣдствіе оное впасть. Кто знаетъ, ты ли, или тотъ, котораго презираешь, лучшій и честнѣйшій есть у Бога, Который смотритъ на сердце и судитъ по внутренности, а не по наружности, и по вѣрѣ и дѣламъ той сообразнымъ, а не по именамъ и титуламъ? 6)~Богу возможно какъ изъ богатаго нищимъ, изъ нищаго богатымъ, такъ и изъ высокаго низкимъ, и изъ низкаго высокимъ сдѣлать, по писанному: \textit{воздвизаяй отъ земли нища, и отъ гноища возвышаяй убога, посадити его съ князи, съ князи людей Своихъ}\footnote{Пс.~112,~7 и 8.}. Почто же человѣку возноситься тѣмъ, что сегодня его, а заутра не его можетъ быть, и презирать того брата, который сегодня подлѣйшимъ, а по маломъ времени честнѣйшій его можетъ быть? Видимъ таковыхъ примѣровъ довольно въ мірѣ семъ. Гордость всегда смирится, но смиреніе всегда возносится. \textit{Богъ гордымъ противится: смиреннымъ же даетъ благодать}\footnote{1~Петр.~5,~5.}. Смотри въ зеркало смиренія Хрістова, и послѣдуй Тому, и всегда будеши въ высотѣ твоей. 7)~Мы вси, высокіи и низкіи, едино есмы. Единаго Создателя признаемъ и исповѣдуемъ и призываемъ Бога. Отъ единаго отца Адама ведемъ родъ нашъ. Единаго Хріста, Сына Божія, кровію и смертію искупилися мы. Едино крещеніе, едину вѣру имѣемъ мы; къ единому вѣчному блаженству Словомъ Божіимъ позваны мы; къ единому таинству святѣйшей Евхаристіи приступаемъ вси; вси нарицаемся хрістіане отъ единаго Хріста. О! когда бы быть всѣмъ хрістіанами, то бы никто другаго не презиралъ. Когда вси едино есмы, какъ уды въ тѣлѣ, то почто одному другаго презирать и уничтожать? Презираяй презрѣнъ, и уничтожаяй уничтоженъ будетъ. 8)~Вси человѣки, отъ высшихъ до низшихъ, равны суть. Показуетъ то день рожденія и исхода. \textit{Единъ бо входъ всѣмъ есть въ житіе}, глаголетъ Соломонъ, \textit{подобенъ же и исходъ}\footnote{Прем.~7,~6.}. Вси наги раждаются; ничего въ міръ не вносятъ, ничего и не износятъ изъ міра. Никто не раждается княземъ, вельможею, господиномъ, богатымъ, славнымъ, но простымъ человѣкомъ, нищимъ, убогимъ, нагимъ, и съ плачемъ. Равно вси всякимъ перемѣнамъ и бѣдамъ подлежатъ. Видимъ тое въ мірѣ. Высокіи смиряются, и низкія возносятся; славніи презираются, и презрѣнніи прославляются; богатіи обнищеваютъ, и нищіи богатѣютъ. Какъ море то возноситъ волны своя, то низпущаетъ: такъ и въ мірѣ дѣлается; то восходятъ люди, то нисходятъ. Равна всѣмъ и кончина. Вси отъ міра сего исходятъ безъ всего; вси богатство, честь, славу, имена и титулы здѣ оставляютъ; равно всѣхъ малый гробъ воспріемлетъ; равно всякъ въ треаршинной ямѣ зарывается и землею засыпается; равно и тлѣнію предается. Гдѣ богатство, гдѣ честь, гдѣ слава, гдѣ имена и титулы? Смотри на сіе, человѣче. Вижу, гдѣ положенъ князь, гдѣ положенъ вельможа, гдѣ положенъ господинъ, гдѣ положенъ славный, гдѣ положенъ богатый, и гдѣ положенъ рабъ и убогій; вижу, что тутъ лежитъ земля, и есть и является земля, и въ ней только кости нагія; и воистину не могу разпознать господина и раба его, славнаго и подлаго, богатаго и нищаго, понеже вижу только землю. Слышу, что въ семъ"=то мѣстѣ такой"=то господинъ положенъ; но его не вижу, а только землю вижу, и како назову его именемъ его? понеже не его самого, но землю вижу. Смотри на сіе, человѣче, и познавай и признавай, что вси равны суть. Будешь и ты таковъ же, каковыхъ видишь здѣ. Почто же равному равнаго презирать и уничтожать? почто земля и пепелъ гордится? Смотри на смиреніе Хрістово. Слышите, хрістіане, Хріста глаголющаго: \textit{вы глашаете Мя Учителя и Господа; и добрѣ глаголете: есмь бо. Аще убо Азъ умыхъ ваши нозѣ, Господь и Учитель, и вы должни есте другъ другу умывати нозѣ. Образъ бо дахъ вамъ, да, якоже Азъ сотворихъ вамъ, и вы творите. Аминь, аминь глаголю вамъ: нѣсть рабъ болій господа своего, ни посланникъ болій пославшаго его. Аще сія вѣсте, блажени есте, аще творите я}\footnote{Іоан.~13,~13"=17.}.

Но посмотримъ уже, хрістіанине, спасительныхъ Хрістовыхъ страданій; посмотримъ страданій, которыя Іисусъ Хрістосъ, Господь нашъ, насъ ради волею претерпѣлъ; посмотримъ страданій, въ которыхъ слава, утѣшеніе, радость и животъ нашъ состоитъ; посмотримъ страданій, и, смотря на нихъ, помедлимъ, да и въ насъ \textit{вообразится Хрістосъ} распятый; посмотримъ страданій, и тѣхъ сообщники будемъ, да и славы Его причастники будемъ. Тако и здѣ, и тамо отъ Него не отлучимся; и въ страданіи, и въ славѣ съ Нимъ будемъ, яко уды съ главою, и невѣста съ женихомъ, и раби съ господемъ.

Видимъ въ святомъ Евангеліи, что тіи, которыхъ Онъ пришелъ спасти, которыи Его ожидали, которымъ Онъ великая благодѣянія показалъ, "--- тіи, говорю, не приняли Его, якоже о томъ святый Евангелистъ Его съ жалостію всему міру свидѣтельствуетъ: \textit{во своя пріиде, и свои Его не пріяша}\footnote{1,~11.}. И не только не приняли Его, но и злобились на Него, и искали убить, какъ тое видимъ въ святомъ Его Евангеліи. "--- Видишь здѣ, хрістіанине: 1)~Благость и человѣколюбіе Хрістово непобѣдимо никакою злобою. Зналъ Хрістосъ Господь нашъ, что Ему имѣютъ дѣлать неблагодарныи Его люди; но однакожъ пришелъ къ нимъ, пришелъ спасти ихъ, пришелъ взыскати ихъ, погибшихъ, пришелъ призвати и привести ихъ къ небесному Своему Отцу. Пожилъ съ ними, обращался между ими, училъ ихъ, показывалъ имъ путь къ спасенію вѣчному, обращалъ ихъ къ покаянію, увѣщавалъ ихъ, показывалъ имъ Себе знаменіями и чудесами, что Онъ Тотъ Мессія, Который имъ отъ Бога обѣщанъ, пророками проповѣданъ, и отъ нихъ ожидаемъ былъ. \textit{Шедше возвѣстите Іоаннови, яже слышите и видите: слѣпіи прозираютъ, и хроміи ходятъ, прокаженніи очищаются, и глусіи слышатъ, мертвіи востаютъ, и нищіи благовѣствуютъ}\footnote{Мѳ.~11,~4 и 5.}. То"=есть, пришло тое время, въ которое проповѣди пророческія исполняются, "--- и кто Я, "--- изъ знаменій, и чудесъ и дѣлъ можете познать. Все сіе видѣли Іудеи, но Его не приняли, родъ, строптивый и развращенный. \textit{Во своя пріиде, и свои Его не пріяша}. 2)~Долготерпѣніе Хрістово. Аще бы царь какій пришелъ во градъ свой, и подданныи бы его не приняли, но обезчестили и озлобили бы его; велико бы и несносно было ему досажденіе. Царь небесный и Господь пришелъ къ подданнымъ Своимъ, Іудеямъ, и не приняли Его, но и обезчестили и озлобили Его; но Онъ кротко съ ними поступалъ, и долготерпѣлъ имъ, и увѣщавалъ ихъ, и благотворилъ имъ, и желалъ и искалъ имъ вѣчнаго спасенія. О долготерпѣнія Твоего, Іисусе! О благости Твоей и человѣколюбія! Воистину отъ сего можешь познать, человѣче, что Онъ единородный Сынъ небеснаго Отца, сіяніе славы Отчія и образъ ѵпостаси Его, образъ невидимаго Бога, Который \textit{солнце, свое сіяетъ на злыя и благія, и дождитъ на праведныя и на неправедныя}\footnote{Мѳ.~5,~45.}. Божественный и прелюбезный нравъ Отчій видится въ Сынѣ Его. Кто отъ человѣкъ тако можетъ стерпѣть? Скоро кротость и терпѣніе человѣческое перемѣняется. 3)~Отсюду видимъ, како зависть и злоба человѣка ослѣпляетъ. Знали книжники и фарисеи отъ пророчествъ, что надобно тогда пріити Хрісту, когда \textit{оскудѣетъ князь отъ Іуды}\footnote{Быт.~49,~10.}. Оскудѣлъ князь отъ Іуды, и \textit{пришелъ Іисусъ Хрістосъ}\footnote{Мѳ.~11,~1.}. Знали, что надобно Ему \textit{родитися въ Виѳлеемѣ іудейстѣмъ}\footnote{Мих.~5,~2.}. \textit{Тамо} родился Іисусъ\footnote{Лук.~2,~4--8.}. Знали о крестителѣ Іоаннѣ, что онъ \textit{пророкъ} былъ\footnote{1,~76; Пс.~131,~17; Ис.~40,~3--5.}. Іоаннъ \textit{свидѣтельствовалъ о Іисусѣ}\footnote{Лук.~3,~16; Мѳ.~3,~11; Іоан.~1,~6--36.}. Знали, что Мессія, отъ нихъ ожидаемый, преславная знаменія и чудеса сотворитъ. Іисусъ сотворилъ знаменія и чудеса неслыханная. Откуду народы, видѣвше преславная Его чудеса, восклицали: \textit{николиже явися тако во Израили}\footnote{Мѳ.~9,~33.}. Откуду и Никодимъ ко Хрісту глаголалъ: \textit{равви! вѣмъ, яко отъ Бога пришелъ еси Учитель: никтоже бо можетъ знаменій сихъ творити, яже Ты твориши, аще не будетъ Богъ съ нимъ}\footnote{Іоан.~3,~2.}. И сами о Іисусѣ Хрістѣ совѣтовали и говорили: \textit{что сотворимъ, яко человѣкъ сей много знаменія творитъ}\footnote{11,~47.}? Но завистію и злобою ослѣпленніи не приняли Его. Тако зависть и злоба ослѣпляетъ! О, люто зло зависть и злоба! Хрістіанине! берегись сего зла, да не и твои ослѣпитъ внутреннія очи, и постигнетъ тебе всеконечное зло, вѣчная погибель. 4)~Не приняли Хріста книжники и фарисеи и прочіи Іудеи, но многіи и приняли. Приняли и языки, которыи о Немъ и не слыхали, и поклонилися Ему. Слава Богу, что намъ свѣтъ Его блеснулъ. Пріимемъ убо Его достойно, и возлюбимъ святое явленіе Его, и послужимъ Ему, яко Царю нашему и Богу, и въ вѣрѣ и исповѣданіи Его до конца пребудемъ, да и насъ исповѣсть предъ Отцемъ Своимъ небеснымъ, якоже глаголетъ: \textit{всякъ, иже исповѣсть Мя предъ человѣки, исповѣмъ его и Азъ предъ Отцемъ Моимъ, Иже на небесѣхъ}\footnote{Мѳ.~10,~32.}. Когда исповѣдуемъ Его Сына Божія, Бога и Господа и Царя нашего, то и послушаемъ Его, и поработаемъ Ему, яко Царю нашему, Господу и Богу, и по правилу святаго Евангелія Его поживемъ. Тако будемъ \textit{Его} здѣ, и тамо за \textit{Своихъ} признаетъ насъ предъ Отцемъ Своимъ небеснымъ. Слыши, хрістіанине, какія примѣты объявляетъ Апостолъ тѣхъ, который Хрістовы суть. \textit{Иже}, рече, \textit{Хрістовы суть, плоть распяша со страстьми и похотьми}\footnote{Гал.~5,~24.}. Вотъ примѣты Хрістовыхъ людей. И иначе они называются \textit{человѣцы Божіи}, какъ видимъ на многихъ святаго Писанія мѣстахъ\footnote{1~Тим.~6,~11; Евр.~4,~9; 1~Петр.~2,~10.}. Будемъ убо, хрістіанине, не единымъ исповѣданіемъ Хрістовы, но и дѣломъ и истиною, да и здѣ, и на второмъ Своемъ пришествіи, за Своихъ насъ признаетъ Хрістосъ. \textit{Твердое убо основаніе Божіе стоитъ, имущее печать сію: позна Господь сущія своя, и да отступитъ отъ неправды всякъ именуяй имя Господне}\footnote{2~Тим.~2,~19.}. \textit{Знаю Моя, и знаютъ Мя Моя}, глаголетъ Господь. И паки: \textit{овцы Моя гласа Моего слушаютъ, и Азъ знаю ихъ; и по Мнѣ грядутъ. И Азъ животъ вѣчный дамъ имъ; и не погибнутъ во вѣки, и не восхититъ ихъ никтоже отъ руки Моея}\footnote{Іоан.~10,~14,~27 и 28.}.

Видимъ въ святомъ Евангеліи Его, что единъ отъ учениковъ Его, именемъ Іуда Искаріотскій, продалъ Его искавшимъ убити Его, и продалъ за тридесять сребренниковъ, и лестнымъ лобзаніемъ предалъ Его въ руки ихъ. Примѣтилъ льстецъ лукавый, что архіереи, книжники и фарисеи искали убить Іисуса, а сердце сребролюбивое имѣлъ; пришелъ къ нимъ, жаждущимъ неповинной крови, и сказалъ имъ: \textit{что ми хощете дати, и азъ вамъ предамъ Его? Они же поставиша ему тридесять сребренникъ}\footnote{Мѳ.~26,~15.}. И за такъ малую цѣну продалъ беззаконный Безцѣннаго!? Видѣлъ еси, Іисусе, умышленія и совѣты лукавыя, которыя на Тя совѣщавали врази Твои; видѣлъ еси беззаконный торгъ, куплю и согласіе; видѣлъ, что за такъ малую цѣну Безцѣнный продаешися; видѣлъ, но терпѣлъ неистовство беззаконныхъ людей Твоихъ. Слава долготерпѣнію Твоему, милостиве! Проданъ былъ еси, благоутробне, такъ дешево, да мене, проданнаго подъ грѣхъ, искупиши. Пою и славлю человѣколюбіе Твое! "--- Видимъ здѣ, хрістіанине: 1)~Коль велико зло есть сребролюбіе. Душа сребролюбивая ничего не ужасается дѣлать, на все дерзаетъ, чтобы только скверный прибытокъ достать. Беззаконный Іуда не ужаснулся продать и предать Іисуса, Господа и Учителя своего, Того, Котораго видѣлъ преславная и чудная дѣла творящаго. Сребролюбіе ослѣпило очи его. Корень и начало такъ беззаконному дѣлу сребролюбіе его было. Такъ пагубное зло есть сребролюбіе! Хрістіанине! берегись сребролюбія, да не во всякое зло вринетъ тебе. 2). Подражаютъ Іудѣ беззаконныи военачальники, которыи за сребро многія тысящи въ руки враговъ предаютъ, и такъ несносный вредъ отечеству дѣлаютъ. Подражаютъ беззаконныи судіи, которыи за мзду правду и истину продаютъ, и праваго виноватымъ и виноватаго правымъ дѣлаютъ. Подражаютъ ложныи свидѣтели, которыи деньгами подкупаются и ложь на судѣ свидѣтельствуютъ. Подражаютъ и прочій беззаконники, которыи какъ нибудь истину за сребро продаютъ, и лжею скверный прибытокъ себѣ достаютъ. 3)~Подражаютъ Іудѣ тіи люди, которыи на пастырей своихъ клевещутъ и порочными именами ихъ обносятъ; подражаютъ дѣти, которыи родителей своихъ, подражаютъ раби, которыи господъ своихъ, подражаютъ ученики, которыи учителей и наставниковъ своихъ, подражаютъ прочіи люди, которыи благодѣтелей своихъ оклеветаютъ и порочатъ предъ другими. Вси таковыи какъ бы продаютъ благодѣтелей своихъ, и послѣдуютъ Іудѣ предателю, который Господа и Учителя своего продалъ и предалъ въ руки враговъ Его. О таковыхъ людяхъ писано: \textit{иже воздаетъ злая за благая, не подвигнутся злая изъ дому его}\footnote{Притч.~17,~13.}. Берегись, хрістіанине, быть подражателемъ предателю Іудѣ, да не съ нимъ часть наслѣдиши. 4)~Іуда, предая Господа своего въ руки беззаконныхъ, лобзалъ Его, и говорилъ Ему: \textit{радуйся, Равви!} но въ сердцѣ своемъ зло мыслилъ на Господа своего, и лестнымъ лобзаніемъ предавалъ Его. \textit{Предаяй же Его, даде имъ знаменіе, глаголя: егоже аще лобжу, той есть: имите его. И абіе приступль ко Іисусови, рече: радуйся, Равви! и облобыза Его}\footnote{Мѳ.~26,~48 и 49.}. О кротости Іисуса, Который терпѣлъ льстивное лобзаніе врага Своего! О скверныхъ устенъ предателя, который не ужаснулся приступить и лестно лобзать святыхъ святѣйшаго, и источника святыни, Господа! Сему лукавому лобзателю послѣдуютъ и подобятся ему тіи хрістіане, которыи лестно, коварно и хитро съ ближними своими обходятся; иное языкомъ произносятъ, иное въ сердцахъ мыслятъ; на языкахъ медъ сладости, но въ сердцахъ желчь горести имѣютъ; устами благословляютъ, а сердцами кленутъ; устами говорятъ: \textit{здравствуй}, но въ сердцахъ думаютъ, какъ бы того, котораго привѣтствуютъ словомъ, уловить и его повредить. Вси таковыи лестнаго лобзателя Іуды нравъ имѣютъ. Хрістіанине! берегись лестно съ ближнимъ обходиться, да не сыномъ погибельнымъ, якоже Іуда, сдѣлаешися. Богъ \textit{сердца и утробы испытуетъ всѣхъ}, и видитъ, како поступаеши и обходишися съ ближнимъ твоимъ, како говорили и мыслили о немъ. 5)~Видимъ, како сатана перьво въ единъ грѣхъ, потомъ въ другій, а далѣе и въ прочіи грѣхи приводитъ человѣка. Іудѣ прежде вложилъ въ сердце сребролюбіе, потомъ предательство Господа своего, а далѣе и во всеконечное отчаяніе его привелъ. Тако онъ и нынѣ дѣлаетъ. Прежде научаетъ человѣка въ праздности и лѣности жить, а въ праздности живущему всякіи злыя мысли предлагаетъ, и ко всякому злу и грѣху приводитъ. Прежде сребролюбіе въ сердце человѣку ввергаетъ, а отъ того къ хищенію, воровству, грабленію, насилію, лжи, разбою, убійству и прочимъ тяжкимъ беззаконіямъ приводитъ. \textit{Корень бо всѣмъ злымъ есть сребролюбіе}\footnote{1~Тим.~6,~10.}. Перьво въ высокоуміе и надменіе, его тщаніемъ, приходитъ человѣкъ; а потомъ уже презираетъ, уничтожаетъ и осуждаетъ ближняго. Тако и въ прочіихъ грѣхахъ онъ дѣлаетъ. Хрістіанине! берегись единаго грѣха, да не въ другіи и иныи многіи грѣхи впадеши, и пріидеши во глубину золъ. 6)~Іуда предатель, пришедши въ отчаяніе, \textit{удавился}\footnote{Мѳ.~27,~5.}. Позналъ онъ величество грѣха, но не позналъ величества милосердія Божія. Тако и нынѣ многіи дѣлаютъ, и послѣдуютъ Іудѣ. Познаютъ множество грѣховъ своихъ, но не познаютъ множества щедротъ Божіихъ, и тако отчаиваются спасенія своего. Хрістіанине! тяжкій и послѣдній ударъ есть діавольскій "--- отчаяніе. Онъ, прежде грѣха, \textit{милостиваго} Бога предлагаетъ; но послѣ грѣха "--- \textit{правосудливаго}. Сія его есть хитрость. Ты противу того дѣлай. Прежде грѣха правосудіе Божіе представляй себѣ, да не согрѣшиши; но когда проступишися и согрѣшиши, помышляй о величествѣ милосердія Божія, да не впадеши въ отчаяніе Іудино. Якоже бо величество Его, тако и милость Его. Сколько у тебе грѣховъ ни есть, и какъ бы велики ни были, у Бога еще болѣе милости и щедротъ; только кайся и впредь грѣшить берегись, да не дознаеши на себѣ правосудіе Божіе. 7)~Видимъ, какъ люто и тяжко есть мученіе злыя совѣсти. Іуда не стерпѣлъ мученія совѣсти своея злыя; и того ради удавился. Злая совѣсть его къ тому привела. Сего ради лучше онъ избралъ умереть, нежели мученіе совѣсти терпѣть. Смерть животу предпочелъ.

Хрістіанине! берегись совѣсть раздражать, да не подобное Іудѣ постраждеши; но храни совѣсть свою такъ, какъ животъ свой храниши. Нѣтъ злѣйшаго мучителя, какъ совѣсть злая. Берегись убо ее уязвлять и безпокоить, каковое безпокойствіе отъ грѣховъ бываетъ. Избирай лучше умереть, нежели согрѣшить противу совѣсти. Сего и хрістіанская должность требуетъ отъ тебе. 8)~Аще здѣ совѣсть злая такъ мучитъ человѣка, что лучше избираетъ умереть, нежели жить; то какъ уже будетъ мучить въ будущемъ вѣкѣ осужденныхъ, когда имъ содѣланные ими всѣ грѣхи и гнѣвъ Божій и вѣчное отчаяніе будетъ представлять! Отъ чего пожелаютъ умереть, но никогда не умрутъ. И сіе"=то есть \textit{смерть вторая и смерть вѣчная}!

Видимъ въ святомъ Евангеліи, что когда Іуда предатель съ своими злыми единомысленниками пришелъ ко Хрісту, тогда бывшему въ вертоградѣ со Своими учениками, хотя Его предати въ руки враговъ Его, "--- Іисусъ Хрістосъ, Господь нашъ, Самъ къ нимъ изшелъ и сказалъ имъ: кого ищете? И когда отвѣщали Ему: Іисуса Назореа; сказалъ имъ Іисусъ: \textit{Азъ есмь}. А какъ только сказалъ имъ: \textit{Азъ есмь}; "--- симъ гласомъ устрашилися и пали на земли, и проч. \textit{Іисусъ же вѣдый вся грядущая нань, изшедъ рече имъ: кого ищете? Отвѣщаша Ему: Іисуса Назореа. Глагола имъ Іисусъ: Азъ есмь. Стояше же и Іуда, иже предаяше Его, съ ними. Егда же рече имъ: Азъ есмь, идоша вспять, и падоша на земли}, и проч.\footnote{Іоан.~18,~4--6.} Здѣ видимъ, хрістіанине: 1)~Іисусъ Хрістосъ Господь нашъ все будущее и имѣющее Ему приключитися зналъ: \textit{Іисусъ же вѣдый вся грядущая нань}. Будущее знать единаго Бога дѣло есть. Видимъ убо, что Хрістосъ Богъ нашъ есть, пришедый въ міръ спасти насъ, и пострадавый за насъ. 2)~Іуда предатель и прочіи единомысленники его гласомъ Хрістовымъ такъ устрашилися, что пали на землю. Отъ сего можно было имъ познать, коль великъ есть и силенъ, Который ихъ гласомъ Своимъ, какъ громомъ поразилъ; однакожъ не усумнѣлись приступить, и беззаконныя своя руки на Іисуса возложить, и взять и связать \textit{Всесильнаго}. Видишь, какъ злоба и всякій грѣхъ ослѣпляетъ человѣка! 3)~Что Іисусъ Хрістосъ \textit{Самъ} ко пришедшимъ врагамъ Его \textit{вышелъ}, и сказалъ имъ: \textit{Азъ есмь}, Егоже ищете. Отъ сего видимъ, что Онъ \textit{волею Своею предалъ Себе} на страданіе за насъ, и вся, приключившаяся Ему, злостраданія претерпѣлъ.

Благоволилъ Отецъ, чтобы чашу страданій единородный Сынъ Его испилъ за спасеніе наше; соблаговолилъ Тому и возлюбленный Сынъ Его, и данную Ему отъ небеснаго Отца Своего чашу испилъ. \textit{Послушливъ бывъ даже до смерти, смерти же крестныя}\footnote{Фил.~2,~8.}; якоже рече: \textit{чашу, юже даде Мнѣ Отецъ, не имамъ ли пити ея}\footnote{Іоан.~18,~11.}? Видишь, хрістіанине, горячую единороднаго Сына Божія къ намъ любовь, видишь и небеснаго Его Отца. Возлюбилъ насъ недостойныхъ Богъ: возлюбилъ и Сынъ Его единородный; Предалъ Отецъ Сына Своего за насъ на смерть и страданіе: предалъ Себе и Сынъ Его за насъ. Слава Богу, тако благоволившему! Слава Божественному человѣколюбію Его, слава благости Его, слава милосердію и щедротамъ Его! \textit{Господи, что есть человѣкъ, яко познался еси ему, или сынъ человѣчь, яко вмѣняеши его}\footnote{Пс.~143,~3.}? Іисусъ Хрістосъ Господь нашъ моглъ уклониться отъ рукъ беззаконныхъ враговъ Его, но не хотѣлъ, яко пришло уже тогда время Ему за насъ пострадать. Хрістіанине! не убѣжимъ и мы и не уклонимся отъ креста страданій, каковыи Отецъ небесный наложитъ намъ; хотя и можемъ уклониться, но понесемъ его безъ роптанія, и послѣдуемъ Хрісту даже до Голгоѳы, аще тако воля небеснаго Отца восхощетъ. 4)~Моглъ Хрістосъ всѣхъ, пришедшихъ Его яти, поразить, но не хотѣлъ, и попустилъ имъ взять Себе и связать. Послѣдуемъ и мы въ томъ Начальнику нашему, Іисусу; и хотя можемъ злотворящимъ намъ отмстить и повредить ихъ, однакожъ да не отмщеваемъ, ни повреждаемъ ихъ, но все ихъ злотвореніе съ кротостію претерпимъ. Тако Хрісту, Начальнику нашему, яко уды главѣ, послѣдовать будемъ, тако Ему сообразны будемъ, тако съ \textit{Нимъ постраждемъ, да и съ Нимъ прославимся}\footnote{Римл.~8,~17.}.

Но повидимъ уже, како и что за насъ пострадалъ Хрістосъ Господь нашъ. Видимъ, что Онъ, проданный и преданный отъ предателя Своего, былъ связанъ, связанный былъ веденъ къ беззаконному собору, былъ отъ учениковъ оставленъ: \textit{вси оставльше Его, бѣжаша}\footnote{Мѳ.~26,~56.}. Былъ отъ беззаконнаго суда судимъ, былъ посмѣянъ и поруганъ, былъ заушаемъ и въ ланиту ударяемъ, былъ осужденъ на смерть. \textit{Они же отвещавше рѣша: повиненъ есть смерти}\footnote{Ст. 66.}. Тотъ, у Котораго въ руцѣ смерть и животъ всякаго, былъ веденъ къ Пилату игемону, и предъ нимъ оклеветаемъ, яко развратникъ; Тотъ, Который училъ истинѣ и проповѣдывалъ царствіе Божіе приближившееся, и путь къ тому показывалъ, былъ отъ Пилата веденъ къ беззаконному Ироду, и предъ нимъ такожде оклеветаемъ, и отъ него посмѣянъ и поруганъ былъ, и паки возвращенъ былъ къ Пилату; и тако по стогнамъ іерусалимскимъ туды и сюды водимъ былъ съ посмѣяніемъ Господь славы. Былъ съ великимъ усилованіемъ отъ беззаконныхъ людей Своихъ просимъ у Пилата на распятіе: \textit{распни Его}, (слышалъ еси, Господи, беззаконная словеса сія, и терпѣлъ), и испрошенъ былъ на крестную смерть; былъ отъ воиновъ вѣнчанъ терніемъ; былъ, яко царь, отъ нихъ поздравляемъ съ поруганіемъ: \textit{радуйся, царю Іудейскій!} былъ оплеваемъ и тростію по главѣ біемъ \textit{воистину Царь небесный и Господь славы}; былъ веденъ на распятіе и неслъ крестъ, всего міра грѣхами отягченный, \textit{Агнецъ Божій}; и ведены были съ Нимъ два злодѣя; и послѣдовало множество народа, и видѣли Іисуса, \textit{яко Агнца на заколеніе ведома}; и изведенъ былъ внѣ града на распятіе и смерть крестную \textit{Сынъ Божій. Видѣна быша шествія Твоя, Боже, шествія Бога моего Царя}\footnote{Пс.~67,~25.}. Тако приведенный на мѣсто лобное, мѣсто смерти, и уже вида и доброты не имѣющій, \textit{краснѣйшій добротою паче сыновъ человѣческихъ}, нагъ ко кресту пригвожденъ, и ископаша руцѣ и нозѣ Его беззаконницы; и на большее поруганіе и безчестіе Его, повѣшенъ между двухъ разбойниковъ, \textit{слава и хвала Израилева}; и со беззаконными вмѣнися, \textit{единъ Праведный и оправдаяй насъ}. Тако поруганнаго и обезчещеннаго и умученнаго и страждущаго, мимоходя, хуляху врази Его, \textit{покивающе главами своими и глаголюще: разоряяй церковь и треми деньми созидаяй! спасися Самъ. Аще Сынъ еси Божій, сниди со креста}, и проч.\footnote{Мѳ.~27,~39 и 40.} И тако къ болѣзни язвъ Его приложиша беззаконницы. Тако умученный и страждущій Сынъ Божій въ жаждѣ своей вмѣсто воды напоенъ былъ оцта; якоже глаголетъ: \textit{и дата въ снѣдь желчь, и въ жажду Мою напоиша Мя оцта}\footnote{Пс.~68,~22.}.

Наконецъ вся Своя страданія окончалъ смертію, \textit{смертію же крестною}, Іисусъ Хрістосъ Господь нашъ. Слава Тебѣ, Боже нашъ, слава Тебѣ! Отъ Твоего страданія намъ проистекло утѣшеніе; Твоя смерть нашъ животъ есть; Твоя скорбь намъ породила радость; Твое безчестіе и поруганіе намъ исходатайствовало честь и славу вѣчную; Твое смиреніе насъ вознесло падшихъ и смирившихся; Твои язвы насъ исцѣлили; Твои узы насъ, грѣхами связанныхъ, разрѣшили. \textit{Разтерзалъ еси узы моя: тебѣ пожру жертву хвалы}\footnote{Пс.~115,~7 и 8.}, Твое проданіе насъ, подъ грѣхъ проданныхъ, искупило; Твой судъ и осужденіе насъ отъ вѣчнаго суда спасло; Твое посмѣяніе и поруганіе насъ отъ діавольскаго поруганія и посмѣянія избавило. Поемъ человѣколюбіе Твое, покланяемся страстемъ Твоимъ, Человѣколюбче, лобызаемъ милосердіе Твое, прославляемъ благость Твою, благодаримъ Тя помилованніи и искупленніи Тобою грѣшники, и спасенніи погибшіи Твои люди. Слава Тебѣ, Сыне Божій, съ благоутробнымъ Твоимъ Отцемъ и Пресвятымъ Твоимъ Духомъ, аминь. \textit{Что убо речемъ къ симъ? Аще Богъ по насъ, кто на ны? Иже убо Сына Своего не пощадѣ, но за насъ всѣхъ предалъ есть Его: како убо не и съ Нимъ вся намъ дарствуетъ}\footnote{Римл.~8,~31 и 32.}? "--- Хрістіанине! видимъ изъ Евангелія страданія Хріста Бога нашего; постоимъ здѣ, и умными нашими очами посмотримъ на страсти Хрістовы, въ которыхъ все наше блаженство состоитъ. Видимъ здѣ: 1)~Хрістосъ Господь нашъ и душею и тѣломъ страдалъ. Въ душѣ имѣлъ скорбь, печаль, тугу и ужасъ несказанный и умомъ непостижимый. \textit{И начатъ}, глаголетъ Евангелистъ, \textit{ужасатися и тужити; и глагола имъ} (апостоламъ): \textit{прискорбна есть душа Моя до смерти}\footnote{Марк.~14,~33 и 34.}. Отъ чего на пресвятомъ тѣлѣ Его \textit{былъ потъ, яко капли крове каплющія на землю}\footnote{Лук.~22,~44.}! Безчестіе, поношеніе, поруганіе, посмѣяніе, которое Ему сотворилося отъ беззаконныхъ, неизреченно пресвятую душу Его уязвляло; потому что хула тая \textit{всему лицу} приключилася, Которое есть Богъ и человѣкъ, Хрістосъ бо есть совершенный Богъ и совершенный человѣкъ. Тѣмъ весьма уязвлялась пресвятая душа. Его, что въ Немъ Богъ хулимъ и поругаемъ былъ. Тѣло Его пресвятое, пречистое, непорочное, \textit{Слову Божію соединенное}, билося, ранилося, уязвлялося, озлоблялося, заушалося, оплевалося, окровавливалося, терніемъ прободалося, обнажалося, ко кресту пригвождалося, и прочая. Смотри, хрістіанине, страданіе и болѣзнь Хрістову. Ты же хощеши въ роскошахъ и веселостяхъ міра сего жить! 2)~Къ болѣзни и страданію Хрістову и тое приложилося, что Онъ отъ \textit{своихъ} людей пострадалъ, якоже писано есть: \textit{во своя пріиде, и свои Его не пріяша}\footnote{Іоан.~1,~11.}. Пострадалъ отъ тѣхъ, которымъ Онъ обѣщанъ, которыи Его ожидали, которыхъ Онъ пришелъ спасти, которымъ Онъ премногая и преславная благодѣянія сотворилъ, "--- отъ таковыхъ людей пострадалъ Хрістосъ Господь нашъ. Сія ихъ была неблагодарность ко Господу Благодѣтелю своему. Тяжка убо Ему была и неблагодарность людей \textit{своихъ}. Ктому"=жъ пострадалъ въ преславномъ и именитомъ градѣ Іерусалимѣ, и предъ множествомъ народа, тогда на праздникъ Пасхи отъ всѣхъ странъ собравшагося; и въ большее поношеніе и поруганіе, повѣшенъ былъ между двухъ злодѣевъ, и со беззаконными вмѣнися \textit{единъ Святый} и \textit{Праведный}. Смотри, хрістіанине, безчестіе и поруганіе Хрістово! Ты же называешися хрістіаниномъ отъ Хріста; но славы и чести міра сего ненасытно желаеши и ищеши. Самъ убо разсуждай, что еси внутрь тебе, и надлежиши ли до Хріста и Хрістовыхъ рабовъ. 3)~Хрістосъ Господь нашъ, какъ ради насъ въ міръ пришелъ спасти насъ, такъ \textit{за насъ} и пострадалъ. Грѣхи наши виною страданія Его были; грѣхи наши взялъ Онъ на Себе; взялъ и претерпѣлъ казнь, грѣхамъ послѣдующую. \textit{Той язвенъ бысть за грѣхи наша, и мученъ бысть за беззаконія наша}\footnote{Ис.~53,~5.}. Чудна благость, милосердіе и промыслъ Божій о насъ! мы согрѣшили, и Хрістосъ Господь нашъ казнь за насъ претерпѣлъ. Раби согрѣшили, и наказанъ бысть Владыка. Слава человѣколюбію Его! Мы вознеслися, но такъ глубоко смирился, ради насъ вознесшихся, славы Господь. Мы похитили славу Божію съ нашими прародителями, и Онъ за то воздавалъ, якоже чрезъ пророка глаголетъ: \textit{яже не восхищахъ, тогда воздаяхъ}\footnote{Пс.~68,~5.}. Чрезъ непослушаніе и грѣхи наши всякое бѣдствіе, злополучіе и окаянство вошло въ насъ: но чрезъ страданіе Хрістово всякое блаженство, какое ни имѣемъ и будемъ имѣть, къ намъ возвратилося. Чрезъ грѣхъ мы пали: чрезъ Хрістово страданіе возстали. Грѣхомъ мы умерли: страданіемъ и смертію Хрістовою ожили. Хрістосъ за насъ умеръ, и смертію Своею оживилъ насъ. Желалъ нѣкогда Давидъ, царь Израилевъ, умереть вмѣсто сына своего Авессалома (о горячей любви!) и рыдая о немъ, тако глаголалъ: \textit{сыне мой Авессаломе, сыне мой, сыне мой Авессаломе! кто дастъ смерть мнѣ вмѣсто тебе? Азъ вмѣсто тебе, Авессаломе, сыне мой, сыне мой, сыне мой Авессаломе}\footnote{2~Цар.~18,~33.}! Тако Хрістосъ Сынъ Божій, и Сынъ Давидовъ по плоти, видя насъ умершихъ и погибшихъ, плакалъ за насъ, и возжелалъ за насъ умереть и пострадать, и самымъ дѣломъ умеръ и пострадалъ, дабы насъ умершихъ оживить. Слава неизреченной любви Его! Умеръ Онъ, и ожили мы. Смерть Его животъ нашъ есть. Грѣхомъ мы отъ Бога удалилися: Хрістовымъ страданіемъ къ Богу возвратилися и приближилися. Грѣхомъ мы растлѣлися: Хрістовымъ страданіемъ обновилися. Грѣхомъ плѣнилъ насъ діаволъ, и торжествовалъ и ярился надъ нами, яко мучитель лютый: Хрістосъ Господь нашъ Своимъ страданіемъ и смертію того лютаго мучителя побѣдилъ и посрамилъ, и отъ рукъ его насъ силою Своею исхитилъ, и связалъ гордаго того врага нашего, и того отдалъ въ попраніе и посмѣяніе вѣрнымъ Своимъ рабамъ. \textit{Се даю вамъ власть наступати на змію и на скорпію, и на всю силу вражію}\footnote{Лук.~10,~19.}. Грѣхъ причиною былъ, что мы, отступивши отъ Бога, такъ помрачилися умомъ и заблудили, и въ такой пребывали прелести, что, вмѣсто Бога живаго, идоловъ и, вмѣсто Творца, тварь боготворили и почитали. Въ такое заблужденіе человѣкъ, разумомъ одаренный, человѣкъ, по образу Божію и по подобію созданный, въ такое, говорю, заблужденіе пришелъ, что неразумной и бездушной твари покланялся, и яко Бога своего почиталъ, и отъ той помощи себѣ просилъ и искалъ. Великое и ужасное помраченіе ума "--- кланяться немощной твари и боготворить ее, и помощи отъ нея искать. Тако діаволъ, плѣнивши, прельщалъ насъ! Хрістосъ Господь нашъ Своимъ пришествіемъ и страданіемъ прелесть сію упразднилъ, и проповѣдію святаго Евангелія и силою креста Его и страданія взысканы мы, и отъ идоловъ къ Богу живому обратилися, и начали славить святое имя Отца и Сына и Святаго Духа, Единаго Тріѵпостаснаго Бога. Отъ востокъ солнца до западъ славится имя Бога живаго. Грѣхомъ подпали мы клятвѣ, и той послѣдующему вѣчному осужденію. Хрістіанине! страшна есть клятва: она къ вѣчному мученію ведетъ человѣка. Хрістосъ Господь нашъ въ Своемъ страданіи \textit{клятвою за насъ} учинился, единъ Благословенный во вѣки, и вмѣсто клятвы благословеніе Божіе намъ исходатайствовалъ. \textit{Хрістосъ ны искупилъ есть отъ клятвы законныя, бывъ по насъ клятва. Писано бо есть: проклятъ всякъ, висяй на древѣ: да въ языцѣхъ благословеніе Авраамле будетъ о Хрістѣ Іисусѣ, да обѣтованіе Духа пріимемъ вѣрою}\footnote{Гал.~3,~13 и 14.}. Грѣхомъ весьма разгнѣвали мы Бога, и не имѣли къ Нему дерзновенія никакого и приступа: Хрістосъ Господь нашъ Своимъ страданіемъ умилостивилъ Бога, и приступъ къ Нему учинилъ намъ, и дерзновеніе подалъ намъ; Бога намъ, и насъ Богу примирилъ; Отца Своего нашимъ Отцемъ сотворилъ, якоже глаголетъ: \textit{восхожду ко Отцу Моему и Отцу вашему, и Богу Моему и Богу вашему}\footnote{Іоан.~20,~17.}. И молитися Ему научилъ тако: \textit{Отче нашъ, Иже еси на небесѣхъ}, и проч.\footnote{Мѳ.~11,~9--13.} Грѣхомъ заключилося намъ небо и рай, и отворился адъ намъ и все вѣчное злополучіе. Хрістосъ Господь нашъ страданіемъ Своимъ отъ ада искупилъ насъ, и двери къ небесному царствію отверзлъ, и прочая. Спасительные и пресладкіе Хрістова страданія и всего Его о насъ смотрѣнія плоды изрядно, и къ нашему живому утѣшенію, изображаются во Евангеліи, Апостольскихъ посланіяхъ и книгахъ Пророческихъ. Но они надлежать только до вѣрныхъ, и въ истинномъ покаяніи находящихся; прочимъ ничего не пользуютъ, какъ ниже сказано будетъ. 4)~Хрістосъ Господь нашъ вся страданія Своя \textit{волею} претерпѣлъ, какъ и выше сказано. Восхотѣлъ предатель продать и предать Его въ руки беззаконныхъ; не противился ему, "--- и предалъ. Восхотѣли взять Его и связать беззаконники; не противлялся имъ, "--- и взяли и связали Его. Восхотѣли вести Его къ беззаконному сонмищу; не противился, "--- и привели Его. Восхотѣли судить Судію всѣхъ; не противился, "--- и осудили. Восхотѣли поругаться, посмѣяться и оплевать Его; не противился, "--- и поругались, посмѣялись и оплевали славы Господа. Восхотѣли озлобить и уязвить; не противился, "--- и озлобили и уязвили. Восхотѣли, яко осужденника, на смерть вести; не противился, "--- и вели на смерть. Восхотѣли ко кресту пригвоздить, и между двухъ злодѣевъ повѣсить; не противился, "--- и пригвоздили и повѣсили. Восхотѣли и прочая злая Ему сдѣлать; не противился, "--- и сдѣлали. Тако Онъ о Себѣ и чрезъ пророка глаголетъ: \textit{Азъ не противлюся, ни вопреки глаголю. Плещи Моя вдахъ на раны, ланиты же Мои на заушенія, лица же Моего не отвратихъ отъ студа заплеваній}, и проч.\footnote{Ис.~50,~5 и 6.} И во Евангеліи глаголетъ: \textit{сего ради Мя Отецъ любитъ, яко Азъ душу Мою полагаю, да паки пріиму ю. Никтоже возметъ ю отъ Мене: но Азъ полагаю ю о Себѣ; область имамъ положити ю, и область имамъ паки пріяти ю: сію заповѣдь пріяхъ отъ Отца Моего}\footnote{Іоан.~10,~17 и 18.}. И кто бы моглъ Ему что сдѣлать, аще бы Онъ не попустилъ? Аще бо Онъ есть истинный Богъ, какъ и подлинно есть; то кто моглъ бы Его озлобить, Того, Который въ руцѣ Своей всѣхъ животъ и смерть имѣетъ? Воистину маніемъ и во мгновеніи ока всѣхъ бы враговъ Своихъ поразилъ, аще бы того восхотѣлъ. Но попустилъ тако быть, да насъ спасенныхъ увидитъ. 5)~Отъ Хрістовыхъ страданій видишь, хрістіанине, коль великое и лютое зло грѣхъ есть. Грѣхъ оскорбляетъ и прогнѣвляетъ непостижимое и неописанное Божіе величество. Нѣтъ большаго зла, какъ грѣхъ. Человѣкъ согрѣшилъ, и тѣмъ Бога прогнѣвилъ. Но надобно было Сыну Божію Своимъ страданіемъ и кровію очищать грѣхъ, и Бога, грѣхомъ прогнѣваннаго, умилостивлять, и тако человѣка Богу примирить, и отъ вѣчныя казни, грѣху послѣдующія, избавить. Человѣкъ бо, согрѣшивши Богу и того прогнѣвавши, вѣчному наказанію себе подвергнулъ. Сколько въ мірѣ ни есть болѣзней, бѣдъ, напастей и золъ, грѣхъ причиною есть. Легко человѣкъ согрѣшить можетъ, но не легко можетъ грѣхъ очищать. Надобно очищать его горькими слезами и кровію единороднаго Сына Божія. Иначе дознаетъ человѣкъ на себѣ горькій его плодъ "--- вѣчную смерть, \textit{Оброцы бо грѣха смерть}\footnote{Римл.~6,~23.}. Хрістіанине! ничего не берегись такъ, какъ грѣха, яко всепагубнаго зла, аще хощеши истиннымъ хрістіаниномъ быть и не погибнуть вѣчно. Лучше избирай умереть, когда того нужда требуетъ, нежели согрѣшить. 6)~Въ Хрістовомъ страданіи видимъ, что \textit{три свойства Божія} показались "--- правда, милосердіе и премудрость, и исполнилися. Видимъ, что \textit{правда Божія} грѣхъ безъ наказанія не оставляетъ. Требовала правда Божія того, чтобы человѣкъ согрѣшившій вѣчно казненъ былъ за грѣхъ. \textit{Милосердіе Божіе} хотѣло человѣка, согрѣшившаго помиловать. \textit{Премудрость Божія} изобрѣла способъ, которымъ и правдѣ и милосердію Его удовлетворилося. Іисусъ Хрістосъ Сынъ Божій Своимъ страданіемъ и смертію правдѣ Божіей удовлетворилъ. А милосердіе Божіе возъимѣло мѣсто дѣйствовать наше спасеніе. Тако видимъ, что чрезъ страданіе Хрістово правдѣ и милосердію Божіему удовлетвореніе сотворилося. Правда удовольствіе свое получила, и мы милосердіемъ Божіимъ спасаемся. Слава премудрому Богу, благоволившему тако! Хрістіанине! лобызай милосердіе Божіе, и кайся за грѣхи, и всѣмъ сердцемъ ищи вѣчнаго спасенія; ищи, пока дѣйствуетъ милосердіе Божіе, да не дознаешь на себѣ вѣчный правды Божія судъ. 7)~Отъ Хрістовыхъ страданій видимъ, коликую Богъ, коль горячую и пламенѣющую \textit{любовь} имѣетъ къ роду человѣческому, такъ что и Сына Своего ради насъ не пощадѣлъ; и видимъ тое дѣломъ, что Хрістосъ Сынъ Божій словомъ Своимъ изъяснилъ: \textit{тако возлюби Богъ міръ, яко и Сына Своего единороднаго далъ есть, да всякъ вѣруяй въ Онь не погибнетъ, но имать животъ вѣчный}\footnote{Іоан.~3,~16.}. Якоже бо чадолюбивый отецъ, видя дѣтей своихъ въ плѣненіи или въ иномъ какомъ великомъ несчастіи, отъ любви къ нимъ соболѣзнуетъ и состраждетъ имъ; любовь бо есть сострадательна: тако человѣколюбивый Богъ, видя насъ плѣненныхъ отъ діавола, и въ погибели, сострадалъ намъ и умилосердился надъ нами. Сего ради и не тяжко Ему было послать Сына Своего въ міръ и за насъ на страданіе и смерть Его предать, чтобы насъ спасенныхъ видѣть. Видишь, человѣче, любовь Божію къ тебѣ. Сколько разъ видишь образъ Хрістовыхъ страданій, или сколько разъ слышишь о нихъ, или сколько разъ поминаешь о нихъ, столько разъ долженъ ты и о любви Божіей поминать, и той удивлятися. Не неради убо и о себѣ, но тщись, дабы какъ хотѣніе и желаніе Божіе спасенія твоего, такъ и Хрістово страданіе плодъ свой возъимѣло въ тебѣ, то"=есть, вѣчное спасеніе. Возлюбилъ Богъ тебе, и послалъ Сына Своего ради тебе, и благоволилъ Ему пострадать за тебе: удовлетворяй убо любви и святому хотѣнію Его, и буди тщателенъ о спасеніи твоемъ. 8)~Оказавшаяся, "--- хотя и во всѣхъ дѣлахъ, однакожъ наипаче въ страданіи Хрістовомъ, "--- любовь Божія къ намъ, возбуждаетъ насъ Его \textit{взаимно любить}, яко Отца благоутробнаго. Богъ высочайшее, вѣчное и непремѣняемое добро есть, и потому Самъ въ Себѣ достоинъ нашей любви. Кто бо великаго и непремѣняемаго добра не любитъ? Оно само собою, но познанное, къ любви своей всякаго сердце влечетъ. Но дѣла любве Его, а паче страданіе Хрістово, въ которомъ непостижимая любовь Его къ намъ показалась, убѣждаютъ насъ взаимно любить Его. Возлюбилъ Онъ насъ, возлюбилъ недостойныхъ: возлюбимъ и мы Его, достойнаго всякія любве. Онъ нашъ Создатель, Онъ нашъ Промыслитель, Онъ нашъ Искупитель, Онъ нашъ Любитель, Онъ нашъ Отецъ. \textit{Господи! что есть человѣкъ, яко познался еси ему? или сынъ человѣчь, яко вмѣняеши его}\footnote{Пс.~143,~3.}? Чувствуемъ святую Его любовь, хотя и во всемъ, но наипаче въ спасительномъ Его о насъ смотрѣніи. Возлюбимъ убо и мы Его, яко высочайшее наше добро и блаженство, и отъ любви покажемъ Ему послушаніе и соблюдемъ святыя заповѣди Его, и отъ всякаго грѣха, котораго Онъ ненавидитъ, уклонимся. Тако покажемъ любовь нашу къ Нему, то"=есть, когда волю Его сотворимъ; безъ сего бо любовь быть не можетъ\footnote{Іоан.~14,~15 и 21.}. Любителю неотмѣнно должно волю любимаго творить. Иначе ложная и лицемѣрная любовь будетъ.

9)~Таяжде Божія любовь увѣщаваетъ насъ ближняго нашего, то"=есть, \textit{всякаго} человѣка \textit{любить. Аще еще возлюбилъ есть насъ Богъ, и мы должни есмы другъ друга любити}\footnote{1~Іоан.~4,~11.}. Восхощеши ли, хрістіанине, ненавидѣть того, котораго Богъ тако возлюбилъ? Восхощеши ли зло сотворить тому, которому Богъ отъ любви благотворитъ? Восхощеши ли повредить того, ради котораго Хрістосъ Сынъ Божій въ міръ пришелъ, пострадалъ и умеръ? Восхощеши ли обмануть, прельстить, оклеветать, обезчестить того, котораго Хрістосъ Сынъ Божій тако почтилъ? Восхощеши ли украсть, похитить, отнять какое добро у того, за котораго Хрістосъ Сынъ Божій кровь Свою проліялъ? Восхощеши ли не помиловать того, котораго Хрістосъ Сынъ Божій помиловалъ? Восхощеши ли пощадѣть денегъ, одежды, хлѣба, питія, дома и прочаго вещества ради того, ради котораго Хрістосъ Сынъ Божій и Самого Себе не пощадѣлъ? Аще любиши Бога, то надобно любить и того, кого Богъ любитъ. Не любиши Бога, когда не любишь человѣка. Источникъ любви къ человѣку есть любовь къ Богу. Отъ любви къ Богу раждается и любовь къ ближнему. Любовь къ Богу связана есть съ любовію ближняго, и одна безъ другой быть не можетъ. Но любовь къ Богу познается отъ любви ближняго. Когда искренно ближняго любишь, яко отъ Бога возлюбленнаго; то знаменіе есть, что и Бога любишь, любящаго его. Когда ближняго твоего не любишь, котораго Богъ любитъ, то безъ сумнѣнія и Бога не любишь\footnote{1~Іоан.~4,~19--21.}. У любителя съ любимыхъ должно быть единомысліе: что любимый мыслитъ, тое и любитель долженъ мыслить. Богъ мыслитъ о человѣкѣ добро, и любитъ его и всякаго добра желаетъ ему, и благотворитъ ему; тако должно и человѣку, ежели Бога любитъ, добро мыслить о человѣкѣ, и любить его, и всякаго ему добра желать и благотворить. Отсюду послѣдуетъ, что во всякой нуждѣ руку помощи ему подавать долженъ, аще его нелицемѣрно любитъ. Любитель бо любимому ни въ чемъ не отречется помощи, аще только можетъ, какъ"=то сіе видимъ въ дѣйствіи любве между мужемъ и женою, между родителями и дѣтьми, между братіями и сестрами, и между другами нелицемѣрными. Аще убо, человѣче, любишь Хріста, пострадавшаго за тебе; то покажи тую любовь на ближнемъ твоемъ, за котораго Хрістосъ пострадалъ, какъ и за тебе. Любезное Богу созданіе есть человѣкъ; аще убо Бога любишь, то люби и любезное Его созданіе. 10)~Отсюду послѣдуетъ, что безотвѣтны будутъ тіи хрістіане, которыи ближнихъ Своихъ не любятъ и оказываютъ тую нелюбовь свою тѣмъ, что не хотятъ имъ въ нуждахъ ихъ помощи, но отрицаются и оставляютъ ихъ, и часто безстыдно говорятъ о находящемся въ нуждѣ: \textit{что Мнѣ до его нужды}? Сюды надлежать вси тіи богачи, которыи имѣніе свое на роскоши, на созиданіе богатыхъ домовъ, на кареты и кони, на шелковыя и богатыя одѣянія, на богатыя трапезы и вина, на увеселительные сады и пруды и на прочія свои прихоти иждиваютъ; а бѣднымъ, за которыхъ Хрістосъ пострадалъ и кровь Свою проліялъ, помощи не хотятъ. Вси таковыи прихоти своя любятъ, а не Хріста. Сего ради и безотвѣтны будутъ и посрамятся на второмъ Хрістовомъ пришествіи. Человѣче! Хрістосъ за тебе Самого Себе не пощадѣлъ, а ты ради Его и денегъ жалѣешь! Надобно всякому хрістіанину къ тому готовымъ быть, дабы въ случаѣ и умереть за Хріста не отреклся. Аще убо денегъ твоихъ жалѣешь ради Хріста, восхощеши ли и пожелаеши ли въ случаѣ умереть за Хріста? Нѣтъ дражае человѣку, какъ животъ; како животъ твой отдашь за Хріста, когда денегъ жалѣешь ради Хріста? Малаго жалѣешь, не пожалѣеши ли великаго? Кто тебѣ въ семъ повѣритъ? 11)~Хрістосъ Господь нашъ за всѣхъ людей, сколько ни было и есть и будетъ, пострадалъ и умеръ. Якоже бо Богъ \textit{всѣмъ человѣкамъ хощетъ спастися, и въ разумъ истины пріити}\footnote{1~Тим.~2,~4.}, тако и Сына Своего единороднаго въ міръ послалъ всѣхъ спасти. \textit{Тако бо возлюби Богъ міръ, яко и Сына Своего единороднаго далъ есть, да всякъ вѣруяй въ Онь не погибнетъ, но имать животъ вѣчный}\footnote{Іоан.~3,~16.}. \textit{Вѣрно слово и всякаго пріятія достойно, яко Хрістосъ Іисусъ пріиде въ міръ грѣшники спасти}\footnote{1~Тим.~1,~15.}. \textit{Вси согрѣшиша и лишени суть славы Божія}\footnote{Римл.~3,~23.}. Вси убо грѣшники суть, убо всѣхъ спасти пришелъ Хрістосъ Іисусъ. \textit{Пріиде Сынъ человѣческій взыскати и спасти погибшаго}\footnote{Лук.~19,~10.}. Вси погибли, убо всѣхъ взыскати и спасти пришелъ, и за всѣхъ пострадалъ Хрістосъ. \textit{Хрістосъ за всѣхъ умре}, свидѣтельствуетъ Павелъ святый\footnote{2~Кор.~5,~15.}. \textit{Нѣсть бо на лица зрѣнія у Бога}\footnote{Римл.~2,~11.}. Всѣмъ хощетъ спастися; ради всѣхъ и Сына Своего въ міръ послалъ. Слава человѣколюбію Его! Однакожъ отъ Бога получаютъ милость и спасаются тіи только люди, который вѣрою пріемлютъ Его, и любятъ спасительное явленіе Его, и Ему сердечно благодарятъ, и слушаютъ спасительныхъ словесъ Его. Прочіи въ своемъ нечестіи и нераскаяніи погибаютъ. Благодать бо Божія хотящихъ спасаетъ, а не нехотящихъ. Уготовалъ всѣмъ Хрістосъ спасеніе; требуется отъ всѣхъ, чтобы вси хотѣли и спастися, и хотѣли истинно и дѣйствительно. А кто не хощетъ, самъ собою уже погибаетъ. Человѣче! Богъ хощетъ сердечно тебѣ спастися, какъ видишь изъ Хрістова страданія; да будетъ убо хотѣніе и твое, и тако благодатію Его спасешися. 12)~Отсюду послѣдуетъ утѣшеніе всякому вѣрному, истинному хрістіанину. Аще бо Хрістосъ за всѣхъ пострадалъ, то и за тебе. Аще за всѣхъ умеръ, то и за тебе. Аще всѣхъ примирилъ небесному Своему Отцу, то и тебе. Аще всѣмъ приступъ къ Нему отворилъ, то и тебѣ. Аще всѣмъ содѣлалъ спасеніе и царствіе Божіе отверзлъ, то и тебѣ. Буди убо благонадеженъ и миренъ, и безъ сомнѣнія ожидай вѣчныхъ благихъ. Только и вѣренъ буди до смерти, по Хрістову словеси: \textit{буди вѣренъ даже до смерти, и дамъ ти вѣнецъ живота}\footnote{Апок.~2,~10.}. 13)~Утѣшеніе хрістіанамъ, во искушеніяхъ, бѣдахъ и напастяхъ находящимся. Аще бо Хрістосъ толикая за насъ пострадалъ, то оставитъ ли насъ въ нуждѣ нашей? оставитъ ли тѣхъ, за которыхъ пострадалъ и умеръ? Предалъ Себе за насъ и умереть не отреклся за насъ: отречется ли въ нуждѣ помощи намъ? Никако; неотмѣнно приспѣетъ помощь Его къ намъ, ради которыхъ такъ великое любве Своея дѣло содѣлалъ. Онъ смотритъ и ожидаетъ подвига и терпѣнія въ нуждѣ нашей, и невидимо помощь намъ подаетъ, и въ насъ побѣждаетъ, и за побѣду вѣнецъ готовитъ. Стой убо, хрістіанине, мужайся и крѣпися въ приключающейся нуждѣ твоей, и молись и призывай Іисуса, и ожидай помощи Его, и почувствуеши укрѣпляющую руку Его, Того, Котораго діаволъ и весь адъ трепещетъ. 14)~Утѣшеніе кающимся и смущаемымъ въ совѣсти за грѣхи. Бываетъ, что злый духъ кающемуся шепчетъ: \textit{ты"=де толикая и толикая злая содѣлалъ: какого уже надѣешися спасенія}? Чрезъ сіе злый духъ намѣреваетъ и хощетъ человѣка во отчаяніе привести и своея погибели сообщникомъ сдѣлать. Хрістіанине! берегись соизволить злому злаго духа совѣту Надежда наша Хрістосъ Господь. Аще бо Онъ за тебе пострадалъ и умеръ, како обратившагося и кающагося не пріиметъ и не помилуетъ? Онъ того и хощетъ, и чрезъ слово Свое и чрезъ проповѣдниковъ увѣщаваетъ, и ожидаетъ, дабы вси къ Нему грѣшники обратились и покаялися, и тако бы спаслися. Алчетъ бо и жаждетъ всѣхъ спасенія Который за всѣхъ Себе предалъ и пострадалъ и умеръ. Вѣрно сіе слово и всякаго пріятія достойно. И хотя велики твои грѣхи и ужасны были бы, и великое бы множество ихъ было, однакожъ вси загладятся и потребятся благодатію Его. Велики твои грѣхи, но \textit{безконечное} есть милосердіе Его. Множество у тебе грѣховъ, но \textit{безчисленніи} и щедроты Его. Якоже капля воды противу моря, тако вси твои грѣхи противу милосердія Божія и безцѣнныхъ заслугъ Хрістовыхъ. Съ радостію отпуститъ тебѣ Хрістосъ вся согрѣшенія твоя, и тѣхъ уже не помянетъ тебѣ; понеже, когда каешися, хотѣніе Его святое и плодъ страстей Его исполняется. И уже прежнее твое житіе, скаредное и скверное, не повредитъ тебѣ, какъ тьма вышедшему на свѣтъ. Тьма "--- нераскаянное житіе и заблужденіе: свѣтъ "--- покаяніе и исправленіе. Ты, отрекшися прежняго своего житія и наченши новое, какъ изъ тьмы на свѣтъ вышелъ и, возненавидѣвши тьму, свѣтъ возлюбилъ, и потому \textit{новымъ} человѣкомъ сдѣлался. Мужайся убо и крѣпися, и ожидай вѣчнаго опасенія, какъ и вси истинніи хрістіане ожидаютъ. Только того берегись, чтобы на прежнее житіе не возвратиться, якоже песъ обращается на свои блевотины, и свинія омывшаяся въ калъ тинный. Смотри, берегись сего! Сатана всѣми силами старается, чтобы тебе паки на прежнее возвратить: ты стой и крѣпись, и совершай тое до конца, что началъ. Не начало, но конецъ похваляется. И какія немощи видишь внутрь себе, исправляй ихъ повседневнымъ покаяніемъ и молитвою, и тако день отъ дне лучшимъ будеши. А во всемъ призывай \textit{всесильнаго} помощника Іисуса Хріста; безъ Него бо ничего не можемъ. Сильно нападаетъ на тебе врагъ: крѣпко и ты стой противу его съ помощію Хрістовою, и тако онъ посрамится, и ты спасешися. 15)~Хотя и за вся Божія благодѣянія, которыя отъ Него получили и получаемъ туне; однакожъ наипаче должно намъ благодарить Ему за сіе, что къ намъ, погибшимъ, Сына Своего послалъ, и благоволилъ Ему за насъ пострадать, и тако спасти насъ. Въ семъ дѣлѣ чудная и непостижимая Его благость и любовь въ намъ открылася. Такъ чудно мы спасены! Сынъ Божій, Царь и Царя небеснаго Сынъ, за спасеніе наше пострадалъ и умеръ: кто можетъ осудить человѣка, истиннаго хрістіанина? Сынъ Божій за него умеръ и пострадалъ, и кровію Своею очистилъ его и освятилъ, и Богъ его оправдаетъ. Поминай убо, хрістіанине, великое сіе и непостижимое дѣло, и за тое сердечно Богу благодари. \textit{Благословенъ Господь Богъ Израилевъ, яко посѣти и сотвори избавленіе людемъ Своимъ, и воздвиже рогъ спасенія намъ, въ дому Давида отрока Своего}\footnote{Лук.~1,~68 и 69.}. 16)~Страданіе и смерть Хрістова научаетъ насъ грѣху умереть и Богу правдою жить. И сіе"=то есть, что Апостолъ написалъ: \textit{Хрістосъ за всѣхъ умре, да живущіи не ктому себѣ живутъ, но умершему за нихъ и воскресшему}\footnote{2~Кор.~5,~15.}. Хрістіанине! когда хощеши Хрісту Богу жить, то долженъ ты грѣху и себѣ умереть. Грѣху и Хрісту жить не возможно. Хрістосъ купилъ тебе Себѣ кровію Своею: рабъ убо еси Его купленный. Онъ твой Господь: ты Его рабъ, цѣною Крове Его купленный, Самъ убо разсуждай, кому долженъ ты жить "--- себѣ, или Ему, Господу твоему; по своей, или по Его волѣ жить. Самъ знаешь, что вѣрный рабъ господина своего волѣ, а не своей угождаетъ. Невѣрный тотъ рабъ, который не по волѣ господина, но по своей волѣ и прихотямъ своимъ живетъ: тако и хрістіанинъ, "--- не истинный хрістіанинъ, но ложный и невѣрный рабъ Хрістовъ, который не по Хрістовой, но по своей волѣ живетъ, Откуду и Хрістосъ Господь глаголетъ таковымъ: \textit{что Мя зовете: Господи, Господи, и не творите, яже глаголю}\footnote{Лук.~6,~40.}? Умри убо волѣ своей и грѣху, то"=есть, прихотямъ, своимъ, и будеши рабъ Хрістовъ, будеши Ему жить и Ему работать. Плодъ смерти и распятія Хрістова есть отъятіе нашихъ грѣховъ. \textit{Иже} (Христосъ) \textit{грѣхи наша Самъ вознесе на тѣлѣ Своемъ на древо, да отъ грѣхъ избывше, правдою поживемъ}\footnote{1~Петр.~2,~24.}. Гдѣ жъ убо плодъ сей въ тебѣ, когда грѣшишь и неправдою живешь? Знакъ есть, что еще во грѣхахъ пребываеши, и отъ грѣховъ не свободился ты; а тако отъ страданій и смерти Хрістовой никакой пользы не имѣешь ты, хотя она и \textit{всесильное} есть врачевство. Покайся убо и умертви грѣховныя прихоти твоя, и тогда смерть Хрістова будетъ пользовать тебѣ. Распялся Хрістосъ за грѣхи наши, должно и намъ распинать плоть свою со страстьми и похотьми, когда хощемъ Хрістовыми быть. \textit{Иже бо Хрістовы суть, плоть распяша со страстьми и похотьми}\footnote{Гал.~5,~24.}. Вотъ знаки и примѣты Хрістовыхъ рабовъ: \textit{плоть распяша со страстьми и похотьми}. Умеръ Хрістосъ за грѣхи наша: должно и намъ отрещися ихъ, и попрощаться съ ними, и умереть имъ, да чрезъ Хріста и со Хрістомъ оживемъ. Осмотрись убо, хрістіанине, чій еси рабъ, кому живешь, кому работаешь, къ какому концу идешь, къ какой части надлежишь, ко Хрісту, или противнику Его діаволу. Чію волю кто творитъ, кому угождаетъ, работаетъ, того и рабъ есть. 17)~Что Хрістосъ Господь нашъ былъ презрѣнъ, уничиженъ, посмѣянъ, поруганъ и обнаженъ въ своемъ страданіи, "--- отъ того учимся мы, хрістіане, богатства, чести и славы въ мірѣ семъ не искать, но все тое презирать, и всего ради нужды нашей, а не ради роскоши и плотоугодія употреблять. Стыдно хрістіанину чести и славы въ мірѣ семъ искать, когда Хрістосъ Господь его посмѣянъ и поруганъ былъ за него. Срамно хрістіанину богатства желать и искать, когда Хрістосъ Господь его въ страданіи Своемъ и рубища не имѣлъ. Безстыдно хрістіанину въ роскошахъ и сластяхъ валяться, когда Хрістосъ Господь его горькую страданія чашу за него пилъ. Бѣдный убо тотъ хрістіанинъ, который хощетъ въ мірѣ семъ обогатитися, прославитися, въ честь произойти и въ роскошахъ и веселостяхъ жить: знакъ есть, что онъ забылъ, что Хрістосъ за него умеръ и пострадалъ; и видно, что не за велико почитаетъ страданіе и смерть Сына Божія. Осмотри убо себе, хрістіанине, въ семъ важномъ дѣлѣ, и обрати умъ твой къ страстямъ Хрістовымъ, и самъ познаешь, что ты противно хрістіанской вѣрѣ дѣлаешь. 18)~Отъ Хрістовыхъ страданій и всего спасительнаго Его промысла о человѣкѣ примѣчаемъ и видимъ, коль великое достоинство, честь и благородіе человѣческое. Самъ Богъ, "--- что всякій разумъ и удивленіе превосходитъ, "--- Самъ Богъ пришелъ къ человѣку, отступившему отъ Него, и въ человѣка вообразился. Воистину \textit{велія благочестія тайна! Богъ во плоти явися}, и, что болѣе того, пострадалъ за человѣка. Надобно быть \textit{великому}, ради чего такъ велико и умомъ непостижимое смотрѣніе Божіе содѣлалося; надобно \textit{быть дорогой} вещи, за которую такъ дорогая цѣна, кровь Сына Божія, дана. Видно, что и Богу нашему дорогъ есть \textit{человѣкъ}, ради котораго Самъ въ міръ пришелъ и живоносною Своею плотію пострадалъ. Дивное и благороднѣйшее созданіе Божіе "--- \textit{человѣкъ!} Особливымъ Божіимъ совѣтомъ сотворенъ: \textit{сотворимъ человѣка}. По \textit{образу Божію} и по подобію Божію сотворенъ человѣкъ. Но когда палъ и погиблъ человѣкъ, дивный и непостижимый о немъ промыслъ Богъ показалъ. Кромѣ того, что подалъ ему законъ Свой, подалъ ему слово Свое, яко свѣщу сіяющую въ темномъ мѣстѣ, послалъ къ нему пророковъ, повелѣлъ ангеламъ Своимъ хранить его; кромѣ того, что безчисленными благами, видимыми и невидимыми, снабдѣваетъ его, "--- кромѣ всего того, Самъ съ небеснымъ Своимъ воинствомъ пришелъ \textit{взыскати погибшую сію драхму}, и человѣка плѣненнаго отъ врага избавить и свободить, и царское ему благородіе возвратить. Ахъ, бѣдный человѣкъ, любезное и высокопочтенное, но падшее и погибшее созданіе! смотри, какой ты чести отъ Бога сподобился, какъ высоко почтилъ тебе Богъ! Самъ Богъ со ангелами Своими пришелъ къ тебѣ, взыскати и спасти тебе, и пострадалъ за тебе. Видиши убо честь твою! Сколько паденіемъ обезчестился человѣкъ, столько воплощеніемъ Сына Божія и спасительнымъ Его смотрѣніемъ почтенъ человѣкъ. Въ подлѣйшее состояніе грѣхъ его вринулъ: \textit{человѣкъ въ чести сый не разумѣ, приложися скотомъ несмысленнымъ, и уподобися имъ}\footnote{Пс.~48,~13.}; но пришествіемъ Хріста Сына Божія высше всея твари превознеслся. О, когда бы человѣкъ такъ Бога почиталъ, какъ онъ почтенъ отъ Бога! Познавай, человѣче, подлость и бѣдность твою, отъ грѣха тебѣ прибывшую, и смиряйся; познавай и честь твою, сотворенную тебѣ отъ Хрістова пришествія и страданія, и сердечно и со всякимъ смиреніемъ благодари Ему. \textit{Господи! что есть человѣкъ, яко познался еси ему? или сынъ человѣчь, яко вмѣняеши его? Человѣкъ суетѣ уподобися}\footnote{143,~3 и 4.}. Подлый человѣкъ въ себѣ сдѣлался; но дивная къ нему благость Божія высоко почтила и превознесла его. Человѣче! дорогъ ты Богу еси: за дражайшее и смотрѣніе Его о тебѣ имѣй. Почтилъ тебе такъ Богъ: сердечно и ты почитай Бога, почетшаго тебе. Дивную благость и любовь изліялъ на тебе Богъ: лобызай и ты сію благость и любовь Его. Богъ истинно почитается, когда каемся за грѣхи, которыми Онъ оскорбляется, и всякаго грѣха бережемся, да не оскорбимъ Его, и угодное волѣ Его святой творимъ. Кайся убо и берегись грѣха, и твори, чего воля Его святая хощетъ, и будешь Бога, почетшаго тебе, почитать. 19)~Отсюду послѣдуетъ, что и намъ должно всякаго человѣка почитать. Почтилъ Хрістосъ Господь нашъ человѣка, надобно и намъ Его почитать. Почтенному отъ царя земнаго человѣку достойно отдаемъ честь, кольми паче тому, котораго Царь небесный почтилъ, должно отдавать честь. Како будешь уничижать, презирать, клеветать, ругать, злословить, смѣяться тому, котораго Самъ Богъ почтилъ? Смотри, хрістіанине, берегись презирать и уничижать всякаго человѣка, котораго Царь небесный такъ высоко почтилъ. 20)~Отсюду видишь, что тяжко грѣшатъ тіи хрістіане, которыи людей презираютъ, уничижаютъ, ругаютъ и поносятъ. Дѣлается обида и непочтеніе самому царю, когда почтенному отъ него человѣку не отдается честь. Тако дѣлается обида и непочтеніе Самому Царю небесному, Іисусу Хрісту, когда почтеннаго отъ Него человѣка презираемъ и уничтожаемъ. Смотри, хрістіанине, разсуждай, что ты дѣлаешь, когда брата своего уничижаешь и поносишь. Уничижитель уничиженъ будетъ, презиратель презрѣнъ будетъ, ругатель поруганъ будетъ. 21)~Хрістосъ Господь нашъ Своимъ страданіемъ и смертію отъ вѣчныя смерти и всего во адѣ мученія избавилъ насъ, и даровалъ намъ вѣчный животъ. Отсюду видишь, хрістіанине коль великое бѣдствіе есть вѣчное мученіе, и коль великое блаженство вѣчныя жизни. Ничимъ мы не могли, мы грѣшники, избавитися отъ вѣчнаго онаго зла, и вѣчное оное блаженство получить, какъ только страданіемъ и смертію единороднаго Сына Божія. Самъ Себе Хрістосъ Господь нашъ не пощадѣлъ ради того; кровь Свою ради искупленія нашего подалъ цѣну. Слава человѣколюбію Его, слава благости Его и милосердію Его! Великое зло, отъ котораго великою цѣною избавляемся: и великое есть добро и дорогое, которое великою и дорогою цѣною купуется. Надобно быть безконечному и ужасному злу "--- вѣчному мученію, отъ котораго безцѣнною смерти Хрістовой цѣною искуплены мы: надобно быть и блаженству оному великому и непостижимому, за которое безцѣнная дана цѣна. Грѣхомъ мы его лишилися, и тому бѣдствію подпали. Берегись убо грѣха, хрістіанине, да не вринетъ тебе въ бѣдственное оное вѣчное состояніе. 22)~Отъ Хрістова страданія и всего спасительнаго Его о насъ смотрѣнія учимся, что первѣйшее наше дѣло и тщаніе должно быть о спасеніи нашемъ; прочее все, къ міру сему и временной жизни надлежащее, послѣднейшее. Весьма желаетъ Богъ, какъ видимъ изъ Хрістова страданія и всего Его смотрѣнія, спастися намъ, должно быть желаніе и наше. Аки алчетъ и жаждетъ преблагій Богъ спасенія нашего, должна быть и наша алчба и жажда къ тому. Якоже алчущій хлѣба и жаждущій питія желаетъ, тако мы должны желать и искать спасенія. Всякое тщаніе, какъ видимъ, милосердый Богъ полагаетъ, чтобы насъ къ покаянію обратити и спасти; да будетъ и наше все тщаніе о томъ. Нужна намъ пища, одѣяніе, домъ и прочее; но спасеніе такъ нужно, что безъ того весь міръ ничто. Предрагая цѣна дана за спасеніе наше, кровь Хрістова; да будетъ убо и намъ оно дорого, и дражайшее паче всего міра, дражайшее паче неба и земли, дражайшее паче всего сокровища міра сего; безъ того бо все ничто. Нѣтъ никакой пользы, гдѣ нѣтъ спасенія души. Ищемъ благихъ міра сего; кольми паче искать должно благъ вѣчныхъ. Печемся о здравіи тѣла смертнаго; кольми паче должно пещися, чтобы оздоровѣла безсмертная душа. Хранимъ временный животъ; кольми паче должно берещи вѣчный животъ, ради котораго все и самый временный животъ оставитъ должны мы, когда того нужда потребуетъ. Всѣхъ истинныхъ хрістіанъ первѣйшее дѣло и тщаніе было и есть, спасенія своего, яко дорогою цѣною купленнаго, со всякимъ усердіемъ желать и искать. Хрістіанине! да будетъ и твое сіе первѣйшее дѣло и тщаніе, когда хощешь истиннымъ быть христіаниномъ. Хрістосъ Господь Самъ Собою пріобрѣлъ тебѣ спасеніе: берегись же, чтобы сатана не исхитилъ его изъ рукъ твоихъ. Грѣхъ всякій и пристрастіе къ міру затворяетъ двери къ вѣчному спасенію, берегись же всего того.

23)~Отъ Хрістовыхъ страданій учимся, что слѣдуетъ и хрістіанамъ въ мірѣ семъ страдать, быть посмѣянными, поруганными и уничиженными, и прочее озлобленіе терпѣть отъ любителей міра сего. Надобно бо со страждущею главою и удамъ ея, и съ господиномъ страждущимъ рабамъ его страдать. Хрістіане отъ Хріста, суть раби Хрістовы, суть уды Хрістовы духовные. Міръ, прелестію помраченный, Хріста Господа ненавидѣлъ и гналъ: ненавидитъ и гонитъ и хрістіанъ, яко Хрістовыхъ, Хріста придержащихся, Хріста любящихъ и Ему послѣдующихъ. \textit{Нѣсть бо рабъ болій господа своего}\footnote{Іоан.~15,~20.}. Изгнали Хріста Господа, чего ожидать и рабамъ Его, кромѣ гоненія? Предсказано сіе отъ Самого Бога и въ святомъ Писаніи написано: \textit{будете ненавидими всѣми имене Моего ради}\footnote{Мѳ.~10,~22.}. \textit{Въ мірѣ скорбни будете. Аще міръ васъ ненавидитъ, вѣдите, яко Мене прежде васъ возненавидѣ. Аще отъ міра бысте были, міръ убо свое любилъ бы: но Азъ избрахъ вы отъ міра, сего ради ненавидитъ васъ міръ}\footnote{Іоан.~16,~33; 15,~18 и 19.}. \textit{Вси хотящіи благочестно жити о Хрістѣ Іисусѣ, гоними будутъ}\footnote{2~Тим.~3,~12.}. \textit{Многими скорбьми подобаетъ намъ внити въ царствіе Божіе}\footnote{Дѣян.~14,~22.}. И гласъ небесный свидѣтельствуетъ о избранныхъ Божіихъ: \textit{сіи суть, иже пріидоша отъ скорби великія}, и проч.\footnote{Апок.~7,~14.} Предсказана христіанамъ скорбь, да пріуготовляютъ себе къ терпѣнію скорби. \textit{Чадо, аще приступаеши работати Господеви Богу, уготови душу твою во искушеніе}\footnote{Сир.~2,~1.}. Нѣтъ, чего убо ради хрістіанамъ златые дни мечтать и ожидать. Скорбь имъ предсказана, "--- и видимъ тое. Отъ кого"=де хрістіанамъ скорбь терпѣть? Мучители престали, хрістіане живутъ между хрістіанами. Престали мучители, подлинно: слава Богу о семъ! Но не престалъ діаволъ, первѣйшій мучитель, который непрестанно воздвигаетъ гоненіе на благочестивыхъ, по Писанію: \textit{и разгнѣвася змій на жену, и иде сотворити брань со оставшимъ сѣменемъ ея, иже заповѣди Божія соблюдаютъ, и имѣютъ свидѣтельство Іисусъ Хрістово}\footnote{Апок.~12,~17.}. Вотъ первѣйшій мучитель христіанскій! Іисусъ Хрістосъ Господь нашъ отъ Своихъ пострадалъ, якоже есть писано: \textit{во своя пріиде, и свои Его не пріяша}\footnote{Іоан.~1,~11.}. Пророки Его отъ своихъ единоплеменниковъ пострадали, какъ видимъ въ Писаніи\footnote{Мѳ.~13,~57; 23,~37.}. Тако и хрістіане отъ своихъ хрістіанъ, но ложныхъ, страждутъ. Злый сосѣдъ добраго сосѣда ненавидитъ и гонитъ; злый мужъ добрую жену, и злая жена добраго мужа ненавидитъ и гонитъ; злый братъ и сестра добраго брата и сестру ненавидитъ и гонитъ. \textit{И врази человѣку домашніи его}\footnote{Мѳ.~10,~36.}. Вотъ хрістіанскіе гонители "--- лживіи хрістіане! Сюды принадлежатъ господа, которыи крестьянъ своихъ или выше мѣры наказуютъ и мучатъ, или безчестными и поносными словами укоряютъ и ругаютъ, или хуже псовъ своихъ имѣютъ тѣхъ, за которыхъ Хрістосъ пострадалъ и умеръ, или излишними работами и оброками отягчаютъ, такъ что не имѣютъ они пропитанія и одѣянія отъ скудости; все ихъ добро единъ господинъ пожираетъ. Вотъ хрістіанскіе мучители, но политичніи! Видишь убо, хрістіанине, отъ кого хрістіане страждутъ. Ожидай убо и самъ страданія и готовься къ терпѣнію, когда хощешь благочестно жить. Злый языкъ "--- великій есть гонитель хрістіанскій. Какъ тѣло уязвляется мечемъ и жезломъ, тако душа уязвляется поноснымъ словомъ. Но сказано хрістіанамъ отъ Хріста во утѣшеніе: \textit{блажени есте, егда поносятъ вамъ, и ижденутъ, и рекутъ всякъ золъ глаголъ на вы лжуще Мене ради. Радуйтеся и веселитеся, яко мзда ваша многа на небесахъ}\footnote{5,~11 и 12.}. 24)~Отъ великаго терпѣнія Хрістова учимся и мы всякое злостраданіе, приключающееся намъ, великодушно терпѣть. Не противился Хрістосъ врагамъ Своимъ, не воспротивимся и мы. Не отмщевалъ Хрістосъ врагамъ Своимъ: не отмстимъ и мы. Кротко терпѣлъ Хрістосъ поношенія, посмѣянія и поруганія, кротко претерпимъ и мы. Умеръ Хрістосъ за насъ, умремъ и мы за Него, когда нужда потребуетъ. \textit{Хрістосъ пострада по насъ, намъ оставль образъ, да послѣдуемъ стопамъ Его}, и проч.\footnote{1~Петр.~2,~21.} Хрістосъ за распинателей Своихъ молился: \textit{Отче, отпусти имъ}\footnote{Лук.~23,~34.}. Да молимся и мы за враговъ нашихъ: \textit{Господи, не постави имъ грѣха сего}\footnote{Дѣян.~7,~60.}. Тако \textit{отречемся себе}, тако \textit{понесемъ крестъ свой}, тако \textit{послѣдовать будемъ Хрісту}, тако \textit{будемъ новая тварь о Хрістѣ}, тако \textit{будемъ Ему сообразны}; тако \textit{будемъ съ Нимъ страдать, да и съ Нимъ прославимся}\footnote{Мѳ.~16,~24; 2~Кор.~5,~17; Римл.~8,~29 и 17.}. Будемъ истинніи хрістіане, будемъ истинніи раби Его, будемъ живіи уды Его, будемъ съ Нимъ здѣ, и тамо будемъ; будемъ имѣть въ пришествіи Его второмъ знаменіе и свидѣтельство, что мы въ мірѣ семъ Его были, и тако Онъ насъ за Своихъ признаетъ тогда. 25)~Страданіе Хрістово и всѣ спасительныя Его заслуги, ради насъ сотворенныя, суть безконечныя важности, силы и достоинства ради лица Его, Которое есть \textit{совершенный Богъ и совершенный человѣкъ}. Сего ради кающемуся и сердечно вѣрующему во Хріста всѣ и всякіе грѣхи, какіе бы онъ ни сотворилъ, коль бы многіи, тяжкіи и ужасныи ни были, отпущаются, и отпущаются отъ единой благодати. Понеже заслуги Христовы высшіи и сильнѣйшіи всѣхъ и всякихъ нашихъ грѣховъ. Погубляетъ убо человѣка не величество, не множество грѣховъ, но нераскаянное и ожесточенное сердце.

26)~Хрістосъ Господь нашъ Своимъ страданіемъ за насъ обдолжилъ насъ вѣчно, понеже Онъ, пострадавый за насъ, есть Богъ, единъ Сый Святыя Троицы; мы грѣшники и раби неключиміи, за которыхъ пострадалъ. Обдолжилъ Онъ насъ вѣчно, понеже отъ вѣчныя смерти и муки избавилъ насъ не сребромъ или златомъ, но честною Своею кровію. Обдолжилъ Онъ насъ вѣчно, понеже къ вѣчной жизни искупилъ насъ, и къ вѣчному Божіему царствію дверь отворилъ. Вѣчное благодѣяніе, и такъ чуднымъ образомъ, и отъ безконечнаго лица сотворенное, вѣчнаго требуетъ и благодаренія. Сего ради достойно со всею святою церковію вѣчное благодареніе, хваленіе, пѣніе и славословіе приносимъ Ему со Отцемъ и Святымъ Духомъ. Слава Богу, благоволившему тако! 27)~Отъ страданія Хрістова учимся, что какъ \textit{оправданіе наше}, такъ \textit{и спасеніе вѣчное въ единой благодати Божіей и заслугахъ Хрістовыхъ, вѣрою воспріятыхъ, состоитъ}. О семъ на многихъ святаго Писанія мѣстахъ свидѣтельствуется. Однакожъ вѣру тую должно отъ дѣлъ показать, по апостольскому ученію: \textit{покажи ми вѣру твою отъ дѣлъ твоихъ}\footnote{Іак.~2,~18.}. Истинному бо покаянію и вѣрѣ неотмѣнно послѣдуетъ исправленіе и обновленіе сердца и внѣшняго житія. Истинное покаяніе и вѣра обновляетъ и исправляетъ человѣка, отвращаетъ отъ суеты міра, подвигаетъ къ желанію и исканію вѣчнаго живота, страху Божію научаетъ, противу всякаго грѣха подвизаться и добро творить, и Богу угождать безпрестанно увѣщаваетъ. Хрістіанине! спасеніе тебѣ уготовано отъ Хріста: берегись тое нерадѣніемъ своимъ потерять. Буди убо въ истинномъ покаяніи и вѣрѣ до конца, и благодатію Божіею получишь спасеніе, "--- чего тебѣ сердечно желаю. 28)~Хрістово страданіе хрістіанамъ, безъ покаянія и исправленія живущимъ, не только не пользуетъ, но и будетъ во обличеніе и горшее осужденіе. Хрістосъ, Который за грѣхи міра умеръ и пострадалъ, \textit{Тойже будетъ и судить} міру за грѣхи\footnote{Іоан.~5,~22; 2~Кор.~5,~10.}. Видѣлъ міръ Хріста живущаго на земли, яко человѣка, тогда увидитъ, яко Бога. Видѣлъ міръ Хріста на земли въ смиреніи, тогда увидитъ въ славѣ Божественной и страшной. Видѣлъ міръ Хріста на земли въ кротости и долготерпѣніи, тогда увидитъ гнѣвъ Его праведный на грѣшниковъ нераскаянныхъ. Видѣлъ міръ любовь и милосердіе Хрістово ко грѣшникамъ на земли, тогда увидитъ правду Его. Видѣлъ міръ Хріста на земли, отпущающаго грѣшникамъ кающимся согрѣшенія, тогда увидитъ воздающаго всѣмъ по дѣламъ ихъ. Видѣлъ міръ на земли Хріста, судимаго отъ беззаконныхъ, тогда увидитъ судящаго беззаконнымъ. Слышалъ міръ вопль беззаконниковъ на Хріста: \textit{возми, возми, распни Его}, "--- тогда услышитъ міръ вопль нераскаянныхъ грѣшниковъ горамъ и каменію: \textit{падите на ны и покрыйте ны отъ лица сѣдящаго на престолѣ, и отъ гнѣва Агнча: яко пріиде день великій гнѣва Его, и кто можетъ стати}\footnote{Апок.~6,~15 и 16.}! Видѣлъ міръ Хріста, на смерть осужденнаго отъ беззаконниковъ, тогда увидитъ Его, осуждающаго беззаконниковъ на вѣчную смерть: \textit{идите отъ Мене проклятіи во огнь вѣчный, уготованный діаволу и аггеломъ его}\footnote{Мѳ.~25,~41.}. Видѣлъ міръ Хріста, ко кресту пригвожденнаго и между двухъ злодѣевъ висящаго, тогда увидитъ Его на престолѣ славы сѣдящаго и небеснымъ воинствомъ окружаемаго, и неизреченнымъ свѣтомъ сіяющаго, и весь отъ страха славы тоя ужаснется и вострепещетъ. Души же благочестивіи, радуйтеся и веселитеся, яко приближися избавленіе ваше. Отъ Хрістова бо страданія, какъ благочестивымъ и кающимся проистекаетъ живое утѣшеніе, такъ нечестивымъ и нераскаяннымъ грѣшникамъ будетъ обличеніе и осужденіе большее. Видимъ въ святомъ Евангеліи, что, по воскресеніи Хрістовомъ изъ мертвыхъ, \textit{знаки язвъ Хрістовыхъ}, на тѣлѣ Его святомъ бывшихъ, \textit{осталися}\footnote{Іоан.~20,~27.}. Промысломъ тое Божіимъ учинилося. Сіи знаки во второмъ Хрістовомъ пришествіи будутъ во утѣшеніе и радость вѣрнымъ и святымъ Его, "--- во обличеніе и большее осужденіе нечестивымъ и грѣшникамъ нераскаяннымъ, что они неблагодарны явилися къ Тому, Который за спасеніе ихъ пострадалъ и ко кресту пригвожденъ былъ и умеръ, и такъ дорогою цѣною спасеніе имъ пріобрѣтенное пренебрегли. Пишется о Іудеяхъ, распеншихъ Хріста: \textit{воззрятъ нань, Егоже прободоша}\footnote{Іоан.~19,~37; Зах.~12,~10.}. \textit{Воззрятъ} и нечестивіи грѣшники нань, и увидятъ язвы рукъ и ногъ и ребра Его, и обыметъ ихъ страхъ, трепетъ и ужасъ. И тако имъ страданіе Хрістово будетъ во обличеніе и большее осужденіе, что толикою благодатію Божіею не хотѣли пользоватися и спастися. Къ сему числу принадлежатъ блудники, прелюбодѣи, малакіи и вси скверноживущіи, воры, хищники, грабители, судіи "--- мздоимцы, помѣщики, крестьянъ своихъ отягчающіи, купцы, обманывающіи въ товарахъ, и дешевую вещь за дорогую, и худую за добрую продающіи, удерживающіи мзду наемничу, клеветники, ругатели, лживіи, хитрецы и лукавцы, піяницы, родителямъ и властямъ своимъ противляющіися, чары творящіи и призывающіи ихъ, и прочіи беззаконники, въ нераскаяніи и неисправленіи живущій. Всѣмъ таковымъ Хрістово страданіе и святое Его Евангеліе и Самъ Хрістосъ, за всѣхъ пострадавый, не токмо ничего не пользуетъ, но въ большее осужденіе и обличеніе будетъ. Покайся, хрістіанине, и начни новое житіе, прежнему беззаконному житію противное, и будетъ Хрістосъ \textit{твой} со всѣми заслугами Своими и вѣчными благими. 29)~Видишь, хрістіанине, коликая происходитъ польза отъ памяти и размышленія страстей Хрістовыхъ. Вѣрное и прилѣжное разсужденіе Хрістовыхъ страданій научаетъ насъ: каятися и жалѣть за грѣхи, которыи причиною были страстей и болѣзней Хрісту Господу нашему; познавать, коль лютое и пагубное зло есть грѣхѣ; берещися всякаго грѣха, яко смертоноснаго и всепагубнаго зла; презирать міръ со всѣми прелестьми и похотьми его, не желать чести, славы, богатства и веселости міра сего; горняя мудрствовать, а не земная; дорого почитать сокровище вѣчнаго спасенія, яко дражайшею цѣною крови Хрістовой купленное; познавать, что мы чрезъ грѣхъ сдѣлалися и что чрезъ Хрістово страданіе пріобрѣли, въ какое бѣдствіе чрезъ грѣхъ впали, и коль великое и непостижимое блаженство чрезъ Хрістово страданіе къ намъ возвратилося, и за все сіе усердно Богу, тако насъ помиловавшему, благодарить; познавать гнѣвъ Божій за грѣхи, неумолимый судъ Божій, непремѣняемую правду Божію, яко она грѣха безъ наказанія не оставляетъ, "--- непостижимую Божію премудрость, которая тамо изобрѣтаетъ способъ спасенія, гдѣ не видится, и гдѣ намъ не возможно, тамо Богу все возможно; познавать горячую Божію къ роду человѣческому любовь, и взаимно любитъ Его и, по слову Его, ближняго нашего любить и почитать, и добро творить ему, и проч. Тако разсужденіе Хрістовыхъ страданий перемѣняетъ и обновляетъ человѣка, и дѣлаетъ его инымъ, нежели какъ прежде былъ. Хрістово страданіе есть какъ книга спасительная, изъ которой всего добра, то"=есть, покаянія, вѣры, богопочитанія, любви ближняго, смиренія, кротости, терпѣнія, презрѣнія міра и всея суеты его учимся, и, аки шпорами, поощряемся и возбуждаемся къ непрестанному желанію и прилѣжному исканію будущія жизни и вѣчныхъ ея благъ. Обращай убо, хрістіанине, сердечныя очи твои къ страстямъ Хрістовымъ, и часто ихъ поминай и разсуждай, и обновишися. Кто что любитъ и почитаетъ, у того всегда тое на умѣ и размышленіи, въ томъ умъ и сердце его. Чрезъ страданіе Хрістово вѣчныя погибели мы избавилися, и вѣчное получили блаженство, и какъ сего хрістіанину не поминать? Сладко и благопріятно поминать способъ и образъ тотъ, которымъ мы отъ великаго бѣдствія избавилися и всякое вѣчное добро получили. Хрістосъ Господь нашъ тое намъ Своимъ страданіемъ сотворилъ. Воспоминаніе прежняго бѣдствія и разсужденіе настоящаго блаженства утѣшаетъ, оживляетъ, радостотворитъ и услаждаетъ человѣка. Погибшіи мы были, но Хрістовымъ страданіемъ спаслися, и вѣчную славу, жизнь, радость и вся благая получили. Сладко и радостно сіе поминать. Только берегись, хрістіанине, чтобы прежнему бѣдствію не подпасть. Двери царствія Божія отверсты смертію Хрістовою. Входятъ въ тое кающіися грѣшники и творящіи дѣла покаянія. Берегись ты тѣхъ дверей себѣ заключить грѣхами и нераскаяннымъ житіемъ. Имѣютъ люди обычай на картинахъ написывать сраженія, подвиги воиновъ противу враговъ и надъ тѣми одержанныя побѣды; и, на тыя картины взирая, съ радостію поминаютъ побѣды надъ врагами одержанныя, и тако утѣшаются. Изрядный и душеспасительный обычай приняла Церковь святая писать образы страстей Хрістовыхъ и тыя вѣрнымъ представлять къ превеликой пользѣ и утѣшенію. Хрістіанине! на сихъ картинахъ представляется тебѣ подвигъ, въ который вступилъ за Насъ Сынъ Божій, и подвизался не оружіемъ, но терпѣніемъ креста и страданія; подвизался противу діавола и всѣхъ нашихъ враговъ, и надъ ними преславную и вѣчную одержалъ побѣду, и тую даровалъ намъ такъ, что надѣемся благодатію Его нѣкогда торжествовать и восклицать: \textit{гдѣ твое, смерте, жало? гдѣ твоя, аде, побѣда?} и проч. \textit{Богу же благодареніе, давшему намъ побѣду Господемъ нашимъ Іисусъ Хрістомъ}\footnote{1~Кор.~15,~55 и 57.}. Утѣшительно и радостно подвигъ сей Сына Божія и побѣду Его надъ врагами нашими одержанную поминать, и на тую взирать. За насъ Онъ въ подвигъ тотъ вступилъ, намъ и побѣду надъ врагами нашими одержалъ. Не могли мы сами никакимъ образомъ побѣдить враговъ нашихъ и отъ нихъ избавиться. Хрістосъ убо Господь и заступникъ нашъ за насъ вступился и побѣдилъ и попралъ ихъ, и восторжествовалъ надъ ними, и тако насъ отъ Нихъ похитилъ и избавилъ. Слава смерти и ада побѣдителю, Іисусу! Поемъ Ему радостно побѣдную пѣснь: \textit{славно бо прославися!} Сію преславную и душеспасительную побѣду представляетъ намъ образъ Хрістовыхъ страданій. Образъ есть какъ всегдашняя книга, въ которой образуемое читаемъ. Въ книгѣ читаемъ, что и какъ сдѣлалось. Образъ и картина все тое не токмо душевнымъ, но и тѣлеснымъ очамъ представляетъ, и живо изображаетъ, и тако въ сердце смотрящаго ударяетъ. Не возможно бо человѣку, смотрящему на образъ Хрістовыхъ страданій, не тронуться и не содрогнуться, страхъ или утѣшеніе, или сокрушеніе и жалѣніе за грѣхи въ сердцѣ не почувствовать, наипаче, когда съ разсужденіемъ и благоговѣніемъ смотритъ. Толикую пользу подаетъ образѣ Хрістовыхъ страданій написанный! Аще убо хощеши, хрістіанине, всегдашнюю память Хрістовыхъ страданій имѣть, и отъ той душѣ своей пользу получать, имѣй у себе въ домѣ написанныя страсти Хрістовы, и на тыя часто съ благоговѣніемъ смотри, и будутъ тебѣ вмѣсто всегдашняго чтенія и очевидной исторіи. Выбрось изъ дома твоего маскарадныя картинки, которыя соблазняютъ, разжигаютъ и разслабляютъ плоть твою, а напиши оныя подвига и побѣды Хрістовой надъ врагами нашими знаменія, которыи созидаютъ душу твою. Изъ нихъ всего добра учиться будешь. Онѣ будутъ тебѣ напоминать всегда, кто есть Спаситель твой и Искупитель, отъ чего и къ чему и чимъ Онъ Тебе искупилъ; напоминать любовь Его къ тебѣ, и твою къ Нему должность; напоминать прежнее твое бѣдствіе, и настоящее и будущее блаженство, что ты прежде былъ, и что нынѣ имѣешися. Сія спасительная картина представитъ тебѣ, что ты въ мірѣ еси, и какъ ты долженъ въ мірѣ обращаться. На сію картину смотря, не захощешь въ мірѣ семъ богатитися, славитися, веселитися и въ роскошахъ быть; но и всего того, что имѣешь, недостойнымъ будешь почитать себе. Она всегда будетъ тебе увѣщавать не отмщевать, но прощать обиды ближнему, благотворить не токмо другамъ, но и врагамъ, и молиться за всѣхъ. Словомъ, что во Евангеліи святомъ написано, тое все въ Хрістовомъ страданіи изображено, и къ подражанію нашему представляется. 30)~Какъ все спасительное Хрістово смотрѣніе, такъ и Страданіе Его, и Самого пострадавшаго Хріста будешь почитать, когда великое сіе искупленія твоего дѣло Его за дорого будешь имѣть и во всегдашней памяти содержать, и сердечно Ему за тое благодарить, и въ истинномъ покаяніи и жалѣніи за грѣхи свои будешь, яко тіи толикую болѣзнь и страданіе Хрісту содѣлали, отъ великаго грѣха удаляться будешь, и все, что волѣ Его святой угодно, творить будешь, и, возненавидѣвши суету міра сего, Ему единому, яко свѣту и животу нашему, прилѣпишися, и смиреніемъ, любовію, кротостію и терпѣніемъ будеши послѣдовать. Тако Хрістосъ и спасительное Его смотрѣніе истинно почитается. Многіи хрістіане почитаютъ Хрістово страданіе и пострадавшаго Хріста устами, но сердцемъ отвращаются отъ Него. О таковыхъ глаголетъ Господь: \textit{приближаются Мнѣ людіе сіи усты своими, и устами чтутъ Мя: сердце же ихъ далече отстоитъ отъ Мене}\footnote{Мѳ.~15,~8.}. 31)~Отъ Хрістовыхъ страданій учимся со смиреніемъ и вѣрою приступать къ небесному Отцу, и во имя Хрістово просить у Него всего, что къ спасенію нашему нужно и полезно, то"=есть, отпущенія грѣховъ, благодати, обновленія и прекрасныхъ оныхъ \textit{Духа Святаго плодовъ}, то"=есть, \textit{любви, радости, мира, долготерпѣнія, благости, милосердія, вѣры, кротости, воздержанія}, наконецъ, жизни вѣчныя и небесныхъ благъ\footnote{Гал.~5,~22 и 23.}. Аще бо Онъ Сына Своего къ намъ послалъ, и на страданіе и смерть предалъ за насъ: како не подастъ всего того, что волѣ Его угодно и намъ полезно? \textit{Иже убо Сына своего не пощадѣ, но за насъ всѣхъ предалъ есть Его, како убо не и съ Нимъ вся намъ дарствуетъ}\footnote{Римл.~8,~32.}? Только и сами о себѣ да не нерадимъ. \textit{Просите, и дастся вамъ; ищите, и обрящете; толцыте, и отверзется вамъ}, ободряетъ и обнадеживаетъ насъ самъ Хрістосъ Господь нашъ\footnote{Мѳ.~7,~7.}. \textit{Сердце чисто созижди во мнѣ, Боже, и духъ правъ обнови во утробѣ моей. Не отвержи мене отъ лица Твоего, и Духа Твоего Святаго не отъими отъ мене. Воздаждь ми радость спасенія Твоего, и Духомъ владычнимъ утверди мя}\footnote{Пс.~50,~12--14.}. 32)~Вѣруемъ мы, что \textit{во Хрістѣ два естества}, Божество и человѣчество во единомъ лицѣ соединены, но не сліяны, и потому Хрістосъ Господь нашъ есть совершенный Богъ и совершенный человѣкъ, пострадавый за насъ. Но пострадалъ и умеръ по естеству человѣческому, а не по Божеству. Ибо Божество безстрастно есть, не можетъ терпѣть, страдать болѣзней и умереть; есть бо всегда непремѣнно въ Своемъ всесовершенномъ блаженствѣ. Страдалъ убо Хрістосъ Богъ нашъ Своею душею и Своимъ тѣломъ, какъ выше сказано; а Божество Его безстрастно пребывало въ страданіи Его. Берегись убо, хрістіанине, приписывать болѣзней и страданія Хрістову Божеству. Читай и слушай со вниманіемъ стихи церковныя, чтомыя въ среду, пятокъ и въ воскресные дни, и самъ увидишь тое. Хрістосъ Господь нашъ, поелику Богъ, и тогда, когда по землѣ ходилъ въ человѣческомъ и смиренномъ образѣ, и когда страдалъ за насъ, и тогда, говорю, на престолѣ славы Своея со Отцемъ и Святымъ Духомъ покланяемый и прославляемый отъ ангеловъ былъ. \textit{Во гробѣ плотски, во адѣ же съ душею яко Богъ, въ раи же съ разбойникомъ, и на престолѣ былъ еси, Хрісте, со Отцемъ и Духомъ, вся исполняяй неописанный}.

Но повидимъ далѣе, что Хрістосъ Господь нашъ насъ ради сотворилъ? Умерши Хрістосъ за грѣхи наши, въ третій день восталъ изъ мертвыхъ и \textit{чрезъ четыредесять дней являлся} ученикамъ Своимъ и \textit{апостоламъ}, и прочимъ вѣрнымъ Своимъ, и различно доказывалъ имъ Свое востаніе, и \textit{глаголалъ о царствіи Божіи}\footnote{Дѣян.~1,~3.}. Тако Онъ дѣло Свое сотворилъ, ради котораго въ міръ пришелъ, и предъ всѣми ими очевидно \textit{отъ горы Елеонской вознеслся на небо}, откуду и пришелъ, и сѣлъ одесную Бога Отца\footnote{Дѣян.~1,~9 и 12.}. Тако Онъ, хрістіанине, сдѣлалъ дѣло Свое, и отшелъ. \textit{Царства земная пойте Богу, воспойте Господеви, вошедшему на небо небесе на востоки}\footnote{Пс.~67,~33 и 34.}. Пою Тя, слухомъ бо, Господи, услышахъ и ужасохся. До мене бо пришелъ еси, мене ища заблуждшаго. Тѣмъ многое Твое снисхожденіе, еже для мя, прославляю, многомилостиве. Хрістіанине, постоимъ и здѣ мало, и посмотримъ на спасительное востаніе и вознесеніе Хрістово. 1)~Восталъ изъ мертвыхъ Хрістосъ. Отсюду великое утѣшеніе и радость намъ проистекаетъ. Восталъ Хрістосъ, и тѣмъ показалъ намъ, что Онъ надъ всѣми нашими врагами, діаволомъ, грѣхами, смертію и адомъ, яко сильный въ крѣпости, восторжествовалъ и даровалъ намъ побѣду надъ ними. Уже бо грѣховъ, діавола, смерти и ада не боимся. Хрістосъ наше \textit{оправданіе, освященіе и избавленіе, Иже преданъ бысть за прегрѣшенія наша, и воста за оправданіе наше}\footnote{1~Кор.~1,~30; Римл.~4,~25.}. \textit{Аще бо Богъ по насъ, кто на ны}\footnote{8,~31.}? Сіе утѣшеніе и радость подаетъ намъ востаніе Хрістово. Откуду, по востаніи Своемъ, женамъ мѵроносицамъ глаголалъ: \textit{радуйтеся!}\footnote{Мѳ.~28,~9.} и Апостоламъ благовѣствовалъ: \textit{миръ вамъ}\footnote{Іоан.~20,~20 и 26.}. Какъ бы сказалъ: не бойтеся. Я ваши грѣхи Своею кровію очистилъ; Бога вамъ, и васъ Богу примирилъ; Отца Моего вашимъ Отцемъ сдѣлалъ. \textit{Восхожду ко Отцу Моему и Отцу вашему, и Богу Моему и Богу вашему}\footnote{17.}. Діавола побѣдилъ и посрамилъ, и отъ того власти васъ исхитилъ; отъ смерти и ада васъ избавилъ, и царствіе Божіе вамъ отворилъ. О семъ и всѣмъ концамъ земли проповѣдуйте. \textit{И рече имъ: шедше въ міръ весь, проповѣдите Евангеліе всей твари}, и проч.\footnote{Марк.~16,~15.} Сея радости и утѣшенія вси вѣрніи пріобщаемся. \textit{О пасха велія и священнѣйшая, Хрісте! О мудросте, и Слове Божій, и сило! подавай намъ истѣе Тебе причащатися въ невечернѣмъ дни царствія Твоего}. 2)~Восталъ Хрістосъ; надобно и намъ со Хрістомъ востать, да и на небо съ Нимъ вознесемся. Двоякое есть востаніе, тѣлесное и душевное. Тѣлесное востаніе будетъ въ послѣдній день; о семъ глаголемъ въ Сѵмволѣ святомъ: \textit{чаю воскресенія мертвыхъ}. Душевное востаніе есть отстать отъ грѣховъ, и отъ суеты міра отвратиться, и быть въ истинномъ покаяніи и вѣрѣ, противу всякаго грѣха подвизаться, волю небеснаго Отца творить, правдою Ему жить, и Хрісту Сыну Божію смиреніемъ, любовію, кротостію и терпѣніемъ послѣдовать. И сіе"=то есть новая тварь, о которой глаголетъ Апостолъ: \textit{аще кто во Хрістѣ, нова тварь}, есть новый человѣкъ, обновленный покаяніемъ и вѣрою, есть истинный хрістіанинъ, есть живый удъ Хрістовъ и наслѣдникъ царствія Божія\footnote{2~Кор.~5,~17.}. Новаго сего человѣка дѣло есть: смиренно на земли жить, славы, чести и роскоши всякой убѣгать, горняя мудрствовать, а не земная, зла за зло не воздавать, и досажденія за досажденіе; любить враговъ своихъ, благословить кленущихъ себе, добро творить ненавидящимъ себе, и молиться за творящихъ себѣ напасть и изгонящихъ себе, и проч.\footnote{Мѳ.~5,~44.} Сіи суть дѣла человѣка, воставшаго отъ душевной смерти и въ новой жизни живущаго. Тако кто нынѣ отъ мертвыхъ востанетъ, въ послѣдній день воскреснетъ въ вѣчную жизнь. \textit{Блаженъ и святъ, иже имать часть въ воскресеніи первѣмъ: на нихже смерть вторая не имать области}\footnote{Апок.~20,~6.}. 3)~Вознеслся Хрістосъ на небо, и исполнилося тое, что Самъ Онъ сказалъ: \textit{пріиде Сынъ человѣческій взыскати и Спасти погибшаго}\footnote{Лук.~19,~10.}. Взыскалъ и спаслъ погибшаго человѣка, и вознеслся на небо и привелъ того ко Отцу Своему небесному, и повелѣлъ силамъ Своимъ небеснымъ радоватися о томъ, глаголя: \textit{радуйтеся со Мною, яко обрѣтохъ драхму погибшую}\footnote{15,~9.}. Якоже бо пастырь, видя отлучившуюся овцу отъ стада, исходитъ на взысканіе тоя, и ищетъ ея по горамъ и пустынямъ и, сыскавши, возлагаетъ ее на рамена своя и съ радостію возвращается къ своему стаду, и тому присовокупляетъ ее: тако Пастырь добрый Іисусъ Хрістосъ Господь нашъ, видя человѣка, отлучившагося отъ лика ангеловъ, яко овцу отъ стада, и блудящаго по пустыни міра сего, изшелъ на взысканіе того; и взыскалъ и, возложивъ на рамена Своя, принеслъ къ небесному Своему Отцу, и причислилъ ликамъ ангельскимъ, \textit{глаголя имъ: радуйтеся со Мною, яко обрѣтохъ овцу Мою погибшую}\footnote{6.}. Или якоже царь человѣколюбивый и сильный, видя плѣненныхъ людей своихъ, исходитъ съ воинствомъ своимъ и идетъ въ слѣдъ плѣненныхъ людей своихъ и плѣнившаго ихъ врага, и того поразивши, своихъ изъ рукъ его похищаетъ, и съ радостію возвращается во отечество твое, и тогда радость бываетъ во всемъ отечествѣ о возвращенныхъ изъ плѣна людехъ: тако Царь небесный, Іисусъ Хрістосъ, Господь крѣпокъ и силенъ, и человѣколюбивый Царь, видя человѣка, плѣненнаго отъ врага діавола, изшелъ съ небеснымъ Своимъ воинствомъ, и поразилъ плѣнившаго супостата, и человѣка плѣненнаго изъ рукъ его мучительскихъ похитилъ, и того возвратилъ въ въ небесное отечество; и тако сотворилъ всѣмъ небеснымъ жителямъ радость. О семъ пророкъ Ему святый воспѣлъ, провидя возвращенный отъ Него нашъ плѣнъ: \textit{возшелъ еси на высоту, плѣнилъ еси плѣнъ}\footnote{Пс.~67,~19.}. О семъ буди Ему слава отъ насъ со Отцемъ и Святымъ Духомъ, аминь! 4)~Вознеслся Хрістосъ на небо предъ святыми учениками Своими, и всѣмъ вѣрнымъ Своимъ путь туды показалъ. Вознеслся Хрістосъ Глава на небо, вознесутся и святіи уды Его, истинніи хрістіане. Затворенъ былъ туды путь человѣкамъ, но смертію Хрістовою отворенъ. \textit{Раздралася церковная завѣса} въ смерти Хрістовой, и \textit{открылся путь} и входъ въ царствіе небесное вѣрнымъ\footnote{Мѳ.~27,~51; Евр.~10,~19.}. Хрістіанине! показанъ намъ путь и отворенъ въ царствіе небесное; и вошелъ туды Хрістосъ Господь нашъ, и насъ туды къ Тебѣ призываетъ, насъ, за которыхъ пострадалъ и умеръ; да не нерадимъ убо о себѣ. Когда хощемъ туды вознестися и со Хрістомъ быть, надобно и здѣ въ мірѣ семъ съ Нимъ быть. \textit{Аще кто Мнѣ служитъ, Мнѣ да послѣдствуетъ: и идѣже есмь Азъ, ту и слуга Мой будетъ}\footnote{Іоан.~12,~26.}. Надобно умомъ и сердцемъ, живущему на земли, отлучитися отъ міра сего, и преселитися на небо, и тамо скрывать свое сокровище, кто хощетъ туды взойти, по апостольскому словеси: \textit{наше жилище на небесѣхъ есть, отонудуже и Спасителя ждемъ, Господа нашего Іисуса Хріста}\footnote{Филип.~3,~20.}. Написано о Хрістѣ: \textit{не сія ли подобаше пострадати Хрісту, и внити въ славу Свою}\footnote{Лук.~24,~26.}. Но слышимъ и мы, хрістіане, отъ Писанія святаго: \textit{многими скорбьми подобаетъ намъ внити въ царствіе Божіе}\footnote{Дѣян.~14,~22.}. Надобно убо и хрістіанину благочестиво, свято, смиренно, любовно, кротко и терпѣливо жить на земли; и тако Хрісту послѣдовать, когда хощетъ въ царствіе Божіе внити.

Низкій и смиренный сей путь есть, но безопасный и въ царствіе Божіе ведетъ. 5)~Какъ страданіе и смерть Хрістова, такъ воскресеніе и вознесеніе Его ничего не пользуетъ тѣмъ хрістіанамъ, которыи въ неисправности живутъ и отъ смерти душевно не востали. \textit{Востани} убо \textit{спяй, и воскресни отъ мертвыхъ, и освѣтитъ тя Хрістосъ}\footnote{Еф.~5,~14.}. Видимъ, хрістіанине, первое пришествіе Хрістово; ожидаемъ еще втораго: потщимся убо перваго причастниками быть, да и второе пришествіе Его съ радостію срѣтимъ.

\section{109. Граждане, ожидающіи царя своего во градъ свой.}

Видимъ, что когда граждане ожидаютъ пришествія царя своего во градъ ихъ, къ достойному срѣтенію его пріуготовляются, и между собою часто говорятъ: \textit{царь пріидетъ! царь пріидетъ! когда"=то онъ будетъ, нощію или днемъ? поутру или въ вечеру? и съ какою"=то свитою пріидетъ?} Сія и подобная симъ говорятъ между собою граждане, царскаго ожидая пришествія. Хрістіанине! мы ожидаемъ Царя небеснаго, Іисуса Хріста пришествія къ намъ. Видѣли мы первое Его къ намъ пришествіе, ожидаемъ и втораго, и увидимъ тое. Видимъ, что первое Его пришествіе было въ смиреніи, нищетѣ, кротости и долготерпѣніи; второе будетъ въ страшной и Божественной славѣ. Въ первомъ пришелъ тихо, и \textit{снизшелъ яко дождь на руно и яко капля каплющая на землю}\footnote{Пс.~71,~6; Суд.~6,~37 и 35.}. Во второмъ возсіяетъ, \textit{яко молнія}, которая на востокѣ блистаетъ и является на западѣ\footnote{Мѳ.~24,~27.}. Пріидетъ тогда не страдати за насъ (было уже тое), но судити насъ и воздати всѣмъ по дѣламъ ихъ. Видимъ въ Писаніи святомъ, что будетъ тогда. Видимъ, что \textit{пріидетъ день Господень, яко тать въ нощи, въ оньже небеса убо съ шумомъ мимо пойдутъ, стихіи же сжигаемы разорятся, земля же, и яже на ней дѣла сгорятъ}\footnote{2~Петр.~3,~10.}. Видимъ, что тогда \textit{небо отлучится яко свитокъ свиваемо, и всяка гора и островъ отъ мѣстъ своихъ двинутся, и царіе земстіи и вельможи и богатіи, и тысящницы и сильніи, и всякъ рабъ и всякъ свободь скрыются въ пещерахъ и каменіи горстѣмъ, и возглаголютъ горамъ и каменію: падите на ны и покрыйте ны отъ лица сѣдящаго на престолѣ, и отъ гнѣва Агнча: яко пріиде день великій гнѣва Его, и кто можетъ стати}\footnote{Апок.~6,~14--17.}? Видимъ, что тогда \textit{вси}, отъ начала міра умершіи, востанутъ изъ гробовъ своихъ, \textit{и изыдутъ сотворшіи благая въ воскрешеніе живота, а сотворшіи злая въ воскрешеніе суда}\footnote{Іоан.~5,~28 и 29.}. Видимъ, что тогда отъ всѣхъ концевъ земли \textit{соберутся вси языки}, и станутъ предъ Царемъ небеснымъ, праведнымъ Судіею. Видимъ, что тогда собравшіися вси человѣки на двѣ части раздѣлятся: одни станутъ по правую сторону Судіи, другіи по лѣвую. Тогда всеконечное и послѣднее разлученіе будетъ другъ отъ друга. Видимъ, что тогда на вѣки безконечные разлучится жена отъ мужа, отецъ и мать отъ дѣтей своихъ, цари отъ подданныхъ своихъ, князи, вельможи и господа отъ рабовъ своихъ, други отъ друговъ своихъ, и знаеміи отъ знаемыхъ своихъ. Видимъ и тое, что тогда жена станетъ по правую сторону Судіи, а мужъ по лѣвую, и мужъ по правую, а жена по лѣвую; отецъ и мать по правую, а дѣти ихъ по лѣвую, и дѣти по правую, а отецъ и мать по лѣвую; единъ братъ по правую, а другій по лѣвую. Видимъ, что тогда тоежде будетъ начальникамъ и подначальнымъ ихъ; видимъ, что царь станетъ по правую, а подданніи его по лѣвую, и подданніи по правую, а царь по лѣвую; князи, вельможи и господа по правую, а раби и слуги ихъ по лѣвую, и раби и слуги ихъ по правую, а господа, князи и вельможи по лѣвую станутъ. Видимъ и тое, что святые нынѣ мниміи и почитаеміи отъ людей по лѣвую сторону, а мниміи грѣшники и аки нечестивіи отъ міра вмѣняеміи по правую станутъ. Святіи и праведніи, которыи по правую сторону; грѣшніи и нечестивіи, которыи по лѣвую сторону станутъ. Тогда всякаго познается правда и неправда, добродѣтель и грѣхъ, святыня и скверна. Ибо разобраніе будетъ тогда по совѣсти, а не по чинамъ; по дѣламъ, а не по лицамъ; по внутренности, а не по внѣшности и наружности. Ибо Судія будетъ Богъ, Который испытуетъ сердца и утробы, и судитъ по внутренности сердца, а не по наружности дѣлъ. Человѣкъ смотритъ на внѣшняя человѣка, но Богъ смотритъ на сердце человѣческое. Часто бо бываетъ, что человѣкъ предъ человѣками добръ, но предъ Богомъ золъ; предъ человѣками святъ, но предъ Богомъ скверненъ; предъ человѣками добродѣтеленъ, но предъ Богомъ пороченъ. Видимъ, что по правую сторону стоящіи будутъ сіять, яко свѣтила небесная: \textit{тогда бо праведницы просвѣтятся, яко солнце}\footnote{Мѳ.~13,~43.}. По лѣвую же сторону стоящіи почернѣютъ и въ гнусномъ безобразіи явятся; ибо грѣхи, которыи въ нихъ крыются нынѣ, изыдутъ на верхъ и ужасное въ нихъ содѣлаютъ безобразіе. Тогда покажется всему міру, что они не токмо явно, но и тайно дѣлали, и какія мысли и начинанія въ сердцахъ своихъ имѣли; всякое тогда души безобразіе, отъ грѣховъ ей прилѣпшее, на тѣлѣ грѣшника явится. Тако они увидятъ свое безобразіе и мерзость грѣховную, въ которой въ мірѣ семъ живучи пребывали. Слышимъ, что тогда праведный Судія сущимъ одесную Себе глаголетъ: \textit{пріидите благословенніи Отца Моего, наслѣдуйте уготованное вамъ царство отъ сложенія міра}; а сущимъ ошуюю глаголетъ: \textit{идите отъ Мене проклятіи во огнь вѣчный, уготованный діаволу и аггеломъ его}\footnote{25,~34 и 31.}. Сей жребій есть стоящихъ по правую, и стоящихъ по лѣвую сторону! Хрістіанине, сладко и радостно будетъ слышать: \textit{пріидите}; страшно и горестно будетъ слышать: \textit{идите отъ Мене}. Но неотмѣнно всякъ или тотъ, или сей услышитъ гласъ. Сими двумя гласами окончится страшное и всемірное оное позорище. Вси тогда, услышавши отвѣтъ, отъ Судіи себѣ реченный, пойдутъ въ свои мѣста, но различныя; пойдутъ, но не равно. Одни пойдутъ съ радостію и веселіемъ и торжествомъ въ вѣчную жизнь: другіи съ плачемъ, рыданіемъ и отчаяніемъ въ вѣчную муку. \textit{И идутъ сіи}, глаголетъ Господь, \textit{въ муку вѣчную: праведницы же въ животъ вѣчный}\footnote{46.}. Хрістіанине! таковаго мы пришествія Царя нашего Іисуса Хріста ожидаемъ: приготовимъ убо себе къ достойному срѣтенію Его. Да будутъ и наши разговоры чаще о томъ дни; да разговариваемъ и мы другъ съ другомъ: Царь небесный \textit{идетъ!} Царь небесный \textit{идетъ} къ намъ! \textit{Идетъ} спасти праведники, грѣшники же мучити. Когда"=то \textit{пріидетъ!} въ нощи или во дни? поутру или въ вечеру? въ которомъ мѣсяцѣ, дни и часѣ? Проповѣдники Его то и знаютъ, что возглашаютъ намъ: Царь небесный грядетъ, Царь небесный грядетъ: готовьтеся къ срѣтенію Его! О любезніи благовѣстники, скажите намъ: когда Онъ пріидетъ къ намъ? когда Солнце праведное явится намъ? когда молнія Его блеснетъ и освѣтитъ вселенную? когда поставится престолъ Его святый на судъ, и узримъ Его вси? Скоро идетъ, отвѣтствуютъ они намъ, скоро идетъ и не закоснитъ. \textit{Пришествіе Господне приближися. Се Судія предъ дверми стоитъ}\footnote{Іак.~5,~8 и 9.}. Но не вѣсте дне и часа, въ который Онъ пріидетъ. \textit{Бдите убо, яко не вѣсте, въ кій часъ Господь вашъ пріидетъ}\footnote{Мѳ.~24,~42.}. \textit{Глаголетъ свидѣтельствуяй сія: ей гряду скоро, аминь! Ей, гряди, Господи Іисусе}\footnote{Апок.~22,~20.}! Тому слава во вѣки вѣковъ, аминь.

\section{110. Единъ поемлется, другій оставляется.}

Видимъ сіе въ мірѣ, что \textit{единъ поемлется, другій оставляется}, "--- напримѣръ: единъ поемлется въ честь, а другій оставляется; единъ поемлется въ воинство, а другій оставляется; единъ поемлется на пиръ, другій оставляется, и проч. Хрістіанине! тое будетъ и во второе пришествіе Хрістово. \textit{Тогда будета два на одрѣ единомъ: единъ поемлется, а другій оставляется. Будетъ двѣ вкупѣ мелющѣ: едина поемлется, а другая оставляется. Два будета на селѣ: единъ поемлется, а другій оставляется}\footnote{Лук.~17,~34--36.}. Благочестивый поемлется, а нечестивый оставляется. Мужъ благочестивый поемлется, а жена нечестивая оставляется; и жена благочестивая поемлется, а мужъ нечестивый оставляется. Отецъ благочестивый поемлется, а сынъ и дщерь нечестивыи оставляются; и сынъ и дщерь благочестивыи поемлются, а отецъ нечестивый оставляется. Братъ и сестра благочестивыи поемлются, а другій братъ и сестра нечестивыи оставляются. Сосѣдъ благочестивый поемлется, а другій, нечестивый, оставляется. Начальникъ благочестивый поемлется, а подначальный нечестивый оставляется; и подначальный благочестивый поемлется, а начальникъ нечестивый оставляется. Князь, вельможа и господинъ благочестивый поемлется, а рабъ и слуга его нечестивый оставляется; и рабъ и слуга благочестивый поемлется, а князь, вельможа и господинъ нечестивый оставляется. Богачъ благочестивый поемлется, а другій нечестивый оставляется. Нищій благочестивый поемлется, а другій, нечестивый, оставляется. Судія благочестивый поемлется, а другій, нечестивый, оставляется. Господинъ благочестивый поемлется, а другій, нечестивый, оставляется. Купецъ благочестивый поемлется, а другій, нечестивый, оставляется. Земледѣлецъ благочестивый поемлется, а другій, нечестивый, оставляется. Мастеръ и художникъ благочестивый поемлется, а другій, нечестивый, оставляется. Рабъ и слуга благочестивый поемлется, а другій, нечестивый, оставляется. Тако единъ поемлется, а другій оставляется. Куды? ко Хрісту, въ царствіе небесное. \textit{Идѣже тѣло, тамо соберутся и орли}\footnote{Лук.~17,~37.}. Гдѣ глава, тамо и уды ея будутъ. Гдѣ Хрістосъ, тамо и хрістіане будутъ. \textit{Идѣже есмь Азъ, ту и слуга Мой будетъ}\footnote{Іоан.~12,~26.}. Будетъ бо избраніе и собраніе по вѣрѣ и правдѣ, а не по сану, званію, внѣшности, именамъ и титуламъ. Богъ лица не пріемлетъ, но судитъ по внутренности, по вѣрѣ и совѣсти. Якоже бо злато и сребро, царскую печать и надписаніе имѣющее, избирается и пріемлется въ казну, не имѣющее же отметается: тако душа, имѣющая печать правды Хрістовой, избирается и поемлется въ царство Божіе, а не имѣющая тоя отметается. Правда Хрістова въ благочестивой душѣ есть какъ печать царская, которая показываетъ, что душа тая есть Хрістова. Таковая душа поемлется въ царство Божіе, яко Хрістова. Тогда бо повелѣніемъ Божіимъ разыдутся ангели по всей вселеннѣй, и соберутъ благочестивыхъ, вѣрою оправданныхъ, печать правды Хрістовой имущихъ, яко царскія знаменанныя монеты въ казну, въ вѣчную жизнь. \textit{Послетъ бо} Господь \textit{ангелы Своя съ трубнымъ гласомъ веліимъ, и соберутъ избранныя Его отъ четырехъ вѣтръ, отъ конецъ небесъ до конецъ ихъ}\footnote{Мѳ.~24,~31.}. Тогда исполнится слово сіе: \textit{единъ поемлется, а другій оставляется}; благочестивый поемлется, а нечестивый оставляется. \textit{Якоже было во дни Ноевы: тако будетъ и въ пришествіе Сына Божія}. Прежде потопа ѣли, пили, женилися, посягали, и нечаянно пришелъ потопъ, и погубилъ всѣхъ. Такимъ образомъ будетъ и въ пришествіе Сына Божія: будутъ люди ѣсть, пить, банкетовать, веселиться, жениться, посягать, и нечаянно пріидетъ \textit{день Господень}\footnote{ст. 37--39.}. Тогда услышится гласъ трубный: идите на судъ! Тогда услышится вопль: \textit{се Женихъ грядетъ, исходите въ срѣтеніе Его}\footnote{Мѳ.~25,~6.}. Тогда исполнится апостольское слово: \textit{егда рекутъ, миръ и утвержденіе; тогда внезапу нападетъ на нихъ всегубительство}\footnote{1~Сол.~5,~3.}. \textit{Близъ же уже есть день Господень}\footnote{Апок.~1,~3.}. Вотъ нечаянно \textit{явится знаменіе Сына человѣческаго на небеси: и тогда восплачутся вся колѣна земная, и узрятъ Сына человѣческаго грядуща на облацѣхъ небесныхъ съ силою и славою многою. Якоже бо молнія исходитъ отъ востокъ и является до западъ: тако будетъ и пришествіе Сына человѣческаго}\footnote{Мѳ.~24,~30 и 27.}. Хрістіанине! покаемся, и сокровище благочестія внутрь себе потщимся имѣть, да и мы тогда не оставлены, но да взяты \textit{будемъ и восхищены на облацѣхъ въ срѣтеніе Господне на воздусѣ, и тако всегда съ Господемъ будемъ}\footnote{1~Сол.~40,~17.}.

\section{111. Присяга.}

Видимъ, что когда человѣкъ, или въ воинство избирается, или на честь какую возводится, или къ какому иному дѣлу Государеву опредѣляется, "--- присягаетъ и съ клятвою обѣщается вѣрно и праведно Государю и обществу служитъ, и по чистой совѣсти поступать. Сія есть сила присяги. Тако вси хрістіане, входя въ церковь святую, пріемля честь и достоинство высокаго имени хрістіанскаго и записываяся въ воинство небеснаго Царя, присягаютъ въ томъ до конца пребывать; клянутся и обѣщаются, отрекшися сатаны, и всѣхъ злыхъ дѣлъ его, вѣрою и правдою служить единому Хрісту Господу небесному Царю, во все время житія своего. Видимъ сію присягу, клятву и обѣщаніе хрістіанамъ. Читай всякъ форму, изображенную предъ крещеніемъ святымъ. Тамо всякъ трижды возглашаетъ: \textit{отрицаюся сатаны, и всѣхъ дѣлъ его, и всѣхъ аггелъ его, и всего служенія его, и всея гордыни его}; и плюетъ на него. Такожде преступаетъ ко Хрісту и глаголетъ: \textit{сочетаваюся Хрісту}; и сіе глаголетъ трижды. Хрістіанине! Что сіи отрицанія и обѣты значатъ, аще не присягу и клятву, которою обѣщаемся, отрекшися сатаны и грѣховъ и похотей нашихъ и гордыни и суеты міра сего, вѣрою, смиреніемъ, любовію, кротостію и терпѣніемъ въ слѣдъ Хріста Господа ходить? Сія есть сила обѣтовъ нашихъ, учиненныхъ отъ насъ предъ святымъ крещеніемъ. Тяжко грѣшатъ люди, которыи присяги Государю и обществу не хранятъ. Таковыи суть невѣрніи общества сынове и вредительнѣйшіи обществу паче внѣшнихъ враговъ, и называются безчестнымъ именемъ, \textit{измѣнники}; измѣняютъ бо Государю и обществу своему, и подобно дѣлаютъ поврежденному уду, который самъ гніетъ и всю цѣлость тѣла повреждаетъ. Тако они мнятся быть уды общества, но суть уды гниліи, которыи цѣлость всего общества повреждаютъ. Злый человѣкъ не можетъ быть добрый начальникъ; и когда въ чести будетъ, далеко болѣе повредитъ общество, нежели явный злодѣй и внѣшній врагъ. Сіе довольно уже дознано отъ исторіи, и нынѣ тоежде видимъ. Тако весьма тяжко грѣшатъ хрістіане, которыи въ святомъ крещеніи присягали, клялися и обѣщалися Хрісту Господу служить, но обѣты своя нарушили и солгали. Сюды надлежать блудники, прелюбодѣи и вси сквернители, злобники, убійцы, тати, хищники, грабители, насильники, вдовицъ и сиротъ утѣснители, клеветники, ругатели, хитрецы, прелестники, лживіи, лукавцы, родителямъ и властямъ не покоряющіися, хульники, чародѣи и ихъ къ себѣ призывающіи, любящіи суету міра сего, въ гордости и пышности живущія, и прочія Божію слову противляющіися. Вси таковіи нарушили хрістіанскую присягу; клятвы и обѣтовъ своихъ, бывшихъ въ крещеніи, не сохранили; разорвали спасительный со Хрістомъ союзъ и отъ Него отлучились и въ слѣдъ сатаны, котораго отрицались и на котораго плевали, паки обратилися; и Хрісту Господу своему и Царю, Которому присягали, измѣнили. Глаголетъ бо Господь: \textit{иже нѣсть со Мною, на Мя есть}\footnote{Мѳ.~12,~30.}. Вси таковіи не со Хрістомъ. \textit{Кое бо причастіе правдѣ къ беззаконію? или кое общеніе свѣту ко тьмѣ? Кое же согласіе Хрістови съ веліаромъ}\footnote{2~Кор.~16,~14 и 15.}? Убо суть противу Хріста. Къ сему бѣдствію и окаянству приводитъ нарушеніе обѣщаній, бывшихъ на крещеніи. Нарушившіи присягу Государю и обществу своему отчисляются отъ числа добрыхъ людей, и, по законамъ гражданскимъ, казнь пріемлютъ, яко лживіи и посмѣятели присяги: тако хрістіане, нарушившіи присягу, Господу Іисусу Хрісту учиненную, и Тому солгавшіи и къ нечестію обратившіися, отъ добрыхъ хрістіанъ, яко козлища отъ овецъ, отлучатся и казнены будутъ вѣчнымъ огнемъ, яко измѣнники, болѣе, нежели Турки и идолопоклонники. \textit{Страшливымъ и невѣрнымъ и сквернымъ, и убійцамъ, и блудъ творящимъ, и чары творящимъ, идоложерцемъ, и всѣмъ лживымъ, часть имъ въ озерѣ горящемъ огнемъ и жупеломъ, еже есть смерть вторая}\footnote{Апок.~21,~8~.}. Хрістіанине! помяни, что и ты присягу сію учинилъ, и осмотрись, осмотрись, пожалуй, въ семъ такъ важномъ дѣлѣ, храниши ли ты ее. Когда не храниши, то и крещеніе твое ничего тебѣ не пользуетъ. Хрістосъ Господь, яко человѣколюбецъ, всѣхъ отвратившихся къ Себѣ призываетъ и ожидаетъ, и обѣщаетъ милость Свою явить имъ. Обратися убо, возлюбленне, къ Нему съ покаяніемъ и плачемъ, и пріиметъ тебе. И уже впредь не отлучайся отъ Него, но всегда буди при Немъ. Онъ бо \textit{единъ} нашъ свѣтъ и животъ, надежда и утѣшеніе, радость, веселіе и блаженство въ семъ и въ будущемъ вѣкѣ. Кромѣ Его и безъ Него здѣ и тамо бѣдствіе, окаянство и горесть едина. Обратися убо къ Нему, пока время не ушло. \textit{Се нынѣ время благопріятно! се нынѣ день спасенія}\footnote{2~Кор.~6,~2.}! Внимай сему и ты, душе моя, и помни, что со Хрістомъ быть есть вѣчный животъ; безъ Хріста быть есть вѣчная смерть. \textit{Се удаляющіи себе отъ Тебе}, Господи, \textit{погибнутъ}\footnote{Пс.~72,~27.}.

\section{112. Баня.}

Видимъ, что люди, вшедши въ баню, омываются отъ сквернъ и пороковъ тѣлесныхъ, и исходятъ изъ бани чисти и одѣяны въ бѣлую рубашку. Тако хрістіане, вшедши въ баню святаго крещенія, омываются отъ сквернъ грѣховныхъ, очищаются и освящаются, и одѣваются пресвѣтлою и предрагою правды Хрістовой одеждою, яко порфирою царскою; и дѣлаются сынами небеснаго Царя и наслѣдниками небеснаго царствія; и исходятъ оттуду чисти, святи, праведни, якоже Апостолъ имъ утѣшительно глаголетъ: \textit{омыстеся, освятистеся, оправдистеся именемъ Господа нашего Іисуса Хріста, и Духомъ Бога нашего}\footnote{1~Кор.~6,~11.}. Откуду святое крещеніе называется отъ Апостола \textit{банею пакибытія}\footnote{Тит.~3,~5.}. Потому что тѣмъ вновь раждаемся, и погибшіи спасаемся и обновляемся, очищаемся и омываемся, и дѣлаемся \textit{новая тварь о Хрістѣ}. Отсюду видимъ, хрістіанине, коль великое есть человѣколюбіе Божіе къ намъ: человѣка, грѣхами оскверненнаго и врага своего, толикія благодати, милости и чести сподобляетъ! Благословенъ Богъ, благоволивый тако! Слава благости Его! Слава человѣколюбію Его! Сію благодать и милость \textit{заслужилъ} намъ единородный Сынъ Божій, \textit{подаетъ} намъ \textit{туне} небесный Отецъ, \textit{совершаетъ} Духъ Святый. Слава Тріѵпостасному Богу! Хрістіане, которыи къ міру и грѣху любовію обращаются и беззаконнуютъ (о каковыхъ сказано выше, подъ числомъ 3), все тое святое и великое Божественное сокровище теряютъ. Почему \textit{случися имъ истинная притча: песъ возвращся на свою блевотину, и свинія, омывшися, въ калъ тинный}\footnote{2~Петр.~2,~22.}. Сіи подобны тѣмъ людемъ, которыи, вшедши въ баню и омывшися, самовольно входятъ въ грязь и мараются: тако они, омывшися святымъ крещеніемъ и забывши великую тую милость Божію, имъ показанную, самовольно въ скверны грѣховныя предаются, и болѣе и болѣе тѣми сквернятся. Самъ разсуди, человѣче. Что таковымъ святое крещеніе пользуетъ? Воистину ничто. Откуду бываетъ, что таковыи хрістіане горшіи нравами бываютъ язычниковъ и Турковъ. Ибо многіи язычники таковыми грѣхами гнушаются отъ естественнаго закона, на каковыи развращенніи хрістіане безстрашно дерзаютъ. Не видна въ язычникахъ политичныхъ таковая неправда, хищеніе, грабленіе, насиліе, хитрость, лесть, лукавство, ненасытное сребролюбіе, мерзкая нечистота, какъ въ развращенныхъ хрістіанахъ видится. А многіи уже до того пришли, что грѣховъ явныхъ, закономъ Божіимъ запрещенныхъ, за грѣхи не ставятъ. Сія есть великая слѣпота, помраченіе и заблужденіе ума, и происходитъ отъ неблагодарности человѣческой. Тако человѣкъ, пришедши во глубину золъ, нерадитъ. Должно было хрістіанину, яко обновившемуся Божіею благодатію, въ лучшее успѣвать и расти о Хрістѣ и \textit{приходить въ мужа совершенна}\footnote{Еф.~4,~13.}. Но онъ въ горшее успѣваетъ, и дѣлается злѣйшимъ язычника и идолопоклонника. Таковому самое крещеніе будетъ во обличеніе въ день суда Хрістова, когда не покается и не исправится, и не омыетъ сквернъ своихъ сокрушеніемъ сердца и слезами. Отсюду видишь, хрістіанине: 1)~Коль полезно и нужно есть хрістіанамъ напоминать о святомъ крещеніи, о отрицаніи и обѣтахъ ихъ бывшихъ, то"=есть, кого они и чего отрицались тогда, какіе обѣты чинили Господу Іисусу Хрісту, и коль великія милости отъ Бога сподобилися, дабы все тое помнили, и въ страсѣ Божіи жили, и дара небеснаго, въ крещеніи имъ даннаго, не потеряли, а потерявши, искали бы того покаяніемъ и сокрушеніемъ сердца и исправленіемъ житія своего и нравовъ. Сіе напоминаніе наипаче до пастырей надлежитъ. 2)~Коль полезно и нужно есть доброе и въ страхѣ Божіи дѣтей воспитаніе, и о святомъ крещеніи и обстоятельствахъ его имъ напоминаніе, дабы, памятуя тое все и разсуждая, не развратились, и дара, въ крещеніи даннаго, не лишились. Сія должность есть родителей ихъ. Родители родили ихъ къ временному житію, должны убо благодатію Божіею и къ вѣчному житію отрождать. А которыи родители дѣтей своихъ добрѣ не воспитываютъ, а паче примѣромъ злымъ соблазняютъ; тіи къ временному житію раждаютъ ихъ, но къ вѣчной смерти дверь отворяютъ: лучше таковымъ не родитися. Возлюбленный хрістіанине! якоже вси хрістіане, тако и ты высочайшія оныя милости и благодати Божія и небеснаго онаго дара въ святомъ крещеніи сподобился туне; и ты отъ служителя Божія слышалъ тогда пресладкое привѣтствіе: \textit{омылся еси, освятился еси, оправдался еси}, и проч. Помяни сіе и осмотрись, не потерялъ ли и ты великаго онаго и небеснаго сокровища. Когда по правилу слова Божія не живеши, но противно тому живеши, и послѣдуеши вышеписаннымъ грѣшникамъ: извѣстный знакъ есть, что и ты духовное тое сокровище потерялъ. Почему и тебѣ притча оная приличествуетъ: \textit{песъ возвращся на свою блевотину, и свинія, омывшися, въ калъ тинный}. Жалѣешь и плачешь, когда потеряешь тѣлесное и тлѣнное сокровище; коль многаго плача и слезъ надобно "--- оплакать потерянное небесное оное сокровище, и слезами омывать осквернившуюся душу, и просить, искать и толкать въ двери милосердія Божія, да паки оное къ намъ возвратится! Когда тое погубляется, то погубляется и вѣчное спасеніе. Возвратится неотмѣнно, когда мы возвратимся къ Богу съ покаяніемъ и плачемъ и слезами. \textit{Щедръ бо и милостивъ Господь есть}\footnote{Пс.~44,~8.}. \textit{Пріемлетъ кающихся} и обращающихся къ Нему, яко благоутробный отецъ блуднаго сына принялъ\footnote{Лук.~15,~20--24.}. Осмотрись и ты, душе моя: что мыслиши, что мудрствуеши? что любиши, чего ненавидиши? къ чему прилѣпляешися? чего отвращаешися? гдѣ живеши, гдѣ обращаещися? къ чему стремишися? какія желанія, какія движенія внутрь себе чувствуеши? Небесное оное сокровище внутрь сердца сокровенно есть; но бытіе свое чрезъ небесныя движенія и желанія показуетъ. Подобное къ подобному стремится. \textit{Идѣже сокровище ваше, ту и сердце ваше будетъ}\footnote{Мѳ.~6,~21.}. \textit{Помяни щедроты Твоя, Господи, и милости Твоя, яко отъ вѣка суть. Грѣхъ юности моея и невѣдѣнія моего не помяни, по милости Твоей: помяни мя Ты, ради благости Твоея, Господи}\footnote{Пс.~24,~6 и 7.}. Тако, обратившися отъ грѣховъ къ Создателю нашему, да воздыхаемъ въ Нему всегда, и помилуетъ насъ.

\section{113. Обрученная дѣва мужу.}

Видимъ въ мірѣ, что дѣвы обручаются мужамъ въ невѣсты. Тако хрістіанскія души вѣрою на крещеніи обручаются небесному Жениху Хрісту, яко дѣвы чистыя въ невѣсту, якоже апостолъ глаголетъ хрістіанамъ: \textit{ревную по васъ Божіею ревностію; обручихъ бо васъ единому Мужу дѣву чисту представити Хрістови}\footnote{2~Кор.~11,~2.}. Тайна сія велика есть. Велико и умомъ непостижимое таинство, и честь и слава и достоинство толикое, каковаго болѣе быть не можетъ, душѣ человѣческой обрученною быть въ невѣсту небесному Царю и \textit{краснѣйшему добротою паче всѣхъ сыновъ человѣческихъ}, Сыну Божію\footnote{Пс.~44,~3.}. Кто сіе человѣколюбіе Божіе можетъ уразумѣть и изслѣдовать? Кто и души хрістіанскія, небесному оному Жениху обручившіися, благородіе, честь, славу и достоинство изрещи можетъ? Внимай сему, душе моя. Помяни, хрістіанине, что и твоя душа преславному оному Жениху обручена въ невѣсту. Невѣста единаго жениха своего любитъ, и тому единому угождаетъ, какъ видимъ: тако душѣ хрістіанской единаго Хріста, Жениха своего, должно любить и Ему единому угождать. Не нравится жениху невѣста и отъ ней отвращается, когда она ко инымъ любителямъ любовію обращается: тако отвращается Хрістосъ отъ души хрістіанской, которая ко грѣху и міру любовію своею обращается и тому угождаетъ. О, коль тяжко грѣшитъ предъ Хрістомъ таковая душа! коль неблагодарно и безстыдно дѣлаетъ! коль великое бѣдствіе ея, яко, пресладкаго и прелюбезнаго небеснаго Жениха отвратившися, къ нечистой и мерзкой грѣха и міра любви обращается! Хрістіанине! осмотрись, пожалуй осмотрись: не обратился ли и ты къ міру и грѣху отъ Хріста Спасителя твоего? не любиши ли чего равно, или, что горше, болѣе Его? Его должно любить тебѣ паче отца и матери, паче жены и дѣтей, паче братіи и друговъ, и паче себе самого\footnote{Мѳ.~10,~37--39.}. Сего Онъ отъ тебе требуетъ, яко и Самъ Онъ такъ тебе возлюбилъ, что и предалъ Себе за тебе. Видишь, коль красное небо, солнце, луна и звѣзды и прочее созданіе Его; но ни о чемъ тако не благоволилъ Онъ, какъ о душѣ твоей. Видишь, какъ Онъ почтилъ тебе, какой чести и славы сподобилъ душу твою. Хвали, душе моя, Господа! \textit{Слыши дщи и виждь, и приклони ухо твое, и забуди люди твоя и домъ отца твоего. И возжелаетъ Царь доброты твоея; зане Той есть Господь твой: и поклонишися Ему}\footnote{Пс.~44,~11 и 12.}. Слыши и ты, душе моя, слыши гласъ небеснаго Жениха твоего, и желай прекрасныя и святѣйшія доброты Его. \textit{Что бо ми есть на небеси? и отъ Тебе что восхотѣхъ на земли? Изчезе сердце мое и плоть моя: Боже сердца моего, и часть моя Боже во вѣкъ! Яко се удаляющіи себе отъ Тебе, погибнутъ: потребилъ еси всякаго любодѣющаго отъ Тебе. Мнѣ же прилѣплятися Богови благо есть, полагати на Господа упованіе мое}\footnote{Пс.~72,~25--28.}. Да мерзѣетъ тебѣ, душе моя, любовь мерзкая грѣха и міра и свое безчинное самолюбіе. Люби \textit{единаго}, любящаго тебе, Іисуса; желай Его \textit{единаго}, желающаго тебе; прилѣпляйся Ему \textit{единому}, ищущему тебе, и буди со Тщащимся о тебѣ тщательна, съ Пекущимся о тебѣ попечительна, съ Чистымъ чиста, со Святымъ свята, съ Любителемъ любительна, съ Кроткимъ кротка, со Смиреннымъ смиренна, съ Долготерпѣливымъ терпѣлива, съ Милостивымъ и Милосердымъ милостива и милосерда. Берегись отъ Него отлучиться здѣ, да и во ономъ вѣкѣ съ Нимъ неотлучно пребудеши. \textit{Мнѣ же прилѣплятися Богови благо есть}.

\section{114. Корабль.}

Видимъ въ мірѣ, что имѣются корабли, ради различныхъ нуждъ устроенніи, и по морямъ плаваютъ, и съ мѣста на мѣсто переходятъ. Что корабль на морѣ, тое Церковь святая въ мірѣ. Церковь кораблю подобна. Корабль на морѣ плаваетъ: Церковь святая на морѣ міра сего имѣется. Корабль управляетъ кормчій: Церковь святую управляетъ кормчій Іисусъ Хрістосъ, Господь нашъ. Корабль, пока на морѣ плаваетъ, всякой бурѣ, непогодѣ и вѣтрамъ подлежитъ и волнами колеблется: Церковь святая, пока въ мірѣ имѣется, всякой бурѣ бѣдъ, напастей и искушеній подлежитъ, и соблазнами міра сего, яко волнами, колеблется, и отъ любителей міра гоненіе страждетъ. Но сказано о ней во утѣшеніе ея отъ Кормчія "--- Іисуса Хріста: \textit{и врата адова не одолѣютъ ей}\footnote{Мѳ.~16,~18.}. Корабль, плавая на морѣ, къ пристанищу своему, къ которому идетъ, стремится: Церковь святая, плавая на морѣ міра сего, стремится къ тихому вѣчнаго живота пристанищу; тамо ей покой будетъ. Имѣющіися въ кораблѣ то и думаютъ и мыслятъ и желаютъ всегда, дабы мѣста, къ которому идутъ, безбѣдно достигнуть: тако хрістіане истинніи, находящіися въ Церкви святой, тое всегда и на умѣ имѣютъ, о томъ всегда тщатся и пекутся, дабы въ пристанище вѣчнаго покоя пріитить. Сего ради о временныхъ и мірскихъ, какъ"=то: о чести, славѣ, богатствѣ и прочемъ міра сего сокровищѣ, небрегутъ, а довольствуются тѣмъ, что имѣютъ отъ благости Божіей. Они со апостоломъ говорятъ: \textit{ничтоже внесохомъ въ міръ сей: явѣ, яко ниже изнести что можемъ; имѣюще же пищу и одѣяніе, сими довольни будемъ}\footnote{1~Тим.~6,~7 и 8.}. Корабельники, когда великая буря и непогода востаетъ, имѣютъ обычай котвы или якори во глубину морскую ввергать и тако корабль свой утверждать и сохранять: тако истинніи хрістіане, когда на нихъ великая буря искушеній и напастей востаетъ, котвы упованія своего во глубину милосердія и обѣщанія Божія и святѣйшія клятвы Его повергаютъ\footnote{Евр.~6,~17--19.}. И ко Іисусу, яко Кормчію своему, аки спящему, съ моленіемъ и слезами приступаютъ и возбуждаютъ Его, глаголюще: \textit{Господи, спаси ны: погибаемъ}\footnote{Мѳ.~8,~25.}. Изрядно предобразовалъ Церковь святую ковчегъ Ноевъ. Въ ковчегѣ Ноевомъ собраны были различніи звѣри, скоти и птицы: тако въ Церковь святую различныи народы, яко звѣри дикіе, собралися. \textit{И се иноплеменницы, и Тѵръ, и людіе Еѳіопстіи, сіи быша тамо. Мати Сіонъ речетъ: человѣкъ, и человѣкъ родися въ немъ}, поетъ о Церкви пророкъ святый\footnote{Пс.~86,~4 и 5.}. Чудно, что въ ковчегѣ Ноевомъ звѣри лютіи кротки и согласны были: Божіимъ то повелѣніемъ учинилося; и тако лютость и свирѣпость свою отложили они, иначе бы не могли помѣститься. Тако вшедшіи въ Церковь святую прежніи свои нравы, яко звѣрскіи и скотскіи, отлагаютъ и бываютъ кротки, согласны и мирны. Крещеніемъ бо вси обновляются и пресвятый нравъ Хрістовъ на себе пріемлютъ, и тако Церкви святой причисляются. Откуду написано въ Дѣяніяхъ Апостольскихъ о первыхъ хрістіанахъ: \textit{народу вѣровавшему бѣ сердце и душа едина}\footnote{Дѣян.~4,~32.}. Вси человѣки, звѣри, скоти и птицы, внѣ ковчега имѣвшіися тогда, водою погибли: тако вси люди, внѣ Церкви святой имѣющіися, вѣчно погибаютъ. Хрістіанине, разсуждай, имѣешися ли внутрь Церкви святой. Церковь есть \textit{свята}\footnote{Еф.~2,~21; 5,~26 и 27.}. Надобно быть и сынамъ ея \textit{святыми}. Соравняй убо нравъ свой съ правиломъ слова Божія, и примѣчай, сходенъ ли онъ тому. Тамо изображенъ \textit{новый человѣкъ} или истинный хрістіанинъ и сынъ Церкве. Что бо пользуетъ именоваться христіаниномъ, но не быть хрістіаниномъ; числиться въ Церкви предъ людьми, но предъ очами Божіими быть, аки извергу, внѣ Церкви? Вси беззаконники и въ гордости и пышности міра сего живущіи внѣ Церкви имѣются, хотя и хвалятся исповѣданіемъ имени Хрістова. \textit{Кое бо причастіе правдѣ къ беззаконію? или кое общеніе свѣту ко тмѣ? Кое же согласіе Хрістови съ веліаромъ}\footnote{2~Кор.~6,~14 и 15.}? Разсуждай сія, хрістіанине, и кайся и исправляйся и молись, да будеши истинный удъ Церкви святыя, и спасешися отъ грядущаго гнѣва Божія, яко Ной отъ всемірнаго потопа. Самъ разсуди, како возможно быть внутрь Церкви \textit{святой} тому, который льститъ, лукавитъ и хитритъ яко лисъ; тому, который похищаетъ яко волкъ; тому, который ярится и свирѣпѣетъ яко левъ лютый; тому, который похотствуетъ яко конь; тому, который злобится яко ехидна; тому, который гордится яко павлинъ; тому, который обжирается и сластолюбствуетъ яко свинья; тому, который такъ мірскихъ вещей желаетъ и ищетъ какъ язычники, не имѣющіи упованія вѣчнаго живота, и какъ бы онъ имѣлъ вѣчно въ мірѣ семъ жить, и аки бы не было воскресенія мертвыхъ? А что когда въ единомъ всѣ скотскіе нравы показуются? то уже и горшій скота есть. Не внѣшній бо видъ и образъ, но внутренній нравъ человѣческій показуетъ человѣка. Самъ убо видишь, хрістіанине, что невозможно неисправному и не очистившемуся покаяніемъ хрістіанину внутрь Церкве Божіей быть. И потому опасно, чтобы вѣчно не погибнулъ, якоже вси, обрѣтшіися внѣ ковчега Ноева, погибли, аще не покается и не исправится. \textit{Плоть бо и кровь царствія Божія наслѣдити не могутъ}\footnote{1~Кор.~15,~50.}. \textit{Внѣ псы и чародѣи, любодѣи и убійцы, идолослужители и всякъ любяй и творяй лжу}\footnote{Апок.~22,~15.}. Покайся убо и исправься, и перемѣни себе и скотскій нравъ въ хрістіанскій измѣни, и тако, обновившися, будеши истинный сынъ Церкве и въ доброй надеждѣ вѣчнаго живота.

\section{115. Чей ты?}

Слышимъ, что единъ другаго вопрошаетъ: \textit{чей ты?} И отвѣщаетъ другій: я"=де такого"=то господина рабъ, или такого"=то отца сынъ, и проч. Тако когда хрістіанина вопросятъ, что ему должно отвѣтствовать? Подлинно всякому хрістіанину должно отвѣщать: \textit{Хрістовъ рабъ есмь}; ибо \textit{Хріста исповѣдуетъ Господа своего}\footnote{1~Кор.~7,~22; Римл.~10,~9; Кол.~3,~24.}. Но посмотри, хрістіанине, въ совѣсть твою, работаеши ли ты Хрісту, угождаеши ли Хрісту, яко рабъ господину. Мученики святіи, когда ихъ мучители вопрошали "--- \textit{чіи вы}, отвѣтствовали дерзновенно: \textit{мы Христіане, мы Хрістовы раби есмы}. Но они такъ вѣрны Хрісту Господу своему были, что крови и живота своего ради имене Его святаго не щадѣли; такъ Его любили, что не только честь, слава и богатства міра сего, плачъ и слезы отцевъ и матерей, женъ и дѣтей, но и ужасное мученіе и \textit{самая смерть отъ Хрістовой любви не могла разлучить ихъ}\footnote{Римл.~8,~38 и 39.}. То"=то истинніи раби Хрістовы, то"=то истинніи хрістіане, сынове свѣта, сынове церкве святыя, кроткія овцы Хрістовы, храбріи воины Хріста Царя, прелюбезныи уды Хрістовы, чада Бога вышняго, и наслѣдники Царствія Его! А тебѣ, хрістіанине, который Хріста исповѣдуешь, но противно Хрісту живешь, тебѣ, говорю, какъ сказать "--- \textit{Хрістовъ рабъ есмь}? Подлинно велико есть быть \textit{Хрістовымъ рабомъ}. Сія честь и достоинство несравненно большее и высшее есть, нежели быть сыномъ славнаго князя или царя земнаго. \textit{О Господѣ похвалится душа моя}\footnote{Пс.~33,~3.}. Многіи хвалятся: я"=де того"=то высокаго господина рабъ, или того"=то князя или вельможи слуга, или такого"=то царя придворный. Но хрістіане истинно и достойно хвалятся о Господѣ своемъ: мы Царя небеснаго \textit{Іисуса Хріста раби}. Подлинно истинная сія похвала есть, истинно высокая слава, честь и достоинство быть и называться \textit{Хрістовымъ рабомъ}. Но какъ тому называть себе Хрістовымъ рабомъ, который грѣху, міру и себѣ работаетъ, а не Хрісту? Что бо пользуетъ называться Хрістовымъ рабомъ, а не быть Хрістовымъ рабомъ; внѣ и исповѣданіемъ быть хрістіаниномъ, но внутрь, нравомъ быть язычникомъ, или, что горше того, скотомъ? Апостолъ святый объявляетъ намъ примѣты Хрістовыхъ рабовъ, да искушаемъ и познаемъ себе, Хрістовы ли мы есмы. \textit{Иже}, рече, \textit{Хрістовы суть, плоть распяша со страстьми и похотьми}\footnote{Гал.~5,~24.}. Вотъ знаки Хрістовыхъ людей или рабовъ! Распни убо, хрістіанине, плоть свою со страстьми и похотьми, и будеши истинно \textit{Хрістовъ рабъ}, исповѣданіемъ и вещію хрістіанинъ. Иначе ложный тотъ хрістіанинъ, который называется хрістіаниномъ, а нрава хрістіанскаго не имѣетъ, и не живетъ по"=хрістіански. Буди убо внутрь хрістіаниномъ, а не точію внѣ, и тогда истиннымъ \textit{Хрістовымъ рабомъ} будеши. \textit{Что Мя зовете, Господи, Господи, и не творите яже глаголю}\footnote{Лук.~6,~46.}, глаголетъ Господь. Твори убо, что хощетъ Господь, и тогда называй Его Господемъ своимъ, и Онъ признаетъ тебе за \textit{раба Своего}.

\section{116. Онъ съ нимъ за одно.}

Слышимъ часто отъ людей слово сіе: \textit{онъ съ нимъ заодно}. Слово сіе говорится о тѣхъ людехъ, которіи единомысліе, согласіе и миръ между собою въ чемъ нибудь имѣютъ. Тако вѣрный рабъ съ господиномъ своимъ за одно, послушливый сынъ со отцемъ своимъ за одно, другъ вѣрный съ другомъ за одно, блудникъ съ блудницею за одно, разбойникъ съ разбойникомъ за одно, тать съ татемъ за одно, когда хотятъ что украсть и похитить, и проч. Тако имѣется и въ духовныхъ. Всякъ человѣкъ или со Хрістомъ, или съ противникомъ Его діаволомъ за одно. Каковый и чій кто духъ имѣетъ, съ тѣмъ за одно; съ кѣмъ единомысліе, согласіе и миръ имѣетъ, съ тѣмъ за одно. Кто во Хріста Сына Божія истинно и сердечно вѣруетъ, и Его единаго Искупителя и Спасителя своего исповѣдуетъ и признаетъ, и кромѣ Его инаго посредствія къ полученію вѣчнаго блаженства не знаетъ, и Ему сердечно прилѣпляется, въ нуждахъ и случаяхъ съ моленіемъ къ Нему прибѣгаетъ, и Его защитника и помощника во всемъ признаетъ и имѣетъ; Его единаго любитъ, и всякаго человѣка, по слову Его; противу всякаго грѣха борется и подвизается и не попущаетъ ему собою обладать; горняя мудрствуетъ, а не земная; о всемъ Богу благодаритъ, и волю Его святую творитъ; ближнему своему оставляетъ обиды, и не отмщеваетъ ему; бѣдствующимъ и страждущимъ состраждетъ сердцемъ; всѣмъ усердно желаетъ спастися; не токмо надъ прочими человѣками, но и надъ врагами своими умилостивляется и добро творитъ имъ; и крестъ, отъ небеснаго Отца ему посланный, безропотно несетъ, и Хрісту, Спасителю своему, со смиреніемъ и кротостію послѣдуетъ: таковый воистину \textit{со Хрістомъ за одно}, единомысліе, согласіе и миръ съ Нимъ имѣетъ. \textit{Прилѣпляяйся бо Господеви, единъ духъ есть съ Господемъ}\footnote{1~Кор.~6,~17.}. \textit{Аще кто любитъ Мя}, глаголетъ Господь, \textit{слово Мое соблюдаетъ: и Отецъ Мой возлюбитъ его, и къ нему пріидемъ, и обитель у него сотворимъ. Вы друзи Мои есте, аще творите, елика Азъ заповѣдаю вамъ}\footnote{Іоан.~14,~23; 15,~14.}. Вси таковіи, который небеснаго сего дара сподобились, \textit{со Хрістомъ за одно}. О прелюбезное единомысліе! о пресладкое согласіе! о превожделѣнный миръ! о спасительный союзъ! о небесное дружество! о всевожделѣнное блаженство "--- любимымъ быть отъ Хріста, и любить Хріста; прилѣпляться Хрісту, и единъ духъ быть со Хрістомъ; возлюбленнымъ быть отъ Бога, и храмомъ и обителію быть Бога; другомъ быть и нарицаться небеснаго Царя! \textit{Богъ любы есть, и пребываяй въ любви въ Бозѣ пребываетъ, и Богъ въ немъ пребываетъ}\footnote{Іоан.~4,~16.}. Хрістіанине! велико есть со Хрістомъ имѣть единомысліе, согласіе и миръ, и \textit{быть съ Нимъ за одно}. Что сего болѣе можетъ быть? что полезнѣе и спасительнѣе? Сіе всякую честь и славу, и богатство и красоту, и художество и мудрость и знаніе вѣка сего несравненно превосходитъ. Сіе бо есть истинная и небесная премудрость, истинное и небесное богатство, честь и слава. И сіе то есть \textit{быть со Хрістомъ}, то"=есть, Ему сердцемъ прилѣпляться, съ Нимъ единомысліе, согласіе и миръ имѣть, и Ему угождать. Сего отъ насъ требуетъ Апостолъ святый: \textit{сіе да мудрствуется въ васъ, еже и во Хрістѣ Іисусѣ}\footnote{Филип.~2,~5.}. Хрістіанине! будемъ тако здѣ со Хрістомъ, да и въ будущемъ вѣкѣ съ Нимъ будемъ. Ибо и на крещеніи святомъ Ему сочетались и соединились, и чрезъ все житіе наше съ Нимъ и при Немъ быть обѣщалися. \textit{Мнѣ же прилѣплятися Богови благо есть}\footnote{Пс.~72,~28.}. Но посмотримъ, которіи хрістіане спасительный и пресладкій сей союзъ разорвали, и въ прежнее свое бѣдствіе впали. Глаголетъ Господь: \textit{иже нѣсть со Мною, на Мя есть}\footnote{Мѳ.~12,~30.}. Страшно есть сіе слово, хрістіанине, но истинно есть. Діаволъ начальникъ и изобрѣтатель грѣха. Прародители наши въ раи, когда послушали его злаго совѣта, предъ Богомъ согрѣшили тяжко, и отъ Него отступили, и сдѣлалися \textit{съ діаволомъ за одно}, и тако съ нимъ подвергли себя вѣчной смерти и мученію. Тому же бѣдствію подпали и потомки ихъ, вси человѣки. Хрістіане благодатію Божіею чрезъ святое крещеніе отъ діавола отстаютъ и свобождаются, и отъ грѣховъ очищаются, и Богу примиряются, и дѣлаются новою тварію о Хрістѣ, и людьми Божіими, и \textit{Хрісту}, яко лозѣ истинной вѣтви, \textit{прицѣпляются} и соединяются\footnote{Іоан.~15,~5.}. Сему великому хрістіанъ блаженству діаволъ завидуя, якоже прародителей въ раи, чрезъ злые помыслы прельщаетъ хрістіанъ, и ко всякому грѣху и любви суетнаго міра приводитъ. Хрістіане, которіи всепагубнаго его совѣта слушаютъ и ему согласуютъ, и въ слѣдъ его уклоняются, съ нимъ за одно, хотя того и не разумѣютъ; ибо умъ, и сердечныя очи помраченніи и душевныя уши оглушенніи отъ него имѣютъ, да не внимаютъ слову Божію, и не видятъ своего бѣдствія и пагубы. Убійца, злобный и отмщеніемъ дышущій человѣкъ съ діаволомъ за одно; ибо діаволъ есть духъ злобы и человѣкоубійца. Въ гордости и пышности живущій съ діаволомъ за одно; ибо діаволъ есть духъ гордый. На себе и на силу свою надѣющіеся съ діаволомъ за одно; ибо діаволъ на себе, свою силу и хитрость надѣется. Блудникъ, прелюбодѣй и нечистоты любитель съ діаволомъ за одно; ибо діаволъ есть духъ нечистый. Всякъ, кто какъ нибудь навѣтуетъ человѣку и пакости и обиды ему творитъ, съ діаволомъ за одно, ибо діаволъ есть супостатъ и навѣтникъ человѣческій. Клеветникъ съ діаволомъ за одно; ибо діаволъ есть клеветникъ, и отъ того имя свое имѣетъ; діаволъ бо есть греческое имя и на нашемъ діалектѣ значитъ \textit{клеветника}. Хульникъ, ругатель и злорѣчивый съ діаволомъ за одно; ибо діаволъ есть хулитель и ругатель. Завистливый и ненавистливый съ діаволомъ за одно: ибо діаволъ есть духъ зависти и ненависти. Воръ, тать и хищникъ съ діаволомъ за одно; ибо діаволъ всегда похищаетъ себѣ славу Божію и спасеніе человѣческое. Властолюбецъ и славолюбецъ съ діаволомъ за одно; ибо діаволъ всегда славы и поклоненія ищетъ отъ человѣковъ. Чародѣй и его къ себѣ призывающій съ діаволомъ за одно; ибо себе ему отдаютъ и помощи отъ него просятъ. Словомъ, всякъ, кто противно слову Божію живетъ и діавольскую волю творитъ и отъ произволенія грѣшитъ, съ діаволомъ за одно. Чью бо кто волю творить и кому согласуетъ, съ тѣмъ за одно. Сему согласуетъ и апостольское ученіе: \textit{всякъ творяй грѣхъ, и беззаконіе творитъ: и грѣхъ есть беззаконіе. И вѣсте, яко Онъ явися, да грѣхи наша возметъ: и грѣха въ Немъ нѣсть. Всякъ, иже въ Немъ пребываетъ, не согрѣшаетъ: всякъ согрѣшаяй не видѣ Его, ни позна Его. Чадца! никтоже да льститъ васъ. Творяй правду праведникъ есть, якоже Онъ праведенъ есть. Творяй грѣхѣ отъ діавола есть: яко исперва діаволъ согрѣшаетъ. Сего ради явися Сынъ Божій, да разрушитъ дѣла діаволя. Всякъ, рожденный отъ Бога, грѣха не творитъ, яко сѣмя Его въ немъ пребываетъ; и не можетъ согрѣшати, яко отъ Бога рожденъ есть. Сего ради явлена суть чада Божія, и чада діаволя}\footnote{1~Іоан.~3,~4--10.}. Отсюду видимъ, хрістіанине: 1)~Въ коль бѣдное состояніе пришелъ человѣкъ, человѣкъ по образу Божію и по подобію сотворенный: съ діаволомъ, противникомъ Божіимъ, за одно учинился. Послушалъ злаго его совѣта и согласился съ нимъ, и отъ Бога отсталъ, и сдѣлался съ противникомъ Его за одно. Сего оплакать довольно не можемъ. \textit{Тебѣ, Господи, правда, намъ же стыдѣніе лица}\footnote{Дан.~9,~7.}. Господи, пощади насъ! 2)~Всякъ человѣкъ или со Хрістомъ, или съ діаволомъ есть; неотмѣнно или къ той, или къ противной части надлежитъ. \textit{Иже нѣсть со Мною, на Мя есть}, глаголетъ Господь\footnote{Мѳ.~12,~30.}. Разсуждай сіе, хрістіанине, и смотри, которыя части еси. 3)~Хрістіане, которіи беззаконнуютъ, тяжко предъ Богомъ грѣшатъ и болѣе, нежели язычники. Ибо, отрекшися діавола въ крещеніи, пристали ко Хрісту, и паки, отставши отъ Хріста, уклонилися въ слѣдъ діавола. \textit{Быша имъ послѣдняя горша первыхъ. Лучше бо бѣ имъ не познати пути правды, нежели познавшимъ возвратитися вспять отъ преданныя имъ святыя заповѣди}\footnote{2~Петр.~2,~20 и 21.}. 4)~Демонъ на демона не востаетъ, но другъ за друга стоитъ; но бѣдный человѣкъ на подобнаго себѣ и сроднаго человѣка востаетъ. Должно было всякимъ образомъ человѣку помогать, и всѣмъ человѣкамъ противу демоновъ стоять и бороться, и другъ другу помогать и защищать; но противное дѣлается діавольскою хитростію. Человѣкъ на человѣка востаетъ и обиждаетъ и гонитъ его, что есть великое заблужденіе и ужасное помраченіе ума. 5)~Сіи люди, которіи на людей востаютъ и обиждаютъ и гонятъ ихъ, діавольскій духъ въ себѣ имѣютъ, и отъ діавола обладаеми суть. Чего ради сожалѣть о нихъ должно, да не вѣчными плѣнниками его будутъ. 6)~Истиннымъ хрістіанамъ слѣдуетъ отъ діавола искушеніе и бореніе; ибо они противятся ему, и злымъ его совѣтамъ не соизволяютъ; сего ради и онъ востаетъ на нихъ и боретъ ихъ. 7)~Діаволъ, чего самъ не можетъ истинному хрістіанину сдѣлать, тое дѣлаетъ чрезъ злыхъ людей, служителей своихъ. Отсюду видимъ различные навѣты злыхъ людей на благочестивую душу. 8)~Отсюду должно благочестивымъ опасно и осторожно жить, да не уловлены будутъ навѣтами и сѣтьми діавольскими и злыхъ людей, служителей его. Къ сему увѣщаваетъ Апостолъ: \textit{трезвитеся, бодрствуйте, зане супостатъ вашъ діаволъ, яко левъ рыкая, ходитъ, искій кого поглотити}\footnote{1~Петр.~5,~8.}. 9)~Отсюду благочестивымъ послѣдуетъ гоненіе. Діаволъ, когда не можетъ благочестивую душу прельстить и въ слѣдъ себе совратить, воздвигаетъ на тую гоненіе чрезъ злыхъ людей, дабы тако ее съ добраго пути совратить и отъ Хріста отлучить, и къ своей части привлещи. Сія его хитрость и тщаніе есть. Возлюбленный хрістіанине! стой, мужайся и крѣпися и претерпи все, взирая на будущую славу и великое терпѣніе Хрістово. Тако всякую скорбь приключающуюся побѣдиши. 10)~Хрістіанине, который въ слѣдъ сатаны уклонился! помяни твои обѣты, учиненные на крещеніи, и покайся, съ жалѣніемъ и сокрушеніемъ прибѣгни ко Хрісту, Который за тебе умеръ и пострадалъ, и пріиметъ тебе, яко благъ и человѣколюбецъ. Онъ ожидаетъ тебе, да къ Нему возвратишися. Возвратися убо, пока ждетъ. \textit{Нѣсть спасенія} и блаженства, \textit{кромѣ Его и безъ Него}\footnote{Дѣян.~4,~12.}. Горе душѣ, которая не есть со Хрістомъ! вѣчная бѣда и погибель постигнетъ ее. \textit{Мнѣ же прилѣплятися Богови благо есть}. Съ Нимъ быть есть животъ: безъ Него быть есть явная смерть. 11)~Душа благочестивая! когда въ чемъ проступишися и согрѣшишь, не медли во грѣхѣ твоемъ, да не къ противной части уклонишися; но тотчасъ, признавши грѣхъ свой, кайся и молися Господу: \textit{согрѣшихъ, Господи, помилуй мя!} и отпустится согрѣшеніе твое. Но впредь берегись того, какъ зміина жала: \textit{жало смерти грѣхъ}\footnote{1~Кор.~15,~56.}. Берегись сего жала, да не умреши. Согрѣшить есть человѣческое, но во грѣхѣ быть и лежать есть діавольское. Діаволъ какъ согрѣшилъ, съ того времени непрестанно во грѣхѣ и ожесточеніи лежитъ и во вѣки въ томъ пребудетъ. Берегись убо грѣхъ къ грѣху прилагать, да не съ діаволомъ будеши.

\section{117. Царь подданнаго своего къ себѣ указомъ зоветъ.}

Бываетъ, что царь, хотячи позвать къ себѣ подданнаго своего, посылаетъ къ нему указъ и тѣмъ къ себѣ зоветъ его. Тако Царь небесный Іисусъ Хрістосъ зоветъ всякаго хрістіанина къ Себѣ на оный вѣкъ; зоветъ же чрезъ смерть. Видитъ человѣкъ приближившуюся кончину свою или смерть: тутъ къ нему невидимо приходитъ указъ отъ Царя небеснаго, которымъ къ Нему зовется. Что тогда у бѣднаго человѣка на сердцѣ его? Какій страхъ, какій трепетъ и ужасъ, какій мятежъ, волнованіе, какое отчаяніе колеблетъ его! Ахъ, зоветъ мене Царь небесный, а я не готовъ! Указъ Его ко мнѣ пришелъ, чтобъ мнѣ къ Нему явиться, а я неисправенъ. Вижу я мою кончину приближившуюся, а я о ней никогда и не думалъ. Приближилась мнѣ моя смерть, о которой я никогда не помнилъ. Отворяются мнѣ уже врата къ вѣчности, о которой я никогда не помышлялъ. Боюсь я суда Судіи Бога, котораго прогнѣвлялъ. Совѣсть моя мене обличаетъ и мучитъ, которая представляетъ мнѣ моя грѣхи. Неблагополучная и мучительная вѣчность, въ которую грѣшники нераскаянніи отходятъ, страхомъ и ужасомъ всего мене колеблетъ. Почто я о семъ страшномъ часѣ никогда не помышлялъ? почто въ суетѣ умъ мой занятъ былъ? Почто я столько собиралъ себѣ? почто гонялся за славою и честію міра сего? почто такіе"=то и такіе грѣхи дѣлалъ? почто не внималъ я слову Божію, которое предостерегало мене? Что мнѣ нынѣ пользуетъ богатство, честь и слава сысканная? что пышность и гордость житейская? что богатый домъ мой и различно украшенный? что кареты, кони, многіи слуги, земли и деревни, различныя имена и титулы? что увеселительные сады, галдареи, и пруды? что други, которыхъ я часто банкетами и различными винами увеселялъ, и самъ съ ними веселился? Сія и прочая міра сего суета помрачила и ослѣпила умъ мой, и тако я не моглъ распознать прелести и истины, зла и добра, вреда и пользы, грѣха и добродѣтели, окаянства и блаженства истиннаго; и отняла память отъ мене, память о смерти, къ которой я нынѣ приближился, и которая всему оному мнимому утѣшенію конецъ полагаетъ, и память о вѣчности, въ которую я нынѣ иду. Нынѣ я познаю, что есть прелесть и истина, что есть истинное мое добро и истинное зло. Нынѣ я вижу, что истинно слово Божіе и истинѣ научаетъ. Блажени, которіи внимаютъ тому! Окаянніи, которіи не внимаютъ! О міръ, міръ суеты и прелести исполненный міръ, како ты бѣднаго человѣка прельщаешь! Все я твое сокровище нынѣ оставляю; и, вмѣсто богатства и краснаго дома, въ малый гробъ вселяюся; и, вмѣсто шелковыхъ и вѵссонныхъ одеждъ, чернымъ одѣяніемъ покрыюся; и, вмѣсто многихъ земель и вотчинъ, въ треаршинной ямѣ земной зарыюся; и, вмѣсто богатства, чести и славы, мертвость и тлѣніе наслѣдую; и, вмѣсто роскоши, которою себе утѣшалъ, снѣдь червямъ буду. Прощайте вси! прощайте жена и дѣти, прощайте други и знаеміи мои, прощайте слуги и крестьяне мои, прощайте вотчины и земли мои! я васъ нынѣ оставляю. Прощай міръ! я и тебе нынѣ оставляю, и все твое тебѣ оставляю. \textit{Нагъ изыдохъ отъ чрева матере моея: нагъ и} нынѣ \textit{отхожду} въ путь всея земли. Нынѣ я вижу, что все, что я ни имѣлъ, не мое было; яко все, что ни имѣлъ, нынѣ оставляю, и какъ въ міръ вошелъ ни съ чимъ, такъ и отъ міра отхожду безъ всего. Царь небесный зоветъ мене нынѣ, и иду къ Нему; по трепещу праведнаго и страшнаго суда Его. Онъ лица не пріемлетъ; судитъ по совѣсти нашей и по дѣламъ нашимъ, а не по лицамъ. У Него цари и подданніи, вельможи, князи, господа и раби ихъ, богатіи и нищіи равны суть. Хрістіанине! помяни, что и къ тебѣ пріидетъ указъ отъ небеснаго Царя; и пріидетъ, когда не знаешь, и позоветъ тебе; и что прочіимъ при кончинѣ ихъ случается, и самъ видишь, тое будетъ и тебѣ. Буди убо разуменъ и мудръ, и заблаговременно къ часу тому покаяніемъ и сокрушеніемъ сердца приготовляйся; приготовляйся къ часу, въ которомъ всякому врата къ вѣчности отворятся. Страшенъ тотъ часъ не токмо грѣшникамъ, но и святымъ, которіи, на тотъ всегда взирая, сокрушались и плакали. Отъ того часа слѣдуетъ всякому или вѣчно спастися, или вѣчно погибнуть. \textit{Помни смерть}, и не захощешь съ міромъ веселитися. Воистину вся суета и роскошь міра сего омерзѣетъ тебѣ. Будешь болѣе искать плача и слезъ, нежели веселія и утѣхи.

\section{118. Завтра пріиду.}

Слышимъ слово сіе отъ людей: \textit{завтра пріиду}. Слово сіе говоритъ человѣкъ, когда его другій къ себѣ зоветъ; тогда онъ, или дѣломъ какимъ занятъ, или иную какую нужду имѣя и отъ дома своего отлучитися не могучи, зовущему его отвѣщаетъ тако: \textit{завтра пріиду}. Хрістіанине! Хрістосъ насъ зоветъ къ Себѣ, и зоветъ всегда и непрестанно: \textit{пріидите ко Мнѣ вси труждающіися и обремененніи, и Азъ упокою вы}\footnote{Мѳ.~11,~28.}. Здѣ Хрістосъ Господь призываетъ на покаяніе и чрезъ покаяніе всѣхъ грѣшниковъ къ Себѣ. Но многіи хрістіане отлагаютъ тое покаяніе, и какъ бы отказываютъ Хрісту, и хотя не устами, но сердцемъ говорятъ: \textit{завтра пріиду}. У всѣхъ таковая мысль есть, и слово сіе, \textit{завтра пріиду}, въ сердцѣ имѣется, которыи день отъ дня обращеніе и покаяніе истинное отлагаютъ. Вси таковіи говорятъ въ сердцѣ своемъ: \textit{завтра пріиду}. Говоритъ блудникъ, прелюбодѣй и нечистоты любитель: \textit{завтра пріиду}. Говоритъ піяница и сластолюбецъ: \textit{завтра пріиду}. Говоритъ въ гордости, пышности и въ суетѣ міра сего живущій: \textit{завтра пріиду}. Говоритъ сребролюбецъ, воръ и хищникъ: \textit{завтра пріиду}. Говоритъ всякъ грѣшникъ, который въ грѣхахъ живетъ и не исправляется: \textit{завтра пріиду}. А многіи до болѣзни, многіи до кончины своей покаяніе свое отлагаютъ. Суть такіи, которыи и не думаютъ о томъ. Таковіи неотмѣнно говорятъ въ сердцѣ своемъ: \textit{нѣсть Богъ}\footnote{Пс.~13,~1.}. О бѣдный грѣшникъ! почто утренній день обѣщаешь себѣ, который не въ твоей, но въ Божіей власти есть? Что, когда завтрашняго дня не дождешися? Что, когда Царскій указъ къ тебѣ пріидетъ тотчасъ, и позоветъ тебе тѣмъ Царь небесный Господь не къ покаянію уже, но къ отвѣту и суду?! Какій страхъ, трепетъ, ужасъ и отчаяніе будетъ колебать тогда душу твою! Смерть невидимою дорогою за всякимъ ходитъ, и восхищаетъ человѣка, когда не чаетъ, и гдѣ не чаетъ, и какъ не чаетъ. Что, когда она и къ тебѣ въ такихъ мысляхъ пріидетъ, и безъ голоса возгласитъ тебѣ: иди, человѣче; Господь Вседержитель зоветъ тебе! Что? будешь ли говорить тогда: \textit{завтра пріиду?} нѣтъ! хотя и не хощешь, но пойдешь. Но съ какою надеждою, не знаю. "--- Грѣшная душа! Богъ обѣщалъ намъ благодать Свою и милость; но утрешняго дня не обѣщалъ. А Духъ Святый глаголетъ: \textit{днесь, аще гласъ Его услышите, не ожесточите сердецъ вашихъ}\footnote{94,~7 и 8; Евр.~3,~7 и 8; 4,~7.}. Проповѣдники Хрістовы то и знаютъ, что покаяніе проповѣдуютъ и грѣшникамъ возглашаютъ: \textit{покайтеся}, и пріидите ко Хрісту. Но грѣшная душа отвѣщаетъ въ сердцѣ своемъ: \textit{завтра пріиду}. Слѣпый грѣшникъ! разсуди и осмотрись, кому ты тако отказываешь въ сердцѣ своемъ: \textit{завтра пріиду}. Зоветъ тебе Богъ и Господь твой, Создатель твой, Царь царствующихъ и Господь господствующихъ; зоветъ щедрый и милостивый, долготерпѣливый и многомилостивый Господь; зоветъ любезно, зоветъ Іисусъ Искупитель твой, Который пострадалъ и умеръ за спасеніе твое; зоветъ, да не впадеши въ вѣчную погибель и смерть; зоветъ въ вѣчное Свое царство и блаженство. Онъ жалѣетъ о тебѣ, яко милосердъ, да не погибнеши. Но ты Ему отвѣщаеши мыслію и нераскаяннымъ нравомъ и неисправнымъ сердцемъ: \textit{завтра пріиду!} Безстыдно царю земному, паче же низшему властелину, отказывать и говорить: \textit{завтра пріиду}; весьма тяжко и безстыдно небесному Царю отказывать и отвѣтствовать: \textit{завтра пріиду}. Тяжко и безстыдно, но и страшно. Ибо будетъ время, когда восхотятъ грѣшники пріитить къ Нему; но \textit{затворены будутъ двери}. Тогда услышатъ: \textit{не вѣмъ васъ}; вы Мене не знали, и Я васъ не знаю; вы Мене не слушали, и Я васъ не слушаю\footnote{Лук.~13,~25--27; Мѳ.~25,~10.}. \textit{Понеже звахъ, и не послушасте, и простиралъ словеса, и не внимасте, но отметасте Моя совѣты, и Моимъ обличеніямъ не внимасте: убо и Азъ вашей погибели посмѣюся, порадуюся же, егда пріидетъ вамъ пагуба}\footnote{Притч.~1,~24--26.}. Сего ради глаголетъ: \textit{днесь, аще гласъ Его услышите, не ожесточите сердецъ вашихъ}. Не медли убо, грѣшникъ, обратитися ко Господу, да не, вмѣсто милости Божія, судъ Божій на себѣ дознаешь. Ко Хрісту приходимъ не ногами, но сердцемъ, не премѣненіемъ мѣста, но премѣненіемъ воли и нравовъ въ лучшее. Кто внутрь себе измѣнится, и, отъ злаго обычая отставши, покаяніемъ себе очищаетъ и всякаго бережется грѣха, и волѣ Хрістовой угождаетъ, тотъ ко Хрісту идетъ, и уже не говоритъ: \textit{завтра пріиду}, но съ пророкомъ глаголетъ: \textit{готово сердце мое Боже, готово сердце мое}\footnote{Пс.~107,~2.}. Се иду! \textit{Воставъ иду ко Отцу моему, и реку Ему: Отче, согрѣшихъ на небо и предъ Тобою; и уже нѣсмь достоинъ нарещися сынъ Твой: сотвори мя яко единаго отъ наемникъ Твоихъ}\footnote{Лук.~15,~18.}. О коль любезно смотритъ на таковую душу возвратившуюся и идущую къ Себѣ Іисусъ Хрістосъ! Ахъ, сынъ Мой изгибшій ко Мнѣ возвращается цѣлъ\footnote{24.}! Милуя помилую его, и щедря ущедрю его\footnote{Іер.~31,~20.}. Радуйтеся, ангели, яко обрѣтеся погибшая драхма. Сынъ Мой, отлучившійся отъ Мене, ко Мнѣ возвратился здравъ. Любезное Мое созданіе, человѣкъ, по образу Моему и по подобію сотворенный, но погибшій, цѣлъ обрѣтеся. Бѣдный грѣшникъ! востанемъ мы тако, и поскорѣе поспѣшимъ къ благоутробному Отцу нашему. Онъ ждетъ насъ, и пришедшихъ распростертыми руками обыметъ. Нигдѣ и ни въ чемъ покоя мы не сыщемъ, кромѣ Его. Онъ единъ упокоеваетъ насъ: \textit{пріидите ко Мнѣ вси труждающіися и обремененніи, и Азъ упокою вы}. Полно уже на чужой странѣ медлить; полно злому господину работать; полно грѣхами, яко рожцами негодными, питаться. \textit{Воставше}, пойдемъ ко Отцу нашему, и будетъ питать насъ трапезою, каковая сынамъ Его представляется. \textit{Отче! согрѣшихъ на небо и предъ тобою; и уже нѣсмь достоитъ нарещися сынъ твой: сотвори убо мя, яко единаго отъ наемникъ Твоихъ!}

\section{119. Садовникъ неплодное дерево посѣкаетъ.}

Видимъ, что садовникъ, довольно ожидая отъ яблони или инаго какого дерева плода, и усмотря, что не даетъ желаемаго ему плода, посѣкаетъ дерево тое, яко негодное и безплодное, и во огнь вметаетъ. Тако судъ Божій постигаетъ и посѣкаетъ нераскаяннаго грѣшника. Богъ, яко преблагій и человѣколюбивый, ожидаетъ отъ грѣшника покаянія, и долготерпитъ ему; но когда уже видитъ, что онъ въ нераскаяніи живетъ, какъ и жилъ, тогда праведнымъ Своимъ судомъ посѣкаетъ его, и въ вѣчный огнь вметаетъ. Тако ожидалъ Богъ много времени покаянія отъ нечестивыхъ, которыи были прежде потопа; но когда уже увидѣлъ, что нѣтъ въ нихъ покаянія, ужаснымъ и всемірнымъ наводненіемъ погубилъ ихъ. Тако и отъ Содомлянъ ожидалъ покаянія, и непокаявшихся и въ нечестіи своемъ пребывшихъ огнемъ съ небесе пожеглъ. Тако и отъ Фараона, царя египетскаго, не мало времени ожидалъ покаянія: но когда долготерпѣніе Его пренебреглъ, и во ожесточеніи своемъ остался, мстительную руку Божію на себѣ дозналъ; погрязнулъ, яко камень, во глубинѣ морской со всѣмъ воинствомъ своимъ. Тако и отъ Израильтянъ, изшедшихъ изъ Египта, видѣвшихъ преславная Божія чудеса, но беззаконновавшихъ, ожидалъ покаянія и обращенія; но когда непремѣнны пребыли и въ своемъ нечестіи утвердились, судомъ Божіимъ различно въ пустыни погибли. Тако дозналъ на себѣ судъ Божій Авессаломъ, сынъ Давидовъ, который искалъ убить отца своего и завладѣть царствомъ Израилевымъ, но на древѣ обвѣсился, и между небомъ и землею погибнулъ. Той же судъ Божій и нынѣ постигаетъ беззаконнующихъ, и о долготерпѣніи Божіи нерадящихъ, и каятися и исправитися не хотящихъ. Слышимъ дивныя и страшныя судьбы Божія. Блудники и блудницы часто въ самомъ беззаконномъ дѣлѣ поражаются; хищники, воры и грабители въ самомъ злодѣяніи часто судомъ Божіимъ восхищаются. Страшный судъ постигаетъ убійцъ, злобниковъ, мстителей, клеветниковъ, ругателей, лживыхъ, прелестниковъ, обманщиковъ и прочихъ, которыи ближнимъ своимъ хитрыи навѣты соплетаютъ, и часто падаютъ въ ровъ тотъ, который другимъ изрываютъ и ископываютъ. Тако всякаго нечестиваго дивный и страшный судъ Божій постигаетъ. И хотя не вси беззаконники, по \textit{невѣдомымъ} судьбамъ Божіимъ, здѣ казнятся; однакожъ вси смертію, аки сѣкирою, посѣкаются и во огнь вѣчный ввергаются, и тако заслуженный свой жребій пріемлютъ. И сіе"=то есть, что Предтеча святый сказалъ: \textit{уже и сѣкира при корени древа лежитъ: всяко убо древо, еже не творитъ плода добра, посѣкаемо бываетъ, и во огнь вметаемо}\footnote{Мѳ.~3,~10.}. Хрістіанине, который небрежешь о покаяніи!

убойся Божія суда и покайся, да не нечаянно постигнетъ тебе и отошлетъ во огнь геенскій! Не знаешь самъ, гдѣ и какъ и когда постигнетъ тебе судъ; и гдѣ чаешь себѣ добра и корысти, тамо будетъ тебѣ зло. Фараонъ ожесточенный гналъ въ слѣдъ Израиля, и чаялъ себѣ корысти; но сыскалъ себѣ погибель и гробъ въ морѣ Чермномъ. Подобный жребій и прочіи беззаконники ожесточенныи на себѣ дознаютъ. Смотри убо на дивныя судьбы Божія, и престани грѣшить; и кайся о томъ, что грѣшилъ, да милостивъ будетъ тебѣ Господь. Терпѣлъ тебѣ доселѣ Богъ, видя твои беззаконныя дѣла, и ожидалъ отъ тебе покаянія; но впредь стерпитъ ли, неизвѣстно. Сіе разумѣть должно и о всемірномъ и страшномъ судѣ Божіи. Долготерпитъ Господь нечестивымъ и нераскаяннымъ грѣшникамъ, и ожидаетъ отъ нихъ покаянія и исправленія, якоже долготерпѣлъ бывшимъ прежде потопа нечестивымъ, и ожидалъ покаянія отъ нихъ; но когда уже увидитъ, что нѣтъ покаянія въ грѣшникахъ, нѣтъ обращенія, нѣтъ исправленія, міръ въ нечестіи своемъ упорно стоитъ: тутъ нечаянно, \textit{какъ молнія}, явится Судія Хрістосъ Господь, и возгласятъ трубы: идите на судъ Хрістовъ! И исполнится апостольское слово: \textit{егда рекутъ, миръ и утвержденіе; тогда внезапу нападетъ на нихъ всегубительство, якоже и болѣзнь во чревѣ имущей: и не имутъ избѣжати}\footnote{1~Сол.~5,~3.}. Тогда вси нераскаянніи грѣшники, якоже прежде потопа бывшіи въ водахъ погибли, погибнутъ и погрузятся \textit{въ езерѣ огненномъ}, и будутъ снѣдать чрезъ всю вѣчность беззаконія и нечестія своего плоды. И сіе"=то есть, что Апостолъ Петръ написалъ: \textit{не коснитъ Господь обѣтованія, якоже нѣцыи коснѣніе мнятъ; но долготерпитъ на насъ, не хотя, да кто погибнетъ, но да вси въ покаяніе пріидутъ}, и проч.\footnote{2~Петр.~3,~9.} Смотри, грѣшникъ! Богъ ожидаетъ отъ тебе покаянія; не неради убо о благости и долготерпѣніи Божіи, и покайся, да не въ нераскаяніи и неисправности или смерть твоя, или судъ Хрістовъ застанетъ тебе. Блаженъ и мудръ, кто въ осторожности живетъ, и отъ бѣдствія иныхъ знаетъ уклоняться бѣды. \textit{Помилуй мя Боже, по велицѣй милости Твоей, и по множеству щедротъ Твоихъ очисти беззаконіе мое}, и проч.\footnote{Пс.~50,~3.}

\section{120. Сѣти, простертыя на пути.}

Видимъ, что люди на тѣхъ мѣстахъ, на которыхъ птицы и звѣри ходятъ, тайныя простираютъ сѣти, дабы ихъ тѣми уловить. Тако сатана, хрістіанине, простираетъ намъ сѣти свои и тщится насъ уловить въ нихъ. Мы на святомъ крещеніи, благодатію Божіею, взошли на путь спасительный, и тѣмъ тщимся пріитить къ небесному нашему отечеству, къ которому отъ Создателя нашего созданы мы, и падшіи искуплены, и словомъ Божіимъ позваны, и банею пакибытія обновлены. На семъ нашемъ пути многоразличныя сѣти распростираетъ врагъ нашъ, дабы насъ тѣми въ свою погибель уловить. Сіе его дѣло и тщаніе есть. Сѣти его пагубныя суть: 1)~Многоразличныя \textit{ереси, расколы и суевѣрія}. О, коль многія души хрістіанскія сими сѣтьми уловлялъ и уловляетъ хитрый и лукавый врагъ, прешедшіе вѣки свидѣтельствуютъ, и нынѣ тое съ сожалѣніемъ и воздыханіемъ видимъ! Онъ о томъ и тщится, чтобы въ сѣти ереси и раскола и суевѣрія запутать человѣка, и чрезъ того иныхъ многихъ соблазнить, уловить и повредить. Ересь бо есть, какъ моровая язва, которая, въ единомъ человѣкѣ наченшися, многихъ заражаетъ. Ересь есть мнѣніе и ученіе, Богу и святому слову Его противное, и происходитъ отъ незнанія и неразумѣнія священнаго Писанія. Почему, чтобы намъ сими сѣтьми лукаваго не запутаться, должно неуклонно держаться святаго Божія слова. Божіе бо слово есть правило вѣры нашея, какъ пророкъ святый глаголетъ: \textit{свѣтильникъ ногама моима законъ Твой, и свѣтъ стезямъ моимъ}\footnote{Пс.~118,~103.}. Святое Писаніе протолковали и изъяснили богоносніи отцы и учители церковные. Полезно къ разумѣнію того и ихъ книги прочитывать. Держись убо, хрістіанине, Божія слова, да не запутаешися въ сѣти ереси и раскола и суевѣрія. А когда услышишь отъ человѣка слово произносимое, противное Божію слову, отъ таковаго, какъ моровою язвою зараженнаго, берегись. Духъ непріязненный отъ злаго запаха познается. Слово человѣческое есть свидѣтель сердца человѣческаго и духа въ немъ живущаго. Отъ сихъ сѣтей вражіихъ предостерегаетъ насъ Божіе слово. \textit{Возлюбленніи, не всякому духу вѣруйте, но искушайте духи, аще отъ Бога суть: яко мнози лжепророцы изыдоша въ міръ}\footnote{1~Іоан.~4,~1.}. И паки: \textit{внемлите отъ лживыхъ пророкъ, иже приходятъ къ вамъ во одеждахъ овчихъ, внутрь же суть волцы хищницы. Отъ плодъ ихъ познаете ихъ}\footnote{Мѳ.~7,~15 и 16.}. 2)~Простираетъ сѣти \textit{любве міра сего, гордости и суеты его}. Представляетъ всякой душѣ хрістіанской честь, славу, богатство и роскоши міра сего, и шепчетъ во уши, какъ хорошо и пріятно быть въ чести, отъ всѣхъ почитаніе и поклоненіе имѣть, отъ всѣхъ славнымъ и хвалимымъ быть, быть въ богатствѣ, жить въ богатомъ и красномъ домѣ, имѣть многихъ слугъ, одѣваться шелковою и пригожею одеждою, проѣзжаться каретою и цугомъ, знаться со знатными и славными людьми, представлять по вся дни богатую трапезу и веселиться, и пріѣзжающихъ гостей веселить, и проч. Тако шепча, замышляетъ лукавый духъ, дабы бѣдный человѣкъ въ такихъ суетныхъ мысляхъ запутался и не помышлялъ о должности и званіи своемъ хрістіанскомъ, и забылъ бы, что онъ искупленъ и позванъ и обновленъ къ вѣчному животу. О, коль многіи хрістіане и сими сѣтьми уловляются, наипаче въ нынѣшнее время! Хрістіанине, отвѣщай злому духу тако: \textit{ничтоже внесохомъ въ міръ сей, явѣ, яко ниже изнести что можемъ; имѣюще же пищу и одѣяніе, сими довольни будемъ}\footnote{1~Тим.~6,~7 и 8.}. Отъ сихъ сѣтей предостерегаетъ насъ слово Божіе. \textit{Не любите міра, ни яже въ мірѣ. Аще кто любитъ міръ, нѣсть любве Отчи въ немъ; яко все, еже въ мірѣ, похоть плотская, и похоть очесъ, и гордость житейская, нѣсть отъ Отца, но отъ міра сего есть. И міръ преходитъ, и похоть его: а творяй волю Божію, пребываетъ во вѣки}\footnote{1~Іоан.~2,~15--17.}. И паки: \textit{горняя мудрствуйте, а не земная}\footnote{Кол.~3,~2.}. \textit{Не можете Богу работати и мамонѣ}\footnote{Мѳ.~4,~24.}. \textit{Отврати очи мои, Господи, еже не видѣти суеты; въ пути Твоемъ живи мя. Постави рабу Твоему слово Твое въ страхъ Твой}\footnote{Пс.~118,~37 и 38.}. 3)~Простираетъ сѣти свои намъ врагъ, когда ко всякому грѣху насъ \textit{прельщаетъ}, и тѣмъ, какъ сѣтію, уловить и запутать тщится. О, коль многіи сими пагубными его сѣтьми уловляются человѣки, и тѣми запутываются! Злобники и убійцы сими сѣтьми уловлены отъ него; прелюбодѣи, блудники и вси нечистоты любители уловлены отъ него; тати, хищники, грабители и лихоимцы уловлены отъ него; піяницы и сластолюбцы уловлены отъ него; сквернословцы и кощунники уловлены отъ него; ссорящіися и другъ друга угрызающіи уловлены отъ него; родителямъ и властямъ противящіися уловлены отъ него; клеветники, хульники и ругатели уловлены отъ него; лживіи, прелестники и обманщики уловлены отъ него; чародѣи и ихъ къ себѣ призывающіи уловлены отъ него; въ праздности живущія, туне и лестно чуждый хлѣбъ поядающіи уловлены отъ него; и прочіи беззаконнующіи и творящіи неправду уловлены отъ него. Отъ сихъ сѣтей предостерегаетъ насъ Божіе слово. \textit{Пріидите чада, послушайте Мене, страху Господню научу васъ. Кто есть человѣкъ хотяй животъ, любяй дни видѣти благи? Удержи языкъ твой отъ зла, и устнѣ твои еже не глаголати льсти; уклонися отъ зла и сотвори благо; взыщи мира, и пожени его}\footnote{Пс.~33,~12--14.}. И паки: \textit{да отступитъ отъ неправды всякъ именуяй имя Господне. Оброцы бо грѣха смерть}\footnote{2~Тим.~2,~19; Римл.~6,~23 и на прочихъ премногихъ мѣстахъ.}. Въ предосторожность нашу, написаны въ святомъ Писаніи и казни, бывшія за грѣхи. Сего бо ради онѣ представляются намъ, да бережемся дѣлать того, что дѣлали беззаконники и казни пострадали. Берегись убо хрістіанине грѣха всякаго. Діавольская сѣть есть грѣхъ. Берегись сѣти, да не впадеши въ сѣть и увязнеши въ сѣти. 4)~Простираетъ намъ діаволъ сѣти своя, когда намъ влагаетъ чрезъ злыя мысли \textit{невѣріе, сумнѣніе, всякія хулы} на имя Божіе, \textit{отчаяніе}, и прочая. Сія и прочая чувствуютъ на себѣ души благочестивыя и противу его подвизающіяся. Отъ сихъ сѣтей предостерегаетъ насъ Божіе слово: \textit{трезвитеся, бодрствуйте, зане супостатъ вашъ діаволъ, яко левъ, рыкая, ходитъ, искій кого поглотити; ему же противитеся тверди вѣрою}\footnote{1~Петр.~5,~8 и 9.}. Хрістіанине! берегись сихъ сѣтей вражіихъ. Сколько разъ чувствуешь таковую злую мысль, столько разъ простираетъ тебѣ врагъ сѣть свою, и хощетъ тебе уловить. Мерзкій духъ отъ смердящаго запаха познаемъ. 5)~Видимъ его и сіи сѣти, что онъ, когда самъ чрезъ себе не можетъ уловить человѣка, насылаетъ къ нему \textit{злыхъ людей}, служителей своихъ, чрезъ которыхъ, яко истое свое орудіе, тщится уловить его. Отсюду видимъ много прелестниковъ, хитрецовъ, обманщиковъ, которыи притворяютъ себе добрыми и тщатся вкрасться въ сердце благочестивое и въ дружество съ нимъ, но внутрь суть волки въ овчіихъ кожахъ, и тайное орудіе діавольское. Чрезъ таковыхъ служителей своихъ діаволъ тщится уловить благочестивую душу, и въ сѣти свои запутать. Таковыи волки опаснѣйшіи суть, нежели самъ діаволъ. Но познаются отъ плодовъ своихъ: ибо сколько ни хитрятъ, и скрываютъ себе, однакожъ ядъ свой, внутрь ихъ крыющійся, оказываютъ. Таковыхъ людей берегись, хрістіанине! Сказано намъ отъ Спасителя нашего: \textit{отъ плодъ ихъ познаете ихъ}\footnote{Мѳ.~6,~16.}. 6)~Простираетъ сѣти свои намъ діаволъ, когда возставляетъ на насъ \textit{гоненіе}, злобу, ненависть, посмѣяніе и руганіе людей. Отсюду"=то бываетъ, что благочестивыя души много страждутъ отъ любителей міра сего и слышатъ таковыя клеветы, на языкахъ и устахъ человѣческихъ носимыя, каковыхъ на себѣ не знаютъ. Сія есть кознь и хитрость діавольская, дабы человѣкъ благочестивый, таковыхъ клеветъ и ненависти людской не стерпѣвши, совратился съ добраго пути. Тако на древнихъ хрістіанъ ложныя и тяжкія клеветы хитростію діавольскою вымышляли идолопоклонники, дабы они, хрістіанское благочестіе оставивши, къ нечестію ихъ обратилися. Тойжде злохитрый духъ и нынѣ дѣйствуетъ чрезъ ложныхъ хрістіанъ. Хрістіанине! отъ сихъ сѣтей предостерегаетъ насъ Хрістосъ Господь, и укрѣпляетъ насъ на пути Своемъ, глаголя: \textit{буди вѣренъ даже до смерти, и дамъ ти вѣнецъ живота}\footnote{Апок.~2,~10.}. И паки: \textit{претерпѣвый до конца, той спасется}\footnote{Мѳ.~24,~13.}. 7)~Тайныя и очень сокровенныя сѣти его суть, когда онъ \textit{подъ видомъ добра зло}, и подъ видомъ добродѣтели порокъ, яко ядъ подъ медомъ сокровенный, представляетъ намъ. Тако научаетъ мужа жену свою, и жену мужа своего оставлять подъ видомъ воздержанія; но тѣмъ замышляетъ лукавый духъ ввергнуть ихъ въ ровъ прелюбодѣянія. Много сего зла на свѣтѣ есть. Влагаетъ человѣку въ сердце отъ нѣкоторыхъ снѣдей воздерживаться (отъ всѣхъ бо не возможно); но тѣмъ замышляетъ тое, чтобы онъ созданіе Божіе порочилъ, и въ высокоуміе пришелъ, и другихъ ядущихъ презиралъ и уничтожалъ. Много и сего зла въ мірѣ есть. \textit{Всякое же созданіе Божіе добро, и ничтоже отметно, со благодареніемъ пріемлемо: освящается бо словомъ Божіимъ и молитвою}\footnote{1~Тим.~4,~4 и 5.}. Научаетъ человѣка изъ дома исходить и другихъ посѣщать подъ видомъ любве; но тѣмъ хощетъ его удалить отъ молитвы, полезнаго размышленія и дѣлъ званія его, и ввергнуть въ праздность, празднословіе, оклеветаніе и осужденіе и прочая злая, и бываетъ, что человѣкъ не тотъ уже въ домъ возвратится, который изъ дома вышелъ. Нѣтъ полезнѣе человѣку, какъ дома и уединенія держаться. Научаетъ человѣка чести искать, подъ видомъ служенія обществу: \textit{будешь"=де, на чести будучи, обществу служить}; но тѣмъ замышляетъ развратить его и въ безчисленная злая вринуть, и соблазнами его многихъ повредить. Хорошо обществу служить, но зло есть неправду дѣлать и \textit{мамонѣ работать}. Скоро оказываютъ себе таковіи служители. Воистину многіи бы святыми были, когда бы въ чести не были. Честь неразумному есть какъ мечь, которымъ и себе и другихъ убиваетъ. Надобно прежде научиться управлять себе, а потомъ уже другихъ. Влагаетъ въ сердце человѣку врагъ собирать богатство ради подаянія милостыни: \textit{будешь"=де тѣмъ снабдѣвать убогихъ, и оттуду не малую пользу имѣть}; но тѣмъ замышляетъ вкоренить въ сердце его сребролюбіе и ко всякой неправдѣ привести. Отсюду бываетъ, что многіи щедро другимъ даютъ, но у другихъ отнимаютъ; многіи господа нищимъ и убогимъ щедры бываютъ, но своимъ крестьянамъ скупы, и ихъ въ нищету и убожество приводятъ. Полезно милостыню давать, но пагубно другихъ обиждать. Какая, се милостыня есть "--- у одного отнимать, а другому давать! Не милостыня, но безчеловѣчіе есть тое. Первѣйшая хрістіанская милость есть никого не обидѣть. "--- Дѣлаетъ и тое врагъ, что подъ видомъ любовнаго угощенія, научаетъ человѣка обѣды и пиршества строить и къ тѣмъ созывать пріятелей своихъ и угощать ихъ; но тѣмъ замышляетъ въ піянство, безчиніе, роскоши и расточеніе имѣнія и прочая злая привести хозяина и другихъ. Отсюду видимъ, коль много таковыхъ любовныхъ угощеній, но душамъ вредныхъ умножилося. Хорошо отворять двери дома своего и принимать гостей и угощать ихъ, но всѣхъ безъ разбору, не имущихъ чего ясти, странныхъ, благообразно и безъ всякаго безчинія; таковое страннолюбіе есть хрістіанская добродѣтель. Но худо прихотямъ своимъ угождать и піянствовать и имѣніе на прихоти иждивать, и прочая злая, которыя при банкетахъ и пиршествахъ бываютъ, дѣлать. Многіи на банкеты и пиршества довольно изнуряютъ, но нищимъ и убогимъ ничего не даютъ. Многіи богатыхъ и славныхъ на обѣды въ себѣ зовутъ, но бѣднымъ, не имѣющимъ пропитанія, двери затворяютъ. Къ сему злу супостатъ приводитъ человѣка подъ видомъ любовнаго угощенія, и проч. Сія и прочія многоразличныя сѣти распростираетъ врагъ на пути, которымъ хрістіане идутъ къ вѣчному животу. 8)~Отъ сихъ и прочихъ сѣтей вражіихъ никакій человѣкъ \textit{самъ собою} не можетъ избавитися. Что жъ убо намъ, хрістіанине, дѣлать, чтобы въ сѣти вражія не впасть? Пророкъ святый своимъ примѣромъ указуетъ на посредствіе, которымъ отъ сѣтей его избавляемся, и глаголетъ: \textit{возведохъ очи мои въ горы, отнюдуже пріидетъ помощь моя. Помощь моя отъ Господа, сотворшаго небо и землю}. Но далѣе говоритъ, и остерегаетъ насъ: \textit{не даждь во смятеніе ноги твоея, ниже воздремлетъ храняй тя. Се не воздремлетъ, ниже уснетъ храняй Израиля. Господь сохранитъ тя, Господь покровъ твой на руку десную твою. Во дни солнце не ожжетъ тебе, ниже луна нощію. Господь сохранитъ тя отъ всякаго зла, сохранитъ душу твою Господь. Господь сохранитъ вхожденіе твое, и исхожденіе твое, отъ нынѣ и до вѣка}\footnote{Пс.~120,~1--8.}. Вотъ посредствіе спасительное "--- помощь Божія. Все наше тщаніе и трудъ и стараніе \textit{безъ помощи Божіей} есть безполезно и суетно\footnote{Іоан.~15,~5.}. Нужна убо намъ всегдашняя и усердная молитва, дабы намъ помоглъ Господь въ такомъ важномъ дѣлѣ. Знаетъ Онъ Самъ, что нужна намъ помощь Его; но хощетъ, чтобы мы \textit{просили, искали и толкали} въ двери милосердія Его\footnote{Мѳ.~7,~7.}. 9)~Когда Израильтяне вышли изъ Египта и прошли сквозѣ Чермное море, и шли пустынею въ землю обѣтованную, то хотя Моѵсей святый вождь ихъ былъ, и преемникъ его Іисусъ Навинъ, однакожъ Самъ Господь невидимо присутствовалъ имъ, и велъ ихъ и враговъ ихъ, противящихся имъ, поражалъ, и тако ввелъ ихъ въ землю обѣтованную, якоже поетъ Ему пророкъ: \textit{Боже, ушима нашима услышахомъ, и отцы наши возвѣстиша намъ дѣло, еже содѣлалъ еси во днехъ ихъ, во днехъ древнихъ. Рука Твоя языки потреби, и насадилъ я еси; озлобилъ еси люди, и изгналъ еси я. Не бо мечемъ своимъ наслѣдиша землю, и мышца ихъ не спасе ихъ: но десница, Твоя, и мышца Твоя, и просвѣщеніе лица Твоего, яко благоволилъ еси въ нихъ}\footnote{Пс.~43,~2--4.}. Мы чрезъ святое крещеніе избавилися отъ работы діавольскія, и путемъ житія сего, какъ \textit{пустынею}, идемъ въ небесное наше отечество, къ которому кровію Хрістовою искуплены мы; но діаволъ съ своими козненными сѣтьми, міръ съ соблазнами, плоть со страстьми и похотьми противятся намъ и запинаютъ намъ, и не допущаютъ насъ до того. Однакожъ мы своего дѣла да не оставляемъ, но далѣе и далѣе путемъ нашимъ да поступаемъ. Ибо \textit{Тойже} Господь Іисусъ Хрістосъ и намъ невидимо присутствуетъ, по неложному Своему обѣщанію: \textit{се Азъ съ вами есмь во вся дни до скончанія вѣка, аминь}\footnote{Мѳ.~28,~20.}. И ведетъ насъ и приведетъ къ небесному отечеству, аще только вѣрою и любовію, смиреніемъ и кротостію послѣдуемъ Ему. \textit{Вѣренъ бо есть обѣщавый}\footnote{Евр.~10,~23.}. Даетъ, что обѣщалъ. \textit{Овцы Моя гласа Моего слушаютъ, и Азъ знаю ихъ, и по Мнѣ грядутъ. И Азъ животъ вѣчный дамъ имъ, и не погибнутъ во вѣки, и не восхититъ ихъ никтоже отъ руки Моея}, глаголетъ Господь\footnote{Іоан.~10,~27 и 28.}. Хрістіанине! будемъ только овцы Хрістовы, и отдадимъ себе и весь животъ нашъ во всесильныя Хрістовы руки: тогда неотмѣнно въ небесной оградѣ будемъ и на пажити вѣчнаго живота. И хотя много искушеній, бѣдъ и напастей въ мірѣ семъ претерпимъ, однакожъ въ вѣчномъ покоѣ будемъ. И на насъ исполнится псаломническое слово: \textit{яко искусилъ ны еси, Боже, разжеглъ ны еси, якоже разжизается сребро. Ввелъ ны еси въ сѣть; положилъ еси скорби наша на хребтѣ нашемъ. Возвелъ еси человѣки на главы наша. Проидохомъ сквозѣ огнь и воду, и извелъ еси ны въ покой}\footnote{Пс.~65,~10--12.}. Да помышляемъ и тое, что Апостолъ написалъ: \textit{непщую, яко недостойны страсти нынѣшняго времене къ хотящей славѣ явитися въ насъ}\footnote{Римл.~8,~18.}. Надежда добрая и воздаяніе всякій трудъ, подвигъ, болѣзнь и печаль облегчеваютъ.

\section{121. Велико.}

Видимъ, что люди много кое"=чего за велико поставляютъ; но хрістіанамъ едино только тое \textit{велико}, что Божественное и вѣчное. Ибо они видятъ и думаютъ, что все земное и временное, какъ тѣнь, преходитъ, и потому все имъ тое, какъ ничто. Сіе есть правое мнѣніе, сіе есть истина непреоборимая. За велико люди почитаютъ рожденнымъ быть отъ царя, и быть и называться чадомъ царскимъ; но хрістіанамъ \textit{велико} есть рожденнымъ быть отъ Бога, и называться и быть чадомъ Божіимъ. Сему удивляется Апостолъ святый, и со удивленіемъ глаголетъ: \textit{видите, какову любовь далъ есть Отецъ намъ, да чада Божія наречемся, и есмы: сего ради міръ не знаетъ насъ, зане не позна Его. Возлюбленніи, нынѣ чада Божія есмы, и не у явися, что будемъ: вѣмы же яко, егда явится, подобна Ему будемъ; ибо узримъ Его, якоже есть}\footnote{1~Іоан.~3,~1 и 2.}. \textit{Елицы пріяша Его} (Хріста), \textit{даде имъ область чадомъ Божіимъ быти, вѣрующимъ во имя Его, иже не отъ крове, ни отъ похоти плотскія, ни отъ похоти мужескія, но отъ Бога родишася}\footnote{Іоан.~1,~12 и 13.}. \textit{Вси бо вы сынове Божіи есте вѣрою о Хрістѣ Іисусѣ. Елицы бо во Хріста крестистеся, во Хріста облекостеся}\footnote{Гал.~3,~26 и 27.}. Откуду и Господь глаголетъ апостоламъ и прочимъ вѣрнымъ Своимъ; \textit{восхожду ко Отцу Моему и Отцу вашему, и Богу Моему и Богу вашему}\footnote{Іоан.~20,~17.}. Сего ради и Отцемъ Отца Своего намъ нарицать повелѣлъ и молитися къ Нему тако: \textit{Отче нашъ, Иже еси на небесѣхъ}, и проч.\footnote{Мѳ.~6,~9.} Хрістіанине! подлинно \textit{велико} есть, велика честь и слава быть и называться чадомъ Божіимъ, и всю и всякую славу міра сего несравненно превосходитъ, но отъ любителей міра сего презирается. Хрістіанине! ищи того \textit{великаго}, и будеши подлинно честенъ и славенъ. Чада подобны родителямъ раждаются: надобно и хрістіаномъ подобными быти Богу, когда хотятъ быть чадами Божіими. \textit{Святи будите, яко Азъ святъ есмь}\footnote{1~Петр.~1,~16; Лев.~11,~44; 19,~2.}. \textit{Будите милосерди, якоже и Отецъ вашъ милосердъ есть}\footnote{Лук.~6,~36.}. \textit{Бывайте убо подражатели Богу, якоже чада возлюбленная}\footnote{Еф.~5,~1.}. За велико люди почитаютъ подлому человѣку съ царемъ бесѣдовать; но хрістіанамъ велико есть съ Богомъ небеснымъ и вѣчнымъ Царемъ бесѣдовать. И подлинно, какъ не велико человѣку, который есть \textit{земля и пепелъ}, ктомуже и \textit{грѣшникъ}, приступать къ Богу, Царю царствующихъ и Господу господствующихъ и во свѣтѣ живущему неприступномъ, приступать, говорю, и стоять предъ Нимъ, и бесѣду свою предлагать Ему! Симъ и архангелъ Гавріилъ хвалится: \textit{азъ есмь Гавріилъ, предстояй предъ Богомъ}\footnote{Лук.~1,~19.}. Воистину и хрістіанинъ предъ Богомъ стоитъ и бесѣдуетъ съ Богомъ, когда истинно и сердечно Ему молится и поетъ Его\footnote{Пс.~144,~1--8; 37,~10; 18,~15.}. Ангельское дѣло есть Богу предстоять и пѣть Его. Ангели Богу предстоятъ и поютъ: \textit{святъ, святъ, святъ Господь Саваоѳъ}\footnote{Ис.~6,~3.}. Ангеламъ подражаютъ и люди, когда Богу сердечно молятся и хвалятъ Его, якоже поютъ хрістіане на литургіи: \textit{иже херувимы тайно образующе, и животворящей Троицѣ трисвятую пѣснь припѣвающе}, и проч. Видишь, хрістіанине, какъ велико есть Богу молиться и хвалить Его. Но премногіи хрістіане о томъ \textit{великомъ} небрегутъ; многіи оставляютъ тое, многіи дѣлаютъ, но съ крайнимъ небреженіемъ. Ибо или съ великимъ поспѣшеніемъ, и какъ можетъ языкъ исправиться, или безъ всякаго разумѣнія, молятся и поютъ, и, какъ сказать, не знаютъ, кому и о чемъ молятся, кого и о чемъ поютъ. Таковымъ приличествуетъ слово Божіе: \textit{приближаются Мнѣ людіе сіи усты своими, и устнами чтутъ Мя; сердце же ихъ далече отстоитъ отъ Мене}\footnote{Мѳ.~15,~8.}. Почему пророкъ поющимъ Бога глаголетъ: \textit{пойте Богу нашему, пойте; пойте Цареви нашему, пойте, яко Царь всея земли Богъ; пойте разумно}\footnote{Пс.~46,~7 и 8.}. Хрістіанине, хощеши ли предъ Богомъ стоять всегда и Ему бесѣдовать? Буди всегда умомъ и сердцемъ твоимъ съ прелюбезнымъ Богомъ, и сердцемъ и умомъ покланяйся Ему, и молись Ему и хвали Его, и будешь всегда Ему предстоять и бесѣдовать съ Нимъ. Сіе возможно будетъ тебѣ дѣлать на всякомъ мѣстѣ и всегда, днемъ и нощію, по утру и въ вечеру, и когда хощеши, въ народѣ и уединеніи, въ дѣлѣ и безъ дѣла, на пути и въ домѣ, стоя, ходя, сидя и на ложѣ лежа, словомъ, всегда и на всякое время, и вездѣ. Понеже Богъ вездѣ есть, и всегда, и вездѣ готовъ слушать тебе\footnote{Іов.~19,~23; Пс.~33,~16.}. Всегда и всегда двери къ Нему отверсты. Но кто хощетъ съ пользою Богу въ молитвѣ предстоять, и съ Нимъ бесѣдовать и хвалить Его, и тако ангельское званіе совершать, тотъ долженъ ангеламъ и подражать чистотою и добродѣтельми. За велико люди почитаютъ видѣть земнаго царя, почему, гдѣ онъ бываетъ, вси туды спѣшатъ, чтобы видѣти его; но хрістіанамъ \textit{велико} есть видѣти небеснаго Царя, Іисуса Хріста, Царя царей и Господа господей, Того, Который, сый \textit{соестественъ и соприсносущенъ} Богу Отцу, Слово Отчее и единородный Сынъ Его, Царь и Царевъ Сынъ, Сый въ нѣдрѣхъ Отчихъ, благоволилъ вообразитися въ человѣка, и на земли пожилъ съ человѣками, и пострадалъ и умеръ за человѣковъ, и изъ мертвыхъ воскреслъ, и на небо возшелъ, и сѣдитъ одесную Бога Отца во славѣ Отчей, "--- Того Царя видѣти хрістіанамъ \textit{велико} есть. Ибо видѣти Его "--- крайнее блаженство и вѣчная жизнь, и радость паче всякія радости, и сладость паче всякія сладости; видѣти Его есть ангельская пища и пресладкое питіе избранныхъ Божіихъ. \textit{Когда пріиду и явлюся лицу Божію}\footnote{41,~3.}? Когда пріиду и увижу Того, Который ради мене, бѣднаго грѣшника, въ міръ пришелъ и бѣдствовалъ, и страдалъ и умеръ, и тако отъ діавола, смерти и ада искупилъ мене? Когда пріиду и увижу такого и толикаго моего Любителя, моего Благодѣтеля, моего Искупителя, моего Спасителя? \textit{Что бо ми есть на небеси, и отъ Тебе что восхотѣхъ на земли? Исчезе сердце мое и плоть Моя: Боже сердца моего, и часть моя Боже во вѣкъ}\footnote{72,~25 и 26.}. Ничего я не хощу и не желаю ни на небеси, ни на земли, кромѣ Тебе единаго, Слове Божій и Дѣвы Сыне, Іисусе Хрісте, Боже мой, Боже боговъ, Господи! Хрістіанине! пожелаемъ сего Царя видѣть, и очистимъ себе покаяніемъ, и убѣлимъ одежды наша въ крови Агнчей, да внидемъ въ преславный чертогъ Его, и увидимъ того Царя нашего, Егоже царствію не будетъ конца. "--- За велико люди почитаютъ въ домъ свой принять царя земнаго, но хрістіанамъ \textit{велико} есть принять въ домъ сердца своего Царя небеснаго, Іисуса Хріста. И воистину \textit{велико} есть. Какая бо слава, какая честь, какая радость и веселіе въ томъ дому будетъ, въ который небесный и преславный сей Гость пріидетъ, сказать не возможно. Миръ и радость тамо небесная, и царствіе Божіе тамо. Ибо гдѣ Хрістосъ Богъ, тамо и царствіе Божіе и блаженство Его. \textit{Се стою при дверехъ и толку: аще кто услышитъ гласъ Мой, и отверзетъ двери, вниду къ нему, и вечеряю съ нимъ, и той со Мною}\footnote{Апок.~3,~20.}. О благости, о человѣколюбія Твоего, Господи! Какая Тебѣ, сѣдящему на престолѣ славы Твоея и живущему во свѣтѣ неприступномъ, какая польза съ того, что хощеши внити въ домы смиренныхъ душъ нашихъ? Не довлѣетъ ли Тебѣ храмъ славы Твоея? \textit{Царство бо Твое царство всѣхъ вѣковъ, и владычество Твое во всякомъ родѣ и родѣ}\footnote{Пс.~144,~13.}. Но глаголетъ человѣколюбивый Господь: \textit{се стою при дверехъ и толку}, и прочая; ибо Онъ, яко благій и человѣколюбивый, нашей пользы, а не Своей, нашего блаженства, а не Своего, хощетъ и ищетъ. Наше спасеніе и блаженство любимая Его корысть. Царь земный, по прошенію, къ подданному своему въ домъ входитъ, и уготованными отъ хозяина благими учреждается. Царь небесный Іисусъ Хрістосъ не тако: Онъ \textit{Самъ} къ дверямъ сердецъ нашихъ приходитъ, и толкаетъ въ двери, и хощетъ къ намъ внити; и, вшедши, пищи нашея отъ насъ не требуетъ, но Самъ Свою пищу приноситъ, приноситъ пресладкую и небесную пищу; и не нашими благими учреждается, но Своими невидимыми и небесными насъ учреждаетъ. \textit{Хвали, душе моя, Господа!} Но бѣдность и окаянство человѣческое есть, что великаго сего и дражайшаго и небеснаго Гостя не чувствуетъ человѣкъ. Онъ приходитъ и толкаетъ въ двери всякаго; но бѣдная душа, любовію міра сего и различныхъ плотскихъ прихотей шумомъ оглушенная, не слышитъ Его пресладкаго гласа, и тако человѣколюбивый Іисусъ, постоявши при дверехъ и ничего не успѣвши, отходитъ празденъ. Человѣче! царя земнаго, или и низшаго какого властелина, въ домъ нашъ хотящаго внити, не пустить стыдно: кольми паче стыдно не пустить Царя небеснаго. Горе душѣ безъ Хріста! подобна есть дому безъ хозяина; подобна есть граду, разоренному и опустошенному отъ враговъ; подобна есть кораблю, на морѣ плавающему безъ кормчія; подобна есть овцѣ, безъ пастыря блудящей по пустынѣ; подобна есть человѣку, совратившемуся съ праваго пути и заблуждающему; подобна есть больному, отъ лѣкаря и прочіихъ оставленному. Но вина въ самой душѣ есть, что Хріста въ себѣ живущаго не имѣетъ. \textit{Аще кто любитъ Мя, слово Мое соблюдетъ: и отецъ Мой возлюбитъ его, и къ нему пріидемъ, и обитель у него сотворимъ}, глаголетъ Господь\footnote{Іоан.~14,~23.}. Возлюбленный хрістіанине, возлюбимъ Хріста и другъ друга, и пріидетъ со Отцемъ и Святымъ Духомъ небесный и прелюбезный Гость сей къ намъ, и обитель у насъ сотворитъ. Тогда мы подлинно блажени будемъ, хотя и весь міръ возненавидитъ насъ. За велико люди почитаютъ побѣдить множество народа и взять крѣпкіе грады, но хрістіанамъ \textit{велико} есть самого себе, то"=есть, прихоти своя, самолюбіе, сребролюбіе, нечистоту, зависть, гнѣвъ, славолюбіе и прочее зло, внутрь сердца крыющееся, побѣдить, или, какъ Апостолъ написалъ, \textit{плоть распять со страстьми и похотьми}\footnote{Гал.~5,~24.}. То"=то преславная побѣда, то"=то преславный побѣдитель и воинъ Хрістовъ! Многіи побѣждаютъ много народа, и твердые берутъ грады, но себе побѣдить не могутъ, и дѣлаются плѣнники своихъ страстей. \textit{Лучше мужъ долготерпѣливъ паче крѣпкаго, и мужъ разумъ имѣяй паче земледѣльца великаго: удерживаяй же гнѣвъ паче вземлющаго градъ}\footnote{Притч.16,~32.}. Хрістіанине, потщимся побѣдить себе, и одержимъ преславную побѣду. \textit{Глаголетъ Духъ церквамъ: побѣждающему дамъ ясти отъ древа животнаго, еже есть посредѣ рая Божія}; и паки: \textit{побѣждающему дамъ ясти отъ манны сокровенныя}, и прочая\footnote{Апок.~2,~7 и 17.}. Твоя, Господи, побѣда, Твое торжество, Твоя и слава, яко похвала силы нашея Ты еси. \textit{Сладцѣ убо похвалюся паче въ немощехъ моихъ, да вселится въ мя сила Хрістова}\footnote{Кор.~12,~9.}. За велико люди почитаютъ здоровое, крѣпкое и доброцвѣтущее тѣло имѣть; но хрістіанамъ \textit{велико} есть здоровую, крѣпкую и благоцвѣтущую душу имѣть. Добро есть и тѣлесное здравіе. Но что пользуетъ тѣло здравое, а душу немощную и разслабленную имѣть? Какая польза человѣку "--- хрістіанину, что тѣло его добрѣ видитъ, но душа слѣпотствуетъ въ знаніи Бога и воли Его святой и себе самыя? Тѣло его слышитъ добрѣ, но душа глуха есть и не слышитъ слова Божія; тѣло право, но душа крива и погорблена; тѣло не хромаетъ, но душа хромаетъ; тѣло крѣпко и движется и ходитъ, но душа разслабленна и ни къ какому дѣлу доброму недвижима; тѣло не трясется лихорадкою и не жжется огневицею, но душа гнѣвомъ и яростію трясется и палима есть; тѣло не мучится насиліемъ бѣсовскимъ, но душа страстьми и грѣхами бѣснуется. Лютый бѣсъ есть страсть и грѣхъ. Что пользуетъ, что тѣло чисто есть, а душа прокаженна и вся отъ ранъ грѣховныхъ смердитъ; тѣло красно и доброобразно, но душа гнусна и безобразна? Тѣло бо есть смертно, но душа безсмертна. Тѣлесное здравіе до смерти намъ только служитъ, а по смерти разсыплется въ прахъ, какъ и немощное. Тутъ смотри, человѣче, красоты и благообразія тѣлеснаго! Воистину не узнаешь, здравое ли, или немощное тѣло было. Но здравіе души неотлучно отъ души бываетъ, и на оный вѣкъ съ нею отходитъ и спутствуетъ ей, и предъ Богомъ является съ нею, и благопріятна святымъ очамъ Его бываетъ. Богъ бо любитъ здоровую и святую душу. Немощь и разслабленіе тѣла смертію временною грозитъ; но немощь и разслабленіе души смертію вѣчною грозитъ. Рѣдко бываетъ, что въ здоровомъ тѣлѣ здоровая душа живетъ, и по большей части и почти всегда \textit{въ немощномъ}\footnote{2~Кор.~12,~10.}. Здравіе тѣла ко многимъ прихотямъ и грѣхамъ отворяетъ человѣку двери, но немощь тѣла затворяетъ. Конь свирѣпый и необученый бѣснуется, и часто стремится на свою пагубу; но уздою воздерживается, и біется, и мучится, и удручается, и тако кроткимъ бываетъ. Плоть наша безъ болѣзни и немощи, какъ конь, свирѣпѣетъ и на вся пагубныя страсти стремится; но немощію и болѣзнію, какъ уздою, воздерживается и укрощается, и покоряется духу. О коль великое милосердіе Богъ дѣлаетъ съ тѣмъ, на кого болѣзнь посылаетъ! Сокрушаетъ тѣло его, да душа въ здравіе свое пріидетъ; предаетъ во изможденіе плоть, \textit{да духъ спасется въ день Господа нашего Іисуса Хріста}\footnote{1~Кор.~5,~5.}. Знаютъ и признаютъ сію истину вси, находящіися въ болѣзняхъ, а паче безбрачно живущіи. Евангельскій богачъ здоровъ былъ, и облачался въ порфиру и виссонъ, веселяся на вся дни свѣтло; но по кончинѣ своей пошелъ въ пламень огненный, и по веселостяхъ временныхъ въ муку вѣчную. Лазарь нищъ былъ и гноенъ, и лежалъ предъ враты его; но по кончинѣ своей отнесенъ ангелами въ лоно Авраамле\footnote{Лук.~16,~19--25.}. Тако единъ по временномъ страданіи въ покой вѣчный, а другій по временныхъ веселостяхъ въ вѣчное отнюдь мученіе. Тоежде и нынѣ дѣлается: многіи и нынѣ воспріемлютъ благая своя въ животѣ своемъ, но тамо страждутъ; и многіи здѣ страждутъ, но тамо утѣшаются. Что убо пользуетъ тѣло здравое имѣть, но душу разслабленную и немощную? \textit{Велико} есть здравіе души имѣть, души безсмертныя и по образу Божію сотворенныя, хотя бы тѣло немощное и гнилое было. Подлинно вси мы желаемъ здравое тѣло имѣть, но не умѣемъ тѣмъ обладать. Лучше убо тѣломъ тлѣть и гнить, когда тое Богу угодно и намъ полезно и нужно, только бы душа свое здравіе получила. Слово Божіе и вѣра святая научаетъ насъ, что со здравою душею и тѣло соединится нѣкогда, и будетъ \textit{здравое, сильное, красное, духовное, нетлѣнное, безсмертное и благоцвѣтущее во вѣки. Аще и внѣшній нашъ человѣкъ тлѣетъ, но внутренній обновляется по вся дни}\footnote{1~Кор.~15,~42--44 и 53; Фил.~3,~21; 2~Кор.~4,~16.}. За велико люди почитаютъ имѣть богатство міра сего; но хрістіанамъ \textit{велико} есть имѣть богатство душевное, внутрь себе, то"=есть, вѣру живую, благодать Божію и плодъ духовный: \textit{любовь, радость, миръ, долготерпѣніе, благость, милосердіе, вѣру, кротость, воздержаніе}\footnote{Гал.~5,~22 и 23.}. Сіе сокровище души хрістіане за \textit{велико} почитаютъ. Богатство міра сего до смерти только служитъ человѣку, а при смерти отступаетъ отъ него, и отходитъ онъ на оный вѣкъ, яко единъ отъ нищихъ и убогихъ. \textit{Не бойся, егда разбогатѣетъ человѣкъ, или егда умножится слава дому его: яко внегда умрети ему, не возметъ вся, ниже снидетъ съ нимъ слава его}\footnote{Пс.~68,~17 и 18.}. Сокровище душевное и здѣ внутрь себе носитъ человѣкъ, и на оный вѣкъ съ нимъ отходитъ, и съ тѣмъ является небесному Отцу, и показуетъ свидѣтельство о себѣ, что онъ именемъ и вещію есть хрістіанинъ. А богачъ, который много вещественнаго богатства имѣетъ, но душевнаго не имѣетъ, съ чимъ на оный вѣкъ отыдетъ, и какое о себѣ свидѣтельство покажетъ, что онъ въ мірѣ семъ былъ хрістіанинъ? Развѣ тое, что много имѣлъ? Много имѣютъ и Турки и идолопоклонники. Или тое, что имя Божіе исповѣдывалъ? \textit{И бѣси вѣруютъ, и трепещутъ}\footnote{Іак.~2,~19.}. Что убо пользуетъ, сундуки веществомъ наполненные, но душу праздную имѣть? Нищъ тотъ богачъ, который внутрь сокровища душевнаго не имѣетъ, хотя и все внѣ себе имѣетъ. Ублажаютъ люди его, яко много видятъ у него, но предъ Богомъ онъ \textit{окаяненъ}, яко Богъ ничего не видитъ въ немъ, кромѣ нищеты\footnote{Апок.~3,~17.}. Хрістіанине! потщимся душевное сокровище имѣть, внутрь насъ, и будемъ истинно богати; признаемъ нашу нищету, да Самъ Богъ благодатію Своею обогатитъ насъ. \textit{Вѣсте благодать Господа нашего Іисуса Хріста, яко васъ ради обнища богатъ сый, да вы нищетою Его обогатитеся}\footnote{2~Кор.~8,~9.}. Чудная нищета, которая обогащаетъ насъ, чудное и богатство, которое отъ нищеты бываетъ! Хрістово сіе дѣло есть, Который \textit{богатъ сый насъ ради обнищалъ}; Тому все возможно. Но и тое писано есть: \textit{вся возможна вѣрующему}\footnote{Мѳ.~17,~20; Марк.~9,~23.}. За велико люди почитаютъ быть свободнымъ человѣкомъ, быть благороднымъ, никому не работать, но самому другимъ приказывать и повелѣвать; но у хрістіанъ \textit{велико} есть быть свободнымъ отъ грѣха, имѣть благородную душу, никакому грѣху не работать, но единому Богу вѣрою и правдою работать. То"=то прекрасная \textit{свобода}, и то"=то истинное \textit{благородіе!} Всякое внѣшнее благородіе и свобода до гроба только человѣку служитъ, а тутъ отъ него отступаетъ, и дѣлается онъ, какъ единъ отъ подлыхъ. Тутъ познать не возможно, гдѣ рабъ, гдѣ господинъ его лежитъ, и кто рабъ, кто господинъ былъ, не видно. А что, когда въ \textit{благородномъ} тѣлѣ подлая жила душа, и тотъ, которому люди служили, работалъ грѣху (\textit{всякъ бо творяй грѣхъ, рабъ есть грѣха})\footnote{Іоан.~8,~34.}: съ чимъ уже такому на всемірномъ ономъ позорищѣ явиться? Со грѣхомъ? Страшно сіе. Тамо не такъ, какъ здѣ бываетъ, будетъ: тамо не будетъ пріятія лицъ; благородніи и подліи, господа и раби рядомъ станутъ предъ Судіею. Не тако душевная свобода. Она и здѣ съ человѣкомъ, и отъ міра сего съ нимъ исходитъ, и сопутствуетъ ему, и клеветникамъ уста заграждаетъ, и предъ Богомъ является, и о немъ ходатайствуетъ. Вотъ коль \textit{велико есть "--- душевная свобода и благородіе!} И хотя хрістіане не безъ грѣха, однакожъ не лишаются своея свободы и благородія. Понеже, чувствуя и видя немощи свои въ себѣ, непрестанно воздыхаютъ къ небесному Отцу, и, призная грѣхи свои, согласно возносятъ гласъ свой къ Нему: \textit{Отче! остави намъ долги наша}; и получаютъ, чего просятъ, и тако свободы своея не лишаются. Видишь, хрістіанине, и внѣшнюю человѣческую свободу и внутреннюю, и міра сего благородіе и хрістіанское благородіе. Самъ убо разсуди, что пользуетъ внѣ быть свободнымъ, но внутрь быть подлымъ и рабомъ; тѣломъ отъ людей принимать служеніе, но душею работать грѣху. Благородныя души есть, стоять противу всякаго грѣха и противу того подвизаться и не попущать ему обладать собою. Лучше работать человѣку, нежели грѣху и грѣхомъ діаволу. \textit{Аще Сынъ васъ свободитъ, воистину свободни будете}\footnote{Іоан.~8,~36.}, глаголетъ Господь"=Свободитель нашъ. То"=то преславная свобода, истинное благородіе наше, достоинство наше, честь наша, слава наша, красота наша, доброта наша, и вѣчное наше блаженство! \textit{Возврати Господи плѣненіе наше, яко потоки югомъ. Сѣющіи слезами, радостію пожнутъ; ходящіи хождаху и плакахуся, метающе сѣмена своя; грядуще же пріидутъ радостію, вземлюще рукояти своя}\footnote{Пс.~125,~4--6.}. \textit{Помилуй мя Боже по велицѣй милости Твоей, и по множеству щедротъ Твоихъ очисти беззаконіе мое}, и проч.\footnote{50,~3.} Сія и прочая люди за велико почитаютъ; но у хрістіанъ едино тое только \textit{велико}, что невидимое, духовное, божественное и вѣчное. Кто что за велико почитаетъ, того со усердіемъ желаетъ и ищетъ. Вѣра едина есть руководительница къ невидимымъ, духовнымъ и небеснымъ вещамъ. \textit{Вѣра бо есть уповаемыхъ извѣщеніе, вещей обличеніе невидимыхъ}\footnote{Евр.~11,~1.}. Она невидимая аки видимая, и будущая аки настоящая представляетъ. Потому вѣрная душа, живу вѣру имѣющая, тая вся видитъ и къ тѣмъ стремится, мало что, или ничего не промышляя о настоящихъ. Но кто таковой вѣры не имѣетъ, тотъ тое только за велико почитаетъ и желаетъ и ищетъ, что видитъ. Хрістіанине, потщимся свѣтильникъ вѣры въ сердцѣ нашемъ имѣть, и той намъ все невидимое, духовное, божественное и вѣчное будетъ показывать, и тогда будемъ хрістіанами. Отъ вѣры бо живой все существо хрістіанскаго блаженства зависитъ.

\section{122. Не бойся, я съ тобою.}

Бываетъ, что мать, видя свое дитя скорбящее и плачущееся, утѣшаетъ тое, и говоритъ ему: \textit{не бойся, я съ тобою}. Тако милосердый и человѣколюбивый Богъ, Иже есть Создатель и Отецъ щедротъ и Богъ всякія утѣхи, вѣрной душѣ, находящейся во искушеніяхъ и напастяхъ, скорбящей и сѣтующей и боящейся, глаголетъ: \textit{не бойся, Я съ тобою}; Я твой Создатель, Я твой Искупитель, Я твой Спаситель, Я твой Помощникъ и Заступникъ, Я, Который въ руцѣ Своей все содержу и Которому вся повинуются, Я Той съ тобою. \textit{Рече Сіонъ: остави мя Господь, и Богъ забы мя. Еда забудетъ жена отроча свое, еже не помиловати исчадія чрева своего? Аще же и забудетъ сихъ жена, но Азъ не забуду тебе, глаголетъ Господь}\footnote{Ис.~49,~14 и 15.}. \textit{Аще преходиши сквозѣ воду, съ тобою есмь, и рѣки не покрыютъ тебе: и аще сквозѣ огнь пройдеши, не сожжешися, и пламень не опалитъ тебе. Яко Азъ Господь Богъ твой, Святый Израилевъ спасаяй тя}\footnote{43,~2 и 3.}. Тако былъ Господь съ вѣрнымъ Своимъ Ноемъ, и сохранилъ его отъ всемірнаго потопа. Былъ съ вѣрнымъ Своимъ Лотомъ, и сохранилъ праведника отъ казни содомской. Былъ съ рабами Своими Авраамомъ, Исаакомъ и Іаковомъ, и сохранилъ ихъ на земли пришельствія ихъ, яко писано о нихъ: \textit{не остави человѣка обидѣти ихъ, и обличи о нихъ цари: не прикасайтеся помазаннымъ Моимъ, и во пророцѣхъ Моихъ не лукавнуйте}\footnote{Пс.~104,~14--15.}. Былъ со Іосифомъ, и въ нашедшихъ ему искушеніяхъ и страданіяхъ сохранилъ его и прославилъ его. Былъ со Израилемъ во Египтѣ, откуду и глаголалъ: \textit{видя видѣхъ озлобленіе людей Моихъ, иже во Египтѣ, и вопль ихъ услышахъ отъ дѣлъ приставниковъ}\footnote{Исх.~3,~7.}. Былъ съ тѣмъ же Израилемъ во исходѣ его отъ Египта, и раздѣлилъ ему Чермное море, и сотворилъ ему путь посредѣ водъ, и провелъ его сквозѣ Чермное море, и спаслъ его отъ Фараона мучителя; откуду и воспѣлъ благодарственную пѣснь Спасителю своему Богу. Былъ съ тѣмъ же Израилемъ въ пустыни, и питалъ Его чудесною манною, и поражалъ предъ лицемъ его враговъ его, и ввелъ его въ гору святыни Своея. Былъ съ тѣмъ же Израилемъ, живущимъ въ землѣ обѣтованной, и спаслъ ихъ, якоже поетъ пророкъ: \textit{яко аще не Господь бы былъ въ насъ, да речетъ убо Израиль: яко аще не Господь бы былъ въ насъ, внегда востати человѣкомъ на ны, убо живыхъ пожерли быша насъ}, и прочая\footnote{Пс.~123,~1 и 2.}. Былъ съ Давидомъ, помазанникомъ Своимъ, въ различныхъ искушеніяхъ и гоненіяхъ, и сохранялъ раба Своего отъ враговъ его. Былъ со Іоною во глубинѣ морской, и сохранилъ его во чревѣ китовѣ, и избавилъ его отъ звѣря морскаго. Былъ съ тремя отроками въ пещи вавилонской, и угасилъ имъ силу огненную, и научилъ ихъ пѣти благодарственную пѣснь. Былъ съ Даніиломъ въ ровѣ, и заградилъ уста львовъ, и избавилъ его оттуду. Былъ съ апостолами, былъ съ мучениками, и сохранилъ ихъ посредѣ ужасныхъ мученій; былъ съ пустынниками, жившими въ пещерахъ, въ вертепахъ и пропастяхъ земныхъ, и сохранилъ ихъ отъ козней вражіихъ; былъ и есть и будетъ до скончанія вѣка съ вѣрными рабами Своими, по неложному Своему обѣщанію: \textit{се Азъ съ вами есмь во вся дни до скончанія вѣка, аминь}\footnote{Мѳ.~28,~20.}. \textit{О Божественнаго, о любезнаго, о сладчайшаго Твоего гласа! Съ нами бо неложно обѣщался еси быти до скончанія вѣка, Хрісте. Егоже вѣрніи утвержденіе надежды имуще радуемся}. Откуду вѣрніи раби Его, чувствуя при себѣ присутствіе Его и живое утѣшеніе, дерзаютъ, восклицаютъ и поютъ пѣснь въ радости духа: \textit{Богъ намъ прибѣжище и сила, помощникъ въ скорбехъ, обрѣтшихъ ны зѣло. Сего ради не убоимся, внегда смущается земля и прелагаются горы въ сердца морская}, и проч. \textit{Господь силъ съ нами, заступникъ нашъ Богъ Іаковль}\footnote{Пс.~45,~2,~3 и 12.}. И паки: \textit{аще и пойду посредѣ сѣни смертныя, не убоюся зла, яко Ты со мною еси}, Боже\footnote{22,~4.}. И паки: \textit{Господь просвѣщеніе мое и Спаситель мой, кого убоюся? Господь защититель живота моего, отъ кого устрашуся?} и проч.\footnote{Пс.~26,~1.} Смотри, хрістіанине, буди только Божій, а Богъ \textit{своего не оставитъ}\footnote{33,~23; 28,~11; 31,~11.}. Вѣруй сердечно Ему, яко Богу; угождай Ему вѣрою и правдою; всю надежду твою полагай на Него, и отъ сердца призывай Его; а Онъ близъ тебе есть, съ тобою есть сохраняяй тя Святый Израилевъ; и гдѣ ни будеши, въ какой скорби и искушеніи ни находишися, съ тобою есть, и смотритъ на подвигъ твой, и невидимою рукою укрѣпляетъ тебе и помогаетъ тебѣ; и хотя вси злыи люди востанутъ на тебе, и бѣсовскіе полки обыдутъ тебе, ничего не успѣютъ. \textit{Господь сохранитъ тя, Господь покровъ твой на руку десную твою. Во дни солнце не ожжетъ тебе, ниже луна нощію. Господь сохранитъ тя отъ всякаго зла, сохранитъ душу твою Господь. Господь сохранитъ вхожденіе твое и исхожденіе твое, отъ нынѣ и до вѣка}\footnote{120,~5--8.}. \textit{Не остави мене, Господи Боже мой, не отступи отъ мене; вонми въ помощь мою, Господи спасенія моего}\footnote{37,~22 и 23.}. \textit{Господи силъ, съ нами буди, иного бо, развѣ Тебе, помощника въ скорбехъ не имамы. Господи силъ, помилуй насъ}.

\section{123. Отрыжка.}

Что отрыжка желудка, тое отъ сердца происходящее слово и дѣло. Отрыжка зачнется прежде въ желудкѣ и исходитъ вонъ изъ устъ: такъ, что внутрь сердца у человѣка имѣется, добро или зло, внѣ, чрезъ слово или дѣло оказывается. Когда въ сердцѣ будетъ молитва и избыточествуетъ, то непремѣнно окажетъ себе или возведеніемъ очесъ на небо, или воздѣяніемъ рукъ, или словомъ, сходнымъ сердечной молитвѣ. Сердечная бо молитва сыщетъ слова, приличныя себѣ. Тогда воззоветъ человѣкъ: \textit{о Господи, помилуй! о Господи, услыши! о Господи, пощади!} или иное что, подобное симъ. Когда въ сердцѣ чувствуется Божія милость и благодѣяніе и за то благодарность, то исходитъ такожде чрезъ слова, приличныя тому. Тутъ человѣкъ восклицаетъ и радостнымъ духомъ и устнами радости взываетъ Богу благодателю: \textit{слава Тебѣ Боже!} или: \textit{благословенъ Богъ!} или: \textit{благослови, душе моя, Господа!} или подобная симъ. Когда въ сердцѣ чувствуется печаль за грѣхи, оказываетъ себе воздыханіемъ или слезами. Воздыханіе и слезы, отъ сокрушеннаго сердца происходящіи, суть слова сердечныя молитвы. Тогда человѣкъ не языкомъ и устами, но сердцемъ молится и вопіетъ къ Богу: \textit{моленіе мое внуши и слезъ моихъ не премолчи}, и проч.\footnote{Пс.~38,~13.} Когда любовь Божія въ сердцѣ чувствуется и сладость тоя ощущается, тогда съ пророкомъ воззоветъ къ Богу: \textit{возлюблю Тя, Господи, крѣпосте моя}, и проч.\footnote{17,~2.} Или: \textit{щедръ и милостивъ Господь, долготерпѣливъ и многомилостивъ}, и проч.\footnote{102,~8.} Или другое что, подобное тому. Любовь бо отрыгнетъ знаки своя. Когда смиреніе имѣется на сердцѣ и чувствуется, тогда человѣкъ уничтожаетъ себе, и признаетъ себе всего недостойнымъ, и вездѣ ищетъ послѣднѣйшаго мѣста. Когда въ сердцѣ человѣческомъ имѣется любовь и милосердіе къ ближнему, то оказываетъ себе радостію о благополучіи ближняго, и соболѣзнованіемъ и состраданіемъ бѣдствію его. Таковая душа не оставитъ безъ помощи ближняго своего; и когда не можетъ помощи, къ Богу, Помощнику всѣхъ, воздохнетъ, о немъ, и проч. Отсюду видишь, хрістіанине, что есть истинная молитва, благодареніе и истинное хрістіанское благочестіе, то есть, когда оно въ сердцѣ имѣется, и отъ сердца происходитъ. А чего въ сердцѣ нѣтъ, того и въ самой вещи нѣтъ. Такожде и какое зло въ сердцѣ у человѣка имѣется, аки отрыжка отъ желудка, чрезъ слово или дѣло вонъ исходитъ и оказываетъ себе. Гнѣвъ оказываетъ себе яростію, крикомъ, хуленіемъ, проклинаніемъ, плесканіемъ рукъ, терзаніемъ власъ и проч. Злоба или памятозлобіе оказываетъ себе отмщеніемъ чрезъ слово или дѣло, поврежденіемъ, безславіемъ, клеветою и убійствомъ. Сіи суть плоды гнѣва и злобы. Блудная похоть оказываетъ себе скверными помыслами, сквернымъ мечтаніемъ, и является чрезъ похотливое воззрѣніе, негодныя шутки, осязаніе, страстныя слова, и самое скверное дѣло, и проч. Сребролюбіе оказываетъ себе чрезъ проискиваніе всякимъ образомъ богатства, чрезъ скупость и храненіе того, чрезъ воровство, хищеніе, грабленіе, насиліе, отъятіе чуждаго добра, и прочая злая. Ложь, лицемѣріе, хитрость, лукавство оказываютъ себе чрезъ слово, несогласное совѣсти и мысли, и прочіе знаки, внутренности противные. Гордость оказываетъ себе проискиваніемъ чести и славы міра сего, и приготовленіемъ богатыхъ домовъ, красныхъ одеждъ, дорогихъ каретъ и коней и прочія пышности міра сего. Гордость бо весьма не терпитъ въ презрѣніи и незнаемости быть. Она вездѣ и во всемъ ищетъ себе показать, и предъ людьми нѣчто быть. Высокоуміе оказываетъ себе выставленіемъ своихъ добрыхъ дѣлъ, презрѣніемъ и уничтоженіемъ другихъ и осужденіемъ ихъ, и себе самаго возношеніемъ паче прочіихъ. Оно фарисейскія слова хотя внѣ и не произноситъ, однакожъ внутрь сердца своего говоритъ: \textit{нѣсмь, якоже прочіи человѣцы}\footnote{Лук.~18,~11.}. О когда бы сіи слова въ противномъ и сердцемъ думалъ, и устами произносилъ: \textit{нѣсмь якоже прочіи}, то"=есть, вси лучшіи мене, и вопилъ бы къ Богу: \textit{Боже милостивъ буди мнѣ грѣшному}\footnote{13.}, "--- подлинно бы многихъ лучшій былъ, и былъ бы праведенъ не своею, но Божіею правдою, грѣшники оправдающею! Многіи думаютъ и говорятъ, что на Бога надежду имѣютъ; но отъ случая, какъ отъ отрыжки, познаются, на кого они надѣются. Искушеніе показуетъ, на кого мы надѣемся, на Бога, или на человѣка, или на иное созданіе. Кто въ нуждѣ своей къ кому прибѣгаетъ и помощи ищетъ, на того и надѣется. Къ человѣку ли прибѣгаешь и помощи ищешь отъ него въ какой бѣдѣ, "--- на человѣка, а не на Бога надѣешися. Чрезъ богатство и дары отъ напасти избавляешися, "--- на богатство, а не на Бога, надѣешися. Чрезъ разумъ и хитрость и коварство свое тщишися избавитися, "--- на себе, а не на Бога надѣешися. Чрезъ санъ и достоинство свое хощешь избавитися, "--- на честь и санъ свой надѣешися, а не на Бога, и проч. Но кто на Бога истинно надѣется, тотъ все, что кромѣ Бога есть, оставивши, къ единому Богу во всѣхъ случаяхъ и приключеніяхъ прибѣгаетъ и помощи проситъ, якоже изъ псалмовъ видимъ. Таковая душа во всякихъ нуждахъ и приключеніяхъ сердечно съ пророкомъ глаголетъ: \textit{Господь мнѣ помощникъ, и не убоюся, что сотворитъ мнѣ человѣкъ; Господь мнѣ помощникъ, и азъ воззрю на враги моя}, и проч.\footnote{Пс.~117,~6 и 7.} \textit{На Господа уповахъ, како речете души моей: превитай по горамъ яко птица}\footnote{10,~1.}. Почему и Господь глаголетъ о ней: \textit{яко на Мя упова, и избавлю его}, и проч.\footnote{Пс.~90,~14.} Тако, какая надежда крыется въ сердцѣ у человѣка, случай и искушеніе отрыгаетъ и показуетъ. Многіи мечтаютъ, что они любятъ Бога и ближняго; но понеже въ сердцѣ любви не имѣютъ, обманываются въ томъ. Дѣла ихъ показываютъ, что они прихоти своя и самихъ себе любятъ, а не Бога и ближняго своего. Велико есть любовь, и нѣтъ ничего болѣе паче любве. \textit{Всѣ дарованія безъ любви ничтоже суть}\footnote{1~Кор.~13,~1--3.}. Любовь въ сердцѣ должна быть, а не въ словахъ, и познается отъ дѣлъ, а не отъ словъ. Непрестанное хотѣніе и желаніе Божія славы, и исполненіе воли Его, сколько человѣку возможно въ мірѣ семъ, есть знаменіе сердечныя любве. Многіи думаютъ, что они кротки и терпѣливы суть; но случай и искушеніе, какъ отрыжка отъ желудка, показуетъ противное, и сами тогда узнаютъ, что нѣтъ въ нихъ кротости и терпѣнія. Многіи думаютъ, что они по вся дни и часто молятся, яко много поклоновъ творятъ и много молитвъ написанныхъ читаютъ; но понеже безъ разума и внутренности дѣлаютъ тое, то или мало когда, или никогда не молятся. Все бо внѣшнее безъ внутренняго ничто. Всякое убо добро и благочестіе должно быть прежде въ сердцѣ, и отъ сердца или отъ внутренности происходить, какъ отрыжка отъ желудка происходитъ. Пророкъ святый глаголетъ: \textit{отрыгну сердце мое слово благо}\footnote{Пс.~44,~2.}. Тако и ты, хрістіанине, дѣлай. Молишися ли, сердце да отрыгаетъ молитву твою. Благодариши ли и поеши Богу, сердце да отрыгаетъ благодареніе и пѣніе твое. Смиряешися ли, сердце да отрыгаетъ смиреніе твое. Милуеши ли ближняго твоего, сердце да отрыгаетъ милость твою. Тихо и ласково говориши ближнему твоему, сердце да отрыгаетъ тихое и ласковое слово твое. Объемлеши и лобзаеши ближняго твоего, сердце да отрыгаетъ объятіе и лобзаніе твое. Въ домъ пріемлеши и угощаеши ближняго твоего, сердце да отрыгаетъ пріятіе и угощеніе твое. Главу и колѣна преклоняеши Господу Богу твоему, сердце да отрыгаетъ преклоненіе и поклоны твоя. Руки воздѣваешь и очи возводишь въ Живущему на небеси, сердце да отрыгаетъ воздѣяніе рукъ и возведеніе очесъ. Поешь со пророкомъ Богу: \textit{возлюблю Тя, Господи}, и проч., "--- сердце да отрыгаетъ слово сіе. Глаголеши Богу: \textit{на Тя, Господи, уповахъ, да не постыжуся во вѣкъ}, и проч.\footnote{Пс.~30,~2.}, "--- сердце да отрыгаетъ слово сіе и упованіе твое. Каешися Господу твоему и исповѣдуеши грѣхъ твой, глаголя: \textit{согрѣшилъ, Господи}, и проч., "--- сердце да отрыгаетъ покаяніе и исповѣданіе твое. Исповѣдуеши и признаеши предъ Богомъ бѣдность, окаянство, нищету, убожество и ничтожество свое, сердце да отрыгаетъ исповѣданіе и признаніе твое. Сего ради внимай, хрістіанине, слѣдующимъ пунктамъ: 1)~Люди чего не знаютъ, тому учатся, и чего не имѣютъ, того ищутъ. Не знаютъ художествъ и наукъ, того ради входятъ въ академіи и школы, и тамо обучаются, и научаются, и бываютъ разумными: тако и намъ должно дѣлать въ хрістіанскомъ званіи. Не знаемъ молитися, должно учиться тому. Откуду и апостоли ко Господу молились: \textit{Господи, научи насъ молитися}\footnote{Лук.~11,~1.}. Изрядно научаютъ насъ молитися и славословити Бога каноны, о покаяніи, о страстяхъ Хрістовыхъ, о воскресеніи Хрістовомъ сложенніи, и прочіи въ праздники чтомые, и прочія церковныя молитвы; научаютъ и подвигаютъ къ истинной молитвѣ и славословію, но съ разсужденіемъ и разумомъ чтомыи, а безъ того ничего не пользуютъ. Не имѣемъ любви, смиренія, кротости и терпѣнія, "--- должно ихъ искать, и обучаться; и тако чего ищемъ, тое и обрящемъ, и чего не знаемъ, тому научимся. 2)~Какъ къ молитвѣ, такъ и ко всякому добру духовному, то"=есть, любви, смиренію, кротости, терпѣнію и прочимъ хрістіанскимъ добродѣтелемъ должно себе нудить и убѣждать, хотя и не хощетъ и отвращается сердце. Таковый трудъ и тщаніе наше видя, Господь подастъ намъ охоту и усердіе и истинную молитву, и прочее хрістіанское добро; \textit{подастъ} Тотъ, Который можетъ подать и обѣщалъ\footnote{Мѳ.~7,~11; 21,~22; Іоан.~16,~23.}. 3)~Познаніе бѣдности научаетъ человѣка избавляться отъ бѣдности. Недостатокъ хлѣба учитъ искать хлѣба, чтобы отъ голода не умереть; недостатокъ питія убѣждаетъ искать питія, чтобы отъ жажды не умереть; недостатокъ одѣянія понуждаетъ искать одѣянія, чтобы нагому не ходить; немощь познанная убѣждаетъ искать лѣкаря, и прочая. Тако имѣется и въ хрістіанскомъ дѣлѣ. Когда познаемъ скудость и бѣдность душъ нашихъ, будемъ искать блаженства нашего. Худо быть хрістіанину безъ молитвы, безъ любви, безъ смиренія, безъ кротости и прочіихъ хрістіанскихъ добродѣтелей, и не имѣть ихъ; явная оттуду бѣда слѣдуетъ. Надобно убо искать ихъ съ прилѣжаніемъ. Тако познанное бѣдствіе убѣждаетъ человѣка блаженства своего искать. Познай убо, хрістіанине, растлѣніе, бѣдность, окаянство, нищету и убожество сердца твоего, и сіе самое познаніе научитъ тебе молитвѣ и хрістіанскимъ добродѣтелямъ. 4)~Изрядное училище молитвы и благочестія есть скорбь и страданіе. Израильтяне, будучи во Египтѣ, и отъ приставниковъ Фараона мучителя озлобленіе терпя, прилѣжно молилися и вопили къ Богу, якоже Самъ Господь глаголалъ о нихъ: \textit{видя видѣхъ озлобленіе людей Моихъ, и вопль ихъ услышахъ}\footnote{Исх.~3,~7.}. Анна, мати Самуила Пророка, будучи въ поношеніи ради неплодствія и тѣснотѣ и скорби, сердечно молилася Богу, и услышана отъ Бога\footnote{1~Цар. гл. 1.}. Како Давидъ царь въ скорбехъ и гоненіяхъ своихъ возбуждался къ сердечной молитвѣ, "--- псалмы его свидѣтельствуютъ. Пророкъ Іона, во чревѣ китовѣ, яко во адѣ, изъ глубины сердца вопилъ ко Господу, якоже самъ глаголетъ: \textit{возопихъ въ скорби моей ко Господу Богу моему}\footnote{Іоан.~2,~3.}. Сосанна въ Вавилонѣ, неправедно отъ беззаконныхъ старцевъ оклеветанна, и на смерть осужденна, въ сей тѣснотѣ и скорби возстенала ко Господу, и услышана отъ Господа\footnote{Дан.~13,~42--44.}. Тако скорбь научаетъ сердечной молитвѣ. Когда бо усерднѣе молимся, какъ во время болѣзни, бѣды, напасти, искушенія, въ нашествіи иноплеменниковъ, во время голода, моровой заразы и прочаго бѣдствія, которое грозитъ намъ смертію? Тогда изъ глубины сердца исходитъ и востаетъ молитва. Такожде и хрістіанскихъ добродѣтелей нигдѣ лучше не научаемся, какъ въ крестѣ скорби и страданіи, и искушеніи. Воинъ не тотъ искусенъ, который много научается регулѣ воинской внутрь отечества, но тотъ, который довольно въ сраженіяхъ противу непріятеля бывалъ: тако и хрістіанинъ тотъ искусенъ бываетъ въ званіи хрістіанскомъ, который сквозь огнь и воду искушеній, бѣдъ, напастей и скорбей проходитъ, и подвизается противу невидимыхъ враговъ. И сія"=то есть между прочими причина, чего ради Богъ попущаетъ на рабовъ Своихъ скорбь и страданіе, то"=есть, чтобы они училися истинной и сердечной молитвѣ, и къ Нему прибѣгали, и помощи искали, и хрістіанскимъ пріобучались добродѣтелямъ. \textit{Скорбь терпѣніе содѣловаетъ, терпѣніе же искусство, искусство же упованіе, упованіе же не посрамитъ}\footnote{Рим.~5,~3--5.}. Безъ соли мясо и рыба гніетъ: тако безъ скорби хрістіанинъ портится. Соль выгоняетъ червей изъ мяса и рыбы: тако скорбь изгоняетъ растлѣніе и страсти отъ души. Соль хранитъ цѣлость вещей: тако скорбь соблюдаетъ цѣлость души. Горька соль, но здорова тѣлу: горька и скорбь, но здорова душѣ. \textit{Блаженъ человѣкъ, егоже аще накажеши, Господи}\footnote{Пс.~93,~12.}. \textit{Егоже любитъ Господь, наказуетъ; біетъ же всякаго сына, егоже пріемлетъ}\footnote{Евр.~12,~6.}. 5)~Ученики въ школахъ смотрятъ на регулы и примѣръ, отъ учителей показанный, и тако научаются художествъ и наукъ. "--- Хрістіанине, намъ регула есть Евангеліе святое, и житіе и нравы Спасителя нашего Іисуса Хріста суть живый примѣръ. На сіе священное правило и живый примѣръ добродѣтелей должно намъ смотрѣть и учиться художеству хрістіанскому, то"=есть, добродѣтельному житію. Все, что ни видимъ въ насъ противное Евангелію и житію Хрістову, есть порокъ. Евангеліе и житіе Хрістово есть свѣтъ. Все убо, что свѣту сему противно есть въ насъ, есть тьма. Якоже убо представляемъ зеркало предъ лицемъ нашимъ, и, изъ того усмотрѣвъ пороки на лицѣ нашемъ, стираемъ ихъ: тако должно намъ представлять души наша предъ зеркаломъ Евангелія и житія Хрістова, и все тому противное, яко порокъ, покаяніемъ и сокрушеніемъ сердца очищать. \textit{Азъ есмь свѣтъ міру: ходяй по Мнѣ, не имать ходити во тьмѣ, но имать свѣтъ животный}, глаголетъ Господь\footnote{Іоан.~8,~12.}. Приступимъ убо къ дивному свѣту сему, и пойдемъ въ слѣдъ его, да не пребудемъ во тьмѣ. 6)~Противу всякаго зла, какое изъ сердца ни востаетъ, стоять и подвизаться, и не допущать тому въ дѣло происходить, и внутрь сердца угнетать тое, и мечемъ глагола Божія отсѣкать. Противу злобы и памятозлобія помнить Хрістово слово: \textit{оставите, и оставится вамъ; не оставляете вы, не оставится и вамъ}\footnote{Мѳ.~6,~14 и 15.}. Противу сребролюбія и гордости и пышности міра сего полагать апостольское слово: \textit{ничтоже внесохомъ въ міръ сей, явѣ, яко ниже изнести что можемъ; имѣюще же пищу и одѣяніе, сими довольни будемъ}\footnote{1~Тим.~6,~7 и 8.}. И Хрістосъ глаголетъ: \textit{не можете Богу работати и мамонѣ}\footnote{Мѳ.~6,~24.}. Блудную похоть пресѣкать тѣмъ, что за сей грѣхъ сильно мучитъ и терзаетъ совѣсть. Высокоумію довольная узда есть память прежнихъ грѣховъ, и разсужденіе своея бѣдности и окаянства, и что можетъ человѣкъ быть мерзостнѣйшимъ и сквернѣйшимъ грѣшникомъ паче прочихъ, когда Богъ руку Свою отниметъ отъ него, какъ довольно такихъ примѣровъ видимъ. \textit{Не высокомудрствуй, но бойся}\footnote{Римл.~11,~20.}. Тако и въ прочихъ грѣхахъ поступать и противу ихъ подвизаться. Вездѣ и всегда полагать предъ собою присутствіе Божіе и всевѣдѣніе Его и правду Его. Богъ вездѣ невидимо присутствуетъ и все видитъ, и можетъ грѣшника въ самомъ грѣхѣ поразить. Бойся убо грѣшить не токмо словомъ и дѣломъ, но и мыслію, яко Богъ и мысли видитъ, да не судъ Божій постигнетъ тебе. Такожде прилично ли хрістіанину тѣ грѣхи творить, за которые Хрістосъ Господь горькую страданія чашу испилъ? Всякій грѣхъ заключаетъ дверь къ вѣчному животу: \textit{оброцы бо грѣха смерть}\footnote{Римл.~6,~23.}. Берегись убо всякаго грѣха, да не погибнеши. Хрістіанинъ безъ доброй совѣсти быть не можетъ. Лучше убо хрістіанину умереть, нежели согрѣшить и совѣсть обезпокоить и раздражить. Сей подвигъ противу грѣха всѣмъ хрістіанамъ нуженъ, которыи ни хотятъ спастися. \textit{Убивай} убо, хрістіанине, беззаконнаго \textit{младенца}, пока малъ есть, да не возрастетъ и убіетъ тебе\footnote{Пс.~136,~9.}. Убивай похоть, да не дѣломъ исполнится; убивай малый гнѣвъ, да не въ ярость и злобу обратится; убивай блудную мысль, да не болѣе умножится и осквернитъ тебе; убивай желаніе богатства, да не рабомъ мамонѣ будеши; убивай малое киченіе, да не вознесешися, и, вознесшися, смиришися; убивай всякое зло, пока мало есть, да не возрастетъ и погубитъ тебе. Труденъ сей подвигъ, подлинно труденъ, но нуженъ. Подвизайся убо, да и тебѣ подастся отъ подвигоположника Хріста вѣнецъ живота. 7)~Всякое наше тщаніе и подвигъ, какъ о добрѣ, такъ и противу грѣха, не силенъ есть безъ помощи Божіей, понеже мы весьма растлѣнны и немощны. Почему и сказано намъ отъ Спасителя нашего: \textit{безъ Мене не можете творити ничесоже}\footnote{Іоан.~15,~3.}. Сего ради отъ Него просить и искать должно всего добра и крѣпости и силы противу грѣха. Книги святыя научаютъ насъ знать, и что творить и отъ чего уклоняться; но Хрістосъ Господь просвѣщаетъ разумъ нашъ, и подаетъ намъ силу и крѣпость къ хотѣнію и творенію, якоже лоза сокъ свой сообщаетъ розгамъ. Должно убо ко Хрісту прилѣпляться и къ Нему воздыхать, и отъ Него силы и крѣпости во всемъ просить и ожидать. \textit{Вѣренъ бо есть обѣщавый}, и хощетъ помощи тѣмъ, ради которыхъ въ міръ пришелъ и пострадалъ\footnote{Евр.~10,~23.}.

\section{124. Лодка или судно на рѣкѣ.}

Что лодка на рѣкѣ, тое человѣкъ въ житіи. Видимъ, что лодка сама собою внизъ по рѣкѣ плыветъ, а противу рѣки, или въ верхъ рѣки, никакъ плыть не можетъ; но, когда надобно ей итить вверхъ рѣки, нужны гребцы сильніи, или парусъ со способнымъ вѣтромъ, подвигающій и женущій ее. Тако имѣется и человѣкъ. По плоти, по прихотямъ и страстямъ и по злой волѣ своей, яко растлѣнный, самъ собою и удобно и легко живетъ, какъ судно внизъ рѣки само собою плыветъ. \textit{Зане прилѣжитъ помышленіе человѣку прилѣжно на злая отъ юности его}\footnote{Быт.~8,~21.}. \textit{Отъ сердца бо исходятъ помышленія злая, убійства, прелюбодѣянія, любодѣянія, татьбы, лжесвидѣтельства, хулы}\footnote{Мѳ.~15,~19.}. Страсти и прихоти наши съ нами родятся, и потому имъ послѣдовать и волю ихъ творить легко намъ и угодно, какъ въ суднѣ по рѣкѣ плыть. Но противу злой своей воли жить, и тую побѣждать и волѣ Божіей покорять, и противу прихотей и страстей стоять и подвизаться, и тыя распинать и умерщвлять, и тако благочестно и по"=хрістіански жить намъ самимъ такъ неудобно и невозможно, какъ судну самому безъ гребцовъ и парусовъ противу быстрины рѣчной плыть. Отсюду видимъ, сколько написано книгъ, которыи стремленіе страстей воспящаютъ; сколько проповѣдники трудятся и противу страстныя плоти гремятъ, и судомъ Божіимъ человѣка устрашаютъ, и страхомъ то временныя, то вѣчныя казни смущаютъ и удерживаютъ; и самъ человѣкъ, слыша слово, ужасается и смущается, и кается часто; но однакожъ по воли страстей и прихотей внизъ низходитъ, какъ судно по рѣкѣ стремится. Такое"=то растлѣнное естество имѣемъ мы, хрістіанине, въ такую подлость и окаянство упалъ человѣкъ чрезъ грѣхъ, въ такое безчестіе предивная оная доброта "--- Божій образъ хитростію діавольскою пришелъ! Что жъ убо намъ дѣлать? По прихотямъ и страстямъ жить хрістіанству и вѣрѣ святой и слову Божію противно и явная бѣда и пагуба; противу стремленія страстей стоять и подвизаться и ихъ усмирять и побѣждать, сами собою, не можемъ. Надобно убо неотмѣнно вышеестественной силѣ намъ помогать, и насъ, яко судно противу рѣки, подвигать, и на всякое время возбуждать и укрѣплять, и противу стремленія страстей возбуждать, и тыя побѣждать, и сердце наше къ воздыханію и сердечной молитвѣ и прочіимъ хрістіанскимъ добродѣтелямъ подвигать. Все сіе дѣйствуетъ Божія благодать, живущая въ человѣкѣ. Съ нею все можетъ человѣкъ, безъ ней ничего не можетъ. Божія благодать подвигаетъ человѣка и увѣщаваетъ на всякое время, и помогаетъ ему противу страстей стоять, и тыя побѣждать, и по"=хрістіански жить, якоже гребцы подвигаютъ судно противу быстрины рѣчной. На всякое убо время, часъ и минуту требуемъ Божіей благодати. Божія благодать есть животъ душъ нашихъ; безъ благодати Божіей душа жива быть не можетъ. Сего ради повелѣно намъ молитися, \textit{просить, искать и толкать}\footnote{Мѳ.~7,~7.}. Познаемъ убо, хрістіанине, растлѣніе и бѣдность нашу (въ низъ идемъ, а не въ верхъ, сами чрезъ себе), и, познавше, смиримся предъ Господемъ, да подастъ намъ благодать Свою. \textit{Зане Богъ гордымъ противится, смиреннымъ же даетъ благодать}\footnote{1~Петр.~5,~5.}. Видишь нищету свою тѣлесную, и почто не познаешь душевной нищеты? Ищешь богатства тлѣннаго, и почто не ищешь нетлѣннаго и душевнаго? Познаешь немощь тѣлесную, и исцѣляешь, и почто не познаешь немощи душевныя, и не исцѣляешь? Убѣгаешь отъ бѣды и смерти временной, и почто не убѣгаешь отъ бѣды и смерти душевныя, которая къ вѣчной смерти ведетъ? Стараешися быть или искуснымъ философомъ, или искуснымъ стихотворцемъ, или краснорѣчивымъ риторомъ, или мудрымъ звѣздочетцемъ, или искуснымъ землемѣромъ, или добрымъ архитектономъ, или искуснымъ купцемъ, или инымъ какимъ художникомъ, и почто не стараешися быть истиннымъ и добрымъ хрістіаниномъ? Сіе всякое художество и искусство и мудрость вѣка сего превосходитъ; безъ того все ничто. Надобно стараться намъ о томъ наипаче, что съ нами во вѣки пребываетъ. \textit{Сердце чисто созижди во мнѣ, Боже, и духъ правъ обнови во утробѣ моей. Не отвержи мене отъ лица Твоего, и Духа Твоего Святаго не отъими отъ мене. Воздаждь ми радость спасенія Твоего, и Духомъ владычнимъ утверди мя}\footnote{Пс.~50,~12--14.}.

\section{125. Познаніе бѣды или неблагополучія убѣждаетъ искать избавленія.}

Человѣкъ изъ двухъ частей состоитъ, изъ души и тѣла. Имѣетъ тѣло свою бѣду и неблагополучіе, имѣетъ и душа. Тѣло видимо, и бѣда его видима; душа невидима, и бѣда ея невидима. Тѣло тлѣнно и смертно, и бѣда его кончится; душа нетлѣнна и безсмертна, и бѣда ея конца не имѣетъ, но во вѣки съ нею пребываетъ, когда отъ той не свободится. Душа, яко разумная, безсмертная и по образу Божію сотворенная, далеко честнѣйшая есть паче тѣла, потому и бѣда ея далеко опаснѣйшая и лютѣйшая паче бѣды тѣлесной. Бѣда бо тѣлесная съ тѣломъ умирающимъ умираетъ и престаетъ; но душевная бѣда съ безсмертною душею никогда не умираетъ, когда отъ той бѣды здѣ не избавится. Сего ради внимай, хрістіанине, слѣдующимъ: 1)~Тяжко человѣку быть въ нищетѣ тѣлесной, но далеко тягчае быть въ нищетѣ душевной. Тяжко человѣку имѣть тѣло немощное и разслабленное, но далеко тягчае имѣть душу немощную и разслабленную. Тяжко человѣку работать тѣломъ злому какому мучителю, но далеко тягчае работать душею грѣху и грѣхомъ діаволу. \textit{Всякъ, творяй грѣхъ, рабъ есть грѣха}\footnote{Іоан.~8,~34.}. Тяжко человѣку связаннымъ быть узами желѣзными, но далеко тягчае связаннымъ быть узами грѣховными: узы желѣзныя тѣло, но узы грѣховныя душу мучатъ. Тяжко человѣку терпѣть гнѣвъ царскій, но далеко тягчае терпѣть гнѣвъ Божій. Тяжко человѣку быть плѣненнымъ тѣломъ отъ нѣкоего врага, но далеко тягчае быть душею плѣненнымъ отъ діавола. Тяжко человѣку удаленнымъ быть отъ отечества и дома своего, и сродниковъ и друговъ своихъ, но далеко тягчае душѣ удаленной быть отъ небеснаго отечества и дома Божія и самаго Бога, и избранныхъ Его святыхъ. Тяжко человѣку сидѣть въ темницѣ, и мало что видѣть свѣта, и терпѣть неволю, скорбь и озлобленіе, но далеко тягчае душѣ быть въ темницѣ адской, и бѣдствіе тое терпѣть во вѣки. Тяжко человѣку тѣломъ умирать, но далеко тягчае душею умирать. Тѣло умираетъ, и тягость его кончится: душа никогда не умираетъ, и смерть ея и тягота ея никогда не умираетъ. Сколько душа дражайшая паче тѣла, столько и неблагополучіе ея лютѣйшее и горчайшее паче тѣлеснаго. Сего ради болѣе и искать должно избавленія отъ душевнаго неблагополучія, нежели отъ тѣлеснаго. Тѣлесное бо неблагополучіе, коликое бы ни было, и сколь долгое бы ни было, все кончится, но душевное никогда не кончится. 2)~Самая бѣда приводитъ въ познаніе бѣды, и познаніе бѣды убѣждаетъ человѣка искать избавленія. Видятъ люди тѣлесную бѣду, и чувствуютъ тоя горесть, сего ради и убѣгаютъ отъ той, и всякимъ образомъ избавиться отъ той ищутъ, якоже видимъ. Не видятъ душевныя бѣды, далеко горшія паче тѣлесныя, и не чувствуютъ горести ея, и потому мало кто отъ той свободитися усердно желаетъ и ищетъ, развѣ кто познаетъ ее, и чимъ болѣе познаетъ, тѣмъ усерднѣе ищетъ отъ ней избавитися. Слово Божіе душевнымъ нашимъ очамъ представляетъ душевную бѣду, и въ познаніе тоя приводитъ насъ, и образъ избавленія представляетъ намъ "--- Хріста. \textit{Аще Сынъ васъ свободитъ, воистинну свободни будете}\footnote{Іоан.~8,~36.}. Слава Богу человѣколюбцу о семъ! Идолопоклонники познали свою прелесть и заблужденіе, и отъ того всепагубную свою бѣду; познали отъ слова Божія и преславныхъ Его чудесъ, и обратилися ко Хрісту, и свободилъ ихъ отъ всепагубныя бѣды, и получили вѣчное душевное блаженство. Познали многіи, въ древности пожившіи, и сдѣлали пещеры, вертепы, пустыни и пропасти земныя жилищами себѣ. Познали мученики, и изволяли лучше умереть и тягчайшее мученіе терпѣть, нежели отрещися Хріста, понеже Хріста отрещися есть отрещися всего душевнаго блаженства и вѣчнаго живота, и подпасть душевной пагубѣ и вѣчной смерти. Познали душевное бѣдствіе многіи разбойники, убійцы, блудники и блудницы, и прочіи тягчайшіи грѣшники и усердно поискали избавленія, и сыскали. Познаютъ и нынѣ многіи и познавше ужасаются, и ищутъ избавленія такъ, что и о тѣлесныхъ бѣдахъ небрегутъ, только бы избавиться отъ душевной бѣды. Страхъ бо душевныя бѣды превосходитъ и недѣйствительнымъ дѣлаетъ страхъ тѣлесныя бѣды, якоже великій шумъ малый и тихій голосъ уничтожаетъ. Тако познанная бѣда подвигаетъ человѣка къ исканію избавленія. Познай убо, хрістіанине, и ты душевное твое бѣдствіе и неблагополучіе, и непремѣнно ни о чемъ такъ стараться не будешь, какъ чтобы отъ того бѣдствія избавитися; точно страхъ и ужасъ обыметъ тебе, познавающаго бѣдствіе души твоея, и болѣе пожелаешь плакать и рыдать, нежели веселиться. Блаженъ человѣкъ, воистину блаженъ, кто заблаговременно познаетъ бѣдствіе души своея. Познаніе бѣды есть начало блаженства, какъ познаніе болѣзни начало здравія. Познавай убо тако себе, хрістіанине, и будеши блаженъ. 3)~Бѣды тѣлесныя временныя, какъ"=то: болѣзни, немощи, нищета, гоненіе, посмѣяніе, уничтоженіе, поруганіе, клевета, ссылка, плѣненіе, темница, узы, лишеніе богатства и чести, біеніе, раны и самая смерть ничего душѣ нашей не вредятъ, паче же \textit{любящимъ Бога вся поспѣшествуютъ во благое}\footnote{Римл.~8,~28.}. И \textit{блажени} суть, какъ слово Божіе учитъ, которыи вся сія Хріста ради страждутъ\footnote{Мѳ.~5,~11; 10,~22.}. Тѣлесныя убо и временныя бѣды не токмо не отнимаютъ у душъ нашихъ блаженства, но и умножаютъ. \textit{Аще и внѣшній нашъ человѣкъ тлѣетъ, обаче внутренній обновляется по вся дни. Еже бо нынѣ легкое печали нашея по преумноженію въ преспѣяніе тяготу вѣчныя славы содѣловаетъ намъ, не смотряющимъ намъ видимыхъ, но невидимыхъ: видимая бо временна, невидимая же вѣчна}\footnote{2~Кор.~4,~16--18.}. Писано же и тое: \textit{судими, отъ Господа наказуемся, да не съ міромъ осудимся}\footnote{1~Кор.~11,~32.}. И паки: \textit{егоже любитъ Господь, наказуетъ; біетъ же всякаго сына, егоже пріемлетъ}, и проч.\footnote{Евр.~12,~6.} Коль многіи бѣдами тѣлесными и временными подвиглися къ истинному покаянію, исторія не токмо церковная, но и священная свидѣтельствуетъ. Видимъ тое и нынѣ. На сей бо конецъ и посылаетъ человѣколюбивый Богъ бѣды на насъ, да подвигнемся къ истинному покаянію, и въ страхѣ Божіи поживемъ. И бѣды убо и напасти благость и человѣколюбіе Божіе къ намъ свидѣтельствуютъ. И въ бѣдахъ убо и напастяхъ должно человѣколюбивому Богу благодарить, яко все отъ Него къ доброму нашему концу происходитъ. Той бо печется о насъ. Блудный сынъ, когда расточилъ имѣніе отеческое и въ такую пришелъ скудость и нищету, что началъ съ голода погибать, тогда уже въ себе пришелъ, и сказалъ: \textit{колико наемникомъ отца моего избываютъ хлѣбы, азъ же гладомъ гиблю! Воставъ иду ко отцу моему, и реку ему: отче, согрѣшихъ на небо и предъ тобою}, и прочая\footnote{Лук.~15,~17 и 18.}. Тако и грѣшника бѣда подвигаетъ возвратитися къ небесному Отцу и истинному покаянію. Бѣдами и напастьми стѣсненный грѣшникъ такъ въ себѣ разсуждаетъ: не имѣю я тѣлесныхъ и временныхъ благъ, поищу убо душѣ вѣчныхъ и небесныхъ благъ. Нищета моя, скудость и убожество увѣщаваетъ мене искать душевнаго богатства; поищу убо того у Хріста, Который, \textit{богатъ сый, мене ради обнищалъ}. Тѣлесная моя немощь научаетъ мене искать здравія душевнаго у Хріста; Онъ есть Врачъ душъ нашихъ и тѣлесъ. Терплю я безславіе и посмѣяніе отъ людей; поищу убо славы и похвалы у Бога. Но глаголетъ Господь: \textit{токмо прославляющія Мя прославлю}\footnote{1~Цар.~2,~30.}. Терплю я гнѣвъ царскій или господина моего; поищу убо милости у Бога. \textit{Уповающаго же на Господа милость обыдетъ}\footnote{Пс.~31,~10.}. Работаю я тѣломъ такому"=то человѣку, и не имѣю свободы и благородія тѣлеснаго; поищу убо свободы и благородія душевнаго. Вси мене люди презираютъ и оставляютъ; прибѣгну я къ Богу и прилѣплюся къ Нему, и Онъ не оставитъ созданія Своего, ниже презритъ: никого бо Онъ не оставляетъ и не презираетъ. \textit{Не остави мене Господи Боже мой, не отступи отъ мене; вонми въ помощь мою, Господи спасенія моего: упованіе всѣхъ концевъ земли и сущихъ въ мори далече}\footnote{37,~22 и 23; 64,~6.}. Удаленъ я отъ отечества своего и дома, и отлученъ отъ сродниковъ и друговъ моихъ; потщуся убо усерднѣе искать небеснаго отечества и со ангелами и святыми Божіими вѣчное дружество имѣть. Темница сія, въ которой я сижу, и узы сіи, которыми я связанъ, убѣждаютъ мене покаяніемъ избыть отъ темницы адской и узъ грѣховныхъ и адовыхъ, и прочая. Слава Богу, \textit{хотящему всѣмъ спастися и въ разумъ истины пріити}\footnote{1~Тим.~2,~4.}. Онъ посылаетъ мнѣ скорбь, но тою мене обращаетъ и пресѣкаетъ путь къ погибели, и къ Себѣ возвращаетъ: \textit{воставъ} убо \textit{иду ко Отцу моему}, и прочая. Тако бѣды тѣлесныя и временныя не токмо не вредятъ душамъ, но и весьма пользуютъ. Бѣды бо сіи аки женутъ насъ къ покаянію и къ Самому Богу, и какъ бы убѣждаютъ насъ внити въ царствіе Божіе. Слава человѣколюбцу Богу о семъ, слава дивному Его о насъ промыслу! Чрезъ бѣды и напасти наши ищетъ нашего истиннаго блаженства. Тако все наше бѣдствіе и окаянство обращаетъ въ наше блаженство. Но тѣлесное и временное благополучіе часто и по большей части развращаетъ человѣка и погубляетъ. Видимъ сіе на евангельскомъ богачѣ, который не умѣлъ богатствомъ своимъ и благополучіемъ временнымъ владѣть, и сошелъ въ погибель\footnote{Лук.~16,~23.}. Тако и нынѣ многіи не умѣютъ владѣть богатствомъ, и не по волѣ Божіей, но по своимъ прихотямъ расточаютъ тое, и развращаются. Многіи не умѣютъ владѣть здравіемъ тѣла своего, и развращаются и на вся злая дерзаютъ. Благополучіе бо тѣлесное и временное, хотя и добро есть, однакожъ растлѣнному человѣческому сердцу есть, какъ безумному мечь, которымъ себе самого убиваетъ. Блудный сынъ, пока еще имѣлъ имѣніе отеческое въ рукахъ своихъ, не думалъ ко отцу своему возвратитися; но когда уже потерялъ тое, и началъ лишатися, и голодъ терпѣть, тогда уже вздумалъ, какъ выше сказано. Почему и написано намъ въ предосторожность: \textit{богатство аще течетъ, не прилагайте сердца}\footnote{Пс.~61,~11.}. Якоже убо тѣлесная и временная бѣда причиною бываетъ покаянія и исправленія душевнаго, тако тѣлесное и временное благополучіе часто причиною бываетъ развращенія и бѣды и пагубы душевныя, а наипаче, когда юный человѣкъ будетъ имѣть благополучіе. Таковому весьма трудно не развратиться въ благополучіи. И сія"=то есть между прочими причина, чего ради Богъ на рабовъ Своихъ попущаетъ бѣды и напасти, то"=есть, чтобы они съ добраго пути не совращались и тако не лишилися бы вѣчнаго блаженства. Въ благополучіи бо человѣкъ, яко немощенъ, удобно совращается и развращается. Горькое убо лѣкарство "--- временная и тѣлесная скорбь, но душѣ здоровое. 4)~Душевная бѣда всегда вредна и пагубна есть, всегда лютое и великое есть зло, яко Богу и святой волѣ Его противное. Душевная бѣда есть самолюбіе, гордость, высокоуміе, славолюбіе, любоначаліе, самомнѣніе, киченіе, презрѣніе и уничтоженіе ближняго, тьма и заблужденіе въ разумѣ, невѣдѣніе Бога и истины и правды, невѣріе и суевѣріе, зависть, гнѣвъ, злоба и памятозлобіе, сребролюбіе и желаніе богатства, блудная похоть, лицемѣріе и лукавство, хитрость и прочее зло. Сіе есть бѣдствіе душъ нашихъ; сей пагубный ядъ въ сердцѣ нашемъ крыется, и душу нашу снѣдаетъ и умерщвляетъ, и ни къ чему иному, какъ къ вѣчной смерти, бѣднаго человѣка ведетъ. Познавай убо, хрістіанине, бѣдствіе души твоея, да усердно поищеши избавленія отъ того. Познанная бо бѣда убѣждаетъ искать способа избавиться отъ бѣды, и противнаго той желать и искать блаженства. Человѣколюбче Іисусе! \textit{Ты еси прибѣжище отъ скорби, обдержащія мя. Радосте моя, избави мя отъ обышедшихъ мя}\footnote{Пс.~31,~7.}. Грѣшная и бѣдная, но по образу Божію сотворенная душа моя стонетъ и вопіетъ къ Тебѣ, Избавителю своему. 5)~Всякая бѣда наипаче отъ самаго искуса познается. Болѣзнь наипаче тѣ познаютъ, которіи страждутъ болѣзнь. Коль люто мучитъ человѣка зубная болѣзнь, тотъ только знаетъ, кто тою мучится. Тоежде разумѣй и о прочихъ бѣдахъ тѣлесныхъ и временныхъ. Тако и душевное бѣдствіе, внутрь сердца крыющееся, наипаче во искушеніи познается; и чимъ болѣе во искушеніи находится человѣкъ, тѣмъ лучше бѣдность и окаянство душевное свое познаетъ. Искушеніе бо все открываетъ, что внутрь сердца крыется, якоже лѣкарство, называемое \textit{рвотное}, показуетъ, что въ желудкѣ содержится. А чимъ болѣе познаетъ человѣкъ душевное свое бѣдствіе и окаянство, тѣмъ болѣе смиряется, познаніе бо бѣды смиряетъ человѣка. Чимъ же болѣе смиряется, тѣмъ болѣе обрѣтаетъ благодать у Бога. \textit{Смиреннымъ Богъ даетъ благодать}\footnote{1~Петр.~5,~5.}. Многіи о себѣ мечтаютъ, что они не имѣютъ высокоумія, гордости, зависти, гнѣва, злобы, и проч.; но искушеніе нашедшее показуетъ, что въ сердцахъ ихъ тое зло крыется, и тако обманываются. Блудная похоть нѣсколько времени крыется и не познается, но при случаѣ, какъ змій ядовитый, изъ тайнаго мѣста показываетъ свою главу и жаломъ своимъ хощетъ уязвить душу. Зависть, въ сердцѣ нашемъ крыющуюся, благополучіе ближняго трогаетъ и показуетъ; тогда она изъ сердца исходитъ и подымается, когда ближняго увидитъ въ счастіи. Гнѣвъ и злоба въ обидѣ, отъ ближняго наносимой, познается. Жестокосердіе и немилосердіе, бѣда и страданіе ближняго, отъ человѣка презрѣнное, показуетъ. Сребролюбіе въ потеряніи богатства и имѣнія, славолюбіе въ безславіи и безчестіи, любоначаліе въ лишеніи чести и сана, наипаче оказываютъ себе. Видимъ, сколько болѣзнуетъ и скорбитъ человѣкъ, сколько сѣтуетъ, плачетъ и рыдаетъ, лишившися или богатства, или сана и чести, или славы міра сего. Многіи не стерпя болѣзни и скорби, внутрь сердца ихъ крыющіяся, сами себе умерщвляютъ. О слѣпоты! о нечувствія! о бѣдности "--- самого себе погубить ради того, что въ себѣ почти ничто, какъ тѣнь. Что бо кто любитъ, о томъ и жалѣетъ, когда тое потеряетъ; и чимъ болѣе что любитъ, тѣмъ болѣе о томъ потерянномъ жалѣетъ. Тако и прочее бѣдствіе душевное во искушеніи наипаче познается. Искушеніе бо открываетъ сердце наше, и тайная сердца нашего явными дѣлаетъ. Богъ же всю глубину и бездну сердца нашего испытуетъ, и что въ немъ крыется, видитъ. Сего ради попущаетъ на насъ Богъ искушеніе, и тѣмъ показываетъ намъ, что въ сердцѣ нашемъ крыется. Познаемъ убо хрістіанине, душевную нашу бѣду и зло, да тако и избавленія усердно поищемъ. 6)~Отъ сего бѣдствія душевнаго никто насъ не можетъ избавить, никто, кромѣ Іисуса Хріста. Онъ нашъ Избавитель. Онъ нашу и бѣдность знаетъ лучше, нежели мы сами. Онъ ради того и въ міръ пришелъ, и въ мірѣ пожилъ и пострадалъ, и умеръ, чтобы души наша отъ того бѣдствія и зла избавить, и первое наше намъ блаженство и благородіе и свободу, то"=есть, прекрасную оную доброту "--- Божій образъ возвратить, въ томъ бо все блаженство и слава наша состоитъ. Признаемъ убо бѣдность нашу, и, смирившися, у Него поищемъ избавленія. Бѣдная оная жена кровоточивая, нигдѣ не сыскавши болѣзни своей исцѣленія, ко Іисусу съ вѣрою пришла, и только коснувшися святыхъ ризъ Его, получила исцѣленіе\footnote{Марк.~5,~25--29.}. Жена сія увѣщеваетъ и насъ приходить ко Іисусу, и просить у Него помощи и избавленія душамъ. Ибо Онъ и \textit{можетъ}, яко всесиленъ, и \textit{хощетъ}, яко человѣколюбецъ, насъ избавить, и невидимо \textit{съ нами обращается}. Да приступаемъ убо и мы съ вѣрою, и духовно да касаемся святыхъ ризъ Его, и день ото дня почувствуемъ умаляющееся бѣдствіе наше и свободу души возвращающуюся. Но всегда истинно есть: \textit{познаніе бѣды убѣждаетъ искать избавленія отъ бѣды}. Познавай убо бѣду, и будешь искать избавленія. Въ семъ бо вся сила состоитъ, чтобы себе самого познать и свое бѣдствіе душевное.

\section{126. Нищій.}

Видимъ въ мірѣ, что много имѣется нищихъ. Нищій же тотъ есть, который ничего у себе не имѣетъ: ни хлѣба, ни одѣянія, ни дома, ни денегъ, ни прочаго, къ житію сему надлежащаго, но всего у людей проситъ, и отъ нихъ получаетъ. Таковую нищету всякъ человѣкъ отъ высокихъ и до низкихъ, и отъ богатыхъ до убогихъ, и отъ господъ до рабовъ, имѣетъ, и подлинно нищъ есть, хотя и не познаетъ своея нищеты. 1)~Всякъ человѣкъ, что ни имѣетъ, не свое \textit{имѣетъ}, но Божіе, не отъ себе самого имѣетъ, но отъ Бога получаетъ\footnote{1~Кор.~1,~7.}. Богъ, яко благій и щедрый, отверзаетъ Своя сокровища, и всѣмъ, яко нищимъ и убогимъ, все щедрою Своею рукою подаетъ. Злато, сребро и мѣдь Божіе добро есть, и намъ въ нуждахъ нашихъ служитъ. Хлѣбъ, которымъ питаемся; питіе, которымъ напаяемся, прохлаждаемся и утѣшаемся; одежда, которою прикрываемся и согрѣваемся; домъ, въ которомъ упокоеваемся и отъ воздушной непогоды и бури защищаемся; огнь, которымъ согрѣваемся, и варится пища наша; воздухъ, которымъ сохраняется животъ нашъ; свѣтъ, которымъ просвѣщаемся; скоти, звѣри, птицы и рыбы, которіи служатъ нуждамъ нашимъ, и все прочее "--- Божіе добро есть\footnote{Пс.~23,~1.}. Все тое Онъ, яко человѣколюбивый, нищимъ, убогимъ и бѣднымъ намъ подаетъ по единой благости Своей. Откуду всего того, всего къ нашему житію нужнаго, просимъ отъ Него въ молитвѣ: \textit{хлѣбъ нашъ насущный даждь намъ днесь}. Чрезъ \textit{хлѣбъ} здѣ разумѣется все, къ содержанію нашего житія нужное. Видишь, человѣче, твою нищету. Богъ подаетъ тебѣ все, яко нищему и убогому. Подаетъ тебѣ все Свое, понеже ты ничего своего не имѣешь. Иначе бы ты и малѣйшаго времени не моглъ жить. Богъ убо предваряетъ твою скудость и нищету, и отворяетъ сокровища Своя, и оттуду подаетъ блага Своя тебѣ къ содержанію и храненію живота твоего. Показуетъ нищету твою, часъ рожденія и часъ исхода твоего. \textit{Нагъ родишися}, и входиши въ міръ ни съ чимъ; \textit{нагъ и исходиши} отъ міра, и отходиши ни съ чимъ\footnote{Іов.~1,~21; 1~Тим.~6,~7.}. Все, что ни имѣешь, отъ рожденія до смерти только служитъ тебѣ, и все тое Божіе добро есть, а не твое. Богъ все тое подаетъ тебѣ яко нищему и убогому, ибо безъ того жить не можешь. Отсюду учися: 1)~Познавать свою нищету и бѣдность, и тако смиряться. Нищъ еси и бѣденъ, понеже ничего своего не имѣешь, но все отъ Бога получаешь. 2)~Познавать Божію благость, которая все намъ, нищимъ и убогимъ, подаетъ. 3)~Богу, яко дателю всего добра, благодарить за все. Отъ Него всякое получаемъ добро, яко нищіи и убогіи, и получаемъ безъ всякихъ нашихъ заслугъ. Достойно убо и праведно за все и всегда Ему благодарить. Слава Богу благодателю о всемъ! 4)~Отсюду видишь, что вси люди, яко нищіи и убогіи, въ себѣ равны суть. Равная всѣхъ нищета равными всѣхъ дѣлаетъ. Много ли кто, или мало кто имѣетъ, чужое добро, а не свое имѣетъ. 5)~Отсюду никто надъ другимъ не долженъ возноситься, и никого не презирать. Вси чужое добро имѣемъ, а не свое. Почтожъ чужимъ возноситься? 6)~Отсюду слѣдуетъ, что должны мы другъ другу помогать въ нуждахъ. Все Божіе добро имѣемъ, по Божіей убо волѣ должно и расходъ того чинить. Домъ нашъ и ближнимъ нашимъ ко упокоенію долженъ служить. Хлѣбъ нашъ и ближнихъ нашихъ питать, сребро и злато наше и ближнихъ нашихъ нуждамъ помогать, скоты наши и прочимъ человѣкамъ служить должны, и проч. Сего и прочаго отъ познанія нищеты и Божія къ намъ благости научаемся. И сія есть \textit{тѣлесная} нищета, которую вси мы имѣемъ. "--- 2)~Когда человѣкъ посмотритъ внутрь сердца своего, и разсудитъ внутреннее состояніе свое, то увидитъ \textit{душевную} нищету, горшую паче тѣлесной. Ничего бо въ себѣ, кромѣ бѣдности, окаянства, грѣха и тьмы, не имѣетъ. Не имѣетъ истинной и живой вѣры, истинной и сердечной молитвы, истиннаго и сердечнаго благодаренія, своея правды, любве, чистоты, благости, милосердія, кротости, терпѣнія, покоя, тишины, мира и прочаго душевнаго добра. Такъ нищъ и убогъ человѣкъ! Но кто имѣетъ тое сокровище, отъ Бога тое получаетъ, а не отъ себе имѣетъ; благодати Божіей, а не своей силѣ приписывать тое долженъ. Откуду все тое сокровище плодъ Духа Святаго въ Божіемъ словѣ называется. \textit{Плодъ духовный есть любы, радость, миръ, долготерпѣніе, благость, милосердіе, вѣра, кротость, воздержаніе}\footnote{Гал.~5,~22 и 23.}. Познаемъ убо, хрістіанине, нашу душевную нищету, и поищемъ сокровища нашего душевнаго у Хріста, \textit{Который насъ ради обнищалъ богатъ сый, да мы нищетою Его обогатимся}\footnote{2~Кор.~8,~9.}. Не имѣемъ сердечной вѣры, "--- поищемъ, да обогатитъ насъ вѣрою. \textit{Господи, помоги моему невѣрію}\footnote{Марк.~9,~24.}. Не имѣемъ сердечной молитвы, "--- поищемъ тоя у Него, да научитъ насъ чрезъ Духа Своего Святаго молитися. \textit{Господи, научи насъ молитися}\footnote{Лук.~11,~1.}. Не знаемъ истинно и сердечно благодарить, "--- попросимъ и того у Него, да научитъ насъ благодарити и пѣти. Не имѣемъ страха Божія и любве, и прочаго душевнаго сокровища, "--- да толкаемъ въ двери милосердія Его, да отверзетъ Свое небесное сокровище. О всемъ семъ Самъ Онъ повелѣлъ намъ молитися, и обѣщалъ подать. \textit{Просите, и дастся вамъ; ищите, и обрящете; толцыте, и отверзется вамъ: всякъ бо просяй пріемлетъ, и ищай обрѣтаетъ и толкущему отверзется}\footnote{Мѳ.~7,~7 и 8.}. \textit{Приклони, Господи, ухо Твое, и услыши мя; яко нищъ и убогъ есмь азъ}\footnote{Пс.~85,~1.}. \textit{Призри на мя и помилуй мя, по суду любящихъ имя Твое}, и проч.\footnote{118,~132.} \textit{На кого воззрю? токмо на кроткаго и молчаливаго, и трепещущаго словесъ Моихъ}, глаголетъ Господь\footnote{Ис.~66,~2.}. \textit{Призрѣ на молитву смиренныхъ, и не уничижи моленія ихъ. Да напишется сіе въ родъ инъ}\footnote{Пс.~101,~18 и 19.}. "--- 3)~Нищъ человѣкъ: понеже не имѣетъ вѣрныхъ и искреннихъ друговъ. Дружество истинное не можетъ быть безъ взаимной и сердечной любви, безъ единомыслія и согласія. Рѣдка есть любовь въ сынахъ человѣческихъ. Лукавство, хитрость, ложь, лицемѣріе почти во всѣхъ показуется. \textit{Умалишася истины отъ сыновъ человѣческихъ. Суетная глагола кійждо ко искреннему своему, устнѣ льстивыя въ сердцѣ; и въ сердцѣ глаголаша злая}\footnote{Пс.~11,~2 и 3.}. Тотъ обманетъ, другій солжетъ, третій и четвертый тоежде сдѣлаетъ, и такъ надобно всякаго опасаться и бояться. Сердце человѣческое \textit{глубоко} и лукаво\footnote{Іер.~17,~9; Пс.~63,~7.}. Много случается намъ имѣть друговъ, но потомъ вскорѣ врагами нашими дѣлаются; такъ вотъ и узнаемъ, что они ложныи намъ други были. Другъ искренній всегда другъ есть. Въ счастіи нашемъ много друговъ при себѣ, какъ тѣнь во время сіянія при тѣлѣ нашемъ, имѣемъ; но когда скрыется сіяніе счастія нашего и день благополучія нашего помрачится, тогда они, какъ тѣнь, отъ насъ отступаютъ и скрываются. А отсюду видно, что и дружество ихъ ложное было. Истинная бо любовь всегда съ любимымъ соединенна есть, и въ счастіи и въ несчастіи. Ибо истинная любовь двухъ едино дѣлаетъ, и потому гдѣ единъ, тамо неотлучно находится и другій, хотя не всегда тѣломъ, но всегда сердцемъ. Истинный убо нашъ другъ есть, который и въ несчастіи насъ не оставляетъ. Ищи же такого друга, въ нынѣшнее наипаче время, въ которое люди въ томъ и поучаются, чтобы другъ друга обмануть и прельстить, посмѣяться и оклеветать. О бѣдніи хрістіане, какъ вы далеко отступили отъ тѣхъ хрістіанъ, у которыхъ \textit{было сердце и душа едина}\footnote{Дѣян.~4,~32.}! А отсюду видите сами, что хрістіанство ваше ложное есть. Отецъ лжи, діаволъ, научаетъ людей, чтобы другъ другу лгали, обманывали и прельщали, и тако бы вся вѣрность и любовь изъ сердецъ человѣческихъ изсякла. Пощади, Господи, достояніе Твое и сохрани останки Израилевы! Суетная глаголетъ всякъ ко искреннему своему, всякъ поучается лжи, всякъ языкъ свой, яко бритву, остритъ, остритъ на клевету, посмѣяніе, уничтоженіе и поруганіе. \textit{Спаси мя, Господи, яко оскудѣ преподобный}\footnote{Пс.~11,~2.}. Хрістіанине, какихъ здѣ надѣяться друговъ? Здѣ явная бѣда человѣку, бѣда отъ сосѣдъ, бѣда отъ домашнихъ, бѣда отъ друговъ, бѣда отъ братіи, бѣда отъ слугъ и рабовъ. Не можетъ быть большее дружество, какъ между мужемъ и женою; но и здѣ коль много видимъ вражды, сколько слышимъ жалобъ отъ нихъ другъ на друга, и клеветъ и прочаго зла! Какое и сіе дружество есть другъ на друга жалиться и клеветать? Видишь, хрістіанине, что мы не имѣемъ вѣрныхъ и искреннихъ друговъ, и потому нищіи есмы, да и сыскать ихъ негдѣ. Обратимся убо ко Хрісту, и къ Нему пристанемъ и прилѣпимся сердцемъ нашимъ. Безъ сумнѣнія Онъ любитъ насъ и недостойныхъ, яко Человѣколюбецъ, такъ, что и душу Свою за насъ положилъ. Возлюбимъ и мы Его, яко нашего Любителя, и тако между нами будутъ дружество, но вѣрное, истинное, искреннее непрестающее, блаженное и вѣчное. Взаимная бо любовь дружество содѣловаетъ. Тогда хотя и весь міръ будетъ намъ враждовать, воистину ничего не повредитъ. \textit{Вы друзи Мои есте} (о благости, о человѣколюбія Іисусова, о чести и достоинства хрістіанскаго!), \textit{аще творите, елика Азъ заповѣдаю вамъ}, глаголетъ Господь\footnote{Іоан.~15,~14.}. \textit{Услыши, Господи, гласъ мой, имже воззвахъ, помилуй мя, и услыши мя. Тебѣ рече сердце мое: Господа взыщу, взыска Тебе лице мое, лица Твоего, Господи, взыщу. Не отврати лица Твоего отъ мене, и не уклонися гнѣвомъ отъ раба Твоего: помощникъ мой буди, не отрини мене, и не остави мене, Боже Спасителю мой}\footnote{Пс.~26,~7--9.}. \textit{Мнѣ же прилѣплятися Богови благо есть, полагати на Господа упованіе мое}\footnote{72,~28.}.

\subsection{О томжде.}

Люди нищіи и бѣдніи, во всякой скудости и недостаткѣ живущіи, зная нѣкоего человѣка богатаго и щедраго, всѣ къ нему прибѣгаютъ и проситъ отъ него всякъ въ скудости своей помощи. Иный проситъ у него денегъ на нужду свою, иный хлѣба, иный одѣянія, иный инаго, и получаютъ, чего просятъ. Хрістіанине! мы всѣ нищіи, бѣдніи и окаянніи, какъ выше сказано. Нищетѣ нашей, наипаче душевной, никто не можетъ помощи, кромѣ Хріста. Знаемъ мы, что Онъ богатъ, яко Богъ, и щедръ и милостивъ и человѣколюбивый. Все убо и можетъ и хощетъ намъ подать. \textit{Онъ ради насъ обнищалъ, да мы нищетою Его обогатимся}\footnote{2~Кор.~8,~9.}. Онъ нашу нищету на Себе воспріялъ, да подастъ намъ богатство благости Своея. Онъ убо насъ нищихъ, бѣдныхъ и убогихъ прибѣжище и обогатитель. Къ Нему предки и отцы наши въ своей нищетѣ, бѣдности, убожествѣ и окаянствѣ прибѣгнули, и получили помощь своимъ недостаткамъ, и обогатились всещедрою рукою Его. Къ Нему и нынѣ вси знающіи Его прибѣгаютъ, и получаютъ свое удовольствіе. Пристанемъ и мы и прилѣпимся къ благочестивой дружинѣ сей, и прибѣжимъ къ Нему съ вѣрою и надеждою, да и мы отъ Него получимъ милость, и обогатитъ насъ Своею благодатію. Онъ никого не отгоняетъ отъ Себе. \textit{Грядущаго}, рече, \textit{ко Мнѣ не изжену вонъ}\footnote{Іоан.~6,~37.}. \textit{Уповающаго же на Господа милость обыдетъ}\footnote{Пс.~31,~10.}. Паче же всѣхъ насъ, бѣдныхъ и убогихъ, къ Себѣ призываетъ: \textit{пріидите ко Мнѣ вси труждающіися и обремененніи, и Азъ упокою васъ}\footnote{Мѳ.~11,~28.}. Тѣлесная и временная благая всѣмъ Онъ подаетъ, всѣмъ, знающимъ Его и незнающимъ. Вси отъ щедрой Его руки получаютъ оная. Подаетъ и намъ, и довольствуемся тѣми. Слава благости Его! Однакожъ должно всего у Него просить, яко нищимъ и убогимъ, да познаемъ и признаемъ нищету свою, и сокровище благости Его, которое намъ бѣднымъ отверзается, и тако будемъ Ему благодарить. Душевная благая подаетъ единымъ только знающимъ Его и вѣрующимъ въ Него, и съ вѣрою просящимъ ихъ у Него. Познаемъ убо и мы неизчерпаемое Его сокровище и нашего Его Искупителя, Спасителя и Обогатителя. Его небесный Отецъ къ намъ послалъ, насъ, нищихъ и бѣдныхъ, обогатить. Въ Немъ все сокровище блаженства нашего содержится. Прибѣгнемъ убо къ Нему съ вѣрою, и объявимъ Ему свою скудость и нищету, да призритъ на насъ и подастъ намъ прошеніе наше, и вмѣсто нищеты нашея подастъ намъ богатство благодати Своея. Но неотмѣнно требуется отъ насъ, чтобы всякъ позналъ и призналъ свою нищету, а безъ того ничего не получимъ. \textit{Зане Богъ гордымъ противится: смиреннымъ же даетъ благодать}\footnote{1~Петр.~5,~5.}. Смиреніе безъ познанія и признанія нищеты и бѣдности не можетъ быть. О нищіи и убогіи человѣки! Чимъ намъ гордиться и возноситься? Нищи мы тѣломъ, нищи и душею. Хотя и имѣемъ тѣлесное и временное богатство, но тое чужое добро, а не наше есть. Все бо Божіе, ибо \textit{Господня земля, и исполненіе ея}\footnote{Пс.~23,~1.}. Душевнаго сокровища не имѣемъ. Но тѣмъ паче бѣднѣйшіи, что нашея сея нищеты не познаемъ. Слѣпы, и не познаемъ слѣпоты, бѣдны и не познаемъ бѣдности, убогіи и не познаемъ убожества, окаянніи и не познаемъ окаянства нашего. Богатство благодати Божія всѣмъ готово и отверзается; но человѣкъ не хощетъ познать и признать своея нищеты и смириться, и тако получить Божію благодать. \textit{Смиреннымъ бо дается благодать}. О какъ люто заразилъ и ослѣпилъ бѣднаго человѣка грѣхъ! Въ такую подлость упалъ образъ Божій! Бѣденъ человѣкъ и своея бѣдности не познаетъ и не признаетъ, и того ради ничего не получаетъ. Богъ хощетъ его обогатить, но онъ не хощетъ признать своея нищеты. Хрістіанине! въ семъ вся сила состоитъ, чтобы мы познали и признали свою нищету и бѣдность предъ Богомъ, а богатство благодати Его готово намъ. Наша убо вина, что мы душевнаго сокровища не имѣемъ. Осмотримся убо, и познаемъ нашу нищету, да скудость нашу и недостатки благодатію Своею исполнитъ Господь. \textit{Всякъ возносяйся, смирится: смиряяй же себе, вознесется}\footnote{Лук.~18,~14.}.

\section{127. Сокровище.}

Имѣютъ сыны вѣка сего свое сокровище, имѣютъ и хрістіане свое сокровище. У сыновъ вѣка сего сокровище есть богатство тлѣнное, злато, сребро, мѣдь и прочее вещество. У хрістіанъ сокровище есть Божія благодать, и есть сіе сокровище небесное, духовное, въ сердцахъ ихъ пребывающее, якоже Апостолъ написалъ: \textit{имамы сокровище сіе въ скудельныхъ сосудѣхъ, да премножество силы будетъ Божія, а не отъ насъ}\footnote{2~Кор.~4,~7.}. Люди, имѣющіи сокровище тлѣнное, всякія свои нужды и недостатки тѣмъ исполняютъ. Не имѣютъ хлѣба, тѣмъ себѣ купятъ хлѣбъ. Не имѣютъ одежды, тѣмъ достаютъ себѣ одежду, и прочая. Тако Божія благодать, небесное сокровище, въ хрістіанскихъ сердцахъ живущая, всѣ нужды ихъ и недостатки душевные исполняетъ. Человѣкъ самъ въ себѣ слѣпъ есть, "--- Божія благодать просвѣщаетъ сердечныя очи его, показуетъ ему суету міра сего, краткость временнаго житія и долготу вѣчности; показуетъ, яко все, что ни есть въ мірѣ семъ, какъ тѣнь преходитъ, и что человѣкъ отъ рожденія до смерти есть странникъ и путникъ, идущій къ небесному отечеству, "--- и увѣщеваетъ его на всякое время всего берещися, и всего, въ мірѣ семъ имѣющагося, со страхомъ касаться, и къ нуждѣ, а не къ плотоугодію употреблять, и соблазновъ міра сего берещися. Почему таковая душа не ищетъ въ мірѣ семъ ни богатства, ни чести, ни славы, ни прочаго міра сего сокровища, якоже сыны вѣка сего дѣлаютъ; но всегда смотритъ и стремится къ вѣчному блаженству, и тое всеусердно желаетъ и тщится получить. Почему воздыхаетъ всегда къ своему Создателю, дабы тое благодатію Его возмогла получить, и тамо со всѣми избранными хвалить и пѣть Его во вѣки. Немощенъ человѣкъ и не силенъ противу стремленія страстей и грѣха стоять, и побѣждать, "--- Божія благодать укрѣпляетъ его и помогаетъ ему въ семъ важномъ подвигѣ стоять, и показуетъ ему мерзость страстей и грѣха, и на всякое время увѣщаваетъ его берещися того зла. Откуду таковый человѣкъ крайне бережется всякаго грѣха и отъ случаевъ, которіи ко грѣху приводятъ, убѣгаетъ, и совѣсть свою чистую во всемъ тщится хранить. Таковому человѣку тяжко согрѣшить, и лучше изволяетъ умереть, нежели согрѣшить, вѣдая, яко всякимъ грѣхомъ величество Божіе оскорбляется, и совѣсть раздражается и безпокойствуется, что душѣ благочестивой несносно есть. Многихъ грѣховъ человѣкъ не видитъ въ себѣ: \textit{грѣхопаденія бо кто разумѣетъ}\footnote{Пс.~18,~13.}? Таковые грѣхи Божія благодать обличаетъ въ немъ. Все бо тое грѣхъ есть, что противу воли Божіей дѣлается. Отсюду востаетъ въ сердцѣ человѣческомъ печаль, тоска и воздыханіе, а иногда и мученіе чувствуется. Отсюду сердце сокрушается, и знаки своя, горячія слезы извергаетъ, плачетъ и рыдаетъ, аки бы нѣчто великое потерялъ. И самыя слезы таковому сердцу вмѣсто прохлажденія и утѣшенія бываютъ. Тако благодать Божія сокрушаетъ сердце человѣческое и исцѣляетъ, опечаляетъ и утѣшаетъ, мертвитъ и живитъ, и все на пользу нашу строитъ. Таяжде Божія благодать показуетъ человѣку не токмо свое его, да и всего рода человѣческаго окаянство и бѣдность. Откуду въ человѣческомъ сердцѣ востаетъ соболѣзнованіе и къ прочимъ людемъ. Вси бо человѣки суть единаго естества, и вси имѣютъ едину немощь и окаянство. Сего ради душа благочестивая не только о своей, но и о прочіихъ людей бѣдности воздыхаетъ и плачетъ. "--- Не знаетъ и не имѣетъ человѣкъ отъ себе истиннаго покаянія и жалѣнія за грѣхи, "--- Божія благодать дѣйствуетъ въ немъ истинное покаяніе и жалѣніе. Истинное покаяніе и жалѣніе есть жалѣть и болѣзновать за грѣхи не ради погибели своей, но ради того, что Бога преблагаго, Который есть благоутробный Отецъ и вѣчная благостыня и любовь, грѣхами оскорбляемъ и прогнѣвляемъ. Человѣкъ, разсуждая свою подлость и Божіе непостижимое величество и презѣльную Его любовь къ себѣ и ко всѣмъ, весьма печалится и сердцемъ уязвляется, что онъ, \textit{земля и пепелъ}, и отъ Бога любимый, Любителя своего оскорбилъ и прогнѣвилъ. Того ангели и архангели и вси небесныя силы со страхомъ и любовію почитаютъ, поютъ и покланяются, но онъ подлый, \textit{земля и пепелъ}, не почиталъ Его и не слушалъ. Отъ сего презѣльную печаль чувствуетъ человѣкъ на сердцѣ своемъ, и тою, яко стрѣлою, уязвляется. И сія"=то есть \textit{печаль по Бозѣ}, которая \textit{покаяніе нераскаянное во спасеніе содѣловаетъ}\footnote{2~Кор.~7,~10.}. Сію печаль Божія благодать содѣловаетъ. Таковую печаль имѣющій будетъ скорбѣть и тужить, хотя бы и геенны не было. "--- Не имѣетъ человѣкъ истинной и сердечной любве къ Богу отъ себе, "--- Божія благодать возбуждаетъ въ сердцѣ его любовь, показуя ему, что Богъ есть несозданное, безначальное, безконечное, непремѣняемое, присносущное, прелюбезное Естество и самоверховнѣйшее Добро, отъ Котораго вся благая и благодѣянія, какъ отъ приснотекущаго источника ручьи, проистекаютъ, и какъ вси созданія и человѣки, такъ и онъ созданъ и падшій искупленъ къ вѣчному блаженству отъ Него.

Тако благодать Божія, просвѣщая сердце человѣческое, зажигаетъ въ немъ огнь любве Божія. Таковую любовь чувствуя человѣкъ въ сердцѣ своемъ, отрыгаетъ словеса любве: \textit{возлюблю Тя, Господи, крѣпосте моя}, и проч.\footnote{Пс.~17,~2.} И паки: \textit{что ми есть на небеси? и отъ Тебе что восхотѣхъ на земли? Исчезе сердце мое и плоть моя: Боже сердца моего, и часть моя Боже во вѣки}\footnote{72,~25 и 26.}. Истинный бо Божій любитель ни на земли, ни на небеси ничего не желаетъ, кромѣ единаго Бога. Таковая душа со пророкомъ вопіетъ внутрь себе: \textit{имже образомъ желаетъ елень на источники водныя, сице желаетъ душа моя къ Тебѣ, Боже! Возжада душа моя къ Богу крѣпкому, живому}, и проч.\footnote{Пс.~41,~2 и 3.} Таковому съ Богомъ быть и во адѣ рай, безъ Бога быть и въ небѣ мученіе. Сего ради таковый во всемъ тщится Богу угодить, и, многія видя въ себѣ немощи и недостатки къ тому, скорбитъ и печалится. А любя Бога, любитъ и ближняго своего, вѣдая, что Богъ его любитъ и велитъ его любить. "--- Не имѣетъ человѣкъ истиннаго смиренія отъ себе, "--- Божія благодать показуетъ ему его нищету, убожество, бѣдность, окаянство, подлость и ничтожество, и тако, познавая себе, человѣкъ смиряется, и, не надѣяся на себе ни въ чемъ, \textit{единому} всемогуществу, премудрости, благости и милосердію Божію предается. "--- Не имѣетъ человѣкъ правды въ себѣ, съ которою бы предъ судомъ Божіимъ явиться моглъ и оправдаться, но видитъ себе грѣшника, которому на судѣ Божіи постоять никакъ не возможно, "--- благодать Божія утѣшаетъ его, и увѣщаваетъ правды ради къ Богу воздыхать, и о Хрістѣ, за грѣшниковъ умершемъ и возкресшемъ, искать оправданія. Таковый съ мытаремъ вопіетъ къ Богу: \textit{Боже, милостивъ буди мнѣ грѣшнику}\footnote{Лук.~18,~13.}. \textit{Боже, отврати лице Твое отъ грѣхъ моихъ}, и обрати на лице Хріста Твоего, Искупителя моего, Іисуса, Который мене ради грѣшника пострадалъ и умеръ. Въ Тебѣ, Іисусе Святе мой, оправдаюся. "--- Не имѣетъ человѣкъ отъ себе истинной и сердечной молитвы, "--- Божія благодать научаетъ его, о чемъ молиться, и возбуждаетъ въ немъ истинную и сердечную молитву. И какъ дымъ благовоннаго порошка, положеннаго на углѣ горящемъ, восходитъ: тако возбужденная благодатію Божіею молитва отъ сердца востаетъ и восходитъ къ Богу. Тогда человѣкъ кратко, но усердно молится и вопіетъ: \textit{о Господи, помилуй! о Господи, ущѣдри! о Господи, услыши и спаси!} Таковыя слова суть слова сердечныя, и устнами произносятся. Сіи слова отъ избытка сердца уста глаголютъ. Таковая молитва, хотя краткая, но проходитъ небеса и входитъ во уши Господа Вседержителя. "--- Не знаетъ человѣкъ истинно и сердечно Богу благодарить, "--- Божія благодать его научаетъ того, показуя ему безчисленная Божія благодѣянія, ему поданная и подаваемая, которыхъ онъ подлый, убогій, нищій и ничего недостойный, туне и безъ всякихъ своихъ заслугъ, отъ благости Божія сподобляется. Чего бо грѣшникъ достоинъ, кромѣ единыя казни? Но Богъ преблагій и человѣколюбивый, не смотря на то, что онъ грѣшникъ и Его уничижитель, безчисленная Своя благая подаетъ ему. Сію благость и человѣколюбіе Божіе и свою подлость и недостоинство разсуждая, человѣкъ сердечно Богу благодаритъ, яко своему высочайшему Благодѣтелю, сердечныя колѣна преклоняетъ предъ Нимъ и падаетъ, и хотя не устнами, но сердцемъ вопіетъ къ Нему: кто есмь азъ, Господи, что Ты мнѣ толикая и толикая благая Своя подаеши? Я грѣшникъ, я тварь предъ Тобою согрѣшившая и огорчившая Тебе, и на всякій день предъ Тобою согрѣшающая, но Ты мене и такого не оставляешь, но благостію и милостію Твоею жалуешь. \textit{Господи, что есть человѣкъ, яко познался еси ему, или сынъ человѣчь, яко вмѣняеши его? Человѣкъ суетѣ уподобися}, и прочая\footnote{Пс.~143,~3 и 4.}. \textit{Благослови душе моя Господа, и вся внутренняя моя имя святое Его. Благослови душе моя Господа, и не забывай всѣхъ воздаяній Его}, и прочая\footnote{102,~1 и 2.}. У благодарнаго бо сердца всегда благая благодѣтеля своего предъ душевными очами обращаются, и, на тыя смотря и свое недостоинство, усердно благодѣтелю своему благодаритъ; якоже неблагодарнаго сердца знакъ есть забвеніе благодѣяній, якоже пишется о Израильтянахъ неблагодарныхъ: \textit{и забыша благодѣянія Его, и чудеса Его, яже показа имъ}\footnote{Пс.~77,~11.}. Божія благодать отъ того зла отвращаетъ человѣка, и на всяко время напоминаетъ ему Божія благодѣянія, и тыя въ памяти содержать, и за нихъ усердно Богу "--- Благодѣтелю благодарить, и свое признавать недостоинство. "--- Не имѣетъ человѣкъ отъ себе страха Божія, "--- Божія благодать научаетъ его страху Господню, представляя ему предъ сердечныя очи Божіе величество, всемогущество, вездѣсущіе и всевѣдѣніе, и тако увѣщаваетъ его опасно и со страхомъ жительствовать, въ словахъ, дѣлахъ и мысляхъ всякаго грѣха берещися, яко тѣмъ Богъ прогнѣвляется, и человѣкъ согрѣшающій милости Божія лишается, и вездѣ и всегда обращаться такъ предъ вездѣсущимъ и всевидящимъ Богомъ, какъ дѣти обращаются предъ отцомъ своимъ, и раби предъ господиномъ своимъ, и подданніи предъ царемъ своимъ, и ничего непристойнаго не смѣютъ дѣлать. Откуду таковая богобоящаяся душа вездѣ и всегда, тайно и явно, предъ людьми и безъ людей, внѣ и внутрь, опасно поступаетъ и всякаго бережется зла. Страхомъ бо Божіимъ окружаема и содержима, неподвижима бываетъ ни на какое зло. И аще какое искушеніе бѣсовское и злая мысль находитъ ей, тутъ ужасается того и къ Богу вопіетъ: \textit{Господи, помози мнѣ!} и тако противу зла стоитъ и борется. Тако страхъ Божій есть корень всѣхъ благъ. \textit{Начало премудрости страхъ Господень}\footnote{110,~10.}. Кто убо премудръ? Тотъ, кто вездѣ и всегда опасно поступаетъ, и невидимаго Бога, аки видимаго, предъ собою имѣетъ. \textit{Блюдите, како опасно ходите, не якоже немудри, но якоже премудри}\footnote{Еф.~5,~15.}. Сея премудрости начало есть страхъ Господень. Страху же Господню человѣка внутрь \textit{научаетъ} Божія благодать\footnote{Пс.~33,~2.}. Таяжде Божія благодать научаетъ человѣка радоватися о Бозѣ Спасѣ своемъ, и содѣловаетъ въ сердцѣ его радость истинную, духовную, небесную, и нѣкое восклицаніе и играніе, и покой и миръ въ совѣсти его, "--- что все есть предвкушеніе вѣчныя жизни и аки крупицы отъ небесныя трапезы, на сердце человѣческое падающія. Таяжде Божія благодать измѣняетъ человѣка, и дѣлаетъ его любительнымъ, милосердымъ, сострадательнымъ, кроткимъ, такъ, что таковый и надъ врагами своими умилостивляется и сожалѣетъ и соболѣзнуетъ имъ, и тако бываетъ терпѣливъ къ нимъ. Божія благодать иногда тако запаляетъ любовію сердце человѣческое, что хотѣлъ бы всѣхъ, безъ изъятія, любве своея объятіями обнять, и всѣхъ спасенныхъ видѣть. Сія и прочая суть дѣйствія Божія благодати, въ сердцѣ человѣческомъ живущія. Сіе есть сокровище хрістіанъ, и есть внутрь ихъ, а не внѣ, и тое вездѣ и всегда съ собою носятъ. Тако они нищіи и богатіи: нищіи въ себѣ, но богатіи милостію Божіею. И хотя сокровище сіе небесное внутрь себе имѣютъ, однакожъ познаютъ и признаютъ себе нищими и убогими, и всегда того желаютъ и просятъ отъ Создателя своего, и непрестанно алчутъ и жаждутъ правды. Божія бо благодать, которая въ нихъ пребываетъ, таяжде и смиряетъ ихъ. Познаемъ убо, хрістіанине, нашу нищету и богатство Божія благодати, и смиримся предъ Господемъ нашимъ, да подастъ намъ Свою благодать. \textit{Зане Богъ гордымъ противится: смиреннымъ же даетъ благодать}\footnote{1~Петр.~5,~5.}. Смиренное сердце, и отъ любви міра и прихотей плотскихъ испраздненное, удобно есть къ воспріятію Божія благодати, якоже праздный сосудъ удобенъ есть пріяти въ себе воду или иное вещество. Вода всегда съ горъ внизъ течетъ, и Божія благодать съ небесныхъ горъ въ праздное, умаленное и низкое сердце снисходитъ. \textit{Сердце чисто созижди во мнѣ, Боже, и духъ правъ обнови во утробѣ моей. Не отвержи мене отъ лица Твоего, и Духа Твоего Святаго не отъими отъ мене. Воздаждь ми радость спасенія Твоего, и духомъ владычнимъ утверди мя}\footnote{Пс.~50,~12--14.}.

\section{128. Свидѣтели.}

Видимъ въ мірѣ, что многіи свидѣтели имѣются, которыи свидѣтельствуютъ о случающихся въ мірѣ вещахъ. Свидѣтельствуютъ люди на судѣ о правомъ и виноватомъ; свидѣтельствуютъ домъ и зданіе о архитекторѣ и здателѣ; свидѣтельствуетъ книга о сочинителѣ и разумѣ его; свидѣтельствуетъ пища вареная о поварѣ, свидѣтельствуетъ письмо о писателѣ; свидѣтельствуютъ слѣды человѣческіе о человѣкѣ ходившемъ, слѣды скотскіе о скотѣ ходившемъ, слѣды птичьи о птицѣ ходившей, и проч. Тако весь міръ свидѣтельствуетъ о Бозѣ Создателѣ своемъ, и безъ гласа о Немъ говоритъ намъ, разумной твари: \textit{Той мя создалъ}. Велико и чудно зданіе великаго и чуднаго создателя показуетъ. Свидѣтельствуетъ великая и непостижимая обширность неба: \textit{Той мя создалъ. Небеса повѣдаютъ славу Божію}\footnote{18,~2.}. Велико и обширно небо, но Создатель несравненно большій, Которому \textit{небо престолъ, земля же подножіе}\footnote{Ис.~66,~1.}. 1)~Свидѣтельствуютъ солнце, луна и звѣзды: \textit{Той насъ создалъ}; имѣемъ мы свѣтъ, отъ Него намъ данный, и вамъ, о человѣки, служимъ тѣмъ; но Создатель нашъ несравненно лучшій, краснѣйшій, превосходнѣйшій и чуднѣйшій \textit{Свѣтъ} есть, Той, Который толикій свѣтъ намъ подалъ\footnote{1~Іоан.~1,~5.}. \textit{Одѣяйся свѣтомъ, яко ризою}\footnote{Пс.~103,~2.}. Мы просвѣщаемъ тѣлесныя очи ваши; но Той просвѣщаетъ душевныя ваши очи. Мы показуемъ вамъ путь, добро и зло, пользу и вредъ тѣлесный; но Той словомъ Своимъ святымъ показуетъ вамъ путь къ вѣчному животу, добро и зло, пользу и вредъ душевный. Свидѣтельствуетъ воздухъ: \textit{Той мене создалъ}. Я оживляю тѣлеса ваша, но Той оживляетъ души ваши; я животъ тѣлесъ вашихъ, но Той есть животъ душъ вашихъ; безъ мене не могутъ животныя жити, безъ Того не могутъ живы быть души ваши; я оживляю смертныя ваши тѣлеса, но отъ Того душамъ вашимъ ищите живота. \textit{Боже, оживиши ны, и имя Твое призовемъ}\footnote{79,~19.}. Свидѣтельствуетъ премѣненіе дней и нощей, и молча вопіетъ намъ о Бозѣ: \textit{Той устроилъ тако. День дни отрыгаетъ глаголъ, и нощь нощи возвѣщаетъ разумъ. Не суть рѣчи, ниже словеса, ихже не слышатся гласи ихъ}\footnote{18,~3 и 4.}. На востокѣ и западѣ, на сѣверѣ и полудни свидѣтельствуютъ безъ гласа. Нощь служитъ намъ ко упокоенію, день къ дѣланію. \textit{Изыдетъ человѣкъ на дѣло свое, и на дѣланіе свое до вечера}\footnote{103,~23.}. Свидѣтельствуютъ источники, рѣки, озера и моря, и живущія въ нихъ: \textit{Той насъ создалъ}. Мы служимъ нуждамъ вашимъ; но Той повелѣлъ намъ служить вамъ, да вы познаете Создателя своего и служите Ему. Мы служимъ вамъ, вы служите, разумная тварь, Создателю своему и благодарите Ему. Свидѣтельствуетъ земля: \textit{Той мя создалъ}, Той мя ко обитанію вашему подалъ вамъ. \textit{Небо небесе Господеви, землю же даде сыновомъ человѣческимъ}\footnote{Пс.~113,~24.}.

Вы отъ мене начало имѣете, но паки въ нѣдра моя возвратитеся, откуду и взяты. Познайте убо, что вы странники и пришельцы въ мірѣ семъ есте. Ищите убо отечества и дому своего, въ которомъ неподвижимо надобно вамъ жить и упокоеваться. \textit{Нова небесе и новы земли, по обѣтованію Его, чаемъ, въ нихже правда живетъ}\footnote{2~Петр.~3,~13.}. Свидѣтельствуютъ плоды земніи: \textit{Той насъ создалъ}. Мы, о человѣки, то питательную, то цѣлительную силу имѣемъ; Той намъ ради васъ таковую силу подалъ. Мы питаемъ и исцѣляемъ тѣлеса ваша; но отъ Того душамъ вашимъ ищите пищи и исцѣленія: Той бо есть пища и исцѣленіе душъ вашихъ. Вкушая насъ, \textit{вкусите и видите, яко благъ Господь}, Который толикая благая вамъ къ пропитанію, утѣшенію и исцѣленію подалъ\footnote{Пс.~33,~9.}. Мы созданіе Его добро, но Создатель нашъ несравненно лучшій; мы добро тѣлесъ вашихъ, но Той есть добро душъ вашихъ. Того любите, Тому благодарите, Того ищите, къ Тому прилѣпляйтеся. \textit{Вкусите и видите, яко благъ Господь}. Свидѣтельствуютъ древеса различная: \textit{Той насъ создалъ}. Мы служимъ вамъ, о человѣки, то плодами нашими, то осѣненіемъ во время зноя; служимъ то къ созиданію домовъ вашихъ, то къ согрѣянію покоевъ вашихъ, то къ варенію пищи вашей, и въ прочихъ различныхъ нуждахъ. Служимъ вамъ: вы служите Создателю нашему и благодарите Ему. Свидѣтельствуютъ звѣри, скоти и птицы различныя: \textit{Той насъ создалъ}, Той повелѣлъ намъ служить вамъ, о человѣки! Мы служимъ вамъ то пищею, то работою, то одеждою; отъ насъ вы получаете то пищу въ снѣдь, то работное служеніе, то кожи ко одѣянію; работаемъ мы вамъ, и служимъ нуждамъ вашимъ: \textit{Той намъ тако повелѣлъ}. Вы работайте и служите Господу Создателю нашему, и благодарите и хвалите Его, \textit{яко благъ Господь}, Который на службу вашу насъ устроилъ. Свидѣтельствуютъ облака, туды и сюды преходящія, и воду, яко мѣхи, переносящія, и тою воздухъ, землю и плоды земные орошающія, и насъ прохлаждающія: \textit{Той насъ создалъ}. Чудно созданіе, чудно и устроеніе его! На тонкой воздуха стихіи мѣсто свое имѣетъ; такъ великое воды бремя въ себѣ содержитъ, но не падаетъ; не вдругъ, но по малу тое на землю изливаетъ, и одну напоивши страну, въ другую переходитъ, и тамо изливаетъ свое сокровище и жаждущую напаяетъ землю. Тако всесильный и премудрый устроилъ Создатель, \textit{возводя облаки отъ послѣднихъ земли}\footnote{Пс.~134,~7.}. Смотрите на насъ, о человѣки, безъ гласа вопіютъ они намъ; смотрите на насъ, и познавайте силу и премудрость Создателя нашего. Той устроилъ и повелѣлъ намъ тако быть. Мы творимъ всесильное слово Его, съ мѣста на мѣсто переходимъ и разносимъ воду, яко раби Его, и орошаемъ воздухъ, и прохлаждаемъ васъ, напаяемъ нивы ваша и возращаемъ плоды ваша. Мы служимъ вамъ: вы служите Создателю своему, Который намъ повелѣлъ служить вамъ, и Тому благодарите и хвалите Его, \textit{яко благъ Господь. Не несвидѣтельствована Себе остави Богъ, благотворя, съ небесе намъ дожди дая и времена плодоносна, исполняя пищею и веселіемъ сердца наша}\footnote{Дѣян.~14,~17.}. Свидѣтельствуютъ молніи и громы: \textit{Той насъ создалъ. Молніи въ дождь сотвори}\footnote{Пс.~134,~7.}. Мы, о человѣки, вопіютъ они намъ, дѣлаемъ вамъ страхъ; но вы паче убойтеся Того, Который насъ сотворилъ; страшны мы вамъ, но Онъ \textit{страшенъ зѣло}\footnote{Сир.~1,~7.}. Убойтеся убо Того, убойтеся, Который и душу и тѣло можетъ погубить въ гееннѣ огненнѣй; воистину Того убойтеся, и покайтеся, да не погибнете. Свидѣтельствуютъ вѣтры: \textit{Той насъ сотворилъ, изводяй вѣтры отъ сокровищъ своихъ}\footnote{Пс.~134,~7.}. Мы, о человѣки, вопіютъ они, пригоняемъ вамъ дождевыя облака, очищаемъ воздухъ и дѣлаемъ его благораствореннымъ, прогоняемъ мглу, всякую гнилость и вредъ отъ него отнимаемъ, и дѣлаемъ того вамъ полезнымъ и здоровымъ, пособствуемъ плодамъ вашимъ земнымъ расти, возрастати, зрѣти и созрѣвати, и тако служимъ вамъ. Мы служимъ вамъ: вы служите Господу Создателю нашему, и благодарите Ему, \textit{яко благъ Господь}. Свидѣтельствуютъ зима и лѣто, весна и осень: Той \textit{насъ создалъ}. Свидѣтельствуетъ и вся тварь: \textit{Той насъ сотворилъ}. Изъ ничего сотворенная свидѣтельствуютъ всемогущую силу Творца своего. Премудро сотворенная свидѣтельствуютъ премудрость Творца своего. \textit{Вся премудростію сотворилъ еси}\footnote{103,~24.}. Добра тварь, и добра зѣло свидѣтельствуетъ непостижимую доброту и благость Творца своего. Свидѣтельствуетъ совѣсть человѣческая о Бозѣ Создателѣ нашемъ. Вѣрный сей свидѣтель никогда не молчитъ, но всегда вопіетъ человѣку: \textit{есть Богъ, есть и праведный судъ Его; есть Богъ, Который награждаетъ добродѣтель, и казнитъ грѣхъ}. Откуду за грѣхъ обличаетъ человѣка, безпокойствуетъ и мучитъ, "--- за добродѣтель утѣшаетъ и увеселяетъ. Свидѣтельствуетъ о Бозѣ слово Божіе, содержащееся въ священной Библіи. \textit{Законъ Господень непороченъ, обращаяй души. Свидѣтельство Господне вѣрно, умудряющее младенцы. Оправданія Господня права, веселящая сердце. Заповѣдь Господня свѣтла, просвѣщающая очи. Страхъ Господень чистъ, пребываяй во вѣкъ вѣка. Судьбы Господни истинны, оправданны вкупѣ, вожделѣнны паче злата и камене честна многа, и слаждша паче меда и сота}\footnote{Пс.~18,~8--11.}. \textit{Свидѣнія Твоя, Господи, увѣришася зѣло}\footnote{92,~5.}. Свидѣтельствуютъ преславная и чудная дѣла Божія о Бозѣ, которыя весь міръ видитъ, и со удивленіемъ на нихъ смотритъ. Кто всемірный потопъ навелъ, и посредѣ толикихъ водъ праведнаго Ноя съ домомъ его и съ нѣкоторыми животными въ ковчегѣ сохранилъ? Богъ. Кто различные и дикіе оные звѣри, въ ковчегѣ содержащіеся, укротилъ? Богъ. Кто Содомъ и Гоморръ съ окрестными градами огнемъ съ небесе пожеглъ, и праведнаго Лота оттуду восхитилъ? Богъ. Кто поразилъ язвами Египетъ? Богъ. Кто извелъ Израиля отъ работы того? Богъ. Кто раздѣлилъ Чермное море, и сквозѣ Его провелъ Израиля? Богъ. Кто потопилъ Фараона мучителя со всѣмъ воинствомъ въ мори томъ? Богъ. Кто провелъ Израиля пустынею страшною? Богъ. Кто питалъ толикое народа множество въ пустынѣ четыредесять лѣтъ? Богъ, Богъ, \textit{даяй пищу всякой плоти}\footnote{135,~25.}. Кто поразилъ предъ лицемъ ихъ великихъ царей, Сіона, царя аморрейска, и Ога, царя васанска? Богъ. Кто привелъ въ землю обѣтованную Израиля? Богъ Кто изъ земли той выгналъ и истребилъ языковъ, и поселилъ въ ней Израиля? Богъ. Кто сохранилъ отъ окрестныхъ языковъ живущаго въ той землѣ Израиля? Богъ. Что былъ Израиль противу всѣхъ языковъ окружающихъ его? Какъ единъ человѣкъ противу тысящи людей, или какъ два человѣка противу двадесяти тысящей, или какъ горсть воды противу моря. Но цѣлъ сохранился Израиль; не могли многотысящные народы, окружающіе его, истребить его. Кто его сохранялъ, аще не всесильный Богъ, Котораго онъ призывалъ и почиталъ? И сіе"=то есть, что Псаломникъ поетъ: \textit{яко аще не Господь бы былъ въ насъ, да речетъ убо Израиль: яко аще не Господь бы былъ въ насъ, внегда востати человѣкомъ на ны, убо живыхъ пожерли быша насъ; внегда прогнѣватися ярости ихъ на ны, убо вода потопила бы насъ}, и прочая\footnote{Пс.~123,~1--3.}. И хотя работалъ иногда Израиль иноплеменникамъ, но тое имъ было посылаемо отъ Бога въ наказаніе за грѣхи ихъ и беззаконія; ибо Богъ и въ своихъ людехъ, знающихъ и почитающихъ Его, грѣховъ и беззаконій ненавидитъ, и за тыя казнь имъ посылаетъ. Въ новой благодати дѣла, отъ Хріста Сына Божія сотворенная и въ святомъ Евангеліи написанная, извѣстное и твердое подаютъ свидѣтельство о Бозѣ, и, какъ мы на солнце перстомъ указуемъ, Бога показуютъ и въ познаніе намъ приводятъ. Таковыхъ преславныхъ и вышеестественныхъ дѣлъ преисполнено Евангеліе. Воду въ вино претворить, разслабленнаго \textit{словомъ} сдѣлать здоровымъ, слѣпому отъ рождества подать прозрѣніе, прокаженнаго очистить, демона изъ человѣка выгнать, мертваго воскресить, и прочая симъ подобная дѣла творить, никакой силѣ, обрѣтающейся въ поднебеснѣй, невозможно. Надобно здѣ быть и присутствовать всемогущей силѣ, которая все можетъ, и изъ ничего нѣчто, и изъ зла добро, изъ немощнаго здоровое и изъ тьмы свѣтъ, изъ мертваго живое и изъ не сущаго сущее, и изъ небытія въ бытіе творить, "--- такой, говорю, силѣ, которыя хотѣнію все повинуется и которая, что хощетъ, все творитъ, которой ничто не невозможно. Таковая сила есть Богъ нашъ, Который единъ все, что хощетъ, творитъ. И, да не кто о преславныхъ сихъ и чудесныхъ Божіихъ дѣлахъ усомнится, прописаны и самыя обстоятельства: гдѣ, въ какой странѣ, въ какомъ городѣ, или веси, на пути, или въ пустынѣ; когда, поутру, или въ вечеру; на комъ, и кто при томъ былъ, и что тому послѣдовало, что говорено, что дѣлалося, и проч. Всѣ сіи обстоятельства всякое сумнѣніе отъ человѣка отнимаютъ и убѣждаютъ за истину имѣть, что святое Евангеліе повѣствуетъ. Претворилъ Хрістосъ Господь воду въ вино. Гдѣ сіе чудо было? въ Канѣ галилейстѣй. Въ какомъ случаѣ? при бракѣ. Кто при томъ былъ? Пресвятая Мати Іисусова и ученики Его. Ради чего претворена вода въ вино? Понеже не достало вина. Кто о семъ Хрісту Господу, вся вѣдущему, доложилъ? Пресвятая Мати Его: \textit{вина не имутъ}. Что Ей Хрістосъ Господь отвѣщалъ? \textit{Что есть Мнѣ и тебѣ, жено?} и проч. Что потомъ Мати Іисуса сказала слугамъ? \textit{Еже аще глаголетъ} (Іисусъ) \textit{вамъ, сотворите}. Въ чемъ вода въ вино претворена? въ водоносахъ, наполненныхъ до верха. Каково вино показалося архитриклину, вино, бывшее отъ воды? доброе. Откуду, пригласивши жениха, сказалъ ему: \textit{всякъ человѣкъ прежде доброе вино полагаетъ, и егда упіются, тогда худшее: ты же соблюлъ еси доброе вино доселѣ}, и проч.\footnote{Іоан.~2,~1--11.} Такожде и въ прочихъ чудесныхъ Божіихъ дѣлахъ прописуются обстоятельства, которыя истину Евангельскую паче солнца яснѣйшую показуютъ, такъ что всякъ здравое имѣющій разсужденіе, убѣждается той вѣровать. Сего ради безотвѣтны будутъ, которыи святѣйшей Евангельской истинѣ не вѣруютъ, а паче тіи, которыи отъ хрістіанъ рождены и чрезъ крещеніе вступили въ хрістіанство: обличитъ бо ихъ собственная ихъ совѣсть. Сильно свидѣтельствуетъ, и аки гремитъ непрестанно всему міру обращеніе языковъ въ хрістіанскую вѣру. Дѣло сіе подлинно велико и чудно есть, и непрестанно во всѣхъ концахъ земли, на востокѣ и западѣ, на сѣверѣ и полудни, всѣмъ вопіетъ безгласно: \textit{есть Богъ}, Который все можетъ; \textit{есть Богъ}, Который изъ тьмы свѣтъ творитъ; \textit{есть Богъ}, Который о всей твари, а паче разумной, то"=есть, человѣкѣ, промышляетъ; \textit{есть Богъ}, Который хощетъ всѣмъ человѣкамъ спастися и въ разумъ истины пріити; \textit{есть Богъ}, Который послалъ Хріста Своего въ міръ обратити заблуждшихъ, взыскати и спасти погибшихъ, и привести къ Себѣ удалившихся человѣковъ, по Своему образу сотворенныхъ. Сіе великое и чудное дѣло, аки перстомъ, указуетъ на Бога \textit{всесильнаго, преблагаго, человѣколюбиваго, милосердаго, долготерпѣливаго и премудраго}, и непрестанная есть во всѣхъ концахъ земли о Хрістѣ, Спасителѣ міра, безгласная проповѣдь. Чудное дѣло! Язычники, хитростію, мечтаніями и привидѣніями діавольскими прельщенніи и утвержденніи въ нечестіи идолопоклонническомъ, и отъ прародителей своихъ тое содержащіи и въ томъ закоснѣвшіи и состарѣвшіися, и зная тое плотскому мудрованію сходно и угодно, оставили тое и за прелесть и ложь почли. Надобно быть здѣ \textit{Свѣту вышеестественному}, Который душевныя ихъ очи просвѣтилъ, и тьму и прелесть показалъ. Къ кому обратились язычники отъ своей прелести? ко Хрісту, отъ нихъ никогда неслыханному, отъ Іудей ожиданному, но отверженному и отъ нихъ пострадавшему, на крестѣ распятому и между двумя разбойниками повѣшенному и умершему, и въ третій день воскресшему, и вознесшемуся на небо. И того за истиннаго Бога своего признали, и всю надежду вѣчнаго блаженства въ Немъ единомъ положили и утвердили; инаго посредствія, кромѣ Его, къ вѣчному блаженству не признали, и Ему, яко истинному своему Богу, поклонилися и прославили. \textit{Сила распятаго Хріста} привлекла ихъ къ Нему, Который за всѣхъ умеръ. Признали единаго истиннаго Бога тіи, который въ многобожіи закоснѣли и состарѣлися, признали Бога \textit{единаго естествомъ}, но \textit{троичнаго въ лицахъ}; повѣрили небеснымъ и вышеестественнымъ тайнамъ, воскресенію мертвыхъ, небеснымъ и невидимымъ благимъ, и прочая; оставили плотоугодное и беззаконное житіе, суеты и пышности міра сего, и возлюбили смиренное и добродѣтельное житіе Хрістово, и, взявши всякъ свой крестъ, за Хрістомъ, Спасителемъ своимъ, пошли. \textit{Свѣтъ вышеестественный} въ сердцахъ ихъ блеснулъ, и привлеклъ въ слѣдъ Хріста. "--- Чрезъ кого безчисленніи народы, яко дикіи звѣри, въ вѣру хрістіанскую обращены? Чрезъ дванадесять апостоловъ. Чимъ обращены? не оружіемъ, но словомъ. Кто они были? невѣжи и препростіи. Уму человѣческому непонятно дѣло сіе! И не иное что, какъ \textit{сила распятаго Хріста} все сіе дѣйствовала. Сколько діаволъ препятствія полагалъ растущей вѣрѣ хрістіанской, и чрезъ служителей своихъ то проповѣдникамъ слова, то прочіимъ вѣрнымъ, какихъ ни изобрѣталъ мученій, но ничего не успѣлъ: разсыпался весь совѣтъ его злый \textit{силою распятаго Хріста Бога нашего}. Чимъ болѣе мучили и убивали хрістіанъ, тѣмъ болѣе умножалось число ихъ; и премногіи изъ нихъ на мученія за Хріста, какъ на пресладкій пиръ, спѣшили и сквозѣ огнь и воду входили въ вѣчный покой. Удивленія достойно дѣло! Видѣли люди тягчайшая мученія, каковыя тогда были хрістіанамъ; но дерзновенно шли въ вѣру хрістіанскую и на вся дерзали мученія. Вѣчныхъ благихъ не видѣли, но къ нимъ, аки видимымъ стремились, и другъ друга къ подвигу тому и вѣчнымъ онымъ благимъ поощряли. Не человѣческое, но Божіе тое дѣло было; \textit{сила Божія} тое дѣйствовала. Богъ, Котораго они познали и прославляли, подавалъ имъ свѣтъ Свой, крѣпость и силу Свою, по писанному: \textit{Господь крѣпость людемъ своимъ дастъ}\footnote{Пс.~28,~11.}. И тако вси навѣты вражіи и тягчайшая мученія одолѣли и побѣдили. И сіе"=то есть, что Хрістосъ Спаситель нашъ прежде страсти Своея изреклъ: \textit{аще Азъ вознесенъ буду отъ земли, вся привлеку къ Себѣ}\footnote{Іоан.~12,~32.}. И паки: \textit{ины овцы имамъ, яже не суть отъ двора сего: и тыя Ми подобаетъ привести, и гласъ Мой услышатъ: и будетъ едино стадо и единъ Пастырь}\footnote{10,~16.}. Привлеклъ къ Себѣ всѣхъ заблуждшихъ овецъ Своихъ, и покланяются Ему со Отцемъ и Святымъ Духомъ всѣ концы земли. \textit{Благословенъ Господь Богъ Израилевъ, яко посѣти и сотвори избавленіе людемъ Своимъ, и воздвиже рогъ спасенія намъ, въ дому Давида отрока Своего}\footnote{Лук.~1,~68 и 69.}. Видишь, хрістіанине, сколько свидѣтелей о Бозѣ, которыи непрестанно увѣщаваютъ вѣровать, почитать и славить Создателя нашего. Почтимъ убо Его и мы вѣрою, страхомъ, любовію, покореніемъ и послушаніемъ; познаемъ Его, яко Создателя, Бога, Господа, Промыслителя и высочайшаго нашего Благодѣтеля и едино наше Блаженство и высочайшее Добро, да и Онъ познаетъ и признаетъ насъ за людей Своихъ. Въ семъ бо состоитъ и животъ вѣчный. \textit{Се есть животъ вѣчный, да знаютъ Тебе единаго истиннаго Бога, и Егоже послалъ еси Іисусъ Хріста}, глаголетъ Господь\footnote{Іоан.~17,~3.}.

\section{129. Когожъ мнѣ и любить, какъ не Его?}

Слышимъ, что единъ о другомъ слово сіе говоритъ: \textit{когожъ мнѣ и любить, какъ не Его?} Слово сіе говорится отъ того, которому другій говоритъ: знать, что ты любишь его, то"=есть, такого"=то человѣка. Тогда любитель отвѣщаетъ: \textit{когожъ мнѣ и любить, какъ не его?} Хрістіанине! намъ слово сіе о Бозѣ нашемъ сердечно говорить приличествуетъ: \textit{когожъ мнѣ и любить, какъ не Его?} Богъ есть существенное, безначальное, безконечное, непостижимое, вѣчное, непремѣняемое, высочайшее и прелюбезное Добро, отъ Котораго, яко источника, вся благая, видимая и невидимая, на небеси и на земли, происходятъ. Когожъ убо и любить намъ, какъ не Его? Всякъ бо добро любитъ, и чимъ большее добро, тѣмъ болѣе любитъ. И хотя зло люди любятъ, но любятъ не яко зло, но подъ видомъ добра. \textit{Никтоже благъ, токмо единъ Богъ}\footnote{Мѳ.~19,~17.}. \textit{Когожъ убо и любить намъ, какъ не Его единаго?} Какъ огнь всегда горячъ есть, и всегда грѣетъ; какъ свѣтъ всегда свѣтъ есть, и свѣтитъ; какъ медъ всегда сладокъ есть, и услаждаетъ: такъ Богъ всегда и непрестанно благъ есть и всегда благотворитъ. И какъ огнь не можетъ не согрѣвать, и свѣтъ не можетъ не просвѣщать, и медъ не можетъ не услаждать: такъ Богъ не можетъ не благотворить. Естество бо Его такое есть, чтобы благотворить. Отъ Него единаго всякое добро, какое ни есть и можетъ быть; отъ Него утѣшеніе, радость, веселіе и блаженство происходитъ. Безъ Него истинное блаженство, утѣшеніе, радость и веселіе быть не можетъ. Богъ и тогда благотворитъ намъ, когда наказуетъ насъ. Ибо наказуетъ насъ, да исправитъ насъ; біетъ насъ, да помилуетъ насъ; сокрушаетъ насъ, да исцѣлитъ насъ; опечаляетъ насъ, да увеселитъ насъ. Все бо творитъ, да ублажитъ насъ. \textit{Когожъ убо и любить намъ, какъ не Его?} Богъ есть Создатель нашъ. Отъ Того мы начало и бытіе свое имѣемъ. Душа и тѣло и весь составъ нашъ отъ Него есть. Рукъ Его дѣло мы. Не было насъ, и се есмы и живемъ. Всемогущая сила Его изъ небытія въ бытіе привела насъ. \textit{Руцѣ Твои сотвористѣ мя, и создастѣ мя}\footnote{Пс.~118,~73.}. Создалъ Онъ насъ не безчувственною, не безсловесною, но разумною тварію, разумомъ насъ почтилъ. И, что болѣе того всего, по образу Своему и по подобію насъ сотворилъ. О любезное и высокопочтенное созданіе, \textit{человѣкъ!} Образъ Божій, яко печать и знамя небеснаго Царя, въ себѣ носитъ. Такъ высоко почтилъ насъ, о человѣки, преблагій Создатель нашъ! Не объ ангелахъ, ни о иной какой твари, но о человѣкѣ сказано: \textit{сотворимъ человѣка по образу Нашему и по подобію}\footnote{Быт.~1,~26.}. \textit{Когожъ убо и любить намъ, какъ не Его?} Богъ есть высочайшій нашъ благодѣтель, и столько благодѣяній намъ показуетъ, сколько дыхаемъ мы, такъ что и минуты безъ благихъ Его жить не можемъ мы. Окружены и заключены мы въ любви Божіей и благихъ Его. Отъ него пищу, отъ Него питіе, отъ Него одежду, отъ Него домъ получаемъ мы; Его свѣтъ свѣтитъ намъ и показуетъ путь, добро и зло, пользу и вредъ; Его огнь согрѣваетъ насъ, и варитъ пищу нашу; воздухомъ Его сохраняется животъ нашъ; Его повеленіемъ работаютъ намъ скоти. Какое бы житіе наше было, когда бы хотя нѣкоторая отъ благихъ Своихъ отнялъ Богъ у насъ? Что бы намъ пользовали глаза наши, когда бы свѣтъ Свой отнялъ Богъ у насъ? блудили бы, какъ слѣпіи. Подала ли бы земля плоды своя намъ, аще бы Богъ дождя Своего не низпослалъ на нее? \textit{Ибо Господь дастъ благость, и земля наша дастъ плодъ свой}\footnote{Пс.~84,~13.}. Безъ воздуха и минуты не можетъ жить человѣкъ. Сія и прочая безчисленная видимая суть Божія благодѣянія, которая Онъ подаетъ намъ отъ единой Своей любви къ намъ. \textit{Когожъ убо и любить намъ, какъ не Его?} Кто благодѣтеля своего не любитъ? развѣ безумный и неблагодарный. "--- Діаволъ съ демонами своими непрестанно ополчается противу насъ и окружаетъ насъ невидимо, и, \textit{яко левъ рыкая, ходитъ, искій кого поглотити}\footnote{1~Петр.~5,~8.}. Кто бы отъ насъ сохраниться моглъ отъ сего всегубительнаго врага, аще бы всемогущая десница Божія не хранила насъ? Богъ насъ защищаетъ и сохраняетъ отъ него, у Котораго въ руцѣ вси концы земли; и ангели Его святіи \textit{ополчаются окрестъ насъ}\footnote{Пс.~33,~8; Быт.~32,~1; Евр.~1,~14.}. И хотя искушаетъ насъ врагъ нашъ, и терпимъ противу воли нашей нападеніе и мучительство его; но тое бываетъ попущеніемъ Божіимъ на пользу нашу, якоже тое искусившіяся познаютъ. И столько попущаетъ ему Богъ, сколько мы можемъ и намъ полезно\footnote{1~Кор.~10,~13.}. Сіе Божіе высочайшее есть благодѣяніе, но, понеже невидимо есть, мало отъ кого познается. Тако много получаемъ Божіихъ благодѣяній на всякъ день и часъ, которыхъ не видимъ и часто не знаемъ. И кто можетъ ихъ узнать и исчислить? Какъ рѣка непрестанно течетъ, и живый источникъ непрестанно источаетъ воду: тако благодѣянія Божія непрестанно на насъ изливаются. Таковый любитель и благодѣтель нашъ Богъ есть! \textit{Богъ спасеній нашихъ, Богъ нашъ, Богъ еже спасати}\footnote{Пс.~67,~20--21.}. \textit{Когожъ убо и любить намъ, какъ не Его}; толикаго благодѣтеля и любителя? "--- Хотя и весь міръ свидѣтельствуетъ о Бозѣ и Его намъ въ познаніе приводитъ, и какое почитаніе разумная тварь Творцу своему должна отдавать, безъ гласа увѣщаваетъ насъ, однакожъ умъ нашъ, тьмою грѣха помраченный, слѣпотствуетъ въ познаніи Бога и истины и святѣйшія Божія воли. Сего ради преблагій и человѣколюбивый Создатель нашъ подалъ намъ чрезъ избранныхъ рабовъ Своихъ слово Свое святое. Тое, яко свѣтильникъ, свѣтитъ намъ и открываетъ Бога и Божественныя свойства Его, и святую волю Его, и небесныя тайны Его; и показуетъ намъ, что есть добро и что есть зло, что есть истина и прелесть, что есть добродѣтель и порокъ, что мы Господу Богу нашему и другъ другу человѣки должны, и тако насъ къ совершенству и блаженству нашему ведетъ. \textit{Всяко бо писаніе богодухновенно и полезно есть ко ученію, ко обличенію, ко исправленію, къ наказанію, еже въ правдѣ: да совершенъ будетъ Божій человѣкъ, на всякое дѣло благое уготованъ}\footnote{2~Тим.~3,~16 и 17.}. Держащіися сего Богодухновеннаго свѣтильника, и тому внимающіи, во свѣтѣ ходятъ; удаляющіеся отъ него во тьмѣ пребываютъ и блудятъ, яко слѣпіи. Надобно бо неотмѣнно во тьмѣ быть тому, кто отъ свѣта удаляется. Божіе слово учитъ насъ любить Бога отъ всего сердца нашего, и ближняго яко себе. \textit{Когожъ убо любить намъ, какъ не Его}, тако о насъ промышляющаго и пекущагося? "--- \textit{Тако возлюби Богъ міръ, яко и Сына Своего Единороднаго далъ есть, да всякъ вѣруяй въ Онь не погибнетъ, но имать животъ вѣчный}\footnote{Іоан.~3,~16.}. Погибли мы, о человѣки! Богъ, милосердый Создатель нашъ, такъ насъ возлюбилъ, насъ, недостойныхъ любве, и такъ чудный о насъ промыслъ показалъ, что Сына Своего Единороднаго послалъ къ намъ, взыскати и спасти насъ. Слава человѣколюбію Его! Сія Его любовь къ намъ убѣждаетъ и насъ Его любить. \textit{Когожъ убо и намъ любить, какъ не Его}? Богъ Отецъ нашъ есть. Онъ съ нами, яко чадами Своими, въ словѣ Своемъ святомъ бесѣдуетъ; и въ сердцахъ нашихъ чувствуемъ сладчайшую бесѣду Его; и мы къ Нему молимся и вопіемъ: \textit{Отче нашъ, Иже еси на небесѣхъ}, и проч. Онъ о насъ промышляетъ, печется, воспитываетъ насъ, вѣчное наслѣдіе готовитъ намъ. \textit{Когожъ убо и намъ любить, какъ не Его, Отца своего}? Какое бо чадо отца своего не любитъ, развѣ юродивое и неблагодарное? Богъ и въ семъ великую къ намъ являетъ любовь и милость, что наказуетъ насъ: \textit{егоже бо любитъ Господь, наказуетъ; біетъ же всякаго сына, егоже пріемлетъ. Судими бо, отъ Господа наказуемся, да не съ міромъ осудимся}\footnote{Евр.~12,~6; 1~Кор.~11,~32.}. Тѣхъ ли"=де любитъ Господь, которыхъ предаетъ въ толикая страданія? Воистину любитъ! Какій отецъ дѣтей своихъ не наказуетъ, и когда наказуетъ, не любитъ ихъ? Никакъ: для того и наказуетъ, что любитъ, наказуетъ ихъ, но имъ наслѣдіе готовитъ. Тако и Богъ наказуетъ насъ, но вѣчнаго живота и царствія наслѣдіе готовитъ намъ. Наказаны пророки, наказаны апостоли, наказаны мученики, наказаны и вси святіи; но были возлюбленная Божія чада, и нынѣ въ дому небеснаго своего Отца водворяются, и въ вѣки вѣковъ восхвалятъ Его. Кто бы не пожелалъ съ ними тамо быть, съ тѣми, которыи такъ наказаны здѣ? Наказуетъ насъ Богъ на пользу нашу, \textit{да причастимся святыни Его}\footnote{Евр.~12,~10.}. \textit{Блаженъ убо человѣкъ, егоже аще накажеши, Господи}\footnote{Пс.~93,~12; Іов.~5,~17.}. Великую убо любовь являетъ намъ Господь, когда наказуетъ насъ. \textit{Когожъ убо и любить намъ, какъ не Его}, и за отеческое Его наказаніе Ему благодарить? Человѣче, аще бы увидѣлъ ты славу тѣхъ, которіи здѣ наказаніе отъ руки Господни приняли, "--- безъ сумнѣнія всякое наказаніе Божіе облобызалъ бы, и желалъ бы болѣе страданія, нежели благоденствія. Истинно есть Божіе слово: \textit{егоже любитъ Господь, наказуетъ}. Когда болѣе человѣкъ развращается и погибаетъ, какъ когда въ благополучіи живетъ? Богъ преблагій, наказуя насъ, до того насъ не допущаетъ. Слава дивному Его о насъ промыслу! \textit{Когожъ убо и любить намъ, какъ не Его}, и за сей спасительный Его о насъ промыслъ благодарить Ему? Но писано есть: \textit{не любите міра, ни яже въ мірѣ. Аще кто любитъ міръ, нѣсть любве Отчи въ немъ: яко все, еже въ мірѣ, похоть плотская, и похоть очесъ, и гордость житейская, нѣсть отъ Отца, но отъ міра сего есть}\footnote{1~Іоан.~2,~15 и 16.}. И паки: \textit{имѣяй заповѣди Моя и соблюдаяй ихъ, той есть любяй Мя}. И нижше: \textit{не любяй Мя, словесъ моихъ не соблюдаетъ}, глаголетъ Господь\footnote{Іоан.~14,~21 и 24.}. И паки: \textit{любы міра сею вражда Богу есть; иже бо восхощетъ другъ быти міру, врагъ Божій бываетъ}\footnote{Іак.~4,~4.}. Хрістіанине! разсуждай сія Божія слова, и осмотрись, да не вмѣсто любителя Божія между врагами Божіими будеши. Неотмѣнно въ томъ числѣ находится человѣкъ, который безстрашно заповѣди Божія нарушаетъ, и сердцемъ міру прилѣпляется, хотя и имя Божіе исповѣдуетъ. Писано бо о таковыхъ: \textit{Бога исповѣдаютъ вѣдѣти, а дѣлы отмещутся Его: мерзцы суще и непокориви, и на всяко дѣло благое неискусны}\footnote{Тит.~1,~16.}. О, колико враговъ Божіихъ и между тѣми, которіи мнятся Его почитать, которіи въ храмы входятъ и молятся Ему, и животворящихъ Таинъ Хрістовыхъ причащаются, и проч. \textit{Врази Господни солгаша Ему}\footnote{Пс.~80,~16.}. Писано сіе о Іудеяхъ беззаконнующихъ; но тое слово приличествуетъ и хрістіанамъ, которіи святымъ крещеніемъ омылися, освятилися, оправдалися и обѣщалися Господу Богу вѣрою и правдою работать, но уклоняются въ слѣдъ похотей своихъ. Таковыи суть блудники и прелюбодѣи, сквернители, хищники, тати, воры, насильники, ругатели, лживіи, клятвопреступники и прочіи беззаконники, и вси въ пышности и гордости міра сего живущіи. Не вниди, душе моя, въ совѣтъ ихъ. \textit{Не Богу ли повинется душа моя? отъ Того бо спасеніе мое. Ибо Той Богъ мой, и спасъ мой, заступникъ мой}\footnote{61,~2.}. О Боже, вѣчная любы и благосте! сподоби мене любити Тебе, Тебе, Который мене создалъ, Который мене падшаго искупилъ, заблудшаго обратилъ, словомъ Своимъ святымъ просвѣтилъ, Который питаешь мене, одѣваешь мене, заступаешь мене, сохраняешь мене отъ навѣтовъ вражіихъ, и прочая безчисленная благая изливаешь на мене, "--- сподоби мене любити Тебе, высочайшее добро и блаженство мое, и отъ любви славити Тя, благодарити Тебѣ, угождати Тебѣ, творити святую волю Твою, \textit{духомъ и истиною} покланятися Тебѣ, пѣти и хвалити Тебе, Котораго поютъ и хвалятъ непрестанно ангели на небеси, да и я сердечно со пророкомъ пою Тебѣ радостнымъ духомъ: \textit{возлюблю Тя, Господи, крѣпосте моя! Господь утвержденіе мое, и прибѣжище мое, и избавитель мой, Богъ мой, помощникъ мой, и уповаю на Него, защититель мой, и рогъ спасенія моего, и заступникъ мой}\footnote{Пс.~17,~2 и 3.}.

\section{130. Сосудъ полный и праздный.}

Что сосудъ есть, тое есть сердце человѣческое. Сердце подобно есть сосуду. Сосудъ полный и исполненный водою или другимъ чимъ, ничего инаго въ себе не вмѣщаетъ. Напротивъ того, сосудъ праздный удобенъ есть къ воспріятію всего. Сего ради люди испражняютъ сосудъ, когда другое что хотятъ въ него влить или положить. Тако и сердце человѣческое имѣется. Когда праздное есть и не имѣетъ въ себѣ прихотей мірскихъ и плотскихъ, удобно есть къ воспріятію Божія любве; а когда любовію міра сего и плотскими похотьми и грѣховными пристрастіями наполнено, тогда любовь Божія въ него вмѣститься не можетъ. Сребролюбіемъ, самолюбіемъ, славолюбіемъ, сластолюбіемъ, гнѣвомъ и памятозлобіемъ, завистію, гордостію и прочими беззаконными пристрастіями исполненное сердце, како можетъ вмѣстить въ себе любовь Божію? \textit{Кое бо причастіе свѣту со тьмою?}\footnote{2~Кор.~6,~14.} Любовь Божія высочайшее есть Божіе дарованіе, и тамо вмѣщается, гдѣ сердце истиннымъ покаяніемъ, сокрушеніемъ и жалѣніемъ испражнено отъ злыхъ пристрастій и грѣховныхъ обычаевъ, и очищено и уготовано есть къ воспріятію того небеснаго дара. Аще убо хощемъ, хрістіанине, чтобы любовь Божія въ сердце наше вселилась, испразднимъ тое отъ любви міра сего и того прихотей и грѣховныхъ обычаевъ, и обратимъ сердце наше къ Богу, единому существенному нашему добру и блаженству, и вѣчнымъ благимъ: тогда сладчайшая Божія любовь вселится въ сердца наша, и тогда \textit{вкусимъ и увидимъ, яко благъ Господь}\footnote{Пс.~33,~9.}. Якоже бо горесть, тако и сладость вещи отъ вкуса познается. Коль сладокъ медъ, тіи только познаютъ, которыи вкушаютъ его; коль сладокъ Богъ и любовь Его святая, тіи только познаютъ, которыи любятъ Бога, и чувствуютъ въ сердцахъ своихъ сладчайшую любовь Его. \textit{Имѣяй ухо слышати, да слышитъ, что Духъ глаголетъ церквамъ: побѣждающему дамъ ясти отъ манны сокровенныя}, и проч.\footnote{Апок.~2,~17.}

\section{131. Кто что любитъ, того и ищетъ.}

Видимъ, что люди что любятъ, того и ищутъ со тщаніемъ, и чимъ болѣе что любятъ, тѣмъ усерднѣе того и ищутъ. Кто любитъ богатство, богатства и ищетъ; кто любитъ честь и славу, тотъ чести и славы и ищетъ. А кто чего ищетъ, тотъ посредствіе и способъ къ снисканію желаемаго удобный употребляетъ, и всякаго препятствія уклоняется. Тако купцы, хотячи собрать богатство, по чужимъ странамъ скитаются и ѣздятъ, и всякихъ препятствій, желанію ихъ противныхъ, уклоняются и убѣгаютъ. Тако кто истинно Бога, высочайшее оное добро, любитъ, Того со всякимъ усердіемъ и прилѣжаніемъ ищетъ, и посредствіе и способъ къ тому употребляетъ, и всякаго препятствія, которымъ желаніе его пресѣкается, убѣгаетъ. Препятствіе, которымъ къ снисканію вѣчнаго онаго добра препинаемся, есть грѣховный обычай и любовь суетнаго міра. Сего ради, когда хощемъ Бога любить и Его искать, должно отъ грѣховъ отвратиться и покаяться, и міръ съ прелестію и суетою оставить, и тако \textit{единаго} Бога желать и искать. Хрістіанине! когда хощешь Бога любить, ненавиди грѣхъ, Богу противный и ненавидимый. Грѣхъ и Бога любить никакъ невозможно. Когда хощешь Бога искать, то не ищи въ мірѣ семъ ни чести, ни славы, ни богатства, ни прочихъ угодій; все сіе оставить надобно ищущему Бога; надобно отъ всего сего сердце испразднить, чтобы въ немъ мѣсто было Богу "--- вѣчному добру съ любовію Его святою. Давидъ святый ничего не желалъ ни на небеси, ни на земли, кромѣ единаго Бога; почему и воспѣлъ пресладкую пѣснь: \textit{что ми есть на небеси? и отъ Тебе что восхотѣхъ на земли? Исчезе сердце мое и плоть моя: Боже сердца моего, и часть моя Боже во вѣкъ}\footnote{Пс.~72,~25 и 26.}. Тако своимъ примѣромъ научаетъ насъ пророкъ Божій, ничего не желать и не искать, кромѣ единаго Бога, когда хощемъ Его искать и сыскать. Способъ и посредствіе, которымъ Богъ ищется, есть наипаче истинное сердечное смиреніе и усердная молитва Когда Богъ видитъ сія въ человѣкѣ, тогда приходитъ къ нему съ сладчайшею Своею любовію и дѣлаетъ сердце его храмомъ Своимъ. \textit{Аще кто любитъ Мя, слово Мое соблюдетъ; и Отецъ Мой возлюбитъ его, и къ нему пріидемъ, и обитель у него сотворимъ}, глаголетъ Господь\footnote{Іоан.~14,~23.}. \textit{Не оставилъ еси взыскающихъ Тя, Господи}\footnote{Пс.~9,~11.}. \textit{Богатіи обнищаша и взалкаша: взыскающіи же Господа не лишатся всякаго блага}\footnote{33,~11.}. \textit{Да возвеселится сердце имущихъ Господа. Взыщите Господа и утвердитеся; взыщите лица Его выну}\footnote{104,~3 и 4.}. Аще бы кто сказалъ, гдѣ Бога искать, въ пустынѣ ли, или въ монастырѣ, или во Іерусалимѣ, или въ иномъ какомъ мѣстѣ? \textit{Отвѣтъ}: Тамо ищи, гдѣ находишися и живеши. Богъ на всякомъ мѣстѣ присутствуетъ, и нѣтъ такого мѣста, гдѣ бы Бога не было; вездѣ есть и вся исполняетъ, хотя намъ и невидимо и непостижимо присутствіе Его. Тако поетъ Псаломникъ: \textit{камо пойду отъ Духа Твоего? и отъ лица Твоего камо бѣжу? Аще взыду на небо, Ты тамо еси; аще сниду во адъ, тамо еси; аще возму крилѣ мои рано, и вселюся въ послѣднихъ моря, и тамо бо рука Твоя наставитъ мя, и удержитъ мя десница Твоя}\footnote{138,~7 и 10.}. Убо Вездѣприсутствующаго вездѣ и искать можешь. Только вездѣ внимай себѣ, и не послѣдуй тѣмъ, о которыхъ написано: \textit{не предложиша Бога предъ собою}\footnote{53,~5.}; но послѣдуй Псаломнику, который о себѣ глаголетъ: \textit{предзрѣхъ Господа предо мною выну}\footnote{15,~8.}. И отъ всякаго грѣха уклоняйся, и міръ съ своею прелестію и суетою позади себе оставляя, на единаго Бога взирай, и Его \textit{всѣхъ желаній твоихъ конецъ полагай}, и сердцемъ сокрушеннымъ и смиреннымъ призывай. Тогда узнаешь и внутрь себе почувствуешь, что близъ тебе Онъ есть, и воспоешь Ему съ пророкомъ: \textit{близъ еси Ты, Господи, и вси путіе Твои истина}\footnote{Пс.~118,~151.}. \textit{Близъ бо Господь всѣмъ призывающимъ Его во истинѣ}\footnote{144,~18.}. И паки: \textit{близъ Господь сокрушеннымъ сердцемъ}\footnote{33,~19.}. Однакожъ нигдѣ удобнѣе не ищется Богъ, какъ во уединеніи и покоѣ. Въ народѣ молва и соблазны, которыи препятствіе полагаютъ боголюбивой душѣ; во уединеніи тишина и миръ: тутъ очи не видятъ и уши не слышатъ ничего, чимъ бы боголюбивая душа возмутиться могла. Откуду нѣкоторый святый отецъ премудро говорилъ, когда его вопрошали другіи: ради чего ты отъ насъ убѣгаешь? Онъ отвѣщалъ: \textit{не могу купно быть съ Богомъ и человѣками}. Сего ради ищущему Бога должно держаться уединенія; и когда нужду имѣетъ изыти въ народъ, вездѣ должно внимать себѣ, и предъ душевными очами Бога вездѣ присутствующаго предлагать, дабы въ чемъ предъ Нимъ не проступиться и Его не оскорбить. \textit{Блюдите, како опасно ходите, не якоже немудри, но якоже премудри}\footnote{Еф.~5,~15.}.

\section{132. Познанное добро ищется.}

Видимъ, что люди, когда добро познаютъ, и ищутъ того. Тако познаютъ богатство добро себѣ, и ищутъ его; познаютъ хлѣбъ добро себѣ, и ищутъ его; познаютъ одѣяніе добро себѣ, и ищутъ его; познаютъ домъ добро себѣ, и ищутъ его; познаютъ свѣтъ добро себѣ, и ищутъ его; познаютъ искусство, художество, разумъ и мудрость добро себѣ, и ищутъ тѣхъ, и прочая. И чемъ большее добро познаютъ, тѣмъ усерднѣе и ищутъ того. А непознанное добро не ищется никогда; како бо искать того, чего не знаемъ? Сего ради надобно прежде познать добро, и тогда его искать. Тако Богъ "--- высочайшее, естественное и вѣчное добро познанное ищется; и чимъ кто болѣе познаетъ Его, тѣмъ усерднѣе и прилежнѣе ищетъ Его, и ничего ни на земли, ни на небеси, кромѣ Его не желаетъ и не ищетъ, поя съ пророкомъ въ сердцѣ своемъ: \textit{что ми есть на небеси? и отъ Тебе что восхотѣхъ на земли?} и проч.; но, все оставивши позади себе, жаждущимъ духомъ спѣшитъ къ Нему. \textit{Имже образомъ желаетъ елень на источники водныя: сице желаетъ душа моя къ Тебѣ, Боже. Возжада душа моя къ Богу крѣпкому, живому: когда пріиду и явлюся лицу Божію}\footnote{Пс.~41,~2 и 3.}? Хрістіанине! когда хощемъ Бога искать (а неотмѣнно должно), то надобно прежде Его познать, и тогда искать; и чимъ болѣе Его будемъ познавать, тѣмъ усерднѣе Его будемъ искать.

Великое и драгоцѣнное добро съ великимъ тщаніемъ и усердіемъ ищется. Сребро отъ любителей своихъ съ великимъ усердіемъ ищется, злато съ большимъ, камень драгоцѣнный далеко съ большимъ тщаніемъ ищется. Богъ паче всѣхъ созданій несравненно лучшее и превосходнѣйшее добро есть, яко всѣхъ Создатель. Вси созданія добры суть, и \textit{добры зѣло}; но несравненно и превосходно лучшій Самъ Тотъ, отъ Котораго созданія добрыми сдѣлались. Свѣтелъ воздухъ, но далеко свѣтлѣйшее солнце, отъ котораго воздухъ просвѣщается. Сладка вода, медомъ растворенная; но сладчайшій самъ медъ, который воду услаждаетъ. Тако и Богъ несравненно лучшій свѣтъ имѣетъ, нежели солнце, которому свѣтъ даровалъ; и несравненно лучшую доброту и сладость имѣетъ нежели всѣ созданія, которыхъ добрыми, сдѣлалъ. Насадилъ Богъ любовь и милосердіе въ сердцахъ человѣческихъ; но Самъ несравненно большую любовь и милосердіе имѣетъ. Якоже величество Его, тако любовь и милосердіе Его. И не токмо имѣетъ любовь и милосердіе, но и есть самая любовь и милосердіе; не токмо благъ, но и есть самая благость; не токмо сладокъ, но и есть самая сладость; не токмо премудръ, но и есть самая премудрость; не токмо красенъ, но и есть самая красота, всѣхъ красотъ краснѣйшая, и прочая. Но писано есть: \textit{вкусите и видите, яко благъ Господь}\footnote{Пс.~33,~9.}. Медъ отъ вкуса, и добровонный цвѣтъ отъ уханія познается; тако благость и доброта Божія отъ вкушенія и ощущенія познавается. Двѣ книги суть, изъ которыхъ Богъ познается: 1)~\textit{книга естества}, то"=есть міръ, небо и земля съ исполненіемъ ихъ. Смотрящему на небо и землю и прочая созданія Божія, и сія прилѣжно разсуждающему, не можно не пріити въ познаніе Создателя и Бога. Откуду и самыи язычники отъ разсужденія сего приходили въ познаніе Бога, и коль великъ есть Богъ, который такъ великій міръ создалъ, познавали. \textit{Небеса повѣдаютъ славу Божію}, поетъ пророкъ\footnote{18,~2.}. Надобно быть великому, который такъ великую машину изъ ничего сотворилъ. Надобно быть премудрому, который премудро все сотворилъ: \textit{вся премудростію сотворилъ еси}\footnote{103,~24.}. Надобно быть доброму, который вся добрая сотворилъ. Подалъ свѣтъ солнцу, лунѣ и звѣздамъ: Самъ въ Себѣ несравненно лучшій свѣтъ есть. \textit{Богъ свѣтъ есть, и тьмы въ Немъ нѣсть ни единыя; одѣяйся свѣтомъ, яко ризою}\footnote{1~Іоан.~1,~5; Пс.~103,~2.}! Сдѣлалъ всѣ вещи добрыми: Самъ въ Себѣ несравненно лучшій есть. Подалъ разумъ человѣку: Самъ несравненно лучшій разумъ имѣетъ. Подалъ уши и очи животнымъ: Самъ несравненно лучшее видѣніе и слышаніе имѣетъ. \textit{Насаждей ухо, не слышитъ ли? или создавый око, не сматряетъ ли?}\footnote{99,~9.} Подалъ силу питательную хлѣбу: Самъ несравненно лучшую имѣетъ. \textit{Не о хлѣбѣ единомъ живъ будетъ человѣкъ, но о всякомъ глаголѣ, исходящемъ изо устъ Божіихъ}\footnote{Мѳ.~4,~4.}. Онъ силенъ есть безъ хлѣба и пищи питать насъ далеко лучше, нежели хлѣбъ питаетъ. Подалъ силу цѣлительную зелію и травѣ: Самъ несравненно лучшую имѣетъ, Который всѣхъ \textit{словомъ} исцѣляетъ. Прохлаждаетъ и оживляетъ насъ вода жаждущихъ; но Самъ Онъ несравненно лучше прохлаждаетъ и оживляетъ насъ. \textit{Отвѣща Іисусъ и рече ей: всякъ піяй отъ воды сея, вжаждется паки: а иже піетъ отъ воды, юже Азъ дамъ ему, не вжаждется во вѣки; но вода, юже Азъ дамъ ему, будетъ въ немъ источникъ воды, текущія въ животъ вѣчный}\footnote{Іоан.~4,~13 и 14.}. Согрѣваетъ насъ огнь Его; но самъ Онъ далеко лучше согрѣваетъ насъ. Онъ согрѣлъ сердце Клеопы и товарища его, когда съ ними, яко путникъ, шествовалъ и глаголалъ имъ на пути; откуду и сами призналися: \textit{не сердце ли наше горя бѣ въ насъ, егда глаголаше нама на пути, и яко сказоваше нама Писанія}\footnote{Лук.~24,~32.}? Услаждаетъ медъ насъ; но несравненно лучше Самъ Онъ услаждаетъ насъ, когда благости и любви Его сладость чувствуемъ въ сердцахъ нашихъ, и проч. Добро убо и весьма добро созданіе Божіе; но Самъ Создатель несравненно лучшій\footnote{Быт.~1,~31; 1~Тим.~4,~4.}. Якоже убо изъ книги разумъ сочинителя, и изъ зданія мудрость архитектора, и изъ теплоты огнь познается; тако изъ созданій познается Создатель. "--- 2)~Лучшій и совершеннѣйшій способъ, которымъ Богъ познается, есть \textit{книга Священнаго Писанія}. Сія книга представляетъ намъ и воображаетъ Бога съ божественными Его свойствами. Тамо представляется намъ, что Тойжде Богъ, Который весь міръ создалъ, и въ созданномъ мірѣ преславная дѣла сотворилъ и творитъ. Но истинно Бога безъ Бога познать не можемъ. Что Богъ благъ есть, есть всемогущій, есть праведенъ, есть истиненъ, есть вездѣсущій, и проч. "--- надобно человѣку въ сердцѣ почувствовать и \textit{вкусить и видѣть, яко благъ Господь}; тогда нѣсколько Богъ познается. Откуду глаголетъ Господь: \textit{никтоже знаетъ Сына, токмо Отецъ; ни Отца кто знаетъ, токмо Сынъ, и емуже аще волитъ Сынъ открыти}\footnote{Мѳ.~11,~27.}. И пророкъ святый молится: \textit{открый очи мои, и уразумѣю чудеса отъ закона Твоего}\footnote{Пс.~118,~18.}. Чтобы очи наши видѣли созданныя вещи, нуженъ есть естественный свѣтъ: чтобы разумъ нашъ моглъ познать Бога и видѣть, нужно, дабы блеснулъ вышеестественный свѣтъ. Богъ сокровенный есть и \textit{во свѣтѣ живетъ неприступномъ}\footnote{1~Тим.~6,~17.}; но открываетъ и показуетъ Себе тѣмъ, которыи любятъ Его, и суть смиренни и просты, яко младенцы; а отъ премудрыхъ и совопросниковъ вѣка сего сокрывается. Откуду глаголетъ Господь: \textit{утаилъ еси сія отъ премудрыхъ и разумныхъ, и открылъ еси та младенцемъ}\footnote{Мѳ.~11,~25.}. Хрістіанине! познаемъ Бога, и неотмѣнно, все оставивши, Его единаго будемъ искать съ любовію и прилѣжаніемъ, и ничего ни на земли, ни на небеси, кромѣ Его единаго, не будемъ желать и искать. Онъ единъ и здѣ и въ будущемъ вѣкѣ совершенное наше будетъ блаженство.

\section{133. Всякая вещь въ своемъ мѣстѣ упокоевается.}

Видимъ, что вси вещи въ сродныхъ себѣ мѣстахъ упокоеваются. Огнь къ верху идетъ, и тамо покой себѣ обрѣтаетъ; воздухъ въ воздухѣ упокоевается; земная вещь къ землѣ идетъ, и всякое тѣло на землѣ упокоевается; рыба воды ищетъ, и упокоевается въ ней; воды въ море текутъ, и тамо упокоеваются; словомъ, всякая вещь въ сродномъ себѣ естествѣ покой обрѣтаетъ. И хотя препятствуется, однакожъ къ своему мѣсту и покою стремится. Огнь хотя и воспящается, однакожъ къ верху стремится; всякая тяжесть хотя къ верху и бросается, однакожъ паки къ землѣ обращается, яко къ своему мѣсту и покою. Тако душа человѣческая въ единомъ Бозѣ упокоевается, и нигдѣ не можетъ сыскати себѣ покоя, кромѣ единаго Бога. Единъ Богъ души есть покой. Душа есть духъ невещественный, образъ и подобіе Божіе въ себѣ имѣющій: гдѣ жъ убо ей и покой имѣть, какъ не въ Создателѣ своемъ, яко своемъ первообразномъ? Богъ Самъ нигдѣ благопріятнѣе не живетъ, какъ въ душѣ человѣческой, яко Своемъ образѣ и подобіи; тако и душа человѣческая ничимъ удовольствоватися не можетъ, какъ своимъ Создателемъ, и нигдѣ покоя сыскать не можетъ, какъ въ прелюбезномъ Божествѣ. Что бо душу можетъ успокоить и удовольствовать, яко духовное и Богу сообразное и подобное существо? Пища ли, одежда ли, драгоцѣнный уборъ, домъ, богатство, красота міра сего, высокая честь, сласть и всякое видимое увеселеніе? Нѣтъ! ничимъ отъ сихъ душа успокоиться и удовольствоваться не можетъ. Все бо тое тлѣнное есть, душа же нетлѣнна, тое временное, душа же вѣчна; тое видимо, душа же невидима. Со всѣмъ тѣмъ душа никакого сходства не имѣетъ: како убо тѣмъ успокоиться и удовольствоваться можетъ, что ей несходно и несродно? Духъ тѣломъ, и безсмертное смертнымъ и нетлѣнное тлѣннымъ, и невещественное вещественнымъ, и невидимое видимымъ никакъ не можетъ довольствоваться. Есть иное нѣчто, чимъ душа упокоевается и довольствуется. Добро есть свѣтъ солнца, луны и звѣздъ, но тѣлесное добро, тѣло просвѣщаетъ. Добро есть воздухъ, но тѣлесное добро, тѣло оживляетъ. Добро есть хлѣбъ и всякая пища, но тѣлесное добро, тѣло укрѣпляетъ. Добро есть питіе, но тѣлесное добро, тѣло утѣшаетъ. Добро есть вода, но тѣлесное добро, тѣло омываетъ и прохлаждаетъ. Добро есть домъ, но тѣлесное добро, тѣло упокоеваетъ. Добро есть огнь, но тѣлесное добро, тѣло согрѣваетъ. Добро есть одежда, но тѣлесное добро, тѣло одѣваетъ. Добро есть плодъ древесъ и травъ, но тѣлесное добро, тѣло того вкушаетъ. Добро есть медъ, но тѣлесное добро, тѣло услаждаетъ, и проч. Все сіе и прочее подобное тому души удовольствовать не можетъ. Есть иное, что душу довольствуетъ и упокоеваетъ. Есть животъ, которымъ душа оживляется; есть свѣтъ, которымъ душа просвѣщается, есть хлѣбъ и пища, которою душа укрѣпляется; есть питіе, которымъ прохлаждается; есть домъ, въ которомъ упокоевается; есть сладость, которою услаждается; есть одѣяніе, которымъ одѣвается, и проч. Богъ единъ душѣ, яко образу и подобію Своему, есть свѣтъ, животъ, пища, питіе, укрѣпленіе, прохлажденіе, утѣшеніе, увеселеніе, радость, покой, миръ, богатство, честь, слава и все блаженство. Безъ Бога и кромѣ Бога душа жить, упокоиться и блаженна быть не можетъ. \textit{Богъ любы есть, и пребываяй въ любви въ Богѣ пребываетъ, и Богъ въ немъ пребываетъ}\footnote{1~Іоан.~4,~16.}. Сколько люди изобрѣтаютъ способовъ, чтобы душу свою успокоить и удовольствовать, "--- но не могутъ изобрѣсти въ мірѣ ничего, чимъ бы она успокоилася и удовольствовалася. Сколько собираютъ богатства, сколько стараются въ высокую честь произойти, сколько тщатся великую славу пріобрѣсти, сколько вымышляютъ родовъ пищи, напитковъ, сластопитанія, красныхъ домовъ и строеній, одеждъ и прочихъ прихотей, "--- но не могутъ душъ своихъ упокоить и удовольствовать, и день отъ дне болѣе и болѣе желаютъ и вымышляютъ новостей и перемѣнъ во всемъ. Не могутъ упокоить и удовольствовать душъ своихъ: "--- какая тому причина? Понеже ищутъ того, чимъ душа не упокоевается и не довольствуется; а не ищутъ того, въ чемъ \textit{единомъ} душа покой и удовольствіе свое обрѣтаетъ. Вся созданія душу, яко образъ и подобіе Божіе, упокоить и удовольствовать не могутъ. Къ Богу ведутъ ее, но она только въ Бозѣ упокоевается. Вѣдалъ сіе добрѣ Давидъ святый. Сего ради, хотя царемъ Израилевымъ былъ, богатство, честь и славу великую имѣлъ, однакожъ все тое за ничто вмѣнялъ, но душею своею спѣшилъ къ \textit{живому источнику} "--- Богу. \textit{Имже образомъ желаетъ елень на источники водныя: сице желаетъ душа моя къ Тебѣ, Боже! Возжада душа моя къ Богу крѣпкому, живому}\footnote{Пс.~41,~2 и 3.}. И на другомъ мѣстѣ поетъ: \textit{насыщуся, внегда явитимися славѣ Твоей}\footnote{16,~15.}; то"=есть, ничимъ душа моя не можетъ насытитися и удовольствоватися, какъ только Тобою единымъ, о Боже! Тогда я насыщуся и удовольствуюся, когда возсіяетъ мнѣ слава святѣйшаго лица Твоего. Хрістіанине! сей Пророкъ святый примѣромъ своимъ научаетъ и насъ ни въ чемъ покоя и удовольствія не искать душамъ нашимъ, какъ только \textit{въ единомъ Бозѣ}. Почто жъ убо и искать покоя и удовольствія тамо, гдѣ сыскать не можемъ? Духъ въ дусѣ, а не въ веществѣ упокоевается. Душа, яко образъ и подобіе Божіе въ себѣ имущая, достоинствомъ и драгостію своею весь міръ превосходитъ; ибо небо и земля съ исполненіемъ ихъ единой души человѣческой не стоитъ. Низшая Бога, но высшая всѣхъ тварей "--- душа человѣческая. Честенъ портретъ царя земнаго ради царя, яко образъ и подобіе царское въ томъ изображается; честна и душа человѣческая и многоцѣнна и великолѣпна, яко образъ и подобіе небеснаго Царя въ ней изображено. Почто жъ убо ее подлѣйшему созданію порабощать? Всякое созданіе недостойно ея есть, яко всякое созданіе низшее души есть и подлѣйшее. Сего ради Хрістосъ Спаситель нашъ, промышляя о покоѣ и блаженствѣ душъ нашихъ, и ведя насъ къ подлѣйшему состоянію уклонившихся и въ земныхъ и суетныхъ вещахъ упражняющихся, и тако безъ всякой нашей пользы трудящихся и утруждающихся, и о нашемъ сожалѣя и соболѣзнуя бѣдствіи, къ Себѣ насъ призываетъ, и у Себе обѣщаетъ душамъ покой. \textit{Пріидите ко Мнѣ вси труждающіися и обремененніи, и Азъ упокою вы}\footnote{Мѳ.~11,~28 и 29.}. О заблудшіи и бѣдніи человѣки! почто трудитеся и ищете покоя тамо, гдѣ покоя сыскать не можете? Оставите суетныи и безполезныи замыслы и тщанія ваша, и пріидите ко Мнѣ, \textit{и обрящете покой душамъ вашимъ}. Нигдѣ бо обрѣсти покоя не можете, кромѣ Мене. \textit{Услыши Господи гласъ мой, имже воззвахъ, помилуй мя и услыши мя! Тебѣ рече сердце мое: Господа взыщу. Взыска Тебе лице мое; лица Твоего, Господи, взыщу. Не отврати лица Твоего отъ мене. Насыщуся, внегда явится мнѣ слава Твоя}\footnote{Пс.~26,~7--9; 16,~15.}.

\section{134. Пресельникъ.}

Видимъ въ мірѣ, что люди съ мѣста на мѣсто, изъ веси въ другую весь, и изъ села въ другое село, и изъ града въ другій градъ переселяются и переходятъ жить, и все свое имѣніе движимое туды переносятъ, и называются \textit{пресельники}. Тако истинныя хрістіане, который живую вѣру, яко свѣтильникъ, въ сердцахъ своихъ имѣютъ, сердцами и мыслями своими изъ міра сего въ оный вѣкъ переселяются; они въ мірѣ раждаются, но въ небесное отечество переселяются. О семъ Павелъ святый, избранный Хрістовъ сосудъ, со всѣми вѣрными глаголетъ: \textit{наше житіе на небесѣхъ есть, отонудуже и Спасителя ждемъ Господа нашего Іисуса Хріста}\footnote{Фил.~5,~20.}. Мы"=де тѣломъ на земли живемъ, но душею на небеси обращаемся. Туды пошелъ Хрістосъ Господь, Глава наша; туды спѣшимъ, стремимся и воздыхаемъ и мы, уды Его. Туды пошелъ Предтеча нашъ Іисусъ Хрістосъ Господь; туды взираемъ и мы, и слѣдуемъ. Онъ насъ, яко своихъ, туды влечетъ. Тамо наше отечество, тамо нашъ домъ, тамо наше наслѣдіе, тамо наше имѣніе, богатство, честь, слава, утѣшеніе, радость и все блаженство. Почему таковыи люди ничего въ мірѣ семъ, кромѣ нужнаго, не ищутъ, и ничто ихъ въ мірѣ семъ не можетъ увеселить, что сыновъ вѣка сего увеселяетъ. Таковыи \textit{не сокрываютъ себѣ сокровищъ на земли, идѣже червь и тля тлитъ, и идѣже татіе подкопываютъ и крадутъ: но скрываютъ себѣ сокровище на небеси, идѣже ни червь, ни тля тлитъ, идѣже татіе не подкопываютъ, ни крадутъ}\footnote{Мѳ.~6,~19 и 20.}. Нѣтъ у нихъ въ сердцахъ того, чтобы созидать богатые и красные домы, прибавлять и умножать земли, вотчины и крестьяны, въ высокія происходить чести, славу и похвалу въ мірѣ семъ пріобрѣсти, и прочее, о чемъ сыны вѣка сего стараются и пекутся. Ибо \textit{гдѣ сокровище ихъ, тамо и сердце ихъ}. Сокровище ихъ на небеси: тамо ихъ и сердце обращается. А мірскимъ и временнымъ тѣмъ довольствуются, что имѣютъ, помышляя и глаголя со Апостоломъ: \textit{ничтоже внесохомъ въ міръ сей, явѣ, яко ниже изнести что можемъ; имѣюще же пищу и одѣяніе, сими довольни будемъ}\footnote{1~Тим.~6,~7 и 8.}. Сіи суть знаки и примѣты истиннаго хрістіанина и вѣрнаго раба Хрістова. Таковый, хотя богатство имѣетъ, хотя честь и славу имѣетъ въ мірѣ семъ, однакожъ ничимъ тѣмъ не утѣшается, но всегда сердце его на небеси, гдѣ истинное сокровище его. Таковому вездѣ въ мірѣ семъ жить равно есть, гдѣ бы онъ ни жилъ; ибо житіе въ мірѣ семъ ему есть ссылка и изгнаніе. Онъ вездѣ, гдѣ ни находится, къ любезному своему отечеству "--- небу плачевныя свои возводитъ очи, и къ тому воздыхаетъ, \textit{въ жилище свое небесное облещися желая}\footnote{2~Кор.~5,~2.}, и со святымъ пророкомъ сердечно вопіетъ: \textit{коль возлюбленна селенія Твоя, Господи силъ! желаетъ и скончавается душа моя во дворы Господни}, и проч.\footnote{Пс.~83,~2 и 3.} Хрістіанине! преселимся и мы сердцемъ нашимъ въ небесное наше отечество, и ничего въ мірѣ, кромѣ нужнаго, не пожелаемъ и не поищемъ. Тамо намъ все уготовано отъ небеснаго нашего Отца. Почто ищемъ того, что не наше? Поищемъ онаго небеснаго сокровища, которое истинно наше есть и съ нами во вѣки будетъ. Мірское все въ мірѣ оставимъ, а мы наги отыдемъ отъ міра, какъ и наги вошли въ міръ. \textit{Не имамы здѣ пребывающаго града, но грядущаго взыскуемъ}\footnote{Евр.~13,~14.}.

\section{135. Раби, отъ господина посланныи звати гостей на обѣдъ.}

Видимъ въ мірѣ, что люди богатіи и славніи дѣлаютъ пиръ въ домахъ своихъ, и посылаютъ рабовъ своихъ звати различныхъ гостей на пиръ тотъ, и зовутъ. Тако Царь небесный, Богъ, богатый въ милости, сотворилъ великую и пресладкую вечерю вѣчнаго живота, и посылаетъ избранныхъ Своихъ рабовъ звати на вечерю тую всѣхъ людей безъ разбора, богатыхъ и нищихъ, славныхъ и подлыхъ, мужескій полъ и женскій, старыхъ и младыхъ, словомъ, всѣхъ и всякаго званія людей. \textit{Грядите, яко уже готова суть вся}\footnote{Лук.~14,~17.}. Посылалъ прежде пророковъ, и чрезъ нихъ звалъ; посылалъ потомъ апостоловъ, и чрезъ нихъ звалъ: \textit{грядите, яко уже готова суть вся}. Посылаетъ нынѣ епископовъ и пресвитеровъ, и велитъ имъ такожде звати всѣхъ. Епископы бо и пресвитеры въ мѣсто оныхъ избранныхъ мужей вступаютъ, и должность ихъ на себе воспріемлютъ. Сего ради должны звать не лѣностно, всѣхъ увѣщавать и молить, яко посланники Божіи: \textit{грядите, яко уже готова суть вся}. О возлюбленне (епископъ и пресвитеръ!) посланникъ еси и рабъ небеснаго Царя: твори и исполняй повелѣніе Господа твоего; трудись нелѣностно, ходи и зови на вечерю Господа твоего, и не преставай звати всѣхъ; толкай въ двери сердечныхъ домовъ ихъ, да, услышавше, поспѣшатъ на славную оную вечерю, и возлягутъ со Авраамомъ, Исаакомъ и Іаковомъ во царствіи Божіи. Зови, возлюбленне, зови, пока двери отверсты; зови, но и самъ иди; и иди напередъ предъ ними, и показывай имъ путь въ преславный оный небеснаго Царя домъ. Говори имъ громогласно: \textit{грядите, яко уже готова суть вся}. И къ чему прочихъ зовешь, не стой и самъ, но поспѣшай; не буди столпъ, на пути стоящій, который указываетъ путь ко граду, но самъ съ мѣста не движется; но буди вождь, который и прочимъ указываетъ путь, и самъ напередъ идетъ. Тогда дѣйствительно будетъ званіе твое, когда самъ туды будешь итить, куды прочихъ зовешь; иначе мало что успѣешь. Люди болѣе примѣру послѣдуютъ, нежели слову. Сильно слово званія твоего будетъ, когда примѣръ житія твоего слову твоему согласенъ будетъ. А когда люди слышатъ званіе твое, а видятъ, что самъ не движешися: едва ли будутъ вѣрить и слову твоему. И тако словомъ будешь звать, но примѣромъ будешь удерживать. О возлюбленне! берегись сего, да не и тебѣ приличествуетъ слово оное Хрістово: \textit{иже не собираетъ со Мною, расточаетъ}\footnote{Мѳ.~12,~30.}. Звали пророки, звали апостоли, звали преемники ихъ святіи, пастыри и учители церковніи, въ древности пожившіи; но и сами жаждущимъ духомъ къ почести вышняго онаго званія спѣшили. И ничто ихъ званія и теченія того воспятить не могло. Не токмо богатство, честь, славу и всякое угодіе міра сего ни во что вмѣняли, но и узы, темницы, изгнанія, біенія, раны, мученія, злостраданія и всякія смерти презирали, \textit{и чрезъ оныя скорби входили въ царствіе Божіе}\footnote{Дѣян.~14,~22.}. Сего ради и званніи ими люди, видя предводителей своихъ таковое теченіе къ вышнему званію, со всякимъ усердіемъ за ними спѣшили. Примѣръ ихъ, слову званія сообразный, поощрялъ и привлекалъ къ тому званныхъ людей. Было тое нѣкогда. Нынѣ люди къ богатству, чести, славѣ, угодіямъ, банкетамъ, операмъ и прочіимъ міра сего забавамъ и веселостямъ, веселыми и скорыми ногами спѣшатъ, спѣшатъ, понеже видятъ тое, къ чему спѣшатъ. А великія оныя вечери, которую Царь небесный Господь по Своей благодати уготовалъ всѣмъ, не видятъ; окомъ бо вѣры, а не тѣлесными глазами видится она. Не видятъ тую люди, почему и не спѣшатъ туды. Соблазны міра сего, день отъ дне умножаеміи, помрачаютъ душевныя очи человѣческія, и тако угашаютъ свѣтильникъ вѣры. Откуду бываетъ, что люди, какъ скоти, къ тому стремятся, что видятъ, а чего не видятъ, того и не ищутъ. Сей есть скотскій нравъ. О, когда бы хотя малую частицу увидѣли люди вечери оныя, "--- все бы міра сего сокровище бросивши, со всякимъ усердіемъ и поспѣшностію туды стремились! \textit{Тогда праведницы просвѣтятся, яко солнце во царствіи Отца ихъ}\footnote{Мѳ.~13,~43.}. Пастырямъ сказано: вы \textit{есте свѣтъ міра}. Когда свѣтъ потемнѣетъ, чимъ уже людямъ просвѣтиться? Пастырямъ сказано: \textit{вы есте соль земли}. Когда соль обуяетъ, чимъ уже людемъ себе растворить? О возлюбленный пастырь! Свѣтъ еси міру: свѣти убо не только словомъ, но и житіемъ своимъ. Соль еси земли: берегись обуять, да не и прочіи, смотря на тя, обуяютъ. \textit{Тако да просвѣтится свѣтъ вашъ предъ человѣки, яко да видятъ ваша добрая дѣла, и прославятъ Отца вашего, Иже на небесѣхъ}\footnote{Мѳ.~5,~13--16.}. Вступилъ ты, возлюбленне, въ званіе и должность апостольскую; подражай убо ученіемъ и житіемъ апостоламъ, да и съ ними участіе на вечери оной будешь имѣть, и многихъ за собою повлечешь.

\section{136. Сторожъ.}

Видимъ въ мірѣ, что различные сторожи поставляются: иніи хранятъ градъ, иніи хранятъ домъ, иніи иное хранить поставляются; и хранятъ, къ чему приставлены бываютъ. Тако епископы и пресвитеры избираются и постановляются хранить души хрістіанскія, въ смотрѣніе имъ порученныя. Имъ приличествуетъ слово оное Божіе, которое святому пророку Іезекіилю сказано: \textit{сыне человѣчь, въ стража дахъ тя дому Израилеву}\footnote{Іез.~23,~7.}. Сторожа, которыи приставлены бываютъ хранить домы и грады и прочая, хранятъ отъ злыхъ людей. Епископы и пресвитеры отъ кого хранить должны души хрістіанскія? \textit{Отвѣтъ}. Апостолъ святый указуетъ, глаголя: \textit{трезвитеся, бодрствуйте, зане супостатъ вашъ діаволъ, яко левъ рыкая, ходитъ искій, кого поглотити}\footnote{1~Петр.~5,~8.}. О какъ лютый врагъ, отъ котораго надобно храниться хрістіанамъ! О какъ искуснымъ и бодрственнымъ надобно быть сторожамъ, который хрістіанъ отъ врага того сохраняютъ! Желаетъ и ищетъ онъ не домовъ нашихъ, ни городовъ нашихъ взять, но души наши плѣнить; не имѣніе наше, злато, сребро и прочее вещество отнять у насъ, но спасеніе вѣчное, кровію и смертію Хрістовою сысканное, похитить тщится. И столько лукавъ и хитръ, сколько золъ. Учится того злаго своего искусства отъ начала міра; знаетъ какъ приступать, какъ обмануть, какъ прельстить человѣка, какъ къ нему подкрасться, и проч. Крадетъ и похищаетъ онъ спасеніе наше чрезъ невѣріе, ереси, расколы и всякій грѣхъ. Отъ такого и такъ лютаго и хитраго врага хранить души хрістіанскія поставляются епископы и пресвитеры. Не дремли убо, о возлюбленне, но трезвися и бодрствуй, и храни порученныя тебѣ души; остерегай и возвѣщай людямъ злый и пагубный приходъ его. Будешь на стражи бодренно стоять, когда будешь непрестанно людямъ Божіе слово говорить, увѣщавать и молить ихъ по правилу того жить, и житіе свое исправлять; въ сердцахъ ихъ насаждать вѣру, страхъ Божій вкоренять, предостерегать отъ грѣха и поощрять къ добродѣтели; напоминать имъ о Бозѣ, чего Онъ отъ насъ хощетъ и требуетъ, "--- о Хрістѣ Спасителѣ міра, о пожитіи Его на земли, страданіи, смерти и воскресеніи Его, и чего ради такъ велико и чудно дѣло сотворилося, и что мы должны Ему за сію такъ великую Его любовь къ намъ показанную; такожде показывать, въ чемъ вѣра и невѣріе, въ чемъ добродѣтель и грѣхъ, ложь и истина состоитъ, и проч. Сія есть должность епископа и пресвитера, душъ хрістіанскихъ сторожа. Стереги убо и храни \textit{домъ Израилевъ} "--- церковь, тебѣ порученную, которая есть домъ Божій. Стереги церковь, но прежде самъ себе храни и стереги, и стереги со всякимъ опасеніемъ. Ибо врагъ нашъ никого болѣе не ищетъ низложить, какъ пастыря. На войнѣ видимой врагъ о томъ болѣе тщится, какъ бы караулъ скрасть, который воинство остерегаетъ: тако врагъ хрістіанскій, діаволъ, ни о чемъ болѣе не тщится, какъ пастырей, душъ хрістіанскихъ стражей, восхитить и плѣнить. Берегись убо, возлюбленне; береги людей, и самъ берегись. Онъ невидимо обходитъ насъ, и ищетъ насъ поглотить. Насъ онъ видитъ, но мы его не видимъ. Однакожъ злый духъ отъ злаго запаха и вони познается. Убо когда другихъ бережешь и предостерегаешь, то прежде береги себе и предостерегай. Како будешь другихъ предостерегать и берещи, когда самъ себе не будешь берещи? Како другихъ будешь отъ грѣха отвращать, когда самъ отъ грѣха не отвращаешися? Како будешь другихъ къ добродѣтели поощрять, когда самъ добродѣтели не прилѣжишь? Услышишь въ совѣсти твоей обличительное слово: \textit{врачу, исцѣлися самъ}\footnote{Лук.~4,~23.}. Читай первое и второе къ Тимоѳею и къ Титу посланіе Апостола Павла, и разсуждай прилѣжно, и увидишь тамо, како должно тебѣ и себе самого и порученныя тебѣ души хрістіанскія хранить. Буди убо, возлюбленне, сторожъ вѣрный, бодрый и неусыпный себе самого и душъ хрістіанскихъ, не сребромъ и златомъ, но Хрістовою кровію купленныхъ. Имѣеши за всѣхъ ихъ отвѣтъ дати предъ Судіею праведнымъ, Который пречистую кровь Свою за нихъ изліялъ. Разсуждай сіе, помни, внимай себѣ и всему стаду твоему, берегись и береги. Сильно есть оружіе противу врага невидимаго \textit{молитва}. Молись убо самъ, и научай молитися людей твоихъ. Взаимная молитва есть, какъ крѣпкая стѣна противу врага, то"=есть, когда пастырь за людей усердно и съ любовію молится, и люди за пастыря отъ сердца къ Богу вопіютъ. За себе самого молиться нужда и бѣда убѣждаетъ, но за ближняго молиться убѣждаетъ любовь. На видимой войнѣ вси воины полководца берегутъ и защищаютъ отъ врага, яко въ цѣлости его и ихъ цѣлость состоитъ: тако и на невидимой войнѣ, всѣмъ хрістіанамъ начальника своего "--- пастыря молитвою должно защищать противу врага; отъ его бо цѣлости и ихъ цѣлость зависитъ. Когда пастырь добръ и мудръ, то и овцы находятся въ добромъ состояніи. Когда таковая взаимная любовь показуется, тогда Самъ Богъ хранитъ и пастыря и стадо, безъ Котораго всякое тщаніе не сильно, по писанному: \textit{аще не Господь созиждетъ домъ, всуе труждаются зиждущіи; аще не Господь сохранитъ градъ, всуе бдитъ стрегій}\footnote{Пс.~126,~1.}. Пророкъ святый своимъ примѣромъ показуетъ, откуду намъ приходитъ помощь и храненіе, и глаголетъ: \textit{возведохъ очи мои въ горы, отнюдуже пріидетъ помощь моя. Помощь моя отъ Господа, сотворшаго небо и землю}, и прочее псалма СХХ, до конца.

\subsection{О томжде.}

Видимъ, что когда сторожи дома даютъ знать о находящемъ злѣ на домъ, то"=есть, или о злыхъ людехъ нашедшихъ, или о пожарѣ наченшемся; тогда жители дома востаютъ и со всякимъ прилѣжаніемъ тщатся защитить и сохранить домъ. Тако, когда пастырь возвѣщаетъ людемъ грядущій гнѣвъ Божій за грѣхи, и временную и вѣчную казнь слѣдующую, и обращаетъ къ истинному покаянію, не должно людемъ пренебрегать о томъ, но должно очувствоваться, обратиться и каяться за грѣхи, и тако Бога умилостивлять, и души своя благодатію Его спасать. Сего требуетъ Апостолъ: \textit{повинуйтеся наставникомъ вашимъ и покаряйтеся; тіи бо бдятъ о душахъ вашихъ, яко слово воздати хотяще, да съ радостію сіе творятъ, а не воздыхающе; нѣсть бо полезно вамъ сіе}\footnote{Евр.~13,~17.}. О хрістіанине! возвѣщаетъ тебѣ сторожъ временное зло, и бережешися: кольми паче должно берещися, когда вѣчное возвѣщается тебѣ зло. Бережешь имѣніе тлѣнное отъ вора и злаго человѣка, кольми паче должно берещи нетлѣнное вѣчнаго спасенія сокровище отъ злаго духа. Хранишь домъ твой отъ пожара; кольми паче должно хранить душу и тѣло, да не вѣчнымъ огнемъ сгоритъ. Все сіе пастырь"=сторожъ твой возвѣщаетъ тебѣ, и даетъ тебѣ знать. Познай убо наступающую бѣду твою, и храни себе, да не впадеши въ бѣду, отъ которой изыти не возможеши. Огнь тамо не угасаетъ, и червь не умираетъ. Все и всякое временное бѣдствіе противу вѣчнаго ничто.

\subsection{О томжде.}

Когда сторожъ дома возвѣщаетъ домашнимъ о находящемъ на домъ злѣ, а жители о томъ небрегутъ и не тщатся домъ защищать и хранить, и тако домъ расхищается и разоряется; тогда сами бѣдствію своему виновны бываютъ, а сторожъ безъ вины остается: тако, когда пастырь всякимъ образомъ тщится обратить людей къ покаянію, увѣщаваетъ и молитъ ихъ къ тому, но они о томъ небрегутъ, и не обращаются и не каются; тогда сами своей погибели виновны бываютъ, а отъ пастыря тое уже не взыщется. И сіе"=то есть, что Богъ пророку Іезекіилю глаголетъ: \textit{аще ты проповѣси нечестивому путь его, еже обратитися отъ него, и не обратится съ пути своего; той въ нечестіи своемъ умретъ, а ты душу твою избавилъ еси}\footnote{Іез.~33,~9.}.

Сколько Самъ Хрістосъ Господь увѣщавалъ и обращалъ къ покаянію ожесточенный Іерусалимъ; но наконецъ сказалъ съ жалѣніемъ и болѣзнію сердца Своего: \textit{Іерусалиме, Іерусалиме, избивый пророки и каменіемъ побиваяй посланныя къ тебѣ! коль краты восхотѣхъ собрати чада твоя, якоже собираетъ кокошъ птенцы своя подъ крилѣ, и не восхотѣсте! Се оставляется вамъ домъ вашъ пустъ}\footnote{Мѳ.~23,~37 и 38.}. Сіе страшное слово приличествуетъ и хрістіанамъ нераскаянымъ, которыи часто и всегда слышатъ Божіе слово и проповѣдь покаянія, но не обращаются, и въ своемъ нечестіи пребываютъ, и не только не каются за грѣхи, но и грѣхи ко грѣхамъ, и беззаконія къ беззаконіямъ прилагаютъ, и тако въ своемъ нечестіи успѣваютъ, и идутъ во глубину золъ; и какъ слѣпый къ ямѣ, въ которую имѣетъ впасть, приближается, тако они, помрачившися прелестію грѣха и суетою міра сего, къ вѣчной погибели стремятся, и имѣютъ въ тую впасть, аще истиннымъ сердцемъ не обратятся къ Богу, и сокрушеніемъ сердца и теплыми слезами не омыютъ нечестія своего. Хрістіанине! страшно Хрістово слово сіе: \textit{се оставляется вамъ домъ вашъ пустъ}. Берегись убо, да и не тебѣ оное приличествуетъ. Однакожъ пастырь никогда не долженъ переставать, но всегда гремѣть словомъ, а сердцемъ сожалѣть и соболѣзновать погибающимъ грѣшникамъ, и ни о комъ не отчаиваться, пока въ мірѣ живетъ.

\subsection{О томжде.}

Бываетъ, что сторожъ дому воздремавши заснетъ, или отлучится куды отъ своего мѣста, гдѣ поставленъ; а въ тое время найдутъ злые люди на домъ, или пожаръ начнется, и тако некому будетъ возвѣстить домашнимъ о нашедшемъ злѣ: тогда господинъ дома разгнѣвается на того сторожа, и истязуетъ его и жестокой предаетъ казни; понеже нерадѣніемъ его зло на домъ нашло. Тако сдѣлается и съ пастыремъ дремлющимъ и не хранящимъ душъ хрістіанскихъ, и не возвѣщающимъ имъ наступающаго зла. Церковь Божія есть \textit{домъ Божій}\footnote{1~Тим.~3,~15.}. Господинъ дому того есть Хрістосъ Господь; домашніи Его суть хрістіане; епископъ и пресвитеръ есть сторожъ дому того. Аще убо дремлетъ или спитъ и небрежетъ о домѣ томъ, а супостатъ діаволъ, который никогда не спитъ, но на погибель человѣческую бодрствуетъ, и ищетъ кого поглотить, находитъ на домъ тотъ, и, видя дремлющаго сторожа, души домашнихъ плѣняетъ: тогда Дому"=Владыка Хрістосъ праведно разгнѣвается на пастыря, сторожа дому своего, и будетъ судить его, яко нерадиваго и невѣрнаго, \textit{и протешетъ его полма, и часть его съ невѣрными положитъ: ту будетъ плачь и скрежетъ зубомъ}\footnote{Мѳ.~24,~51; Іез.~33,~8.}. Дремлетъ пастырь, сторожъ дому Господня, когда молчитъ и не проповѣдуетъ слово Божіе, не приводитъ людей къ покаянію, не обличаетъ ихъ за грѣхи, не научаетъ святаго хрістіанскаго житія, не предохраняетъ отъ козней діавольскихъ, не показуетъ пути къ вѣчному животу, не устрашаетъ грядущимъ гнѣвомъ Божіимъ за грѣхи; словомъ, дремлетъ, когда не тщится порученныхъ себѣ людей спасти въ вѣчную жизнь; сія бо есть сила званія и должности его. Спитъ и глубоко спитъ пастырь, сторожъ дому Господня, когда самъ о своемъ спасеніи небрежетъ, но беззаконнуетъ. Тогда не только не стережетъ дома Господня, но и разоряетъ, и соблазнами своими души хрістіанскія за собою влечетъ въ погибель. Таковому пастырю приличествуетъ слово Хрістово: \textit{иже нѣсть со Мною, на Мя есть; и иже не собираетъ со Мною, расточаетъ}\footnote{Мѳ.~12,~30.}. Нѣтъ большей радости діаволу, врагу душъ хрістіанскихъ, какъ когда пастырь, сторожъ дому Господня, въ такомъ состояніи находится; тогда онъ, что хощетъ, съ хрістіанами дѣлаетъ. Горе людемъ, у которыхъ таковый пастырь! Но большее горе пастырю тому, который себе только пасетъ, а не людей порученныхъ себѣ! \textit{Аще свѣтъ тма, то тма кольми} паче\footnote{6,~23.}. Аще свѣтильникъ угаснетъ: чимъ домашніи просвѣтятся? Аще пастырь похитится отъ волка: отъ кого сохранятся овцы? Аще пастырь, \textit{соль земли, обуяетъ}: какое уже буйство въ людяхъ будетъ\footnote{5,~13.}! Аще вождь съ пути совратится и заблудитъ: въ какомъ заблужденіи уже будутъ путники! О возлюбленный пастырь! поставленъ ты сторожемъ стерещи домъ не вещественный, но невещественный, не стѣны каменныя и деревянныя и прочее вещество, но души хрістіанскія. Стереги убо ихъ, и самъ стерегись. Они любимое стяжаніе Хріста Сына Божія, Душелюбца. Стяжалъ ихъ Себѣ Онъ не сребромъ и златомъ, но честною Своею кровію. И тебѣ ихъ поручилъ стерещи и пасти: \textit{паси овцы Моя}\footnote{Іоан.~21,~15--17.}. И ты, вступая въ должность и званіе сіе, взялъ ихъ въ храненіе себѣ. Храни же ихъ, береги и \textit{паси}. Взыщетъ Онъ стяжанія Своего отъ рукъ твоихъ. Страшно едину душу хрістіанскую погубить; далеко страшнѣе многія погубить души. Разсуждай сіе, возлюбленне, и спасай, и спасайся самъ. Пастырь и Начальникъ, Іисусъ Хрістосъ Господь да поможетъ тебѣ, да умудритъ тебе, да уцѣломудритъ тебе, да сохранитъ тебе съ людьми твоими. Слыши апостольское слово, сказанное тебѣ и прочимъ пастырямъ: \textit{внимайте себѣ и всему стаду, въ немже васъ Духъ Святый постави епископы, пасти церковь Господа и Бога, юже стяжа кровію Своею}\footnote{Дѣян.~20,~28.}.

\section{137. Не касайся, ты того поднять не можешь.}

Слышимъ, что единъ другому слово сіе говоритъ: \textit{не касайся, ты того поднять не можешь}. Слово сіе говорится тому, который хощетъ взять въ руки свои камень или иную какую тягость, высшую силъ его, и поднять тую и понести; тогда правильно ему говорится: \textit{не касайся, ты того поднять не можешь}. Хрістіанине! честь хрістіанину великая есть тягость; и чимъ большая и высшая честь, тѣмъ большая ея тягость есть. Аще убо въ честь идешь, разсуждай силы твоя, можешь ли тягость сію понести. Аще не можешь поднять и понести ея, \textit{не касайся}, да не отяготитъ тебе, и падеши и подъ тяжкимъ тѣмъ бременемъ будеши лежать, а не нести его. Надобно тому многіи глаза имѣть, кто хощетъ многихъ смотрѣть. Надобно тому научиться прежде себе управлять, кто хощетъ другихъ управлять; себе исправить, кто хощетъ другихъ исправлять; себе научить, кто хощетъ другихъ учить; быть \textit{свѣтомъ}, кто хощетъ другихъ просвѣщать; быть \textit{солію}, кто хощетъ другихъ растворять; быть образомъ и зеркаломъ незамараннымъ, въ которое многіи будутъ смотрѣть; быть отцемъ, который о дѣтяхъ своихъ отечески промышляетъ; быть недремлющимъ и бодреннымъ сторожемъ, который не градъ и стѣны и имѣніе вещественное, но души хрістіанскія, за которыя Хрістосъ Господь кровь Свою изліялъ, стережетъ; быть искуснымъ вождемъ, который ведетъ не къ вещественному граду, но къ вѣчному животу; надобно тому сильнымъ быть, кто долженъ многихъ немощи носить. Чимъ высшее дерево, тѣмъ чаще отъ всякой бури и непогоды колеблется. На малое дерево, и въ низкомъ мѣстѣ стоящее, мало какій вѣтръ вѣетъ; а высокое, и на горѣ стоящее, всякому вѣтру подлежитъ и отъ того колеблется; надобно ему корень свой глубоко въ землѣ утвержденный имѣть, да не отъ нашедшей бури исторгнется и падетъ и сокрушится. Во уединеніи и низкости живущему хрістіанину надобно нести крестъ свой и послѣдовать Хрісту, \textit{яко многи скорби} хрістіанамъ бываютъ\footnote{Пс.~33,~20.}; но въ чести находящемуся слѣдуетъ большій и тяжелѣйшій крестъ нести. И чимъ высшая честь хрістіанина, тѣмъ тягчайшее бремя ея. Діаволъ, супостатъ нашъ, на всякаго хрістіанина востаетъ и боретъ, но наипаче на начальника"=хрістіанина. Онъ всѣ силы свои употребляетъ на то, чтобы начальника"=хрістіанина низложить и плѣнить. Врагъ на войнѣ міра сего ни на кого болѣе не стрѣляетъ, какъ на полководцевъ и начальниковъ: тако и врагъ хрістіанскій, діаволъ, ни на кого болѣе не вооружается, какъ на тѣхъ хрістіанъ, которыи и прочими хрістіанами управляютъ, и себе и прочихъ хотятъ спасти. Любимая ему корысть есть хрістіанинъ начальствующій низложенный и плѣненный отъ него. Всякъ же плѣнникъ его есть, который противно слову Божію живетъ, своей, а не подначальныхъ ищетъ корысти, соблазняетъ, а не управляетъ и исправляетъ людей. Тогда онъ удобно и прочія хрістіанскія души плѣняетъ: какъ врагъ видимый, плѣнивши начальника, дѣлаетъ замѣшательство въ подначальныхъ воинахъ его, а тако и ихъ удобно плѣняетъ. Имѣетъ онъ своихъ служителей, ложныхъ хрістіанъ, которыи, какъ плевелы между пшеницею, находятся и растутъ. Чрезъ тѣхъ много навѣтовъ, козней и хитростей дѣлаетъ начальствующему хрістіанину. О семъ злѣ всѣ вѣки прешедшіе свидѣтельствуютъ; примѣчаемъ тое и нынѣ и видимъ. Сіи навѣтники опаснѣйшіи суть хрістіанину"=начальнику паче самаго діавола; понеже суть волки, но подъ овчими кожами сокровенны; они ласково говорятъ и, какъ Іуда, привѣтствуютъ, но въ сердцѣ на преданіе поучаются. Надобно остроумнымъ и прозрительнымъ быть начальнику, который хощетъ и себе и подначальныхъ своихъ спасать. Сіи души много вымышляютъ о начальникѣ, и въ подначальныхъ разсѣваютъ; и бываетъ, что на языкахъ человѣческихъ носится тое, что начальнику и на умъ не пришло. А тако великое въ подначальныхъ дѣлается замѣшательство, нестроеніе, подозрѣніе и сумнѣніе о начальникѣ. Все сіе зло и слухъ злый ударяетъ душу начальника. Надобно сильнымъ ему быть, когда хощетъ тую тягость поднять и понести, а неотмѣнно долженъ. Многіи хрістіане ни о чемъ такъ не стараются, какъ въ честь произойти, и въ высшую честь. Но не знаютъ сами, о чемъ стараются. Неразумному и неисправному хрістіанину честь какъ безумному мечь есть, которымъ и себе и другихъ погубляетъ. Вси хрістіане присягаютъ и кленутся предъ Богомъ всемогущимъ и Сердцевѣдцемъ, идучи въ честь; но многіи, вышедши изъ церкви, и забываютъ и нарушаютъ клятвенное свое обѣщаніе. Видно, что у такихъ людей на устахъ и языкѣ только клятва была, но въ сердцѣ безбожіе сокровенно. Много отъ такихъ присяжныхъ людей стонетъ и страждетъ общество. Лучше бы имъ не родитися, или родившимся землю дѣлать, и тако себе питать, нежели въ честь итить и многихъ неповинно слезы проливать. Горе таковому начальнику! Богъ, сирыхъ Отецъ и Судія вдовицъ, всякую каплю слезы озлобленныхъ людей Своихъ взыщетъ отъ него. \textit{Разумѣйте безумніи въ людехъ, и буіи нѣкогда умудритеся! Насаждей ухо, не слышитъ ли? или создавый око, не сматряетъ ли? Наказуяй языки, не обличитъ ли, учай человѣка разуму}\footnote{Пс.~93,~8--10.}? Сего ради внимай сему, хрістіанине, \textit{и не касайся, чего поднять не можешь}. Сіе хотя всякому, кто ни идетъ въ честь, говорится, но паче тому, который въ званіе и должность пастырскую вступаетъ и хощетъ быть іереемъ или епископомъ. \textit{Не касайся, чего понести не можешь}; не ищи и не бери бремени тягчайшаго силъ твоихъ, да не отяготитъ и погрузитъ тебе въ бездну. И когда зовешися на честь, разсуждай себе, можеши ли сдержать на себѣ бремя тое. А когда изволеніе Божіе будетъ быть тебѣ въ чести, то хотя и убѣгать будешь отъ чести, будешь на чести. О семъ многіи примѣры въ Церковной Исторіи видимъ. И воистину тако есть! Ибо Божіе опредѣленіе не мимоидетъ.

\section{138. Познаннаго зла всякъ уклоняется.}

Видимъ, что вси люди, познавши зло, того уклоняются. Тако зная, что змій ядомъ своимъ вредитъ и умерщвляетъ, того уклоняются; вѣдая, что зубы звѣрей вредъ и смерть содѣлаютъ, убѣгаютъ звѣрей; вѣдая, что огнь жжетъ и опаляетъ и умерщвляетъ, глубина водная потопляетъ и умерщвляетъ, берегутся воды и огня; вѣдая, что злодѣй и разбойникъ обнажаетъ и убиваетъ, вси таковыхъ убѣгаютъ пагубныхъ людей. Тако и прочаго познаннаго зла люди уклоняются. Хрістіанине! познаемъ и мы наше зло, которое истинно есть зло, и намъ пагубно. Познаемъ, что грѣхъ горькое и лютое наше зло есть, и будемъ уклоняться его, яко познаннаго зла. Кто бо, познавши зло и пагубу, отъ зла и пагубы не уклоняется? Невозможно, воистину невозможно познавшему зло не ненавидѣть зла и не уклоняться. Что люди любятъ грѣхъ и отъ него не уклоняются, то оттуду бываетъ, что сего зла не знаютъ и горькихъ плодовъ его. Познавшему грѣхъ, яко зло, невозможно не уклоняться отъ него. Ради чего убѣгаешь змія? понеже жаломъ своимъ умерщвляетъ. \textit{Жало смерти грѣхъ}, которымъ человѣкъ умерщвляется\footnote{1~Кор.~15,~56.}. Како убо, смертоносное сіе жало познавши, отъ него не будешь уклоняться? Грѣхъ познается изъ закона Божія и горькихъ его плодовъ. 1)~Великое зло есть грѣхъ. Ибо грѣхъ есть преступленіе и разореніе вѣчнаго и непремѣняемаго закона Божія. \textit{Грѣхъ есть беззаконіе}\footnote{1~Іоан.~3,~4.}. Что Богъ повелѣлъ и узаконилъ, тое чрезъ грѣхъ разоряется, и Богу повелѣвшему и узаконившему въ противность дѣлается, и непослушаніе показуется. 2)~Великое есть зло грѣхъ. Понеже всякимъ грѣхомъ \textit{великій и безконечный} Богъ оскорбляется, и прогнѣвляется Тотъ, Который насъ создалъ изъ ничего, создалъ по образу и по подобію Своему; Тотъ, Который промышляетъ о насъ, печется о насъ, благодѣтельствуетъ намъ, питаетъ насъ, одѣваетъ насъ, сохраняетъ и упокоеваетъ насъ; Тотъ, который такъ насъ возлюбилъ, что и Сына Своего Единороднаго въ міръ послалъ для насъ, и наслѣдіе вѣчнаго живота и царствія уготовалъ намъ, и прочая безчисленная благая подаетъ намъ; таковый и толикій любитель нашъ и высочайшій намъ благодѣтель грѣхомъ отъ насъ оскорбляется и прогнѣвляется. О какъ люто зло есть грѣхъ! Создателя и Отца нашего тѣмъ оскорбляемъ и прогнѣвляемъ. Воистину лучше умереть, нежели хотя легко согрѣшить! Богъ \textit{святый} и \textit{праведный} грѣхомъ прогнѣвляется. Ангели Того поютъ, славятъ, почитаютъ и покланяются Ему со страхомъ и благоговѣніемъ; но бѣдный человѣкъ, \textit{земля и пепелъ}, дерзаетъ грѣхомъ оскорблять и прогнѣвлять Его. О, како мы, бѣдніи человѣки, проданы подъ грѣхъ! \textit{како потемнѣ злато, измѣнися сребро доброе}\footnote{Пл. Іер.~4,~1.}! Како помрачилася краснѣйшая душъ нашихъ доброта! Гдѣ любезная оная красота наша "--- образъ Божій? \textit{Помяни, Господи, что бысть намъ: призри и виждь укоризну нашу. Разсыпася радость сердецъ нашихъ, обратися въ плачъ ликъ нашъ, спаде вѣнецъ съ главы нашея: горе намъ, яко} с\textit{огрѣшихомъ! О семъ смутися сердце наше, о семъ померкнуша очи наши}\footnote{5,~1,~15--17.}. \textit{Не еже хощу, сіе творю: но еже ненавижду, то содѣловаю; не еже бо хощу доброе, творю, но еже не хощу злое: сіе содѣваю. Окаяненъ азъ человѣкъ: кто мя избавить отъ тѣла смерти сея}\footnote{Римл.~7,~15,~19 и 14.}? \textit{Тебѣ, Господи, правда, намъ же стыдѣніе лица}\footnote{Дан.~9,~7.}. Великое зло есть грѣхъ, понеже великое мученіе дѣлаетъ совѣсти человѣческой. Грѣхомъ совѣсть уязвляется, раздражается и согрѣшившаго безпокойствуетъ, обличаетъ, мучитъ и терзаетъ. Бѣжитъ согрѣшившій, хотя и никто не гонитъ его: совѣсть, паче всякаго гонителя, гонитъ его. Когда люди что говорятъ между собою, онъ думаетъ, что о немъ говорятъ и совѣтуютъ; слышитъ слово о грѣхѣ, которому онъ повиненъ, помышляетъ, что ему то говорится, хотя и всѣмъ вообще простирается слово. Совѣсть тое злая въ немъ дѣйствуетъ, которая грѣхъ его, яко домашній судія и обличитель, судитъ и обличаетъ, и его безпокойствуетъ и мучитъ. Сіе внутреннее и домашнее зло всегда и вездѣ грѣшникъ съ собою носитъ, пока истиннымъ покаяніемъ и сокрушеніемъ сердца содѣланнаго грѣха не очиститъ. А которыи грѣшатъ и не чувствуютъ мучительства совѣсти (чему, кажется, быть невозможно), у таковыхъ совѣсть спитъ до времени. Но когда пробудится и начнетъ ихъ обличать, и содѣланныи ими грѣхи имъ представлять, тогда они узнаютъ, коль люто мучительство ея, и коль горекъ грѣха плодъ. Сіе мучительство такъ люто бываетъ, что бѣдный грѣшникъ изволяетъ всякое зло, нежели то мучительство терпѣть; а многіи, не терпя того, сами себе умерщвляютъ, какъ то сдѣлалъ Іуда, предатель Хрістовъ\footnote{Мѳ.~27,~5.}. Видимъ тое зло и нынѣ. Въ таковомъ мучительствѣ совѣсти практикою или самымъ дѣломъ научается человѣкъ познавать грѣхъ и горькій плодъ его. Аще же гдѣ въ мірѣ семъ такъ совѣсть за грѣхъ человѣка мучитъ и жжетъ: что уже будетъ въ будущемъ вѣкѣ, гдѣ всѣ ему грѣхи, отъ перваго возраста до смерти содѣланныи отъ него, вѣдомыи и невѣдомыи, въ совѣсти его открыются ясно, и ему будутъ представляться, обличать, мучить и терзать его?.. Разсуждай сіе, бѣдный грѣшникъ, и познавай грѣхъ. Сладокъ тебѣ кажется грѣхъ, но горекъ плодъ его, котораго будешь вкушать неотмѣнно, когда не обратишися къ Богу и истиннымъ покаяніемъ не очистишися. 4)~Видимъ, что за грѣхъ различныя казни отъ Бога посланы и посылаются. Ибо Богъ \textit{праведный} грѣхъ весьма ненавидитъ, и за той казнитъ согрѣшившаго. За грѣхъ аггели съ небесе свержены, и демонами сдѣлались, и на вѣчное мученіе осуждены. За грѣхъ прародители изъ рая выгнаны, и въ міръ, яко ссылку, посланы. За грѣхъ всемірный потопъ наведенъ. За грѣхъ Содомъ и Гоморръ съ окрестными градами огнемъ съ небесе пожжены. За грѣхъ Фараонъ, мучитель Израильскій, со всѣмъ воинствомъ своимъ въ морѣ Чермномъ потопленъ. За грѣхъ Израильтяне, исшедшіи изъ Египта, въ пустынѣ и землѣ обѣтованной многоразличныя казни претерпѣли, какъ видимъ въ святыхъ Ветхаго Завѣта книгахъ. За грѣхъ и нынѣ бываютъ различныя казни, кровавыя брани, моровыя язвы, междоусобныя брани, пожары, голоды и трясеніе земли, многоразличныя болѣзни и прочія безчисленныя бѣды. Всѣмъ бѣдамъ, какія ни бываютъ въ мірѣ, грѣхъ причиною есть. Грѣшитъ міръ, и за грѣхъ страждетъ бѣды и напасти. Тако правда Божія казнитъ противный Себѣ грѣхъ. Грѣхъ безъ казни не бываетъ. Согрѣшишь, человѣче, "--- уже ожидай себѣ казни, или внутрь или внѣ тебе, или купно обѣихъ. Сей бо есть плодъ грѣха. Ничего бо онъ не раждаетъ, кромѣ казни. 5)~За грѣхъ заключается входъ въ вѣчный животъ. \textit{Ибо не внидетъ туды ничтоже скверно}\footnote{Апок.~21,~27.}. Всякъ грѣхъ оскверняетъ сотворшаго его.

\textit{Внѣ псы и чародѣи, и любодѣи и убійцы, и идолослужители и всякъ любяй и творяй лжу}\footnote{22,~15; 1~Кор.~6,~8 и 9; Гал.~5,~19--21.}. 6)~За грѣхъ вѣчному огню и мученію предается человѣкъ, и съ діаволомъ и аггелами его безъ конца будетъ мучиться. Огнь ихъ не угасаетъ, и червь ихъ не умираетъ. \textit{Страшливымъ и невѣрнымъ, и сквернымъ и убійцамъ, и блудъ творящимъ, и чары творящимъ, идоложерцемъ и всѣмъ лживымъ, часть имъ въ езерѣ горящемъ огнемъ и жупеломъ, еже есть смерть вторая}\footnote{Апок. 21~,8.}. Видишь, хрістіанине, горькіе плоды грѣха. Горекъ убо и грѣхъ, который горькіе плоды раждаетъ. Таковые бо плоды, таковое и сѣмя. Сѣмя бо подобный себѣ плодъ раждаетъ. Злы и горьки грѣха плоды: горькое убо и злое сѣмя есть грѣхъ; и нѣтъ никакого большаго зла, какъ грѣхъ. Злое и горькое сѣмя есть гордость и высокоуміе; злое и горькое сѣмя непослушаніе и непокореніе Богу и властямъ, отъ Него посланнымъ; злое и горькое сѣмя непочитаніе родителей, и тѣмъ противленіе и показуемая неблагодарность, злое и горькое сѣмя есть злоба и памятозлобіе, отмщеніе и убійство; злое и горькое сѣмя есть блудъ, прелюбодѣяніе и всякая нечистота; злое и горькое сѣмя есть воровство, хищеніе, грабленіе, насиліе и всякая неправда; злое и горькое сѣмя есть презрѣніе, уничтоженіе и посмѣяніе ближняго; злое и горькое сѣмя есть оклеветаніе и осужденіе ближняго; злое и горькое сѣмя есть ругательство и хуленіе ближняго, злое и горькое сѣмя есть ложь, лесть, лукавство и хитрость и всякое законопреступленіе. Познай убо, хрістіанине, сіе зло и горесть, и будешь берещися того зла. Бережешися змія, бережешися зубовъ звѣриныхъ, бережешися огня, бережешися злодѣя и разбойника, бережешися всякаго зла, яко тѣло твое вредитъ и временный отнимаетъ животъ. Грѣхъ паче змія, паче зубовъ звѣриныхъ, паче всякаго злодѣя и разбойника, паче всякаго инаго зла вредитъ тебѣ, и отнимаетъ не временный, но вѣчный животъ, погубляетъ не тѣло, но душу, и содѣловаетъ не временную, но вѣчную смерть. Что сего зла ужаснѣйшее можетъ быть? Сіе зло вредитъ намъ болѣе паче самаго демона. Ибо и демона грѣхъ сдѣлалъ демономъ, который прежде грѣха свѣтлый ангелъ былъ. Почто жъ такъ великаго и ужаснаго зла не берещися? Развѣ хощешь самовольно въ явную бѣду себе вдать, какъ дѣлаютъ тіи люди, который сами себе ядомъ умерщвляютъ, или мечемъ прободаютъ, или обвѣшиваются и себе удавляютъ, или во огнь бросаются и сгараютъ, или во глубину водную погружаются и утопаютъ, или въ прочая явная злая себе вдаютъ? Правильно сіи люди несмысленными и безумными называются, потому что сами себе погубляютъ. Не паче ли безуменъ и несмысленъ есть тотъ человѣкъ, который зная, что грѣхъ есть ужасное и смертоносное зло, самовольно того касается и вдается ему? Хрістіанине! не видишь сего зла сими очами, но чувствуешь горькіе плоды его въ совѣсти твоей. Чувствуетъ и міръ весь, когда бѣдами и напастями за грѣхъ казнится. Мудрымъ и блаженнымъ вси того называютъ, который предостерегаетъ себе отъ всякаго зла, и отъ того уклоняться знаетъ; далеко мудрѣйшій и блаженнѣйшій есть, который бережется всякаго грѣха. Той тѣло и временный животъ, а сей душу и вѣчный животъ сохраняетъ. Се есть мудрость хрістіанская. Се есть разумъ, Божественнымъ просвѣщенный свѣтомъ "--- что жить въ мірѣ опасно, и берещися всякаго грѣха и всего со страхомъ касаться, и суеты міра удаляться и горняя мудрствовать, и предъ душевными очами вездѣ и всегда Бога имѣть, Его вѣрою и святымъ послушаніемъ почитать! \textit{Блюдите убо, како опасно ходите, не якоже немудри, но якоже премудри}\footnote{Еф.~5,~15.}.

\subsection{О томжде.}

Бываетъ, что родители дѣтей своихъ, и прочіи люди другихъ людей предостерегаютъ отъ вредныхъ вещей и случаевъ, и говорятъ: берегись того и того. Но когда дѣти и люди прочіи, не послушавше полезнаго совѣта, касаются того, отъ чего предостерегаемы бываютъ, "--- повреждаются, и тако познаютъ вредъ того, отъ чего были предостерегаемы. Тако малыя дѣти касаются углія огненнаго, и обжигаются, и познаютъ, что огнь жжетъ; касаются люди ядовитаго зелія, и вредятся, и познаютъ того вредъ. Хрістіанине! тако Богъ человѣколюбивый, жалѣя и промышляя о насъ, предостерегаетъ насъ въ словѣ Своемъ святомъ отъ всякаго грѣха. \textit{Пріидите чада, послушайте Мене, страху Господню научу васъ. Кто есть человѣкъ, хотяй животъ, любяй дни видѣти благи? Удержи языкъ твой отъ зла, и устнѣ твои, еже не глаголати льсти; уклонися отъ зла, и сотвори благо}, и проч.\footnote{Пс.~33,~12--15.} Но когда мы его не слушаемъ (что есть слѣпота и безуміе наше), и касаемся пагубнаго грѣха, какъ малыя дѣти огня, "--- повреждаемся отъ того, и яко отъ огня обжигаемся, и тогда уже познаемъ, коль люто и пагубно зло есть грѣхъ. Тако предостерегалъ Богъ прародителей нашихъ въ раи отъ грѣха: \textit{отъ древа, еже разумѣти доброе и лукавое, не снѣсте отъ него; а въ оньже аще день снѣсте отъ него, смертію умрете}\footnote{Быт.~2,~17.}. Не повѣрили прародители наши святой Божіей заповѣди, не послушали заповѣдавшаго Бога Создателя своего, коснулися запрещеннаго плода, и такъ тяжко согрѣшили, и тогда уже узнали, коль люто и пагубно есть заповѣдь Божію преступать. Узнали, коль горькое сѣмя есть грѣхъ, когда горькаго плода его вкусили "--- смерти: \textit{оброцы бо грѣха смерть}\footnote{Римл.~6,~23.}. Тое жъ дѣлается и нынѣ съ хрістіанами. Предостерегаетъ Богъ всѣхъ отъ грѣха чрезъ законъ Свой святый: \textit{не убій, не укради, не прелюбы сотвори}, и проч.; предостерегаетъ чрезъ казни, которыя за грѣхъ претерпѣли грѣшники, казни, прописанныя въ святомъ словѣ Его, и аки глаголетъ чрезъ тыя всѣмъ: \textit{сынове человѣчестіи, берегитесь грѣха!} Видите, что согрѣшившіи пострадали за грѣхъ: постраждете и вы, когда будете грѣшить. Блаженъ и мудръ, кто отъ бѣдствія другихъ опаснѣйшимъ бываетъ и знаетъ уклониться бѣды. Но когда хрістіане, не смотря на то, согрѣшаютъ; тогда уже узнаютъ отъ горькихъ слѣдствій грѣха "--- казней, коль великое и горькое зло есть грѣхъ. Совѣсть мучительная и всякая нашедшая казнь показуетъ силу и горесть грѣха. Тогда уже всякъ узнаетъ, что есть грѣхъ. Истинно и спасительно есть слово Божіе, которое предостерегаетъ насъ отъ грѣха. Предостерегаетъ Богъ грѣшниковъ отъ вѣчныя казни и муки. На сіе бо и прописана вѣчная оная казнь, прописанъ огнь вѣчный, червь неусыпающій, скрежетъ зубомъ, тьма кромѣшная, и прочая злая. Чрезъ сія аки глаголетъ Господь: бѣдніи грѣшники! сія вся злая согрѣшившіи постраждутъ въ будущемъ вѣкѣ; покайтеся убо, да не впадете въ оная злая. \textit{Не хощу бо смерти грѣшника, глаголетъ Господь, но еже обратитися и живу быти ему}\footnote{Іез.~33,~11.}. Но когда грѣшники не вѣрятъ тому, и отъ грѣховъ своихъ не обращаются и не каются; то уже самымъ дѣломъ узнаютъ въ будущемъ вѣкѣ, какъ великое зло есть грѣхъ, когда горькихъ его плодовъ чрезъ всю вѣчность будутъ вкушать. Узнаютъ нынѣ отшедшіи отъ міра сего и непокаявшіися; узнаютъ блудники, воры, хищники и прочіи законопреступники, не очистившіи грѣховъ своихъ покаяніемъ, и низверженніи во адъ пріяти по дѣломъ своимъ. Дознаютъ тое и нынѣ на земли живущіи беззаконники, когда не обратятся и не покаются; узнаютъ самымъ дѣломъ, что \textit{оброцы грѣха смерть}, когда горести смертной безъ конца будутъ вкушать, пожелаютъ умереть, и не умрутъ. Хрістіанине! истинно есть слово Божіе, а не ложно. \textit{Небо и земля мимоидетъ, но слово Божіе не мимоидетъ}\footnote{Мѳ.~24,~35; Марк.~13,~31; Лук.~21,~33.}. Что возвѣщаетъ, истинно есть. Будетъ вѣчная жизнь праведникамъ; будетъ и вѣчная мука грѣшникамъ: \textit{пойдутъ сіи въ муку вѣчную; праведницы же въ животъ вѣчный}\footnote{Мѳ.~25,~46.}. Покайся убо, да благодатію Хрістовою получиши вѣчную жизнь. Берегись грѣха, да не вѣчно умреши. \textit{Оброцы бо грѣха смерть}.

\section{139. Малыя дѣти.}

Видимъ, что малыя дѣти несмысленны суть. Когда игралища ихъ и шуточныя вещи отнимаются у нихъ, "--- плачутъ и рыдаютъ; но когда злодѣи, нашедше, домъ и имѣніе расхищаютъ, "--- смѣются и небрегутъ о томъ. Таковымъ дѣтямъ подобный имѣются многіи хрістіане. Когда у нихъ временное что отнимается, "--- плачутъ и рыдаютъ, а часто и смерти себе предаютъ; а что вѣчное блаженство чрезъ грѣхъ отъ нихъ отнимается, о томъ нерадятъ. Лишаются или богатства, или чести, или славы временной, "--- сѣтуютъ и печалуютъ, плачутъ и рыдаютъ; но что злодѣй душъ человѣческихъ, діаволъ, хитростію своею лишаетъ ихъ вѣчнаго живота, богатства, чести, и славы и всего блаженства онаго, небрегутъ о томъ. Къ сему числу надлежатъ блудники, прелюбодѣи и вси нечистоты любители, воры, хищники, грабители, чародѣи и призывающіи ихъ, хульники, клеветники, ругатели, хитрецы, лукавцы, лживіи и прочіи во грѣхахъ пребывающіи. Видно, что вси таковыи небрегутъ о вѣчномъ сокровищѣ спасенія, отъ нихъ хитростію діавольскою отнимаемомъ. Знать, что они, чего не видятъ, о томъ небрегутъ; а что видятъ, о томъ пекутся, и о потерянномъ жалѣютъ: якоже не видятъ вѣчнаго живота, и о погубленіи того не жалѣютъ. Сами убо сіи люди видятъ, что у нихъ на устахъ только вѣра, а въ сердцѣ невѣріе и безбожіе крыется. Невозможно бо, воистину невозможно имѣющему свѣтильникъ вѣры въ сердцѣ горящій къ вѣчнымъ онымъ благимъ со всякимъ усердіемъ не стремиться, а тако и въ истинномъ покаяніи не быть, и отъ всякаго грѣха не удаляться. Вѣра бо человѣка обновляетъ и отъ грѣховъ и суеты міра отвращаетъ, и къ будущимъ обѣщаннымъ благимъ руководствуетъ, и оная яко видимая ему представляетъ. О! когда бы человѣкъ хотя малую частицу вѣчныя жизни увидѣлъ, "--- всю бы суету міра сего бросивши, съ великимъ стремленіемъ къ оной спѣшилъ. Но понеже не видитъ оной, а Божію слову, которое оныя жизни блаженство различно изображаетъ, не вѣритъ, и потому не ищетъ оныя, и о погубляемой не жалѣетъ; а только стремится къ тому, что чувства ему представляютъ, и лишаяся того, жалѣетъ о томъ. Се есть малыхъ и несмысленныхъ дѣтей дѣло, или паче, скотское житіе. Скоти бо, что видятъ, къ тому и стремятся; а чего не видятъ, того и не желаютъ. \textit{И человѣкъ въ чести сый не разумѣ, приложися скотомъ несмысленнымъ, и уподобися имъ}\footnote{Пс.~48,~21.}. Не тако хрістіанская и благочестивая душа, которая свѣтильникъ вѣры въ сердцѣ своемъ горящій имѣетъ, дѣлаетъ. Она со всякимъ стремленіемъ спѣшитъ къ вѣчному животу, и того лишиться весьма бережется; почему всякаго грѣха и суеты міра, яко къ тому явныхъ препятствій, отвращается, и о временныхъ благихъ по апостольскому слову разсуждаетъ: \textit{ничтоже внесохомъ въ міръ сей, явѣ, яко ниже изнести что можемъ; имѣюще же пищу и одѣяніе, сими довольни будемъ}\footnote{1~Тим.~6,~7 и 8.}. А когда лишается временныхъ, не сокрушается такъ, какъ сынове вѣка сего дѣлаютъ, но съ праведнымъ Іовомъ глаголетъ: \textit{Господь даде, Господь отъятъ; яко Господеви изволися, тако бысть: буди имя Господне благословенно во вѣки}\footnote{Іов.~1,~21.}. Хрістіанине! великое безуміе есть и явная пагуба жалѣть и плакать о томъ, что какъ тѣнь преходитъ, что нынѣ имѣемъ, а вскорѣ и не имѣемъ (кончина бо жизни нашей всему полагаетъ конецъ); а не жалѣть о томъ, что, единожды сысканное, во вѣки пребываетъ. Вѣрно слово Божіе; что объявляетъ, тое неотмѣнно будетъ. Прекрасно и различно изображаетъ и представляетъ намъ вѣчную жизнь, и къ полученію оныя путь показуетъ, то"=есть, истинное покаяніе и живую вѣру. Повѣрь убо живо и дѣйствительно слову Божію, и безъ сумнѣнія перемѣнишися и обновишися, и будущую жизнь окомъ вѣры увидишь, и, еще на земли живучи, малыя крупицы сладости ея будешь вкушать. Тогда и ты со Псаломникомъ будешь сердцемъ говорить: \textit{коль возлюбленна селенія Твоя, Господи силъ! желаетъ и скончавается душа моя во дворы Господни}\footnote{Пс.~83,~2 и 3.}. Тогда, истину тебѣ говорю, весь міръ со всею своею суетою, гордостію и красотою, омерзѣетъ тебѣ. \textit{Братіе! не дѣти бывайте умы, но злобою младенствуйте, умы же совершенна бывайте}\footnote{1~Кор.~14,~20.}. \textit{Хощу васъ мудрыхъ убо быти во благое, простыхъ же въ злое}\footnote{Римл.~16,~19.}.

\section{140. Ядущіи и піющіи на трапезѣ, но прочіи алчущіи, къ той не допущаемыи за вину свою.}

Бываетъ, кто въ дому нѣкоемъ одни ядятъ и піютъ на трапезѣ и утѣшаются; но прочіихъ господинъ дома не допущаетъ къ трапезѣ той за нѣкую вину ихъ, и тако объемлетъ ихъ скорбь и печаль; хотятъ бо ясти, и къ тому не допущаются. Видятъ прочіихъ ядущихъ и піющихъ и утѣшающихся, но сами алчутъ и жаждутъ, и тако печалію снѣдаются. Тоежде будетъ и грѣшникамъ. Обыметъ и ихъ скорбь, печаль и туга несносная и жалѣніе поздное и безполезное, когда увидятъ праведныхъ во утѣшеніи, чести, славѣ, блаженствѣ вѣчномъ, ядущихъ и піющихъ на трапезѣ Господней, себе же всего того блаженства лишаемыхъ и въ великомъ мученіи находящихся. И сіе"=то есть, что глаголетъ Господь Іудеомъ: \textit{ту будетъ плачь и скрежетъ зубомъ, егда узрите Авраама и Исаака и Іакова и вся пророки во царствіи Божіи, васъ же изгонимыхъ вонъ}, и проч.\footnote{Лук.~13,~28.} И чрезъ пророка глаголетъ: \textit{се работающіи Ми ясти будутъ, вы же взалчете; се работающіи Ми пити будутъ, вы же возжаждете; се работающіи Ми возрадуются, вы же посрамитеся; се работающіи Ми возвеселятся въ веселіи сердца, вы же возопіете въ болѣзни сердца вашего, и отъ сокрушенія духа восплачетеся}\footnote{Ис.~65,~14.}. Тако увидѣлъ евангельскій богачъ убогаго Лазаря въ лонѣ Авраамлѣ утѣшаемаго, себе же страждущаго въ пламени огненнѣмъ, увидѣлъ и возгласилъ: \textit{отче Аврааме! помилуй мя}, и проч., но услышалъ отвѣтъ: \textit{чадо! помяни, яко воспріялъ еси благая твоя въ животѣ твоемъ, и Лазарь такожде злая: нынѣ же здѣ утѣшается, ты же страждеши. И надъ всѣми сими между нами и вами пропасть велика утвердися: яко да хотящіи прейти отсюду къ вамъ не возмогутъ, ни иже оттуду къ намъ преходятъ}\footnote{Лук.~16,~24--26.}. Такойжде отвѣтъ услышатъ и прочіи нераскаянніи грѣшники, которыи нынѣ благими Божіими утѣшаются, но Бога забываютъ и убогихъ Лазаревъ презираютъ. Наипаче же обыметъ скорбь и несносная болѣзнь господъ, князей и вельможъ, которыи рабами своими гнушалися, и ихъ за подножіе и аки скотовъ имѣли, но ихъ въ славѣ, а себе въ безчестіи увидятъ; обыметъ богатыхъ, когда нищихъ и убогихъ, отъ нихъ презрѣнныхъ, увидятъ во царствіи Божіи, себе же внѣ того; обыметъ всѣхъ ругателей, когда тѣхъ, которыхъ здѣ въ мірѣ поносили, злословили, ругали, безчестили и аки гной вмѣняли, увидятъ въ славѣ несказанной, себе же въ безчестіи и укоризнѣ. И сіе"=то есть, что въ книгѣ Премудрости Соломоновой написано: \textit{тогда станетъ въ дерзновеніи мнозѣ праведникъ предъ лицемъ оскорбившихъ его, и отметающихъ труды его. Видящіи смятутся страхомъ тяжкимъ, и ужаснутся о преславномъ спасеніи его. И рекутъ въ себѣ кающеся, и въ тѣснотѣ духа воздыхающе: сей бѣ, егоже имѣхомъ нѣкогда въ посмѣхъ, и въ притчу поношенія. Безумніи житіе его вмѣнихомъ неистово, и кончину ею безчестну. Како вмѣнися въ сынѣхъ Божіихъ, и въ святыхъ жребій его есть? Убо заблудихомъ отъ пути истиннаго, и правды свѣтъ не облиста намъ, и солнце не возсія намъ}.

\textit{Беззаконныхъ исполнихомся стезь и погибели, и ходихомъ въ пустыни непроходимыя, пути же Господня не увѣдѣхомъ}\footnote{Прем. Сол.~5,~1--7.}. Хрістіанине! видимъ въ святомъ Писаніи пресладкую оную трапезу, различно къ поощренію и утѣшенію нашему изображаемую. Видимъ ядущихъ и піющихъ на трапезѣ оной; видимъ алчущихъ и жаждущихъ къ трапезѣ оной, но праведнымъ судомъ Божіимъ не допущаемыхъ и изгонимыхъ отъ той. Тамо возлежатъ вѣрою и правдою послужившіи Господу и угодившіи Ему; возлежатъ святіи патріархи, возлежатъ пророки, возлежатъ апостоли, возлежатъ святители, возлежатъ мученики, пустынники, преподобніи, и вси святіи и праведніи; но грѣшники на тую трапезу издалеча смотрятъ; хотятъ къ ней пріити, но не допущаются; желаютъ ясти и пити, но не дается имъ; каются и жалѣютъ, но уже поздно. И тако объемлетъ ихъ скорбь и тѣснота, печаль и воздыханіе поздное и безполезное. Хрістіанине! мы, слава Богу, еще въ мірѣ семъ живемъ, еще надежда наша не погибла, еще двери къ трапезѣ оной отверсты, и вси отъ милостиваго Господа къ оной призываются. Не нерадимъ убо о себѣ, да не и мы отъ пресладкой оной трапезы изгонимся. Покаемся убо, и вѣрою и правдою Господу нашему Іисусу Хрісту угодимъ, да и насъ общниками сотворитъ трапезѣ оной. А понеже многія скорби срѣтаютъ идущихъ къ утѣшенію оному, не возвратимся вспять, стужаеми отъ нихъ, но паче претерпимъ ихъ. \textit{Претерпѣвый бо до конца, той спасется: и многими скорбьми подобаетъ намъ внити въ царствіе Божіе}\footnote{Мѳ.~24,~13; Дѣян.~14,~22.}. И гласъ съ небесе свидѣтельствуетъ о возлежащихъ на трапезѣ оной: \textit{сіи суть, иже пріидоша отъ скорби великія}, и проч.\footnote{Апок.~7,~14.} Надежда чаемаго добра поощряетъ къ подвигу, и въ подвигѣ утверждаетъ. Надежда богатства купца по чужимъ странамъ скитаться, надежда плодовъ земледѣльца трудиться и потѣть, надежда разума ученика въ училищахъ обращаться и всякую нужду претерпѣвать, надежда побѣды воина противу враговъ подвизаться поощряетъ и укрѣпляетъ. Тако и наипаче насъ надежда будущаго онаго блаженства да подвигнетъ и укрѣпитъ въ подвигѣ до конца. Конецъ бо добрый блаженство наше совершаетъ, по реченному: \textit{буди вѣренъ даже до смерти, и дамъ ти вѣнецъ живота}, глаголетъ Господь\footnote{2,~10.}.

\section{141. Входящіи въ чертогъ царскій.}

Видимъ, что люди, хотящіи внити въ чертогъ царскій, одѣваются въ свѣтлыя и чистыя одежды, и тако входятъ; а въ рубищахъ и гнусныхъ одѣяніяхъ туды не допущаются. Тако въ чертогъ небеснаго Царя входятъ тіи только души, которыи имѣютъ свѣтлыя и чистыя одежды, одѣянія брачная и чистый виссонъ; но въ рубищахъ грѣховныхъ туды не допущаются. \textit{Ибо не внидетъ туды ничтоже скверно}\footnote{21,~27.}. Хрістіане во святомъ крещеніи благодатію Божіею совлекаются грѣховныхъ рубищъ, и одѣваются въ чистую одежду и одѣяніе брачное, и тако дѣлаются достойными входа въ небесный оный чертогъ, по писанному: \textit{омыстеся, освятистеся, оправдистеся именемъ Господа нашего Іисуса Хріста, и Духомъ Бога нашего}\footnote{1~Кор.~6,~11.}. Сіи невозбранно входятъ въ домъ небеснаго Царя, и всего того блаженства участниками бываютъ. Аще кто сію богоданную одежду хранитъ, блаженъ есть, и воистину блаженъ! \textit{Блаженъ бдяй, и блюдый ризы своя, да не нагъ ходитъ, и узрятъ срамоту его}\footnote{Апок.~16,~15.}. Но многіи хрістіане (о бѣдности и окаянства человѣческаго!), многіи боготканную оную одежду потеряли, и облеклися паки въ рубища грѣховныя. Таковіи суть блудники, прелюбодѣи и вси нечисто живущіи; таковіи суть сребролюбцы, воры, хищники, грабители и вси чуждая похищающіи; таковіи суть клеветники, ругатели и лживыи и вси беззаконно живущія; вси таковіи въ рубищахъ грѣховныхъ ходятъ, и не имѣютъ одѣянія брачнаго. Самъ убо, человѣче, разсуди, како въ такъ гнусной одеждѣ въ небесный чертогъ человѣку войти? Въ палату земнаго царя не допущаются во одѣяніи рубищномъ: въ небесную ли палату небеснаго Царя въ рубищахъ допустятся? Нѣтъ! надобно такъ гнусной одежды совлещися и облещися въ чистый виссонъ хотящему внити въ прекрасный оный чертогъ. Пресвѣтлый домъ тотъ и Божественною славою осіявается: надобно и входящимъ въ тотъ и пребывающимъ въ немъ чистыми и свѣтлыми быть. Иначе какое согласіе тьмѣ со свѣтомъ? \textit{Господи, кто обитаетъ въ жилищи Твоемъ? или кто вселится во святую гору Твою?} (Отвѣщаетъ Господь:) \textit{ходяй непороченъ, и дѣлаяй правду, глаголяй истину въ сердцѣ своемъ; иже не ульсти языкомъ своимъ, и не сотвори искреннему своему зла, и поношенія не пріятъ на ближнія своя. Уничиженъ есть предъ нимъ лукавнуяй, боящіяжеся Господа славитъ. Кленыйся искреннему своему, и не отметаяйся, сребра своего не даде въ лихву, и мзды на неповинныхъ не пріятъ: творяй сія не подвижится во вѣкъ}\footnote{Пс.~14,~1--5.}. Вотъ примѣты души, носящія одѣяніе брачное и входящія въ чертогъ небеснаго Царя. "--- Хрістіанине! осмотримся мы, не потеряли ли и мы спасительнаго онаго одѣянія, и вмѣсто того не носимъ ли рубища грѣховнаго, которое носящіи въ небесный чертогъ не допущаются. Аще примѣтимъ сіе, потщимся покаяніемъ и вѣрою грѣховнаго рубища совлещися, и очистить и убѣлить ризы душъ нашихъ въ крови Агнчей, да и мы со входящими въ прекрасный оный чертогъ внидемъ и явимся лицу Божію, и въ томъ чертогѣ жить и благость Божію хвалить во вѣки вѣковъ будемъ. Дается туне благодатію Божіею на крещеніи всякому хрістіанину боготканная и спасительная тая риза; но потерявшимъ тую надобно уже съ немалымъ трудомъ, покаяніемъ, жалѣніемъ, стенаніемъ, сокрушеніемъ сердца, плачемъ и слезами искать. \textit{Блажени живущіи въ дому Твоемъ: въ вѣки вѣковъ восхвалятъ Тя, Господи}\footnote{83,~5.}. Писано же есть: \textit{просите, и дастся вамъ; ищите и обрящете; толцыте, и отверзется вамъ}\footnote{Мѳ.~7,~7.}. \textit{Чертогъ Твой вижду, Спасе мой, украшенный; и одежды не имамъ, да вниду въ онь: просвѣти одѣяніе души моея, Свѣтодавче Хрісте Боже, и спаси мя!}

\section{142. Свѣтъ и тьма.}

Есть свѣтъ тѣлесный и видимый: есть душевный и невидимый свѣтъ; такожде и тьма есть тѣлесная и видимая: есть душевная и невидимая. Тѣлеснымъ свѣтомъ тѣло, но душевнымъ свѣтомъ душа просвѣщается; такожде тѣлесною и видимою тьмою тѣло, но душевною и невидимою тьмою душа помрачается и потемняется. Богъ и познаніе Божіе есть свѣтъ душевный; но незнаніе Бога есть тьма душевная. Вѣра есть свѣтъ душевный, но невѣріе и суевѣріе есть тьма душевная; страхъ Божій есть свѣтъ душевный, но безстрашіе есть тьма душевная. Добродѣтель всякая есть свѣтъ душевный; но всякій грѣхъ есть тьма душевная. Смиреніе, умаленіе и уничтоженіе себе самого есть свѣтъ душевный; но гордость, высокоуміе и величаніе есть тьма душевная. Любовь къ Богу и ближнему есть свѣтъ душевный; но самолюбіе неумѣренное есть тьма душевная. Надежда и упованіе на Бога есть свѣтъ душевный; но надежда на человѣка и прочее созданіе есть тьма душевная. Истинная и нелицемѣрная молитва, пѣніе и хваленіе имени Божіяго есть свѣтъ душевный; но небреженіе о томъ есть тьма душевная. Беззлобіе и непамятозлобіе есть свѣтъ душевный; но злоба и памятозлобіе есть тьма душевная. Постъ и умѣренность есть свѣтъ душевный; но объяденіе и піянство есть тьма душевная. Любленіе и храненіе чистоты есть свѣтъ душевный; но блудъ, прелюбодѣяніе и всякая нечистота есть тьма душевная. Милость, милосердіе и состраданіе ближнему есть свѣтъ душевный; но немилосердіе и жестокосердіе есть тьма душевная. Щедрое подаяніе милостыни отъ чиста сердца есть свѣтъ душевный; но скупость есть тьма душевная. Презрѣніе суеты и пышности міра сего есть свѣтъ душевный; но любленіе того есть тьма душевная. Память смерти, суда Хрістова и вѣчности блаженной и неблагополучной есть свѣтъ душевный; но забвеніе тѣхъ есть тьма душевная. Словомъ, истинное покаяніе и тому сообразные плоды, добрыя дѣла "--- свѣтъ душевный; но нераскаяніе и тому сопряженные злые плоды, темныя дѣла "--- тьма душевная. Видишь, хрістіанине, свѣтъ и тьму: убѣгай отъ тьмы, да не пребудеши во тьмѣ, и въ вѣчную и кромѣшную тьму вверженъ не будеши; возлюби свѣтъ, да во свѣтѣ пребудеши и къ вѣчному свѣту прейдешь. Во тьмѣ ходитъ и не знаетъ, куды идетъ, яко тьма ослѣпила ему очи, кто во грѣхахъ живетъ: во свѣтѣ ходитъ, кто Бога знаетъ, и въ истинномъ покаяніи находится, и темныхъ дѣлъ убѣгаетъ, и плоды покаянія приноситъ. \textit{Азъ есмь свѣтъ міру}, глаголетъ Господь: \textit{ходяй по Мнѣ, не имать ходити во тмѣ, но имать свѣтъ животный}\footnote{Іоан.~8,~22.}. Ходимъ за Господемъ не ногами, но сердцемъ, волею и нравами. Ходитъ за Господемъ, кто вѣрою, смиреніемъ, любовію, кротостію, терпѣніемъ и прочіими добродѣтельми подражаетъ Ему; и во свѣтѣ ходитъ, яко въ слѣдъ свѣта ходитъ. Не ходитъ за Господемъ, кто по своей волѣ живетъ, и по примѣру житія Господня нравовъ своихъ не исправляетъ; и во тьмѣ ходитъ, яко отъ свѣта удаляется. Надобно бо тому во тьмѣ быть, кто удаляется отъ свѣта. Писано же есть: \textit{сей есть судъ, яко свѣтъ пріиде въ міръ, и возлюбиша человѣцы паче тму, неже свѣтъ: бѣша бо ихъ дѣла зла. Всякъ бо дѣлаяй злая, ненавидитъ свѣта, и не приходитъ къ свѣту, да не обличатся дѣла его, яко лукава суть: творяй же истину, грядетъ ко свѣту, да явятся дѣла его, яко о Бозѣ суть содѣлана}\footnote{Іоан.~3,~19--21.}. Всякъ, кто добро творитъ, не убѣгаетъ свѣта и не ищетъ тьмы; не боится бо обличенія, яко права дѣла его. Но кто зло творитъ, убѣгаетъ отъ свѣта и ищетъ тьмы; темныя бо дѣла во тьмѣ совершаются. Блудникъ, воръ, хищникъ и всякъ законопреступникъ тьмы и сокровеннаго мѣста ищетъ. О бѣдный человѣкъ! гдѣ ни сокрывайся, во тьмѣ, или подъ землею, или въ пустынѣ, или между стѣнами, "--- отъ Бога и домашняго твоего свидѣтеля и обличителя "--- совѣсти "--- нигдѣ сокрыться не можешь. Богъ вездѣ, и всякое дѣло видитъ твое; и совѣсть твоя вездѣ, и всякое обличаетъ дѣло твое. Она во всякомъ беззаконіи твоемъ не молчитъ, но дерзновенно говоритъ; худо ты, человѣче, дѣлаешь, страшный судъ Божій будетъ тебѣ! На томъ вся обличатся злая дѣла твоя. Тогда Богъ, праведный Судія, воздастъ тебѣ по дѣломъ твоимъ. Хрістіанине! возненавидимъ тьму, и возлюбимъ свѣтъ, \textit{да сынове свѣта будемъ}\footnote{13,~36.}. Отвратимся дѣлъ, которыи во тьмѣ творятся: сотворимъ дѣла сообразная дню, да просвѣтимся. \textit{Нощь убо прейде, а день приближися. Отложимъ убо дѣла темная, и облечемся во оружія свѣта. Яко во дни благообразно да ходимъ, не козлогласованіи и піянствы, не любодѣяніи и студодѣяніи, не рвеніемъ и завистію; но облецытеся Господемъ нашимъ Іисусъ Хрістомъ, и плоти угодія не творите въ похоти}\footnote{Римл.~13,~12--14.}. Хрісте, свѣте истинный! просвѣти очи сердецъ нашихъ, да увидимъ Тебе, Свѣта незаходимаго; и направи ноги наша въ слѣдъ Тебе ходить, \textit{да имамы свѣтъ животный}.

\section{143. Стыдно мнѣ на тебе смотрѣть.}

Слышимъ, что слово сіе единъ другому говоритъ; \textit{стыдно мнѣ на тебе смотрѣть}. Слово сіе говоритъ тотъ, который немало благодѣянія отъ нѣкоего человѣка получилъ, но его много оскорбилъ, и тако неблагодаренъ благодѣтелю своему показался; а потомъ, пришедши въ сожалѣніе, кается предъ нимъ и винность свою ему объявляетъ, и грѣхъ свой исповѣдуетъ, и стыдится и съ сожалѣніемъ ему говоритъ: \textit{стыдно мнѣ на тебе смотрѣть}. Тако хрістіанская душа, много Господу согрѣшившая, но потомъ въ чувство пришедшая и кающаяся, прилично Богу, Создателю своему, съ жалѣніемъ говоритъ: \textit{стыдно мнѣ очи свои къ Тебѣ возвести}. Ты мой Создатель: я Твое созданіе и дѣло рукъ Твоихъ, \textit{Руцѣ Твои сотвористѣ мя, и создастѣ мя}\footnote{Пс.~118,~73.}. Но я сего не разумѣлъ, согрѣшая и беззаконнуя предъ Тобою. Создалъ Ты мене не безчувственною и безсловесною тварію, но человѣкомъ, чувствами и разумомъ и, что болѣе всего того, образомъ Своимъ мене почтилъ. \textit{Образъ есмь неизреченныя Твоея славы}. Но я того великаго Твоего дарованія не разумѣлъ, не слушалъ Тебе, грѣшилъ и беззаконновалъ предъ Тобою, и тако оскорблялъ Тебе, Создателя и Благодѣтеля моего толикаго, Сего ради стыдъ лице мое покрываетъ; \textit{стыдно мнѣ и очи мои къ Тебѣ возвести. Тебѣ, Господи, правда, мнѣ стыдѣніе лица}\footnote{Дан.~9,~7.}. Ты мене, оскверненнаго грѣхомъ, въ беззаконіяхъ зачатаго и во грѣсѣхъ рожденнаго, яко человѣколюбивый и не хотяй смерти грѣшника Богъ, банею пакибытія омылъ, освятилъ и оправдалъ. Но и великаго Твоего сего и непостижимаго дара не разумѣлъ, паки себе (увы мнѣ!) многими грѣхами осквернилъ, и такъ тяжко предъ Тобою согрѣшилъ и Тебе оскорбилъ, \textit{яко песъ на свою блевотину возвратился, и яко свинія омывшися въ калъ тинный}\footnote{1~Петр.~2,~22.}. Се есть студъ лица моего! Горе мнѣ, яко согрѣшилъ Тебѣ! О семъ смутися сердце мое. О семъ померкнуша очи мои\footnote{Пл. Іер.~5,~17.}. \textit{Стыдно убо мнѣ очи свои къ Тебѣ возвести}. Стыжусь я на пресвятое лице Твое смотрѣть. Ты мене къ вѣчному животу позвалъ, отворилъ мнѣ двери царствія Твоего, и того наслѣдіе человѣколюбно обѣщалъ. Но я сего не разумѣлъ, но обратился къ суетѣ міра сего, и богатства, чести и славы того искалъ, и о вѣчномъ ономъ сокровищѣ, отъ Тебе мнѣ обѣщанномъ, безумно пренебрегалъ, и тако тяжко оскорблялъ Тебе. \textit{Стыдно убо мнѣ къ Тебѣ смотрѣть}. Стыдъ лица моего покрываетъ мене. "--- Ты мене питаешь, одѣваешь, и все, и всякое добро, какое ни имѣю я, щедрою Твоею рукою подаешь мнѣ. Но я преизобильнаго Твоего благодѣянія не разумѣлъ, оскорблялъ Тебе, Благодѣтеля моего, и къ Тебѣ Благодателю неблагодаренъ явился. Се есть студъ лица моего! \textit{Стыдно убо къ Тебѣ смотрѣть}, и проч. И нынѣ признаю и исповѣдую неблагодарность и грѣхи моя къ Тебѣ; и знаю и признаюсь, что я подлинно недостоинъ неба и земли, и укруха хлѣба, и никакого малѣйшаго добра, и самаго временнаго живота, но только наказанія достоинъ. Грѣхи мои и неблагодарность моя мнѣ тое заслужили. Но, о милостиве и человѣколюбивый Господи, якоже величество Твое, тако и милость Твоя да будетъ на мнѣ грѣшномъ! Не погубилъ Ты мене согрѣшающаго: не погуби уже кающагося. Потерпѣлъ Ты беззаконнующаго, и ожидалъ обращенія, помилуй и пріими обращающагося къ Тебѣ. Помиловалъ Ты разбойниковъ, убійцъ, блудниковъ, хищниковъ, грабителей и прочихъ тяжкихъ грѣшниковъ кающихся: помилуй и мене подобнаго имъ. Видишь, Сердцевѣдче, печаль, тугу и стенаніе сердца моего, и стыдъ лица моего, яко Тебѣ, Создателю и Богу моему, согрѣшилъ. \textit{Помилуй мя Боже, по велицѣй милости Твоей, и по множеству щедротъ Твоихъ очисти беззаконіе мое}\footnote{Пс.~50,~1 и проч. до конца псалма.}. Не хощу уже болѣе я ни въ чемъ Тебѣ согрѣшить; но, якоже слово Твое научаетъ, тако поступать и по правилу того себе исправлять постараюсь. Господи, помози мнѣ, и простри мнѣ руку Твою святую, и укрѣпи мя, яко немощенъ есмь. \textit{Милость Твоя, Господи, поженетъ мя вся дни живота моего}\footnote{22,~6.}. Хрістіанине! разсуди добрѣ и познай, кто и кому ты согрѣшаешь, и кого грѣхомъ твоимъ оскорбляешь, когда ни согрѣшаешь. Ты, земля и пепелъ и малый червячокъ, Божіе непостижимое величество оскорбляешь грѣхомъ твоимъ; Бога великаго, Котораго ангели и вси небесныя силы со страхомъ и благоговѣинствомъ поютъ и славословятъ непрестанно, и Котораго хотѣнію и манію вся тварь повинуется и волю исполняетъ, ты единъ не слушаешь Его и не почитаешь, но грѣхомъ оскорбляешь. Онъ твой Создатель, Искупитель и Благодѣтель, какого нѣтъ и не можетъ быть большій; но ты Его не почитаешь, яко Ему согрѣшаешь, и тако оскорбляешь Его. Самъ разсуди, не великое ли безуміе и слѣпота, не почитать и оскорблять Того, отъ Котораго бытіе наше имѣемъ, и всякое добро, какое ни имѣемъ, получаемъ? Какій подданный смѣетъ не почтить и оскорбить царя своего? развѣ безумный, и на свою явную пагубу. Какій сынъ отца своего не почитаетъ? развѣ безчувственный. Какій рабъ господина своего не боится? развѣ лишенный ума. Какій человѣкъ благодѣтеля своего восхощетъ оскорбить? развѣ худшій скота и звѣря. И скоти бо и звѣри знаютъ своихъ благодѣтелей и помнятъ благодѣянія ихъ. Ты все сіе вышеписанное неистовство Господу Богу твоему показуешь, когда грѣшишь. И сіе"=то есть, что глаголетъ Господь: \textit{сынъ славитъ отца, и рабъ господина своего убоится. И аще отецъ есть Азъ: то гдѣ слава Моя? И аще Господь есмь Азъ: то гдѣ есть страхъ Мой? глаголетъ Господь Вседержитель}\footnote{Малах.~1,~6.}. Тяжко, воистину тяжко, человѣку благодѣтеля своего, рабу господина своего, сыну и дщери отца своего, подначальному начальника своего, далеко тягчае того подданному царя своего оскорбить: коль несравненно тягчае Бога, Царя царствующихъ и Господа господствующихъ и Отца родителей нашихъ и Благодѣтеля благодѣтелей нашихъ оскорбить! Грѣхомъ оскорбляется Богъ, яко достодолжное Ему послушаніе не показуется. О лютое зло "--- грѣхъ! Сладокъ человѣку грѣхъ, но весьма горьки плоды его. И не думай, человѣче, что ты, когда человѣку согрѣшаешь, Богу не согрѣшаешь. Нѣтъ, неправое сіе мнѣніе есть, ибо кто противу человѣка согрѣшитъ, тотъ и противу Бога согрѣшитъ; и кто человѣка оскорбитъ, тотъ и Бога оскорбитъ. Какъ? Богъ повелѣлъ намъ человѣка не обидѣть и не оскорблять, но паче любить его, и помогать ему. Аще убо не дѣлаемъ того, что Онъ повелѣлъ, то не показуемъ Ему послушанія и противимся святой волѣ Его, и тако безумно оскорбляемъ Его. Видишь убо, человѣче, что грѣхъ противу человѣка не можетъ быть безъ грѣха противу Бога, и оскорбленіе человѣка не бываетъ безъ оскорбленія Божія.

Ненавидишь человѣка, котораго Богъ любитъ и велѣлъ тебѣ любить: ненавидишь и Самого Бога. Дѣлаешь обиду человѣку, что Богъ тебѣ запретилъ: дѣлаешь обиду Самому Богу. Видишь, куды грѣхъ твой къ человѣку, оскорбленіе твое и обида твоя, ему показанная, видишь, куды восходитъ?! Къ Самому Богу. А отсюду послѣдуетъ, что, когда хощешь покаяться и примириться Богу, то надобно тебѣ прежде примириться ближнему твоему, котораго ты обидѣлъ и оскорбилъ. И сіе"=то есть, что глаголетъ Господь: \textit{аще убо принесеши даръ твой ко олтарю, и ту помянеши, яко братъ твой имать нѣчто на тя; остави ту даръ твой предъ олтаремъ, и шедъ прежде смирися съ братомъ твоимъ, и тогда пришедъ принеси даръ твой}\footnote{Мѳ.~5,~23 и 24.}. И, что умножаетъ тяжесть и мерзость грѣха и слѣпоту и безуміе человѣческое показуетъ, всякій грѣхъ, какъ и всякое дѣло, предъ Богомъ и всевидящими очами Его совершается, яко Богъ вездѣ есть и вся видитъ. Кто дерзнетъ предъ царемъ земнымъ, или паче, и низшею властію безчинствовать? Грѣшникъ слѣпый дерзаетъ тое дѣлать предъ Богомъ и всевидящимъ окомъ Его. Когда все сіе и вышеписанное разсудитъ человѣкъ, неотмѣнно пріидетъ въ истинное покаяніе, сожалѣніе и печаль сердечную, и будетъ жалѣть и сокрушаться сердцемъ, что грѣшилъ и Бога оскорблялъ и прогнѣвлялъ; и, къ Богу не смѣя возвести очесъ своихъ, стыдъ свой будетъ признавать \textit{стыдно мнѣ и очи мои къ Тебѣ, Господи, возвести!} И уже впредь пожелаетъ лучше умереть, нежели согрѣшить. Познай убо, хрістіанине, кому ты согрѣшаешь, и кого оскорбляешь грѣхомъ, и неотмѣнно, истину тебѣ говорю, будешь сердечно каятися и всякаго берещися грѣха. Тако просвѣщеніе и познаніе Бога, себе самого и грѣха, есть начало покаянія и спасенія. \textit{Тако глаголетъ Господь, Господь Святый Израилевъ: егда возвратився воздохнеши, тогда спасешися, и уразумѣеши, гдѣ еси былъ}\footnote{Ис.~30,~15.}.

\section{144. Междоусобная брань.}

Видимъ и слышимъ, что въ мірѣ бываетъ междоусобная брань. Она бываетъ тогда, когда единаго отечества или единаго города жители другъ на друга востаютъ. Тако хрістіане дѣлаютъ междоусобную брань, когда едины на другихъ востаютъ и обиждаютъ ихъ. Хрістіане тѣсный союзъ между собою имѣютъ, и большимъ союзомъ, нежели единаго отечества или единаго града жители, связаны суть. \textit{Едину} вѣру имѣютъ, \textit{единаго} Бога Отца и Сына и Святаго Духа исповѣдуютъ и Того призываютъ, и молятся Ему и поютъ Его, \textit{едино} Божіе Слово слушаютъ, \textit{единымъ} крещеніемъ крещаются, въ \textit{едину} церковь молиться входятъ, \textit{единыхъ} Таинъ святыхъ причащаются, къ \textit{единому} вѣчному животу позваны, отъ \textit{единаго} Хріста называются \textit{Хрістіанами} и проч. Видишь, хрістіанине, какъ хрістіане между собою связаны; воистину болѣе, нежели сограждане и братія плотская. Но когда едины противу другихъ востаютъ, тогда подвигаютъ междоусобную брань. Сюды надлежитъ: 1)~Воры, хищники, грабители и прочіи подобніи симъ, которыи какъ нибудь похищаютъ у хрістіанъ добро ихъ, то"=есть, или деньги, или платье, или хлѣбъ, или домъ, или землю, или скотъ, или рыбную ловлю, или рощу, или иное что. 2)~Купцы, которыи въ продажѣ товаровъ своихъ хрістіанъ обманываютъ, и болѣе за товаръ просятъ, нежели онъ чего стоитъ. 3)~Господа, которыи крестьянъ своихъ хрістіанъ или оброками, или работами тяжкими обременяютъ, или безчеловѣчно мучатъ, или ругательными словами ихъ называютъ. 4)~Судіи, мздою растлѣнніи, которыи не хотятъ хрістіанъ обижденныхъ и озлобленныхъ удовольствовать безъ мзды. 5)~Лживіи клеветники и ругатели, которыи хрістіанъ какъ нибудь ругаютъ и поносятъ. 6)~Хитрецы, лукавцы, обманщики, которыи какъ нибудь и въ чемъ нибудь хрістіанъ обманываютъ, и имъ ровъ ископываютъ. 7)~Наконецъ вси, которыи какую нибудь дѣлаютъ обиду и озлобленіе хрістіанамъ. Вси сіи и вышеописанніи христіане востаютъ противу хрістіанъ, и противу ихъ брань возставляютъ. О первыхъ хрістіанахъ написано: \textit{народу вѣровавшему бѣ сердце и душа едина}\footnote{Дѣян.~4,~32.}. И Хрістосъ Господь того требуетъ отъ хрістіанъ: \textit{заповѣдь новую даю вамъ, да любите другъ друга: якоже возлюбилъ вы, да и вы любите себе. О семъ разумѣютъ вси, яко Мои ученицы есте, аще любовь имате между собою}\footnote{Іоан.~13,~34--35.}. Было тое нѣкогда, хрістіанине! Нѣтъ, того нынѣ уже не ищи. Изсякла бо любовь многихъ; умножилися беззаконія; день отъ дне усиливается грѣхъ; болѣе и болѣе возрастаютъ обиды и озлобленія. "--- Кому? хрістіанамъ. Отъ кого? отъ хрістіанъ! Братія на братію востали, \textit{и врази человѣку домашніи его}\footnote{Мѳ.~10,~36.}. Вездѣ слышится жалоба и плачъ; воздухъ самый шумитъ, плачевными гласами наполненный. Тамо вдовица жалуется и плачетъ: той"=де сильный человѣкъ отнялъ у мене землю, или рощу, и проч. Индѣ убогій и подлый человѣкъ стенетъ: гдѣ"=де мнѣ жить? такой"=то насильникъ согналъ мене съ земли и домъ мой разорилъ. На другомъ мѣстѣ крестьянинъ рыдаетъ: столько"=то дней въ недѣлѣ работаю я на господина своего; когда уже мнѣ на домъ мой и домашнихъ моихъ работать? Иный подобно тому болѣзнуетъ: весь трудъ мой единъ господинъ или госпожа поядаетъ; чимъ уже мнѣ съ домашними моими питаться? Вси почти согласно жалятся: въ такомъ"=де мѣстѣ и въ такомъ не находимъ удовольствія обидамъ нашимъ; куды уже намъ итить и искать удовольствія? Нигдѣ безъ денегъ не судятъ. Иный печалится и жалѣетъ: такой"=де купецъ въ томъ и въ томъ товарѣ мене обманулъ. А другій ему говоритъ: гдѣ того нѣтъ? въ какомъ городѣ? куды ни пойдешь, и чего ни хощешь купить, вездѣ берегись обмана. Понеже вездѣ бóльшія цѣны просятъ, какъ товаръ стоитъ, и тое призываніемъ имени Божія утверждаютъ, и вси уже къ тому привыкли, и проч. Се суть нынѣшняго вѣка хрістіане! О коль далеко отступили таковіи хрістіане отъ первыхъ хрістіанъ, у которыхъ было \textit{сердце и душа едина!} А отсюду видно, что и хрістіанство ихъ ложное есть. Прежде хрістіане другъ другу помогали, но нынѣ другъ друга гонятъ и озлобляютъ; и отъ сего видно, что пришествіе Господне приближися, и нечаянно день Господень, яко тать въ нощи, пріидетъ. О ослѣпленніи и бѣдніи хрістіане! кто васъ тако научилъ жить? какій пророкъ, какій апостолъ и учитель? Не вси ли научаютъ насъ въ страхѣ Божіи жить, и другъ друга любить, и другъ другу помогать, согласіе и миръ между собою имѣть? Осмотритеся, и увидите, что дѣла ваши отъ злаго духа суть. Онъ любви, согласія и мира ненавидитъ, яко духъ вражды и злобы; почему всякимъ образомъ тщится любовь искоренить и вражду всѣять, и тако научаетъ обиды и озлобленія хрістіанамъ дѣлать. Душа богобоящаяся! претерпи все, что ни дѣлаютъ тебѣ ложніи хрістіане и лицемѣры. \textit{Претерпѣвый бо до конца, той спасется. Буди вѣренъ даже до смерти}, глаголетъ тебѣ Господь, \textit{и дамъ ти вѣнецъ живота}\footnote{Мѳ.~24,~13; Апок.~2,~10.}.

\subsection{О томжде.}

Единаго отечества жителямъ, яко единаго общества удамъ, должно было другъ другу помогать и защищать, и противу внѣшнихъ враговъ единодушно вооружаться; но когда не дѣлаютъ того, отдаютъ себе всѣмъ въ посмѣяніе и своимъ врагамъ въ попраніе, и свое отечество въ разореніе. Ничимъ бо такъ отечество не разоряется, какъ междоусобною и внутреннею бранію. Тако когда хрістіане на хрістіанъ возстаютъ и ихъ озлобляютъ, "--- отдаютъ себе въ посмѣяніе діаволу и служителямъ его идолопоклонникамъ. Радуется діаволъ, врагъ хрістіанскій, о таковыхъ хрістіанахъ. Должно было христіанамъ всѣмъ единодушно противу того врага стоять и подвизаться, и другъ другу помогать, и другъ друга противу его поощрять, и тако съ нимъ единымъ брань творить, якоже сего Господь требуетъ\footnote{Еф.~6,~10--18; 1~Петр.~5,~9.}. Но когда хрістіане на хрістіанъ возстаютъ, то уже не противу діавола, но съ діаволомъ противу хрістіанъ вооружаются, и ему пособствуютъ, и тако угодное ему творятъ. Истинно сіе есть, хотя того и не примѣчаютъ таковіи бѣдніи хрістіане. Кто бо чію волю творитъ, тому и угождаетъ и работаетъ. О! въ коль бѣдное состояніе пришло хрістіанство: другъ друга угрызаютъ и снѣдаютъ! Въ прежнихъ временахъ хрістіане берегли и защищали другъ друга отъ идолопоклонниковъ; нынѣ хрістіане на хрістіанъ возстаютъ, и другъ друга берегутся. Понеже невозможно никому вѣрить. Всякъ бо льститъ и обманываетъ, лукавнуетъ и хитритъ: на языкѣ медъ, а на сердцѣ желчь горести носитъ; устами привѣтствуетъ, но сердцемъ враждуетъ; предъ глазами ласкаетъ и пріятствуетъ, но отшедши поноситъ и ругаетъ. Всякъ того и ищетъ и въ сердцѣ поучается, какъ бы чужимъ чимъ завладѣть. О лютое время! О како возмогла діавольская хитрость, како усилился грѣхъ, како исчезла любовь! Чего въ такое лютое время хрістіанской богобоящейся душѣ ожидать, кромѣ озлобленія и гоненія? Израильтяне, когда думали изъ Египта вытить, тогда тягчайшее озлобленіе отъ Фараона мучителя терпѣли; тако, когда святыя и благочестивыя души отъ міра сего оскудѣваютъ, жесточае на нихъ вооружается діаволъ и озлобляетъ ихъ то чрезъ себе самого, то чрезъ своихъ служителей, и чимъ ближайшій міра конецъ, тѣмъ бо́льшее и жесточайшее благочестивымъ слѣдуетъ страданіе; діаволъ наипаче востаетъ на тѣхъ, которіи противу его стоятъ и подвизаются; а о тѣхъ небрежетъ, которіи волѣ его повинуются. Онъ всѣ силы на то употребляетъ, чтобы сіе \textit{малое стадо} озлобить. Израильтяне, исходя изъ Египта, хотя и озлоблялъ ихъ Фараонъ, однакожъ вышли съ сребромъ и златомъ, по писанному: \textit{изведе я съ сребромъ и златомъ}\footnote{Пс.~104,~37.}; тако хрістіане, хотя и много отъ діавола и служителей его при концѣ страждутъ, однакожъ отъ міра исходятъ съ безцѣннымъ вѣчнаго блаженства сокровищемъ. Терпи убо и утверди сердце твое, благочестивая душа! Мірское въ мірѣ все останется: ты изыдешь отсюду съ вѣчнымъ и небеснымъ сокровищемъ. \textit{Долготерпите убо, братіе моя, до пришествія Господня. Се земледѣлецъ ждетъ честнаго плода отъ земли, долготерпя о немъ, дондеже пріиметъ дождь раннь и позденъ. Долготерпите убо и вы, утвердите сердца ваша, яко пришествіе Господне приближися. Не воздыхайте другъ на друга, братіе, да не осуждени будете: се Судія предъ дверьми стоитъ}\footnote{Іак.~5,~7--9.}.

\section{145. Пчела, жаломъ уязвляющая.}

Сказываютъ, что пчела, когда жаломъ своимъ кого уязвитъ, то уже сама погибаетъ. Тоежде страждетъ и хрістіанинъ, когда какъ нибудь ближняго своего обидитъ и озлобляетъ. Не можетъ онъ обидѣть ближняго своего безъ большія и тягчайшія своея обиды; и чимъ большую ближнему своему дѣлаетъ обиду, тѣмъ болѣе себе обиждаетъ; и чимъ болѣе другаго уязвляетъ, тѣмъ болѣе себе уязвляетъ. Понеже, обиждая ближняго своего, обидою своею касается Самого Бога; яко заповѣдь Божію разоряетъ, которая не велитъ никого обиждать, но всякаго любить и всякому благотворить.

\textit{Возлюбиши искренняго твоего, яко себе самого}\footnote{Лев.~19,~1,~8; Мѳ.~22,~39.}. Болѣе таковый обиждаетъ себе самого, нежели другаго, "--- почему? потому что другаго на тѣлѣ, а себе самого на душѣ обиждаетъ; другаго тѣло, а свою душу уязвляетъ и озлобляетъ. Чимъ бо лучшая и честнѣйшая душа паче тѣла, тѣмъ большая обида, язва и озлобленіе ея паче тѣлеснаго. Всякимъ бо грѣхомъ, какій ни дѣлаетъ человѣкъ, душа его уязвляется и озлобляется. Согрѣшаетъ къ ближнему своему, убо и язву на душѣ своей пріемлетъ. Грѣхомъ своимъ, яко жаломъ, себе уязвляетъ. Сатана, яко духъ человѣку враждебный, всегда ищетъ и тщится человѣка ко грѣху привести. Грѣшишь ли убо, человѣче, "--- отдаешь себе въ обиду діаволу. Гонишь ли человѣка, "--- уже гонитъ тебе діаволъ. Похищаеши ли добро у человѣка, "--- уже похитилъ добро души твоея діаволъ. Прельщаеши ли человѣка и обманываешь, "--- уже прельстилъ и обманулъ тебе діаволъ. Клевещеши ли на человѣка, "--- уже отдалъ ты себе клеветѣ діавольской. Хулиши ли и ругаешися человѣку, "--- уже отдалъ ты себе въ поруганіе діаволу. Біеши ли и уязвляеши человѣка, "--- уже біетъ и уязвляетъ душу твою діаволъ. Смѣешися ли человѣку, "--- уже смѣется тебѣ діаволъ. Презираеши ли и уничижаеши человѣка, "--- уже и діаволъ презираетъ и уничижаетъ тебе, и проч. Тако всякъ человѣкъ грѣшитъ и казнитъ себе. Самый грѣхъ его казнь ему есть. Обиждаетъ другаго и обиждается самъ; уязвляетъ, и уязвляется; озлобляетъ, и озлобляется; біетъ, и біется; убиваетъ, и убивается; лишаетъ и лишается; клевещетъ и оклеветаемъ бываетъ; осуждаетъ, и осужденъ бываетъ; хулитъ, и хулится; ругаетъ и ругается; прельщаетъ, и прельщается; обманываетъ, и обманывается; уничижаетъ; и уничиженъ бываетъ; посмѣвается, и посмѣваемъ бываетъ. Словомъ, когда ни дѣлаетъ ближнему зло, себѣ большее зло дѣлаетъ; яко ближняго тѣлесно и временно, себе же душевно и вѣчно озлобляетъ. Тако грѣшникъ мѣру, которою мѣритъ ближнему своему, себѣ наполняетъ со излишествомъ. Сего ради глаголетъ Духъ Святый: \textit{Пріидите чада, послушайте Мене, страху Господню научу васъ. Кто есть человѣкъ, хотяй животъ, любяй дни видѣти благи? Удержи языкъ твой отъ зла, и устнѣ твои еже не глаголати льсти; уклонися отъ зла и сотвори благо; взыщи мира, и пожени и. Очи Господни на праведныя, и уши Его въ молитву ихъ: лице же Господне на творящія злая, еже потребити отъ земли память ихъ}\footnote{Пс.~33,~12--17.}.

\section{146. Уязвленный уязвляется.}

Бываетъ, что мучитель, довольно ранъ наложивши человѣку, паки біетъ его и раны къ ранамъ прилагаетъ, и тако лютѣйшую болѣзнь ему содѣловаетъ. Видитъ міръ таковое мучительство. Тако дѣлаютъ тіи именующіися хрістіане, которыи бѣднымъ людямъ какую нибудь дѣлаютъ обиду, и тако бѣдствіе къ бѣдствію ихъ придаютъ. Таковіи суть: 1)~Которыи печальнаго, или въ болѣзни находящагося, или иное что страждущаго, коварно опечаляютъ, и ему досаждаютъ. Тако сотворили жестоковыйніи и жестокосердечніи жиды, которыи Хріста Сына Божія, уже поруганнаго, уязвленнаго, умученнаго и на крестѣ страждущаго, хулили и ругали. \textit{Мимоходящіи хуляху Его, покивающе главами своими, и глаголюще: разоряяй церковь, и треми денми созидаяй, спасися самъ. Аще Сынъ еси Божій, сниди со креста}, и проч.\footnote{Мѳ.~27,~39--41.} И тако \textit{къ болѣзни язвъ Его приложиша}\footnote{Пс.~68,~27.}. О лютаго безчеловѣчія! О долготерпѣнія Твоего, Іисусе! 2)~Судіи, которыи озлобленному человѣку удовольствія не дѣлаютъ, и его обвиняютъ и осуждаютъ невиннаго аки виновнаго, и тако болѣзнь къ болѣзни придаютъ.

3)~Господа, которыи съ бѣдныхъ и убогихъ крестьянъ своихъ послѣднее ихъ имѣніе берутъ, или ихъ излишними работами отягощаютъ, и тако оставляютъ ихъ безъ пропитанія. 4)~Сильніи, но безстрашніи люди, которыи у вдовицъ, сиротъ и прочихъ бѣдныхъ и беззаступныхъ людей отнимаютъ землю или иное что, отъ чего они пропитаніе себѣ получаютъ. 5)~Тіи безсовѣстныи люди, которыи, видя пожаръ, что нибудь у хозяина, котораго домъ горитъ, похищаютъ и крадутъ. 6)~Сребролюбивіи богачи и прочіи люди, которыи бѣднымъ людямъ даютъ взаимъ, но съ нихъ процентъ берутъ, и чимъ болѣе процентовъ берутъ, тѣмъ болѣе имъ бѣды дѣлаютъ. 7)~Тіи люди, которыи у наемниковъ мзду удерживаютъ. 8)~Тіи господа и приставники, которыи людямъ, себѣ подчиненнымъ, жалованья, отъ Государя имъ опредѣленнаго, безъ правильной причины не даютъ. 9)~Сюды надлежать и тіи безстудніи насильники, которыи, ради скверной своей похоти, женъ у другихъ отнимаютъ, и проч. Сіе и прочее зло видитъ міръ. Вси сіи люди подобно дѣлаютъ тому мучителю, который уязвленнаго уже уязвляетъ еще, и умученнаго уже еще мучитъ. Люто есть безчеловѣчіе! Таковыя беззаконія видятся въ хрістіанахъ. Таковыи люди и послѣднюю искру не токмо хрістіанства, но и человѣчества потеряли. Таковыя беззаконія, какъ содомскіе и гоморрскіе грѣхи, на небо вопіютъ. Въ которомъ градѣ и странѣ злая сія дѣлаются, тамо ищутъ люди себѣ погибели. Хрістіанине, какой ты милости чаешь себѣ у Бога, у Бога Отца сирыхъ и Судіи вдовицъ, ты, который какое нибудь вышеписанное зло бѣдному дѣлаешь? и чего себѣ ожидаешь въ будущемъ вѣкѣ, который таковое мучительство человѣку подобному тебѣ показуешь? Развѣ хощешь съ тѣми безчеловѣчными людьми быть, которыи нѣкогда хрістіанъ, какъ мясо снѣдное, пекли, и въ огнѣ жгли, и въ водѣ потопляли, и зубамъ звѣринымъ предавали, и прочее безчеловѣчное мученіе имъ дѣлали? Будешь, когда не покаешися, и беззаконій твоихъ слезами не омыешь. Аще бо \textit{за праздное слово отвѣтъ воздадятъ} люди, по словеси Хрістову, \textit{въ день судный}\footnote{Мѳ.~12,~36.}; кольми паче за беззаконія и такъ злая дѣла истязаны будутъ. Отраднѣе будетъ тогда Содомлянамъ, Туркамъ и идолопоклонникамъ, нежели таковымъ хрістіанамъ, которыи всегда слышатъ слово Божіе, обличающее и наказующее ихъ, но исправиться не только не хотятъ, но еще и въ горшее успѣваютъ. Познай убо, хрістіанине, свою пагубу, и покайся, пока время не ушло; покайся, когда не хощешь въ гееннѣ огненной, какъ желѣзо во огнѣ, вѣчно горѣть, но никогда не сгарать. \textit{Огнь ихъ не угасаетъ, и червь ихъ не усыпаетъ}\footnote{Ис.~64,~24; Марк.~9,~44.}.

\section{147. Зеркало пороки на лицѣ показуетъ.}

Что пороки на лицѣ, тое грѣхи на душѣ. Что зеркало лицу и порокамъ, тое есть обличительное слово душѣ и грѣхамъ. Зеркало показуетъ намъ тое, что имѣется на лицѣ: когда имѣются на лицѣ пороки, пороки и показуетъ; когда нѣтъ ихъ на лицѣ, то и не показуетъ ихъ. Тако имѣется и обличительное слово; якоже зеркало, грѣхи на душѣ показуетъ. Какіе грѣхи въ душѣ имѣются, тѣ показуетъ и обличаетъ; какихъ грѣховъ душа не имѣетъ, тѣхъ и слово уже не обличаетъ. Когда человѣкъ, усмотрѣвши на лицѣ своемъ пороки, заглаждаетъ ихъ, то уже и зеркало ихъ не показуетъ: тако, когда душа, отставши отъ грѣховъ, очиститъ ихъ покаяніемъ и вѣрою, уже и слово обличительное ея не касается. А что слово обличительное, тое и совѣсть дѣлаетъ душѣ человѣческой. Совѣсть бо согласна съ закономъ и словомъ обличительнымъ. Сего свидѣтеля внутренняго свидѣтельство вѣрно есть: что видитъ въ душѣ, тое свидѣтельствуетъ и обличаетъ; чего не видитъ, того и не обличаетъ. Тако два свидѣтеля и обличителя душѣ человѣческой положилъ Богъ: внѣ, \textit{законъ Свой}; внутрь души, \textit{совѣсть}. Оба сіи свидѣтели вѣрно и согласно свидѣтельствуютъ и обличаютъ насъ. Они будутъ свидѣтели всякому и на второмъ Хрістовомъ пришествіи. Что мы ни дѣлали нынѣ, о томъ они тамо будутъ свидѣтельствовать. Грѣшили ли мы здѣ, "--- грѣхи наши тамо будутъ обличать. Покаялися ли и загладили грѣхи наши здѣ, "--- уже и они не будутъ ихъ тамо обличать. Добрая ли дѣла творили мы здѣ, "--- и они тамо о нихъ будутъ свидѣтельствовать, и похвалятъ насъ. Сіе видимъ и нынѣ. Когда мы добро творимъ, то они насъ похваляютъ. Похваляетъ законъ за добро, похваляетъ и совѣсть; милость похваляетъ законъ, похваляетъ и совѣсть; кроткихъ и терпѣливыхъ ублажаетъ законъ, ублажаетъ ихъ и совѣсть; неповинно страждущаго руганіе, хуленіе, или ссылку, или узы, или иное какое зло, утѣшаетъ слово Божіе, утѣшаетъ его и совѣсть. Тако какъ самый грѣхъ есть человѣку казнь, такъ самая добродѣтель есть ему мзда и награжденіе. Отсюду послѣдуетъ: 1)~Коль нужна есть проповѣдь закона Божія, дабы люди изъ того познали грѣхи своя, якоже изъ зеркала познаютъ пороки на лицѣ, и тако бы познавши, заглаждали ихъ покаяніемъ и вѣрою, и впредь бы ихъ береглись, "--- которое дѣло до пастырей наипаче надлежитъ. 2)~Не должно гнѣваться никому за слово обличительное. Не гнѣваешися, человѣче, на зеркало, что пороки на лицѣ твоемъ показуетъ: почто жъ гнѣваться на проповѣдника, что онъ словомъ грѣхи твои тебѣ показуетъ? Зеркало показуетъ тебѣ тое, что имѣется на лицѣ твоемъ: тако обличительное слово обличаетъ тебе въ томъ, что есть въ душѣ твоей. 3)~Кто гнѣвается за слово обличительное, видно, что онъ тѣ грѣхи въ себѣ имѣетъ, которые слово обличаетъ вообще. Ибо знаменіе есть, яко, что въ словѣ онъ слышитъ, въ томъ обличаетъ его и совѣсть его. Когда убо, человѣче, гнѣваешися за обличительное слово на проповѣдника, то гнѣвайся и на совѣсть твою, которая тебе обличаетъ. Однакожъ, какъ ни ярись, она не престанетъ тебе обличать, всегда будетъ тебѣ говорить, что видитъ въ тебѣ. Лучше убо гнѣваться на себе самого, что грѣшилъ и совѣсть свою уязвлялъ. Покайся убо и престани дѣлать тое, чимъ совѣсть уязвляется; и совѣсть уязвлять тебе уже не будетъ, и слово обличительное не будетъ тебе касаться. Тогда покой, миръ и радость въ душѣ твоей благодатію Божіею будетъ вселяться. 4)~Якоже зеркало ради того поставляется, чтобы пороки на лицѣ познавалися, и тако бы стирались: тако обличительное слово не ради иной какой причины должно говорить, но ради того только, дабы люди познали свои грѣхи и каялися, и тако бы себе исправили.

\section{148. Какъ мнѣ его забыть!}

Слышимъ, что слово сіе единъ о другомъ говоритъ: \textit{какъ мнѣ его забыть!} Говорится оно въ такомъ случаѣ, когда единъ другому говоритъ: \textit{часто ты его поминаешь}, то"=есть, такого"=то человѣка; тогда поминающій отвѣщаетъ: \textit{какъ мнѣ его забыть!} я"=де отъ него много благодѣяній получилъ. Христіанине, мнѣ и тебѣ наипаче о Бозѣ нашемъ сердечно должно говорить: \textit{какъ мнѣ Его забыть!} Мы отъ Бога нашего созданы, искуплены, и столько благодѣяній отъ Него получили и получаемъ всегда, что не только словомъ изъяснить, но и умомъ понять ихъ не можемъ: безмѣрна бо и безчисленна суть, вѣдома и невѣдома; словомъ, заключаемся въ Божіихъ благодѣяніяхъ, и безъ нихъ и минуты жить не можемъ. \textit{Какъ убо намъ Его забыть}, Его, толикаго нашего Благодѣтеля! Не забудемъ убо Его, хрістіанине! Къ ветхому Израилю чрезъ пророка Своего говорилъ Богъ: \textit{сотворихъ тя раба Моего, Іакове, и ты, Израилю, не забывай Мене}\footnote{Ис.~46,~21.}. Тако христіанину глаголетъ Тойжде Богъ: \textit{раба Моего сотворихъ тя}, хрістіанине, \textit{и ты не забывай Мене}. Когда сотворенъ, когда крестился хрістіанинъ во имя Отца и Сына и Святаго Духа: тутъ сотворенъ хрістіанинъ рабомъ Божіимъ отъ Бога; прежде былъ рабомъ діавольскимъ, но въ крещеніи святомъ сотворился уже рабомъ Божіимъ. Помяни сія, хрістіанине, яко рабъ Божій еси ты: \textit{сотворилъ тя} Господь \textit{раба Своего, и ты}, хрістіанине, \textit{не забывай Его}. Именемъ Хрістовымъ хрістіанинъ, яко царскою печатію назнаменанъ, \textit{хрістіанинъ} отъ Хріста называется. О древнихъ Израильтянахъ слышимъ, что когда они идолу поклонилися, то \textit{и забыша Бога, спасающаго ихъ}, и проч.\footnote{Пс.~105,~21.} Тако хрістіане забываютъ Бога, спасающаго ихъ, когда послѣ крещенія обращаются къ беззаконіямъ, и, сдѣлавшися рабами Божіими, дѣлаются самовольно рабами грѣха. \textit{Всякъ} бо, \textit{творяй грѣхъ, рабъ есть грѣха}\footnote{Іоан.~8,~34.}. Всякъ убо законопреступникъ и беззаконникъ забываетъ Бога"=Благодѣтеля своего, яко заповѣди Его святыя забываетъ. Забываетъ блудникъ и прелюбодѣй; забываетъ воръ, хищникъ и грабитель; забываетъ клеветникъ, ругатель и всякъ злорѣчивый; забываетъ прелестникъ, обманщикъ и лживый; забываетъ всякъ, заповѣдь Божію безстрашно разоряющій. Когда убо говоришь, хрістіанине, о Бозѣ, \textit{какъ мнѣ Бога забыть}: то не забывай и заповѣдей Его святыхъ, но имѣй ихъ всегда предъ собою, яко заповѣди Божія, и помни творити ихъ всегда. Когда сіе будешь творить, то подлинно и Самого Бога не будешь забывать. Иначе хотя на всяку минуту имя Божіе поминаешь, а о заповѣдяхъ Божіихъ небрежешь, то забываешь Бога"=Благодѣтеля своего, и тако Ему неблагодарнымъ показуешися. Память бо твоя о Немъ только на устахъ и языкѣ, а въ сердцѣ нѣтъ ея, но забвеніе. Человѣкъ бо, истинно къ своему благодѣтелю благодарный, не токмо его на устахъ, но и въ сердцѣ его поминаетъ, и благодѣянія его помнитъ: тако хрістіанинъ не токмо на устахъ носитъ имя Божіе, но и въ сердцѣ имѣть память Его, и любить Его и благодарить Ему долженъ. Память бо и благодарность и любовь суть неразлучны. Кто любитъ благодѣтеля своего, тотъ всегда поминаетъ его; кто поминаетъ, тотъ къ благодаренію благодѣяніемъ его возбуждается. Невозможно бо не благодарить благодѣтелю, и искренно не помнить благодѣянія его. Ибо самое благодѣяніе возбуждаетъ благодарность. Самые скоты помнятъ благодѣяніе, и любятъ благодѣтеля и услуживаютъ ему. Аще убо, хрістіанине, устами не забываешь Бога, то не забывай и сердцемъ; безъ того память твоя ничто. Аще молишися Богу: \textit{помяни мя, Господи, во царствіи Твоемъ}; то поминай и ты Его здѣ въ мірѣ семъ такъ, какъ выше сказано. Аще говоришь къ Богу: \textit{не забуди мене}; и Онъ тебѣ говоритъ: \textit{и ты}, хрістіанине, \textit{не забывай Мене} и Моихъ заповѣдей.

\subsection{О томжде.}

Человѣкъ къ воспоминанію своего благодѣтеля возбуждается, видя его благодѣянія, то"=есть, или домъ, который онъ ему состроилъ, или одежду, которою его одѣлъ, или иное что. Сію"=де одежду тотъ мой благодѣтель подалъ мнѣ; сей хлѣбъ, который я вкушаю, отъ него я получилъ; сей домъ, въ которомъ я живу, онъ мнѣ по своей милости построилъ, и проч. Тако хрістіанина къ любовному воспоминанію Бога"=Благодѣтеля вся Его благодѣянія подвигнуть должны. Хрістіанине! все и всякое добро какое ни имѣешь у себе, Божіе добро есть и благодѣяніе. Все сіе возбуждаетъ тебе къ воспоминанію Бога. Что убо ни видишь, помышляй и говори въ себѣ тако: сей домъ, въ которомъ я живу и упокоеваюся; сія пища, которой я вкушаю и укрѣпляюся; сіе питіе, которымъ я прохлаждаюся и утѣшаюся; сія одежда, которою я прикрываюся и согрѣваюся; сей скотъ, который мнѣ служитъ; сей огнь, который мене согрѣваетъ и варитъ пищу мнѣ, сіи свѣтила (солнце, луна и звѣзды), которыя мнѣ и всей твари свѣтятъ, и прочая, "--- суть Божіе добро и благодѣяніе. Всегда получаешь отъ Бога благодѣяніе, и всегда Божіимъ добромъ пользуешися и утѣшаешися такъ, что и минуты безъ того быть не можешь. Всегда убо и на всякое время долженъ ты Бога поминать, и Ему за все благодарить, и со Псаломникомъ Его благословить: \textit{благословлю Господа на всякое время, выну хвала Его во устѣхъ моихъ}\footnote{Пс.~33,~2.}. Погибаетъ тое время, день, часъ и минута, въ которое Господь отъ насъ не благословляется; понеже безпрестанно, какъ рѣки, текутъ благодѣянія Божія и на насъ щедро изливаются. \textit{Щедръ и милостивъ Господь, долготерпѣливъ и многомилостивъ. Благъ Господь всяческимъ, и щедроты Его на всѣхъ дѣлѣхъ Его. Да исповѣдятся Тебѣ, Господи, вся дѣла Твоя, и преподобніи Твои да благословятъ Тя}\footnote{Пс.~144,~8--10.}. Изрядно на память приводятъ намъ Бога и благодѣянія Его святые образа. \textit{Образъ Благовѣщенія} показуетъ намъ, како человѣколюбивый Господь, хотя отъ Дѣвы плотію родитися и пріити въ міръ, и міру явитися, и тако спасеніе наше содѣлать, послалъ архангела Своего благовѣстить тое человѣколюбіе Свое преблагословеннѣй Дѣвѣ, а чрезъ Нее и всему міру. \textit{Образъ спасительнаго Рождества Христова} на память приводитъ намъ, како Богъ безначальный начался, и невидимый показался, и невещественный воплотился, и неосязаемый осязался, и ветхій деньми младенствовалъ, и всесильный безсильнымъ сотворился, и содержай въ руцѣ Своей вси концы земли матерними руками носился, и даяй пищу всякой плоти матернимъ млекомъ питался, и одѣяйся свѣтомъ яко ризою, пеленами младенческими повивался, о чудо "--- Богъ младенствовалъ! Мене ради и тебе, человѣче, сотворилъ сіе Господь нашъ. Благословите Господа вси сыны человѣчестіи! Благослови душе моя Господа! Помяни сіе, хрістіанине, \textit{и не забывай} Господа. \textit{Образъ Хрістовыхъ страданій} представляетъ намъ, како Онъ грѣхи наши очищалъ, и правдѣ Божіей за насъ удовлетворялъ, и Бога, небеснаго Своего Отца, умилостивлялъ, и Его намъ, и насъ Ему примирялъ и отъ діавола, смерти и ада насъ искуплялъ, и двери Божіяго царствія, нашими грѣхами затворенныя, отверзалъ, и тако намъ спасеніе и блаженство вѣчное содѣловалъ. Сія и прочая высочайшая Божія благодѣянія представляетъ намъ образъ Хрістовыхъ страданій. Поминай сія, хрістіанине, и \textit{не забывай} пострадавшаго за тебе Іисуса Хріста Сына Божія. Такожде все священное Писаніе, яко слово Божіе, вси въ немъ прописанныя чудесныя Божія дѣла и благодѣянія Его, роду человѣческому показанныя, вси праздники Божіи, отъ святой церкви уставленные, на память намъ приводятъ Бога, и высочайшая Его къ намъ благодѣянія, и къ благодаренію и славословію имени Божія возбуждаютъ насъ, якоже тое слышимъ въ церковныхъ стихахъ и пѣсняхъ, въ святые праздники поемыхъ. Внимай сему, хрістіанине, и \textit{не забывай} Бога"=Благодѣтеля своего. Точно знай, что забываютъ Бога тіи хрістіане, которыи безъ страха Божія живутъ, и заповѣдямъ Его святымъ не внимаютъ, и противу Его согрѣшаютъ и беззаконнуютъ. Сіи вси во тьмѣ невѣдѣнія Божія ходятъ, Бога не знаютъ, хотя Его и исповѣдуютъ, и тако забываютъ Его. Божіими благими довольствуются, но Бога"=Благодѣтеля забываютъ. Къ таковымъ глаголетъ пророкъ: \textit{разумѣйте сія} (то"=есть, что Богъ грѣхи ваши предъ лицемъ вашимъ представитъ), \textit{забывающіи Бога, да не когда похититъ, и не будетъ избавляяй}\footnote{Пс.~49,~22.}. Берегись убо, хрістіанине, Бога и святыя заповѣди Его забывать, да не отъ Бога во вѣки забвенъ будеши. \textit{Жива будетъ душа моя, и восхвалитъ Тя: и судьбы Твоя помогутъ мнѣ. Заблудихъ яко овча погибшее: взыщи раба Твоего, яко заповѣдей Твоихъ не забыхъ}\footnote{118,~175 и 176.}. \textit{Помяни мя, Господи, во царствіи Твоемъ}\footnote{Лук.~23,~42.}.

\section{149. Тамо намъ безопасно будетъ.}

Слышимъ, что люди говорятъ слово сіе: \textit{тамо намъ безопасно будетъ}. Говорятъ слово сіе отлучившіися отъ отечества и дома своего, и желающіи къ тому возвратитися, и тако утѣшая себе говорятъ: \textit{тамо намъ безопасно будетъ}, то"=есть, во отечествѣ и домѣ своемъ. Такожде говорится слово сіе и о другомъ мѣстѣ, гдѣ люди надѣются безопасно жить. Однакожъ, о какомъ мѣстѣ міра сего ни говорится слово сіе, не совсѣмъ истинно говорится. Почему? потому что нѣтъ въ мірѣ семъ совершенно безопаснаго мѣста, какъ ниже увидимъ. Гдѣ ни живетъ человѣкъ, пока въ мірѣ семъ живетъ, не безъ опасности живетъ. Человѣку отъ утробы матерней до гроба непрестанная брань. Того ради прилично и правильно о вѣчномъ животѣ слово сіе говорить намъ должно, хрістіанине: \textit{тамо намъ воистину безопасно будетъ}. Тамо ничего мы опасаться не будемъ; тамо будемъ жить безъ всякаго страха, будемъ жить во градѣ покойномъ и мирномъ; тамо совершенный миръ и покой будемъ имѣть. Откуду вѣчный животъ въ священномъ Писаніи называется покой\footnote{Мѳ.~11,~29; Евр.~4,~11.}. Здѣ живя, опасаемся отъ брани и нашествія иноплеменниковъ и враговъ: тамо того не будетъ; ибо едино царствіе Божіе, исполненное мира, правды и радости будетъ. Здѣ опасаемся отъ пожара, глада, моровой язвы: тамо ничего того не будетъ. Здѣ опасаемся отъ злыхъ людей, которыи то явно, то тайно дѣлаютъ намъ навѣты: тамо того не будетъ; ибо тамо они мѣста не будутъ имѣть. Здѣ опасаемся воровъ и хищниковъ, которыи то явно, то тайно, то лестно имѣніе наше отнимаютъ у насъ: тамо того не будетъ; понеже оттуду они вси изгонятся. Здѣ опасаемся отъ необузданнаго языка человѣческаго, не малый бо гонитель нашъ есть клеветникъ: тамо того не будетъ; ибо мѣсто оное неприступно есть таковымъ. Здѣ опасаемся отъ лжебратіи, \textit{и врази человѣку домашніи его}: тамо того не будетъ; ибо тамо единомысленная и прелюбезная братія. Здѣ не безопасное и дружество бываетъ: тамо беззазорное, истинное и прелюбезное дружество будетъ; ибо вси будутъ имѣть едино сердце, мысль, волю, хотѣніе и совершенную другъ къ другу любовь. Здѣ опасаемся отъ еретиковъ и развратниковъ святыя вѣры: тамо того не будетъ; ибо оттуду они удалятся и заключатся въ своихъ мѣстахъ. Здѣ опасаемся отъ плоти и соблазновъ міра: тамо того не будетъ. Здѣ опасаемся отъ грѣха: тамо того не будетъ; ибо будетъ совершенная правда и любовь. Здѣ опасаемся отъ болѣзни и печали: тамо того не будетъ; ибо тамо не будетъ болѣзни, печали и воздыханія. Здѣ опасаемся отъ діавола и злыхъ демоновъ: тамо того не будетъ; ибо они заключены будутъ въ своемъ мѣстѣ. Здѣ опасаемся отъ смерти и ада: тамо того не будетъ; тамо надъ смертію и адомъ вѣчно будемъ торжествовать. \textit{Гдѣ твое, смерте, жало? гдѣ твоя, аде, побѣда? Жало же смерти, грѣхъ: сила же грѣха, законъ. Богу же благодареніе, давшему намъ побѣду Господемъ нашимъ Іисусъ Хрістомъ}\footnote{1~Кор.~15,~55--57.}. Видишь, хрістіанине, опасность и безопасность. Въ мірѣ семъ живущему вездѣ и всегда опасно жить: будущая жизнь никакой опасности не имѣетъ. О ней правильно и утѣшительно можемъ и должны мы говорить: \textit{тамо намъ безопасно будетъ}. Тамо мы не будемъ бояться ни брани, ни враговъ, ни голода, ни пожара, ни моровой язвы, ни навѣтниковъ, ни болѣзни, ни клеветниковъ, ни ругателей, ни татей и хищниковъ, ни діавола и злыхъ его духовъ, ни соблазновъ, ни смерти и ада, "--- словомъ, никакого неблагополучія. Ибо всякое неблагополучіе оттуду удалено, и всякое совершенное благополучіе, \textit{совершенное}, но вѣчное будетъ тамо. А отсюду видишь: 1)~Коль бѣдное въ мірѣ семъ житіе наше. Непрестанно бѣды и напасти окружаютъ насъ, и, какъ терніе, бодутъ насъ. Одна прошла бѣда, тотчасъ другая наступаетъ, по другой третія, по третьей четвертая, и далѣе. И тако не ино что въ мірѣ семъ дѣлаемъ, какъ только бѣдствуемъ; и яко корабль отъ вѣтровъ и волнъ на морѣ, тако мы въ житіи семъ отъ бѣдъ и напастей колеблемся. Гдѣ убо и какой безопасности въ мірѣ семъ искать? Отыдешь въ пустыню, и отъ всѣхъ людей и навѣтниковъ удалишися? Тамо большее нападеніе діавола и множайшія козни его обыдутъ тебе. 2)~Отсюду послѣдуетъ, какъ опасно обращаться должны мы, въ мірѣ семъ живучи. \textit{Блюдитеся, како опасно ходите}, глаголетъ Апостолъ\footnote{Еф.~5,~15.}. Какъ птица, сѣдя всюду смотритъ и бережется стрѣльца и ловца, тако хрістіанину всего въ мірѣ семъ опасаться должно. 3)~Тщаніе наше безъ помощи Божіей не сильно, паче отъ невидимыхъ враговъ. Сего ради непрестанно должно воздыхать и молиться Богу, дабы Самъ насъ всесильною Своею десницею сохранилъ. Хрістіанине! читай псалмы святые, и увидишь, како Давидъ святый молитвою и хваленіемъ имени Божія, аки крѣпкою стѣною, отъ враговъ своихъ защищался. \textit{Хваля призову Господа, и отъ врагъ моихъ спасуся}\footnote{Пс.~17,~4.}. Дѣлай и ты такожде, да спасешися отъ навѣтовъ вражіихъ. \textit{Яко умножилъ еси милость Твою, Боже: сынове} же \textit{человѣчестіи въ кровѣ крилу Твоею надѣятися имутъ}\footnote{35,~8.}. 4)~Аще такъ опасное и бѣдное въ мірѣ семъ житіе наше, хрістіанине: почто намъ желать долгаго въ мірѣ семъ житія? Въ мірѣ семъ жить не иное что есть, какъ всегда бѣдствовать, убо и желать долго въ мірѣ семъ жить не иное что, какъ долго бѣдствовать. Сего ради пророкъ святый о семъ сѣтуетъ и воздыхаетъ, глаголя: \textit{увы мнѣ, яко пришельствіе мое продолжися}\footnote{Пс.~119,~5.}. Лучшій день блаженныя смерти, нежели рожденія. Раждаемся на бѣды, но блаженная смерть полагаетъ конецъ всѣмъ бѣдамъ. Желай убо, хрістіанине, не долгаго житія, какъ сыны вѣка сего дѣлаютъ; но желай блаженныя кончины. 5)~Аще въ вѣчномъ животѣ едина только безопасность, миръ, покой, радость, веселіе, и истинное и совершенное и вѣчное блаженство: поспѣшимъ туды, хрістіанине, поспѣшимъ, презрѣвши всю суету міра сего; \textit{тамо намъ безопасно будетъ. Блаженъ, егоже избралъ еси и пріялъ: вселится во дворѣхъ Твоихъ, исполнится во благихъ дому Твоего: святъ храмъ Твой, дивенъ въ правдѣ. Услыши ны Боже Спасителю нашъ, упованіе всѣхъ концей земли, и сущихъ въ мори далече}\footnote{64,~5 и 6.}. \textit{Коль возлюбленна селенія Твоя, Господи силъ! Желаетъ и скончавается душа моя во дворы Господни} (и прочее псалма 83"~го до конца).

\section{150. Образъ живописный портится.}

Аще бы образъ живо отъ нѣкоего художника написанъ былъ, а сыскался бы другій какій художникъ неискусный, и той бы образъ переправилъ по своему хотѣнію, "--- всякъ бы тому безумному дѣлу смѣялся, и живописцу, написавшему образъ, немалая бы обида была и досажденіе. Подобно дѣлаютъ тіи безстудныя жены, которыя лица своя намазываютъ, и бѣлилами и красками украшаютъ ихъ. Богъ и Создатель нашъ, яко премудрый живописецъ, какъ весь составъ тѣла и души всякому человѣку подалъ, такъ и изображеніе лица всякому подаетъ. Но когда намазываютъ и украшаютъ лица своя люди, "--- дѣло рукъ Его святыхъ передѣлываютъ и портятъ, и тако, какъ Самому Богу Создателю своему дѣлаютъ обиду и досажденіе, такъ и людямъ разумнымъ отдаютъ себе въ посмѣхъ, или паче, сожалѣніе, а юнымъ сердцамъ въ соблазнъ. Вотъ куды ведетъ мазаніе и украшеніе женскихъ лицъ! Дѣло Божіе, аки бы не добрѣ сотворенное (что мыслить и говорить страшно), дѣло Божіе передѣлываютъ таковыя безстудницы. Въ юныхъ сердцахъ плоть, страстною похотію жегомую, болѣе разжигаютъ. Дѣлаютъ тое тіи, которыи Хріста, Сына Божія, за грѣхи наши распятаго и умершаго, исповѣдуютъ! О прелести, о хитрости діавольской! О како возмогло злаго духа коварство! Великое орудіе діавольское есть таковое лицъ украшеніе. Ничимъ онъ болѣе не уловляетъ человѣческихъ сердецъ, какъ сею своею сѣтію; ничимъ болѣе не боретъ хрістіанъ, какъ симъ своимъ орудіемъ. Безъ поджоги горитъ плоть, а паче у юнаго; но вотъ еще отъ безстудницъ подлагается поджога. Горе имъ, яко огонь ко огню придаютъ! Какъ не стыдно таковымъ безстудницамъ въ народъ выходить! Скажи, скажи пожалуй, ради чего ты съ таковою личиною въ люди исходишь? Показаться? подлинно показуешь себе, и, что въ сердцѣ твоемъ крыется, показуешь. Видятъ и люди, но иніи отвращаются тебе, яко страшилища; иніи обращаютъ очи свои на тебе, и \textit{зачинаютъ болѣзнь, и рождаютъ беззаконіе}. Худо таковіи дѣлаютъ, и себѣ и ближнимъ во вредъ великій, что на обѣды и прочая собранія тако приходятъ; но далеко хуже и въ большую себѣ пагубу дѣлаютъ, что съ таковымъ безобразіемъ и въ храмы святые, гдѣ слово Божіе проповѣдуется, и святое имя Божіе славословится, и общая молитва отъ вѣрныхъ къ Богу приносится, и святыя и страшныя тайны совершаются, и Самъ Богъ особливо присутствуетъ, "--- въ храмы, говорю, святые дерзаютъ входить. Аще бы кто вопросилъ таковую: за чимъ ты сюды, въ сіе \textit{святое и священное мѣсто} пришла, "--- и аще бы отвѣщала: Богу молиться; то должно ей во обличеніе сказать: съ сею ли утварію приходятъ на молитву? Въ молитвѣ къ Богу приступаемъ, Богу предстоимъ, бесѣдуемъ и милости просимъ у Него: такая ли утварь и уборъ требуется здѣ? Иное слово, иное уборъ и утварь твоя являетъ. Молиться Богу пришла ты, но иное значитъ утварь твоя. Молитвѣ приличная утварь: воздыханіе, слезы, плачь, колѣнопреклоненіе, біеніе въ перси, а не тѣла и лица украшеніе. Аще воздыхать и плакать будеши, уборъ твой обличитъ тебе, и покажетъ твое лицемѣріе. Показать убо себе, а не молиться приходишь въ такомъ уборѣ. Осмотрись убо, за чимъ и въ какое мѣсто съ такою утварію приходишь, да не съ тягчайшимъ грѣхомъ изыдеши. \textit{Свято и страшно мѣсто сіе, яко домъ Божій}\footnote{Быт.~28,~17.}. Берегись убо въ святомъ и страшномъ мѣстѣ нечистоты исполненное сердце имѣть и прочихъ къ тому соблазнять, и храмъ Божій безчестнымъ мѣстомъ дѣлать, да не дознаеши судъ Божій на себѣ. Какъ осмотришися, то сама послѣ будешь каятися и жалѣть, что такою утварію себе украшала, и что не токмо въ храмы святые, но и въ прочія мѣста тако выходила, "--- чего тебѣ сердечно желаю. \textit{Имже (женамъ) да будетъ, не внѣшняя плетенія власъ, и обложенія злата, или одѣянія ризъ лѣпота, но потаенный сердца человѣкъ, въ неистлѣніи кроткаго и молчаливаго духа, еже есть предъ Богомъ многоцѣнно. Тако бо иногда и святыя жены, уповающія на Бога, украшаху себе, повинующеся своимъ мужемъ: якоже Сарра послушаше Авраама, господина того зовущи}, и проч.\footnote{1~Петр.~3,~3--6.} \textit{Такожде и жены во украшеніи лѣпотномъ, со стыдѣніемъ и цѣломудріемъ да украшаютъ себе, не въ плетеніихъ, ни златомъ или бисерми, или ризами многоцѣнными, но еже подобаетъ женамъ, обѣщавающимся благочестію, дѣлы благими}\footnote{1~Тим.~2,~9 и 10.}.

\section{151. Я не твой братъ.}

Слышимъ, что единъ другому слово сіе говоритъ: \textit{я"=де не твой братъ}. Чудно, что человѣкъ человѣку говоритъ, но не стыдится говорить: \textit{я"=де не твой братъ!} Осмотрись, человѣче, отъ какого духа слово сіе произносишь: \textit{я не твой братъ}; и самъ увидишь, что духъ гордости запахъ свой злый издаетъ тако. Когда ты не его братъ, то чій? Онъ человѣкъ, а тебе какъ назвать "--- ангеломъ, или бѣсомъ? скажи, скажи пожалуй; самъ бо говоришь человѣку \textit{не твой братъ}. Я"=де высокій, а онъ низкій; я"=де богатъ, а онъ нищъ; я"=де благороденъ, а онъ подлый; я"=де господинъ, а онъ рабъ; я"=де честенъ, а онъ безчестенъ; я"=де добрый человѣкъ, а онъ злый, и проч. О человѣче! посмотри на Хріста, Сына Божія: кто Его выше, кто Его благороднѣе, кто Его богатѣе, кто Его честнѣе, кто Его славнѣе, кто Его лучше, кто Его премудрѣе? Никто Ему сравниться не можетъ и ни въ чемъ. Есть бо Богъ вѣчный, есть Царь царствующихъ и Господь господствующихъ, и во свѣтѣ живый неприступномъ. И когда такъ великъ Онъ и славенъ, однакожъ не стыдится человѣковъ \textit{братіею Своею} нарицати, глаголя: \textit{возвѣщу имя твое братіи Моей}; и паки: иди \textit{ко братіи Моей, и рцы имъ: восхожду ко Отцу Моему и Отцу вашему, и Богу Моему и Богу вашему}\footnote{Евр.~2,~12; Пс.~21,~23; Іоан.~20,~17.}. А ты кто, который говоришь человѣку: \textit{я"=де не твой братъ}, и не хощешь подобнаго себѣ братомъ назвать? Высокій ли ты еси, но такойжде человѣкъ, какъ и подлый. Благородный ли еси ты, но такойжде человѣкъ, какъ и худородный. Властелинъ ли еси ты, но такойжде человѣкъ, какъ и подкомандный твой. Господинъ ли еси ты, но такойжде человѣкъ, какъ и рабъ твой. Богатый ли еси ты, но такойжде человѣкъ, какъ и нищій, и проч. А что добрымъ себе человѣкомъ называешь, а другаго злымъ, тое неизвѣстно, кто лучшій, ты ли, или онъ, котораго злымъ называешь. Не тотъ добръ, кто себе называетъ добрымъ, но тотъ, кто добро творитъ, и кого Богъ, праведный Судія, похваляетъ. Посмотри во гробы мертвыхъ, и увидишь, что и ты \textit{братъ} всякому человѣку. \textit{Вы же} (князи, вельможи, славніи, благородніи, господа, богатіи и вси высокіи, яко бози въ мірѣ почитаеміи) \textit{яко человѣцы умираете}\footnote{Пс.~13,~7.}. Осмотрись убо, возлюбленне, что и ты человѣкъ такойжде, какъ и прочіи, что день рожденія и смерти показуетъ тебѣ, и, отложивши гордость, возлюби смиреніе Хрістово; тогда будеши всякаго человѣка, и самаго подлѣйшаго, \textit{братомъ} своимъ называть. \textit{Всякъ возносяйся смирится: смиряяй же себе вознесется}\footnote{Лук.~18,~14.}.

\section{152. Путь.}

Что путь отъ села въ село, и отъ града во градъ, тое житіе наше, хрістіанине! И житіе бо наше путь есть, по которому непрестанно идемъ. Спимъ ли или бодрствуемъ, "--- всегда путемъ симъ идемъ. На путь сей всходимъ, когда раждаемся; оканчиваемъ его, когда умираемъ. Путь сей иному должайшій, иному кратчайшій; но конецъ его никому неизвѣстенъ; не знаемъ бо, когда окончаемъ его. Тако промышляющій о насъ Господь опредѣлилъ, да всегда конца его ожидаемъ, и къ тому себе пріуготовляемъ. Видимъ, что видимый путь иный пространный и широкій, иный тѣсный и узкій. Тако и путь житія нашего иный пространный и широкій, иный узкій и тѣсный. Но повидимъ, какій путь есть пространный, и какій тѣсный житія нашего, а оттуду познаемъ, къ какому концу тотъ и другій ведетъ. На пространномъ пути имѣется невѣріе, на тѣсномъ пути имѣется живая вѣра. На пространномъ пути безстрашіе, на тѣсномъ пути страхъ Божій. На пространномъ пути самоволіе и непослушаніе, на тѣсномъ пути повиновеніе и послушаніе. На пространномъ пути неумѣренное самолюбіе, на тѣсномъ пути боголюбіе и братолюбіе. На пространномъ пути любовь суеты мірскія, на тѣсномъ пути отвращеніе отъ той. На пространномъ пути исканіе чести, славы и богатства міра сего, на тѣсномъ пути всего того презрѣніе. На пространномъ пути роскошь и плотоугодіе, на тѣсномъ пути умѣренность, постъ и воздержаніе. На пространномъ пути гордость и пышность, на тѣсномъ пути смиренномудріе. На пространномъ пути всякій грѣхъ и беззаконіе, на тѣсномъ пути всякая добродѣтель. На пространномъ пути блудъ, прелюбодѣяніе и всякая нечистота, на тѣсномъ пути цѣломудріе и чистота. На пространномъ пути піянство, и отъ того всякое беззаконіе, на тѣсномъ пути трезвость и благочиніе. На пространномъ пути воровство, хищеніе, грабленіе, насиліе и всякая неправда, на тѣсномъ пути отъ всего того удаленіе, и твореніе правды. На пространномъ пути гнѣвъ, ярость, памятозлобіе, и словомъ и дѣломъ отмщеніе, на тѣсномъ пути презрѣніе мщенія, кротость и терпѣніе. На пространномъ пути жестокосердіе, свирѣпость и лютость, на тѣсномъ пути милосердіе и состраданіе. На пространномъ пути клевета, презрѣніе, осужденіе и поношеніе ближняго, на тѣсномъ пути воздержаніе отъ того всего, и благоразумное молчаніе. На пространномъ пути ложь, лукавство, хитрость и лицемѣріе, на тѣсномъ пути простосердечіе и слово, сердечному помышленію согласное. Словомъ, на пространномъ пути слово, дѣло и помышленіе, Божіей волѣ и святому слову Его противное; на тѣсномъ пути истинное покаяніе, и того плоды "--- добрыя дѣла. Вотъ видишь, хрістіанине, пространный и тѣсный путь житія нашего. Пространный путь Богу противенъ, и потому не угоденъ Ему; тѣсный путь волѣ Его святой согласенъ, и потому Ему благоугоденъ. Пространный путь ведетъ человѣка въ погибель, но тѣсный путь вводитъ въ животъ. На пространный путь зоветъ и привлекаетъ всякаго сатана, но отъ того отзываетъ и призываетъ на тѣсный путь всѣхъ Хрістосъ Господь, пострадавый и умерый за всѣхъ. Разсуди убо, человѣче, кого слушать "--- Хріста, или сатану, и какимъ путемъ итить, пространнымъ, или тѣснымъ, въ погибель или въ животъ. Хрістосъ Господь хощетъ тебе привести въ животъ вѣчный, яко любитель и избавитель твой; но сатана, яко истинный супостатъ твой, хощетъ тебе съ собою въ вѣчную привести погибель. Слыши убо достойныя памяти слова Спасителя твоего, и углуби ихъ въ сердцѣ твоемъ, и поучайся въ нихъ, и внимай имъ и себѣ: \textit{внидите узкими враты: яко пространная врата и широкій путь вводяй въ пагубу, и мнози суть входящіи имъ. Что узкая врата и тѣсный путь вводяй въ животъ, и мало ихъ есть, иже обрѣтаютъ его}\footnote{Мѳ.~7,~13 и 14.}. Согласно сему и святый Апостолъ Его научаетъ: \textit{многими скорбьми подобаетъ намъ внити въ царствіе Божіе}\footnote{Дѣян.~14,~22.}. \textit{Настави мене Господи на путь Твой, и пойду во истинѣ Твоей: да возвеселится сердце мое боятися имене Твоего}\footnote{Пс.~85,~11.}. О чемъ весь псал. 118"~й поучаетъ насъ; поучаетъ, како намъ молитися подобаетъ, дабы Самъ Господь наставилъ насъ на путь Свой, и на томъ содержалъ, и по тому велъ.

\section{153. Чрево.}

Видимъ, что чрево ненасытно есть, всегда требуетъ снѣди и пищи; безъ того бо быть не можетъ. Сегодня насытится; на другой день, и третій, и болѣе вновь требуетъ пищи. Тако имѣется роскошь. Роскошь подобна чреву, все пожирающему. И роскошь бо есть прихотлива, и ничимъ и никогда удовольствоватися не можетъ. Но повидимъ, что дѣлаетъ роскошь, и оттуду узнаемъ, коль вредна и пагубна роскошь есть.

1)~Видимъ, что роскошь все новое замышляетъ, и все хощетъ передѣлывать. Ей надобно все по своему мудрованію перемѣнять. Домъ"=де мой сей не хорошъ, надобно мнѣ его перестроить, или вновь другій сдѣлать. Одежда сія не нравится мнѣ, надобно мнѣ перешить ее, или новую пошить. Пища сія мерзѣетъ мнѣ, надобно приказать повару, чтобы такую и такую"=то варилъ, тѣмъ и тѣмъ"=то ее приправлялъ. Водки простой и простаго винограднаго не могу я пить, надобно мнѣ вейневой водки, и лучшаго винограднаго купить. Слугамъ моимъ неприлично въ такомъ"=то платьѣ мнѣ служить, надобно ихъ лучше убрать. На такой"=то коляскѣ и коняхъ стыдно мнѣ ѣздить, надобно лучшую коляску и лучшихъ коней достать. Скучно мнѣ безъ музыки быть; надобно постараться, чтобы музыка у мене была. Не малую веселость придаетъ собачья охота; постараюсь я и тую имѣть. Печальныя мысли и скуку прогоняетъ картежная игра; что мнѣ возбраняетъ и тою себе утѣшать? Не малую веселость придаютъ хорошіе пруды, сады и галдареи; построю и тыя, и проч. Тако замышляетъ и размышляетъ тотъ, который знаетъ, что Хрістосъ Господь за него пострадалъ и распятъ былъ на крестѣ, и въ житіи Своемъ не имѣлъ, гдѣ главы подклонить, и которому отъ того же Господа показанъ путь узкій къ вѣчному животу! О бѣдный хрістіанине, который такъ себе разширяешь! осмотрись, и увидишь, куды роскошь твоя тебе ведетъ. Когда тебѣ въ такихъ мысляхъ подумать о Бозѣ, о Хрістѣ распятомъ за грѣхи твои? когда о душѣ твоей? когда о смерти? когда о судѣ Хрістовомъ, которому предстанешь, и за все отвѣтъ воздашь? когда о блаженной и неблагополучной вѣчности? Что нищимъ и бѣднымъ, которыи Хрістова ради имени просятъ, подашь, когда все на роскошь изнуряешь? Я"=де и нищимъ даю. \textit{Отвѣтъ}: даешь, но по копѣйкѣ или по денежкѣ, и то для показанія себе самаго, а не недостатки ихъ дополняешь. Но \textit{кому много дано, отъ того много и требуется}. Тако роскошь и отъ мыслей душеполезныхъ отъ подаянія милостыни человѣка отвращаетъ, и развращаетъ. 2)~Роскошно жить учатся люди другъ отъ друга, какъ видимъ. Единъ сдѣлалъ по своимъ прихотямъ тое и тое, такіе и такіе построилъ покои, такую и такую пошилъ одежду, и проч. Дѣлаетъ тоежде и другій, и прочіи дѣлаютъ. И тако люди другъ за другомъ идутъ, по подобію скотовъ неразумныхъ, которыи туды идутъ, куды одна пойдетъ скотина, а не разсуждаютъ, полезно ли имъ тое, въ чемъ другимъ подражаютъ. Чувствамъ послѣдуютъ, а у здраваго разума и вѣрнаго совѣтника, Божія слова, не совѣтуются, и тако заблуждаютъ. О человѣче! не тое дѣлай, что люди дѣлаютъ, но что здравому разуму и святому Писанію согласно. Воистину не заблудишь, когда сего святаго свѣтильника держаться, и тому послѣдовать будешь. 3)~Роскошь требуетъ, чтобы человѣку пространно жить. А на тое не мало суммы надобно. Что жъ замышляетъ роскошный? Откуду тую взять? готовой нѣтъ. Надобно убо роскошному всякую дѣлать неправду. Властелину надобно съ подначальныхъ сбирать; помѣщику излишніе оброки на своихъ крестьянъ налагать, или принуждать ихъ болѣе дней въ недѣлѣ на него работать; купцу дешевую вещь за дорогую продавать, лгать и купующихъ обманывать; иному мзду наемничу удерживать; иному жалованья, отъ Государя опредѣленнаго, подкоманднымъ своимъ не давать; иному на воровство, хищеніе и всякую неправду обращаться надобно.

Сему и всякому злу роскошь причиною бываетъ. Отсюду видимъ, что многіи во всякомъ убожествѣ и недостаткѣ живутъ, многіи не имѣютъ домовъ, дневнаго пропитанія и одежды Все сіе отъ роскоши бываетъ. Роскошь научаетъ людей обиждать и обнажать. Однимъ много надобно, слѣдственно другіи ни съ чимъ остаются. Одни лишаютъ, другій лишаются. Одни пресыщаются, другіи алчутъ и жаждутъ. Одни порфирою и виссономъ украшаются, другіи полунаги ходятъ. Одни расширяютъ и украшаютъ домы свои, а другіи и хижинъ не имѣютъ. Одни каретами и цугами проѣзжаются, другимъ землю пахать и прочую отправлять работу нечимъ, и проч. Такая"=то у нынѣшнихъ хрістіанъ правда и любовь, у тѣхъ, которыи всегда говорятъ: \textit{чаю воскресенія мертвыхъ и жизни будущаго вѣка}. Чаютъ таковіи воскресенія мертвыхъ и жизни будущаго вѣка; но такъ живутъ и дѣлаютъ, какъ бы воскресенія и вѣчнаго живота не было. 4)~Что отъ роскошныхъ людей тайно дѣлается, о томъ срамно есть и глаголати. Гдѣ страха Божія нѣтъ, какого тамо чаять добра, кромѣ всякаго зла? Плоть у всякаго человѣка свирѣпѣетъ, но у роскошнаго, который узду воздержанія у нея отнялъ, наипаче. Но какъ таковіи ни скрываются съ темными своими дѣлами, однакожъ отъ всевидящаго Божія ока сокрытися не могутъ. Онъ какъ мысли и начинанія ихъ, такъ и дѣла ихъ, во тьмѣ творимая, ясно видитъ, и объявитъ имъ въ день страшнаго Своего суда, якоже глаголетъ: \textit{обличу тя, и представлю предъ лицемъ твоимъ грѣхи твоя. Разумѣйте убо сія забывающіи Бога, да не когда похититъ, и не будетъ избавляяй}\footnote{Пс.~49,~21 и 22.}. 5)~Мысли прихотливыя и роскошныя сатана, врагъ душъ человѣческихъ, представляетъ человѣку, и въ тѣхъ запутываетъ его: какъ"=де хорошо и благопріятно веселиться, тое и тое дѣлать, тѣмъ и тѣмъ себе утѣшать, въ гости ѣздить и гостей принимать, и проч. Тако замышляетъ супостатъ, чтобы человѣкъ міръ сей за свое отечество и рай веселія имѣлъ, а о будущемъ бы блаженствѣ забылъ, и тако бы погиблъ; такожде бы на всякую неправду и обиды бѣдныхъ людей, чему роскошь научаетъ, стремился, и тако бы удобнѣе, во всѣхъ злыхъ запутавшися, погибнулъ. Сія его хитрость есть и замыслъ. Сильная и дѣйствительная удица діавольская есть роскошь, которою души хрістіанскія уловляетъ и за собою въ вѣчную погибель влечетъ. 6)~Читаемъ въ исторіяхъ, что многіи города и государства отъ роскоши погибли. Роскошь бо все и всякое добро, какъ чрево или какъ пучина пожираетъ, и людей и самыхъ сильныхъ безсильными дѣлаетъ и разслабляетъ, и неугодными въ брани дѣлаетъ. Радость бываетъ окрестнымъ непріятелямъ, когда въ государствѣ, противномъ имъ, роскошь умножается. Горе убо той странѣ и государству, въ которомъ роскошь умножилась! Ибо съ роскошію и всякое зло тамо умножается. Отъ чего и праведный Божій гнѣвъ виситъ надъ тѣмъ. Оттуду не инаго чего ожидать, какъ погибели. 7)~Видишь, хрістіанине, что есть роскошь, и что она дѣлаетъ. Берегись убо роскоши, да не развратишися и погибнеши. Прочіи"=де тое и тое дѣлаютъ, какъ и мнѣ не дѣлать? Что говоришь, о человѣче, прочіи тое и тое дѣлаютъ!? Прочіи, дѣлая беззаконно, идутъ въ погибель: убо и тебѣ надобно имъ послѣдовать? Когда будешь дѣлать, что они дѣлаютъ, то за ними будешь послѣдовать и въ погибель. Вѣдь ты хрістіанинъ; тебѣ свѣтитъ свѣтильникъ слова Божія, и показуетъ тебѣ, что добро и что зло, что польза и что вредъ, что добродѣтель и что порокъ, къ какому концу путь узкій и къ какому пространный путь ведетъ; надобно убо тебѣ дѣлать, когда хощешь спастися, не что люди дѣлаютъ, но чего Божіе поучаетъ слово. Въ Содомѣ вси беззаконновали; но праведный Лотъ на нихъ не смотрѣлъ, а жилъ свято и богоугодно. Буди убо и ты въ мірѣ, какъ Лотъ въ Содомѣ. Хотя и вси въ роскошахъ и беззаконныхъ дѣлахъ будутъ утопать, ты не смотри на нихъ, но дѣлай, чего Божіе научаетъ слово, и живи такъ, какъ истиннымъ хрістіанамъ прилично. Когда о блаженной и мучительной вѣчности будешь размышлять, то размышленіе сіе, какъ вѣтръ мглу, мысли твои о прихотяхъ и роскошахъ развѣетъ, и ничего не потребуешь, кромѣ нужнаго. Похоти и роскоши много надобно; естество малымъ довольствуется. \textit{Молю васъ, братіе, щедротами Божіими, представите тѣлеса ваша жертву живу, святу, благоугодну Богови, словесное служеніе ваше; и не сообразуйтеся вѣку сему, но преобразуйтеся обновленіемъ ума вашего, во еже искушати вамъ, что есть воля Божія благая и угодная и совершенная}\footnote{Римл.~12,~1 и 2.}.

\section{154. Въ яму человѣкъ впадаетъ, которую самъ ископалъ.}

Бываетъ, что человѣкъ ископаетъ яму ради нѣкоей своей потребы, но самъ по случаю въ тую яму впадаетъ. Тако навѣтниковъ и злодѣевъ зло, ближнимъ своимъ отъ нихъ уготованное, часто постигаетъ. Часто ядъ, который ближнимъ своимъ уготовляютъ, сами испиваютъ, и умерщвляются; часто мечемъ, который на ближняго своего подымаютъ, себе самихъ ударяютъ и убиваютъ; часто въ большее поношеніе впадаютъ, когда замышляютъ ближнихъ своихъ оклеветать и опорочить; часто своей земли и прочаго имѣнія лишаются, когда чуждымъ хотятъ и стараются завладѣть, и тако впадаютъ въ яму, которую ради другихъ ископываютъ. Видимъ сего довольно въ мірѣ. Тако праведный и дивный судъ Божій беззаконниковъ постигаетъ! Тако постигнулъ судъ Божій Фараона мучителя, который гнался въ слѣдъ Израиля, свободившагося отъ работы его, и хотѣлъ"=было его паки покорить, поработить и озлобить, но вмѣсто того погибель себѣ сыскалъ; и гдѣ чаялъ себѣ корысти, тамо смерть свою увидѣлъ, и въ водѣ морской, яко во гробѣ, со всѣмъ воинствомъ своимъ погреблся\footnote{Исх.~14,~28.}. Авессаломъ, сынъ Давидовъ, восхотѣлъ завладѣть царствомъ Израилевымъ, и умыслилъ и искалъ убить святаго и неповиннаго отца своего; но вмѣсто того обвѣсился на древѣ и погиблъ между небомъ и землею, и тако впалъ въ ровъ, который дѣлалъ ради праведнаго отца своего\footnote{2~Цар.~18,~9 и 14.}. Подобный судъ дозналъ на себѣ Аманъ; на той бо висѣлицѣ, которую пріуготовилъ неповинному Мардохею, царевымъ повелѣніемъ повѣшенъ\footnote{Есѳ.~7,~9 и 10.}. Тако гордыхъ и своевольныхъ судъ Божій постигаетъ, и нечаянно впадаютъ въ ровъ, который ради другихъ изрываютъ. Подобныя судьбы и нынѣ беззаконныхъ постигаютъ. Слышите, навѣтники и злодѣи! Въ яму впадаютъ люди, которую для ближнихъ ископываютъ; и ядъ сами испиваютъ, который ради другихъ пріуготовляютъ; и сами тое зло страждутъ, въ которое другихъ вринуть хотятъ. Ужаснитеся убо и берегитеся ближнимъ зло дѣлать, да не сами прежде въ тое впадете зло. \textit{Се болѣ неправдою, зачатъ болѣзнь, и роди беззаконіе. Ровъ изры, и ископа и, и падетъ въ яму, юже содѣла. Обратится болѣзнь его на главу его, и на верхъ его неправда его снидетъ}\footnote{Пс.~7,~15--17.}.

\section{155. Угожденіе.}

Видимъ, что люди людямъ угождаютъ. Угождаютъ подданніи царямъ своимъ, подвластніи властямъ своимъ, раби господамъ своимъ, дѣти родителямъ своимъ, и прочіи другимъ людямъ угождаютъ. Тако и наипаче намъ, хрістіанине, Богу должно угождать. Требуетъ сего отъ насъ 1)~Вѣра и исповѣданіе наше. Вѣра бо не должна быть праздна и суетна, но плодовита, любовію и добрыми дѣлами поспѣшествуема, якоже сего требуетъ Апостолъ отъ насъ: \textit{покажи ми вѣру твою отъ дѣлъ твоихъ}\footnote{Іак.~2,~18.}. 2)~Требуетъ сего отъ насъ обѣщаніе наше, которое мы учинили Богу при святомъ крещеніи. Ибо тогда, отставши отъ сатаны, обѣщались Ему работать и угождать. 3)~Требуетъ сего самая совѣсть наша. Ибо и самою совѣстію убѣждаемся Богу угождать, и Его высочайшимъ почитаніемъ почитать, что безъ угожденія быть не можетъ. 4)~Требуетъ сего отъ насъ необходимая нужда и дѣло спасенія нашего. Надобно бо Тому угождать неотмѣнно, у Котораго въ руцѣ смерть и животъ нашъ, отъ Котораго временнаго и вѣчнаго спасенія просимъ, и всякое добро душевное и тѣлесное желаемъ получить. Кто бо не угождаетъ тому, у котораго милости и всякаго добра проситъ и ищетъ? Самый разумъ естественный убѣждаетъ его къ тому. Хощемъ ли убо и мы, хрістіане, у Бога милость получить, должны Ему и угождать. Угождаютъ люди царямъ своимъ. Богъ есть Царь нашъ, и Царь вѣчный, Царь царствующихъ, и страшенъ Царь по всей земли. Должны убо Ему угождать, яко вѣчному Царю нашему. "--- Угождаютъ люди властямъ и начальникамъ своимъ. Богъ есть Властелинъ и Начальникъ нашъ верховнѣйшій, Котораго власть и начальство вѣчное. Должно убо Ему угождать, яко Властелину и Начальнику нашему вѣчному. "--- Угождаютъ люди господамъ своимъ. Богъ есть Господь нашъ высочайшій, и есть Господь господствующихъ. Должно убо Ему угождать, яко высочайшему Господу нашему. "--- Угождаютъ люди своимъ благодѣтелямъ. Богъ есть высочайшій Благодѣтель нашъ, отъ Котораго безчисленная получили и получаемъ благодѣянія, и каковаго нѣтъ и не можетъ быть большій благодѣтель. Его бо есмы созданіе, и все, что ни имѣемъ, отъ Него имѣемъ. Должно убо Ему, яко высочайшему Благодѣтелю, угождать. "--- Угождаютъ люди отцамъ своимъ. Богъ есть Отецъ нашъ; словомъ Своимъ святымъ и банею крещенія духовнаго родилъ насъ. Должно убо Ему, яко благоутробному Отцу нашему, угождать. "--- Должно угождать Тому, Которому вси святіи ангели со страхомъ, любовію и радостію угождаютъ, Тому, Которому вся тварь работаетъ и угождаетъ; Тому, Котораго вся трепещутъ; Тому, Которому противитися и раздражати страшно; Тому, Который можетъ и душу и тѣло въ гееннѣ огненной погубить; Тому, у Котораго въ руцѣ всѣхъ животъ и смерть; Тому наконецъ, отъ Котораго чаемъ воскресенія и жизни будущаго вѣка. И угождать Ему должно искренно, сердечно и нелицемѣрно; то"=есть, должно тое угожденіе происходить отъ вѣры и сердечнаго упованія. Ибо Онъ, яко Сердцевѣдецъ, глубину сердецъ нашихъ испытуетъ и видитъ, како Ему угождаемъ, искренно или лицемѣрно. Кому жъ убо хрістіанине, угождать намъ, какъ не Богу нашему? Но повидимъ, кто Богу угождаетъ, и кто не угождаетъ, и како Ему угождать должно. 1)~Угождаетъ Богу, кто волю Его святую творитъ; не угождаетъ Богу, кто воли Его святой не творитъ. Угождаетъ Богу, кто смиренно поступаетъ и живетъ; не угождаетъ Богу, кто возносится и гордится, въ гордости и пышности живетъ. \textit{Богъ гордымъ противится: смиреннымъ же даетъ благодать}\footnote{Петр.~5,~5.}. Угождаетъ Богу, кто всю надежду на Него полагаетъ, надежду какъ временнаго, такъ и вѣчнаго спасенія; не угождаетъ Богу, кто на себе, на свою честь, на князей и высокихъ людей и прочее созданіе надѣется. Угождаетъ Богу, кто Его отъ чиста сердца призываетъ и молится Ему, и проситъ отъ Него въ нуждахъ и бѣдствіяхъ своихъ помощи и заступленія; не угождаетъ Богу, кто оставляетъ молитву, и въ нуждѣ своей прибѣгаетъ къ немощному созданію, и глаголетъ или человѣку или злату и сребру: \textit{ты моя надежда}, что единому Богу говорить прилично и должно. Угождаетъ Богу, кто Ему за вся благая сердечно благодаритъ, и хвалитъ Его и поетъ: \textit{восхвалю имя Бога моего съ пѣснію, возвеличу Его во хваленіи: и угодно будетъ Богу паче тельца юна, роги износяща и пазнокти}\footnote{Пс.~68,~31 и 32.}. Не угождаетъ Богу, кто сію оставляетъ должность, и тако Богу неблагодаренъ бываетъ. Угождаетъ Богу, кто святое имя Его съ почтеніемъ и благоговѣніемъ поминаетъ; не угождаетъ Богу, кто имени Его святому не отдаетъ достойныя чести, но безъ всякой нужды, безъ страха и почтенія, и во всякихъ случаяхъ поминаетъ, какъ"=то вошли въ обычай негодныя божбы и поминанія страшнаго и святаго имени Божія: \textit{ей Богу! на то Богъ! Свидѣтель Богъ!} и прочая. Беззаконнѣе того дѣлаетъ, кто въ шуткахъ и во лжи страшное имя Божіе поминаетъ. Угождаетъ Богу, кто своимъ родителямъ, властямъ и прочіимъ, въ чести находящимся, отдаетъ достойную честь; не угождаетъ Богу, кто таковыя имъ не отдаетъ чести. Угождаетъ Богу, кто не только свой, но и у ближняго своего сохраняетъ животъ; не угождаетъ Богу, кто о томъ небрежетъ. Угождаетъ Богу, кто въ честномъ и цѣломудренномъ супружествѣ, или безъ того чисто и непорочно живетъ; не угождаетъ Богу, кто не хранитъ чистоты, но или прелюбодѣйствуетъ, какъ то брачніи дѣлаютъ, или блудодѣйствуетъ, какъ то дѣлаютъ живущіи безъ брака законнаго. \textit{Честна женитва во всѣхъ, и ложе нескверно: блудникомъ же и прелюбодѣемъ судитъ Богъ}\footnote{Евр.~13,~4.}. Угождаетъ Богу, кто отъ похищенія, воровства и всякія неправды отвращается; не угождаетъ Богу, кто на беззаконныя тѣ дѣла руку простираетъ. Угождаетъ Богу, кто языкъ свой отъ клеветы, осужденія, злословія, празднословія и всякихъ негодныхъ и гнилыхъ словъ удерживаетъ; не угождаетъ Богу, кто языкъ свой не обузданъ имѣетъ. Угождаетъ Богу, кто искренно, простосердечно и нелицемѣрно съ ближнимъ своимъ поступаетъ и обходится; не угождаетъ Богу, кто съ ближнимъ своимъ ложно, лестно, лукаво, коварно поступаетъ. Угождаетъ Богу, кто ничего чуждаго безъ воли хозяина не желаетъ; не угождаетъ Богу, кто желаетъ похитить, или отнять какое чуждое добро. Словомъ, угождаетъ Богу, кто по правилу святаго слова Его себе исправляетъ и живетъ; не угождаетъ Богу, кто о томъ небрежетъ. \textit{Твердое убо основаніе Божіе стоитъ, имущее печать сію: позна Господь сущія Его, и да отступитъ отъ неправды всякъ именуяй имя Господне}\footnote{2~Тим.~2,~19.}. 2)~Богу угождать должны мы, хрістіанине, паче царей нашихъ, паче властей нашихъ, паче господъ нашихъ, паче родителей нашихъ и паче всѣхъ, кому мы ни подчинены. Ибо и бояться и любить и почитать Его паче всякаго созданія, именуемаго на небеси, или на земли, должно. Онъ бо есть верховнѣйшій Господь нашъ, и есть Царь царствующихъ и Господь господствующихъ. Сего ради и подданніи царямъ, и подвластніи властямъ, и раби господамъ, и дѣти родителямъ повиноваться и угождать \textit{Господа ради} должны, якоже Апостолъ глаголетъ: \textit{повинитеся всякому человѣчу начальству, Господа ради: аще царю, яко преобладающу; аще ли же княземъ, яко отъ Него посланнымъ}, и проч.\footnote{1~Петр.~2,~13 и 14.} 3)~Аще что, какъ то бываетъ, приказываютъ и повелѣваютъ намъ власти, но тое есть противно волѣ Божіей; то не должны мы въ томъ ихъ слушать, хотя бы и смертію намъ грозили. Честни должны быть намъ власти человѣчи; но несравненно честнѣйшій долженъ быть Богъ, Которому и самыя власти со страхомъ повиноватися должны. Тако научаютъ насъ примѣромъ своимъ апостоли святіи, которымъ беззаконніи жиды запрещали учить и глаголать о имени Хрістовомъ; но они не слушали ихъ, и со дерзновеніемъ учили и глаголали Божіе слово, и отвѣщали запрещающимъ: \textit{повиноватися подобаетъ Богови паче, нежели человѣкомъ}\footnote{Дѣян.~5,~29.}. 4)~Когда хощемъ Богу угождать, то не должны ближняго нашего презирать и отвращаться отъ него, но миловать его и въ нуждахъ его снабдѣвать. \textit{Кійждо васъ ближнему да угождаетъ во благое къ созиданію}\footnote{Рим.~15,~2.}. Требуетъ ближній нашъ совѣта отъ насъ, "--- подадимъ совѣтъ, когда можемъ. Требуетъ пищи отъ насъ, "--- подадимъ пищу. Требуетъ одежды, "--- одѣемъ его. Требуетъ упокоенія въ домѣ нашемъ "--- упокоимъ его. Требуетъ послуженія нашего, "--- послужимъ ему. Требуетъ въ скорби и печали утѣшенія, "--- утѣшимъ его по нашей возможности, и проч. Сего бо хощетъ воля Божія, и требуетъ хрістіанская любовь. \textit{Благотворенія и общенія не забывайте; таковыми бо жертвами благоугождается Богъ}\footnote{Евр.~13,~16.}. 5)~Отсюду послѣдуетъ, что и враговъ нашихъ любить, и имъ добро творить должны мы, когда хощемъ Богу угождать. Ибо воля Божія хощетъ сего\footnote{Мѳ.~5,~44.}. Аще убо хощемъ волю Божію творить, и тако Богу угождать; то должно намъ и отъ враговъ нашихъ любве нашей не отнимать, и имъ въ нуждахъ ихъ помогать; сіе бо наипаче хрістіанина и богоугоднаго человѣка показуетъ. Аще царю земному хощешь угодить, и тако милость отъ него получить; то не отречешися и врагу твоему послужить, когда онъ хощетъ того отъ тебе, и повелѣваетъ тебѣ: кольми паче должно тебѣ дѣлать сіе ради Царя небеснаго, ради Котораго и земному царю угождать должно, и враговъ своихъ любить, и въ нуждахъ ихъ служить имъ. Въ семъ бо и хрістіанская истинная добродѣтель состоитъ, то"=есть, себе самого побѣдить и воздавать добро за зло. \textit{Аще убо алчетъ врагъ твой, ухлѣби его; аще ли жаждетъ, напой его. Сіе бо творя, угліе огненно собираеши на главу его. Не побѣжденъ бывай отъ зла, но побѣждай благимъ злое}\footnote{Римл.~12,~20 и 21.}. 6)~Чтобы все сіе угодно Богу было, то должно намъ тое все творить, то"=есть, уклоняться отъ зла и творить благое, и не ради тщеславія и похвалы человѣческія, но ради \textit{славы святаго имени Его}. О семъ научаетъ насъ Хрістосъ Господь: \textit{тако да просвѣтится свѣтъ вашъ предъ человѣки, яко да видятъ ваша добрая дѣла, и прославятъ Отца вашего, Иже на небесѣхъ}\footnote{Мѳ.~5,~16.}. Иначе какъ угодно можетъ быть Богу тое, что не ради Бога дѣлается? Отсюду слѣдуетъ, что не всякое доброе дѣло видимое истинно доброе есть, но тое, которое добрѣ творится, то есть, на добрый конецъ, который есть слава имени Божія. Хрістіанине! Богъ міръ весь ради насъ сотворилъ, и все насъ ради творитъ, и Сына Своего ради насъ въ міръ послалъ, и на смерть Его предалъ ради насъ, и такъ отъ грѣховъ, діавола, смерти и ада искупилъ насъ, и вѣчное уготовалъ царствіе: что болѣе уже можетъ Богъ сотворити ради насъ? Сотворимъ убо и мы все ради Его, и возлюбимъ Его, и почтимъ Его, и послужимъ Ему, и заповѣди Его святыя сотворимъ ради Его Самого, и тако отъ чиста сердца угодимъ Ему. 7)~Богоугоднымъ дѣламъ и святому предложенію послѣдуетъ искушеніе какъ отъ діавола, такъ и отъ злыхъ людей. Дѣлается тое попущеніемъ Божіимъ ко искушенію человѣческому, "--- ради Бога ли дѣлаетъ онъ тое, что началъ, и твердо ли стоитъ въ начатомъ дѣлѣ. Сего ради все, что ни приключается противное ему, должно, аки отъ руки Господни посылаемое, принимать и великодушно претерпѣвать, и въ начатомъ не ослабѣвать, но паче успѣвать. \textit{Потерпи Господа, мужайся, и да крѣпится сердце твое: и потерпи Господа}, увѣщаваетъ пророкъ\footnote{Пс.~26,~14.}. 8)~Богу угождать безъ Самого Бога не можемъ, хрістіанине; великая бо есть наша слѣпота и слабость и растлѣніе. Сего ради должно со усердіемъ Богу молиться и къ Нему воздыхать, чтобы Самъ наставилъ насъ на путь Ему угодный, и на томъ бы насъ содержалъ, и по тому велъ насъ. Сему насъ научаетъ примѣромъ своимъ Давидъ святый, который чрезъ весь псаломъ 118"~й со всякимъ усердіемъ молился Богу, и желалъ святаго и богоугоднаго житія. Да послѣдуемъ убо и мы ему въ томъ. \textit{Научи мя}, Господи, \textit{творити волю Твою, яко Ты еси Богъ мой}\footnote{Пс.~142,~10.}. Господи! дай мнѣ хотѣти, начинати, мыслити, творити и глаголати, что волѣ Твоей святой угодно. Не допусти, Господи, ни хотѣти, ни начинати, ни мыслити, ни творити, ни глаголати, что волѣ Твоей не угодно. \textit{Настави мя Господи, на путь Твой, и пойду во истинѣ Твоей: да возвеселится сердце мое боятися имене Твоего}\footnote{85,~11.}.

\subsection{О томжде.}

Когда подданніи предъ царемъ своимъ, подначальніи предъ начальникомъ своимъ, раби предъ господиномъ своимъ, дѣти предъ отцемъ своимъ имѣются, ходятъ и обращаются; то со всякимъ страхомъ и опасеніемъ обращаются, чтобы въ чемъ не проступиться, и тако бы ихъ не оскорбить и не прогнѣвать; почему всякихъ словъ и дѣлъ непристойныхъ берегутся, и тако имъ угождаютъ. Хрістіанине! да послѣдуемъ и мы въ томъ таковымъ людямъ, которіи тако людямъ угождаютъ, и предъ Богомъ и Царемъ нашимъ вѣчнымъ и Отцемъ нашимъ, Иже есть на небесѣхъ, да поступаемъ тако и обращаемся, и всего, что Ему противно и неугодно, да бережемся и тако угодимъ Ему. Всегда бо и вездѣ, гдѣ мы ни имѣемся, предъ Богомъ имѣемся и обращаемся; и что ни дѣлаемъ, говоримъ, мыслимъ и начинаемъ, предъ Нимъ все дѣлаемъ, мыслимъ, начинаемъ и говоримъ. Ибо Богъ на всякомъ мѣстѣ присутствуетъ, и всякое дѣло наше видитъ, и всякое слово наше слышитъ. Сего ради, въ домѣ ли имѣемся, "--- предъ Нимъ имѣемся. Дѣло ли какое дѣлаемъ, "--- предъ Нимъ дѣлаемъ. Рѣчь ли какую говоримъ, "--- предъ Нимъ говоримъ. Мыслимъ ли и начинаемъ что, "--- предъ Нимъ мыслимъ и начинаемъ. Съ людьми ли имѣемся и бесѣдуемъ, "--- предъ Нимъ бесѣдуемъ. Ядимъ ли, или піемъ, "--- предъ Нимъ ядимъ и піемъ. Купуемъ ли, или продаемъ, "--- предъ Нимъ купуемъ и продаемъ. Любимъ ли что, или ненавидимъ, желаемъ ли чего, или отвращаемся, "--- все тое предъ Нимъ бываетъ. На ложи ли имѣемся и упокоеваемся, "--- предъ Нимъ почиваемъ. По пути ли идемъ, "--- Онъ съ нами есть, и предъ Нимъ идемъ. Въ пустынѣ ли или во уединеніи, во градѣ ли или селѣ, въ тайномъ или явномъ мѣстѣ имѣемся, "--- Онъ съ нами присутствуетъ, и предъ Нимъ обращаемся. Словомъ, всегда и вездѣ, гдѣ ни имѣемся, Онъ съ нами есть, и мы предъ Нимъ обращаемся, и что ни дѣлаемъ, мыслимъ и говоримъ, все Ему явно и откровенно\footnote{Іер.~23,~24; Іов.~34,~21; Сир.~23,~27--29; Евр.~4,~13.}. Да поступаемъ убо предъ Нимъ вездѣ тако, какъ поступаютъ подданніи предъ царемъ своимъ и вси подвластніи предъ властію своею, и тако отдадимъ Ему, яко Богу нашему, Царю нашему, Господу нашему и Отцу нашему, достойную Ему честь, и симъ образомъ угодимъ Ему. Не уподобимся тѣмъ хрістіанамъ, которіи предъ людьми благочинно поступаютъ и показуютъ себе благочестивыми, но въ тайныхъ мѣстахъ беззаконнуютъ и таковая творятъ, которыхъ срамно есть и глаголати. Таковіи хрістіане суть лицемѣры, нечестивіи и безбожніи. Они людей стыдятся, но Бога не стыдятся; людей боятся, но Бога не боятся. О таковыхъ написалъ пророкъ Божій: \textit{не предложиша Бога предъ собою}\footnote{Пс.~53,~5.}. О человѣче, гдѣ ты съ темнымъ своимъ дѣломъ укрыться можешь отъ Того, Который вездѣ присутствуетъ и все видитъ? Убѣгаешь отъ людей; но Богъ вездѣ предваряетъ тебе, и никуды отъ Него убѣжать не можешь\footnote{Амос.~9,~2--4.}. Сокрываешися отъ людскихъ очей; но отъ Божіяго всевидящаго ока нигдѣ и никакъ сокрытися не можешь. Бережешися, чтобы не слышали люди негодныхъ и гнилыхъ твоихъ словъ; но Богъ ихъ слышитъ. Сокрываешь въ тайнѣ сердца твоего лесть, лукавство, хитрость и ложь, и бережешися, чтобы люди сего твоего зла не познали; но Богъ, \textit{сердца и утробы испытуяй}, все тое явно видитъ\footnote{Пс.~7,~10; 1~Пар.~28,~9.}. "--- Я"=де"=Бога не вижу? "--- Правда, и видѣть Его не возможно никому. \textit{Бога никтоже видѣ нигдѣже}\footnote{Іоан.~1,~18.}. Богъ есть \textit{Духъ}, никакимъ чувствамъ не подлежащій, и потому видѣть Его невозможно, а только вѣрою и умомъ познается. Вѣра святая и Божіе слово научаетъ насъ, что Богъ вездѣ есть, и все наше видитъ, и всякое дѣло, слово и помышленіе Ему откровенно, какъ то сіе изображается наипаче во псалмѣ 138"~мъ: \textit{Господи, искусилъ мя еси}, и проч. Аще вѣришь слову Божію, то вѣруй, что и Богъ на всякомъ мѣстѣ есть, и ты предъ Нимъ ходишь и обращаешися, хотя Его и не видишь; и Онъ всякое твое дѣло видитъ, и нигдѣ отъ Него и ни съ чимъ утаитися не можешь. Всему бо тому слово Божіе научаетъ. Аще ли же не вѣришь Божію слову, то почто называешися и хрістіаниномъ? \textit{Блюдите убо, како опасно ходите, не якоже немудри, но якоже премудри}\footnote{Еф.~5,~15.}.

\subsection{О томжде.}

Примѣчаемъ и тое, что когда люди хотятъ царямъ, властямъ и отцамъ своимъ угодить, то дѣлаютъ тое, что они дѣлаютъ, и тако имъ, послѣдуя нравамъ ихъ, угождаютъ. Хрістіанине! будемъ подражатели и мы Богу, небесному нашему Царю и Отцу; да послѣдуемъ Божественнымъ нравамъ Его, и да дѣлаемъ тое, что Онъ дѣлаетъ и тако угодимъ Ему. Любитъ Онъ всѣхъ, что показуется и отъ всѣхъ созданій Его, и отъ святаго слова Его, и отъ воплощенія единороднаго Сына Его; потщимся и мы любить Его отъ всего сердца нашего, Его, возлюбившаго насъ, и другъ друга любить. \textit{Исполненіе закона любовь есть}\footnote{Римл.~13,~10.}. Онъ святъ есть, будемъ и мы святи, творяще святыню въ страсѣ Божіи. Онъ праведенъ есть, будемъ и мы праведни, отдая Ему всякую честь и славу, и ближнимъ нашимъ должная имъ. Онъ милостивъ есть и милосердъ, и состраждетъ бѣдствію нашему; будемъ и мы милосерди, якоже Отецъ нашъ небесный милосердъ есть. Онъ благотворитъ всѣмъ, добрымъ и злымъ, солнце Свое сіяетъ на злыя и благія, и дождитъ на праведныя и грѣшныя; да творимъ и мы добро всѣмъ, сродникамъ и несродникамъ нашимъ, друзьямъ и врагамъ нашимъ, знаемымъ и незнаемымъ нашимъ. Онъ долготерпитъ всѣмъ, ожидаетъ всѣхъ на покаяніе; будемъ и мы долготерпѣливы, и не воздадимъ зла за зло и досажденіе за досажденіе. Онъ истиненъ есть и вѣренъ въ словесѣхъ Своихъ; будемъ и мы истинни и простосердечни. Онъ всякаго грѣха и беззаконія ненавидитъ; возненавидимъ и мы всякій грѣхъ и беззаконіе, и отъ того отвратимся. Онъ хощетъ всѣмъ спастися и въ разумъ истины пріити; да будетъ таковое хотѣніе и наше. Онъ всѣмъ кающимся грѣхи отпущаетъ, отпустимъ и мы грѣшники грѣшникамъ согрѣшенія ихъ; отпустимъ, да и намъ отпустятся отъ Него согрѣшенія наша. Онъ всѣхъ молящихся слушаетъ, и прошенія ихъ исполняетъ; не отвратимъ и мы ушей нашихъ отъ просящихъ насъ, и прошенія ихъ исполнимъ. Тако уподобимся Ему, тако будемъ подражатели Ему, тако будемъ послѣдовать пресвятымъ нравамъ Его, тако будемъ чада свѣта, и живіи и истинніи уды Хріста, Сына Божія, тако покажемъ, что мы не напрасно называемъ Его Отцемъ: \textit{Отче нашъ, Иже еси на небесѣхъ}, и проч. И тако угодимъ Ему. Сего бо и воля Его святая, и слово Его святое, и обѣщаніе наше, учиненное нами въ крещеніи, и должность наша хрістіанская требуютъ. \textit{Бывайте убо подражатели Богу, якоже чада возлюбленная}\footnote{Еф.~5,~1.}.

\subsection{О томжде.}

Видимъ, что когда хотятъ подданіи царямъ своимъ, подвластніи властямъ своимъ, раби господамъ своимъ, и дѣти отцамъ своимъ угодить, то не что себѣ угодно, но что имъ угодно видятъ, дѣлаютъ, и тако имъ угождаютъ. Хрістіанине! тако сотворимъ и мы. Да дѣлаемъ не что намъ угодно, но что Богу угодно, и тако угодимъ Ему. Божіе слово показуетъ намъ, что Богу угодно, и что неугодно. Сей священный свѣтильникъ да свѣтитъ намъ во всѣхъ нашихъ дѣлахъ, словахъ, мысляхъ и начинаніяхъ; на то бо онъ и поставленъ намъ отъ милосердаго нашего Небеснаго Отца. \textit{Свѣтильникъ ногама моима законъ Твой, и свѣтъ стезямъ моимъ}\footnote{Пс.~118,~105.}. Угодно намъ въ праздности и лѣности жить, но Богу тое неугодно; сотворимъ убо угодное Богу, и, оттрясши сонъ лѣности и праздности, пребудемъ въ полезныхъ и благословенныхъ трудахъ. \textit{Въ потѣ лица твоего снѣси хлѣбъ твой}\footnote{Быт.~3,~19.}. Угодно намъ въ гордости и пышности міра сего жить, но Богу тое не угодно, угодно же смиренномудріе наше, возлюбимъ убо смиреніе, угодное Богу и \textit{въ смиреніи поживемъ}\footnote{1~Петр.~5,~5 и 6.}. Угодно намъ пространнымъ прихотей, роскошей и веселостей путемъ итить, но Богу тое неугодно; уклонимся убо отъ пространнаго и широкаго пути, и сотворимъ Богу угодное, и взыдемъ на тѣсный путь, сей бо Ему угоденъ есть. \textit{Внидите узкими враты: яко пространная врата и широкій путь вводяй въ пагубу, и мнози суть входящіи имъ. Что узкая врата и тѣсный путь вводяй въ животъ, и мало ихъ есть, иже обрѣтаютъ его}, глаголетъ Господь\footnote{Мѳ.~7,~13 и 14.}. Угодно намъ гнѣваться, и обиду за обиду, и зло за зло, и досажденіе за досажденіе воздавать, но Богу тое неугодно; сотворимъ убо угодное Богу, и простимъ обидящимъ насъ, простимъ отъ сердца нашего. \textit{Не себе отмщающе, возлюбленніи, но дадите мѣсто гнѣву; писано бо есть: Мнѣ отмщеніе, Азъ воздамъ, глаголетъ Господь}\footnote{Римл.~12,~19.}. Неугодно намъ любить враговъ нашихъ, благословить кленущихъ насъ, и добро творить ненавидящимъ насъ, но Богу тое угодно; сотворимъ убо угодное Богу, и потщимся любить не токмо друговъ, но и враговъ нашихъ, и добро творить ненавидящимъ насъ, и проч. \textit{Любите враги ваша, благословите кленущія вы, добро творите ненавидящимъ васъ, и молитеся за творящихъ вамъ напасть, и изгонящія вы}\footnote{Мѳ.~5,~44.}. Угодно намъ страстную похоть исполнять, но Богу тое противно, а угодно Ему цѣломудріе и святость наша; сотворимъ убо угодное Богу и \textit{въ цѣломудріи и святости поживемъ}\footnote{Тит.~2,~12.}. Угодно намъ языкомъ нашимъ празднословить, клеветать, осуждать и прочія негодныя и гнилыя слова произносить, но Богу тое неугодно; обуздаемъ убо языкъ нашъ и возлюбимъ благоразумное молчаніе, и всякое слово, какое ни говоримъ, солію разума растворенно и къ созиданію ближняго нашего да будетъ; сіе бо есть угодно предъ Богомъ\footnote{Еф.~4,~29.}. Тако и въ прочемъ поступимъ, и сотворимъ не тое, что нашей волѣ и плоти угодно, но тое, что воля Божія хощетъ, и тако угодимъ Ему. Снидемъ со страхомъ и усердіемъ съ того сѣдалища и престола, на которомъ сѣдимъ, и дадимъ на томъ мѣсто Богу нашему, и почтимъ Его, и поклонимся Ему, и всегда и во всѣхъ словахъ, дѣлахъ и начинаніяхъ да глаголемъ Ему со смиреніемъ и страхомъ: \textit{Отче, да будетъ воля Твоя}, а не моя! да будетъ хотѣніе Твое, а не мое! Хощу и я, что Ты хощешь: не хощу и я, чего Ты не хощешь. \textit{Господи, помози мнѣ, аминь}.

\section{156. Хищникъ.}

Слышимъ, что тіи люди, которіи чуждое какимъ нибудь образомъ, то"=есть, или явно, или тайно, или лестно похищаютъ, называются безчестнымъ именемъ: \textit{хищники, тати и воры}; которое беззаконное дѣло есть противу заповѣди Божіей: \textit{не укради}. Тако и наипаче суть хищники, тати и воры тіи люди, которіи славу и похвалу и прочее, единому Богу подобающее, какимъ нибудь образомъ восхищаютъ и присвояютъ себѣ. Сюды надлежать: 1)~Тіи, которіи Божіе слово проповѣдуютъ ради похвалы и славы своея. Божіе бо слово дано намъ ради спасенія нашего и славы имене Божія; сего ради должно быть и проповѣдуемо. Но когда люди ради своея похвалы и славы тое проповѣдуютъ, то славу, Богу единому подобающую, себѣ восхищаютъ. Откуду бываетъ, что таковіи проповѣдники мало что людямъ пользуютъ, хотя ихъ люди и похваляютъ. Не на такій бо конецъ, на какій дано слово Божіе, проповѣдуютъ. Почему и недѣйствительно бываетъ въ сердцахъ слышащихъ. 2)~Тіи богачи, которіи изнуряютъ богатство свое на строеніе богатыхъ домовъ, на богатую одежду, на богатыя кареты и коней, на богатые столы, на богатое убраніе слугъ и прочую пышность и суету, дабы оттуду похвалу и славу отъ міра имѣть. Богатство бо и имѣніе наше есть Божіе добро; и дано намъ ради бѣдности и нужды нашей, не ради похвалы и славы нашей; дано, чтобы мы и сами тѣмъ пользовалися умѣренно, и ближнихъ нашихъ, въ честь и славу имене Божія, пользовали. Но когда того не творимъ, но отъ того ищемъ себѣ похвалы и славы, то похищаемъ себѣ славу, Богу единому подобающую. Все бо Божіе добро есть. \textit{Господня бо земля и исполненіе ея}\footnote{Пс.~23,~1.}. Да будетъ убо все и всякое добро Божіе во славу Божію, а не нашу. Довольно намъ Божіимъ добромъ пользоватися, и за тое Богу благодарить и славить святое имя Его, а не себѣ похвалы и славы отъ того искать. 3)~Тіи люди, которіи милостыню и прочія добрыя дѣла ради показанія себе самихъ творятъ; такожде которіи созидаютъ храмы Божіи и украшаютъ ихъ, созидаютъ богадельни, дабы славили и хвалили ихъ люди. Вси таковіи своей, а не Божіей славы ищутъ и потому единому Богу подобающее похищаютъ. 4)~Сюды надлежатъ и тіи, которіи или разумъ свой и премудрость, или краснорѣчіе оказываютъ, да славими и хвалими будутъ отъ человѣкъ. Вси таковіи, и вышереченніи, и прочіи, которіи какое нибудь добро дѣлаютъ, но отъ того хотятъ прославится, славу Божію восхищаютъ. Богъ есть начало и источникъ всякаго добра; сего ради Ему единому отъ всякаго добра слава и похвала подобаетъ. Человѣку, яко нищему и бѣдному и туне, безъ всякихъ своихъ заслугъ отъ Бога добро получающему, довольно того, что онъ Божіимъ добромъ \textit{туне} пользуется; а славу и благодареніе единому Богу, своему благодѣтелю, восписывать и отдавать должно. Но когда за Божіе добро себѣ славы и похвалы ищетъ, то Божіе себѣ восхищаетъ, и сердцемъ отъ Бога отступаетъ, и себе самаго, что страшно и говорить, боготворитъ, и на томъ мѣстѣ, на которомъ долженъ Бога имѣть, себе поставляетъ, и тую честь, которую долженъ Богу отдавать, себѣ привлекаетъ, "--- что есть премерзкій и тяжкій грѣхъ, и подобный діаволову грѣху, которымъ отъ Бога и Создателя своего отступилъ. Смотри, человѣче, и разсуждай, куды славолюбіе и гордость твоя тебе приводитъ! 5)~Что человѣкъ въ нынѣшнемъ вѣкѣ дѣлаетъ и въ сердцѣ своемъ сокрываетъ, тое на судѣ Хрістовомъ явится и всему міру въ познаніе пріидетъ, и все, что ни дѣлаетъ, аки взаимъ Богу даетъ, и тогда воспріиметъ. Прославляетъ ли Бога здѣ, живя въ мірѣ, "--- тамо самъ отъ Бога прославится. Презираетъ ли и уничижаетъ Бога, "--- тамо самъ презрѣнъ и уничиженъ будетъ. Сего ради глаголетъ Господь: \textit{токмо прославляющія Мя прославлю, и уничижаяй Мя безчестенъ будетъ}\footnote{1~Цар.~2,~30.}. 6)~Сего ради внимай, хрістіанине, что и для кого дѣлаешь; Божіей, или своей волѣ угождаешь, Божіей чести, или своей въ дѣлѣ твоемъ ищешь. Когда своей волѣ угождаешь, а не Божіей, то свою волю Божіей предпочитаешь, что тяжко и страшно есть. Когда въ дѣлѣ твоемъ своей, а не Божіей чести ищешь, то Бога оставляешь, и себе, аки бога, почитаешь и боготворишь, какъ самъ тое можешь видѣть, что такожде страшно и беззаконно. Богъ, Который сердца и утробы испытуетъ, видитъ и твое сердце, и на какій конецъ что дѣлаешь, видитъ. Берегись убо, да не, вмѣсто богочтеца, врагомъ Божіимъ будеши. Въ такую пагубу самолюбіе, своеволіе и славолюбіе приводитъ человѣка, хотя онъ того, яко слѣпый, и не видитъ. Самолюбіе бо и славолюбіе ослѣпляетъ человѣка. Осмотрись убо, возлюбленне, и мерзкаго гордости идола выбрось изъ твоего сердца; и съ того сѣдалища, на которомъ сѣдишь и почитаешь себе, сойди, и на томъ дай мѣсто Господу силъ, Которому со страхомъ покланяются ангели и вся тварь работаетъ; и отдавай Ему славу Его и хвалу Его и честь Его, и покланяйся Ему \textit{духомъ и истиною}, и почитай Его, яко Бога и Создателя своего. Тому \textit{единому} подобаетъ всякая слава, честь и поклоненіе. \textit{Прославите убо Бога въ тѣлесѣхъ вашихъ, и въ душахъ вашихъ, яже суть Божія}\footnote{1~Кор.~6,~20.}. \textit{Убойтеся Бога, и дадите Ему славу}\footnote{Апок.~14,~7.}. \textit{Вся въ славу Божію творите}\footnote{1~Кор.~10,~31.}.

\section{157. При случаѣ подобное поминается.}

Видимъ, что человѣкъ при случаѣ подобное поминаетъ; когда видитъ что или слышитъ, что"=то подобное ему на память приходитъ. Напримѣръ, когда видитъ ученый человѣкъ въ школу идущаго ученика, поминаетъ, како и самъ въ школу ходилъ; земледѣлецъ, видя другаго земледѣльца, землю дѣлающаго, поминаетъ и свое земледѣліе, и проч. Хрістіанине! поминай и ты при случаѣ подобное, что къ созиданію души твоея надлежитъ. Видишь, какъ смиренно, кротко и благоразумно, или дѣти предъ отцемъ своимъ, или раби предъ господиномъ своимъ, или подвластніи предъ властію своею, ходятъ и обращаются: помяни, что мы предъ Богомъ, яко вездѣсущимъ и на все смотрящимъ, ходимъ, живемъ и обращаемся, мыслимъ, говоримъ и дѣлаемъ. Аще убо человѣки предъ человѣками со страхомъ и благочинно ходятъ, и всего берегутся, чтобы ихъ не прогнѣвать, и наказанію не подпасть; кольми паче намъ должно предъ Богомъ нашимъ со страхомъ и благоговѣніемъ ходить, жить и обращаться, и ничего не дѣлать, что волѣ Его святой противно. Вездѣ бо, гдѣ ни имѣемся, предъ Богомъ имѣемся, яко вездѣсущимъ. Ибо Богъ вездѣ и на всякомъ мѣстѣ есть, якоже о томъ во псалмѣ 138"~мъ и прочихъ святаго Писанія мѣстахъ свидѣтельствуется, хотя мы Его и не видимъ (и видѣть бо Его невозможно), и како на всякомъ мѣстѣ есть, не понимаемъ. Берегись убо предъ Богомъ грѣшить, да не дознаешь на себѣ мстительную руку Его. Стыдишися человѣка, кольми паче должно стыдиться Бога. Угождаешь человѣку, кольми паче должно угождать Богу. Боишися человѣка, кольми паче должно бояться Бога, Котораго вся тварь боится и трепещетъ. \textit{Богъ нашъ огнь есть, поядаяй} нечествующихъ\footnote{Евр.~12,~29; Второз.~4,~24.}. Берегись убо, да не и тебе въ нечестіи поразитъ. Отдавай Ему вездѣ и на всякомъ мѣстѣ, во дни и нощи, въ явномъ и тайномъ мѣстѣ, во уединеніи и собраніи людей, въ домѣ и на пути, и во всякомъ дѣлѣ отдавай достойную Ему честь, славу и послушаніе. Сего отъ тебе требуютъ и совѣсть твоя, и слово Божіе, и вѣра хрістіанская. Аще убо Богъ на всякомъ мѣстѣ есть, какъ и подлинно есть, и все наше дѣло и помышленіе видитъ, и слово слышитъ; отсюду послѣдуетъ: 1)~Нигдѣ и ни въ какомъ мѣстѣ отъ Него скрытися и утаитися невозможно. Понеже куды ни пойдемъ, и гдѣ ни подумаемъ скрытися, Онъ прежде насъ тамо есть. Человѣче! убѣгаешь отъ людей, но отъ Бога убѣжати нигдѣ не можешь. Онъ вездѣ предваряетъ тебе. Скрываешися отъ очей человѣческихъ, но отъ Божіихъ скрытися не можешь: Онъ вездѣ видитъ тебе. Бережешися, чтобы люди не слышали гнилаго слова твоего; но отъ Божіихъ ушей уберещися не можешь: Онъ вездѣ и всякое слышитъ слово твое. Стыдишися зло дѣлать предъ людьми; но Богъ паче всего міра единъ видитъ зло твое: постыдися убо и убойся предъ Нимъ вездѣ и всякое зло дѣлать. Что дѣлаеши нынѣ, въ мірѣ семъ живучи, тайно или явно, тое все на всемірномъ ономъ судѣ покажетъ тебѣ, и представитъ предъ тобою, якоже глаголетъ: \textit{обличу тя, и представлю предъ лицемъ твоимъ грѣхи твоя}\footnote{Пс.~49,~21.}. 2)~Отсюду видишь, хрістіанине, коль тяжко досаждаютъ Богу тіи хрістіане, которіи беззаконнуютъ и не боятся закона Божія предъ Богомъ нарушать. Аще бы какій беззаконникъ дерзнулъ предъ царскимъ лицемъ скакать, плясать, кричать и прочія безчинія показывать: не было ли бы великое безчестіе, досажденіе и обида царскому лицу? Самъ видишь, что великое безчестіе показалъ бы царю таковый безчинникъ. Таковое, или паче несравненно большее, показуютъ безчиніе и дѣлаютъ безчестіе и досажденіе Богу тіи люди, которыи предъ Богомъ вездѣсущимъ и всевидящимъ беззаконнуютъ, и тако предъ Нимъ безчинствуютъ. Сюды надлежатъ блудники, прелюбодѣи и всякіе сквернители; надлежатъ другъ друга ругающіи, другъ друга укоряющіи, другъ съ другомъ ссорящіися, и другъ друга біющіи; надлежать клеветники и злорѣчивіи, которіи ближнихъ своихъ языкомъ, яко мечемъ, уязвляютъ; надлежатъ лукавцы, хитрецы, прелестники, обманщики и лицемѣры, которіи съ ближними своими лестно и коварно обходятся; надлежатъ чары творящіи, и ихъ къ себѣ призывающіи; надлежать скверныя и неподобныя пѣсни поющіи, и кличи и вопли возносящіи; надлежать безчинно и неблагообразно пирующіи и банкетующіи; надлежать танцующіи и пляшущіи безчинники и безчинницы; надлежать въ кулачныхъ бояхъ находящіися и на тыя смотрящіи; надлежатъ тати, хищники, грабители, насильники и вси, чуждое добро тайно, или лестно, или явно похищающіи; надлежатъ купцы, которіи въ товарахъ обманываютъ и большія цѣны, нежели товаръ стоитъ, просятъ; а паче тіи беззаконники, которіи во лжи не боятся имене Божія призывать, и тѣмъ (о долготерпѣнія Твоего, Господи!), тѣмъ ложь свою, аки истину, утверждаютъ; надлежатъ судіи и приказные служители, которіи по страсти своей и по мздѣ, а не по правдѣ и по силѣ присяги, дѣла свои отправляютъ; надлежатъ кленущіися именемъ Божіимъ во лжу; надлежатъ всѣ сквернословцы и кощунники; словомъ, всякъ, не стыдящійся и не боящійся беззаконновать. Всякое бо беззаконіе предъ Богомъ дѣлается, тайно ли или явно дѣлается оно. 3)~Отсюду видишь, коль тяжко таковіи хрістіане грѣшатъ предъ Богомъ. Они слышатъ часто слово Божіе, слышатъ и тое, что Богъ вездѣ и на всякомъ мѣстѣ есть; но не внимаютъ тому и небрегутъ о томъ, и себе не исправляютъ. 4)~Видишь паки, коль великая благость Божія и долготерпѣніе. Люди предъ Нимъ беззаконнуютъ, и безчестіе и досажденіе Ему показуютъ, но Онъ не абіе казнитъ ихъ, но долготерпитъ имъ, и ожидаетъ ихъ на покаяніе. Какій кроткій царь, безчиніе предъ собою показуемое видя, стерпитъ? Скоро кротость человѣческая обращается въ ярость. Богъ нашъ не тако: видитъ беззаконія, предъ лицемъ Своимъ отъ человѣкъ творимая; видитъ, и \textit{долготерпитъ}, не хотя, да беззаконнующіи погибнутъ, \textit{но да вси въ покаяніе пріидутъ}\footnote{2~Петр.~3,~9.}. Видишь благодать Божію, видишь и долготерпѣніе Его. 5)~Но слыши, бѣдный грѣшникъ, что тебѣ Апостолъ глаголетъ: \textit{или о богатствѣ благости Его и кротости и долготерпѣніи нерадиши, не вѣдый, яко благость Божія на покаяніе тя ведетъ? По жестокости же твоей и непокаянному сердцу, собираеши себѣ гнѣвъ въ день гнѣва и откровенія праведнаго суда Божія, Иже воздастъ коемуждо по дѣломъ его}\footnote{Римл.~2,~4 и 5.}. Видишь предстоящаго или подданнаго царю, или раба господину, и бесѣдующаго съ нимъ, и просящаго отъ него милости; видишь, како онъ предъ нимъ стоитъ: стоитъ со смиреніемъ и благоговѣніемъ, и съ преклоненіемъ главы и колѣнъ, и къ нему со умиленіемъ смотритъ, и со вниманіемъ проситъ, чего хощетъ получить. Помяни здѣ, что тако истинніи хрістіане въ молитвѣ Богу предстоятъ: предстоятъ со смиреніемъ и благоговѣніемъ и вниманіемъ, и главы и колѣна предъ Нимъ преклоняютъ, и вѣрою къ Нему взираютъ, и припадаютъ Ему, и нужды и требованія своя объявляютъ Ему, и милости и помощи отъ Него просятъ, и грѣхи своя исповѣдуютъ Ему, и отпущенія просятъ. Отсюду видишь: 1)~Истинная молитва состоитъ не въ единыхъ наружныхъ словахъ и устномъ глаголаніи; но истинная молитва состоитъ \textit{въ дусѣ и истинѣ}\footnote{Іоан.~4,~23.}. Когда молимся Богу, то должно предъ Нимъ стоять не токмо тѣломъ, но и духомъ, и молитву глаголати не токмо устами, но и умомъ и сердцемъ, и не токмо главу и колѣна преклонять, но и сердце наше предъ Нимъ, и къ Нему умныя наши очи возводить со смиреніемъ. Всякая убо молитва должна отъ сердца происходить; и что языкъ глаголетъ, умъ и сердце глаголати должны. Сего ради ничему такъ пріобучатися не должно, какъ истинной молитвѣ. Богъ смотритъ на сердце, а не на слова наши, и внимаетъ воздыханію сердечному, а не глаголанію устному. \textit{Господи! научи насъ молитися}\footnote{Лук.~11,~1.}. 2)~Таковая молитва, то"=есть, въ дусѣ и истинѣ совершаемая, на всякомъ мѣстѣ можетъ творитися. Духомъ бо вездѣ и на всякомъ мѣстѣ, въ домѣ и внѣ дома, въ явномъ и тайномъ мѣстѣ, при народѣ и безъ народа, ходя, и сѣдя, и лежа, и на пути, и на торжищи, и всегда и на всякомъ мѣстѣ предъ Богомъ свободно стоять и покланяться и молиться можно. Духъ бо всегда свободенъ есть, и вездѣ и всегда можетъ къ Богу приступать, и съ Нимъ бесѣдовать и Ему покланяться. И къ Богу, яко вездѣсущему, и на всѣхъ насъ взирающему, и о всѣхъ насъ промышляющему, вездѣ и на всякое время свободенъ приступъ есть; и всякаго вездѣ и всегда готовъ слышати. Къ человѣку не всегда можно приступать съ прошеніемъ нашимъ: къ Богу всегда двери отверсты суть. И хотя бы вси человѣки, по лицу всея земли разсѣянніи, къ Нему въ дусѣ и истинѣ приступили и молились, "--- всѣхъ бы услышалъ. \textit{Услыши ны Боже, Спасителю нашъ, упованіе всѣхъ концевъ земли и сущихъ въ мори далече}\footnote{Пс.~64,~6.}. 3)~Молитва должна быть со смиреніемъ и благоговѣніемъ. Хрістіанине! помяни, кто и предъ кѣмъ стоишь въ молитвѣ твоей. Земля и пепелъ, къ тому же грѣшникъ, предъ Богомъ святымъ, великимъ и непостижимымъ стоишь! Помяни, съ кѣмъ бесѣдуешь въ молитвѣ, кому говоришь: \textit{помилуй, Господи; подай, Господи; Ты, Господи; Тебѣ, Господи}, и прочая! Къ человѣку царю, или господину, или какой власти смиренно и благоговѣйно говоримъ; несравненно большее смиреніе и благоговѣніе нужно есть, когда предъ Богомъ стоимъ и бесѣдуемъ съ Нимъ; Богъ бо несравненно всякаго человѣка превосходитъ, и величество Его непостижимо. 4)~Коль велико дѣло и честь человѣческая есть "--- предъ Богомъ стоять и бесѣдовать съ Нимъ! Преславно съ царемъ земнымъ бесѣдовать: кольми паче съ Богомъ, Царемъ небеснымъ, Который есть Царь царствующихъ и Господь господствующихъ. Нѣтъ большія чести человѣку, какъ съ Богомъ бесѣдовать. Слава Тебѣ, Господь, яко сея чести насъ, бѣдныхъ грѣшниковъ, удостоилъ еси. Буди имя Господне благословенно во вѣки! 5)~Молитвы, каноны и стихи церковные, со скоростію и безъ разсужденія и вниманія чтомые, не ино что суть, какъ только единъ шумъ, какъ самъ видишь, хрістіанине; и болѣе они, тако чтомые, Бога раздражаютъ, нежели умилостивляютъ. Таковіи, хотя часто мнятся молитися, однакожъ никогда не молятся. Лучше предъ Богомъ сказать отъ сердца и со смиреніемъ и благоговѣніемъ два или три слова, нежели много прочитать молитвъ и каноновъ безъ разсужденія и со скоростію. Богъ внимаетъ внутренности, а не внѣшности, и слушаетъ вопіенія сердечная, а не устная. Моисей ничего устами не говорилъ, но только сердцемъ къ Богу молился, когда привелъ Израиля къ морю Чермному; но Богъ ему глаголалъ: \textit{что вопіеши ко Мнѣ}\footnote{Исх.~14,~15.}! Такожде Анна, мати Самуилова, ничего не говорила устами, но только сердцемъ единымъ воздыхала и вопила ко Господу; однакожъ услышана была молитва ея, и получила желаемый плодъ молитвы своея\footnote{1~Цар.~1,~13 и 20.}. Тако Богъ внимаетъ сердцу, а не устамъ нашимъ. Учися убо, хрістіанине, Богу молитися духомъ и истиною, и тако Ему покланятися, и Его призывати, и Его почитати. Богъ на всякомъ мѣстѣ есть, и всегда готовъ насъ слушать; всегда и вездѣ можешь Ему молитися и покланятися духомъ и истиною. 6)~Отсюду видишь, что молитва можетъ быть безъ голосу и внѣшнихъ словъ, умомъ и сердцемъ творимая, и есть дѣйствительна. Но внѣшняя молитва, въ единыхъ словахъ состоящая, не есть истинная молитва, но только голосъ безъ разума. Сіе же тогда бываетъ, когда человѣкъ читаетъ молитву, но иное думаетъ. Сего ради должно неотмѣнно внимать чтомымъ молитвы словамъ, и тѣмъ умъ и разсужденіе свое привязывать, и думать, что въ молитвѣ предъ Богомъ стоимъ и Ему говоримъ и молимся и милости отъ Него просимъ, "--- и тако пріобучаться истинной молитвѣ. Молитва бо, какъ есть великое добро и всѣхъ благъ виновна, то многаго труда и обученія требуетъ. И ничему такъ діаволъ препятствовать не тщится, какъ молитвѣ нашей, вѣдая, что тою вся благая отъ Бога получаемъ. Почему должно со всякимъ усердіемъ тщиться, чтобы молитва отъ сердца происходила, и умъ и разсужденіе чтомымъ молитвы словамъ было привязано. Хрістіанине, начиная молитися, поминай, что ты хощешь предъ Богомъ стать, и стоять въ молитвѣ, и Ему говорить, и отъ Него милости просить, такъ, какъ рабъ предъ господиномъ, или поданный предъ царемъ стоитъ, и кланяется ему, и милости отъ него проситъ. Вѣруй и думай, что Богъ \textit{близъ тебе и предъ тобою есть}, и тако возбудится истинная, сердечная и благоговѣйная молитва. Тогда будешь предъ Нимъ падать со смиреніемъ и кланяться, и воздыхать и молиться, и глаголати: \textit{о Господи, помилуй! о Господи, ущедри! о Господи, услыши!} Тогда и сердце и умъ согласенъ будетъ словамъ молитвы твоея. \textit{Близъ Господь всѣмъ призывающимъ Его, всѣмъ призывающимъ Его во истинѣ}\footnote{Пс.~144,~18.}. Слышишь благопріятное пѣніе или въ церкви, или индѣ: помяни тогда, что ангели на небеси тако поютъ Создателя и Бога нашего\footnote{Ис.~6,~3.}. Подражай и самъ ангеламъ святымъ, и пой на земли Создателя и Бога твоего, да и на небеси сподобишися нѣкогда пѣти Его съ ними. Приставай и ты къ святымъ онымъ пѣвцамъ, и пѣснь свою съ пѣснію ихъ связывай. А чтобы пѣніе твое Богу благопріятно было, то чистотою и житіемъ послѣдуй небеснымъ онымъ чинамъ, и, поя устами, пой и сердцемъ. Сладкая музыка есть, когда голоса хороши и между собою согласны: тако благопріятна Богу пѣснь бываетъ, когда пѣвецъ свято живетъ и добрые нравы имѣетъ, и устному пѣнію сердечное согласуетъ, и со устами и сердце поетъ. Ангельское дѣло есть пѣть Бога. Ангели бо не иное что на небеси дѣлаютъ, какъ непрестанно Бога поютъ. И мы убо, когда Бога поемъ, ангеламъ подражаемъ, и дѣло ихъ творимъ. Коль же сіе преславно есть, самъ видишь. Преславно на земли жить, и небеснымъ жителямъ послѣдовать, и свое пѣніе и голосъ съ пресладкимъ ихъ пѣніемъ и голосомъ совокуплять, и пѣть святую и животворящую Троицу. Но чтобы пѣніе наше согласно ихъ пѣнію было, да послѣдуемъ и святому житію ихъ, и будемъ любовны, согласны и мирны между собою, якоже они пребываютъ: тогда и пѣніе наше согласно будетъ пѣнію ихъ. Иначе не красна пѣснь во устахъ грѣшника. \textit{Восхвалю имя Бога моего съ пѣснію, возвеличу Его во хваленіи: и угодно будетъ Богу паче тельца юна, роги износяща и пазнокти}\footnote{Пс.~68,~31 и 32.}. Видишь трапезу богатую, и на той ядущихъ и піющихъ людей, или самъ съ ними яси и піеши: помяни трапезу и вечерю царствія Божія; помяни, что тако на трапезѣ Господни во царствіи Его возлягутъ благочестивіи, святіи и праведніи, по реченному: \textit{мнози отъ востокъ и западъ пріидутъ, и возлягутъ со Авраамомъ и Исаакомъ и Іаковомъ во царствіи небесномъ}\footnote{Мѳ.~8,~11.}. И паки глаголетъ Господь: се \textit{работающіи Ми ясти будутъ, се работающіи Ми пити будутъ, се работающіи Ми возрадуются, се работающіи Ми возвеселятся въ веселіи сердца}\footnote{Ис.~65,~13 и 14.}. Но помяни писанное: \textit{мнози суть звани, мало же избранныхъ}\footnote{Лук.~14,~24.}; и паки: \textit{многими скорбьми подобаетъ намъ внити въ царствіе Божіе}\footnote{Дѣян.~14,~22.}. Живи убо такъ въ мірѣ, чтобы радости и веселія онаго не лишитися. Работай Господеви вѣрою и правдою, и всякую приключающуюся скорбь безропотно претерпѣвай, да и во царствіи Божіи сподобишися ясти, пити и веселится; претерпи временную скорбь, да вѣчныя жизни получиши радость; вкуси горести крестныя, да и сладости во царствіи Божіи вкусишь. \textit{Помяни мя Господи, во царствіи Твоемъ}\footnote{Лук.~23,~48.}. "--- Смотришь на солнце, и удивляешися красотѣ его: помяни писанное: \textit{тогда праведницы, просвѣтятся, яко солнце, во царствіи Отца ихъ}\footnote{Мѳ.~13,~43.}. Видишь, коль великая слава избранныхъ Божіихъ будетъ. \textit{Возлюбленніи}, глаголетъ Апостолъ, \textit{нынѣ чада Божія есмы, и не у явися, что будемъ. Вѣмы же, яко егда явится, подобни Ему будемъ, и узримъ Его, якоже есть}\footnote{1~Іоан.~3,~2.}. Ибо Той \textit{преобразитъ тѣло смиренія нашего, яко быти ему сообразну тѣлу славы Его}\footnote{Филип.~3,~21.}. Хрістіанине! когда хощемъ тамо Хрісту сообразны и подобны быти въ славѣ, то должны и здѣ въ житіи и терпѣніи подобны и сообразны Ему быть. Сего ради глаголетъ Апостолъ: \textit{всякъ, имѣяй надежду сію Нань, очищаетъ себе, якоже Онъ чистъ есть}\footnote{1~Іоан.~3,~3.}. Вси хотятъ со Хрістомъ прославленнымъ и превознесеннымъ быть; но Хрісту послѣдовать и со Хрістомъ крестъ носить, и поруганіе, уничиженіе, посмѣяніе и скорбь терпѣть мало кто хощетъ. Но глаголетъ Онъ: \textit{иже не пріиметъ креста своего, и въ слѣдъ Мене грядетъ, нѣсть Мене достоинъ}\footnote{Матѳ.~10,~38.}. Хотящему убо быти со Хрістомъ во царствіи и славѣ Его, надобно и здѣ, въ мірѣ семъ, быти съ Нимъ, и смиреніемъ и терпѣніемъ Ему послѣдовать, и тако крестъ свой носить. "--- Когда имѣешися въ бани, и чувствуешь горячесть и жженіе своего тѣла; или находишися въ горячкѣ, или въ лихорадкѣ, или иной какой тяжкой болѣзни: помяни о вѣчной мукѣ, помяни, како осужденніи во огни геенскомъ и душею и тѣломъ будутъ страдать во вѣки вѣковъ. \textit{Огнь ихъ не угасаетъ, и червь ихъ не усыпаетъ}\footnote{Марк.~9,~44.}. \textit{Тамо будетъ плачь и скрежетъ зубомъ}\footnote{Мѳ.~25,~30.}. Поминай оное зло, да не впадеши въ тое зло. Снисходи нынѣ умомъ во адъ, да не потомъ душею и тѣломъ снидеши. Память геенны не допуститъ впасть въ геенну. Грѣхъ всякій къ бѣдствію оному приводитъ: берегися всякаго грѣха, да не вринетъ тебе во оное бѣдствіе. Тамо люди каются, но поздно; тамо люди воздыхаютъ, но поздно; тамо люди плачутъ, но поздно. Хрістіанине! Богу слава, ты еще не погибъ, ты еще на земли живешь, ты еще на оное мученія мѣсто не пришелъ; еще благодатію Божіею можешь спастися, и зла и бѣдствія онаго убѣжать. Кайся убо нынѣ, пока покаяніе полезно; воздыхай и плачи, да не къ оному воздыханію и плачу пріидеши; молись и толкай въ двери милосердія Божія, да тебѣ отверзутся, которыя всѣмъ истинно кающимся отворяются. \textit{Помилуй мя, Боже, по велицѣй милости Твоей, и по множеству щедротъ Твоихъ очисти беззаконіе мое}\footnote{Пс.~50,~3.}. \textit{Страшливымъ и невѣрнымъ, и сквернымъ и убійцамъ, и блудъ творящимъ, и чары творящимъ, идоложерцемъ и всѣмъ лживымъ, часть имъ въ езерѣ горящемъ огнемъ и жупеломъ, еже есть смерть вторая}\footnote{Апок.~21,~8.}. "--- Видишь, что малый скотъ большому скоту, малый звѣрь большому звѣрю, малая птица большой птицѣ уступаютъ: помяни, что тако люди, малые въ себѣ самихъ, другимъ уступаютъ, и даютъ мѣсто гнѣву; укоряющихъ не укоряютъ, злословящихъ не злословятъ, отъ востающихъ уклоняются, біющимъ не противятся. Сіе въ нихъ дѣйствуетъ истинное смиреніе и кротость. Отсюду, видишь, что есть смиреніе и кротость. А наипаче тогда познается смиреніе и кротость, когда равный равному уступаетъ; и хотя можетъ востающему противится, но не хощетъ того, и уступаетъ ему. Большее еще смиреніе и кротость показуется, когда большій меньшему уступаетъ, и не противится ему ни дѣломъ, ни словомъ. О, любезное зрѣлище, когда высокій человѣкъ низкому и подлому уступаетъ! Таковый человѣкъ внѣ высокъ, но внутрь себе низокъ; внѣ великъ, но въ сердцѣ своемъ малъ; внѣ богатъ, но внутрь себе нищъ и убогъ. Сія высокая подлость отъ міра презрѣна, но отъ Бога возносится. \textit{Всякъ смиряяй себе вознесется}\footnote{Лук.~18,~14.}. Тако уступалъ противникамъ своимъ Давидъ святый; и хотя моглъ имъ зло за зло воздать, однакожъ не хотѣлъ. Спасителя нашего смиреніе и кротость все святое Евангеліе проповѣдуетъ, Который всѣмъ Своимъ врагамъ и хулителямъ со всякою кротостію уступалъ. Столько Онъ смиренъ и кротокъ, сколько великъ. Почему и намъ велитъ учитися отъ Себе смиренію и кротости: \textit{научитеся отъ Мене, яко кротокъ есмь и смиренъ сердцемъ}\footnote{Матѳ.~11,~29.}. Не тако безумная гордость. Она вездѣ хощетъ себе оказать, не хощетъ никому уступить, но или словомъ, или дѣломъ противится. Оттуда востаютъ ссоры, взаимное руганіе, злословіе и драки. Одинъ говоритъ другому: ты плутъ, или мотъ, или иное что; гордость отвѣтствуетъ: ты самъ такой. Тако дѣлаютъ и скоты, звѣри и птицы, когда видятъ себе другимъ равными, не уступаютъ имъ, но противятся имъ; почему и дерутся дотолѣ, доколѣ одинъ другаго переможетъ. Симъ несмысленнымъ скотамъ послѣдуютъ люди гордыи и несмысленныи, и другъ другу уступить не хотятъ, но по подобію скотовъ ссорятся и дерутся, и другъ друга или дѣломъ или словомъ уязвляютъ. Видишь убо, человѣче, что есть неотрожденный человѣкъ, и Божіею благодатію необновленный. Какая гордость и гнѣвъ въ скотѣ и звѣрѣ, таковая въ немъ имѣется. Но писано есть: \textit{Богъ гордымъ противится, смиреннымъ же даетъ благодать}\footnote{1~Петр.~5,~5.}. \textit{Тому похвала, честь и слава во вѣки вѣковъ буди, буди!} Аминь.

\begin{center}\small\textsc{Конецъ четвертаго тома.}\end{center}
