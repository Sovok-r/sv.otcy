
\chapter*{Жизнеописание\\епископа Игнатия Брянчанинова,\\\normalfont\small\textsc{составленное его ближайшими учениками в 1881 году}}
\addcontentsline{toc}{chapter}{Жизнеописание епископа Игнатия Брянчанинова (1881)}
\begin{quotation}

«\textit{Поминайте наставники ваша, иже глаголаша вам слово Божие: ихже взирающе на скончание жительства, подражайте вере их}»

\hfill(Евр. 13: 7).

\end{quotation}
\section{Вступление}

Истекло 12 лет со дня мирной кончины приснопамятного святителя"=инока Церкви Русской XIX века, преосвященнейшего епископа \textit{Игнатия} Брянчанинова. Близко еще время его к нам, живы еще многие его современники, спостники, ученики, и между тем светлая личность святопочившего святителя Божия высоко уже стоит над нами, светло светит нам светом христианских его добродетелей, подвигами строго"=иноческого его жития и аскетическими его писаниями. Краса иночества нашего века, святитель является деятельным учителем иноков и не только в писаниях своих, но и во всей жизни своей представляет дивную картину самоотвержения, близкого к исповедничеству, борьбы человека со страстями, скорбями, болезнями, картину жизни, которая при помощи и действии обильной благодати Божией увенчалась победой, привлекла к подвижнику многие редкие дары Святого Духа. С благоговением следя за этим многострадальным и многоскорбным шествием подвижника к преуспеянию духовному и ясно созерцая при этом особое водительство Промысла Божия во всей его жизни, невольно ощущаешь живое познание веры в отеческое попечение о нас Бога, Творца и Спасителя нашего, и проникаешься желанием подражать, по мере сил, этому, современному нам, образцу совершенства христианского. Предоставляя будущему биографу подробную и обстоятельную оценку плодотворной деятельности незабвенного святителя, мы, в настоящую минуту, решились предложить только краткое жизнеописание в Бозе почившего преосвященного Игнатия, составленное по запискам ближайших его учеников и родного брата его Петра Александровича Брянчанинова, глубоко преданного ему в отношении духовном, разделявшего с ним уединение последних лет жизни его на покое в Николо"=Бабаевском монастыре и пользовавшегося полным доверием и любовью блаженного святителя, так равно и сподвижника друга его, от ранних лет юности и до глубокой старости, Сергиевой пустыни схимонаха Михаила Чихачова, с которым начал он свой подвиг иноческий и вместе с ним проходил его до самого епископства, "--- друга, пред которым святитель не таил ни одного из событий своей жизни, и, наконец, "--- главное, "--- руководились собственными повествованиями архипастыря"=инока о своих немощах, борениях, скорбях, чувствах и благодатных ощущениях, которые изложены им в его творениях. Все сочинения вообще, а духовно"=нравственные преимущественно, обладают тем свойством, что в них вполне точно выражается внутренняя жизнь их авторов. Таким образом сочинения дают обильный материал биографу для начертания характеристики лица, этой существенной части жизнеописания; но чтобы в неложных чертах изобразить жизнь преосвященного Игнатия, надлежит самому изучить и испытать нечто такое, что он изучал и испытывал. Изучение же здесь так мало доступно, опыты столь исключительны, что всего менее зависят от собственных усилий и воли человека. Кто Промыслом Божиим поставлен на подобную дорогу и отчасти введен в горнило подобных испытаний, лишь тот может знать всю особенность таких опытов, и с этой стороны правильнее оценить деятельность представителя их. Жизнеописания особенно замечательных или передовых людей отличаются тем признаком, что в них, преимущественно, выказывается какая"=нибудь одна сторона, с которой, деятельность этих лиц особенно проявляется, которая отличает их резкими, характеристическими чертами и сосредоточивает на себе все внимание: это как бы лицевая сторона всей их деятельности, скрывающая за собою все прочие. В жизнеописаниях таких личностей необходимо схватывать этот признак и проводить его вполне от начала до конца жизнеописания; тогда оно будет иметь свойственную выдержку. В этом отношении жизнь преосвященного Игнатия имеет особенное преимущество: она представляет такую отличительную сторону, которая совершенно выделяет его личность в ряду прочих, современных ему, духовных деятелей. Такую сторону его жизни составляет полное самоотвержение ради точного исполнения Евангельских заповедей в потаенном иноческом духовном подвиге, послужившем предметом нового, аскетически"=богословского учения в нашей духовной литературе, "--- учения о внутреннем совершенствовании человека в быту монашеском и отношений его к другим духовным существам, влияющим на него, как по внутреннему человеку, так и со стороны внешней или физической. Вот та особенность, которая отличает преосвященного Игнатия в ряду прочих духовных писателей нашего времени, "--- особенность резкая, однако не всеми точно усматриваемая, верно различаемая.

\section{Глава I}

Преосвященный Игнатий был избран на служение Богу от чрева матери. Такое избрание "--- удел весьма редких и нарочитых служителей Божиих, "--- предзнаменовалось следующим обстоятельством. Родители преосвященного сочетались браком в ранней молодости. В начале супружества у них родилось двое детей, но родители не долго утешались ими; оба детища умерли на первых днях младенчества, и юная чета пребывала долго бездетною. В глубокой печали о своем продолжительном бесчадии, молодые супруги обратились к единственной помощи "--- помощи Небесной. Они предприняли путешествие по окрестным святым местам, чтобы усердными молитвами и благотворением исходатайствовать себе разрешение неплодия. Благочестивое предприятие увенчалось успехом: плодом молитв скорбящих супругов был сын, нареченный Димитрием, в честь одного из первых чудотворцев Вологодских "--- преподобного Димитрия Прилуцкого. Таким образом, очевидно, неплодство молодых Брянчаниновых было устроением Промысла Божия, чтобы рожденный после неплодства первенец, испрошенный молитвою, впоследствии сделался ревностным делателем ее и опытным наставником. Младенец Димитрий родился 6 февраля 1807 года в с. Покровском, которое было родовым имением его отца и находится в Грязовецком уезде, Вологодской губернии. Будущий инок имел счастливую участь провести свое детство в уединении сельской жизни, в ближайшем соприкосновении с природою, которая, таким образом, явилась первою его наставницею. Она вселила в него наклонность к уединению: отрок часто любил оставаться под тенью вековых дерев обширного сада и там, одинокий, погружался в тихие думы, содержание которых, без сомнения, заимствовалось из окружающей природы. Величественная и безмолвная, она рано начала влиять на него своими вдохновляющими образами: она внушала его детской душе, еще незапятнанной житейскою мелочностью, иные более возвышенные стремления, какими бывает полна жизнь пустынная, она восхищала его сердце более живыми, чистыми чувствованиями, какие способно доставить только уединение. Отрок рано научился понимать этот безмолвный голос природы и отличать его от шума житейского. Явления домашней жизни не действовали на него впечатлительно, "--- он более углублялся в себя и среди изящной светской обстановки казался питомцем пустыни. Искра Божественной любви запала в его чистое сердце. Она сказалась в нем безотчетным влечением к иночеству, к его высоким идеалам, которыми так полна родная сторона, особенным расположением ко всему священному и истинно"=прекрасному, сколько это доступно для детского возраста. С этой ранней поры жизни дальнейший путь ее уже определился. Отрок духовно был отделен от мира. Такое настроение малолетнего Димитрия не могло рассчитывать на сочувствие со стороны родителей. Его отец Александр Семенович Брянчанинов, потомок древних дворян Брянчаниновых, фамилии весьма известной и чтимой в Вологде, был в полном смысле слова светский человек. Паж времени Императора Павла Петровича, он имел необыкновенно развитый вкус к изяществу в домашней обстановке и представлял собою совершенный тип современного передового русского помещика. Наследовав от своих родителей значительное имение, он должен был истощить большую часть его на уплату огромных долгов, после чего ему осталось около 400 душ крестьян, да живописное село Покровское, издавна бывшее местопребыванием его предков, "--- родина будущего святителя. Супруга его, мать преосвященного Игнатия, Софья Афанасьевна происходила также из фамилии Брянчаниновых и, как женщина замечательного образования, весьма благочестивая, памятуя, что муж есть глава, во всем подчинялась влиянию мужа, разделяя его взгляды и понятия. Александр Семенович по справедливости считался в числе передовых образованных помещиков своего времени и любил просвещение\footnote{Все время жизни своей в с. Покровском он содержал постоянно, на полном своем иждивении, приходское двух"=классное училище, в котором обучалось до 50 чел. крестьянских детей.}, а потому и детям своим старался дать, по возможности, основательное воспитание, чтобы приготовить из них истинных сынов отечества, преданных престолу, верных Православию. Давая такое воспитание, он не чужд был честолюбия видеть, впоследствии, сыновей своих занимающими почетные должности на государственной службе. От проницательности юного Димитрия не могла укрыться эта черта его родителя, черта совершенно противоположная намерениям и стремлениям юноши, и вот начало внутренней борьбы, начало страданий и испытаний, сделавшихся потом уделом всей жизни почившего владыки.

Все дети в семействе Брянчаниновых, братья и сестры Димитрия Александровича, воспитывались вместе, связанные взаимной дружбой, но все сознавали главенство Димитрия и сознавали не потому только, что он был старший, а вследствие особого, высшего, так сказать, склада его ума и характера, вследствие нравственного его превосходства. Пользуясь всегдашним уважением от братьев и сестер и превосходя всех их в научных способностях и других дарованиях, Димитрий Александрович не обнаруживал ни малейшего превозношения или хвастовства. Зачатки иноческого смиренномудрия высказывались в тогдашнем его поведении и образе мыслей; по нравственности и уму он был несравненно выше лет своих, и вот причина, почему братья и сестры относились к нему даже с некоторым благоговением, а он, в свою очередь, сообщал им свои нравственные качества.

С возрастом, религиозное настроение Димитрия Александровича обнаруживалось заметнее: оно проявлялось в особенном расположении к молитве и чтению книг духовно"=нравственного содержания. Он любил часто посещать церковь, а дома имел обыкновение молиться часто в течение дня, не ограничиваясь установленным временем "--- утром и вечером. Молитва его не походила на урочное вычитывание, часто торопливое и машинальное, что так обыкновенно у детей; он приучался к внимательной молитве, которая начинается с благоговейного предстояния и неспешного произношения слов молитвенных, и так преуспевал в ней, что еще в детстве наслаждался ее благодатными плодами. Учась молиться внимательно, он с благоговением относился ко всему священному, внушая это благоговение и прочим своим братьям и сестрам; Евангелие всегда читал с умилением, размышляя о читанном. Любимою его книгою было: «Училище благочестия» в пяти томах старинного издания. Книга эта, содержащая краткое изложение деяний Святых и избранные изречения их, весьма соответствовала настроению отрока, или, вернее, она настраивала его дух, предоставляя Святым повествованиям и изречениям Духоносных мужей самим действовать на него, без посредства посторонних пояснений. Способности Димитрия Александровича были весьма многосторонни: кроме установленных занятий в науках, он с большим успехом упражнялся в каллиграфии, рисовании, нотном пении и даже музыке, притом, на самом трудном инструменте, какова скрипка. Выучивая очень скоро свои уроки, свободные часы он употреблял на чтение и разные письменные упражнения, в которых также начинало выказываться его литературное дарование. Наставниками его в это время были профессора Вологодской семинарии и учителя гимназии. Домашним учителем был студент семинарии Левитский, живший в семействе Брянчаниновых. Он же преподавал и закон Божий. Левитский отличался замечательным благонравием и основательным знанием своего предмета. Он так хорошо умел ознакомить своего ученика с начальными истинами богословия, что Димитрий Александрович до конца жизни сохранял благодарное воспоминание о нем.

Жизнь Димитрия Александровича в доме родительском продолжалась до 16"~го года его возраста; этот первый период жизни уже был труден для него в духовном отношении тем, что внешние и внутренние условия жизни в доме родителей не допускали возможности открывать, кому бы то ни было, заветные желания и цели, наполнявшие тогда его душу. В заключение периода детства автора «Аскетических опытов», весьма назидательно привести собственное его поведание об этом детстве. Вот как трогательно он говорит о себе в статье «Плач мой»: «Детство мое было преисполнено скорбей. Здесь вижу руку Твою, Боже мой! Я не имел кому открыть моего сердца; начал изливать его пред Богом моим, начал читать Евангелие и жития святых Твоих. Завеса, изредка проницаемая, лежала для меня на Евангелии; но Пимены Твои, Твои Сисои и Макарии производили на меня чудное впечатление. Мысль, часто парившая к Богу молитвою и чтением, начала мало"=помалу приносить мир и спокойствие в душу мою. Когда я был пятнадцатилетним юношею, несказанная тишина возвеяла в уме и сердце моем. Но я не понимал ее, "--- я полагал, что это обыкновенное состояние всех человеков»\footnote{Аскетические опыты. Том 1. Статья «Плач мой».}.

В конце лета 1822 года, когда Димитрию Александровичу шел шестнадцатый год от рождения, родитель повез его в С."=Петербург для определения в Главное инженерное училище, куда он был подготовлен домашним учением. Дорогой, близ Шлиссельбурга, отец внезапно обратился к сыну с следующим вопросом: «Куда бы ты хотел поступить на службу?» Пораженный такой небывалой откровенностью отца, сын не хотел более скрывать от него своей сердечной тайны, которой до сих пор никому не открывал; сперва он испросил у него обещание не сердиться, если ответ ему не понравится; затем, с твердостью воли и силой вполне искреннего чувства, сказал, что желает идти «в монахи». Решительный ответ сына, по"=видимому, не подействовал на отца; он или не дал ему значения на основании молодости отвечавшего, или не хотел возражать по кажущейся несбыточности желания, которое совершенно расходилось с планами, какие он строил о будущности своего сына. В Петербурге Димитрий Александрович сдал блистательно вступительный экзамен\footnote{В этом году на 30 вакансий было 130 конкурентов. Из числа их Брянчанинов не только был первым, но исключительно он один удовлетворил требованиям для поступления во 2"~й кондукторский класс.}. Благообразная наружность и отличная подготовка в науках обратили на молодого Брянчанинова особенное внимание Его Высочества Николая Павловича, бывшего тогда генерал"=инспектором инженеров. Великий Князь приказал Брянчанинову явиться в Аничковский дворец, где представил его своей супруге, Государыне Великой Княгине Александре Феодоровне, и рекомендовал, как отлично приготовленного не только к наукам, требуемым в инженерном училище, но знающего даже латинский и греческий языки. Ее Высочество благоволила приказать зачислить Брянчанинова Ее пенсионером. Сделавшись Императором, Николай Павлович и Императрица Александра Феодоровна продолжали оказывать свое милостивое расположение Брянчанинову. По сдаче экзамена Димитрий Александрович зачислен был в кондукторскую роту Главного инженерного училища, а действительная служба его стала считаться со дня принесения им присяги 19 января 1823 года. Успехи по наукам\footnote{В самом непродолжительном времени Брянчанинов стал 1"~м учеником своего класса, и сохранил это место по наукам до самого выхода из училища.}, отличное поведение и расположение Великого Князя выдвигали его на первое место между юнкерами"=товарищами: к концу 1823 года, с переводом в верхний кондукторский класс, он был назначен фельдфебелем кондукторской роты; в 1824 году был переведен из юнкерских классов в нижний офицерский (что ныне Николаевская инженерная академия) и 13 декабря произведен в инженер"=прапорщики. Редкие умственные способности и нравственные качества Димитрия Александровича привлекали к нему профессоров и преподавателей училища; все они относились к нему с особенною благосклонностью, отдавая явное предпочтение пред прочими воспитанниками.

Наряду с служебно"=учебной деятельностью Димитрий Александрович имел успехи и в светском обществе своими личными достоинствами. Родственные связи ввели его в дом тогдашнего президента Академии художеств Оленина. Там, на литературных вечерах он сделался любимым чтецом, а поэтические и вообще литературные дарования его приобрели ему внимание тогдашних знаменитостей литературного мира: Гнедича, Крылова, Батюшкова и Пушкина. Такое общество, конечно, благодетельно влияло на литературное развитие будущего писателя. Преосвященный Игнатий до конца жизни сочувственно отзывался о советах, какие ему давали тогда некоторые из этих личностей.

Описанный круг светского знакомства, к которому принадлежала, имевшая большие связи, тетка Димитрия Александровича А. М. Сухарева, только внешним образом влиял на жизнь молодого человека, внутренняя жизнь которого развивалась самостоятельно, независимо от родственных и общественных связей. Димитрий Александрович и в шуме столичной жизни остался верен своим духовным стремлениям, какие испытал в уединении отдаленной родины: он всегда искал в религии живого, опытного знания и, хранимый благодатию, не поддавался ни тлетворному влиянию чуждых учений, ни приманкам светских удовольствий. Вот с какою подробностью он сам, в вышеприведенной статье «Плач мой», описывает тогдашнее свое душевное состояние: «Вступил я в военную и вместе ученую службу не по своему избранию и желанию. Тогда я не смел, "--- не умел желать ничего; потому что не нашел еще Истины, еще не увидел Ее ясно, чтобы пожелать Ее! Науки человеческие, изобретение падшего человеческого разума, сделались предметом моего внимания: к ним я устремился всеми силами души; неопределенные занятия и ощущения религиозные оставались в стороне. Протекли почти два года в занятиях земных: родилась и уже возросла в душе моей какая"=то страшная пустота, явился голод, явилась тоска невыносимая по Боге. Я начал оплакивать нерадение мое, оплакивать то забвение, которому я предал веру, оплакивать сладостную тишину, которую я потерял, оплакивать ту пустоту, которую я приобрел, которая меня тяготила, ужасала, наполняя ощущением сиротства, лишения жизни! И точно "--- это было томление души, удалившейся от истинной жизни своей, Бога. Воспоминаю: иду по улицам Петербурга в мундире юнкера, и слезы градом льются из очей»…

«Понятия мои были уже зрелее, я искал в религии определительности. Безотчетные чувствования религиозные меня не удовлетворяли, я хотел видеть верное, ясное, Истину. В то время разнообразные религиозные идеи занимали и волновали столицу северную, препирались, боролись между собою. Ни та, ни другая сторона не нравились моему сердцу; оно не доверяло им, оно страшилось их. В строгих думах снял я мундир юнкера и надел мундир офицера. Я сожалел о юнкерском мундире: в нем можно было, приходя в храм Божий, стать в толпе солдат, в толпе простолюдинов, молиться и рыдать сколько душе угодно. Не до веселий, не до развлечений было юноше! Мир не представлял мне ничего приманчивого: я был к нему так хладен, как будто мир был вовсе без соблазнов! Точно их не существовало для меня: мой ум был весь погружен в науки, и вместе горел желанием узнать, где кроется истинная вера, где кроется истинное учение о ней, чуждое заблуждений и догматических и нравственных»\footnote{Аскетические опыты. Том 1. Статья «Плач мой».}.

\section{Глава II}

Начало духовной деятельности, когда она предпринимается с определенною целью и становится преобладающею, чтобы затем сделаться вполне исключительною, сопровождается обыкновенно внутреннею бранью помыслов и страстных чувствований. Брань эта столь сильна, что противостоять ей собственными силами нет никакой возможности, "--- нужна помощь Свыше. Димитрий Александрович обратился к молитве, творя ее внутренно, внимательно и непрестанно. Такая молитва, образуя внутреннего монаха, настраивает сообразно себе всю душевную деятельность человека, но такой молитве необходимо правильно обучаться, что и составляет предмет монашеского духовного делания. Он занимался умною молитвою, и столь рачительно упражнялся в ней, что она творилась у него самодейственно. «Бывало с вечера, "--- рассказывал он впоследствии о себе, "--- ляжешь в постель и, приподняв от подушки голову, начнешь читать молитву, да так, не изменяя положения, не прекращая молитвы, встанешь утром идти на службу, в классы». Таким образом, будучи монахом по душе и, еще на 16 году жизни, испытав благодатное действие молитвы, набожный сей юноша не мог довольствоваться установленным в училище обычаем "--- только однажды в год приступать к таинствам исповеди и Св. Причастия, а нуждался в более учащенном подкреплении себя этою духовною пищею, почему для удовлетворения своего желания он обратился к законоучителю и духовнику училища. Такое необычайное, среди юношества, явление, вызвало удивление духовника, особенно, когда исповедающийся сказал, что «борим множеством греховных помыслов». Не делая различия между «греховными помыслами» и «политическими замыслами», отец протоиерей счел своею обязанностью довести об этом обстоятельстве до сведения училищного начальства. Начальник училища генерал"=лейтенант граф Сиверс подверг обвиняемого юношу формальному допросу о значении помыслов, им самим признанных «греховными». Немецкое начальство\footnote{Инспектором Училища был Инженер Ген."=Майор Барон Эльснер, с трудом объяснявшийся на русском языке.}, не уяснив себе значения этого выражения, за Брянчаниновым стало следить. Неосмотрительность духовника повергла Брянчанинова в тяжкую ответственность пред своим начальством и довела до болезненного состояния; он принужден был избрать себе другого духовника. По сему Брянчанинов обратился к инокам Валаамского подворья, стал ходить туда каждую субботу и воскресенье для исповеди и Св. Причащения и, наученный опытом, старался делать это скрытно от училищного начальства. В этом святом деле к нему присоединился товарищ по училищу, Чихачов, из дворян Псковской губернии, одновременно с ним поступивший в училище и весьма любимый Государем Николаем Павловичем. Димитрий Александрович привязался к Чихачову самою искреннею дружбою, несмотря на несходство их характеров: первый был серьезен, задумчив, сосредоточен в себе, другой "--- весельчак, говорун, с душой нараспашку. Чихачов предался Брянчанинову скорее, как сын отцу, нежели как брат брату: таково было влияние Димитрия Александровича на своего сотоварища. Самое первое знакомство этих двух молодых товарищей полно умиления и истинно"=христианского характера. Однажды в дружеских разговорах Димитрий Александрович прервал веселую болтовню Чихачова, сказав ему: «Будь ты христианином!» "--- «Я никогда не бывал татарином», "--- возразил товарищ ему. "--- «Так, "--- сказал первый, "--- да надо слово это исполнить делом и углубиться поприлежнее в него». С того времени оба они ходили к инокам на подворье, исповедывались и причащались, молились, назидались душеспасительными беседами, подвизались. Вот как эти хождения описывает в своих записках сам Чихачов, где откровенно говорит, какое они производили на него действие: «В одну субботу слышу приглашение от товарища своего идти к священнику. "--- «Зачем?» "--- «Да обычай у меня исповедаться, а в воскресенье приобщаться Св. Христовым Тайнам; смотри и ты не отставай». Бедная моя головушка пришла тогда в изумление и великое смятение. Страх и ужас: что и как, не готов, не могу! "--- «Не твое дело, а духовника», "--- отвечает храбро товарищ, и любовью своею влечет за собою. Юность и здоровье, и все внешние обстоятельства и вся обстановка, да к тому же и внутреннее сильное восстание страстей и привычек, разъяренных противодействием им, страшно волновали душу, и могла ли бы она своею немощью устоять, если б не была невидимая сила, свыше поддерживавшая ее? "--- И при всем этом, не будь у меня такого друга, который и благоразумием своим меня вразумлял, и душу свою за меня всегда полагал, и вместе со мною всякое горе разделял, не уцелел бы я на этом поприще, "--- поприще мученичества добровольного и исповедничества».

Иноки Валаамского подворья с любовью принимали молодых людей, потому что видели в них искреннее стремление к Богу и желание пути спасительного, но они, как люди без научного образования, по преимуществу, ограничивавшиеся внешним подвигом, не могли удовлетворить вполне их духовных потребностей, почему и посоветовали молодым людям обращаться за душеназиданием к инокам Невской Лавры. Там, в это время, пребывали некоторые ученики старцев отца Феодора и отца Леонида, мужей опытных в духовной жизни, получивших монашеское образование первый у известного старца Паисия Величковского, архимандрита Молдавского Нямецкого монастыря, а второй у учеников его. Таковы были монах Аарон, монахи Харитон, Иоанникий и другие. Молодые люди стали ходить к этим инокам; через них познакомились они с лаврским духовником отцом Афанасием, который своим истинноотеческим, любвеобильным обхождением поддержал их живое стремление к христианскому благочестию. Молодые люди радовались, нашедши себе истинных наставников, понимавших их духовные нужды, и могущих пользовать обильно. Они усугубили свою ревность к подвигам благочестия, участили посещения свои к инокам, услаждались богослужением Лавры, которое производило на них благое впечатление, потому что было величественнее и продолжительнее, чем на Валаамском подворье. Они совещались с иноками, как с духовными отцами, обо всем, что касается внутреннего монашеского делания, исповедывали свои помыслы, учились как охранять себя от страстей, греховных навыков и преткновений, какими руководствоваться книгами из писаний святых Отцов и т. п. Добрые иноки, особенно отец Иоанникий и духовник отец Афанасий делились с монахолюбивыми и любомудрыми юношами всем, что составляло достояние их многолетней духовной опытности. Часто Димитрий Александрович удивлял их своими вопросами, которые касались таких сторон жизни духовной, какие свидетельствуют о довольно зрелом духовном возрасте. Такая тесная дружба с иноками имела соответственное себе действие. Димитрий Александрович сделался совершенным аскетом по душе, обложил себя творениями святых Отцов, преимущественно подвижнического содержания, которые перечитывая с жадностью, еще более углублялся в самосозерцание и видимо охладел к светскому обществу. В «Плаче» своем так говорит он о себе:

«Пред взорами ума уже были грани знаний человеческих в высших окончательных науках. Пришедши к граням этим, я спрашивал у наук: что вы даете в собственность человеку? Человек вечен, и собственность его должна быть вечна. Покажите мне эту вечную собственность, это богатство верное, которое я мог бы взять с собою за пределы гроба! Науки молчали».

«За удовлетворительным ответом, за ответом существенно нужным, жизненным, обращаюсь к вере. Но где ты скрываешься, вера истинная и святая? Я не мог тебя признать в фанатизме, который не был запечатлен евангельскою кротостью; он дышал разгорячением и превозношением! Я не мог тебя признать в учении своевольном, отделяющемся от церкви, составляющем свою новую систему, суетно и кичливо провозглашающем обретение новой истинной веры христианской, чрез осмнадцать столетий по воплощении Бога"=Слова. Ах! В каком тягостном недоумении плавала душа моя!»…

«И начал я часто, со слезами, умолять Бога, чтобы Он не предал меня в жертву заблуждению, чтоб указал мне правый путь, по которому я мог бы направить к Нему невидимое шествие умом и сердцем. Внезапно предстает мне мысль… сердце к ней, как в объятия друга. Эта мысль внушала изучить веру в источниках "--- в писаниях святых Отцов. «Их святость, "--- говорила она мне, "--- ручается за их верность: их избери в руководители». Повинуюсь. Нахожу способ получать сочинения святых угодников Божиих, с жадностью начинаю читать их, глубоко исследовать. Прочитав одних, берусь за других, читаю, перечитываю, изучаю. Что прежде всего поразило меня в писаниях Отцов Православной Церкви? "--- Это их согласие, согласие чудное, величественное»… Какое между прочим учение нахожу в них? "--- Нахожу учение, повторенное всеми Отцами, учение "--- что единственный путь к спасению, "--- последование неуклонное наставлениям святых Отцов. «Видел ли ты, "--- говорят они, "--- кого прельщенного лжеучением, погибшего от неправильного избрания подвигов "--- знай: он последовал себе, своему разуму, своим мнениям, а не учению Отцов», из которых составляется догматическое и нравственное предание Церкви»…

«Мысль эта была для меня первым пристанищем в стране истины. Здесь душа моя нашла отдохновение от волнения и ветров. Мысль благая, спасительная! Мысль "--- дар бесценный всеблагого Бога, хотящего всем человекам спастись и придти в познание истины! Эта мысль соделалась камнем основным для духовного созидания души моей! Эта мысль соделалась моею звездою путеводительницею! Она начала постоянно освящать для меня многотрудный и многоскорбный, тесный, невидимый путь ума и сердца к Богу».

«Таковы благодеяния, которыми ущедрил меня Бог мой! Таково нетленное сокровище, наставляющее в блаженную вечность, ниспосланное мне свыше от горнего престола Божественной милости и премудрости»… «Бог, Сам Бог мыслью благою уже отделил меня от суетного мира. Я жил посреди мира, но не был на общем, широком, углажденном пути: мысль благая повела меня отдельною стезею, к живым, прохладным источникам вод, по странам плодоносным, по местности живописной, но часто дикой, опасной, пересеченной пропастями, крайне уединенной. По ней редко странствует путник».

«Чтение Отцов с полною ясностью убедило меня, что спасение в недрах Российской Церкви несомненно, чего лишены вероисповедания западной Европы, как несохранившие в целости ни догматического, ни нравственного учения первенствующей Церкви Христовой. Оно открыло мне, что сделал Христос для человечества, в чем состоит падение человека, почему необходим Искупитель, в чем заключается спасение, доставленное и доставляемое Искупителем. Оно твердило мне: должно развить, ощутить, увидеть в себе спасение, без чего вера во Христа "--- мертва, а христианство "--- слово и наименование без осуществления его! Оно научило меня смотреть на вечность, как на вечность, пред которой ничтожна и тысячелетняя земная жизнь, не только наша, измеряемая каким"=нибудь полустолетием. Оно научило меня, что жизнь земную должно проводить в приготовлении к вечности, как в преддвериях приготовляются ко входу в великолепные царские чертоги. Оно показало мне, что все земные занятия, наслаждения, почести, преимущества "--- пустые игрушки, которыми играют и в которые проигрывают блаженство вечности взрослые дети»\footnote{Аскетические опыты. Том 1. Статья «Плач мой».}.

\section{Глава III}

Духовные стремления юного подвижника, его ревность, усердие к молитве, выдерживали тяжкое испытание. Первыми врагами на пути спасения явились его домашние. Александр Семенович приставил для служения к своему сыну человека, который был предан ему до самозабвения, это был старик лет 60"~ти по имени Доримедонт, послуживший век свой верой и правдой своему господину. Он был, так сказать, надзирателем всех поступков Димитрия Александровича, и сообщал их Александру Семеновичу. Тяжелы были эти известия родителю. Он вспомнил тогда о выраженном на пути в Петербург желании сына и убедился теперь, что то не был детский каприз. Он тогда же написал обо всем начальнику училища графу Сиверсу, своему бывшему товарищу по службе в пажах, и просил его наблюсти за воспитанником Брянчаниновым; написал также родственнице своей Сухаревой, прося ее отвлечь его сына от предпринятого им намерения. Училищное начальство приняло свои меры, переведя Брянчанинова с частной квартиры в казенную, в стены Михайловского инженерного замка, под строгий надзор, а Сухарева, "--- особа влиятельная, озаботилась довести до сведения тогдашнего митрополита Петербургского Серафима, что ее племянник Брянчанинов, любимый Государем Императором, свел знакомство с лаврскими иноками, что лаврский духовник Афанасий склоняет его к монашеству, и что если об этом будет узнано при Дворе, то и ему "--- митрополиту не избежать неприятностей. Митрополит призвал к себе духовника Афанасия и сделал ему строгий выговор, воспретив впредь принимать на исповедь Брянчанинова и Чихачова. Тяжелы были для Димитрия Александровича эти обстоятельства, которыми стеснялась свобода его духовной деятельности; он решился сам представиться митрополиту и лично объясниться. Митрополит сначала не верил бескорыстному стремлению юноши, когда тот в разговоре объявил ему свое непременное желание вступить в монашество; но потом, выслушав внимательно искренние заявления молодого человека, митрополит позволил ему по"=прежнему ходить в Лавру к духовнику.

Таково было стремление Брянчанинова к жизни иноческой; это было не прихотливое желание представлять из себя оригинала в обществе, не было следствием простого разочарования жизнью, которой горечи и удовольствий он еще не успел испытать: это было чистое намерение, чуждое всяких расчетов житейских, искреннее, святое чувство любви божественной, которая одна способна с такою силою овладевать существом души, что никакие препятствия не в состоянии преодолеть ее.

Практика монастырской жизни определительно указывает, что чистосердечно избирающие ее готовы на всякие пожертвования и на совершенное самоотвержение. Вот какие чувства изливаются в «Плаче», где автор «Аскетических опытов» говорит:

«Охладело сердце к миру, к его служениям, к его великому, к его сладостному! Я решился оставить мир, жизнь земную посвятить для познания Христа, для усвоения Христу. С этим намерением я начал рассматривать монастырское и мирское духовенство. И здесь встретил меня труд; его увеличивали для меня юность моя и неопытность. Но я видел все близко, и, по вступлении в монастырь, не нашел ничего нового, неожиданного. Сколько было препятствий для этого вступления! Оставляю упоминать о всех; самое тело вопияло мне: «Куда ведешь меня? Я так слабо и болезненно. Ты видел монастыри, ты коротко познакомился с ними; жизнь в них для тебя невыносима и по моей немощи, и по воспитанию твоему, и по всем прочим причинам». Разум подтверждал доводы плоти. Но был голос, голос в сердце, думаю, голос совести, или может быть Ангела Хранителя, сказывавшего мне волю Божию, потому что голос был решителен и повелительный. Он говорил мне: это сделать твой долг, долг непременный. Так силен был голос, что представления разума, жалостные, основательные, по"=видимому, убеждения плоти, казались пред ним ничтожными»\footnote{Аскетические опыты. Том 1. Статья «Плач мой».}.

Кроме случаев и обстоятельств, зависящих от воли людей, самая природа ставила препятствия благочестивым намерениям юного Димитрия. Весною 1826 года он заболел тяжкою грудною болезнью, имевшею все признаки чахотки, так что не в силах был выходить. Государь Император Николай Павлович приказал собственным медикам пользовать больного и еженедельно доносить ему о ходе болезни. Доктора объявили Димитрию Александровичу об опасности его положения, сам он считал себя на пороге жизни и частыми молитвами готовился к переходу в вечность. Но случилось не так, как предсказывали знаменитые врачи столицы; болезнь получила благоприятный переворот и послужила для больного опытным доказательством того, что без воли Божией самые настоятельные законы естества не сильны воздействовать на нас.

Все благочестивые упражнения Димитрия Александровича служили подготовкою для того решительного переворота, который он должен был совершить, чтобы осуществить свои давнишние намерения и желания. Но чтобы произвести этот переворот, т. е. чтобы совсем порвать все связи с миром, нужен был человек, который бы содействовал этому разрыву, который бы силою своего духа увлек за собою, "--- нужен был свой Моисей, чтоб вывести нового израильтянина из Египта мирской жизни. Таким Моисеем явился для Димитрия Александровича вышеупомянутый иеромонах Леонид\footnote{Имя в схиме "--- Лев.}. Отец Леонид отличался духовною мудростью, святостью жизни, опытностью в монашеском подвиге; под его руководством образовались многие истинные подвижники благочестия и наставники иночества. Об этом старце много наслышан был Димитрий Александрович от лаврских иноков. Наконец представился случай познакомиться с ним. Отец Леонид прибыл по делам своим в Петербург и остановился в Невской Лавре. Там в одинокой беседе с этим представителем тогдашнего монашеского подвижничества, Димитрий Александрович почувствовал такое влечение к этому старцу, что как бы век жил с ним: это были великие минуты, в которые старец породил его духовно себе в сына… О впечатлении этой первой беседы Димитрий Александрович высказался после своему другу Чихачову так: «Сердце вырвал у меня отец Леонид, "--- теперь решено: прошусь в отставку от службы и последую старцу; ему предамся всею душою и буду искать единственно спасения души в уединении». После этой первой встречи Димитрий Александрович уже не принадлежал более миру, решительный переворот был произведен, требовалось только некоторое время, чтобы окончательно распутать мирские узы.

Вознамерившись совсем оставить службу и удалиться в монастырь, Димитрий Александрович сперва должен был выдержать великую нравственную борьбу с одной стороны с родителями своими, с другой с сильными мира сего. Эта борьба стоила ему больших усилий. Как физические силы его подрывались постоянно болезнями, так теперь он должен был уготовиться нравственно, чтоб принять напор со стороны власти родительской и государственной, которые устремлялись подавить, сокрушить то, что для него было всего дороже и вожделеннее. Сугубую выдерживал он борьбу в молодых летах своих "--- физическую и нравственную; но как в первой он всегда торжествовал силою духа своего над слабостью плоти, так и во второй явился искусным и надежным борцом со стихиями земной жизни, обещавшей ему много сладостного, великого и славного. В этой последней борьбе окончательно выработался его твердый характер, необходимый для прохождения многотрудной иноческой жизни, требующей самоотвержения, особенной непоколебимости воли, неустрашимости, постоянства и готовности на всякую крайность. Вот та дверь, чрез которую приходилось вступить юному подвижнику на тесный и прискорбный путь иночества.

В июне 1826 года Димитрий Александрович получил трехмесячный отпуск от службы и для поправления здоровья отправился на родину, в дом своих родителей. Зная честолюбивое намерение своего отца и не желая, притом, огорчить родителей решительным объявлением им своей воли, Димитрий Александрович старался исподволь и осторожно приготовить их к предполагаемой перемене жизни, но и это не помогло; "--- Александр Семенович не мог примириться с мыслью о монашестве своего первенца. Он сердился на него, отказывал наотрез, отстранял его от себя, как сына непокорного. Все должен был выносить кроткий и чувствительный юноша, послушный заповеди Спасителя: «\textit{Иже любит отца или матерь паче Мене, несть Мене достоин}»\footnote{Мф. 10: 37.}. С глубокою скорбью, не получив желаемого согласия, он уехал из дома родительского в столицу. Здесь ему предстояла необходимость сначала сдать окончательный экзамен в Инженерном училище, что он исполнил в конце декабря, и хотя без конкуренции с товарищами по выпуску, сдавшими экзамен гораздо ранее, но по числу баллов он и тут сохранил свое первенство; затем освободившись от зависимости училищной, он подал в отставку от службы. Тут встретила его новая буря: он должен был иметь дело с высшею властью, должен был отстоять свое заветное желание даже пред Монархом, которому всецело был обязан воспитанием, образованием и благодарностью за милостивое высокое к нему внимание. Трудно ему было убеждать мирских людей в правдивости своих духовных стремлений, понятных только некоторой горсти чернецов в Невской Лавре; тут нужна была решимость отважная; надо было противостоять лишь самоотвержением и силою воли; а не доводами и очевидными указаниями. Ясно, что спор был неравный: надлежало или поддаться, уступить, или показать пример непоколебимого мужества, доблести мученической, прямого исповедничества.

Государь Император Николай Павлович, узнав о поданной Брянчаниновым просьбе и о желании его идти в монастырь, поручил своему Августейшему брату Великому Князю Михаилу Павловичу отговорить всеми любимого воспитанника от такого предприятия. В первых числах января 1827 года Димитрий Александрович был потребован во дворец к Великому Князю. Там было собрано все высшее начальство инженерного училища. 19"~ти летний юноша с трепетным сердцем, но твердою волею предстал пред собранием. Великий Князь сообщил ему, что Государь Император, зная его способности к службе, вместо отставки, намерен перевести его в гвардию и дать такое положение, которое удовлетворит и его Брянчанинова самолюбию и его честолюбию. Молодой человек сказал на это, что, не имея достаточных денежных средств, он не может служить в гвардии. "--- «Заботы об этом Государь изволит принять на себя», "--- прервал Великий Князь. "--- Расстроенное мое здоровье, "--- продолжал юноша, "--- о чем известно Его Величеству из донесений лечивших меня медиков, поставляет меня в совершенную невозможность нести труды служебные, и предвидя скорую смерть, я должен позаботиться о приготовлении себя к вечности, для чего и избираю монашеское звание. Великий князь заметил, что он может получить службу в южном климате России, и что гораздо почетнее спасать душу свою, оставаясь в мире. Брянчанинов отвечал: «остаться в мире и желать спастись, "--- это, Ваше Высочество все равно, что стоять в огне и желать не сгореть». Не смотря на убеждения Великого Князя, прибегавшего и к ласке и к угрозе, Брянчанинов оставался тверд в своем намерении и просил оказать ему милость "--- уволить от службы. Тогда Великий Князь решительно возразил ему, что так как он остается непреклонен в своем упорстве, то объявляется ему высочайшая воля: Государь Император отказывает ему в увольнении от службы и делает ему лишь ту милость, что предоставляет самому избрать крепость, в которую он должен быть послан на службу. Брянчанинов отклонил от себя добровольное избрание. Великий Князь обратился к графу Оперману, своему помощнику по званию генерал"=инспектора инженеров; тот указал на Динабург. Великий Князь одобрил указание, и в тот же вечер состоялось назначение Брянчанинова в Динабургскую инженерную команду, с приказанием в 24 часа выехать из С."=Петербурга к месту нового служения.

Начальник Динабургской команды был в то время генерал"=майор Клименко; ему сообщено было о настроении Брянчанинова и предписано иметь строгий надзор за его поведением. Товарищи по службе сперва не совсем доверчиво относились к Димитрию Александровичу, но потом переменили свое мнение, увидев истинное благочестие, кротость и благоразумие его. Они даже сделались преданными ему, разделяя его труды по службе, вследствие болезненного его состояния. Служебные занятия офицера Брянчанинова состояли в наблюдении за производством разных построек и земляных работ в крепости; он же до того был слаб здоровьем, что принужден был по нескольку недель сряду держаться безвыходно в квартире, а потому необходимо нуждался в помощи товарищей по исполнению служебных обязанностей. Одна только переписка с отцом Леонидом поддерживала Димитрия Александровича в этом одиночестве духовном, так как и с любимым другом своим Чихачовым он был разлучен. Осенью 1827 года Великий Князь Михаил Павлович посетил Динабургскую крепость и, убедившись в физической несостоятельности офицера Брянчанинова к отправлению службы, склонился на его непременное желание получить отставку.

\section{Глава IV}

6"~го ноября 1827 года Димитрий Александрович получил вожделенную отставку. Он был уволен с чином поручика и немедленно чрез Петербург отправился в Александро"=Свирский монастырь к отцу Леониду, чтоб под его руководством начать подвиг иночества. Прибыв в Петербург в одежде простолюдина, в нагольном тулупе, он остановился в квартире Чихачова. Здесь условлено было обоим поступить в монастырь и по возможности немедленно. Чихачов тотчас написал прошение, выставляя причиною домашние обстоятельства, но не получил удовлетворения и должен был еще повременить на службе.

Выход из службы Димитрия Александровича совершился без ведома родителей, а потому естественно навлек на себя гнев их. Они отказали сыну в вещественном вспомоществовании и даже прекратили с ним письменные сношения. Таким образом полная нищета материальная сопровождала вступление Димитрия Александровича в монастырь; он буквально выполнил заповедь нестяжания при самом начале иночества, и вполне справедливо мог сказать с Апостолом, как истинный ученик Христов: «\textit{Се мы оставихом вся и в след Тебе идохом}»\footnote{Мф. 19: 27.}. В «Плаче» своем он так выразил свои чувствования, с которыми вступал на этот новый путь жизни: «Вступил я в монастырь, как кидается изумленный, закрыв глаза и отложив размышление, в огонь, или пучину, "--- как кидается воин, увлекаемый сердцем, в сечу кровавую, на явную смерть. Звезда руководительница моя, мысль благая, пришла светить мне в уединении, в тишине, или правильнее, во мраке, в бурях монастырских»\footnote{Аскетические опыты. Том 1. Статья «Плач мой».}.

Беспрекословное послушание и глубокое смирение отличали поведение послушника Брянчанинова в монастыре. Первое послушание было назначено ему служить при поварне. Поваром был бывший крепостной человек Александра Семеновича Брянчанинова. В самый день вступления в поварню случилось, что нужно было идти в амбар за мукой. Повар сказал ему: «ну"=ка, брат, пойдем за мукой!» и бросил ему мучной мешок, так что его всего обдало белою пылью. Новый послушник взял мешок и пошел. В амбаре растянувши мешок обеими руками и по приказанию повара прихватив зубами, чтоб удобнее было всыпать муку, он ощутил в сердце новое, странное духовное движение, какого еще не испытывал никогда: собственное смиренное поведение, полное забвение своего «я», так усладили его тогда, что он во всю жизнь поминал этот случай. В числе прочих послушников, он назначен был тянуть рыболовный невод в озере Свирского монастыря. Раз как"=то невод запутался в глубине. Монах из простолюдинов, заведывающий ловлею, зная, что Брянчанинов хорошо умел плавать и долго мог держаться под водою, послал его распутать невод. Несмотря на сильный осенний холод, Димитрий Александрович беспрекословно исполнил приказание, которое отозвалось крайне зловредно на его слабом здоровье "--- он сильно простудился. Подобные случаи послушания и смирения сделали то, что вся монастырская братия стала с явным уважением относиться к Брянчанинову, отдавая ему предпочтение пред прочими, чем он очень тяготился, потому что, живя в среде монастырского братства, он даже старался скрывать свое происхождение и образование, радуясь, когда незнавшие считали его за недоучившегося семинариста.

Поступив в монастырь, Димитрий Александрович всею душою предался старцу отцу Леониду в духовное руководство. Эти отношения отличались искренностью, прямотою, представляли совершенное подобие древнего послушничества, которое не решалось сделать шагу без ведома, или позволения наставника. Всякое движение внутренней жизни таких послушников происходит под непосредственным наблюдением старца; ежедневная исповедь помыслов дает им возможность тщательно наблюдать над собою, она предохраняет новоначального инока от вредного действия этих помыслов, которые будучи исповеданы, подобно скошенной траве, не могут уже возникать с прежнею силою. Опытный взор старца"=духовника обнаруживает самые сокровенные тайники души, указывает гнездящиеся там страсти и таким образом удивительно способствует самонаблюдению. Чистосердечная исповедь, всегдашняя преданность старцу и всецелое пред ним отсечение воли вознаграждаются духовным утешением, легкостью и мирным состоянием духа, какие свойственны бесстрастию.

Такой род начального подвижничества и в древнее время, когда духовными старцами обиловали пустыни и монастыри, был уделом немногих послушников, тем реже он встречается ныне, при заметном оскудении духовного старчества. Димитрий Александрович, как сказано, во всем повиновался воле своего духовного отца, все вопросы и недоумения разрешались непосредственно им. Старец не ленился делать замечания своему юному питомцу, вел его путем внешнего и внутреннего смирения, обучая деятельной жизни.

«Однажды, "--- рассказывает И. А. Барков, человек весьма благочестивый и достойный всякого вероятия, "--- ко мне приехал из Свирского монастыря отец Леонид зимой: был жестокий мороз и вьюга, старец приехал в кибитке. Когда вошел он ко мне, я захлопотал о самоварчике и подумал: не один же старец приехал, вероятно, есть какой"=нибудь возница "--- и я стал просить старца, чтобы он позволил ему также войти. Старец согласился. Я позвал незнакомца, и не мало был удивлен, когда предстал предо мной молодой, красивой наружности человек, со всеми признаками благородного происхождения. Он смиренно остановился у порога. «А, что перезяб, дворянчик», "--- обратился к нему старец, и затем сказал мне: знаешь ли кто это? "--- Это Брянчанинов». «Тогда я низко поклонился вознице».

Такой крайне"=смиряющий образ руководства был предпринят отцом Леонидом в отношении ученика своего, молодого офицера Брянчанинова, без сомнения, для того, чтобы победить в нем всякое высокоумие и самомнение, которые обыкновенно присущи каждому благородному и образованному человеку, вступающему в среду простецов. Старец поступал, как нелицемерный наставник, в духе истинного монашества, по примерам святых Отцов; он постоянно подвергал своего ученика испытаниям, и такие опыты смирения не могли не нравиться благородному послушнику, с искреннею любовью к Богу предавшемуся иноческим подвигам.

Спустя год, представилась надобность отцу Леониду со всеми учениками переселиться из Свирского монастыря, по причине многолюдства этой обители, в другое место. Он направился в Площанскую пустынь, Орловской епархии; Димитрий Александрович, в числе прочих учеников, следовал за старцем. В это время прибыл в Площанскую пустынь и Чихачов. Друзья обменялись сердечными приветствиями, порадовались, что опять соединились в тихом приюте монастырского уединения, и стали жить по"=прежнему, совокупно, связывая себя союзом святейшей дружбы. На такую жизнь вдвоем, отдельно от других учеников, благословил их и старец Леонид.

\section{Глава V}

Молодые послушники предались вполне подвижнической жизни: они держались уединения, избегали многолюдства, хранили себя всячески от вредных для безмолвия впечатлений окружающей среды, избегали ненужных встреч и лишних знакомств, чтобы держать себя в строгом молчании и блюдении ума. Все силы души были направлены у них к богомыслию и молитве. Отдельное помещение в монастырском саду, вне всяких сообщений, доставляло им желанный покой: молодые подвижники радовались своему отшельничеству. Так провели они зиму 1829 года. Димитрий Александрович, от природы наделенный способностью литературного творчества, любил созерцать картины природы и из них извлекать содержание для своего богомыслия, которое и изображал искусным пером. Здесь он написал свой «Сад во время зимы». К тому же роду литературного творчества принадлежит и другое его произведение: «Древо зимою пред окнами келлии», написанное немного прежде, в Свирском монастыре. В этих двух произведениях высказались взгляды и чувствования богомысленной души, предавшейся религиозной созерцательности, под влиянием молитвенных состояний, что испытывается лишь безмолвниками. Но недолго пришлось молодым отшельникам пользоваться мирным приютом в Площанской пустыне: им готовилось тяжкое испытание. Между строителем пустыни иеромонахом Маркеллом и старцем отцом Леонидом возникли неудовольствия, вынудившие последнего оставить Площанскую пустынь и переселиться в скит Оптиной Введенской пустыни, находящейся в Калужской губернии. Брянчанинов и Чихачов также получили приказание немедленно выбыть из обители и отправиться «\textit{куда угодно}». Поскорбели монастырские братия на безвинное изгнание никому, ничем не досадивших, благонравных молодых послушников, и проводили с чувствами глубокого сожаления и уважения за их тихую и строгую жизнь, дав им на дорогу пять рублей, собранных складчиной. Трудно было с тощим кошельком странствовать двоим товарищам по неизвестной стороне, не имея в виду определенного места; они старались, как можно сократить свое путешествие и направлялись к Белобережской пустыне в той же Орловской губернии. На пути они были в Свенском монастыре, где в то время подвизался в затворе иеромонах Афанасий, один из учеников вышеупомянутого Молдавского старца Паисия Величковского. Димитрий Александрович посетил затворника и много пользовался его душеназидательной беседой о благотворности плача, о чем вспоминает в своих «Аскетических опытах», приводя слова затворника, глубоко запавшие в его душу: «В тот день, в который я не плачу о себе как о погибшем, считаю себя в самообольщении». Белобережская пустыня не приютила однако ж на жительство бедных странников, и они, продолжая путь далее, прибыли в Оптину пустынь, где поселился их старец отец Леонид с учениками. Настоятель Моисей не соглашался было принять их к себе, но старшая братия сжалилась над бедственным положением скитальцев, и уговорила игумена не отгонять их. В мае 1829 года Брянчанинов и Чихачов поселились в Оптиной пустыне, держась того же порядка жизни, какой был у них заведен в Площанской обители.

Пребывание Димитрия Александровича и его товарища в Оптиной пустыне далеко было не таково, как в Площанской. Настоятель смотрел на них неблагосклонно, братия относились не совсем доверчиво. Приходилось много скорбеть им при уединенном образе жизни; самая пища монастырская, приправленная постным маслом дурного качества, вредно действовала на слабый и болезненный организм Димитрия Александровича. Они решились сами для себя изготовлять пищу; с немалым трудом выпрашивали круп или картофеля и варили похлебку в своей келлии; ножом служил им топор; готовил пищу Чихачов. Такая трудная и неблаговидная обстановка, конечно, не могла долго продолжаться: изнурительная слабость телесных сил была последствием ее для того и другого. Сперва пострадал от нее Димитрий Александрович, настолько, что не мог держаться на ногах; за ним ухаживал Чихачов, который был крепче его телосложением; но вскоре свалился и он, пораженный лихорадкою. Тогда за больным товарищем ухаживал Димитрий Александрович; он хотя усердно исполнял это служение, но тут же сам падал от конечного изнеможения.

Мать Димитрия Александровича была больна. Болезнь "--- предвестница смерти "--- обыкновенно изменяет расположение человеческого сердца. Софья Афанасьевна простила в душе поступок своего сына; материнское чувство заговорило в ней; она пожелала видеться с сыном. Александр Семенович под влиянием этого обстоятельства сам смягчился и написал сыну, что не будет препятствовать его намерениям, пусть он приедет к матери и одновременно с письмом прислал за ним крытую бричку. Димитрий Александрович поспешил к родителям. Он отправился вместе с больным товарищем своим Чихачовым, так как Александр Семенович был столь внимателен, что не забыл пригласить и того. Но встреча в доме родительском была далеко не такова, какую обещало приглашение. Больная родительница Брянчанинова несколько поправилась здоровьем и мирное чувство, внезапно явившееся в отце по поводу угрожавшего обстоятельства "--- болезни жены, исчезло. Он принял сына холодно. Мать, хотя и была приветлива, но обошлась со сдержанностью. Таким образом скитальничество из одного монастыря в другой, тяжкое положение в последнем, болезнь матери и вследствие ее мгновенная вспышка родительских чувств, "--- все это послужило только к тому, чтобы извлечь молодых людей из приюта святой обители и поставить на прежнюю дорогу лицом к лицу с мирским соблазном. Врагу человеческого спасения нет выгоднейшей сети, как оставление молодыми послушниками стен монастыря, под какими бы то ни было благовидными предлогами. Самовольный выход из монастыря "--- всегда его затея.

Молодые люди расположились под мирским кровом в отдельном уединенном флигеле дома с намерением продолжать свои иноческие подвиги, обращаясь за духовными потребностями к местному сельскому священнику, считая свое пребывание здесь только временным. Но не так думал Александр Семенович. Он обратился к прежней своей мысли возвратить сына к мирской жизни, и всеми мерами стал склонять его к поступлению на государственную службу; взоры родных и знакомых обращались к нему с тою же мыслью; мать, хотя внимала иногда учению сына о душеспасении и других высоких истинах христианской жизни, но не имела столько самостоятельности, чтоб отдаться вполне его внушениям. Вращающиеся пред глазами соблазны смущали подвижников; шумная толпа нарушала их безмолвие. Молодые люди стали тяготиться своим пребыванием среди мирян и помышляли о том, как бы им скорее выбраться из светского общества несовместного с монашеством, и водвориться опять где"=нибудь в монастыре. Проживя начало зимы 1829 года в селе Покровском, в следующем 1830 году, в феврале месяце отправились оба друга искать себе удобного приюта в стенах монастыря; они направили путь свой в Кирилло"=Новоезерский монастырь. В это время там жительствовал на покое архимандрит Феофан, знаменитый своею святою жизнью и примерным управлением обителью, а настоятельствовал игумен Аркадий, его присный ученик и подражатель его образа правления. Отец Аркадий отличался простотою нрава; он провидел в двух молодых пришельцах дух истинного монашества, и с любовью принял их в свою обитель. Но недолго радовались друзья новому месту жительства: неумолимая природа доказала им, что человек состоит не только из души, но и тела. Новоезерский монастырь расположен на острове обширного озера. Сырой климат от испарения воды наделяет жестокою лихорадкою непривычные и слабые организмы. Вскоре почувствовал его вредное влияние Димитрий Александрович; он заболел лихорадкой и три месяца испытывал ее мучительные симптомы без всяких медицинских пособий. Под конец у него стали пухнуть ноги, так что он не мог вставать уже с постели. В июне, когда лихорадка особенно там свирепствует, родители прислали за сыном экипаж, чтобы привезти его в г. Вологду. Тяжело было это время для Димитрия Александровича: он вынужден был возвратиться опять туда, откуда хотел спастись бегством. В Вологде Димитрий Александрович поместился у своих родных и стал пользоваться медицинскими средствами от мучившей его лихорадки, которая так глубоко проникла в его организм, что оставила свои следы на весь остальной век. Чихачов, также пострадавший от климата Новоезерской обители, отправился в Псковскую губернию для свидания с своими родителями 13 августа того же 1830 года. Друзья расстались, чтоб каждому отдельно испытать свои силы в борьбе с стихиями мирской жизни.

\section{Глава VI}

Рука Промысла, доселе невидимо покрывавшая бесприютного скитальца, коснулась сердца преосвященного Стефана, епископа Вологодского: архипастырь проник душевные стремления молодого Брянчанинова и расположился к нему. Преосвященный Стефан так полюбил Димитрия Александровича, что принял в нем самое живое участие, и эта любовь владыки была видимым знаком благоволения Божия к жертве сердца, которую приносил новый Авель: она возвещала благоприятный исход всех понесенных на пути к иночеству испытаний, потому что архипастырь держал в руке своей тот лавр, которым надлежало повить голову юного борца, измученного в брани с миром, плотью и диаволом. Оправившись от болезни, Димитрий Александрович не хотел возвратиться к родителям, а по благословению владыки поместился в Семигородной пустыне. Местность этой обители благоприятствовала восстановлению его здоровья; он с новою ревностью предался своим обычным духовным занятиям: богомыслию и молитве в тишине келейного уединения. Здесь написал он свой «Плач инока», в котором выразилось печалующее состояние души, усиленно стремящейся к Богу, но разбитой треволнениями жизни, вследствие чего уделом ее стал только плач на развалинах ее стремлений. Недолго пожил Димитрий Александрович и в Семигородной пустыне; вскоре, 20 февраля 1831 года, он был перемещен по его просьбе преосвященным в более уединенный, пустынный Глушицкий Дионисиев монастырь, где и зачислен послушником. К этому времени относится первое знакомство преосвященного Игнатия с бывшим настоятелем Николо"=Угрешского монастыря архимандритом Пименом. Отец Пимен, тогда еще молодой купеческий сын, так описывает наружность послушника Брянчанинова: «В первый раз довелось мне увидеть Брянчанинова на набережной реки Золотухи (в Вологде): я был на левом берегу, а он шел по правому. Как сейчас вижу его: высокого роста, стройный и статный, русый, кудрявый, с прекрасными темно"=карими глазами; на нем был овчинный тулуп, крытый нанкою горохового цвета, на голове послушническая шапочка». Далее повествователь восхищается его благородной осанкой, скромной поступью, в высшей степени благоговейным предстоянием в церкви за богослужением и, наконец, самою беседою, которую описывает следующими словами: «не взирая еще на молодые лета, видно было, что Брянчанинов много читал отеческих книг, знал весьма твердо Иоанна Лествичника, Ефрема Сирина, Добротолюбие и писания других подвижников, и потому беседа его назидательная и увлекательная, была в высшей степени усладительна»\footnote{Воспоминания архимандрита Пимена.}.

Между тем родитель Димитрия Александровича, и во время пребывания его в Глушицком монастыре, не переставал выражать желание исполнения своих требований: он настойчиво добивался того, чтоб сын оставил монастырскую жизнь и поступил в государственную службу. Тогда новоначальный послушник стал просить архиерея оказать ему милость, и в виду семейных обстоятельств, поспешить постричь его в монашество. Преосвященный, зная хорошо духовное настроение Брянчанинова, решился исполнить его просьбу. Исходатайствовав разрешение Св. Синода, он вызвал Димитрия Александровича из Глушицкого монастыря в Вологду, и велел готовиться к пострижению; вместе с тем он приказал ему хранить это в тайне от родных и знакомых, чтоб избежать каких"=либо притязаний со стороны их, могущих воспрепятствовать делу, так как намеревался постричь его неожиданно для всех. Стеснительно было такое положение в столь важное время: готовящийся к пострижению вынужден был остановиться на постоялом дворе и среди мирской молвы приготовляться к великому обряду.

28 июня 1831 года преосвященный Стефан совершил обряд пострижения Брянчанинова в малую схиму в кафедральном Воскресенском соборе, и нарек Димитрия "--- Игнатием, в честь священномученика Игнатия Богоносца, память которого празднуется Церковью 20 декабря и 29 января. Инок Игнатий сначала в первый, потом в последний из этих дней праздновал свое тезоименитство. Это имя Игнатия указывает еще на преподобного Игнатия "--- князя, Вологодского чудотворца, мощи которого почивают в Прилуцком монастыре, где покоятся мощи и преподобного Димитрия Прилуцкого "--- ангела новопостриженного инока от крещения. Таким образом произведена над ним перемена имен двух чудотворцев, почивающих в одной обители. С именем одного, данным при крещении, соединено воспоминание об обстоятельствах рождения, а имя другого наречено при пострижении, как бы в ознаменование сходства земной участи новопостриженного "--- с преподобным из княжеского рода. Родные Брянчанинова, прибывшие 28 июня в собор к богослужению, были крайне изумлены неожиданным священнодействием, которого они сделались зрителями. 4"~го июля того же года инок Игнатий был рукоположен преосвященным Стефаном в иеродиакона, а 25"~го того же месяца "--- в иеромонаха, и временно оставлен при архиерейском доме, который в Вологде находится при кафедральном соборе, в одной с ним ограде, образуемой стенами Кремля, времен царя Иоанна Грозного. Для обучения священнослужению новорукоположенный был приставлен к городской церкви Спаса обыденного под руководство священника Василия Нордова, впоследствии протоиерея и настоятеля Вологодского кафедрального собора.

Родители новопостриженного, разумеется, с неудовольствием отнеслись к этому событию, особенно Александр Семенович был поражен им; его воля, на которой он так упорно настаивал, "--- не состоялась: все планы относительно светской карьеры сына рушились, мечты о его блестящей будущности исчезли. Сын, в глазах отца, сделался бесполезным членом общества, утратившим все, что отец доставил ему воспитанием. Женское сердце, менее упорное в противодействиях обстоятельствам и всегда податливее на взаимности, расположило Софью Афанасьевну благосклоннее смотреть на поступок сына; но духовная сторона была также чужда ей, и мирские понятия брали верх. Все это, конечно, ничего не значило для монаха, который сам добровольно поставляет себя в положение, заставляющее забыть все мирские связи и родственные чувства; но обстоятельства инока Игнатия были не таковы, чтоб это неудовольствие родителей было для него нечувствительно. По пострижении он должен был приютиться в загородном доме своего дяди и крестного отца Димитрия Ивановича Самарина и вынужден был принять денежное вспомоществование от одной из своих родственниц (г"=жи Воейковой). Пребывание в Вологде заставляло его часто обращаться в кругу родных и знакомых: многие из них стали его посещать и требовали от него взаимных посещений себе. Молодой годами, красивый наружностью, он интересовал все вологодское общество, все о нем говорили, все желали сблизиться с ним. Это необходимо вовлекало его в мирскую рассеянность и прямо противоречило тем обетам, какие он только что произнес у алтаря. Вся внешняя обстановка пустыннолюбивого инока была противна его влечениям, он соскучился городскою молвою и стал просить покровителя своего преосвященного Стефана отпустить его в Глушицкий монастырь; но преосвященный, намереваясь дать ему место, соответственное его способностям и благочестивому направлению, а также приличное по отношению его к обществу, удерживал его при себе. В скором времени открылось такое место: в конце 1831 года скончался строитель Пельшемского Лопотова монастыря иеромонах Иосиф. Обряд погребения поручено было совершить иеромонаху Игнатию. 6"~го января 1832"~го года он был назначен на место умершего, а 14"~го дано звание строителя, причем возложен был на него набедренник.

\section{Глава VII}

Лопотов монастырь, основанный преподобным Григорием Пельшемским, Вологодским чудотворцем, находится в Кадниковском уезде, Вологодской губернии, в 40 верстах от Вологды и в 7"~ми от Кадникова, расположен на берегу реки Пельшмы, впадающей в Сухону, в местности лесной и болотистой. Монастырь был почти в разрушенном состоянии, так что предположено было его упразднить: церковь и прочие здания крайне обветшали, доходы были скудные, чувствовался недостаток в самом необходимом к продовольствию, а потому и братии было очень мало. Много надо было употребить трудов и забот, чтоб все исправить, обновить, пополнить скудость во всех отношениях. Новый настоятель не унывал; он принялся за дело с энергией. Вскоре потекли пожертвования от благочестивых жителей Вологды, чествовавших память преподобного Григория; монашествующие из тех монастырей, где проживал послушником строитель Игнатий, стали собираться в его обитель и в короткое время составили в ней братство до 30 человек. Богослужение приведено в надлежащий порядок: обитель и внешне и внутренне обновилась, сделалась неузнаваема против того положения, в каком принял ее строитель Игнатий. Но чего стоило это ему самому?… По рассказам одного очевидца, посетившего Лопотов монастырь в зиму 1832 года, строитель Игнатий помещался в сторожке у Святых ворот, когда производилась постройка новой настоятельской келлии.

Смягчилось сердце Александра Семеновича, когда он увидел молодого сына своего в таком сане, какой приличен старческому возрасту, следовательно многое обещавшего впереди. Там, где не могла подействовать внутренняя, духовная сторона, взяла внешняя, и она вполне оказала благотворное влияние свое на Софии Афанасьевне. Строитель сын часто стал бывать в доме родителей: его могучему слову об истинах загробной жизни покорилось сердце матери, часто болевшей и чувствовавшей себя близкою к смерти. Мать напиталась духовными беседами сына; понятия ее изменились; из плотских сделались духовными: она благодарила Бога, что сподобил ее иметь первенца своего в числе Его служителей, тогда как прежде почитала это для себя великим несчастьем. Такая перемена с родительницей на пороге ее жизни несказанно радовала священно"=инока сына. Напутствованная его назиданиями и молитвами, Софья Афанасьевна мирно скончалась 25 июля 1832 года. Строитель Игнатий сам совершил обряд отпевания в храме села Покровского. Замечательно, что при этом богослужении, сын не выронил ни одной слезы над бездыханным телом матери! И это происходило не от сдержанности, приличествующей предстоятелю священнослужения, или от холодности родственного чувства, а составляло особую черту от духовного характера. Чувство в нем было живо, сыновняя любовь к матери в своей естественной мере, но в нем душевный человек был заменен духовным; чувство плотского родства было вполне проникнуто духовною любовью, которая побуждала не о временной потере жалеть, а желать единственно блаженной участи усопшей "--- в вечности. Потому такие родственные чувства в иноке Игнатие никогда не обнаруживались своим обычным образом; они отражались в нем глубокою думою и молитвенным, безмолвным благоговением, при полном внешнем спокойствии.

В Лопотовом монастыре строитель Игнатий имел утешение встретиться и опять соединиться по жительству с любимым своим другом Чихачовым. Чихачов сделался деятельным помощником строителя Игнатия по устройству обители; он обладал отличным голосом, знал хорошо церковное пение и составил очень хороший певческий хор, который не мало содействовал к привлечению в обитель многих богомольцев. Настоятель Игнатий облек его в рясофор и руководил в духовной жизни.

Вступив на новое поприще начальника иноческого общежития, отец Игнатий был в полном смысле слова Аввою общества иноков. Следующий отрывок из его аскетических сочинений изображает нам, каким духом он водился в деле назидания иноков: «Скажу здесь о монастырях российских мое убогое слово, слово "--- плод многолетнего наблюдения. Может быть, начертанное на бумаге, оно пригодится для кого"=нибудь! "--- Ослабела жизнь иноческая, как и вообще христианская, ослабела иноческая жизнь потому, что она находится в неразрывной связи с христианским миром, который, отделяя в иночество слабых христиан, не может требовать от монастырей сильных иноков, подобных древним, когда и христианство, жительствовавшее посреди мира, преизобиловало добродетелями и духовною силою. Но еще монастыри, как учреждение Святого Духа, испускают лучи света на христианство; еще есть там пища для благочестивых; еще есть там хранение евангельских заповедей, еще там строгое и догматическое и нравственное православие; там хотя редко, крайне редко, обретаются живые скрижали Святого Духа. Замечательно, что все духовные цветы и плоды возросли в тех душах, которые, в отдалении от знакомства внутри и вне монастыря, возделали себя чтением Писания и святых Отцов, при вере и молитве, одушевленной смиренным, но могущественным покаянием. Где не было этого возделания, там "--- бесплодие».

«В чем состоит упражнение иноков, для которого "--- и самое иночество? Оно состоит в изучении всех заповеданий, всех слов Искупителя, в усвоении их уму и сердцу. Инок соделывается зрителем двух природ человеческих: природы поврежденной, греховной, которую он видит в себе, и природы обновленной, святой, которую он видит в Евангелии. Десятословие Ветхого Завета отсекало грубые грехи, Евангелие исцеляет самую природу, болезнующую грехом, стяжавшую падением свойства греховные. Инок должен при свете Евангелия вступить в борьбу с самим собою, с мыслями своими, с сердечными чувствованиями, с ощущениями и пожеланиями тела, с миром, враждебным Евангелию, с миродержителями, старающимися удержать человека в своей власти и плене. Всесильная истина освобождает его\footnote{Ин. 8: 32.}; освобожденного от рабства греховных страстей запечатлевает, обновляет, вводит в потомство Нового Адама, всеблагий Дух Святой»…\footnote{Аскетические опыты. Том 1. Статья «Плач мой».}.

Преосвященный вологодский Стефан, видя неутомимые и полезные труды строителя Игнатия по возобновлению и благоустройству Лопотовой обители, возвел его в сан игумена 28 мая 1833 года; но болотистая местность Лопотова монастыря уносила последние остатки здоровья, а наконец совсем уложила его на одр болезни. Чихачов томился душой за своего настоятеля, и не видя никакого другого исхода бедственному положению, осмелился предложить ему свою мысль "--- переселиться из Лопотова монастыря куда"=либо в другое место. Мысль эта была одобрена игуменом и решено было ехать Чихачову на свою родину, в Псковскую губернию, хлопотать о перемещении их в один из тамошних монастырей. Напутствованный благословением своего настоятеля, отправился Чихачов в преднамеренный путь. Приехав в Петербург, он обратился к графине Анне Алексеевне Орловой"=Чесменской, с которою незадолго прежде имел случай познакомиться. Это было в первую его поездку из Лопотова монастыря, когда он ездил на свою родину для устройства дел семейных; тогда в первый раз встретил он графиню в Новгородском Юрьеве монастыре, в келлиях настоятеля, знаменитого архимандрита Фотия. Графиня ласково приняла Чихачова и пожертвовала в Лопотов монастырь несколько книг и 800 рублей денег. С тех пор Брянчанинов и Чихачов пользовались милостивым расположением графини Орловой, что продолжалось до самой ее кончины. На этот раз графиня Анна Алексеевна также радушно приняла Чихачова, дала ему помещение в своем доме, снабдила всем нужным, и деятельно стала хлопотать о перемещении игумена Игнатия из Лопотова монастыря.

Чихачов, находясь в столице, в кругу знатного общества, посещавшего графиню, намеревался уже возвратиться обратно в Лопотов монастырь, но графиня его удержала и советовала ему представиться Московскому митрополиту Филарету, который тогда находился в Петербурге. Чихачов явился на Троицкое подворье. Высокопреосвященный милостиво принял Лопотовского монаха, и сказал: «мне не безызвестны жизнь и качества игумена Игнатия», "--- и предложил тому настоятельское место в Николо"=Угрешском третьеклассном монастыре, своей епархии, если пожелает он туда переместиться, обещаясь потом доставить и лучшее. Чихачов поблагодарил милостивого владыку, и осмелился выразить пред ним опасение, что игумену Игнатию неудобно будет самому проситься из вологодской епархии, так как он пострижен лично вологодским архиереем, который может оскорбиться таким поступком своего постриженца. «Хорошо, "--- сказал митрополит, "--- я сделаю предложение об этом в Синоде и надеюсь, что мне не откажут». На другой день был послан из Синода указ в Вологду к преосвященному Стефану о перемещении игумена Лопотова монастыря Игнатия в Николо"=Угрешский монастырь, куда, по сдаче своего монастыря, и предписывалось его немедленно отправить.

Преосвященный Стефан доброжелательно отнесся к этому событию. Напутствовав игумена Игнатия своим благословением на новое служебное место, он сделал следующий отзыв о нем в своем отношении к митрополиту Московскому от 28 ноября 1833 года: «Игумен Игнатий по пострижении в 1831 году, по указу Святейшего Правительствующего Синода, в монашество, состоя в числе братства третьеклассного Глушицкого монастыря, похвальными своими качествами и образованностью своею в науках, всегда обращал на себя особое мое внимание, почему взят был в вологодский архиерейский дом и, по рукоположении во иеродиакона, а потом в иеромонаха, употребляем был для соборного священнослужения, где более и более замечая в нем отличные способности, украшаемые похвальным поведением, в 1832 году января 6, я определил его, Игнатия, на место умершего в Лопотове монастыре строителя иеромонаха Иосифа строителем, и будучи он в сей новой возложенной на него должности, образом примерной своей жизни, учреждением в монастыре порядка, согласно правилам и уставам монастырским, точным наблюдением должного в монастыре благоприличия, обращая на себя от публики особенное внимание, успел возродить в почитателях святой обители усердие, и тем достиг возможности Лопотов монастырь, пришедший уже в совершенный упадок и расстройство привести ныне в короткое время в наилучшее состояние, как то: 1) заведением многоценных серебряных святых сосудов, Евангелия и облачений, и многих других для благолепия церковного служащих вещей, и 2) устроением настоятельских и братских келлий, а потом поправкою многих ветхих монастырских строений, каковая его Игнатия полезная для обители святой служба, а притом и отзывы публики, о похвальных его качествах, убедили меня сего года мая 28 дня, для поощрения его к дальнейшей таковой же службе, произвести в игумена, с оставлением в том же заштатном Лопотове монастыре настоятелем, о каковой его игумена Игнатия отлично"=похвальной службе за нужное почел довести при сем до сведения Вашего Высокопреосвященства».

Чихачов, обрадованный столь успешным исходом своего ходатайства, отправился из Петербурга на родину в Псковскую губернию, чтоб навестить своих родителей. Здесь, вскоре по приезде получает он письмо от графини Орловой"=Чесменской, в котором она извещает его, что все события жизни игумена Игнатия и его самого дошли до сведения Государя Императора Николая Павловича, и что Его Императорское Величество изволил вспомнить бывших своих воспитанников и приказал митрополиту Московскому вызвать игумена Игнатия не в Москву, а в Петербург, для личного представления ему, причем прибавил, что если Игнатий ему так же понравится, как и прежде, то он его митрополиту Филарету не отдаст. Высокопреосвященный Филарет, во исполнение этой Высочайшей воли, официальным письмом от 15 ноября 1833 года на имя вологодского епископа Стефана, просил его, как можно скорее, отправить игумена Игнатия прямо в Петербург, а частным собственноручным письмом своим к игумену Игнатию требовал, чтобы он, нисколько не медля, прибыл к нему в Петербург, на Троицкое подворье. «Это распоряжение должно быть исполнено безотлагательно, "--- писал Московский владыка, "--- потому что это воля не моя».

27"~го ноября игумен Игнатий сдал Лопотов монастырь своему казначею, а 30"~го ноября выехал в С."=Петербург. К этому времени возвратился туда и Чихачов, с нетерпением ожидавший приезда своего игумена. Приехав в столицу, игумен Игнатий немедля представился митрополиту Филарету, который приютил его на своем Троицком подворье, где и поджидал он времени, когда будет назначено ему явиться к Государю.

В назначенный день и час игумен Игнатий представился Государю в Зимнем дворце. Государь обрадовался, увидев своего воспитанника, «а радость, "--- пишет Чихачов, "--- предстать любимому Царю, полнота благодарного чувства за все его монаршии милости, доводили до благоговейного восторга теплую душу инока верноподданного». После некоторых объяснений Государь изволил сказать: «Ты мне нравишься, как и прежде! Ты у меня в долгу за воспитание, которое я тебе дал, и за мою любовь к тебе. Ты не хотел служить мне там, где я предполагал тебя поставить, избрал по своему произволу путь, "--- на нем ты и уплати мне долг твой. Я тебе даю Сергиеву пустынь, хочу, чтоб ты жил в ней, и сделал бы из нее монастырь, который в глазах столицы был бы образцом монастырей». Затем он повел игумена на половину к Государыне Императрице Александре Феодоровне. Войдя к ней, спросил ее: узнает ли она этого монаха? На отрицательный ответ он назвал игумена по фамилии. Государыня очень милостиво отнеслась к своему бывшему пенсионеру и заставила благословить всех детей ее. Государь тут же изволил послать за обер"=прокурором Синода Нечаевым, который доложил Его Величеству, что Сергиева пустынь имеет особое назначение, "--- она отдана викарному епископу при С."=Петербургском митрополите и доходами ее пользуется епископ взамен содержания от духовной администрации. Тогда Государь приказал справиться, как велика сумма дохода, получаемая викарным епископом от монастыря, и в этом размере производить ему выдачу суммы из кабинета, а монастырь сдать в полное управление назначенного им настоятеля. Обер"=прокурор объявил Святейшему Синоду Высочайшую волю, и преосвященному Венедикту, бывшему тогда викарным, дан указ Синода сдать пустынь игумену Игнатию, а самому получать 4000 руб. ассигнациями содержания от кабинета. Тогда же, по распоряжению Синода, игумен Игнатий был возведен в сан архимандрита, что исполнено было в Казанском соборе 1"~го января 1834 года, а 5 числа того же месяца новый настоятель выехал в свою обитель в сопровождении Чихачова и только что принятого в келейники двадцатидвухлетнего юноши Иоанна Малышова, который впоследствии, чрез 23 года, сделался преемником своего старца в настоятельстве обители, с саном архимандрита.

\section{Глава VIII}

Намерение игумена Игнатия переселиться из Лопотова монастыря имело в основании чисто физическую причину. Его надломленному организму нужен был климат, если не южный, то по крайней мере сухой, а не болотистый. Счастливый вниманием Московского владыки, он довольствовался бы Николо"=Угрешским монастырем; но державная воля поставила его на более широкую деятельность.

Местность Сергиевой пустыни, в климатическом отношении, не представляла даже тех удобств, какими обладал Лопотов монастырь. Береговая сторона Финского залива, волны которого разливаются в виду самой обители, никак не могла служить к восстановлению физических сил. В духовно"=нравственном же отношении новое место жительства представляло гораздо более неудобств сравнительно с прежним; оно требовало сугубого духовного подвига, так как более было обстановлено тернием житейской молвы и суеты, которое неминуемо должно было уязвлять духовного человека. Только живая вера в Промысл Божий и добрая совесть в исполнении иноческого обета послушания, какое архимандрит Игнатий оказывал царской воле, могли подкреплять его при вступлении на это новое поприще. Он вступал туда как истинный монах; враг личных интересов, он заботился единственно о благе вверенной ему обители. Как верноподданный и инок, он твердо решился в точности исполнить волю возлюбленного Монарха, сделав вверенную ему пустынь образцовою обителью во всех отношениях.

Сергиева пустынь, основанная в 1734"~м году и расположенная близ самого Петербурга, немного в сторону от нынешней Петергофской железной дороги, находилась, как сказано, под управлением викарных епископов. Такое административное положение далеко не благоприятствовало ее материальному состоянию, а близость столицы делала ее перепутьем для проезжающих столичных жителей, что весьма невыгодно влияло на духовный быт братства обители. Здания монастырские, начиная с церкви преподобного Сергия до последних монастырских служб, были давно запущены. В церкви, когда приступлено было к ее поправке, оказались годными только одни стены; согнивший внутри настоятельский корпус почти не существовал, он стоял запертым и не мог быть отапливаем; помещения новоприбывшему настоятелю вовсе не было и он принужден был остановиться в инвалидном доме, устроенном при монастыре на иждивение графов Зубовых, и состоящем на их содержании. Там ему отведены были две комнаты, в которых он и поместился с 8"~ю человеками братии, собственно монахов; все найденное им до него бывшее братство обители состояло из 13"~ти человек: восьми монашествующих, трех послушников и двух подначальных. Несмотря на такое незначительное число братии, в среде их не было порядка, приличествующего монастырю. Запущенность в материальном, распущенность в нравственном отношениях царили во всей силе. В таком положении застал Сергиеву пустынь новый настоятель. Обитель требовала такого настоятеля. Судьбами Промысла, или молитвами преподобного Сергия, чрез сто лет от основания началось ее восстановление, как вещественное, так и духовное. Представительная личность настоятеля, его аскетическая духовность соответствовали положению обители и тому назначению, с каким он принял ее. Но труды и заботы по внешнему возобновлению и благоустройству и отношения всякого рода к высшим и низшим положили печать свою на болезненного и строго"=подвижного инока. По собственному его признанию, скорби от человеков, постигавшие его доселе, были «умеренные». "--- «Чтоб испытать их, "--- говорит он в своем Плаче, "--- нужно было особенное поприще. Непостижимыми судьбами Промысла я помещен в ту обитель, соседнюю северной столице, которую, когда жил в столице, не хотел даже видеть, считая ее по всему несоответствующею моим целям духовным. В 1833 году я был вызван в Сергиеву пустынь и сделан ее настоятелем. Негостеприимно меня приняла Сергиева пустынь. В первый же год, по прибытии в нее, я поражен был тяжкою болезнью, на другой год другою, на третий третьею; они унесли остатки скудного здоровья моего и сил, сделали меня изможденным, непрестанно страждущим. Здесь поднялись и зашипели зависть, злоречие, клевета, здесь я подвергся тяжким, продолжительным, унизительным наказаниям, без суда, без малейшего исследования, как бессловесное животное, как истукан бесчувственный; здесь я увидел врагов, дышущих непримиримою злобою и жаждою погибели моей»\footnote{Аскетические опыты. Том 1. Статья «Плач мой».}.

Из этого очерка вступления отца архимандрита Игнатия в новую обитель видно, что его настоятельская деятельность с самого начала должна была делиться на две отрасли: по внешнему устройству и внутреннему благочинию. Первым делом настоятеля было возобновление храма преподобного Сергия и капитальное исправление корпуса настоятельских келлий. Вот что он писал в 1834"~м году в прошении своем к тогдашнему С."=Петербургскому митрополиту Серафиму о дозволении произвести в обители необходимые постройки и исправления: «Обозревая монастырские здания, я нашел оные безысключительно в весьма неблагоприятном положении. Такое состояние видели предместники мои в управлении монастырем преосвященные епископы Ревельские и потому приготовили заблаговременно денежную сумму (до 50 тысяч рублей ассигнациями) и значительное количество кирпича, имея непременною целью починку ветхих и постройку новых зданий». Работы были начаты с разрешения С."=Петербургского епархиального начальства, которым дозволено употребить собранные 50 тысяч рублей ассигнациями и заготовленный кирпич. Графиня Орлова также много помогала своими щедрыми даяниями. Церковь и корпус настоятельских келлий, как однофасадные здания были соединены новым, двух"=этажным корпусом, в верхнем этаже которого весьма удобно устроена была обширная братская трапеза, а в нижнем расположены кухня и пекарня и другие хозяйственные помещения. Во время производства этих построек, в том же 1834"~м году, летом, совершенно неожиданно посетил обитель Государь Император. Приехав из Петергофа около 6"~ти часов пополудни, он один вошел в церковь и спросил встреченного монаха: «Дома ли архимандрит?» «Скажи, что прежний товарищ хочет его видеть». "--- Пришел архимандрит в сопровождении неизменного товарища его, отца Михаила Чихачова. Милостиво и ласково обошелся с ними Государь, спрашивал о третьем их товарище Федорове, вместе с ними поступившем в монастырь, и на ответ, что Федоров возвратился в мир и вновь на службе, заметил: «видно ему монастырский хлеб сух показался, а тебе, "--- обратился к Чихачову, значительно пополневшему, "--- пошел в прок». "--- В это же время приехали Государыня Императрица и Государь Наследник. Народу естественно собралось много, и весь этот народ был свидетелем того отеческого внимания, которое всегда составляло отличительную черту в отношениях покойного Императора к почившему владыке Игнатию. Затем Государь изволил подробно осматривать производившиеся постройки и нашел необходимым возобновить соборный храм, для чего велел архимандриту представить смету в порядке служебных инстанций. Представленная смета утверждена 25 декабря 1835 года, и по государственной росписи назначено было выдать из казначейства С."=Петербургской казенной палаты 96808 руб. 19 коп. на поправление Троицкого соборного храма. Возобновление было окончено в 1838 году, а в 1842 году в этом храме были устроены богатые клиросы на сумму, пожертвованную Императрицею Александрою Феодоровною.

Первою заботою настоятеля по внутреннему духовному благосостоянию обители было установление во всем строгого порядка, согласно монастырским уставам: церковное богослужение стало совершаться в стройном чине, с величием и торжественностью, которые дополняли привлекательное хоровое пение, внятное чтение, чинное стояние, поклонения по положению и вообще благообразные движения, благочинное пребывание в трапезе, весьма приличная одежда; а опрятность и чистота во всем придавали всей обстановке вид изящества, соединенного с простотою, которые облагораживали самые нравы иноков. Настоятель вникал в келлейную жизнь каждого, настраивал к спасительному препровождению времени в чтении монашеских книг, к несению посильных трудов по послушаниям, проводил дух истинного монашества в братство, располагая всякого ко вниманию, к принятию совета и назидания в нравственной жизни, к устроению себя по руководству святоотеческих писаний. Он сообщал братству свой образ мыслей и взгляды на монашество, был отцом и наставником всех, принимая к себе на откровение помыслов, для созидания и настроения духовного. Двери келлии отца архимандрита были открыты; к нему входили свободно от престарелых монахов священнослужителей до юных послушников, вследствие чего все братство стало составлять одну великую семью, управляемую одним отцом, связанную союзом согласия и духовного единения, одушевляемую и руководимую высоким учением отца"=наставника. Особенно, говорит Чихачов, помогало деятельности настоятеля его уменье выбирать людей и его знание сердца человеческого, которым он умел привязывать людей к делу им доверяемому. Он искал развить в человеке преданность поручаемому ему делу и поощрял ее одобрениями и даже наградами и повышениями. Окружая себя людьми со способностями и силами, он быстро достигал своих целей и приводил намерения свои в точное исполнение.

\section{Глава IX}

Много было трудов, препятствий, неудач, скорбей и искушений как для самого настоятеля, так и для окружавшей его, пришедшей с ним братии; самое неудобство местоположения монастыря, стоящего на бойком перепутье загородных жилищ столицы, было для них тяжелым внутренним крестом, незримым для очей мира. Здесь архимандрит Игнатий, опытно обучая своих духовных чад внутреннему крестоношению, которое бывает уделом всякого благочестивого христианина, а тем более инока, сам служил для них примером благодушного терпения и безропотного несения креста своего, в чем, при содействии благодати, достиг столь великой духовной силы, что такое крестоношение во многих случаях было для него любезно. Так, продолжая вышеприведенную статью «Плача», он говорит о себе: «Здесь Милосердый Господь сподобил меня познать невыразимые словом радость и мир души; здесь сподобил Он меня вкусить духовную любовь и сладость в то время, как я встречал врага моего, искавшего головы моей, и соделалось лице этого врага в глазах моих, как бы лицом светлого Ангела. Опытно познал я таинственное значение молчания Христова пред Пилатом и архиереями иудейскими. Какое счастье быть жертвою подобно Иисусу! Или нет! Какое счастье быть распятым близ Спасителя, как был некогда распят блаженный разбойник, и вместе с этим разбойником, от убеждения души, исповедывать: «достойно по делом моим приемлю: \textit{помяни меня, Господи, во Царствии Твоем}»\footnote{Лк. 23: 41, 42} —\footnote{Аскетические опыты. Том 1. Статья «Плач мой».}. В слове утешения к скорбящим инокам духовный крестоносец так поучает тому предмету, который, можно сказать, составляет насущный хлеб истинного монашеского жительства. «Последуем Христу! Смиримся подобно Ему! Подобно Ему не откажемся прослыть льстецами и умоисступленными, "--- не пощадим чести нашей, не отвратим лица от заплеваний и ланит от заушений; не будем искать ни славы, ни красоты, ни наслаждений, принадлежащих миру сему; совершим земное странствование, как странники, не имеющие, где главу подклонить; примем, примем поношения, уничижения и презрение от людей, как неотъемлемые принадлежности избранного нами пути; будем явно и тайно бороться с помыслами гордыни, всеусильно низлагать эти помыслы нашего ветхого человека, ищущего оживить свое «я», под различными правдоподобными предлогами. Тогда Сын Божий, сказавший \textit{вселюся в них и похожду}\footnote{2 Кор. 6: 16.}, явится в сердце нашем, и дарует нам власть и силу связать крепкого, расхитить сосуды его, наступить на аспида и василиска, попрать их».

«Отвергнем ропот, отвергнем жалобы на судьбу нашу, отвергнем сердечную печаль и тоску, от которых слабые души страдают более, нежели от самых скорбей. Отвергнем всякую мысль о мщении и воздаянии злом за зло. \textit{Мне отмщение, Аз воздам}\footnote{Рим. 12: 19.}, "--- сказал Господь».

«Хочешь ли переносить скорби с легостью и удобством? "--- Смерть за Христа да будет вожделенна тебе. Эта смерть да предстоит непрестанно пред очами твоими. Умерщвляй себя ежедневно воздержанием от всех греховных пожеланий плоти и духа; умерщвляй себя отвержением своей воли и отвержением самооправданий, приносимых лжеименным разумом и лукавою совестью ветхого человека; умерщвляй себя, живо представляя себе и живописуя неминуемую смерть твою. Нам дана заповедь последовать Христу, взяв крест свой. Это значит: мы должны быть всегда готовы с радостью и веселием умереть за Христа»…

«Желающий умереть за Христа, какой напасти, какого оскорбления не претерпит великодушно?»\footnote{Аскетические опыты. Том 1. Статья «Слово утешения к скорбящим инокам».}.

Вместе с необходимыми постройками и учреждением порядков внутри обители, настоятель архимандрит Игнатий должен был обратить свою деятельность и на другие отрасли ее благоустройства, именно, на поземельную собственность и сельское хозяйство. По вступлении в управление монастырем, он не нашел ни одного межевого знака на монастырской земле. По делопроизводству обители оказалось, что всею землею, которая была приобретена покупкою еще основателем пустыни, пользовались незаконно экономические крестьяне деревни Подмонастырской слободы, монастырь же имел лишь 25½ десятин, занимаемых огородом и покосом, "--- и что все хлопоты монастыря о восстановлении его прав на эту землю, несмотря на неоспоримость его документов, остались безуспешными. В 1835 году архимандрит вошел с прошением о восстановлении на монастырской земле межевых знаков и о скорейшем разборе прав на владение землею, неправильно присвоенною крестьянами, а также о наделе монастыря лесным участком, согласно объявленному 4 июня 1835 года Высочайшему повелению, в силу которого монастыри, в видах поддержания их в способах существования, должны быть наделены для устройства земледельческого хозяйства примерно от 100 до 150 десятин земли. Спорное с крестьянами дело о земле решено в 1836 году тем, что хотя земля признана принадлежащею монастырю, но в видах затруднения выселиться с нее крестьянам, основавшимся на ней с 1765 года, положено, по соглашению с настоятелем, разделить землю на две части: восточную сторону, на которой стоит монастырь, по линии от севера к югу отдать ему, а западную, на которой поселились крестьяне, уступить им. Согласие на эту уступку прекрасно выражено настоятелем в письме его к бывшему статс"=секретарю у принятия прошений князю А. Н. Голицыну. "--- «Ваше Сиятельство, "--- писал архимандрит, "--- обычное снисхождение ваше внушает мне смелость беспокоить вас покорнейшею просьбою, впрочем весьма для вас легкою; она состоит в следующем: наша обитель давно ведет процесс о земле по купчим, плану и межевой книге генерального межевания ей принадлежащей, но оспариваемой казенными крестьянами, которые самовольно на ней населились. Министр финансов, рассмотрев дело, хотя и нашел, что земля по всей справедливости принадлежит Сергиевой пустыне, однако, затрудняясь переселением крестьян, положил землю разделить так: сторону, на коей стоит монастырь, отдать монастырю, а на коей поселились крестьяне "--- крестьянам. Сие мнение его поступило в комитет министров. Думаю, что преподобный Сергий лучше бы согласился уступить часть достояния своего, чем причинить огорчение крестьянам, переселением их, чему простые сии люди не иначе повинуются, как предаваясь неутешной печали и горьким слезам. Посему и я, поверенный преподобного Сергия, как в сем деле, так и в прочих, до обители его касающихся, должен соображаться с благоутробием своего Настоятеля и решением министра финансов быть довольным. Поддержите сие решение в комитете министров. Вот в чем состоит вся просьба к вашему Сиятельству от поверенного обители преподобного Сергия».

Во время ведения дела о поземельном владении настоятель положил хозяйственные начала тем, что прекратил отдачу в арендное содержание небольшого участка земли, оставшегося во владении обители, завел на ней огородничество, улучшил садоводство, значительно прибавил яблонь в саду. Отрезанная в монастырское владение дача, с самого начала тяжбы, остававшаяся без всякого возделывания, поросла кустарником и обратилась в болото. Очистка и осушка болота было дело одного года. Задумав обеспечить содержание монастыря введением рационального сельского хозяйства, архимандрит Игнатий просил у митрополита Серафима разрешение занять из капитала комиссии духовных училищ 45000 руб., с рассрочкою уплаты капитала и процентов на 8 лет. Сумма эта нужна была на заведение скотоводства, земледельческих орудий, рабочих лошадей, найма людей и другие хозяйственные нужды и постройки. По Высочайшему разрешению комиссия отпустила только 30000 руб., но вклад отца Михаила Чихачова, пожертвовавшего 40000 руб. ассигнациями в личное, безотчетное распоряжение архимандрита, дал средство окончить это дело и достигнуть возможного расширения и улучшения скотоводства и всего сельского хозяйства. Монастырь стал пользоваться круглый год овощами из своих огородов; ржаного хлеба доставало нередко на весь год, несмотря на значительно увеличивавшееся число братии, на огромный расход его для раздачи богомольцам в праздничные дни и на продовольствие всех монастырских рабочих; овса и сена было столько, что ежегодно продавалось на сумму от одной до полуторы тысячи рублей. Скотоводство доставляло для братской трапезы в изобилии молочные продукты; кроме хуторных земледельческих построек, стоивших до 20000 руб. ассигнациями, с ригою для сушки хлеба и с сараями для клевера, "--- были сделаны и другие при обители необходимые деревянные постройки.

Улучшения и распространения хозяйства и доходы от богомольцев, во множестве стекавшихся в обитель, дали возможность содержать большее число братии, нужду в которой увеличивали, как сами богомольцы, так и частые требования начальства о командировании на флотскую службу иеромонахов. Архимандрит стал просить в 1836 году епархиальное начальство об увеличении штата монашествующих шестью иеромонахами. Святейший Синод, на усмотрение которого представлено было это дело, 23 мая 1836 года постановил следующее определение: «Принимая в рассуждение, "--- писал Синод, "--- что Сергиева пустыня, находясь близ столицы, посещается многими богомольцами, что в ней при управлении нынешнего настоятеля нравственное состояние братии, благочестие и порядок между ними и в самом церковном служении ощутительно улучшаются, и что монашествующие ее нередко командируются во флот для морских кампаний, Святейший Синод нашел справедливым и полезным вместо предполагаемого епархиальным начальством только увеличения в этой пустыне штата иеромонахов шестью человеками, возвести оную из второго в первый класс, с присвоением ей штата людей и содержания общего для монастырей первоклассных и тем доставив пустыне способ приуготовлять большее число хороших монашествующих, поддержать достоинство ее во мнении народа, для богомоления туда стекающегося». "--- Это постановление, которым выражается признание высшим духовным правительством современного преуспеяния обители Сергиевской, было по Всеподданнейшему докладу Высочайше утверждено 3 июня 1836 года. Следствием этого было быстрое увеличение числа братии, дошедшего к 1837 году уже до 42 человек. Недоставало помещения для всех. Промысл Божий явил скоро помощь: около этого времени поступил в пустыню происходивший из петербургских купцов Макаров, скончавшийся в этой пустыне схимонахом. Он пожертвовал в пользу обители весь свой благоприобретенный капитал, составлявший 50000 руб. ассигнациями. Из числа этой суммы употреблено 40000 на построение в 1840 году внутри монастырской ограды большого деревянного на каменном фундаменте корпуса братских келлий; 10000 поступили на удовлетворение существеннейших нужд обители.

Круг деятельности архимандрита Игнатия расширился еще назначением его 22 июня 1838 года благочинным всех монастырей С."=Петербургской епархии. Служение это он нес до самого выбытия его на епископскую кафедру. "--- И на этом служении он приобрел общее доверие и уважение как настоятелей монастырей, так и монашествующей братии, соединяя для всех в лице своем и разумно"=твердого представителя власти административной, и для искавших духовного совета "--- старца, опытного руководителя, всегда готового, с любовью помочь ближнему советом духовным и словом духовного утешения.

\section{Глава X}

Любовь к служению иночеству собственным примером и писаниями не сама собою возгорелась и пламенела в архимандрите Игнатие: он веровал в призвание свое Свыше к этому служению и относился к нему, как к Божественному делу, стараясь усугублять данный ему талант; а потому такая любовь в нем торжествовала над всеми превратностями жизни. В нем витал дух живой веры в Промысл Божий, что было ощутительно для всех знавших его и что видно ясно из его творений. Он признавал, что жизнь человека, всецело предающего себя водительству Провидения, располагается по некоему Божественному плану, первообраз которого начертан в священных событиях избранного народа Божия. Смотря на иноческую жизнь, как на странствование по земной пустыне и приготовление ко входу в обетованную землю вечности, он учил, что надо соглядать эту вечность еще при настоящем земном существовании, чтобы обеспечить себе вступление в нее за пределами гроба. Это было не простое, поверхностное уподобление, а приобретенное духовною деятельностью сознание, разительные примеры чего он видел на себе самом; часто, когда естественный источник его благих желаний иссякал от зноя страстей и бурь житейских, он находил в себе новые ключи благодатных мыслей, внезапно истекавшие и обновлявшие изнемогшие силы; горечи жизни растворялись благодатною силою терпения и чрез это делались сладкими, приятными для духовного вкуса. "--- Он имел особенный дар смотреть на все духовно; малейшие случаи, ничтожные по"=видимому обстоятельства часто получали у него глубокий духовный смысл и всегда находили отголосок в нравственном учении, которым он руководился; они доставляли обильную пищу его уму и сердцу и нередко в дивной мелодии слова раздавались с его духовно"=поэтической лиры. Таковы его произведения: «Блажен муж», «Песнь под сению креста», «Молитва преследуемого человеками», «Плач инока» и многие другие. Из таких особенностей духовного призвания и настроения явствует, что высказываться письменно было духовною потребностью архимандрита Игнатия.

Тщась раскрыть сущность монашеского жительства, архимандрит Игнатий подвизался олицетворить в себе самом, и живописью слова изобразил другим духовную красоту нравов древнего египетского монашества, которое было идеалом его жизни. Иночествование по учению и примерам святых Отцов, преимущественно египетских, было с детства заветною его мыслью. Руководимый этим учением, он питал беспримерную в наше время любовь к киновиальному иночеству, и эта любовь была вполне осмыслена: он смотрел на новоначалие иноческое, как на основание аскетической науки, где зарождаются и развиваются монашеские нравы; а вообще на монашество "--- как на науку из наук. В таком духе он наставлял всякого расположенного к вступлению в иночество и, силою собственного стремления к своим высоким идеалам, производил могущественное влияние на юные, неиспорченные жизнью души. Он охотно принимал таких в духовное родство с собою и руководил опытным духоносным словом своим, которое столь было действительно, что обращало сердца, отрешало от многолетних навыков, ослабляло привычки, изменяло нравы многих. Способностью принимать исповедь помыслов, что составляет весьма редкое явление в наше время, архимандрит Игнатий обладал в совершенстве; многосторонняя опытность, глубокая проницательность, постоянное и точное самонаблюдение делали его искусным в целении душевных струпов, к которым он всегда прикасался самым тонким резцом духовного слова. Умея владеть собою во всяких случайностях жизни, не падая духом в самых стеснительных обстоятельствах, он сообщал ту же твердость и тем, которые исповедывали ему свои помыслы: угнетавшая печаль, после исповеди у него, казалась им пустым призраком. Правильное воззрение на страстную природу человека "--- плод многолетнего самонаблюдения, изложенное им в статье: «Отношение христианина к страстям его»\footnote{Аскетические опыты. Том 1. (издания 1886 и 1905 г.). Эта статья была издана по смерти епископа Игнатия в 1870 году в брошюре: Три статьи, не бывшие в печати, епископа Игнатия (Брянчанинова). "--- Ярославль : П. Брянчанинов, 1870. "--- 30 с.}, служило источником утешения для его питомцев; оно заставляло их при откровении помыслов высказываться с полною свободою, доверием и безбоязненно; они всегда слышали ответ, вполне примиряющий их с самим собою; часто пример из собственной жизни, приводимый старцем, или указание на какое"=либо в книгах описанное событие, так близко подходили к исповедываемому случаю, что не оставалось никакого сомнения или недоумения в душе исповедывающегося; ученик всегда уходил с утешением от старца.

Исповедь помыслов новоначальным иноком старцу всегда лежала в основах монашеского жительства; она входила, как непременное условие, в круг духовного воспитания архимандрита Игнатия. Борьба с помыслами мучительна, особенно в начале подвига, когда еще нововступивший не навык ратовать против них орудием молитвы; настроение себя по назиданию книги полезно и необходимо, "--- но недостаточно. Трудно юному управить себя по духовной стезе, не имея в виду примера; а враг особенно сильно ратует именно на тех, которые избирают монастырь с прямою целью спасения, отвергая все мирские преимущества и выгоды: для таких"=то духовное руководство живым словом, при исповеди помыслов, истинная находка; оно служит оплотом против наветов врага и делает собственную волю устойчивою. Все это хорошо в том случае, когда старец настолько мудр и опытен, что в состоянии уразумевать открываемые помыслы и постигать их причины и следствия; иначе его совет будет действовать разрушительно, как неверно поданное лекарство. Благоустроению духовного быта новоначального содействует и то обстоятельство, когда старец его находится во главе управления: где многоначалие или зависимость старца, там несвобода духовных отношений. Архимандрит Игнатий соединял в себе и то и другое, т. е. и мудрость духовную и внешнюю власть, а потому жительство под его руководством и в его обители было драгоценным приобретением для искавших монашествовать разумно. Не смотря на свою болезненность, он принимал на себя труд ежедневно выслушивать исповедь помыслов; "--- у учеников его было даже обыкновение вести дневную запись их, и они открывали свои помыслы чистосердечно, с прямотою, потому что старец был способен принимать такую исповедь вполне бесстрастно. Польза от исповеди помыслов была для всех очевидна. При этом старец не подвергал учеников своих тягостным испытаниям, а сообразовался с физическими способностями каждого и умственным развитием, так что состояние под его духовным водительством было даже льготно, как в физическом, так и в нравственном отношении.

Вот мнение самого старца Игнатия об исповеди помыслов, основанное на строгом следовании учению святых Отцов: «Все Отцы согласны в том, что новоначальный инок должен отвергать греховные помыслы и мечтания в самом начале их, не входя в прение, ниже в беседу с ними. Для отражения греховных помыслов и мечтаний отцы предлагают два орудия: 1) немедленное исповедание помыслов и мечтаний старцу и 2) немедленное обращение к Богу с теплейшею молитвою о прогнании невидимых врагов. Преподобный Кассиан говорит: завсегда наблюдай главу змея, т. е. начала помыслов, и тотчас сказывай их старцу: тогда ты научишься попирать зловредные начинания змея, когда не постыдишься открывать их, все без изъятия твоему старцу. Этот образ борьбы с бесовскими помыслами и мечтаниями был общий для всех новоначальных иноков в цветущие времена монашества. Новоначальные, находившиеся постоянно при своих старцах, во всякое время исповедывали свои помышления, как это можно видеть из жития преподобного Досифея, а новоначальные, приходившие к старцу своему в известное время, исповедывали помышления, однажды в день, вечером, как это можно видеть из Лествицы и других Отеческих книг. Исповедание своих помыслов и руководство советом духоносного старца древние иноки признавали необходимостью, без которой невозможно спастись»… «Наставления духоносного старца постоянно ведут новоначального инока по пути Евангельских заповедей, и ничто так не разобщает его с грехом и началом греха "--- демоном, как постоянное и усильное исповедание греха в самых его началах. Такое исповедание уставляет между человеком и демоном спасительную для человека непримиримую вражду. Такое исповедание, уничтожая двоедушие или колебание между любовью к Богу и любовью к греху, дает благому произволению необыкновенную силу, а потому преуспеянию инока необыкновенную быстроту, в чем можно убедиться опять из жития преподобного Досифея. Те иноки, которые не могли действовать против греха постоянною и учащенною исповедью греховных помыслов по неимению старца, действовали против него постоянною и учащенною молитвою»\footnote{Приношение современному монашеству. Том 4, издания 1865 г. (Том 5, издания 1886 и 1905 г.). Глава XLIV. Первый образ борьбы с падшими ангелами.}.

\section{Глава XI}

Зиму 1846 года архимандрит Игнатий пробыл безвыходно в келлии по причине тяжкой болезни, а с наступлением весны 1847 года, он подал прошение о сложении с него настоятельской должности и увольнении на покой в Николо"=Бабаевский монастырь Костромской епархии. Вместо увольнения на покой, ему разрешен был только 11"~ти месячный отпуск для поправления здоровья в указанный им Бабаевский монастырь. По отъезде архимандрита в этот отпуск, Государь Император, встретив однажды Чихачова, спросил о здоровье его товарища и приказал написать ему, что нетерпеливо ожидает его возвращения.

Летом 1847 года архимандрит Игнатий прибыл в Николо"=Бабаевский монастырь, где и занялся серьезным лечением. Ему отведены были келлии, состоявшие из четырех маленьких комнат, в отдельном мезонине над келлиями настоятеля. Помещение это, разобщенное с прочими жильями, весьма удобно было для безмолвия. С одной стороны из окон келлии открывался величественный вид обширной местности, орошаемой рекою Волгою, и представлял усладительное зрелище для отшельника в минуты отдохновения. Здесь архимандрит Игнатий написал много духовных назидательных писем к разным лицам, в числе их ряд писем к некоему иноку Леониду, озаглавленных так: «К иноку, занимающемуся умным деланием». Здесь написана была статья: «Бородинский монастырь», которая не вошла в собрание сочинений; поводом написания этой статьи служило посещение архимандритом Бородинского монастыря, на пути следования в монастырь Бабаевский, по приглашению тогдашней настоятельницы, игуменьи Марии Тучковой. Чрез 11 месяцев, в 1848 году архимандрит Игнатий возвратился в Сергиеву пустынь. Продолжительное безмолвие в пребывании уединенном, на Бабайках, расположило его еще более к совершенному отшельничеству, к которому он постоянно стремился.

Понеся великую утрату в кончине возлюбленного Монарха своего Государя Императора Николая Павловича и не оставляя намерения переселиться на покой, архимандрит Игнатий в 1856 году предпринял путешествие в скит Оптиной пустыни исключительно с целью устроить там желанное пребывание на безмолвии. Он уже совсем было условился с Оптинским настоятелем о приготовлении для себя келлии в скиту и о переделке ее, дал 200 рублей задатку, и возвратился в С."=Петербург, где, по обстоятельствам, не от него зависящим, должен был на неопределенное время отложить исполнение своей мысли о переселении на покой и предать дальнейшую участь свою воле Божией.

В 1856"~м году скончался С."=Петербургский митрополит Никанор; назначенный на его место митрополит Григорий хорошо знал архимандрита Игнатия и даже состоял с ним в духовно"=близких отношениях, а потому в видах пользы для Церкви Божией предложил ему епископскую кафедру в Ставрополе Кавказском. По получении Высочайшего соизволения 23"~го октября 1857 года происходило в Святейшем Синоде наречение архимандрита Игнатия во епископа Кавказского и Черноморского, а 27"~го "--- самая хиротония в Казанском соборе, при весьма многочисленном стечении народа. На другой день новопоставленный епископ совершил литургию в лаврской Крестовой церкви, и затем три дня следующего месяца провел в Сергиевой пустыне. 2"~го ноября посетила пустынь Великая Княгиня Мария Николаевна "--- «чтобы проститься с епископом Игнатием», как она изволила выразиться. 3"~го ноября, в воскресный день, епископ Игнатий отслужил в пустыне Божественную литургию, участвовал на общей братской трапезе, и простившись со всеми, окончательно оставил Сергиеву пустынь. Он переехал в Невскую Лавру.

4"~го ноября по назначению вдовствующей Императрицы Александры Феодоровны, епископ Игнатий ездил в Царское Село для представления Ее Величеству. Государыня изволила принимать его в своем кабинете, причем пожаловала ему панагию, украшенную бриллиантами и рубинами, сказав: «С соизволения Государя, даю вам эту панагию в память обо мне и о покойном Государе». 9"~го ноября епископ откланивался у Великого Князя Константина Николаевича и Великой Княгини Александры Иосифовны, причем имел продолжительную духовную беседу с Великой Княгиней, а 10"~го, в Царском Селе имел счастье откланиваться сначала у Государя, потом у Государыни особо, на их половинах, при этом Императрице угодно было говорить с ним очень серьезно о монашестве вообще и о Сергиевой пустыне в особенности. 17"~го преосвященный участвовал в хиротонии Соловецкого архимандрита Александра во епископа Архангельского и Холмогорского. В этом же богослужении наместник Сергиевой пустыни иеромонах Игнатий возведен в сан архимандрита, и по рекомендации епископа Игнатия и единодушному желанию братства, назначен настоятелем пустыни.

Прожив в Сергиевой пустыни без двух месяцев двадцать четыре года, епископ Игнатий оставил ее в весьма цветущем состоянии. В его управление обитель украсилась тремя новыми великолепными храмами.

Из воспитанников по монашеству архимандрита Игнатия Брянчанинова, Сергиева пустынь дала шестнадцать настоятелей: десять архимандритов, пять игуменов и одного строителя.

\section{Глава XII}

Епископ Игнатий в столь продолжительное настоятельствование свое в Сергиевой пустыне не только не скопил себе никакого капитала, но дошел до такой нестяжательности, что при отбытии своем не имел даже собственных средств на дальную дорогу. Он всегда был чрезвычайно щедр на милостыню; отказа не делал никому, если имел что подать; когда не случалось денег, то подавал вещами, так что келлейные старались наделять просящих деньгами, сколько было возможно, предупреждая просителей от личных отношений к архимандриту. Поэтому, когда надо было выехать из Петербурга, епископ вынужден был прибегнуть за денежным вспомоществованием к одному близкому ему по духовным отношениям лицу, которое и снабдило его пособием в тысячу рублей. 25"~го ноября он оставил Петербург и 26"~го прибыл в Москву, откуда ездил в Сергиеву Лавру. Пробыв в Москве с неделю, он отправился на Харьков и Ростов (на Дону), останавливался в губернских городах у епархиальных архиереев, по приглашению их литургисал в Курске и Харькове, заезжал в Святогорский монастырь, и 24"~го декабря, утром, выехал из Бахмута еще по летней сухой дороге. В ночь на Рождество он был застигнут в степи страшною снежною метелью, которая подвергала его жизнь большой опасности. Только к семи часам утра, 25"~го декабря, он кое"=как добрался до жилья, в санях, в которых выехали его отыскивать священнослужители ближайшего села, извещенные кучерами, оставившими экипаж в степи и верхами поскакавшими на звон колоколов искать помощи.

4"~го января 1858 года, в четвертом часу пополудни епископ Игнатий прибыл в Ставрополь Кавказский. Архиерейского дома не было; новоприбывший владыка остановился в приготовленной для него квартире в доме купца Стасенкова. Существовал небольшой деревянный домишко, похожий на хижину, который лет за 14 до этого подарил ставропольский купец Волобуев, для временного помещения первого епископа Иеремии. К этой же хижине была пристроена столько же незатейливая половина из двух небольших комнат, названных залой и гостинной, "--- последняя служила вместе и моленной для архиерея, так как из нее были сделаны окошечко и входная дверь, в пристроенную к этой хижине, небольшую каменную церковь Крестовую. Тогда же и этот плохой домишко пришел в крайнее разрушение, так что граждане посовестились принять в него епископа; движимые благожеланием, они наняли для него приличную квартиру на свой счет. В день приезда, преосвященный принимал в своей квартире духовенство кафедрального собора, граждан с хлебом"=солью и начальника губернии генерал"=лейтенанта А. А. Волоцкого; вечером слушал дома всенощную, и на другой день (воскресный), 5"~го января, служил литургию в рядской соборной церкви. В день Богоявления, по совершении литургии, освящал воду в бассейне, расположенном в средине города, при значительном собрании войск, с окроплением знамен и при многочисленном стечении народа, который принял владыку приветливо.

Епархия Ставропольская, совсем неустроенная, потребовала от преосвященного Игнатия больших трудов. Она учреждена была около 1840 года. Первым ее епископом был Иеремия, который по ревности к Православию слишком строго отнесся к раскольникам, которых весьма значительное число в Кавказском линейном казачьем войске. Вследствие этого Кавказское линейное начальство ходатайствовало об изъятии линейного казачьего населения из ведомства епархиального архиерея и о передаче в ведение обер"=священника Кавказской армии. Таким образом, в только что учрежденной епархии из пятисот тысяч душ, составлявших ее паству, половина отошла из управления епископа, который никак не подозревал возможности такого отделения. Так как епархия эта была поставлена в разряд третьеклассных, то епископ ее должен был содержаться на жалованье в 285 руб. серебром в год. В виду того, что такая сумма была крайне недостаточна для вновь открывшейся кафедры, которая требовала во всем необходимого устройства, Святейший Синод определил временно отпускать епископу из синодских сумм по 1000 руб. серебром в год карманных денег и по 1500 руб. серебром на содержание архиерейского дома, впредь до полного устроения кафедры.

Преосвященный Игнатий нашел в Ставрополе, как выше замечено, гражданским губернатором Волоцкого, Вологодского уроженца, своего сверстника по детству, с которым одновременно приехал в Петербург определяться на службу; начальником войск был генерал"=лейтенант Филипсон, человек весьма благочестивый, а наместником князь А. И. Барятинский. Вскоре Волоцкой уехал в отпуск и место его занял Ставропольский вице"=губернатор П. А. Брянчанинов, родной брат епископа. При отправлении из С."=Петербурга обер"=прокурор Синода граф А. П. Толстой уверял преосвященного, что ему будут продолжать выдачу тех пособий от Синода, которые получали его предместники, но не смотря на многократно повторенную просьбу, ни одного из помянутых прежде получаемых окладов ни разу не дали; между тем епископу было бы нечем жить, если бы особые случайные обстоятельства, как то: сближение сперва с товарищем детства, потом с родным братом не дали возможности восполнять недостатки. Ко времени приезда епископа Игнатия на кафедру, архиерейскому дому были даны только штатные служители, кусок земли пахотной около двухсот десятин вдали от города, да лесная дача вблизи города, в который она входит одною частью своею, примыкая к Андреевской церкви, принадлежащей архиерейскому дому, при коей помещается ныне викарный епископ. С этою лесною дачею архиерейскому дому пришлось получить в свое ведение спорное дело с одним частным лицом, сделавшим в этой даче захват земли, на которой им устроен был кирпичный завод.

Первым делом епископа было благодарить граждан города за заботливость их о его помещении, причем он представил им неудобство жить ему, окруженному монахами, в светском доме семьянина, просил их помочь ему устроить хотя небольшое помещение при Крестовой церкви вместо вышеупомянутой развалившейся хижины. Граждане с усердием согласились. Тотчас собрана была сумма более 4"~х тысяч рублей и с открытием весны началась постройка деревянного на каменном фундаменте дома, о восьми комнатах с девятою "--- моленною. В этом доме поместились: архиерей, два иеромонаха, несколько послушников и прислуга. Одновременно с этим было сделано епископом сношение с гражданским губернатором, о соглашении управления государственными крестьянами заменить штатных служителей, которые наряжались палатою, взносом ежегодно денег по местной уменьшенной справочной цене по 40 рублей в год за рабочего. Соглашение это, представленное Кавказскому наместнику, было утверждено; точно так же по Высочайшему повелению, испрошенному наместником, заменены были и угодья, законом определенные архиерейским домам, ежегодным соответственным денежным окладом Кавказскому архиерейскому дому из оброчных сборов палаты государственных имуществ. "--- Этим и ограничилось улучшение содержания епископа в бытность преосвященного Игнатия на Ставропольской кафедре. Видя представления свои Святейшему Синоду, об увеличении средств существования, по разным причинам неуваженными, епископ Игнатий вошел об этом с ходатайством непосредственно к наместнику Кавказскому князю Барятинскому, представив ему вполне положение епископа Кавказского, нищего, по тем средствам содержания, которые определялись ему лично (285 руб. серебром годового жалованья), и поставленного на такое административное место, которое непременно требует соответственной внешней обстановки и некоторой независимости в материальных средствах; при этом был представлен наместнику штат Таврической епархии, по примеру которой епископ просил определить содержание Кавказской кафедре. «Не для себя забочусь я об этом окладе, "--- писал преосвященный князю, "--- ибо меня несомненно уже не будет тогда, когда разрешится это мое представление, а забочусь для самого дела; лучше закрыть кафедру, чем оставлять ее в таком положении нищенском, понуждающем чиновников и лиц, какими обставлен епископ, искать неправильных средств к существованию». Это ходатайство, подкрепленное участием фельдмаршала, увенчалось успехом, но уже тогда, когда преосвященный Игнатий оставил Кавказскую кафедру, теперь одну из самых обеспеченных в материальном отношении.

Первою заботою епископа Игнатия, по управлению паствою, было устроение богослужения в надлежащем церковном чине и восстановление должных отношений между духовенством и народом, как в городах, так и в селах. Сам он в обращении с духовенством был приветлив, прост и прям, постоянно заботился о его быте, образовании и взаимных отношениях, приличествующих священному сану; внимательно входил в действия благочинных, в нравственное состояние тех, кои подвергались наказаниям и замечаниям, стараясь, сколько возможно, отделять служебную виновность лица от его семейных обстоятельств и всяких домашних нужд. В исходе августа преосвященный объехал юго"=восточную часть епархии, посетил города Моздок и Кизляр, и селениями восточной части губернии возвратился в Ставрополь, где на зиму и перешел во вновь построенный дом.

Преосвященный Игнатий обращал свое внимание также и на воспитание юношества в подведомственных ему учебных заведениях. Так он нашел необходимым ограничить меры наказания, бывшие в обычаях в этих училищах, предложением семинарскому правлению, чтобы воспитатели обращали более внимания на нравственную сторону взысканий, развивали в детях и юношах совестливость, как более гарантирующую их доброе поведение, и тщательно обсуждали правдивость наказаний, соразмеряя их с виновностью; чтоб самые наказания эти были разумно"=человечные, без увлечения гневом или горячностью.

\section{Глава XIII}

С наступлением весны 1859"~го года преосвященный Игнатий весьма сочувственно отнесся к делу об улучшении быта крестьян с уничтожением крепостного права. Его глубоко огорчило кривое толкование воли Государя"=Освободителя, извращение смысла слов «свобода», "--- «воля», благоволением Монарха дарованных своим подданным. По этому поводу преосвященным были сделаны два предложения, разосланные циркулярно по епархии: в том и другом излагался евангельский взгляд на дело, и предлагалось духовенству руководствоваться этим взглядом в случаях сношения с прихожанами, когда эти будут обращаться за советом или вразумлением касательно крестьянского вопроса.

Летом того же года преосвященный объезжал западную часть своей епархии; посетил приходы и монастыри Черноморья и около двух месяцев прожил в Тамани. С осени он занялся приисканием и устройством нового помещения для семинарии, помещавшейся в наемном по контракту доме, тесном и неудобном, которому уже приближался исходный срок найма. При содействии местного гражданского начальства состоялся весьма выгодный наем на продолжительный срок частных домов под это заведение, в нагорной и самой здоровой части города, рядом с церковью Св. Апостола Андрея Первозванного, что на архиерейской лесной даче, примыкающей в этом месте к главной торговой площади города. Множество следственных дел, тянувшихся производством, множество давних неудовольствий между прихожанами и духовенством, и еще большее число дел, возникших от немирных столкновений уездных гражданских властей с духовными или были окончены мирными соглашениями, или разобраны и порешены; из множества их осталось разве ничтожное количество неоконченных, так что вообще можно сказать, что епархия была приведена в полное благосостояние.

Но скорби не оставляли преследовать епископа Игнатия и в этом его положении, а кончина митрополита Григория лишила его ближайшего человека, принимавшего в нем дружеское участие. К тому же его постигла очень тяжкая болезнь: натуральная оспа, соединенная с сильною горячкою. Продолжительно тянулось его выздоровление, силы его стали заметно слабеть, он решился проситься прямо на покой в знакомый уже Николо"=Бабаевский монастырь; в конце июля 1861"~го года, подал о том рапорт в Синод и обратился с письмом к Государю Императору.

Вот содержание этого письма:

«Августейший Монарх, Всемилостивейший Государь!»

«Чувствуя изнеможение сил от болезненности, продолжающейся около 40 лет, и постоянно питая в душе моей желание окончить дни в уединении, я подал в Святейший Синод рапорт, в котором, донося о состоянии своего здоровья, прошу об увольнении меня от управления епархиею и предоставлении мне в управление общежительного Николо"=Бабаевского монастыря, на Волге, в Костромской епархии, по тому образцу, как это делалось для многих архиереев, уволенных от дел епархиальных. То милостивое внимание, которого удостоивали меня Ваши Августейшие Родители, называя меня своим воспитанником, дозволяет мне обратиться к Вашему Императорскому Величеству, с всеподданнейшею и убедительнейшею просьбою. "--- Не во внимание к какой"=либо заслуге или достоинству "--- коих нет у меня "--- в память Ваших почивших Родителей окажите мне милость, повелите удовлетворить моему прошению, даруйте мне просимый приют, в котором я мог бы окончить в мире дни мои, вознося недостойные и убогие молитвы к Богу о благоденствии Вашем и всего Вашего Августейшего Дома, о покое и вечном блаженстве Ваших приснопамятных Родителей».

«С чувствами верноподданническими благоговейнейшего уважения и совершеннейшей преданности имею счастье быть», и проч. «24 июля 1861 года».

Августа 5"~го состоялось увольнение с назначением пенсии по 1000 руб. в год; впоследствии она по Высочайшему повелению увеличена прибавкою 500 рублей.

19"~го сентября 1861 года Государь посетил Кавказ, но в Ставрополе не был, а осматривая вновь покоренные земли за Кубанью, спрашивал у графа Евдокимова (главная квартира которого была в Ставрополе) о преосвященном, и чрез него прислал ему орденские знаки Св. Анны 1"~й степени, которые уже не застали владыку в Ставрополе, а отправлены к нему по почте на новое место жительства.

При отъезде из Ставрополя, так же, как и прежде из Петербурга, у преосвященного не имелось собственных денежных средств; он должен был опять прибегнуть к посторонней помощи, чтоб рассчитаться с некоторыми долгами и покрыть путевые издержки. На пути следования чрез Москву, он остановился у старого знакомца своего преосвященного Леонида, епископа Дмитровского, викария Московского, и прогостил у него за болезнью до двух недель. "--- У него сделался нервный удар в правой ноге, и хотя принятыми медицинскими мерами и получил некоторое облегчение, но, с того времени, стал постоянно страдать слабостью этой ноги.

Во время пребывания своего на Кавказе, преосвященный Игнатий не оставлял своих духовно"=литературных трудов. Кроме устно сказанных поучений, он написал здесь всю книгу «Приношение современному монашеству», составляющую 5"~й том\footnote{\normitfont Том 4 издания 1865 г.; том 5 издания 1886 и 1905 г. "--- Ред.} его творений. В ней преподает он современному монашеству советы жизни иноческой в правилах наружного поведения и во внутренней душевной деятельности, применительно к тем многоразличным служениям, какие возлагаются на монашествующих в наше время. Составил слова: «О различных состояниях естества человеческого по отношению к добру и злу», «О видении духов», и «О спасении и христианском совершенстве».

Слово о видении духов составлено преосвященным на основании собственных опытов. Такого рода опыты он изведывал в течение большей половины своей духовной жизни и изображал их в своих книгах, смотря на них как с духовной, так и физической сторон.

\section{Глава XIV}

В Николо"=Бабаевский монастырь преосвященный приехал 13 октября 1861 года. С ним прибыли: управлявший при нем Кавказским архиерейским домом игумен Иустин, присный духовный сын епископа, ризничий иеромонах Каллист, из послушников Сергиевой пустыни, иеромонах Феофан, состоявший духовником при Андреевской церкви Ставропольского архиерейского дома, из иноков Никифоровской пустыни Олонецкой епархии, и несколько послушников. Последние состояли на Кавказе под духовным руководством иеромонаха Феофана, который, равно как и игумен Иустин и иеромонах Каллист, пользовался сам советами и духовными наставлениями владыки Игнатия. Таким образом все лица, прибывшие с Кавказа в Бабаевскую обитель, составляли одну духовную семью, и их дружный образ действий скоро повлиял на весь внутренний и внешний быт обители.

Николо"=Бабаевский монастырь расположен на правом берегу реки Волги, при впадении в нее небольшой речки Солоницы, которая отделяет Костромской монастырский берег от границы Ярославской губернии. За обеими реками в этой смежной губернии монастырь владеет небольшими сенокосными лугами. Жители Ярославля издавна расположены к Николо"=Бабаевскому монастырю и с особенным благоговением и верою чтут имеющуюся в нем святыню "--- чудотворную икону Святителя Николая. По"=видимому выгодное положение обители, стоящей на полпути между двумя губернскими городами Ярославлем и Костромой, и при одном из первых водных сообщений должно намекать на материальное благосостояние ее, но на самом деле было не то: пришельцы с Кавказа застали этот общежительный монастырь всего с 60 рублями наличных денег, при двух тысячах рублей долгу, притом, пред самым вступлением в зиму, без всякого запаса хлеба и заготовки дров; хлебопашеством в монастыре вовсе не занимались. Благоустроению обители много помогло то обстоятельство, что, еще до приезда преосвященного Игнатия, бывший настоятель ее игумен Парфений, по собственному желанию и просьбе, был перемещен епархиальным начальством в Надеевскую пустынь той же Костромской епархии, и должность его была сряду занята игуменом Иустином: благочинным монастыря был назначен иеромонах Каллист, а иеромонах Феофан сделан вторым духовником. Чин богослужения, порядки келлейного жительства, братская трапеза и жилища, все было поведено к улучшению. Заведено правильное хлебопашество на земле, принадлежащей к монастырю, около 80 десятин, частью болотной, частью наносно"=песчаной, для чего была произведена разбивка полей, прорыты канавы для осушки болот и чрез них болотная вода спущена в Волгу. Начата перестройка келлий, назначенных для самого епископа, капитальное исправление корпуса братских келлий и настоятельских. Корпус этот, с одной стороны обращенный к Волге, "--- двухэтажный, с другой, внутрь монастыря "--- одноэтажный; его нашли нужным сшить поперечными железными полосами, так как он от сырости места и от неравномерной осадки значительно истрескался в этом направлении. Гостиница для приезжающих, расположенная у самой ограды монастыря, была вся переделана внутри и прилично меблирована.

В первый год по прибытии владыки Игнатия на Бабайки, посетил его старинный друг М. Чихачов, и это свидание их было последнее в жизни. Обоими признано было, что последнему не надо покидать своего места жительства "--- Сергиевой пустыни, в которую он положил все свое состояние и пользовался в ней всеобщим уважением братства и теми удобствами, какие необходимы были для него в болезненной старости. В конце июня 1862"~го года приехал в Бабаевскую обитель на жительство родной брат владыки П. А. Брянчанинов, испросив увольнение от службы с должности ставропольского гражданского губернатора, и поселился в монастыре на правах богомольца.

В мае 1862 года епископ Игнатий посетил преосвященных: Платона в Костроме, Нила в Ярославле и Иринея в Толгском монастыре, близ Ярославля; после этой поездки уже никуда не выезжал из монастыря, кроме прогулки в экипаже по окрестностям монастыря в хорошую погоду. В семь часов утра он пил чай, который признавал необходимостью, как средство, согревающее кровь, и говаривал: «вот что значит старость, не напившись чаю и Богу неспособно помолиться». С 9"~ти часов принимался за дела или выходил к литургии или осматривал производящиеся работы; принимал посетителей, по большей части крестьян больных, пользовавшихся от владыки медикаментами (гомеопатией). Таких больных стекалось очень много; один из келлейников записывал имена их в книгу, лета и род болезни, а владыка отмечал, какое кому дать лекарство, число приемов и диету, если таковая оказывалась нужною. Лечение шло успешно, но чрез три года было прекращено по причине многолюдного стечения больных, нарушавших уединение святителя. В исходе 12"~го часа дня владыка обедал. Стол его был простой и кушал он очень умеренно. В 3"~м часу кушал чай, к которому всегда призывал кого"=либо из братии и завершал угощение душеспасительною беседою. После вечерни, до 8"~ми часов принимал всех, имевших к нему духовную нужду иноков и послушников, а также и посторонних посетителей. С 8"~ми часов вечера владыка запирался в своих келлиях; он спал обыкновенно не раздеваясь, на ночь надевал валенные сапоги, по причине болезни ног, издавна простуженных. Такова была обстановка келлейной жизни преосвященного Игнатия на Бабайках.

С приездом преосвященного Игнатия в монастырь, стечение народа к богослужениям значительно увеличилось; церковь Святителя Николая, вмещающая не более 600 человек, стала тесна, вся монастырская братия и окрестные жители стали выражать желание построить новый храм, вместо пришедшего в опасное состояние соборного храма Иверской Божией Матери, шести"=придельного, в четырех приделах которого уже воспрещено было отправлять богослужение, так как образовались большие трещины по всем направлениям каменных сводов. Явился жертвователь "--- подрядчик каменных работ, Ярославский мещанин Федотов, обещавший всю каменную работу исполнить своими рабочими безвозмездно, и дать на начальное действие 1000 рублей. Преосвященный вызвал из С."=Петербурга, знакомого ему архитектора, профессора академии художеств И. И. Горностаева, и передал ему свою величественную идею нового храма, которую тот осуществил в своем проекте. Чтоб разобрать разрушавшийся соборный храм, надо было исходатайствовать чрез обер"=прокурора Святейшего Синода Высочайшее разрешение; а также подлежало Высочайшему утверждению и разрешение постройки нового храма. Между тем монастырю необходимо было спешить разборкою храма, так как рабочие Федотова, в случае неразрешения, готовились отправиться в Петербург на свои заработки, тогда и Федотов вынужден был бы отказаться от обещанного, что составило бы большую потерю для храмоздателей.

В субботу на первой неделе Великого поста Государь, по приобщении Святых Таин, благоволил вспомнить, что у обер"=прокурора есть нужные к докладу дела и потребовал представить их. Обер"=прокурор, директор его канцелярии, начальник отделения, "--- все, чрез руки которых шло дело о Бабаевском храме, были в этот день причастниками. Государь принял во внимание пожертвования Федотова и благоволил разрешить разборку старого храма. В тот же день разрешение было передано по телеграфу чрез Ярославль в Бабаевский монастырь; вся братия монастыря были причастники, но владыке об этом было доложено на другой день "--- в воскресенье недели Православия после литургии, которую он совершал сам. Таким образом совершилось достойное замечания событие: первый шаг к сооружению нового храма был сделан лицами, которые все без изъятий только что сподобились причаститься Св. Христовых Таин, "--- видимый признак благословения Божия на начатое дело. С понедельника 2"~й недели поста началась разборка старого храма; дело шло быстро, кирпич заготовлялся на заводах, устроенных при монастыре, добывался на месте дикий камень, который пошел в цоколь здания, распланирована была местность, так как новый храм воздвигался немного далее в глубь берега, размеры фундамента обозначены; твердый грунт не требовал свайных укреплений; предположено было с наступлением весны приступить к сооружению здания; ждали утверждения проекта, которое замедлилось. Министр путей сообщения находил, что архитектурные линии в нем слишком смелы, чтоб могли быть благонадежно исполнены; только по личным объяснениям архитектора Горностаева, министерство согласилось на утверждение. Таким образом проект храма в честь чудотворной иконы Иверской Божией Матери был приготовлен к всеподданнейшему докладу, который в свою очередь разными случайностями оттянулся до конца мая, и лишь в 21"~е число того месяца, день празднования Божией Матери иконы Владимирской, Государь изволил утвердить проект. «Над построением храма Богоматери, "--- писал владыка своему брату, "--- очевиден перст Богоматери. Даруется человекам труждающимся в деле помощь; вместе даруется им побороться с препятствиями и поскорбеть для их же душевной пользы, чтобы очистить дело от примеси тщеславия и других увлечений, чтобы оно было совершено в богомудром смиренномудрии. Таков обычный ход дел, покровительствуемых Богом».

Сооружению нового храма доставляли денежные средства расположенные к монастырю граждане гг. Ярославля и Костромы. Преосвященный Ярославский Нил сочувственно отнесся к желанию граждан ежегодно переносить с крестным ходом чудотворную икону Святителя Николая из Бабаевского монастыря в Ярославский кафедральный собор. Костромской преосвященный Платон выразил свое одобрение и Святейший Синод, указом от 4"~го июля 1866 года, уважил и разрешил ходатайство Ярославского архипастыря. С тех пор икона эта носится ежегодно в Ярославль, встречаемая и провожаемая десятками тысяч жителей.

\section{Глава XV}

Живя на покое в Бабаевской обители, свободный от служебных обязанностей, преосвященный Игнатий все свободные часы своего дня отдал пересмотру и пополнению своих аскетических сочинений. Труды по напечатанию их принял на себя, поселившийся при нем, родной брат его Петр Александрович. Между всеми изданными в это время сочинениями особенно замечательным является «Слово о смерти», в первый раз напечатанное издателем журнала «Домашняя Беседа» "--- Аскоченским. Впоследствии составлено было преосвященным Игнатием особое «Прибавление к слову о смерти»; это «Прибавление» вошло в издание «Аскетических Опытов» в конце 2"~го тома, а потом, вновь значительно пополненное автором, было издано вместе с «Словом о смерти» отдельною книжкою по кончине его, в 1869 и 1880 годах.

Книгопродавец "--- издатель И. И. Глазунов, старинный знакомец владыки Игнатия, вошел с ним в соглашение о напечатании всех его сочинений, принимая издержки издания на себя, и тем ввел преосвященного в усиленное занятие пересмотром, исправлением, пополнением и приведением в одно целое всех статей, писанных им в разное время в сане архимандрита, а потом и епископа. Таким образом составились первые два тома под названием «Аскетических Опытов», изданные в 1864 году; последние два: «Аскетическая проповедь» и «Приношение современному монашеству», состоящее в советах наружного поведения и духовного делания, напечатанные в 1867 году, пред самою кончиною владыки. Пятый том, под заглавием «Отечник», содержащий изречения святых отцов и повести из жизни их, издан также Глазуновым уже по кончине составителя. В предисловии 1"~го тома автор объясняет причины, побудившие его к изданию своих сочинений; именно, он признает себя обязанным дать христианскому обществу отчет в соглядании им земли обетованной, точащей духовные блага, какою является иноческая жизнь, проводимая по учению и преданию Восточной Церкви и созерцаемая в живых представителях ее. В этом предисловии говорится, что разнообразные статьи «Аскетических Опытов» были составляемы по поводу возникавших вопросов в обществе иноков и боголюбивых мирян, находившихся в духовном общении с автором. Все они, в целом составе, изображают православный христианский подвиг в его порядке, постепенности; остерегают подвижника от увлечений и заблуждений, от несвоевременного стремления к высоким духовным состояниям; научают полагать прочное основание на делании евангельских заповедей, на покаянии и покаянном плаче.

Здесь, не не у места привести слова самого преосвященного Игнатия, не раз в откровенных беседах повторенные им брату его, Петру Александровичу, "--- что он ни о каком духовном делании не говорил, а тем более не писал, не проверив своим собственным опытом того учения или делания и его последствий, которые он передавал слушателю или читателю, указывая в то же время на Писания Священное и отеческие, говорившие о том же предмете, "--- что впрочем ясно усматривается и из самых творений его.

\section{Глава XVI}

Задолго до кончины своей преосвященный Игнатий стал готовиться к ней, и в разговорах своих часто касался распоряжений на случай смерти. За пять лет (в 1862 году) он сделал духовное завещание, засвидетельствованное 20"~го июля 1863 года в Костромской палате гражданского суда, коим все свои сочинения передавал в собственность и распоряжение брата своего Петра Александровича Брянчанинова. В августе 1864 года он говорил своему брату: «Матушка наша была также больна пред смертью, как и я; все на ногах, и аппетит был порядочный; а пришло время, "--- в три дня болезнь покончила все дело. Прошу, когда я буду умирать, не вздумайте посылать за доктором, дайте мне умереть христианином, "--- не подымайте суматохи. О кончине моей родных не уведомлять и к похоронам их не ожидать, а предав земле, тогда уведомите… Я тебе говорю вперед, чтоб ты знал, и чтоб об этом в час болезни предсмертной не забыть и не заботиться. О том, как и где похоронить меня ничего не говорю и не завещаю потому, что не желаю связывать действий ближних, за пределами моей жизни, и притом в том, что никогда почти не исполняется».

Наступил 1866 год, "--- печатались 3"~й и 4"~й томы творений преосвященного Игнатия, его «Аскетическая проповедь» и «Приношение современному монашеству» или «Советы»; между тем, физические силы его самого видимо упадали, так что приезжавшие из Петербурга для посещения его духовные дети поражались тою переменою, какая представилась им при виде духовного отца, изнуренного болезнью и преждевременною дряхлостью. Не смотря однако же на такое падение физических сил, душевная бодрость не оставляла его. «Не бойтесь, "--- писал он одному из своих духовных чад, занимавшемуся корректурою издаваемых его сочинений, "--- я не умру до тех пор, пока не кончу дела своего служения человечеству и не передам ему слов истины, хотя действительно так ослабел и изнемог в телесных силах, как это вам кажется».

14"~го августа 1866 года, посетили Николо"=Бабаевскую обитель Их Императорские Высочества Государь Наследник Александр Александрович и Великий Князь Владимир Александрович. Владыка, поднося Цесаревичу святую икону Благоверного Князя Александра Невского, встретил его следующею речью: «Всемогущий Бог, в трудные времена России, осенивший небесным благословением и небесною помощью Благоверного Великого Князя Александра Невского, да осенит этим благословением и этою помощью и Ваше Императорское Высочество в предстоящем Вам великом служении Богу и человечеству». Потом, вручая Владимиру Александровичу икону Святого равноапостольного Князя Владимира, сказал: «Ваше Императорское Высочество! В древности два Великие Князя "--- Равноапостольный и Мономах "--- носили имя Владимира. Благочестием, мудростью, мужеством ознаменовалась жизнь их. "--- И ныне Великий Князь, носящий имя вожделенное для России, да возрадует Россию этими качествами, столько благодетельными для народов, когда народы озаряются ими из святилища "--- из Царственного Дома». Келлейная беседа владыки с высокими посетителями касалась монастырей. "--- «Монастыри лечебницы, "--- говорил преосвященный, "--- это приют для людей, которые, сознав бессилие свое сохранить себя, душу свою, живя в мире, идут в это убежище, и приносят в него свои понимания, свои привычки, свои пороки, свои страсти, развитые тем образованием, которое они получили в мире, "--- поэтому нравственное состояние монастырей находится в совершенной зависимости от нравственного настроения народа. Народ развращается, развращаются и монастыри. В них много вкралось предосудительного, много дурного; но при всем том, они сохраняют характер свой "--- убежища желающим сохраниться от конечной погибели; они больницы для душ безнадежно"=больных, они приют верности Церкви Православной и престолу. Извольте, Ваше Высочество, обратить внимание на то обстоятельство, что нет другого сословия, кроме монашеского, в котором не было бы ковов на измену престолу. Монашество и монастыри потому особенно гонимы партиями злонамеренными, что они преданы вере и престолу и поддерживают эти чувства в тех, которые сближаются с ними и подчиняются их духовному направлению. Одною ногою я уже стою в могиле, и для себя ничего не ищу, и мне нечего искать, а докладываю Вашему Высочеству сущую истину, ради истины; умоляю Ваше Высочество, поддерживайте монастыри по тому благу, которое приносит их существование». "--- Их Высочества обошлись весьма благосклонно с владыкой, утешили его своим вниманием к словам его. Посещение их оставило самое приятное впечатление в преосвященном, он называл его своим окружающим «зрением восходящих светил».

\section{Глава XVII}

Зиму с 1866 на 1867 год преосвященный Игнатий провел в заботах о приготовлении к печати избранных им изречений и повестей из жизни святых иноков, из которых составился 6"~й том его творений, изданный уже по кончине его, под наименованием «Отечник». С тем вместе он не оставлял продолжать и другие частью начатые, частью пополняемые им, прежденаписанные статьи. В эту зиму он написал статью «О терпении скорбей», «Об отношениях человека к страстям его», и значительно пополнил «Разговор старца с учеником его о молитве Иисусовой».

Дни шли за днями, ничто по"=видимому не изменялось. Окружающие привыкли видеть преосвященного постоянно болящим, слабым в силах, но притом постоянно одетым, постоянно занятым работою за письменным столом, или в молитвенном подвиге "--- и ничто по"=видимому не выражало близости кончины его; хотя он жаловался иногда на боль сердца, на болезнь ног и другие недуги, но все это проходило, как явления временные и довольно обычные, не изменявшие нисколько порядка дневных занятий. Не смотря на разнообразные недуги, о которых сообщал он окружающим "--- никто никогда не слыхал его болезненного стона. Он не раз говорил, что, заставляя себя не стонать в болезнях, он приучал себя претерпевать все находящее, а по обычаю Афонских подвижников, не раздеваясь ни днем, ни ночью, до самого часа кончины, он как бы скрыл от окружающих этим внешним порядком жизни и самую опасность своего положения.

16"~го апреля 1867 года, в день Светлого Христова Воскресенья, совершив литургию, преосвященный так утомился, что с трудом довели его до келлии. Нужен был ему получасовой отдых, чтоб собраться с силами принять пищу. "--- В этот же день объявил он окружающим, чтоб после вечерни никто его не беспокоил, ибо с этого часа дня он никого принимать не будет, объявив причиною этого распоряжения «необходимость свою готовиться к смерти».

На другой день, 17"~го числа, день рождения Государя Императора Александра Николаевича, преосвященный стоял литургию в алтаре, но выходил служить благодарственный молебен, причем читал окончательную благодарственную молитву с таким сильным, глубоко"=благодатным выражением, что обратил общее внимание на это обстоятельство. Кто мог полагать, что это был последний выход святителя из его келлий, "--- возвратившись в которые он уже более не выходил, хотя обычная жизнь его в трудах, в подвиге, в болезнях потекла неизменно"=обычною чредою.

21"~го апреля получены были присланные из Петербурга, только что вышедшие из печати, 3 и 4 томы его творений. Преосвященный перекрестился, и, дав славу Богу, не развернув, не посмотрев книг, приказал оставить их до приезда из Петербурга брата его Петра Александровича. Равнодушие это было совершенно противоположно заботливому и от природы деятельному характеру, и прежнему вниманию преосвященного к изданию его творений, на что он смотрел, как на обязательное исполнение долга своего. Нельзя не заметить притом, что, так как приезд брата его был обстоятельством совершенно неопределенным, то помянутым распоряжением Преосвященный совершенно устранял себя от дела, которое в естественном порядке было ему, как пастыреначальнику и автору, ближайшим из всех его земных дел. Около этого времени объясняя архимандриту Иустину свое духовное состояние, он передавал ему, что потерял всякое сочувствие ко всему земному, потерял даже внимание ко вкусу пищи, причем прибавил: «я не долго потяну». "--- Любимому своему келлейнику Василью Павлову\footnote{Впоследствии иеромонах С."=Петербургской Невской Лавры.} он неоднократно повторял, что очень полезно просить Господа об извещении дня кончины. «Очень хорошо, "--- говорил он, "--- если кого Господь известит о приближающейся кончине, только эти извещения бывают почти всегда не точно определяемы, ради того, чтоб человек пребывал в непрестанном страхе. Святитель Тихон молил Господа: скажи мне, Господи, когда я умру? "--- Ему и сказано было: «в день недельный», но не сказано в какой именно. Значит и готовься на каждое воскресенье». "--- 23"~го апреля, день воскресный, недели св. Апостола Фомы, преосвященный пролежал весь день на кровати, по причине общего нездоровья. "--- На другой день, в понедельник, он писал настоятелю Николо"=Угрешского монастыря, Московской епархии, архимандриту Пимену, что он так слаб, что ждет смерти, "--- и далее говорит: «вчера (в воскресенье) весь день пролежал, ждал смерти, а сегодня опять брожу».

Еще в Страстную Седмицу владыка сказывал, что у него был маленький удар, но так как он не оставил никаких следов болезненности, то обстоятельство это не возбудило никаких серьезных опасений. 25"~го апреля удар повторился. Архимандрит Иустин просил благословения послать за доктором, но преосвященный с твердостью, в мирном и покойном настроении духа сказал решительно: «не надо» "--- повторил несколько раз: «мне так легко "--- хорошо!»

27"~го, в четверг, преосвященный просил одного из присных своих, иеромонаха Каллиста, потереть его сосновым маслом: по окончании натирания он просил прощения у Каллиста и сказал, что принял от него эту послугу в последний раз. На вопрос иеромонаха: разве ему не нравится масло? "--- отвечал: «нет, но дни мои сочтены».

28"~го, в пятницу, после обеда, преосвященный по обычаю лег отдыхать, но вскоре встал, приказал подать чаю. Келлейник Василий, заметив необыкновенную красноту лица, спросил о причине. Владыка объяснил это следствием слабого удара, который, хотя не произвел никакого особого повреждения, но вообще он чувствует себя столько не хорошо, что ожидает смерти. При этих словах, поразивших юношу скорбью и ужасом, первая мысль его выразилась вопросом: «как мне жить без вас, владыко, "--- ведь нынче очень трудно?» "--- Владыко ответил: «да, батюшка, очень, очень трудно, так трудно, что ты себе и представить не можешь; и я думал о тебе и предал как себя, так и тебя воле Божией». "--- Когда пришел к нему старший келлейник, заведовавший хозяйством его келлейным, иеродиакон Никандр, и предложил послать за доктором, то владыка отверг это предложение. Прежде он говорил не раз окружающим: «когда я буду умирать, не посылайте за доктором, дайте мне умереть, как следует христианину "--- во внимании, не смущая и не рассеевая меня вашей тревогой». Архимандрит Иустин сказывает, что еще в начале прошлой зимы, по поводу разговора о лице, подвергшемся параличному удару, владыка сказал: «и я умру ударом». Архимандрит начал было возражать, говоря, что при его телосложении, худобе и образе жизни это невероятно, но владыка, кратко подтвердив свои слова, переменил разговор. "--- К вечеру в пятницу владыка успокоился, и приказал на субботу приготовить ванну, но встав по утру довольно бодрым говорил, что ему лучше, и прибавив «а вчера чуть не умер», отменил распоряжение о ванне, сказав: «уж не нужно».

В эти последние дни жизни своей преосвященный был воодушевлен ко всем необыкновенною милостью, как бы растворенною жалостью. Эта милость и с нею неземная радость сияли на лице болящего. В один из этих последних дней владыка, прощаясь с келлейником своим, ответил на его поклон и прощание благоговейным поклоном до земли, сказав: «ты меня, батюшка, прости». Полный благолепного смирения вид старца тронул келлейника до слез. "--- В эти дни не раз говорил ему владыка, что «ему трудно низводить ум к земным занятиям», и уклоняясь от общения со всеми, он видимо уже не жил на земле.

30"~го апреля, воскресенье, недели Жен Мироносиц, к 7 часам утра, келлейник Василий, войдя в спальную\footnote{Владыка занимал всего две комнаты, приемную, она же и столовая, и кабинет, она же и спальная.} преосвященного, нашел орлец\footnote{\textit{Орлец} "--- коврик круглой формы, с вышитым на нем орлом, подлагаемый под ноги епископам при богослужении и молитве.} не убранным пред иконами, что случалось очень редко "--- большею частью преосвященный, всегда употреблявший его при келлейном правиле, сам убирал его. Умывшись, он, по обычаю, выпил Богоявленской воды и вышел в столовую комнату пить чай, приказав Василию \textit{скорее} убрать спальную. Выпив две чашки чаю, он поспешил в свою внутреннюю келлию. В исходе 8"~го часа, пред самым благовестом к поздней литургии, Василий, войдя к нему с обычною молитвою, нашел его лежащим на кровати, на левом боку, лицом к стене. Видя, что владыка, всегда очень чуткий, не обращает внимания на вход его, келлейник сначала приписал это особенно углубленному молитвенному деланию, что иногда с ним случалось. Постояв несколько, Василий повторил молитву, но "--- ответа не было. Вглядываясь пристальнее, он заметил, что рука владыки покрыта смертною бледностью; подошел ближе и убедился, что владыка уже скончался. Голова его, лежавшая на подушке, была несколько наклонена вперед, ладонь левой руки воздета кверху, как бы в молитве, правая рука, опущенная вдоль тела на кровать к стороне стены, лежала близь раскрытого канонника. Вообще благообразное положение тела было причиною, что келлейник не мог скоро решиться признать его уже отшедшим в вечность. Смерть, придя к святителю Христову, нашла ум его занятым молитвословием; начав оное на земле, он был призван к бесконечному славословию Бога "--- на небо.

Давно готовился и ждал епископ Игнатий прихода смерти, вооруженный непрестанною молитвою именем Господа Христа Иисуса, и смерть, побежденная Христом, почтила жизнь во Христе, прийдя к рабу Христову, сообразно с выраженным им его желанием, в тишине уединения, в час молитвы, при внимании углубленном в молитвословие; "--- избрала даже то положение телу, которое не нарушило бы \textit{благостояния} отходившего святителя, посвятившего всю жизнь свою духовным деланиям, заповеданным Господом: покаянию и плачу. Шедший этим путем, не мог не придти к блаженствам, обетованным Евангелием за эти добродетели.

Лице почившего епископа, по перенесении тела на стол, сияло радостью светлою, не земною. На левом виске заметна была синяя жилка, спускавшаяся около уха по щеке, полоской красноватого цвета, "--- вероятно след пути, которым смерть вошла в тело.

Беседуя с одним из близких ему учеников о заповедях Евангельских, сказал святитель Игнатий: «всякая явная добродетель, не моя добродетель, по учению Самого Господа, заповедовавшего всякое Евангельское добро делать в тайне». И точно, все величие всежизненного подвига его, в его неописанном объеме, осталось тайной его душевной клети, исповеданной и открытой, насколько то возможно было, в его сочинениях, но в полноте своей ведомой единому Богу. Этою таинственностью, отличительною чертою всей земной деятельности своей, по точному смыслу Евангельских заповедей, запечатлел преосвященный Игнатий и свой конечный, предсмертный подвиг. Сближая его поведание келлейнику об извещении свыше святителя Тихона о дне его кончины «в день недельный» с письмом преосвященного к архимандриту Пимену, что он все Фомино воскресенье пролежал, ожидая смерти, и наконец в день кончины (воскресенье), приказание его келлейнику поспешить скорейшею уборкою его спальни, наводят на мысль, что и ему был открыт день его кончины и определен подобно, как и святителю Тихону «днем недельным».

Для утешения нам, осиротевшим духовным чадам своим, владыка оставил определительное указание о земном пути своем, о том, куда стремился он жизнью, и куда "--- веруем "--- достиг. «Взят я, "--- говорит он в предисловии к 4"~му тому сочинений своих, "--- восхищен с широкого пути, ведущего к вечной смерти, и поставлен на путь тесный и прискорбный, ведущий в живот. Путь тесный имеет самое глубокое значение: он подъемлет с земли, выводит из омрачения суетою, возводит на небо, возводит в рай, возводит к Богу, поставляет пред Лице Его в незаходимый свет, для вечного блаженства».

Замечательно, что в этот же праздник, воскресенье недели Жен Мироносиц, скончался и преподобный Нил Сорский, известный делатель умной молитвы; это сходство дней кончины как бы подтверждает замечаемое сходство внутреннего подвига нашего современного скитянина, как по нраву, так и по плодам их, с основателем в древности скитского жительства в России. Все это, конечно, может быть знаменательным не для всех: но те, которые ведают молитвенное подвижничество святителя из личного с ним сопребывания, или пользуются его писаниями, не могут не слагать этого в сердце, в созидание священной для них памяти о своем духовном Отце и наставнике.

Трое суток стояло тело епископа Игнатия в келлиях его, неизменно сохраняя светлое выражение лица, затем оно стало припухать и было перенесено в соборную монастырскую церковь святителя Николая. По миновании 6 суток, 5"~го мая, в пятницу, совершена была заупокойная литургия и отпевание преосвященным Ионафаном, епископом Кинишемским, викарным Костромским. По его распоряжению, отпевание совершалось по чину служения пасхального; по окончании отпевания, он произнес надгробное слово и простился с почившим, за ним прощалось духовенство, сослужащие, монастырское братство и все присутствовавшие, во главе их начальник Костромской губернии Т. С. Дорогобужинов. Затем тело в открытом гробе было обнесено с крестным ходом кругом церкви святителя Николая и внесено в больничную монастырскую церковь св. Иоанна Златоустого и преподобного Сергия Радонежского, где после обычной литии закрыли крышу и гроб был опущен в склеп за левым клиросом.

По общему отзыву, отпевание усопшего произвело на всех впечатление скорее церковного торжества, чем печального обряда. Ученики владыки припоминали его слова: «можно узнать, "--- говорил он, "--- что почивший под милостью Божиею, если, при погребении тела его печаль окружающих растворена какою"=то непостижимою отрадою».

Хотя все предшествовавшие погребению дни собрание народа было довольно значительно, но в день погребения, не смотря на разлив Волги, затруднявший переправу в монастырь заречным жителям, стечение народа было до 5 тыс. человек.

\noprelistbreak\begin{quotation}

\textit{Блаженни яже избрал и приял еси, Господи! Память их в род и род!}

\end{quotation}


