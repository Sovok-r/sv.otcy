
\section[Статья 1-я. О Евангеліи и вѣрѣ.]{статья первая.\\\bfseries О Евангеліи и вѣрѣ.}

Хрістіанинъ отъ вѣры; вѣра отъ Евангелія начинается, какъ ниже сказано будетъ; того ради въ разсужденіи о Хрістіанствѣ, нужно прежде о Евангеліи и вѣрѣ вкратцѣ разсужденіе предложить.


\subsection[Глава 1-я. О Евангеліи.]{глава первая.\\\bfseries О Евангеліи.}

\begin{quotation}\textit{Духъ Господень на Мнѣ, Егоже ради помаза Мя, благовѣстити нищимъ посла Мя, исцѣлити сокрушенныя сердцемъ, проповѣдати плѣнникомъ отпущеніе и слѣпымъ прозрѣніе}, и проч.\footnote{Ис.~61,~1 и слѣд.; Лук.~4,~18 и слѣд.}\end{quotation}

\paragraph*{§\:277.} \textit{Евангеліе} есть слово греческое, и значитъ \textit{благопріятную и радостную вѣсть}, то"=есть, проповѣдаетъ Хріста Сына Божія, пришедшаго въ міръ грѣшники спасти, якоже во утѣшеніе грѣшникамъ Апостолъ Хрістовъ написалъ: \textit{вѣрно слово и всякаго пріятія достойно, яко Хрістосъ Іисусъ пріиде въ міръ грѣшники спасти}\footnote{1~Тим.~1,~15.}. И Самъ Хрістосъ о Себѣ объявилъ: \textit{пріиде Сынъ человѣческій взыскати и спасти погибшаго}\footnote{Лук.~19,~10.}. Какая убо пріятнѣйшая намъ грѣшникамъ можетъ быть вѣсть, какъ слышать туне проповѣдуемое отпущеніе грѣховъ, вмѣсто гнѣва Божія милость Божію, вмѣсто клятвы благословеніе Божіе, вмѣсто осужденія оправданіе, вмѣсто ада царствіе небесное отверстое, и съ Богомъ великимъ, святымъ и вѣчнымъ вѣчное примиреніе и общеніе\footnote{Римл.~5"~я; 1~Іоан.~1,~3,~7 и 9.}? Воистину не можетъ быть вожделѣннѣйшая грѣшникамъ вѣсть. Не такъ желаемо больнымъ здравіе, алчущимъ хлѣбъ, жаждущимъ холодная вода, заключеннымъ свобода, плѣннымъ отпущеніе, слѣпымъ и сѣдящимъ во тьмѣ свѣтъ, какъ грѣшникамъ отпущеніе грѣховъ и оправданіе, которые вѣчному Божію гнѣву и отверженію подлежали. Сіе значитъ \textit{Евангеліе}, хрістіане! Сію \textit{благопріятную вѣсть} приноситъ намъ Евангеліе! Симъ утѣшайся душа, печалію за грѣхи и страхомъ суда Божія сокрушенная: яко Богъ, по богатству благости Своея, всѣмъ грѣшникамъ, кающимся и вѣрующимъ во имя единороднаго Сына Его, отворилъ двери милосердія и вѣчнаго блаженства.

Что есть Евангеліе, лучше познается изъ сравненія закона съ Евангеліемъ. Правда, что закона и Евангелія творецъ единъ есть Богъ. И какъ законъ всѣмъ данъ, и всѣхъ одолжаетъ къ совершенному послушанію, тако и Евангеліе всѣмъ проповѣдати повелѣно, якоже писано есть: \textit{проповѣдите Евангеліе всей твари}\footnote{Марк.~16,~15.}. И Хрістосъ, во Евангеліи откровенный и всей твари проповѣданный, всѣхъ къ Себѣ призываетъ: \textit{пріидите ко Мнѣ вси труждающіися и обремененніи и Азъ упокою вы}\footnote{Матѳ.~11,~28.}. И какъ законъ на то данъ былъ, чтобъ человѣка, совершенно его исполнившаго, оправдать и спасти: \textit{сотворивый та} (въ законѣ писанная) \textit{человѣкъ живъ будетъ въ нихъ}\footnote{Лев.~18,~5; Гал.~3,~12.}; тако Евангелія конецъ есть вѣрующихъ оправдать, и проч. Однакожъ разнствіе между закономъ и Евангеліемъ примѣчается слѣдующее: 1)~\textit{Законъ}, то"=есть писанный, данъ былъ Моисеомъ (ибо на сердцахъ и у прародителей нашихъ, Адама и Евы, написанъ былъ): \textit{Евангеліе} Хрістомъ бысть, якоже Евангелистъ святый написалъ: \textit{законъ Моѵсеомъ данъ бысть; благодать же и истина Іисусъ Хрістомъ бысть}\footnote{Іоан.~1,~17.}. 2)~\textit{Законъ} даетъ заповѣди, что дѣлать и чего не дѣлать должны мы: \textit{Евангеліе} проповѣдуетъ отпущеніе грѣховъ, благодать Божію и заслуги Хрістовы. \textit{Тако возлюби Богъ міръ, яко и Сына Своего единороднаго далъ есть, да всякъ, вѣруяй въ Онь, не погибнетъ, но имать животъ вѣчный}\footnote{3,~16.}. 3)~\textit{Законъ} учитъ, что дѣлать и отъ чего отвращаться, но помощи къ дѣйствію не подаетъ: \textit{Евангеліе} обѣщаетъ благодать Святаго Духа, Которымъ на сердцахъ вѣрующихъ написуется законъ и дѣйствуется\footnote{Іер.~31,~33.}. 4)~\textit{Законъ} грѣхъ показуетъ, \textit{закономъ бо познаніе грѣха}\footnote{Римл.~3,~20.}, согрѣшившаго обличаетъ, обвиняетъ, устрашаетъ, гнѣвъ Божій возвѣщаетъ ему\footnote{4,~15.}, клятвою поражаетъ его и осуждаетъ\footnote{Гал.~3,~10.}, но того не отнимаетъ: \textit{Евангеліе} грѣхъ прикрываетъ, немощь грѣховную врачуетъ, гнѣва Божія боящихся ободряетъ и утѣшаетъ, благодать и вѣчный животъ вѣрующимъ обѣщаетъ. 5)~\textit{Законъ} надлежитъ до людей безстрашныхъ, неисправныхъ, небрежливыхъ, нечестивыхъ, которымъ ихъ должно устрашать, сокрушать и къ покаянію приводить. \textit{Уже и сѣкира при корени древа лежитъ: всяко убо древо, еже не творитъ плода добра, посѣкаемо бываетъ, и во огнь вметаемо}\footnote{Матѳ.~3,~10.}. \textit{Евангеліе} проповѣдуется боящимся, смущающимся, страхомъ Божіимъ и печалію за грѣхи сокрушеннымъ, милости Божіей и утѣшенія алчущимъ. Откуду Хрістосъ глаголетъ: \textit{благовѣстити нищимъ посла Мя, исцѣлити сокрушенныя сердцемъ}, и проч. Отсюду послѣдуетъ, что проповѣдники Божія слова должны опасно поступать въ своемъ ученіи, то"=есть: законъ проповѣдывать и гнѣвъ Божій возвѣщать тѣмъ, которые безстрашно живутъ; а тѣмъ людямъ, которыхъ примѣчаютъ сокрушенныхъ и страхомъ Божіимъ уязвленныхъ, такожде умирающимъ и кающимся Евангеліе, милость Божію и отсюду утѣшеніе предлагать, дабы безстрашныхъ въ большее безстрашіе, печальныхъ и смущенныхъ Божіимъ страхомъ въ пагубу отчаянія не привести, но паче чтобъ безстрашные въ страхъ и чувство пришли, сокрушенные же утѣшеніе живое въ сердцахъ почувствовали, и въ вѣрѣ утвердилися.

\paragraph*{§\:278.} Евангеліе святое, какъ есть \textit{смотрѣніе тайны, сокровенныя отъ вѣковъ въ Бозѣ}, по ученію Апостольскому\footnote{Еф.~3,~9; Римл.~14,~24 и 25; 1~Кор.~2,~7; Кол.~1,~26; 1~Петр.~1,~20.}, такъ отъ начала міра проповѣдуется. Ибо первому человѣку Адаму открылъ Богъ милостивое Свое о немъ и его потомкахъ, родѣ человѣческомъ, благоволеніе, которое чрезъ оныя слова, къ змію сказанныя отъ Бога: \textit{Той твою сотретъ главу} (\textit{Той} разумѣется Хрістосъ "--- благословенное Сѣмя), означается, по общему святыхъ отецъ и учителей церковныхъ разумѣнію\footnote{Быт.~3,~15.}. Сіе милосердое Божіе обѣщаніе, какъ теплый солнца лучь сквозь облаковъ, и какъ свѣтъ во тьмѣ сѣдящимъ, просіявало праотцамъ нашимъ Аврааму, Исааку и Іакову и прочіимъ, и согрѣвало ихъ сердца вѣрою и надеждою \textit{грядущаго} въ міръ Солнца правды "--- Хріста\footnote{12,~3 и проч.; 49,~10 и проч.}. Откуду Хрістосъ объявилъ Іудеямъ: \textit{Авраамъ отецъ вашъ радъ бы былъ, да бы видѣлъ день Мой}, то есть во плоти: \textit{и видѣ}, то"=есть вѣрою, \textit{и возрадовася}\footnote{Іоан.~8,~56.}. А что о Авраамѣ, тое и о прочіихъ праотцахъ и отцахъ, въ Ветхомъ Завѣтѣ пожившихъ и спасшихся, разумѣется, якоже въ Дѣяніяхъ Апостольскихъ пишется: \textit{благодатію Господа Іисуса Хріста вѣруемъ спастися якоже и они}, то"=есть, отцы наши\footnote{Дѣян.~15,~11.}. Сей благодати Божіей свѣтъ пророкамъ открывался, и ими проповѣданъ былъ. Моисей явно проповѣдалъ сынамъ Израилевымъ: \textit{Пророка отъ братіи твоея, якоже мене, возставитъ тебѣ Господь Богъ твой: Того послушайте}\footnote{Второз.~18,~15.}. Что и святый Стефанъ первомученикъ приводитъ\footnote{Дѣян.~7,~37.}. О томъ же и прочіи святые пророцы проповѣдали, которые пришествіе, рожденіе отъ Дѣвы по плоти, на земли пожитіе, страданіе, смерть, погребеніе, воскресеніе и славное на небо восшествіе Спасителя міру Хріста согласно предвозвѣстили и различно изобразили, какъ о томъ въ святыхъ книгахъ ихъ пространно написано. Откуду и Хрістосъ глаголетъ къ двумъ ученикамъ: \textit{о несмысленная и косная сердцемъ, еже вѣровати о всѣхъ, яже глаголаша пророцы! Не сія ли подобаше пострадати Хрісту, и внити во славу свою? И наченъ отъ Моѵсеа и отъ всѣхъ пророкъ, сказаше има отъ всѣхъ писаній, яже о Немъ}\footnote{Лук.~24,~25--27.}. И въ тойже главѣ глаголетъ: \textit{яко подобаетъ скончатися всѣмъ написаннымъ въ законѣ Моѵсеовѣ и пророцѣхъ и псалмѣхъ о Мнѣ}\footnote{ст.~44.}. Отсюду заключается правильно, что и въ Ветхомъ Завѣтѣ Евангеліе святое проповѣдывалося. Ибо все, что въ ономъ завѣтѣ ни обѣщало отпущеніе грѣховъ, наше съ Богомъ примиреніе и благодать отрожденія, тое надлежитъ до Хріста и Евангелія. И самыя жертвы оныя, яко прознаменовавшія Хріста, за спасеніе міра заклавшагося, Евангеліе значатъ. Ибо не сами въ себѣ жертвы оныя очищали грѣхи людскіе, но Хрістосъ, жертвами оными прознаменованный, очищалъ. \textit{Не возможно бо крови юнчей и козлей отпущати грѣхи}, глаголетъ Апостолъ\footnote{Евр.~10,~4.}. И \textit{Іисусъ Хрістосъ вчера и днесь, Тойже и во вѣки}\footnote{13,~8.}. И \textit{Агнецъ} (Іисусъ Хрістосъ) \textit{закланъ} въ жертвахъ \textit{отъ сложенія міра} (силою и дѣйствіемъ, не самымъ дѣломъ)\footnote{Апок.~13,~8.}. \textit{Егда же пріиде кончина лѣта}\footnote{Гал.~4,~4.}, явися отъ начала міра проповѣданный, и отъ всего міра не иначе, какъ дождь отъ земли жаждущей, ожиданный Мессія Хрістосъ, \textit{сый сіяніе славы Отчія}\footnote{Евр.~1,~3.}, и, какъ солнце, всю поднебесную Своимъ явленіемъ просвѣтилъ и согрѣлъ. Сей великій \textit{Посланникъ и Святитель исповѣданія нашего}\footnote{3,~3.}, \textit{и великаго совѣта Ангелъ}\footnote{Ис.~9,~6.}, возвѣстилъ намъ всю благость, милость и человѣколюбіе Божіе къ намъ, и призываетъ всѣхъ грѣшниковъ: \textit{покайтеся и вѣруйте во Евангеліе}\footnote{Марк.~1,~15.}; призываетъ всѣхъ къ Себѣ труждающихся и обремененныхъ: \textit{пріидите ко Мнѣ вси труждающіися и обремененніи}, и обѣщаетъ покой душамъ: \textit{и обрящете покой душамъ вашимъ}\footnote{Матѳ.~11,~28 и 29.}. Блудникамъ, мытарямъ, разбойникамъ и прочіимъ грѣшникамъ, приходящимъ къ Нему съ покаяніемъ и вѣрою, двери милосердія Божія и рай, грѣхомъ заключенный, отворилъ, и восхищаютъ его благодатію Его мытари и любодѣйцы и разбойники кающіися\footnote{21,~31; Лук.~23,~43.}. Сіе богатство благости Божіей въ весь міръ пронести повелѣлъ Апостоламъ Своимъ: \textit{шедше въ міръ весь, проповѣдите Евангеліе всей твари}\footnote{Марк.~16,~15.}. \textit{Они же изшедше, проповѣдаша всюду, Господу поспѣшствующу, и слово утверждающу послѣдствующими знаменми}\footnote{20.}. Которые согласно всѣ намъ благовѣствуютъ: \textit{проповѣдуемъ Хріста распята, Божію силу и Божію премудрость}\footnote{1~Кор.~1,~23 и 24.}, и молятъ насъ, яко посланники Хрістовы, примиритися съ Богомъ: \textit{по Хрістѣ убо молимъ, яко Богу молящу нами: молимъ по Хрістѣ, примиритеся съ Богомъ. Невѣдѣвшаго бо грѣха по насъ грѣхъ сотвори, да мы будемъ правда Божія о Немъ}\footnote{2~Кор.~5,~20 и 21.}.

\paragraph*{§\:279.} Видишь возлюбленный хрістіанине, что есть Евангеліе. А именно, есть проповѣдь о Хрістѣ, пришедшемъ въ міръ \textit{грѣшники спасти}. Комужъ оно проповѣдуется? Не тѣмъ грѣшникамъ, которые радуются о чести, славѣ и богатствѣ міра сего; не тѣмъ, которые утѣшаются различными міра сего веселостями, веселятся по вся дни свѣтло, день отъ дня препровождаютъ съ ликами и тимпанами; не тѣмъ, которые веселятся о любимицахъ своихъ и беззаконномъ ложѣ ихъ; не тѣмъ, которые утѣшаются о беззаконной побѣдѣ, озлобленіи ближняго своего; не тѣмъ, которые услаждаются о беззаконномъ и скверномъ, отъ лихоиманія, хищенія, насилія, коварно и хитро происходящемъ прибыткѣ, ни прочіимъ симъ подобнымъ беззаконникамъ. Имъ гнѣвъ Божій, грядущій на нихъ, возвѣщается, когда не обратятся. \textit{Не льстите себе: ни блудницы, ни идолослужители, ни прелюбодѣи, ни сквернители, ни малакіи, ни мужеложницы, ни лихоимцы, ни татіе, ни піяницы, ни досадители, ни хищницы, царствія Божія не наслѣдятъ}\footnote{1~Кор.~6,~9 и 10.}. Имъ глаголется: \textit{покайтеся}; глаголется: \textit{очистите руцѣ грѣшницы, исправите сердца ваша двоедушніи, постраждите и слезите и плачитеся; смѣхъ вашъ въ плачь да обратится, и радость въ сѣтованіе. Смиритеся предъ Господемъ}, и проч.\footnote{Іак.~4,~8--10.} А когда сіе сотворятъ, то имъ сія Божія благодать и утѣшеніе, какъ сладкая трапеза по постѣ и благословенныхъ трудахъ, представляется. "--- Кому убо возвѣщается Евангеліе? Отвѣщаетъ Хрістосъ: \textit{благовѣстити нищимъ посла Мя, исцѣлити сокрушенныя сердцемъ}. Вотъ кому проповѣдуется Евангельское утѣшеніе: \textit{нищимъ}, то"=есть, тѣмъ, которые духовную свою нищету признаютъ, и ищутъ богатства благости, милости и человѣколюбія Божія; которые, не находя никакія правды въ себѣ предъ Богомъ, но паче всякое усматривая окаянство, алчутъ и жаждутъ правды Божіей о Хрістѣ Іисусѣ, и тако смиряются предъ Нимъ; \textit{сокрушеннымъ сердцемъ}, то"=есть, которые печалію за грѣхи, какъ стрѣлою, уязвленное имѣютъ сердце. Симъ благовѣствуется, симъ глаголется: \textit{вѣруйте во Евангеліе}. Симъ открывается всякое человѣколюбія Божія богатство. \textit{Благословенъ Богъ и Отецъ Господа нашего Іисуса Хріста, Отецъ щедротъ и всякія утѣхи, утѣшаяй насъ о всякой скорби нашей}\footnote{2~Кор.~1,~3 и 4.}.

\paragraph*{§\:280.} Законъ, хотя и того ради данъ былъ, чтобы оправдать человѣка, какъ сказано выше, однакожъ не могъ человѣка оправдать, не отъ себе, но отъ немощи человѣческой; яко никто не могъ его совершенно исполнить, якоже писано есть: \textit{вси согрѣшиша, и лишени суть славы Божія}\footnote{Римл.~3,~23.}; "--- и тако съ оправданіемъ своимъ отступилъ отъ насъ. \textit{Съ оправданіемъ}, глаголю; ибо что надлежитъ до творенія его, то и нынѣ его творить должны хрістіане, ему повиноваться, по правилу и ученію его житіе свое провождать. Иначе, кто не хощетъ по правилу его жить, клятвы и осужденія, въ законѣ объявленныя, не избѣжитъ. А когда законъ съ оправданіемъ отъ насъ отсталъ: \textit{яко не оправдится отъ дѣлъ закона всяка плоть}\footnote{Гал.~2,~16.}, яко не могли его исполнить, тѣмъ самымъ отсылаетъ насъ къ Евангелію и, яко немощныхъ, поручаетъ Хрісту, въ Евангеліи проповѣдуемому, Который \textit{туне безъ дѣлъ закона, оправдаетъ и спасаетъ вѣрующихъ въ Него}\footnote{Римл.~3,~24--28.}. Въ семъ, кажется, разумѣ Апостолъ написалъ: \textit{законъ пѣстунъ намъ бысть во Хріста, да отъ вѣры оправдимся}\footnote{Гал.~3,~24.}. Ибо кромѣ того что научаетъ, что дѣлать и чего не дѣлать, и тако отъ грѣховъ тщится отвести насъ, "--- какъ бы за руку вземши, ведетъ ко Хрісту насъ, яко неисправныхъ и немощныхъ; яко, чему учитъ насъ, того не дѣлаемъ, и тако обличаетъ насъ и осуждаетъ, а не оправдаетъ; устрашаетъ, а не утѣшаетъ; немощь нашу намъ показуетъ, а не отнимаетъ; а тѣмъ самымъ какъ бы убѣждаетъ насъ искать другаго посредствія, чрезъ которое бы избавиться намъ отъ окаянства нашего. Въ такой совѣсти тѣснотѣ убѣждаемся къ милостивому Божіему о Хрістѣ Іисусѣ обѣщанію, къ святому Его Евангелію и Хрісту, во Евангеліи откровенному и проповѣдуемому, прибѣгать, Который не сотворившихъ закона, но вѣрующихъ въ Него \textit{туне оправдаетъ} и всего избавляетъ бѣдствія и окаянства, якоже Самъ глаголетъ: \textit{аще Сынъ вы свободитъ, воистину свободни будете}\footnote{Іоан.~8,~36.}, "--- да тако, отъ закона оправдитися не могучи, \textit{отъ вѣры оправдимся}\footnote{Гал.~3,~24.}. Ибо Хрістосъ осужденіе закона, которому мы, яко грѣшники и преступники закона, подлежали, на Себе взялъ, да намъ благословеніе подастъ; и \textit{грѣхи наши Самъ вознесе на тѣлѣ Своемъ на древо}\footnote{1~Петр.~2,~24.}, да намъ подастъ правду Свою вѣрою, якоже Апостолъ написалъ: \textit{Богъ невѣдѣвшаго грѣха по насъ грѣхъ сотвори, да мы будемъ правда Божія о Немъ}\footnote{2~Кор.~5,~21.}. О семъ пространнѣе ниже предложится. "--- Отъ вышереченныхъ послѣдуетъ: 1)~Евангеліе тоежде было отъ начала міра, которое нынѣ проповѣдуется. 2)~Таяжде вѣра была отъ начала міра, которая нынѣ проповѣдуется, яко Ветхаго Завѣта отцы вѣровали въ \textit{грядущаго} Хріста, а мы вѣруемъ въ \textit{пришедшаго}. 3)~Тойжде образъ оправданія и спасенія у нихъ былъ, какой у насъ есть, то"=есть, \textit{вѣра во Хріста}. Ибо въ предвѣчномъ Божіемъ совѣтѣ положено "--- человѣка согрѣшившаго спасти не иначе, какъ только чрезъ Хріста, \textit{вѣрою въ Него}, и что опредѣлено тогда, тое нынѣ дѣлается, и спасаются вѣрою, которые ни спасаются. Ибо Евангеліе есть \textit{смотрѣніе тайны сокровенныя отъ вѣковъ въ Бозѣ}, какъ сказано прежде. 4)~Какъ хрістіанинъ отъ вѣры, такъ вѣра отъ Евангелія начинается. Безъ Евангелія бо не можетъ быть вѣра; надобно бо вѣрить милостивому Божію обѣщанію, которое до Евангелія надлежитъ, когда хощемъ вѣру въ сердце воспріяти. 5)~Чтобы вѣру въ сердце воспріять отъ Евангелія, должно почувствовать въ сердцѣ силу, обличеніе, осужденіе и страхъ суда Божія отъ закона; чего ради нужна есть проповѣдь закона Божія. 6)~Отсюду видно, какъ нужно есть поучатися въ Словѣ Божіемъ, то"=есть, въ законѣ и Евангеліи, яко отъ закона \textit{познается грѣхъ} и ощущается гнѣвъ Божій противу грѣха\footnote{Римл.~3,~20.}: отъ Евангелія утѣшеніе и живая вѣра въ сердце вселяется силою Духа Святаго. 7)~Евангеліе неисправнымъ, некающимся и непрестающимъ отъ грѣха, ничего не пользуетъ; яко тѣмъ только благовѣствуется, которые Бога и суда Его праведнаго боятся, и страхомъ и печалію за грѣхи уязвлены. Ибо симъ утѣшеніе отъ Евангелія нужно и предлагается; симъ глаголется: \textit{не бойся, токмо вѣруй}; симъ Хрістосъ, во Евангеліи проповѣдуемый, Утѣшитель и Врачъ есть, якоже Самъ глаголетъ: \textit{не требуютъ здравіи врача, но болящіи}\footnote{Матѳ.~9,~12.}.

\subsection[Глава 2-я. О вѣрѣ.]{глава вторая.\\\bfseries О вѣрѣ.}

\begin{quotation}\textit{Праведный отъ вѣры живъ будетъ}\footnote{Аввак.~2,~4; Римл.~1,~17; Гал.~3,~11; Евр.~10,~38.}.\end{quotation}
\begin{quotation}\textit{Есть же вѣра уповаемыхъ извѣщеніе, вещей обличеніе невидимыхъ}\footnote{11,~1.}.\end{quotation}

\paragraph*{§\:281.} Вѣра есть тое, чего не видимъ, или умомъ не постигаемъ, но вѣруемъ тако быти. Тако не видимъ Бога (хотя изъ созданія міра познаемъ Его, но познаніе тое, яко несовершенное, вѣра совершаетъ, и такъ болѣе отъ вѣры, нежели отъ разума познаемъ), но вѣруемъ, что есть Богъ, есть единъ. Не постигаемъ умомъ, како Богъ \textit{единъ естествомъ}, но \textit{троиченъ въ лицахъ}, но вѣруемъ, словомъ Божіимъ наставляеми. Не видимъ Хріста Сына Божія, вѣчнаго живота, небеснаго царствія, но вѣруемъ. Не постигаемъ умомъ воскресенія мертвыхъ, въ святѣйшей Евхаристіи Тѣла и Крови Хрістовой, но вѣра утверждаетъ насъ въ томъ. Въ сихъ и прочіихъ тайнахъ Божіихъ разумъ вѣрѣ послѣдовати долженъ, а не вѣра разуму; то"=есть, познаніе ихъ не отъ разума зависитъ, но отъ вѣры, и не того ради вѣруемъ, понеже познаемъ, но того ради познаемъ, понеже вѣруемъ; и такъ познаніе ихъ не есть разума нашего плодъ, но вѣры, которая утверждается на твердомъ святаго Божія слова основаніи и истинѣ и всемогуществѣ Божіи. Сіе на такій конецъ предлагается, дабы не испытовати умомъ того, что единыя вѣры требуетъ, но \textit{плѣнять разумъ въ послушаніе Хрістово}\footnote{2~Кор.~10,~5.}. Ибо знаменіе невѣрія извѣстнѣйшее есть вопрошати о тайнахъ Божіихъ: како сіе можетъ быть? Откуду святый Златоустъ поучаетъ: «идѣже вѣра, не потребно есть изстязаніе; идѣже не подобаетъ испытовати, какая потреба истязанія? Истязаніе вѣры есть разорительно. Ибо истязуяй не обрѣте»\footnote{Бес.~1"~я на 1"~е посл. къ Тим.}. Ибо ежелибъ умомъ постигали мы, то бы не нужна намъ была и вѣра. Вѣра бо тамъ нужна, гдѣ умъ нашъ не постигаетъ. Но сія вѣра, то"=есть, едино вѣрою познаніе таинъ святыхъ и догматовъ православныхъ, не довлѣетъ ко спасенію, какъ ниже о томъ предложится.

\paragraph*{§\:282.} \textit{Начало} евангельской \textit{спасающей} вѣры. Сказано выше, что закона собственно есть обличать, немощь показывать и гнѣвомъ Божіимъ устрашать и осуждать законопреступниковъ; Евангелія же дѣло есть: немощь врачевать, Хріста "--- Врача показывать, утѣшать, страхъ отнимать и благодать Хрістову обѣщати. Откуду нужно есть, чтобъ евангельская вѣра въ сердцѣ зачалась, изъ закона познать немощь свою, гнѣвъ Божій, клятву, судъ и осужденіе грѣшникамъ слѣдующее почувствовать въ сердцѣ, и тако вѣра въ сердцѣ, симъ страхомъ, какъ огнемъ, очищенномъ и предуготованномъ, отъ Духа Святаго зачинается. Ибо вѣра сія есть даръ Духа Святаго. Сего ради Предтеча святый, уготовляя путь Хрісту Сыну Божію, Который имѣлъ съ Евангеліемъ Своимъ изыти, началъ проповѣдь свою отъ закона, которымъ обличалъ грѣхи человѣческіе, и судомъ Божіимъ устрашалъ сердце ихъ. \textit{Уже и сѣкира при корени древа лежитъ: всяко убо древо, еже не творитъ плода добра, посѣкаемо бываетъ, и во огнь вметаемо}\footnote{Матѳ.~3,~10.}. Откуду и вопрошали его народы глаголюще: \textit{что сотворимъ}\footnote{Лук.~3,~10,~12 и 14.}? Тако и въ Дѣяніяхъ Апостольскихъ пишется, что народъ, по проповѣди Апостола Петра, которою доказывалъ Іисуса распятаго быть Мессію Хріста, Отцемъ обѣщаннаго, и приложилъ: \textit{твердо убо да разумѣетъ весь домъ Израилевъ, яко и Господа и Хріста Его Богъ сотворилъ, сего Іисуса, Егоже вы распясте}, слышавше умилишася сердцемъ, и рѣша Петру и прочіимъ апостоломъ: \textit{что сотворимъ, мужіе братіе}\footnote{Дѣян.~2,~14--36 и 37.}? Узнали грѣхъ свой, почувствовали отъ того и гнѣвъ Божій, и тако совѣта требовали: \textit{что сотворимъ, мужіе братіе? И иже убо любезно пріяша слово его}, какъ таможде пишется, \textit{крестишася}\footnote{ст.~41.}. Тако законъ предуготовляетъ путь къ Евангелію и ведетъ ко Хрісту, Который во Евангеліи открывается, яко Врачъ немощныхъ, печальныхъ Утѣшитель, грѣшныхъ Спаситель, отчаявающихся въ себѣ и своихъ силахъ Надежда. Отсюду слѣдуетъ, что вѣра спасающая евангельская есть не иное что, какъ утѣшительное Евангелія воспріятіе, въ сердцѣ содѣловаемое Духомъ Святымъ; или есть сердечное упованіе и несумнѣнная надежда благодати Божія, туне, ради Хріста обѣщанныя, то"=есть, отпущенія грѣховъ и вѣчнаго живота. Сего ради начинающій вѣровати не неприлично уподобитися можетъ \textit{немощному}, который, видя свою неисцѣльную болѣзнь, желаетъ и ищетъ искуснаго врача: тако грѣшникъ, видя отъ закона грѣховную свою немощь, отъ которой своею силою никакъ свободиться не можетъ, желаетъ и ищетъ врача, который бы отъ той немощи его могъ свободить. Таковый открывается ему во Евангеліи "--- Хрістосъ, къ Которому прибѣгаетъ и ищетъ отъ Него исцѣленія, и милостивно исцѣляется. Паки уподобляется \textit{плѣненному}, который ищетъ избавителя: тако грѣшникъ, плѣненный отъ діавола, отъ котораго избавитися своею хитростію и разумомъ никакъ не можетъ, ищетъ и желаетъ избавителя, каковый Избавитель представляется ему во Евангеліи "--- Хрістосъ, якоже Самъ глаголетъ: \textit{аще Сынъ вы свободитъ, воистину свободни будете}\footnote{Іоан.~8,~36.}, "--- Котораго признавая за Избавителя своего, свобождается отъ ига діавольскаго. Паки уподобляется \textit{убѣгающему отъ страха и ищущему безопаснаго мѣста и защищенія}: тако грѣшникъ, убѣжати желая отъ страха суда Божія, осужденія и ада, ищетъ защищенія и мѣста безопаснаго, но не можетъ, ибо никуды отъ суда Божія укрытися невозможно. Таковое мѣсто убѣжища открывается ему во Евангеліи "--- Хрістосъ, въ которое вѣрою и сокрывается безопасно отъ суда Божія и гнѣва, на всю вселенную грядущаго. Паки уподобляется \textit{малому отроку}, который нагъ стоитъ предъ матерію своею и проситъ одѣянія: тако грѣшникъ, обнаженъ одежды правды и спасенія, стоитъ и ищетъ прикрытія наготѣ своей, но не можетъ сыскати. Таковая одежда, \textit{одежда оправданія}, открывается и показуется ему во Евангеліи, которая есть правда Хрістова, которую вѣрою туне пріемлетъ, и прикрывается нагота его и срамота. Откуду Хрістосъ Сынъ Божій представляется намъ во Евангеліи \textit{Врачь}, яко врачуетъ немощи наша; \textit{Избавитель и Искупитель}, яко кровію Своею искупляетъ и избавляетъ насъ, вѣрующихъ въ Него, отъ діавола, грѣха, ада и прочаго бѣдствія; \textit{градъ и мѣсто прибѣжища и Ходатай}, яко къ Нему прибѣгая вѣрою, спасаемся отъ гнѣва Божія, и \textit{Его Ходатая имамы ко Отцу}\footnote{1~Іоан.~2,~1; Римл.~8,~34.}; \textit{оправданіе наше}, яко Его правдою прикрываемся и оправдываемся, да не наги и скверны явимся предъ святѣйшимъ лицемъ Божіимъ, о чемъ ниже предложится яснѣе. Откуду Самъ Хрістосъ о Себѣ глаголетъ: \textit{благовѣстити нищимъ посла Мя} Отецъ небесный, \textit{исцѣлити сокрушенныя сердцемъ, проповѣдати плѣненнымъ отпущеніе и слѣпымъ прозрѣніе, и отпустити сокрушенныя во отраду}\footnote{Лук.~4,~18.}. Отъ сего видно: 1)~Что никто не можетъ оправдатися предъ Богомъ и спастися безъ Хріста и кромѣ Хріста, но токмо вѣрою во Хріста: \textit{нѣсть бо иного имени подъ небесемъ, даннаго въ человѣцѣхъ, о немже подобаетъ спастися намъ}, глаголетъ Апостолъ\footnote{Дѣян.~4,~12.}. Ибо никто не можетъ отъ діавола, грѣха, клятвы законныя и ада избавитися безъ Хріста, что все въ краткомъ семъ словѣ Хрістовомъ заключается: \textit{аще Сынъ вы свободитъ, воистину свободни будете}. Къ сему убо должно намъ вѣрою прибѣгать, яко \textit{граду убѣжища}, когда хощемъ убѣжать отъ гнѣва Божія; яко \textit{врачу}, когда хощемъ отъ язвъ грѣховныхъ исцѣлитися. Сего просить вѣрою, дабы избавилъ насъ отъ плѣненія и ига діавольскаго. Къ сему подобаетъ возводити очи наши, да наготу и срамоту нашу прикрыетъ ризою правды Своея, да не явимся наги и осквернени грѣхами предъ чистѣйшими очами небеснаго Его Отца. 2)~Не можетъ быть истинное покаяніе безъ познанія, ощущенія и признанія немощи, грѣха, бѣдности и окаянства своего, что все изъ закона и клятвы законныя познается и ощущается. Чего ради всякому, хотящему истинно каятися, должно приникнуть въ законъ и поставить совѣсть свою противу закона, и разсмотрѣть, въ чемъ ее законъ обличаетъ, уязвляетъ и утѣсняетъ; въ такой тѣснотѣ, то"=есть, чувствуя законную клятву и гнѣвъ Божій противу содѣянныхъ грѣховъ, вѣрою прибѣгать ко Хрісту и искать милостиваго избавленія отъ гнѣва Божія и казни, послѣдующія гнѣву. 3)~Не можетъ быть истинная евангельская вѣра безъ познанія, ощущенія и признанія своея бѣдности и окаянства. Ибо, чтобы утѣшеніе евангельское почувствовать, должно прежде страхъ и печаль имѣть, что бываетъ отъ познанія бѣдности и окаянства своего. Утѣшеніе бо печальныхъ утѣшеніе есть, а не веселыхъ. А гдѣ живое чувствуется утѣшеніе чрезъ слово Евангелія, тамо и вѣра мѣсто свое имѣетъ. Вѣра бо евангельская не можетъ быть безъ утѣшенія, какъ ниже скажется. 4)~Какъ нужна проповѣдь святаго закона Божія, чтобы безстрашные въ страхъ пришли, очувствовалися и спасительную печаль возымѣли; и проповѣдь Евангелія, чтобы сердца, страхомъ гнѣва Божія сокрушенныя, вѣру воспріяли и живое утѣшеніе въ себѣ почувствовали, который долгъ пастырямъ Хрістова стада необходимо надлежитъ. 5)~Отъ сего видно, ради чего надмѣнные \textit{мнимою} своею правдою не требуютъ и не пріемлютъ Евангелія, яко не познаютъ и не признаютъ своея немощи и окаянства. Напротивъ того, явные грѣшники удобнѣйшіе бываютъ къ тому, ибо видятъ свою бѣдность и окаянство. Откуду пишется, что \textit{бяху приближающеся Ему} (Хрісту) \textit{вси мытаріе и грѣшницы, послушати Его}; напротивъ того, \textit{роптаху} на Хріста \textit{книжники} и \textit{фарисеи}, мнимою своею святостію надмѣнные, \textit{глаголюще, яко Сей грѣшники пріемлетъ, и съ ними ястъ}\footnote{Лук.~15,~1 и 2.}. Ибо кто не видитъ своея немощи и не признаетъ, тотъ и врача не ищетъ, но единіи болящіи, которые болѣзнуютъ сердцемъ, сердце стрѣлою грѣха и печали уязвленное имѣютъ. \textit{Не требуютъ бо здравіи врача, но болящіи, глаголетъ Хрістосъ}\footnote{Мѳ.~9,~12.}. 6)~Вѣра святая евангельская въ сердцѣ свое мѣсто имѣетъ. Какъ бо грѣхъ, страхъ и печаль сердце занимаютъ и сокрушаютъ, такъ и благодать Хрістова, которая грѣхъ, страхъ и печаль отъемлетъ, надежду и утѣшеніе содѣловаетъ, въ сердце вѣрою вселяется. Откуду всякому должно тщиться вѣру имѣть въ сердцѣ, а не токмо въ разумѣ и на языкѣ. Вѣра сія удобнѣе познается отъ свойствъ и дѣйствій своихъ, о которыхъ въ слѣдующемъ параграфѣ объяснится.

\paragraph*{§\:283.} \textit{Свойства и дѣйствія вѣры спасающей}. 1)~Сія вѣра всѣ благодѣянія Божія, всему міру отъ Бога вообще показанныя, себѣ привлекаетъ и присвояетъ. Она научаетъ вѣрующаго отъ сердца глаголати: \textit{Вѣрую во единаго Бога Отца}, Который какъ всѣхъ, такъ и мене изъ ничего создалъ, изъ небытія въ бытіе привелъ, душу и тѣло, жизнь и дыханіе мнѣ подалъ, согрѣшившаго и погибшаго мене чрезъ единороднаго Сына Своего взыскалъ, и двери небеснаго царствія заключенныя мнѣ отверзлъ. \textit{Вѣрую во единаго Господа} моего \textit{Іисуса Хріста Сына Божія}, Который мене заблудшаго взыскалъ, погибшаго спаслъ отъ грѣховъ, клятвы, смерти, ада и діавола, избавилъ не сребромъ, ни златомъ, ни иною какою цѣною, но честною Своею кровію, неповинною Своею смертію и страданіемъ, и мене удаленнаго и отъ лица Божія отверженнаго присвоилъ и привелъ въ Свое царство, въ которомъ вѣрніи Его, кровію Его оправданніи, находятся и чаютъ воскресенія мертвыхъ и жизни будущаго вѣка. \textit{Вѣрую въ Духа Святаго Господа животворящаго}, Который банею пакибытія мене отродилъ и обновилъ \textit{служити Богу преподобіемъ и правдою предъ Нимъ вся дни живота моего}\footnote{Лук.~1,~15.}; Который мене духовно оживотворяетъ, просвѣщаетъ, наставляетъ на путь правый и на путь спасенія, подаетъ мнѣ благодатныя движенія въ сердцѣ моемъ и помоществуетъ мнѣ въ подвигѣ противу грѣха и діавола, отводитъ отъ грѣха и научаетъ ходить достойно званія хрістіанскаго, и проч. И сіи великія и высокія милости подаетъ мнѣ человѣколюбивый Богъ, въ тріехъ Лицахъ мною вѣруемый и покланяемый, не по заслугамъ моимъ, но туне, по единой Своей безприкладной благости и человѣколюбію. Сего ради всякому, истинную вѣру въ сердцѣ имущему, должно со святымъ Апостоломъ Павломъ держати и исповѣдати: \textit{вѣрою живу Сына Божія, возлюбившаго мене и предавшаго Себе по мнѣ}\footnote{Гал.~2,~20.}. Сынъ Божій весь міръ возлюбилъ, и предалъ Себе за весь міръ; но Павелъ святый сію высокую Его любовь и благодѣяніе себѣ вѣрою присвояетъ. Сіе и на прочіихъ святаго Писанія мѣстахъ примѣчается, и въ церковныхъ пѣсняхъ и стихахъ премногихъ видѣть можно поющему и читающему со вниманіемъ, како вѣрная душа великое сіе человѣколюбія Божія дѣло себѣ присвояетъ. Дамаскинъ святый поетъ Хрісту: «Спаслъ еси всего мя человѣка»\footnote{Глас.~2"~го пѣснь 4"~я.}. И паки: «до мене идеши, мене ища заблуждшаго»\footnote{тамъ же.}. И паки: «Ты моя крѣпость, Господи! Ты моя и сила» и проч.\footnote{Пѣснь 4"~я гласа 8"~го.} Тако вѣрная душа присвояетъ благодѣянія Божія и заслуги Хрістовы себѣ, которыя человѣколюбивый Богъ всей твари вообще показалъ и показуетъ. Отсюду послѣдовательно въ сердцѣ вѣрующаго востаетъ горячая къ Богу любовь, благодареніе, прославленіе имени Его святаго, и тако вѣра самохотно творитъ добрыя дѣла. "--- 2)~Вѣра спасительная, что вѣруетъ, то небоязненно, гдѣ должно и нужно, исповѣдуетъ, какъ учитъ Апостолъ: \textit{сердцемъ вѣруется въ правду, усты же исповѣдуется во спасеніе}\footnote{Римл.~10,~10.}. Откуду Давидъ святый глаголетъ о себѣ: \textit{вѣровахъ, тѣмже возглаголахъ}\footnote{Пс.~115,~1.}. И Апостолъ: \textit{и мы вѣруемъ, тѣмже и глаголемъ}\footnote{2~Кор.~4,~13.}. Отсюду востаетъ небоязненное засвидѣтельствованіе истины предъ врагами истины; отсюду ревность по Бозѣ, наипаче гдѣ слава имени Его требуетъ. Отсюду то бываетъ, что \textit{вси, хотящіи благочестно жити о Хрістѣ Іисусѣ, гоними будутъ}\footnote{2~Тим.~3,~12.}; ибо міръ не любитъ истины, которую вѣрніи раби Божіи свидѣтельствуютъ. Отсюду толико тысящъ мучениковъ и исповѣдниковъ просіяло, которые не только устами, но и изліяніемъ крови своея истину предъ врагами Божіими засвидѣтельствовали. Отсюду Хрістосъ глаголетъ: \textit{мните ли, яко миръ пріидохъ дати на землю? ни, глаголю вамъ, но раздѣленіе. Будутъ бо отселѣ пять въ единомъ дому раздѣлени, тріе на два, и два на три. Раздѣлится отецъ на сына, и сынъ на отца; мати на дщерь, и дщерь на мать, свекры на невѣсту свою, и невѣста на свекровь свою}\footnote{Лук.~12,~51--53.}. Отсюду бываетъ, что нечестивый мужъ благочестивую жену свою и нечестивая жена благочестиваго мужа своего, нечестивый отецъ благочестиваго сына и дщерь свою, нечестивая свекровь благочестивую невѣстку свою, нечестивый братъ благочестиваго брата своего, и нечестивая сестра благочестивую сестру, нечестивый сосѣдъ благочестиваго сосѣда, и всякъ нечестивый благочестиваго ненавидитъ, гонитъ и озлобляетъ; понеже благочестивый правду и истину словомъ и дѣломъ свидѣтельствуетъ, и обличаетъ неправду и ложь, чего нечестивые не любятъ. 3)~Вѣра сія имѣетъ дерзновеніе и упованіе на милосердіе Божіе, человѣколюбіе и помощь. Понеже высоко почитаетъ благодатныя Божія обѣщанія, которыми обѣщалъ милость, помощь и избавленіе всѣмъ, вѣрующимъ въ Него и вѣрою призывающимъ Его, подати. Она вѣруетъ несумнѣнно обѣщавшему: \textit{всякъ, иже аще призоветъ имя Господне, спасется}\footnote{Римл.~10,~13.}; и паки: \textit{призови Мя въ день скорби твоея, и изму тя}\footnote{Пс.~49,~15.}. Откуду вѣра называется \textit{уповаемыхъ извѣщеніе}, по ученію Апостола\footnote{Евр.~11,~1.}. Ибо невидимая какъ видимая, и будущая какъ настоящая зритъ, якоже о святомъ Авраамѣ, отцѣ вѣрныхъ, пишется: \textit{извѣстенъ бывъ, яко, еже обѣща Богъ, силенъ есть сотворити}\footnote{Римл.~4,~21.}. И Хрістосъ глаголетъ къ разслабленному: \textit{дерзай чадо, отпущаются ти грѣси твои}\footnote{Мѳ.~9,~2.}. Тако милосердіемъ Божіимъ ободряетъ и всякаго вѣрнаго вѣра, въ сердцѣ его живущая: \textit{дерзай}, яко Богъ \textit{милостивъ} есть, и потому помилуетъ тя; \textit{истиненъ} есть, и потому исполнитъ Свое обѣщаніе, услышитъ тя призывающаго; \textit{всесиленъ} есть, и потому можетъ избавити тебе. Отсюду бываетъ, что вѣрный съ горячимъ любви дерзновеніемъ Бога \textit{своимъ} Богомъ называетъ: \textit{Боже, Боже мой, къ Тебѣ утренюю}\footnote{Пс.~62.}. \textit{Боже мой, милость моя}\footnote{58,~18.}; называетъ своею крѣпостію, утвержденіемъ, прибѣжищемъ, и проч.: \textit{возлюблю Тя, Господи, крѣпосте моя! Господь утвержденіе мое, и прибѣжище мое, избавитель мой, Богъ мой, помощникъ мой, и уповаю на Него; защититель мой, и рогъ спасенія моего}\footnote{Пс.~17,~2 и 3.}. Таковое вѣры дерзновеніе изображается и во псалмѣ 90"~мъ: \textit{живый въ помощи Вышняго}, и проч. Тако Дамаскинъ, духомъ радуяся и играя, поетъ въ своихъ къ Богу пѣсняхъ: «Ты моя крѣпость, Господи! Ты моя и сила! Ты мой Богъ! Ты мое радованіе!» и проч.\footnote{Пѣснь 4"~я гласа 8"~го.} Отсюду въ сердцахъ вѣрныхъ радость и веселіе духовное востаетъ, такъ что не токмо готовыми себе къ страданію за Бога показываютъ, но и съ радостію воспріимаютъ оное, якоже пишется о апостолѣхъ: \textit{они же убо идяху радующеся отъ лица собора, яко за имя Господа Іисуса сподобишася безчестіе пріяти}\footnote{Дѣян.~5,~41.}. Радость сія преизобильно растетъ и въ духовное играніе и воскликновеніе сердца ихъ претворяетъ, когда взираютъ вѣрою на будущую славу, скорбь терпящимъ и Бога любящимъ уготованную, поминаютъ утѣшительное Хрістово слово: \textit{радуйтеся и веселитеся, яко мзда ваша многа на небесѣхъ}\footnote{Мѳ.~5,~12.}. 4)~Вѣра очищаетъ сердце вѣрующаго отъ всего, что къ міру принадлежитъ, то"=есть, отъ пристрастія къ чести, славѣ, богатству, сладострастію, и отъ всякаго грѣховнаго обычая. Ибо неравное съ неравнымъ, напр. огнь съ водою не помѣщается; тако вѣра, яко даръ небесный, съ земными и тлѣнными не помѣщается, но всегда къ небеси стремится и вѣрующаго сердце восхищаетъ туды, откуду произошла, ищетъ невидимыхъ и вѣчныхъ, и того ради все, къ міру сему надлежащее, какъ суету, или какъ прелесть, вмѣняетъ. Откуду вѣрнымъ повелѣвается: \textit{не любите міра, ни яже въ мірѣ}\footnote{1~Іоан.~2,~15.}. Отсюду въ вѣрныхъ востаютъ воздыханія и гласы плачевные: \textit{увы мнѣ, яко пришельствіе мое продолжися}\footnote{Пс.~119,~5.}. Востаютъ желанія святыя: \textit{Боже, Боже мой, къ Тебѣ утренюю: возжада Тебѣ душа моя}, и проч.\footnote{62,~2.} \textit{Имже образомъ желаетъ елень на источники водныя: сице желаетъ душа моя къ Тебѣ, Боже}, и проч.\footnote{41,~2.} \textit{Коль возлюбленна селенія Твоя, Господи силъ! Желаетъ и скончавается душа моя во дворы Господни}, и проч.\footnote{83,~2 и 3.} Ибо вѣрніи, истинніи хрістіане суть пришельцы и странники въ мірѣ семъ; почему отечество ихъ, домъ ихъ, богатство, честь, слава, наслѣдіе и вся благая не суть въ мірѣ семъ, но на небеси, гдѣ и Отецъ ихъ, къ Которому воздыхаютъ и молятся: \textit{Отче нашъ, Иже еси на небеси!} и проч. И какъ отлучившійся отъ отечества и дома своего человѣкъ всегда помышляетъ о домѣ и родителяхъ своихъ, и сердцемъ и желаніемъ своимъ къ тому стремится: тако вѣрніи раби Божіи, доколѣ въ мірѣ семъ и въ тѣлѣ живомъ, отходятъ отъ Господа \textit{(вѣрою бо ходятъ, а не видѣніемъ)}; того ради, \textit{воздыхаютъ, въ жилище свое небесное облещися желающе}\footnote{2~Кор.~5,~2 и 7.}. 5)~Вѣра сія отвлекаетъ человѣка вѣрующаго отъ надежды на созданіе, на князей, сыновъ человѣческихъ, на честь, богатство, силу свою, благочестіе свое, на себе самого, и отъ всего того, что кромѣ Бога есть; но на единаго Бога надѣятися, въ Немъ единомъ почивати, отъ Него единаго милости, защищенія, избавленія и спасенія искати и надѣятися научаетъ. Надежда бо отъ вѣры неотлучна, но съ вѣрою всегда пребываетъ; и потому какъ вѣра единаго Бога зритъ, такъ и надежду съ собою влечетъ и къ единому Богу привязываетъ, и на Него единаго надѣятися вѣрующаго научаетъ. Люди плотскіе, то"=есть, по плоти и міру сему живущіе, на честь свою, на богатство, силу и хитрость свою надѣются, и потому въ нуждахъ своихъ и случаяхъ къ тѣмъ прибѣгаютъ, и защищенія и помощи отъ тѣхъ просятъ. Вѣрный Божій рабъ не тако: всю сію, яко суетную, надежду оставивши, къ единому Богу, яко отроча малое къ матери своей, прибѣгаетъ и прилѣпляется, и отъ Него единаго всего проситъ съ надеждою, крѣпко держася за милостивое Божіе обѣщаніе: \textit{призови Мя въ день скорби твоея, и изму тя, и прославиши Мя}\footnote{Пс.~49,~15.}; и паки: \textit{еда забудетъ жена отроча свое, еже не помиловати исчадія чрева своего? аще же и забудетъ сихъ жена, но Азъ не забуду тебе, глаголетъ Господь}\footnote{Ис.~49,~15.}, и прочія милостивыя Его обѣщанія поминая. Откуду святое Божіе слово тѣхъ единыхъ ублажаетъ, которые на Бога надѣются, какъ сіе на премногихъ святаго Писанія мѣстахъ читаемъ; напротивъ того, отзываетъ всѣхъ отъ надежды на человѣка и прочее созданіе: \textit{не надѣйтеся на князи, на сыны человѣческія, въ нихже нѣсть спасенія}\footnote{Пс.~145,~3.}, и на прочіихъ мѣстахъ; и клятвою поражаетъ таковыхъ, которые надежду свою полагаютъ на человѣка: \textit{проклятъ человѣкъ, иже надѣется на человѣка, и утвердитъ плоть мышцы своея на немъ, и отъ Господа отступитъ сердце его}\footnote{Іер.~17,~5.}. Ибо кто на кого надѣется, тотъ въ того и вѣруетъ, и потому кто на человѣка надѣется, въ человѣка, какъ идола, вѣруетъ, и тако отъ Бога сердцемъ отступаетъ, какъ пророкъ глаголетъ, хотя устами исповѣдуетъ и почитаетъ Его. Таковый бо грѣшитъ противу заповѣди Божіей: \textit{Азъ есмь Господь Богъ твой}, и тако духовное совершаетъ идолопоклонство. 6)~Вѣра свобождаетъ вѣрующаго отъ грѣха, смерти, клятвы, ада, діавола и прочаго бѣдствія, и дѣлаетъ его духовно свободнымъ. Духовно, глаголю: понеже тѣлеснѣ можетъ быть порабощенъ, можетъ быть рабъ человѣку, можетъ быть въ плѣненіи, во узахъ, темницѣ, связанъ и окованъ быть; якоже многіе таковые вѣрные имѣются, каковы были мученики святіи и прочіе правды ради страждущіе; и нынѣ имѣются подъ игомъ нечестивыхъ находящіеся и раби благочестивіи, своимъ господамъ работающіи. Но таковые всѣ \textit{духовно} свободни суть. Ибо духа никто не можетъ поработить, плѣнить, связать, заключить, умертвить. \textit{Идѣже бо Духъ Господень, ту свобода}\footnote{2~Кор.~3,~17.}, "--- которая свобода въ ономъ Хрістовомъ обѣщаніи означается: \textit{аще Сынъ вы свободитъ, воистину свободни будете}. 7)~Вѣра обновляетъ человѣка, содѣлываетъ его доброхотнымъ, любительнымъ, милостивымъ, терпѣливымъ, кроткимъ и прочіихъ хрістіанскихъ добродѣтелей любителемъ. Ибо вѣра есть Божественное сѣмя, отъ котораго духовно раждаются хрістіане, и потому такіе плоды творятъ, каково сѣмя есть. \textit{Не можетъ бо древо добро плоды злы творити}, глаголетъ Хрістосъ\footnote{Мѳ.~7,~18.}. Каково сѣмя, таковы и плоды его. Якоже бо злаго сѣмене зміина, которое посѣялъ на сердцѣ прародителя нашего, суть злые плоды: ненависть, злоба, вражда и прочіе грѣхи; тако сѣмене Хрістова, которое есть вѣра святая, суть благіе плоды: любовь, милость, кротость, благость, милосердіе, воздержаніе и прочая\footnote{Гал.~5,~22.}. Откуду Апостолъ глаголетъ: \textit{о Хрістѣ Іисусѣ ни обрѣзаніе что можетъ, ни не обрѣзаніе, но вѣра, любовію поспѣшествуема}\footnote{Гал.~5,~6.}. Откуду апостоли святіи вездѣ въ посланіяхъ своихъ увѣщаваютъ къ познанію плодовъ вѣры, то"=есть, къ добрымъ дѣламъ. 8)~Вѣра сія содѣловаетъ радость и веселіе въ сердцѣ вѣрующаго. Радость сія есть не о пищѣ, питіи, ни о чести, ни о богатствѣ, златѣ, сребрѣ, ни о прочемъ, о чемъ сыны вѣка сего радуются, ибо сія радость плотская; но есть радость духовная, радость о Господѣ Спасѣ, о благости и человѣколюбіи Его, утѣшеніе и спокойство въ совѣсти, какъ Апостолъ учитъ: \textit{оправдившеся вѣрою, миръ имамы съ Богомъ Господемъ нашимъ Іисусъ Хрістомъ}, и прочая\footnote{Римл.~5,~1 и слѣд.}. Ибо Евангеліе святое есть вѣсть радостная, и вѣра есть сердечное принятіе Евангелія; того ради пріемлющимъ тое непремѣнно послѣдуетъ радость духовная въ сердцахъ ихъ, якоже о стражѣ темничномъ, въ Дѣяніяхъ Апостольскихъ поминаемомъ, пишется: \textit{возрадовася со всѣмъ домомъ своимъ, вѣровавъ Богу}\footnote{Дѣян.~16,~34.}. Откуду радость сія на многихъ Писанія святаго мѣстахъ, какъ пресладкая духовная пища, вѣрнымъ предлагается, якоже Псалмы, Евангеліе и посланія Апостольская читающему примѣтить можно. 9)~Вѣра сія злостраданіе и скорбь умягчаетъ. Ибо вѣдаютъ вѣрніи, яко скорбію и злостраданіемъ сообразными дѣлаются Хрісту Сыну Божію, неповинно за всѣхъ пострадавшему: \textit{аще бо съ Нимъ страждутъ, съ Нимъ и прославятся}\footnote{Римл.~8,~17.}. Хотя лишаются чести, богатства, славы, имени добраго, но несравненно большее богатство, честь, славу обрѣтаютъ во Хрістѣ, Спасителѣ своемъ; оскорбляютъ ихъ, но они большее утѣшеніе обрѣтаютъ во Хрістѣ, Господѣ своемъ; проклинаютъ ихъ, но они благословеніе находятъ и имѣютъ во Хрістѣ Іисусѣ, въ Котораго вѣруютъ, и клятва злыхъ обращается имъ въ благословеніе, якоже глаголетъ Пророкъ: \textit{проклянутъ тіи, и Ты благословиши}\footnote{Пс.~108,~28.}. Откуду Апостолъ святый глаголетъ: \textit{не точію же, но и хвалимся въ скорбѣхъ, вѣдяще, яко скорбь терпѣніе содѣловаетъ, терпѣніе же искусство; искусство же упованіе; упованіе же не посрамитъ}\footnote{Римл.~5,~3 и 4.}. \textit{(Смотри пунктъ 3"~й сего параграфа)}. 10)~Вѣра сія имѣетъ истинный Божій страхъ, которымъ вѣрующій отвращается отъ всякаго грѣха, всякаго безчиннаго пристрастія; всего, что позволено, напр. пищи, питія и прочіихъ міра сего благихъ, не къ сладострастію, но къ нуждѣ, къ подкрѣпленію немощныя плоти употребляетъ, дабы возмоглъ дѣла по званію своему творити. Откуду Апостолъ вѣрующимъ пишетъ: \textit{аще Отца называете нелицемѣрно судяща комуждо по дѣлу, со страхомъ житія вашего время жительствуйте}, и прочая\footnote{1~Петр.~1,~17 и слѣд.}. И Павелъ святый: \textit{со страхомъ и трепетомъ свое спасеніе содѣвайте}\footnote{Филип.~2,~12.}. Страхъ здѣ разумѣется не тотъ, который любви не причастенъ, каковый страхъ и самые нечестивые въ совѣсти своей дознаютъ, который имъ мукою вѣчною претитъ, его паче же и бѣси имѣютъ яко \textit{вѣруютъ и трепещутъ}, по словеси Апостола святаго\footnote{Іак.~2,~10.}, но страхъ съ любовію сопряженный. И хотя вѣрныхъ часто ударяетъ страхъ суда Божія, осужденія и ада, что наипаче бываетъ въ великомъ духовномъ искушеніи; однакожъ они силою вѣры и молитвы страхъ тотъ побѣждаютъ, и, якоже злато въ горнилѣ, въ семъ искушенія огнѣ чистѣйшими дѣлаются. 11)~Вѣра сія спасительное Хрістово смотрѣніе, рожденіе, на землѣ пожитіе, страданіе, смерть и воскресеніе дорого и высоко почитаетъ, яко едино такое посредствіе, которымъ вѣчно спасаемся. Она, со благоговѣніемъ на распятаго Хріста взирая, помышляетъ: кто и коликъ есть Который тако за подлаго Своего раба пострадалъ, и что симъ страданіемъ ему пріобрѣлъ! Откуду вѣрующій возбуждается ко взаимной любви и сердечному благодаренію, и тщится всякимъ образомъ тое исполнять. Но понеже видитъ, что не можетъ того учинити \textit{(что бо воздамы Господеви о всѣхъ, яже воздаде намъ?)}\footnote{Пс.~115,~3.}, падаетъ предъ Нимъ со смиреніемъ и любовію, сердечно исповѣдая свои недостатки, и единое сердечное къ благодаренію желаніе, усердіе и любовь, какую можетъ въ вѣкѣ семъ, приноситъ высочайшему своему Благодѣтелю. Отъ сего возбуждается въ вѣрномъ желаніе все терпѣть за имя Господа Іисуса, что воля Божія восхощетъ. Откуду Апостоли святіи въ своихъ посланіяхъ вездѣ почти высокую сію благодать нашимъ душевнымъ очесамъ предлагаютъ: \textit{вѣсте благодать Господа нашего Іисуса Хріста, яко васъ ради обнища, богатъ сый, да вы нищетою Его обогатитеся}\footnote{2~Кор.~8,~9 и на прочіихъ мѣстахъ.}, и тѣмъ къ благодаренію возбуждаютъ вѣрныхъ. Вѣра сія подвизается противу грѣха (о чемъ смотри въ параграфѣ 13"~мъ). И сіе свойство въ вѣрѣ примѣчается, что иногда ясно является и въ высоту сердце вѣрующаго восхищаетъ; иногда такъ сокрывается, что аки бы ея не было, отъ чего вѣрующій унылъ бываетъ, печалится, боится суда Божія, осужденія, геенны и прочаго, но, воспоминая Божія милостивыя обѣщанія и евангельскія утѣшенія, паки ободряется, укрѣпляется и упованіе свое и дерзновеніе воспріемлетъ, что во Псаломникѣ и на прочіихъ святыхъ, какъ во святомъ Писаніи, такъ и въ церковной Исторіи поминаемыхъ, примѣтить можно. 13)~Вѣра душу вѣрнаго \textit{таинственно} соединяетъ со Хрістомъ, аки невѣсту съ женихомъ, якоже пророкъ глаголетъ въ лицѣ Господа: \textit{обручу тя Себѣ въ вѣрѣ, и увѣси Господа}\footnote{Ос.~2,~20.}. И Апостолъ сему соотвѣтствуетъ, глаголя: \textit{обручихъ васъ} (вѣрнымъ глаголетъ) \textit{единому Мужу дѣву чисту представити Хрістови}\footnote{2~Кор.~11,~2.}. Сего ради Хрістосъ таковую душу вѣрную очищаетъ отъ грѣха и всякія скверны, и содѣловаетъ ее святу, отъемлетъ отъ нея бѣдствія духовныя "--- клятву, осужденіе, смерть и всякое духовное неблагополучіе, и вмѣсто того подаетъ Своя духовная благая: благословеніе, избавленіе и вѣчный животъ, якоже Апостолъ о семъ научаетъ: \textit{Хрістосъ бысть намъ премудрость отъ Бога, правда же и освященіе и избавленіе}\footnote{1~Кор.~1,~30.}; и, какъ жену какую, совлекши съ нея раздранная рубища и отъ смрада омывши, облекаетъ въ чистую правды Своея багряницу, да предъ очесами Его и небеснаго Его Отца явится чиста и, какъ дщерь царская, духовною утварію \textit{одѣяна и преиспещрена}\footnote{Пс.~44,~10.}. Откуду пророкъ святый, о семъ духомъ радуяся, восклицаетъ, глаголя: \textit{да возрадуется душа моя о Господѣ! облече бо мя въ ризу спасенія и одеждою веселія одѣя мя; яко на жениха, возложи на мя вѣнецъ, и яко невѣсту украси мя красотою}\footnote{Ис.~61,~10.}. Изъ вышеписанныхъ видишь, хрістіанине: 1)~Что въ вѣрѣ все существо хрістіанскаго блаженства состоитъ, яко кто истинную живую вѣру имѣетъ, тотъ имѣетъ Божіе благословеніе, правду Хрістову, святость, избавленіе, свободу духовную, вѣчный животъ и блаженство. 2)~Отсюду познай, человѣче, коль великое богатство благости Божіей, что человѣка отверженнаго и окаяннаго въ Сынѣ такъ возлюбилъ и превознеслъ, яко чимъ болѣе человѣкъ обезчестился въ себѣ, тѣмъ паче во Хрістѣ Сынѣ Божіемъ, по Божію человѣколюбію, вознесенъ. 3)~Отсюду поучайся сердцемъ и устами благодарить Богу, Который тебѣ, бѣдному, окаянному, проклятому, отверженному, осужденному въ себѣ самомъ, подаетъ вѣрою въ Сынѣ Своемъ вмѣсто грѣха твоего правду, вмѣсто клятвы благословеніе, вмѣсто окаянства блаженство, вмѣсто смерти животъ вѣчный, вмѣсто ада и геенны наслѣдіе небеснаго царствія, и всѣхъ вѣчныхъ благъ сокровища отверзаетъ. Умъ не постигаетъ сея къ намъ Божія милости и благоволенія! О бѣдные и окаянные грѣшники! коль великія милости и человѣколюбія Божія туне, безъ всякихъ нашихъ заслугъ, сподобилися мы! Кто за сіе человѣколюбіе Божіе Богу достойно возблагодарить можетъ? \textit{Что воздамы Господеви о всѣхъ, яже воздаде намъ?} 4)~Отсюду познай силу истинныя и живыя во Хріста вѣры, которая вся благая и духовная дарованія къ себѣ привлекаетъ и отъ Бога получаетъ. 5)~Когда отъ таковаго бѣдствія благодатію Божіею избавились мы, и къ толикому блаженству возведены, какъ должно намъ берещися отъ всякаго грѣха, отъ чего всякое бѣдствіе приходитъ, да не паки впадемъ въ бѣдствіе; и какъ усердно работать Богу, Который толикую милость намъ грѣшнымъ показалъ! 6)~Какъ тщиться хранить одежду \textit{оправданія}, данную при крещеніи, да не паки съ великою нашею бѣдою лишимся тоя! \textit{Блаженъ бдяй и блюдый ризы своя, да не нагъ ходитъ, и узрятъ срамоту его}, глаголетъ Хрістосъ\footnote{Апок.~16,~15.}. 7)~Несоблюдшему сихъ богоданныхъ ризъ едина надежда осталася ко взысканію ихъ "--- чистосердечное покаяніе (о которомъ \textit{смотри въ статьѣ первой книги первыя, части вторыя}).

\paragraph*{§\:284.} Вѣра святая многоразличному подлежитъ искушенію. 1)~Искушаетъ ее плоть со страстьми и похотьми, когда на всякое безчинное мудрованіе и похотѣніе движетъ вѣрное сердце\footnote{Іак.~1,~14.}. Хрістіанамъ, яко Хрістовымъ рабамъ, должно \textit{плоть распинать со страстьми и похотьми}\footnote{Гал.~5,~24.}. 2)~Соблазны міра сего, то"=есть, развращенное и слову Божію противное ученіе, примѣръ развращеннаго житія, искру ея угасить нудятся. Сюды принадлежитъ лесть богатства, чести, славы и сластей. 3)~Сатана и слуги его тщатся ее искоренить и какъ искру заченшуюся, наводненіемъ бѣдъ, напастей, гоненій и скорбей многихъ угасить. Противу всего сего бѣдствія вѣрному сердцу надобно подвизатися, и терпѣніемъ вѣру свою хранить. 4)~Нѣтъ большаго и тягчайшаго вѣрѣ искушенія, какъ когда на вѣрное сердце находитъ духовная напасть: смущеніе, страхъ и ужасъ суда Божія, отчаянія, страхъ геенны и ада; когда совѣсти жестокое удареніе и мученіе чувствовать принуждается вѣрная душа; когда помыслы, какъ волны, востаютъ и ударяютъ отчаяніемъ вѣрное сердце: \textit{нѣтъ спасенія ему въ Бозѣ его}\footnote{Пс.~3,~3.}. Что бо тягчае можетъ быть душѣ, ищущей спасенія, какъ когда врагъ хощетъ отчаяніемъ отнять у ней вѣчное спасеніе? Сіи суть \textit{стрѣлы лукаваго разженныя}, мещемыя на вѣрную душу, Божіимъ въ нашу пользу попущеніемъ, которыя \textit{щитомъ вѣры}, по ученію Апостольскому, \textit{угашати подобаетъ}\footnote{Еф.~6,~16.}.

\paragraph*{§\:285.} Вѣра святая укрѣпляется и растетъ 1)~прилѣжнымъ чтеніемъ или слушаніемъ Божія слова. Какъ бо \textit{вѣра} бываетъ \textit{отъ слуха, слухъ же глаголомъ Божіимъ}\footnote{Римл.~10,~17.}, такъ укрѣпляется и растетъ вниманіемъ и поученіемъ всегдашнимъ отъ тогожде Божія слова. Ибо, по ученію премудраго Павла, \textit{всяко писаніе Богодухновенно и полезно есть ко ученію, ко обличенію, ко исправленію, къ наказанію, еже въ правдѣ, да совершенъ будетъ Божій человѣкъ, на всякое дѣло благое уготованъ}\footnote{2~Тим.~3,~16 и 17.}. Сего бо ради и \textit{далъ есть Хрістосъ овы убо апостолы, овы же пророки, овы же благовѣстники, овы же пастыри и учители, къ совершенію святыхъ въ дѣло служенія, въ созиданіе тѣла Хрістова: дондеже достигнемъ вси въ соединеніе вѣры, и познанія Сына Божія, въ мужа совершенна, въ мѣру возраста исполненія Хрістова}\footnote{Еф.~4,~11--13.}. Откуду \textit{ублажается тотъ, кто въ законѣ Господни поучается день и нощь}\footnote{Пс.~1,~2.}; яко прекрасные плоды отъ того поученія обѣщаются не иначе, какъ отъ древа, которое при водахъ насажденно, и влагою тѣхъ водъ напаяемо бываетъ: \textit{и будетъ} (поучающійся въ законѣ Господни) \textit{яко древо, насажденное при исходищихъ водъ, еже плодъ свой дастъ во время свое, и листъ его не отпадетъ, и, вся, елика аще творитъ, успѣетъ}\footnote{ст.~3.}. 2)~Укрѣпляется и растетъ вѣра размышленіемъ чудныхъ Божіихъ дѣлъ, милостивыхъ Его обѣщаній исполненныхъ. Обѣщалъ Богъ Ною праведному, что его съ фамиліею его отъ всемірнаго потопа спасетъ, и спаслъ\footnote{Быт. гл.~6--8.}. Обѣщалъ Аврааму, что сѣмя его изъ Египта, въ которомъ имѣло тяжкую претерпѣвать работу, изведетъ и введетъ въ землю Хананейскую, и учинилъ тое, якоже читаемъ въ книгахъ Моисеовыхъ и псалмахъ: \textit{помяну слово святое Свое, еже ко Аврааму рабу Своему. И изведе люди Своя въ радости, и избранныя Своя въ веселіи. И даде имъ страны языкъ, и труды людей наслѣдоваша}\footnote{Пс.~104,~42--44.}. Обѣщалъ сыновъ Израилевыхъ изъ плѣненія Вавилонскаго свободить, и свободилъ, какъ написано въ книгахъ Ветхаго Завѣта. Обѣщалъ послать Спасителя міру, и исполнилъ тое, и спасаются вѣрующіи во имя Его. Обѣщалъ услышать вѣрою призывающихъ Его, и слушаетъ, какъ на многихъ святаго Писанія мѣстахъ читаемъ. Услышалъ ихъ, услышитъ и насъ, когда вѣрою Его призовемъ. Богъ бо на лица не зритъ, но на вѣру и сердце человѣческое. Всѣмъ готовъ подать благодать и милость Свою, когда сердце видитъ удобное къ пріятію благодати Его. А когда не получаемъ, чего желаемъ и просимъ, наша винность, а не Его. А отъ бывшихъ заключаемъ и о будущихъ, и отъ исполнившихся Божіихъ пророчествъ и обѣщаній милостивыхъ несумнѣнно подтверждаемъ о тѣхъ, которыя еще не исполнилися, но имѣютъ въ свое время исполнитися. Обѣщалъ воскресеніе мертвыхъ и вѣрныхъ Своихъ рабовъ прославить и подать наслѣдіе небеснаго царствія, непремѣнно будетъ тое. И сіе такъ истинно и извѣстно есть, что какъ бы оно уже въ самой вещи было. Богъ бо, яко \textit{истиненъ}, солгати не можетъ; яко \textit{всемогущій}, все сдѣлаетъ, что хощетъ, и какъ удобно Ему было міръ изъ ничего создать, такъ удобно разсыпанный прахъ тѣлесъ нашихъ собрать и въ безсмертіе облещи и прославить; яко \textit{благъ}, хощетъ разумной Своей твари блаженство подать; и яко \textit{праведенъ}, вѣрныхъ рабовъ Своихъ, въ вѣрѣ подвизавшихся, непремѣнно наградитъ. Тако вѣра пріемлетъ подкрѣпленіе отъ разсужденія чудныхъ Божіихъ дѣлъ и свойствъ Его, въ святомъ Его Словѣ открытыхъ, и вѣрное сердце оттудуже почерпаетъ себѣ живое утѣшеніе, якоже Апостолъ святый написалъ: \textit{елика преднаписана быша, въ наше наказаніе преднаписашася, да терпѣніемъ и утѣшеніемъ писаній упованіе имамы}\footnote{Рим.~15,~4.}. 3)~Укрѣпляется и растетъ вѣра усердною и частою молитвою, якоже учитъ Хрістосъ: \textit{просите, и дастся вамъ; ищите и обрящете; толцыте, и отверзется вамъ. Всякъ бо просяй пріемлетъ, и ищай обрѣтаетъ, и толкущему отверзется}\footnote{Матѳ.~7,~7 и 8.}. Ибо вѣра, яко даръ Божій, Божіею помощію сохраняется и умножается; яко безъ Него не можемъ творити ничесоже, якоже глаголетъ Хрістосъ: \textit{безъ Мене не можете творити ничесоже}\footnote{Іоан.~15,~5.}. Откуду апостоли молились Хрісту: \textit{приложи намъ вѣру}\footnote{Лук.~17,~5.}. Тако подобаетъ и намъ молитися и толкати въ двери милосердія Его: \textit{Господи, приложи намъ вѣру}; или паче со отцемъ отрока бѣсноватаго молитися со слезами: \textit{Господи, помози нашему невѣрію}\footnote{Марк.~9,~24.}. Ибо въ вѣрѣ все хрістіанское блаженство содержится, какъ выше сказано и ниже увидимъ. 4)~Укрѣпляется и умножается вѣра святая причащеніемъ пресвятыхъ Хрістовыхъ Таинъ. Ибо \textit{прилѣпляяйся Господеви, единъ духъ есть съ Господемъ}, глаголетъ Павелъ святый\footnote{1~Кор.~6,~17.}; и Хрістосъ: \textit{ядый Мою плоть, и піяй Мою кровь, во Мнѣ пребываетъ, и Азъ въ немъ}\footnote{Іоан.~6,~56.}; \textit{и иже будетъ во Мнѣ и Азъ въ немъ, той сотворитъ плодъ многъ}\footnote{15,~5.}. Якоже бо плоть человѣческая укрѣпляется и растетъ естественною пищею, тако вѣра Хрістова, въ сердцѣ человѣческомъ заченшаяся, укрѣпляется и растетъ таинственною святѣйшаго Тѣла и Крови Хрістовой пищею. Ибо Тѣло и Кровь Хрістова есть \textit{животворящая}, и потому, кто достойно того пріобщается, духовно оживотворяется и отъ немощей душевныхъ исцѣляется. 5)~Сохраняется вѣра удаленіемъ отъ бесѣдъ и обращеній со злыми и развращенными. Ибо \textit{тлятъ обычаи благи бесѣды злы}, глаголетъ Апостолъ\footnote{1~Кор.~15,~33.}. Откуду и повелѣваетъ отъ таковыхъ отлучатися: \textit{повелѣваемъ вамъ, братіе, о имени Господа нашего Іисуса Хріста, отлучатися вамъ отъ всякаго брата безчинно ходяща, а не по преданію, еже пріяша отъ насъ}\footnote{2~Сол.~3,~6.}. Чимъ болѣе вѣра растетъ, тѣмъ множайшіе ея плоды являются; якоже чимъ болѣе растетъ и вѣтви своя распростираетъ плодовитое древо, тѣмъ болѣе плодовъ приноситъ.

\paragraph*{§\:286.} Отъ вышеописанныхъ примѣчаются несумнѣнные \textit{невѣрія знаки}: 1)~Развращенное и безстрашное житіе и закону Божію противное. О каковыхъ нравахъ поминается на многихъ святаго Писанія мѣстахъ\footnote{во Псалмахъ: 5,~10; 7,~15 и 16; 9,~26--32; 11,~3 и 5; 35,~2--5; 36,~12,~14,~21,~32,~35; 49,~16--21; 51,~3--6; 58,~2--4,~8 и 13; 63,~4--7; 72,~6,~7 и 11; 93,~5--7; 139,~2--6, и въ прочіихъ. Въ посланіяхъ Апостольскихъ: Римл.~1,~21--32; 1~Кор.~6,~9 и 10; 2~Кор.~12,~27; Гал.~5,~19--21; Еф.~5,~3--5; Фил.~3,~18 и 19; Кол.~3,~5 и 8; 1~Тим.~1,~9 и 10; 2~Тим.~3,~2--7; Тит.~1,~10 и 16; Апок.~21,~8; 22,~15 и на прочіихъ мѣстахъ.}. 2)~Пристрастіе къ временнымъ и мірскимъ вещамъ, то"=есть: чести, богатству, злату, сребру, славѣ, роскоши и прочіимъ. Ибо \textit{любы міра сего вражда Богу есть: иже бо восхощетъ другъ быти міру, врагъ Божій бываетъ}\footnote{Іак.~4,~4.}, "--- что съ вѣрою помѣститься не можетъ; вѣра бо отъ всего того отвращается, и единаго Бога ищетъ, любитъ и прилѣпляется Ему. Чего ради и Апостолъ отъ сей любви отзываетъ насъ: \textit{не любите міра, ни яже въ мірѣ: аще кто любитъ міръ, нѣсть любве Отчи въ немъ}\footnote{1~Іоан.~2,~15.}. Откуду читаемъ въ притчѣ о званыхъ на вечерю велію, что \textit{начаша вкупѣ отрицатися вси}. Первый рече ему, рабу посланному звати: \textit{село купихъ, и имамъ нужду изыти, и видѣти е: молютися, имѣй мя отречена}. И другій рече: \textit{супругъ воловныхъ купихъ пять, и гряду искусити ихъ: молю тя, имѣй мя отречена}. И другій рече: \textit{жену пояхъ, и сего ради не могу пріити}\footnote{Лук.~14,~18--20.}. Велія вечеря есть великое и непостижимое въ небесномъ царствіи наслажденіе. На сію вечерю Богъ, небесный Царь, зоветъ всѣхъ чрезъ рабовъ Своихъ, пророковъ, апостоловъ и проповѣдниковъ слова Своего. Но къ суетѣ міра сего пристрастившіеся вси купно отрицаются, хотя не устами, но сердцами. Ибо \textit{идѣже сокровище ихъ, ту и сердце ихъ}, по словеси Хрістову\footnote{Матѳ.~6,~21.}. Сюды принадлежитъ оное Хрістово ученіе: \textit{никтоже можетъ двѣма господинома работати: любо единаго возлюбитъ, а другаго возненавидитъ; или единаго держится, о друзѣмъ же нерадити начнетъ. Не можете Богу работати и мамонѣ}\footnote{ст.~24.}. Кто работаетъ мамонѣ, а не Богу, безъ сумнѣнія не имѣетъ вѣры; ибо \textit{безъ вѣры невозможно угодити Богу}\footnote{Евр.~11,~6.}, слѣдственно и работати. Работати бо и угождати Богу едино есть. Сюды принадлежитъ слово Хрістово: \textit{како вы можете вѣровати, славу другъ отъ друга пріемлюще, и славы, яже отъ единаго Бога, не ищете}\footnote{Іоан.~5,~44.}? И паки: \textit{кая польза человѣку, аще міръ весь пріобрящетъ, душу же свою отщетитъ}\footnote{Матѳ.~16,~26.}? Вѣра бо очищаетъ сердце отъ любви сего. Слѣдственно, въ которомъ сердцѣ любы міра сего есть, въ томъ вѣры нѣтъ. 4)~Лицемѣріе есть такожде знаменіе невѣрія. Ибо лицемѣръ внѣ только оказываетъ свое благочестіе, а внутрь нѣтъ того. Лицемѣріе бо есть притворство благочестія и святости; вѣра же святая не въ наружности, но въ сердцѣ имѣетъ мѣсто свое, и внѣ оказываетъ себе въ случаѣ исповѣдываніемъ и прочіими плодами. И лицемѣръ людямъ угождаетъ, а не Богу; отъ людей ищетъ славы и похвалы, а не отъ Бога, что противно есть вѣрѣ, по реченному: \textit{како можете вѣровати, славу другъ отъ друга пріемлюще, и славы, яже отъ единаго Бога, не ищете?} Убо въ лицемѣрѣ нѣтъ вѣры. 4)~Презрѣніе и оставленіе хрістіанскихъ добродѣтелей есть знаменіе невѣрія. Какъ бо \textit{древо отъ плодовъ}, такъ вѣра отъ добрыхъ дѣлъ \textit{познается}\footnote{Лук.~6,~44; Іак.~2,~18.}. Всяко древо доброе плоды добрые творитъ, а древо злое плоды злы творитъ. \textit{Не можетъ древо добро плоды злы творити, ни древо зло плоды добры творити}\footnote{Матѳ.~7,~18.}. Слѣдственно гдѣ нѣтъ плодовъ вѣры, то"=есть, добрыхъ дѣлъ, тамо нѣтъ и самой вѣры. Вѣра бо праздна и безплодна быть не можетъ. 5)~Надежда на себе самого, на свое благочестіе, на честь, богатство, силу свою, на князей и прочее созданіе, доказываетъ невѣріе сердечное. Ибо надежда на Бога отъ вѣры неотлучна; и истинная вѣра отъ всея надежды, кромѣ Бога тщится быть свободна, и въ единомъ Бозѣ утверждается. \textit{(Смотри о надеждѣ главу первыя книги)}. 6)~Презрѣніе и оставленіе слушанія или чтенія Божія слова. Ибо человѣкъ, яко забытливъ, слѣпъ и немощенъ самъ въ себѣ, требуетъ частаго поминовенія о вѣрѣ и плодахъ ея, требуетъ обличенія, наказанія, исправленія, утѣшенія въ нужныхъ случаяхъ, поощреній, возбужденія, "--- что отъ слова Божія почерпается. \textit{Всяко бо писаніе Богодухновенно и полезно есть ко ученію, ко обличенію, ко исправленію, къ наказанію, еже въ правдѣ, да совершенъ будетъ Божій человѣкъ, на всякое дѣло благое уготованъ}\footnote{2~Тим.~3,~16 и 17.}. Слушаніе бо, или чтеніе святаго Писанія съ размышленіемъ, есть какъ елей, который, вѣрѣ въ сердцѣ горящей подливаемый, не попускаетъ угаснути. Слѣдственно какъ свѣтильнику безъ елея, такъ свѣтильнику вѣры безъ поученія въ словѣ Божіемъ слѣдуетъ угаснути. 7)~Оставленіе молитвы доказательство есть потерянныя вѣры. Человѣкъ бо, пока въ мірѣ семъ живетъ, всякому бѣдствію и искушенію плоти, діавола и злыхъ духовъ, и злыхъ людей подверженъ, которому бѣдствію, яко немощенъ, противитися никакъ не можетъ самъ собою, безъ помощи Божіей, которая молитвою испрашивается. Молитва бо есть оружіе хрістіанское противу врага діавола и прочіихъ враговъ. Того ради оставившему оружіе слѣдуетъ оставить и брань, и покориться врагу. Откуду усердно и часто, паче же \textit{непрестанно повелѣно намъ молитися}\footnote{Лук.~18,~1--7; Римл.~12,~12; 1~Сол.~5,~17; Еф.~6,~18.}. 8)~Явное невѣрія знаменіе есть отчаяніе милосердія Божія, которое оказываетъ себе или весьма развратнымъ житіемъ, о каковыхъ пишется: \textit{иже въ нечаяніе вложшеся, предаша себе студодѣянію, въ дѣланіе всякія нечистоты въ лихоиманіи}\footnote{Еф.~4,~19.}, "--- или безмѣрною печалію, которая часто и къ самоубійству приводитъ, какъ то случилося Іудѣ предателю, который \textit{шедъ удавися}\footnote{Матѳ.~27,~5.}, и братоубійцѣ Каину, который, по убіеніи брата своего Авеля и обличеніи Божіи, сказалъ ко Господу Богу: \textit{вящшая вина моя, еже оставитися ми}\footnote{Быт.~4,~13.}. Однакожъ кто кается за грѣхи и безмѣрную печаль страждетъ, но тую упованіемъ на милосердіе Божіе тщится преодолѣвать и побѣдить, какъ то бываетъ въ великомъ духовномъ искушеніи, "--- тамо подвизающаяся вѣра еще не угасла, хотя и въ великой слабости есть.

\paragraph*{§\:287.} Аще бы кто сказалъ, что истинная вѣра есть правое содержаніе и исповѣданіе правыхъ догматовъ, правду бы сказалъ; ибо вѣрному неотмѣнно нужно есть православное догматовъ содержаніе и исповѣданіе. Но сіе верное знаніе и исповѣданіе не дѣлаетъ человѣка вѣрнымъ и истиннымъ хрістіаниномъ. Заключается всегда православныхъ догматовъ содержаніе и исповѣданіе въ истинной во Хріста вѣрѣ, но не всегда истинная во Хріста вѣра во исповѣданіи православномъ заключается. Иное бо есть знаніе и исповѣданіе вѣры, иное есть вѣра во Хріста, какъ изъ вышеписанныхъ видѣть можно. Знаніе правыхъ догматовъ имѣется въ разумѣ, которое часто бываетъ безплодно, надмѣнно и возносливо. Откуду бываетъ, что многіи, имѣющіи знаменіе догматовъ, суть беззаконнаго житія; многіи и проповѣдуютъ вѣру, поучаютъ, наставляютъ другихъ и показуютъ путь ко спасенію, но сами не идутъ по тому пути, по подобію столповъ, на пути поставленныхъ, которые отъ града до града указуютъ идущимъ путь, но сами недвижно стоятъ. Почему знаніе сіе и исповѣданіе и ученіе имъ самимъ ничего не пользуетъ. Истинная же во Хріста вѣра есть въ сердцѣ, какъ сказано выше, и есть плодовита, смиренна, терпѣлива, любительна, милосерда, человѣколюбива, сострадательна, алчущая и жаждущая правды, и проч., отъ мірскихъ похотей удаляется и единому Богу прилѣпляется, всегда къ небеснымъ и вѣчнымъ стремится и ищетъ ихъ, противу всякаго грѣха подвизается, и отъ Бога помощи непрестанно ищетъ и проситъ къ тому. Отсюду апостоли святіи къ показанію плодовъ вѣры, то"=есть, добрыхъ дѣлъ, вездѣ въ посланіяхъ своихъ поощряютъ вѣрующихъ: \textit{подадите въ вѣрѣ вашей добродѣтель: въ добродѣтели же разумъ, въ разумѣ же воздержаніе, въ воздержаніи же терпѣніе, въ терпѣніи же благочестіе, въ благочестіи же братолюбіе, въ братолюбіи же любовь}\footnote{2~Петр.~1,~5--7.}, и на прочіихъ мѣстахъ. Отсюду къ подвигу противу діавола и грѣха возбуждаютъ: \textit{трезвитеся, бодрствуйте! зане супостатъ вашъ діаволъ, яко левъ рыкая, ходитъ искій, кого поглотити; емуже противитеся тверди вѣрою}\footnote{1~Петр.~5,~8 и 9.}. Отсюду отцы святіи, хрістіанскую вѣру проповѣдуя, поучаютъ. Августинъ глаголетъ: «съ любовію хрістіанина есть, безъ любви вѣра демона есть»\footnote{въ кн.~10"~й о любви.}. Ибо \textit{и бѣси вѣруютъ, и трепещутъ}, учитъ Апостолъ\footnote{Іак.~2,~19.}. Амвросій святый учитъ: «вѣра есть корень всѣхъ добродѣтелей»\footnote{Въ кн. о Каинѣ и Авелѣ.}. Златоустъ святый глаголетъ: «вѣра есть матерь всѣхъ благихъ»\footnote{Бес.~3"~я на посл. къ Римл.}. Правильно убо заключается, что живая вѣра есть такая, которой нравы вѣрующаго соотвѣтствуютъ: напротивъ того мертвая вѣра есть въ томъ, въ комъ нѣтъ плодовъ вѣры, кто не Богу, но грѣху, не умершему за всѣхъ и воскресшему Хрісту, якоже Апостолъ требуетъ: \textit{Хрістосъ за всѣхъ умре, да живущіи не ктому себѣ живутъ, но умершему за нихъ и воскресшему}\footnote{2~Кор.~5,~15.}, "--- но себѣ и своимъ прихотямъ живетъ. \textit{Якоже бо тѣло}, учитъ Апостолъ, \textit{безъ духа мертво есть, тако и вѣра безъ дѣлъ мертва есть}\footnote{Іак.~2,~26.}. Примѣчай и сіе, хрістіанине, что вѣрнымъ о Хрістѣ отверстое царствіе небесное проповѣдуютъ апостоли, научаютъ отцы наши, и вся церковь содержитъ тако. Слѣдственно, ежели бы знаніе только едино, и исповѣданіе догматовъ православныхъ было вѣра нелицемѣрная и истинная, то бы и блудники, и хищники, и прочіи беззаконники невозбранно входили въ царствіе небесное; ибо многіе отъ нихъ знаютъ и исповѣдуютъ православные догматы. Но таковымъ всѣмъ затворяетъ входъ въ царствіе небесное Божіе слово. \textit{Или не вѣсте}, глаголетъ Апостолъ, яко неправедницы царствія Божія не наслѣдятъ\textit{? Не льстите себе, ни блудницы, ни идолослужители, ни прелюбодѣи, ни сквернители, ни малакіи, ни мужеложницы, ни лихоимцы, ни татіе, ни піяницы, ни досадители, ни хищницы царствія Божія не наслѣдятъ}\footnote{1~Кор.~6,~9 и 10.}. И \textit{внѣ суть псы и чародѣи, и любодѣи, и убійцы, и идолослужители, и всякъ любяй и творяй лжу}\footnote{Апок.~22,~15.}. Хотя бо исповѣдуютъ Бога, но дѣлами отмещутся Его, якоже глаголетъ Апостолъ: \textit{Бога исповѣдуютъ вѣдѣти, а дѣлы отмещутся Его, мерзцы суще и непокориви, и на всяко дѣло благое неискусни}\footnote{Тит.~1,~16.}. Исповѣдуешь, хрістіанине, единаго Бога: хорошо. Но вѣруеши ли сердцемъ твоимъ въ Бога, якоже учитъ Апостолъ: \textit{сердцемъ вѣруется въ правду, усты же исповѣдуется во спасеніе}\footnote{Римл.~10,~10.}? Безъ сердечной бо вѣры устное исповѣданіе не иное что, какъ лицемѣріе. Можетъ быть знаменіемъ вѣры устное исповѣданіе; но знаменіе ложное бываетъ, когда исповѣдующаго житіе противорѣчитъ вѣрѣ. Слѣдуетъ отъ вѣры устное исповѣданіе, какъ отъ заченшагося во умѣ помышленія слово, по оному: \textit{вѣровахъ, тѣмже возглаголахъ}\footnote{Пс.~115,~1.}. Но не всегда отъ устнаго исповѣданія истинная вѣра доказывается, какъ и слово не всегда съ помысломъ сходно бываетъ. Ибо часто бываетъ, что люди не тое говорятъ, что въ сердцѣ имѣютъ; какъ и исповѣдующіи Бога не тое помышляютъ въ сердцѣ и дѣлаютъ, чего исповѣданіе вѣры учитъ. Исповѣдуютъ устами Бога, но сердцемъ и дѣлами отмещутся Его. Устами проповѣдуютъ Бога, но въ сердцѣ глаголютъ: \textit{нѣсть Богъ}\footnote{13,~1.}. "--- Называешь Бога Отцемъ, Господемъ, Царемъ, Защитникомъ и Помощникомъ, якоже въ молитвѣ Господней, псалмахъ и прочіихъ пѣснехъ вѣрные Ему молятся и поютъ: хорошо, вѣры знакъ есть. Онъ вѣрнымъ есть Отецъ, Господь, Царь, Защитникъ и Помощникъ. Но смотри, тщишися ли Ему, яко Отцу, послушаніе показывать и подобные Ему нравы имѣти? Сынъ бо отцу подобенъ есть. Работаеши ли Ему, яко Господу своему? Не работаеши ли мамонѣ и прочіимъ идоламъ? Не царствуетъ ли надъ тобою грѣхъ, когда Царемъ своимъ Его называешь? Не ищешь ли защищенія и помощи отъ сыновъ человѣческихъ и прочаго созданія? Слыши, что Богъ о устномъ исповѣданіи, но не отъ сердца происходящемъ, и которому сердце не согласуетъ, чрезъ пророка глаголетъ: \textit{приближаются Мнѣ людіе сіи усты своими, и устнами чтутъ Мя: сердце же ихъ далече отстоитъ отъ Мене}\footnote{Матѳ.~15,~8; Ис.~29,~13.}. "--- Глаголешь: вѣрую во Хріста Сына Божія: хорошо. Но пріемлешь ли сердцемъ Его святое Евангеліе? послѣдуешь ли Его ученію, во Евангеліи написанному? не стыдишися ли смиренія, нищеты, терпѣнія Его? несеши ли крестъ свой, который всѣмъ вѣрнымъ предлагаетъ Онъ: \textit{аще кто хощетъ по Мнѣ ити, да отвержется себе, и возметъ крестъ свой, и по мнѣ грядетъ}\footnote{Матѳ.~16,~24.}? Послушай, что Онъ глаголетъ о неносящихъ креста своего: \textit{иже не пріиметъ креста своего, и въ слѣдъ Мене грядетъ, нѣсть Мене достоинъ}\footnote{10,~38.}, "--- и о стыдящихся Его и святыхъ словесъ Его: \textit{иже аще постыдится Мене и Моихъ словесъ въ родѣ семъ прелюбодѣйнѣмъ и грѣшнѣмъ, и Сынъ человѣческій постыдится его, егда пріидетъ во славѣ Отца Своего, со ангелы святыми}\footnote{Марк.~8,~38.}. А о вѣрующихъ истинно во Хріста написано: \textit{се полагаю въ Сіонѣ камень краеуголенъ, избранъ, честенъ; и вѣруяй въ Онь не постыдится}\footnote{1~Петр.~2,~6; Ис.~28,~16; Римл.~9,~33.}; и во Апокалипсисѣ глаголется: \textit{буди вѣренъ даже до смерти, и дамъ ти вѣнецъ живота}\footnote{2,~10.}. Отсюду заключается, что не тотъ вѣрный Хрістовъ рабъ есть, кто устами исповѣдуетъ Хріста, но кто и сердцемъ вѣруетъ во Хріста, и не стыдится Его и словесъ Его, послѣдуетъ смиренію, нищетѣ, терпѣнію и кротости Его, якоже Самъ того научаетъ насъ: \textit{научитеся отъ Мене, яко кротокъ есмь и смиренъ сердцемъ}\footnote{Матѳ.~11,~29.}, "--- и носитъ крестъ свой, и до смерти вѣренъ пребываетъ. "--- Глаголешь: \textit{вѣрую въ Духа Святаго Господа}: хорошо. Но смотри, не противишься ли Духу Святому жестокосердечіемъ и нераскаяннымъ житіемъ, якоже Стефанъ святый обличаетъ Іудеевъ: \textit{жестоковыйніи и необрѣзанніи сердцы и ушесы! вы присно Духу Святому противитеся}\footnote{Дѣян.~7,~51.}. "--- Глаголешь: \textit{вѣрую во едину святую Соборную и Апостольскую Церковь}: хорошо. Но смотри, истинный ли ты еси удъ церкви святыя? Она \textit{свята} есть, кровію Хріста Сына Божія \textit{освященна и освящается}\footnote{Еф.~5,~26 и 27.}; \textit{служитъ Богу своему преподобіемъ и правдою предъ Нимъ}\footnote{Лук.~1,~75.}. Ты храниши ли святость, вѣрою дарованную тебѣ въ крещеніи\footnote{1~Кор.~6,~11.}? Аще же потерялъ тую нерадѣніемъ, ищеши ли тую покаяніемъ, премѣненіемъ житія и вѣрою? удаляешися ли \textit{отъ всякія скверны плоти и духа, творяще святыню въ страсѣ Божіи}\footnote{Кор.~7,~1.}? Когда нѣтъ того, то нѣси удъ церкви святыя. Церковь бо есть \textit{свята и непорочна}\footnote{Еф.~5,~27.}, слѣдственно и удамъ ея подобаетъ быти святымъ и непорочнымъ. "--- Глаголеши: \textit{чаю воскресенія мертвыхъ и жизни будущаго вѣка}: хорошо; надежды знаменіе есть. Но смотри, не имѣешь ли нравовъ такихъ, которые показываютъ, что воскресенія мертвыхъ нѣтъ? Не живешь ли такъ, какъ звѣри и скоты, которые тое только дѣлаютъ, что хотятъ, и чувства имъ свои представляютъ, при скончаніи которыхъ и духъ исчезаетъ? Не творишь ли такихъ дѣлъ, которыя заключаютъ двери къ вѣчному животу? \textit{Не льстите себе}, глаголетъ Апостолъ, \textit{ни блудницы, ни идолослужители, ни прелюбодѣи, ни сквернители, ни малакіи, ни мужеложницы, ни лихоимцы, ни татіе, ни піяницы, ни досадители, ни хищницы, царствія Божія не наслѣдятъ}\footnote{1~Кор.~6,~9 и 10; Гал.~5,~19--21.}. О каковыхъ дѣлахъ и на прочіихъ Писанія мѣстахъ свидѣтельствуется. Кто сердечно вѣруетъ во Хріста, и чаетъ воскресенія мертвыхъ и жизни будущаго вѣка, тотъ отъ таковыхъ дѣлъ бережется, якоже Апостолъ написалъ: \textit{всякъ, имѣяй надежду сію Нань, очищаетъ себе, якоже Онъ чистъ есть}\footnote{1~Іоан.~3,~3.}, понеже въ животъ вѣчный ничто не внидетъ скверно и нечисто\footnote{Апок.~21,~27.}. \textit{Внѣ бо псы и чародѣи, и любодѣи, и убійцы, и идолослужители и всякъ любяй и творяй лжу}\footnote{22,~15.}.

\paragraph*{§\:288.} Хотя изъ вышеописанныхъ можно познать, что вѣра святая есть корень и источникъ добрыхъ дѣлъ; однакожъ въ семъ параграфѣ представляется тое ради простѣйшихъ, како отъ вѣры, какъ на древѣ отъ кореня плоды, происходятъ добрыя дѣла. Вѣрный человѣкъ въ святомъ Писаніи уподобляется древу: \textit{будетъ яко древо насажденное при исходищихъ водъ, еже плодъ свой дастъ во время свое, и листъ его не отпадетъ, и вся, елика аще творитъ, успѣетъ}\footnote{Пс.~1,~3; Матѳ.~7,~17 и 18.}. Корень древа углубляется въ землѣ: вѣра въ человѣколюбіи и милосердіи Божіи о Хрістѣ Іисусѣ. Корень все древо содержитъ: вѣра вѣрное сердце содержитъ. Отъ корня древо начинается и растетъ: отъ вѣры хрістіанинъ начинается и растетъ; безъ вѣры бо не можетъ быть хрістіанинъ, якоже безъ корня не можетъ быти древо. Чимъ болѣе корень у древа растетъ, тѣмъ большее древо бываетъ: тако, чимъ болѣе вѣра растетъ, тѣмъ болѣе успѣваетъ вѣрный. Что у древа вѣтви, листвіе и плоды, тое у вѣрнаго человѣка дѣла, слова и помышленія. Древо тѣмъ болѣе кверху возвышается, чимъ болѣе растетъ: тако вѣрный, чимъ болѣе въ вѣрѣ и плодахъ ея возрастаетъ, тѣмъ болѣе къ небеснымъ и вѣчнымъ стремится и восхищается. Древо чимъ болѣе плодами обременяется, тѣмъ болѣе вѣтви его книзу ниспускаются: тако вѣрный, чимъ множайшими плодами вѣры украшается, тѣмъ болѣе смиряется; ибо чимъ болѣе кто просвѣщается, тѣмъ болѣе видитъ въ себѣ немощей, недостатковъ и окаянства. Древо, какіе отъ корня плоды пріемлетъ, всѣмъ въ снѣдь подаетъ ихъ: тако вѣрное сердце, какую милость и благодать отъ Бога вѣрою пріемлетъ, всѣмъ тоя удѣляетъ; бываетъ милостивъ, щедръ, миротворецъ, всѣмъ хощетъ и ищетъ спасенія молитвою, совѣтомъ и благовременнымъ обличеніемъ. Древо чимъ болѣе отребляется и очищается, тѣмъ лучшіе и множайшіе плоды приноситъ: тако вѣрный, чимъ болѣе наказуется, гоненія терпитъ и страждетъ, тѣмъ множайшіе и сладчайшіе добродѣтелей плоды приноситъ\footnote{Іоан.~15,~2.}. Древо отъ бури и вѣтровъ колеблется: такъ вѣрное сердце отъ плоти, міра и діавола искушается. Древо корень, въ землѣ водруженный, укрѣпляетъ, чтобы отъ бури и вѣтровъ не упало: тако вѣрнаго вѣра, въ человѣколюбіи Божіи утвержденная, содержитъ и укрѣпляетъ, чтобы отъ искушеній не былъ побѣжденъ. Когда корень у древа начнетъ портиться, древо начнетъ сохнуть и плоды оскудѣваютъ: тако, когда вѣра начнетъ оскудѣвать, и вѣрный оскудѣваетъ въ дѣлахъ своихъ. Какъ корень у древа согніетъ, и древо сохнетъ: тако, когда вѣра въ вѣрномъ совсѣмъ оскудѣетъ, человѣкъ вѣрный развратится. Оскудѣваетъ же вѣра и исчезаетъ отъ невниманія и лѣности и оставленія тѣхъ посредствій, которыми она укрѣпляется; вѣра убо есть всѣхъ благихъ дѣлъ источникъ. Вѣра представляетъ будущая благая, яко настоящая, и невидимая, яко видимая. \textit{Вѣра бо есть уповаемыхъ извѣщеніе, вещей обличеніе невидимыхъ}, по описанію Павла\footnote{Евр.~11,~1.}, "--- отъ чего раждается \textit{надежда}. Вѣра показуетъ Бога, яко Отца человѣколюбива, Отца щедротъ и всякія утѣхи\footnote{2~Кор.~1,~3.}, "--- отъ сего происходитъ \textit{любовь къ Богу}; а отъ любви къ Богу послѣдуетъ \textit{любовь къ ближнему}. Вѣра научаетъ, что Богъ посылаетъ крестъ вѣрнымъ Своимъ не отъ гнѣва, но отъ любви; ибо, \textit{егоже любитъ Господь, наказуетъ}\footnote{Евр.~12,~6.}\textit{}; тѣмъ утѣшаетъ ихъ въ страданіи, и научаетъ \textit{терпѣнію}. Вѣра, когда показуетъ человѣческую немощь, тѣмъ научаетъ, что искушеніе понести самъ собою человѣкъ не можетъ; а тако убѣждаетъ его искать помощи у Бога, Который вся можетъ, и обѣщался искушаемымъ помощи\footnote{Матѳ.~19,~26; 1~Кор.~10,~13; Евр.~2,~18.}; отсюду въ вѣрномъ востаетъ \textit{воздыханіе и молитва}. Вѣра воображаетъ вѣрному величество и всемогущество Божіе, и свое Ему окаянство и подлость представляетъ; отъ того въ вѣрномъ бываетъ \textit{смиреніе} и самого себе уничтоженіе. Вѣра увѣщаваетъ вѣрнаго опасно ходити и обращаться въ прелестномъ мірѣ семъ, чтобы не окаляться сквернами его, и тако не лишиться милости Божіей и не подпасть праведному Его гнѣву, отъ сего послѣдуетъ \textit{страхъ Божій}, который человѣка отъ всякаго зла отводитъ. И тако вѣра истинная бываетъ всѣхъ добрыхъ дѣлъ корень. О семъ еще яснѣйше увидишь въ слѣдующихъ разсужденіяхъ.


\section[Статья 2-я. О святой Церкви и крещеніи.]{статья вторая.\\\bfseries О святой Церкви и крещеніи.}

Отъ Евангелія начинается вѣра, отъ вѣры хрістіанинъ, какъ сказано въ предшедшей статьѣ. Собраніе же хрістіанъ составляетъ церковь, въ которую входъ не инымъ чимъ бываетъ, какъ токмо вѣрою и святымъ крещеніемъ. Того ради въ сей статьѣ вкратцѣ предлагается о церкви и крещеніи святомъ.

\subsection[Глава 1-я. О святой Церкви.]{глава первая.\\\bfseries О святой Церкви.}

\begin{quotation}\textit{Тѣмже убо ктому нѣсте странни и пришельцы, но сожители святымъ и присніи Богу, наздани бывше на основаніи апостолъ и пророкъ, сущу краеугольну Самому Іисусу Хрісту, о Немже всяко созданіе составляемо растетъ въ церковь святую о Господѣ}\footnote{Еф.~2,~19--21.}.\end{quotation}

\paragraph*{§\:289.} Церковь здѣ разумѣется не храмъ Божій, какъ обще наипаче простые люди называютъ; хотя то и храмъ Божій можетъ назватися церковію, потому что въ немъ собираются вѣрные на общую молитву, славословіе Божіе, благодареніе, слышаніе Божія слова, пріобщеніе Таинъ святыхъ, и прочая. Но разумѣется церковь "--- собраніе вѣрныхъ, по всему міру живущихъ, въ Бога и Хріста Сына Божія право и истинно вѣрующихъ, проповѣдію Божія слова просвѣщаемыхъ, и Тайны святыя право содержащихъ. Изъ греческаго языка \textit{Екклесіа} "--- церковь знаменуетъ \textit{вызваніе}; потому что вѣрніи, въ церкви находящіися, \textit{вызваны} изъ области сатанины въ царство Хрістово, отъ тьмы въ чудный Его свѣтъ, какъ учитъ Апостолъ: \textit{вы родъ избранъ, царское священіе, языкъ святъ, люди обновленія, яко да добродѣтели возвѣстите изъ тьмы васъ призвавшаго въ чудный Свой свѣтъ}\footnote{1~Петр.~2,~9; Дѣян.~26,~18.}.

\paragraph*{§\:290.} Церковь святая имѣетъ различныя именованія и уподобленія, которыя ей святое Божіе слово приписуетъ. 1)~Церковь святая называется \textit{духовное Хрістово тѣло}; сынове ея святіи "--- уды духовные, преблагословенную Главу "--- Хріста имѣющіи и признающіи, и вѣрою и любовію оной Главѣ соединенные. Тако именуетъ ее Апостолъ святый: \textit{якоже во единомъ тѣлеси} естественномъ \textit{многи уды имамы: уды же вси не тожде имутъ дѣланіе: такожде мнози едино тѣло есмы о Хрістѣ, а по единому, другъ другу уди}\footnote{Римл.~12,~4 и 5. См. еще: 1~Кор.~10,~17; 12,~27; Еф.~1,~22 и 23; 4,~12 и 16; 5,~23 и 30; Кол.~1,~18 и 24.}. 2)~Называется церковь \textit{домъ Божій}: яко въ ней, какъ въ дому Своемъ, кровію Хрістовою освященномъ, благоволитъ милостивно обитати и сохраняти его: \textit{да увѣси, како подобаетъ въ дому Божіи жити, яже есть церковь Бога жива}\footnote{1~Тим.~3,~15. См. еще: Евр.~3,~6; 1~Петр.~4,~17.}. 3)~Нарицается церковь \textit{невѣста Хрістова}. Тако Апостолъ ее нарицаетъ: \textit{обручихъ васъ} (къ вѣрнымъ глаголетъ) \textit{единому Мужу дѣву чисту представити Хрістови}\footnote{2~Кор.~11,~2. См. еще: Ис.~62,~5; Ос.~2,~19 и 20; Еф.~5,~32; Апок.~19,~7; Пс.~44,~11 и 12.}. Обручители ея суть пророки и апостоли. О чемъ святый Златоустъ на слово оное Павлово: \textit{обручихъ васъ}, и проч., тако бесѣдуетъ: «въ мірѣ семъ дѣвы пребываютъ прежде брака, по брацѣ же не суть дѣвы: здѣже не тако; но аще и не суть дѣвы прежде брака сего, но послѣ брака дѣвы бываютъ. Тако вся церковь дѣва есть. Ибо ко всѣмъ бесѣдуяй глаголетъ и къ мужамъ и женамъ посягшимъ. Но повидимъ, какъ обручилъ насъ, какое вѣно, какіе дары нося: не сребро, ни злато, но царствіе небесное». И мало спустя: «сего образъ было дѣло, бывшее при Авраамѣ. Ибо онъ послалъ раба вѣрнаго обручити отроковицу языческую\footnote{Быт. гл.~24"~я.}. И здѣ послалъ Богъ рабы Своя обручити Сыну Своему церковь, и пророки, свыше вѣщающіе сія: \textit{слыши дщи и виждь и приклони ухо твое, и забуди люди твоя и домъ отца твоего. И возжелаетъ Царь доброты твоея}\footnote{Пс.~44,~11 и 12.}. Виждь и Апостола, со многимъ дерзновеніемъ сей глаголъ вѣщающа и глаголюща: \textit{обручихъ васъ единому Мужу дѣву чисту представити Хрісту}»\footnote{Бес.~23"~я.}. 4)~Называется церковь \textit{матерь вѣрныхъ. Возвеселися неплоды нераждающая, возгласи и возопій нечревоболѣвшая, яко многа чада пустыя паче, нежели имущія мужа}, утѣшаетъ ее Самъ Богъ чрезъ пророка\footnote{Ис.~54,~1.}. Она чадъ своихъ \textit{раждаетъ не отъ сѣмени истлѣнна, но неистлѣнна словомъ живаго Бога и пребывающа во вѣки}\footnote{1~Петр.~1,~23.}. Воспитываетъ ихъ и возращаетъ пищею Божія слова и святыхъ Таинъ\footnote{См. еще Гал.~4,~26 и 27.}. 5)~Нарицается церковь \textit{дворъ овчій}. Тако Хрістосъ нарицаетъ ее: \textit{не входяй дверьми во дворъ овчій, но прелазяй инудѣ, той тать есть и разбойникъ}\footnote{Іоан.~10,~1 и слѣд.}. Въ семъ святомъ дворѣ вѣрные, яко овцы кроткія, незлобивыя, мирны, любовны, простосердечны, водворяются и пасутся отъ Господа, Пастыря своего, душу Свою за нихъ положившаго, якоже поетъ пророкъ: \textit{Господь пасетъ мя, и ничтоже мя лишитъ. На мѣстѣ злачнѣ, тамо всели мя; на водѣ покойнѣ воспита мя}\footnote{Пс.~22,~1 и 2.}. Называется церковь Хрістова \textit{виноградъ}, якоже поетъ пророкъ: \textit{виноградъ изъ Египта принеслъ еси, изгналъ еси языки, и насадилъ еси его}\footnote{79,~9. См. еще: Іер.~2,~21; 12,~10; Матѳ.~20,~1--16; Марк.~12,~1--12; Лук.~20,~9--18.}. Лоза въ виноградѣ семъ есть Хрістосъ Сынъ Божій; дѣлатель есть Богъ и Отецъ Господа Іисуса Хріста, якоже Самъ Хрістосъ глаголетъ: \textit{Азъ есмь лоза истинная и Отецъ Мой дѣлатель}\footnote{Іоан.~15,~1.}. Истинной сей лозѣ прицѣпившіеся рождіе суть вѣрніи Его, которые вѣрою и любовію оной лозѣ присоединены, сокомъ благодати ея оживляются, напояются и плодъ творятъ. 7)~Именуется церковь Хрістова \textit{гора Господня}, Сіонъ святый и вѣрный, въ которомъ Хрістосъ, Сынъ Давидовъ по плоти, со Отцемъ и Святымъ Духомъ жилище Свое имѣетъ. О сей Господней горѣ глаголетъ пророкъ: \textit{будетъ въ послѣдняя дни явлена гора Господня, и домъ Божій на версѣ горъ, и возвысится превыше холмовъ; и пріидутъ къ ней вси языцы, и пойдутъ языцы мнози, и рекутъ: пріидите, и взыдемъ на гору Господню, и въ домъ Бога Іаковля}\footnote{Ис.~2,~2 и 3. См. еще: Пс.~14,~1; 2,~6; 23,~3; 47,~2; 67,~16; 73,~2 и проч.}. 8)~Уподобляется церковь Хрістова \textit{кораблю}, который на морѣ міра сего плаваетъ, и къ тихому вѣчнаго покоя пристанищу стремится. Якоже бо корабль на морѣ волнами, тако церковь Хрістова въ мірѣ семъ бѣдами искушеній непрестанно обуревается, но не погружается. Кормчій сего спасительнаго корабля есть Самъ Хрістосъ, Сынъ Божій, \textit{вѣтрамъ и морю повелѣвающій}\footnote{Матѳ.~8,~26 и 27.}. Сей великій и безопасный корабль прознаменованъ ковчегомъ Ноевымъ, по общему святыхъ отецъ и учителей ученію. Якоже бо Ной святый со всѣмъ домомъ своимъ въ ковчегѣ спасся отъ всемірнаго потопа, а вси, внѣ ковчега бывшіи, погибли: тако и нынѣ тіи только спасаются отъ потопа грѣховнаго, гнѣва Божія и вѣчнаго осужденія, которые въ церкви святой находятся, и истинными сынами ея пребываютъ; прочіи же всѣ, внѣ пребывающіи, погибаютъ и потопляются въ потопѣ бездны адской. Суть и иная имена и подобія церкви святой, въ святомъ Писаніи положенная; но сія оставляю читающимъ со вниманіемъ Писаніе святое. Сія простѣйшимъ людямъ къ наставленію и душевной пользѣ довлѣютъ.

\paragraph*{§\:291.} Сіи имена, воистину великія и преславныя, церкви святой отъ Духа Святаго приписанныя, ко утѣшенію и наставленію нашему, хрістіанине, служатъ. Ко \textit{утѣшенію}: яко когда такъ высоко церковь отъ Бога превознесена, то и ты, аще истинный ея сынъ еси, той чести и славы пріобщаешися; и отъ сего разсужденія можетъ въ сердцѣ твоемъ послѣдовати такая радость, что никакая скорбь и печаль и бѣда, въ мірѣ семъ случающаяся, истинно опечалити тя не можетъ; и тако посредѣ самыхъ свирѣпыхъ искушенія волнъ будеши дерзати и съ Апостоломъ глаголати: \textit{кто ны разлучитъ отъ любве Божія, яже о Хрістѣ Іисусѣ Господѣ нашемъ}\footnote{Римл.~8,~35 и 39.}? \textit{Къ наставленію} служитъ разсужденіе сіе: ибо ежели церковь есть \textit{тѣло Хрістово духовное}, и ты удъ благословеннаго того тѣла хощешь быти, то разсуди самъ, какъ свято и чисто жити тебѣ, какую любовь какъ къ святѣйшей Главѣ той, такъ и къ святымъ ея удамъ, то есть, истиннымъ хрістіанамъ, имѣти должно; какъ союзно съ прелестною Главою и членами ея жити! Въ сіе святое общеніе ничто не входитъ скверное и нечистое. \textit{Кое бо причастіе правдѣ къ беззаконію? или кое общеніе свѣту ко тьмѣ?} глаголетъ Апостолъ\footnote{2~Кор.~6,~14.}. Отсюду святіи Апостоли вездѣ въ Посланіяхъ своихъ такъ сильно увѣщаваютъ и поощряютъ хрістіанъ къ святому житію и къ взаимной любви и миру. Примѣчай сіе, хрістіанине, когда хощешь удомъ церкви святыя быти. \textit{Святъ Господь}, Глава церкви; \textit{свято и тѣло Его}, церковь. \textit{Очистимъ себе} и мы \textit{отъ всякія скверны плоти и духа, творяще святыню въ страсѣ Божіи}, по увѣщанію апостольскому\footnote{2~Кор.~7,~1.}, когда не хощемъ быть отверженными отъ святаго сего общества. Ежели церковь есть \textit{домъ Божій}, жилище и градъ небеснаго Царя, и ты гражданинъ и житель преславнаго сего града хощеши быть: съ какою охотою и усердіемъ должно тебѣ повиноватися закону небеснаго Царя, слушати Его и работати Ему, чтобы не лишитися гражданства и общенія святыхъ и благословенныхъ гражданъ, живущихъ въ преславномъ градѣ семъ! Самъ разсуди, кто царю своему не работаетъ? кто законъ его не хранитъ, и безъ наказанія бываетъ? Кто и Хрістовъ рабъ можетъ быть, когда Хрісту Царю не работаетъ вѣрою и правдою? Како и безъ наказанія можетъ быть, когда Ему не работаетъ, противится оному Духа Святаго повелѣнію: \textit{работайте Господеви со страхомъ}\footnote{Пс.~2,~11.}. Ежели церковь есть \textit{Хрістова невѣста}, и въ томъ духовномъ и таинственномъ бракосочетаніи находишься: какую вѣрность, чистоту и любовь хранить долженъ ты къ безсмертному и небесному Жениху "--- Хрісту, Сыну Божію; какъ храниться отъ \textit{похоти плотской, похоти очесъ и гордости житейской}\footnote{1~Іоан.~2,~16.}, яко любодѣйцы вавилонской! Ежели церковь Хрістова есть \textit{мать вѣрныхъ} и твоя: какъ должно тебѣ ее любить, почитать, слушать, яко родившую тя водою и духомъ, и повиноваться спасительнымъ ея наставленіямъ! Ежели церковь Хрістова есть \textit{дворъ овчій}, и ты имѣешися между овцами, въ томъ дворѣ находящимися: помышляй, какъ должно тебѣ слушать гласа Хріста, пастыря твоего, \textit{Иже есть пастырь добрый}, и есть пастырь \textit{Своимъ} овцамъ, якоже Самъ глаголетъ: \textit{овцы Моя гласа Моего слушаютъ}\footnote{Іоан.~10,~11,~14 и 27.}. Какъ, глаголю, должно повиноватися пастырю тому доброму, когда не хощешь отъ стада Его благословеннаго заблудить и быть снѣдію волка пагубнаго, \textit{діавола}, который, \textit{яко левъ рыкая, ходитъ искій, кого поглотити}\footnote{1~Петр.~5,~8.}! Ежели церковь Хрістова есть \textit{виноградъ Божій}, и ты въ томъ находишься: какъ крѣпко должно тебѣ держаться вѣрою и любовію \textit{истинной лозы} "--- Хріста, и плодъ творить угодный \textit{дѣлателю} "--- Богу, чтобы, какъ \textit{розгѣ} непотребной, не творящей добра плода, не извергнуться вонъ, и, яко изсохшей, не предаться на сожженіе вѣчнаго огня, якоже глаголетъ Хрістосъ: \textit{аще кто во Мнѣ не пребудетъ, извержется вонъ, якоже розга, и изсышетъ, и собираютъ ю, и во огнь влагаютъ, и сгараетъ}\footnote{Іоан.~15,~6.}; и Предтеча Его святый: \textit{всяко древо, еже не творитъ плода добра, посѣкаемо бываетъ, и во огнь вметаемо}\footnote{Матѳ.~3,~10.}. Ежели церковь Хрістова есть \textit{гора Господня}, и ты въ сей горѣ водворяешися; то тебѣ должно отъ скверныхъ міра сего болотъ и удолій, въ которыхъ всякая нечистота и смрадъ находится, удаляться и возвышаться духомъ къ \textit{небесной горѣ}, въ которой есть блаженныхъ духовъ жилище, покой и торжество\footnote{Евр.~12,~22 и 23.}. Аще наконецъ церковь Хрістова есть \textit{корабль}, спасающій насъ отъ гнѣва Божія и страшнаго потока адскаго, и ты въ немъ находишися, "--- берегись прогнѣвлять кормчія "--- Хріста, дабы изъ сего спасительнаго корабля не выпасть и въ вѣчномъ гнѣва Божія и суда потопѣ не погибнуть. Ибо въ семъ единомъ ковчегѣ спасаются вси, которые ни спасаются. Видиши убо, возлюбленный хрістіанине, что есть церковь Хрістова, какое ея великолѣпіе, и блажени суть, которые истинными ея сынами имѣются. "--- А отъ вышеписанныхъ именъ церкви и приложенныхъ наставленій примѣтить можешь: 1)~Кто есть истинный церкви святыя сынъ. То"=есть тотъ, который право и сердечно въ Бога и Хріста Сына Божія вѣруетъ, банею святаго крещенія омовенъ, и вѣры своея и новаго духовнаго рожденія плоды показуетъ, благочестно о Хрістѣ Іисусѣ живетъ, истинный страхъ Божій имѣетъ, противу всякаго грѣха подвизатися и добрыя дѣла творить тщится. 2)~Что хрістіане, безстрашно живущіи, покаянія и плодовъ его не творящіи, до церкви святой не надлежатъ, хотя и крещены во имя Святыя Троицы, понеже вѣры, какъ сказано въ предшедшей статьѣ, не имѣютъ. Безъ вѣры бо удомъ церкви святыя быти невозможно. 3)~Кто отъ нихъ чистосердечно обратится и покаянія плоды сотворитъ, тотъ паки въ церковное сообщеніе пріимется милосердіемъ Божіимъ, и будетъ истиннымъ церкви святыя сыномъ. О семъ ниже еще узриши яснѣе.

\paragraph*{§\:292.} Церковь святая, какъ была и есть, такъ и до скончанія вѣка будетъ, по обѣщанію Господню: \textit{врата адова не одолѣютъ ей}\footnote{Матѳ.~16,~18.}. \textit{Се Азъ съ вами есмь во вся дни до скончанія вѣка}, глаголетъ Хрістосъ\footnote{28,~20.}.

\paragraph*{§\:293.} Святая церковь есть \textit{едина}. Понеже 1)~\textit{единъ} есть \textit{Богъ} "--- Отецъ, Сынъ и Святый Духъ, Которому вѣруетъ, служитъ, поклоняется и почитаетъ. 2)~\textit{Едино основаніе} церкви есть Іисусъ Хрістосъ\textit{. Основанія инаго никтоже можетъ положити паче лежащаго, еже есть Іисусъ Хрістосъ}\footnote{1~Кор.~3,~11.}. 3)~\textit{Едино} есть \textit{ученіе Божія слова}, которымъ наставляется, просвѣщается, укрѣпляется и спасается. 4)~\textit{Едина} есть \textit{вѣра}, которую имѣетъ во единаго Тріѵпостаснаго Бога. 5)~\textit{Едины Тайны} святыя, которыми спасается. 6)~\textit{Едино} есть духовное и таинственное \textit{тѣло}, котораго глава Хрістосъ. 7)~\textit{Единъ духъ}, которымъ оживотворяется, водится и освящается. 8)~\textit{Едина надежда} воскресенія мертвыхъ и жизни будущаго вѣка. Сіе все Апостолъ святый вкратцѣ заключилъ: \textit{едино тѣло, единъ духъ, якоже и звани бысте во единомъ упованіи званія вашего; единъ Господь, едина вѣра, едино крещеніе; единъ Богъ и Отецъ всѣхъ, Иже надъ всѣми и чрезъ всѣхъ и во всѣхъ насъ}\footnote{Еф.~4,~4--6.}. Аще бо и разсѣяны суть вѣрніи и святіи Божіи по лицу всея земли, однакожъ едино благословенное общество составляютъ, ради вышеписанныхъ причинъ.

\paragraph*{§\:294.} Церковь святая, которая на земли въ мірѣ семъ имѣется, называется \textit{воинствующая}, яко съ помощію Божіею подвизается противу враговъ своихъ: діавола, плоти, міра и грѣха. Къ ней глаголетъ Апостолъ святый, храбрый Хрістовъ воинъ: \textit{братіе моя! возмогайте о Господѣ, и въ державѣ крѣпости Его; облецытеся во вся оружія Божія, яко возмощи вамъ стати противу кознемъ діавольскимъ: яко нѣсть наша брань противу крови и плоти, но} противу \textit{началъ и} противу \textit{міродержителей тьмы вѣка сего}, противу \textit{духовъ злобы поднебесныхъ}. \textit{Сего ради пріимите вся оружія Божія, да возможете противитися въ день лютъ, и вся содѣявше стати. Станите убо препоясани чресла ваша истиною, и оболкшеся въ броня правды, и обувше нозѣ во уготованіе благовѣствованія мира; надъ всѣми же воспріимше щитъ вѣры, въ немже возможете вся стрѣлы лукаваго разженныя угасити, и шлемъ спасенія воспріимше, и мечь духовный, иже есть глаголъ Божій}\footnote{Еф.~6,~10--17.}. Которая \textit{(церковь)} въ небесномъ отечествѣ по трудахъ и подвигахъ упокоевается, и чаетъ воскресенія мертвыхъ, и по немъ совершеннѣйшаго съ прославленными тѣлесами, съ которыми здѣ подвизалася, воздаянія, "--- называется отъ святыхъ отецъ и учителей \textit{торжествующая}, яко надъ діаволомъ, смертію и адомъ побѣдоносно торжествуетъ и совокупно съ ликомъ ангельскимъ поетъ и славитъ Бога. Примѣчай, хрістіанине: 1)~Кто въ воинствующей церкви имѣется и подвизается противу діавола и грѣха, тотъ при окончаніи временнаго живота преселяется въ торжествующую церковь, аще до конца вѣренъ пребудетъ. \textit{Претерпѣвый бо до конца, той спасется}\footnote{Матѳ.~24,~13.}. И паки: \textit{буди вѣренъ даже до смерти, и дамъ ти вѣнецъ живота}, глаголетъ Хрістосъ\footnote{Апок.~2,~10.}. 2)~По общемъ воскресеніи и окончаніи всемірнаго суда Хрістова, едина торжествующая будетъ церковь, которая безъ конца будетъ \textit{видѣть Бога лицемъ къ лицу}\footnote{1~Кор.~13,~12.}, и отъ того радоватися, веселитися, восклицати, и въ радости духа хвалить безконечную Его благость. 3)~Мы, любезный хрістіанине, потщимся истинными сынами быти церкви святыя подвизающіяся, да и въ торжествующую достигнемъ благодатію Хріста Спаса нашего, Которому со Отцемъ и Святымъ Духомъ буди честь и слава и благодареніе во вѣки. Аминь.

\textit{Помяни насъ, Господи, во благоволеніи людей Твоихъ, посѣти насъ спасеніемъ Твоимъ, видѣти во благости избранныя Твоя, возвеселитися въ веселіи языка Твоего, хвалитися съ достояніемъ Твоимъ}\footnote{Пс.~105,~4 и 5.}.

\subsection[Глава 2-я. О святомъ крещеніи.]{глава вторая.\\\bfseries О святомъ крещеніи.}

\begin{quotation}\textit{Иже вѣру иметъ и крестится, спасенъ будетъ}\footnote{Марк.~16,~16.}.\end{quotation}
\begin{quotation}\textit{Егда благодать и человѣколюбіе явися Спаса нашего Бога, не отъ дѣлъ праведныхъ, ихже сотворихомъ мы, но по Своей Его милости, спасе насъ, банею пакибытія, и обновленія Духа Святаго, Егоже излія на насъ обильно, Іисусъ Хрістомъ Спасителемъ нашимъ, да оправдившеся благодатію Его, наслѣдницы будемъ по упованію жизни вѣчныя}\footnote{Тит.~3,~4--7.}.\end{quotation}

\paragraph*{§\:295.} Крещеніе святое есть какъ дверь, которою крестящіися входятъ въ церковь Божію. Вѣра во Хріста, Сына Божія, есть какъ ключь, которымъ спасительная сія отверзается дверь. Ибо хотящему креститися должно прежде вѣровать во Хріста, яко единаго Избавителя и Спасителя міра и инаго посредствія къ полученію вѣчныя жизни не знать, кромѣ Его, яко \textit{никтоже пріидетъ ко Отцу, токмо Имъ}\footnote{Іоан.~14,~6.}, "--- и тако съ сею вѣрою \textit{креститься во имя} Тріѵпостаснаго Бога "--- \textit{Отца и Сына и Святаго Духа}\footnote{Матѳ.~28,~19.}. Тако евнухъ, поминаемый въ Дѣяніяхъ Апостольскихъ, первѣе исповѣдуетъ вѣру во Хріста, Сына Божія: \textit{вѣрую Сына Божія быти Іисуса Хріста}, и потомъ \textit{крестися}\footnote{Дѣян.~8,~37 и 38.}. И Хрістосъ глаголетъ: \textit{иже вѣру иметъ и крестится, спасенъ будетъ}.

\paragraph*{§\:296.} При крещеніи святомъ бываетъ: 1)~Отрицаніе сатаны, и всѣхъ дѣлъ его, и аггелъ его и проч., какъ въ уставѣ о крещеніи изображено. Сіе отрицаніе возрастные сами чинятъ, а вмѣсто младенцевъ воспріемники ихъ. Сатана есть духъ лукавый, созданный отъ Бога добрымъ, но своею волею отступилъ отъ Создателя своего, и тако сдѣлался злымъ и начальникомъ всякія злобы и грѣха; аггели его суть такожде духи, созданные отъ Бога добрыми, но съ нимъ отступили отъ Творца своего, и суть такожде духи злые и роду человѣческому, какъ и начальникъ ихъ, весьма враждебные\footnote{2~Петр.~2,~4; Іуд. ст.~6; Іоан.~8,~44.}. Дѣла его суть дѣла темная, грѣхи и беззаконія. Онъ есть начальникъ и изобрѣтатель грѣха, въ который и прародителей нашихъ, а съ ними и насъ, лукавствомъ своимъ вринулъ и отъ Бога отвелъ\footnote{Быт.~3,~1--7.}, и въ свою темную область\footnote{Дѣян.~26,~18.} привлеклъ. Служитъ сатанѣ, кто, волю его злую исполняя, творитъ грѣхъ, яко \textit{творяй грѣхъ отъ діавола есть, яко исперва діаволъ согрѣшаетъ}, по свидѣтельству Апостола\footnote{1~Іоан.~3,~8.}. Сего всего, то"=есть, сатаны, злыхъ его дѣлъ, злаго и пагубнаго его служенія, при крещеніи отрекаемся. Ибо вси, которые внѣ церкви безъ вѣры истинной во Хріста находятся, въ царствіи и области діавольской имѣются, служатъ и работаютъ ему, какъ плѣнники\footnote{Іоан.~3,~44; 1~Іоан.~3,~8; 2~Тим.~2,~26; Дѣян.~26,~18; 2~Петр.~2,~9 и слѣд.}. \textit{Имже кто побѣжденъ бываетъ, сему и работенъ есть}\footnote{19~ст.}. Вѣрующіи же истинно и сердечно во Хріста Избавителя и крестящіися, отъ области его свобождаются, по свидѣтельству Самаго Сына Божія: \textit{аще Сынъ вы свободитъ, воистину свободни будете}\footnote{Іоан.~8,~36.}. Сего нашего свобожденія прообразованіемъ было свобожденіе сыновъ Израилевыхъ изъ работы Египетскія\footnote{Исх.~14"~я.}. Якоже бо Израильтяне чрезъ прешествіе Чермнаго моря силою Божіею свободилися отъ работы Фараоновой, тако хрістіане, новый Израиль и духовные сынове Авраамли, вѣрою и святымъ крещеніемъ избавляются отъ работы діавольской. Тамо предводитель и свободитель, съ помощію Божіею, былъ Моисей: здѣ же Хрістосъ, Сынъ Божій, Избавитель и Искупитель міра, Который вѣрующихъ въ Него изводитъ отъ области и работы діавольскія, и приводитъ въ пресладкую царствія Своего свободу\footnote{Гал.~5,~1.}. 2)~Отрекшися сатаны и сквернаго служенія его, обѣщаемся до смерти служить Хрісту "--- Сыну Божію со Отцемъ и Святымъ Духомъ\footnote{Евр.~9,~14.}, \textit{служить же преподобіемъ и правдою предъ Нимъ вся дни живота нашего}\footnote{Лук.~1,~75.}. И какъ воини міра сего и прочіи, въ чины и ранги входящіи, обѣщаются и обѣщаніе свое присягою утверждаютъ, что монарху и обществу имѣютъ служить вѣрою и правдою (сія есть сила присяги): тако крестящіися записываются въ службу Хрістову и обѣщаются Ему со Отцемъ и Духомъ вѣрно работать, повелѣнія Его слушать и исполнять, безъ чего вѣрная служба быть не можетъ; и тое обѣщаніе свое троекратно, какъ и отрицаніе, повторяютъ, чего и знаменіе на себѣ носятъ, то"=есть, нарицаются \textit{хрістіанами}, отъ Хріста, Царя своего. Тако, какъ видишь, хрістіанине, въ святомъ крещеніи въ завѣтъ съ Богомъ вступаемъ. Мы Богу обѣщаемся вѣрою и правдою служить, то"=есть, послушаніе Ему показывать, отъ всякаго грѣха удаляться и правду хранить; а Онъ насъ въ высочайшую Свою милость пріемлетъ, о чемъ яснѣйше въ слѣдующихъ увидиши.

Слѣдственно 1)~должно намъ, хрістіанине, отрекшимся сатаны и злыхъ дѣлъ его, не возвращаться паки къ нему и къ злымъ его дѣламъ, да не во вѣки въ темной его области съ крайнею бѣдою нашею останемся. 2)~Обѣты, учиненные Богу при крещеніи, хранить. \textit{Онъ вѣренъ пребываетъ: отрещися бо Себѣ не можетъ}\footnote{2~Тим.~2,~13.}. Какъ пріемлетъ въ милость Свою вѣрующихъ въ Него, такъ и во вѣки содержать въ милости обѣщается: \textit{яже отъ Мене къ тебѣ милость не оскудѣетъ}\footnote{Ис.~54,~10.}. И на прочіихъ святаго Писанія мѣстахъ тоежде глаголется намъ во утѣшеніе наше. Должно убо и намъ вѣрными быть, и быть до смерти, да получимъ Его милостивое обѣщаніе, по реченному: \textit{буди вѣренъ даже до смерти, и дамъ ти вѣнецъ живота}\footnote{Апок.~2,~10.}. 3)~Сіи обѣты да обращаются предъ очесами родителей, и да воспоминаютъ ихъ дѣтямъ своимъ малымъ, какъ только начнутъ въ возрастъ и разумъ приходить и понимать ученія слово; и тако юность, ко злу склонную, ими, какъ уздою, да удерживаютъ отъ страстей безчинныхъ, и въ страсѣ Божіи да содержатъ. Да обращаются всегда предъ пастырями и поощряютъ ихъ прилѣжно хранить стадо Хрістовыхъ овецъ, имъ порученныхъ, и имъ тыежде часто предлагать. Сіи обѣты да отвращаютъ судей отъ неправды, и да научаютъ хранить правду, удерживать руки свои отъ мздоиманія, не пріимать въ судѣ лица, отирать слезы обиженныхъ, и обидящихъ смиряти. Сіи обѣты да поощряютъ господъ съ рабами своими хрістіанами обходиться, яко съ братіею; яко господа и раби единаго Господа на небесѣхъ имѣютъ, Которому обои обѣщались работать вѣрно. Да обращаются наконецъ предъ всѣми, всякаго чина, званія, обоего пола и всякаго возраста людьми, которые въ крещеніи имена свои и себе записали и отдали Хрісту, и да побуждаютъ къ истинному благочестію, да тако вѣрными Его рабами будутъ, и не постыдятся въ день славнаго Его втораго пришествія. 4)~Крещеніе святое не пользуетъ ничего хрістіанамъ тѣмъ, которые обѣтовъ своихъ, при крещеніи учиненныхъ, не тщатся хранить, но \textit{возвращаются вспять отъ преданныя имъ святыя заповѣди}\footnote{2~Петр.~2,~22.}. Образъ сего видимъ на Израильтянахъ, которые вышли изъ Египта и перешли Чермное море, но сердцами своими паки возвратилися во Египетъ, и \textit{похотѣша желанію въ пустыни}\footnote{Пс.~105,~14 и пр.}: \textit{иже сѣдоша ясти и пити, и восташа играти, иже соблудиша, иже искусиша Бога, иже ропташа и клеветаша на Бога}, и прочая беззаконія дѣлаша\footnote{1~Кор.~10,~7--10.}. Ибо не дошли до земли обѣтованной; но поражени суть въ пустыни, якоже о томъ пишется въ книгахъ Моисеовыхъ, и Апостолъ приводитъ въ вышеписанномъ посланіи: \textit{не хощу васъ не вѣдѣти, братіе, яко отцы наши вси подъ облакомъ быша, и вси сквозѣ море проидоша, и вси въ Моѵсеа крестишася во облацѣ и въ мори}, и прочая; \textit{но не во множайшихъ ихъ благоволи Богъ: поражени бо быша въ пустыни}, то"=есть, за беззаконія. \textit{Сія же образы намъ быша}, и прочая\footnote{ст.~1--6.}. Тако не пользуетъ крещеніе хрістіанамъ, которые сердцами своими обращаются къ міру, какъ Египту, и работаютъ похотямъ плоти своея, любятъ честь, славу, богатство и прочую міра сего суету, и тое все, какъ бога своего, имѣютъ и почитаютъ. Все бо тое человѣку вмѣсто Бога есть, къ чему онъ сердцемъ своимъ приложился. Не достигнутъ и сіи въ страну живыхъ и селеніе праведныхъ, гдѣ вси праотцы и отцы наши упокояются, веселятся и благоугождаютъ предъ Господемъ\footnote{Пс.~104,~8.}. Ибо \textit{внѣ суть псы и чародѣи, и любодѣи, и убійцы, и идолослужители, и всякъ любяй и творяй лжу}\footnote{Апок.~22,~15.}. Таковымъ хрістіанамъ, которые, по пріятіи дара въ крещеніи святомъ, обращаются на нечестивое и безбожное житіе, случается тое, что Апостолъ написалъ: \textit{быша имъ послѣдняя горша первыхъ. Лучше бо бѣ имъ не познати пути правды, нежели познавшимъ возвратитися вспять отъ преданныя имъ святыя заповѣди. Случися бо имъ истинная притча: песъ возвращся на свою блевотину, и свинія, омывшися, въ калъ тинный}\footnote{2~Петр.~2,~20--22.}. И Хрістосъ тое изображаетъ, глаголя: \textit{егда нечистый духъ изыдетъ отъ человѣка, преходитъ сквозѣ безводная мѣста, ища покоя, и не обрѣтаетъ. Тогда речетъ: возвращуся въ домъ мой, отнюдуже изыдохъ: и пришедъ обрящетъ празденъ, пометенъ и украшенъ. Тогда идетъ, и пойметъ съ собою седмь иныхъ духовъ лютѣйшихъ себе, и вшедше живутъ ту: и будутъ послѣдняя человѣку тому горша первыхъ}\footnote{Матѳ.~12,~43--45.}. Въ крещеніи святомъ освобождаемся отъ духа нечистаго и темныя его власти, и всякому глаголется, что сказано отъ Хріста разслабленному исцѣлѣвшему: \textit{се здравъ еси, ктому не согрѣшай, да не горше ти что будетъ}\footnote{Іоан.~5,~14.}. Слышатъ вси крестившіися отъ служителей Хрістовыхъ реченное Апостоломъ: \textit{омыстеся, освятистеся, оправдистеся именемъ Господа нашего Іисуса Хріста и Духомъ Бога нашего}\footnote{1~Кор.~6,~11.}. Но когда крестившійся человѣкъ на первый нечестія путь обращается, и \textit{отбѣгше сквернъ міра познаніемъ Господа и Спаса нашего Іисуса Хріста, сими же паки сплетшеся, побѣжденъ бываетъ}\footnote{2~Петр.~2,~20.}: \textit{бываютъ ему послѣдняя горша первыхъ}, по ученію Хрістову и Апостольскому. Откуду заблуждшій и развращенный хрістіанинъ горшій бываетъ язычника честнаго, такъ что которыми грѣхами и беззаконіями язычники честніи гнушаются, тѣми развращенный хрістіанинъ услаждается. Отсюду бываетъ, что у таковыхъ хрістіанъ за мало, или, что горше того, за ничто поставляется похищать чужое, красть, присягу, предъ Богомъ"=Сердцевѣдцемъ учиненную, нарушать, неправду въ судѣ дѣлать, неповиннаго обвинять, виновнаго оправдать, слезы бѣдныхъ, вдовицъ, сиротъ и прочіихъ беззаступныхъ проливать, льстить, обманывать, лукавновать, піянствовать, прелюбодѣйствовать и на прочія беззаконія безстрашно дерзать, отъ которыхъ политичные язычники отвращаются, какъ читаемъ въ исторіяхъ и нынѣ слышимъ. И тако гдѣ нечистый единъ духъ былъ, тамо уже \textit{седмь духовъ лютѣйшихъ} бываютъ, и бѣднаго человѣка отъ грѣха въ грѣхъ и отъ беззаконія въ беззаконіе женутъ, чего ради \textit{бываютъ ему послѣдняя горша первыхъ}. 6)~Отсюду послѣдуетъ, что таковые хрістіане большему и ужаснѣйшему суду Божію и вѣчному осужденію подлежатъ, нежели язычники. Ибо, по словеси Хрістову, \textit{рабъ, вѣдѣвый волю господина своего, и не уготовавъ, ни сотворивъ по воли его, біенъ будетъ много: невѣдѣвый же, сотворивъ же достойная ранамъ, біенъ будетъ мало}\footnote{Лук.~12,~47 и 48; Римл.~2,~5 и 6.}. 7)~Согрѣшившимъ противу обѣтовъ своихъ, при крещеніи учиненныхъ, осталася надежда "--- истинное покаяніе (о которомъ сказано въ первой книгѣ). Того ради, когда хотятъ у Бога милость получить, должно имъ чистосердечно обратитися къ Богу, Котораго оставили. Аще бо они и не сохранили вѣры своея къ Богу, но Богъ вѣренъ пребываетъ, Который обѣщалъ и покаяніе грѣшникамъ, и кающихся паки въ Свою милость принять. Имѣемъ образы согрѣшившихъ и покаявшихся: Давида, Манассію, Петра и прочіихъ, и паки милость получившихъ у Бога. Богъ нынѣ каятися всѣмъ повелѣваетъ, и призываетъ всѣхъ къ покаянію. Нынѣ Его милость по большей части дѣйствуетъ; нынѣ зоветъ, обѣщаетъ и пріемлетъ кающихся; нынѣ слушаетъ вопіющихъ и воздыхающихъ къ Нему; нынѣ помогаетъ обращающимся и труждающимся въ дѣлѣ спасенія. \textit{Се нынѣ время благопріятно, се нынѣ день спасенія}\footnote{2~Кор.~6,~2.}, \textit{просить, искать и толкать}\footnote{Мѳ.~7,~7.}, толкати въ двери милосердія Его. Будетъ время, когда не будетъ мѣсто прошенія, но суда; когда правда Его въ свое дѣйствіе вступитъ, и воздастъ \textit{коемуждо по дѣломъ его}\footnote{Римл.~2,~6.}. Нынѣ Богъ зоветъ грѣшниковъ на покаяніе, и обѣщаетъ имъ милость: тогда призоветъ на судъ и осудитъ грѣшниковъ. Нынѣ глаголетъ: \textit{пріидите ко Мнѣ}\footnote{Мѳ.~11,~28.}: тогда возглаголетъ: \textit{идите отъ Мене}\footnote{25,~41.}. Послушаемъ убо, грѣшниче, Бога призывающаго, и покаемся, да и мы отъ Него милость получимъ, и получимъ, что \textit{любящимъ Его обѣщалъ}\footnote{1~Кор.~2,~9.}.

\paragraph*{§\:297.} Понеже въ крещеніи святомъ сподобляемся вѣрою отъ Бога всѣхъ духовныхъ благъ; то должно, хотя вкратцѣ, коснуться того, отъ чего сія Божія благодать на насъ изливается.

\textit{Источникъ} сей \textit{спасительный}, отъ котораго проистекаетъ намъ Божественная благодать, есть: 1)~Любовь Божія къ намъ недостойнымъ. \textit{Тако бо возлюби Богъ міръ, яко и Сына Своего единороднаго далъ есть, да всякъ, вѣруяй въ Онь, не погибнетъ, но имать животъ вѣчный}\footnote{Іоан.~3,~16.}. 2)~Вольное Хріста Сына Божія къ роду нашему снисхожденіе и умаленіе. \textit{Иже}, какъ учитъ Апостолъ, \textit{въ образѣ Божіи сый, не восхищеніемъ непщева быти равенъ Богу; но Себе умалилъ, зракъ раба пріимъ, въ подобіи человѣчестѣмъ бывъ, и образомъ обрѣтеся, якоже человѣкъ; смирилъ Себе, послушливъ бывъ даже до смерти, смерти же крестныя}\footnote{Филип.~2,~6--8.}. Сіе Сына Божія насъ ради снисхожденіе, которое принялъ своею волею и благоволеніемъ небеснаго Своего Отца, и содѣйствіемъ Святаго Духа, есть виновное всего нашего духовнаго блаженства, которое отъ щедрой Божіей руки въ крещеніи вѣрою въ Него получаемъ. Насъ ради явился на земли, насъ ради родился плотію, да мы родимся духомъ; насъ ради пожилъ на земли, да насъ, изъ рая изгнанныхъ, на небо возведетъ; насъ ради хуленія страшная претерпѣлъ, да врагу нашему, діаволу, который \textit{оклеветаваетъ насъ предъ Богомъ нашимъ день и нощь}\footnote{Апок.~12,~10.}, заградитъ уста. Но посмотримъ еще на подвигъ страданія Хрістова, въ который насъ ради вступилъ. Внидемъ вопервыхъ умомъ нашимъ, внидемъ и сердцемъ въ весь, нарицаемую \textit{Геѳсиманія}, и прилѣжно посмотримъ на подвигъ Его, которымъ грѣхъ ради нашихъ тамо подвизался. \textit{И поятъ Петра и Іакова и Іоанна съ Собою, и начатъ ужасатися и тужити; и глагола имъ: прискорбна есть душа Моя до смерти}\footnote{Марк.~14,~33 и 34.}. Ужасается и тужитъ Сынъ Божій Господь нашъ, да отъ насъ отъиметъ страхъ, трепетъ и ужасъ вѣчнаго во адѣ осужденія, вѣчную тугу и тоску, которымъ мы грѣхъ ради нашихъ подвержены были. Скорбитъ душею до смерти, да насъ душевныя скорби избавитъ и подастъ вѣчную радость. Посмотримъ далѣе на подвигъ Хрістовъ. Продается и предается отъ неблагодарнаго ученика въ руки беззаконныхъ\footnote{Марк.~14,~10,~11,~41--46.} да насъ, \textit{проданныхъ подъ грѣхъ}, искупитъ\footnote{Ис.~50,~1; Римл.~7,~14.}. Оставляется единъ отъ всѣхъ учениковъ, "--- \textit{тогда ученицы вси, оставльше Его, бѣжаша}\footnote{Мѳ.~26,~56.}, "--- да насъ оставленныхъ, беззаступныхъ, безпомощныхъ и отверженныхъ отъ лица Божія, заступитъ и поможетъ намъ, якоже Самъ чрезъ пророка глаголетъ: \textit{и воззрѣхъ, и не бѣ помощника; и помыслихъ, и никтоже заступи: и избави ихъ мышца Моя}\footnote{Ис.~63,~5.}. Емлется отъ беззаконныхъ, и связуется заступникъ нашъ Іисусъ\footnote{Іоан.~18,~12.}, да наши растерзаетъ узы. Мы за свои грѣхи достойны были связаны быть узами адскими, и отъ діавола и аггеловъ его на судъ Божій вестися и принять по дѣломъ нашимъ; но Іисусъ, Сынъ Божій, неправедно связанъ былъ, да наши растерзаетъ узы, якоже съ Давидомъ благодарно поемъ Ему: \textit{растерзалъ еси узы моя. Тебѣ пожру жертву хвалы}\footnote{Пс.~115,~7 и 8.}. Сію благодать подалъ и служителямъ Своимъ: \textit{елика аще разрѣшите на земли, будутъ разрѣшена на небесѣхъ}\footnote{Мѳ.~18,~18.}. "--- Пойдемъ умомъ и сердцемъ нашимъ во дворъ беззаконнаго архіерея, куды приведенъ Іисусъ, Заступникъ нашъ, и посмотримъ на судъ, на которомъ связанъ стоитъ связавый море повелѣніемъ Своимъ, и судяй праведно всему міру отъ беззаконныхъ судится. \textit{Архіерее же и весь сонмъ}, глаголетъ Евангелистъ, \textit{искаху на Іисуса свидѣтельства, да умертвятъ Его: и не обрѣтаху}\footnote{Марк.~14,~55.}. Ищутъ свидѣтельства, но не обрѣтаютъ, ибо ищутъ на Того, \textit{Иже грѣха не сотвори, ни обрѣтеся лесть во устѣхъ Его}\footnote{1~Петр.~2,~22.}. \textit{Мнози лжесвидѣтельствоваху на Него, и равна свидѣтельства не бяху}\footnote{Марк.~14,~56.}. Ибо гдѣ не истина, но ложь свидѣтельствуется, тамо нужно быть разногласію. Посмотри еще производства суда того. \textit{И воставъ}, глаголетъ Евангелистъ, \textit{архіерей посредѣ, вопроси Іисуса, глаголя: не отвѣщаваеши ли ничесоже? что сіи на Тя свидѣтельствуютъ}\footnote{ст.~60.}? Что отвѣщаетъ на сіе Заступникъ нашъ? какіе отвѣты даетъ на вопросъ беззаконный? Не слышимъ ничего! не отвѣщаетъ ничего! \textit{Онъ же молчаше}, глаголетъ Евангелистъ, \textit{и ничтоже отвѣщаваше}\footnote{ст.~61.}. Стоитъ и молчитъ Заступникъ нашъ, аки повинный, самая \textit{неповинность}; и не отвѣщаваетъ, аки обличенный, \textit{Иже грѣха не сотвори}, яко агнецъ незлобивый, не отверзаетъ устъ Своихъ, о Которомъ прореклъ Исаія: \textit{яко овца на заколеніе ведеся, и яко агнецъ предъ стригущимъ его безгласенъ: тако не отверзаетъ устъ Своихъ}\footnote{Ис.~53,~7.}. Вмѣсто насъ стоитъ на судѣ Іисусъ, Заступникъ нашъ, и праведный за насъ беззаконныхъ отъ неправедныхъ судится. Мы суду Божію подлежали; мы связанніи достойны вестися на судъ Божій и судитися, и вѣчно осудитися. Насъ не ложные, но праведные обличаютъ свидѣтели. Насъ праведно обличаетъ законъ Божій, нами безстрашно разоренный; на насъ лежитъ рукописаніе, \textit{еже бѣ сопротивно намъ}\footnote{Кол.~2,~14.}; на насъ праведно вопіетъ совѣсть наша, законопреступленіемъ раздраженная, вопіетъ паче многихъ свидѣтелей. Здѣ мы стоимъ безгласны, безъ отвѣта, со стыдомъ, яко праведно обличаемы, яко повинные. Ибо \textit{повиненъ есть весь міръ Богови. Вси бо согрѣшиша, и лишени суть славы Божія}\footnote{Римл.~3,~19 и 23.}. \textit{Вси уклонишася, вкупѣ неключими быша: нѣсть творяй благостыню, нѣсть до единаго}\footnote{Пс.~13,~3.}. Сего ради достойни суда и осужденія, и яко \textit{неключимые раби, ввержены быть во тьму кромѣшнюю, гдѣ плачь и скрежетъ зубомъ}\footnote{Мѳ.~26,~30.}. Но здѣ на судѣ неправедномъ праведный и неповинный Іисусъ, Заступникъ нашъ, стоитъ и вмѣсто насъ повинныхъ судится, ложная свидѣтельства слышитъ и терпитъ, да насъ, повинныхъ противу клеветы діавольскія, законнаго осужденія, грызенія совѣстнаго, защититъ и избавитъ, якоже Самъ глаголетъ: \textit{аще Сынъ вы свободитъ, воистинну свободни будете}\footnote{Іоан.~8,~36.}. О семъ Защитникѣ и Избавителѣ своемъ дерзаютъ вѣрующіи во имя Его и любящіи Его. На сего уповая, вопіютъ съ Павломъ: \textit{кто поемлетъ на избранныя Божія? Богъ оправдаяй. Кто осуждаяй? Хрістосъ Іисусъ умерый, паче же и воскресый, Иже есть одесную Бога, Иже и ходатайствуетъ о насъ}\footnote{Римл.~8,~33 и 34.}. Постоимъ еще здѣ, и посмотримъ на судь сей, на которомъ неправда правду, ложь истину и злоба неповинность судитъ; и послушаемъ, какое опредѣленіе на беззаконномъ томъ судѣ воспослѣдовало на Заступника нашего. Слышимъ, что осудили Его на смерть. \textit{Они же вси}, глаголетъ Евангелистъ, \textit{осудиша Его повинна смерти}\footnote{Марк.~14,~64.}. \textit{Отвѣщавше рѣша: повиненъ есть смерти}\footnote{Мѳ.~26,~66.}. А за что? \textit{Яко хулу глагола}, вопіетъ беззаконный архіерей\footnote{ст.~65.}. Какую хулу? \textit{яко Себе Сына Божія сотворилъ}\footnote{Іоан.~19,~7.}, о Которомъ Отецъ съ небесе свидѣтельствуетъ: \textit{Сей есть Сынъ Мой возлюбленный, о Немже благоволихъ}\footnote{Мѳ.~3,~17; 17,~5.}. "--- Посмотримъ еще, что за симъ беззаконнымъ опредѣленіемъ послѣдовало. \textit{И начаша нѣцыи}, глаголетъ Евангелистъ, \textit{плевати на Него и прикрывати лице Его, и мучити Его и глаголати Ему: прорцы}\footnote{Марк.~14,~65.}, \textit{прорцы намъ Хрісте, кто есть ударей Тя}\footnote{Мѳ.~26,~68.}. Вотъ что послѣдовало по беззаконномъ судѣ и опредѣленіи. Оплевается, заушается, прикрывается и ударяется, посмѣвается и поругается Іисусъ!.. "--- Кто Онъ? Сынъ Божій, Ѵпостасное Слово Божіе, Которому со страхомъ предстоятъ херувимы и серафимы! "--- Отъ кого? Отъ Своего непотребнаго созданія, которое пришелъ спасти; отъ Своихъ людей, которымъ законъ далъ и пророковъ послалъ! "--- За кого? Кто онъ такой, такъ честенъ, такъ дорогъ, что Богъ во плоти такъ безчестится и мучится? За Свое созданіе погибшее, за грѣшника непотребнаго! Мы, человѣче, мы погибшіи, тому виновны; вмѣсто насъ оплевается, заушается, прикрывается, ударяется и такое поруганіе пріемлетъ; за насъ долгъ сей платитъ милостивый Искупитель нашъ и Творецъ. Мы съ прародителями нашими хотѣли быть \textit{яко бози}, и Божію честь похитить себѣ, яко татіе и разбойницы, и того ради достойны были, яко хульники, на вѣчную смерть осужденными быти, на вѣчное поношеніе, посмѣяніе и біеніе діаволу и аггеломъ его злымъ предатися. Намъ должно, прикрывши лица своя, предъ праведнымъ Богомъ со стыдомъ стоять, и, винность свою признавая, исповѣдатися: \textit{Тебѣ, Господи, правда, намъ же стыдѣніе лица}\footnote{Дан.~9,~7.}. Но Сынъ Божій казнь тую, нашимъ грѣхамъ должную, здѣ претерпѣваетъ; Онъ вмѣсто насъ посмѣвается, ругается, закрывается и біется, да намъ подастъ \textit{область сынами Божіими быти}\footnote{Іоан.~1,~12.}, со дерзновеніемъ ко \textit{Отцу Своему небесному приступати}\footnote{14,~6; Еф.~2,~18.}, и открытымъ и непосрамленнымъ лицемъ на \textit{пресвятое Его лице смотрѣть}\footnote{1~Кор.~13,~12; Іоан.~3,~2.}. "--- Посмотримъ далѣе на подвигъ Хрістовъ, въ который за насъ вступилъ и подвизался. "--- Видимъ, что беззаконный соборъ, поругавшися Сыну Божію и неправедно осудивши на смерть Его, предаетъ Понтійскому Пилату игемону: \textit{и связавше Его}, глаголетъ Евангелистъ, \textit{ведоша, и предаша Его Понтійскому Пилату игемону}\footnote{Мѳ.~27,~2.}. Обличаютъ и тутъ паки неповиннаго, аки повиннаго, и паки судятъ неправедные праведнаго! \textit{Начаша}, глаголетъ Евангеліе, \textit{нань вадити, глаголюще: сего обрѣтохомъ развращающа языкъ нашъ, и возбраняюща кесареви дань даяти}\footnote{Лук.~23,~2.}. О злости человѣческой! Называютъ развратникомъ Того, Который есть \textit{путь, истина и животъ}\footnote{Іоан.~14,~6.}, Который училъ ихъ пути истины, и бѣсныхъ и хромыхъ и слѣпыхъ и прокаженныхъ ихъ исцѣлилъ, и мертвецовъ ихъ воскресилъ! Кесарю противникомъ оглашаютъ Того, Который глаголалъ имъ: \textit{воздадите убо кесарева кесареви, и Божія Богови}\footnote{Мѳ.~22,~21.}. "--- Пилатъ нечестивый къ лукавому лису Ироду отсылаетъ Его\footnote{Лук.~23,~7.}. И тутъ архіерее и книжницы съ своею клеветою на неповинность! \textit{Стояху же}, глаголетъ Писаніе, \textit{архіерее и книжницы, прилежно вадяще нань}\footnote{ст.~10.}. Иродъ, поругавшися высшему всякія чести, паки къ Пилату возвращаетъ Его. \textit{Укоривъ же Его Иродъ съ вои своими, и поругався, оболкъ Его въ ризу свѣтлу, возврати Его къ Пилату}\footnote{Лук.~23,~11.}. И тако Сынъ Божій, связанный и обезчещенный, отъ безчестія къ безчестію и отъ поруганія къ поруганію приводится; водится по улицамъ іерусалимскимъ, водится отъ враговъ Своихъ, водится съ поруганіемъ, водится въ позоръ ангеломъ и человѣкомъ и всей твари: \textit{приводится}, и туды и сюды, \textit{и возвращается}! Въ мою и твою совѣсть, о человѣче, страшное сіе позорище ударяетъ. Мы грѣхъ ради нашихъ блудили по пустынѣ міра сего, тьмою невѣдѣнія Божія, аки узами связани. \textit{Вси}, глаголетъ пророкъ, \textit{яко овцы заблудихомъ: человѣкъ отъ пути своего заблуди}\footnote{Ис.~53,~6.}. \textit{Мы заблудихомъ отъ пути истиннаго, и правды свѣтъ не облиста намъ, и солнце не возсія намъ; беззаконныхъ исполнихомся стезь и погибели, и ходихомъ въ пустыни непроходимыя: пути же Господня не разумѣхомъ}\footnote{Прем.~5,~6 и 7.}. Но Хрістосъ, Сынъ Божій, за насъ сталъ; за насъ съ поруганіемъ по улицамъ водится, да насъ наставитъ на путь истины, да намъ заблуждшимъ путь къ небеси покажетъ и устроитъ, и приведетъ ко Отцу Своему, якоже глаголетъ: \textit{Азъ есмь путь и истина и животъ: никтоже пріидетъ ко Отцу, токмо Мною}\footnote{Іоан.~14,~6.}. Онъ учинился \textit{поношеніе человѣковъ и уничиженіе людей: Ему вси видящіи Его поругашася, глаголаша устнами, покиваша главою}\footnote{Пс.~21,~7 и 8.}. И такимъ образомъ наше отъ насъ безчестіе отъялъ, и вмѣсто безчестія вѣчную славу заслужилъ вѣрующимъ во имя Его. Одеждою поруганія одѣялся, дабы намъ чистую \textit{оправданія одежду} подать, да, тою одѣяни, явимся непосрамлени предъ святѣйшимъ небеснаго Его Отца лицемъ, и возрадуемся со пророкомъ о Господѣ: \textit{да возрадуется душа моя о Господѣ: облече бо мя въ ризу спасенія, и одеждою веселія одѣя мя}\footnote{Ис.~61,~10.}. "--- Посмотримъ еще, что дѣлается съ Заступникомъ нашимъ, отъ Ирода къ Пилату возвращеннымъ; и послушаемъ, что Пилатъ и людіе его, домъ Израилевъ, которые ожидали пришествія Его, которыхъ Онъ пришелъ спасти, якоже глаголетъ Самъ: \textit{нѣсмь посланъ, токмо ко овцамъ погибшимъ дому Израилева}\footnote{Мѳ.~15,~24.}, "--- что о Немъ судятъ? какое опредѣленіе съ обѣихъ сторонъ полагаютъ? Здѣ слышимъ разные о Немъ гласы. Пилатъ язычникъ, незнающій истиннаго Бога, закона Его и пророковъ, познаетъ и признаетъ Его неповинность: \textit{ни единыя обрѣтаю въ человѣцѣ семъ вины, яже нань вадите; но ни Иродъ: послахъ бо Его къ нему, и се ничтоже достойно смерти сотворено есть о Немъ}\footnote{Лук.~23,~14 и 15.}. Но сонмъ іудейскій вопіетъ: \textit{возми Сего, отпусти же намъ Варавву, иже бѣ за нѣкую крамолу, бывшую во градѣ и убійство вверженъ въ темницу}\footnote{18 и 19.}. Пилатъ паки признаетъ и объявляетъ Его неповинность, и хощетъ Его отпустити: \textit{паки же Пилатъ возгласи, хотя отпустити Іисуса}\footnote{ст.~20.}. Но народъ Божій возглашаетъ, глаголя: \textit{распни, распни Его}\footnote{21.}. И еще Пилатъ оправдаетъ Его, и хощетъ отпустити Его. \textit{Онъ же третицею рече къ нимъ: что бо зло сотвори сей? ничесоже достойна смерти обрѣтохъ въ Немъ: наказавъ убо Его отпущу}\footnote{22.}. Но Израиль Божій крѣпится и прилѣжно проситъ Его на распятіе. \textit{Они же прилѣжаху гласы великими, просяще Его на распятіе}\footnote{23.}. Что отъ сихъ разныхъ гласовъ воспослѣдовало? какій конецъ воспріялъ несогласный судъ сей? Видимъ, что пересилила злоба Израиля правду язычника. Не успѣлъ ничего Пилатъ съ своимъ оправданіемъ противу неправды беззаконныхъ людей. Превозмогли гласы народа и архіерейскіе. \textit{И превозмогаху}, глаголетъ Евангелистъ\textit{, гласи ихъ и архіерейстіи Пилатъ же суди быти по прошенію ихъ}\footnote{Лук.~23 и 24.}. Отпущается крамольникъ, злодѣй и убійца: предается же на смерть Сынъ Божій, самая неповинность!.. \textit{Отпусти} игемонъ \textit{за крамолу и убійство всажденнаго въ темницу, егоже прошаху: Іисуса же предавъ воли ихъ}\footnote{ст.~23,~25.}. Мы, о человѣче, тому причиною, что злодѣй предпочтенъ неповинному Іисусу: наши грѣхи, возверженные на Него, переважили убійцы грѣхи. Ибо Онъ всего міра грѣхи на Себе взялъ: яко \textit{агнецъ Божій, вземляй грѣхи міра}\footnote{Іоан.~1,~29.}. На насъ совѣсть наша, грѣхами нашими раздраженная, вопіяла къ Богу; на насъ законъ Божій, нами разоренный, суда у Бога просилъ, и на вѣчную осуждалъ смерть. Но Хрістосъ Заступникъ вмѣсто насъ сталъ; неповинный за повинныхъ осуждается и предается на смерть. \textit{Господь предаде Его грѣхъ ради нашихъ}, вопіетъ пророкъ\footnote{Ис.~53,~6.}. \textit{Преданъ бысть за прегрѣшенія наша}, проповѣдуетъ намъ Апостолъ Его\footnote{Римл.~4,~25.}. \textit{Невѣдѣвшаго грѣха по насъ грѣхъ сотвори Богъ, да мы будемъ правда Божія о Немъ}, тойже Апостолъ открываетъ намъ\footnote{2~Кор.~5,~21.}. "--- Посмотримъ еще, что далѣе дѣлается по беззаконномъ томъ судѣ съ неповиннымъ нашимъ Заступникомъ. Ужасное еще представляется очамъ нашимъ видѣніе! Видимъ, что Іисусъ, преданный въ руки безчеловѣчныхъ воиновъ, обнажается отъ нихъ и одѣвается хламидою червленою на посмѣяніе и поруганіе, вѣнчается терновымъ вѣнцемъ; въ руцѣ Его святыя, сотворшія чудеса, подается трость въ поруганіе. Тако Ему, яко царю, на поруганіе увѣнчанному, покланяются, и поздравляютъ Его съ поруганіемъ, какъ царя: \textit{радуйся царю Іудейскій!} И симъ не довольна бываетъ безчеловѣчная беззаконниковъ злость! Еще беззаконіе къ беззаконію и лютость къ лютости придаютъ: плюютъ на пресвятое лице Его, предъ которымъ ангели и архангели благоговѣютъ, и біютъ тростію по главѣ Его! \textit{Тогда}, повѣствуетъ Матѳей святый, \textit{воини игемоновы, пріемше Іисуса на судище, собраша нань все множество воинъ; и совлекше Его, одѣяша Его хламидою червленою; и сплетше вѣнецъ отъ тернія, возложиша на главу Его, и трость въ десницу Его; и поклоншеся на колѣну предъ Нимъ, ругахуся Ему, глаголюще: радуйся царю Іудейскій! И плюнувше нань пріяша трость, и біяху по главѣ Его}\footnote{Мѳ.~27,~27--30.}. За насъ стоитъ и подвизается тако заступникъ нашъ Іисусъ. Насъ грѣхъ нашъ обнажилъ первыя боготканныя одежды, и облеклъ въ кожаныя одежды\footnote{Быт.~3,~6,~7 и 21.}. Мы восхотѣли, по совѣту діавольскому, славу Божію похитити, и тою вѣнчатися \textit{яко бози}\footnote{ст.~5.}. Мы дали рукописаніе на себе врагу нашему, которымъ насъ, яко закона преступниковъ, обличалъ, въ своей власти, яко плѣнниковъ, держалъ, и съ собою въ вѣчную погибель влеклъ. Насъ совѣсти наши боли паче тернія, и уязвляли паче всякаго жезла. Но Хрістосъ вмѣсто насъ совлекается ризъ Своихъ, да насъ ризою спасенія и одеждою веселія одѣетъ; терніемъ вѣнчается, да намъ подастъ вѣнецъ неистлѣненъ; трость въ руцѣ пріемлетъ, да рукописаніе обличающее раздеретъ; по главѣ біется, да наши язвы, которыми насъ сатана уязвилъ, исцѣлитъ и отъиметъ совѣстное боденіе; одѣвается хламидою червленою, и, яко царь, съ насмѣяніемъ поздравляется, да \textit{насъ сотворитъ цари и іереи Богу и Отцу Своему}\footnote{Апок.~1,~6.}. Тако \textit{язвенъ бысть за грѣхи наша, и мученъ бысть за беззаконія наша} Избавитель нашъ\footnote{Ис.~53,~5.}, \textit{Его язвою мы исцѣлѣхомъ}\footnote{1~Петр.~2,~24.}. "--- Еще здѣ постоимъ, и посмотримъ далѣе на спасительный подвигъ Заступника нашего. Видимъ далѣе, что \textit{воини, поругавшеся, ведоша Его на пропятіе}\footnote{Мѳ.~27,~31.}. \textit{И нося крестъ Свой, изыде на глаголемое лобное мѣсто, еже глаголется еврейски Голгоѳа}\footnote{Іоан.~19,~17.}. Посмотримъ здѣ прилѣжнѣе, откуду изыде, и что носячи, изыде. Изъ града за градъ изыде, отъ мѣста поруганія и страданія на мѣсто большаго поруганія и страданія изыде. Что несетъ предъ очесами Божіими и ангельскими поруганный и уязвленный Сынъ Божій? Крестъ Свой, на которомъ слѣдуетъ Ему пригвоздитися и вознестися; и на крестѣ всего міра грѣхи несетъ Агнецъ Божій, на Котораго указуя, Предтеча глаголалъ: \textit{се Агнецъ Божій, вземляй грѣхи міра}\footnote{1,~29.}. На семъ древѣ мои и твои, о человѣче, грѣхи положенные несетъ Агнецъ Божій, которые насъ имѣли погрузить во дно адово. Ибо сего тяжкаго бремене не токмо воздухъ, но и земля не можетъ держать, но во дно адово носящаго погружаетъ, аще не возметъ его Сей непорочный и пречистый Агнецъ. Сего ради Агнецъ Божій взялъ ихъ на Себе, и вмѣсто насъ несетъ ихъ, да мы отъ нихъ избудемъ. \textit{Той грѣхи наша Самъ вознесе на тѣлѣ Своемъ на древо, да, отъ грѣхъ избывше, правдою поживемъ}, проповѣдуетъ Апостолъ\footnote{1~Петр.~2,~24.}. \textit{Тѣмже Іисусъ, да освятитъ люди Своею кровію, внѣ вратъ пострадати изволилъ}\footnote{Евр.~13,~12.}. "--- Пойдемъ и мы за Хрістомъ, и послѣдуемъ Ему, носящему крестъ Свой, нашими грѣхами отягченный, и тако взыдемъ на Голгоѳу, и повидимъ еще, что тамо съ Заступникомъ нашимъ дѣлается. Здѣ намъ паки страшное представляется позорище! Видимъ, что тамо распяли Его и двухъ разбойниковъ съ Нимъ въ большее Ему поношеніе; единаго по правую, другаго по лѣвую сторону распяли, посредѣ же ихъ Іисуса. \textit{Егда пріидоша}, глаголетъ Евангелистъ, \textit{на мѣсто нарицаемое лобное, ту распяша Его и злодѣевъ, оваго убо одесную, а другаго ошуюю}\footnote{Лук.~23,~33; Іоан.~19,~18.}. "--- Постоимъ еще мало на мѣстѣ семъ, еже есть сказаемо лобное мѣсто, и очеса наша съ сердцами возведемъ на распятаго Хріста, \textit{да вообразится въ насъ Хрістосъ}\footnote{Гал.~4,~19.}. Посмотримъ: на чемъ распяли Его? На древѣ крестномъ. Какъ? Всего на древѣ томъ растягнули, руки и ноги Его къ тому пригвоздили, и между небомъ и землею повѣсили. Кто Онъ, Которому такое безчеловѣчіе показали? \textit{Господа славы распяли}, проповѣдуетъ всему міру вселенныя учитель Павелъ\footnote{1~Кор.~2,~8.}. Сынъ Божій между разбойниками повѣшенъ въ позоръ ангеломъ и человѣкомъ! Судія праведный \textit{со беззаконными вмѣнися}\footnote{Ис.~53,~12.}! Еще посмотримъ, любезный хрістіанине, на Хріста, насъ ради распятаго, и прилѣжнѣе разсудимъ, за что такъ ужасно страждетъ, \textit{Иже грѣха не сотвори, ни обрѣтеся лесть во устѣхъ Его}, какъ учитъ Апостолъ Его\footnote{1~Петр.~2,~22.}. Мои и твои, о человѣче, грѣхи на Него возвержены. \textit{Господь предаде Его грѣхъ ради нашихъ}, свидѣтельствуетъ Исаія святый\footnote{Ис.~53,~6.}. За тебе и за мене правдѣ Божіей, раздраженной грѣхами нашими, платитъ Сынъ Божій, Іисусъ, Заступникъ нашъ, якоже глаголетъ чрезъ пророка: \textit{яже не восхищахъ, тогда воздаяхъ}\footnote{Пс.~68,~5.}. Мы, которые хотѣли славу и честь Божію со Адамомъ похитить, какъ хищники и тати, достойны были вѣчно во огнѣ геенскомъ страдать, и тако правдѣ Божіей огорченной платить; но Хрістосъ, Сынъ Божій, сей долгъ на Себе взялъ, и тако здѣ воздаетъ за него, \textit{воздаетъ, чего не похищалъ}; такимъ образомъ долгъ нашихъ грѣховъ, которымъ мы Царю нашему Богу обдолжились, и не имѣли чимъ отдать, кромѣ смерти вѣчной\footnote{Мѳ.~18,~24--35.}, Своею неоцѣненною кровію загладилъ, якоже глаголетъ: \textit{Азъ есмь заглаждаяй беззаконія твоя Мене ради}\footnote{Ис.~43,~25.}. Свои руки и ноги ко кресту пригвоздити попустилъ, да очиститъ грѣхи наша, руками и ногами содѣланные: \textit{язвенъ бысть за грѣхи наша}. Онъ вознеслся на древо и повѣсился, да искупитъ насъ отъ клятвы законныя. \textit{Хрістосъ}, глаголетъ Апостолъ Хрістовъ, \textit{насъ искупилъ есть отъ клятвы законныя, бывъ по насъ клятва; писано бо есть: проклятъ всякъ висяй на древѣ: да во языцѣхъ благословеніе Авраамле будетъ о Хрістѣ Іисусѣ}\footnote{Гал.~3,~13 и 14.}. Якоже Моисей вознесе змію въ пустыни, на которую взирая, уязвленніи отъ зміевъ исцѣлялись\footnote{Числ.~21,~8 и 9.}: \textit{тако подобало вознестися Сыну человѣческому, да всякъ, вѣруяй въ Онь, не погибнетъ, но имать животъ вѣчный}\footnote{Іоан.~3,~14 и 15.}. Постоимъ еще и послушаемъ, не говоритъ ли чего въ лютомъ Своемъ страданіи Сынъ Божій; страждай не претитъ ли распинателямъ; не жалуется ли на враговъ Своихъ Отцу Своему небесному, не проситъ ли на нихъ суда праведнаго, что такъ звѣрски съ Нимъ поступаютъ. Нѣтъ! Не слышимъ того, не отверзаетъ на то устъ Своихъ непорочный Агнецъ Божій, но паче противное слышимъ. Слышимъ, что за нихъ, не на нихъ, молится; имъ отпущенія проситъ, \textit{Іисусъ же глаголаше}, повѣствуетъ Его святый Евангелистъ: \textit{Отче! отпусти имъ: не вѣдятъ бо, что творятъ}\footnote{Лук.~23,~34.}. О долготерпѣнія Твоего, Владыко, милостивый нашъ Искупитель! О неисповѣдимой Твоей любви и милосердія ко грѣшникамъ! И въ такъ зѣльномъ мученіи не такъ о Себѣ, какъ о погибели грѣшниковъ, да еще явныхъ враговъ, болѣзнуешь!.. Благословенъ еси Господи, толикаго терпѣнія образъ показавый намъ! Да постыдится человѣческая гордость, за поносное и укорительное слово нетерпящая и ропщущая и отмщенія брату ищущая! "--- Еще постоимъ, хрістіанине, на лобномъ мѣстѣ и посмотримъ на Хріста, нашихъ ради грѣховъ ко кресту пригвожденнаго, и повидимъ, что еще дѣлается съ Искупителемъ нашимъ. Видимъ здѣ, что недовольна бываетъ злоба такъ лютымъ и кровавымъ мучительствомъ; еще къ мученію мученіе придаетъ! Слышимъ, что и ядовитымъ языкомъ, какъ стрѣлами, умученнаго и распятаго уязвляютъ; мимоходя хулятъ неповиннаго, насмѣваются такъ лютое мученіе страждущему. \textit{И мимоходящіи хуляху Его}, повѣствуетъ Евангелистъ, \textit{покивающе главами своими и глаголюще: уа, разоряяй церковь, и треми деньми созидаяй! спасися Самъ, и сниди съ креста. Такожде и архіерее ругающеся, другъ ко другу съ книжники глаголаху: ины спасе, себе ли не можетъ спасти? Хрістосъ царь израилевъ да снидетъ нынѣ съ креста, да видимъ и вѣру имемъ Ему}\footnote{Марк.~15,~29--32.}. О семъ Самъ чрезъ пророка къ небесному Своему Отцу глаголетъ: \textit{вси видящіи Мя поругашася Ми, глаголаша устнами, покиваша главою: упова на Господа, да избавитъ Его, да спасетъ Его, яко хощетъ Его}. И ниже: \textit{обыдоша Мя тельцы мнози, юнцы тучніи одержаша Мя. Отверзоша на Мя уста своя, яко левъ восхищаяй и рыкаяй}. И паки таможде: \textit{обыдоша Мя пси мнози, сонмъ лукавыхъ одержаша Мя; ископаша руцѣ Мои и нозѣ Мои; исчетоша вся кости Моя: тіи же смотриша и презрѣша Мя}\footnote{Пс.~21,~8,~9,~13,~14,~17 и 18.}. Наши совѣсти, человѣче, страшное сіе позорище уязвляетъ. За насъ сіи язвительныя насмѣянія и руганія терпитъ страждущій \textit{Сынъ Бога живаго}. Мы достойны язвительныхъ насмѣяній и руганій діавольскихъ. Мы храмъ Божій, тѣлеса наша, разорили грѣхами нашими, и законъ Божій нарушили, за что должны въ вѣчное діаволу и аггеломъ его, врагамъ нашимъ, посмѣяніе и поруганіе отданы быть. Но Сынъ Божій, Заступникъ нашъ, вмѣсто насъ и здѣ стоитъ; за насъ терпитъ ядовитыя сіи посмѣяній стрѣлы; за насъ симъ неизреченнымъ Своимъ безчестіемъ и поруганіемъ платитъ, дабы насъ отъ вѣчнаго посмѣянія и укоризны избавить. Слышитъ неповинно: \textit{уа, разоряяй церковь}, да насъ храмы Духа Святаго содѣлаетъ\footnote{1~Кор.~6,~19.}. Слышитъ съ насмѣяніемъ: \textit{спасися самъ}\footnote{Мѳ.~27,~40.}, да намъ вѣчное спасеніе устроитъ. Слышитъ: \textit{аще Сынъ еси Божій, сниди со креста}\footnote{ст.~40.}, да мы \textit{всыновленіе воспріимемъ}\footnote{Гал.~4,~5.}, да \textit{сынове Божіи будемъ вѣрою о Хрістѣ Іисусѣ}\footnote{Гал.~3,~26.}. "--- Постоимъ еще, хрістіанине, на Голгоѳѣ предъ Хрістомъ, насъ ради грѣшныхъ страждущимъ, и повидимъ, что еще Онъ ради насъ страждетъ. Видимъ, что пречистыя Его уста, благовѣствовавшія миръ и Евангеліе, запеклися отъ зѣльной болѣзни, и языкъ, проповѣдавшій отпущеніе грѣховъ грѣшникамъ, \textit{прилпе гортани Его}\footnote{Пс.~21,~16.}. И слышимъ, что въ томъ жесточайшемъ страданіи \textit{вопіетъ: жажду}\footnote{Іоан.~19,~28.}. Кто Онъ есть, Который вопіетъ "--- \textit{жажду}? Сей есть Тотъ, \textit{Котораго земля и исполненіе ея}\footnote{Пс.~23,~1.}, есть \textit{напаяяй горы отъ превыспреннихъ Своихъ}\footnote{103,~13.}, \textit{собираяй, яко мѣхъ, воды морскія, полагаяй въ сокровищахъ бездны}\footnote{32,~7.}, Который \textit{изведе воду изъ камене, и низведе яко рѣки воды}\footnote{77,~16.}, \textit{порази камень, и потекоша воды}\footnote{ст.~20.}; \textit{разверзе камень, и потекоша воды}\footnote{106,~41.}. Той плотію, насъ ради воспріятою, въ страданіи жаждетъ, и въ жаждѣ Своей вопіетъ: \textit{жажду!} За насъ стоитъ Заступникъ, и въ страданіи Своемъ жажду терпитъ. Намъ слѣдовало въ пламени геенскомъ вѣчно страдать, жаждать, кричать и просить, но просить безполезно, капли воды, ради прохлажденія языка нашего\footnote{Лук.~16,~24.}. Но Хрістосъ здѣ на Голгоѳѣ жаждетъ, да отъ насъ отъиметъ вѣчную въ гееннѣ жажду, и подастъ намъ прохлажденіе во царствіи Своемъ небесномъ, да устроитъ насъ \textit{возлещи со Авраамомъ и Исаакомъ и Іаковомъ во царствіи небеснѣмъ}\footnote{Мѳ.~8,~11.}, гдѣ возлежащіи \textit{не взалчутъ, ни вжаждутъ, ниже поразитъ ихъ зной, ниже солнце}\footnote{Ис.~49,~11.}; яко \textit{Агнецъ, Иже посредѣ престола, упасетъ ихъ, наставитъ ихъ на животныя источники водъ}\footnote{Апок.~7,~16 и 17.}. Тако Хрістосъ жаждою Своею отъимаетъ вѣчную нашу жажду и подаетъ намъ \textit{воду живую, отъ которой піяй не вжаждется во вѣки}\footnote{Іоан.~4,~14.}. "--- Посмотримъ, чѣмъ жаждущаго въ страданіи Сына Божія напаяютъ. Видимъ, что и воды студеныя не сподобляется \textit{Покрываяй небо облаки}, но оцтомъ напаяютъ Его. \textit{Исполнивше губу оцта, и на трость вонзше, придѣша ко устомъ Его}\footnote{19,~29.}. И самъ чрезъ пророка глаголетъ: \textit{даша въ снѣдь Мою желчь, и въ жажду Мою напоиша Мя оцта}\footnote{Пс.~68,~22.}. Желчи вкушаетъ и горькимъ оцтомъ напаяется Избавитель нашъ, да отъиметъ горесть нашу, душамъ нашимъ отъ снѣди заповѣданнаго древа прилѣпшую, и подастъ намъ \textit{ясти и пити на трапезѣ Своей во царствіи Его}\footnote{Лук.~22,~30.}, якоже церковь святая поетъ Ему: «да избавимся мы сластнаго грѣха, желчи вкусилъ еси, Сладосте жизненная»\footnote{Пѣснь 7 гласа 5"~го, въ пятокъ на утрени.}! Постоимъ еще на Голгоѳѣ, и посмотримъ конца страшнаго сего и спасительнаго позорища. И се, видимъ уже, что Заступникъ нашъ, Сынъ Божій окончилъ подвигъ Свой, которымъ за насъ подвизался. \textit{Егда же пріятъ оцетъ Іисусъ, рече: совершишася! И преклонь главу, предаде духъ}, проповѣдуетъ возлюбленный Его ученикъ и наперсникъ Іоаннъ святый\footnote{Іоан.~19,~30.}. Вотъ чимъ окончилъ, хрістіанине, подвигъ Свой Хрістосъ, насъ ради воспріятый! \textit{Смертію, смертію же крестною} окончалъ!.. "--- Посмотримъ прилѣжнѣе на крестную Сына Божія смерть, хрістіанине, отъ которой животъ нашъ зависитъ, ибо онъ душу Свою за насъ положилъ\footnote{10,~15,~17 и 18.}. Намъ согрѣшившимъ слѣдовало не токмо временно, но и вѣчно умирать, \textit{оброцы бо грѣха смерть}, глаголетъ Апостолъ\footnote{Римл.~6,~23.}. Но Хрістосъ, Сынъ Божій, на крестѣ умеръ, да нашу смерть Своею смертію умертвитъ, и насъ умершихъ Своею смертію оживитъ. \textit{Составляетъ свою любовь къ намъ Богъ}, глаголетъ Павелъ, \textit{яко, еще грѣшникомъ намъ сущимъ, Хрістосъ за ны умре}\footnote{Римл.~5,~8.}. Аще бо и умираютъ временно вѣрующіи во имя Его; но понеже \textit{жало смерти, грѣхъ}\footnote{1~Кор.~15,~56.}, Хрістосъ Своею смертію \textit{истребилъ}\footnote{Кол.~2,~14; 1~Кор.~15,~55.}, то и сія смерть \textit{умирающимъ о Господѣ} не есть смерть, но прехожденіе отъ временной къ вѣчной жизни. Аще бо и разрушается земная ихъ храмина тѣла, однакожъ вѣра святая утѣшаетъ и утверждаетъ ихъ, яко \textit{созданіе отъ Бога имѣютъ, храмину нерукотворенну, вѣчну на небесѣхъ}\footnote{2~Кор.~5,~1.}. \textit{И якоже зерно пшенично, падшее въ землю умираетъ, и плодъ многъ сотворитъ}\footnote{Іоан.~12,~24.}, тако усопшихъ о Господѣ тѣлеса, въ землѣ погребенная и растлѣнная, услышавше \textit{повелѣніе Божіе и гласъ архангельскій}\footnote{1~Сол.~4,~16.}, яко посѣянная сѣмена изъ земли, изъ гробовъ изыдутъ; \textit{вострубитъ бо, и мертвіи востанутъ нетлѣнни}\footnote{1~Кор.~15,~52.}. Симъ упованіемъ укрѣпляяся и утѣшаяся, вѣрніи дерзаютъ съ Павломъ: \textit{пожерта бысть смерть побѣдою! Гдѣ ти, смерте жало? гдѣ ти, аде, побѣда}\footnote{ст.~55.}? Оставляютъ міръ сей, отцевъ, матерей, братію, сродниковъ, но несравненно большая и совершеннѣйшая приобрѣтаютъ. \textit{Приходятъ бо къ Богу Отцу, и являются святѣйшему лицу Его}\footnote{Пс.~41,~3.}, видѣти \textit{Его лицемъ къ лицу}\footnote{1~Кор.~13,~12.}. Обрѣтаютъ благопріятное дружество со святыми ангелами, патріархами, пророками, апостолами, мучениками и всѣмъ соборомъ небеснымъ. Тамо сіяетъ прекрасное Солнце правды "--- Хрістосъ, которое никогда не заходитъ, ни облаками покрывается, ни мглою не помрачается, но всегда въ силѣ своей блистаетъ, радостотворитъ и увеселяетъ святыхъ Своихъ. Тамо наслѣдствуютъ благая, \textit{ихже око не видѣ, и ухо не слыша, и на сердце человѣка не взыдоша, яже уготова Богъ любящимъ Его}\footnote{1~Кор.~2,~9.}. Тамо славу и богатство нетлѣнное пріемлютъ отъ руки Господни. О сихъ вѣрное разсужденіе утѣшаетъ вѣрныхъ и смерти горесть растворяетъ, паче же желанною дѣлаетъ, такъ что съ Симеономъ святымъ радостнымъ духомъ срѣтаютъ ее, и глаголютъ: \textit{нынѣ отпущаеши раба Твоего, Владыко, по глаголу Твоему, съ миромъ: яко видѣстѣ очи мои спасеніе Твое, еже еси уготовалъ предъ лицемъ всѣхъ людей}\footnote{Лук.~2,~29--31.}, "--- и съ Павломъ великимъ: \textit{желаніе имѣю разрѣшитися и со Хрістомъ быти}\footnote{Филип.~1,~23.}. Тако убо \textit{Хрістосъ умре грѣхъ нашихъ ради}\footnote{1~Кор.~15,~3.}, \textit{да смертію упразднитъ имущаго державу смерти, сирѣчь, діавола; и избавитъ сихъ, елицы страхомъ смерти чрезъ все житіе повинни бѣша работѣ}\footnote{Евр.~2,~14 и 15.}. Се видимъ, хрістіанине, какъ дорого стало Хрісту, Сыну Божію, спасеніе наше! Жесточайшимъ страданіемъ, глубочайшимъ смиреніемъ, неизреченнымъ безчестіемъ, поносною смертію, смертію же крестною искупилъ насъ отъ смерти, ада и всего бѣдствія Искупитель нашъ, и возвратилъ намъ спасеніе, которое мы самовольно потеряли. "--- Посмотримъ еще и на погребеніе святѣйшія и живоносныя плоти Его, которая по многоболѣзненныхъ трудахъ и страданіяхъ, нашего ради спасенія подъятыхъ, три дни и три нощи почивала во гробѣ. \textit{Якоже бо бѣ Іона во чревѣ китовѣ три дни и три нощи: тако бысть и Сынъ человѣческій въ сердцы земли три дни и три нощи}\footnote{Мѳ.~12,~40.}. Погребеся Хрістосъ, погреблъ и грѣхи наша, и погрузилъ неправды наша, и ввергнулъ во глубины морскія вся грѣхи наша, дабы не явилися предъ лицемъ Божіимъ, аще истинно и сердечно вѣруемъ въ Него, насъ ради умершаго и погребеннаго, якоже о семъ пророкъ Его святый предсказалъ. \textit{Той обратитъ, и ущедритъ ны, и погрузитъ неправды наша, и ввержетъ въ глубины морскія вся грѣхи наша}\footnote{Мих.~7,~19.}. Почилъ Хрістосъ отъ дѣлъ Своихъ, которая нашего ради спасенія началъ и совершилъ: тако устроилъ и намъ \textit{субботу}, которую праздновать будемъ вѣчно, \textit{почивши отъ дѣлъ нашихъ и трудовъ, и вшедши въ покой оный}\footnote{Евр.~4,~1--11; Апок.~14,~13.}, который никакого труда, печали и болѣзни непричастенъ, но только слышится гласъ радости и веселія, якоже поетъ пророкъ: \textit{гласъ радости и спасенія въ селеніяхъ праведныхъ}\footnote{Пс.~117,~15.}. Положился Хрістосъ во гробѣ, и наши гробы не страшны намъ учинилися, но сотворилися какъ ложа, въ которыхъ, по многоболѣзненной жизни сей почивши, упокоеватися будемъ, пока \textit{услышимъ гласъ Сына Божія, и услышавше оживемъ}, по неложному Его обѣщанію\footnote{Іоан.~5,~25.}. Воскресе Хрістосъ, воскреснемъ и мы вѣрующіи во имя Его. И что \textit{сѣется въ тлѣніе, востаетъ въ нетлѣніи; сѣется не въ честь, востаетъ въ славѣ; сѣется въ немощи, востаетъ въ силѣ; сѣется тѣло душевное, востаетъ тѣло духовное}\footnote{1~Кор.~15,~42--44.}. Вознесеся Хрістосъ на небо, вознесемся и мы, аще \textit{до смерти Ему вѣрни пребудемъ}\footnote{Апок.~2,~10.}, и \textit{претерпимъ до конца}\footnote{Мѳ.~24,~13.}. Да \textit{идѣже глава, тамо и уды его будутъ; и идѣже, Господь, ту и слуги Его будутъ}\footnote{12,~26.}, и \textit{идѣже Самъ Онъ, и вѣрніи Его съ Нимъ будутъ, да видятъ славу Его}\footnote{Іоан.~17,~24.}. Откуду Апостолъ вѣрнымъ во утѣшеніе глаголетъ: \textit{наше житіе на небесѣхъ есть, отонудуже и Спасителя ждемъ Господа нашего Іисуса Хріста, Иже преобразитъ тѣло смиренія нашего, яко быти сему сообразну тѣлу славы Его}\footnote{Филип.~3,~20 и 21.}. И паки: \textit{вѣмы, яко аще земная наша храмина тѣла разорится, созданіе отъ Бога имамы, храмину нерукотворенну, вѣчну на небесѣхъ. Ибо о семъ воздыхаемъ, въ жилище наше небесное облещися желающе}\footnote{2~Кор.~5,~1 и 2.}.

Слава Богу, помиловавшему насъ и такъ чудно устроившему спасеніе намъ, во вѣки вѣковъ, аминь!

Отъ вышеписанныхъ примѣчай, хрістіанине, слѣдующая: 1)~Видишь отчасти страданіе Хрістово, о которомъ намъ святое Евангеліе и прочее Писаніе повѣствуетъ, и вѣра святая исповѣдаетъ; но лучше его увидишь, когда разсудишь вѣрно обстоятельства его. Аще бо хощемъ что лучше разумѣти, должно не токмо самую вещь, но и ея обстоятельства разсуждать. Такъ, когда и страстей Хрістовыхъ важность и величество хощемъ нѣсколько разумѣть (\textit{нѣсколько}, глаголю: ибо постигнуть ихъ умъ человѣческій не можетъ), и въ размышленіи ихъ съ пользою нашею поучаться, должно обстоятельства ихъ, о которыхъ Божіе слово намъ представляетъ, вѣрно разсуждати. Образъ страстей Хрістовыхъ видишь въ храмахъ святыхъ на дскахъ; но безъ размышленія обстоятельствъ не иное что видишь, какъ только образъ человѣка умученнаго и ко кресту пригвожденнаго. Но вѣрное ихъ разсужденіе изъ слова Божія представляетъ тебѣ важность и силу ихъ. Примѣчай убо: кто пострадалъ? Отвѣтствуетъ намъ со удивленіемъ и проповѣдуетъ богомудрый Апостолъ Хрістовъ Павелъ: \textit{исповѣдуемо велія есть благочестія тайна: Богъ явися во плоти}\footnote{1~Тим.~3,~16.}. Той убо Богъ, который явися во плоти, во плоти Своей пострадалъ. \textit{Господа славы распяли}, тойже Апостолъ вопіетъ вселеннѣй\footnote{1~Кор.~2,~8.}. И церковь святая поетъ Его, и молится пострадавшему: \textit{распныйся Хрісте Боже, смертію смерть поправый, единъ сый Святыя Троицы, спрославляемый Отцу и Святому Духу, спаси насъ}\footnote{Стихъ, на литург. поемый.}. Тоже и на прочіихъ мѣстахъ. "--- Отъ кого пострадалъ? Отвѣщаетъ Евангелистъ святый: \textit{во своя пріиде, и свои Его не пріяша}\footnote{Іоан.~1,~11.}. Отъ своихъ неблагодарныхъ людей, \textit{иже суть Израильтяне, ихже всыновленіе и слава, и завѣти и законоположеніе и служеніе и обѣтованіе; ихже отцы, и отъ нихже Хрістосъ по плоти, сый надъ всѣми Богъ благословенъ во вѣки}, воздыхаетъ о нихъ Павелъ\footnote{Римл.~9,~4 и 5.}. Отъ тѣхъ пострадалъ, которыхъ въ особливой милости содержалъ паче прочихъ языковъ; которыхъ пришелъ спасти, якоже Самъ глаголетъ: \textit{нѣсмь посланъ, токмо ко овцамъ погибшимъ дому Израилева}\footnote{Мѳ.~15,~24.}; которыхъ училъ истинѣ, которымъ проповѣдалъ царствіе небесное, показывалъ путь къ небеси; у которыхъ слѣпыхъ, хромыхъ, прокаженныхъ, бѣсноватыхъ и прочіихъ недужныхъ исцѣлилъ и мертвыхъ воскресилъ, и прочія чудесныя и вышеестественныя дѣла содѣлалъ. Отъ сихъ людей \textit{своихъ} пострадалъ Хрістосъ, Сынъ Божій, ихъ Благодѣтель, Творецъ и Богъ! Преданъ и проданъ въ руки беззаконныхъ отъ \textit{своего} ученика неблагодарнаго Учитель, сшедый съ небесе и учивый истинѣ. За колико проданъ? \textit{за тридесять сребренниковъ}\footnote{Мѳ.~26,~15.}, "--- такъ малую цѣну Безцѣнный, Котораго есть небо и земля со всѣмъ исполненіемъ ихъ\footnote{Пс.~49,~12; 23,~1; Ис.~19,~5.}! Тяжко намъ отъ всякаго терпѣть, но тягчае отъ того, кто отъ насъ благодѣяніе получаетъ. Не такъ и Хрістову святую и непорочную душу уязвляла неправда язычника Пилата, какъ злоба беззаконнаго Израиля, толикими Его благодѣяніями одарованнаго. "--- Что пострадалъ? Всякое безчестіе, хуленіе, руганіе, насмѣяніе, біеніе, оплеваніе, заушеніе, уязвленіе, вязаніе, обнаженіе, съ мѣста на мѣсто перевожденіе, терновное вѣнчаніе, ко кресту пригвожденіе, смерть крестную и всякое звѣрское безчеловѣчіе, какое могла злоба человѣческая, діаволомъ подстрѣкаемая, вымыслить, "--- такое, какого и помыслить ужасно, пострадалъ! Откуду слово Божіе представляетъ намъ ужасный видъ страждущаго Хріста. \textit{Якоже ужаснутся о Тебѣ мнози, тако обезславится отъ человѣкъ видъ Твой, и слава Твоя отъ сыновъ человѣческихъ}, глаголетъ о Немъ пророкъ Исаія\footnote{Ис.~52,~14.}. \textit{Нѣсть вида Ему, ниже славы: и видѣхомъ Его, и не имяше вида, ни доброты; но видъ Его безчестенъ, умаленъ паче всѣхъ сыновъ человѣческихъ: человѣкъ въ язвѣ сый, и вѣдый терпѣти болѣзнь}, и проч.\footnote{53,~2,~3 и слѣд.} И паки Самъ чрезъ пророка: \textit{яко вода изліяхся, и разсыпашася вся кости Моя; бысть сердце Мое, яко воскъ таяй посредѣ чрева Моего. Изше, яко скудель, крѣпость Моя, и языкъ Мой прилпе гортани Моему}, и проч.\footnote{Пс.~21,~15,~16 и слѣд.} Но сколь тяжкое было тѣлесное мученіе, яко мягкое, чистое, дѣвственное Божіе и непорочное тѣло уязвлялося, прободалося и терзалося, столь мучительная была внутренняя печаль и болѣзнь, которою непорочная, святѣйшая и Божественная Его душа уязвлялася, якоже глаголетъ Самъ, вступая въ подвигъ страданія Своего: \textit{прискорбна есть душа Моя до смерти}\footnote{Матѳ.~26,~38.}. И паки: \textit{поношеніе чаяше душа Моя и страсть: и ждахъ соскорбящаго, и не бѣ, и утѣшающихъ, и не обрѣтохъ}\footnote{Пс.~68,~21.}. Ужасное убо и неизреченное было мученіе на тѣлѣ и душѣ Сына Божія, Который хотѣлъ и пришелъ тѣло и душу нашу, смертно уязвленную, исцѣлить и оживить. "--- Какъ пострадалъ? "--- Неповинно и терпѣливо. \textit{Неповинно}: ибо \textit{грѣха не сотвори, ни обрѣтеся лесть во устѣхъ Его}\footnote{1~Петр.~2,~32.}, но паче есть сокровище и образъ всѣхъ добродѣтелей. \textit{Терпѣливо}: ибо \textit{яко овча на заколеніе ведеся, и яко агнецъ предъ стригущимъ его безгласенъ, тако не отверзалъ устъ Своихъ}\footnote{Ис.~53,~7.}. "--- Гдѣ и когда пострадалъ? "--- Въ столичнѣмъ градѣ іудейскомъ Іерусалимѣ, куды на праздникъ Пасхи собралося отъ всѣхъ странъ безчисленное множество народа. При собраніи убо и позорѣ толикаго народа пострадалъ, яко единъ отъ осужденниковъ, Искупитель нашъ, хрістіанине! Посредѣ двухъ разбойниковъ, въ позоръ ангеломъ и человѣкомъ повѣшенъ былъ на древѣ, яко злодѣй, единъ Праведный и Святый!.. И тако \textit{бысть по насъ клятва}\footnote{Гал.~3,~13.} единъ благословенный во вѣки!.. Болѣзнь и страданіе непорочнаго сердца Его умножаетъ и тое, что Онъ паче злодѣя и убійцы отъ \textit{Своихъ} людей вмѣненъ, и испрошенъ на свободу убійца, а Онъ на смерть испрошенъ отъ Пилата, якоже обличаетъ ихъ Петръ святый: \textit{отвергостеся Его предъ лицемъ Пилатовымъ, суждшу оному пустити. Вы же святаго и праведнаго отвергостеся, и испросисте мужа убійцу дати вамъ: начальника же жизни убисте}\footnote{Дѣян.~3,~13--15.}. За кого пострадалъ? "--- За тебе, и мене непотребнаго, окаяннаго, беззаконнаго, нечестиваго, проклятаго раба Своего, или паче врага Своего, единъ Царь царствующихъ и Господь господствующихъ! Я и ты причиною ужаснаго Его страданія были; мои и твои грѣхи и скверны очищалъ тако на олтарѣ крестномъ непорочный Агнецъ, Сынъ Божій. Мы виноваты, но Сынъ Божій казнь терпѣлъ. Мы законъ Божій нарушили, но Хрістосъ безгрѣшный со \textit{беззаконными вмѣнися}. Мы должники Богу тьмою талантъ, но Хрістосъ, Который ничего не долженъ былъ, долгъ нашъ за насъ кровію Своею платилъ, якоже глаголетъ: \textit{яже не восхищахъ, тогда воздаяхъ}\footnote{Пс.~68,~5.}. "--- 2)~Страданіе и смерть Хрістову разумѣть должно по плоти, отъ Дѣвы насъ ради воспріятой. Ибо Божество есть безстрастно и безсмертно, и потому страдать и умереть не можетъ. Приписуется же Сыну Божію воплотившемуся потому, что Онъ Своею плотію, отъ святой Дѣвы воспріятою, пострадалъ и умеръ. Ибо во Хрістѣ вѣруемъ и исповѣдуемъ \textit{едино лице, но два естества, Божество и человѣчество}, въ единомъ лицѣ неслитно соединившіяся, и потому Хрістосъ есть совершенный Богъ и совершенный человѣкъ. Но, ради единости лица, единъ есть Хрістосъ. И такъ, когда глаголемъ: \textit{Хрістосъ пострадалъ}, то разумѣемъ, что не простый человѣкъ, но Богочеловѣкъ пострадалъ, хотя, какъ выше сказано, страданіе тое разумѣется не по Божеству, но по человѣчеству. "--- 3)~Страданіе Хрістово было \textit{вольное}. Самъ восхотѣлъ чашу сію страданія и смерти насъ ради грѣшныхъ пити, и испилъ, какъ видѣли мы. Тако о Немъ свидѣтельствуетъ святое Божіе слово: \textit{тогда рѣхъ: се пріиду! Въ главизнѣ книжнѣ писано есть о Мнѣ, еже сотворити волю Твою, Боже Мой, восхотѣхъ}, глаголетъ чрезъ пророка\footnote{Пс.~39,~8 и 9.}, которое слово и Апостолъ святый о вольномъ Его страданіи приводитъ\footnote{Евр.~10,~7.}. И паки: \textit{сего ради Мя Отецъ любитъ, яко Азъ душу Мою полагаю, да паки пріиму ю. Никтоже возметъ ю отъ Мене, но Азъ полагаю ю о Себѣ: область имамъ положити ю, и область имамъ паки пріяти ю}\footnote{Іоан.~10,~17 и 18.}. Тако вѣруетъ и исповѣдуетъ святая церковь, якоже поетъ Ему: «Благословенную нарекій Твою Матерь, Хрісте, пришелъ еси на страсть \textit{вольнымъ} хотѣніемъ»\footnote{въ воскр. Богород.~6"~го гл. и на прочіихъ премногихъ мѣстахъ.}. "--- 4)~Таковымъ страданіемъ Своимъ Сынъ Божій заслужилъ намъ у небеснаго Отца милость, которыя мы лишилися"=было грѣхъ ради нашихъ; заслужилъ отпущеніе грѣховъ, и намъ согрѣшившимъ оправданіе. Тако избавилъ насъ отъ власти діавольскія, осужденія, смерти, ада и вѣчнаго мученія, и привелъ насъ въ царствіе Свое, присвоилъ насъ Себѣ отпадшихъ отъ Него, и отворилъ намъ дверь въ вѣчный животъ, блаженство и славу вѣчную. Тако Своею смертію насъ умершихъ оживилъ, Своею язвою насъ исцѣлилъ: \textit{язвою Его мы исцѣлѣхомъ}\footnote{Ис.~53,~5.}. Своимъ смиреніемъ насъ воставилъ, Своею скорбію намъ радость, Своими узами намъ избавленіе и свободу, Своимъ безчестіемъ и поношеніемъ намъ честь и славу, "--- словомъ, Своимъ неизреченнымъ бѣдствіемъ, которое за насъ претерпѣлъ, намъ вѣчное и неизреченное блаженство исходатайствовалъ. "--- 5)~Отсюду видишь, коль великое, тяжкое и ужасное зло есть грѣхъ, который толикое бѣдствіе ввелъ, яко его никто не моглъ съ плодомъ его пагубнымъ отъяти, кромѣ Іисуса Хріста, Сына Божія, Господа славы и Господа силъ. Великъ и тяжекъ есть грѣхъ, такъ что и малѣйшій, по мнѣнію человѣческому, силенъ есть согрѣшившаго погрузити во дно адово, когда не приспѣетъ милость Божія о Хрістѣ Іисусѣ. Ибо всякій грѣхъ \textit{безконечное} величество Божіе оскорбляетъ и \textit{безконечную} правду раздражаетъ. \textit{Сего ради явися Сынъ Божій, да разрушитъ дѣла діаволя}, глаголетъ Іоаннъ святый\footnote{1~Іоан.~3,~8.}. "--- 6)~Отъ сего познавай, коль великую любовь и милость намъ бѣднымъ показалъ Богъ, что и \textit{Сына Своего не пощадѣлъ, но за насъ предалъ Его}\footnote{Римл.~8,~32; 5,~8.}, Который такъ ужасно за наши грѣхи обезчещенъ, посмѣянъ, уязвленъ, мученъ и поносною смертію умеръ. Умъ созданный сея благости и милости Божіей постигнуть не можетъ, хрістіанине! "--- 7)~Видишь, какъ дорого стало Богу и Сыну Божію, Искупителю нашему, спасеніе наше, въ прародителяхъ нами потерянное. \textit{Яко не истлѣннымъ сребромъ или златомъ избавихомся отъ суетнаго житія нашего, отцы преданнаго, но честною кровію яко Агнца непорочна и пречиста Хріста}\footnote{1~Петр.~1,~18 и 19.}, \textit{Егоже язвою мы исцѣлѣхомъ}\footnote{2,~24.}. "--- 8)~Отсюду примѣчай, коль усердно Богу, хотя и за всѣ Его благодѣянія, наипаче за сіе великое дѣло благодарить, любить, почитать, прославлять и хвалить Его должно. 9)~Коль усердное послушаніе Ему показывать обдолжаемся, безъ котораго истинное благодареніе и почитаніе быть не можетъ. 10)~Какъ опасно намъ, которые во Хріста вѣруемъ, и святое имя Его исповѣдуемъ, и крещеніемъ святымъ отъ сквернъ грѣховныхъ очистились, берещися отъ грѣховъ должно, яко за грѣхи наши Хрістосъ муки претерпѣлъ. 11)~Отсюду видишь, что вѣра истинная есть матерь добрыхъ дѣлъ: яко вѣра вся сія благодѣянія Божія представляетъ вѣрующему и научаетъ его, что вся сія благая и ему отъ Бога подаются, которая всему міру изливаются; и все себѣ присвояетъ, что всѣмъ вообще подалося. Самъ разсуди, како захочешь противу совѣсти твоея грѣшить и Богу послушанія не показывать, чѣмъ Богъ оскорбляется и раздражается, когда истинно вѣруешь въ сердцѣ твоемъ, что Богъ есть \textit{твой} Богъ, Господь и Искупитель, Который Сына Своего, какъ за всѣхъ, такъ и за тебе, то"=есть, какъ за грѣхи всѣхъ, такъ и за твои, на смерть предалъ, чего вѣры истинныя свойство требуетъ. 12)~Понеже Хрістосъ есть Богъ \textit{безконечный}, Который насъ ради грѣшныхъ пострадалъ, то и заслуги Его божественныя \textit{безконечны}. А отсюду послѣдуетъ, что и всякіе грѣхи, какъ бы ни были велики, тяжки и ужасны, и сколько бы ихъ ни было, истинно кающемуся и вѣрующему сердечно во Хріста отпущаются ради заслугъ Его, такъ что никакой грѣхъ и никакое множество грѣховъ вѣрующаго побѣдити не можетъ, но всякій грѣхъ и великое грѣховъ множество очищаетъ кровь Іисуса Хріста, ради грѣшника изліянная: яко кровь Хрістова есть \textit{безконечныя} важности и силы, яко кровь Хрістова есть \textit{Божія кровь}\footnote{Дѣян.~20,~28.}. Чего ради и Апостолъ глаголетъ: \textit{кровь Іисуса Хріста Сына Божія очищаетъ насъ}, то"=есть, вѣрующихъ во Хріста, \textit{отъ всякаго грѣха}, то"=есть, какъ бы онъ великъ и тяжекъ ни былъ\footnote{1~Іоан.~1,~7,~13.}. Отъ сего источника согрѣшившимъ и кающимся живое проистекаетъ утѣшеніе, что всякій грѣхъ, ради имени Хрістова, кающемуся отпущается. А гдѣ отпущеніе грѣховъ, тамо вся благая, отъ Бога обѣщанная (о которыхъ въ четвертомъ пунктѣ упомянуто), послѣдуютъ. Симъ утѣшайся, согрѣшившая и сѣтующая за грѣхи душа! 14)~Отъ страстей Хрістовыхъ учиться мы должны свои страсти усмирять и побѣждати, хрістіанине, якоже глаголетъ Хрістосъ: \textit{научитеся отъ Мене}\footnote{Матѳ.~11,~29.}. Скорбь Его душевная, которою \textit{скорбѣлъ до смерти}\footnote{26,~38.}, да научитъ насъ каятися и скорбѣти за грѣхи содѣланные, но не отчаяватися, но на Его благодать и на Божіе милосердіе возлагать надежду. Проданіе Его, отъ неблагодарнаго ученика учиненное, да отвратитъ насъ отъ проданія благодати Его, туне намъ подаваемыя, но паче да увѣщаетъ употреблять тую во славу Божію и ближняго нашего пользу. Вязаніе святѣйшихъ рукъ Его да развяжетъ сердце наше отъ скупости, и руки наши разверзетъ къ помоществованію ближнему и подаянію милостыни, да научитъ отпущати долги должникамъ нашимъ, рукописаніе ихъ раздирать, и въ темницѣ насъ ради сѣдящихъ отпущати на свободу. Претерпѣніе неправеднаго суда и осужденія и слышанныя отъ Него хуленія да удержатъ насъ отъ осужденія, оклеветанія, злословія и празднословія. Претерпѣнное отъ Него посмѣяніе и поруганіе да отвратитъ насъ отъ насмѣянія, руганій и прочіихъ язвительныхъ словъ, на ближнихъ нашихъ отрыгаемыхъ. Оплеваніе святѣйшаго Его лица да не попуститъ женамъ намазывати лицъ бѣлилами, красками и прочіими мастями на показаніе и прельщеніе юныхъ. Обнаженіе святѣйшаго тѣла Его да научитъ одежду имѣть ради прикрытія наготы и согрѣтія плоти немощной, а не ради щегольства, пышности, гордости и тщеславія. Терновное вѣнчаніе и біеніе Божественныя Его главы да увѣщаетъ насъ изгоняти отъ ума и сердца нашего помыслы суетные, злые, скверные и прочіи, волѣ Божіей противные. Крестоношеніе Его да научитъ насъ себе, то"=есть, своея воли и прихотей своихъ отрещися, и, вземши крестъ свой, въ слѣдъ Его ходити, якоже глаголетъ: \textit{аще кто хощетъ по Мнѣ ити, да отвержется себе, и возметъ крестъ свой, и по Мнѣ грядетъ}\footnote{Матѳ.~16,~24.}. Распятіе Его и пригвожденіе Божественныхъ Его удовъ ко кресту да подвигнетъ насъ распинать плоть нашу со страстьми и похотьми, якоже глаголетъ Апостолъ: \textit{иже Хрістовы суть, плоть распяша со страстьми и похотьми}\footnote{Гал.~5,~24.}, "--- и \textit{умертвитъ уды наша, яже на земли, блудъ, нечистоту, страсть, похоть злую и лихоиманіе, еже есть идолослуженіе}, по увѣщанію тогоже Апостола\footnote{Кол.~3,~5.}. Вкушеніе желчи и оцта да удержитъ насъ отъ сладострастія, сластопитанія и роскоши. Смерть Его живоносная да убѣдитъ насъ умрети грѣху, міру и плоти, и жити Ему, за насъ умершему, по увѣщанію Апостола: \textit{Хрістосъ за всѣхъ умре, да живущіи не ктому себѣ живутъ, но умершему за нихъ и воскресшему}\footnote{2~Кор.~5,~15.}. Словомъ, глубочайшее Его смиреніе да отвратитъ насъ отъ гордости. Всесовершенное Его послушаніе, учиненное до \textit{смерти, смерти же крестныя}\footnote{Филип.~2,~8.}, да отвратитъ насъ отъ непокоренія, непослушанія, и да убѣдитъ покоритися и повиноватися Богу и властямъ о Господѣ. Нищетою Его да учимся богатство міра сего, безчестіемъ Его "--- славу, горчайшимъ мученіемъ Его сладострастіе презирать, какъ увѣщаваетъ Апостолъ: \textit{не любите міра, ни яже въ мірѣ. Аще кто любитъ міръ, нѣсть любве Отчи въ немъ. Яко все, еже въ мірѣ, похоть плотская, и похоть очесъ, и гордость житейская, нѣсть отъ Отца, но отъ міра сего есть}\footnote{1~Іоан.~2,~15 и 16.}. Отъ терпѣнія и кротости Его да навыкнемъ нетерпѣніе наше, роптаніе, гнѣвъ и злобу нашу усмирять. Тако почтимъ страсти Хрістовы, и Его насъ ради пострадавшаго. "--- 15)~Видишь, наконецъ, хрістіанине, какъ въ опасномъ состояніи находятся хрістіане тіи, которые безстрашно, безъ покаянія живутъ, но своимъ прихотямъ работаютъ, Хріста исповѣдаютъ, но дѣлами отмещутся Его. Почувствуютъ они, что"=то есть грѣхъ, коль великая горесть его, когда отъ лица Божія на вѣки отвергнутся, и въ гееннѣ огненнѣй съ діаволомъ и аггелами его горестную вѣчнаго мученія чашу будутъ пити, ибо \textit{оброцы грѣха смерть}\footnote{Рим.~6,~23.}. Аще бо Хрістосъ Сынъ Божій за чужіе грѣхи такъ страшно мученъ былъ, что уже постраждутъ за свои грѣхи тіи, которые волею согрѣшаютъ? \textit{Аще въ суровѣ древѣ}, глаголетъ страждущій Хрістосъ, \textit{сія творятъ, въ сусѣ что будетъ}\footnote{Лук.~23,~31.}? \textit{Страшливымъ и невѣрнымъ, и сквернымъ и убійцамъ, и блудъ творящимъ, идоложерцамъ, и всѣмъ лживымъ, часть имъ въ езерѣ, горящемъ огнемъ и жупеломъ, еже есть смерть вторая}\footnote{Апок.~21,~8.}.

\paragraph*{§\:298.} Въ крещеніи святомъ всѣхъ тѣхъ благъ сподобляемся, которыя Хрістосъ страданіемъ и смертію Своею заслужилъ намъ, какъ сказано въ параграфѣ предшедшемъ. 1)~Получаемъ отпущеніе грѣховъ всѣхъ, какіе бы крестящійся ни имѣлъ. О семъ Богъ чрезъ пророка милостивно обѣщалъ: \textit{воскроплю на вы воду чисту, и очиститеся отъ всѣхъ нечистотъ вашихъ}\footnote{Іез.~36,~25.}. Откуду и Петръ святый Апостолъ глаголетъ: \textit{покайтеся, и да крестится кійждо васъ во имя Іисуса Хріста во оставленіе грѣховъ}\footnote{Дѣян.~2,~38.}. И Павелъ святый глаголетъ: \textit{омыстеся, освятистеся, оправдистеся именемъ Господа нашего Іисуса Хріста, и духомъ Бога нашего}\footnote{1~Кор.~6,~11.}. 2)~Свобождаемся отъ смерти и діавольской власти, и приходимъ въ царство Хріста Сына Божія; сотворяемся овцами святаго стада Его и удами церкви святыя, *а послѣдовательно и Его, яко Главы церкве святыя,* такъ что нарицаемъ Его \textit{нашимъ} Царемъ и Господемъ. 3)~Дѣлаемся наслѣдниками вѣчнаго живота и блаженства, по неложному Его обѣщанію: \textit{иже вѣру иметъ и крестится, спасенъ будетъ}\footnote{Марк.~16,~16.}. Откуду Апостолъ о крещеніи глаголетъ: \textit{по своей Его милости спасе насъ банею пакибытія и обновленія Духа Святаго, Егоже излія на насъ обильно Іисусъ Хрістомъ Спасителемъ нашимъ, да, оправдившеся благодатію Его, наслѣдницы будемъ по упованію жизни вѣчныя}\footnote{Тит.~3,~5--7.}. "--- Когда"=де въ крещеніи получаемъ оставленіе грѣховъ, извѣстно то о возрастныхъ крещаемыхъ, а въ младенцахъ какіе грѣхи? Отвѣщаетъ на сіе пророкъ Божій: \textit{въ беззаконіихъ зачатъ есмь, и во грѣсѣхъ роди мя мати моя}\footnote{Пс.~50,~7.}. Видишь, что какъ онъ, такъ и мы во грѣхахъ зачинаемся и раждаемся. Того ради и младенцы требуютъ омовенія отъ сквернъ грѣховныхъ, дабы, очистившися, вошли въ вѣчный животъ.

\paragraph*{§\:299.} Крещеніемъ святымъ раждается вновь, или отраждается человѣкъ не по естеству, (ибо по естеству тойжде пребываетъ человѣкъ, тоежде тѣло, душа, чувства, дыханіе и прочія естественныя силы и дѣйствія), но по внутреннему сердца, мысли, разума, воли, хотѣнія обновленію и просвѣщенію. Откуду крещеніе святое называется \textit{баня пакибытія}\footnote{Тит.~3,~5.}, и крестившійся \textit{новая тварь: аще кто во Хрістѣ, нова тварь}\footnote{2~Кор.~5,~17.}. И тако человѣкъ, вѣру хрістіанскую и крещеніе воспріявшій, \textit{двоякое} имѣетъ \textit{рожденіе: первое "--- плотское}, въ которомъ зачинается въ беззаконіяхъ и раждается во грѣсѣхъ, и по сему рожденію всякъ человѣкъ раждается \textit{чадомъ гнѣва}\footnote{Еф.~2,~3.}; и сіе рожденіе происходитъ отъ ветхаго Адама, по которому рожденію называемся вси, отцы наши и мы, сыны Адамовы. \textit{Второе рожденіе} есть \textit{крещеніе}, которое водою и Духомъ бываетъ, и есть духовное, святое, благословенное, блаженное, и происходитъ отъ Хріста. И якоже ветхій Адамъ есть отецъ нашъ по плоти, тако Хрістосъ есть Отецъ нашъ по сему духовному рожденію, и Отецъ вѣчный, \textit{Отецъ будущаго вѣка}\footnote{Ис.~9,~6.}.

Сіе духовное рожденіе дѣйствуется: 1)~Вѣрою во Хріста, якоже глаголетъ Апостолъ: \textit{всякъ вѣруяй, яко Іисусъ есть Хрістосъ, отъ Бога рожденъ есть}\footnote{1~Іоан.~5,~1.}. 2)~Духомъ Святымъ: \textit{аще кто не родится водою и Духомъ, не можетъ внити въ царствіе Божіе}\footnote{3,~5; 1~Кор.~6,~11.}. 3)~Крещеніемъ святымъ, якоже научаетъ Хрістосъ: \textit{шедше научите вся языки, крестяще ихъ во имя Отца и Сына и Святаго Духа}\footnote{Матѳ.~28,~19.}. Возрастнымъ нужно есть наставленіе изъ слова Божія прежде крещенія, яко отъ слышанія слова Божія зачинается вѣра, по реченному: \textit{вѣра отъ слуха, слухъ же глаголомъ Божіимъ}\footnote{Римл.~10,~17.}. И Петръ святый глаголетъ къ вѣрнымъ: \textit{порождени не отъ сѣмене истлѣнна, но неистлѣнна, словомъ живаго Бога}\footnote{1~Петр.~1,~23.}. Чего ради возрастнымъ прежде крещенія, а младенцамъ крещеннымъ нужно есть необходимо наставленіе, какъ только начнутъ въ возрастъ приходити и ученіе понимать.

\paragraph*{§\:300.} Понеже въ человѣкѣ, который хрістіанскую вѣру и крещеніе воспріялъ, ветхое или плотское, новое или духовное рожденіе и потому два рожденія имѣются: отъ сихъ двухъ рожденій возстаетъ въ немъ \textit{брань духа и плоти}, или, какъ простѣе сказать, двоякое наклоненіе, движеніе, побужденіе и поощреніе; понеже два сіи рожденія суть противны себѣ, и свое свойство каждое отъ нихъ содержитъ, и къ тому человѣческое наклоняетъ и побуждаетъ сердце, что которому свойственно есть. О чемъ Апостолъ святый тако глаголетъ: \textit{плоть похотствуетъ на духа, духъ же на плоть; сія же другъ другу противятся}\footnote{Гал.~5,~17.}. Плотское рожденіе наклоняетъ хрістіанина къ плотскому мудрованію: духовное къ мудрованію духовному. Плотское къ исканію чести, славы, богатства земнаго движетъ: духовное отводитъ отъ того, и движетъ къ исканію вѣчныхъ благъ, славы и блаженства вѣчнаго. Плотское побуждается къ гордости и высокоумію: духовное къ смиренномудрію. Плотское поощряетъ ко гнѣву, ярости, злобѣ, мщенію: духовное отвращаетъ отъ того, и поощряетъ къ терпѣнію, тихости, незлобію, кротости. Плотское побуждаетъ къ ненависти ближняго: духовное отъ того уклоняетъ, и склоняетъ къ любви. Плотское къ немилосердію и жестокости, духовное къ милосердію движетъ. Плотское къ сребролюбію и міролюбію: духовное къ любви Божіей влечетъ. Плотское сердце къ ссорѣ и враждѣ: духовное къ миру, согласію и единомыслію благочестивому привлекаетъ. Плотское къ піянству, невоздержанію, нечистотѣ: духовное къ трезвости, воздержанію и цѣломудрію побуждаетъ. Плотское ко лжи, лести, лицемѣрію: духовное къ истинѣ, простосердечію ведетъ. Вкратцѣ, плотское рожденіе наклоняетъ человѣка къ самолюбію, исполненію воли своея и угожденію себѣ самому: духовное къ Боголюбію, къ послушанію Богу и творенію воли Божіей, и отъ всего того отводитъ, что волѣ Божіей противно есть. Сія брань въ единомъ и томъ же хрістіанинѣ есть. Сіе есть всегдашнее сраженіе между плотію и духомъ, яко \textit{плоть похотствуетъ на духа, духъ же на плоть}. Въ семъ состоитъ подвигъ хрістіанскій, въ которомъ должно намъ до конца жизни нашея подвизатися, плоть духу покорять, \textit{да не яже хощемъ, сія творимъ}\footnote{Гал.~5,~17.}, но что воля Божія, законъ Божій и новое рожденіе отъ насъ требуетъ. И сіе"=то есть\textit{: плоть распинать со страстьми и похотьми}\footnote{24.}, \textit{огребатися отъ плотскихъ похотей, яже воюютъ на душу}\footnote{1~Петр.~2,~11.}, \textit{плотоугодія не творити въ похоти}\footnote{Римл.~13,~14.}, \textit{отложити по первому житію ветхаго человѣка, тлѣющаго въ похотехъ прелестныхъ: обновлятися же духомъ ума нашего, и облещися въ новаго человѣка, созданнаго по Богу въ правдѣ и въ преподобіи истины}\footnote{Еф.~4,~22--24.}; \textit{духомъ дѣянія плотская умерщвляти}\footnote{Римл.~8,~13.}; \textit{похоти плотскія не совершати}\footnote{Гал.~5,~16.}; \textit{не по плоти ходити, но по духу}\footnote{Римл.~8,~1 и 4.}; \textit{не себѣ жити, но умершему за насъ и воскресшему Хрісту}\footnote{2~Кор.~5,~15.}. Ибо въ хрістіанинѣ отрожденномъ какъ два рожденія, такъ два и человѣка съ свойствами своими имѣются, \textit{ветхій}, то"=есть, и \textit{новый}. Ветхій свое свойство злое показуетъ, и наклоняетъ хрістіанина ко злу и всякому грѣху, какъ выше сказано: новый также свое свойство доброе являетъ, и возбуждаетъ человѣка"=хрістіанина къ добру. Хрістіанинъ, когда хощетъ новое рожденіе сохранить и истиннымъ хрістіаниномъ быти, долженъ ветхаго человѣка не слушать, дѣяній его не творить, и такъ всякаго зла уклоняться; а послѣдовать свойствамъ новаго рожденія, новаго человѣка, и тако всякому доброму дѣлу прилѣжать. Къ сему подвигу апостоли святіи вездѣ въ посланіяхъ своихъ святыхъ поощряютъ насъ. Тоежде и Хрістосъ въ святомъ Евангеліи Своемъ предлагаетъ намъ, когда велитъ \textit{внити узкими враты}\footnote{Матѳ.~7,~13.}, \textit{отрещися себе, и взять крестъ свой, и послѣдовати Ему}\footnote{16,~24.}, \textit{взяти иго Свое, и учитися отъ Него}\footnote{11,~29.}. И Отецъ небесный съ небесе глаголетъ: \textit{Того послушайте}\footnote{17,~5.}. Къ тому и святіи отцы, пастыріе и учители наши насъ увѣщаваютъ.

\paragraph*{§\:301.} Отъ вышеписанныхъ плотскаго и духовнаго рожденія свойствъ и движеній \textit{двоякое} бываетъ \textit{мудрованіе} "--- плотское и духовное. Кто плотскому рожденію или ветхому человѣку послѣдуетъ, волю его исполняетъ, тотъ плотское мудрствуетъ, дѣла плотская творитъ. Кто дѣянія плотская духомъ умерщвляетъ, и новаго рожденія плоды показуетъ, тотъ духовная мудрствуетъ. Дѣла плотская Апостолъ святый показуетъ намъ: \textit{явлена суть дѣла плотская, яже суть: прелюбодѣяніе, блудъ, нечистота, студодѣяніе, идолослуженіе, чародѣянія, вражды, рвенія, завиды, ярости, разжженія, распри, соблазны, ереси, зависти, убійства, піянства, безчинны кличи, и подобная симъ}\footnote{Гал.~5,~19--21.}. Сюда принадлежатъ исчисленныя дѣла\footnote{1~Кор.~6,~9 и 10; Римл.~1,~29--32; 1~Тим.~1,~9 и 10; 2~Тим.~3,~2--5 и на прочіихъ мѣстахъ.}. Плодъ духовный тойжде Апостолъ показуетъ: \textit{плодъ духовный есть любы, радость, миръ, долготерпѣніе, благость, милосердіе, вѣра, кротость, воздержаніе}\footnote{Гал.~5,~22.}.

Отъ сего видно: 1)~что есть духовный, и что есть плотскій человѣкъ. \textit{Духовный человѣкъ} есть, который духовная мудрствуетъ, плотская дѣянія умерщвляетъ, и плоды новаго рожденія показуетъ, хотя и жену имѣетъ, дѣтей раждаетъ, ружье носитъ, кавалеріею украшается. \textit{Плотскій} человѣкъ есть, который по плоти живетъ, дѣла плотская творитъ, хотя бы рясою, клобукомъ и мантіею покрывался, или наружнымъ крестомъ украшался. Не платье бо и прочая наружность дѣлаетъ хрістіанина и духовнаго человѣка, но вѣра живая, плодами украшенная, якоже, напротивъ, плотскаго дѣлаетъ невѣріе съ беззаконными плодами, какъ изъ апостольскаго ученія и вышепрописаннаго разсужденія можно всякому видѣть. 2)~Видишь, что есть \textit{истинный хрістіанинъ}. То есть, человѣкъ духовный, который не міру сему прилѣпляется, но, оставляя его, ищетъ духовныхъ и небесныхъ благъ; противу грѣха подвизается, и правыя, истинныя и сердечныя вѣры плоды, то"=есть, добрыя дѣла тщится творити. 3)~Видишь наконецъ, что есть \textit{ложный хрістіанинъ}. То есть, тотъ, который хотя Бога и Хріста Сына Божія исповѣдуетъ, и крещенъ во имя Святыя Троицы, но дѣлами своими, какъ древо злое злыми плодами своими, себе не отъ числа хрістіанъ, но отъ числа язычниковъ показуетъ; какъ бо древо доброе отъ вкуса плодовъ добрыхъ, такъ хрістіанинъ отъ дѣлъ добрыхъ познается; и какъ древо злое отъ вкуса злыхъ плодовъ, такъ ложный хрістіанинъ отъ злыхъ дѣлъ показуется.

\paragraph*{§\:302.} Плотскаго и духовнаго \textit{мудрованія конецъ} слово Божіе показуетъ. Плотскаго конецъ: \textit{аще по плоти живете, имате умрети}\footnote{Рим.~8,~13.}, то"=есть, вѣчною смертію, ибо временно вси, плотскіи и духовныи, святіи и грѣшніи, умираютъ\footnote{6,~21 и 23; 8,~6; 1~Кор.~6,~10; Гал.~5,~21; Филип.~3,~19 и проч.}. Духовнаго мудрованія конецъ: \textit{ни едино нынѣ осужденіе сущимъ о Хрістѣ Іисусѣ, не по плоти ходящимъ, но по духу}\footnote{Римл.~8,~1.}; и ниже: \textit{аще духомъ дѣянія плотская умерщвляете, живи будете}\footnote{ст.~13 и проч.}. Конецъ убо, къ которому ведетъ плотское мудрованіе, есть вѣчная смерть и мука, \textit{оброцы бо грѣха смерть}\footnote{6,~23.}; духовнаго же мудрованія конецъ, къ которому духовная мудрствующіи идутъ, есть вѣчный животъ. Отъ сего видно, что хрістіанамъ неисправнымъ святое крещеніе и православныхъ догматовъ содержаніе ничего не пользуетъ, пока пребываютъ неисправны. Чего ради всякому хрістіанину должно осматриваться, имѣетъ ли вѣру спасающую, какъ Апостолъ увѣщаваетъ: \textit{себе искушайте, аще есте въ вѣрѣ, себе искушайте}\footnote{2~Кор.~13,~5.}.


\section[Статія 3-я. О должности хрістіанской къ Богу.]{статія третья.\\\bfseries О должности хрістіанской къ Богу.}

Хотя изъ вышеписанныхъ можно видѣть, что мы, хрістіане, должны Богу нашему, помиловавшему и спасшему насъ; однакожъ въ сей статьѣ пространнѣе и яснѣе о томъ представляется, что хрістіанинъ, по силѣ вѣры и званія своего, обдолжается Богу показывать, о чемъ на своихъ мѣстахъ предложится.

\subsection[Глава 1-я. О должномъ Богу послушаніи.]{глава первая.\\\bfseries О должномъ Богу послушаніи.}

\begin{quotation}\textit{Овцы Моя гласа Моего слушаютъ, и Азъ знаю ихъ, и по Мнѣ грядутъ. И Азъ животъ вѣчный дамъ имъ}, глаголетъ Хрістосъ\footnote{Іоан.~10,~27.}.\end{quotation}
\begin{quotation}\textit{Иже есть отъ Бога, глаголовъ Божіихъ послушаетъ: сего ради вы не послушаете, яко отъ Бога нѣсте}, глаголетъ Хрістосъ Іудеомъ\footnote{8,~47.}.\end{quotation}

\paragraph*{§\:303.} Послушаніе, которое хрістіанинъ долженъ Богу своему показывать, состоитъ въ томъ, чтобы онъ волю свою волѣ Божіей во всемъ старался покорять, и что Богъ ни повелѣваетъ, на тое всеохотнымъ себе показывать, и не смотрѣть, что міру или плоти угодно, но что Богу есть благоугодно, якоже Апостолъ написалъ: \textit{якоже чада свѣта ходите (плодъ бо духовный есть во всякой благостыни и правдѣ и истинѣ), искушающе, что есть благоугодно Богови}\footnote{Еф.~5,~9 и 10.}. Истинное послушаніе не смотритъ, трудно ли, или удобно, тяжко ли, или легко есть, что повелѣвается; но на всякое повелѣніе, какое бы оно ни было, готовымъ себе показываетъ, съ пророкомъ отдая сердце свое въ послушаніе Богу: \textit{готово сердце мое}\footnote{Пс.~107,~2.}. Оно не повелѣваемое, но повелѣвающаго лице, волю и хотѣніе разсуждаетъ, кто и коликъ есть, и что хощетъ отъ него повелѣвающій; довольствуется тѣмъ, что Богъ его есть Той, Который повелѣваетъ, и того, что повелѣвается, хощетъ отъ него. Богъ, глаголю, Создатель и Промыслитель его, а не иной кто. Сего ради не повелѣнія, но повелѣвающаго разсуждаетъ силу и власть. Откуду и охотнымъ себе на повелѣніе Его показуетъ. И тако, что хощетъ Повелѣвающій, тое хощетъ и творитъ; и чего не хощетъ, не хощетъ и не дѣлаетъ того и оно. Тако послушаніе показалъ Богу Авраамъ, праотецъ святый, какъ читаемъ въ книгѣ Бытія. Когда ему Богъ повелѣлъ: \textit{изыди отъ земли твоея, и отъ рода твоего, и отъ дому отца твоего, и иди въ землю, юже ти покажу}, и прочая\footnote{Быт.~12,~1 и слѣд.}, "--- онъ не отреклся учинить того, оставилъ отечество свое, родъ свой и домъ отца своего, и пошелъ въ землю чуждую, незнаемую, якоже Апостолъ о томъ написалъ: \textit{вѣрою зовомъ Авраамъ, послуша изыти на мѣсто, еже хотяше пріяти въ наслѣдіе, и изыде невѣдый, камо грядетъ}\footnote{Евр.~11,~8.}. Еще большаго удивленія достойно дѣло учинилъ патріархъ святый, что по повелѣнію Божію сына своего Исаака привелъ и вознеслъ на жертвенникъ на всесожженіе, и хотя не дѣломъ, но произволеніемъ заклалъ его\footnote{Быт.~22,~1--12.}. Сынъ былъ единородный, сынъ возлюбленный отцу, сынъ по обѣщанію Божію данный и въ старости рожденный, сынъ, о которомъ сказано ему отъ Бога, что \textit{о Исаацѣ наречется тебѣ сѣмя}\footnote{Быт.~21,~12; Евр.~11,~18.}. Но вѣрный и искренній послушникъ не отрекся на всесожженіе вознести его; не смотрѣлъ на такъ тяжкое и неудобное повелѣніе, но Повелѣвающаго власти, силѣ и всемогуществу внималъ, \textit{помысливъ}, глаголетъ Апостолъ, \textit{яко и изъ мертвыхъ воскресити силенъ есть Богъ}\footnote{Евр.~11,~19.}. А отъ искренняго своего послушанія искреннюю свою въ Бога вѣру показалъ праотецъ святый. Вѣра его къ послушанію возбуждала, и отъ послушанія показалася вѣра его, яко отъ добраго плода доброе древо. Что Апостолъ святый о немъ написалъ: \textit{Авраамъ отецъ нашъ не отъ дѣлъ ли оправдася, вознесъ Исаака сына своего на жертвенникъ? Видиши ли, яко вѣра поспѣшествоваше дѣломъ его, и отъ дѣлъ совершися вѣра}\footnote{Іак.~2,~21 и 22.}.

\paragraph*{§\:304.} Святый Авраамъ, который выше во образъ искренняго послушанія приведенъ, означенъ отъ Бога \textit{отцемъ вѣрующихъ}, ради вѣры и послушанія его, якоже рече ему Богъ: \textit{отца многихъ языковъ положихъ тя}\footnote{Быт.~17,~4 и 5; Римл.~4,~16 и 17.}. И вѣрніи суть сынове Авраамли, по ученію Апостола: \textit{сущіи отъ вѣры, сіи суть сынове Авраамли}\footnote{Гал.~3,~7.}, то"=есть, не по плоти, но по духу сынове Авраамли, не плотская чада, но духовная. Того ради, хрістіанине, когда мы хощемъ чадами Авраамлими быти, и благословеніе Авраамле получити, \textit{благословитися съ вѣрнымъ Авраамомъ}\footnote{ст.~9.}, должно намъ вѣрѣ Авраама, яко \textit{отца вѣрныхъ}, подражать, и вѣру нашу къ Богу отъ послушанія нашего показывать, якоже отецъ вѣрныхъ Авраамъ учинилъ, какъ мы видѣли въ параграфѣ предшедшемъ. Аще чада Авраамли хощемъ быти, дѣла Авраамля должны творити, якоже глаголетъ Хрістосъ Іудеомъ: \textit{аще чада Авраамля бысте были, дѣла Авраамля бысте творили}\footnote{Іоан.~8,~39.}. Сіе слово сказано Іудеомъ тѣмъ, которые о семъ великомъ патріархѣ хвалились, нарицая его отцемъ своимъ: \textit{отца имамы Авраама}\footnote{Матѳ.~3,~9.}, \textit{отецъ нашъ Авраамъ есть}\footnote{Іоан.~8,~39.}, но въ вѣрѣ и дѣлахъ отцу тому не послѣдовали. Не повелѣваетъ намъ Богъ оставити отечество наше, родъ и домъ нашъ, и ити въ землю чужую, хотя и тое учинить должны мы въ случаѣ, ради имени Его святаго, когда того честь имени Его требуетъ\footnote{Матѳ.~19,~29.}, но повелѣваетъ оставити міръ съ прелестьми и похотьми. \textit{Не любите міра, ни яже въ мірѣ. Аще кто любитъ міръ, нѣсть любве Отчи въ немъ. Яко все, еже въ мірѣ, похоть плотская, и похоть очесъ, и гордость житейская, нѣсть отъ Отца, но отъ міра сего есть}\footnote{Іоан.~2,~15 и 16.}. Повелѣваетъ намъ Богъ, \textit{яко пришельцамъ и странникамъ, огребатися отъ плотскихъ похотей, яже воюютъ на душу}\footnote{1~Петр.~2,~11.}, и \textit{горняя мудрствовати, а не земная}\footnote{Кол.~3,~2.}. Не повелѣваетъ закалать сыновъ нашихъ, или себе умерщвлять, хотя въ случаѣ за честь имени Его и животъ нашъ пожрети должны, но глаголетъ: \textit{умертвите уды ваша, яже на земли, блудъ, нечистоту, страсть, похоть злую, и лихоиманіе, еже есть идолослуженіе}\footnote{ст.~5.}. Глаголетъ: \textit{представите тѣлеса ваша жертву живу, святу, благоугодну Богови, словесное служеніе ваше; и не сообразуйтеся вѣку сему, но преобразуйтеся обновленіемъ ума вашего, во еже искушати вамъ, что есть воля Божія благая и угодная и совершенная}\footnote{Римл.~12,~1 и 2.}. Словомъ, повелѣваетъ умрети грѣху, міру и плоти, и жити Ему, то"=есть, Ему работати, Ему благоугодная творити, глаголати и мыслити, якоже глаголетъ Апостолъ: \textit{Хрістосъ за всѣхъ умре, да живущіи не ктому себѣ живутъ, но умершему за нихъ и воскресшему}\footnote{2~Кор.~5,~15.}. Сего бо вѣра наша, которую имѣемъ въ Него, отъ насъ требуетъ, и къ сему вѣра вѣрное поощряетъ сердце. Аще убо хощемъ вѣрными быти, чадами вѣрнаго Авраама быть, Хрістовыми быть, "--- \textit{иже} бо \textit{Хрістовы суть, плоть распяша со страстьми и похотьми}\footnote{Гал.~5,~24.}, "--- то покажемъ вѣру нашу въ Него отъ послушанія нашего къ Нему; покажемъ вѣру нашу отъ дѣлъ нашихъ, якоже Апостолъ требуетъ: \textit{покажи ми вѣру твою отъ дѣлъ твоихъ}\footnote{Іак.~2,~18.}. \textit{Якоже бо тѣло безъ духа мертво есть, тако и вѣра безъ дѣлъ мертва есть}\footnote{ст.~5,~26.}.

\paragraph*{§\:305.} Послушаніе показывать Богу, то"=есть, отъ зла уклонятися и благое творити должно намъ, хрістіанине; но надежду избавленія и спасенія полагать на единаго Хріста. Ибо 1)~какъ отъ вѣчныя смерти, ада и мученія избавляемся, такъ и вѣчный животъ получаемъ не нашимъ, но Хрістовымъ послушаніемъ, \textit{вѣрою о Хрістѣ Іисусѣ, Иже бысть намъ премудрость отъ Бога, правда же и освященіе и избавленіе}\footnote{1~Кор.~1,~30.}, \textit{Иже смирилъ Себе, послушливъ бывъ даже до смерти, смерти же крестныя}\footnote{Фил.~2,~8.}. \textit{Якоже бо ослушаніемъ единаго человѣка}, Адама, \textit{грѣшни быша мнози: сице и послушаніемъ единаго}, Іисуса Хріста, \textit{праведни будутъ мнози}, глаголетъ Апостолъ\footnote{Римл.~5,~19.}. Какъ убо оправдаемся предъ Богомъ вѣрою во Хріста, такъ слѣдственно и спасаемся. Идѣже бо оправданіе, тамо и спасеніе. Ибо какъ оправданію спасеніе послѣдуетъ, такъ безъ оправданія спасенія не бываетъ. Ослушаніемъ нашимъ гнѣву Божію и вѣчной смерти подпадаемъ, но послушаніемъ своимъ отъ того избавитися и вѣчный животъ заслужити не можемъ. Надобно о томъ ко Хрісту Сыну Божію, яко \textit{Ходатаю нашему}\footnote{1~Іоан.~2,~1.} вѣрою прибѣгать. Онъ намъ \textit{всесовершеннѣйшимъ} Своимъ послушаніемъ, какъ избавленіе отъ смерти, такъ и наслѣдіе вѣчнаго живота заслужилъ, когда чистосердечно въ Него вѣруемъ. 2)~Намъ стараться о томъ должно, чтобы того спасенія, которое Онъ намъ заслужилъ, своимъ невѣріемъ и ослушаніемъ не потерять. Откуду повелѣно намъ: \textit{со страхомъ и трепетомъ спасеніе свое содѣвайте}\footnote{Филип.~2,~12.}. \textit{Со страхомъ житія вашего время жительствуйте, вѣдяще, яко не истлѣннымъ сребромъ или златомъ избавистеся отъ суетнаго вашего житія, отцы преданнаго, но честною кровію яко Агнца непорочна и пречиста Хріста}\footnote{1~Петр.~1,~17--19.}. Когда бо хощемъ получить спасеніе оное, то должно намъ отъ всего того берещися, что заключаетъ дверь къ спасенію тому. Заключаетъ же непослушаніе и грѣхъ. \textit{Всякъ, имѣяй надежду сію Нань, очищаетъ себе, якоже Онъ чистъ есть}, глаголетъ Апостолъ\footnote{1~Іоан.~3,~3.}, то"=есть, бережется отъ грѣховъ и о прежнихъ усердно кается. 3)~Послушаніе наше есть должное Богу, ибо созданіе Создателю своему должно послушаніе и почитаніе отдавати. Подданный царю своему и рабъ господину своему долженъ повиновеніе и послушаніе показывать, кольми паче мы Богу, Который есть Царь и Господь нашъ, и есть \textit{Царь царствующихъ и Господь господствующихъ}\footnote{1~Тим.~6,~15.}, повиновеніе, послушаніе и почитаніе, которое безъ послушанія не можетъ быть, показывать должны. Обида и оскорбленіе бываетъ царю, когда *подданный ему, и господину, когда* рабъ ему должнаго послушанія не отдаетъ; много паче Богу обида отъ насъ бываетъ, когда Ему, яко Господу и Царю своему, повиновенія и послушанія не показываемъ. Откуду глаголетъ Господь чрезъ пророка: \textit{сынъ славитъ отца, и рабъ господина своего убоится. И аще Отецъ есмь Азъ, то гдѣ слава Моя? И аще Господь есмь Азъ, то гдѣ есть страхъ Мой? глаголетъ Господь Вседержитель}\footnote{Мал.~1,~6.}. Сего ради, когда послушаніе показуемъ Богу, то должное Ему отдаемъ, и тѣмъ заслужить у Него ничего не можемъ, много паче вѣчный животъ, который есть безконечное добро. Откуду глаголетъ Хрістосъ: \textit{егда сотворите вся повелѣнная вамъ, глаголите, яко раби неключими есмы: яко еже должни бѣхомъ сотворити, сотворихомъ}\footnote{Лук.~17,~10.}. 4)~Животъ вѣчный есть даръ Божій, который Хрістосъ, Сынъ Божій, кровію Своею заслужилъ вѣрующимъ во имя Его, якоже глаголетъ Апостолъ: \textit{дарованіе Божіе "--- животъ вѣчный о Хрістѣ Іисусѣ Господѣ нашемъ}\footnote{Рим.~6,~23.}. Человѣкъ бо, яко грѣшникъ и самъ въ себѣ проклятый, и не иному чему, какъ вѣчному осужденію подлежащій, не можетъ заслужить и не заслуживаетъ, какъ только смерть, якоже глаголетъ: \textit{оброцы грѣха смерть}\footnote{тамъ же.}. Самъ себе погубить можетъ, и погубляетъ, но самъ себе спасти не можетъ. Надобно Сыну Божію, яко Пастырю доброму, \textit{взять его, яко овцу заблуждшую, на рамена Своя}\footnote{Лук.~15,~4--6.}, и принести ко Отцу Своему, якоже глаголетъ Хрістосъ: \textit{Азъ есмь путь и истина и животъ: никтоже пріидетъ ко Отцу, токмо Мною}\footnote{Іоан.~14,~16.}. Онъ овецъ Своихъ, то"=есть, вѣрующихъ въ Него, и слушающихъ гласа его, и по Немъ грядущихъ, пасетъ, спасаетъ и отверзаетъ имъ дверь вѣчнаго живота, якоже глаголетъ: \textit{овцы Моя гласа Моего слушаютъ, и Азъ знаю ихъ, и по Мнѣ грядутъ, и Азъ животъ вѣчный дамъ имъ}\footnote{10,~27.}. 5)~Послушаніе наше Богу не можемъ безъ Божіей благодати показывать. Тако бо свидѣтельствуетъ Хрістосъ: \textit{безъ Мене не можете творити ничесоже}\footnote{15,~5.}. И Апостолъ Хрістовъ: \textit{Богъ есть дѣйствуяй въ васъ, и еже хотѣти и еже дѣяти о благоволеніи}\footnote{Филип.~2,~13.}. Откуду повелѣно намъ просити, искати и толкати, чтобъ подалася на то намъ Божія благодать. \textit{Просите, и дастся вамъ; ищите, и обрящете; толцыте, и отверзется вамъ}\footnote{Матѳ.~7,~7.}. Како убо можемъ заслужить послушаніемъ нашимъ вѣчный животъ, когда и послушаніе истинное есть не наше собственное, но Божіей благодати дѣло, намъ же приписуется, что мы Божіей благодати не противимся, но \textit{дѣйствующей въ насъ} содѣйствуемъ? Откуду Іудеомъ ожесточеннымъ сказано: \textit{жестоковыйніи и необрѣзанніи сердцы и ушесы! вы присно Духу Святому противитеся}\footnote{Дѣян.~7,~51.}. 6)~Послушаніе наше есть несовершенное. Аще бо и тщится вѣрная душа послушаніе Богу своему показывать, но часто немощію плоти своея запинается въ томъ, \textit{духъ бо бодръ, плоть же немощна}\footnote{Матѳ.~26,~41.}. Откуду и святіи въ немощи сей находятся, и о отпущеніи молится всякъ преподобный\footnote{Пс.~31,~11.}, и воздыхаютъ къ небесному Отцу: \textit{остави намъ долги наша}, и проч.: \textit{грѣхопаденія бо кто разумѣетъ}\footnote{18,~13.}? Послушаніе убо должны мы показывать Богу нашему, да засвидѣтельствуемъ отъ послушанія того вѣру нашу, или, по словеси Апостола, \textit{покажемъ вѣру нашу отъ дѣлъ нашихъ}\footnote{Іак.~2,~18.}. А *отъ вѣчныя смерти избавленія и* вѣчныя жизни наслѣдія вѣрою во Хріста просить и искать должно намъ отъ Бога; чему и молитвы самыя, отъ церкви святой пріятыя и чтомыя, научаютъ насъ, въ которыхъ просимъ у Бога нашего милости, благодати, отпущенія грѣховъ, оправданія, святости, избавленія отъ ада и вѣчныя смерти, и наслѣдія вѣчнаго живота вѣрою о Хрістѣ Іисусѣ Господѣ нашемъ. Читай самъ со вниманіемъ молитвы тѣ, и разсуждай прилѣжно, то и самъ признаеши сіе за истину.

\paragraph*{§\:306.} \textit{Причины}, которыя къ послушанію поощряютъ насъ, суть сіи: 1)~\textit{Благость Божія} оная, которою мы изъ небытія въ бытіе приведены и образомъ Его Божественнымъ почтены, можетъ и должна насъ воздвигнуть повелѣніямъ Его повиноватися. Онъ нашъ Создатель, Богъ и Господь; убо яко Создателя, Бога и Господа своего всеохотно слушати должны, и святая Его повелѣнія исполняти. Отцу по плоти, который насъ родилъ, господину нашему, который насъ стяжалъ, Государю нашему, котораго власти временно мы подчинены, словомъ, человѣку, который надъ нами не самъ собою, но по Божію строенію власть имѣетъ, повинуемся; кольми паче Богу, Который насъ создалъ и промышляетъ о насъ, и всѣми властями и подчиненными владѣетъ, повиноватися подобаетъ. Безстыдно убо и беззаконно не слушать Того, отъ Котораго и сами мы созданы, и все, что ни имѣемъ, отъ Него получаемъ и всевысочайшей и вѣчной Его власти подчинены. 2)~Къ послушанію Его да подвигнетъ насъ Божественная \textit{Его любовь}, которую Онъ о Хрістѣ Іисусѣ показалъ къ намъ. Сію любовь Его проповѣдалъ намъ Хрістосъ, Сынъ Его: \textit{тако возлюби Богъ міръ, яко и Сына Своего единороднаго далъ есть, да всякъ, вѣруяй въ Онь, не погибнетъ, но имать животъ вѣчный}\footnote{Іоан.~3,~16.}. Проповѣдалъ Павелъ святый: \textit{составляетъ Свою любовь къ намъ Богъ, яко, еще грѣшникомъ сущимъ намъ, Хрістосъ за ны умре}\footnote{Римл.~5,~8.}. И Іоаннъ святый: \textit{о семъ явися любы Божія въ насъ, яко Сына Своего единороднаго посла Богъ въ міръ, да живи будемъ Имъ}\footnote{1~Іоан.~4,~9.}. И удивляется сей любви: \textit{видите, какову любовь далъ есть Отецъ намъ, да чада Божія наречемся, и есмы}\footnote{3,~1.}! Сіе Его великое и умомъ непостижимое человѣколюбіе можетъ и должно насъ возбудить къ послушанію Его, и творенію повелѣній Его. Возлюбилъ Онъ насъ, какъ отецъ, возлюбимъ и мы Его, какъ отца чада, и отъ любви послушаніе Ему, какъ Отцу милостивому, показать потщимся, яко \textit{чада послушанія}\footnote{1~Петр.~1,~14.}. \textit{Возлюбимъ} и мы \textit{другъ друга, якоже заповѣдь далъ есть намъ}\footnote{1~Іоан.~3,~23.}. \textit{Аще сице возлюбилъ есть насъ Богъ, и мы должни есмы другъ друга любити}\footnote{4,~11.}. 3)~\textit{Обѣты}, учиненные нами при крещеніи святомъ, да подвигнутъ насъ къ послушанію. Тогда отрицались мы сатаны и сквернаго его служенія, и обѣщались Богу служить, \textit{служить преподобіемъ и правдою предъ Нимъ вся дни живота нашего}\footnote{Лук.~1,~75.}, что не можетъ быть безъ усерднаго послушанія. Служить бо Богу и послушанія Ему не показывать суть противна себѣ, якоже служить господину и не слушать его есть невозможное дѣло. Аще же сихъ обѣтовъ не исполнимъ, то самые оные противу насъ станутъ на судѣ Его, и обличатъ и осудятъ насъ, яко лживыхъ и невѣрныхъ Господу своему, и тако посрамимся предъ всѣмъ міромъ, яко обѣщались Богу служить вѣрою и правдою, и не служили, но солгали Ему. 4)~Отъ послушанія истиннаго показуется истинный и нелицемѣрный хрістіанинъ, якоже непослушаніе есть доказательство нехрістіанскаго сердца, ложнаго и лицемѣрнаго хрістіанина. \textit{О семъ разумѣемъ}, глаголетъ Апостолъ святый, \textit{яко познахомъ Его, аще заповѣди Его соблюдаемъ. Глаголяй, яко познахъ Его, и заповѣди Его не соблюдаетъ, ложь есть, и въ семъ истины нѣсть}\footnote{1~Іоан.~2,~3 и 4.}. Якоже бо раба къ господину своему вѣрность отъ усерднаго его послушанія и работы, и невѣрность отъ непослушанія и лицемѣрія познается: тако и вѣрный рабъ Божій отъ сего примѣчается, что онъ усердное показуетъ послушаніе Господу своему, и Ему вѣрно работаетъ, какъ невѣрный отъ непослушанія и непокоренія. Коль же велико есть быть истиннымъ и вѣрнымъ рабомъ Божіимъ, якоже невѣрнымъ и ложнымъ весьма бѣдственно, что изъ слѣдующихъ узриши, читатель! 5)~Отъ непослушанія христіанинъ, невѣрнымъ Богу учинившись, лишается всей той милости Божіей, которой въ крещеніи святомъ сподобился"=было, и подпадаетъ праведному Его гнѣву и вѣчному осужденію. Якоже бо отецъ плотскій отрекается сына своего непослушливаго и неисправнаго, и лишаетъ его наслѣдія; тако Богъ, небесный Отецъ, отрекается ложнаго хрістіанина; не признаетъ его за Своего, яко онъ самъ не знаетъ и не слушаетъ Его; выключаетъ его изъ числа вѣрныхъ своихъ, и тако заключаетъ ему дверь царствія небеснаго, якоже глаголетъ Хрістосъ: \textit{не всякъ глаголяй Ми: Господи, Господи, внидетъ въ царствіе небесное, но творяй волю Отца Моего, Иже есть на небесѣхъ}\footnote{Матѳ.~7,~21.}. "--- И прочія причины, которыя святое Божіе слово представляетъ.

\paragraph*{§\:307.} \textit{Знаки} тщательнаго \textit{послушанія} примѣчаются сіи: 1)~Чтеніе, или слышаніе святаго Писанія, и въ томъ прилѣжное поученіе на такій конецъ, дабы волю Божію, въ немъ откровенную, познать и ее творить. Ибо святое Писаніе научаетъ насъ, чего отъ насъ воля Божія хощетъ и требуетъ. 2)~Исканіе и требованіе полезнаго совѣта у разумныхъ и по Бозѣ живущихъ людей, которые и сами тщатся волю Божію творить, и другихъ къ тому могутъ наставить. 3)~Удаленіе отъ такихъ бесѣдъ и собраній, которыя дѣлаютъ соблазнъ и ударяютъ въ совѣсть, и подаютъ случай ко грѣху, и тако отъ послушанія отводятъ. \textit{Тлятъ бо}, глаголетъ Апостолъ, \textit{обычаи благи бесѣды злы}\footnote{1~Кор.~15,~33.}. 4)~Молчаніе, на такій конецъ воспріятое, дабы празднословіемъ не согрѣшить, и тако бы отъ послушанія Богу должнаго не отпасть. Ибо ничѣмъ такъ не согрѣшаетъ человѣкъ, и тако заповѣди Божіей не разоряетъ, какъ необузданнымъ языкомъ\footnote{Іак.~3,~5--8.}. 5)~Усердная и частая молитва, ради того бываемая, чтобъ Самъ Богъ наставилъ на путь заповѣдей Своихъ, и творить волю Свою научилъ, и прочая.

\paragraph*{§\:308.} \textit{Послушанію}, которое истинные хрістіане хотятъ и тщатся показывать небесному Отцу, \textit{препятствіе} чинятъ: 1)~сатана со злыми своими аггелами. Сей есть оный змій древній, который прародителей нашихъ въ раю отъ послушанія Богу должнаго отвратилъ, завидя ихъ блаженству, и подвергъ всякому бѣдствію\footnote{Быт.~3"~я.}. Той всехитрый, врагъ, видя и нынѣ хрістіанъ, Божіею благодатію возставленныхъ и возобновленныхъ и въ милости Его находящихся, и завидуя ихъ блаженству, которое имъ въ будущемъ вѣкѣ уготовано, тщится ихъ отъ Бога отвратить чрезъ преслушаніе, и въ прежнее бѣдственное состояніе, и тако съ собою въ вѣчную погибель вринуть, якоже съ прародителями учинилъ. 2)~Слѣпая и слабая плоть наша такожде хощетъ насъ къ преслушанію привести. Сія, какъ Ева въ раи Адаму яблоко отъ заповѣданнаго древа, духу нашему міръ со своими веселостями и красотами, похотію плотскою и похотію очесъ и гордостію житейскою, что \textit{нѣсть отъ Отца, но отъ міра сего есть}\footnote{1~Іоан.~2,~16.}, предлагаетъ, и безстрашно отъ тѣхъ похотей вкушать совѣтуетъ, да тако, вкусивше отъ того, что запрещено, заповѣдь Божію разоримъ и Заповѣдавшаго прогнѣваемъ. 3)~Соблазны злыхъ и развращенныхъ людей, которые, или развращеннымъ ученіемъ, или злымъ и ядовитымъ примѣромъ дѣлъ своихъ, развращаютъ неосторожныхъ и въ тоежде бѣдствіе хотятъ насъ вринуть. Сему всему злу должны хрістіане противитися, и заповѣди Бога своего хранить. И хотя вѣра и духъ нашъ усердно хощетъ и нудится послушаніе Богу показывать и святыя Его повелѣнія исполнять; но отъ сихъ враговъ удобно запинается, когда всесильная Божія помощь не приспѣетъ и не укрѣпитъ. Того ради нужна намъ необходимо помощь Божія, нужна противу врага діавола, нужна противу свирѣпѣющія плоти и безчинныхъ ея страстей, нужна противу соблазновъ злаго міра. А чтобъ помощь Божія въ нужномъ случаѣ къ намъ приспѣла и укрѣпила насъ, или паче всегда съ нами пребывала, яко всегда ея требуемъ, нужна есть усердная и частая молитва, которою благодать Божія помогающая испрашивается. Откуду повелѣно намъ бдѣти и молитися: \textit{бдите и молитеся, да не внидете въ напасть}\footnote{Матѳ.~26,~41.}, "--- молитися: \textit{не введи насъ во искушеніе, но избави насъ отъ лукаваго}, Отче!\footnote{6,~13; Лук.~11,~4.}, "--- всегда \textit{молитися}\footnote{18,~1.}, "--- \textit{непрестанно} молитися\footnote{1~Сол.~5,~17.}; "--- повелѣно просить, искать, толкать: \textit{просите и дастся вамъ; ищите, и обрящете; толцыте, и отверзется вамъ}\footnote{Матѳ.~7,~7.}.

\paragraph*{§\:309.} Безъ истиннаго послушанія ничто Богу не угодно: 1)~ни исповѣданіе имени Его, якоже глаголетъ Хрістосъ: \textit{что Мя зовете, Господи, Господи, и не творите, яже глаголю}\footnote{Лук.~6,~46.}? \textit{Да отступитъ убо отъ неправды всякъ, именуяй имя Господне}, глаголетъ Апостолъ\footnote{2~Тим.~2,~19.}. Неправда бо преслушаніе, якоже правда послушаніе заключаетъ въ себѣ. Ибо Господь во всѣхъ заповѣдяхъ Своихъ какъ правду хранить, такъ отъ неправды удаляться повелѣваетъ; и потому кто имя Господне святое и страшное нарицаетъ, исповѣдуетъ и призываетъ, долженъ какъ правду, въ заповѣдяхъ Его повелѣнную, хранить, такъ отъ неправды, въ тѣхъ же заповѣдяхъ запрещенныя, удаляться. Безъ чего исповѣданіе имени Господня какъ Господу не угодно, такъ и исповѣдующему не полезно. "--- 2)~Ни молитва. Ибо писано есть: \textit{яко грѣшники Богъ не послушаетъ; но аще кто богочтецъ есть, и волю Его творитъ, того послушаетъ}\footnote{Іоан.~9,~31.}. И самъ Богъ чрезъ пророка глаголетъ беззаконнымъ: \textit{егда прострете руки ваша ко Мнѣ, отвращу очи Мои отъ васъ, и аще умножите моленіе, не услышу васъ; руки бо ваша исполнены крове}\footnote{Ис.~1,~15.}. "--- 3)~Ни пѣніе, ни хваленіе, ни славословіе, ни прочее устное почитаніе Богу не пріятно безъ сердечнаго почитанія, якоже о таковыхъ глаголетъ Богъ: \textit{приближаются Мнѣ людіе сіи усты своими, и устнами чтутъ Мя: сердце же ихъ далече отстоитъ отъ Мене}\footnote{Матѳ.~15,~8; Ис.~29,~13.}. Надобно убо неотмѣнно очистить себе истиннымъ покаяніемъ, и тогда призывать, пѣть, хвалить и славить святое имя Божіе, которое святіи ангели со благоговѣніемъ поютъ\footnote{Ис.~6,~3.}. Всякое бо устное почитаніе безъ сердечнаго есть лицемѣріе. "--- 4)~И проповѣданіе слова Божія безъ послушанія Богу неугодно, и проповѣдующему не полезно. \textit{Грѣшнику бо рече Богъ: вскую ты повѣдаеши оправданія Моя, и воспріемлеши завѣтъ Мой усты твоими? Ты же возненавидѣлъ еси наказаніе, и отверглъ еси словеса Моя вспять}, и проч.\footnote{Пс.~49,~16,~17 и слѣд.} Надобно бо прежде исправить себе, и тогда людей исправлять; себе научить, и потомъ людей научать, якоже Апостолъ глаголетъ: \textit{научая инаго, себе ли не учиши?}\footnote{Римл.~2,~21.}, дабы не приличествовала притча сія: \textit{врачу, исцѣлися самъ}\footnote{Лук.~4,~23.}. "--- 5)~Ни даръ чудотворенія, ни прочія дарованія намъ не пользуютъ, и тѣми угодити не можемъ Богу безъ послушанія. О семъ глаголетъ Хрістосъ: \textit{мнози рекутъ Мнѣ во онъ день}, въ день страшнаго суда: \textit{Господи, Господи, не въ Твое ли имя пророчествовахомъ, и Твоимъ именемъ бѣсы изгонихомъ, и твоимъ именемъ силы многи сотворихомъ? И тогда исповѣмъ имъ: яко николиже знахъ васъ. Отъидите отъ Мене дѣлающіе беззаконіе}\footnote{Матѳ.~7,~22 и 23.}. Симъ показуетъ Хрістосъ, что не токмо вѣра, но и знаменій твореніе, творящему ничего не пользуетъ безъ добродѣтели. Глаголетъ на сіе слово Златоустъ святый, и, мало спустя, полагаетъ причину сему: «Ты не дивися сему; всякое бо дарованіе отъ милости дарующаго бываетъ. Они же отъ себе ничего не трудились, и того ради достойны наказанія, яко неблагодарны и нечувственны были къ Тому, Который ихъ такъ почтилъ, что и недостойнымъ подалъ дарованіе»\footnote{Бес.~24"~я.}. Не знаменія бо творити, но волю Свою исполнити и любити Господь нашъ повелѣлъ намъ, якоже написалъ Апостолъ: \textit{конецъ завѣщанія есть, любы отъ чиста сердца и совѣсти благи и вѣры нелицемѣрныя}\footnote{1~Тим.~1,~5.}. Знаменія бо и чудеса творити не наше, но единыя Его всемогущія силы дѣло есть. И хотя и волю Божію творити и заповѣди Его исполняти, безъ благодати Божіей, не можемъ; однакожъ отъ нашей стороны требуется тщаніе, трудъ, соизволеніе и послушаніе \textit{дѣйствующей благодати}. Богъ бо помогаетъ трудящимся, а не лежащимъ. Просящимъ дается, ищущіе обрѣтаютъ, толкущимъ отверзается. \textit{Всякъ бо просяй пріемлетъ, и ищай обрѣтаетъ, и толкущему отверзается}, глаголетъ Хрістосъ\footnote{Матѳ.~7,~8.}. О семъ пунктѣ смотри еще 1~Кор.~13,~1--3\textit{}, гдѣ показуется, что безъ любви намъ ничто не пользуетъ. Любовь же къ Богу познается отъ соблюденія заповѣдей Его, яко глаголетъ Хрістосъ: \textit{имѣяй заповѣди Моя, и соблюдаяй ихъ, той есть любяй Мя}\footnote{Іоан.~14,~21.}, слѣдственно и отъ послушанія. Ибо тотъ соблюдаетъ заповѣди Божія, кто Богу послушаніе показуетъ. И тако видишь, что безъ послушанія ничего Богу угоднаго сдѣлати не можемъ, что ни дѣлаемъ. Причина тому сія есть: понеже, когда человѣкъ не тщится Богу послушанія показывать, "--- во гнѣвѣ у Бога находится; и тако, когда самое лице во гнѣвѣ у Бога имѣется, то и всякое дѣло его неугодно Ему бываетъ. Богъ бо велитъ намъ \textit{каятися, и творити плоды достойны покаянія}\footnote{Матѳ.~3,~2 и 8.}, что не можетъ быть безъ оставленія грѣховъ; велитъ вѣровати во имя Іисуса Хріста, Сына Его, и любить другъ друга: \textit{сія есть заповѣдь Его, да вѣруемъ во имя Сына Его Іисуса Хріста, и любимъ другъ друга}\footnote{1~Іоан.~3,~23.}. Но когда человѣкъ сего творити не хощетъ, не показуетъ Богу послушанія, и такъ во гнѣвѣ у Бога находится, слѣдственно ничего благоугоднаго сдѣлати не можетъ. Того ради должно прежде обратитися чистосердечно къ Богу, оставити грѣхи, перемѣнить сердце свое, и начать новое житіе, вѣрою во Хріста примиритися Богу, и тако \textit{въ новости жизни благоугодно жити}\footnote{Римл. гл.~6"~я.}.

\paragraph*{§\:310.} \textit{Плоды непослушанія или преслушанія} суть. 1)~Грѣхъ. Грѣхъ бо не иное что есть, какъ беззаконіе или противленіе закону Божію, по описанію Апостола: \textit{грѣхъ есть беззаконіе}\footnote{1~Іоан.~3,~4.}. То"=есть: что Богъ въ законѣ Своемъ повелѣваетъ, того человѣкъ не дѣлаетъ, и такъ противится закону Божію, а противяся закону Божію, беззаконіе совершаетъ. Тако праотецъ Адамъ, не послушавъ заповѣди Божіей и отъ заповѣданнаго древа вкусивъ, \textit{согрѣшилъ}\footnote{Быт.~3,~6.}. И тако \textit{единѣмъ человѣкомъ грѣхъ въ міръ вниде}\footnote{Римл.~5,~12.}. Тако и нынѣ, когда не слушаетъ человѣкъ Бога и заповѣди Его святыя нарушаетъ, согрѣшаетъ противу величества Божія. 2)~Плодъ преслушанія есть безпокойствіе совѣсти, которая, непослушаніемъ и законопреступленіемъ раздраженная, не престаетъ обличать, безпокойствовать и снѣдать человѣка. Тако прародители наши, въ раи согрѣшивши, \textit{сокрылися отъ лица Божія посредѣ древъ райскихъ}\footnote{Быт.~3,~8.}, чего прежде согрѣшенія не дѣлали. Ибо какъ прежде согрѣшенія совѣсть ихъ покойна была, и ничего не боялися, потому и не искали убѣжища: такъ, по согрѣшеніи, начала ихъ обличать и устрашать, почему и начали укрываться. Тоежде зло и нынѣ согрѣшающихъ постигаетъ: \textit{бѣгаетъ бо нечестивый ни единомуже гонящу}\footnote{Притч.~28,~1.}. 3)~Временная и вѣчная казнь есть плодъ преслушанія.

\paragraph*{§\:311.} Коль великое и тяжкое зло и безстыдство есть непослушаніе, которое человѣкъ Богу показуетъ, отсюду видно, что 1)~Бога святаго, великаго и безконечнаго, и Господа, Который весь міръ въ руцѣ содержитъ, Которому ангели святіи съ великимъ благоговѣинствомъ покланяются, предъ Которымъ долженъ трепетать, благоговѣинствовать и со всякимъ смиреніемъ покланятися и повиноваться, "--- онъ, \textit{земля и пепелъ}, но разумомъ одаренный, безстрашно и безстыдно не слушаетъ. "--- 2)~Законъ Его святый, вѣчный и непремѣняемый и на то данный, чтобъ цѣлъ и ненарушимъ былъ, дерзаетъ безстрашно нарушать. Ибо въ законѣ святомъ изображена непремѣнная правда, которая учитъ всякому свое отдавать; изображено тое, что такъ должно быть, какъ повелѣно, а не иначе. Напримѣръ, Бога почитать, и ближнему не дѣлать того, чего не хощешь себѣ, и дѣлать то, что хощешь себѣ; сего вѣчная правда и самая совѣсть наша отъ насъ требуетъ, и убѣждаетъ такъ, а не иначе дѣлать. Но грѣшникъ безстрашный дерзаетъ тое перемѣнять, что само въ себѣ непремѣняемо, и нарушать тое, что свято, цѣло и ненарушимо во вѣки должно быть. Отсюду въ грѣшникѣ непокаявшемся пробудится и возстанетъ совѣсть, таковымъ его безстрашіемъ и безстыдствомъ раздраженная, и чрезъ всю вѣчность будетъ его обличать и мучить, что такъ безстрашно и безстыдно противу Бога и святаго Его закона поступалъ. Отсюду и гнѣвъ Божій праведный возгорится на грѣшника, который вовѣки на себѣ будетъ чувствовать, и тѣмъ претяжко мучитися, такъ что пожелаетъ въ ничто обратитися и умереть, но не можетъ. "--- 3)~Отсюду паки видѣти можемъ, коль страшною слѣпотою изъ чрева матерня пораженъ человѣкъ, такъ что непросвѣщенніи благодатію Божіею \textit{сѣдящимъ во тьмѣ}, которые ничего не видятъ, уподобляются. \textit{Сѣдящіи во тьмѣ видѣша свѣтъ велій}\footnote{Матѳ.~4,~16.}. Чего оплакать довольно не можемъ, что человѣкъ, разумомъ одаренный и образомъ Божіимъ почтенный, толь сильно обезумился, обезобразился и ослѣпился праотеческимъ грѣхомъ. Разсуди всякъ, не крайняя ли се слѣпота? Царя земнаго, который надъ тѣломъ только власть имѣетъ, и то не безъ воли Божіей, и такойже человѣкъ, какъ и прочіи, боится, почитаетъ и слушаетъ человѣкъ; но Бога, Царя небеси и земли, у Котораго \textit{весь міръ въ руцѣ}\footnote{Пс.~94,~4.}, Который \textit{и душу и тѣло можетъ погубити въ гееннѣ}\footnote{Матѳ.~10,~28.}, не боится, не почитаетъ и не слушаетъ человѣкъ! Царя земнаго, котораго судъ и гнѣвъ есть только временный, и тѣла единаго касается, прогнѣвать опасается; но Бога вѣчнаго, Котораго судъ и гнѣвъ вѣчному предаетъ мученію, прогнѣвлять не опасается!.. И тако временнаго суда и наказанія боится и трепещетъ, но вѣчнаго наказанія, противу котораго всякое временное наказаніе ничтоже есть, не боится и не трепещетъ, послѣдуя несмысленнымъ младенцамъ, которые мертвыя личины боятся и убѣгаютъ, но горячаго углія касаются и обжигаются. Отца по плоти, отъ котораго рожденъ и воспитанъ, любитъ и почитаетъ; но Бога, отъ Котораго и онъ самъ и отецъ его созданъ, и всякое довольствіе къ житію, пищу, питіе и одежду получаетъ, и на всякую минуту безъ благихъ Его не можетъ быти, яко \textit{о Немъ живемъ, движемся и есмы}\footnote{Дѣян.~17,~28.}, не любитъ, не почитаетъ и волѣ Его не повинуется. И сіе"=то есть, что пророкъ Моѵсей въ пѣсни своей выговариваетъ непокоривому Израилю: \textit{роде строптивый и развращенный, сія ли Господеви воздаете? Сіи людіе буіи и немудріи. Не самъ ли Сей Отецъ твой стяжа тя, и сотвори тя, и созда тя?} И ниже: \textit{Бога, рождшаго тя, оставилъ еси, и забылъ еси Бога питающаго тя}\footnote{Второз.~32,~5,~6 и 18.}. Человѣку благодѣтелю, отъ котораго какое нибудь добро получилъ или получаетъ (хотя и то все Божіе есть, ибо все Божіе есть, что ни имѣемъ), усердно благодаритъ и угождаетъ; но Богу, отъ Котораго безчисленныя благодѣянія получаетъ, не благодаритъ и не угождаетъ. Друга любителя своего любитъ и оскорбить бережется; но Бога, Любителя своего, Котораго человѣколюбіе къ намъ неизреченно и непонятно, оскорбить преступленіемъ закона Его не бережется. Самъ человѣкъ отъ человѣка, напримѣръ, отецъ отъ сына, господинъ отъ раба, и всякая власть отъ подвластнаго своего требуетъ послушанія, и гнѣвается, когда не видитъ того; но самъ Богу, Который есть Отецъ и Господь всѣхъ, и власть вѣчную надъ всѣми и надъ нимъ имѣетъ, не хощетъ послушанія показывать и не чувствуетъ того. Видишь, какъ слѣпъ и безуменъ человѣкъ, непросвѣщенный благодатію Божіею. "--- 4)~Отъ сего видишь, что непослушаніе отъ самолюбія происходитъ, которое почитаетъ болѣе свою волю, нежели повелѣвающаго Господа; якоже послушаніе отъ Боголюбія, которое болѣе любитъ Бога, нежели себе, и такъ волю Его волѣ своей предпочитаетъ, и потому заповѣди Его соблюдаетъ, якоже глаголетъ Хрістосъ: \textit{аще любите Мя, заповѣди моя соблюдите}; и паки: \textit{имѣяй заповѣди моя и соблюдаяй ихъ, той есть любяй Мя}\footnote{Іоан.~14,~15 и 21.}. Видишь наконецъ, что какъ истинная и сердечная вѣра есть матерь послушанія и добрыхъ дѣлъ, такъ невѣріе есть корень непослушанія и злыхъ дѣлъ и всякихъ грѣховъ.

\paragraph*{§\:312.} Изъ вышеописанныхъ видишь, хрістіанине, что хрістіане, которые не показуютъ Богу послушанія, но по своей волѣ и прихотямъ живутъ, послѣдуютъ ветхому Адаму, который, оставивши Божію заповѣдь, послушалъ совѣта діавольскаго, и тако лишился Божіей благодати, подпалъ праведному Божію гнѣву, клятвѣ и смерти. Тако и хрістіане неисправные, послѣдуя ему, оставляютъ заповѣди Божія, и какъ Псаломникъ глаголетъ, \textit{отвергаютъ словеса Божія вспять}\footnote{Пс.~49,~17.}, и слушаютъ тогожде врага совѣта, и волю его злую въ плотскихъ своихъ похотяхъ исполняютъ; а тако показуютъ, что они, хотя и крещены, не имѣютъ новаго духовнаго рожденія, но въ ветхомъ пребываютъ, находятся внѣ числа истинныхъ хрістіанъ, внѣ церкви святыя, и тако находятся во гнѣвѣ Божіи, клятвѣ и вѣчному осужденію подлежатъ. А послѣдовательно Хрістосъ, Котораго имя исповѣдуютъ, ничего имъ не пользуетъ, по словеси Хрістову: \textit{не всякъ глаголяй Ми, Господи, Господи, внидетъ въ царствіе небесное, но творяй волю Отца Моего, Иже есть на небесѣхъ}\footnote{Матѳ.~7,~21.}, пока себе не исправятъ и не покаются. А когда обратятся и чистосердечно каяться будутъ, и покаяніе свое плодами покаянія засвидѣтельствуютъ, "--- паки Хрістосъ "--- \textit{ихъ Хрістосъ} будетъ со всѣми своими Божественными заслугами. Напротивъ того, хрістіане, которые тщатся послушаніе Богу показывать, "--- волю Божію и заповѣди Его соблюдаютъ, послѣдуютъ Хрісту, Сыну Божію, \textit{новому безгрѣшному Адаму, и Господу съ небесе}\footnote{1~Кор.~15,~47.}, Который совершенно волю Божію и законъ Божій исполнилъ, Который \textit{грѣха не сотвори, ни обрѣтеся лесть во устѣхъ Его}\footnote{1~Петр.~2,~22.}, Который \textit{смирилъ Себе, послушливъ бывъ даже до смерти, смерти же крестныя}\footnote{Филип.~2,~8.}; и тѣмъ показуютъ, что они истинные хрістіане суть, истинно во Хріста вѣруютъ, истинно Хрістовы суть: \textit{иже плоть распяша со страстьми и похотьми}\footnote{Гал.~5,~24.}; истинно \textit{отъ Бога рождени суть, Бога знаютъ}\footnote{1~Іоан.~4,~7; 5,~1.}, и въ Бога вѣруютъ, и находятся у Бога въ милости и благодати. И тако видишь, что какъ отъ послушанія истиннаго истинный хрістіанинъ, такъ отъ непослушанія ложный познается хрістіанинъ. Ибо гдѣ послушаніе истинное къ Богу, тамо истинная вѣра въ Бога имѣется, которая человѣка Богу послушливымъ дѣлаетъ; а гдѣ нѣтъ истиннаго послушанія, тамо нѣтъ вѣры истинныя, но притворная. Истинная бо вѣра не можетъ быть безъ послушанія, якоже видѣли мы о Авраамѣ, который во образъ и отца вѣрнымъ положенъ. Смотри о семъ святаго Писанія мѣста: \textit{аще чада Авраамля бысте были, дѣла Авраамля бысте творили. Нынѣ же ищете Мене убити, человѣка, иже истину вамъ глаголахъ, юже слышахъ отъ Бога: сего Авраамъ нѣсть сотворилъ}, глаголетъ Хрістосъ Іудеомъ\footnote{Іоан.~8,~39 и 40.}. \textit{Нѣсте отъ овецъ Моихъ, якоже рѣхъ вамъ}. \textit{Овцы Моя гласа Моего слушаютъ, и Азъ знаю ихъ, и по Мнѣ грядутъ}\footnote{10,~26 и 27.}. \textit{Еже видѣхомъ и слышахомъ, повѣдаемъ вамъ, да и вы общеніе имате съ нами: общеніе же наше со Отцемъ и съ Сыномъ Его Іисусомъ Хрістомъ}, и проч. \textit{Богъ свѣтъ есть, и тьмы въ Немъ нѣсть ни единыя. Аще речемъ, яко общеніе имамы съ Нимъ, и во тьмѣ} (то"=есть, во грѣховной тьмѣ) \textit{ходимъ, лжемъ и не творимъ истины. Аще же во свѣтѣ ходимъ, якоже Самъ Той есть во свѣтѣ, общеніе имамы другъ ко другу, и кровь Іисуса Хріста Сына Его очищаетъ насъ отъ всякаго грѣха}, глаголетъ Іоаннъ святый съ прочіими апостолами, братіею своею, всѣмъ вѣрнымъ\footnote{1~Іоан.~1,~3--7.}. \textit{О семъ разумѣемъ, яко познахомъ Его, аще заповѣди Его соблюдаемъ. Глаголяй, яко познахъ Его, и заповѣди Его не соблюдаетъ, ложь есть, и въ семъ истины нѣсть}\footnote{2,~3 и 4.}. \textit{Аще вѣсте, яко праведникъ есть Богъ, разумѣйте, яко всякъ, творяй правду, отъ Него родился}\footnote{1~Іоан.~2,~29.}. \textit{Возлюбленніи! нынѣ чада! Божіи есмы, и не у явися, что будемъ; вѣмы же, яко, егда явится, подобни Ему будемъ: узримъ Его, якоже есть. И всякъ, имѣяй надежду сію Нань, очищаетъ себе, якоже Онъ чистъ есть}\footnote{3,~2 и 3.}. \textit{Всякъ, иже въ Немъ пребываетъ, не согрѣшаетъ: всякъ согрѣшаяй не видѣ Его, ни позна Его}\footnote{ст.~6.}. И паки: \textit{творяй грѣхъ отъ діавола есть: яко исперва діаволъ согрѣшаетъ}\footnote{ст.~8.}. И паки: \textit{всякъ рожденный отъ Бога, грѣха не творитъ, яко сѣмя Его въ немъ пребываетъ; и не можетъ согрѣшати, яко отъ Бога рожденъ есть. Сего ради явлена суть чада Божія и чада діаволя}, и проч.\footnote{ст.~9,~10 и слѣд. См. еще: 4,~7 и 8; 5,~18; Іак.~2,~18 и проч.; Іоан.~8,~34 и 44; 15,~2; Гал.~5,~24; Пс.~14, и проч.} Внимай сему и разсуждай, хрістіанине! Вѣры свойство есть Бога \textit{своимъ} Богомъ, Господемъ и Царемъ признавати, исповѣдати и нарицати, якоже читаемъ въ книгахъ пророческихъ, апостольскихъ и святыхъ отцевъ и учителей церковныхъ: такожде и въ церковныхъ пѣсняхъ видимъ тое. Того ради, когда ты вѣруешь, что Богъ \textit{твой} есть Богъ, Господь, Царь и Избавитель, то должно тебѣ вѣру твою отъ дѣлъ твоихъ показать, должно тебѣ Богу, Господу, Царю и Избавителю послушаніе и повиновеніе показывать. Какій бо рабъ господина своего *и подданный царя своего* не слушаетъ? Рабу господина какого за господина своего признавать и его не слушать, и рабомъ господина того нарицаться и ему не работать, уму невмѣстимо и невозможное дѣло есть. Такъ и Бога признавать отъ сердца за \textit{своего} Бога, Господа и Царя, и Ему не показывать должнаго послушанія, есть невозможная вещь. Отъ того слѣдуетъ, что кто повиновенія и послушанія Богу не показываетъ, знаменіе есть, что Его и за Господа и Бога своего не имѣетъ, и потому вѣры въ Него истинныя не имѣетъ, хотя устами и исповѣдуетъ Его. Не едиными бо устами исповѣдывать, но и сердцемъ вѣровать въ Бога есть истинная и спасительная вѣра, якоже пишется: \textit{сердцемъ вѣруется въ правду, усты же исповѣдуется во спасеніе}\footnote{Римл.~10,~10.}.

\subsection[Глава 2-я. О прославленіи имени Божія.]{глава вторая.\\\bfseries О прославленіи имени Божія.}

\begin{quotation}\textit{Тако да просвѣтится свѣтъ вашъ предъ человѣки, яко да видятъ ваша добрая дѣла, и прославятъ Отца вашего, Иже на небесѣхъ}\footnote{Матѳ.~5,~16.}.\end{quotation}
\begin{quotation}\textit{Аще ясте, аще ли піете, аще ли ино что творите, вся въ славу Божію творите}\footnote{1~Кор.~10,~31.}.\end{quotation}
\begin{quotation}\textit{Принесите Господеви славу и честь, принесите Господеви славу имени Его}\footnote{Пс.~28,~1 и 2.}.\end{quotation}

\paragraph*{§\:313.} Имя Божіе само въ себѣ какъ свято, такъ славно и препрославлено есть; того ради отъ насъ не требуетъ прославленія нашего. И якоже солнце, хвалится ли, или хулится, всегда свѣтло въ себѣ пребываетъ, и лучи свѣта своего на всю поднебесную низпущаетъ: тако имя Божіе, славится ли, или хулится отъ человѣкъ, равно всегда, всегда славно, свято и страшно пребываетъ, и лучи славы своея издаетъ; издаетъ въ созданіяхъ, ибо \textit{небеса повѣдаютъ славу Божію}\footnote{18,~2.}, \textit{и исполнена вся земля славы Его}\footnote{Ис.~6,~3.}, издаетъ въ дивныхъ и страшныхъ дѣлахъ Его, преславныхъ чудесахъ Его. И потому какъ отъ хуленія человѣческаго не умаляется, такъ и отъ прославленія не умножается слава имени Его. Слава бо имени Божія вѣчна, безконечна и непремѣняема есть, какъ и Самъ Богъ; того ради ни умножитися, ни умалитися въ себѣ не можетъ. Но наша хрістіанская должность требуетъ того, чтобы мы какъ сами въ себѣ славили оное, такъ пресѣкали тое, чѣмъ оно хулится, и хулящимъ заграждали уста, сколько можемъ. Ибо сынъ добрый о славѣ отца своего, и рабъ вѣрный о славѣ господина своего всегда и вездѣ печется и, чѣмъ слава ихъ и честь помрачается, тое пресѣкаетъ; кольми паче хрістіанамъ, яко сынамъ и рабамъ Божіимъ, о славѣ Отца небеснаго и Господа своего пещися и ревновать должно, якоже Самъ Господь глаголетъ: \textit{сынъ славитъ отца, и рабъ господина своего убоится. И аще Отецъ есмь Азъ, то гдѣ слава Моя? И аще Господь есмь Азъ, то гдѣ страхъ Мой? глаголетъ Господь Вседержитель}\footnote{Мал.~1,~6.}.

\paragraph*{§\:314.} Великое имя Божіе заключаетъ въ себѣ Божественныя Его свойства, никакой твари несообщаемыя, но Ему единому собственныя, какъ"=то: единосущіе, присносущіе всемогущество, благость, премудрость, вездѣсущіе, всевѣдѣніе, правду, святость, истину, духовное существо, и прочая. Сіи собственныя свойства открываетъ намъ Духъ Святый въ Словѣ Своемъ, и различно изображаетъ ихъ къ просвѣщенію нашему и прославленію имени Божія. Истинное и дѣйствительное познаніе Божіихъ свойствъ (сколько можно человѣку въ семъ вѣкѣ познать), и, какъ сказать, \textit{вкушеніе}, по пророческому ученію, якоже глаголетъ: \textit{вкусите и видите, яко благъ Господь}\footnote{Пс.~33,~9.}, содѣловаетъ въ человѣкѣ просвѣщеніе и премѣненіе, и тако къ прославленію, имени Божія приводитъ. Отъ \textit{познанія единосущнаго Божества} послѣдуетъ, что мы будемъ единаго Бога знать и почитать, и кромѣ Его никакихъ другихъ боговъ не признавать и не почитать, якоже Самъ глаголетъ: \textit{Азъ есмь Господь Богъ твой: да не будутъ тебѣ бози иніи, развѣ Мене}\footnote{Исх.~20,~2 и 3.}; и паки: \textit{слыши, Израилю! Господь Богъ нашъ, Господь единъ есть}\footnote{Втор.~6,~4.}. Аще бо и вѣруемъ и исповѣдуемъ Тріѵпостаснаго Бога, Отца и Сына и Святаго Духа, но едино Божество въ тріехъ Ѵпостасѣхъ почитаемъ. Отъ \textit{познанія присносущнаго бытія}, безсмертія и вѣчности Божія послѣдуетъ, что Онъ \textit{единъ существенно живетъ}, и есть источникъ живота нашего: отъ Него единаго животъ нашъ зависитъ, и житіе наше не иное что, какъ тѣнь древа движущагося, которая тогда движется, когда движется древо. А тако научаемся, что мы въ себѣ, и отъ кого животъ нашъ зависитъ: отъ Того, Который всегда живетъ, и не можетъ не жити, и все оживляетъ и движетъ. Тако убѣждаемся приписывать Ему бытіе, животъ и движеніе наше со Апостоломъ: \textit{о Немъ живемъ и движемся и есмы}\footnote{Дѣян.~17,~28.}. А отъ того гордость, пышность и надмѣніе наше низпадаетъ. "--- \textit{Познанію вездѣсущія Божія} послѣдуетъ благоговѣинство къ Богу; яко когда Богъ вездѣ есть, то и со мною и тобою и со всякимъ есть, и предъ Нимъ ходимъ; и что ни дѣлаемъ, говоримъ и мыслимъ, предъ Нимъ дѣлаемъ, говоримъ и мыслимъ. Тако удержится человѣкъ отъ безчинія, но будетъ предъ лицемъ Его яко Создателя, Царя и Господа своего, благоговѣинство показывать. Аще бо предъ лицемъ земнаго царя нашего и человѣка стыдимся и опасаемся всякое, и малѣйшее безчиніе показывать; кольми паче предъ лицемъ Бога, Царя небесе и земли, Иже \textit{Царь царствующихъ и Господь господствующихъ}\footnote{Тим.~6,~15.}! Сему послѣдуетъ дерзновеніе противу страха вражія: яко Богъ съ нами есть, яко можемъ со Псаломникомъ въ нужномъ случаѣ утѣшать себе: \textit{аще и пойду посредѣ сѣни смертныя, не убоюся зла, яко Ты со мною еси, Боже}\footnote{Пс.~22,~4.}! Вездѣ и на всякомъ мѣстѣ можемъ Его призывать и молитися Ему, яко вездѣ съ нами есть. Никуды отъ суда Его убѣжать не можемъ, яко вездѣ постигаетъ и предваряетъ насъ Своимъ присутствіемъ\footnote{Іерем.~23,~24; Пс.~138,~7--12.}. Но тѣмъ самымъ убѣждаемся смиряться предъ Нимъ, съ умиленіемъ падать и просить прощенія и помилованія. "--- \textit{Познаніе невещественнаго и духовнаго существа Божія} научаетъ насъ почитать Его не тѣлесными обрядами и церемоніями, не наружностію единою, но духомъ и истиною, страхомъ, любовію, покореніемъ воли нашея, и надеждою на Него: яко \textit{Духъ есть Богъ, и иже кланяется Ему, духомъ и истиною достоитъ кланятися}, по словеси Хрістову\footnote{Іоан.~4,~24.}. Богъ Духъ есть: духовнаго почитанія и поклоненія требуетъ. \textit{Познаніемъ истины Божіей} научаемся во всемъ Ему вѣрить, что Онъ ни глаголетъ, открываетъ, обѣщаетъ и предсказываетъ намъ; яко Богъ солгати не можетъ, яко вѣчная Истина, но непремѣнно тое, что открываетъ, такъ есть, какъ открываетъ, и что обѣщаетъ и предсказываетъ, тое непремѣнно въ свое время сбудется. \textit{Вѣренъ} бо \textit{Господь во всѣхъ словесѣхъ Своихъ}\footnote{Пс.~144,~13.}. \textit{И небо и земля мимоидетъ: словеса же Божія не мимоидутъ}\footnote{Мѳ.~24,~35.}. Отсюду послѣдуетъ страхъ суда Божія, который Богъ въ словѣ Своемъ возвѣщаетъ, и отъ того истинное покаяніе. Оттуду надежда и утѣшеніе кающемуся, яко Богъ кающихся обѣщалъ помиловать; надежда въ молитвѣ, яко молящихся обѣщалъ услышать; надежда воскресенія мертвыхъ, вѣчнаго живота и будущихъ благъ, \textit{яже уготова Богъ любящимъ Его}\footnote{1~Кор.~2,~9.}: "--- \textit{Познаніе правды Божіей}, которая всякому воздаетъ по дѣломъ и требуетъ, чтобъ грѣхъ безъ наказанія не оставленъ былъ, приводитъ въ страхъ суда Божія, въ истинное покаяніе и сокрушеніе за грѣхи, яко со пророкомъ убѣждаемся смиренно молитися Богу: \textit{не вниди въ судъ съ рабомъ Твоимъ, яко не оправдится предъ Тобою всякъ живый}\footnote{Пс.~142,~2.}. Отводитъ отъ грѣха, яко правда Божія за грѣхъ казнь наводитъ на грѣшника; и приводитъ въ познаніе грѣховъ, яко когда наказуемся, долженствуемъ правду признавать и прославлять: \textit{праведенъ еси, Господи, и прави суди Твои}\footnote{118,~137.}! И содѣлываетъ терпѣніе, яко должны тое терпѣть, что грѣхами себѣ заслужили. Богъ бо не обидитъ никого, но что дѣлаетъ, праведно дѣлаетъ; хотя"=то Богъ, яко милосердъ, не по мѣрѣ грѣховъ нашихъ наводитъ на насъ казни, но во гнѣвѣ милости и щедроты Своя поминаетъ. "--- \textit{Вкушеніе благости Божіей}, отъ которой, какъ источника живаго и приснотекущаго, изливаются на насъ неисчетныя щедроты, милости и благодѣянія, возбуждаетъ насъ къ истинной и сердечной Божіей любви. Ибо естественнымъ разума закономъ убѣждаемся благо и благодѣтеля любить, благодарить и отъ любви прославлять и почитать Его. И хотя часто бываетъ, что люди и зло любятъ; но любятъ тое не поелику зло есть, но подъ видомъ добра. Тако любятъ нечистоту, сласть, роскошь и прочія безчинныя страсти, яко тѣми плоть ихъ утѣшается и услаждается, хотя душа и бѣдствуетъ. "--- \textit{Познанію премудрости Божіей} послѣдуетъ несумнѣнная на Бога надежда: яко, хотя что нашему слѣпому уму непостижимо и противно кажется, тое Онъ въ пользу намъ обратить можетъ и обращаетъ; и гдѣ кажется намъ, что способа нѣтъ къ избавленію и спасенію нашему, тамо способъ обрѣтаетъ, яко премудрый; и что по мнѣнію нашему къ худому концу идетъ, тое Онъ по премудрости Своей къ доброму концу приводитъ. Отсюду послѣдуетъ терпѣніе во всякихъ искушеніяхъ, бѣдствіяхъ и тѣснотахъ, ожиданіе милостиваго избавленія, или помощи, якоже читаемъ о святомъ Іосифѣ, который отъ братіи своей проданъ былъ во Египетъ въ раба; но премудрымъ Божіимъ промысломъ сдѣлался господиномъ Египта, питателемъ страны тоя, отца, братіи и домашнихъ ихъ, какъ самъ братіи своей исповѣдалъ: \textit{посла мя Богъ предъ вами оставити вамъ останокъ на земли, и препитати вашъ останокъ велій}\footnote{Быт.~45,~7.}; и паки: \textit{не бойтеся! Божій бо есмь азъ. Вы совѣщаете на мя злая: Богъ же совѣща о мнѣ во благая, дабы было якоже днесь, и препиталися бы людіе мнози}\footnote{50,~19 и 20.}. Такожде Израильтяне изшедши изъ Египта и пришедши къ морю, думали, что отъ Египтянъ гонящихъ озлоблени будутъ, якоже Моисей къ нимъ глаголалъ, утѣшая ихъ: \textit{дерзайте, стойте, и зрите спасеніе, еже отъ Господа, еже сотворитъ вамъ днесь}\footnote{Исх.~14,~13.}! Но Божія премудрость изобрѣла имъ спасеніе, котораго не чаяли, и показала къ тому путь, гдѣ пути не было, такъ что отъ которыхъ надѣялися погибели, тѣхъ самихъ погибшихъ увидѣли, и гдѣ себѣ чаяли смерти, тамо спасеніе получили\footnote{гл. таже.}. "--- Отъ \textit{познанія всевѣдѣнія Божія} послѣдуетъ опасное храненіе не токмо отъ дѣлъ, словъ, но и отъ помышленій, совѣтовъ сердечныхъ, умышленій, намѣреній богопротивныхъ, лукавства, лжи и лицемѣрія. Ибо Богъ не токмо дѣла и слова, но и помышленія, намѣренія и умышленія наша, куды онѣ клонятся, ясно видитъ и записываетъ ихъ въ книзѣ Своей\footnote{Пс.~138,~1--5 и слѣд.}. Напротивъ того, будетъ тщаніе о добрыхъ дѣлахъ, словахъ, помышленіяхъ и намѣреніяхъ, о чистосердечіи, простосердечіи и истинѣ. "--- \textit{Познаніе святости Божіей} научаетъ насъ \textit{очищать себе отъ всякія скверны плоти и духа, творяще святыню въ страсѣ Божіи}\footnote{2~Кор.~7,~1.}, когда съ Нимъ общеніе хощемъ имѣти. Ибо осквернившемуся пороками грѣховными общенія съ Нимъ, яко съ самою чистѣйшею святостію, имѣти не можно: \textit{яко Богъ свѣтъ есть}, по ученію Апостола, \textit{и тьмы въ Немъ нѣсть ни единыя. Аще речемъ, яко общеніе имамы съ Нимъ, и во тьмѣ ходимъ, лжемъ и не творимъ истины}\footnote{1~Іоан.~1,~5 и 6.}. Отсюду послѣдуетъ въ грѣшникѣ истинное покаяніе, чрезъ которое приходитъ къ святѣйшему Богу, и съ Нимъ общеніе получаетъ вѣрою; въ праведникѣ большая осторожность отъ сквернъ грѣховныхъ, и поученіе тщательнѣйшее въ добродѣтельномъ хрістіанскомъ житіи. "--- Когда \textit{ощущаемъ всемогущество Божіе} въ сердцѣ нашемъ, тогда отгоняется сумнѣніе о Божественныхъ Его тайнахъ, умомъ нашимъ непостижимыхъ. Яко когда Богъ есть всемогущій, то все можетъ сдѣлать, что открылъ и обѣщалъ. Тако о воскресеніи мертвыхъ разумъ нашъ самъ въ себѣ волнуется и не вѣритъ; но ощущеніемъ всемогущества Божія въ вѣрѣ укрѣпляется и безъ сумнѣнія надѣется того, твердо увѣряяся, яко Который изъ ничего все создалъ, Той много паче разсыпанный прахъ тѣлесъ силою и словомъ Своимъ можетъ собрать и оживить; Который \textit{рече, и быша, повелѣ и создашася}\footnote{Пс.~148,~5.}, Той речетъ и все будетъ, повелитъ и все возсозиждется, что въ словѣ Своемъ обѣщалъ. Симъ утверждайся, хрістіанская душа, яко Богу все возможно. "--- Когда воображаетъ человѣкъ въ сердцѣ своемъ величество Божіе, тогда уничтожаетъ себе, яко \textit{аще вси языцы предъ Богомъ, какъ ничто}\footnote{Ис.~40,~17.}, кольми паче единъ человѣкъ, я или ты единъ. Отсюду послѣдуетъ смиреніе, послѣдуетъ, что со пророкомъ смиренно глаголати будемъ къ Нему: \textit{не намъ, Господи, не намъ, но имени Твоему даждь славу о милости Твоей и истинѣ Твоей}\footnote{Пс.~113,~9.}. Отсюду видишь, хрістіанине, что 1)~отъ истиннаго богопознанія послѣдуетъ истинное богопочитаніе и благочестіе. 2)~Чѣмъ кто болѣе Бога познаетъ, тѣмъ усерднѣе Его почитаетъ и заповѣди Его соблюдаетъ. 3)~Беззаконное житіе знаменіе есть незнанія Божія, по ученію Апостола: \textit{глаголяй, яко познахъ Его и заповѣдей Его не соблюдаетъ, ложь есть, и въ семъ истины нѣсть}\footnote{1~Іоан.~2,~4.}. 4)~Отсюду видишь, что не тотъ Бога знаетъ, кто много о Богѣ говоритъ и учитъ, а ученію несогласно живетъ; но тотъ, кто благочестиво живетъ, Бога боится, любитъ и заповѣди Его соблюдаетъ. 5)~Видишь паки, что истинная вѣра и истинное богопознаніе суть неразлучны, то есть, гдѣ истинное богопознаніе, тамо и истинная вѣра; и гдѣ истинная вѣра, тамо и истинное богопознаніе. 6)~Видишь, что хрістіане неисправные какъ Бога не знаютъ истинно, такъ и истинныя въ Бога вѣры не имѣютъ, хотя и исповѣдуютъ Бога. 7)~Отсюду послѣдуетъ, что они внѣ церкви святой находятся, хотя и въ церковь ходятъ съ вѣрными, и мнятся Бога призывать; и Хрістосъ имъ ничего не пользуетъ, пока чистосердечно не обратятся и не престанутъ отъ грѣховъ.

\paragraph*{§\:315.} Прославляется имя Божіе: 1)~Когда святое писаніе, пророками и апостолами проповѣданное, за истинное Его слово пріемлемъ, и яко даръ Его великій, намъ недостойнымъ отъ Него посланный, почитаемъ, который пророки и апостоли Его принесли отъ Него и вручили намъ; и всему, что въ Немъ благоволилъ открыть, безъ сумнѣнія вѣруемъ, яко тако исповѣдуемъ Бога истиннаго, якоже пишется о Авраамѣ: \textit{во обѣтованіи Божіи не усумнѣся невѣрованіемъ, но возможе вѣрою, давъ славу Богу}\footnote{Римл.~4,~20.}. 2)~Когда истинно вѣруемъ во имя единороднаго Сына Его Іисуса Хріста, въ святомъ Писаніи Откровеннаго, по писанному: \textit{сія писана быша, да вѣруете, яко Іисусъ есть Хрістосъ Сынъ Божій, и да, вѣрующе, животъ имате во имя Его}\footnote{Іоан.~20,~31.}. 3)~О Божественныхъ Его свойствахъ, въ святомъ Писаніи откровенныхъ, право мудрствуемъ и исповѣдуемъ, наипаче гдѣ честь и слава имени Его святаго требуетъ, какъ"=то предъ врагами Божіими, которые или не признаютъ Его, или злѣ и хулительно о Немъ мудрствуютъ. Отсюду бываетъ въ рабахъ Божіихъ горячая ревность по славѣ Божіей предъ врагами Его; отсюду самовольное страданіе и смерть бываетъ въ нихъ, которые желаютъ страдати паче и умереть, нежели имя Бога своего презираемо и хулимо слышать, что учинили святіи мученики и исповѣдники, какъ читаемъ въ церковной священной Исторіи. Отсюду то бываетъ, что \textit{вси, хотящіи благочестно жити о Хрістѣ Іисусѣ, гоними будутъ}\footnote{2~Тим.~3,~12.}; яко благочестивые какъ дѣломъ житія своего, такъ и словомъ о истинѣ свидѣтельствуютъ, чего злый міръ не любитъ, и того ради гонитъ свидѣтелей истины. 4)~Прославляется имя Божіе, когда повѣдаемъ дивная Его дѣла и чудеса, и поемъ сотворшаго ихъ. На то бо и написаны онѣ въ священныхъ книгахъ, да, на нихъ взирая, познаемъ силу и всемогущество Того, Который оная сотворилъ, и удивляяся дивнымъ дѣламъ Божіимъ, познаемъ, коль дивенъ и чуденъ, Который такъ славная дѣла творитъ. Каковыхъ чудесныхъ Божіихъ дѣлъ преисполнены книги Ветхаго Завѣта. Въ Новомъ Завѣтѣ чудо чудесъ есть, что \textit{Богъ явися во плоти}\footnote{1~Тим.~3,~16.}, на земли съ человѣками пожилъ, отъ рабовъ своихъ злыхъ за своихъ рабовъ грѣшныхъ волею пострадалъ, умеръ, воскресъ и вознесеся во славѣ. Видимъ сіе великое и умамъ нашимъ непостижимое Бога нашего дѣло; удивляемся и познаемъ оттуду, яко Богъ, который міръ создалъ, тойже милостивно и промышляетъ о немъ. Буди слава Ему во вѣки! 5)~Прославляется имя Божіе неповиннымъ рабовъ Божіихъ терпѣніемъ, когда они за истину и честь Бога своего не токмо имѣній своихъ, но и живота своего не щадятъ, тѣмъ бо доказываютъ, что Богъ, за котораго страждутъ, силенъ есть, который немощную плоть на такое страданіе и терпѣніе укрѣпляетъ; и надежда, ради которой страждутъ, не суетна, но извѣстна есть. Откуду читаемъ въ Церковной исторіи, что и самые враги ихъ терпѣнію ихъ удивлялися; и познавали силу Божію, укрѣпившую ихъ, и тако къ тѣмъ, которыхъ мучили, приставали и прославляли имя Божіе. 6)~Прославляется имя Божіе несумнѣнною на Него надеждою. Тако бо показываемъ, что Богъ есть истиненъ, Который обѣщалъ избавить; всесиленъ, Который можетъ спасти; премудръ, Который знаетъ, какъ спасти, и благъ, Который хощетъ спасти уповающихъ на Него. И тако несумнѣнною надеждою на Бога исповѣдуемъ Божію истину, всемогущество, прѣмудрость и благость. А тако отдая Богу тое, что Ему собственно есть, отдаемъ Ему и славу Его. «Самая великая Ему отъ насъ слава и честь, глаголетъ Златоустъ святый, когда мы уповаемъ на силу Его, хотя и противно кажется тому, что сими очами видимъ. И что ты дивишися тому, что оттуду Богу есть великая честь, ежели мы не усумнѣваемся? И подобные намъ человѣки въ великую то себѣ поставляютъ честь, когда мы не усумнѣваемся, но вѣруемъ ихъ обѣщаніямъ, въ которыхъ нѣчто временное и тлѣнное представляютъ намъ»\footnote{Бес.~39"~я на Быт.}. 7)~Прославляется имя Божіе искреннимъ благодареніемъ за показанныя Его и показуемыя благодѣянія. Ибо въ благодареніи признаемъ наше недостоинство и Божію милость и благость, которая и недостойнымъ благая Своя подаетъ, якоже Псаломникъ поетъ: \textit{исповѣмся Тебѣ, Господи Боже мой, всѣмъ сердцемъ моимъ, и прославлю имя Твое въ вѣкъ: яко милость Твоя велія на мнѣ, и избавилъ еси душу мою отъ ада преисподнѣйшаго}\footnote{Пс.~85,~12 и 13.}. И сіе наипаче бываетъ, когда Ему въ скорби и подъ крестомъ благодаримъ и исповѣдуемся, якоже Давидъ святый дѣлалъ: \textit{благо мнѣ, Господи, яко смирилъ мя еси}\footnote{118,~71.}. Ибо \textit{егоже любитъ Господь, наказуетъ}\footnote{Евр.~12,~6.}. 8)~Прославляется имя Божіе благочестивымъ хрістіанъ житіемъ. Тако велитъ Хрістосъ имя небеснаго Отца славить. \textit{Тако да просвѣтится свѣтъ вашъ предъ человѣки, яко да видятъ ваша добрая дѣла, и прославятъ Отца вашего, Иже есть на небесѣхъ}\footnote{Мѳ.~5,~16.}. Сіе прославленіе не гласомъ бываетъ, но подаяніемъ случая къ познанію и прославленію имени Божія. Тако \textit{небеса повѣдаютъ славу Божію}\footnote{Пс.~13,~2.}. Тако \textit{хвалитъ Его солнце, луна, звѣзды, свѣтъ, небеса небесъ} и все созданіе\footnote{148,~3,~4 и слѣд.}. Ибо взирая на дивное Божіе созданіе, и отъ того несказанную пріемля пользу, убѣждаемся прославлять Создателя и удивляться всемогуществу, премудрости и благости Его, Который такъ дивная дѣла ради насъ, разумной твари, сотворилъ. Тако, взирая люди на благочестивое и святое хрістіанъ житіе, не могутъ не прославлять Господа ихъ, Который такихъ рабовъ у Себе имѣетъ, что хотя общаго естества и немощи съ прочими суть, но отмѣнные имѣютъ нравы, и, яко свѣтъ отъ тьмы, день отъ нощи, злато отъ блата, разнствуютъ отъ нихъ. Тако бываетъ отцу слава, когда дѣти его, и господину честь, когда раби его постоянно живутъ. Не могутъ бо люди отца за постоянство дѣтей, и господина за постоянство рабовъ не хвалить. Знать"=де отецъ добрый, который тако дѣтей содержитъ, и господинъ разумный, который такъ рабовъ своихъ учитъ и наставляетъ! Тако незнающіи истиннаго Бога въ познаніе приходятъ, видя честное и постоянное хрістіанъ житіе; и вѣруютъ въ того Бога, Который такъ постоянныхъ у Себе рабовъ имѣетъ; и \textit{прославляютъ} сами имя \textit{Отца нашего, Иже есть на небесѣхъ}. Откуду читаемъ въ Церковной исторіи, что многіи язычники приходили въ познаніе истиннаго Бога, и пріимали хрістіанскую вѣру, видя хрістіанъ добронравіе и благочестивое житіе. 9)~Наконецъ прославляется имя Божіе, когда не нашему, но Его имени ищемъ отъ добрыхъ дѣлъ славы и похвалы, и все, что доброе дѣлаемъ, Ему восписуемъ, яко всякаго добра Виновнику и Началу, да Тому \textit{единому} будетъ честь и слава за добро, отъ Котораго всякое добро происходитъ. Святый Василій великій глаголетъ: «тогда во славу Божію все дѣлаемъ, когда все ради Бога, по заповѣди Его, дѣлаемъ, ни въ чемъ похвалы человѣческой не ищемъ». И паки: «тотъ"=де во славу Божію ястъ и піетъ, кто въ пищи и питіи поминаетъ Божія благая, и ястъ не яко рабъ чрева, ради сласти, но яко дѣлатель Божій, да въ твореніи дѣлъ Его крѣпчайшій можетъ быти»\footnote{Въ вопросѣ 195"~мъ краткихъ правилъ.}.

\paragraph*{§\:316.} \textit{Хулится имя Божіе} святое и страшное различнымъ образомъ: 1)~Когда слову Его святому и въ немъ открытымъ тайнамъ и обѣщаніямъ не вѣритъ человѣкъ. Ибо хотя и не называетъ Его невѣрующій ложнымъ, но самымъ дѣломъ тое показуетъ, якоже глаголетъ Апостолъ: \textit{не вѣруяй Богови, лжа сотворилъ есть Его}\footnote{1~Іоан.~5,~10.}. Кто бо не вѣруетъ слову Божію, тотъ самому Богу не вѣруетъ, и такъ страшно хулитъ Его. Сюды принадлежатъ непризнающіи воскресенія мертвыхъ и прочіихъ таинъ, которыя слово Божіе открываетъ. 2)~Хулится имя Божіе, когда человѣкъ явно отрыгаетъ на Него хулу, называетъ Его или неправеднымъ, или немилостивымъ, или присносущнаго Его бытія, или промысла о созданіи своемъ не признаетъ, или прочія подобныя симъ хульныя слова на Него произноситъ. А хотя кто хульнымъ словомъ и не касается имени Его, но въ сердцѣ своемъ содержитъ мысли, Божественнымъ Его свойствамъ противныя, такожде имя Божіе хулитъ. Ибо Богъ глубину сердца человѣческаго видитъ, и сердечнымъ, равно какъ устнымъ хуленіемъ величество Его безчестится. 3)~Страшно имя Божіе хулится, когда дѣло Его Божественное приписуется противнику Его діаволу, каковые были хулители книжники и фарисеи во время Хрістова на земли пожитія, "--- или иной какой твари, какъ тое учинили оные Израильтяне, которые великое избавленія своего отъ Египта дѣло приписали бездушному тельцу: \textit{сіи бози твои Израилю, иже изведоша тя изъ земли Египетскія}\footnote{Исх.~32,~4.}. 4)~Хулится имя Божіе, когда человѣкъ тую славу и честь, которую Богу долженъ воздавать, отъ Него отъемлетъ и себѣ приписуетъ. Тако бо поставляетъ себе, какъ идола, на томъ мѣстѣ на которомъ Бога долженъ поставлять и почитать, и такъ отъ Бога сердцемъ отступаетъ и въ духовное впадаетъ идолопоклонство. Таковые суть, которые или надѣются на свою силу, премудрость, разумъ, искусство, хитрость, или доброе дѣло сотворенное себѣ приписуютъ, или отъ добрыхъ своихъ дѣлъ ищутъ себѣ похвалы и славы. Извѣстно бо, что безъ Божіей помощи сила, разумъ и искусство наше ничего не можетъ; такожде и добра безъ Бога не можемъ дѣлати, зло творить и грѣшить сами можемъ и грѣшимъ, но добро творить безъ Него не можемъ\footnote{Іоан.~15,~5.}. 5)~Хулится имя Божіе ложною присягою, когда люди не тое въ сердцѣ имѣютъ, что словомъ объявляютъ; кленутся именемъ Божіимъ, Бога въ свидѣтеля совѣсти своей призываютъ и обѣщаются правду хранить, на честь восходя, но въ сердцѣ замышляютъ, какъ бы, на чести будучи, обогатиться мздоиманіемъ и неправдою. Такожде свидѣтельствомъ ложнымъ хулится имя Божіе, когда свидѣтели кленутся именемъ Божіимъ, что истину объявляютъ, но въ самой вещи лгутъ. И прочіи, которые ни кленутся именемъ Божіимъ во лжи, имя Божіе хулятъ. Ибо вси таковые помышляютъ въ сердцахъ своихъ, что или нѣтъ Бога, или есть, но не знаетъ таинъ человѣческихъ, или неправосудливъ, или иное что злое и хульное о Немъ думаютъ. 6)~Безчестится имя Божіе, когда не въ нужныхъ и важныхъ случаяхъ, но въ подлыхъ и безъ всякой правильной причины поминается и пріемлется, какъ то у людей въ злые обычаи вошло, при всякомъ почти словѣ поминать тое: \textit{ей Богу! на то Богъ! на то Хрістосъ! свидѣтель Богъ! Богъ видитъ!} и проч. Хотя бо таковые и не примѣчаютъ, по застарѣлому обычаю, какъ они безчестятъ имя Божіе, яко обычай ослѣпляетъ разумъ; но когда осмотрятся, увидятъ и признаютъ за истину сіе. Ибо чѣмъ имени Божію отъемлется честь, тѣмъ оно безчестится. Но всякъ можетъ видѣть, что таковымъ безстрашнымъ и непочтительнымъ поминаніемъ отъемлется Ему честь достойная. Имя бо Божіе, яко великое, святое и страшное, должно со страхомъ и почтеніемъ поминать. Но когда человѣкъ того Ему не показуетъ, тѣмъ самымъ отъемлетъ Ему достойную честь и такъ безчеститъ. И сія есть сатанинская хитрость, который ищетъ, чрезъ посредствіе человѣка, Бога безчестить и самаго человѣка погубить. Монарха земнаго имя люди съ почтеніемъ поминаютъ; но имя Божіе, какъ подлую вещь (что и писать страшно, но нужно, ради злонравія и застарѣлаго злаго обычая безстрашныхъ хрістіанъ, чтобы отъ того обычая отстали), подлую, глаголю, и легкую вещь во всякихъ подлѣйшихъ словахъ, и, что горше того, въ шуточныхъ приводить не устрашаются!.. О долготерпѣнія Твоего Боже, Который все сіе видишь и слышишь, и терпишь безумію человѣческому! 7)~Хулится имя Божіе роптаніемъ противу Бога, что бываетъ въ несчастіи и бѣдахъ, какъ"=то обычай есть у таковыхъ роптать: \textit{что"=де я сдѣлалъ? что я согрѣшилъ? неужели я грѣшнѣйшій отъ прочіихъ?} и прочія безумныя рѣчи. Ибо какъ все, такъ бѣды и напасти не безъ промысла Божія приключаются; и потому кто въ бѣдахъ ропщетъ противу Бога, хотя и не называетъ Его неправеднымъ, но самымъ дѣломъ и негодованіемъ своимъ показуетъ тое, что аки неправедно посылаются на него отъ Бога. Не такъ вѣрное и смиренное сердце: оно въ напастяхъ и бѣдахъ правду Божію прославляетъ: \textit{праведенъ еси Господи, и прави суди Твои!} Оно смиряемо, Богу покланяется и благодаритъ: \textit{благо мнѣ, яко смирилъ мя еси}. 8)~Безчестится и хулится имя Божіе преступленіемъ святаго закона Его и житіемъ беззаконнымъ, якоже Апостолъ глаголетъ: \textit{преступленіемъ закона Бога безчествуеши}, "--- и изъ пророка приводитъ: \textit{имя Божіе вами хулится во языцѣхъ}\footnote{Римл.~2,~23 и 24; Ис.~52,~5.}. Сіе отъ Апостола глаголется о Іудеяхъ, которые законъ Божій, врученный имъ, содержали и имъ хвалилися, но законъ Божій преступали и разоряли. Разумѣется оно и о хрістіанахъ, которые вмѣсто Іудеевъ, Хріста непріявшихъ, вступили, людьми Божіими учинилися, и законъ Божій, который Іудеи содержали, взяли, и хранить его, вступая въ хрістіанское общество, обѣщалися; но самымъ дѣломъ исполнять его не хотятъ и не тщатся, но паче безстрашно его нарушаютъ. Сего ради, какъ Іудеями тогда, такъ и нынѣ хрістіанами беззаконными великое имя Божіе хулится во языцѣхъ. Якоже бо добродѣтельное хрістіанъ житіе къ познанію и прославленію имени Божія, какъ выше сказано, тако беззаконное къ отвращенію и хуленію того людямъ подаетъ случай. Смотря бо язычники на беззаконное хрістіанъ житіе, и примѣчая свое лучшее и сходнѣйшее разуму, нежели ихъ (ибо нѣтъ такихъ беззаконій, хищеній, лихоиманія, грабленія, насилія, лжи, лукавства, хитрости, насмѣшекъ язвительныхъ, нечистоты, піянства и прочаго безчинія въ честныхъ язычникахъ, которые естественнымъ только закономъ управляются, какъ въ хрістіанахъ, именемъ хрістіанскимъ, какъ волкахъ овчею кожею покрытыхъ и закономъ Божіимъ написаннымъ хвалящихся), "--- на сія, глаголю, смотря, отвращаются отъ хрістіанской вѣры, и Бога, Творца хрістіанскаго закона, оставляютъ, и тако въ нечестіи своемъ утверждаются и вѣчно погибаютъ. Горе таковымъ хрістіанамъ! судъ Божій жесточайшій имъ слѣдуетъ, нежели язычникамъ. Не поможетъ имъ ничего имени Хрістова нарицаніе: \textit{Господи, Господи!} услышатъ отъ Хріста: \textit{не вѣмъ васъ}\footnote{Мѳ.~7,~21--23; Лук.~13,~25--28; Мѳ.~25,~11 и 12.}, какъ себе ни ласкаютъ и утѣшаютъ злую свою совѣсть. 9)~Богъ вездѣ есть, и нѣтъ такого мѣста, гдѣ бы Богъ не присутствовалъ. Оттуду слѣдуетъ, что все дѣлаемъ предъ Нимъ, что ни дѣлаемъ. Отсюду разсуди, хрістіанине, какое безчестіе показуетъ Ему человѣкъ, когда беззаконнуетъ. Ежели бы какій безчинникъ предъ монархомъ земнымъ дерзнулъ безчинствовать и законы его нарушать, "--- не великое ли бы безчестіе и стыдъ сдѣлалъ монарху? Истина сія безспорна, всякъ сіе признаетъ. Тако несравненно большее безчестіе показуетъ Богу человѣкъ, когда предъ Нимъ законъ Его святый безстрашно нарушать дерзаетъ, и тако предъ святѣйшимъ Его присутствіемъ безчинствовать, не бояся правды Его, не почитая свѣта чистѣйшія святости Его, не стыдяся безконечнаго величества Его. Съ блудницею ли обнимается и беззаконнуетъ, "--- Богъ смотритъ на тое, хулитъ ли Его, или безстрашно въ подлыхъ вещахъ и въ шуткахъ, или въ ложной присягѣ, или въ иной какой лжи пріемлетъ, и въ подтвержденіе лжи своея, аки истины, приводитъ; грабитъ ли и похищаетъ чужое, обманываетъ и прельщаетъ ближняго своего, біетъ или злословитъ, укоряетъ и осуждаетъ его, "--- дѣлаетъ тое предъ Богомъ, Который и на обидящаго и на обидимаго смотритъ. Сквернословитъ ли и кощунствуетъ, "--- слышатъ святѣйшія уши Его. Словомъ, какое ни дѣлаетъ беззаконіе словомъ, дѣломъ и помышленіемъ, предъ безконечнымъ величествомъ Его и святынею дѣлаетъ. Отсюду можеши видѣть, какое Ему непочтеніе, а тако и безчестіе, показуетъ безстрашный беззаконникъ. "--- Кто? Человѣкъ, Его подвластный, Его рабъ, Его созданіе, \textit{земля и пепелъ}! Человѣкъ, о которомъ Онъ промышляетъ, котораго питаетъ, напаяетъ, одѣваетъ, сохраняетъ, ради котораго Сына Своего послалъ въ міръ, да спасется! "--- Кому дѣлаетъ такое непочтеніе? Безначальному, безконечному, небо и землю горстію содержащему, непостижимому, великому и страшному Богу, Создателю и высочайшему Благодѣтелю своему, Котораго долженъ бояться, почитать и покланяться! "--- Отъ вышеписанныхъ видиши, хрістіанине: 1)~коль великая діавольская злоба и хитрость, который всегда тщится Бога похулить или языкомъ, или дѣломъ, или помышленіемъ человѣческимъ, и тако Его прогнѣвать, а человѣка отъ Бога отвести и погубити! Тако онъ поступилъ съ прародителями нашими, которыхъ злымъ своимъ совѣтомъ въ хуленіе и тяжкій грѣхъ вринулъ. Богъ имъ объявилъ: \textit{смертію умрете, въ оньже день снѣсте отъ заповѣданнаго древа}\footnote{Быт.~2,~17.}; но онъ напротивъ говорилъ: \textit{не умрете, но будете яко бози}\footnote{3,~4 и 5.}. Они, послушавши лжи, не повѣрили Богу, аки бы неистинно слово Его было. \textit{Всякъ бо, не вѣруяй Богови, лжа Его творитъ}\footnote{1~Іоан.~5,~10.}. И тако тяжко Его похулили. Тоже дѣлается и нынѣ, когда люди слову Божію, истинѣ Его и заповѣдямъ Его святымъ не вѣруютъ, и тако безчестятъ великое имя Божіе. 2)~Коль тяжкая слѣпота и неблагодарность человѣческая, яко Бога, Котораго, яко Создателя, Отца, Господа, Благодѣтеля и Промыслителя своего, долженъ слушать, почитать, славить и благодарно покланятися, Того не слушаетъ, не почитаетъ, и тако къ Нему неблагодаренъ является! Чего Богъ ради человѣка не сдѣлалъ? Небо солнцемъ, луною и звѣздами украсилъ ради его; воздухъ ради его проліялъ; облака на немъ повѣсилъ, которыя, яко мѣхи, воду всюду разносятъ и проливаютъ на землю; землю различными плодами наполнилъ; моря, озера, рѣки и источники проліялъ и различные роды рыбъ въ нихъ умножилъ. Слово свое святое, яко благопріятное посланіе, къ нему чрезъ пророковъ и апостоловъ послалъ, и въ немъ волю Свою и благословеніе къ нему открылъ. Сына Своего единороднаго въ міръ послалъ, да спасется человѣкъ. Но человѣкъ ослѣпленный не чувствуетъ того. Такого Благодѣтеля своего не хочетъ почитать; но или хульнымъ языкомъ касается страшной славы Его, или невѣріемъ сердца своего хулитъ Его, или презрѣніемъ и преступленіемъ святаго Его и вѣчнаго закона безчеститъ Его, и тако Бога почетшаго его, глаголетъ Златоустъ, безчествуетъ\footnote{Бес.~6"~я на послан. къ Римл. на оное слово апостольское: \textit{преступленіемъ закона Бога безчествуеши}.}. 3)~Видишь отсюду, коль великое Божіе есть долготерпѣніе, Который грѣшнику, хульнику и безчествователю Своему терпитъ, и не тотчасъ казнитъ его. Который монархъ земный такъ терпѣливъ и кротокъ есть, который бы рабу своему, подобному человѣку, предъ собою безчинствующему и его безчестящему, стерпѣлъ? Не слышно того нигдѣ. Скоро человѣческая кротость премѣняется и распаляется. Богъ нашъ не тако. Видитъ и терпитъ и ждетъ покаянія грѣшнича. Отсюду видишь, что \textit{никтоже благъ, токмо единъ Богъ}, по ученію Хрістову\footnote{Мѳ.~19,~17.}. 4)~Отъ сего можемъ видѣть, что на грѣшниковъ неблагодарныхъ и нераскаянныхъ, которые долготерпѣнія Божія не хотятъ себѣ въ покаяніе и спасеніе принять и употребить, гнѣвъ Божій возгорится; яко чѣмъ болѣе человѣкъ грѣшитъ и о долготерпѣніи Божіи нерадитъ, тѣмъ болѣе собираетъ себѣ гнѣва Божія, по апостольскому слову: \textit{о человѣче! или о богатствѣ благости и кротости и долготерпѣніи Его нерадиши, не вѣдый, яко благость Божія на покаяніе тя ведетъ? По жестокости же твоей и непокаянному сердцу, собираеши себѣ гнѣвъ въ день гнѣва и откровенія праведнаго суда Божія}\footnote{Римл.~2,~4 и 5.}. Хотя бо Богъ благъ есть и милостивъ, но есть и праведенъ, Котораго правда требуетъ, чтобы грѣшникъ нераскаянный, яко неблагодарный и вѣчнаго Божія закона и Самаго Бога презиратель, вѣчно казненъ былъ. Сіе неотмѣнно будетъ нераскаяннымъ беззаконникамъ, хотя они что ни вымышляютъ къ умягченію и утѣшенію злыя и грызущія своея совѣсти. \textit{Богъ бо поругаемъ не бываетъ}\footnote{Гал.~6,~7.}. Благость Божія и долготерпѣніе въ семъ показуются что Онъ грѣшнику, святый Его законъ нарушающему, терпитъ, не тотчасъ казнитъ его, но ожидаетъ его на покаяніе, или, какъ Апостолъ учитъ, \textit{ведетъ его на покаяніе}. Но когда грѣшникъ \textit{сіе благости Божіей богатство} презритъ и пребудетъ въ своемъ нераскаяніи и ожесточеніи, тогда правда Божія въ свое дѣло вступитъ, и \textit{воздастъ ему по дѣломъ его}\footnote{Римл.~2,~6.}. 5)~Мы, слава Богу, еще не погибли; еще благость Божія терпитъ намъ, еще насъ \textit{на покаяніе ведетъ}; еще можемъ покаятися и спастися. Нужъ, обратимся къ Богу и покаемся, да спасемся. Потщимся впредь отъ грѣховъ берещися, яко тѣми великое Божіе имя безчестится; и потерпѣвшему грѣхи наша Богу отъ сердца возблагодаримъ, и самихъ себе постыдимся, что предъ Богомъ такъ безстыдно поступали, и тако подобающія Ему чести и славы не отдавали. Попечемся учинить сіе, пока благость Божія терпитъ намъ и \textit{на покаяніе ведетъ} насъ, да не, вознерадивше о благости Божіей, правды Его дѣло на себѣ узнаемъ. Не тщетно бо слово Божіе, но, какъ Самъ Богъ, такъ и слово Его истинно, и что возвѣщаетъ, тое неотмѣнно будетъ и исполнится. Исполнится тое, что сказано: \textit{воздастъ комуждо по дѣломъ его}.

\subsection[Глава 3-я. О жертвѣ хрістіанской.]{глава третія.\\\bfseries О жертвѣ хрістіанской.}

\begin{quotation}\textit{Пожрите жертву правды, и уповайте на Господа}\footnote{Пс.~4,~6.}.\end{quotation}
\begin{quotation}\textit{Пожри Богови жертву хвалы, и воздаждь Вышнему молитвы твоя}\footnote{48,~14.}.\end{quotation}
\begin{quotation}\textit{Жертва Богу духъ сокрушенъ: сердце сокрушенно и смиренно Богъ не уничижитъ}\footnote{Пс.~50,~19.}.\end{quotation}
\begin{quotation}\textit{Молю васъ, братіе, щедротами Божіими, представите тѣлеса ваша жертву живу, святу, благоугодну Богови, словесное служеніе ваше}\footnote{Римл.~12,~1.}.\end{quotation}

\paragraph*{§\:317.} Іудеи приносили въ жертву Богу различныхъ животныхъ, какъ о томъ пишется въ книгахъ Моисеовыхъ и пророческихъ. И сія ихъ жертва прознаменовала смерть Хріста Сына Божія, Который, яко Агнецъ непорочный, имѣлъ заклатися на древѣ крестномъ за спасеніе міра. Но когда сія святѣйшая жертва принеслася, оная, яко сѣнь, прообразовавшая истину, престала. Въ новой благодати духовную жертву, безъ которой и Іудеямъ оная жертва неполезна была, хрістіане вѣрою приносить Богу должны.

\paragraph*{§\:318.} \textit{Хрістіанская жертва} изъ святаго Писанія примѣчается сія: 1)~Духъ сокрушенъ и сердце сокрушенно, якоже глаголетъ Давидъ святый: \textit{жертва Богу духъ сокрушенъ: сердце сокрушенно и смиренно Богъ не уничижитъ}. Сіе бываетъ, когда мы за грѣхи печалію сокрушаемся, печаль въ сердцѣ имѣемъ, которая есть \textit{печаль по Бозѣ и покаяніе нераскаянно во спасеніе содѣловаетъ}\footnote{2~Кор.~7,~10.}. Чего отъ насъ Самъ Богъ требуетъ: \textit{расторгните сердца ваша, а не ризы ваша, и обратитеся ко Господу Богу вашему}\footnote{Іоил.~2,~13.}. Таковый духъ и сердце есть жертва Богу пріятна. Тако сокрушенное сердце Богъ не уничижитъ. На такую жертву съ высоты Своея призираетъ Богъ и на приносящаго милосердыми Своими отеческими очами зритъ. Такую жертву принесъ Петръ Апостолъ, когда отвергся Хріста, \textit{и изшедъ вонъ плакася горько}\footnote{Мѳ.~20,~75.}. Сію жертву и мы, хрістіане, на жертвенникѣ сердца нашего приносить должны, да и на насъ призритъ милосердіемъ Господь. 2)~Жертва хваленія и благодаренія, когда мы отъ сердца благодаримъ Ему за показанныя и показуемыя отъ Него намъ неисчетныя благодѣянія. Онъ насъ создалъ, и разумною душею и образомъ Своимъ почтилъ, да славимъ Его. Онъ насъ падшихъ возставилъ, и заблуждшихъ взыскалъ и погибшихъ искупилъ \textit{не сребромъ или златомъ, но честною кровію Сына Своего}\footnote{1~Петр.~1,~18 и 19.}, да усердно и отъ любви благодаримъ Ему. Онъ, яко Отецъ чадолюбивый, милостивно промышляетъ о насъ, питаетъ, напаяетъ, одѣваетъ, отъ козней вражіихъ сохраняетъ насъ, Духомъ Своимъ Святымъ просвѣщаетъ и оживляетъ насъ, движетъ насъ къ желанію и исканію небесныхъ Его благъ, поощряетъ къ истинному покаянію, и кающимся отпущаетъ грѣхи, и вѣрующимъ отверзаетъ двери небеснаго царствія, и прочая благая туне подаетъ намъ, да \textit{приносимъ жертву хваленія выну Богу, сирѣчь плодъ устенъ исповѣдающихся имени Его}\footnote{Евр.~13,~15.}. Таковыя жертвы требуетъ отъ насъ Богъ: \textit{пожри Богови жертву хвалы}. Такая жертва Ему благоугодна, якоже глаголетъ пророкъ: \textit{восхвалю имя Бога моего съ пѣснію, возвеличу Его во хваленіи; и угодно будетъ Богу паче тельца юна, роги износяща и пазнокти}\footnote{Пс.~68,~31.}. Сію жертву должно хрістіанамъ приносить въ вечеръ, утро и полудне, по примѣру Псаломника: \textit{въ вечеръ и заутра, и полудне повѣмъ и возвѣщу}\footnote{54,~18.}, "--- паче же на всякое время благословить Господа, якоже тойже пророкъ Божій о себѣ глаголетъ: \textit{благословлю Господа на всякое время, выну хвала Его во устѣхъ моихъ}\footnote{Пс.~33,~2.}. Аще убо кто хощетъ Бога хвалить въ вѣчной жизни, той да не оставляетъ Его здѣ хвалити и славити. 3)~Жертва милости и милосердія. Якоже бо Самъ Господь есть щедръ, долготерпѣливъ и многомилостивъ; тако за пріятное приношеніе Себѣ вмѣняетъ, когда мы братіи нашей милость являемъ, страждущимъ состраждемъ, бѣдствующимъ подаемъ руку помощи. \textit{Таковыми бо жертвами благоугождается Богъ}\footnote{Евр.~13,~16.}. Откуду Самъ Господь глаголетъ: \textit{милости хощу, а не жертвы}\footnote{Ос.~6,~6.}. Къ чему Хрістосъ Іудеевъ отсылаетъ: \textit{шедше научитеся, что есть: милости хощу, а не жертвы}\footnote{Мѳ.~9,~13.}. Сіе жертвоприношеніе предъ всѣмъ міромъ превознесетъ Хрістосъ, и къ приносившимъ любезно возглаголетъ: \textit{взалкахся, и дасте Ми ясти, возжадахся, и напоисте Мя; нагъ, и одѣясте Мя; страненъ бѣхъ, и введосте Мене}, и проч.\footnote{25,~35 и 36.} 4)~Жертва правды и истины, якоже глаголетъ Соломонъ: \textit{творити праведная, и истинствовати угодна Богу паче, нежели жертвъ кровь}\footnote{Притч.~21,~3.}. Богъ бо есть \textit{праведенъ и правду возлюби}\footnote{Пс.~10,~7.}, или паче самая Правда и Истина есть, сего ради правду и истину любитъ. Яко правдою вѣчную Его правду, истиною вѣчную истину почитаемъ, когда дѣлаемъ тое, чего правда Его хощетъ, и хранимся отъ лжи, льсти, лукавства и лицемѣрія, отъ чего вѣчная Его отвращается истина. Сего ради пророкъ увѣщаваетъ насъ: \textit{пожрите жертву правды}\footnote{4,~6.}. И Апостолъ глаголетъ: \textit{глаголите истину кійждо ко искреннему своему}\footnote{Еф.~4,~25.}, \textit{яко истины взыскуетъ Господь}\footnote{Пс.~30,~24.}. 5)~Жертва удовъ или членовъ плоти нашея, которая возстаетъ на духъ нашъ, и тѣлесными членами, какъ орудіями, боретъ насъ, и вѣру нашу умертвить хощетъ. Сія жертва тогда приносится отъ хрістіанъ, когда они \textit{умерщвляютъ уды своя, сущія на земли, блудъ, нечистоту, страсть, похоть злую и лихоиманіе, еже есть идолослуженіе}, по увѣщанію Апостола\footnote{Кол.~3,~5.}; очищаютъ сердце свое и обрѣзуютъ обрѣзаніемъ нерукотвореннымъ; умъ очищаютъ отъ помысловъ суетныхъ, злыхъ, гордыхъ начинаній и намѣреній богопротивныхъ; волю отъ злаго похотѣнія отвращаютъ, и волѣ Божіей покоряютъ; изъ памяти злобу и всякую непотребность изгоняютъ; \textit{отвращаютъ очи свои, еже не видѣти суеты}\footnote{Пс.~118,~37.}; отвращаютъ уши отъ клеветы, злословія, пѣсней и словъ соблазнительныхъ; удерживаютъ языкъ отъ злословія, осужденія, клеветы, проклинанія, хулы, сквернословія, празднословія и прочіихъ золъ; отвращаютъ руки отъ хищенія, лихоиманія, грабленія, біенія, нечистаго осязанія, удерживаютъ чрево отъ объяденія и піянства; во всѣхъ удахъ, какъ безсловесныхъ животныхъ, безчинныя страсти: гордость, высокоуміе, гнѣвъ, ярость, зависть, объяденіе, піянство, нечистоту и прочая закалаютъ. Тако умерщвляя уды неправды и въ удесѣхъ грѣхъ, \textit{да представляютъ себе Богови яко отъ мертвыхъ живыхъ, и уды своя орудія правды Богови}, какъ учитъ Апостолъ\footnote{Римл.~6,~13.}; умъ къ размышленію о чудесныхъ Божіихъ дѣлахъ, къ разширенію славы Его, своея и ближняго пользы; волю къ угожденію воли Божіей; сердце къ любви Божіей и любви ближняго; языкъ къ славословію и прославленію имени Божія, къ созиданію ближняго; очи къ зрѣнію дивнаго Божія созданія, и въ созданіи Создателя; уши къ слышанію слова Божія, славы и похвалы Божіей, къ плачу и прошенію бѣдныхъ, просящихъ милости и защищенія; руки къ подаянію милостыни и помоществованію бѣдствующимъ, къ воздѣянію въ молитвѣ къ Богу; ноги къ хожденію во храмы Божія на славословіе и молитву, и прочая. Тако, когда умерщвляютъ уды плоти своея, \textit{оружія правды Богови}, приносятъ жертву живу, святу, благоугодну Богови, словесное служѣніе свое, якоже учитъ Апостолъ: \textit{молю васъ, братіе, щедротами Божіими, представите тѣлеса ваша жертву живу, святу, благоугодну Богови, словесное служеніе ваше}\footnote{Римл.~12,~1.}. «Егда, глаголетъ Златоустъ святый, убиваешь ветхаго человѣка, егда умертвишь уды, яже на земли, егда міръ распнешь себѣ, и проч., священникъ своему тѣлу бываешь, и душевныя добродѣтели, яко се, егда цѣломудріе приносиши, егда милостыню, егда кротость, егда беззлобіе: сія вся творя, приносиши словесную службу, сирѣчь ничесоже имущую плотское, ничтоже дебѣлое, ничтоже чувственное»\footnote{Бесѣд.~20"~я на посл. къ Римл.}. \textit{Жертвенникъ}, на которомъ благоугодную Богу жертву сію приносятъ хрістіане, есть сердце, вѣрою во Хріста очищенное отъ сквернъ міра сего. \textit{Храмъ} есть составъ души и тѣла, въ которомъ жертву сію заколаютъ и возносятъ. \textit{Ножь} есть мечь духовный, сирѣчь, глаголъ Божій, которымъ закалаютъ духовную сію жертву. \textit{Огнь}, которымъ сожигаютъ оную, есть любовь Божія. \textit{Священники} сами суть, по писанному: \textit{сотворилъ есть насъ цари и іереи Богу и Отцу Своему} Іисусъ Хрістосъ\footnote{Апок.~1,~6.}. Удивленія достойная вещь! Іудеи когда приносили жертву, потребенъ былъ храмъ, потребенъ жертвенникъ, потребна жертва, потребенъ огнь, потребны были іереи, и проч. У хрістіанъ не тако. Сами іереи, сами храмъ, сами жертвенникъ, сами съ душами и тѣлесами жертва, внутрь огнь носятъ и мечь глагола Божія: \textit{сами яко каменіе живо зиждутся въ храмъ духовенъ, святительство свято, возносити жертвы духовны, благопріятны Богови Іисусъ Хрістомъ}\footnote{1~Петр.~2,~3.}. Къ приношенію жертвы сея увѣщаваетъ и молитъ насъ щедротами Божіими Апостолъ святый, якоже видѣли. Принесемъ убо и мы, хрістіанине, жертву сію, и въ прочіихъ пунктахъ прописанную, да не явимся тощи и праздны предъ Господемъ Богомъ нашимъ, якоже пророкъ глаголетъ: \textit{да не явишися предъ Господемъ Богомъ твоимъ тощъ}\footnote{Второз.~16,~16.}. Смиримъ и сокрушимъ сердца наша предъ Нимъ, да и на насъ призритъ милосердыми очами Своими; покажемъ Ему благодареніе сердцемъ и устами за безчисленныя Его къ намъ милости; сотворимъ милость братіи нашей, якоже отъ Него сподобляемся милости, яко \textit{таковою жертвою благоугождается Богъ}\footnote{Евр.~13,~16.}. Сотворимъ правду, и истину возлюбимъ, якоже сего требуетъ отъ насъ Богъ. Умертвимъ уды наши, сущія на земли, и \textit{плоть распнемъ со страстьми и похотьми, аще хощемъ Хрістовы быти}\footnote{Гал.~5,~24.}. Ибо всего сего вѣра наша отъ насъ требуетъ. Покажемъ убо вѣры нашея плоды, да не безъ плодовъ и вѣра наша суетна будетъ.

\paragraph*{§\:319.} Хрістіане"=де, которые безстрашно живутъ, истиннаго покаянія и плодовъ его не творятъ, приносятъ ли Богу жертву? Никакъ. Видѣлъ ты, въ чемъ хрістіанская жертва состоитъ. Они того не тщатся дѣлать, слѣдственно и жертвы никакой не приносятъ Богу. "--- Многіи"=де отъ нихъ храмы Божіи созидаютъ, Евангеліе окладываютъ сребромъ и златомъ, ризы шьютъ въ церковь, свѣщи и ѳиміамъ приносятъ? Ничего имъ сія не пользуютъ, понеже сами неисправны и нераскаянны пребываютъ. Богъ бо, яко Духъ\footnote{Іоан.~4,~24.}, ничѣмъ не благоугождается, какъ только духовною жертвою, то"=есть, вѣрою, доброхотною волею, покореніемъ воли своея волѣ Его и прочими вѣры плодами. А они хотя и творятъ вышеписанныя дѣла, но не хотятъ принести воли своея Богу и послушанія, безъ котораго ничто Богу неугодно. Ктомужъ многіи отъ нихъ храмы Божіи каменные или деревянные созидаютъ, но одушевленные оставляютъ, или, что горше того, разоряютъ. Евангеліе украшаютъ, но чему Евангеліе учитъ, и перстомъ коснутися не хотятъ, и проч. Многіи ризы шьютъ, и прочее дѣлаютъ, но нагихъ не одѣваютъ, и, что горше, другихъ обнажаютъ, и въ прочемъ неправду дѣлаютъ. "--- Многіи"=де ходятъ въ церковь на молитву, славословіе, и поютъ Бога и благодарятъ? И то имъ не пользуетъ. Ибо устами поютъ и благодарятъ Бога, но сердцемъ отстоятъ отъ Него, якоже о нихъ глаголетъ Богъ: \textit{приближаются Мнѣ людіе сіи усты своими, и устнами чтутъ Мя: сердце же ихъ далече отстоитъ отъ Мене}\footnote{Мѳ.~15,~8.}, "--- и житіемъ беззаконнымъ безчестятъ и хулятъ имя Божіе. "--- Въ опасномъ"=де они состояніи находятся? Подлинно, въ опасномъ. Ибо съ идолопоклонниками, съ которыми въ заблужденіи находятся и Бога единаго не почитаютъ, но грѣху и въ грѣхѣ бѣсу волю и душу свою пожираютъ, вѣчному осужденію подлежатъ, когда чистосердечно не обратятся и не покаются. Они"=де вѣруютъ въ Бога? Никакъ. Вѣра бо истинная благоплодна, и отъ дѣлъ злыхъ удаляется. А они \textit{исповѣдуютъ Бога}, но \textit{сердцемъ и дѣлами отмещутся Его}, якоже учитъ Апостолъ\footnote{Тит.~1,~14.}. О сихъ изъ слѣдующихъ лучше познаешь. Однакожъ ты на нихъ и вѣру и дѣла ихъ не смотри; но внимай, чему учитъ Божіе слово, и толкователи его отцы святіи и учители церковные.

\subsection[Глава 4-я. О подражаніи Богу.]{глава четвертая.\\\bfseries О подражаніи Богу.}

\begin{quotation}\textit{Бывайте подражатели Богу, якоже чада возлюбленная}\footnote{Еф.~5,~1.}.\end{quotation}
\begin{quotation}\textit{По звавшему васъ Святому, и сами святи во всемъ житіи будите. Зане писано есть: святи будите, яко Азъ святъ есмь}\footnote{1~Петр.~1,~15; Лев.~11,~44; 19,~2.}.\end{quotation}
\begin{quotation}\textit{Возлюбленніи, аще сице возлюбилъ есть насъ Богъ, и мы должны есмы другъ друга любити}\footnote{1~Іоан.~4,~11.}.\end{quotation}

\paragraph*{§\:320.} Хрістіане во святомъ крещеніи вѣрою во Хріста Сына Божія отъ Бога родилися. \textit{Всякъ бо вѣруяй, яко Іисусъ есть Хрістосъ, отъ Бога рожденъ есть}\footnote{5,~1.}. Слѣдственно Бога Отца имѣютъ, и \textit{суть сынове Божіи вѣрою о Хрістѣ Іисусѣ}\footnote{Гал.~3,~26.}. \textit{Елицы бо пріяша Его, даде имъ область чадомъ Божіимъ быти, вѣрующимъ во имя Его, иже не отъ крове, ни отъ похоти плотскія, ни отъ похоти мужескія, но отъ Бога родишася}\footnote{Іоан.~1,~12 и 13.}. Почему и Хрістосъ Сынъ Божій повелѣлъ имъ Бога призывати какъ \textit{Отца}, Бога \textit{Отцемъ} своимъ нарицати, и молитися тако: \textit{Отче нашъ, Иже еси на небесѣхъ}, и проч.\footnote{Мѳ.~6,~9.} Должность убо хрістіанская того требуетъ, чтобы хрістіане Богу, яко Отцу своему, подражали, и Ему, яко чада возлюбленная, своими подобилися нравами, якоже Апостолъ ихъ къ тому увѣщаваетъ: \textit{бывайте подражатели Богу, якоже чада возлюбленная}\footnote{Еф.~5,~1.}. Въ хрістіанинѣ бо, который \textit{водою и Духомъ рожденъ есть}\footnote{Іоан.~3,~5.}, долженъ быть наченшійся образъ Божій, по писанному: \textit{елицы во Хріста крестистеся, во Хріста облекостеся}\footnote{Гал.~3,~27.}; образъ же неотмѣнно долженъ быть подобенъ первообразному, якоже въ зерцалѣ образъ являющійся подобенъ есть первообразному, кто смотритъ въ зерцало. Иначе бы не былъ образъ, когда бы не былъ подобенъ первообразному. Отсюду послѣдуетъ, что въ хрістіанѣхъ, которые благодатію Божіею и вѣрою о Хрістѣ Іисусѣ сдѣлалися сынами Божіими и образъ Божій наченшійся въ себѣ имѣютъ, должны быть богоподобные нравы, и воля, елико возможно въ семъ вѣцѣ, согласна съ волею Божіею. И чѣмъ лучшая воля и согласнѣйшая волѣ Божіей, и богоподобнѣйшіе нравы въ коемъ хрістіанинѣ имѣются, тѣмъ чистѣйшій и яснѣйшій Божій образъ въ немъ сіяетъ: якоже чѣмъ чистѣйшее зерцало, тѣмъ яснѣе образъ смотрящаго въ него является. На который образъ Свой Богъ, яко Отецъ на сына своего, смотря и видя Самого Себе въ немъ, веселится. И якоже въ зерцалѣ таковое подобіе изображается, каково смотрящее есть лице, и что дѣлаетъ первообразное лице, которое въ зерцало смотритъ, тоежде въ зерцалѣ видится; словомъ, во всемъ подобится, во всемъ подражаетъ образъ въ зерцалѣ являющійся смотрящему лицу въ зерцало; тако подобаетъ хрістіанамъ во всемъ, что возможно силѣ ихъ, подражати Отцу небесному, яко своему \textit{первообразному}. И якоже зерцало погубляетъ образъ лица человѣческаго когда отъ лица того, которое въ него смотрѣло, отвращается; и къ чему обращается, таковый и образъ въ себе воспріемлетъ: тако и душа человѣческая образъ Божій погубляетъ, когда отъ Бога отвращается, и то въ ней изображается, къ чему обращается и прилѣпляется; напримѣръ, когда любитъ честь, славу, богатство и сласть міра сего, тогда въ ней земный и безсловесный образъ изображается. Тако праотецъ Адамъ, когда отвратился отъ Бога, потерялъ образъ Божій, и облекся во образъ земный и скотскій. Сего ради слово Божіе хрістіанъ, яко водою и Духомъ отъ Бога рожденныхъ, отводитъ отъ любви міра сего: \textit{не любите міра, ни яже въ мірѣ. Аще кто любитъ міръ, нѣсть любве Отчи въ немъ. Яко все, еже въ мірѣ, похоть плотская, похоть очесъ и гордость житейская, нѣсть отъ Отца, но отъ міра сего есть}\footnote{1~Іоан.~2,~15 и 16.}.

\paragraph*{§\:321.} Святое Божіе слово, якоже чистое зерцало, представляетъ намъ, какій въ человѣкѣ, водою и Духомъ отрожденномъ, долженъ быть образъ Божій, и въ чемъ онъ состоитъ, которое все до того насъ руководствуетъ, дабы въ насъ образъ возобновился, который по паденіи Адамовомъ растлѣлъ: \textit{всяко бо писаніе богодухновенно и полезно есть ко ученію, ко обличенію, ко исправленію, къ наказанію, еже въ правдѣ: да совершенъ будетъ Божій человѣкъ, на всякое дѣло благое уготованъ}\footnote{2~Тим.~3,~16 и 17.}. Ибо якоже ветхаго человѣка образъ земный и скотскій, тако новаго образъ небесный и Божій въ немъ описуется, и повелѣвается намъ \textit{отложити, по первому житію, ветхаго человѣка, тлѣющаго въ похотехъ прелестныхъ, обновлятися же духомъ ума нашего, и облещися въ новаго человѣка, созданнаго по Богу въ правдѣ и преподобіи истины}\footnote{Еф.~4,~22--24.}. Сіе убо намѣреваетъ слово Божіе, когда повелѣваетъ намъ \textit{быть милосердымъ, якоже Отецъ небесный милосердъ есть}\footnote{Лук.~6,~36.}; \textit{быть совершенными, яко Той совершенъ есть}\footnote{Мѳ.~5,~48.}; \textit{быть святымъ, яко Онъ святъ есть}\footnote{1~Петр.~1,~15 и 16.} и проч. Сего ради какъ въ зерцало смотримъ ради того, да пороки, на лицѣ усмотрѣвше, очистимъ: тако часто подобаетъ хрістіанамъ посматривать въ ясное святаго Писанія зерцало, въ которомъ живо Божій образъ описуется, дабы, въ тое взирая, могли очищать покаяніемъ и вѣрою во Хріста пороки, къ душамъ ихъ прилѣпившіеся. А чѣмъ кто болѣе очищать будетъ душевные свои пороки, тѣмъ чистѣйшая будетъ у него душа; чѣмъ же чистѣйшая будетъ душа, тѣмъ яснѣе въ ней будетъ блистать Божій образъ, прекрасная и любезная ея доброта. О семъ, хрістіанине, помышляй, о семъ старайся. На сіе бо и Хрістосъ пришелъ въ міръ, да въ насъ образъ Божій возставитъ и возобновитъ, который мы во Адамѣ потеряли. Отсюду видно: 1)~Коль нужно всякому хрістіанину чтеніе, или слушаніе прилѣжное святаго Писанія. 2)~Хрістіанское житіе все не иное что есть, какъ всегдашнее до кончины жизни покаяніе. Ибо всякъ, соравняя себе съ зерцаломъ священнаго Писанія, усмотритъ, что на всякъ день покаяніемъ и вѣрою во Хріста въ себѣ очищать и исправлять, \textit{много бо согрѣшаемъ вси}\footnote{Іак.~3,~2.}; чего ради на всякій день имѣютъ нужду просить отпущенія грѣховъ и молитися хрістіане: \textit{Отче! остави намъ долги наша, якоже и мы оставляемъ должникомъ нашимъ}\footnote{Мѳ.~6,~12.}.

\paragraph*{§\:322.} Писаніе святое, яко Божественное правило и ясное святаго хрістіанскаго житія зерцало, представляетъ намъ, чего мы уклоняться и что творить должны, когда хощемъ подражать небесному Отцу. Оно научаетъ насъ \textit{уклоняться отъ зла, и творить благое}\footnote{Пс.~33,~15.}, да тако можемъ послѣдовать Богу, Творцу и Отцу нашему; яко всякое зло Богу противно и насъ отъ Бога отводитъ, и благообразіе души нашей помрачаетъ и растлѣваетъ; и всякое добро Богу сообразно, и Богу насъ уподобляетъ оно. Когда учитъ насъ Богу подражать, учитъ не міръ созидать, не мертвыя воскрешать, не по водамъ ходить, и прочія чудеса творить. Сіе бо единаго всемогущества Его дѣло есть, хотя по вѣрѣ и чудеса \textit{возможна вѣрующему}\footnote{Марк.~9,~23.}. Но когда велитъ подражателями Божіими быть, представляетъ намъ къ подражанію Его благость, милосердіе, щедроты, правду, истину, святость, долготерпѣніе Его и прочая, да, на сіи Его Божественныя свойства взирая, и сами учимся дѣлать тое, что Онъ дѣлаетъ: якоже дѣти, что видятъ отца дѣлающа, и сами тоежде дѣлать тщатся. Въ семъ святомъ зерцалѣ видимъ \textit{истину Божію}, которая солгать и прельстить не можетъ, но что объявляетъ, открываетъ и обѣщаетъ намъ, тое такъ есть неотмѣнно, какъ объявляется и открывается, хотя умъ нашъ и не постигаетъ, что показуютъ многая пророчества, самымъ дѣломъ исполнившіяся. На истину Отца небеснаго взирая, хрістіане не должны лгать, льстить, лукавновать и лицемѣрить; но должны быть простосердечными, \textit{истину глаголати въ сердцѣ своемъ, не льстить языкомъ своимъ}\footnote{Пс.~14,~2 и 3.}; что языкомъ объявляютъ, тое и въ сердцѣ имѣть, и каковыми внѣ предъ людьми являются, таковыми и внутрь въ сердцѣ быть; и что согласное и непротивное слову Божію обѣщаютъ ближнему, тое исполнять, аще въ силѣ ихъ тое состоитъ. "--- Представляется тамо \textit{правда Божія}, которая всякому свое отдаетъ. Тѣмъ увѣщаваются и хрістіане \textit{всѣмъ воздавать должная: емуже убо урокъ, урокъ; а емуже дань, дань; а емуже страхъ, страхъ; и емуже честь, честь}\footnote{Римл.~13,~7.}. Чего не хотятъ себѣ, того другимъ не дѣлать; и чего хотятъ себѣ, тое и ближнимъ творить. Въ семъ состоитъ правда, которая всѣмъ должная воздавать учитъ. Не хощемъ себѣ зла, "--- и ближнему нашему, то"=есть, всякому человѣку, не должны желать и дѣлать зла. Хощемъ себѣ добра, тоежъ и ближнему должны желать и дѣлать. Не хощемъ, чтобъ кто у насъ похитилъ, отнялъ, укралъ что, имя наше опорочилъ, укорилъ насъ, и прочее зло сдѣлалъ намъ, "--- и сами отъ того уклонятися должны. Хощемъ, чтобъ ближній нашъ въ нуждѣ насъ не оставилъ, и сами тоже ближнему нашему творить должны. Того и Хрістосъ нашъ отъ насъ требуетъ: \textit{вся елика аще хощете, да творятъ вамъ человѣцы, тако и вы творите имъ. Се бо есть законъ и пророцы}\footnote{Матѳ.~7,~12.}. Въ томъ чистѣйшемъ зерцалѣ видимъ и удивляемся \textit{благости} Божіей, которая \textit{солнце свое сіяетъ на злыя и благія, и дождитъ на праведныя и на неправедныя}\footnote{5,~45.}. Отсюду научаются хрістіане не токмо друговъ, но и \textit{враговъ своихъ любить, благословлять кленущія ихъ, добро творить ненавидящимъ ихъ, и молиться за творящихъ имъ напасть и изгонящія ихъ}\footnote{ст.~44.}; \textit{не воздавать зла за зло, или досажденія за досажденіе}\footnote{1~Петр.~3,~9.}; но \textit{аще алчетъ врагъ, насытить его; аще ли жаждетъ, напоить его: не побѣжденными быть отъ зла, но побѣждать благимъ злое}\footnote{Римл.~12,~20 и 21.}. \textit{Аще бо любите любящія вы}, глаголетъ Хрістосъ хрістіанамъ, \textit{кая вамъ благодать есть? ибо и грѣшницы любящія ихъ любятъ}\footnote{Лук.~6,~32.}. Ничтоже высше отъ язычниковъ дѣлаютъ хрістіане, которые должны, какъ свѣтъ отъ тьмы, благочестивымъ житіемъ отъ нихъ разнствовать\footnote{Филип.~2,~15.}, когда только любящихъ любятъ, и благотворящимъ только благотворятъ. Смотря въ зерцало святаго Писанія, видимъ святость Божію, яко свѣтъ, никакія тьмы непричастный: \textit{яко Богъ свѣтъ есть, и тьмы въ немъ нѣсть ни единыя}\footnote{1~Іоан.~1,~5.}. Тако подобаетъ и хрістіанамъ, \textit{якоже чадамъ свѣта, ходити}\footnote{Еф.~5,~8.}; яко чадамъ Божіимъ, о святомъ житіи пещися, \textit{очищать себе отъ всякія скверны плоти и духа, творяще святыню въ страсѣ Божіи}\footnote{2~Кор.~7,~1.}; \textit{по звавшему Святому, и самимъ святыми во всемъ житіи быти, зане писано есть: святи будите, яко Азъ святъ есмь}\footnote{1~Петр.~1,~15 и 16.}. \textit{Не призва бо насъ Богъ на нечистоту, но во святость}\footnote{1~Сол.~4,~7.}. «Святость же разумѣется не токмо тое, чтобъ хранить себе отъ скверны, блуда и нечистоты, но и отъ лихоиманія, зависти, гордости, тщеславія и прочаго зла», по ученію святаго Златоустаго\footnote{2~Кор.~7,~1. Бес. на оное слово Апостол. \textit{очистимъ себе}, и проч.}. Паки видимъ тамо щедроты и милости Божія, и самымъ дѣломъ вкушаемъ на всякій день и часъ сладости щедротъ и милостей Его. Отсюду научаются и сами хрістіане милостивыми и щедрыми быти, и, какъ чада, милостивому и щедрому Отцу своему подобиться; удѣлять требующимъ отъ благихъ, которыя отъ щедрой Его получили десницы; не затворять ушей отъ просящихъ, и не отвращаться отъ требующихъ помощи. Безконечное милосердіе ко грѣшникамъ и отеческое Божіе состраданіе немощамъ нашимъ таможде видимъ. Милосердый бо тотъ есть, который утробою подвигается и сердцемъ болѣзнуетъ, видя бѣдность другаго. Таковое милосердіе Отца небеснаго представляется намъ въ словѣ Его святомъ, когда, ради насъ погибшихъ, и \textit{Сына Своего единороднаго не пощадѣлъ, но за насъ всѣхъ предалъ Его}\footnote{Римл.~8,~32.}, да насъ \textit{погибшихъ}, чрезъ Него \textit{взыщетъ и спасетъ}\footnote{Лук.~19,~10.}; "--- такожде когда намъ, согрѣшившимъ Ему и кающимся, отпущаетъ грѣхи. Въ томъ послѣдовать и мы должны Отцу нашему небесному. Онъ о насъ милосердствуетъ, "--- и мы должны о братіи нашей милосердствовать, по писанному: \textit{будите милосерди, якоже и Отецъ вашъ милосердъ есть}\footnote{6,~36.}. Онъ намъ бѣдствующимъ состраждетъ, "--- и мы должны братіи нашей напаствующей сострадати и соболѣзновати, \textit{плакати съ плачущими}\footnote{Римл.~12,~15.}, \textit{поминати юзники, аки съ ними связани, озлобленныя, аки и сами суще въ тѣлѣ}\footnote{Евр.~13,~3.}. Онъ намъ отпущаетъ грѣхи кающимся, отъ казни грѣхамъ послѣдующія избавляетъ, "--- и мы должны братіи нашей отъ милосердія \textit{оставляти согрѣшенія ихъ}\footnote{Матѳ.~6,~14.}. Да якоже намъ отъ Отца небеснаго оставляются долги многіе, такъ и сами оставимъ долгъ малый клевретамъ братіи нашей\footnote{18,~27 и 33.}. "--- Усматриваемъ въ томъжде священномъ зерцалѣ безприкладное \textit{долготерпѣніе} небеснаго Отца, Который столько всего міра грѣховъ, неправдъ, беззаконій, идолопоклоненій и прочіихъ тяжкихъ беззаконій терпѣлъ и терпитъ; и сами, во вся дни согрѣшая, дознаемъ тое. Толикое видя долготерпѣніе милосердаго нашего Отца, должны и мы терпѣливы быти къ согрѣшающей братіи нашей; не гнѣваться, ни злобиться на нихъ, не отмщевать имъ за обиды, намъ отъ нихъ сотворенныя, но съ терпѣніемъ и кротостію ожидать ихъ исправленія, да тако пріобрящемъ ихъ Хрісту, и себе истинными братіями содѣлаемъ, по подобію небеснаго Отца, Который столько грѣшниковъ долготерпѣніемъ Своимъ пріобрѣлъ и пріобрѣтаетъ, якоже Апостолъ написалъ: \textit{Господа нашего долготерпѣніе, спасеніе непщуйте}\footnote{2~Петр.~3,~15.}. \textit{Благость} бо \textit{Божія на покаяніе} всякаго \textit{грѣшника ведетъ}\footnote{Римл.~2,~4.}. "--- Открывается намъ таможде, яко \textit{Духъ есть Богъ}\footnote{Іоан.~4,~24.}. И мы, хотя не можемъ быти духомъ, яко плоть и кости имущіи, но можемъ и должны быть духовными; \textit{должни не плоти, еже по плоти жити, не плотская мудрствовати}, но \textit{духомъ дѣянія плотская умерщвляти}\footnote{Римл.~8,~12; 5,~13.}; \textit{плоти угодія не творити въ похоти}\footnote{13,~14.}; \textit{духомъ ходити, и похоти плотскія не совершати; плоть распинати со страстьми и похотьми}\footnote{Гал.~5,~16 и 24.}; \textit{огребатися отъ плотскихъ похотей, яже воюютъ на душу}\footnote{1~Петр.~2,~11.}: \textit{во еже не ктому человѣческимъ похотемъ, но воли Божіей, прочее во плоти жити время}\footnote{4,~2.}, "--- да не о насъ скажетъ Господь тое, что о первомъ мірѣ сказалъ: \textit{не имать Духъ Мой пребывати въ человѣцѣхъ сихъ во вѣкъ, зане суть плоть}\footnote{Быт.~6,~3.}. И Іуда Апостолъ: \textit{суть тѣлесни, духа не имущіи}\footnote{ст.~19.}. \textit{Сущіи бо по плоти Богу угодити не могутъ. Зане мудрованіе плотское вражда есть на Бога: закону бо Божію не покоряется, ниже бо можетъ}\footnote{Римл.~8,~8 и 7.}. Тако и въ прочемъ хрістіанамъ подобаетъ въ святомъ Писаніи, какъ въ зерцалѣ, взирать на образъ Отца небеснаго, \textit{да якоже облекохомся во образъ перстнаго, облечемся и во образъ небеснаго}\footnote{1~Кор.~15,~49.}; да тако засвидѣтельствуютъ хрістіане новое свое и духовное рожденіе; да покажутъ, что они истинно \textit{водою и духомъ рождены суть}\footnote{Іоан.~3,~5.}, истинно отъ Бога \textit{родишася}\footnote{1,~13.}, истинно и нелицемѣрно Бога Отцемъ своимъ нарицаютъ и молятся: \textit{Отче нашъ, Иже еси на небесѣхъ} и проч.\footnote{Матѳ.~6,~9; Лук.~11,~2.}, "--- истинно \textit{суть сынове Божіи вѣрою о Хрістѣ Іисусѣ}\footnote{Гал.~3,~26.}, яко Ему, яко сынове Отцу, подобятся нравами своими, и образъ Его Божественный въ себѣ носятъ. Якоже бо отъ Адама раждаемся грѣшными, скверными, тако отъ Бога раждаемся водою и Духомъ святыми и чистыми. И якоже отъ Адама раждаемся злонравными, лживыми, грѣхолюбивыми, нелюбительными, завистливыми, злобными, гордыми, похотливыми, славолюбивыми, сластолюбивыми и самолюбивыми; тако отъ Бога, яко благаго Отца, рожденнымъ должно быть добронравными, истинными, правдолюбивыми, любительными, милосердыми, милостивыми, кроткими, терпѣливыми, смиренными, смиренномудренными, воздержными, благолюбивыми и братолюбивыми. Якоже отъ Адама раждаемся плотская и земная мудрствующими, тако рожденнымъ отъ Бога подобаетъ быти духовная и небесная мудрствующими, яко \textit{Духъ есть Богъ}\footnote{Іоан.~4,~24.}. \textit{Рожденное бо отъ плоти, плоть есть; и рожденное} Богомъ \textit{отъ духа, духъ есть}, по словеси Хрістову\footnote{Іоан.~3,~6.}. Ибо какое кто рожденіе имѣетъ, того и свойства имѣть долженъ. Отсюду послѣдуетъ, что хрістіане, яко вновь рожденные отъ Бога, должны Богу, Отцу своему, подобитися нравами своими; должны подражати Богу благому въ благости Его, щедрому въ щедротахъ Его, милостивому и щедрому въ милости и милосердіи Его, истинному въ истинѣ Его, долготерпѣливому, кроткому, святому "--- въ долготерпѣніи, кротости, святости и прочіихъ Божественныхъ свойствахъ, и тако быть благолюбивыми, щедролюбивыми, святолюбивыми, истиннолюбивыми, правдолюбивыми, милостивыми, милосердыми, терпѣливыми, и прочіими добрыми нравами отъ чиста сердца и нелицемѣрно себе украшати.

\paragraph*{§\:323.} Сіе истинныхъ хрістіанъ Отцу небесному подражаніе отъ истинной ихъ вѣры и отъ размышленія Божественныхъ Его свойствъ, которыя въ святомъ Писаніи открываются и къ подражанію представляются, происходитъ. Какъ бо истинное богопочитаніе, такъ и истинное Богу подражаніе отъ истиннаго богопознанія происходитъ. Ибо и подражаніе истинное до богопочитанія надлежитъ, и не иное что есть, какъ усердное богопочитаніе. Ни чѣмъ бо инымъ Бога почитать и Богу угодить не можемъ, какъ такимъ житіемъ, какое Ему пріятно, когда хощемъ и дѣлаемъ тое, что Онъ хощетъ и дѣлаетъ; и не хощемъ и отвращаемся того, чего не хощетъ и отвращается Онъ. Сего истинное богопочитаніе, сего истинное имени Божія познаніе требуетъ отъ насъ. \textit{Возвѣстися тебѣ, человѣче, что добро, или чесого Господь ищетъ отъ тебе, развѣ еже творити судъ, и любити милость, и готову быти, еже ходити съ Господемъ твоимъ}\footnote{Мих.~6,~8.}. Богъ твой и Господь, хрістіанине, никакого зла, грѣха и беззаконія не хощетъ, \textit{отвращается и ненавидитъ его}\footnote{Пс.~5,~4--7.}; яко всякое зло, грѣхъ и беззаконіе не Его, но діавольское изобрѣтеніе и дѣло есть; и отъ тебе ищетъ того, чтобы и ты отъ всякаго грѣха, беззаконія отвращался и гнушался тѣмъ, яко діавольскимъ дѣломъ. Покажи убо вѣру твою отъ дѣлъ твоихъ, и новаго Божественнаго рожденія плодъ; послѣдуй Богу и Господу твоему, и ненавиди грѣхъ и беззаконія, и гнушайся тѣмъ. "--- Богъ твой и Господь \textit{праведенъ есть, и правду любитъ}\footnote{10,~7.}; хощетъ и отъ тебе того, чтобъ и ты, \textit{оправдившися именемъ Господа нашего Іисуса Хріста}\footnote{1~Кор.~6,~11.}, правду любилъ и дѣлалъ правду. Покажи убо плодъ оправданія онаго; послѣдуй Господу твоему и дѣлай правду, да тако покажеши, что ты истинно отъ Него родился, якоже глаголетъ Апостолъ: \textit{аще вѣсте, яко праведникъ есть Богъ, разумѣйте, яко всякъ, творяй правду, отъ Него родися}\footnote{1~Іоан.~2,~29.}. "--- Богъ твой и Господь никого не обидитъ, но все дѣлаетъ праведно, что ни дѣлаетъ\footnote{Евр.~6,~10.}; убо и тебѣ, по примѣру Его, должно никого не обидѣть ни словомъ, ни дѣломъ, ни инымъ какимъ образомъ. "--- \textit{Богъ твой и Господь истиненъ есть, вѣренъ во всѣхъ словесѣхъ Своихъ}\footnote{Пс.~144,~13; 85,~15; Іоан.~17,~17.}, и отъ тебе требуетъ того, и тебѣ должно отъ лести, лжи, лукавства и лицемѣрія храниться, но паче любить и хранить истину въ словахъ и дѣлахъ твоихъ\footnote{Еф.~4,~24.}. "--- Богъ твой и Господь любитъ тебе и всѣхъ, и милуетъ всѣхъ; хощетъ и отъ тебе того; и тебѣ, по примѣру Его, должно всѣхъ любить, по увѣщанію Апостола: \textit{аще сице возлюбилъ есть насъ Богъ, и мы должны есмы другъ друга любити}\footnote{1~Іоан.~4,~11.}, и любви плодъ "--- милость показывать всѣмъ, которые требуютъ милости. "--- Богъ и Господь твой щедръ есть, подаетъ тебѣ и всѣмъ благая Своя; хощетъ, чтобъ и ты Ему въ томъ подражалъ, отъ благихъ Его, данныхъ тебѣ отъ Него, удѣлялъ требующимъ, алчущихъ питалъ, жаждущихъ напаялъ, нагихъ одѣвалъ, странныхъ въ домъ свой вводилъ, и проч. "--- Богъ твой и Господь твоего и всѣхъ спасенія хощетъ и ищетъ, \textit{Иже всѣмъ хощетъ спастися, и въ разумъ истины пріити}\footnote{1~Тим.~2,~4.}; покажи и ты сей человѣколюбія на себѣ нравъ; послѣдуй волѣ благой и человѣколюбивому хотѣнію небеснаго Отца; тщись о своемъ и ближняго твоего спасеніи. "--- Богъ твой и Господь долготерпитъ беззаконія и грѣхи человѣческіе, и тѣмъ ожидаетъ покаянія ихъ; хощетъ и отъ тебе того; и тебѣ убо должно обиды, отъ человѣкъ нанесенныя, претерпѣвать, не гнѣваться, ни злобиться на нихъ, но ожидать ихъ исправленія. "--- Богъ твой и Господь прощаетъ тебѣ и всѣмъ кающимся грѣхи, и не поминаетъ ихъ\footnote{Іезек.~18,~21--23.}; хощетъ и отъ тебе того; и ты убо брату твоему, яко грѣшникъ грѣшнику, и человѣкъ человѣку, и рабъ клеврету твоему, прощать и оставлять согрѣшенія его долженъ ради Господа твоего и не поминать ихъ\footnote{Матѳ.~6,~14; 18,~22--35.}. "--- Богъ твой и Господь святъ есть; ищетъ и отъ тебе того: \textit{святи будите, яко Азъ святъ есмь}\footnote{1~Петр.~1,~16.}. Должно и тебѣ святость, данную тебѣ въ крещеніи, хранить, и ради того хранить себе отъ всякія скверны плоти и духа, уклоняться отъ всякія нечистоты и проч., что отъ общенія съ Богомъ и Сыномъ Его Іисусомъ Хрістомъ отлучаетъ: \textit{яко Богъ свѣтъ есть, и тьмы въ Немъ нѣсть ни единыя. Аще речемъ, яко общеніе имамы съ Нимъ, и во тьмѣ ходимъ, лжемъ и не творимъ истины}\footnote{1~Іоан.~5,~6.}. Сего отъ тебе, человѣче, Богъ твой, Котораго исповѣдуешь, ищетъ. Сего вѣра твоя, сего познаніе имени Его, сего новое рожденіе или крещеніе твое, въ которомъ ты родился водою и Духомъ, отрекся сатаны, міра и похотей его, и обѣщался Богу работать и послѣдовать Ему, требуетъ. Родилися мы въ плотскомъ рожденіи нечистыми, грѣшными, скверными; но въ духовномъ рожденіи вновь отродилися, \textit{омылися, освятилися, оправдалися именемъ Господа нашего Іисуса Хріста, и Духомъ Бога нашего}\footnote{1~Кор.~6,~11.}. Убо сего новаго рожденія плоды должны показывать, которые не иное что суть, какъ хрістіанскія добродѣтели и житіе богоподобное. Пали мы во Адамѣ, убо во Хрістѣ и чрезъ Хріста должны встать, \textit{да, якоже восталъ Хрістосъ отъ мертвыхъ славою Отчею, тако и мы во обновленіи жизни ходити будемъ}\footnote{Римл.~6,~4.}. Въ новомъ рожденіи какъ востаемъ силою и благодатію Хрістовою отъ того паденія, такъ и отраждаемся, или, какъ просто сказать, перераждаемся дѣйствіемъ и силою Святаго Духа, слѣдственно востанія нашего и отрожденія плоды и знаки показывать должны. Какъ бо падшій и воставшій тѣлесно ходитъ и дѣлаетъ дѣла свои, и рожденный младенецъ дѣйствія рожденія своего и живота показуетъ, то"=есть, чувствуетъ, питается, плачетъ, и прочая: тако духовнѣ востающимъ отъ грѣхопаденія Адамова, которое востаніе вѣрою во Хріста и силою Духа Святаго бываетъ, подобаетъ знаки того востанія показывать "--- духовнѣ ходить, духовнѣ питаться, духовная мудрствовать, дѣлать; и рожденнымъ младенцамъ о Хрістѣ, то"=есть, отрожденнымъ водою и Духомъ, должно отрожденія своего дѣйствія показывать, то"=есть, дѣла духовному своему рожденію приличныя и богоугодныя творить. \textit{Рожденное отъ плоти, плоть есть; и рожденное отъ Духа, духъ есть}, глаголетъ Хрістосъ\footnote{Іоан.~3,~6.}. То"=есть, кто отъ плотскаго только родителя родился, не оное что есть, какъ только плоть, и плотская мудрствуетъ: кто рожденъ есть отъ Духа, Духа свойства въ себѣ имѣетъ, и тако \textit{духовная мудрствуетъ}, *якоже и Апостолъ глаголетъ: \textit{сущіи по плоти, плотская мудрствуютъ; а иже по духу, духовная}\footnote{Римл.~8,~5.}.* То"=есть, люди плотскіе не иное что замышляютъ и дѣлаютъ, какъ только тое, что плоти растлѣнной угодно есть; но духовные, которые Духомъ Божіимъ отрождены суть, замышляютъ и тщатся дѣлать тое, что духовному ихъ рожденію свойственно и Богу угодно. И какъ рожденное отъ плоти живое есть и дѣйствующее, напр. отъ живой матери живый младенецъ родится, и потому дѣла плоти приличная дѣлаетъ: тако рожденному отъ Духа, внутреннему или новому человѣку, подобаетъ быть живому, и приличныя духовному своему рожденію дѣлати дѣла, жизнь, силу и дѣйствіе въ себѣ имѣть, и дѣлами подобными Богу, отъ Котораго рожденъ есть, живность свою и силу оказывати. И якоже сынъ плотскаго своего отца свойства изображаетъ, тако отъ Бога рожденному должно свойства Божіи, правды, истины, любви, терпѣнія, святости, и прочая въ себѣ изображати. Иначе не истинное, но мечтательное было бы рожденіе, когда бы рожденный рождшаго свойствъ въ себѣ не имѣлъ. Живность же сія во внутреннемъ или новомъ человѣкѣ не иное что есть, какъ вѣра живая и дѣйствующая, отъ Святаго Духа заченшаяся, которая живность свою, какъ доброе древо плодами, добрыми дѣлами оказываетъ.

\paragraph*{§\:324.} Чтобы свойствамъ и благимъ дѣламъ небеснаго Отца послѣдовать, должно прежде своихъ нравовъ злыхъ отрещися. Чтобы волѣ Божіей послѣдовать, должно прежде свою волю оставить. Аще хощемъ правдѣ Его, истинѣ, благости, терпѣнію, кротости, милосердію, милости и прочіимъ Его Божественнымъ свойствамъ подражати, должны отрещись неправды, лжи, злобы, гнѣва, роптанія, нетерпѣнія, жестокосердія, скупости, нечистоты и прочіихъ злыхъ и богопротивныхъ нашихъ нравовъ. Примѣчай, хрістіанине, что есть отрещися своего злонравія, и тако подражати благости Божіей. Всякъ отъ насъ имѣетъ въ себѣ отъ рожденія своего ветхаго человѣка, и ветхимъ человѣкомъ отъ родителей своихъ раждается. Свойства и нравъ его не что иное есть, какъ самолюбіе, плотоугодіе, славолюбіе, сластолюбіе, неправда, ложь, нечистота, немилосердіе, жестокосердіе, гнѣвъ, ярость, злоба, роптаніе, хуленіе и прочіи грѣхи. Сіи суть ветхаго рожденія и ветхаго человѣка нравы. Сими онъ утѣшается и услаждается. Его свойство есть: любить почитаніе, поклоненіе, похвалу, славу отъ человѣкъ; ему свойственно есть гордиться, возноситься, надыматься, презирать другаго, "--- въ противномъ случаѣ: роптать, гнѣваться, хулить, злобиться, отмщевать словомъ или дѣломъ, зло за зло воздавать, и тѣмъ утѣшаться; ему пріятно все у себе имѣть, а о другомъ нерадѣть; всѣмъ тѣмъ утѣшаться, что чувствамъ подлежитъ; въ томъ единомъ удовольствіе, утѣшеніе, услажденіе полагать, что чувства утѣшаетъ, услаждаетъ и увеселяетъ; отъ того отвращаться, что имъ непріятно, хотя и душѣ полезно и Богу угодно. Онъ о томъ единомъ только тщится, замышляетъ, начинаетъ и дѣлаетъ, что видитъ, что временное есть, что ему богатство, славу, честь и сласть тѣлесную и временную приноситъ. Ему непріятно все тое, что его злонравіе обличаетъ; пріятно все тое, что растлѣнной его волѣ сходно. Отвращается отъ всего того, что самолюбіе, славолюбіе и сластолюбіе пресѣкаетъ; любитъ все тое, что тыя его прихоти умножаетъ. Всякимъ образомъ собирать, богатѣть, собраннымъ утѣшаться, пространно жить, роскоши расширять, банкеты и пированія строить, насыщаться, напиваться, веселиться, украшаться и прочія прихоти исполнять, то его и дѣло, въ томъ свое удовольствіе полагаетъ. Онъ любитъ самъ одѣваться, питаться, упокоеваться пространно, а о ближнемъ нерадитъ. Любитъ всякое довольствіе имѣть, хотя бы было тое съ обидою ближняго и презрѣніемъ Божественнаго закона и Самого Законодавца. Онъ отъ другихъ обиды, укоризны, безчестія, поношенія, клеветы, проклинанія не терпитъ, злобится за тое и отмщеваетъ; но самъ другихъ обидѣть, насиловать, обнажить, укорить, обезчестить, осудить, оклеветать и во всякое бѣдствіе привести не стыдится. Ему нѣтъ стыда тое въ себѣ любить, хвалить, что въ другихъ ненавидитъ и хулитъ. Въ немъ и тое примѣчается, что людей учитъ, наставляетъ, просвѣщаетъ, самъ весь ослѣпленъ, безуменъ и несмысленъ. Словомъ, всякое беззаконіе, грѣхъ, слѣпота, безстыдство въ ветхомъ человѣкѣ крыется. Онъ не иное что, токмо помышленія злая зачинаетъ и родитъ. \textit{Отъ сердца бо человѣческаго исходятъ помышленія злая, прелюбодѣянія, любодѣянія, убійства, татьбы, лихоимства, обиды, лукавствія, лесть, студодѣянія, око лукаво, хула, гордыня, безумство}\footnote{Марк.~7,~21 и 22.}. Въ немъ видится весь нравъ скотскій, и звѣриный и зміиный. Нечистота скотская, обжирство скотское, лютость, гнѣвъ, ярость и лукавство звѣриное. Примѣчай скотовъ и звѣрей, что они дѣлаютъ, и самъ узнаешь и признаешь, что все тое въ человѣкѣ неотрожденномъ и необновленномъ благодатію Божіею видится, что въ нихъ видишь; и что въ каждомъ скотѣ и звѣрѣ примѣчаешь, тое все въ единомъ человѣкѣ имѣется. Лукавъ и хитръ, какъ лисъ; ядовитъ, какъ змій; гнѣвливъ, яростенъ и лютъ, какъ левъ; обжирчивъ, какъ песъ и волкъ; нечистъ, какъ свинья, и проч. Сіе все и прочее злонравіе въ ветхомъ человѣкѣ крыется, и при случаѣ себе внѣ оказываетъ. Сего ветхаго человѣка со злонравіемъ отрещися должно хрістіанину, да тако послѣдуетъ Богу въ преблагихъ нравахъ Его; отрещися лжи, лицемѣрія, лукавства, неправды, нечистоты, сребролюбія, скупости, гнѣва, ярости, сладострастія и прочаго злонравія, и тако послѣдовати Божіей истинѣ, правдѣ, святости, милосердію, долготерпѣнію, кротости, благости и прочіимъ Божественнымъ Его свойствамъ. Какъ бо хотящему на небо смотрѣть, надобно отъ земли очи отвратить и обратить на небо, и хотящему на солнце смотрѣть, должно къ нему обратиться: тако хотящему небесному житію подражать, должно оставить земное и скотское, и хотящему на Бога смотрѣть вѣрою, и Его святымъ нравамъ послѣдовать, должно позади оставлять свое природное злонравіе. И сіе"=то есть \textit{отрещися себе}, воли своея, своего злонравія, то"=есть, не дѣлать того, что ветхій нашъ человѣкъ, живущій въ насъ, хощетъ, но тое, что воля Божія ищетъ и требуетъ отъ насъ, когда хощемъ Божіимъ свойствамъ и нравамъ послѣдовать. Но какъ ветхаго человѣка отлагать въ похотѣхъ тлѣющаго, такъ и въ новаго облещися должно намъ, да тако послѣдуемъ Богу, нашему небесному Отцу. На сіе бо истое какъ ветхій человѣкъ, такъ и новый въ словѣ Божіемъ представляется намъ, да того совлечемся, въ сего облечемся; онаго образъ скотскій и земный помощію и силою Духа Святаго отлагаемъ, сего образъ Божій и небесный воспріимемъ тогожде Духа Святаго силою; онаго безобразія совлечемся, въ сего благообразіе облечемся, якоже Апостолъ повелѣваетъ намъ отложити, по первому житію, ветхаго человѣка, тлѣющаго въ похотѣхъ прелестныхъ: \textit{обновлятися же духомъ ума нашего, и облещися въ новаго человѣка, созданнаго по Богу въ правдѣ и преподобіи истины}\footnote{Еф.~4,~22--24.}. Да, тако ветхаго человѣка отлагая, котораго есть образъ скотскій и звѣриный, мерзкій и скаредный, и въ новаго облекаяся, котораго есть образъ Божій, небесный и святый, "--- Отцу нашему небесному, Который насъ по образу Своему и по подобію создалъ, и тотъ образъ, грѣхомъ растлѣнный, во Хрістѣ обновляетъ силою Святаго Духа, "--- сотворимся подобные благими нравами своими: будемъ милосердыми, якоже Онъ милосердъ есть; будемъ чисти и святи, якоже Онъ святъ есть; будемъ щедры, кротки, терпѣливы и милостивы, якоже Онъ щедръ и милостивъ, долготерпѣливъ и многомилостивъ. И колико будемъ совлекаться ветхаго человѣка, толико будемъ облекаться въ новаго; ибо отложеніе единаго есть *облеченіе другаго, и смерть единаго есть животъ другаго*. Колико бо совлекается и умерщвляется ветхій человѣкъ, толико облекается и оживляется въ насъ новый; колико же облекаться будемъ въ новаго человѣка, толико богоподобнѣйшія будутъ свойства въ насъ; колико же богоподобнѣйшія свойства будутъ, толико будетъ чистѣйшій и яснѣйшій образъ Божій въ насъ. Душа человѣческая подобна есть зеркалу. Какъ въ зеркалѣ, обращенномъ къ солнцу, тѣмъ яснѣе видится солнце, чѣмъ чистѣйшее есть зеркало: тако въ душѣ человѣческой, обратившейся къ Богу, вѣчному Солнцу, тѣмъ яснѣйшій образъ Божій изображается и видится, чѣмъ болѣе очищается она отъ грѣховъ и Божіимъ святымъ свойствамъ и нравамъ послѣдуетъ. И какъ въ зеркалѣ не изобразится подобіе солнца, когда къ солнцу не обратится, или закопчено и замарано будетъ: тако въ душѣ человѣческой не изобразится образъ Божій, когда онъ къ Богу не обратится и отъ грѣховъ очищаться не будетъ. Къ Богу же обращаемся тогда, когда сердце наше, умъ и житіе наше премѣняемъ и начинаемъ житіе новое, и тако совлекаемся ветхаго и облекаемся въ новаго человѣка, который по образу и по подобію Божію созданъ\footnote{Быт.~1,~26 и 27.}.

Отъ вышеписанныхъ видишь, хрістіанине: 1)~что есть истинный хрістіанинъ. "--- То"=есть, который образъ Божій наченшійся въ себѣ имѣетъ, и по правилу святаго Божія слова, въ которомъ образъ Божій описуется, житіе свое исправляти тщится, \textit{совлекается ветхаго человѣка съ дѣяньми его, и облекается въ новаго, обновляемаго въ разумъ, по образу создавшаго Его}\footnote{Кол.~3,~9 и 10.}, то"=есть, плотскія вожделѣнія и прихоти пресѣкаетъ, и новаго рожденія свойства, то"=есть, хрістіанскія добродѣтели показуетъ, и такъ Первообразному, Отцу небесному, подобится, свойствамъ и нравамъ Его благимъ послѣдуетъ; сего бо неотмѣнно новое рожденіе, въ которомъ хрістіанинъ раждается отъ Бога, требуетъ, какъ выше сказано. 2)~Видишь, что ничѣмъ инымъ Богу угодить не можемъ, какъ только тѣмъ служеніемъ, которое Онъ въ словѣ Своемъ объявилъ и повелѣлъ, то"=есть, покаяніемъ, вѣрою и покореніемъ воли нашея волѣ Его святой, когда уклоняемся отъ того, чего Онъ не хощетъ, и тщимся дѣлать тое, что Онъ отъ насъ хощетъ и требуетъ. 3)~Отсюду видишь, что безъ послушанія и покоренія воли человѣческія Богу ничто не пріятно, что бы человѣкъ ни дѣлалъ и вымышлялъ. Надобно неотмѣнно волю свою волѣ Его покорить, чего вѣра и новое рожденіе отъ насъ требуетъ. 4)~Видишь далѣе, что грѣхолюбивое и беззаконное житіе есть противно вѣрѣ, новому рожденію, которое своихъ свойственныхъ плодовъ, то"=есть, хрістіанскихъ добродѣтелей требуетъ, и такъ противно истинному хрістіанству. 5)~Отсюду видишь, что хрістіанинъ неисправный, который противно слову Божію живетъ и ветхаго человѣка прихоти исполняетъ, въ плотскомъ только рожденіи находится, духовнаго и новаго не имѣетъ рожденія, а тако не есть истинный хрістіанинъ, но плотскій, душевный и неотрожденный, хотя и крещенъ, въ церковь ходитъ, молится и прочіе хрістіанства знаки показуетъ. Не едина бо наружность хрістіанства дѣлаетъ хрістіанина, но вѣра живая, живущая въ сердцѣ и живность свою добрыми дѣлами показывающая. Откуду \textit{истинный хрістіанинъ доброму древу}, которое извнутрь себе добрые плоды творитъ, якоже \textit{ложный хрістіанинъ злому древу}, которое такожде извнутрь себе злые плоды творитъ, уподобляется въ Писаніи\footnote{Матѳ.~7,~17 и 18.}. Отсюду, наконецъ, послѣдуетъ, что таковые ложные хрістіане, хотя посредѣ церкви и находятся, но никакого участія въ церкви не имѣютъ; не суть сынове церкви, никакой части въ Хрістѣ не имѣютъ, хотя и исповѣдуютъ Его, пока чистосердечно къ Нему не обратятся и похотей своихъ не оставятъ.

\paragraph*{§\:325.} Убо"=де хрістіанамъ, которые неисправно живутъ, Бога какъ Отца призывать, якоже истинные хрістіане призываютъ, невозможно? "--- Подлинно невозможно правильно и съ пользою ихъ, пока себе не исправятъ и истинно не покаются. Самъ разсуди, како безъ зазрѣнія совѣсти будутъ Бога, яко Отца, призывать, Который есть святый, праведный, истинный, благій, милостивый, щедрый, и проч., а нравами своими показуютъ противное, что они не сыны Его, но противники? Како блудники и прочіе сластолюбцы сказать могутъ Богу святому: \textit{Отче нашъ}, а сами отъ святыни и чистоты удаляются и подобятся скотомъ? Како неправедные, хищники, татіе, лихоимцы и прочіе, подобные симъ, дерзнутъ Богу \textit{праведному} глаголати: \textit{Отче нашъ}, "--- а сами неправдою своею отлучаются отъ Него и противятся Ему? Како льстецы и лукавцы Бога, \textit{Иже есть истина}, Отцемъ призовутъ, а сами подобятся отцу лжи, діаволу, \textit{иже есть отецъ лжи и во истинѣ не стоитъ}, по свидѣтельству Хрістову\footnote{Іоан.~8,~44.}? Тоежде разумѣти должно и о прочіихъ злонравныхъ хрістіанахъ, которые свойства новаго рожденія не показуютъ. "--- Убо"=де Богъ имъ не Отецъ? "--- Богъ, поелику Создатель есть, поелику \textit{солнце Свое сіяетъ на злыя и благія и дождитъ на праведныя и на неправедныя}\footnote{Матѳ.~6,~45.}, поелику \textit{хощетъ всѣмъ спастися, и въ разумъ истины пріити}\footnote{1~Тим.~2,~4.}, всѣхъ есть милостивѣйшій Отецъ; но \textit{собственно} есть тѣхъ только, кои вѣрою во Хріста Сына Божія и Духомъ отрождени суть, и почитаютъ Его яко Отца, боятся и любятъ Его, и нравами своими Ему, какъ Отцу своему, подобятся. Таковымъ и Хрістосъ повелѣлъ Бога Отцемъ призывати: \textit{Отче нашъ, Иже еси на небесѣхъ}\footnote{Матѳ.~6,~9.}! "--- Могутъ ли"=де хотя Господемъ своимъ нарицать? "--- Богъ какъ Отецъ есть, якоже сказано, тако и Господь есть всѣхъ, поелику всѣми владѣетъ, и надъ всѣми господствуетъ, и есть \textit{Царь царствующихъ и Господь господствующихъ}\footnote{1~Тим.~6,~15.}. Но собственно есть тѣхъ, которые Ему работаютъ, слушаютъ Его, повинуются святымъ повелѣніямъ Его. Того ради сіи только Его Господемъ своимъ нарицати и призывати могутъ, а оные правильно Господемъ своимъ называти Его не могутъ, понеже не слушаютъ Его и не работаютъ Ему. Како бо могутъ сказать: \textit{Господи мой и Царю мой}, а сами Господа и Царя того не слушаютъ? Рабъ бо господина своего слушаетъ, и подданный царю своему повинуется. Откуду и Хрістосъ таковымъ призывателямъ глаголетъ: \textit{что Мя зовете: Господи, Господи, и не творите, яже глаголю}\footnote{Лук.~6,~46.}? "--- Убо"=де они не раби Божіи? Суть раби Божіи, поелику власти Его подлежатъ, но неключимые, о каковыхъ сказано: \textit{неключимаго раба вверзите во тьму кромѣшнюю: ту будетъ плачь и скрежетъ зубомъ}\footnote{Матѳ.~25,~30.}. "--- Чіи"=де они собственно раби? Рабъ собственно есть того господина рабъ, которому работаетъ, и по имени бо своему рабъ значитъ работу, и рабъ отъ работы называется рабомъ; кому убо работаетъ и угождаетъ рабъ, того и рабъ есть. Они страстямъ и прихотямъ угождаютъ и работаютъ, убо и раби суть страстей своихъ. Послушай, что Хрістосъ глаголетъ о таковыхъ: \textit{всякъ, творяй грѣхъ, рабъ есть грѣха}\footnote{Іоан.~8,~34.}; и Апостолъ Петръ: \textit{имже кто побѣжденъ бываетъ, сему и работенъ есть}\footnote{2~Петр.~2,~19.}. Аще же кто работаетъ грѣху и страстямъ, тотъ Богу работати не можетъ, слѣдственно и рабомъ Его не можетъ быти. О семъ Хрістосъ свидѣтельствуетъ: \textit{никтоже можетъ двѣма господинома работати: либо единаго возлюбитъ, а другаго возненавидитъ, или единаго держится, о друзѣмъ же нерадити начнетъ. Не можете Богу работати и мамонѣ}\footnote{Матѳ.~6,~24.}. "--- Убо"=де имъ и молиться Богу невозможно? Подлинно невозможно, пока таковыми пребываютъ. "--- Что"=же"=де учинить должно имъ, чтобъ отъ тяжкія работы сея избавиться, и Богу живому работать и имя Его невозбранно призывать могли? \textit{Отвѣтъ}: Должно оставить прихоти свои и обратиться ко Хрісту вѣрою, и молить Его, чтобъ Онъ свободилъ ихъ отъ тяжкія работы тоя, Который единъ тако порабощенныхъ и плѣненныхъ освобождаетъ, по свидѣтельству Его: \textit{аще Сынъ вы свободитъ, воистину свободни будете}\footnote{Іоан.~8,~36.}, "--- и тако благодатію единороднаго Сына Божія возвратиться и пріитить къ небесному Отцу, якоже Самъ глаголетъ: \textit{никтоже пріидетъ къ Отцу, токмо Мною}\footnote{14,~6.}. Тогда Богъ будетъ Богъ ихъ, Господь, Царь и Отецъ ихъ; и они будутъ людіе Божіи, раби Господа вышняго и сынове Отца небеснаго, якоже Самъ милостивно обѣщается: \textit{и буду имъ Богъ, и тіи будутъ Мнѣ людіе}\footnote{2~Кор.~6,~16; Лев.~26,~12.}; и паки: \textit{и буду вамъ во Отца, и вы будете Мнѣ въ сыны и дщери, глаголетъ Господь Вседержитель}\footnote{2~Кор.~6,~18; Іер.~3,~19; Ос.~1,~11.}. О семъ, хрістіанине, ниже яснѣе увидишь. Только ты всегда въ памяти содержи сіе, что не тое должны дѣлать, что нынѣшній свѣтъ дѣлаетъ, но чему Божіе слово учитъ. Ибо соблазны день отъ дня умножаются въ мірѣ, и единъ отъ другаго несмысленно научается зла, и тако истинныхъ хрістіанъ число умаляется. Не вси бо тѣ хрістіане суть, которые крестомъ знаменуются, въ церковь ходятъ, имя Божіе и Хрістово исповѣдаютъ, и прочіе знаки хрістіанства показуютъ. Извѣстно знай, что много и безбожныхъ подъ сими знаками крыется. Безъ вѣры бо истинной и живой, какъ выше сказано и ниже еще скажется, истинный хрістіанинъ быть не можетъ.

\subsection[Глава 5-я. О молитвѣ или призываніи Бога.]{глава пятая.\\\bfseries О молитвѣ или призываніи Бога.}

\begin{quotation}\textit{Сице убо молитеся вы: Отче нашъ, Иже еси на небесѣхъ! да святится имя Твое, да пріидетъ царствіе Твое, да будетъ воля Твоя, яко на небеси, и на земли. Хлѣбъ нашъ насущный даждь намъ днесь, и остави намъ долги наша, якоже и мы оставляемъ должникомъ нашимъ: и не введи насъ въ напасть, но избави насъ отъ лукаваго. Яко твое есть царствіе и сила и слава во вѣки. Аминь!} учитъ Хрістосъ\footnote{Матѳ.~6,~9--13.}.\end{quotation}

\paragraph*{§\:326.} Молитва, или призываніе имени Божія надлежитъ къ должности хрістіанской троякой: къ должности къ Богу, должности къ себѣ самому, и должности къ ближнему. 1)~Къ должности нашей къ Богу надлежитъ: да молитвою и призываніемъ Его покажемъ нашу въ Него вѣру, что мы истинно вѣруемъ въ Него, инаго Бога кромѣ Его не знаемъ, всѣхъ благъ источника Его исповѣдуемъ, надѣемся на Него, какъ Отца истиннаго, Который чадъ Своихъ любитъ и просимое подаетъ; якоже дѣти плотскія не къ чужому отцу, но къ своему, отъ котораго родилися, съ прошеніемъ приходятъ, "--- якоже Апостолъ глаголетъ: \textit{како призовутъ, въ Негоже не вѣроваша}\footnote{Римл.~10,~14.}? И Самъ Господь Богъ глаголетъ: \textit{Азъ есмь Господь Богъ твой}\footnote{Исх.~20,~2.}, то есть, Мене единаго знай, исповѣдуй, на Мя надѣйся единаго, и всего добра отъ Мене ищи. \textit{Призови Мя въ день скорби твоея, и изму тя}\footnote{Пс.~49,~15.}. "--- 2)~Къ должности къ самимъ себѣ: понеже мы какъ въ тѣлесныхъ, такъ и духовныхъ скудны, бѣдны и окаянны; и ради того должно вѣрою искать всего добра отъ Подателя всѣхъ благъ, небеснаго Отца. "--- 3)~Къ должности, до ближняго касающейся, надлежитъ: понеже хрістіанская любовь требуетъ, чтобы мы и за ближняго нашего молилися, и какъ себѣ, такъ и ему просили добра у Бога нашего, якоже Апостолъ глаголетъ: \textit{молитеся другъ за друга, яко да исцѣлѣете}\footnote{Іак.~5,~16.}.

\paragraph*{§\:327.} Знаетъ Богъ и прежде прошенія нашего, чего мы требуемъ, якоже Хрістосъ глаголетъ: \textit{вѣсть Отецъ вашъ небесный, яко требуете сихъ всѣхъ}\footnote{Матѳ.~6,~32.}, то"=есть, пищи, питія и одежды и всѣхъ къ житію нужныхъ; и подаетъ благая Своя праведнымъ и грѣшнымъ, яко милостивъ и щедръ, \textit{яко солнце Свое сіяетъ на злыя и благія, и дождитъ на праведныя и на неправедныя}\footnote{5,~45.}. Такожде вѣдаетъ, что и духовная благая, какъ"=то: вѣра, благодать, миръ совѣстный и прочая, нужна намъ къ спасенію. Но намъ должно знать, что вся благая отъ Бога происходятъ, \textit{яко всякое даяніе благо и всякъ даръ совершенъ свыше есть, сходяй отъ Отца свѣтовъ}\footnote{Іак.~1,~17.}. И ради того ихъ вѣрою отъ Бога просить должно, да познаемъ и признаемъ, что отъ Бога все пріемлемъ; а пріемля благодарить Ему, яко Благодѣтелю, туне насъ милующему.

\paragraph*{§\:328.} Къ молитвѣ слѣдущія причины побуждаютъ всякаго вѣрнаго: 1)~Самъ Богъ \textit{повелѣваетъ} молитися и призывати Его: \textit{призови Мя въ день скорби твоея}\footnote{Пс.~49,~15.}. Просите, ищите, толцыте, глаголетъ Хрістосъ\footnote{Матѳ.~7,~7.}. И на прочіихъ мѣстахъ святаго Писанія таковоежде повелѣніе обрѣтается. 2)~\textit{Нужда} наша, какъ тѣлесная, такъ и духовная, убѣждаетъ насъ къ молитвѣ. Тѣлесная нужда во временныхъ и къ житію временному надлежащихъ; духовная, до спасенія душъ нашихъ касающаяся. 3)~Многіи обстоятъ насъ \textit{бѣды}, напасти, утѣсненія, скорби и искушенія, которыхъ сами, безъ помощи Божіей, которая молящимся подается, одолѣть не можемъ. 4)~Неизреченная \textit{польза}, отъ молитвы происходящая; яко молитва смиренная всего у милостиваго и благаго Господа испрашиваетъ, какъ на многихъ святаго Писанія мѣстахъ примѣры тому обрѣтаются. 5)~\textit{Великое} благости Божіей \textit{дѣло есть} сіе, что человѣкъ сподобляется съ непостижимымъ Богомъ "--- Отцемъ и Сыномъ и Святымъ Духомъ бесѣдовать, нужды свои Ему представлять и просить о нихъ, хвалить, пѣть и благодарить Ему, что посредствіемъ истинныя, вѣрныя и смиренныя молитвы бываетъ, якоже святый Златоустъ глаголетъ: «Егда молишися, не Богу ли бесѣдуеши, рцы ми? Егда чтеши, слышиши Того бесѣдующа тебѣ»\footnote{Бес.~6"~я на 1"~е посл. къ Сол.}. За счастіе свое вмѣняетъ человѣкъ, что съ земнымъ монархомъ разглагольствуетъ: какъ же должно за великое благополучіе хрістіанамъ вмѣнять того, что къ бесѣдѣ съ небеснымъ Царемъ, преблагимъ и милостивымъ Богомъ, допущаются, не токмо допущаются, но и привлекаются къ тому повелѣніемъ и обѣщаніемъ услышанія прошенія ихъ! Отъ сего, возлюбленный хрістіанине, \textit{вкуси и виждь, коль благъ Господь нашъ}\footnote{Пс.~33,~9.}. Яко не токмо допущаетъ насъ недостойныхъ и грѣшныхъ къ себѣ, но и привлекаетъ; и не токмо привлекаетъ, но и научаетъ насъ, какъ приходити къ Нему и молитися, якоже молитва Господня показуетъ. \textit{Сице молитеся}, глаголетъ Хрістосъ\footnote{Матѳ.~6,~9.}. А какъ \textit{сице}? "--- \textit{Отче нашъ!} Отцемъ"=де называйте Бога. "--- Кто мы? "--- Человѣцы, прогнѣвавшіе Его, раби неключимые, грѣшники, земля и пепелъ. О премилосердый Боже нашъ! Вездѣ срѣтаетъ насъ благость Твоя, куды мы ни обратимся; и на что ни посмотримъ, вездѣ находимъ случай удивляться превеликому благости Твоея богатству. 6)~Возбуждаетъ къ молитвѣ безсумнительное услышаніе обѣщанія: \textit{аминь, аминь глаголю вамъ, яко елика аще чесо просите отъ Отца во имя Мое, дастъ вамъ}, глаголетъ Хрістосъ "--- Истина\footnote{Іоан.~16,~23.}.

\paragraph*{§\:329.} Чего у милосердаго Отца небеснаго должно просить намъ? 1)~Всего, что къ славѣ имени Его и къ нашей пользѣ, какъ тѣлесной и временной, такъ духовной и вѣчной, касающагося должни просить; что въ святомъ Его словѣ повелѣно, обѣщано, и что воля Его святая хощетъ, въ томжде святомъ Писаніи открытая. Къ \textit{духовной пользѣ} надлежатъ тая благая, которая къ душевному спасенію и полученію вѣчнаго живота нужна суть, какъ"=то: вѣра, отпущеніе грѣховъ, благодать Его, обновленіе сердца, постоянство въ вѣрѣ, терпѣніе въ скорбяхъ, бѣдахъ и тѣснотахъ, и проч. Къ \textit{тѣлесной пользѣ} надлежитъ все тое, что къ тѣлу нашему, къ содержанію житія сего нужно, какъ"=то: благоцвѣтущее здравіе, разумъ, довольство нужныхъ, пища, одежда, покой, и проч. "--- 2)~Духовная благая должно просить подобающимъ образомъ, и ожидать безъ всякаго сумнѣнія, яко необходимо нужная къ истинному нашему блаженству. Ибо Самъ Богъ сія обѣщалъ намъ подать безъ всякаго изъятія, когда надлежащимъ образомъ просимъ. \textit{Всякъ, иже призоветъ имя Господне, спасется}\footnote{Римл.~10,~13.}. И клятвою подтвердилъ, что \textit{не хощетъ смерти грѣшника}\footnote{Іез.~33,~11.}. \textit{Аще вы, зли суще, умѣете даянія блага даяти чадомъ вашимъ: кольми паче Отецъ, Иже съ небесе, дастъ Духъ Святъ просящимъ у Него}, глаголетъ Хрістосъ\footnote{Лук.~11,~13.}, \textit{Иже всѣмъ человѣкомъ хощетъ спастися, и въ разумъ истины пріити}\footnote{1~Тим.~2,~4.}. Худо убо такъ говорить: Господи, аще хощеши, и мнѣ полезно, остави ми согрѣшенія, подаждь ми благодать Твою, и проч.; ибо извѣстно безъ сумнѣнія, что и Богъ хощетъ, и мнѣ полезно, или паче необходимо нужно къ спасенію отпущеніе грѣховъ, и проч. "--- 3)~Временныхъ благихъ, напр. здравія, продолженія жизни, благополучія и прочіихъ, не просто просить должно, но съ приложеніемъ сего: \textit{ежели волѣ Его святой угодно, и намъ просящимъ полезно. Ибо о чесомъ помолимся, якоже подобаетъ}, не вѣмы, глаголетъ Апостолъ\footnote{Римл.~8,~26.}. Часто бо о томъ просимъ Бога, что намъ вредно и волѣ Его святой противно. Якоже дѣлаютъ дѣти малыя, когда не хотятъ въ баню итить, и отъ нея у матерей своихъ просятся, которая имъ много полезна: тако дѣлаютъ тѣ, которые просятъ у Бога избавленія изъ"=подъ креста, скорби, болѣзни и прочаго бѣдствія, не зная, что много полезнѣйше имъ быть въ неблагополучіи, нежели въ благополучіи. Что вредно и что полезно немощному, лѣкарь знаетъ; и часто немощный у лѣкаря проситъ яда вмѣсто лѣкарства, но благоразумный лѣкарь того не даетъ. Мы вси предъ Богомъ какъ немощніи, и какъ младенцы, не знающіи пользы нашея; и часто просимъ, что намъ смерть душевную можетъ принесть, чего Богъ, какъ милосердый Отецъ и Благопромыслитель, не хощетъ, и того ради, просящимъ сынамъ Своимъ вмѣсто хлѣба \textit{камня}, вмѣсто рыбы \textit{зміи}, и вмѣсто яица \textit{скорпіи не хощетъ подати}\footnote{Лук.~11,~11.}. Того ради всегда временнаго добра должно просить тако: \textit{Господи, аще волѣ Твоей святой угодно и мнѣ полезно, подаждь желаемое мною: аще ли ни, буди воля Твоя}. Сего образъ показалъ намъ Сынъ Божій, егда предъ страданіемъ Своимъ вольнымъ молился ко Отцу Своему небесному тако: \textit{Отче Мой! аще возможно есть, да мимоидетъ отъ Мене чаша сія: обаче не якоже Азъ хощу, но якоже Ты}\footnote{Матѳ.~26,~39.}.

\paragraph*{§\:330.} Молитися, или призывати Бога должно намъ: 1)~Во имя Хріста Іисуса Сына Божія, якоже Самъ глаголетъ: \textit{аминь, аминь глаголю вамъ, яко елика аще чесо просите отъ Отца во имя Мое, дастъ вамъ}\footnote{Іоанн.~16,~23.}. То"=есть, молитися и просити Бога, не на свою правду и дѣла надѣяся, но на правду Хрістову и заслуги Его высочайшія, и ради Его неповиннаго страданія и смерти, своей повинности просить прощенія, по подобію мытаря, молившагося: \textit{Боже, милостивъ буди мнѣ грѣшнику}\footnote{Лук.~18,~13.}! \textit{Зане тѣмъ имамы приведеніе обои во единомъ Дусѣ ко Отцу}\footnote{Еф.~2,~18.}. И: \textit{никтоже пріидетъ ко Отцу, токмо Мною}, глаголетъ Самъ Хрістосъ\footnote{Іоан.~14,~6.}. Ибо какое ни обѣщалъ Богъ подать намъ благословеніе, о имени Его обѣщалъ, \textit{яко въ Немъ благословятся вся колѣна земная}, глаголетъ Писаніе\footnote{Пс.~71,~17; Гал.~3,~8.}. И тако съ сею надеждою приступать ко Отцу небесному, и ради Единороднаго Сына отъ Отца милосердія просить и ожидать безъ сумнѣнія. "--- 2)~Истинною вѣрою. \textit{Вся, елика аще воспросите въ молитвѣ, вѣрующе, пріимете}, глаголетъ Хрістосъ\footnote{Матѳ.~21,~22.}. Вѣра утверждается обѣщаніемъ неложнымъ, всемогуществомъ, истиною, милосердіемъ и премудростію обѣщающаго Бога и заслугами Хрістовыми. А понеже вѣра есть даръ Святаго Духа, Который не обитаетъ въ душѣ, оскверненной грѣхами, того ради \textit{да отступитъ отъ неправды всякъ именуяй имя Господне}\footnote{2~Тим.~2,~19.}. Кто хощетъ къ Богу приступать и желаемое получать, должно тому отступить отъ грѣховъ, и вѣрою во Хріста Сына Божія примириться съ Богомъ. Блудъ и всякая нечистота хищеніе, воровство, вражда, злоба, сребролюбіе, лихоиманіе, клевета, срамословіе, проклинаніе, піянство, ложь, лесть, лукавство и прочая симъ подобная заключаютъ двери къ молитвѣ, пока человѣкъ въ нихъ пребываетъ. Да отступитъ убо отъ всѣхъ сихъ, кто хощетъ призывати имя Господне. «Подобаетъ, глаголетъ святый Златоустъ, молитися, но сквернами незамаранному. Что, аще впаду въ погрѣшеніе, глаголеши? Очисти себе. Какъ? Плачи, стени, подай милостыню, примирися съ тѣмъ, егоже обидѣлъ еси; очисти языкъ, да не болѣе прогнѣваеши Бога, и прочая, да не и къ тебѣ речетъ Господь: \textit{аще умножите моленіе, не услышу васъ}»\footnote{Ис.~1,~15; Бес.~51"~я на Матѳ.}. "--- 3)~Молитися должно не токмо языкомъ, но и сердцемъ, то"=есть, отъ сердца должна происходити молитва. «Изъ глубины, глаголетъ святый Златоустъ, призывай Бога, якоже глаголетъ: \textit{изъ глубины воззвахъ къ Тебѣ, Господи}; изъ внутренности изведи гласъ твой». И мало спустя: «Не человѣку бо молишися, но Богу вездѣсущему, и прежде гласа слышащему, и тайная сердца вѣдущему»\footnote{Бес.~19"~я на Матѳ.}. Въ молитвѣ бо объявляемъ Богу сердечная наша желанія, и потому должно и умомъ думать, и на сердцѣ имать и желать тое, что языкомъ и словами произносимъ. Иначе никакой не будетъ молитвы но праздныя только слова. Сего ради должно намъ себе молитвѣ пріобучать, и о семъ Господа Самаго просить, чтобы Духомъ своимъ Святымъ научилъ насъ молитися, якоже апостолы просили Его о семъ: \textit{Господи, научи ны молитися}\footnote{Лук.~11,~1.}. Отсюду заключается, что истинная молитва безъ дѣйствія Святаго Духа не можетъ быти. Егда бо Сей благій Утѣшитель коснется сердца нашего, тогда, яко отъ ѳиміама зажженнаго куреніе, отъ сердца произыдетъ воздыханіе, святое желаніе и истинная молитва, и приближится къ престолу благодати.

\paragraph*{§\:331.} Молитися должно 1)~\textit{на всякое время}\footnote{Еф.~6,~18.}, \textit{всегда}\footnote{Лук.~18,~1.}, \textit{непрестанно}\footnote{1~Солун.~5,~17.}. Молитися же всегда и непрестанно не тое значитъ, чтобы всегда читать псалмы или молитвы написанныя и кланяться, сіе бо невозможное дѣло есть; ибо надобно всякому хрістіанину дѣло по званію своему дѣлать, такожде плоть утружденная требуетъ упокоенія, сна, и прочая. Но значитъ тое, чтобы часто, во всякомъ начинаніи и дѣлѣ умъ, сердце и воздыханіе къ Богу возводить и просить тѣмъ у Него милости, помощи и защищенія. Причина тому сія есть: понеже на всякое время сатана съ своими злыми слугами навѣтуетъ намъ; такожде плоть всегда похотствуетъ на духъ, "--- которымъ врагамъ сами противитися не можемъ; того ради должно молитвою и отъ той помощію Божіею противу ихъ вооружаться, стоять и крѣпиться, якоже Псаломникъ глаголетъ: \textit{къ Тебѣ возведохъ очи мои, живущему на небеси. Се, яко очи рабъ въ руку господей своихъ, яко очи рабыни въ руку госпожи своея: тако очи наши ко Господу Богу нашему, дондеже ущедритъ ны}\footnote{Пс.~122,~1 и 2.}. "--- 2)~\textit{На всякомъ мѣстѣ} молитися. Тако Апостолъ глаголетъ: \textit{хощу, да молитвы творятъ мужіе на всякомъ мѣстѣ, воздѣюще преподобныя руки безъ гнѣва и размышленія}\footnote{1~Тим.~2,~8.}. 3)~Стоя, сѣдя, ходя, почивая, дѣлая что руками, въ уединеніи и въ собраніи молитися возможно. Ибо всегда на всякомъ мѣстѣ и во всѣхъ нашихъ дѣлахъ, и въ яденіи и питіи, и въ бесѣдахъ богоугодныхъ можемъ умъ нашъ и сердце къ Богу возводить, нужды наша со смиреніемъ и вѣрою предлагать, и милости у Него просить, и глаголати: \textit{Господи помилуй!} Тако Моисей посредѣ многочисленнаго народа, который изъ Египта вывелъ, и увидѣлъ ихъ въ тѣснотѣ, умомъ и сердцемъ воспѣлъ къ Богу, хотя молитва Его словесная и не изображается въ Писаніи, Богъ бо глаголалъ ему: \textit{что вопіеши ко Мнѣ?}\footnote{Исх.~14,~15.} Езекія, благочестивый царь, \textit{на одрѣ лежа, молился, и услышалъ его Богъ}\footnote{Ис.~38,~2,~4, и слѣд.}. Іона \textit{во чревѣ китовѣ} молился, и услышанъ отъ Бога\footnote{Іон.~2,~2--11.}. Тріе отроки \textit{въ пещи огненной} молилися, и спасены суть\footnote{Дан.~3,~24; 25--49,~50--94 и проч.}. Богъ бо не на внѣшній знакъ и расположеніе тѣла, какъ оно имѣется, смотритъ, но на сердце, смиреніе и вѣру и желаніе сердечное. Откуду написано: \textit{желаніе убогихъ}, то"=есть смиренныхъ, \textit{услышалъ еси, Господи, уготованію сердца ихъ внятъ ухо Твое}\footnote{Пс.~9,~38.}.

\paragraph*{§\:332.} \textit{Внѣшніе знаки молитвы} въ святомъ Писаніи примѣчаются сіи: 1)~колѣнопреклоненіе\footnote{3~Цар.~8,~54; Дѣян.~7,~60; Еф.~3,~14.}; 2)~воздѣяніе рукъ\footnote{Исх.~9,~22,~23,~29 и 33; 1~Тим.~2,~8; Пс.~142,~6.}; 3)~возведеніе очесъ на небо\footnote{Іоан.~11,~41; 17,~1; Пс.~120,~1; 122,~1,~2 и проч.}. Однакожъ сіи знаки безъ внутренняго истиннаго благодѣянія и усердія ничего не пользуютъ, яко Богъ на сердце, а не на внѣшность и наружность, смотритъ, и не внѣшняго, но внутренняго, сердечнаго гласа слушаетъ. Сего ради, хотя кто читаетъ много молитвъ, или псалмовъ, или много поклоновъ отправляетъ, но безъ разума, вниманія и сердечнаго смиренія и усердія, таковый никогда не молится. Аще бо когда къ человѣку прошеніе наше предлагаемъ, тутъ умъ, сердце и все наше усердіе имѣемъ, кольми паче сіе должно показывать, когда въ молитвѣ предъ Богомъ, \textit{сердца и утробы испытующимъ}\footnote{Пс.~7,~10.}, стоимъ и прошеніе наше простираемъ; надобно тутъ неотмѣнно весь умъ и сердце собрать, сколько возможно человѣку, плотію немощною обложенному.

\paragraph*{§\:333.} Молитися по хрістіанской должности не токмо за себе самого но и \textit{за вся человѣки}: 1)~\textit{за царя, и за всѣхъ, иже во власти суть: да тихое и безмолвное житіе поживемъ во всякомъ благочестіи и чистотѣ}\footnote{1~Тим.~2,~1 и 2.}. 2)~За проповѣдниковъ слова Божія, пастырей и учителей, \textit{да слово Господне течетъ и славится}\footnote{Сол.~3,~1.}; да въ житіи и въ званіи своемъ будутъ исправны и непорочны; да право учатъ, и, что учатъ, того образъ на \textit{себѣ} показуютъ. 3)~Другъ за друга, якоже Апостолъ увѣщаваетъ: \textit{молитеся другъ за друга, яко да исцѣлѣете}\footnote{Іак.~5,~16.}. Откуду и Хрістосъ повелѣлъ молитися небесному Отцу вообще, какъ единымъ гласомъ, всѣмъ хрістіаномъ, и глаголати: \textit{Отче нашъ, Иже еси на небесѣхъ}\footnote{Лук.~11,~2; Матѳ.~6,~9.}. 4)~Не токмо за братію и друговъ, но и за враговъ молитися хрістіанамъ должно. \textit{Молитеся за творящихъ вамъ напасть и изгонящія вы}, глаголетъ Хрістосъ\footnote{5,~44.}, \textit{да будете}, придаетъ далѣе, \textit{сынове Отца вашего, Иже есть на небесѣхъ, яко солнце свое сіяетъ на злыя и благія, и дождитъ на праведныя и на неправедныя}\footnote{Ст.~45.}. Тако молился Самъ Хрістосъ за враговъ своихъ: \textit{Отче, отпусти имъ}\footnote{Лук.~23,~34.}. И святый Стефанъ: \textit{Господи, не постави имъ грѣха сего}\footnote{Дѣян.~7,~60.}. "--- О чемъ молитися, научаетъ Господня молитва: \textit{Отче нашъ}, о которой въ слѣдующемъ параграфѣ предлагается съ краткимъ изъясненіемъ.

\paragraph*{§\:334.}

\subsubsection*{Молитва Господня:}

\begin{quotation}

\textit{Отче нашъ, Иже еси на небесѣхъ!}

\end{quotation}

Сей стихъ есть предисловіе молитвы, отъ котораго научаемся: 1)~что Богъ есть \textit{истинный Отецъ хрістіанъ, и они суть сынове Божіи вѣрою о Хрістѣ Іисусѣ}\footnote{Гал.~3,~26.}. Слѣдственно, какъ Отца, съ упованіемъ должно Его призывать имъ, якоже дѣти плотскія родителей своихъ призываютъ и руки свои къ нимъ во всякихъ нуждахъ простираютъ. 2)~Богъ хрістіанамъ есть \textit{единъ Отецъ}; слѣдственно они суть братія между собою, яко единаго Отца имущіи. 3)~Должны хрістіане, яко братія духовные, любовь между собою имѣти. 4)~Другъ за друга къ Богу молитися, и аки единъ гласъ отъ сердца испущати къ небесному своему Отцу: \textit{Отче нашъ!} «Научаетъ симъ, глаголетъ святый Златоустъ на сіе слово, общую за братію творити молитву. Не глаголетъ бо: Отче мой, но \textit{Отче нашъ}, "--- за общее тѣло возсылать молитвы, и не свою только пользу, но и ближняго смотрѣти вездѣ». 5)~Когда хрістіане суть братія о Бозѣ, то вси едину честь и славу въ семъ имѣютъ, вси, глаголю, господа и раби, славные и неславные, богатые и нищіе, сановитые и подлые, яко братія; и того ради не должны другъ друга презирать, яко \textit{вси едино суть о Хрістѣ Іисусѣ}\footnote{Гал.~3,~28.}. «Тако, глаголетъ святый Златоустъ на сіе мѣсто, неравенство отъ насъ выводитъ, и великое царя съ нищимъ равночестіе показуетъ». 6)~Какъ опасно отъ грѣховъ берещися, паче же добрыми нравами Богу подобитися должны, яко сынове отцу своему, когда безъ зазрѣнія совѣсти хотятъ Его призывати и нарицати Отцемъ. 7)~Видно отъ сего, что хрістіанинъ неисправный, пока не исправитъ и не очиститъ себе истиннымъ покаяніемъ, не можетъ Бога съ пользою своею призывати, много паче Отцемъ нарицати и сію молитву глаголати. Надобно неотмѣнно оставить грѣхи и прихоти плотскія, и покаятися, и отступити отъ неправды, по ученію Апостола: \textit{да отступитъ отъ неправды всякъ, именуяй имя Господне}\footnote{2~Тим.~2,~19.}. Како бо можетъ Ему сказать: \textit{Отче нашъ}; а самъ скотамъ, или діаволу нравами своими подобится? Ибо которые Бога Отцемъ нарицаютъ, и тако молятся: \textit{Отче нашъ}, должны быть сынами Божіими, а въ сынахъ должны быть свойства, подобныя отцу. Должно убо неотмѣнно, по подобію блуднаго сына, пріитить въ себе, обратиться и возвратиться ко Отцу, и предъ Нимъ со смиреніемъ признать свой грѣхъ: \textit{Отче! согрѣшихъ на небо и предъ Тобою, и уже нѣсмь достоинъ нарещися сынъ Твой}\footnote{Лук.~15,~17--21.}, "--- и впредь не отлучаться отъ Отца небеснаго, но Ему отъ чиста сердца работать, и съ Его домашними, то"=есть, истинными хрістіанами, участіе имѣть, и тако съ ними купно сей общій испущати гласъ въ Нему: \textit{Отче нашъ!} 8)~Глаголется: \textit{на небесѣхъ} "--- не яко Богъ на небесѣхъ заключается, но молящагося отъ земли отводитъ и къ небеснымъ обителямъ возводитъ, глаголетъ Златоустъ; такожде понеже тамо отечество есть избранныхъ, и Богъ въ славѣ Своей тамо Себе показываетъ, и души святыхъ Своихъ увеселяетъ. Иначе Богъ \textit{существенно} на всякомъ мѣстѣ, на небеси и на земли есть, якоже Псаломникъ поетъ: \textit{Богъ нашъ на небеси и на земли}\footnote{Пс.~113,~11.}. "--- Посмотримъ далѣе, о чемъ молитися хрістіанамъ повелѣлъ Хрістосъ.

\begin{quotation}

\textit{Да святится имя Твое}.

\end{quotation}

1) \textit{Да святится имя}, то"=есть, да славится имя Твое, глаголетъ Златоустъ. 2)~Сіе значитъ, чтобы мы ничего инаго, какъ только славы Божіей искали. Достойна молитва того, глаголетъ Златоустъ, который Бога Отцемъ нарицаетъ, ничего не проситъ прежде славы Божіей, но всему предпочитать хвалу Ему. 3)~Имя Божіе свято и славно есть безъ нашего прославленія, но мы должны стараться, чтобы и въ насъ славилося. 4)~Имя Божіе прославляется, когда слово Его чисто проповѣдуется, и мы по слову Его житіе наше исправляемъ, якоже глаголетъ Хрістосъ: \textit{тако да просвѣтится свѣтъ вашъ предъ человѣки, яко да видятъ ваша добрая дѣла, и прославятъ Отца вашего, Иже есть на небесѣхъ}\footnote{Матѳ.~5,~16.}. 5)~Сего сами, безъ Божіей благодати, чинить не можемъ; того ради просимъ небеснаго Отца, да тое милостивно намъ подастъ. 6)~Отсюду примѣчается немощь наша, что мы безъ Бога никакого добра не можемъ творити, якоже глаголетъ Хрістосъ: \textit{безъ Мене не можете творити ничесоже}\footnote{Іоан.~15,~5.}. 7)~Аще просимъ, дабы святилося и славилося имя Божіе, то не должно намъ нашему имени славы искать отъ добрыхъ дѣлъ. 8)~Кто иначе учитъ, или живетъ, какъ слово Божіе научаетъ, тотъ имени Божія не прославляетъ. Сего ради, чтобы до того самъ Богъ не допустилъ насъ, молимся Ему: \textit{да святится имя Твое}.

\begin{quotation}

\textit{Да пріидетъ царствіе Твое!}

\end{quotation}

1) По ученію святаго Златоустаго, царствіе здѣ разумѣется небесное, "--- царствіе, котораго ожидаютъ истинные хрістіане, и молятся: \textit{да пріидетъ царствіе Твое}. И сіе, рече, благоразумнаго сына слово есть, что не прилѣпляется видимымъ, ниже за велико почитаетъ настоящая, но къ отцу спѣшитъ и къ будущимъ стремится. 2)~Кто царствія небеснаго желаетъ и молится тако: \textit{да пріидетъ царствіе Твое}, тому надобно презирать земное царствіе, славу, честь, утѣхи и богатство. Иначе напрасно глаголетъ и молится: \textit{да пріидетъ царствіе Твое}. Надобно бо прежде презрѣть земная, и тогда небесныхъ желати. 3)~Надобно ему имѣть чистую совѣсть, или очистить тую покаяніемъ, и тогда молиться: \textit{да пріидетъ царствіе Твое}. Иначе не можетъ желать того и просить, когда совѣсть, грѣхами оскверненная, судомъ Божіимъ и вѣчною мукою грозитъ. Како бо можетъ молитися и сказать: \textit{да пріидетъ царствіе Твое}, "--- а въ совѣсти ожиданіемъ вѣчныя муки смущается? Неотмѣнно убо требуется, чтобы оставилъ грѣхи, и тыя покаяніемъ, жалѣніемъ и вѣрою очистилъ, кто хощетъ и проситъ, чтобы царствіе Божіе пришло. Царствіе бо Божіе праведнымъ и святымъ уготовано, и пріидетъ неотмѣнно, котораго напрасно ожидаетъ грѣхами оскверненный. Отъ совѣсти чистой, глаголетъ Златоустъ, сіе желаніе происходитъ, и души, отъ земныхъ удаленной.

\begin{quotation}

\textit{Да будетъ воля Твоя, яко на небеси, и на земли}.

\end{quotation}

1) Воля Божія бываетъ и безъ нашего прошенія; того ради не просимъ, чтобы Богъ дѣлалъ, что хощетъ; но просимъ, чтобы мы могли дѣлать тое, что воля Его хощетъ. 2)~Отсюду видимъ, что воли Божіей безъ Бога творить не можемъ. 3)~Воля Божія исполняется, когда благочестіе цѣло хранимъ, и до конца въ томъ пребываемъ, по реченному: \textit{буди вѣренъ даже до смерти}\footnote{Апок.~2,~10.}; такожде, когда грѣшники отъ грѣховъ отстаютъ и каются. Богъ \textit{хощетъ всѣмъ спастися и въ разумъ истины пріити}\footnote{1~Тим.~2,~4.}. Того ради, чтобы сіе было, просимъ Бога: \textit{да будетъ воля Твоя}. 4)~\textit{Яко на небеси, и на земли}. То"=есть, якоже ангели на небеси исполняютъ волю Божію, и живутъ между собою любовно, мирно, согласно: тако и мы, на землѣ подражая имъ, да возможемъ творить волю Его, и жить свято, чисто, мирно, любовно и согласно. 5)~Когда молимся: \textit{да будетъ воля Твоя}, то своей воли отрещися и оставить ее должны.

\begin{quotation}

\textit{Хлѣбъ нашъ насущный даждь намъ днесь}.

\end{quotation}

1) Хлѣбъ насущный, по Златоустаго разумѣнію, разумѣется \textit{повседневный}. 2)~Здѣ не только разумѣется хлѣбъ, но и все къ житію сему временному нужное, напр. питіе, одѣяніе, покой, или домъ и прочая, какъ нѣкоторые толкуютъ. 3)~Не просимъ богатства, но просимъ нужнаго къ содержанію житія сего. Не о деньгахъ, не о роскоши, ни многоцѣнномъ одѣяніи, ни о иномъ подобномъ молитися повелѣлъ, но о хлѣбѣ только, и о хлѣбѣ повседневномъ, да о утреннемъ не печемся, глаголетъ Златоустъ. 4)~Отсюду послѣдуетъ, что хрістіанину не должно пещись о богатствѣ, многоцѣнномъ одѣяніи, богатыхъ домахъ, богатой пищѣ и прочемъ, сему подобномъ. Ибо хрістіанинъ всегда готовъ долженъ быть, когда его позоветъ Господь его, и тогда все сіе принужденъ будетъ оставить. Зоветъ же Господь всякаго къ Себѣ чрезъ смерть. Хощетъ, глаголетъ Златоустъ, всегда намъ готовыми быть и тѣмъ довольствоваться, что естеству нашему нужно. 5)~Когда молимся: \textit{хлѣбъ даждь намъ}, исповѣдуемъ, что мы нищіи, убогіи и бѣдныи, и потому всего у Бога просить, и что ни имѣемъ, Его благости приписывать должны, якоже Псаломникъ поетъ: \textit{очи всѣхъ на Тя уповаютъ, и Ты даеши имъ пищу во благовременіи; отверзаеши Ты руку Твою, и исполняеши всякое животно благоволенія}\footnote{Пс.~144,~15 и 16.}. 6)~Когда глаголемъ: \textit{хлѣбъ нашъ даждь}, тѣмъ показуемъ, что не токмо о своемъ пропитаніи, но и о прочіихъ просимъ отъ любви хрістіанской. Любовь бо хрістіанская требуетъ, чтобы мы не токмо о себѣ, но и о ближнихъ нашихъ старалися. 7)~Извѣстно сіе, что Богъ, яко щедръ, не токмо хрістіанамъ, но и незнающимъ Его подаетъ временная благая; но хрістіанамъ должно всего вѣрою просити у Него, яко сынамъ у отца, и тѣмъ показывать, что все тое есть Божіе добро, что ни имѣютъ нужное къ житію, и тако, пріемля благодѣяніе, благодарить Благодѣтелю.

\begin{quotation}

\textit{И остави намъ долги наша, якоже и мы оставляемъ должникомъ нашимъ}.

\end{quotation}

1) Долги здѣ разумѣются \textit{грѣхи} "--- слова, дѣла и помышленія, закону Божію противныя. Откуду Лука святый въ своемъ благовѣстіи написалъ: \textit{остави намъ грѣхи наша}\footnote{Лук.~11,~4.}. Называются же грѣхи долгами того ради, что какъ въ гражданствѣ бываетъ, что долги обдолжаютъ должника къ отдачѣ заимодавцу, и когда не отдаетъ долговъ должникъ, то въ темницу за долги заключается и держится, доколѣ не отдастъ долговъ: тако и грѣхи обдолжаютъ насъ къ удовлетворенію правдѣ Божіей (всякій бо грѣхъ бываетъ противу правды Божіей), и когда не имѣемъ чѣмъ удовлетворить, то заключаютъ насъ въ вѣчную темницу. 2)~Сихъ долговъ заплатить сами не можемъ чрезъ себе; того ради прибѣгаемъ къ заслугамъ Хрістовымъ и милосердію Божію, чтобы ихъ туне намъ оставилъ, и просимъ Его о семъ: \textit{остави намъ долги наша}. 3)~Когда просимъ тако: \textit{остави намъ долги наша}, чрезъ сіе показуется, что мы не токмо за себе, но и другъ за друга молиться, и другъ другу отпущенія грѣховъ просить должны. 4)~Симъ словомъ: \textit{остави намъ долги наша}, показуется ясно, что и самые святіи согрѣшаютъ, и потому должны себѣ отпущенія грѣховъ просить, якоже написано: \textit{за то}, то"=есть, за оставленіе грѣховъ, \textit{помолится къ Тебѣ всякъ преподобный во время благопотребно}\footnote{Пс.~31,~6.}. Молитва бо сія есть святыхъ; ибо кто Бога Отцемъ истинно и отъ сердца нарицаетъ, надобно тому быть святымъ и чадомъ Божіимъ, и Духа Божія въ себѣ живущаго имѣти, о Немъ же вопіютъ вѣрніи: \textit{Авва Отче}\footnote{Гал.~4,~6.}! 5)~Грѣхи здѣ разумѣются не тыя, которые отъ произволенія и предразсужденія бываютъ, какъ"=то: блудъ, воровство, хищеніе, злоба, лукавство, лесть и прочая, "--- отъ сихъ грѣховъ удаляются святіи, "--- но разумѣются немощи, которыхъ ради слабости своей уберещися не могутъ, хотя и тщатся о томъ. О сихъ грѣхахъ молятся святіи: \textit{Отче! остави намъ долги наша}. 6)~Глаголется: \textit{якоже и мы оставляемъ должникомъ нашимъ}. Симъ словомъ научаемся, чтобы мы и сами оставляли грѣхи ближнимъ нашимъ, когда отъ Бога просимъ и получаемъ оставленіе грѣховъ. Онъ намъ отъ милосердія прощаетъ грѣхи, "--- и мы, подражая Ему, отъ милосердія должны прощать грѣхи братіи нашей. Когда оставляемъ грѣхи братіи нашей, "--- оставляетъ и Богъ намъ грѣхи наши; не оставляемъ мы, "--- не оставляетъ и намъ Богъ, якоже Хрістосъ придаетъ: \textit{аще отпущаете человѣкомъ согрѣшенія ихъ, отпуститъ и вамъ Отецъ вашъ небесный. Аще ли не отпущаете человѣкомъ согрѣшенія ихъ, ни Отецъ вашъ отпуститъ согрѣшеній вашихъ}\footnote{Матѳ.~6,~5 и 6.}, "--- что и притчею о царѣ и должникѣ показуется\footnote{18,~23--35.}. 7)~Кто отъ грѣшниковъ, которые во грѣхахъ находятся и не оставляютъ ихъ, хощетъ молитися и грѣховъ оставленія у Бога просити, тому должно прежде самому оставить грѣхи своя, и тако оставивши просить Бога, чтобы благодатію Своею оставилъ ихъ, и отъ казни, которой грѣхи грѣшника предаютъ, избавилъ его. Ибо не оставляетъ и Богъ намъ грѣховъ, когда мы сами ихъ не оставляемъ. А когда намъ грѣхи не оставляются, то не иное что оттуду послѣдуетъ, какъ казнь за грѣхи, якоже случается должнику, которому долга заимодавецъ не оставляетъ; ибо заключаетъ его въ темницу и держитъ, доколѣ отдастъ долгъ свой: тако и грѣшникъ, которому грѣхи не оставляются, заключится судомъ Божіимъ въ вѣчную темницу, и будетъ правдѣ Божіей за долги грѣховные платить казнію, но никогда не заплатитъ. За прогнѣваніе и оскорбленіе \textit{вѣчнаго и безконечнаго} Бога праведно воздается \textit{вѣчное и безконечное} наказаніе. Ибо правда Божія того требуетъ, чтобы грѣшникъ непокаявшійся достойное воспріялъ наказаніе. Но за разореніе вѣчнаго закона Божія, и тѣмъ оскорбленіе безконечнаго величества Божія ничѣмъ инымъ человѣку удовлетворить невозможно, какъ вѣчною казнію и страданіемъ. Что же кающимся, престающимъ грѣшить, жалѣющимъ за грѣхи, и просящимъ отпущенія, грѣхи отъ Бога оставляются, то дѣлаетъ Богъ отъ единаго милосердія Своего и ради заслугъ Хрістовыхъ, ради которыхъ въ словѣ Своемъ обѣщалъ всѣмъ кающимся и вѣрующимъ во Хріста подать отпущеніе грѣховъ, якоже сіе милосердіе Божіе изображается во Пс.~102: \textit{Благослови душе моя Господа}. Но хотящему каятися, и ради страданій и смерти Хрістовой получить отпущеніе грѣховъ, должно неотмѣнно оставить грѣхи свои. Ибо гдѣ нѣтъ оставленія грѣховъ, тамо нѣтъ истиннаго покаянія; гдѣ нѣтъ истиннаго покаянія, тамо заслуги Хрістовы мѣста не имѣютъ; гдѣ заслуги Хрістовы мѣста не имѣютъ, тамо нѣтъ отпущенія грѣховъ; гдѣ нѣтъ отпущенія грѣховъ, тамо слѣдуетъ осужденіе и вѣчное за грѣхи во адѣ страданіе. И тако грѣшникъ, когда не оставятся ему грѣхи (не оставятся же ему, яко самъ ихъ оставить не хощетъ), за всѣ тѣ грѣхи, яко за долги должникъ, которыми предъ Богомъ обдолжился, истязанъ будетъ; и понеже не имѣетъ чѣмъ искупитися, будетъ \textit{вѣчною} смертію удовлетворяти. Надобно убо хотящему каятися и тако спастися, отъ грѣховъ отстать и прибѣгнуть къ милосердію Божію, и исповѣдать себе со смиреніемъ и жалѣніемъ виноватымъ предъ Богомъ, прощенія просить ради изліянной крови Хрістовой, которая за всякаго грѣшника изліянна есть, и всякій грѣхъ очищаетъ; надобно съ блуднымъ сыномъ оставить чуждую беззаконную страну, и возвратитися ко Отцу съ жалѣніемъ и уничиженіемъ себе самого, и повергнуть себе, яко недостойнаго, предъ милосердыми очами Его и признать: \textit{Отче! согрѣшихъ на небо и предъ Тобою, и уже нѣсмь достоинъ нарещися сынъ Твой: сотвори мя, яко единаго отъ наемникъ Твоихъ}\footnote{Лук.~15,~18 и 19.}. Помилуй мене благодатію и человѣколюбіемъ единороднаго Сына Твоего, Который какъ за всѣхъ, такъ и за мене грѣшнаго кровь Свою изліялъ и умеръ; ради Его неповиннаго страданія и смерти, мнѣ повинному прости и остави долги моя. Тако обратившемуся и со смиреніемъ и жалѣніемъ исповѣдующему грѣхи свои и просящему милости, оставятся безъ сумнѣнія вси долги его отъ Отца небеснаго, не по какимъ его заслугамъ, но по единой благодати. Ибо самая кровь Сына Божія, ради грѣшника изліянная, вопіетъ, да всѣ ему долги оставятся и не помянутся ктому. Узнаетъ на себѣ таковый грѣшникъ милость Отца небеснаго, якоже блудный сынъ возвратившійся узналъ на себѣ милость благоутробнаго отца своего. Увидитъ грѣшника таковаго Отецъ небесный пришедшаго къ Себѣ, и воззритъ на него милосердыми Своими очами, и милъ Ему будетъ. Не услышитъ выговора отъ Отца за своевольство, удаленіе и житіе развращенное; но услышитъ единый благоутробнаго Отца гласъ: \textit{изнесите одежду первую, и облецыте его, и дадите перстень на руку его, и сапоги на нозѣ его; и приведше телецъ упитанный заколите, и ядше веселимся: яко сынъ Мой сей мертвъ бѣ, и оживе; и изгиблъ бѣ, и обрѣтеся}\footnote{Лук.~15,~22--24.}. И \textit{тако}, по словеси Хрістову, \textit{радость будетъ предъ ангелы Божіими о единомъ грѣшницѣ кающемся}\footnote{ст.~7.}. Но такъ великую отъ Бога получившему грѣшнику милость должно и самому являть милость ближнему своему, и прощать ему согрѣшенія его, да не тоежде узнаетъ на себѣ, что узналъ должникъ оный евангельскій, тьмою талантъ царю своему обдолжившійся, который, получивши отъ царя своего милость, не хотѣлъ клеврета своего помиловать, и, великаго долга своего по единой царя своего милости свободившися, не хотѣлъ малаго долга брату своему оставить, "--- чего ради гнѣву царя своего подпалъ, и долгъ свой тяжкій паки на себѣ увидѣлъ, но еще съ выговоромъ: \textit{рабе лукавый! весь долгъ оный отпустихъ тебѣ, понеже умолилъ мя еси: не подобаше ли и тебѣ помиловати клеврета твоего, якоже и азъ тя помиловахъ? И прогнѣвався Господь его, предаде его мучителемъ, дондеже воздастъ весь долгъ свой. Тако}, заключаетъ Хрістосъ притчу тую, \textit{и Отецъ Мой небесный сотворитъ вамъ, аще не отпустите кійждо брату своему отъ сердецъ вашихъ согрѣшенія ихъ}\footnote{Мѳ.~18,~32--35.}.

\begin{quotation}

\textit{Не введи насъ во искушеніе}.

\end{quotation}

1) Искушеніе бываетъ намъ къ добру, или ко злу. Искушеніе къ добру отъ Бога есть, которымъ искуситися Давидъ и прочіи святіи просятъ. \textit{Искуси мя Боже, и увѣждь сердце мое; истяжи мя, и разумѣй стези моя}\footnote{Пс.~138,~23.}. Тако искушаемъ былъ Авраамъ\footnote{Быт.~22.}. Не о семъ искушеніи здѣ слово. "--- Искушеніе ко злу, или прельщеніе, бываетъ или \textit{отъ діавола}, который всякимъ образомъ ищетъ насъ уловить, прельстить, ко грѣху привести и погубить; или \textit{отъ плоти}, которая страстьми и похотьми беретъ насъ; или \textit{отъ міра}, который прелестію, суетою и соблазнами ко злу поощряетъ насъ. О семъ искушеніи есть слово здѣ. 2)~Богъ, яко благъ, ко злу никого не искушаетъ. \textit{Никтоже искушаемъ да глаголетъ, яко отъ Бога искушаемъ есмь: Богъ} бо \textit{нѣсть искуситель злымъ, и не искушаетъ Той никогоже. Кійждо же искушается, отъ своея похоти влекомъ и прельщаемъ}\footnote{Іак.~1,~13 и 14.}. Симъ словомъ: \textit{не введи насъ во искушеніе}, молимъ Бога, чтобы насъ отъ искушенія міра, плоти и діавола Своею благодатію сохранилъ. 4)~А хотя и впадемъ во искушенія, о томъ просимъ, чтобы не попустилъ оными быть намъ побѣжденными, но помоглъ бы намъ ихъ одолѣть и побѣдити. 5)~Отъ сего видно, что мы безсильны и немощны есмы въ себѣ безъ Божіей помощи. Аще бы сами могли мы противитися искушенію, не бы повелѣно намъ помощи въ семъ просити. 6)~Симъ научаемся, коль скоро почувствуемъ искушеніе на насъ находящее, тотчасъ Богу молитися и просить отъ Него помощи. 7)~Учимся паки отъ сего на себе и свою силу не надѣяться, но на Бога. 8)~Когда молимся: \textit{не введи насъ во искушеніе}, то не должны сами себе вдавать во искушеніе. Егда, рече Златоустъ святый, влечемся на подвигъ, мужественно стоять должно; а не позываеми молчать должны и времени подвига ожидать, да и суетныя славы нехотѣніе и мужество покажемъ.

\begin{quotation}

\textit{Но избави насъ отъ лукаваго}.

\end{quotation}

1) По ученію святаго Златоустаго, лукавый здѣ разумѣется діаволъ. Лукаваго, рече, здѣ глаголетъ діавола, повелѣвая намъ неусыпную съ нимъ имѣти брань. Онъ тако называется по превосходству ради великія злобы его, яко неусыпно на насъ вооружается, и всякія бѣды, какъ тѣлесныя, такъ и духовныя, на насъ навести тщится, якоже глаголетъ Апостолъ: \textit{трезвитеся, бодрствуйте, зане супостатъ вашъ діаволъ, яко левъ рыкая, ходитъ, искій кого поглотити}\footnote{1~Петр.~5,~8.}. 2)~Симъ молимся Отцу небесному, чтобы Самъ насъ отъ него защитилъ, отъ котораго сами чрезъ себе сохранитися не можемъ. 3)~Наконецъ дабы въ часъ смерти нашея, гдѣ онъ наипаче противу вѣрныхъ подвизается, сохранилъ насъ отъ него, и по блаженной кончинѣ души наши въ небесное отечество поялъ, и тако бы какъ отъ него, такъ и отъ всѣхъ бѣдъ, отъ него возставляемыхъ, избавивши, вѣчное подалъ намъ блаженство, которое обѣщалъ вѣрующимъ во имя Его. 4)~Симъ словомъ: \textit{избави насъ отъ лукаваго}, къ молитвѣ насъ возбуждаетъ Спаситель нашъ, и молитвою отъ него избавлятися научаетъ. 5)~Поминая о врагѣ, глаголетъ Златоустъ, къ подвигу насъ пріуготовляетъ, и лѣность отъ насъ отсѣкаетъ. 6)~Моляся: \textit{не введи насъ}, а не мене, \textit{во искушеніе, но избави насъ}, а не мене, \textit{отъ лукаваго}, научаемся и увѣщаваемся другъ за друга молитися, и другъ другу отъ Бога помощи, защищенія, избавленія и спасенія просити.

\begin{quotation}

\textit{Яко Твое есть царство и сила и слава, во вѣки вѣковъ. Аминь}.

\end{quotation}

Сіе слово есть заключеніе молитвы Господней, которымъ означается упованіе и дерзновеніе къ милосердому небесному Отцу. Ты \textit{царствуеши} надъ всѣми, Котораго власти вся повинуются; Ты сильнѣйшій паче всѣхъ, противу Котораго никто не можетъ стояти: отсюду имени Твоему \textit{слава} будетъ, что мы бѣдные грѣшники получимъ просимое отъ Тебе, и тако отъ сердца возблагодаримъ Тебѣ, и здѣ, а паче въ будущей жизни, прославимъ имя Твое со избранными Твоими. Дерзновеніе, глаголетъ Златоустъ, намъ подаетъ и возставляетъ насъ, поминая о Царѣ, Которому подчинены мы, и показуетъ Его отъ всѣхъ сильнѣйшаго, глаголя: \textit{яко твое есть царство}, и проч. Убо, аще Его есть царство, никого не подобаетъ боятися, понеже никто не можетъ противу Его стояти и власти Его противитися. Егда глаголетъ: \textit{Твое есть царство}, показуетъ, что и противникъ нашъ Ему подчиненъ, хотя и противится, попущеніемъ Божіимъ. Ибо и онъ отъ числа рабовъ есть, хотя отверженныхъ и прогнѣвляющихъ, и не дерзнетъ ни на кого нападати, аще прежде не пріиметъ позволенія свыше; и не токмо ни на какого человѣка, но ниже на свиней, овецъ и воловъ найти не можетъ безъ попущенія Божія. Убо, хотя и немощенъ еси, долженъ дерзати, толикаго имѣя Царя, Который все удобно и чрезъ тебе можетъ дѣлати.

\paragraph*{§\:335.} \textit{Иные молитвы образцы} здѣ предлагаются вкратцѣ изъ Писанія святаго, которыя можетъ хрістіанинъ въ пристойныхъ и нужныхъ случаяхъ употреблять, и милости у Бога искать и просить вѣрою. Безъ молитвы бо ничего добраго имѣть и творить не можемъ, яко нищіи и немощніи.

\subsubsection*{1. Молитва грѣшника кающагося.}

\textit{Помилуй мя Боже, по велицѣй милости Твоей, и по множеству щедротъ Твоихъ, очисти беззаконіе мое}, и проч.\footnote{Пс.~50.} Сему сходно мытарево моленіе: \textit{Боже! милостивъ буди мнѣ грѣшнику}\footnote{Лук.~18,~13.}. Великимъ грѣхамъ великой милости, и множеству беззаконій множества щедротъ у Бога вѣрою просить должно, да, \textit{идѣже умножися грѣхъ, преизбыточествуетъ благодать}\footnote{Римл.~5,~20.}.

\subsubsection*{2. Молитва кающагося.}

\textit{Помяни щедроты Твоя, Господи, и милости Твоя, яко отъ вѣка суть. Грѣхъ юности моея и невѣдѣнія моего не помяни; по милости Твоей помяни мя Ты, ради благости Твоея, Господи}\footnote{Пс.~24,~6 и 7.}. Много въ юности и въ невѣдѣніи согрѣшаетъ человѣкъ. И когда сія кающійся размышляетъ, и на правду Божію взираетъ, много въ совѣсти смущается и опечаляется. Въ такомъ случаѣ отставшему отъ грѣховъ и кающемуся должно помянуть и о безконечномъ Божіемъ милосердіи и щедротахъ, которыя обѣщалъ въ Словѣ Своемъ святомъ всѣмъ кающимся, и вѣрою во Хріста отпущенія грѣховъ просящимъ, "--- и со Псаломникомъ часто повторяти: \textit{помяни щедроты Твоя}, и проч.

\subsubsection*{3. Молитва чувствующаго гнѣвъ Божій въ совѣсти своей.}

\textit{Господи, да не яростію Твоею обличиши мене, ниже гнѣвомъ Твоимъ накажеши мене}\footnote{6,~2.}. Сходно сему Іеремія пророкъ: \textit{накажи насъ, Господи, обаче въ судѣ, а не въ ярости, да не умаленныхъ насъ сотвориши}\footnote{10,~24.}. И иное Псаломника: \textit{не вниди въ судъ съ рабомъ Твоимъ, яко не оправдится предъ Тобою всякъ живый}\footnote{142,~2.}. И паки: \textit{аще беззаконіе назриши, Господи, Господи, кто постоитъ}\footnote{129,~3.}? Не просто отъ наказанія Божія избавленія просить должно, но отъ гнѣва и строгости правды Божіей, которая грѣшнику всякому вѣчною мукою грозитъ. Отеческаго же наказанія Божія, хотя плоти нашей и горестно, желать должно, яко \textit{егоже любитъ Господь, наказуетъ, біетъ всякаго сына, егоже пріемлетъ}\footnote{Евр.~12,~6.}. И Самъ Хрістосъ глаголетъ: \textit{Азъ, ихже аще люблю, обличаю и наказую}\footnote{Апок.~3,~19.}. Горе же всякому человѣку, когда Богъ съ нимъ по строгости правды Своея въ судъ внидетъ! Не инаго чего, какъ вѣчнаго въ огнѣ геенскомъ мученія, ожидать ему должно. Всякій бо грѣхъ, и малѣйшій, по мнѣнію нашему, силенъ есть погрузить человѣка во дно адово, когда милость Божія не приспѣетъ, ради Сына Его Іисуса Хріста обѣщанная. Что уже о тѣхъ сказать, которые волею согрѣшаютъ? Сего ради должно всякому, кто хощетъ суда Божія убѣжать, отъ грѣховъ отстать и молить Бога вѣрою во Хріста: \textit{Господи, да не яростію Твоею обличиши мене}, и проч.

\subsubsection*{4. Молитва въ различномъ искушеніи.}

\textit{Господи Боже мой! на Тя уповахъ, спаси мя отъ всѣхъ гонящихъ мя, и избави мя. Да не когда похититъ, яко левъ, душу мою, не сущу избавляющу, ниже спасающу}\footnote{Пс.~7. 2 и 3.}. Сатана съ своими злыми аггелами и служителями непрестанно гонитъ и ищетъ вѣрныхъ уловити и похитити, яко левъ, якоже Апостолъ глаголетъ: \textit{трезвитеся, бодрствуйте, зане супостатъ вашъ діаволъ, яко левъ рыкая, ходитъ, искій кого поглотити}\footnote{1~Петр.~5,~8.}, "--- противу которыхъ враговъ инаго посредствія нѣтъ, какъ усердная молитва съ упованіемъ на Бога.

\subsubsection*{5. Молитва, когда помыслы смущаютъ душу, и къ отчаянію милости Божіей склоняютъ.}

\textit{Господи! что ся умножиша стужающіи ми? Мнози востаютъ на мя; мнози глаголютъ души Моей: нѣсть спасенія ему въ Бозѣ его. Ты же, Господи, заступникъ мой еси, слава моя, и возносяй главу мою}\footnote{Пс.~3,~2--4.}. Вѣрному, а наипаче такому, который во грѣхахъ находился и благодатію Божіею отъ нихъ отсталъ и кается, сатана часто помыслы влагаетъ злые, поминая ему прежніе грѣхи, и тѣми совѣсть его утѣсняетъ, глаголя: «престани вѣровать, осудишися; толико и толико ты согрѣшилъ Бога праведнаго, Который всѣмъ воздаетъ по дѣломъ, прогнѣвалъ: чего тебѣ надѣяться, кромѣ единаго мученія?»\footnote{Макар. Егип. бес.~11"~я.} И тако злый и коварный духъ къ отчаянію кающагося хощетъ привести. Въ такомъ случаѣ можно злому духу отвѣщать: «ты клеветникъ, и осужденъ уже, а не судія; судъ же преданъ Хрісту, Который пришелъ въ міръ грѣшники спасти, въ Котораго и я вѣрую, и надѣюся благодатію Его спастися». И должно къ молитвѣ обратиться, и утвердить себе Божіимъ милосердіемъ, котораго никакій грѣхъ побѣдить не можетъ. Сколько бы грѣховъ ни было у кого, и какъ бы велики ни были, у Бога милосердія еще болѣе. Якоже бо Самъ безконеченъ, тако и милость Его безконечна, и Хрістовы заслуги, которыми намъ вѣрующимъ въ Него милость, благодать и отпущеніе грѣховъ у Отца небеснаго заслужилъ, такожде безконечны, якоже Божественны. Симъ человѣколюбіемъ Божіимъ утѣшай себе, кающаяся душа, и благодатію Божіею и молитвою стой въ вѣрѣ. Единымъ бо тѣмъ неотмѣнно слѣдуетъ вѣчное съ діаволомъ осужденіе во адѣ, которые не хотятъ каяться и отъ грѣховъ отстать; а отставшимъ отъ грѣховъ и кающимся милосердія Божія двери отверсты.

\subsubsection*{6. Молитва во всякой скорби и тѣснотѣ.}

\textit{Господи Вседержителю Боже Израилевъ! душа въ тѣснотѣ, и духъ въ стуженіи возопи къ Тебѣ: услыши, Господи, и помилуй, яко еси Богъ милосердъ, и помилуй насъ, яко согрѣшихомъ предъ Тобою}\footnote{Вар.~3,~1 и 2.}. Сходно сему Псаломникъ: \textit{Ты еси прибѣжище мое отъ скорби, обдержащія мя. Радосте моя! избави мя отъ обышедшихъ мя}\footnote{Пс.~31,~7.}. Отъ скорби всякой ничѣмъ инымъ, какъ молитвою и исповѣданіемъ грѣховъ нашихъ предъ Богомъ, избавитися не можемъ. Какъ бо кто вѣрному своему другу сообщаетъ скорбь свою, получаетъ нѣкую отраду и утѣшеніе: такъ наипаче, когда Богу, Иже есть милосердъ естествомъ и близъ есть призывающихъ Его во истинѣ, сообщаемъ скорби сердца нашего, и, грѣхи наша исповѣдая, смиряемся, падаемъ предъ величествомъ Его, получаемъ или избавленіе, когда волѣ Его угодно и намъ полезно, или непремѣнно облегченіе и утѣшеніе: \textit{призови Мя въ день скорби твоея, и изму тя}\footnote{Пс.~49,~15.}. Слыши утѣшительное Божіе слово: \textit{за грѣхъ мало что опечалихъ его, и поразихъ его и отвратихъ лице Мое отъ него: и опечалися, и пойде дряхлъ въ путехъ своихъ: пути его видѣхъ, и исцѣлихъ его, и утѣшихъ его, и дахъ ему утѣшеніе истинно}\footnote{Ис.~57,~17 и 18.}. Всякая скорбь и бѣда отъ грѣха бываетъ, и когда бы грѣха не было, не было бы и скорби. Сего ради посылается скорбь, да грѣхъ очистится; и когда очистится грѣхъ, который скорбь содѣловаетъ, отъимется и самая скорбь. Сего ради должно покаяніемъ, исповѣданіемъ предъ Богомъ, молитвою и вѣрою очищати грѣхи, да тако и скорбь и печаль благодатію Божіею отъимется, и пріидетъ утѣшеніе, какъ солнце по мрачныхъ дняхъ, или какъ пища по постѣ, пріятное и увеселительное.

\subsubsection*{7. Како подобаетъ утѣшати вѣрою скорбящую душу.}

\textit{Вскую прискорбна еси, душе моя? и вскую смущаеши мя? Уповай на Бога, яко исповѣмся Ему, спасеніе лица моего, и Богъ мой}\footnote{Пс.~41,~6.}. Въ великой печали человѣкъ не знаетъ самъ, что дѣлать. Въ такомъ случаѣ, возлюбленный хрістіанине, \textit{возверзи на Господа печаль твою, и Той тя препитаетъ}, глаголетъ тебѣ Давидъ святый\footnote{54,~23.}. Утѣшай себе \textit{безконечнымъ} милосердіемъ Божіимъ, \textit{яко не во вѣкъ отринетъ Господь, яко смиривый помилуетъ по множеству милости Своея}\footnote{Плач. Іер.~3,~31 и 32.}. \textit{Господь мертвитъ, и живитъ, низводитъ въ адъ} скорбей \textit{и возводитъ; Господь убожитъ и богатитъ, смиряетъ и выситъ}\footnote{1~Цар.~2,~6 и 7.}. Аще и глаголютъ тебѣ врази твои, діаволъ и слуги его: \textit{гдѣ есть Богъ твой?} бесѣдуй къ душѣ твоей, утѣшай ее съ Давидомъ святымъ: \textit{вскую прискорбна еси, душе моя?} и проч. Хотя и глаголется тебѣ въ помыслѣхъ: \textit{гдѣ есть Богъ твой?} но ты отвѣщай: \textit{близъ Господь всѣмъ призывающимъ Его во истинѣ}\footnote{Пс.~144,~18.}.

\subsubsection*{8. Молитва о неотступной Божіей помощи.}

\textit{Не остави мене, Господи Боже мой, не отступи отъ мене; вонми въ помощь мою, Господи спасенія моего}\footnote{37,~22 и 23.}. Когда человѣкъ, отставши отъ грѣховъ, міръ презрѣвши и прелесть его, тщится Богу угождать, Хрісту послѣдовать вѣрою и терпѣніемъ, "--- срѣтаетъ его ненависть міра, и самые тѣ, которые другами прежде были, врагами его дѣлаются, какъ въ томже псалмѣ глаголется: \textit{друзи мои и искренніи мои прямо мнѣ приближишася и сташа; и ближніи мои отдалече мене сташа: и нуждахуся ищущіи душу мою, и ищущіи злая мнѣ глаголаху суетная, и льстивнымъ весь день поучахуся}\footnote{Пс.~37,~12 и 13.}. Хотя и всегда, но наипаче въ такомъ случаѣ, должно къ Богу сердце возводити, съ воздыханіемъ, чтобы Онъ не оставилъ и не отступилъ съ помощію Своею. А быть должно противу враговъ и хульниковъ, какъ глухому и нѣмому, якоже Псаломникъ о себѣ въ томже псалмѣ глаголетъ: \textit{азъ же яко глухъ не слышахъ, и яко нѣмъ не отверзаяй устъ своихъ}\footnote{ст.~14.}. Тогда Богъ, на Котораго уповаеши, вмѣсто тебе, возглаголетъ дѣломъ самымъ, яко глаголетъ пророкъ: \textit{открый ко Господу путь твой, и уповай на Него, и Той сотворитъ. И изведетъ, яко свѣтъ правду твою, и судьбу твою яко полудне}\footnote{36,~5 и 6.}. А притомъ самъ тщись не оставить Господа, ни сердцемъ отступить отъ Него, но внимать всегда закону Его, якоже глаголетъ: \textit{внемлите людіе Мои закону Моему}\footnote{77,~1.}.

\subsubsection*{9. Богу благодареніе отъ грѣшника, отъ грѣховъ отставшаго и кающагося.}

\textit{Исповѣмся Тебѣ, Господи Боже мой, всѣмъ сердцемъ моимъ, и прославлю имя Твое въ вѣкъ: яко милость Твоя велія на мнѣ, и избавилъ еси душу мою отъ ада преисподнѣйшаго}\footnote{85,~12 и 13.}. Сходно тому: \textit{растерзалъ еси узы моя. Тебѣ пожру жертву хвалы}\footnote{Пс.~115,~7 и 8.}. Человѣкъ, во грѣхахъ живущій, грѣхами, какъ узами, духовно связанъ, и, какъ плѣнникъ, отъ сатаны содержится, и при вратахъ адовыхъ душею своею имѣется, хотя онъ того, яко ослѣпленъ, не чувствуетъ; но благость Божія, на покаяніе ожидающая его, не попущаетъ душу его сатанѣ во адъ восхитити. А когда отъ грѣховъ благодатію Божіею отстанетъ и кается, отъ узъ тѣхъ свобождается, и отъ области сатанины восхищается, и благодатію Хрістовою свободенъ дѣлается, якоже Хрістосъ глаголетъ: \textit{аще Сынъ вы свободитъ, воистину свободни будете}\footnote{Іоан.~8,~36.}, и отъ вратъ смертныхъ восхищается, и къ числу спасающихся приходитъ. Чего ради всегда ему должно прежнюю свою бѣду и сію великую милость Божію помнить, и усердно Богу благодарить за тое, что отъ той бѣды его избавилъ.

\subsubsection*{10. Благодареніе Богу за отеческое Его наказаніе.}

\textit{Благо мнѣ, Господи, яко смирилъ мя еси, яко да научуся оправданіямъ Твоимъ}\footnote{Пс.~118,~71.}! Надобно отъ сердца признать, что великую милость дѣлаетъ съ нами Богъ, когда насъ отеческимъ наказанія жезломъ біетъ, хотя плоти нашей немощной и горестно. Ибо \textit{егоже любитъ Господь, наказуетъ}\footnote{Евр.~12,~6.}. Чего ради не роптать, но благодарить Ему за сіе должно.

\subsubsection*{11. Молитва о храненіи языка.}

\textit{Положи, Господи, храненіе устомъ моимъ, и дверь огражденія о устнахъ моихъ}\footnote{Пс.~140,~3.}. Никакимъ членомъ тѣлеснымъ не согрѣшаетъ такъ человѣкъ, какъ языкомъ; а наипаче во время скорби и напасти весьма трудно его удержать. Чего ради должно отъ Бога помощи просить, чтобы помоглъ намъ управлять его, и тое только глаголати, что полезно, а о томъ молчать, что вредно.

\subsubsection*{12. Молитва, взятая изъ псалма 50"~го.}

\textit{Отврати лице Твое отъ грѣхъ моихъ, и вся беззаконія моя очисти}\footnote{Пс.~50,~11.}. Отче, Господи небесе и земли! молю Тя вѣрою во имя единороднаго Сына Твоего: отврати лице Твое отъ грѣхъ моихъ, и обрати на лице Сына Твоего, Господа нашего Іисуса Хріста, Егоже послалъ еси въ міръ \textit{грѣшники спасти, отъ нихже первый есмь азъ}\footnote{1~Тим.~1,~15.}, и Его благодатію и человѣколюбіемъ вся беззаконія моя очисти, съ Нимъ же благословенъ еси и съ Пресвятымъ Твоимъ Духомъ во вѣки, аминь.

\subsubsection*{13. Псаломъ 118"~й.}

\textbf{Псаломъ 118"~й} весь научаетъ насъ, како подобаетъ усердно просить намъ Бога, чтобы Самъ наставилъ насъ на путь заповѣдей Своихъ, и отверзлъ умъ разумѣти чудеса отъ закона Своего.

\subsubsection*{14. Молитва общая.}

\textbf{Молитва общая} во время нашествія иноплеменниковъ, и во всякомъ общемъ бѣдствіи. \textit{Воскресни, Господи, помози намъ и избави насъ имене ради Твоего}\footnote{Пс.~43,~27.}!

\textbf{Вторая.} Боже, ущедри ны, и благослови ны, просвѣти лице Твое на ны, и помилуй ны\footnote{66,~2.}!

\textbf{Третія.} \textit{Благоволилъ еси, Господи, землю Твою, возвратилъ еси плѣнъ Іаковль. Оставилъ еси беззаконія людей Твоихъ, покрылъ еси вся грѣхи ихъ. Укротилъ еси весь гнѣвъ Твой, возвратился еси отъ гнѣва ярости Твоея. Возврати насъ, Боже спасеній нашихъ, и отврати ярость Твою отъ насъ. Еда во вѣки прогнѣваешися на ны, или простреши гнѣвъ Твой отъ рода въ родъ? Боже! Ты обращся оживиши ны, и людіе Твои возвеселятся о Тебѣ. Яви намъ, Господи, милость Твою, и спасеніе Твое даждь намъ}\footnote{Пс.~84,~2--8.}! Когда непріятель подымаетъ оружіе и на наше отечество находитъ, сіе не иное что значитъ, какъ гнѣвъ Божій, грѣхами нашими возгорѣвшійся. Такожде, когда нивы наши не приносятъ намъ плода, или частые пожары бываютъ, или моровая язва расширяется, или иное какое бѣдствіе находитъ на насъ: Богъ наказуетъ насъ общимъ бѣдствіемъ, да въ чувство пріидемъ и покаемся. Сей праведный Божій гнѣвъ воздвигаютъ хотя и всякіе грѣхи, а наипаче суды неправедные, мздоимства, клятвопреступленія, обиды и озлобленія вдовъ, сиротъ и прочіихъ бѣдныхъ, и проліянія слезъ неповинныхъ, взаимная неправда въ купляхъ, взаимное похищеніе, воровство, взаимная лесть, лукавство, ложь и обманъ, когда другъ друга берегутся и другъ другу ни въ чемъ не вѣрятъ. Сюды надлежитъ мерзкая нечистота, когда люди не ужасаются ближнихъ своихъ ложа осквернять, и прочія нечистоты совершать, за что первый міръ потопомъ, и Содомъ и Гоморръ огнемъ и жупеломъ, праведнымъ Божіимъ гнѣвомъ погублены. Тойжде праведный гнѣвъ Божій дѣйствуетъ и въ той странѣ, въ которой то пожары, которыми грады и веси опустѣваютъ, то оскудѣнія хлѣба и глады, то паденія многихъ тысячъ людей отъ военнаго оружія слышатся, отъ чего все отечество бѣду пріемлетъ: люди оскудѣваютъ, общая казна истощается и исчезаетъ, осиротѣвшихъ дѣтей, вдовствующихъ женъ, лишившихся сыновъ своихъ, падшихъ на брани, матерей плачь и рыданіе и вопль жалостно возносится; всѣхъ отечества сыновъ печаль и скорбь объемлетъ, страхъ и ужасъ отчаянія неблагополучнаго въ войнѣ конца чувствительно смущаетъ. Въ такихъ бѣдственныхъ обстоятельствахъ, когда люди наказующія Божія руки не чувствуютъ, и не токмо отъ грѣховъ не отстаютъ и не каются, но и грѣхи ко грѣхамъ прилагаютъ; то должно непремѣнно общаго бѣдствія еще большаго ожидать. Сего ради хотя и всегда, а паче въ такихъ случаяхъ, должно всѣмъ обще, познавши и признавши свои грѣхи и беззаконія, единодушно отъ глубины сердца возвысить гласъ: \textit{Господи Боже силъ! обрати ны, и просвѣти лице Твое, и спасемся}\footnote{Пс.~79,~8.}, и приведенныя, или иныя, какія вѣра подастъ, молитвы къ Богу проливать. Общіе бо грѣхи общимъ и публичнымъ наказаніемъ наказуются, чего ради общимъ и покаяніемъ умилостивлять Бога должно. А беззаконныхъ клятвопреступниковъ, хищниковъ, мздоимцевъ, бѣдныхъ слезы проливающихъ, неповинно осуждающихъ и прочіихъ беззаконниковъ, какъ общую отечества язву, кому отъ Бога мечъ правосудія преданъ, должно смирять и судомъ праведнымъ всякому воздавать по дѣламъ его, да не и сами судящіи земли мечемъ праведнаго суда Божія посѣчены будутъ, и жребій съ невѣрными воспріимутъ. Худо и беззаконно тамо снисхожденіе являть, гдѣ строгости и правосудіе употреблять должно. Снисхожденіе тамо нужно, гдѣ отъ немощи грѣхи дѣлаются, чему всякъ и добрый подлежитъ. А безстрашіе и безбожіе строго должно наказывать, какъ общую моровую язву, да и прочіи страхъ имутъ и не заразятся.

\subsubsection*{15. Молитва и исповѣданіе общее вѣрныхъ.}

\textit{Помяни насъ, Господи, во благоволеніи людей Твоихъ, посѣти насъ спасеніемъ Твоимъ, видѣти во благости избранныя Твоя, возвеселитися въ веселіи языка Твоего, хвалитися съ достояніемъ Твоимъ. Согрѣшихомъ со отцы нашими, беззаконновахомъ, неправдовахомъ}, и проч.\footnote{Пс.~105,~4--6 и слѣд.} Должно намъ сердечно признать неправду нашу предъ Богомъ, якоже Даніилъ пророкъ глаголетъ: \textit{Тебѣ, Господи, правда, намъ же стыдѣніе лица}\footnote{Дан.~9,~7.}. Какъ бо отцы наши, то"=есть, предки наши, такъ и мы много согрѣшили и согрѣшаемъ предъ Богомъ нашимъ, беззаконнуемъ и неправдуемъ. \textit{Много бо согрѣшаемъ вси}, глаголетъ Апостолъ\footnote{Іак.~3,~2.}. Чего ради должно хрістіанамъ всѣмъ вообще исповѣдати грѣхи своя предъ Богомъ: \textit{согрѣшихомъ со отцы нашими, беззаконновахомъ, неправдовахомъ} и проч., "--- и молити Его, чтобы помянулъ насъ по милости, человѣколюбію и благоволенію, тому, которое обѣщалъ ради благословеннаго сѣмени онаго, еже есть Іисусъ Хрістосъ, людемъ Своимъ явить; посѣтилъ насъ спасеніемъ Своимъ о томжде благословенномъ сѣмени, да сподобимся \textit{видѣти во благости избранныя Его, яко солнце, сіяющія во царствіи Его}\footnote{Мѳ.~13,~43.}, \textit{возвеселитися въ веселіи языка святаго Его, хвалитися съ достояніемъ Его}, и тѣхъ благъ, \textit{ихже око не видѣ, и ухо не слыша, и на сердце человѣку не взыдоша, яже уготова Богъ любящимъ Его}\footnote{1~Кор.~2,~9.}, въ безконечные вѣки наслаждаться. О сей славѣ избранныхъ Божіихъ всегда намъ памятовати подобаетъ; къ той желанія наши, умъ нашъ, сердце наше и воздыханія возводити, пока въ юдоли сей плачевной странствуемъ, и усердно молитися вѣрою къ Отцу небесному: \textit{помяни насъ, Господи, во благоволеніи}, и проч. Сколько бо ни жить въ бѣдственномъ мірѣ семъ, надобно отсюду на оный свѣтъ преселиться; сколько ни собирать богатства, слѣдуетъ его непремѣнно въ мірѣ семъ оставить; сколько ни веселиться честію и славою міра сего, слѣдуетъ необходимо въ маломъ и темномъ гробѣ заключиться, и землѣ въ землю возвратиться. А какая польза отъ богатства, чести и славы міра сего, когда той чести, славы и богатства, которое избраннымъ Божіимъ на небеси уготовано, лишимся? Что пользы отъ веселостей и роскошей міра сего, когда тѣло червямъ на снѣденіе, а душа демономъ на посмѣяніе и поруганіе предастся? Худая и бѣдственная перемѣна "--- отъ мнимой славы и чести въ истинное и вѣчное поношеніе и укоризну, отъ временнаго и мнимаго сладострастія въ вѣчное и истинное мученіе перейти. Сія пренеблагополучная измѣна срящетъ сластолюбивую, славолюбивую и сребролюбивую душу. Напротивъ того, весьма блаженная и всежелаемая перемѣна "--- отъ малой скорби и временной (отъ \textit{малой}, говорю: ибо какая бы въ мірѣ семъ ни была скорбь, и сколь долговременна бы ни была, ничто есть противу будущей и вѣчной скорби) въ вѣчную радость и веселіе избранныхъ Божіихъ преселитися, и Хріста лицемъ къ лицу, какъ солнце, сіяющее и просвѣщающее избранныхъ Своихъ, видѣти. Преблагополучны суть, и благополучія ихъ не токмо словомъ изобразить, но и умомъ понять невозможно, которые части сей блаженной сподобляются, хотя они въ мірѣ семъ, какъ сметіе непотребное, попираеми бываютъ. Напротивъ того, бѣдные и окаяннѣйшіе суть, которые ради прихотей своихъ онаго лишаются жребія, хотя бы они, какъ бози, въ вѣцѣ семъ почитаеми и покланяеми были. Сего ради достойно всегда вѣрою воздыхать ко Отцу щедротъ и всякія утѣхи Богу: \textit{помяни насъ, Господи!} и проч. Но прежде отъ грѣховъ и суеты міра сего отвратиться и обратиться всѣмъ сердцемъ къ Нему, и тако, признавая свою неправду и заблужденіе, каяться и молитися.

\paragraph*{§\:336.} \textbf{Молитвы ко Хрісту, Сыну Божію, взятыя изъ Святаго Евангелія.}

\subsubsection*{1. Молитва, взятая изъ притчи сѣющаго сѣмя\footnote{Мѳ.~13,~3--8.}.}

Сыне Божій, Іисусе Хрісте, явивыйся во смиренной нашей плоти на земли, и посѣявый доброе вѣры сѣмя на селѣ міра сего! Огради страхомъ Твоимъ сердце мое, да не приходя лукавый восхищаетъ посѣянное слова Твоего сѣмя, и очисти сердце мое отъ терній печали вѣка сего, лести, богатства и всякія суеты, и сотвори тое доброю землею къ пріятію Божественнаго Твоего слова, да сотворю плодъ вѣры и имѣю часть со избранными Твоими, иже плодъ приносятъ Тебѣ ово сто, ово шестьдесятъ, ово тридесять, "--- яко благословенъ еси со Отцемъ и Святымъ твоимъ Духомъ во вѣки, аминь.

\subsubsection*{2. Молитва, взятая изъ исторіи о хананейской женѣ\footnote{15,~22--28.}.}

Помилуй мя, Господи Сыне Божій, Сыне Давидовъ по плоти, якоже помиловалъ еси хананею: душа моя злѣ бѣснуется гнѣвомъ, яростію, похотію злою и прочіими пагубными страстьми. Господи! помози мнѣ, вопію Тебѣ, не на земли ходящему, но одесную Отца на небесѣхъ сѣдящему. Ей, Господи! даждь мнѣ сердце вѣрою и любовію послѣдовати Твоему смиренію, благости, кротости и долготерпѣнію, да и въ вѣчномъ Твоемъ царствіи сподоблюся причащатися трапезы рабовъ Твоихъ, ихже избралъ еси. Аминь.

\subsubsection*{3. Молитва различно искушаемаго, взятая изъ исторіи о бѣдствующихъ на морѣ апостолѣхъ\footnote{Мѳ.~8,~23--27.}.}

Господи! спаси мя, погибаю. Се бо кораблецъ мой бѣдствуетъ отъ искушенія волнъ, и близъ потопленія есть. Ты, яко Богъ милосердъ и сострадателенъ немощемъ нашимъ, властію Твоею всесильною запрети волненію бѣдствій, хотящихъ погрузити мя и низвести во глубину золъ, "--- и будетъ тишина, яко \textit{вѣтры и море послушаютъ Тебе}. Аминь.

\subsubsection*{4. Молитва, взятая изъ исторіи о Преображеніи\footnote{17,~1--9.}.}

Собезначальный Отцу и Духу Слове, насъ ради человѣкъ и нашего ради спасенія отъ Дѣвы родивыйся и на земли поживый! сподоби мя вкусити благости Твоея, которую святіи Твои, любящіи и почитающіи Тя, и нынѣ вкушаютъ, странствующе въ плачевной міра сего юдоли, "--- и зрѣти присносущный свѣтъ Твой умнымъ окомъ, егоже святымъ Твоимъ ученикомъ показалъ еси въ преображеніи святѣйшія плоти Твоея, да, нынѣ вѣрою зря, потомъ лицемъ къ лицу сподоблюся видѣти славу Твою, и со избранными Твоими хвалити Тя и безначальнаго Твоего Отца и Святаго Духа. Аминь.

\subsubsection*{5. Молитва, взятая изъ исторіи о прокаженныхъ\footnote{Лук.~17,~12--19.}.}

Іисусе, Врачу душъ и тѣлесъ нашихъ, помилуй мя! Къ тебѣ, живущему на небесѣхъ и на насъ смиренныхъ призирающему, возношу отъ земли гласъ мой съ воздыханіемъ и вѣрою: очисти мою душу, прокаженную грѣхами, якоже очистилъ еси десять прокаженныхъ мужей, да явлюся чистъ предъ святѣйшими очесами Твоими, егда пріидеши во славѣ Твоей, и имѣю часть со избранными Твоими и съ ними восхвалю Тя со безначальнымъ Твоимъ Отцемъ и Пресвятымъ Духомъ во царствіи Твоемъ. Аминь.

\subsubsection*{6. Молитва, взятая изъ исторіи о слѣпцахъ\footnote{Мѳ.~20,~30--34.}.}

Сыне Божій, Свѣте вѣчный, и даяй всѣмъ свѣтъ! помилуй мя, просвѣти очи сердца моего, мглою грѣховъ и страстей омраченныя, якоже просвѣтилъ еси слѣпыхъ, возопившихъ къ Тебѣ. Да отверзутся и мои очи сердечныя, да узрю Тя, Свѣта незаходимаго, и вѣрою и любовію послѣдую Тебѣ, мене ради окаяннаго ходившему по земли. Аминь.

\subsubsection*{7. Молитва, взятая изъ разныхъ исторій.}

Іисусе Хрісте, Сыне Божій! призри на мя, якоже призрѣлъ еси на грѣшницу въ дому Симоновѣ, на Петра, отвергшагося Тебе, на разбойника на крестѣ, и сопричти мя покаявшимся Тебѣ мытарямъ, блудникамъ и прочіимъ грѣшникамъ, яко Ты еси Богъ и Спасъ нашъ, пришедый въ міръ грѣшники спасти, отъ нихже первый есмь азъ. Аминь.

\subsubsection*{8. Молитва о возобновленіи духовномъ.}

Іисусе Сыне Божій, Спасителю и Обновителю міра, злобою грѣховною обветшавшаго! обнови мене благодатію животворящаго Духа Твоего. Дай мнѣ умъ разумѣти силу спасительнаго пришествія Твоего; дай мнѣ сердце любити Тя "--- вѣчную любовь, утѣшеніе и радость святыхъ; дай мнѣ очи непрестанно зрѣти страсти Твоя; дай мнѣ уши слышати святое слово Твое; дай мнѣ уста глаголати Тебѣ угодная, мнѣ и ближнему полезная; дай мнѣ нозѣ ходити по стези заповѣдей Твоихъ; и все мое, молю Тя, отъими, и подаждь Свое; возми ветхое, и подай новое все, яко Ты еси творяй вся, и \textit{безъ Тебе не можемъ творити ничесоже}\footnote{1~Іоан.~15,~5.}, яко благословенъ еси во вѣки. Аминь.

\subsubsection*{* * *}

Тако можешь, возлюбленный хрістіанине, читаючи Евангеліе святое и прочее Писаніе, возводить умъ и сердце твое вѣрою къ Богу, и милости отъ Него искати, каковыхъ молитвъ умилительныхъ много имѣется между церковными пѣсньми и ирмосами святаго Дамаскина, которыя, въ церкви стоя и прилѣжно внимая чтенію, къ себѣ прилагай, и вѣрою душѣ своей ищи пользы. А извѣстно знай, что безъ прилѣжной молитвы всякое тщаніе, наипаче въ духовныхъ, тщетно бываетъ.

\paragraph*{§\:337.} Здѣ представляется, како вѣрная душа съ Богомъ бесѣдуетъ, и Богъ милосердо и человѣколюбно ей въ святомъ Своемъ словѣ отвѣтствуетъ, на такій конецъ, дабы всякъ, отъ Бога ищущій милости, въ вѣрѣ, которая необходимо въ молитвѣ нужна, утвердился, и кающійся грѣшникъ, который духовно отъ сатаны искушается, утѣшеніе воспріялъ въ сердцѣ своемъ, то"=есть, что кающагося грѣшника Богъ благодатію Своею пріемлетъ. Вѣрная душа глаголетъ: \textit{Ты еси, Господи, прибѣжище отъ скорби обдержащія мя. Радосте моя! избави мя отъ обышедшихъ мя}\footnote{Пс.~31,~7.}. Отвѣщаетъ Богъ: \textit{вразумлю тя, и наставлю тя на путь сей, въ оньже пойдеши; утвержу на тя очи Мои}\footnote{ст.~8.}. Вѣрная душа глаголетъ: \textit{призри на мя и помилуй мя, по суду любящихъ имя Твое}\footnote{118,~132.}. Богъ чрезъ пророка отвѣщаетъ: \textit{на кого воззрю? токмо на кроткаго, и молчаливаго, и трепещущаго словесъ Моихъ}\footnote{Ис.~66,~2.}. "--- Душа вѣрная глаголетъ: \textit{Господи! кто обитаетъ въ жилищи Твоемъ? или кто вселится во святую гору Твою}\footnote{Пс.~14,~1.}? Отвѣщаетъ Богъ: \textit{ходяй непороченъ, и дѣлаяй правду, глаголяй истину въ сердцѣ своемъ, иже не ульсти языкомъ своимъ, и не сотвори искреннему своему зла, и поношенія не пріятъ на ближнія своя. Уничиженъ есть предъ Нимъ лукавнуяй: боящіяжеся Господа славитъ. Кленыйся искреннему своему, и не отметаяся, сребра своего не даде въ лихву, и мзды на неповинныя не пріятъ: творяй сія не подвижится во вѣкъ}\footnote{ст.~2--5.}. "--- Душа вѣрная глаголетъ: \textit{сердце чисто созижди во мнѣ, Боже, и духъ правъ обнови во утробѣ моей}\footnote{50,~12.}. Богъ отвѣщаетъ чрезъ пророка: \textit{дамъ вамъ сердце новое, и духъ новый дамъ вамъ; и отъиму сердце каменное отъ плоти вашея, и дамъ вамъ сердце плотяное}\footnote{Іез.~36,~26.}. "--- Душа вѣрная глаголетъ: \textit{скорби сердца моего умножишася, отъ нуждъ моихъ изведи мя}\footnote{Пс.~24,~17.}. Отвѣщаетъ Богъ: \textit{призови Мя въ день скорби твоея, и изму тя, и прославиши Мя}\footnote{Пс.~49,~15.}. "--- Вѣрная душа глаголетъ: \textit{помяни мя, Господи, егда пріидеши во царствіи Твоемъ}\footnote{Лук.~23,~42.}. Хрістосъ отвѣщаетъ: \textit{буди вѣренъ даже до смерти, и дамъ ти вѣнецъ живота}\footnote{Апок.~2,~10.}. "--- Вѣрная душа глаголетъ: \textit{не остави мене, Господи Боже мой, не отступи отъ мене; вонми въ помощь мою, Господи спасенія моего}\footnote{Пс.~37,~22 и 23.}. Отвѣщаетъ Хрістосъ: \textit{се Азъ съ вами есмь во вся дни до скончанія вѣка}\footnote{Мѳ.~28,~20.}. "--- Вѣрная душа глаголетъ: \textit{Господи Боже мой! на Тя уповахъ: спаси мя отъ всѣхъ гонящихъ мя, и избави мя}\footnote{Пс.~7,~2.}. Отвѣщаетъ Господь: \textit{яко на Мя упова, и избавлю его; покрыю его, яко позна имя Мое}\footnote{90,~14.}. "--- Вѣрная душа глаголетъ: \textit{не отвержи мене отъ лица Твоего}\footnote{50,~13.}. Хрістосъ глаголетъ: \textit{грядущаго ко мнѣ не изжену вонъ}\footnote{Іоан.~6,~37.}. "--- Вѣрная душа глаголетъ: \textit{не вниди въ судъ съ рабомъ Твоимъ, яко не оправдится предъ Тобою всякъ живый}\footnote{Пс.~142,~2.}. Хрістосъ отвѣщаетъ: \textit{тако возлюби Богъ міръ, яко и Сына Своего единороднаго далъ есть, да всякъ, вѣруяй въ Онь, не погибнетъ, но имать животъ вѣчный. Не посла бо Богъ Сына Своего въ міръ, да судитъ мірови, но да спасется Имъ міръ. Вѣруяй въ Онь, не будетъ осужденъ}\footnote{Іоан.~3,~16--18.}. "--- Вѣрная душа глаголетъ: \textit{грѣхъ юности моея и невѣдѣнія моего не помяни}\footnote{Пс.~24,~7.}. Богъ отвѣщаетъ: \textit{беззаконникъ, аще обратится отъ всѣхъ беззаконій своихъ, яже сотворилъ, и сохранитъ вся заповѣди Моя, и сотворитъ судъ, и правду, и милость: жизнію поживетъ, и не умретъ; вся согрѣшенія его, елика сотворилъ, не помянутся ему}\footnote{Іез.~18,~21 и 22.}. "--- Вѣрная душа глаголетъ: \textit{беззаконія моя превзыдоша главу мою, и, яко бремя тяжкое, отяготѣша на мнѣ}\footnote{Пс.~37,~5.}. Отвѣщаетъ Господь: \textit{не пріидохъ призвати праведники, но грѣшники на покаяніе}\footnote{Мѳ.~9,~13.}; и паки чрезъ Апостола: \textit{идѣже умножися грѣхъ, преизбыточествова благодать}\footnote{Римл.~5,~20.}. "--- Вѣрная душа глаголетъ: \textit{реку Богу: Заступникъ мой еси, почто мя забылъ еси}\footnote{Пс.~41,~10.}? И паки: \textit{не забуди убогихъ Твоихъ до конца}\footnote{9,~33.}. Отвѣщаетъ Господь: \textit{еда забудетъ жена отроча свое, еже не помиловати исчадія чрева своего? Аще же и забудетъ сихъ жена, но Азъ не забуду тебе, глаголетъ Господь}\footnote{Ис.~49,~15.}. "--- Вѣрная душа глаголетъ: \textit{не отврати лица Твоего отъ отрока Твоего: яко скорблю, скоро услыши мя}\footnote{Пс.~68,~18.}. Отвѣщаетъ Господь: \textit{за грѣхъ мало что опечалихъ его, и поразихъ его, и отвратихъ лице Мое отъ него: и опечалихся, и пойде дряхлъ въ путехъ своихъ: пути его видѣхъ, и исцѣлихъ его, и утѣшихъ его, и дахъ ему утѣшеніе истинно}\footnote{Ис.~57,~17 и 18.}. "--- Вѣрная душа глаголетъ: \textit{настави мя Господи на путь Твой, и пойду во истинѣ Твоей}\footnote{Пс.~85,~11.}. Отвѣщаетъ Хрістосъ: \textit{Азъ есмь путь и истина и животъ}\footnote{Іоан.~14,~6.}. Азъ есмь путь: \textit{никтоже пріидетъ ко Отцу, токмо Мною}. Азъ есмь истина: \textit{вѣруяй въ Мя и слову Моему не имать прельститися}. Азъ есмь животъ: \textit{вѣруяй въ Мя, аще и умретъ, оживетъ}\footnote{11,~25.}. "--- Вѣрная душа глаголетъ: \textit{имже образомъ желаетъ елень на источники водныя: сице желаетъ душа моя къ Тебѣ, Боже}\footnote{Пс.~41,~2.}. Хрістосъ глаголетъ: \textit{пріидите ко Мнѣ вси труждающіися и обремененніи, и Азъ упокою вы}\footnote{Мѳ.~11,~28.}. "--- Вѣрная душа съ Апостоломъ глаголетъ: \textit{желаніе имѣю разрѣшитися, и со Хрістомъ быти}\footnote{Филип.~1,~23.}. Хрістосъ глаголетъ: \textit{днесь со Мною будеши въ раи}\footnote{Лук.~23,~43.}.


\section[Статья 4-я. О должности хрістіанской ко Хрісту, Сыну Божію.]{статья четвертая.\\\bfseries О должности хрістіанской ко Хрісту, Сыну Божію.}

\subsection[Глава 1-я. О вѣрѣ во Хріста.]{глава первая.\\\bfseries О вѣрѣ во Хріста.}

\begin{quotation}\textit{Тако возлюби Богъ міръ, яко и Сына Своего единороднаго далъ есть, да всякъ, вѣруяй въ Онь, не погибнетъ, но имать животъ вѣчный}\footnote{Іоан.~3,~16.}.\end{quotation}

О вѣрѣ и свойствахъ ея сказано въ главѣ второй первыя статьи сея книги. Здѣ еще о вѣрѣ во Хріста нужно предложити тебѣ, любезный читатель, понеже въ вѣрѣ все наше блаженство состоитъ. А понеже вѣрою во Хріста всѣ спасаются, которые ни спасаются, то къ лучшему понятію вѣры во Хріста нужно предложить здѣ и тое; что насъ вринуло въ ровъ погибели; отъ которой вѣрою во Хріста спасаемся, и вкратцѣ коснуться дивнаго о насъ Божія смотренія, которымъ насъ человѣколюбно благоволилъ спасти, "--- наипаче ради простыхъ людей.

\paragraph*{§\:338.} 1)~Богъ нашъ и Творецъ, въ тріехъ лицахъ покланяемый, когда сотворилъ небо и землю и все украшеніе ихъ, послѣди создалъ человѣка Адама, и помощницу ему Еву: отъ сихъ двухъ весь родъ человѣка произшелъ и умножился; всѣ человѣки имѣютъ сихъ двухъ первыхъ себѣ родителей. 2)~Прародителей нашихъ сихъ преблагій Богъ создалъ не такъ, какъ прочія вещи, но дивнымъ нѣкіимъ и человѣколюбивымъ совѣтомъ. Прочія твари созидая, Богъ глаголалъ: \textit{да будетъ, и бысть тако}\footnote{Быт.~1,~3,~6,~9,~11,~14,~15,~20,~24.}, \textit{рече, и быша; повелѣ, и создашася}\footnote{Пс.~148,~5.}. Человѣка имѣя создать, глаголалъ: \textit{сотворимъ человѣка по образу Нашему и по подобію}\footnote{Быт.~1,~26.}. Такъ высоко человѣка паче прочіихъ тварей превознеслъ, что не токмо Божественнымъ Своимъ совѣтомъ, но и по образу Своему сотворилъ его. Отъ сего познай, человѣче, высокое твое и дивное благородіе первое и непостижимую Бога нашего благость. \textit{По образу Божію созданъ еси}\footnote{ст.~27.}! 3)~Созданнымъ прародителямъ нашимъ, и въ раи, прекрасномъ сладостей мѣстѣ, учиненнымъ, далъ Богъ заповѣдь: \textit{отъ всякаго древа, еже въ раи, снѣдію снѣси: отъ древа же, еже разумѣти доброе и лукавое, не снѣсте отъ него: а въ оньже аще день снѣсте отъ него, смертію умрете}\footnote{2,~16 и 17.}. 4)~Заповѣди Божія прародители наши не сохранили: \textit{отъ заповѣданнаго древа вкусили}\footnote{3,~6.}. Къ сему преслушанію и законопреступленію прельстилъ ихъ змій лукавый и сатана, позавидѣвъ великому ихъ блаженству. И тако они, не послушавши Бога, Создателя своего, послушавши же прелестника и врага своего, весьма тяжко согрѣшили предъ Богомъ; чего ради лишилися милости Божіей и благодати, которою ихъ почтилъ Богъ въ созданіи; потеряли образъ Божій, и изъ святыхъ и праведныхъ сдѣлались нечистыми, скверными и грѣшными, и подпали праведному гнѣву Божію и всякому неблагополучію, временному и вѣчному. 5)~Сею прародителей нашихъ язвою грѣховною заразились и мы, сынове ихъ, и съ тою зачинаемся и раждаемся, якоже о семъ воздыхаетъ и сѣтуетъ Псаломникъ: \textit{въ беззаконіихъ зачатъ есмь, и во грѣсѣхъ роди мя мати моя}\footnote{Пс.~50,~7.}, и тако съ ними временному и вѣчному бѣдствію подвержены учинилися. Слѣдовало не токмо временно, но и вѣчно съ ними всѣмъ намъ умирать, горькую гнѣва Божія чашу и вѣчнаго мученія во адѣ пити, и у діавола, котораго злаго совѣта послушали, во власти и поруганіи быти. \textit{Оброцы бо грѣха смерть}, глаголетъ Апостолъ\footnote{Римл.~6,~23.}. 6)~Богъ, по великой Своей милости и непостижимой благости, \textit{не по беззаконіямъ нашимъ сотворилъ есть намъ, ниже по грѣхомъ нашимъ воздалъ есть намъ}\footnote{Пс.~102,~10.}. Еще прародителямъ нашимъ обѣщалъ послать избавленіе, и спасеніе имъ и всему міру, "--- что означается чрезъ оныя слова, змію отъ Бога сказанныя: \textit{Той} (Хрістосъ, благословенное сѣмя) \textit{твою сотретъ главу}\footnote{Быт.~3,~15.}. 7)~Сіе отеческое Свое благословеніе къ роду человѣческому и милостивое обѣщаніе многократно повторялъ и открывалъ Богъ святымъ патріархамъ, Аврааму, Исааку, Іакову и прочіимъ; открывалъ пророкамъ, и имъ повелѣвалъ возвѣщать и проповѣдывать грядущаго въ міръ Спасителя, какъ о томъ въ книгахъ Ветхаго Завѣта написано, въ Котораго Избавителя \textit{грядущаго} и вѣровали ветхозаконные отцы, и спаслися вѣровавшіе въ Него\footnote{Дѣян.~15,~11.}. 8)~Когда приближилося тое время, въ которое Богъ опредѣлилъ пріитить въ міръ Избавителю и явитися міру, "--- предъ пришествіемъ Его родился пророкъ отъ неплодныхъ и престарѣлыхъ родителей Захаріи священника и Елисаветы праведныхъ именемъ \textit{Іоаннъ}\footnote{Лук.~1,~57--63.}. Сей пророкъ посланъ отъ Бога уготовати путь Избавителю, грядущему въ міръ, якоже о томъ Захарія, отецъ его, Духомъ Святымъ сказалъ ему рожденному: \textit{и ты отроча пророкъ Вышняго наречешися; предъидеши бо предъ лицемъ Господнимъ, уготовати пути Его}\footnote{Лук.~1,~76.}. И самъ Богъ еще прежде о немъ предсказалъ чрезъ пророка: \textit{се Азъ посылаю ангела Моего предъ лицемъ Твоимъ, иже уготовитъ путь Твой предъ Тобою}\footnote{Мал.~3,~1; Марк.~1,~2.}. Откуду называется Предтечею отъ церкви святой, яко предъ пришествіемъ Хрістовымъ, какъ денница предъ солнцемъ, предтеклъ и проповѣдалъ приближившееся Солнце правды, Хріста. Сей пророкъ проповѣдалъ Хріста, уже пришедшаго въ міръ, глаголя: \textit{грядетъ Крѣплій мене, во слѣдъ мене, Емуже нѣсмь достоинъ преклонься разрѣшити ремень сапогъ Его}\footnote{Марк.~1,~7.}. И указывалъ на Хріста, глаголя: \textit{се Агнецъ Божій, вземляй грѣхи міра! Сей есть, о Немже азъ рѣхъ: по мнѣ грядетъ Мужъ, Иже предо мною бысть, яко первѣе мене бѣ}, и проч.\footnote{Іоан.~1,~29,~30 и слѣд.} "--- Пришествіе Хрістово въ міръ тако учинилося: по зачатіи Іоанна Предтечи во чревѣ Елисаветы, въ мѣсяцъ шестый, посланъ бысть ангелъ Гавріилъ отъ Бога къ Дѣвѣ Маріи, Которая тогда жилище свое имѣла во градѣ галилейскомъ Назаретѣ; и пришедъ къ ней, первѣе поздравилъ ее тако: \textit{радуйся Благодатная! Господь съ тобою: благословенна ты въ женахъ!} Потомъ объявилъ ей: \textit{се зачнеши во чревѣ, и родиши Сына и наречеши имя Ему Іисусъ}, и проч. Пресвятая Дѣва, услышавши о семъ неслыханномъ дѣлѣ, сказала ко ангелу: \textit{како будетъ сіе, когда я мужа не знаю?} А когда ангелъ отвѣщалъ ей: \textit{Духъ Святый найдетъ на тя, и сила Вышняго осѣнитъ тя}, и прочія слова во увѣреніе приложилъ, "--- соизволила на тое благодатная Дѣва: \textit{се раба Господня: буди мнѣ по глаголу твоему!} И тако Сынъ Божій Іисусъ Хрістосъ, благоволеніемъ небеснаго Своего Отца и наитіемъ и осѣненіемъ Святаго Духа, во чревѣ дѣвическомъ плотію зачался\footnote{Лук.~1,~24--38.}. Божіе сіе къ намъ грѣшнымъ и погибшимъ благоволеніе празднуется церковію Марта 25"~го дня. Сіе милосердое Божіе къ намъ благоволеніе называетъ церковь святая \textit{Благовѣщеніемъ}, потому что \textit{благая вѣсть} Дѣвѣ Богородицѣ и всей твари отъ небесъ на землю чрезъ ангела Божія принеслася и объявилася. Отсюду бо началося спасеніе наше. 10)~Наконецъ возсіяло Солнце правды изъ дѣвственныя утробы, родился отъ Дѣвы Сынъ Божій, явился на земли Царь небесный, \textit{Слово плоть бысть}, и Ветхій денми пеленами, яко младенецъ, повился, показался Невидимый, и Питаяй всякую плоть матернимъ млекомъ питатися изволилъ. При рожденіи Его ангели святые воспѣли славу и хвалу Божію: \textit{слава въ вышнихъ Богу, и на земли миръ, въ человѣцѣхъ благоволеніе}\footnote{2,~14.}! Рождество по плоти Сына Божія Спаса нашего празднуетъ церковь Декабря 25"~го дня. 11)~Рожденный Сынъ Божій въ осмый день обрѣзанъ по закону Моисеову, и наречено имя ему \textit{Іисусъ}\footnote{ст.~21.}, нареченное ангеломъ прежде даже не зачатся во чревѣ\footnote{Мѳ.~1,~21.}. \textit{Іисусъ} значитъ \textit{Спасителя, яко спасаетъ людей Своихъ отъ грѣхъ ихъ}\footnote{тамъ же.}. "--- Постоимъ здѣ, любезный читатель, и почудимся благости Божіей. Обрѣзался Тотъ, Который не имѣетъ скверны; пріялъ нарицаніе и имя Тотъ, Котораго родъ и имя неизреченно; нареклся \textit{Іисусъ}, да самымъ именемъ покажетъ, что Онъ есть \textit{Спаситель міра}, и ведетъ людей Своихъ не въ Хананею, якоже Іисусъ Навинъ, но въ небесное царствіе; нареклся сыномъ человѣческимъ Тотъ, Который на земли не знаетъ отца; зачался Тотъ, Который начала не имѣетъ; увидѣлся Тотъ, Который по Своему естеству невидимъ есть; осязался матерними руками Тотъ, Который ангеламъ неприступенъ есть; повился пеленами Тотъ, Который одѣваетъ небо облаками. Мене ради и тебе, человѣче, воспріялъ Сынъ Божій великое сіе посольство: \textit{благослови душе моя Господа!} 12)~Какъ четыредесять дней прошло отъ рожденія, святою Матерію принесенъ во храмъ Господень, *дабы поставити Его предъ Господемъ,* по закону Моисеову. Въ тое время, по смотрѣнію Божію, пришелъ въ храмъ Господень Симеонъ праведный, мужъ престарѣлый, которому обѣщано было не видѣти смерти, пока не увидитъ Хріста Господня. Только"=что онъ увидѣлъ младенца Іисуса, принесеннаго въ храмъ Божій, Духомъ Святымъ узналъ, что младенецъ Тотъ есть Мессія, отъ Бога обѣщанный и пророками проповѣданный; и тако взявши Его съ благоговѣніемъ на руки свои, благословилъ Бога и съ радостію воскликнулъ: \textit{нынѣ отпущаеши раба Твоего, Владыко, по глаголу Твоему, съ миромъ: яко видѣстѣ очи мои спасеніе Твое, еже еси уготовалъ предъ лицемъ всѣхъ людей}, и проч.\footnote{Лук.~2,~22--31 и слѣд.} И сіе"=то значитъ праздникъ Срѣтенія Господня, Февраля 2"~го дня. 13)~Будучи двунадесяти лѣтъ, Іисусъ, Сынъ Божій, въ праздникъ Пасхи взошелъ въ храмъ Божій, и, сѣдши посредѣ учителей, послушалъ ихъ и вопрошалъ ихъ. \textit{Ужасахуся же вси, послушающіи Его, о разумѣ и о отвѣтѣхъ Его}\footnote{Лук.~2,~42--47.}. Яко хотя и отрокъ былъ возрастомъ, но въ такомъ возрастѣ дѣйствовала и показывалася ѵпостасная премудрость Божія, яко \textit{Онъ есть Божія сила и Божія премудрость}\footnote{1~Кор.~1,~24.}. 14)~Лѣтъ яко тридесяти пожитія Своего на земли\footnote{Лук.~3,~23.}, крестился во Іорданѣ рѣкѣ отъ пророка и Предтечи Іоанна (о которомъ помянуто въ пунктѣ 8"~мъ). Тогда явился надъ Нимъ Духъ Святый въ видѣ голубинѣ, и Отецъ небесный гласомъ съ небесе засвидѣтельствовалъ: \textit{Сей есть Сынъ Мой возлюбленный, о Немже благоволихъ}\footnote{Мѳ.~3,~16 и 17; Марк.~1,~9--11; Лук.~3,~21 и 22.}. Здѣ тайна Пресвятыя Троицы явно открылася намъ. \textit{Сынъ во плоти} крестился; \textit{Духъ Святый} на Него сошелъ; \textit{Отецъ} съ небесе свидѣтельствовалъ. Тако Тріѵпостасный Богъ "--- Отецъ, Сынъ и Духъ Святый, Создатель нашъ, о спасеніи нашемъ промышляетъ! Сіе спасительное \textit{Богоявленіе} празднуется Января 6"~го дня. "--- Примѣчай здѣ, православный хрістіанине: Хрістосъ крестился не такъ, какъ мы крещаемся. Мы крещеніемъ омываемся отъ сквернъ грѣховныхъ и освящаемся. Хрістосъ грѣха не имѣлъ и потому не было Ему нужды омыватися отъ грѣха; но Своимъ крещеніемъ наше крещеніе освятилъ и установилъ. 15)~По крещеніи возведенъ былъ Духомъ въ пустыню Сынъ Божій, \textit{искуситися отъ діавола}, и искушаемъ былъ отъ Него, и побѣдилъ искусителя\footnote{Мѳ.~4,~1--11.}. Постой и здѣ, и почудись благости и смиренію Сына Божія. Богъ будучи, попустилъ злому и нечистому духу приступить къ Себѣ и искушать различно; а тако показалъ, что всякому, рожденному водою и Духомъ и начинающему Господу работать, слѣдуетъ искушеніе, якоже глаголетъ Сирахъ: \textit{чадо! аще приступаеши работати Господеви Богу, уготови душу твою во искушеніе}\footnote{Сир.~2,~1.}. Побѣдилъ Онъ искусителя, "--- намъ побѣдилъ, да мы побѣжденнаго отъ Него силою Его побѣждаемъ. 16)~По искушеніи претерпѣнномъ и побѣжденіи вышелъ Хрістосъ изъ пустыни и пошелъ на проповѣдь святаго Евангелія, которое отъ нѣдръ небеснаго Своего Отца на землю принеслъ и съ нимъ всю благость Его и человѣколюбіе открылъ намъ, глаголя: \textit{покайтеся и вѣруйте во Евангеліе}\footnote{Марк.~1,~15.}. 17)~Въ началѣ проповѣди Своея собралъ дванадесять учениковъ изъ подлыхъ и простыхъ людей, которыхъ \textit{Апостолами} назвалъ, да будутъ очевидные свидѣтели Его всему міру, якоже Самъ сказалъ имъ по воскресеніи своемъ: \textit{будете Ми свидѣтели во Іерусалимѣ же, во всей Іудеи и Самаріи и даже до послѣднихъ земли}\footnote{Дѣян.~1,~8.}. И Апостолъ Іоаннъ съ братіею своею, прочіими апостолами, глаголетъ: \textit{и животъ явися, и видѣхомъ, и свидѣтельствуемъ, и возвѣщаемъ вамъ животъ вѣчный, Иже бѣ у Отца, и явися намъ}, то"=есть Хрістосъ, Слово Отчее\footnote{1~Іоан.~1,~2.}; и паки: \textit{и мы видѣхомъ и свидѣтельствуемъ, яко Отецъ посла Сына Спасителя міру}\footnote{4,~15.}. 18)~Съ сими дванадесятьми переходя отъ мѣста на мѣсто, и отъ града во градъ, и отъ веси въ весь, приносилъ святое Свое Евангеліе и разсѣвалъ спасительное слова Божія сѣмя; насаждалъ на сердцахъ человѣческихъ вѣру и показывалъ путь къ вѣчному животу и небесному царствію; училъ истинному богопочитанію и поклоненію, которое бываетъ \textit{духомъ и истиною}\footnote{Іоан.~4,~24.}; научалъ любви къ Богу и ближнему, кротости, смиренію, терпѣнію, милосердію и прочіимъ добродѣтелямъ. Чему словомъ училъ, тое дѣломъ и примѣромъ Своимъ показывалъ. Откуду учитися хрістіанскому непорочному житію отъ Себе велѣлъ, яко живаго образа и чистаго зерцала. \textit{Научитеся отъ Мене, яко кротокъ есмь и смиренъ сердцемъ}\footnote{Мѳ.~11,~29.}; и паки: \textit{образъ дахъ вамъ, да якоже Азъ сотворихъ вамъ, и вы творите}\footnote{Іоан.~13,~15.}. 19)~Сотворилъ преславная чудеса: слѣпыхъ, глухихъ, нѣмыхъ, разслабленныхъ исцѣлялъ, прокаженныхъ очищалъ, демоновъ изгонялъ, мертвыхъ воскрешалъ, многія тысячи народа въ пустынѣ малымъ хлѣбомъ питалъ, и прочія вышеестественныя дѣла творилъ. Тѣмъ показывалъ людямъ, что Онъ есть Мессія, отъ Бога обѣщанный и пророками проповѣданный, якоже самъ Іоанну Предтечѣ, чрезъ учениковъ Своихъ вопрошающему Его: \textit{Ты ли еси грядый, или иного чаемъ}, "--- отвѣщалъ имъ: \textit{шедше возвѣстите Іоаннови, яже слышите и видите: слѣпіи прозираютъ, и хроміи ходятъ; прокаженніи очищаются, и глусіи слышатъ; мертвіи востаютъ, и нищіи благовѣствуютъ}\footnote{Мѳ.~11,~2--5.}. Какъ бы сказалъ: изъ дѣлъ Моихъ примѣчайте, что пришло тое время, о которомъ Исаія прорекъ: \textit{тогда отверзутся очи слѣпыхъ и уши глухихъ услышатъ; тогда скочитъ хромый, яко елень, и ясенъ будетъ языкъ гугнивыхъ}\footnote{35,~5 и 6.}. Здѣ примѣчай, хрістіанине: Хрістосъ чудеса творилъ не такъ, какъ святые Его творятъ. Святые молитвою и именемъ Хрістовымъ творятъ чудеса, а не сами собою; Хрістосъ же Самъ чрезъ Себе, яко Богъ, творилъ чудеса. Чудеса бо творить единаго Бога есть. 20)~Народъ, слышавши небесное, спасительное и утѣшительное Хрістово ученіе, и видя Его преславная чудеса, въ великомъ множествѣ за Нимъ, какъ овцы за добрымъ пастыремъ, ходили, иные исцѣлитися отъ недуговъ своихъ, иные послушати святаго ученія Его желая, и всѣ получали отъ Него, чего съ вѣрою желали и искали. \textit{И весь народъ искаше прикасатися Ему: яко сила отъ Него исхождаше, и исцѣляше вся}\footnote{Лук.~6,~19.}. 21)~Услышали гласъ Пастыря добраго и заблуждшія овцы Его, и познали Его. Блудницы, мытари и прочіи грѣшники приходили къ Нему, якоже глаголетъ Лука святый: \textit{бяху приближающеся Ему вси мытаріе и грѣшницы, послушати Его}\footnote{15,~1.}. И пріимали душевныхъ своихъ язвъ исцѣленіе, отпущеніе грѣховъ. Тако, хрістіанине, Хрістосъ Сынъ Божій, живучи на земли и проповѣдая Евангеліе царствія, \textit{бысть всѣмъ вся}: сущимъ во тьмѣ свѣтъ, слѣпымъ прозрѣніе, глухимъ слышаніе, нѣмымъ глаголаніе, разслабленнымъ исцѣленіе, прокаженнымъ очищеніе, мертвымъ воскресеніе и животъ, грѣшникамъ спасеніе. Приступимъ и мы къ Нему, не на земли ходящему, но одесную Бога сѣдящему; и приступимъ вѣрою сердечною, не ногами, и объявивши язвы и недуги наша, возвысимъ къ Нему гласъ: \textit{Іисусе, Сыне Божій, помилуй насъ!} Не презритъ и насъ Онъ, Который не презрѣлъ никого, съ вѣрою къ Нему приходящаго. "--- Но посмотримъ, что воздали Ему люди Его, которыхъ Онъ пришелъ взыскати и спасти. 22)~Слава о Немъ, яко неслыханномъ учителѣ и чудотворцѣ, вездѣ прошедшая, подвигла на зависть фарисеевъ, книжниковъ и законоучителей іудейскихъ; чего ради различно хулили Сына Божія, и тщались славу Его помрачить; но \textit{свѣтъ во тьмѣ свѣтится, и тьма его не объятъ}\footnote{Іоан.~1,~5.}. Зависти другиня "--- злоба послѣдовала, и научила ихъ къ убійству Мессіи своего; чего ради искали Его умертвить, но не могли, яко еще не пришло время страданія Его. 23)~Провидя Сынъ Божій, яко слѣдуетъ Ему чашу страданія испить отъ беззаконныхъ людей своихъ, предсказывалъ о томъ ученикамъ своимъ: \textit{яко подобаетъ Ему ити во Іерусалимъ, и много пострадати отъ старецъ и архіерей и книжникъ, и убіену быти, и въ третій день востати}\footnote{Мѳ.~16,~21 и 24.}. Когда приближалося спасительнаго Его страданія время, вземше трехъ учениковъ Своихъ, Петра, Іоанна и Іакова, взошелъ на высокую гору; гдѣ, какъ началъ молитися, \textit{лице Его, яко солнце, просвѣтилося, и ризы Его были, яко свѣтъ}. Тамо явилися два пророка, Моисей и Илія, и съ Нимъ разговаривали, которыхъ видѣли и апостоли. Услышали потомъ и гласъ, отъ велелѣпныя славы къ Нему пришедшій таковый: \textit{Сей есть Сынъ Мой возлюбленный, о Немже благоволихъ: Того послушайте!} Отъ гласа того Божественнаго апостоли устрашившися, ницъ на землю пали\footnote{17,~1--6; Лук.~9,~28--35.}. Помедлимъ и здѣ, любезный читатель, и вперимъ умныя наши очи на Божественную славу преобразившагося Господа. Увидѣвши апостоли славу Хрістову, сколько могли видѣть, такую радость и сладость въ сердцахъ почувствовали, что Петръ святый не могъ умолчать, но воскликнулъ ко Хрісту: \textit{Господи, добро есть намъ здѣ быти}\footnote{Мѳ.~17,~4.}. Часть нѣкую славы Хрістовой увидѣли апостоли, и толико усладилися: коликая уже радость въ избранныхъ Божіихъ будетъ въ будущемъ вѣкѣ, гдѣ всю Божественную Свою славу покажетъ имъ! Потщимся и мы славу оную видѣть благодатію Его, и славу міра сего презримъ. "--- Преображеніе святѣйшія плоти Сына Божія празднуемъ Августа 6"~го дня. 25)~Когда уже пришло время Хрісту Сыну Божію пострадать за спасеніе міра, всѣдши на ослицѣ, во Іерусалимъ поѣхалъ. Люди, прослышавши о прихожденіи Его, вышли во срѣтеніе Ему изъ города, изъ которыхъ иные постилали вѣтви по пути, по которому ѣхалъ, иные полагали одежды свои, иные держали въ рукахъ ваія отъ финиковыхъ древесъ, и восклицали: \textit{осанна Сыну Давидову! Благословенъ грядый во имя Господне! Осанна въ вышнихъ}\footnote{Мѳ.~21,~7--9; Марк.~11,~7--10; Лук.~19,~35--38; Іоан.~12,~14 и 15.}! Примѣчай здѣ, хрістіанине, съ какою славою Хрістосъ шелъ во Іерусалимъ. На чтожъ? На вольное Свое страданіе и поносную крестную смерть. За кого? За мене и тебе погибшаго. Отсюду познай презѣльную Его любовь, жаждущую нашего спасенія. Примѣчай паки человѣческое непостоянство. Народъ, который восклицалъ Хрісту: \textit{благословенъ грядый во имя Господне}, той послѣ вопилъ на Хріста къ Пилату: \textit{возми, возми, распни Его}\footnote{19,~15.}! Отъ сего учись славу и похвалу людскую презирать, яко кого нынѣ хвалятъ, того послѣ проклинаютъ. "--- Но посмотримъ, что послѣдовало народному ко Хрісту восклицанію. 26)~Восклицаніе народа, Хрісту учиненное, на большую зависть и злобу подвигло книжниковъ, фарисеевъ и архіереевъ іудейскихъ. Чего ради какъ прежде, такъ наипаче тогда собравшися совѣтовали, како Его погубить, Который ихъ пришелъ спасти; но народа ради не могли того явно учинить, понеже всѣ Его держались, послушающе ученія Его. Тогда сыскался сосудъ погибели, единъ отъ двунадесяти ученикъ неблагодарный, Іуда Искаріотскій. Сей присталъ къ пагубному совѣту ихъ, и за тридесять сребрениковъ имъ безцѣннаго своего Учителя продалъ, и лобызаніемъ и лестію предалъ имъ въ руки нощію\footnote{Мѳ.~26,~14--16,~47--50.}. Примѣчай здѣ, хрістіанине, что сребролюбіе дѣлаетъ своимъ любителямъ. Іуда не ужаснулся продать за такъ малую цѣну безцѣннаго Хріста, Благодѣтеля и Учителя своего, и тако купилъ себѣ вѣчную погибель. Тоежде и прочіимъ сребролюбцамъ послѣдуетъ, которые не ужасаются всякое зло дѣлать, чтобы обогатиться: \textit{корень бо всѣмъ злымъ сребролюбіе есть}\footnote{1~Тим.~6,~10.}. 27)~Преданіе Іисуса Хріста Сына Божія тако учинилося отъ предателя: по яденіи ветхозаконныя пасхи и тайной вечери, о чемъ воспоминаніе творитъ церковь въ четвертокъ великій, Сынъ Божій пришелъ съ апостолами въ весь, нарицаемую Геѳсиманія, и взявши Петра, Іакова и Іоанна, учениковъ Своихъ, яко избраннѣйшихъ, отшедъ нѣсколько отъ прочіихъ апостоловъ, \textit{началъ скорбѣти и тужити}\footnote{Мѳ.~26,~36--38.}. Посемъ отъ нихъ, мало отступилъ, молился Отцу Своему небесному съ колѣнопреклоненіемъ и преклоненіемъ лица Своего на землю; и таковую молитву троекратно учинилъ, якоже святый Евангелистъ повѣствуетъ\footnote{Мѳ.~26,~39--44.}. Іуда предатель, понеже зналъ, что Іисусъ многажды на томъ мѣстѣ собирался съ учениками Своими, взявши отъ архіерей и фарисей слугъ, пришелъ туды со свѣтилами и свѣщами и оружіемъ; пришелъ же со свѣщами потому, что нощь была тогда\footnote{Іоан.~18,~2--4.}. А чтобы удобнѣе предать ему Сына Божія въ руки ихъ, далъ имъ знаменіе таковое: \textit{котораго}"=де \textit{я буду лобызать, той есть: имите его}\footnote{Мѳ.~26,~48.}. И тако приступивши ко Хрісту, сказалъ: \textit{радуйся равви} (учителю)! \textit{и облобызалъ Его}\footnote{ст.~49.}. И таковымъ образомъ рабъ и льстецъ предалъ Господа своего въ руки вражія. Тогда пришедшіе яти Хріста, \textit{связали Его}\footnote{Іоан.~18,~12.}. Апостоли, видѣвше таковое злодѣйское начинаніе, \textit{оставили Его и бѣжали}\footnote{Марк.~14,~50.}. И тако взятый Хрістосъ руками беззаконныхъ и связанный приведенъ къ Каіафѣ архіерею, гдѣ книжники и старцы собралися\footnote{Мѳ.~26,~57.}. 28)~Здѣ приведенный Сынъ Божій неправедный судъ терпѣлъ отъ беззаконныхъ судей, гдѣ на Него различныя и ложныя клеветы отъ лжесвидѣтелей приносимы были, противу которыхъ Онъ молчалъ: \textit{Іисусъ же молчаше}\footnote{ст.~59--63.}. Потомъ осудили на смерть Судію живыхъ и мертвыхъ: \textit{они же отвѣщавше рѣша: повинень есть смерти}\footnote{66.}. Осужденный на смерть Безсмертный, смерти и живота Господь, былъ заплеваемъ, по ланитамъ ударяемъ и различно мучимъ и посмѣваемъ чрезъ всю тую нощь\footnote{67.}. 29)~Какъ насталъ день, связавши Сына Божія, привели къ Пилату игемону, и просили у него умертвить смертію крестною, который, хотя и отрицался тое учинить, видя Его неповинность; однакожъ, понеже сильно настояли и устрашали его: \textit{аще"=}де \textit{сего пустиши, нѣси другъ кесаревъ: всякъ, иже царя себе творитъ, противится кесарю}\footnote{Іоан.~19,~12.}, "--- учинилъ по волѣ и прошенію ихъ, предалъ въ руки беззаконныхъ неповиннаго, да распнется\footnote{ст.~16.}. 30)~Тогда, поемши Сына Божія поруганнаго, посмѣяннаго, уязвленнаго, умученнаго, и, возложивши крестъ Его на Него, нашими грѣхами отягченный, повели на большее безчестіе и страданіе, то"=есть, на распятіе: \textit{и нося крестъ Свой, изыде на глаголемое лобное мѣсто}\footnote{ст.~17.}. И тако Іисусъ Хрістосъ, Сынъ Божій, по пріятіи многихъ безчестій, насмѣяній, хуленій, уязвленій, мученій, распятъ на крестѣ между двумя разбойниками, и смертію крестною заключилъ подвигъ Свой, насъ ради и нашего ради спасенія воспріятый\footnote{Іоан.~19,~18.}. День сей, въ который Хрістосъ пострадалъ, отъ церкви святой нарицается пятокъ великій, "--- потому что великое спасенія нашего дѣло въ немъ Самъ Собою совершилъ Сынъ Божій, якоже Самъ, умирая на крестѣ, возгласилъ: \textit{совершишася}\footnote{ст.~30.}! 31)~Смерти Хрістовой послѣдовали знаменія и чудеса сія: \textit{завѣса церковная раздралася на двое, съ вышняго края до нижняго: и земля потряслася, и каменіе распалося: и гробы отверзлися, и многа тѣлеса усопшихъ святыхъ востали}\footnote{Мѳ.~27,~51,~52 и слѣд.}. 32)~Умершій за спасеніе наше Сынъ Божій погребенъ былъ, и почивала святѣйшая Его плоть во гробѣ три дни и три нощи, якоже Самъ о томъ превозвѣстилъ\textit{: якоже бѣ Іона во чревѣ китовѣ три дни и три нощи: тако будетъ и Сынъ человѣческій въ сердцы земли, три дни и три нощи}\footnote{12,~40.}. 33)~Въ третій день по крестной и животворящей Своей смерти восталъ отъ мертвыхъ Сынъ Божій\footnote{Лук.~24,~7 и 21; 1~Кор.~15,~4; Дѣян.~10,~40.}. Востаніемъ отъ мертвыхъ показалъ намъ Хрістосъ, что Онъ есть смерти, грѣха, ада, діавола и всѣхъ нашихъ грѣховъ побѣдитель; и тѣмъ насъ увѣрилъ, что и мы умершіи востанемъ, когда будетъ время, отъ Бога опредѣленное. Сей воскресенія Хрістова день называется \textit{Пасха}. Пасха значитъ \textit{прехожденіе}, ибо вѣрующіе во Хріста, смертію и воскресеніемъ Его отъ смерти къ животу вѣчному, и отъ земли къ небеси \textit{преходятъ}, какъ церковь святая поетъ въ день Воскресенія Хрістова. 34)~По воскресеніи Своемъ Хрістосъ не скоро отъ земли отлучился, но четыредесять дней пребывалъ, и являлся чрезъ тыя дни апостоламъ святымъ и прочіимъ вѣрнымъ, глаголя имъ о царствіи Божіи\footnote{Дѣян.~1,~3.}. 35)~По четыредесяти дняхъ, отъ горы Елеонскія, которая есть близъ Іерусалима, предъ апостолами вознеслся на небо, и сѣлъ одесную Бога Отца\footnote{Марк.~16,~19; Лук.~24,~51; Дѣян.~1,~9--11.}. "--- Постоимъ и здѣ, возлюбленный хрістіанине, и почудимся великости благости Божіей. Смотри, въ какую высоту превознеслося естество человѣческое, ѵпостаснѣ соединенное Слову Божію! Одесную Бога сѣдитъ Хрістосъ съ прославленною Своею святою плотію. Насъ ради на земли родился, пожилъ, пострадалъ, умеръ, воскресъ, вознеслся на небо, и сѣдитъ во славѣ Отчей, и ходатайствуетъ о насъ\footnote{Римл.~8,~34.}. Благодари о семъ Богу, благоволившему тако. Радуйся духомъ и торжествуй, и прихоти міра сего, какъ калъ, поплевавши, тщись достигнуть благодатію Божіею въ часть спасаемыхъ, и лицемъ къ лицу видѣти во славѣ царствующаго Сына Божія, Котораго нынѣ вѣрою видиши. Хрістосъ, Глава вѣрныхъ, вознеслся на небо: вознесутся и духовные уды Его, истинные хрістіане. Прославился Хрістосъ: спрославятся и вѣрніи раби Его, и предстанутъ Ему, какъ Царю своему, въ ризахъ позлащенныхъ одѣяны и преиспещренны\footnote{Пс.~44,~14.}; облекутся чистою оправданія Его порфирою, яко сыны царевы, и воцарятся съ Нимъ во вѣки вѣковъ. Разумѣй сію истинныхъ хрістіанъ славу, и пустые міра сего титулы, ранги, славу, честь, которыми сыны вѣка сего утѣшаются, вмѣняй какъ ничто. Истинно сіе признаешь, когда вѣрою размыслишь будущую сыновъ Божіихъ славу. Да сподобитъ насъ части тоя Искупитель нашъ, Сынъ Божій. 36)~Вознесшійся на небо Хрістосъ послалъ отъ Отца Своего небеснаго Духа Святаго на апостоловъ и на всякую плоть, якоже обѣщался: \textit{и Азъ умолю Отца, и инаго утѣшителя дастъ вамъ, да будетъ съ вами въ вѣкъ, Духъ истины, Егоже міръ не можетъ пріяти}\footnote{Іоан.~14,~16,~17.}. Но и Самъ обѣщался неотлучно пребывати съ вѣрными до скончанія вѣка\footnote{Мѳ.~28,~20.}: ибо \textit{возлюбль Своя сущія въ мірѣ, до конца возлюби ихъ}\footnote{Іоан.~13,~1.}. 37)~Апостоли святые, Духомъ Святымъ просвѣтившися и укрѣпившися, во всю вселенную сіе неизреченное Божіей благости богатство пронесли; Сына Божія, плотію въ міръ пришедшаго, проповѣдали; вѣру въ Него на сердцахъ человѣческихъ насадили, идолобѣсіе опровергли; тьмою невѣдѣнія Божія языки помраченные просвѣтили; очи сердечныя ослѣпленныхъ къ познанію истиннаго Бога отверзли; основали церковь Божію на лицѣ всея земли; едино благословенное стадо изъ разныхъ языковъ, какъ дикихъ звѣрей, собрали; Троицу Святую "--- Отца и Сына и Святаго Духа, единаго Бога, почитати научили; огнь Божіей любви въ сердцахъ вѣрующихъ возжгли, и надеждою живота и блаженства будущаго вѣка утвердили, якоже написано о нихъ: \textit{они же изшедше проповѣдаша всюду, Господу поспѣшствующу и слово утверждающу послѣдствующими знаменми}\footnote{Марк.~16,~20.}. 38)~Спасительнымъ богоглаголивыхъ сихъ проповѣдниковъ ученіемъ церковь святая, по лицу всея земли разсѣянная, просвѣщается, наставляется, созидается, спасается; и отъ плачевной міра сего юдоли, въ горній и небесный Сіонъ, къ Царю своему и безсмертному Жениху преселяется, помощію Святаго и животворящаго Духа, гдѣ не вѣрою, но лицемъ къ лицу сподобляется чистѣйшую и святѣйшую Его доброту видѣти. 39)~По блаженной кончинѣ апостолъ святыхъ, учинились преемники ихъ "--- святители Хрістовы, пастыри и учители, которые тѣмъ же ученія ихъ словомъ церковь Хрістову пасутъ, наставляютъ, утверждаютъ и предпосылаютъ, съ помощію Божіею, въ небесную ограду; и сами, вѣру сохранившіе и теченіе скончавшіе, туды же по трудахъ и подвигахъ преселяются, "--- что и до конца вѣка имѣетъ быть, яко \textit{врата адова не одолѣютъ} созданной на камени вѣры Хрістовой \textit{церкви}\footnote{Матѳ.~16,~18.}.

Се видимъ, \textit{православный хрістіанине}, первое пришествіе Спасителя нашего въ міръ. Ожидаемъ еще и \textit{втораго}, по неложному обѣщанію: \textit{Сей Іисусъ, вознесыйся отъ васъ на небо, такожде пріидетъ, имже образомъ видѣсте Его идущаго на небо}, ангели Господни глаголали апостоламъ, смотрящимъ на возносимаго Хріста\footnote{Дѣян.~1,~11.}. Два сіи пришествія Хрістовы не равны, какъ видимъ въ святомъ Его словѣ. \textit{Первое} было тихое: \textit{сниде бо яко дождь на руно, и яко капля, каплющая на землю}\footnote{Пс.~71,~6.}. \textit{Второе} страшное и скорое: \textit{якоже бо молнія исходитъ отъ востокъ и является до западъ, тако будетъ пришествіе Сына человѣческаго}, глаголетъ Хрістосъ о Себѣ\footnote{Матѳ.~24,~27.}. \textit{Первое} было смиренное, ибо пришелъ въ человѣческомъ и рабіемъ зракѣ къ человѣкамъ и рабамъ бесѣдовати, дабы смиреніемъ Своимъ всѣхъ привлечь къ Себѣ. \textit{Второе} будетъ славное, ибо \textit{пріидетъ Сынъ человѣческій во славѣ Своей, и вси святіи ангели съ Нимъ}\footnote{Матѳ.~25,~31.}. Въ \textit{первомъ} пришелъ \textit{взыскати и спасти погибшаго}\footnote{Лук.~19,~10.}. \textit{Во второмъ} пріидетъ судити и воздати всѣмъ по дѣламъ ихъ: вѣрныхъ, пребывшихъ до конца въ вѣрѣ, ввести въ вѣчное блаженство и небесное царствіе, невѣрныхъ и несоблюдшихъ вѣры вѣчному предати наказанію. И тако \textit{пойдутъ сіи въ муку вѣчную: праведницы же въ животъ вѣчный}\footnote{Матѳ.~25,~46.}. Тако все смотрѣніе Сына окончится. Тогда начнется блаженная на небеси, пренеблагополучная и бѣдственная во адѣ вѣчность; тогда одни царствовать и веселиться безъ конца, другіе бѣдствовать и страдать безъ конца будутъ. Примѣчай здѣ, хрістіанине, что Хрістосъ, Который пришелъ въ міръ грѣшниковъ призвати на покаяніе, пострадалъ и умеръ за грѣшниковъ, Тойже будетъ и судить непокаявшихся грѣшниковъ; и колико тихъ, смиренъ, кротокъ и милостивъ былъ ко грѣшникамъ въ первомъ пришествіи, толико гнѣва и ярости изліетъ на непокаявшихся грѣшниковъ, толикую Его благодать презрѣвшихъ, такъ что возглаголютъ горамъ и каменію: \textit{падите на ны, и покрыйте ны отъ лица сѣдящаго на престолѣ, и отъ гнѣва Агнча: яко пріиде день великій гнѣва Его, и кто можетъ стати}\footnote{Апок.~6,~16 и 17.}?

\paragraph*{§\:339.} Видѣлъ ты, хрістіанине, вкратцѣ представленное Сына Божія въ міръ пришествіе, о которомъ святое Его Евангеліе обстоятельно намъ повѣствуетъ. Но знай, что слышать только Хріста въ міръ пришедшаго, а не имѣть въ Него истинныя и живыя вѣры, ничего не пользуетъ. Ибо многіе и язычники слышатъ, что Хрістосъ въ міръ пришелъ и чудеса сотворилъ, но не вѣруютъ; паче же и самый діаволъ со злыми аггелами своими знаетъ о томъ, и святаго имени Его трепещетъ, но сіе знаніе ихъ ничего не пользуетъ. Тако и хрістіанамъ знаніе Хрістова въ міръ пришествія не пользуетъ, которые не имѣютъ въ сердцѣ живыя въ Него вѣры. А чтобы живую во Хріста имѣть вѣру, должно всякому познать и признать душевную свою бѣдность и окаянство, какъ сказано выше. Истинная бо во Хріста вѣра подаетъ живое и дѣйствительное души утѣшеніе. А чтобы утѣшеніе воспріять, должно почувствовать скорбь, печаль и страхъ суда. Утѣшеніе бо "--- печальнаго утѣшеніе есть, и врачевство болящему подается, якоже глаголетъ Хрістосъ: \textit{не требуютъ здравіи врача, но болящіи}\footnote{Матѳ.~11,~12.}. Истинное праотеческаго грѣха и отъ того всякаго бѣдствія послѣдующаго познаніе не попускаетъ человѣку веселитися безумно, но содержитъ его въ сокрушеніи сердечномъ, страхѣ и печали. Самъ разсуди, какъ возможно тому радоватися, котораго внутрь содержитъ неисцѣльная болѣзнь и очевидною грозитъ смертію. Праотеческаго грѣха зло неисцѣльно заразило душу нашу со всѣми ея силами, такъ что умомъ слѣпъ, волею преслушливъ, сердцемъ отвращенъ отъ Бога учинился человѣкъ. Откуду воспослѣдовало, что между добромъ и зломъ распознать не можетъ, и часто вмѣсто добра зло избираетъ, добродѣтель порокомъ, и порокъ добродѣтелію называетъ; радуется о томъ, что ему зло и вредно, печалится о томъ, что ему полезно. Сія пагубная тьма разумъ человѣческій объяла. Сердце и воля человѣческая не иное что замышляетъ и хощетъ, какъ только противное волѣ Божіей, \textit{зане прилежитъ помышленіе человѣку прилѣжно на злая отъ юности его}\footnote{Быт.~8,~21.}. На маломъ дѣтищѣ примѣчай зло сіе, како въ немъ злое сіе сѣмя прорастаетъ плоды свои злые. Какое въ немъ показуется самолюбіе, гнѣвъ, ярость, зависть и вражда! какъ во гнѣвѣ кричитъ и ярится! Сіи суть плоды зміинаго сѣмени, на сердцѣ человѣческомъ посѣяннаго. Кто бы могъ познать зло тое, въ сердцѣ человѣческомъ крыющееся, когда бы само себе чрезъ внѣшніе знаки не оказывало? Чѣмъ болѣе растетъ человѣкъ мѣрою возраста, тѣмъ большее возникаетъ изъ сердца его зло; и при умноженіи лѣтъ умножаются и поднимаются душепагубныя страсти и похоти плотскія. Тутъ является блудническое похотѣніе, любочестіе, гордость, пышность, самомнѣніе, высокоуміе, злоба, славолюбіе, сребролюбіе; показывается лицемѣріе, ложь, лукавство, лесть, хитрость, пронырство, хищеніе и всякая неправда и всякое зло. Все сіе не къ иному чему, какъ къ вѣчной погибели и смерти ведетъ бѣднаго человѣка. Данъ былъ отъ Бога законъ, чтобы зло сіе воспятить, и человѣка, въ вѣчную погибель идущаго, удержать; но онъ только немощь человѣческую показывалъ, *а не врачевалъ; обличалъ*, а не помогалъ; грѣхъ показывалъ, но грѣха не отнималъ; казнію угрожалъ, но отъ казни не освобождалъ. Сего ради святое Божіе слово заключило, что \textit{вси согрѣшиша и лишени суть славы Божіей, и тако весь міръ учинился повиненъ Богови}\footnote{Римл.~3,~23 и 19.}. А оттуду слѣдовала всему міру казнь вѣчная, по строгости правды Божіей, яко Богъ вѣчный и безконечный грѣхами міра прогнѣванъ и огорченъ. Отъ сего тяжкаго долга никто отъ человѣкъ самъ собою никакимъ образомъ свободитися не моглъ, какъ бы ни тщался о томъ; яко Богъ, огорченный, прогнѣванный и обезчещенный человѣческимъ грѣхомъ, есть \textit{безконечный}. Но слѣдовало неотмѣнно всякому заключеннымъ быть въ темницу \textit{вѣчную} за долгъ свой, и платить правдѣ Божіей раздраженной \textit{вѣчною} казнію. Но и угодить Богу никто отъ человѣкъ не моглъ самъ собою, яко всякъ \textit{естествомъ чадо гнѣва} раждается\footnote{Еф.~2,~3.}, яко \textit{въ беззаконіихъ зачинается, и во грѣсѣхъ раждается}\footnote{Пс.~50,~7.}. И закона Божія, который должно совершенно исполнить, никто совершенно исполнить не моглъ, и не можетъ, праотческимъ грѣхомъ ко всякому злу преклоняемый и влекомый, "--- которое зло чувствуютъ во удесѣхъ своихъ и самые отрожденные\footnote{Римл.~7,~18--24.}. Откуду и самые святые, симъ зломъ порѣваемые и подстрѣкаемые, въ тяжкіе грѣхи падали и падаютъ, якоже въ священной и церковной исторіи читаемъ. А сіи падежи показуютъ тое, что всякій человѣкъ всѣ грѣхи можетъ дѣлать, когда благодать Божія не укрѣпитъ его и не поможетъ ему. Отсюду научись, хрістіанине, познавать немощь твою, бѣдность и окаянство твое духовное. Отсюду примѣчай, коль великое зло, зло, котораго до смерти оплакати не можемъ, грѣхъ Адамовъ, сѣмя зміино ядовитое и смертоносное, разліянное по всѣмъ силамъ душевнымъ и тѣлеснымъ, "--- коль великое, глаголю, зло крыется въ сердцѣ нашемъ, отъ котораго избавитися, и котораго побѣдити силою своею не можемъ. Страшно и жалостно описывается въ святомъ Писаніи человѣкъ, который Святымъ Духомъ не отрожденъ. \textit{Рече безуменъ въ сердцѣ своемъ: нѣсть Богъ. Растлѣша и омерзишася въ начинаніихъ: нѣсть творяй благостыню. Господь съ небесе приниче на сыны человѣческія, видѣти, аще есть разумѣваяй, или взыскаяй Бога. Вси уклонишася, вкупѣ неключими быша: нѣсть творяй благостыню, нѣсть до единаго}\footnote{Пс.~13,~1--3.}. И паки: \textit{гробъ отверстъ гортань ихъ, языки своими льщаху; ядъ аспидовъ подъ устнами ихъ; ихже уста клятвы и горести полна суть; скоры ноги ихъ проліяти кровь. Сокрушеніе и озлобленіе на путехъ ихъ; и пути мирнаго не познаша. Нѣсть страха Божія предъ очима ихъ}, глаголетъ Апостолъ о Іудеяхъ, которые законъ Божій имъ преданный имѣли, но беззаконно жили, якоже придаетъ далѣе: \textit{вѣмы же, яко елика законъ глаголетъ, сущимъ въ законѣ глаголетъ, да всяка уста заградятся, и повиненъ будетъ весь міръ Богови}\footnote{Римл.~3,~11--19.}. Аще убо о тѣхъ, которые имѣли законъ Божій и хвалились о Бозѣ, сія беззаконія и растлѣнія глаголются, что уже сказать о язычникахъ, которые, какъ дикіе звѣри, жили и дѣлали тое только, что чувства ихъ услаждало, и что неразумное сердце ихъ замышляло и похотствовало? Ужасная заблужденія тьма покрывала и обымала сердца ихъ. Мало имъ было солнце, луну, звѣзды и прочія созданія Божія вмѣсто Бога почитати: страсти самыя, нечистоту, піянство, гнѣвъ, ярость, свирѣпость обоготворили. Все сіе зло и мерзость отъ первороднаго грѣха, въ сердцѣ человѣческомъ живущаго, произошла и разлилася. Всякій бо \textit{идолъ} въ себѣ \textit{ничтоже есть}\footnote{1~Кор.~8,~4.}. Сребро, злато, древо, животное суть созданіе Божіе, и какъ на пользу отъ Создателя устроенное; но мерзостію предъ Богомъ бываетъ, и въ погибель человѣку обращается, не отъ себе, "--- \textit{добра} бо \textit{суть зѣло} созданія Божія\footnote{Быт.~1,~31.}, "--- но отъ сердечнаго человѣческаго произволенія, яко человѣкъ тому, какъ Богу, сердце свое отдаетъ и посвящаетъ. Солнце, луна, звѣзды, звѣри, скоты и всякое животное, такожде огнь, вода и прочія стихіи, хлѣбъ, вино и прочая къ служенію человѣку создана, дабы, ихъ употребляя, познавалъ Создателя и Ему благодарилъ и служилъ. Но ослѣпленный человѣкъ превратнымъ образомъ дѣлаетъ: что должно ему служить, тому онъ служитъ, и что ему покоряться должно, тому онъ душу и сердце свое покоряетъ. Всякая бо мерзость прежде внутрь въ сердцѣ человѣческомъ начинается, и потомъ внѣ исходитъ и является. Внѣ убо сердца человѣческаго идолъ ничтоже есть, но всякій идолъ въ сердцѣ человѣческомъ есть. Сердце человѣческое есть мерзкое капище, идоловъ и мерзостей исполненное. Оно вымыслило всякое идолонеистовство, и въ самое дѣло произвело. Тамо всякое зло и богопротивная мерзость крыется. И когда вѣрою не будетъ очищено, не иное что зачинаетъ и раждаетъ, какъ только злые и богопротивные плоды. И хотя многіе, вѣрою неочищенные и Духомъ неотрожденные, показуются внѣ добрыми, но внутрь злые суть, подобны \textit{гробамъ повапленнымъ, которые внѣ являются красны, внутрь же полны суть костей мертвыхъ и всякія нечистоты}\footnote{Матѳ.~23,~27.}. Богъ же по сердцу и внутреннему состоянію судитъ, а не по внѣшнему. Сіе толикое, яко древо великое, зло съ горькими своими плодами отъ сѣмени зміинаго возрасло, которое посѣялъ на сердцѣ праотца Адама, и въ зачатіи и рожденіи нашемъ плотскомъ на наши сердца, яко ядъ смертоносный, изливается, и всѣ наши силы душевныя и тѣлесныя заражаетъ такъ, что отъ природы не иное что можетъ мыслити и дѣлати, какъ только суетное, непотребное и злое. Нынѣ разсуди, хрістіанине, что есть самъ въ себѣ человѣкъ кромѣ Хріста, когда ни помыслити, ни хотѣти, ни дѣлати чего богоугоднаго не можетъ безъ благодати Его, якоже Самъ сказалъ Хрістосъ: \textit{безъ Мене не можете творити ничесоже}\footnote{Іоан.~15,~5.}; но есть какъ розга изсохшая, которая только къ сожженію угодна есть. А тако человѣкъ не иное что предъ Богомъ есть, какъ мерзость и смрадъ, отъ котораго святѣйшія Его отвращаются очеса; а оттуду не иному чему подлежитъ, какъ проклятію, осужденію и мученію вѣчному. Въ сіе зло и бѣдствіе Адамъ насъ вринулъ своимъ грѣхопаденіемъ, но Хрістосъ, Сынъ Божій пришелъ въ міръ отъ всего того зла насъ избавити. Онъ, явившися во плоти, учинился ходатаемъ и сталъ посредникомъ между Богомъ, прогнѣваннымъ нашими грѣхами, и человѣками прогнѣвавшими. Законъ Божій и волю Божію исполнилъ совершенно, такъ что \textit{грѣха не сотвори, ни обрѣтеся лесть во устѣхъ Его}\footnote{1~Петр.~2,~22.}. Грѣхи міра на Себе взялъ, да, якоже Агнецъ непорочный, очиститъ ихъ. \textit{Се Агнецъ Божій, вземляй грѣхи міра}, показывалъ на Него всему міру Предтеча святый\footnote{Іоан.~1,~29.}. \textit{Не вѣдѣвшаго грѣха по насъ грѣхъ сотвори} Богъ, \textit{да мы будемъ правда Божія о Немъ}\footnote{2~Кор.~5,~21.}. \textit{И Той очищеніе есть о грѣхахъ нашихъ, не о нашихъ же точію, но и о всего міра}\footnote{1~Іоан.~2,~2.}, и проч. Вземши наши грѣхи на Себе, взялъ и казнь, за грѣхи намъ должную, ибо грѣхи безъ казни не бываютъ. \textit{Господь предаде Его грѣхъ ради нашихъ. Той язвенъ бысть за грѣхи наша, и мученъ бысть за беззаконія наша}\footnote{Ис.~53,~6 и 5.}. Мы достойны были вѣчнаго отъ лица Божія отверженія, осужденія и во адѣ съ діаволомъ и злыми его аггелами мученія, яко отступники и разорители Святаго Его закона; но Хрістосъ, Сынъ Божій, Царь славы и Господь нашъ, сію нашу казнь взялъ на Себе, и за насъ, рабовъ Своихъ непотребныхъ, такъ ужасно хуленъ, посмѣянъ, обезчещенъ, поруганъ, уязвленъ, мученъ, и такъ поносною смертію умеръ; а тако вѣчнаго наказанія, котораго мы достойны, избавилъ насъ, и Божіе благословеніе, милость, благодать и вѣчный животъ, котораго мы недостойны, у небеснаго Своего Отца заслужилъ намъ. Тако Онъ учинился намъ \textit{совершеннѣйшимъ врачевствомъ} противу всякаго нашего бѣдствія и окаянства, \textit{и живымъ и извѣстнѣйшимъ} всякаго истиннаго \textit{блаженства источникомъ}. Онъ \textit{свободитель} нашъ отъ плѣненія и власти діавольскія; Онъ \textit{исцѣлитель} язвъ нашихъ, которыми мы отъ врага онаго, яко разбойника, смертельно уязвлены; Онъ \textit{свѣтъ} противу тьмы нашей; Онъ \textit{врачь} противу немощей нашихъ; Онъ намъ \textit{премудрость} противу безумія и недоумѣнія нашего; Онъ \textit{утѣшеніе} противу всякой печали нашей; Онъ \textit{животъ} и живота источникъ противу смерти нашей; Онъ \textit{оправданіе} противу грѣховъ нашихъ; Онъ заступникъ противу клеветниковъ нашихъ; Онъ \textit{миръ и покой} противу злыя совѣсти нашея; Онъ \textit{побѣда} противу враговъ нашихъ; Онъ \textit{помощникъ} намъ въ брани нашей противу діавола, плоти, міра, грѣха и прочіихъ враговъ нашихъ; Онъ \textit{ходатай} нашъ противу гнѣва и суда Божія; Онъ \textit{укрѣпленіе} намъ въ слабости, изнемоганіи и малодушіи нашемъ; Онъ \textit{путь и вождь} къ вѣчному животу, блаженству и царствію небесному. Словомъ, Онъ \textit{бысть намъ премудрость отъ Бога, правда же и освященіе и избавленіе}\footnote{1~Кор.~1,~30.}. Слѣпы мы были, "--- Онъ намъ бысть премудрость. Грѣшны мы были, "--- Онъ нашимъ учинился оправданіемъ. Скверны мы были, "--- Онъ намъ сдѣлался освященіемъ. Плѣнены мы отъ діавола были, "--- Онъ нашимъ учинился избавленіемъ. Сіе высочайшее Его благодѣяніе и милость какъ всему міру, такъ мнѣ и тебѣ, человѣче, подалося. Мой и твой Онъ есть избавитель, искупитель, заступникъ, помощникъ, утѣшитель, оправданіе, освященіе, миръ, воскресеніе, надежда въ жизни, при смерти, по смерти вѣчный животъ, вѣчная радость, слава и блаженство, когда истинно, сердечно и нелицемѣрно вѣруемъ въ Него. О семъ буди Ему со Отцемъ и Духомъ Святымъ благодареніе и слава отъ насъ недостойныхъ. Аминь.

\paragraph*{§\:340.} \textit{Вѣровать во Хріста} не иное есть, какъ изъ закона узнавши и сердечно признавши свою бѣдность и окаянство, которое отъ грѣха и грѣху послѣдующаго праведнаго гнѣва Божія бываетъ, "--- изъ Евангелія же познавши благодать Божію, всѣмъ открытую, къ Нему \textit{единому}, то"=есть Хрісту, подъ защищеніе и покровительство прибѣгнуть, Его \textit{единаго} за Избавителя и Спасителя отъ того бѣдствія признать и имѣть, на Него \textit{единаго} всю спасенія вѣчнаго, такожде въ подвигѣ противу діавола, плоти, міра и грѣха во время живота при смерти и по смерти надежду неуклонно и неотступно утверждать, яко на несумнѣнное и непоколебимое вѣчнаго спасенія основаніе. Ради изъясненія слѣдующій примѣръ возми себѣ въ разсужденіе, хрістіанине. Ежели бы какій монархъ разгнѣвался на градъ какой за преступленіе, за которое слѣдовала гражданамъ смертная казнь; въ такомъ бы бѣдственномъ случаѣ вступился любезный монаршій сынъ за гражданъ, и къ отцу разгнѣванному ходатайство употребилъ, со смиреніемъ предая себе волѣ отчей, и изволяя терпѣть за повинныхъ все, что отъ монарха было бы опредѣлено, чтобъ бѣдныхъ осужденниковъ въ милость родителю своему привести, и тако бы всѣхъ, которые бы къ нему ни прибѣгали подъ покровъ и защищеніе, привелъ въ царскую милость: тогда бы вся ихъ спасенія и живота надежда милостивому тому ходатаю достойно приписатися должна была, яко безъ его ходатайства всѣмъ слѣдовало неотмѣнно смертію казненнымъ быти. \textit{Весь міръ} учинился \textit{повиннымъ Богу, яко вси согрѣшиша}\footnote{Римл.~3,~19 и 23.}, и потому всѣмъ слѣдовало праведнымъ судомъ Божіимъ вѣчно умирать. Сынъ Божій, Іисусъ Хрістосъ, за всѣхъ согрѣшившихъ вступился. \textit{Хрістосъ за всѣхъ умре}\footnote{2~Кор.~5,~15.}, и тако сдѣлался \textit{Ходатаемъ Бога и человѣковъ}\footnote{1~Тим.~2,~5.}, Бога разгнѣваннаго вольнымъ Своимъ смиреніемъ и послушаніемъ умилостивляя и разгнѣвавшихъ въ милость Ему приводя. И тако всѣ, которые къ Нему, яко Ходатаю своему, подъ защищеніе и покровъ прибѣгаютъ, въ милость Богу посредствомъ Его приходятъ, якоже Самъ глаголетъ: \textit{никтоже пріидетъ ко Отцу, токмо Мною}\footnote{Іоан.~14,~6.}. И Отецъ съ небесе свидѣтельствуетъ о Немъ: \textit{Сей есть Сынъ Мой возлюбленный, о Немже благоволихъ: Того послушайте}\footnote{Матѳ.~17,~5.}. Слушаемъ Его, когда оставляемъ грѣхи, каемся и вѣруемъ во святое Евангеліе Его, якоже глаголетъ: \textit{покайтеся, и вѣруйте во Евангеліе}\footnote{Марк.~1,~15.}. Тако пріемлющимъ Хріста сына Божія и вѣрующимъ въ Него вся благая отъ Бога человѣколюбно подаются: отпущеніе грѣховъ, оправданіе, обновленіе, благодать Святаго Духа, помоществующая и укрѣпляющая въ немощахъ ихъ, въ подвигѣ противу діавола и прочіихъ враговъ, сыновство, наслѣдіе, вѣчный животъ, и проч., якоже Апостолъ вкратцѣ заключилъ: \textit{Иже Сына Своего не пощадѣ, но за насъ всѣхъ предалъ есть Его: како убо не и съ Нимъ вся намъ дарствуетъ}\footnote{Римл.~8,~32.}? "--- Отъ вышереченныхъ показуется: 1)~Вѣра безпосредственно зритъ и пріемлетъ Хріста, яко всѣхъ благъ, человѣку отъ Бога обѣщанныхъ, виновнаго, яко не чрезъ Него никто не можетъ ко Отцу пріити, и обѣщанныхъ благъ отъ Него сподобитися. И тако всякъ, пріемля Хріста истинною вѣрою, вѣруетъ и въ Бога, пославшаго Его; вѣруяй въ Сына, вѣруетъ и въ Отца, пославшаго Его; и тако посредствомъ Сына ко Отцу приходитъ, якоже рече: \textit{никтоже пріидетъ ко Отцу, токмо Мною}\footnote{Іоан.~14,~6.}. "--- 2)~Какъ отъ ветхаго Адама воспріялъ человѣкъ въ плотскомъ рожденіи всякое зло и бѣдствіе, грѣхъ и клятву, гнѣвъ и смерть: тако вѣрующему во Хріста подобаетъ отъ Него воспріяти всякое добро и блаженство, оправданіе, благословеніе Божіе, благодать, обновленіе и вѣчный животъ. Какъ отъ Адама воспріялъ сердце нечистое, грѣхолюбивое, сребролюбивое, гнѣвливое, злобное, завистливое, непокоривое, немилосердое, жестокое, лестное, лукавое, лживое, лицемѣрное, гордое и всякое злонравіе: тако подобаетъ ему вѣрою отъ Хріста воспріяти сердце чистоты любительное, доброхотное, смиренное, боголюбивое, братолюбивое, терпѣливое, кроткое, щедролюбивое, милостивое, благопослушное, простое, нелестное, и прочее добронравіе. Ибо какъ отъ Адама раждаемся по плоти, и въ семъ рожденіи всякое злонравіе съ нами раждается: тако отъ Хріста раждаемся по духу, и въ семъ рожденіи ветхаго человѣка со страстьми и похотьми совлещися подобаетъ, и \textit{облещися въ новаго, созданнаго по Богу въ правдѣ и преподобіи истины}\footnote{Еф.~4,~22--24.}. И какъ во Адамѣ умерли мы духовно, недѣйствительны учинились, какъ мертвіи, къ творенію добрыхъ дѣлъ: тако во Хрістѣ и Хрістовою благодатію должно намъ воскреснути духовно, то"=есть, оживитися вѣрою и благодатію Его къ творенію богоугодныхъ дѣлъ, \textit{да, якоже воста Хрістосъ отъ мертвыхъ славою Отчею, тако и мы во обновленіи жизни ходити начнемъ}\footnote{Римл.~6,~4.}. Откуду въ обратившемся и истинно вѣрующемъ во Хріста все иное показуется, какъ прежде было: иное склоненіе, иное желаніе, иное мудрованіе, иныя слова, дѣла и замыслы, какъ прежде были, якоже читаемъ въ церковной исторіи о многихъ покаявшихся и вѣрою обновившихся, которые, какъ отъ сна грѣховнаго пробудившися, начали дѣла Божія творити, или паче, какъ отъ мертвыхъ воставши, начали по пути заповѣдей Божіихъ ходити. "--- 3)~Отсюду бываетъ въ единомъ и томжде хрістіанинѣ истинномъ брань между плотію и духомъ, между злонравіемъ ветхаго и добронравіемъ новаго человѣка, яко \textit{плоть похотствуетъ на духа, духъ же на плоть; сія же другъ другу противятся}\footnote{Гал.~5,~17.}. "--- 4)~Видно паки отсюду, что злонравіе и беззаконное житіе есть противно вѣрѣ хрістіанской, и потому въ комъ оно имѣется, въ томъ нѣтъ вѣры, хотя и исповѣдуетъ Хріста и хрістіаниномъ именуется. Блудъ и всякая нечистота, хищеніе, воровство, клятвопреступленіе, неправда въ судахъ и во всякомъ дѣлѣ, піянство, лесть, лукавство, ложь, злоба и прочая симъ подобная изгоняютъ вѣру изъ сердца. Исповѣданіе православныя вѣры можетъ быть въ таковыхъ, ибо многіе и беззаконнаго житія право мудрствуютъ о вѣрѣ, многіе отъ нихъ и проповѣдуютъ вѣру; но всѣ таковые не имѣютъ сердечной вѣры, которая человѣка обновляетъ, изъ злонравнаго въ добронравнаго премѣняетъ, и истиннымъ хрістіаниномъ дѣлаетъ. Иное бо есть знаніе и исповѣданіе вѣры, иное же есть сердечная вѣра, о чемъ выше въ первой статьѣ сказано.

\subsection[Глава 2-я. О любви ко Хрісту, Сыну Божію.]{глава вторая.\\\bfseries О любви ко Хрісту, Сыну Божію.}

\begin{quotation}\textit{Имѣяй заповѣди Мои, и соблюдаяй ихъ, той есть любяй Мя; а любяй Мя, возлюбленъ будетъ Отцемъ Моимъ, и Азъ возлюблю его, и явлюся ему Самъ}. И мало ниже: \textit{аще кто любитъ Мя, слово Мое соблюдетъ: и Отецъ Мой возлюбитъ его, и къ нему пріидемъ, и обитель у него сотворимъ. Не любяй Мя, словесъ Моихъ не соблюдаетъ}, глаголетъ Хрістосъ\footnote{Іоан.~14,~21 и 23.}.\end{quotation}

\paragraph*{§\:341.} Самое имя "--- \textit{Іисусъ}, то"=есть, \textit{Спаситель}, привлекаетъ сердца наша къ любви своей, православный хрістіанине! \textit{Нѣсть бо иного имене подъ небесемъ, даннаго въ человѣцѣхъ, о немже подобаетъ спастися намъ, кромѣ имене Іисусова}, проповѣдуетъ Апостолъ Его Петръ\footnote{Дѣян.~4,~12.}. Видѣлъ ты въ предшедшей главѣ, что мы въ себѣ были, и что чрезъ Него учинилися. \textit{Вси} мы, \textit{яко овцы, заблудихомъ: человѣкъ отъ пути своего заблуди}\footnote{Ис.~53,~6.}. Онъ насъ, яко пастырь добрый, взыскалъ, и взявши на рамена Своя, къ небесному Своему Отцу принеслъ\footnote{Лук.~15,~5.}. Всѣ мы, яко путницы, отъ діавола, какъ разбойника, обнажены и смертоносною уязвлены были раною. Онъ къ намъ, тако бѣднымъ, милосердіемъ Своимъ преклонился, и къ лежащимъ склонился, учинился \textit{ближнимъ нашимъ}, милосердую помощи Своея руку простерлъ къ намъ, и лежащихъ возставилъ, и язвленныхъ, елеемъ милости Своея исцѣлилъ\footnote{10,~30--35.}. Были мы, яко плѣнники, узами врага нашего связанные. Онъ, яко сильный въ крѣпости, врага нашего связалъ, \textit{узы наша растерзалъ}\footnote{Пс.~115,~7.}, и возвратилъ плѣненіе душъ нашихъ\footnote{Матѳ.~12,~29; Пс.~67,~19.}. Были мы, яко преступники и отступники, отъ лица Божія отвержены. Онъ насъ предъ праведнымъ судомъ Божіимъ заступилъ, и между Богомъ и нами сталъ посредникомъ, ходатаемъ и примирителемъ. Были мы всѣ должники тьмою талантъ Царю нашему Богу, и никакъ не могли долга нашего заплатить, но слѣдовало неотмѣнно заключитися по суду правды Его въ темницу вѣчную, и вѣчною платить мукою\footnote{Матѳ.~18,~23--34.}. Онъ за насъ вступился, и долгъ нашъ на Себе взялъ, и правдѣ Божіей заплатилъ. Словомъ, отъ таковаго неблагополучія, каковаго горшее и горестнѣйшее не можетъ быть, избавилъ насъ, и толикое блаженство, котораго лучшее и вожделѣннѣйшее не можетъ быть, милостивно и туне, безъ всякихъ нашихъ заслугъ, исходатайствовалъ намъ Іисусъ Сынъ Божій. Все же сіе учинилъ намъ Іисусъ не чрезъ какое посредство, но Самъ Собою. Самъ, ради насъ заблуждшихъ, въ мірѣ семъ странствовалъ, и взыскалъ: \textit{пріиде Сынъ человѣческій взыскати и спасти заблуждшаго}\footnote{Лук.~19,~10.}; Самъ, ради нашей бѣдности, бѣдствовалъ; Самъ, ради насъ связанныхъ, связанъ былъ, и Своими узами \textit{наша растерзалъ узы}; Самъ, ради насъ язвенныхъ, язвенъ былъ, и \textit{Его язвою мы исцѣлѣхомъ}\footnote{1~Петр.~2,~24.}; Самъ за долгъ нашъ тягчайшій платилъ не сребромъ, или златомъ, но \textit{честною Своею кровію}\footnote{1,~19.}; Самъ, ради насъ умершихъ, на древѣ крестномъ умеръ, и Своею смертію смерть нашу умертвилъ, и насъ умершихъ оживилъ; Своимъ крестомъ двери къ милости Божіей, вѣчному животу и блаженству отверзлъ. Сіе значитъ имя "--- \textit{Іисусъ}, ангеламъ любимое, намъ грѣшнымъ утѣшительное и сладкое, и всей твари дивное. Кто Онъ такой, Который такъ высоко насъ почтилъ и безконечно и вѣчно одолжилъ? Сынъ Божій единородный, Слово Отчее, Царь и Господь славы, Еммануилъ, Богъ невидимый, непостижимый и вѣчный, во плоти подобный намъ, кромѣ грѣха явивыйся, Творецъ нашъ, Который насъ изъ ничего создалъ. Кто мы, такъ высоко и несказанно почтенные отъ великаго Бога? Персть отъ земли, земля и пепелъ, рабы Его непотребные, грѣшники, отступники и враги Его. Подлинно тако, воистинну тако! Павелъ, Апостолъ Его, о семъ увѣряетъ насъ: \textit{врази бывше, примирихомся Богу смертію Сына Его}\footnote{Римл.~5,~10.}. Ибо честь Его Божественную хотѣли похитить и себѣ присвоить, по совѣту отступника діавола\footnote{Быт.~3,~5.}. О, воистину великое сіе и непостижимое благости Его и человѣколюбія дѣло есть, что Онъ насъ, враговъ и отступниковъ Своихъ, тако помиловалъ! Сія убо высочайшая Его къ намъ недостойнымъ любовь требуетъ, чтобы и мы Его взаимно любили, не языкомъ и словомъ, но дѣломъ и истиною. Аще бо простаго человѣка, который знатно насъ одолжилъ, по естественному закону, любимъ, и, не отдая долга сего, неблагодарными дѣлаемся; кольми паче Сыну Божію, такъ высоко насъ одолжившему, сей долгъ отдавати должны мы.

\paragraph*{§\:342.} \textit{Какъ"=де намъ не любить Хріста?} "--- Правда, всякъ отъ тѣхъ, которые имя Хрістово исповѣдуютъ, тое исповѣдуетъ, совѣстію самою убѣждаемый, и долженъ исповѣдывать. И я тоежде признаю и отвѣщаю: какъ не любить Хріста, съ любовію покланяемаго и почитаемаго ангелами и всѣми святыми душами? Хріста, единороднаго Сына Божія, Сына по плоти Пречистыя Дѣвы Богородицы, ради нашего спасенія такъ смирившагося, такъ умалившагося, такъ страшная претерпѣвшаго, такъ поносною смертію умершаго; Хріста "--- оправданіе наше, упованіе наше, воскресеніе наше, вѣчный животъ нашъ, славу нашу, помощника и заступника нашего, побѣдителя смерти, ада и всѣхъ нашихъ враговъ? Какъ не любить такого высокаго благодѣтеля, въ которомъ все наше состоитъ блаженство, такъ высокое и непостижимое добро? "--- Послушай же, хрістіанине, \textit{что то есть любить Хріста}, и отъ чего познается любовь сія, и внимай, чтобы любовь сія не на языкѣ и устахъ, но на сердцѣ была, гдѣ истинная любовь мѣсто свое имѣетъ, дабы вмѣсто любителей Хрістовыхъ врагами Его не быть. 1)~Любитель съ любимымъ всегда совокупно и неразлучно быть желаетъ. Тако Богъ, Иже есть вѣчная любовь, понеже любитъ человѣка, когда человѣкъ отступилъ и отлучился отъ Него, благоволилъ съ нимъ въ воплощеніи соединитися, и тако любовію Своею привлещи къ Себѣ человѣка, и благодатно обитаетъ въ рабахъ Своихъ, кровію Хрістовою очищенныхъ, якоже писано есть: \textit{вы есте церкви Бога живаго}\footnote{2~Кор.~6,~16.}. Сіе бываетъ и между человѣками, которые другъ друга любятъ. Тако другъ съ любимымъ своимъ другомъ всегда хощетъ быть неразлучно, не токмо въ счастіи, но и въ несчастіи. Откуду бываетъ, что когда разлучаются, болѣзнуютъ, сѣтуютъ и плачутъ, любовь бо не терпитъ разлученія отъ любимаго. И хотя вѣрные и любимые друзья тѣломъ разлучаются, но духомъ, сердцами и любовію неразлучны пребываютъ, любовь бо двухъ во едино совокупляетъ и связуетъ. Слѣдственно кто любитъ Хріста сердцемъ, тотъ со Хрістомъ неразлучно быть желаетъ, крестъ свой носить и Ему послѣдовать, поруганнымъ, посмѣяннымъ, озлобленнымъ быть отъ міра, и съ Нимъ и за Него умереть не отрицается. Отсюду познается, что ложная есть любовь тѣхъ, которые вѣчно со Хрістомъ царствовать хотятъ, но смиренія Его, страданія и поруганія срамляются, и убѣгаютъ отъ того. Таковые суть, которые въ благополучіи мнятся Богу благодарить, но въ неблагополучіи ропщутъ; не терпятъ и обращаются къ снисканію помощи отъ созданія, дабы отъ нашедшей бѣды избавиться. Таковые самихъ себе и благополучіе свое любятъ, а не Хріста. Дѣлаютъ они подобно тѣни, которая во время сіянія солнечнаго не отлучается отъ вещи, и стоитъ ли, или движется, и тѣнь съ нею; а когда нѣтъ сіянія солнечнаго, тогда отступаетъ и тѣнь. Тако поступаютъ и неистинные любители Хрістовы: хотятъ быть со Хрістомъ воскресшимъ и прославленнымъ, но съ обезчещеннымъ, поруганнымъ, посмѣяннымъ, уязвленнымъ, распятымъ быть не хотятъ, ужасаются Его и убѣгаютъ. Истинныя бо любви и дружества союзъ ничѣмъ разорваться не можетъ. Тако мучениковъ святыхъ ничто, ни огнь, ни вода, ни мечь, ни смерть, ни животъ, не могли разлучить отъ любимаго Хріста. Тако Павелъ святый, связанная любовію со Іисусомъ Хрістомъ душа, \textit{не токмо связанъ быть хотѣлъ но и умрети во Іерусалимѣ готовъ былъ за имя Господа Іисуса}\footnote{Дѣян.~21,~13.}. 2)~Любитель любимаго волю исполняетъ, дабы не опечалить любимаго, ибо оскорбленіе любимаго противно есть любви, и разрушаетъ союзъ любви. Такъ сынъ добрый волю отца своего, жена добрая волю мужа своего тщится исполнить, чтобы не оскорбить того, котораго естественною любовію любитъ. Тако, кто Хріста Сына Божія любитъ, тщится волю Его святую исполнять. Воля Хрістова въ заповѣдяхъ Его означается; и тако кто Хріста любитъ, тщится заповѣди Его исполнять, якоже глаголетъ Самъ: \textit{имѣяй заповѣди Моя, и соблюдаяй ихъ, той есть любяй Мя}\footnote{Іоан.~14,~21.}. Хрістова же заповѣдь есть, чтобы мы любили другъ друга, якоже паки глаголетъ: \textit{сія заповѣдаю вамъ, да любите другъ друга}\footnote{15,~17.}. (Что есть любить ближняго, и какіе плоды любви сея, сказано въ главѣ \textit{о любви къ ближнему} въ первой книгѣ). И не токмо друговъ и братію свою, но и враговъ нашихъ хощетъ и повелѣваетъ намъ любити Хрістосъ: \textit{любите враги ваша}\footnote{Мѳ.~5,~44.}. Отъ сея бо единыя любви истинный любитель Хрістовъ и познавается, когда не точію друговъ, но и враговъ своихъ любитъ, по ученію Его. Ибо любящихъ себе и язычники и мытари и прочіи грѣшники любятъ, якоже глаголетъ Хрістосъ: \textit{аще любите любящихъ васъ, кую мзду имате? не и мытари ли тожде творятъ}\footnote{Мѳ.~5,~46.}? Враговъ же любить единыя хрістіанскія и любовію Хрістовою исполненныя души дѣло есть, какъ о семъ такожде въ главѣ \textit{о любви ко врагамъ} въ первой книгѣ сказано. "--- \textit{Какъ"=де мнѣ того любить, который на меня враждуетъ?} "--- Правда; правильно сіе отвѣщаетъ о себѣ плотскій, душевный и неотрожденный человѣкъ. Плотскому бо и самолюбивому человѣку сіе весьма трудно и почти невозможно, но вѣрному и благодатію Духа Святаго отрожденному удобно. Вѣрѣ бо и любви истинной ничто неудобно. Когда любишь Хріста, Который за тебе и мене \textit{враговъ} умеръ, то съ охотою и любовію послушаешь того, что Онъ мнѣ и тебѣ велитъ: \textit{любите враги ваша}. Что тебѣ отъ того, что врагъ твой на тебе враждуетъ? Ты на него не враждуй, но смотри, что Хрістосъ, любитель душъ нашихъ, повелѣваетъ, и Его любовію вражду врага твоего преодолѣвай, чтобы ненавистію ко врагу не разорить Хрістову заповѣдь, и тако заповѣдавшаго не оскорбить, что разоряетъ любовь. \textit{Не побѣжденъ бывай отъ зла, но побѣждай благимъ злое}\footnote{Римл.~12,~21.}. "--- \textit{Врагъ"=де худо и беззаконно дѣлаетъ?} Правда; но и ты худо и беззаконно будешь дѣлать, когда не любить, но враждовать на него будешь. Едина бо вражда на грѣхъ и на изобрѣтателя его діавола праведная есть. Онъ худо дѣлаетъ, что зло тебѣ чинитъ; но и ты худо дѣлаешь, что зло за зло воздаешь. \textit{Сердце"=де мое отъ него отвращается?} "--- Ты сердце свое вѣрою и любовію Хрістовою побѣждай и убѣждай, и не что сердце хощетъ, но что вѣра твоя и заповѣдь Хрістова требуетъ отъ тебе, дѣлай. Хрістосъ велитъ, любитель твой велитъ; того вѣчная правда требуетъ; то волѣ Его святой угодно, тебѣ полезно, хотя плотское сердце, мысль и разумъ слѣпый противно хощетъ. Слѣдственно такъ должно дѣлать, какъ Онъ хощетъ, а не какъ страстная плоть наша, когда хощемъ любить Его. Въ томъ бо состоитъ и подвигъ нашъ хрістіанскій, чтобы намъ дѣлать не тое, что плоть наша слѣпая хощетъ, но что заповѣдь Божія. "--- 3)~Любитель любитъ и того, кого любимый любитъ. Тако когда любишь друга твоего, то ради любимаго друга твоего любишь и того, кого другъ любитъ. Убо, когда любишь Хріста, любить долженъ всякаго человѣка, друга твоего и врага твоего. Ибо Хрістосъ всѣхъ возлюбилъ такъ, что за всѣхъ, друговъ и враговъ твоихъ, душу Свою положилъ, якоже глаголетъ Апостолъ: \textit{Хрістосъ за всѣхъ умре}\footnote{2~Кор.~5,~15.}. Сего бо ради и велитъ намъ ближнихъ нашихъ любить, что какъ насъ, такъ и ближнихъ нашихъ любитъ, дабы мы не токмо не дѣлали имъ никакого зла, но и всякое добро дѣлали, какъ себѣ дѣлаемъ, отъ сего бо и любовь къ ближнему познается. Отсюду послѣдуетъ, что ради Хріста, любящаго всѣхъ, и ты долженъ всѣхъ любить, когда хощешь Хріста любить; а отъ сего послѣдуетъ, что ради Хріста, алчущаго напитать, жаждущаго напоить, нагаго одѣть, страннаго въ домъ ввести, больнаго и въ темницѣ сѣдящаго посѣтить, печальнаго утѣшить, о счастіи ближняго твоего радоватися, о несчастіи печалитися не отречешися. Сего бо любовь ко Хрісту и Хрістомъ любимому ближнему твоему отъ тебе требуетъ. Плодъ бо любви къ ближнему есть милость къ ближнему и благотвореніе. Отсюду заключается, что не имѣютъ ко Хрісту любви, которые ближняго не любятъ; не любятъ же его, яко не благотворятъ ему. Горше того дѣлаютъ, которые не токмо не благотворятъ, но и зло творятъ ближнему. Сіи люди, гоня и озлобляя ближняго, гонятъ и озлобляютъ Самаго Хріста, хотя того и не примѣчаютъ. Озлобляяй бо раба касается и господина его; всякъ же человѣкъ рабъ Божій есть. "--- 4)~Любитель любимаго бережется оскорбить. Ибо оскорбленіе и любовь къ любимому купно быть не могутъ въ любителѣ. Развѣ бы оскорбленіе тое отъ невѣдѣнія, или отъ другой какой немощи было, а не отъ умысла и произволенія, или когда любитель любимаго оскорбляетъ ради пользы его. Но сіе самое оскорбленіе отъ любви происходитъ. Тако отецъ оскорбляетъ сына своего, когда біетъ его, дабы исправенъ былъ; тако и другъ, любезнаго своего друга погрѣшности обличая, оскорбляетъ; тако добрые пастыри оскорбляютъ людей, себѣ порученныхъ, да возымѣютъ \textit{печаль по Бозѣ, которая покаяніе нераскаянно во спасеніе содѣловаетъ}\footnote{2~Кор.~7,~10.}. Тако и самъ Господь, \textit{егоже любитъ, наказуетъ, біетъ же всякаго сына, егоже пріемлетъ}\footnote{Евр.~12,~6.}. Но сіе оскорбленіе, яко отъ любви происходящее, любви непротивно, паче же извѣстнѣйшее есть знаменіе и плодъ любви. И потому здѣ слово есть не о семъ оскорбленіи, но о такомъ, которое отъ злости сердечной бываетъ. Хрістосъ всякимъ грѣхомъ оскорбляется, яко вѣчная правда и благостыня и сокровище всѣхъ добродѣтелей. Убо кто любитъ Хріста, тотъ отъ всякаго грѣха бережется; и чѣмъ болѣе кто любитъ Его, тѣмъ болѣе бережется отъ грѣха, ради того единаго, чтобы Его не оскорбить и не опечалить, оставляя прочія причины, которыя могутъ отъ грѣха отвести. "--- 5)~Любитель жалѣетъ, когда любимаго чѣмъ оскорбитъ самъ, или отъ другаго кого оскорбленнымъ видитъ. Тако, кто любитъ Хріста, когда отъ немощи что согрѣшитъ, сокрушается, жалѣетъ, окаеваетъ себе, и со смиреніемъ и горячимъ исповѣданіемъ своего грѣха, и самого себе уничтоженіемъ и укореніемъ, падаетъ предъ Нимъ нелицемѣрно, и самъ себе судитъ всякаго наказанія достойнымъ. Таковый, хотя бы во адѣ былъ за преступленіе, и тамо бы хвалилъ Божію правду. Тако Апостолъ Петръ, теплѣйшій любитель Хрістовъ, когда отреклся Хріста, и грѣхъ свой, которымъ оскорбилъ любимаго Учителя своего, узналъ, \textit{изшедъ вонъ, плакася горько}\footnote{Мѳ.~26,~75.}. Тяжко бо любителю опечалить любимаго, и опечаленіе то безъ послѣдующей печали своей быть не можетъ. Ибо истинная любовь \textit{радуется съ радующимся плачетъ съ плачущимъ} любимымъ\footnote{Рим.~12,~15.}. "--- 6)~Любитель любимаго всегда въ сердцѣ своемъ носитъ, не вещественно (ибо не можетъ тому быть, дабы человѣкъ весь, и тѣломъ и душею, въ сердцѣ нашемъ былъ), но духовно. Любовь бо въ сердцѣ мѣсто свое имѣетъ и съ любимымъ едино бываетъ. Такъ сынъ отца, котораго любитъ, и другъ любезнаго своего друга, хотя и отсутствующаго, въ сердцѣ объемлетъ; чего ради часто его поминаетъ, печется, дабы противное что ему не приключилося, разглагольствуетъ о немъ, спрашиваетъ и провѣдываетъ, како любимый его находится отлучившійся отъ него; а часто и плачется, что долго съ нимъ не видится. Сіе дѣйствіе примѣчается въ естественной любви. Тако, кто Хріста Сына Божія любитъ, всегда въ сердцѣ своемъ духовно объемлетъ и носитъ; а оттуда бываетъ, что часто о Немъ поминаетъ, размышляетъ и разсуждаетъ о человѣколюбномъ и спасительномъ Его смотрѣніи, рожденіи, на земли пожитіи, терпѣніи, страданіи, смерти, воскресеніи, вознесеніи, одесную Бога Отца сѣденіи, и радуется о томъ, удивляется тому, и со усердіемъ Ему за тое со Отцемъ и Святымъ Духомъ Его благодаритъ, и желаетъ лицемъ къ лицу видѣти любимаго, Котораго нынѣ вѣрою видитъ. Таковъ былъ Павелъ святый, который, любовію горя ко Хрісту, взывалъ: \textit{желаніе имѣю разрѣшитися и со Хрістомъ быти}\footnote{Филип.~1,~23.}. "--- 7)~Любитель любимаго нраву послѣдуетъ, и всякъ ищетъ имѣть дружество съ подобнымъ себѣ. Отсюду бываетъ, что истинное дружество не можетъ быть, какъ между добрыми и единонравными. Нѣтъ сообщенія смиренному съ гордымъ, трезвенному съ піяницею, цѣломудренному съ нечистымъ, щедрому и милостивому съ сребролюбцемъ, хищникомъ и жестокосердымъ; но единъ отъ другаго уклоняется, и всякъ ищетъ тое, что любитъ. Тако Хріста любящій тщится послѣдовать преблагимъ и Божественнымъ нравамъ Его: бываетъ смиренъ, терпѣливъ, кротокъ, незлобивъ, любителенъ, миренъ, простосердеченъ, чистосердеченъ, милостивъ, милосердъ, сострадателенъ, и проч., *не ради заслуги какой*, но ради единаго того, что Хрістосъ есть таковъ и такія имѣетъ добродѣтели. "--- 8)~Любитель едино съ любимымъ мыслитъ, и согласуетъ во всемъ ему, и о чемъ тщится любимый, о томъ и любитель. Иначе не было бы въ нихъ согласія и единонравія, когда бы единъ изъ нихъ о томъ, другій о другомъ противномъ мыслилъ, намѣревалъ и начиналъ, "--- что взаимной любви и дружеству противно, и отъ того взаимная любовь разоряется и дружество престаетъ. Дружество бо, какъ сказано, не можетъ быть, какъ между единонравными. Хрістосъ желаетъ и ищетъ всѣмъ спасенія. Се истинно и извѣстно есть. Тое бо значитъ и показуетъ Его въ міръ пришествіе, яко \textit{всѣмъ хощетъ спастися и въ разумъ истины пріити}\footnote{1~Тим.~2,~4.}. Убо кто любитъ Хріста, тотъ какъ о своемъ, такъ и ближняго спасеніи печется. Любитель съ любимымъ едину мысль имѣти долженъ, едино тщаніе и попеченіе. И хотя сія должность до служителей Божіихъ наипаче надлежитъ, однакожъ и всѣхъ хрістіанъ касается, хрістіанство бо есть духовное тѣло. Въ тѣлѣ вещественномъ единъ членъ о другомъ промышляетъ и помогаетъ, напр. рука руку отираетъ, моетъ, глаза все тѣла берегутъ, и проч.: такъ и въ духовномъ тѣлѣ, то"=есть хрістіанствѣ, единъ о другомъ пещися долженъ. Сіе бываетъ молитвою, совѣтомъ, любовнымъ обличеніемъ. Истинная бо любовь хощетъ и ближнему того, чего себѣ хощетъ, и тщится ближняго отъ того отвратить, отъ чего сама убѣгаетъ. Когда Хрістосъ по воскресеніи Своемъ вопрошалъ Апостола Петра: \textit{Симоне Іонинъ! любиши ли Мя?} и Петръ святый отвѣщалъ: \textit{ей, Господи, Ты вѣси, яко люблю Тя?} тогда сказалъ ему: \textit{паси овцы Моя}\footnote{Іоан.~21,~16.}. Аки бы тако сказалъ Господь: \textit{Симоне! аще любиши Мя, паси овцы Моя}. Отъ сего видно, коль дорого почитаетъ спасеніе наше Господь Іисусъ. А тѣмъ научаетъ, что кто хощетъ любить Его, тотъ долженъ не токмо о своемъ, но и о ближняго спасеніи пещися. Откуду Апостолъ съ жалѣніемъ глаголетъ: \textit{вси своихъ си ищутъ, а не яже Хріста Іисуса}\footnote{Филип.~2,~21.}. Отсюду заключается, какъ тяжко согрѣшаютъ противу Хріста пастыри, которые, на паствѣ будучи, не овецъ Хрістовыхъ, но себе только пасутъ, за что страшному суду Божію подлежатъ, \textit{яко взыщетъ Господь овецъ своихъ отъ руки ихъ}\footnote{Іез.~34,~10.}. "--- 9)~Любитель любимому славы и чести хощетъ, и ищетъ, и радуется, когда любимый его славится, "--- печалится, когда обезславляется. Тако сынъ радуется, когда слышитъ, что отецъ его доброе имя имѣетъ, и другъ веселится, когда любимый его другъ прославляется. Любовь бо трогается тѣмъ, яко своимъ, что въ любимомъ видитъ, и тако благополучіе и бѣдствіе любимаго за свое вмѣняетъ. Такъ, кто Хріста любитъ, ищетъ во всемъ славы имени Его, а не своему, и радуется о томъ, когда славится святое имя Его. Славится же имя Хрістово тогда, когда исповѣдующіе имя Его живутъ достойно исповѣданія, и которую вѣру устами исповѣдуютъ, тую добрыми дѣлами оказываютъ. Тако и человѣку, напримѣръ, отцу, когда дѣти его добродѣтельно живутъ, и господину, когда рабы его постоянно обращаются, честь и слава бываетъ. Люди бо, видя честное житіе дѣтей, отца похваляютъ, "--- и постоянство слугъ, господину честь приписуютъ: знать"=де добрый отецъ, что такихъ дѣтей имѣетъ, и разумный господинъ, что такъ порядочно рабовъ содержитъ. Отсюду видно, что беззаконныхъ хрістіанъ житіемъ Хрістосъ безчестится. "--- 10)~Любитель не любитъ того, кто любимому противенъ есть: сынъ противника отца своего, жена противника мужа своего, другъ противника друга своего, не любитъ. Сердце бо, въ которомъ любовь мѣсто свое имѣетъ, раздвоено быть не можетъ, но или къ тому, или къ другому прилѣпляется, и противныхъ между собою вдругъ любити не можетъ, но непремѣнно, одного оставивши, къ другому прилѣпляется, якоже глаголетъ Хрістосъ: \textit{никтоже можетъ двѣма господинома работати: любо единаго возлюбитъ, а другаго возненавидитъ: или единаго держится, о друзѣмъ же нерадѣти начнетъ}\footnote{Мѳ.~6,~24.}. Міръ Хрісту противенъ есть, убо кто любитъ Хріста, тотъ не любитъ міра, по ученію Апостола: \textit{не любите міра, ни яже въ мірѣ}\footnote{1~Іоан.~2,~15.}. Міръ здѣ разумѣется не небо и земля съ исполненіемъ своимъ, но \textit{похоть плотская, похоть очесъ и гордость житейская}, какъ таможде Апостолъ учитъ\footnote{ст.~16.}. Чрезъ сіе сатана насъ, какъ Адама въ раи чрезъ вкушеніе отъ заповѣданнаго древа, отъ Хріста и послушанія Ему должнаго отводитъ. Богатство, честь, слава и сласти міра сего суть пріятное зрѣлище плоти нашей; на сія указуетъ она, сатаною прельщаема, древнимъ онымъ зміемъ, какъ Ева указывала Адаму на яблоко заповѣданнаго древа, и хощетъ отвратить сердце отъ любви Хрістовой и обратить къ созданію Его, и вмѣсто Создателя созданіе любить. Сердце, любовію Хрістовою объятое, отвращается отъ злаго сего совѣта, хотя по внѣшнему виду и добръ быти кажется. При семъ знай, хрістіанине, что иное есть имѣть богатство, честь и славу, а иное любить. Имѣніе богатства, чести и славы правильное не изгоняетъ любве Хрістовой. Многіе были богаты и славны, какъ то: Авраамъ, Исаакъ, Іаковъ, Давидъ царь и прочіи, какъ и нынѣ многіи суть благочестивыи цари и высокія лица, но были истинныи любители Хрістовы. Но любовь богатства, чести и славы міра сего не помѣщается съ любовію Хрістовою, а едина другую изгоняетъ. Не можетъ любить Хріста, пока любитъ сребро, злато, почитаніе и прославленіе отъ человѣкъ; напротивъ того, не любитъ сребра, чести и славы человѣческой, пока любитъ Хріста, по словеси Хрістову: \textit{никтоже можетъ двѣма господинома работати}, и проч.\footnote{Мѳ.~6,~24.} Какъ ни обращай и превращай мнѣніе свое, непремѣнно надобно къ единому изъ нихъ пристать и прилѣпиться "--- или ко Хрісту, и оставить міръ, или къ міру, и оставить Хріста. И отъ сего видно, что пристрастившіеся къ міру, то"=есть, чести, богатству и сласти міра сего, оставляютъ Хріста, и тако вѣру погубляютъ, и отрекаются Его сердцемъ, хотя устами исповѣдуютъ, якоже учитъ Апостолъ: \textit{Бога исповѣдуютъ вѣдѣти, дѣлы же отмещутся Его}\footnote{Тит.~1,~16.}. "--- 11)~Истинная любовь до того достигаетъ, что любитель за любимаго умереть не отрекается. Любовь бо любителя тако крѣпко связуетъ съ любимымъ, что лучше вся терпитъ, и самую смерть, нежели отъ любимаго разлучится, и пріятнѣйшая ему есть смерть и горесть всякая, нежели разлученіе отъ любимаго. Тако Сына Божія любовь къ роду человѣческому убѣдила воплотитися и умереть за отпадшаго человѣка, и тако присвоить его Себѣ, и \textit{причастникомъ Божественнаго естества}\footnote{2~Петр.~1,~4.} и вѣчныя славы сдѣлати, нежели отриновеннаго и погибающаго видѣти. Тако и истинному Хрістову любителю пріятнѣе всякую горесть и всякую смерть претерпѣть, нежели Хріста и Хрістовой любви разлучиться. Истинная бо и горячая любовь такъ крѣпка и сильна, что ее ничто побѣдить, и союза, которымъ съ любимымъ своимъ связана, разорвать не можетъ. Истинный Хрістовъ любитель связанъ, въ темницу заключенъ, окованъ, на уды раздробленъ, сожженъ и умѣрщвленъ быть можетъ, но побѣжденъ быти не можетъ. Сила бо любви есть духовная, а не тѣлесная, и потому, хотя тѣло побѣждается, ранится, мучится и умерщвляется, но любовь побѣдитися не можетъ; паче же тѣмъ болѣе показываетъ свою силу и крѣпость, чѣмъ болѣе претерпѣваетъ и страждетъ. Такую имѣли любовь святые мученики, которые изволяли предавать себе на жесточайшія мученія и горестнѣйшія смерти, нежели отлучитися отъ любимаго Сына Божія; а многіе сами, ревностію по любимомъ Хрістѣ распалившися, на мѣстѣ мученія и предъ мучителей приходили, и самопроизвольно изволяли страдати за имя Господа Іисуса Хріста. Дивно тое намъ, любезный хрістіанине, что тако любятъ Хріста человѣки; но то дивнѣе, и нашъ разумъ превосходитъ, что Хрістосъ насъ грѣшныхъ тако возлюбилъ, что и умереть за насъ изволилъ. Человѣки любятъ Хріста возлюбившаго: \textit{мы}, глаголетъ Апостолъ, \textit{любимъ Его, яко Той первѣе возлюбилъ есть насъ}\footnote{1~Іоан.~4,~10.}: Хрістосъ возлюбилъ насъ, враговъ Своихъ и отступниковъ. Сей любви удивляйся, хрістіанине! Отсюду \textit{вкуси и виждь, яко благъ Господь}\footnote{Пс.~33,~9.}.

Примѣчай, хрістіанине: 1)~Что всѣ сіи дѣйствія и плоды любве отъ того слѣдуютъ, что любитель съ любимымъ сердцемъ и любовію соединенъ, и изъ двухъ едино бываетъ. Любовь бо изъ двухъ сердецъ и душъ едино сердце и душу дѣлаетъ не вещественно, но духовно и единомудренно, якоже о хрістіанахъ, во время апостолъ бывшихъ, Лука святый повѣствуетъ: \textit{народу вѣровавшему бѣ сердце и душа едина}\footnote{Дѣян.~4,~32.}. "--- 2)~Видишь, какъ сильна и крѣпка есть любовь, такъ что союза ея ничто разрушити не можетъ. "--- 3)~Какъ безъ любви всякое дѣло мертво и непотребно есть, такъ любовію всякое дѣло оживляется, и благопріятно всякому бываетъ. Безъ любви никакое не можетъ быть добро, а гдѣ любовь, тамъ все добро. "--- 4)~Истинная ко Хрісту любовь происходитъ отъ истинныя во Хріста вѣры и отъ Духа Святаго. Вѣра бо представляетъ Хріста вѣрному, яко Онъ есть истинный животъ, истинное блаженство, истинная и вѣчная радость и сладость, кромѣ котораго истинное блаженство не можетъ быть. И тако вѣрою просвѣщенное сердце чрезъ благодать Святаго Духа разжигается къ любви высочайшаго того добра, которое есть Іисусъ Хрістосъ, Сынъ Божій, со Отцемъ и Святымъ Духомъ. И чѣмъ болѣе познается Хрістосъ, ощущается благодать Его въ сердцѣ вѣрнаго, тѣмъ горячайшая къ Нему любовь возжигается: чѣмъ болѣе познается добро, тѣмъ болѣе любится. Ибо любить добра безъ познанія того не можемъ, якоже сладости меда познать не можемъ безъ вкушенія меда. Откуду написано: \textit{вкусите и видите, яко благъ Господь}\footnote{Пс.~33,~9.}. "--- 5)~Отсюду видишь, что безстрашное и беззаконное житіе какъ съ вѣрою, такъ и съ любовію Хрістовою помѣститися не можетъ, якоже Самъ Хрістосъ глаголетъ: \textit{не любяй Мя, словесъ Моихъ не соблюдаетъ}\footnote{Іоан.~14,~24.}. Такожде сердце, пристрастившееся къ міру, какъ вѣры, такъ и любве никакой ко Хрісту не имѣетъ, какія бы ни дѣлалъ человѣкъ внѣшнія дѣла. Все бо, что ни противно есть Хрісту и отъ произволенія бываетъ, вѣру и любовь Хрістову изъ сердца изгоняетъ. А оттуду слѣдуетъ, что Апостолъ написалъ: \textit{иже восхощетъ другъ быти міру, врагъ Божій бываетъ}\footnote{Іак.~4,~4.}. Страшно сіе, но истинно есть. Ибо колико разъ человѣкъ грѣшитъ отъ произволенія, толико разъ законъ Божій и вѣчный разоряетъ и противится Законодавцу; и колико разъ къ міру уклоняется и прилѣпляется, толико разъ со Адамомъ къ заповѣданному древу руку свою простираетъ. Богъ бо заповѣдалъ: \textit{не любите міра, ни яже въ мірѣ}\footnote{1~Іоан.~2,~15.}. "--- 6)~Отсюду заключается, что хрістіане, потерявшіе вѣру, и съ вѣрою любовь (вѣра бо безъ любви быть не можетъ), оставили и Самого Хріста, и вмѣсто Хріста похоть плотскую, похоть очесъ и гордость житейскую, какъ тройственнаго бога, почитаютъ. "--- 7)~Едину плотскую, а не хрістіанскую любовь имѣютъ, которые единыхъ друговъ своихъ и любящихъ себе только любятъ. Истинная бо хрістіанская любовь всѣхъ, друговъ и враговъ своихъ, объятіями своими объемлетъ. Такожде, которые сродниковъ своихъ, напр. дѣтей, братію, отцевъ, матерей и прочіихъ кровныхъ обильно награждаютъ, а бѣдныхъ и убогихъ, ради имени Хрістова просящихъ, или пренебрегаютъ совсѣмъ и отсылаютъ, или полушкою довольствуютъ, "--- всѣ таковые себе, плоть и кровь свою любятъ, а не Хріста и ради Хріста. Хрістіанская бо любовь надъ сродниками и несродниками умилостивляется; ей тотъ и сродникъ, кто бѣденъ и нищъ; тамо она и склоняется и руку свою простираетъ, гдѣ бѣдность объявляется и Хрістово поминается имя. "--- 8)~Непохвально дѣлаютъ, и самолюбіемъ недугуютъ, которые другимъ благотворятъ ради того, чтобы ихъ любили; и здѣ бо не ради имени Хрістова, но ради своей пользы благотвореніе бываетъ. Истинный Хрістолюбецъ и нелюбящимъ себе благотворитъ, внимая слову Хрістову: \textit{просящему у тебе дай}\footnote{Мѳ.~5,~42.}. "--- 9)~Сюды принадлежатъ и тѣ, которые того ради даютъ милостыню, чтобы пріемлющіе молились Богу за нихъ; ибо и здѣ своя польза ищется, а не Хрістова честь и ближняго польза. Иначе бы не давали, когда бы не надѣялись отъ ихъ молитвы. Истинная бо любовь \textit{не ищетъ своихъ}\footnote{1~Кор.~13,~5.}. И всякое дѣло отъ конца судится, то"=есть, въ Божію ли славу и ближняго пользу дѣлается, или ради своей какой корысти. Должны пріемлющіе милостыню благодарны быть къ своимъ благодѣтелямъ и молитися за нихъ; но подающимъ должно просто подавать, безъ своей пользы, и взирать только на бѣдность просящаго, котораго Хрістосъ повелѣлъ миловать. А хотя будутъ, или не будутъ молитися пріемлющіе милостыню, милостыня свое воздаяніе неотмѣнно получитъ, по реченному: \textit{блажени милостивіи, яко тіи помиловани будутъ}\footnote{Мѳ.~5,~7.}. Милостыня бо, просто, отъ чистаго и любительнаго сердца подаемая, безъ гласа молится, и болѣе молится, нежели всѣ человѣки. "--- 10)~Истинному Хрістолюбцу не токмо милостыни, но и прочіимъ добрымъ дѣламъ, какъ"=то: кротости, незлобію, терпѣнію, воздержанію, цѣломудрію, смиренномудрію и прочіимъ хрістіанскимъ добродѣтелямъ прилѣжать должно, яко Хрістосъ симъ и словомъ и примѣромъ Своимъ училъ насъ. Аще убо хощемъ Хріста любить и Ему угождать, должно намъ добродѣтельно жить, творить чему Онъ насъ словомъ и дѣломъ училъ.

Чтобы съ помощію Божіею къ любви Хрістовой возбудиться намъ, хрістіанине, разсудимъ нѣкоторыя обстоятельства ея, и съ помощію Божіею поучимся: 1)~Хрістосъ глаголетъ: \textit{любяй Мя возлюбленъ будетъ Отцемъ Моимъ}\footnote{Іоан.~14,~21.}. О коль утѣшительное слово сіе есть! толь великіе и высокіе обѣты предлагаетъ любителю Своему Сынъ Божій, такъ что истинный Хрістовъ любитель дружество имѣетъ со Отцемъ и Сыномъ Его, а гдѣ Отецъ и Сынъ, тамо и Духъ Святый неотлученъ. Дружество бо не иное что, какъ взаимная любовь "--- любитися и любить. Умъ человѣческій не постигаетъ сея благости Божіей. Богъ великій, безсмертный, безконечный и непостижимый съ человѣкомъ, созданіемъ и рабомъ Своимъ, дружество хощетъ имѣть, и имѣетъ, кто самъ себе отъ того не отлучитъ. \textit{Общеніе наше со Отцемъ и съ Сыномъ Его Іисусъ Хрістомъ}, глаголетъ Апостолъ хрістіанамъ во утѣшеніе\footnote{1~Іоан.~1,~3.}. Откуду вѣрнымъ Своимъ, возлюбленнымъ отъ Него и любящимъ Его, глаголетъ: \textit{вы друзи Мои есте, аще творите, елика Азъ заповѣдаю вамъ}\footnote{Іоан.~15,~14.}. Высокое и великое дѣло есть съ земнымъ царемъ дружество имѣть, несравненно большее съ Царемъ небеснымъ. Возлюбимъ убо, любезный хрістіанине, другъ друга, да тако покажемъ и ко Хрісту любовь, и возымѣемъ общеніе со Отцемъ и съ Сыномъ Его Іисусомъ Хрістомъ. "--- 2)~Глаголетъ паки Хрістосъ: \textit{явлюся ему Самъ}\footnote{14,~21.}. \textit{Кому явлюся?} Человѣку, рабу и грѣшнику. Кто? Царь небесе и земли, Господь славы, Сынъ Божій и Слово Отчее. Не чрезъ ангела, или инаго кого посѣщу его, но \textit{Самъ явлюся ему, Самъ} посѣщу его, \textit{Самъ} пріиду къ нему, \textit{явлюся ему Самъ}. Кому? Любителю Моему и соблюдающему заповѣди Моя. О силы и дѣйствія любве, которая Господа къ рабу, и Бога къ человѣку, и Создателя къ созданію привлекаетъ! О благости Твоей, Іисусе! Ты изливаешь любовь въ сердца вѣрующія въ Тя, и къ любящимъ Тя привлекаетъ Тя любовь Твоя. Кто бо Тебе безъ Тебе можетъ истинно любити? Сердце наше безъ Тебе и Твоей благодати не иное что, какъ розга изсохшая, которая никакого плода не творитъ. Твоя любовь предваряетъ всѣхъ, и возжигаетъ въ сердцахъ нашихъ хладныхъ любовь къ Тебѣ; и за благодать Твою, которой мы содѣйствуемъ, новую воздаеши благодать. Тако вездѣ благость Твоя обходитъ и предваряетъ и послѣдуетъ намъ. "--- Како является Хрістосъ любящимъ Его душамъ? Когда во плоти на земли жилъ и проповѣдывалъ слово Свое, всѣмъ Себе являлъ, другамъ и врагамъ Своимъ. Но по воскресеніи Своемъ показывалъ Себе единымъ другамъ, какъ читаемъ. Показалъ Себе съ небесе Стефану святому, когда о Немъ свидѣтельствовалъ предъ врагами его\footnote{Дѣян.~7,~55 и 56.}. И сіе есть \textit{видимое} явленіе Хрістово. \textit{Духовнѣ} является другамъ Своимъ, когда въ сердца ихъ входитъ чрезъ Святаго Своего Духа и просвѣщаетъ, утѣшаетъ и радостотворитъ ихъ. Явится же всѣмъ любящимъ Его, егда пріидетъ въ страшной славѣ Своей, и призоветъ ихъ въ вѣчное Свое царство, глаголя: \textit{пріидите благословенніи Отца Моего, наслѣдуйте уготованное вамъ царствіе отъ сложенія міра}\footnote{Матѳ.~25,~34.}. Явится тогда и врагамъ Своимъ, но инымъ образомъ, то"=есть, яко страшный Царь и грозный Судія, отъ Котораго гнѣва пожелаютъ скрытися въ пещерахъ и разсѣлинахъ каменныхъ\footnote{Апок.~6,~16.}, и отринетъ ихъ отъ Себе \textit{въ огнь вѣчный, уготованный діаволу и аггеломъ его}\footnote{Матѳ.~25,~41.}. Отверземъ и мы, хрістіанине, сердца нашему Любителю Іисусу, да и насъ человѣколюбіемъ Своимъ посѣтитъ превожделѣнный гость Сей. Онъ ко всякому хощетъ пріити, кто не затворитъ предъ Нимъ дому своего. \textit{Се стою при дверехъ, и толку: аще кто услышитъ гласъ Мой, и отверзетъ двери, вниду къ нему, и вечеряю съ нимъ, и той со Мною}\footnote{Апок.~3,~20.}. Отверземъ Ему нынѣ двери сердецъ нашихъ, да тогда отверзетъ намъ двери царствія Своего. "--- 3)~Паки глаголетъ Хрістосъ: \textit{къ нему пріидемъ, и обитель у него сотворимъ}\footnote{Іоан.~14,~23.}. Смотри, хрістіанине, чего благодатію сподобляется любитель Хрістовъ: домомъ и жилищемъ Пресвятыя Троицы бываетъ! Богъ Тріѵпостасный, Отецъ, Сынъ и Святый Духъ благодатно въ немъ изволяетъ обитати! За велико почитаютъ люди царя земнаго въ домъ свой принять; коль несравненно большее преимущество есть Царя небеснаго въ домъ сердца своего принять, и не токмо принять, но и живущаго имѣть, якоже глаголетъ: \textit{къ нему пріидемъ и обитель у него сотворимъ}. Блаженное есть воистинну таковое сердце. Что тамо иное можетъ быть, какъ токмо миръ, покой, радость и веселіе? \textit{Царствіе Божіе внутрь его есть}\footnote{Лук.~17,~21.}. Оно еще въ сей юдоли плачевной чувствуетъ радость, которая изобильно изліется въ сердца избранныхъ вѣчной жизни; оно самою вещію вкушаетъ, \textit{коль благъ Господь}\footnote{Пс.~33,~9.}. \textit{Сицева убо имуще обѣтованія, о возлюбленніи! очистимъ себе отъ всякія скверны плоти и духа, творяще святыню въ страсѣ Божіи}, увѣщаваетъ насъ Апостолъ\footnote{2~Кор.~7,~1.}. Не обитаетъ бо Господь въ сердцахъ, нечистою міра сего любовію оскверненныхъ. \textit{Кое бо причастіе правдѣ къ беззаконію, или кое общеніе свѣту ко тмѣ}\footnote{6,~14.}? \textit{Богъ} бо нашъ \textit{свѣтъ есть, и тьмы въ Немъ нѣсть ни единыя}\footnote{1~Іоан.~1,~5.}, міра же сего любовь есть тьма, \textit{яко все, еже въ мірѣ, похоть плотская, и похоть очесъ и гордость житейская, нѣсть отъ Отца, но отъ міра сего есть}\footnote{1~Іоан.~2,~16.}. Хотящіе царя земнаго принять въ домъ очищаютъ домъ, кольми паче хотящему принять Царя небеснаго должно очистить домъ сердца своего. "--- 4)~Якоже сладость любве Хрістовой привлекаетъ, тако благодѣяніе Его, которое намъ отъ единыя презѣльныя къ намъ любви показалъ, убѣждаетъ насъ любить Хріста. Помяни, хрістіанине, чего не дѣлалъ Сынъ Божій ради насъ, чего не изобрѣлъ, чего не претерпѣлъ и не страдалъ ради убогія и бѣдныя души твоея? какихъ трудовъ и болѣзней не принялъ, чтобы насъ отпадшихъ къ небесному Отцу Своему привести? Ежелибъ кто гнѣвъ царя земнаго, который грозилъ тебѣ смертною казнію, отъ тебе отвратилъ, и вмѣсто смерти животъ тебѣ у него испросилъ, и, недовольствуяся сею высокою къ тебѣ милостію, знатный и высокій чести степень исходатайствовалъ тебѣ, всякое ради того бѣдствіе изволяя терпѣть: о, коль усердно и искренно любилъ бы ты такого благодѣтеля! Самое бо естество убѣждаетъ насъ къ тому. Хрістосъ Сынъ Божій гнѣвъ, который вѣчною намъ всѣмъ грозилъ смертію, отвратилъ отъ насъ, когда вѣруемъ въ Него, а тако отъ вѣчныя смерти избавилъ насъ; сверхъ того такъ въ высокую милость Богу привелъ, что \textit{даде область чадами Божіими быти вѣрующимъ во имя Его}\footnote{Іоан.~1,~12.}, которая честь такъ высока, что вся слава міра сего есть какъ ничто предъ нею. Всякое бо міра сего благородіе и величество есть только пустое имя и титулъ, самой вещи въ себѣ не имущій, который подобенъ пузырю водному, или мѣху воздухомъ надмѣнному, который внѣ нѣчто является, но внутрь ничего не имѣетъ. Хрістіанское же благородіе "--- истинное, вѣчное и непостижимое, которое Хрістосъ намъ кровію Своею заслужилъ. И тако скажи пожалуй, какъ толикаго, такъ высокаго Любителя и Благодѣтеля намъ не любить? Аще бо малый благодѣтель достоинъ любви, кольми паче великій; аще кто отъ временной бѣды, напасти и смерти спасетъ, временное и земное добро сдѣлаетъ намъ, кольми паче Хрістосъ нашъ, о хрістіане, достоинъ того, Который не отъ временныя бѣды и смерти избавилъ насъ, но отъ вѣчныя; не временный, но вѣчный животъ, не временную и земную, но вѣчную честь и славу исходатайствовалъ намъ. О, коль нечувственна ты, грѣшная душа, когда сего не чувствуешь! Нечувственна воистину, когда человѣка, временно и мало одолжившаго (\textit{мало}, глаголю, ибо все временное противу вѣчнаго ничто есть), однакожъ и такого благодѣтеля любишь, благодаришь ему, почитаешь его, услуживаешь ему, и все, что угодно ему есть, дѣлаеши; но Хрістосъ, Который \textit{вѣчно} и \textit{безконечно} одолжилъ тебе, Котораго благодѣяніе, показанное тебѣ, ничѣмъ наградиться не можетъ, и того, чего человѣкъ"=благодѣтель, не сподобляется отъ тебе. Такъ"=то ты подло поставляешь Хрістово ради тебе бывшее воплощеніе, рожденіе, на землѣ пожитіе, терпѣніе, смиреніе, страданіе и крестную смерть, которой благодати ангели удивляются, поютъ, славятъ и величаютъ\footnote{Лук.~2,~14.}. Хрістосъ ради тебе и Себе Самого не пощадѣлъ, ты же ради имени Его мѣди и сребра, да еще Его, (все бо, что имѣешь, Его есть), жалѣешь. Хрістосъ ради тебе смирился, ты же ради Его не хощешь оставити гордости твоей. Хрістосъ ради тебе, богатъ сый, обнищалъ, ты же ради Его не хощешь оставити скупости и сребролюбія твоего. Хрістосъ ради тебе поношеніе, удареніе, заплеваніе, распятіе и крестную смерть претерпѣлъ: ты же ради Его и укорительнаго слова не хощешь стерпѣть. Тебе ради сошелъ съ небесъ, чтобы тебе, изъ рая изгнанную, на небо вознести; тебе ради родился плотію, чтобы тебе, отрожденную Духомъ, Себѣ присвоить; тебе ради смирился, чтобы тебе возвысить; тебе ради обнищалъ, чтобы тебе убогую обогатить; тебе ради безчестіе и раны претерпѣлъ, чтобы тебе исцѣлить и прославить; тебе ради умеръ, чтобы тебе умершую оживить. Къ сему такъ глубокому снисхожденію и смиренію не иное что, какъ презѣльная Его къ тебѣ любовь и сострадательное милосердіе привлекло Его. Сію ли ты такъ высокую Его любовь и благодать пренебрегаешь, и не хощешь Ему любовно служить и угождать, но оставивши Его, истиннаго твоего Любителя, отдаешь сердце противнику Его? "--- 5)~Что ты сыскалъ, грѣшниче, въ Любителѣ твоемъ Іисусѣ неугодное тебѣ? что видишь въ Немъ, что бы тебе къ любви Его не привлекало? Желаетъ человѣкъ блаженства, "--- истинное и вѣчное блаженство у Него. Желаетъ человѣкъ красоты, "--- \textit{Онъ краснѣйшій добротою паче сыновъ человѣческихъ}\footnote{Пс.~44,~3.}. Хощемъ благородства, "--- кто благороднѣйшій Сына Божія? Ищемъ чести, "--- кто честнѣе и высше Царя небеси? Славы ли, "--- Ѵпостасная Божія премудрость Онъ есть. Дружества ли, "--- кто любезнѣйшій и любительнѣйшій паче Его? Веселіе любитъ человѣкъ, "--- радость и веселіе блаженныхъ духовъ и избранныхъ Божіихъ Онъ есть. Утѣшеніе ли потребно тебѣ, "--- кто утѣшитъ тебѣ, кромѣ Іисуса? Миръ потребенъ тебѣ, "--- Онъ есть Царь мира, душевный миръ Онъ есть. Покоя ищеши, "--- Іисусъ обѣщаетъ и подаетъ покой вѣчный душамъ, любящимъ Его. Живота желаешь и ищешь, "--- у Него источникъ живота. Заблудити опасаешися, "--- Іисусъ есть путь. Прельститися боишися, "--- Іисусъ есть истина. Смерти ужасаешися, "--- Іисусъ есть животъ, якоже глаголетъ: \textit{Азъ есмь путь и истина и животъ}\footnote{Іоан.~14,~6.}. Словомъ, все блаженство у Него, и кромѣ Его нѣтъ никакого блаженства. Смотри и примѣчай, о бѣдная, грѣшная душа! Іисусъ Хрістосъ все блаженство у Себе имѣетъ, какое можетъ быть, и блаженство истинное и неподвижимое. Его Отецъ небесный подалъ намъ, яко врача противу всякаго нашего бѣдствія и источника всякаго блаженства. Читай святое Евангеліе Его, какъ должно, и познаеши истину. Аще убо прилѣпишися къ Нему сердечною вѣрою, будеши имѣти Его оправданіе противу грѣховъ твоихъ, избавленіе противу плѣненія твоего, миръ противу злыя совѣсти твоея, отраду противу скорби твоея, утѣшеніе противу печали твоея, защищеніе противу клеветниковъ твоихъ, оправданіе противу суда Божія, помощь противу враговъ твоихъ, убѣжище противу гонителей твоихъ, разумъ и мудрость противу недоумѣнія и безумія твоего, укрѣпленіе въ слабости твоей, животъ противу смерти твоей, "--- словомъ, всякое блаженство противу окаянства твоего. Когда бо Іисусъ блаженство твое, какое бѣдствіе приключится тебѣ? Аще Іисусъ миръ и покой твой, кто можетъ обезпокоити тебе? Аще Іисусъ радость твоя, кто и что можетъ оскорбити тебе? Аще Іисусъ утѣшеніе твое, кто можетъ опечалити тебе? Аще Іисусъ честь и слава твоя, кто можетъ обезславити тебе? Аще Іисусъ заступникъ твой, кто можетъ оклеветати и осудити тебе? Аще Іисусъ помощникъ и поборникъ твой, кто можетъ побѣдити тебе? Аще Іисусъ избавитель твой, кто можетъ плѣнити тебе? Аще Іисусъ оправданіе твое, кто можетъ осудити тебе? Аще Іисусъ пастырь твой, кто можетъ отъ руки Его восхитити тебе? Аще Іисусъ Царь и Господь твой, кто можетъ поработити тебе? Аще Іисусъ животъ твой, кто можетъ умертвити тебе? \textit{Аще Богъ по насъ, кто на ны? Иже убо Своего Сына не пощадѣ, но за насъ всѣхъ предалъ есть Его, како убо не и съ Нимъ вся намъ дарствуетъ? Кто поемлетъ на избранныя Божія? Богъ оправдаяй. Кто осуждаяй? Хрістосъ Іисусъ умерый, паче же и воскресый, Иже и есть одесную Бога, Иже и ходатайствуетъ о насъ}, дерзаетъ и глаголетъ Павелъ съ любящими Бога\footnote{Римл.~8,~31--34.}. Разсуждай, грѣшная душа, что есть Іисусъ, и что любящимъ Его даруетъ. Даруетъ, \textit{ихже око не видѣ, и ухо не слыша и на сердце человѣку не взыдоша, яже уготова Богъ любящимъ Его}\footnote{1~Кор.~2,~9.}. "--- 6)~Нынѣ обрати очи твои на міръ и разсмотри, что онъ такое, и что любителямъ своимъ даетъ. Представляетъ тебѣ образъ его Апостолъ и глаголетъ: \textit{міръ весь во злѣ лежитъ}\footnote{1~Іоан.~5,~19.}. Зло во градѣхъ, въ селѣхъ, зло въ домѣхъ, зло на путехъ. Сосѣдъ на сосѣда враждуетъ; путникъ съ товаромъ боится разбоя: \textit{врази человѣку домашніи его}\footnote{Матѳ.~10,~36.}; нищій богатому завидуетъ, поселянинъ гражданину, сановитый подлаго презираетъ, подлый негодуетъ; жена надъ мужемъ, мужъ надъ женою подзираетъ; господинъ рабу не вѣритъ, рабъ господину льститъ и лукавнуетъ. Всѣ другъ друга опасаются. Единъ крадетъ, другій окраденъ жалуется и проклинаетъ. Тотъ клевещетъ, другій оклеветаемъ усугубляетъ оклеветающему. Иный досаждаетъ, другій отмщеваетъ досадившему. Отъ сего ссоры, брани, взаимныя злословія; за симъ слѣдуютъ драки и кровопролитія; отсюду другъ на друга жалобы, другъ друга судятъ и осуждаютъ, а всѣ неправы. Доходитъ дѣло до суда; тутъ зло злу послѣдуетъ новое и большее. Вымышляютъ другъ на друга язвительныя обличенія, другъ друга погубить, или въ крайнее бѣдствіе вринуть тщатся. Сами себѣ безпокойствуютъ душевредно, безпокойствуютъ и тѣмъ, кому долгъ надлежитъ разбирать праваго и виноватаго. Тако міръ непрестанно, какъ море, волнуется, и любителей своихъ утруждаетъ. Любитель Хрістовъ не тако. Онъ, по подобію корабля, въ тихой терпѣнія гавани стоитъ, ни вѣтра, ни бури, ни волнъ не боится. Хотя и вѣютъ на него вѣтры искушеній и ударяютъ волны бѣдъ, но онъ тишиною терпѣнія безопасенъ стоитъ. Самъ ни на кого не враждуетъ, и враждующимъ прощаетъ, и такъ тихъ, покоенъ, миренъ внутрь себе; всегда и вездѣ сокровище свое съ собою носитъ. Обѣщаетъ міръ и представляетъ любителямъ своимъ блаженство свое, но прелестное и ложное. Блаженство его состоитъ въ чести, славѣ и сласти; но сіе блаженство подобно есть сонному мечтанію, или мѣху надмѣнному воздухомъ, который дотолѣ показуется быть полнымъ и нѣчто имѣющимъ, доколѣ воздухъ не выйдетъ; или подобно водному пузырю, который какъ скоро является, такъ скоро и исчезаетъ. Смотри и разсуждай, сколько за честію и славою гоняются любители міра. Но она коль трудно снискивается, толь удобно теряется. И чѣмъ большая честь, тѣмъ удобнѣе погубляется, ибо столько враговъ себѣ имѣетъ, сколько завистливыхъ есть и властолюбивыхъ сердецъ. Отъ сихъ всѣхъ надобно берещися хотящему не потерять честь свою; понеже, когда завидуютъ сѣдящему на высокомъ мѣстѣ, то хотятъ его въ той высотѣ не видѣть. Какъ трудно единому отъ многихъ уберечься, а паче такихъ, которые извнѣ показуются пріятелями, но внутрь суть истинные и злѣйшіе враги! Слѣдственно или ядомъ, или мечемъ погибнуть, или низпасть съ высоты и въ заключеніи быть, или, что рѣдко бываетъ, во всегдашнемъ страхѣ и попеченіи, по видимому свѣтлое, но въ самой вещи горькое житіе надобно провождать. Сіе свѣтлое, но воистину горькое славолюбивыхъ житіе горчайшею заключается смертію, которой далеко горчайшая другая и вѣчная послѣдуетъ. *Гдѣ нынѣ славные вселенныя побѣдители, которымъ мало было всѣмъ свѣтомъ владѣть? Какъ смертные померли.* Гдѣ слава ихъ, наполнившая вселенную? Погреблась съ ними во гробѣ, или паче исчезла, какъ и сами: заключились въ треаршинномъ мѣстѣ, которые всѣмъ свѣтомъ не довольны были; попираются ногами отъ подлѣйшихъ, которыхъ весь свѣтъ трепеталъ; обратилися въ прахъ, которые хотѣли, какъ боги, почитаемы быть, такъ что знать не можно, гдѣ Александръ, побѣдитель вселенныя, и послѣдній его рабъ лежитъ. И когдабъ съ окончаніемъ живота и сами вовсе окончались. Но нѣтъ; желали бы они того, но не дается имъ. Надобно имъ за все платить правдѣ Божіей, что въ свѣтѣ неправедно сдѣлали; надобно \textit{толико принять мученія, колико прославились здѣ и разсвирѣпѣли}\footnote{Апок.~18,~7.}. О! когда бы хотя единъ отъ нихъ возмоглъ, отъ того бѣдствія, въ которомъ нынѣ находятся, освободившися, явиться въ міръ, непремѣнно бы въ иномъ и далеко лучшемъ мнѣніи находился, нежели прежде, живучи въ надмѣнной и пустой гордости; и желающимъ прославитися на земли и на мнимую прелестную высоту взыти тщащимся здравый бы подалъ совѣтъ. Какъ бы ублажалъ тѣхъ, которые своимъ довольствуются жребіемъ, себе самихъ побѣждать, нежели другихъ, себѣ самимъ повелѣвать, нежели инымъ, вѣчную славу, нежели временную искать болѣе тщатся! Призналъ бы подлинно, что нѣтъ большаго побѣдителя на земли, какъ кто самъ себе побѣждаетъ и надъ своими страстьми господствуетъ. Но се есть смертоносная человѣческая и общая слѣпота, что дотолѣ не вѣримъ бѣдствію другихъ, доколѣ сами на себѣ того не узнаемъ; гоняемся за тѣнью, оставивши самую вещь и истину; ищемъ тамо покоя и утѣхи, гдѣ безпокойствіе и горесть. Коль убо премудръ и блаженъ, кто отъ чуждаго бѣдствія учится бѣды уклоняться! "--- Какая"=де истинная честь и слава? Есть та, которую Хрістосъ любящимъ Себе подаетъ. Она въ семъ мірѣ сокровенна, и только вѣрою постигается, но открыется въ будущемъ вѣкѣ, когда \textit{праведницы просвѣтятся, яко солнце, во царствіи Отца ихъ}\footnote{Матѳ.~13,~43.}. Сію славу будутъ люди имѣть, но никогда ея не лишатся.

Вторая часть, которая горькое міра сего составляетъ блаженство, есть \textit{богатство}. Многіе и сего, какъ чести и славы, желаютъ и ищутъ, но не всѣ обрѣтаютъ, а которые обрѣтаютъ, со многимъ такожде трудомъ обрѣтаютъ, съ большимъ берегутъ, съ великою печалію лишаются. Сколько таковыхъ есть, которые богатства, толикими трудами снисканнаго, въ единъ часъ лишилися и лишаются! Огню и татю и прочему бѣдственному случаю всякое подлежитъ богатство. Но, хотя кто до кончины пребудетъ въ семъ своемъ мнимомъ благополучіи, возметъ ли съ собою тое, исходя отъ міра? Нѣтъ, оставляетъ все въ мірѣ, что въ мірѣ имѣлъ. Сребро, злато, каменіе дорогое, земли, отчины, крестьяне, слуги, кони, кареты и все сокровище въ мірѣ остается; единъ отходитъ отъ свѣта, какъ въ свѣтъ вшелъ. Гробъ малъ какъ нищаго, такъ и богатаго воспріемлетъ и вмѣщаетъ. Все прочее уже не его, другому въ руки перешло. Тутъ праведно всякъ признаетъ, какъ безумно дѣлаетъ, кто блаженство въ богатствѣ полагаетъ. А что когда богатство было собрано неправдою, хищеніемъ, лестію и прочіими беззаконіями, какъ по большей части бываетъ? Тогда, оставляя богатство другимъ, самъ съ собою тяжкое беззаконій бремя относитъ; тогда вмѣсто мнимаго блаженства истинное бѣдствіе срѣтаетъ его: при самомъ исходѣ мучительная совѣсть, судомъ Божіимъ и вѣчною мукою устрашающая; по исходѣ праведное по дѣломъ воздаяніе настаетъ. Тогда узнаетъ прелесть, и познаетъ, что онъ трудился ради прибытка и корысти другихъ, а ради своей погибели. Тогда отъ сердца признаетъ, коль великая есть прелесть \textit{собирать, а не въ Бога богатѣть}\footnote{Лук.~12,~21.}. Весьма убо убогій тотъ богачъ, который деньгами изобилуетъ. Бѣдственное тое богатство, которое отъ истиннаго блаженства отводитъ. Нѣтъ тамо никакой прибыли, гдѣ уронъ душѣ. Весьма худо хрістіанину за временнымъ богатствомъ гоняться, которое вскорѣ оставить принужденъ бываетъ, и оставлять вѣчное, къ которому позванъ. "--- \textit{Какое"=де истинное и непоколебимое богатство?} Тое, которое Хрістосъ предлагаетъ намъ: \textit{скрывайте себѣ сокровище на небеси, идѣже ни червь, ни тля тлитъ, и идѣже татіе не подкопываютъ, ни крадутъ}\footnote{Мѳ.~6,~20.}. Сіе сокровище вѣрою, которая любовію поспѣшествуема бываетъ, собирается здѣ, и открывается на небеси любящимъ Бога. Тамо имъ отверзутся сокровища благъ вѣчныхъ, \textit{ихже око не видѣ, и ухо не слыша, и на сердце человѣку не взыдоша}\footnote{1~Кор.~2,~9.}.

Третія часть блаженства мірскаго состоитъ въ плотоугодіи, въ банкетахъ, пиршествахъ, веселостяхъ и во всемъ, что чувства услаждаетъ. Сіе мнимое блаженство дотолѣ показуется нѣчто быти, доколѣ совершается, и чувствъ касается; а какъ совершится, и чувствованіе престанетъ, то вмѣсто сладости истинная послѣдуетъ горесть. Тако, по совершеніи плотскія похоти, грызеніе злыя совѣсти бываетъ; тако по банкетахъ и піянствѣ разслабленіе членовъ, умноженіе вредныхъ мокротъ, болѣзнь въ головѣ, дрожаніе рукъ и ногъ, и всего состава тѣлеснаго многоразличная немощь наступаетъ. Молчу о ослѣпленіи ума, о безславіи, о безчестіи и прочіихъ сластолюбія плодахъ. Кто блудникомъ, прелюбодѣемъ и піяницею не гнушается? Никто благоразумный и того не похваляетъ, кто часто банкетуетъ. Что сказать о душевныхъ язвахъ, которыхъ сластолюбцы отъ сего мнимаго своего блаженства пріемлютъ? Сласть есть удица, которою сатана души ловитъ, и во адъ, мѣсто себѣ опредѣленное, бросаетъ. Въ самократчайшемъ времени услаждаются сластолюбцы, но во адъ сходятъ безъ конца горестную мученія чашу пить. Посмотри въ святое Евангеліе, и покажетъ тебѣ, что сіе истинно слово есть. Тамо читаемъ, что \textit{человѣкъ нѣкій бѣ богатъ, и облачашеся въ порфиру и виссонъ, веселяся на вся дни свѣтло}. Далѣе повѣствуетъ о немъ, что \textit{умре богатый, и погребоша его. И во адѣ возвелъ очи свои, сый въ мукахъ, узрѣ Авраама издалеча и Лазаря на лонѣ его; и той возгласи и рече: отче Аврааме! помилуй мя, и посли Лазаря, да омочитъ конецъ перста своего въ водѣ, и устудитъ языкъ мой, яко стражду во пламени семъ}\footnote{Лук.~16,~19,~22--24.}. По банкетахъ, по пированіяхъ, по драгихъ винахъ, по веселостяхъ и сластяхъ проситъ капли воды, и не получаетъ, но слышитъ отвѣтъ: \textit{помяни, яко воспріялъ еси благая твоя въ животѣ твоемъ}\footnote{ст.~25.}. Сія есть часть и жребій сластолюбцевъ. Къ сему концу ведетъ роскошь и сластолюбіе. О горестная сладость, за которою толикое бѣдствіе и горесть послѣдуетъ! "--- Какая же"=де истинная сладость? Есть тая, которую святыя души отъ любви Хрістовой чувствуютъ, якоже они ко Хрісту въ радости и сладости съ церковію восклицаютъ: \textit{весь еси желаніе, весь сладость, Слове Божій}, и проч.\footnote{Пѣсн.~9"~я гласа 2"~го; Пѣсн. пѣсн.~5,~16.} Сія сладость отъ печали по Бозѣ начинается и въ смиренныхъ слезахъ умножается, въ скорбяхъ веселитъ, въ напастяхъ утѣшаетъ, въ уныніи ободряетъ, въ отчаяніи обнадеживаетъ, при смерти не отлучается, на оный вѣкъ спутствуетъ, и тамо совершится, гдѣ и любовь свое совершенство пріиметъ. "--- Вотъ въ чемъ состоитъ міра сего блаженство, то"=есть, \textit{въ чести}, которыя любители къ вѣчному безчестію приходятъ; \textit{въ богатствѣ}, которое рачителей своихъ вѣчно нищими и убогими дѣлаетъ; \textit{въ сласти}, которой работающіе вѣчно горести вкушаютъ! А тако видишь, что троечастное есть блаженство, есть такое, какое того, который въ сновидѣніи нѣчто сладкое ястъ и піетъ, отъ всѣхъ почитается и покланяется, великое сокровище нашелъ, но пробудившися алчбу и жажду чувствуетъ, видитъ себе въ презрѣніи и нищетѣ. Тоежде непремѣнно и міролюбцамъ приключается, которые въ краткомъ времени, какъ во снѣ, упиваются изобиліемъ роскошей и веселостей мірскихъ, утѣшаются честію и богатствомъ, но при смерти и по смерти очувствовавшися, какъ отъ сна пробудившися, все противное видятъ. Вмѣсто краткихъ сладостей, вѣчныя чувствуютъ горести; вмѣсто богатства, убожество и нищету свою познаютъ; вмѣсто чести и славы, въ вѣчномъ безчестіи и безславіи себе видятъ. Того ради праведно и истинно, но поздно каются и болѣзнуютъ, и въ тѣснотѣ духа воздыхаютъ; тогда признаютъ, что то суть мірскія утѣхи, коль ложныя и прелестныя суть вѣка сего веселости, за которыми истинныя послѣдуютъ скорби; тогда исповѣдуютъ свое заблужденіе, но не во время. \textit{Убо заблудихомъ отъ пути истиннаго, и правды свѣтъ не облиста намъ, и солнце не возсія намъ. Беззаконныхъ исполнихомся стезь и погибели, и ходихомъ въ пустыни непроходимыя: пути же Господня не увѣдѣхомъ. Что пользова намъ гордыня? и богатство съ величаніемъ что воздаде намъ? Преидоша вся она яко сѣнь, и яко вѣсть претекающая; яко корабль, преходяй волнующуюся воду, егоже проходу нѣсть стопы обрѣсти, ниже стези шествія его въ волнахъ; или яко птицы, прелетающія по аеру, ни едино обрѣтается знаменіе пути} и проч.\footnote{Премуд.~5,~6--11 и слѣд.} Тако міра сего утѣхи и веселости преходятъ такъ, какъ бы ихъ и не было. \textit{Преходитъ бо образъ міра сего}\footnote{1~Кор.~7,~31.}. \textit{И міръ преходитъ, и похоть его}\footnote{1~Іоан.~2,~17.}. Наконецъ не тое есть истинное блаженство, которое люди по слѣпому разуму мнятъ и утверждаютъ, но тое, которое словомъ Божіимъ, которое есть правило и зерцало нашего разсужденія и хрістіанскаго житія, поставляется и утверждается. Послушай убо, душа грѣшная, кому оно приписуетъ блаженство. Глаголетъ оно: \textit{блаженъ мужъ, иже не иде на совѣтъ нечестивыхъ, и на пути грѣшныхъ не ста, и на сѣдалищи губителей не сѣдѣ, но въ законѣ Господни воля Его, и въ законѣ Его поучится день и нощь}\footnote{Пс.~1,~1 и 2.}. \textit{Блажени вси надѣющіися нань}\footnote{2,~12.}. \textit{Блажени, ихже оставишася беззаконія}\footnote{31,~1.}. \textit{Блаженъ языкъ, емуже есть Господь Богъ его, люди, яже избра въ наслѣдіе Себѣ}\footnote{32,~12.}. \textit{Блаженъ разумѣваяй на нища и убога: въ день лютъ избавитъ его Господь}\footnote{40,~2.}. \textit{Блаженъ, егоже избралъ еси и пріялъ: вселится во дворѣхъ Твоихъ}\footnote{64,~5.}. \textit{Блажени живущіи въ дому Твоемъ}, Господи, \textit{въ вѣки вѣковъ восхвалятъ Тя}\footnote{83,~5.}. \textit{Блаженъ мужъ, емуже есть заступленіе его у Тебе}, Господи\footnote{Ст.~6.}. \textit{Блаженъ человѣкъ, егоже аще накажеши Господи, и отъ закона Твоего научиши его}\footnote{Пс.~93,~12.}. \textit{Блажени хранящіи судъ, и творящіи правду во всякое время}\footnote{105,~3.}. \textit{Блаженъ мужъ, бояйся Господа, въ заповѣдехъ Его восхощетъ зѣло}\footnote{111,~1.}. \textit{Блажени непорочніи въ пути, ходящіи въ законѣ Господни. Блажени испытающіи свидѣнія Его: всѣмъ сердцемъ взыщутъ Его}\footnote{118,~1 и 2.}. И Хрістосъ, праведный судія, ублажаетъ: \textit{блажени нищіи духомъ: яко тѣхъ есть царствіе небесное. Блажени плачущіи: яко тіи утѣшатся. Блажени кротцыи: яко тіи наслѣдятъ землю. Блажени алчущіи и жаждущіи правды: яко тіи насытятся. Блажени милостивіи: яко тіи помиловани будутъ. Блажени чистіи сердцемъ: яко тіи Бога узрятъ. Блажени миротворцы: яко тіи сынове Божіи нарекутся. Блажени изгнани правды ради: яко тѣхъ есть царствіе небесное. Блажени есте, егда поносятъ вамъ, и ижденутъ, и рекутъ всякъ золъ глаголъ на вы лжуще, Мене ради. Радуйтеся и веселитеся, яко мзда ваша многа на небесѣхъ}\footnote{Матѳ.~5,~3--12.}. Таковыяжъ блаженства и подобныя и на прочіихъ святаго Писанія мѣстахъ видимъ. Видишь убо, что слово Божіе не богатыхъ, славныхъ и въ веселостяхъ вѣка сего находящихся ублажаетъ, но поучающихся въ законѣ Господни, надѣющихся на Господа, нищихъ духомъ, плачущихъ, алчущихъ и жаждущихъ правды, кроткихъ, милостивыхъ, чистыхъ сердцемъ, миротворцевъ, изгнанныхъ правды ради, поношеніе терпящихъ ради имени Хрістова, и проч. Что же убо остается тебѣ дѣлать, какъ, оставивши ложное блаженство, которое сынове вѣка сего полагаютъ въ чести, славѣ, богатствѣ, сласти, искать истиннаго блаженства, которое намъ хрістіанамъ слово Божіе обѣщаетъ? Аще убо оставиши міръ, и вѣрою и любовію пристанешь ко Хрісту, Онъ тебѣ сіе подастъ; безъ вѣры бо невозможно имѣти истиннаго блаженства; а гдѣ вѣра, тамо и любовь, любовь бо отъ вѣры неотлучна. Аще же пребудешь въ любви міра, погубитъ тебе міръ, хотя и ласкаетъ, какъ многихъ погубилъ; или паче князь міра, то"=есть, сатана, который какъ прародителей нашихъ въ раи яблокомъ прельстилъ и погубилъ, отвелъ отъ источника истиннаго блаженства "--- Бога, и вринулъ въ бѣдствіе, такъ и нынѣ благополучіемъ міра сего, какъ краснымъ яблокомъ, прельщаетъ и отводитъ отъ Хріста, у Котораго \textit{единаго} истинное, непоколебимое и вѣчное блаженство, и ведетъ съ собою въ вѣчное неблагополучіе и погибель. Внимай и помни, что сказалъ Хрістосъ: \textit{кая польза человѣку, аще міръ весь пріобрящетъ, душу же свою отщетитъ? или что дастъ человѣкъ измѣну за душу свою? Пріити бо имать Сынъ человѣческій во славѣ Отца Своего со ангелы Своими: и тогда воздастъ комуждо по дѣяніямъ его}\footnote{Мѳ.~16,~26 и 27.}. Тогда не спроситъ Хрістосъ насъ, были ли мы въ мірѣ славны, богаты, честны, имѣли ли мудрость, художество, краснорѣчіе, похвалу, благородіе, честное и почтенное имя, и проч. Нѣтъ, ничего того не будетъ; но спроситъ, како мы въ мірѣ жили. Хрістіанами нарицались, но имѣли ли хрістіанское святое житіе? Бога единаго исповѣдывали, но единому ли Богу работали? Слово Божіе проповѣдуемо слушали, но жили ли по правилу его? Примѣчай и тое, что написано: \textit{яко любы міра сего вражда Богу есть: иже бо восхощетъ другъ быти міру, врагъ Божій бываетъ}\footnote{Іак.~4,~4.}. Міролюбцевъ убо, какъ враговъ Божіихъ, будетъ Хрістосъ судити, и \textit{воздастъ имъ по дѣломъ ихъ}.

\subsection[Глава 3-я. О почитаніи страстей Хрістовыхъ.]{глава третія.\\\bfseries О почитаніи страстей Хрістовыхъ.}

\begin{quotation}\textit{Помыслите таковое пострадавшаго отъ грѣшникъ на Себе прекословіе, да не стужаете, душами своими ослабляеми}\footnote{Евр.~12,~3.}.\end{quotation}

\paragraph*{§\:343.} Все житіе Іисуса Хріста Господа нашего, которое на земли нашего ради спасенія началъ и совершилъ, было всегдашній крестъ. Въ вертепѣ родился и повитый \textit{положенъ былъ въ яслехъ: зане не бѣ имъ мѣста во обители}\footnote{Лук.~2,~7.}. Не было мѣста во обители Тому, Котораго есть небо и земля и исполненіе ея. Тако изволися Ему насъ ради гордыхъ смиритися. Только"=что на свѣтъ показался, и житіе Свое на земли не успѣлъ начать, отъ беззаконнаго Ирода претерпѣлъ гоненіе, и Младенецъ ничего, кромѣ пеленъ, млека матерня, плача и слезъ, не знающій, \textit{искомъ былъ на убіеніе}\footnote{Мѳ.~2,~16.}. Благочестіе бо не успѣетъ появиться, тотчасъ отъ нечестія начинаетъ гоненіе терпѣти. "--- Далѣе, якоже пишется, \textit{бѣ повинуяся има}, то"=есть, матери и мнимому своему отцу\footnote{Лук.~2,~51.}. Повиновался человѣкамъ Тотъ, Которому ангели со страхомъ и трепетомъ повинуются и вся тварь работаетъ. Насъ ради, мене ради и тебе, грѣшниче, повиновался Создатель нашъ созданію Своему; насъ ради, глаголю, которые Богу, Создателю своему, не хотѣли повиноваться. "--- Отъ сатаны, противника Своего, \textit{искушаемъ былъ}, такъ что и на крилѣ церковномъ поставилъ Его, и сказалъ Ему: \textit{аще Сынъ еси Божій, верзися низу}; и на гору высокую возвелъ, и показалъ Ему вся царствія міра и славу ихъ, и сказалъ Ему: \textit{сія вся Тебѣ дамъ, аще падъ поклонишися мнѣ}\footnote{Мѳ.~4,~5,~6,~8 и 9.}. Толикое дерзновеніе и неистовство претерпѣлъ Господь и Создатель всѣхъ отъ злаго духа, которому маніемъ единымъ запретить моглъ. Насъ ради претерпѣлъ, человѣче, да намъ \textit{искушаемымъ поможетъ}\footnote{Евр.~2,~18.}. "--- Странствуя по земли и проповѣдуя Божественное слово Свое, страшно отъ фарисейскихъ языковъ \textit{хулимъ былъ}. Божественныя Его чудеса веельзевулу, князю бѣсовскому, причитали\footnote{Мѳ.~9,~34; 12,~24; Марк.~3,~22; Лук.~11,~15.}. Святое Его и небесное ученіе, аки хулу на Бога, почитали\footnote{Мѳ.~26,~65; Марк.~2,~7; 14,~64; Лук.~5,~21.}. Отрыгали на Него хулы и, аки стрѣлы язвительныя, бросали языками своими: \textit{бѣса имать, самарянинъ, неистовъ есть, ядца и винопійца, другъ мытаремъ и грѣшникомъ}, и проч.\footnote{Іоан.~8,~48 и 52; Марк.~3,~22 и 21; Мѳ.~11,~19; Лук.~7,~34.} Хулы и клеветы претерпѣлъ единъ Благословенный во вѣки, да насъ отъ клеветы діавольскія защититъ. Пожилъ въ нищетѣ, такъ что \textit{не имѣлъ гдѣ главы подклонити}, Имѣяй престолъ небо и землю подножіе\footnote{Мѳ.~8,~20.}. Насъ ради, \textit{богатъ сый, обнищалъ}, да мы нищетою Его обогатимся\footnote{2~Кор.~8,~9.}. Сюды надлежитъ, что имѣющія быть и въ свое время ему приключитися жесточайшія и умомъ человѣческимъ непостижимыя страданія ясно предвидѣлъ. Откуду апостоламъ и предсказывалъ о тѣхъ: \textit{оттолѣ начатъ Іисусъ сказовати ученикомъ Своимъ, яко подобаетъ Ему ити во Іерусалимъ, и много пострадати отъ старецъ и архіерей и книжникъ, и убіену быти, и въ третій день востати}\footnote{Мѳ.~16,~21.}. Сюды паки надлежатъ всегдашніе труды, подъятые въ прехожденіи съ мѣста на мѣсто, отъ града во градъ, отъ веси въ весь, и въ пренесеніи и проповѣданіи святаго Евангелія которое отъ нѣдръ Отца Своего небеснаго на землю принесъ, и прочая. Тако трудился и потѣлъ Сынъ Божій ради бѣдной души человѣческой, которую сатана подъ свою власть похитилъ было. Наконецъ проданъ и преданъ отъ Своего ученика, отъ прочіихъ учениковъ оставленъ, сужденъ и осужденъ, заушенъ, оплеванъ, поруганъ, уязвленъ, терніемъ вѣнчанъ, какъ злодѣй обнаженъ, и, аки \textit{проклятый}, сый Благословенный на древѣ въ позоръ всѣмъ повѣшенъ\footnote{Гал.~3,~13.}. Повѣшенному посмѣвалися и ругалися мимоходящіе предъ очесами Его, и о одеждѣ Его метали жребія; оцтомъ и желчію въ жаждѣ напоили. Тако умученный и посмѣянный Сынъ Божій на древѣ крестномъ окончалъ житіе Свое. Но и по спасительной смерти \textit{въ ребро прободенъ} былъ\footnote{Іоан.~19,~34.}. \textit{Льстецемъ} называемъ былъ\footnote{Мѳ.~27,~63.}. Воскресеніе Его покрыти тщились, и стражамъ руки мздою исполнили, дабы не говорили того, что видѣли\footnote{28,~12--15.}. Тако Хрістосъ чрезъ все житіе Свое, при смерти и по смерти, насъ ради грѣшныхъ терпѣлъ.

\paragraph*{§\:344.} Аще бы кто впалъ въ какую знатную бѣду, напр. или въ полонъ непріятелю попалъ, или въ смертную болѣзнь, или въ смертную вину, за которую слѣдовало ему по закону умереть, впалъ; а сыскался бы такой добрый человѣкъ, который отъ той бѣды его избавилъ своимъ тщаніемъ: неотмѣнно бы избавленный въ незабвенной памяти знатное тое избавленія своего дѣло содержалъ, и усердно бы почиталъ избавителя своего; иначе бы крайне неблагодаренъ и безчувственъ былъ. Видѣли мы, православный хрістіанине, въ какую бѣду за грѣхи наши впали мы, и какъ, отъ Кого отъ той избавилися: \textit{яко не истлѣннымъ сребромъ или златомъ избавихомся, но честною кровію, яко Агнца непорочна и пречиста Хріста}\footnote{1~Петр.~1,~18 и 19.}. \textit{Хрістосъ ны искупилъ отъ клятвы законныя, бывъ по насъ клятва. Писано бо есть: проклятъ всякъ висяй на древѣ. Да во языцѣхъ благословеніе Авраамле будетъ о Хрістѣ Іисусѣ}, и проч.\footnote{Гал.~3,~13,~14 и слѣд.} Разсуди же, какъ великое сіе избавленія дѣло и Избавителя почитать намъ должно.

\paragraph*{§\:345.} \textit{Вѣры свойство} есть благодѣянія и заслуги Хрістовы, всему міру показанныя, себѣ присвоять. Откуду всякъ вѣрный, ты и я, съ вѣрнымъ Павломъ отъ сердца исповѣдовати долженъ: \textit{вѣрую во Хріста Сына Божія, возлюбившаго мене, и предавшаго Себе по мнѣ}\footnote{2,~30.}. Вѣрую, что какъ ради всѣхъ, такъ и ради мене непотребнаго родился отъ Дѣвы, плотію на землѣ пожилъ, хуленія, безчестія, страданіе, распятіе и смерть крестную претерпѣлъ; и тако какъ всѣмъ, такъ и мнѣ недостойному заслужилъ у Бога милость, отпущеніе грѣховъ, благодать, благословеніе, усыновленіе, наслѣдіе вѣчнаго живота и блаженства, чего я своею силою никакъ не моглъ и не могу заслужить и получить. Онъ смирился, чтобы мене смиреннаго вознести; обнищалъ, чтобы мене нищаго обогатить; хулы и клеветы претерпѣлъ, чтобы мене отъ клеветы діавольскія защитить; связанъ былъ, чтобы мене отъ вѣчныхъ избавить узъ; судимъ былъ и осужденъ, чтобы мене отъ осужденія свободить; язвы и раны пріялъ, чтобы мене уязвленнаго исцѣлить; поруганіе и безчестіе претерпѣлъ, чтобы мене отъ сатаны поруганнаго почтить; умеръ, чтобы умершаго оживить, и тако какъ всѣмъ, такъ и мнѣ, вѣрующему во имя Его, учинился Избавитель, Спаситель, Ходатай и Виновникъ вѣчнаго блаженства. А когда всякому вѣрному должно великое сіе Сына Божія дѣло себѣ привлекать и присвоивать, то и почитать тое должно такъ, какъ бы оно ради его единаго учинилося. Откуду святый Златоустъ глаголетъ: «Всякъ изъ насъ праведно толико долженъ Хрісту благодарить, колико аще бы и тебе ради единаго пришелъ. Ибо не отреклся бы и о единомъ толикое смотрѣніе показать, всякаго бо человѣка толикою любви мѣрою любить, еликою вселенную всю»\footnote{Бес. на посл. къ Гал.~2,~20.}. Ради тебе и мене Хрістосъ пришелъ въ міръ, убо тебѣ и мнѣ Его, такъ великаго Благодѣтеля, яко своего \textit{собственнаго}, благодарнымъ, смиреннымъ, любовнымъ, послушливымъ и терпѣливымъ сердцемъ срѣтать и почитать должно. Отсюду видишь, хрістіанине, что 1)~кто такую вѣру имѣетъ, какъ и всякому хрістіанину должно имѣти, тотъ не захощетъ противу совѣсти грѣшить. 2)~Имѣетъ ко Хрісту любовь и тщится заповѣди Его исполнять. 3)~Житіе порочное и безплодное хрістіанъ доказательствомъ есть невѣрія ихъ, хотя и имѣютъ на устахъ вѣру. Аще бо человѣка"=благодѣтеля, который временное намъ добро сдѣлалъ, совѣсть убѣждаетъ насъ любить; кольми паче вѣра, когда ее имѣемъ, убѣдитъ насъ любить Хріста, Который намъ вѣчное и умомъ нашимъ непонятное блаженство пріобрѣлъ не инымъ чѣмъ, какъ вольнымъ Своимъ страданіемъ и смертію. А гдѣ любовь ко Хрісту есть, тамо и послушаніе Хрісту неотлучно есть; гдѣ послушаніе, тамо тщательное исполненіе заповѣдей Его; гдѣ заповѣдей Его исполненіе, тамо твореніе добрыхъ дѣлъ, заповѣди бо Его добрымъ дѣламъ научаютъ. И тако видишь, что вѣра истинная есть матерь добрымъ дѣламъ, якоже невѣріе есть корень злыхъ.

\paragraph*{§\:346.} \textit{Почитается} спасительное \textit{Хрістово пришествіе и страданіе}: 1)~Вѣрою, когда почитаемъ Его за \textit{единаго} Избавителя, Спасителя и Защитника нашего, намъ отъ небеснаго Его Отца на избавленіе наше посланнаго, и инаго посредствія къ полученію вѣчнаго живота, кромѣ Его, не признаемъ, якоже о Немъ Отецъ небесный свидѣтельствуетъ намъ: \textit{Сей есть Сынъ Мой возлюбленный, о Немже благоволихъ: Того послушайте}\footnote{Мѳ.~17,~5.}. И Апостолъ Его глаголетъ: \textit{разумѣйте посланника и святителя исповѣданія вашего Іисуса Хріста}\footnote{Евр.~3,~1.}. "--- 2)~Всегдашнею памятію, когда съ любовію и благодарностію поминаемъ сіе великое Его дѣло, насъ ради начатое и совершенное. Разсуди всякъ: ради кого родился Хрістосъ, на земли пожилъ, пострадалъ и умеръ? Ради мене и тебе, грѣшниче, ради моей и твоей погибшей души, ради моего и твоего спасенія толико трудился и подвизался Сынъ Божій! Все сіе во всегдашней и благодарной памяти содержать должно намъ. Всякое бо благодѣяніе, показанное намъ, требуетъ отъ насъ, дабы мы всегда тое памятовали, "--- кольми паче великое сіе искупленія дѣло. Откуду Самъ Спаситель глаголетъ: \textit{сіе творите въ Мое воспоминаніе}\footnote{Лук.~22,~19.}. Отсюду церковь святая знатнѣйшія спасительнаго Его смотрѣнія таинства представляетъ намъ къ воспоминанію и празднованію ихъ. Сего ради и на дскахъ и на прочіихъ матеріалахъ изображаемъ преславная дѣла Его, какъ"=то: Благовѣщеніе, Рождество Хрістово, Богоявленіе, Преображеніе, Распятіе, Страданіе, Погребеніе, Воскресеніе и прочая, да, на изображеніе взирая, поминаемъ Самаго Того, Который насъ ради сія, на иконѣ написанная, сотворилъ, и поминая усердно благодаримъ сотворшему. Изображеніе бо сіе есть намъ вмѣсто всегдашней памяти и очевидной исторіи, и незнающимъ чтенія вмѣсто всегдашняго чтенія, и тако, что въ священной исторіи читаемъ, или изъ нея слышимъ, тое изображенное на иконѣ ясно видимъ. "--- 3)~Благодарнымъ хваленіемъ, пѣніемъ и славословіемъ, когда чудесныя и спасительныя дѣла Его поминаемъ, и поемъ благость Его. Откуду святое Божіе слово повелѣваетъ намъ пѣти великое сіе дѣло: \textit{исполняйтеся Духомъ, глаголюще себѣ во псалмѣхъ и пѣніихъ и пѣснехъ духовныхъ, воспѣвающе и поюще въ сердцахъ вашихъ Господеви}\footnote{Еф.~5,~19.}. И паки: \textit{слово Хрістово да вселяется въ васъ богатно, во всякой премудрости учаще и вразумляюще себѣ самѣхъ, во псалмѣхъ и пѣніихъ и пѣснехъ духовныхъ во благодати поюще въ сердцахъ вашихъ Господеви}\footnote{Кол.~3,~16.}. И пѣснь новую пѣти повелѣваетъ: \textit{воспойте Господеви пѣснь нову: яко дивно сотвори Господь}\footnote{Пс.~97,~1.}. Новому бо чудеси новая пѣснь прилична. \textit{Новое чудо} содѣлалося въ насъ, каковаго древніе годы не видѣли. Новое чудо, яко Богъ во плоти явися, и Господь славы пострадалъ за непотребнаго раба Своего. Насъ ради грѣшныхъ тое \textit{новое чудо} сотворилося: намъ новую пѣснь пѣти Господеви подобаетъ, пѣти же въ новости сердца и духа, то"=есть, въ отложеніи ветхаго человѣка и облеченіи въ \textit{новаго}. Нашему благополучію срадуяся, ангели святые воспѣли Богу, помиловавшему насъ: \textit{слава въ вышнихъ Богу, и на земли миръ, въ человѣцѣхъ благоволеніе}\footnote{Лук.~2,~14.}. Предвидѣли сіе великое чудо пророцы святіи, и воспѣли со удивленіемъ. Единъ воспѣлъ: \textit{се Дѣва во чревѣ пріиметъ и родитъ Сына, и нарекутъ имя Ему Еммануилъ}\footnote{Ис.~7,~14.}. Другій, ужаснувшися о семъ, возопилъ: \textit{Господи! услышахъ слухъ Твой, и убояхся. Господи! разумѣхъ дѣла Твоя, и ужасохся. Посредѣ двою животну познанъ будеши}, и проч.\footnote{Аввак.~3,~1,~2 и слѣд.} \textit{Видѣ Авраамъ и возрадовася}\footnote{Іоан.~8,~56.}. Они предвидѣли тое, что мы уже видимъ. Они видѣли будущее, мы видимъ совершившееся; они ожидали, мы дождалися; они подъ мракомъ и тѣнію покрытое, мы откровенное видимъ. Сего ради праведно и прилично намъ пѣти \textit{новую пѣснь: пріидите убо, возрадуемся Господеви, воскликнемъ Богу Спасителю нашему; предваримъ лице Его во исповѣданіи, и во псалмѣхъ воскликнемъ Ему}\footnote{Пс.~94,~1 и 2.}. Къ сему ликованію созываетъ насъ матерь наша, святая церковь: «пріидите, возрадуемся Господеви, настоящую тайну сказующе», и проч.\footnote{На вечерни праздника Рождества Хрістова.} \textit{Благословенъ Господь Богъ Израилевъ! яко посѣти и сотвори избавленіе людемъ Своимъ}\footnote{Лук.~1,~68.}. \textit{Помощникъ и покровитель бысть мнѣ во спасеніе: сей мой Богъ, и прославлю Его; Богъ отца моего, и вознесу Его}\footnote{Исх.~15,~2.}. "--- 4)~Устнымъ исповѣданіемъ, когда вѣру нашу, которую въ сердцѣ имѣемъ, устами исповѣдаемъ небоязненно, гдѣ надлежитъ; не молчимъ, гдѣ честь и слава имени Его требуетъ; не стыдимся отъ Дѣвы рожденнаго и на крестѣ распятаго Богомъ нашимъ нарицать; хотя сіе \textit{Іудеемъ соблазнъ, Еллиномъ же безуміе} показуется, но намъ Хрістосъ \textit{есть Божія сила и Божія премудрость}\footnote{1~Кор.~1,~23,~24 и 30.}. Тако предначинаетъ намъ святая церковь: «аще и ятъ былъ еси, Хрісте, отъ беззаконныхъ мужей, но Ты ми еси Богъ, и не постыждуся; біенъ былъ еси по плещема, и не отметаюся» и проч. Тако Павелъ святый не токмо не стыдится узникомъ Хрістовымъ быть, но и не хощетъ хвалитися, токмо о крестѣ Хрістовомъ: \textit{мнѣ да не будетъ хвалитися, токмо о крестѣ Господа нашего Іисуса Хріста}\footnote{Гал.~6,~14.}. И къ Тимоѳею пишетъ: \textit{не постыдися свидѣтельствомъ Господа нашего Іисуса Хріста}\footnote{2~Тим.~1,~8.}. Тако исповѣдники и мученики святые не токмо не стыдилися предъ мучителями и врагами истины распятаго исповѣдовать Бога своего, но и раны и смерть за славу имени Его претерпѣть не отреклися. Тако и намъ не должно истины молчать, гдѣ потребно, но небоязненнымъ сердцемъ исповѣдовать, хотя бы не токмо чести и имѣнія, но и живота лишиться слѣдовало отъ враговъ истины; ибо сего вѣра наша отъ насъ требуетъ. Преславно бо намъ исповѣдати Того и хвалитися о Томъ, Котораго имя свято и страшно, намъ же сладко есть. Аще бо и зачался въ дѣвической утробѣ, но \textit{отъ Духа свята}\footnote{Мѳ.~1,~18.}, и есть собезначальный Отцу и Святому Духу. Родился въ лѣто отъ Дѣвы Матери\footnote{Лук.~2,~7.}, но \textit{есть надъ всѣми Богъ благословенъ во вѣки}\footnote{Римл.~9,~5.}. Пеленами повился, яко младенецъ\footnote{Лук.~2,~7.}, но Тойжде есть \textit{одѣяйся свѣтомъ яко ризою}\footnote{Пс.~103,~2.}. Во храмъ Младенецъ принеслся руками матерними\footnote{Лук.~2,~22.}, но старца Симеона отъ временнаго къ вѣчному животу отпустилъ\footnote{ст.~29.}. Отъ Ирода гонимъ былъ, но волхвовъ на поклоненіе Себѣ привелъ\footnote{Мѳ.~2"~я.}. Во Іорданѣ крестился, но крещаемому Отецъ съ небеси свидѣтельствовалъ: \textit{Сей есть Сынъ Мой возлюбленный, о Немже благоволихъ}, и Духъ Святый, яко голубь, сходя на крещаемаго, утверждалъ истину\footnote{3,~13--17.}. Искушаемъ былъ отъ діавола, но посрамилъ искусителя\footnote{4,~3--11; Лук.~4,~2--13.}. На земли между человѣками жилъ, но нѣдръ Отеческихъ не отлученъ былъ\footnote{Іоан.~1,~18.}. Въ рабіемъ зракѣ обращался\footnote{Филип.~2,~7.}, но, яко Господь, повелѣвалъ вѣтрамъ и морю, и послушали Его; повелѣвалъ демонамъ, и исходили; повелѣвалъ мертвымъ, и воставали; повелѣвалъ слѣпымъ, и прозирали; глухимъ, и слухъ воспріимали; нѣмымъ, и глаголали; прокаженнымъ, и очищалися\footnote{Мѳ.~15,~30 и 31; 17,~17; 9,~25 и проч.}. Алкалъ и жаждалъ, но многія тысящи пятію хлѣбами питалъ\footnote{17,~17--21.}. Отъ завистливыхъ и злобныхъ хулимъ былъ, но сердечныя помышленія ихъ видѣлъ\footnote{12,~25.}. Со грѣшниками ялъ, но грѣхи ихъ кающихся очищалъ\footnote{Лук.~15,~2--7.}. Отъ неблагодарнаго ученика на сребреникахъ проданъ и преданъ былъ, но насъ, подъ грѣхъ проданныхъ, искупилъ. Скорбѣлъ и тужилъ, но отъ насъ вѣчную скорбь и тугу отнялъ. Отъ беззаконныхъ связанъ былъ, но врага нашего діавола связалъ, и наши узы растерзалъ. Судищу неправедному предстоялъ, но насъ отъ вѣчнаго суда избавилъ. Одежды совлеченъ и обнаженъ былъ, но насъ одеждою правды Своея одѣялъ. Посмѣянъ, обезчещенъ, поруганъ, оплеванъ былъ, но насъ отъ вѣчнаго срама, безчестія и поруганія свободилъ. Терніемъ вѣнчанъ былъ, но намъ вѣчныя славы вѣнецъ устроилъ. На древѣ распялся и умеръ, но солнце помрачилъ, землю поколебалъ, гробы отверглъ, мертвыя оживилъ, сотника и сущихъ съ нимъ устрашилъ\footnote{Мѳ.~27,~45,~51--54.}, разбойника въ рай ввелъ\footnote{Лук.~23,~43.}, и Своею смертію нашу смерть умертвилъ. Во гробѣ погребенъ былъ, яко мертвый; но Своею силою восталъ, яко безсмертный, и намъ умирающимъ востанія надежду подалъ. На земли пожилъ, но вознеслся на небо, и сѣлъ одесную Бога Отца, и паки пріидетъ судити живымъ и мертвымъ, и воздати всѣмъ по дѣломъ ихъ, "--- праведныхъ ввести въ вѣчный животъ и царствіе небесное, грѣшныхъ вѣчной предати мукѣ\footnote{Мѳ.~25,~31--46.}. Тако въ распятаго вѣрующе, не отрицаемся Его исповѣдывать нашего Господа и Бога Который, насъ ради человѣкъ и нашего ради спасенія, бысть человѣкъ, и тако Самъ собою устроилъ намъ путь спасенія. Въ томъ наша премудрость, похвала, слава, животъ, утѣшеніе, веселіе, надежда и вѣчное блаженство. "--- 5)~Понеже наша вѣра и слово Божіе увѣряетъ насъ, что Хрістосъ ради грѣховъ нашихъ пострадалъ, якоже писано есть: \textit{Той язвенъ бысть за грѣхи наша, и мученъ бысть за беззаконія наша}\footnote{Исх.~53,~5.}; то нѣтъ большаго почитанія Сыну Божію, пострадавшему за грѣхи наша, какъ когда ради любве Его ненавидимъ грѣхи, удаляемся отъ нихъ и чѣмъ болѣе ненавидимъ ихъ и убѣгаемъ отъ нихъ, тѣмъ болѣе любимъ и почитаемъ Хріста, въ міръ пришедшаго и пострадавшаго, такъ что безъ сего и почитаніе истинное быть не можетъ, но притворное и лицемѣрное. Идѣже бо грѣхи не оставляются, но дѣлаются, тамо нѣтъ истиннаго покаянія; каяться бо и не оставлять грѣхи суть противна себѣ. Гдѣ нѣтъ истиннаго покаянія, тамо нѣтъ вѣры во Хріста; Хрістосъ бо безъ вѣры быть не можетъ. Гдѣ нѣтъ Хріста, тамо Хрістосъ не почитается, но отвергается и презирается. Внимай сему, грѣхолюбивая душа! Не почитаешь Хріста, тебе такъ высоко почтившаго, когда любишь грѣхъ; любишь же, понеже не оставляешь его: кто бо что любитъ, тотъ того и держится. Услаждаешися тѣмъ, за что Хрістосъ горестную мученія чашу испилъ. Такое ты воздаешь почтеніе и благодареніе Хрісту, пришедшему тебе взыскати и спасти! И дотолѣ такое презрѣніе и неблагодарность Ему будешь показывать, хотя устами и почитаешь Его, доколѣ не оставишь грѣха, и истинно не покаешися. "--- 6)~Наконецъ, почтимъ Хріста, пострадавшаго за насъ, когда, ненавидя грѣхъ, будемъ послѣдовать святому Его слову и примѣру непорочнаго житія Его, якоже глаголетъ: \textit{аще кто Мнѣ служитъ, Мнѣ да послѣдствуетъ}\footnote{Іоан.~12,~26.}; почитать бо Хріста и служить Хрісту едино есть. Научалъ Онъ насъ словомъ и примѣромъ добродѣтельному житію, якоже читаемъ въ святомъ Его Евангеліи. Аще убо хощемъ Его сердечно почитать, должно намъ Его въ томъ слушать. Безъ послушанія бо почитать Его не можемъ, какъ самъ сіе можешь признать. Какое тое почитаніе господину, когда рабъ его не слушаетъ, отцу, когда сынъ ему не повинуется? Не почитаніе сіе есть, но презрѣніе явное и безчестіе. Тако и Хрісту лицемѣрное, а не истинное показываемъ, въ самой же вещи презрѣніе, когда глаголемъ Ему устами: \textit{Господи, Господи}, а не дѣлаемъ того, что Онъ повелѣваетъ, якоже Самъ глаголетъ: \textit{что Мя зовете Господи, Господи, и не творите, яже глаголю}\footnote{Лук.~6,~46.}? Не отъ слова убо единаго, но и отъ дѣла истинное почитаніе показуется. Отъ дѣлъ нашихъ показать намъ, хрістіанине, должно вѣру нашу къ Нему, якоже Апостолъ Его требуетъ отъ насъ: \textit{покажи ми вѣру твою отъ дѣлъ твоихъ}\footnote{Іак.~2,~18.}. Вѣра наша требуетъ отъ насъ любви, милости, милосердія, смиренія, кротости, терпѣнія и прочіихъ хрістіанскихъ добродѣтелей, которымъ Онъ насъ и словомъ и дѣломъ училъ. Сами убо почтимъ почетшаго насъ Хріста и Господа нашего. Велико ли намъ подать просящему ради Его, Который намъ, что ни имѣемъ (все бо Его есть, что ни имѣемъ, кромѣ грѣховъ), и Себе Самого подалъ? Велико ли напитать алчущаго, напоить жаждущаго, одѣть нагаго, ввести въ домъ страннаго, посѣтить болящаго и въ темницѣ заключеннаго, ради Того, Который насъ ради алкалъ, жаждалъ, обнаженъ былъ, странствовалъ, недуги наши понеслъ, болѣзновалъ и связанъ былъ? Велико ли намъ отпустить ближнему согрѣшенія ради Его, Который за наши грѣхи пострадалъ? Тяжко ли намъ терпѣть ради Его, когда Онъ ради насъ ужасныя мученія претерпѣлъ? Не постыдился и не отреклся Онъ, Господь нашъ и Богъ (о благости Твоея, Іисусе Боже нашъ!), насъ ради рабовъ Своихъ непотребныхъ, посмѣянъ, поруганъ, оплеванъ, уязвленъ, заушенъ быти, распятися и умрети: намъ ли, рабамъ Его неключимымъ, \textit{стыдитися Его и словесъ Его въ родѣ семъ прелюбодѣйнѣмъ и грѣшнѣмъ}\footnote{Марк.~8,~38.}, и не терпѣть, что намъ приключится противное, и не послѣдовать тому, чему Онъ словомъ и примѣромъ Своимъ училъ? Воистину ветхій Адамъ царствуетъ надъ нами, а не Хрістосъ. Воистину не Хрістовы есмы, когда не послѣдуемъ слову Его и житію. \textit{Аще кто Духа Хрістова не имать, сей нѣсть Его}\footnote{Римл.~8,~9.}. Ибо кто Духъ Хрістовъ имѣетъ, тотъ мыслитъ тое, что Хрістосъ; поучается въ томъ, что Хрісту угодно; ищетъ и хощетъ того, что хощетъ Хрістосъ; тщится дѣлать тое, чего требуетъ Хрістосъ. Аще убо хощемъ Хрістовы быть, и Хріста, яко Царя своего, почитать, должны едино съ Нимъ мудрствовать, хотѣть и дѣлать, что хощетъ и повелѣваетъ Онъ. Истинное бо почитаніе не въ наружности единой, но и во внутренности, не въ словахъ только, исповѣданіи и церемоніи, но и въ сердцѣ, усердіи, любви, благодарности и послушаніи состоитъ. Господь нашъ и Богъ есть: почтимъ Его страхомъ и послушаніемъ. Любитель вашъ есть искреннѣйшій: почтимъ Его любовію нашею. Благодѣтель нашъ есть: почтимъ Его благодарностію. Искупитель нашъ есть, цѣною бо крови Своея искупилъ насъ Себѣ: почтимъ Его усердною работою. Учитель нашъ есть: почтимъ Его охотнымъ исполненіемъ ученія Его. Пастырь нашъ есть: почтимъ Его слушаніемъ святаго гласа Его\footnote{Іоан.~10,~27.}. Вождь нашъ есть къ небеси вѣрнѣйшій: почтимъ Его вѣрнымъ и любовнымъ послѣдованіемъ. Тако почтимъ, хрістіане, Хріста Царя нашего. Онъ насъ возлюбилъ недостойныхъ: возлюбимъ и мы Его, достойнаго любви, Который есть утѣшеніе и радость ангелъ и святыхъ душъ. Почтилъ Онъ насъ недостойныхъ: почтимъ и мы Его, достойнаго всякаго почитанія, Которому со страхомъ и любовію покланяются ангели и всякое колѣно. Поискалъ Онъ насъ заблуждшихъ: поищемъ и мы Его вѣрою и любовію. Не пощадѣлъ Онъ Себе ради насъ: не пощадимъ и мы себе ради Его. Терпѣлъ Онъ все ради насъ: потерпимъ и мы все ради Его, да съ Нимъ и прославимся во царствіи Его. "--- Господи! настави насъ на путь сей: \textit{Ты бо еси путь и истина и животъ}\footnote{Мѳ.~14,~6.}.

\paragraph*{§\:347.} Всѣ"=ли"=де хрістіане почитаютъ пришествіе и страданія Хрістовы? \textit{Отвѣтъ}. Хрістіане, которые возлюбили Евангеліе Хрістово и творятъ покаяніе и плоды достойные покаянія, почитаютъ; но хрістіане неисправные, напримѣръ, блудники, прелюбодѣи, піяницы, хищники, тати, мздоимцы, льстецы, лукавцы, ненавистники, злобные, клеветники, злорѣчивые и прочіе симъ подобные, не почитаютъ. Ради чего? Понеже 1)~истинное почитаніе безъ вѣры быть не можетъ: житіе беззаконное вѣрѣ противно, и знакъ есть невѣрія. 2)~\textit{Никтоже можетъ двѣма господинома работати}, по словеси Хрістову\footnote{Мѳ.~6,~24.}. Они работаютъ своимъ страстямъ и почитаютъ ихъ, яко, что хотятъ страсти ихъ, тое они исполняютъ. Убо почитая свои страсти, Хрістовыхъ страстей не почитаютъ, и, своимъ прихотямъ работая, Хрісту работать не могутъ; почитать бо Хріста и работать Хрісту едино есть. Аще бо кто почитаетъ Хріста, тотъ неотмѣнно работаетъ Ему. 3)~Беззаконное житіе противно и враждебно есть кресту и страданію Хрістову, ибо Хрістосъ пострадалъ и умеръ за грѣхи наша. Хрістіане беззаконнующіе пренебрегаютъ сію высокую Его благодать и человѣколюбіе, и нераскаяннымъ своимъ житіемъ показуютъ, какъ бы туне умеръ Хрістосъ. Какъ бо тѣмъ, которые престаютъ отъ грѣховъ и каются, страданіе и смерть Хрістова не тщетна бываетъ, но получаетъ плодъ свой, то"=есть, отпущеніе грѣховъ, оправданіе, и вѣчный ходатайствуетъ животъ: тако некающимся, но пребывающимъ во грѣхахъ, никакой пользы не подаетъ, и потому ради нераскаяннаго ихъ житія имъ тщетна бываетъ. И кровь Хрістова, ради всѣхъ, въ которомъ числѣ и они заключаются, изліянная, ради ихъ какъ бы туне изліянна есть; ибо плода своего, который есть обращеніе, покаяніе, новая жизнь и отпущеніе грѣховъ и спасеніе, въ нихъ лишается. Хотя бо Хрістосъ \textit{за всѣхъ умре}, по ученію Апостола\footnote{2~Кор.~5,~15.}, но смерть Хрістова тѣхъ только пользуетъ, которые каются о грѣхахъ и вѣруютъ въ Него, а въ нераскаянныхъ спасительнаго своего плода не получаетъ. И сіе бываетъ не отъ стороны Хріста, Который \textit{всѣмъ хощетъ спастися, и въ разумъ истины пріити}\footnote{1~Тим.~2,~4.}, и \textit{за всѣхъ умре}; но отъ стороны нехотящихъ каятися и пользоватися смертію Хрістовою. Якоже бо лѣкарство, хотя и изрядное и преполезное есть, не всѣхъ пользуетъ, но тѣхъ только, которые пристойнымъ образомъ воспріемлютъ его: тако духовное и животворящее врачевство, смерть Хрістова, тѣ только души врачуетъ и оживляетъ, которыя тое благодарнымъ сердцемъ пріемлютъ, престаютъ отъ грѣховъ, каются и вѣруютъ во Хріста, исцѣляющаго недуги вѣрныхъ Своихъ, а нераскаяннымъ ничего не пользуетъ. И тако кровь Хріста Сына Божія, ради всѣхъ изліянную, тщетну въ себѣ показуютъ, когда, не имѣя истиннаго покаянія, лишаютъ себе того, ради чего изліянна есть. «Благодать бо, аще и благодать есть, хотящихъ спасаетъ, а не нехотящихъ», поучаетъ святый Златоустъ\footnote{Бес.~18"~я на посл. къ Римл.}. Откуду Павелъ святый, теплѣйшій любитель и учитель Хрістовъ, плачетъ о таковыхъ, и называетъ врагами креста Хрістова: \textit{мнози}, глаголя, \textit{ходятъ, ихже многажды глаголахъ вамъ, нынѣ же плачя глаголю, враги креста Хрістова; имже кончина погибель, имже богъ чрево, и слава въ студѣ ихъ, иже земная мудрствуютъ}\footnote{Филип.~3,~18 и 19.}. \textit{Враги креста Хрістова} суть, ибо противятся кресту Хрістову. Какъ? Крестъ бо Хрістовъ учить тѣснымъ путемъ итить: они идутъ пространнымъ. Крестъ учитъ смиряться: они гордятся. Крестъ учитъ прихоти свои отсѣкать: они совершаютъ. Крестъ учитъ терпѣть: они гнѣваются и злобствуютъ. Крестъ учитъ плоть со страстьми и похотьми распинать: они плотская мудрствуютъ. Крестъ учитъ грѣху умирать, а Богу жить: они себѣ и грѣху живутъ. Крестъ учитъ небесныхъ искать: они земная мудрствуютъ. \textit{Хрістосъ бо за всѣхъ умре, да живущіи не ктому себѣ живутъ, но умершему за нихъ и воскресшему}\footnote{2~Кор.~5,~15.}. Многіе"=де отъ нихъ образъ креста Хрістова и образъ Хрістовъ окладываютъ сребромъ и златомъ? "--- \textit{Отвѣтъ}. Окладываютъ симъ веществомъ образъ древа крестнаго, на которомъ Хрістосъ распятъ былъ, и образъ Хрістовъ, но Самаго Хріста не почитаютъ непослушаніемъ, непокореніемъ и противленіемъ, яко противятся Слову Его; не почитаютъ страданія Хрістова, которое почитается не веществомъ, златомъ, сребромъ и прочіимъ, но сердечною вѣрою, любовію, благодарностію и терпѣніемъ. "--- Многіе"=де отъ нихъ ходятъ въ церковь, и поютъ Хріста и страданія Его? "--- Поютъ устами, но сердцемъ отвращаются отъ Него, и обращаются къ міру, якоже о таковыхъ глаголетъ Хрістосъ: \textit{приближаются Мнѣ людіе сіи усты своими, и устнами чтутъ Мя: сердце же ихъ далече отстоитъ отъ Мене}\footnote{Мѳ.~15,~8.}. Внѣшнее бо исповѣданіе и почитаніе безъ внутренняго есть лицемѣріе, какъ выше сказано и ниже скажется. "--- Почему"=де сіи люди дѣлаютъ такъ безстрашно и беззаконно? Потому, что слову Божію, которое учитъ насъ Бога и Хріста Сына Божія почитать не наружностію единою, но сердечною вѣрою, страхомъ и любовію, не внимаютъ, и такъ дѣлаютъ тое, что прихотямъ своимъ угодно. "--- Чтожъ"=де имъ должно учинить, чтобы истинно могли почитать Хріста, распятаго за всѣхъ? "--- Должно оставить прежнее беззаконное и безбожное житіе, и обратиться ко Хрісту, и молить Его усердно, чтобы ихъ принялъ въ общеніе вѣрныхъ Своихъ. Хрістосъ же глаголетъ: \textit{грядущаго ко мнѣ не изжену вонъ}\footnote{Іоан.~6,~37.}. Едины бо вѣрные рабы Хрістовы почитаютъ Хріста, ибо безъ истинныя и живыя вѣры никто не можетъ почитати Его.

\subsection[Глава 4-я. О должной Хрісту работѣ.]{глава четвертая.\\\bfseries О должной Хрісту работѣ.}

\begin{quotation}\textit{Работайте Господеви со страхомъ, и радуйтеся Ему съ трепетомъ}\footnote{Пс.~2,~11.}.\end{quotation}

\paragraph*{§\:348.} Какъ работа, такъ и свобода двоякая: тѣлесная и духовная. Тѣлесная работа есть, когда имѣетъ надъ собою господина человѣка, которому повинуется, служитъ и приказы его исполняетъ, почему и \textit{рабъ его} называется. \textit{Духовная работа} есть, когда человѣкъ работаетъ грѣху, и грѣхомъ діаволу. Грѣхъ бо діавольское дѣло есть, и потому кто діавольское дѣло творитъ, діаволу работаетъ: \textit{творяй} бо \textit{грѣхъ отъ діавола есть, яко исперва діаволъ согрѣшаетъ}\footnote{1~Іоан.~3,~8.}; \textit{имже бо кто побѣжденъ бываетъ, сему и работенъ есть}\footnote{2~Петр.~2,~19.}. О сей работѣ глаголетъ Хрістосъ: \textit{всякъ, творяй грѣхъ, рабъ есть грѣха}\footnote{Іоан.~8,~34.}. Къ сей работѣ надлежатъ сребролюбцы, хищники, татіе, блудники, прелюбодѣи, піяницы, льстецы, лживые, лукавцы, мздоимцы, злобные и прочіе симъ подобные. Отъ сей работы не выключаются и свободные тѣломъ господа, которые какими нибудь страстьми плѣнены суть, и попущаютъ надъ собою грѣху царствовать. Отъ таковой работы предостерегаетъ Апостолъ хрістіанъ: \textit{да не царствуетъ грѣхъ въ мертвеннѣмъ вашемъ тѣлѣ, во еже послушати его въ похотехъ его}\footnote{Римл.~6,~12.}. Отъ сего видно, что есть тѣлесная и духовная свобода. \textit{Тѣлесная свобода} есть, когда человѣкъ другому человѣку не работаетъ, власти его не подлежитъ; таковые суть цари и князи, власти земной не подлежащіе. Такожде тѣлесная свобода есть, когда человѣкъ тѣломъ свободенъ есть, не связанъ, не скованъ, не заключенъ въ темницѣ, и проч. Духовная свобода есть, когда человѣкъ отъ грѣха и діавольскія власти благодатію Божіею освобожденъ, не попущаетъ грѣху и діаволу надъ собою царствовать, противится страстямъ и похотямъ своимъ, плоть духу покоряетъ, единому Богу свободныхъ духомъ служитъ. Таковый можетъ быть тѣломъ рабъ человѣку, связанъ, окованъ, въ темницѣ, во узахъ, въ плѣненіи, но духомъ вездѣ и всегда свободенъ есть. Духа бо поработить и связать никто не можетъ. Тако мученики святые тѣломъ окованы, связаны и въ темницахъ заключены были, но духомъ были свободни. Тако и нынѣ благочестивые рабы господамъ плотскимъ тѣломъ работаютъ, но духомъ свободни суть. Тако и прочіе, благочестно живущіе, хотя въ темницахъ и заключаются и плѣняются, духомъ свободни пребываютъ. Доколѣ бо человѣкъ вѣру имѣетъ и подвизается противу плоти, міра, діавола и грѣха, духовную свободу имѣетъ, хотя тѣломъ и порабощается. Хощеши ли убо, хрістіанине, сію преславную свободу имѣть, подвизайся всегда противу грѣха и діавола, врага своего, и будеши свободенъ.

\paragraph*{§\:349.} По преступленій заповѣди Божіей въ раи, прародители наши, и съ ними мы, чада ихъ, всѣ учинилися рабы грѣха и плѣнники діавольскіе, потеряли духовную свободу и царское духовное благородіе, подпали грѣховной тяжкой работѣ и попали подъ область діавольскую; и гордился надъ нами, яко страшный исполинъ и мучитель надъ плѣнными. Отъ сей тяжкой работы и діавольскаго плѣненія и власти искупилъ насъ Хрістосъ, Сынъ Божій, искупилъ же \textit{не истлѣннымъ сребромъ или златомъ, но честною Своею кровію}\footnote{1~Петр.~1,~18 и 19.}; цѣну за насъ плѣнныхъ дражайшую кровь Свою подалъ. Откуду называется и есть \textit{Искупитель, Свободитель и Избавитель} человѣческаго рода. И тако извелъ насъ на свободу духовную, дабы мы уже не были рабами грѣха и діавола, но противу сихъ враговъ подвизалися, а работали бы Ему со Отцемъ и Святымъ Духомъ, яко \textit{купленные рабы}, не принужденно, но свободно. Ибо Онъ никого къ Своей работѣ и службѣ не принуждаетъ, яко дѣлаютъ земные господа, но только призываетъ. \textit{Пріидите ко Мнѣ вси труждающіися и обремененніи, и Азъ упокою вы. Возмите иго Мое на себе, и научитеся отъ Мене, яко кротокъ есмь и смиренъ сердцемъ, и обрящете покой душамъ вашимъ}\footnote{Мѳ.~11,~28 и 29.}. А чтобы мы охотно могли отправлять благословенную службу сію, обѣщался намъ подать, и вѣрою просящимъ \textit{подаетъ Духа Своего Святаго благодать}\footnote{Лук.~11,~13.}, которая предваряетъ насъ нехотящихъ, дабы усердно хотѣли, хотящимъ помогаетъ, дабы дѣлали, что хощемъ; дѣлающихъ женетъ, дабы всегда въ томъ благословенномъ дѣлѣ пребывали и до конца пребыли, якоже о томъ поетъ и молится Давидъ святый: \textit{милость Твоя поженетъ Мя вся дни живота моего}\footnote{Пс.~22,~6.}.

\paragraph*{§\:350.} Должность убо наша требуетъ отъ насъ, дабы мы Іисусу Хрісту Господу нашему усердно работали. Понеже 1)~Нѣсмы свои, якоже Апостолъ учитъ: \textit{нѣсте свои, куплени бо есте цѣною}\footnote{1~Кор.~6,~19 и 20.}. Убо нѣсмы свои, но Его, ибо Онъ купилъ насъ не сребромъ или златомъ, но честною Своею кровію. Весь составъ нашъ, душа и тѣло наше, Его есть; Ему убо на службу душу и тѣло наше обратить должны мы, якоже Апостолъ увѣщаваетъ насъ: \textit{прославите Бога въ тѣлесѣхъ вашихъ и въ душахъ вашихъ, яже суть Божія}\footnote{1~Кор.~6,~20.}. Внѣшнее служеніе отъ внутреннаго должно происходить, и тѣлесная служба съ душевною согласна быть должна, дабы тое было и внутрь въ сердцѣ, что внѣ словомъ и дѣломъ является, и которая вѣра на устахъ исповѣдуется, тая бы въ сердцѣ была, и при случаѣ внѣ себе оказывала, якоже \textit{древо доброе добрыми плодами себе оказываетъ}\footnote{Мѳ.~7,~17.}. "--- 2)~Когда крещеніемъ святымъ вступали мы въ церковь Его, "--- отрицаяся сатаны и всѣхъ дѣлъ Его, обѣщалися Ему со Отцемъ и Святымъ Духомъ вѣрно работати. Сего ради должно намъ обѣты свои хранить и вѣрно Ему служитъ, да не лживи предъ Нимъ явимся. 3)~Раби Его есмы и \textit{по силѣ созданія}. Онъ нашъ Создатель и Господь, мы же дѣло рукъ Его. \textit{Руцѣ Твои сотвористѣ мя и создастѣ мя}, поетъ Ему Давидъ\footnote{Пс.~118,~75.}, что и всякъ отъ насъ долженъ Ему пѣть. \textit{Той бо сотвори насъ, а не мы: мы же людіе Его и овцы пажити Его}\footnote{99,~5.}. Завладѣлъ"=было нами, Его созданіемъ, лукавый сатана; но Онъ, яко Создатель нашъ, паки изъ рукъ его исторгнулъ насъ, и, яко Своихъ, паки присвоилъ Себѣ; того ради Ему, яко Господу нашему праведному, праведно одолжаемся и служити вся дни живота нашего. "--- 4)~Вся тварь работаетъ Ему, яко Господу своему, якоже поетъ Ему пророкъ: \textit{всяческая работна Тебѣ, Господи}\footnote{118,~91.}, "--- яко \textit{творитъ слово Его}\footnote{148,~8.}. Убо и намъ, яко твари Его разумной, должно сіе званіе отправлять, и Ему, яко Господу нашему, работати. "--- 5)~Вся тварь, по повелѣнію Его, намъ служитъ. Солнце, луна и звѣзды свѣтятъ намъ, огнь согрѣваетъ насъ, облака орошаютъ насъ, вода напаяетъ насъ, земля плоды приноситъ намъ, скоты служатъ намъ и пищу подаютъ намъ, и проч. Работаетъ сіе созданіе намъ, дабы мы Создателю нашему усердно работали, и за службу, которую отъ созданія Его пріемлемъ, Ему благодарили. Не требуетъ Онъ ради Себе ничего, яко \textit{всесовершенно блаженный}, но требуемъ мы. Сего ради, имѣя создать человѣка, прежде сотворилъ небо и землю и все украшеніе ихъ, дабы сотворенныя вещи имущему быть человѣку были, какъ рабы и слуги, готовы къ служенію; и тако сотворилъ человѣка, разумную тварь, дабы ихъ къ потребѣ и служенію своему употреблялъ, и, отъ нихъ услугу себѣ имѣя, Создателю своему, яко ближайшій и разумный рабъ, безпосредственно служилъ. Видиши, хрістіанине, како работаетъ намъ тварь неразумная, дабы мы разумные Богу работали. "--- 6)~Чудно сіе, что видимый свѣтъ, по повелѣнію Божію, работаетъ намъ, но то чуднѣе, что и ангели святые спасенію нашему служатъ. \textit{Ополчается Ангелъ Господень окрестъ боящихся Его}\footnote{Пс.~33,~8.}. \textit{Не вси ли суть служебніи дуси въ служеніе посылаеми за хотящихъ наслѣдовати спасеніе}\footnote{Евр.~1,~14.}? Преисполненъ его Ветхій Завѣтъ, но и въ Новомъ много таковаго служенія обрѣтается. Ангелъ благовѣствуетъ пресвятой Дѣвѣ грядущаго въ міръ Сына Божія, и отъ Нея плотію раждаемаго\footnote{Лук.~1,~26--38.}. Ангелъ Іосифу, обручнику Ея, является\footnote{Мѳ.~1,~20; 2,~13 и 19.}. Ангелъ Захаріи является и благовѣствуетъ зачатіе Предтечи\footnote{Лук.~1,~11.}. Ангелъ пастырямъ благовѣствуетъ рождшагося Хріста Спасителя міру\footnote{2,~9.}. Ангели на гробѣ воскресшаго Хріста сѣдятъ и проповѣдаютъ женамъ воскресеніе\footnote{Лук.~24,~4.}. Ангели при вознесеніи Господни являются апостоламъ, и возвѣщаютъ имъ второе Хрістово пришествіе\footnote{Дѣян.~1,~10.}. Ангелъ Петра изводитъ изъ темницы\footnote{12,~7.}. Ангелъ глаголетъ къ Филиппу: \textit{возстани и иди на полудне}, и проч.\footnote{8,~26.} Ангелъ Корнилію сотнику является\footnote{10,~3 и проч.}. Исторія церковная о томжде повѣствуетъ. «Ангелъ хранитель всякому вѣрному дается, и выну видитъ лице Отца небеснаго», глаголетъ Василій великій\footnote{Въ словѣ о дѣвствѣ и на Пс.~68"~й, т.~1.}. Служатъ сіи святые и блаженные духи нашему спасенію, да мы усердно служимъ Господу ихъ и нашему, и тако спасаемся. Отсюду удивляйся, хрістіанине, благости и человѣколюбію Божію, что не токмо видимой твари повелѣлъ служить нуждамъ нашимъ, но и невидимой, слугамъ Своимъ святымъ, хранить насъ и ополчаться окрестъ насъ. Отъ сего познай хрістіанскую честь и благородіе высокое, что слуги самаго Бога Вседержителя въ служеніе имъ посылаются. Великое бы воистину дѣло было, когда бы царь земный, вѣрнаго своего подданнаго отпущая отъ себе, приказалъ слугамъ своимъ проводить до дому его, дабы какого ему зла не приключилося на пути, коль несравненно большее дѣло есть, что Царь небесный, по крещеніи отпущая вѣрныхъ Своихъ рабовъ на путь міра сего, придаетъ имъ слугъ Своихъ, ангеловъ святыхъ, ради охраненія ихъ, дабы подъ охраненіемъ ихъ пришли въ отечество свое небесное, хотя то и Самъ отъ нихъ не отлучается, по реченному: \textit{се Азъ съ вами есмь во вся дни до скончанія вѣка}\footnote{Мѳ.~28,~20.}. Воистину невозможно надивиться благости и человѣколюбію Божію къ человѣку. Человѣкъ окаянный, грѣшникъ, законопреступникъ и отступникъ Божій, благодатію Сына Божія, вѣрою въ Него, такъ высокой милости отъ Бога сподобляется. О, Боже и Создателю нашъ преблагій! вездѣ насъ любовь Твоя срѣтаетъ, куды мы ни обратимся. Воистину слѣпъ человѣкъ, когда того не видитъ; неблагодаренъ, когда сердечно не признаетъ, окаяненъ, когда, оставивши Бога, Господа своего, Благодѣтеля и Промыслителя своего, работаетъ міру злому и грѣху. Чего ради самъ нѣкогда, какъ словомъ Божіимъ, такъ своею совѣстію, обличится и осудится, ибо противу слова Божія и своея совѣсти согрѣшаетъ. "--- 7)~И работать Хрісту благопріятно есть. Аще бо сынове вѣка сего лучше изволяютъ высокому господину, нежели подлому, "--- доброму, нежели злому, "--- мудрому, нежели безумному, "--- кроткому, милостивому, нежели гнѣвливому и жестокому работати: кто высше паче царя небеснаго? кто лучше паче Іисуса? кто мудрѣйшій паче Сына Божія? Кто кротчайшій и милостивѣйшій паче Хріста? Онъ есть Царь царствующихъ и Господь господствующихъ, Онъ вѣчная благость, Онъ Божія Ѵпостасная мудрость, Онъ есть кротокъ и смиренъ сердцемъ. Чтожъ прочее остается намъ, грѣшниче, какъ Ему отдаться и предаться въ сладкую работу? "--- 8)~Ищутъ люди легкой паче, нежели тяжелой работы. Іисусу работать легко есть. Не велитъ *Онъ каменія носить, не велитъ* горы разрывать, и прочая симъ подобная дѣлать рабамъ Своимъ; нѣтъ, ничего такого не слышимъ отъ Него, "--- но что? \textit{любите другъ друга}\footnote{Іоан.~13,~34; 15,~12 и 17.}. Что бо легчае, какъ любить? Тяжко ненавидѣть, ибо ненависть мучитъ; но любить сладко, ибо любовь веселитъ. Самъ Онъ о семъ свидѣтельствуетъ: \textit{иго Мое благо, и бремя Мое легко есть}\footnote{Мѳ.~11,~30.}. Возмемъ убо, возлюбленный хрістіанине, на себе иго Хрістово благое, и понесемъ бремя Его легкое, и послѣдуемъ Ему. 9)~Хотятъ люди по трудахъ и работѣ упокоиться. Хрістосъ намъ обѣщаетъ \textit{вѣчный покой, вѣчную субботу} подать, которую безпрестанно и безъ конца будемъ праздновать не на земли, но на небеси\footnote{Ис.~66,~23.}. 10)~Желаютъ люди награжденія за работу и труды. Хрістосъ обѣщаетъ намъ вѣчное награжденіе во царствіи Своемъ. \textit{Тако бо глаголетъ Господь: се работающіи Ми ясти будутъ: се работающіи Ми пити будутъ; се работающіи Ми возрадуются; се работающіи Ми возвеселятся въ веселіи сердца}\footnote{65,~13 и 14.}. 11)~Работающіи Іисусу и здѣ награжденіе свое имѣютъ. Ибо истинная добродѣтель сама въ себѣ есть награжденіе имѣющимъ ее. Гдѣ бо истинная добродѣтель есть, тамо есть любовь; *гдѣ любовь*, тамо добрая и покойная совѣсть; *гдѣ покойная совѣсть*, тамо миръ и покой; гдѣ миръ и покой, тамо утѣшеніе, радость, веселіе и сладость. "--- Тако, хрістіанине, работа Іисусу есть отъ насъ \textit{должная}, ибо Онъ нашъ Богъ, Создатель, Господь, Искупитель и Промыслитель. \textit{Славная} намъ, ибо Онъ есть Царь царей и Господь господей. \textit{Сладкая}, ибо иго Его благо, и бремя Его легко есть. \textit{Желанная}, ибо великое и неизреченное награжденіе имѣетъ. Поработаемъ убо отъ искренняго сердца, да и мы получимъ отъ Него мзду съ дѣлающими \textit{въ виноградѣ Его}\footnote{Мѳ.~20,~1 и проч.}.

\paragraph*{§\:351.} \textit{Работа хрістіанская}, которую они Господу и Царю своему Іисусу Хрісту отправляютъ, состоитъ: 1)~Въ истинной, живой и сердечной вѣрѣ во Хріста. \textit{Се есть дѣло Божіе, да вѣруете въ Того, Егоже посла Отецъ}, глаголетъ Хрістосъ\footnote{Іоан.~6,~29.}. Послалъ Отецъ Сына Своего въ міръ, \textit{да спасется Имъ міръ}\footnote{3,~17.}. Дѣло убо Божіе творитъ, кто Его, яко отъ Отца небеснаго посланнаго, сердцемъ пріемлетъ и сердечно въ Него вѣруетъ. А что сердцемъ пріемлется и вѣруется, тое и устами исповѣдуется, гдѣ нужно и честь имени Его требуетъ, якоже глаголетъ Апостолъ: \textit{сердцемъ вѣруется въ правду, усты же исповѣдуется во спасеніе}\footnote{Рим.~10,~10.}. "--- 2)~Состоитъ въ подвигѣ противу діавола и грѣха. Фараонъ, царь египетскій, когда Моисей извелъ Израиля изъ Египта, и работы его избавилъ, гналъ въ слѣдъ его, хотя его паки себѣ поработити\footnote{Исх.~14,~6--10.}: тако діаволъ гонитъ въ слѣдъ насъ хрістіанъ, которыхъ Хрістосъ Сынъ Божій отъ работы его избавилъ, и хощетъ паки себѣ поработити. Тако онъ противится Хрісту: Хрістосъ хощетъ насъ спасти, онъ хощетъ насъ погубити, и тако тщится лукавый духъ недѣйствительными Его заслуги учинить, и тѣмъ славу Его, которая отъ спасаемыхъ ради спасенія Ему бываетъ, умалити. Чѣмъ бо болѣе спасается людей благодатію Хрістовою, тѣмъ большая Хрісту отъ того бываетъ слава. Сіе онъ усматривая и негодуя, ищетъ препятствіе учинить спасенію нашему и Хрістовой славѣ умаленіе. Намъ, съ помощію Божіею, должно противу сего врага нашего и противника Хрістова вѣрою подвизаться, вѣдая, что мы Хрістовы есмы, \textit{цѣною} крови Его \textit{куплены} отъ него\footnote{1~Кор.~6,~19 и 20; 7,~23.}. Прельщаетъ онъ насъ суетою міра сего, *и совѣтуетъ хитро искать чести, славы, богатства и сласти вѣка сего*, да тѣмъ намъ учинитъ препятствіе въ семъ нашемъ подвигѣ, якоже совѣтовалъ прародителямъ нашимъ въ раи отъ заповѣданнаго древа вкусити, чѣмъ запялъ и погубилъ ихъ. Намъ должно злаго его совѣта не слушать, и не любить міра, по увѣщанію апостольскому: \textit{не любите міра, ни яже въ мірѣ. Аще кто любитъ міръ, нѣсть любве Отчи въ немъ: яко все, еже въ мірѣ, похоть плотская и похоть очесъ, и гордость житейская, нѣсть отъ Отца, но отъ міра сего есть}\footnote{1~Іоан.~2,~15 и 16.}. Тако святые мученики, когда ихъ мучители, слуги діавольскіе, къ идолопоклоненію прельщали и убѣждали, отвѣтъ имъ давали такой: «Хрістіане есмы, Хрісту работаемъ, Хрістовы рабы есмы: идоламъ не пожремъ». Такимъ образомъ и намъ, хрістіанине, должно, когда насъ сатана чрезъ злые помыслы, какъ слуги своя, ко грѣху и творенію злыя своея воли прельщаетъ и убѣждаетъ, отвѣщать ему: «Хрістовы рабы есмы, Хрісту должны и работать; не слушаемъ тебе, и злаго твоего совѣта не пріемлемъ». Сколько убо разъ, любезный хрістіанине, востаютъ въ сердцѣ твоемъ богопротивные помыслы, похоть къ плотской нечистотѣ, похоть къ возвышенію въ честь и къ славолюбію, похоть къ скверному прибытку, хищенію, воровству, лихоиманію, мщенію и озлобленію и проч.: столько разъ належитъ тебѣ брань и подвигъ противу врага твоея души. Разумѣй подлинно, что то есть злый совѣтъ врага нашего сатаны, который Хрісту противится, и нашему препятствуетъ спасенію. Хрістосъ бо насъ отъ грѣха отвращаетъ, а онъ въ грѣхъ вводитъ, и тако хощетъ насъ паки уловить въ сѣть свою, и отъ царства Хрістова въ свою покорить власть; вси бо работающіе грѣху во власти діавольской находятся, и потому \textit{не имѣютъ достоянія въ царствіи Хріста Бога}\footnote{Еф.~5,~5.}. Аще убо злый помыслъ востаетъ въ тебѣ, и хощетъ тебе въ грѣхъ вринуть, отвѣщай тогда по подобію мучениковъ: «Хрістосъ мене искупилъ Себѣ, Хрістовъ есмь: Хрісту долженъ я и работать вѣрою и правдою». Словомъ, аще хощещи въ царствіи Хрістовомъ быти, Ему работать и Его Царемъ, Господемъ и Богомъ своимъ истинно нарицать, не долженъ быть рабъ грѣха, \textit{яко всякъ творяй грѣхъ, рабъ есть грѣха}\footnote{Іоан.~8,~34.}. \textit{Никтоже бо можетъ двѣма господинома работати}\footnote{Мѳ.~6,~24.}. "--- 3)~Работаетъ Хрісту, кто, отвергше тяжкое сатанинское иго, грѣхами обремененное, носитъ благое иго Хрістово, по словеси Его: \textit{возмите иго Мое на себе}\footnote{11,~29.}. Благое иго Хрістово значитъ заповѣди Его святыя и легкія, яко \textit{заповѣди Его тяжки не суть}\footnote{1~Іоан.~5,~3.}. Что легчае, какъ любить, какъ выше сказано? Онъ заповѣдаетъ намъ другъ друга любить: \textit{се заповѣдаю вамъ, да любите другъ друга}\footnote{Іоан.~15,~17.}. А отъ любви всѣ добрыя дѣла, какъ отъ источника ручьи, проистекутъ; любовь бо есть \textit{корень добрыхъ дѣлъ}\footnote{1~Кор.~13,~1--8 и 13.}. Аще убо кто истинную хрістіанскую любовь имѣетъ, сей носитъ иго Хрістово и заповѣди Его исполняетъ, и тако Ему работаетъ. "--- 4)~Вѣрнымъ рабамъ Хрістовымъ должно славы Господа своего искать, отъ зла уклоняться и доброе творить на такій конецъ, дабы оттуду имя Господа ихъ славилося, да тако угодна будутъ дѣла ихъ Господу ихъ; иначе не могутъ быть угодна. Како бо угодно Хрісту можетъ быть тое дѣло, которое не ради Хріста дѣлается? Всякое дѣло, какое оно есть, доброе или злое, отъ конца судится. Хотя бо и доброе дѣло повидимому кажется быть, но, когда не ради добраго конца творится, не можетъ быть доброе, но злое. Напр. милостыню даеши ради того, чтобъ похвалу отъ человѣкъ имѣть: доброе дѣло по внѣшнему твориши, но внутрь есть злое, ради бо суетныя славы своея, а не ради Хріста дѣлаеши тое. Сего ради должно не токмо добрыя дѣла творить, но и добрѣ творить, то"=есть, на добрый конецъ, который есть слава Божія. "--- 5)~Вѣрнымъ рабамъ Хрістовымъ должно Господу своему работать всегда, въ благополучіи и неблагополучіи. Рабъ бо истинный и вѣрный долженъ быть неотлученъ отъ господина своего: такъ и Хрістовъ рабъ долженъ быть всегда со Хрістомъ Господемъ своимъ, и Ему неотступно работать. Иначе лицемѣрная и неистинная работа есть, когда въ благоденствіи нашемъ любимъ Хріста и работаемъ Ему, а въ злоденствіи негодуемъ, ропщемъ, и тако сердцемъ отстаемъ отъ Него. Неотступными убо рабами Хрістовыми должны мы быть, и вездѣ Ему послѣдовать, якоже Самъ отъ насъ того хощетъ: \textit{аще кто Мнѣ служитъ, Мнѣ да послѣдствуетъ}\footnote{Іоан.~12,~26.}. "--- 6)~Вѣрные Хрістовы рабы должны быть смиренны, кротки и терпѣливы. Стыдно рабамъ не быть смиренными, терпѣливыми и кроткими, когда господинъ ихъ смиренъ, терпѣливъ и кротокъ: стыдно и хрістіанамъ гордитися, когда Господь ихъ Іисусъ Хрістосъ смиренъ сердцемъ, и гнѣваться и мстить, когда Господь ихъ кротокъ и долготерпѣливъ. Откуду учитися симъ добродѣтелямъ отъ Себе повелѣлъ имъ: \textit{научитеся отъ Мене, яко кротокъ есмь и смиренъ сердцемъ}\footnote{Мѳ.~11,~29.}, рабы бо должны послѣдовать нравамъ господъ своихъ. "--- 7)~Рабамъ Хрістовымъ должно до конца вѣрными быть, якоже Самъ глаголетъ Господь имъ: \textit{буди вѣренъ даже до смерти, и дамъ ти вѣнецъ живота}\footnote{Апок.~2,~10.}. Всякое бо дѣло отъ конца, а не отъ начала, совершенство свое получаетъ. Конецъ все совершаетъ. Похвально и начало доброе, но безъ конца добраго лишается своея похвалы. Славно побѣдить непріятеля съ начала, но конечная побѣда все дѣло совершаетъ. Тако и въ хрістіанскомъ дѣлѣ: похвально подвизаться противу грѣха и діавола, и побѣждать его, но безъ конечной побѣды ничтоже есть. Сего ради до конца должно въ семъ благословенномъ дѣлѣ трудитися.

\paragraph*{§\:352.} Хрістіане"=де многіе работаютъ человѣкамъ, какъ"=то: подданные монархамъ, рабы господамъ своимъ? \textit{Отвѣтъ}: 1)~Работаютъ человѣкамъ тѣломъ, но духомъ единому Хрісту, Царю своему, со Отцемъ и Святымъ Духомъ работаютъ, и потому свободни суть; духа бо никто не можетъ поработити. Какъ окованные узами, сѣдящіе въ темницѣ, плѣненные, тѣломъ въ неволѣ, но духомъ въ свободѣ имѣются благочестивые, пока въ истинномъ находятся благочестіи. 2)~Истинные и вѣрные рабы Хрістовы и человѣку работать должны, но ради Хріста, Который учитъ: \textit{воздадите кесарева кесареви, и Божія Богови}\footnote{Мѳ.~22,~24.}. И Апостолъ глаголетъ: \textit{повинитеся всякому человѣчу начальству Господа ради}\footnote{1~Петр.~2,~13.}. Аще убо кто господину своему работаетъ ради Хріста заповѣдавшаго, тотъ работаетъ не человѣку, но Хрісту Господу. Кто бо чію волю и повелѣніе исполняетъ, тотъ тому и работаетъ. И тако кто повелѣнія Божія слушая, работаетъ господину своему, тотъ Богу работаетъ, а не человѣку, понеже Божія повелѣнія слушаетъ, а не человѣческаго. Якоже бо рабъ господину своему работаетъ не ради того, что онъ человѣкъ есть почтенный, но ради того, что царь ему работать повелѣлъ, и того рабомъ его учинилъ, и тако, повелѣнія царева слушая, рабъ царю своему работаетъ, а господину не просто, но ради царя: тако и хрістіанамъ должно работать человѣку, которому подчинены суть, не просто, но ради Господа повелѣвающаго: \textit{повинитеся}, и прочая. Сего бо хрістіанская вѣра требуетъ, которая отъ всего человѣка освобождаетъ, и единому Богу порабощаетъ свободно. Свободно, глаголю, понеже Богъ никого не принуждаетъ работать Себѣ, какъ выше сказано. О какъ рабъ дотолѣ господина своего слушать долженъ, доколѣ приказуетъ ему тое, что монаршимъ законамъ непротивно; а когда противно что приказуется, тогда вѣрноподданный монарху рабъ не долженъ слушать его: тако и вѣрнымъ Хрістовымъ рабамъ въ томъ только господъ своихъ слушать должно, что непротивно Царю своему Іисусу Хрісту повелѣваютъ. Въ противномъ случаѣ Хрістово повелѣніе предпочитать должно, ибо \textit{повиноватися подобаетъ Богови паче, нежели человѣкамъ}\footnote{Дѣян.~5,~29.}. Тако мученики святые, вѣрные и избранные Хрістовы рабы, чинили. Повелѣвали имъ нечестивые цари дань давати, давали; повелѣвали на брань итить, шли; повелѣвали работать, работали; повелѣвали трапезѣ предстоять, предстояли; повелѣвали руду копать, копали; повелѣвали каменіе и землю носить, носили; повелѣвали въ заточеніе, въ темницу, во узы итить, не противилися; повелѣвали одеждъ совлекатися, совлекалися; повелѣвали и въ прочемъ, закону Божію непротивномъ, слушали. А когда доходило повелѣніе ихъ до презрѣнія Божія, и повелѣваемо было имъ поклониться идоламъ, тутъ стали дерзновенно противу нечестиваго повелѣнія, не восхотѣли повелителей своихъ слушать, внимая заповѣди единаго истиннаго Бога: \textit{Азъ есмь Господь Богъ твой: да не будутъ тебѣ бози иніи, развѣ Мене}\footnote{Исх.~20,~2 и 3.}. А тѣмъ показали святые, что дотолѣ человѣку властелину повиноваться и служить должно намъ, доколѣ повелѣніе его согласно съ Божіимъ повелѣніемъ, или непротивно есть. Аще убо господинъ твой приказуетъ тебѣ что согласное Божію закону, Богъ тебѣ тое приказуетъ; и такъ, когда повинуешься въ томъ господину твоему, Богу повинуешься и работаешь. Притомъ разсуждай, кого ты боишися, когда усердно работаешь господину твоему, Бога или господина твоего, человѣка? Когда Бога боишися, Который повелѣлъ повиноватися властямъ, и тако работаешь господину \textit{Бога ради}, то Богу работаешь, Котораго страхомъ почитаешь, а не человѣку; а когда боишися господина человѣка, чтобы не наказывалъ тебе за нерадѣніе, то подлинно человѣку единому работаешь, а не Богу. Ибо человѣку работаешь не \textit{ради Божія}, но ради человѣческаго страха. Всякъ бо человѣкъ, ради кого что дѣлаетъ, тому и угождаетъ, а кому угождаетъ, тому и работаетъ, ибо работать и угождать есть едино. И тако, когда работаешь господину ради того, чтобы ему угодить, а не Богу, "--- человѣку работаешь, а не Богу. Тако и всякъ, кто отъ грѣха удаляется и доброе дѣло показуется творить не ради того, чтобы Богу угодить, но чтобы людямъ, "--- таковый не Богу, но людямъ угождаетъ, и работаетъ не Богу, но людямъ, рабъ есть человѣковъ, а не Господень. Всякое бо дѣло отъ сердечнаго расположенія и конца судится отъ Бога. Тако всѣ тѣ хрістіане суть рабы человѣческіе, а не Господни, которые указовъ монаршихъ слушаютъ, боясь казни гражданскія, а не Бога. Тако судіи, хрістіанами именующіися, суть рабы человѣческіе, а не Божіи, которые мзды не касаются и правду наблюдаютъ не ради страха Божія, но ради страха человѣческаго. Ибо всѣ таковые, ежели бы наказаній гражданскихъ не боялися, дѣлали бы по своей волѣ, но страхъ гражданскаго суда воспящаетъ ихъ отъ того. И потому, оставивши Бога, человѣку работаютъ, или паче грѣху, и грѣхомъ діаволу, и суть внутрь злыи и неправедные, хотя внѣ и являются добрыми, и Богу Господу своему невѣрные, якоже язычники, незнающіе истиннаго Бога, понеже вѣры не имѣютъ, которая учитъ все \textit{ради Бога} творити. А которые и суда гражданскаго не боятся, и законъ нарушаютъ, тѣ уже страхъ и стыдъ человѣческій потеряли, въ крайнемъ находятся заблужденіи, и злонравіемъ своимъ превосходятъ самихъ язычниковъ политичныхъ, которые общимъ человѣческимъ нравамъ повинуются и наблюдаютъ, что монаршіе указы повелѣваютъ. Словомъ, всякъ, кто человѣку, монарху и господину своему, повинуется не ради Бога, но ради страха ихъ или награжденія, тотъ не Богу работаетъ, но человѣку, не Божій рабъ, но человѣческій, не Богу угождаетъ, но человѣкамъ, ибо дѣло его не отъ вѣры происходитъ. Вѣрный же Хрістовъ рабъ, все минуя, ради единаго Бога работаетъ монарху, или господину своему, взирая на повелѣніе Божіе, понеже Богъ повелѣлъ повиноватися властямъ. Таковый, хотя бы никакого награжденія временнаго за исправность, или наказанія за неисправность не было, будетъ чистосердечно господину своему служить, опасаяся единаго Божія гнѣва за неисправность.

\paragraph*{§\:353.} Убо"=де хрістіане не всѣ единому Іисусу Хрісту работаютъ? "--- \textit{Отвѣтъ}. Видѣлъ ты выше, кто работаетъ Хрісту. Хрістіане, которые чистосердечно вѣруютъ во Хріста и вѣру свою оказываютъ послушаніемъ, тѣ всѣ работаютъ единому Хрісту; а которые не слушаютъ Хріста, и заповѣдей Его не исполняютъ, тѣ Ему не работаютъ. Истинная бо и нелицемѣрная работа не можетъ быть безъ послушанія. Ибо работать господину, и не слушать его, разуму противно есть и въ себѣ противорѣчіе включаетъ, якоже противорѣчіе есть "--- быть рабомъ и не быть рабомъ, служить и не служить, почитать и не почитать, любить и не любить, и проч. "--- Комуже"=де таковые хрістіане работаютъ? "--- Работаютъ своимъ прихотямъ и страстямъ плотскимъ, а въ нихъ самому діаволу, яко волю его злую исполняютъ и слушаютъ его. "--- Страшно"=де и мерзко есть работать діаволу, противнику и врагу Божію? "--- Подлинно страшно и мерзко, но истинно. Ибо кто кому повинуется, тотъ тому и работаетъ: \textit{имже бо кто побѣжденъ бываетъ, сему и работенъ есть}, глаголетъ Апостолъ\footnote{2~Петр.~2,~19.}. "--- Въ чемъ"=де они повинуются и работаютъ ему? "--- Въ дѣлѣхъ его злыхъ, въ гордости, сребролюбіи, хищеніи, воровствѣ, лукавствѣ, лжи, ссорахъ, блудѣ, прелюбодѣяніи, нечистотѣ, піянствѣ, памятозлобіи и прочіихъ симъ подобныхъ дѣлахъ. "--- Хрістосъ"=де избавилъ хрістіанъ отъ работы діавольской? "--- Подлинно избавилъ, какъ выше сказано, но они сами подъ тяжкое его иго возвратилися, и, оставивши Господа своего, Которому, яко рабы, должны были вѣрою и правдою работать, пошли въ слѣдъ сатаны. "--- Убо"=де они не рабы Хрістовы? "--- Какъ Хрістосъ есть Господь и Царь всея твари, и господствуетъ надъ всѣми добрыми и злыми, такъ и всѣ люди, добрые и злые, суть раби Его. Но которые вѣрно Ему работаютъ, вѣрные и добрые раби Его суть. Къ таковымъ глаголетъ Хрістосъ: \textit{добрый рабе, благій и вѣрный! о малѣ былъ еси вѣренъ, надъ многими тя поставлю: вниди въ радость Господа твоего}\footnote{Матѳ.~25,~23.}. А которые не слушаютъ Его и не работаютъ Ему, тѣ такожде раби Его суть, но злые, невѣрные и неключимые. О таковыхъ глаголетъ: \textit{неключимаго раба вверзите во тьму кромѣшнюю: ту будетъ плачь и скрежетъ зубомъ}\footnote{ст.~30.}. "--- Чтожъ"=де имъ надобно учинить, дабы отъ тяжкой діавольской власти и работы свободиться, и работать Хрісту, Царю и Господу всѣхъ? "--- \textit{Отвѣтъ}. Должно чистосердечно съ покаяніемъ и жалѣніемъ обратиться къ Нему и молить Его, чтобы паки ихъ отъ ига того свободилъ и принялъ въ благодатное Свое царство. Онъ, яко милостивъ и человѣколюбецъ и не хотяй смерти грѣшнику, \textit{грядущаго къ Себе не ижденетъ вонъ}\footnote{Іоан.~6,~37.}, понеже и нынѣ всѣхъ призываетъ: \textit{пріидите ко Мнѣ вси труждающіися и обремененніи, и Азъ упокою вы. Возмите иго Мое на себе, и научитеся отъ Мене, яко кротокъ есмь и смиренъ сердцемъ}, и проч.\footnote{Матѳ.~11,~23--30.}, "--- и до скончанія вѣка будетъ призывать и пріимать приходящихъ къ Нему. О семъ буди Ему благодареніе и слава со Отцемъ и Святымъ Духомъ во вѣки. Аминь.

\subsection[Глава 5-я. Яко хрістіанинъ долженъ вѣрою и любовію послѣдовать Хрісту.]{глава пятая.\\\bfseries Яко хрістіанинъ долженъ вѣрою и любовію послѣдовать Хрісту.}

\begin{quotation}\textit{Хрістосъ пострада по насъ, намъ оставль образъ, да послѣдуемъ стопамъ Его: Иже грѣха не сотвори, ни обрѣтеся лесть во устѣхъ Его; Иже укоряемь противу не укоряше, стражда не прещаше; предаяше же судящему праведно}\footnote{1~Петр.~2,~21--23.}.\end{quotation}
\begin{quotation}\textit{Образъ дахъ вамъ, да, якоже Азъ сотворихъ вамъ, и вы творите}, глаголетъ Хрістосъ\footnote{Іоан.~13,~15.}.\end{quotation}
\begin{quotation}\textit{Аще кто мнѣ служитъ, Мнѣ да послѣдствуетъ}, глаголетъ Хрістосъ\footnote{12,~26.}.\end{quotation}

\paragraph*{§\:354.} Якоже приходимъ ко Хрісту и благодатію Его къ небесному Отцу Богу нашему не ногами, но сердцемъ и вѣрою, не премѣненіемъ мѣста, но премѣненіемъ сердца и мысли нашея на лучшее: тако и послѣдуемъ Ему не ногами нашими, не прехожденіемъ съ мѣста на мѣсто, но вѣрою, любовію, изволеніемъ и премѣненіемъ житія нашего. Многіе, во время плотскаго Его пожитія на земли, ходили за Нимъ ногами, но сердцами отъ Него отвращалися; паче же многіе и враждовали и злобилися на Него, каковы были книжники, фарисеи и прочіе Его враги. Такъ и нынѣ многіе отъ хрістіанъ устами приближаются Ему, поютъ Его, хвалятъ и глаголютъ Ему: \textit{Господи, Господи}, но сердцами далеко отстоятъ отъ Него; хотятъ со Хрістомъ \textit{прославленнымъ} быть, но съ \textit{поруганнымъ} быть не хотятъ. Но святая вѣра, которая вѣрнаго со Хрістомъ соединяетъ, требуетъ того, чтобы вѣрный вездѣ и всегда, яко членъ отъ главы, неотлученъ былъ, то"=есть и здѣ крестъ свой носилъ и за Нимъ ходилъ, и въ будущемъ вѣкѣ въ славѣ съ Нимъ царствовалъ. Должно убо хрістіанину не токмо устами приближатися Хрісту, но и сердцемъ Ему прилѣпитися и послѣдовати, когда угодити и вѣрно работати Ему хощетъ.

\paragraph*{§\:355.} Хрістосъ, Сынъ Божій, Господь нашъ, живучи на землѣ, не токмо заслужилъ намъ милость и благословеніе у небеснаго Своего Отца, какъ выше о томъ многократно сказано, но и какъ словомъ научилъ, такъ и дѣломъ показалъ, какъ намъ Богу угождать, свято и по"=хрістіански жить: понеже мы не знали сами, въ чемъ тое богоугодное житіе состоитъ. Ибо, по свидѣтельству святаго Писанія, \textit{Господь съ небесе приниче на сыны человѣческія, видѣти, аще есть разумѣваяй, или взыскаяй Бога. Вси уклонишася, вкупѣ неключими быша; нѣсть творяй благостыню, нѣсть до единаго}\footnote{Пс.~13,~2 и 3.}. Сего ради Сынъ Божій явился на земли во образѣ человѣческомъ, и свято и непорочно пожилъ между человѣками, такъ что \textit{грѣха ни сотвори, ни обрѣтеся лесть во устѣхъ Его}\footnote{1~Петр.~2,~22.}. И святое Свое житіе подалъ намъ во образъ къ подражанію того, да въ оное, какъ въ чистое зерцало, часто посматриваемъ и вѣрою въ Него скверны, къ душамъ нашимъ прилѣпшія, отираемъ, якоже глаголетъ: \textit{образъ дахъ вамъ, да якоже Азъ сотворихъ вамъ, и вы творите}\footnote{Іоан.~13,~15.}. Любовь, смиреніе, терпѣніе, кротость, милосердіе и прочія святаго Своего сердца свойства открылъ, да и мы учинимъ тоежде, что въ Немъ видимъ, якоже глаголетъ: \textit{научитеся отъ Мене}\footnote{Матѳ.~11,~29.}. Откуду и апостоли святые къ подражанію представляютъ намъ житіе Хрістово: \textit{Хрістосъ пострада по насъ, намъ оставль образъ, да послѣдуемъ стопамъ Его}\footnote{1~Петр.~2,~21.}. И Павелъ святый глаголетъ: \textit{сіе да мудрствуется въ васъ, еже и во Хрістѣ Іисусѣ} и проч.\footnote{Филип.~2,~5 и слѣд.} Примѣчайте"=де и внимайте, что сотворилъ Хрістосъ Іисусъ. Онъ тако великъ и высокъ есть, что никто Его большій и высшій есть, ибо есть Богъ; но какъ глубоко Себе насъ ради смирилъ! На сей живый смиренія образъ взирайте и тому же учитеся. \textit{Сіе да мудрствуется въ васъ, еже и во Хрістѣ Іисусѣ}.

\paragraph*{§\:356.} Аще кто хощетъ Хрісту послѣдовать (всякъ же хрістіанинъ долженъ, какъ выше сказано и ниже скажется), тому должно прежде себе отрещися. Какъ бо хотящему на небо смотрѣть, должно первѣе очи свои отвратить отъ земли, и тако на небо смотрѣть: тако кто хощетъ Хрісту, то"=есть, Хрістову небесному житію послѣдовать, тотъ долженъ первѣе себе отрещися, и тако будетъ свободно послѣдовать Хрісту. Ибо какъ невозможно купно и на небо и на землю смотрѣть, такъ невозможно и себѣ и Хрісту послѣдовать. Причина тому сія есть: понеже мы отъ природы нашей злы, отъ грѣшника Адама грѣшники и грѣхолюбивые раждаемся всѣ, не иное бо что замышляемъ, какъ злое и суетное: Хрістосъ же есть самая чистѣйшая благостыня и святыня, есть свѣтъ, никакія тьмы непричастный\footnote{Іоан.~8,~12; 1~Іоан.~1,~5.}, "--- того ради и не можемъ Ему послѣдовать, пока себе, яко злыхъ, не отречемся. Откуду и Самъ Онъ глаголетъ: \textit{аще кто хощетъ по Мнѣ ити, да отвержется себе}\footnote{Матѳ.~16,~24.}.

\paragraph*{§\:357.} Что есть \textit{отрещися} себе, познаемъ, когда посмотримъ на того, кто ближняго своего отрицается. Всякъ, кто ближняго своего отрекся, удаляется отъ него: когда отреченный алчетъ, жаждетъ, наготуетъ, біется, безчестится, болитъ, не чувствуетъ тогда отрекшійся; когда бѣдствуетъ, не помогаетъ ему; когда страждетъ, не состраждетъ ему. Тако должно и намъ, когда хощемъ истинно отрещися себе, съ собою поступать: когда хулятъ, ругаютъ, поносятъ намъ, біютъ и уязвляютъ, лишаютъ чести и имѣнія, заключаютъ въ темницу и облагаютъ оковами, изгоняютъ и отъ дома и отечества удаляютъ, "--- всего сего какъ бы не чувствовать на себѣ, и вмѣнять какъ бы на другомъ комъ, а не на насъ, тое все бѣдствіе дѣлалося. Сіе есть \textit{отрещися себе}. Тако отрещися должно намъ и послѣдовать Хрісту: отрещися воли нашей, и послѣдовать волѣ Хрістовой; отрещися злонравія нашего, и послѣдовать благонравію Хрістову; отрещися гордости, злобы, зависти, ненависти, нетерпѣнія, сребролюбія, самолюбія, славолюбія, и прочаго ветхаго Адама злонравія, и послѣдовать Хрістову смиренію, кротости, любви, терпѣнію, нищетѣ и прочіимъ Божественнымъ нравамъ. Никтоже можетъ купно быть гордымъ и смиреннымъ, злобнымъ и кроткимъ, ропотливымъ и терпѣливымъ, сребролюбцемъ и нестяжательнымъ, завистливымъ и любительнымъ, похотливымъ и цѣломудреннымъ, скупымъ и щедрымъ. Ибо порокъ и противная ему добродѣтель въ единомъ сердцѣ помѣститься не можетъ, но одно другаго изгоняетъ вонъ. Когда порокъ какій изъ сердца исходитъ, тогда добродѣтель тому противная входитъ: когда исходитъ гордость, приходитъ смиреніе; когда отступаетъ злоба, на мѣсто ея наступаетъ кротость; когда изгоняется ненависть, зависть, скупость, невоздержаніе, нечистота, тогда вселяется любовь, милость, щедролюбіе, воздержаніе, цѣломудріе, и проч. И чимъ болѣе порокъ въ человѣкѣ уменьшается, тѣмъ болѣе добродѣтель тому противная умножается: чимъ болѣе въ комъ гордость убавляется, тѣмъ болѣе смиреніе возрастаетъ; чимъ болѣе оскудѣваетъ злоба, ненависть, зависть, скупость, немилосердіе, нечистота, тѣмъ болѣе возрастаетъ кротость, любовь, щедролюбіе, милосердіе, чистота. Аще убо кто хощетъ послѣдовать Хрісту, смиренію, кротости, терпѣнію, любви и прочіимъ Божественнымъ Его нравамъ, тому должно оставить свою гордость, злобу, мщеніе, гнѣвъ, нетерпѣніе, ненависть, зависть и прочія ветхаго человѣка душепагубныя свойства. "--- Какъ"=де тое можетъ быть? "--- \textit{Отвѣтъ}. Таковымъ образомъ тое дѣлается. Востаетъ въ сердцѣ твоемъ похоть къ нечистотѣ, блуду и прелюбодѣянію? "--- Сердце твое сего похотствуетъ; а сердце твое ты самъ еси, ибо сердце человѣческое всего человѣка въ себѣ заключаетъ, и не иное что есть, какъ самъ человѣкъ, внутрь и внѣ со всѣмъ своимъ наклоненіемъ разсуждаемый. Отрекись похоти сея злыя, и съ похотію тебе самаго похотствующаго отрекись, и послѣдуй чистотою и цѣломудріемъ пречистому и святыхъ святѣйшему Хрісту. Поднимается въ сердцѣ твоемъ гордость, и, какъ змій, главу свою возноситъ и хощетъ тебе угрызти? Отрекись сея мерзости, \textit{яко все, еже въ человѣцѣхъ высоко, мерзость есть предъ Богомъ}\footnote{Лук.~16,~15.}, "--- и послѣдуй смиренному Іисусу смиреніемъ. Запаляется сердце твое ко гнѣву, злобѣ, мщенію? Отрекись того, и послѣдуй кротостію кроткому и незлобивому Агнцу, Хрісту Іисусу. Зачинается въ сердцѣ твоемъ зависть и ненависть, какъ младенецъ пагубный отъ зміина сѣмене? Отрицайся того зла, умерщвляй его внутрь тебе, и послѣдуй любовію человѣколюбцу Сыну Божію. Востаетъ стыдъ безчестія и страхъ смерти правды ради? Плоти твоея славолюбивыя и немощныя свойство есть. Ободрися вѣрою и духомъ, отрекися немощи твоея, воззри на Хріста поруганнаго, обезчещеннаго, посмѣяннаго, оплеваннаго, распятаго за грѣхи твои, и любовію послѣдуй Ему, тебе ради пострадавшему. Тако первѣе себе должно отрещися, и тако послѣдовати Хрісту. И сіе"=то есть \textit{распинать плоть со страстьми и похотьми}\footnote{Гал.~5,~24.}; \textit{совлекаться ветхаго человѣка съ дѣяньми его, и облекаться въ новаго человѣка, созданнаго по Богу въ правдѣ и въ преподобіи истины}\footnote{Кол.~3,~9 и 10; Еф.~4,~22--24.}. Въ человѣкѣ бо, хрістіанскую вѣру воспріявшемъ и святымъ крещеніемъ обновленномъ, двоякій человѣкъ, но противный себѣ долженъ быть: \textit{ветхій и новый}, какъ о томъ выше многажды сказано. \textit{Ветхій} человѣкъ не иное что въ насъ, какъ страсти грѣховныя, съ нами родившіяся и ко грѣху насъ склоняющія, какъ"=то: нечистота плотская, неправда, ложь, лукавство, гордость, гнѣвъ, злоба, ненависть, зависть, сребролюбіе, славолюбіе и проч. \textit{Новый} же есть вѣра живая съ плодами своими, подвизающаяся противу ветхаго человѣка, страстьми и похотьми тлѣющаго, котораго подвига отъ всякаго хрістіанина вѣра хрістіанская требуетъ. И въ комъ такого подвига нѣтъ, въ томъ только единъ ветхій человѣкъ имѣется, слѣдственно и вѣры нѣтъ, которая обновляетъ человѣка и подвизается противу грѣха, какъ о томъ такожде выше сказано. Аще убо хощемъ Хрісту послѣдовать, должно намъ ветхаго человѣка, то"=есть, гордости, сребролюбія, славолюбія, сладострастія, гнѣва, злобы, мщенія и прочаго отрещися, и возлюбить и послѣдовать смиренію, терпѣнію, кротости и любви Хрістовой. Ветхій нашъ человѣкъ ужасается безчестія, поношенія, руганія, узъ, темницы, ссылки, изгнанія. Но, когда Хрістовыми быть, Хрісту послѣдовать, не должно намъ въ томъ ветхаго человѣка слушать, но ради Хріста безчестіе и прочее противное, намъ приключающееся, презирать, взирая на Него, якоже Онъ насъ ради не устыдился быть поруганнымъ и прочія страсти претерпѣть. Ветхій нашъ человѣкъ ужасается смерти; но и въ томъ вѣра наша насъ да укрѣпитъ, и живота нашего не щадѣть, гдѣ нужда есть, Хріста ради, Который \textit{душу Свою за насъ положилъ}\footnote{1~Іоан.~3,~16.}. Онъ намъ предшелъ смиреніемъ, терпѣніемъ, кротостію, злостраданіемъ, и показалъ путь отъ земли на небо; того ради на Него взирать, за Нимъ итить, и тѣмъ путемъ Ему послѣдовать, и, ветхаго человѣка, который насъ отъ того пути отвлекаетъ, отрицаяся и оставляя позади, послѣдовать оному \textit{Начальнику вѣры и Совершителю} должно\footnote{Евр.~12,~2.}. Родился Хрістосъ плотію: должно и намъ родитися духомъ. Возрасталъ Хрістосъ: должно и намъ расти духовнѣ, и не всегда младенцами быти о Хрістѣ. Искушаемъ былъ Хрістосъ отъ сатаны\footnote{Матѳ.~4,~1--11.}: слѣдуетъ и намъ искуситися отъ него, и Хрістовою силою побѣдить его. Изшелъ Хрістосъ на проповѣдь святаго своего Евангелія, и училъ правдѣ, и свидѣтельствовалъ истину, якоже Самъ рече: \textit{Азъ на сіе родихся, и на сіе пріидохъ въ міръ, да свидѣтельствую истину}\footnote{Іоан.~18,~37.}, должно и намъ, когда надлежитъ и гдѣ надлежитъ, истины не молчать, но небоязненно слово истины свидѣтельствовать, и, что въ сердцѣ вѣруемъ, устами исповѣдывать. Ненавидимъ былъ и гонимъ Хрістосъ за правду отъ враговъ правды: слѣдуетъ и намъ, правду исповѣдающимъ, тоежъ терпѣти отъ міра, ибо міръ не любитъ правды. Подвизался Хрістосъ до смерти, смерти же крестныя, за истину: такъ должно и намъ до самыя смерти стоять за истину, лживыя уста заграждать, ложь и неправду обличать. Тако пострадалъ Хрістосъ, и вошелъ въ славу Свою, якоже глаголетъ намъ: \textit{не сія ли подобаше пострадати Хрісту, и внити въ славу} Свою\footnote{Лук.~24,~26.}? Тако и намъ слѣдовать за Нимъ и \textit{многими скорбьми подобаетъ внити въ царствіе Божіе}, кровію и смертію Его отверстое\footnote{Дѣян.~14,~22.}. Ибо нѣтъ инаго пути къ оному царствію, кромѣ пути тѣснаго, скорбнаго и крестнаго, \textit{яко пространная врата и широкій путь вводяй въ пагубу}\footnote{Матѳ.~7,~13.}. "--- Какъ"=де себе отрещися? дѣло сіе невозможное есть. \textit{Отвѣтъ}: 1)~Не велитъ Хрістосъ совсѣмъ себе отрещися: надобно бо плоть алчущую питать, жаждующую напаять, нагую одѣвать, утружденную упокоевать, ибо безъ того жить не можемъ. Надобно животъ свой хранить и защищать. 2)~За честь имени Его, въ случаѣ нужды, и плоть и животъ нашъ презрѣть должно, якоже Онъ Себе насъ ради не пощадѣлъ. А безъ таковыя нужды себе и животъ свой берещи. Напр. когда намъ слѣдуетъ или лишиться живота, или согрѣшить Ему, въ такомъ случаѣ должно намъ себе отрещися и лишиться живота, нежели согрѣшить Ему. Безъ того должно себе и животъ свой хранить. 3)~Когда велитъ Хрістосъ отрещися, то отреченіе сіе разумѣется отреченіе злонравія и злыя воли нашея, какъ выше сказано; безъ того бо не можемъ угодить и послѣдовать Ему. 4)~Правда и тое, что и злонравія нашего отрещися и побѣдить нашими силами не можемъ. Ибо злонравіе свое побѣдить, есть себе самого побѣдить и отрещися, что собственной нашей силѣ невозможно. Чего ради къ тому искать намъ должно помощи у Самаго Бога, Который \textit{все можетъ}\footnote{19,~26.}. Откуду повелѣно молитися, просити, искати и толкати. Однакожь требуется и съ нашей стороны тщаніе, Богъ бо помогаетъ труждающимся, а не лежащимъ и унывающимъ.

\paragraph*{§\:358.} \textit{Причины}, ради которыхъ себе отрещися и \textit{послѣдовати Хрісту} должны мы: 1)~Что когда не хощешь себе и злонравія своего отрещися: паки ли хощешь обратитися къ сатанѣ, дѣламъ его, служенію и гордынѣ его, которыхъ при крещеніи святомъ и вступленіи въ хрістіанство отреклися мы, и обѣщалися Богу служить преподобіемъ и правдою? Соединившися Хрісту вѣрою, паки ли хощешь союзъ сей спасительный разорвать? \textit{Познавши путь правды}, паки ли хощешь \textit{возвратитися вспять отъ преданныя святыя заповѣди истинныя}, и на себѣ показати притчу: \textit{песъ возвращся на свою блевотину, и свинія, омывшися, въ калъ тинный}\footnote{2~Петр.~2,~21 и 22.}, "--- и тако попасться въ тяжкую работу діавольскую, отъ которой Сынъ Божій кровію Своею искупилъ тебе и извелъ на свободу? Паки ли хощешь подъ гнѣвомъ Божіимъ и клятвою быть? Ибо вся сія, не хотящему себе и своего злонравія ненавидѣти, отрекатися, умерщвляти, и послѣдовати благому Хрістову нраву, неотмѣнно послѣдуетъ. \textit{Иже не пріиметъ креста своего и въ слѣдъ Мене грядетъ, нѣсть Мене достоинъ}, глаголетъ Хрістосъ\footnote{Матѳ.~10,~38.}. И паки: \textit{иже аще постыдится Мене и Моихъ словесъ въ родѣ семъ прелюбодѣйнѣмъ и грѣшнѣмъ, и Сынъ человѣческій постыдится его, егда пріидетъ во славѣ Отца Своего со ангелы святыми}\footnote{Марк.~8,~38.}. Стыдишися ли Хрістова смиренія, терпѣнія, кротости, и хощеши съ міромъ въ гордости, злобѣ, почитаніи и славѣ быть? Стыдишися и словесъ Хрістовыхъ, которыя поучаютъ смиренію, терпѣнію и кротости; слѣдственно и не послѣдуеши Его смиренію, терпѣнію и кротости, а тако и Самого Хріста стыдишися. Стыдишися ли Самого Хріста? Постыдится и Онъ тебе предъ ангелами святыми и всѣми избранными Своими. Ибо не имѣеши знаменія, что ты Хрістовъ еси, знаменіе бо Хрістова раба есть: вѣра, любовь, терпѣніе, кротость, и проч. Горе же будетъ тѣмъ, которыхъ Хрістосъ постыдится, отречется и скажетъ: \textit{не вѣмъ васъ, откуду есте}\footnote{Лук.~13,~25 и 27.}. \textit{Не вѣмъ васъ}, понеже вы въ гордости своей не знали Мене въ смиреніи Моемъ. Что бо за симъ страшнымъ изреченіемъ послѣдуетъ? Страшнѣйшій гласъ: \textit{отступите отъ Мене вси дѣлателіе неправды: ту будетъ плачь и скрежетъ зубомъ}\footnote{Ст.~27 и 28.}. "--- 2)~Смотри и разсуждай, сколько сынове вѣка сего временныхъ ради благъ трудятся и подвизаются, ничего тяжестнаго и неудобнаго себѣ не вмѣняютъ, дабы намѣренное и желаемое получить. Не страшится воинъ исходить противу непріятеля подъ пули, ядра, мечи и прочія оружія, явною смертію грозящія, чтобы отъ монарха, которому служитъ, честь и славу заслужить. Не боится купецъ по чужимъ странамъ, разбоевъ и прочіихъ опасностей наполненнымъ, скитаться, дабы тлѣнное собрать сокровище. Не тяжко поселянину цѣлое лѣто зноемъ солнечнымъ горѣть и потомъ обливаться и трудиться ради желаемаго плода. Чего не дѣлаютъ честолюбцы? какихъ услугъ не показуютъ княземъ, вельможамъ и прочіимъ высокимъ лицамъ? Не токмо имъ самимъ, но и слугамъ ихъ угождаютъ, ласкаютъ, раболѣпствуютъ, чтобъ честь желаемую достать. Сихъ всѣхъ надежда временнаго блаженства такъ сильно поощряетъ и убѣждаетъ труды всякіе подымать и противности терпѣть. Какъ тебе не поощритъ къ подвигу надежда будущаго вѣчнаго и неизреченнаго блаженства, когда на тое вѣрою взирать будешь! Истину тебѣ говорю, хрістіанине: весь свѣтъ со всею славою, честію, великолѣпіемъ, богатствомъ и прочіими утѣхами своими омерзѣетъ тебѣ. Безчестіе, поношеніе и поруганіе ради Хріста за честь себѣ вмѣниши. Нѣтъ о томъ сумнѣнія тѣмъ, которые вѣрою взираютъ на будущую сыновъ Божіихъ славу. Различно оную изображаетъ слово Божіе, и представляетъ намъ во утѣшеніе и духовную радость. Но Павелъ святый, неизреченныхъ Божіихъ таинъ зритель, краткими словами заключилъ: \textit{ихже око не видѣ, и ухо не слыша, и на сердце человѣку не взыдоша, яже уготова Богъ любящимъ Его}\footnote{1~Кор.~2,~9.}. И Хрістосъ о избранныхъ Своихъ глаголетъ: \textit{тогда праведницы просвѣтятся, яко солнце, во царствіи Отца ихъ}\footnote{Матѳ.~13,~43.}. Временное добро, какъ ни велико, и коль долго ни имѣть его, надобно, съ лишеніемъ живота сего, и того лишиться, а часто и прежде кончины отступаетъ отъ насъ, однакожъ съ толикою охотою ищутъ его люди: кольми паче, ради пріобрѣтенія вѣчнаго блаженства, которое обѣщалъ Богъ любящимъ Его, ничто себѣ тяжестно и неудобно не должно намъ вмѣнять, хотя плоти нашей и горестно кажется, наипаче, когда Самъ Богъ обѣщался въ томъ помогать намъ, и помогаетъ труждающимся и ищущимъ его. \textit{Вся возможна суть вѣрующему} и истинно хотящему\footnote{Марк.~9,~23.}. "--- 3)~Представь себѣ образъ и нравъ обоихъ "--- ветхаго Адама и Хріста, насъ ради бывшаго человѣка, и разсуди прилѣжно: кому лучше послѣдовать? Гордость, высокоуміе, зависть, злоба, нечистота, срамословіе, обжирство, хищеніе, лукавство, лесть, ложь и прочія мерзости ветхаго Адама свойства суть. Скотскій и звѣрскій нравъ ветхаго человѣка нравъ есть: лютостію подобенъ льву, хитростію и лукавствомъ лису, хищеніемъ волку, нечистотою и обжирствомъ псу, злобою василиску, и проч. И которые нравы особенно раздѣлены имѣются въ безсловесныхъ, тѣ въ единомъ ветхомъ человѣкѣ находятся. Таковая мерзость въ ветхомъ человѣкѣ крыется, и въ случаяхъ подобными себѣ плодами себе оказываетъ. Самъ ты, въ ближнемъ твоемъ видя нравъ сей, гнушаешься имъ, хотя и въ своемъ сердцѣ тойжде имѣешь; всѣ бо мы зло сіе въ сердцѣ своемъ носимъ, яко рожденное съ нами. Обрати теперь очи сердца твоего на Хрістовъ нравъ, который живо описуетъ намъ святое Его Евангеліе. Въ Немъ является горячая любовь, объемлющая объятіями своими всѣхъ, друговъ и враговъ, сердце, жаждущее всѣмъ спасенія, смиреніе глубочайшее, кротость и долготерпѣніе къ хулителямъ, святыня паче свѣта чистѣйшая, благосклонность къ просящимъ, милосердіе къ бѣднымъ, состраданіе къ страждущимъ, милость ко грѣшникамъ, благость и снисхожденіе ко всѣмъ. Хощемъ Бога Отца небеснаго видѣть? Во Хрістѣ Его видимъ. Хощемъ Божественный Его нравъ познать, какъ щедръ, милостивъ, долготерпѣливъ и многомилостивъ, какъ жаждетъ нашего спасенія? Во Хрістѣ, единородномъ Сынѣ Его, дознаемъ. Сіе тебѣ о Спасителѣ нашемъ предлагаю, хрістіанине, не на такій конецъ, дабы хвалить Его, "--- высшій Онъ есть нашей хвалы, Котораго хвалятъ небесныя силы, и солнце не требуетъ доказательства къ своему свѣту, "--- но чтобы представить тебѣ Божественный Его и преблагій нравъ, который самъ собою влечетъ и привлекать долженъ сердца наша къ любленію и подражанію Его, и тако бы, оставивши ветхаго человѣка мерзость, послѣдовать Ему возмоглъ ты вѣрою и любовію. На сію убо Божественную доброту должно вѣрою взирать намъ, и, какъ въ зерцало взирая, отираемъ на лицѣ пороки, смотря на тую, омывать грѣхи и пороки, къ душамъ нашимъ прилѣпшіе, покаяніемъ, отверженіемъ своего злонравія и тако съ помощію Его обновляться въ новаго человѣка. Не токмо убо душеспасительно, какъ выше сказано, но и сладко послѣдовать Хрісту, преблагому, милостивому, милосердому, смиренному, долготерпѣливому, кроткому, и всѣхъ небесныхъ добродѣтелей сокровищу и живому образу. Такъ, отречемся же своего злонравія, яко смрадныя вони, и сему Божественному и пресладкому благоуханію послѣдуемъ. \textit{Воня бо мѵра Его паче всѣхъ ароматъ; мѵро изліянное имя Его. Влецы насъ въ слѣдъ Тебе: въ воню мѵра Твоего потечемъ}, сладчайшій Іисусе\footnote{Пѣсн. пѣсн.~1,~2 и 3.}. "--- 4)~Хрістосъ есть глава церкви: сынове церкви вѣрные уды суть Его. Убо подобаетъ намъ, аще истинные уды Его хощемъ быть, неотлучными быть отъ Него, но вѣрою и любовію Ему соединенными, и, что дѣлаетъ *Онъ, дѣлать и намъ, и куды идетъ Онъ*, за Нимъ слѣдовать и намъ; уды бо отъ главы своей не отлучены суть, якоже видимъ въ вещественномъ тѣлѣ. Хрістосъ есть вождь вѣрный и мудрый, Который по пути міра сего ведетъ послѣдующихъ Ему въ небесное отечество: убо должны мы послѣдовать Ему, когда не хощемъ заблудить въ путь погибели. Хрістосъ есть пастырь добрый: убо должно намъ гласа Его слушать, которымъ Онъ въ слѣдъ Себе идти призываетъ насъ, и послѣдовать Ему: \textit{аще кто Мнѣ служитъ, Мнѣ да послѣдствуетъ}\footnote{Іоан.~12,~26.}. Хрістосъ есть учитель и наставникъ нашъ, Который учитъ и наставляетъ насъ на путь истины: убо должно намъ учитися отъ Него, чему Онъ и словомъ и образомъ непорочнаго житія учитъ насъ, и въ словеси Его пребывати, аще хощемъ ученики Его быти, якоже Самъ рече: \textit{аще вы пребудете во словеси Моемъ, воистину ученицы Мои будете}\footnote{Іоан.~8,~31.}. Хрістосъ есть свѣтъ міру\footnote{ст.~12.}: убо когда не хощемъ во тьмѣ быть, должны Свѣту сему послѣдовать, да тьма насъ не обыметъ\footnote{12,~35.}. Хрістосъ есть женихъ вѣрныхъ душъ: убо вѣрнымъ должно вѣрою и любовію Ему прилѣплятися, творити волю Его. Хрістосъ есть истина и животъ\footnote{14,~6.}: убо должно намъ придержатися Его, когда не хощемъ прельститися и умереть. "--- Отъ реченныхъ послѣдуетъ, что аще Хрістосъ есть глава вѣрныхъ, то не суть уды Его, которые вѣрою и любовію Ему не соединены. Не соединены же Ему, яко не послѣдуютъ Ему. Не послѣдуютъ Ему, яко смиренію, терпѣнію, кротости и прочіимъ Божественнымъ нравамъ не подражаютъ. Аще Хрістосъ есть вождь къ небеси, то неотмѣнно заблуждаютъ тѣ, которые Ему не послѣдуютъ. Аще Хрістосъ есть пастырь нашъ, то не суть овцы Его, которыя не слушаютъ гласа Его: \textit{овцы бо Мои}, глаголетъ Онъ, \textit{гласа Моего слушаютъ, и Азъ знаю ихъ, и по Мнѣ грядутъ}\footnote{10,~17.}. Аще Хрістосъ есть учитель и наставникъ нашъ, убо не суть ученицы Его, которые не учатся отъ Него смиренію, терпѣнію, кротости, и проч. Аще Хрістосъ есть свѣтъ, то непремѣнно во тьмѣ ходятъ, которые не ходятъ въ слѣдъ Его; надобно бо во тьмѣ быть неотмѣнно, кто удаляется отъ свѣта. Аще Хрістосъ есть женихъ вѣрнымъ душамъ, убо прелюбодѣйствуютъ отъ Него, которые удаляются отъ Него. Аще Хрістосъ есть истина, неотмѣнно прельщаются и заблуждаютъ, которые не придерживаются Его. Аще Хрістосъ есть животъ, то мертвы суть, которые отлучаются и удаляются отъ Него, якоже глаголетъ: \textit{се удаляющіи себе отъ Тебе погибнутъ}\footnote{Пс.~72,~17.}. Видиши, возлюбленный хрістіанине, какъ нужно намъ послѣдовати Хрісту, такъ что \textit{едино} сіе \textit{есть на потребу}\footnote{Лук.~10,~41.}; напротивъ того, какъ опасно удаляться отъ Него, у Котораго единаго есть животъ и блаженство. Оттрясемъ убо лѣность, уныніе и тягость грѣховную, и потецемъ въ слѣдъ Его, да приведетъ насъ ко Отцу Своему небесному \textit{зрѣти лицемъ къ лицу} доброту Его святѣйшую\footnote{1~Кор.~13,~11.}. Безъ Него бо не можемъ туды пріити, якоже глаголетъ: \textit{никтоже пріидетъ ко Отцу, токмо Мною}\footnote{Іоан.~14,~6.}. "--- 5)~Требуетъ того отъ насъ Хрістова любовь, да любовію послѣдуемъ Ему, возлюбившему насъ. Любя насъ, Онъ сошелъ съ небесъ и пришелъ въ міръ; любя насъ, высокій смирился, Сынъ Божій сыномъ человѣческимъ нарицатися и быть благоволилъ, Богъ во плоти явитися и на земли пожити, Царь и Господь нашъ къ намъ подлымъ, бѣднымъ, окаяннымъ и отверженнымъ рабамъ пріити и въ рабскомъ зракѣ бесѣдовати, и образъ смиренія, терпѣнія, кротости и любве намъ показати, крестъ понести и на крестѣ распятися насъ ради изволилъ. Сія Его Божественная и горячайшая любовь да привлечетъ насъ въ слѣдъ Его. \textit{Хрістосъ пострада за насъ, намъ оставль образъ, да послѣдуемъ стопамъ Его}\footnote{1~Петр.~2,~11.}. \textit{Внѣ вратъ пострадати изволилъ} насъ ради. \textit{Тѣмже убо да исходимъ къ Нему внѣ стана, поношеніе Его носяще}\footnote{Евр.~13,~12 и 13.}. То"=есть, изыдемъ внѣ міра сего, не тѣломъ и ногами, но сердцами нашими; оставимъ позади насъ похоть плотскую, похоть очесъ и гордость житейскую, яко сродниковъ своихъ, и \textit{терпѣніемъ да течемъ на предлежащій намъ подвигъ, взирающе на Начальника вѣры и Совершителя Іисуса}\footnote{Евр.~12,~1 и 2.}. Тому слава, честь и благодареніе буди, со Отцемъ и Святымъ Духомъ во вѣки. Аминь.

\begin{center}\textbf{Краткое изъясненіе статьи сея.}\end{center}

1) Вышеписанныя ко Хрісту Сыну Божію должности хрістіанскія связаны суть между собою, и начало имѣютъ отъ вѣры, и едина отъ другой слѣдуетъ. Аще бо кто сердечно вѣруетъ во Хріста, сей и любитъ Хріста; аще кто любитъ Хріста, сей почитаетъ спасительное Его смотрѣніе, рожденіе, страданіе и пострадавшаго; аще кто почитаетъ Хріста, сей усердно работаетъ Ему; кто усердно работаетъ Ему, сей исполняетъ слово Его, и святому житію Его послѣдовать тщится. Ибо вѣра есть корень любве, любовь же есть источникъ благихъ дѣлъ, которыми Хріста почитаемъ и послѣдуемъ Ему. 2)~Должность хрістіанская ко Хрісту Сыну Божію связана съ должностію къ Богу. Ибо Богъ повелѣлъ намъ во Хріста, Сына Его единороднаго, вѣровать, любить Его, почитать, работать и послѣдовать Ему: \textit{Сей есть Сынъ Мой возлюбленный, о Немже благоволихъ: Того послушайте}\footnote{Мѳ.~17,~5.}. И Хрістосъ глаголетъ: \textit{се есть дѣло Божіе, да вѣруете въ Того, Егоже посла Онъ} (Отецъ)\footnote{Іоан.~6,~29.}. Чему бо учитъ насъ Хрістосъ Сынъ Божій, тое и Отецъ Его небесный глаголетъ намъ, и велитъ слушать Его: \textit{Того послушайте!} И Богъ въ Сынѣ Своемъ глаголалъ намъ, якоже написалъ Апостолъ: \textit{многочастнѣ и многообразнѣ древле Богъ глаголавый отцемъ во пророцѣхъ, въ послѣдокъ дній сихъ глагола намъ въ Сынѣ}\footnote{Евр.~1,~1 и 2.}. Аще убо кто должность исполняетъ ко Хрісту, сей исполняетъ должность и къ Богу; а кто оставляетъ должность ко Хрісту, сей оставляетъ должность и къ Богу. \textit{Иже бо не чтитъ Сына, не чтитъ Отца, пославшаго Его}\footnote{Іоан.~5,~23.}. Не можетъ вѣровати въ Бога, кто не вѣруетъ во Хріста Сына Божія. Не можетъ любить, почитать, работать Богу, кто не любитъ, не почитаетъ и не работаетъ Хрісту. И тако человѣкъ, не имѣющій вѣры въ Бога, есть безбожный, хотя устами исповѣдуетъ Бога; въ сердцѣ бо у человѣка должно быть богопочитаніе. 3)~Должность сія ко Хрісту Сыну Божію разумѣется, поелику Онъ во плоти явился, страсти и смерть крестную претерпѣлъ насъ ради. Иначе ему и тая должность, которая есть къ Богу, принадлежитъ, ибо Хрістосъ есть Богъ, второе Лице Пресвятыя Троицы, со Отцемъ и Святымъ Духомъ покланяемый. 4)~Должности къ Богу и Хрісту Сыну Божію сами собою исполнять не можемъ, понеже \textit{плоть} всегда \textit{похотствуетъ на духа}\footnote{Гал.~5,~17.}. Того ради должно прилѣжно молиться о томъ, дабы Самъ Богъ помогалъ намъ неотступно Своею благодатію. 5)~Кто не тщится должности Хрісту Сыну Божію, яко Искупителю своему, Избавителю и Спасителю, со Отцемъ и Святымъ Духомъ купно отдавать, неблагодаренъ есть, и ходитъ въ слѣпотѣ и заблужденіи, и неотмѣнно имѣетъ погибнуть, когда не очувствуется и не обратится къ Богу. \textit{Емуже бо нѣсть сихъ, слѣпъ есть, мжай забвеніе пріемъ очищенія древнихъ своихъ грѣховъ}\footnote{2~Петр.~1,~9.}.


\section[Статья 5-я. О должности хрістіанской къ самому себѣ.]{статья пятая.\\\bfseries О должности хрістіанской къ самому себѣ.}

Должность хрістіанина къ самому себѣ въ томъ состоитъ, чтобы онъ въ подвигѣ вѣры не ослабѣвалъ съ помощію Божіею, и тако бы спасеніе вѣчное, которое Хрістосъ смертію Своею намъ пріобрѣлъ, получилъ. Аще бо и туне, единою благодатію получаютъ вѣрные животъ вѣчный, по Писанію: \textit{дарованіе Божіе животъ вѣчный о Хрістѣ Іисусѣ Господѣ нашемъ}\footnote{Римл.~6,~23.}; однакожь требуется непрестанный подвигъ противу діавола, міра, плоти, которые хрістіанамъ препятствіе чинятъ, и не допущаютъ ихъ до вѣчнаго блаженства. Потребно много труда противу сихъ враговъ хотящему спастися. Хрістіанство бо какъ есть великая и высокая вещь, такъ многому искушенію и бѣдствію подвержено. Сего ради недремлющимъ окомъ должно его хранить.

\subsection[Глава 1-я. О храненіи вѣры.]{глава первая.\\\bfseries О храненіи вѣры.}

\begin{quotation}\textit{Буди вѣренъ даже до смерти, и дамъ ти вѣнецъ живота}, глаголетъ Хрістосъ\footnote{Апок.~2,~10.}.\end{quotation}

\paragraph*{§\:359.} Понеже вѣра многоразличному искушенію подлежитъ, то должно ее посредѣ толикаго бѣдствія такъ берещи и хранить, какъ душу свою или животъ свой, или паче болѣе. Ибо и животъ нашъ должны мы въ случаѣ пренебрещи, дабы вѣра сохранилася. Аще бо душу нашу или животъ вѣры ради погубимъ, то обрящемъ животъ вѣчный въ будущемъ вѣцѣ, якоже глаголетъ Хрістосъ: \textit{иже погубитъ душу Свою Мене ради и Евангелія, той спасетъ ю}\footnote{Марк.~8,~35.}. Въ вѣрѣ бо все блаженство наше состоитъ. Вѣрою во Хріста избавляемся отъ грѣховъ нашихъ, гнѣва, клятвы, власти діавольскія, вѣчнаго осужденія, ада и вѣчныя муки, и сподобляемся Божія милости, благодати, благословенія, царствія Божія и всѣхъ вѣчныхъ благъ, по неложному Его обѣщанію. Аще убо вѣру соблюдемъ, вся сія благая наша будутъ; аще же вѣру погубимъ, всѣхъ тѣхъ благъ лишимся, и паки подпадемъ всему бѣдствію, гнѣву Божію, клятвѣ, власти темной, вѣчному осужденію и мученію. Хранить тщимся отъ татей сокровище наше тлѣнное и временное; кольми паче должно хранить сокровище душевное, вѣчное, вѣру святую, дабы его сатана, какъ тать хитрый и лукавый, не похитилъ. Бережемъ здравіе тѣлесное и животъ; много болѣе здравіе и животъ душевный берещи должно, да не во вѣки умремъ. Предостерегаемъ себе отъ враговъ, которые тѣло наше тлѣнное погубить и временный животъ отнять у насъ хотятъ; кольми паче отъ враговъ душевныхъ предостерегаться должно, которые душу нашу погубить и вѣчный животъ отнять у насъ стараются. Къ сей осторожности и тщанію увѣщаваетъ насъ Апостолъ Петръ: \textit{трезвитеся, бодрствуйте: зане супостатъ вашъ діаволъ, яко левъ рыкая, ходитъ, искій кого поглотити. Ему противитеся тверди вѣрою}\footnote{1~Петр.~5,~8 и 9.}. Вѣра сохраняется молитвою, чтеніемъ или слушаніемъ слова Божія и проч.

\paragraph*{§\:360.} Вѣра истинная и живая безъ добрыхъ дѣлъ быть не можетъ. \textit{Всякое бо древо доброе плоды добры творитъ, а древо злое плоды злы творитъ. Не можетъ древо доброе плоды злы творити, ни древо зло плоды добры творити}, глаголетъ Хрістосъ\footnote{Мѳ.~7,~17 и 18.}. Тако вѣрное сердце, вѣру, какъ доброе сѣмя, въ себѣ имѣющее, плоды добрыхъ дѣлъ творитъ; всякое бо сѣмя подобный себѣ плодъ раждаетъ. Того ради должно намъ, хрістіанине, вѣру нашу показать отъ добрыхъ дѣлъ, якоже древо доброе отъ добрыхъ плодовъ показуется, какъ Апостолъ глаголетъ: \textit{покажи ми вѣру твою отъ дѣлъ твоихъ}\footnote{Іак.~2,~18.}. Тогожде требуютъ отъ насъ и прочіи апостоли святіи въ посланіяхъ своихъ, и толкователи Писанія отцы святіи, святители Хрістовы и учители. Вѣра и любовь, которая есть \textit{матерь добрыхъ дѣлъ}\footnote{1~Кор.~13,~1--8.}, такъ сопряжены между собою, какъ душа съ тѣломъ, сокъ съ древомъ суровымъ, теплота съ огнемъ. Живетъ тѣло, когда душа въ немъ есть; стоитъ и живетъ древо и плодъ творитъ, пока сокъ въ немъ имѣется; находится тамо огнь, гдѣ теплота чувствуется: тако имѣется тамо вѣра, гдѣ любовь истинная оказываетъ себе. «Вѣры, глаголетъ Златоустъ, знаменіе есть любовь»\footnote{Бес.~4"~я на 1"~е посл. къ Сол.}. Нѣтъ тамо вѣры, гдѣ нѣтъ любви. Якоже бо тѣло человѣческое мертвое видится быть человѣкъ, но въ самой вещи трупъ бездушный есть, а не человѣкъ; и древо изсохшее показуется быть древо, но въ самой вещи видъ только древа, а не древо есть, древо бо безъ соку быть не можетъ; и печь негрѣтая показуется согрѣвати покой, но не грѣетъ, яко огня въ себѣ не имѣетъ, и потому теплоты не издаетъ: тако человѣкъ показуется быть вѣрнымъ, яко Бога и Хріста Сына Божія устами исповѣдуетъ, въ церковь ходитъ и прочіе знаки хрістіанства показуетъ; но ежели любви и добрыхъ дѣлъ не имѣетъ, то видъ только вѣрнаго человѣка и хрістіанина показуетъ, а не истинно вѣрный и хрістіанинъ есть, и въ самой вещи тоежде, что и язычникъ, неисповѣдующій Бога. Не едино бо исповѣданіе дѣлаетъ хрістіанина, но исповѣданіе съ сердечною вѣрою сопряженное. Вѣра учитъ вѣрнаго исповѣдывать и нарицать Бога \textit{своимъ Богомъ}, якоже о семъ на многихъ Писанія мѣстахъ свидѣтельствуется. Аще убо вѣруешь, что Богъ твой есть Богъ, то неотмѣнно тебѣ Его должно почитать, что не можетъ быть безъ послушанія истиннаго. Почитать бо Бога и не слушать Его разуму невмѣстимо есть. А отъ послушанія послѣдуетъ соблюденіе святыхъ Его заповѣдей; а гдѣ соблюденіе заповѣдей Божіихъ, тамо и добрыя дѣла: заповѣди бо научаютъ насъ добрымъ дѣламъ. Вѣруешь, что Богъ есть \textit{прибѣжище, сила, помощникъ и заступникъ} вѣрныхъ Своихъ\footnote{Пс.~45,~2.}: почтожъ надежду твою полагаешь на человѣка смертнаго, на сребро, злато, честь, санъ твой, силу твою и мудрость, и отъ нихъ въ нашедшей напасти помощи и заступленія ищешь? Вѣруешь, что Хрістосъ мученъ былъ какъ за всѣхъ, такъ и за твои грѣхи, "--- \textit{Той бо язвенъ бысть за грѣхи наша, и мученъ бысть за беззаконія наша}\footnote{Ис.~53,~5.}: почтожъ дерзаешь безстрашно грѣшить? За что Хрістосъ такъ горькую страданія чашу испилъ? Хрісту грѣхи твои толикую горесть и мученіе содѣлали, а ты ихъ не оставляешь, но утѣшаешися ими. Утѣшаешися же, понеже не оставляеши; не оставляеши, понеже услаждаешися ими. Услаждаешися же тѣмъ, что Хрісту безчестіе, поношеніе, хулу, болѣзнь, скорбь, оплеваніе, заушеніе, уязвленіе, раны, распятіе и смерть содѣлало; тое тебѣ пріятно и сладко, что Хрісту горестно было. Видишь, человѣче, хрістіаниномъ нарицающійся, какая твоя любовь ко Хрісту, пострадавшему за грѣхи твоя, что тое любишь, что Хрісту горчайшее страданіе учинило! А какая любовь, такая и вѣра твоя; вѣра бо и любовь неразлучны суть. Вѣруешь, что Онъ есть \textit{Пастырь} твой, почтожъ не слушаешь гласа Его? \textit{Овцы моя}, глаголетъ Онъ, \textit{гласа Моего слушаютъ}\footnote{Іоан.~10,~27.}. Вѣруешь, что Онъ Учитель твой есть: почтожъ яко ученикъ, не учишися отъ Него смиренію, терпѣнію, кротости и прочіимъ добродѣтелямъ, которымъ Онъ словомъ и образомъ училъ? Научаетъ тебе вѣра святая, что хрістіане суть братія твоя, единаго Отца имѣющіи Бога, едину матерь признающіи церковь, едино крещеніе, которымъ отъ сквернъ грѣховныхъ омылися, едину вѣру "--- сродство духовное получившіи: почтожъ не любишь ихъ, яко братію свою, и требующимъ милости не дѣлаешь милости, и въ нуждахъ ихъ не помогаешь имъ? Братію свою по плоти любишь и помогаешь имъ, кольми паче братію по духу должно любить. Ибо большее есть сродство духовное, хрістіанское, яко вѣчное, нежели плотское, яко временное и смертію престающее. Такая"=то твоя вѣра есть! Вѣруешь, но противно вѣрѣ своей поступаешь. Исповѣдуешь Бога, но непослушаніемъ отмещешися Его. Хріста Сына Божія пріемлешь, яко глаголешь Ему: \textit{Господи, Господи}; но ученія Его не пріемлеши. Между хрістіанами считаешися, но хрістіанскихъ дѣлъ не твориши, паче же противная хрістіанамъ дѣлаеши. Хрістосъ глаголетъ Іудеомъ: \textit{аще чада Авраамля бысте были, дѣла Авраамля бысте творили}\footnote{8,~39.}. Тоежде и хрістіанину неисправному прилично сказать: аще бы хрістіанинъ ты былъ, дѣла бы хрістіанскія творилъ. \textit{Иже бо Хрістовы суть, плоть распяша со страстьми и похотьми}\footnote{Гал.~5,~24.}. \textit{Аще ли кто Духа Хрістова не имать, сей нѣсть Его}\footnote{Римл.~8,~9.}. Не единымъ исповѣданіемъ, именемъ и церемоніями хрістіанинъ разнствуетъ отъ язычника, но вѣрою сердечною и житіемъ хрістіанскимъ. Видишь убо, хрістіанине, что какъ благочестивое житіе есть доказательство живущія вѣры въ человѣкѣ, такъ неисправное "--- знаменіе безсумнѣнное есть невѣрія, хотя бы человѣкъ Бога исповѣдалъ, крещенъ былъ и хрістіаниномъ назывался. А отъ сего можешь искушать себе, храниши ли вѣру, якоже Апостолъ увѣщаваетъ: \textit{себе искушайте, аще есте въ вѣрѣ, себе искушайте}\footnote{2~Кор.~13,~5.}.

\paragraph*{§\:361.} Вѣра святая не токмо устами отвергается, когда человѣкъ Бога и Хріста Сына Божія отрицается, и обращается къ чувственному идолослуженію, но и злыми дѣлами, не"=хрістіанскимъ и беззаконнымъ житіемъ. О семъ слово Божіе свидѣтельствуетъ. \textit{Бога исповѣдуютъ}, глаголетъ Апостолъ, \textit{вѣдѣти, а дѣлы отмещутся Его, мерзцы суще и непокориви, и на всяко дѣло благое неискусни}\footnote{Тит.~1,~16.}. И лихоиманіе \textit{идолослуженіемъ}\footnote{Кол.~3,~5.}, и лихоимца \textit{идолослужителемъ}\footnote{Еф.~6,~5.} тойжде Апостолъ называетъ. И: \textit{отринувшіи благую совѣсть отъ вѣры отпадоша}\footnote{1~Тим.~1,~19.}. И: \textit{о своихъ, паче же о присныхъ не промышляяй, вѣры отверглся есть, и невѣрнаго горшій есть}\footnote{6,~8.}. И Апостолъ Іаковъ глаголетъ: \textit{аще кто мнится вѣренъ быти въ васъ, и не обуздаваетъ языка своего, но льститъ сердце свое, сего суетна есть вѣра}\footnote{Іак.~1,~26.}. И паки: \textit{кая польза, братіе моя, аще вѣру глаголетъ кто имѣти, дѣлъ же не имать? еда можетъ вѣра спасти его}\footnote{2,~14.}? И ниже въ тойже главѣ: \textit{покажи ми вѣру твою отъ дѣлъ твоихъ}\footnote{ст.~2,~18.}. И паки: \textit{якоже тѣло безъ духа мертво есть, тако и вѣра безъ дѣлъ мертва есть}\footnote{26.}. А когда вѣра мертва, то и нѣтъ ея. Мертвое бо уже не живетъ, хотя и показуется нѣчто быти. Ибо вѣра есть даръ Божій духовный, и потому когда мертва есть, то уже исчезла, якоже свѣтильникъ безъ елея исчезаетъ. Сего ради какъ въ свѣтильникъ подливается елей, чтобы не угаслъ; тако къ свѣтильнику вѣры нуженъ есть елей милости Божіей, безъ котораго онъ угасаетъ. Вѣра бо отъ Бога приходитъ, того ради Божіею благодатію и силою сохраняется. Сего ради глаголетъ Давидъ: \textit{Ты просвѣтиши свѣтильникъ Мой, Господи Боже мой, просвѣтиши тму мою}\footnote{Пс.~17,~29.}. Однакожъ требуется и наше тщаніе къ тому. Богъ бо помогаетъ труждающимся, а не лежащимъ, дѣлающимъ, а не престающимъ. Сего ради глаголетъ: \textit{востани спяй, и воскресни отъ мертвыхъ, и освѣтитъ тя Хрістосъ}\footnote{Еф.~5,~14.}. Надобно первѣе востати и начать дѣлати, и тогда помощь приспѣетъ. А которые оставляютъ сіе благословенное тщаніе и трудъ, сами себѣ виновны бываютъ, что попущаютъ свѣтильнику вѣры своея угасати. Я здѣ не говорю тебѣ, хрістіанине, о той вѣрѣ, которая въ познаніи и содержаніи правыхъ догматовъ состоитъ: сію вѣру имѣютъ и беззаконно живущіи хрістіане. Но говорю о вѣрѣ сердечной, живой и спасающей (о которой сказано выше): сей вѣры нѣтъ въ томъ, кто неисправно и не по"=хрістіански живетъ, хотя бы и исповѣдывалъ и проповѣдывалъ вѣру. Возми въ разсужденіе, что Хрістосъ глаголетъ: \textit{не можете Богу работати и мамонѣ}\footnote{Матѳ.~6,~24.}. Какъ работающіи Богу уже не работаютъ мамонѣ, такъ работающіи мамонѣ уже не работаютъ Богу. Богу бо и мамонѣ вдругъ работати невозможно, по словеси Хрістову. А отъ сего заключается, что и всякому грѣху работающіи уже не работаютъ Богу; а не работая Богу, и вѣры не имѣютъ, яко безъ вѣры работати Богу невозможно. \textit{Безъ вѣры бо невозможно угодити Богу}, учитъ Апостолъ\footnote{Евр.~11,~6.}. Работати и угождати Богу есть едино. Внимай и сему, хрістіанине: вѣрнымъ о Хрістѣ отверсты двери царствія небеснаго, якоже Божіе святое слово увѣряетъ насъ, и церковь святая содержитъ и исповѣдуетъ; но напротивъ того Апостолъ заключаетъ входъ въ царствіе небесное неисправнымъ хрістіанамъ. \textit{Или не вѣсте, яко неправедницы царствія Божія не наслѣдятъ?} и проч.\footnote{1~Кор.~6,~9 и 10; Гал.~5,~19--21 и на прочіихъ мѣстахъ.} И Хрістосъ глаголетъ: \textit{не всякъ глаголяй Ми, Господи, Господи, внидетъ въ царствіе небесное, но творяй волю Отца Моего, Иже есть на небесѣхъ}\footnote{Матѳ.~7,~21.}. Слѣдственно и вѣры не имѣютъ тіи, которые творятъ таковыя дѣла, которыми заключается входъ въ царство и блаженство вѣчное. Внимай паки, хрістіанине, и сему, что Богъ, Который глаголетъ мнѣ и тебѣ: \textit{Азъ есмь Господь Богъ твой}, и повелѣваетъ иныхъ боговъ не имѣти: \textit{да не будутъ тебѣ бози иніи, развѣ Мене}\footnote{Исх.~20,~2 и 3.}, "--- тойжде Богъ повелѣваетъ и въ слѣдующихъ заповѣдяхъ. Слѣдственно какъ первую заповѣдь разоряя, грѣшимъ предъ Богомъ, такъ и прочія не соблюдая, равно грѣшимъ; и какъ тогда, когда оставляя Его, къ инымъ богамъ обращаемся, отрекаемся вѣры и Бога, такъ и другія слѣдующія заповѣди разоряя, отрицаемся вѣры и Бога, хотя не устами, но дѣломъ и непослушаніемъ. Едино бо есть словомъ или дѣломъ отрицаться. Ибо и вѣра должна быть въ сердцѣ, и внѣ дѣлами себе оказывать; которой сердечной вѣрѣ устное исповѣданіе послѣдуетъ и объявляетъ оную, какъ сердечную всякую мысль слово произносимое, по писанному: \textit{сердцемъ вѣруется въ правду, усты же исповѣдуется во спасеніе}\footnote{Римл.~10,~10.}. И потому отверженіе, словомъ или дѣломъ бываемое, равно есть отверженіе. Отвергается бо и самаго повелѣвающаго, кто отвергается его повелѣнія; и хотя языкомъ не дерзаетъ произносити: не знаю и не слушаю тебе, но сердцемъ непокоривымъ и преслушливымъ и самымъ дѣломъ показуетъ тое. Тако заповѣдь Божію разоряя отъ произволенія и предразсужденія, человѣкъ и Самаго Бога отрицается, и тако отступаетъ отъ Него, и падаетъ въ духовное идолопоклонство. Страсти бо, которую совершаетъ, какъ идолу, сердце и волю свою покоряетъ. Разсуди всякъ, не тоежде ли дѣлается со всякимъ хрістіаниномъ, что дѣлалось во время гоненія съ мучениками? Ихъ мучители и слуги ихъ убѣждали къ идолопоклоненію, и которые не соизволяли на тое богопротивное дѣло, сохраняли вѣру, а которые повиновалися, погубляли вѣру: такъ и всякаго хрістіанина убѣждаетъ злая мысль, какъ слуга діавольскій, ко грѣху, отъ чего и совѣсть и Самъ Богъ чрезъ Слово Свое отвращаетъ. Кто злому помыслу, какъ мучителю, противится, и на богопротивное дѣло не соизволяетъ, "--- хранитъ вѣру къ Богу своему, и вѣрно работаетъ Господу своему; но кто тому соизволяетъ, оставляя совѣсть, и Божіе повелѣніе презирая, "--- отъ Бога повелѣвающаго отступаетъ, и тако вѣру погубляетъ. Мученики, которые не стерпѣли мученія, отвергалися Хріста устами: а сей, который богопротивному помыслу повинуется и беззаконнуетъ, сердцемъ отвергается. \textit{Идолъ бо внѣ ничтоже есть}\footnote{1~Кор.~8,~4.}. Сребро, злато и прочая матерія, въ идолѣхъ находящаяся, Божіе есть созданіе, которое и на тое и на другое дѣло по человѣческому соизволенію можетъ обратиться. Но идоли у всякаго въ сердцѣ имѣются "--- страсти безчинніи: похоть нечистая, гордость, зависть, злоба, сребролюбіе и проч. Внутрь имѣются и мучители, къ симъ страстямъ, какъ идоламъ, убѣждающіи, "--- помыслы злые. Внутрь и капище у человѣка есть "--- сердце его. Отъ внѣшнихъ идоловъ удобно можно сохранитися; но отъ сихъ, которыхъ внутрь насъ носимъ, съ великою трудностію, и то не безъ всесильной Божіей помощи сохраняемся и избавляемся, ибо природны намъ суть. Аще убо хощемъ, хрістіанине, вѣру нашу сохранить, должно намъ не токмо отъ внѣшнихъ, но и отъ внутреннихъ идоловъ хранить себе; не токмо златымъ, сребрянымъ и прочіимъ дѣламъ рукъ человѣческихъ колѣна не преклонять, но и внутреннимъ, съ нами, отъ ветхаго человѣка рожденнымъ, то"=есть, мамонѣ, похоти нечистой, гнѣву, злобѣ, зависти, ненависти, гордости и прочіимъ не покоряться. Покоряется же имъ тотъ, кто работаетъ имъ, а работаетъ имъ, кто дѣлаетъ по ихъ похоти. Откуду и называется таковый рабомъ грѣха, якоже глаголетъ Хрістосъ: \textit{всякъ, творяй грѣхъ, рабъ есть грѣха}\footnote{Іоан.~8,~34.}. Кто похоть нечистоты совершаетъ, рабъ нечистоты есть; кто гнѣву и злобѣ повинуется, работаетъ гнѣву и рабъ гнѣва есть. Тоежъ и о прочіихъ разумѣй. И тако столько идоловъ внутрь себе почитаетъ человѣкъ, сколько плотскихъ похотей исполняетъ. Богъ бо въ словѣ Своемъ глаголетъ не къ тѣлу нашему и удамъ тѣлеснымъ; не къ тѣлу глаголетъ: \textit{да не будутъ тебѣ бози иніи развѣ Мене}; не къ рукамъ нашимъ глаголетъ: \textit{не убій, не укради}, но къ душѣ нашей. Душа убо или имѣетъ, или не имѣетъ иныхъ боговъ, развѣ Господа. Душа убиваетъ, или не убиваетъ; крадетъ, или не крадетъ; прелюбодѣйствуетъ, или не прелюбодѣйствуетъ; а тѣло и уды тѣлесныя суть орудія, чрезъ которыя душа начинанія и дѣйствія свои совершаетъ. И тако, хотя не преклоняеши колѣнъ языческимъ богамъ, не простираешь рукъ имъ, не приносишь кадила; но когда волю бѣсовскую, въ которой чрезъ страсти наши насъ убѣждаетъ, исполняешь, столько иныхъ боговъ почитаешь, сколько страстей совершаешь. Ибо и язычники безчинныя страсти человѣческія обоготворяли, и тѣхъ какъ боговъ почитали, и бѣсовъ волю въ идолахъ живущихъ исполняли, какъ читаемъ въ исторіяхъ. Кто же сію волю исполняетъ, тому работаетъ, служитъ и того почитаетъ. Аще убо страстямъ своимъ человѣкъ повинуется, то и работаетъ имъ; аще работаетъ имъ, то какъ господей своихъ имѣетъ ихъ, а тако и почитаетъ ихъ, какъ идоловъ. Не токмо убо отъ внѣшнихъ, сребряныхъ, златыхъ, мѣдныхъ, деревянныхъ и изъ прочіихъ матерій содѣланныхъ идоловъ, но и отъ сердечныхъ берещися должны мы, хрістіанине; когда хощемъ вѣру нашу соблюсти Господу нашему. Сіе значитъ, когда Богъ нашъ не велитъ намъ гнѣватися и злобитися на ближняго, не велитъ прелюбодѣйствовати и красти, и проч. Аще убо, не слушая Бога, страстямъ симъ повинуемся, тѣмъ самымъ отъ Бога отступаемъ, и къ симъ пагубнымъ божкамъ сердцами приступаемъ, и волю нашу вмѣсто жертвы и кадила приносимъ имъ. Ради понятія сего моего разсужденія разсуждай, простосердечный хрістіанине, сей примѣръ, который тебѣ предлагаю. Аще бы рабъ подъ вѣдомствомъ какого господина счислялся, но другому какому господину, противнику того господина, угождалъ, и хотѣнія его исполнялъ: призналъ бы ты, что тотъ лукавый рабъ внѣшнимъ только видомъ и именемъ онаго господина, у котораго въ вѣдомствѣ находится, имѣется рабомъ, а самою вещію сего, которому угождаетъ, истинный рабъ есть. \textit{Никтоже бо можетъ двѣма господинома работати: любо единаго возлюбитъ, а другаго возненавидитъ; или единаго держится, о друзѣмъ же нерадити начнетъ}, учитъ Хрістосъ\footnote{Матѳ.~6,~24.}. Кому бо человѣкъ угождаетъ, тому и работаетъ; а кому работаетъ, того и рабъ есть. Тако имѣются и хрістіане неисправные. Они счисляются между вѣрными Хрістовыми рабами, и Хріста нарицаютъ Господемъ своимъ: \textit{Господи, Господи}; въ церковь ходятъ, молятся, поютъ и покланяются; но волю Его не хотятъ творить, а исполняютъ волю и хотѣніе противника Его, діавола, и тому въ страстяхъ и прихотяхъ своихъ угождаютъ, и тако наружностію только единою имѣются хрістіанами и рабами Хрістовыми, но въ самомъ дѣлѣ суть раби страстей своихъ, грѣха и діавола, которымъ угождаютъ. \textit{Всякъ бо, творяй грѣхъ, рабъ есть грѣха}, глаголетъ Хрістосъ\footnote{Іоан.~8,~34.}. Таковые хрістіане суть ложные хрістіане, лицемѣры и лукавые раби, и въ самой вещи суть язычники; и имъ Хрістосъ ничего не пользуетъ, хотя и исповѣдуютъ имя Его. Къ таковымъ глаголетъ Богъ чрезъ пророка: \textit{приближаются Мнѣ людіе сіи усты своими, и устнами чтутъ Мя: сердце же ихъ далече отстоитъ отъ Мене}\footnote{Матѳ.~15,~8; Исх.~29,~13.}. И Хрістосъ отвѣчаетъ имъ: \textit{что Мя зовете: Господи, Господи, а не творите, яже глаголю}\footnote{Лук.~6,~46.}? Видишь убо хрістіанине, что Хрістосъ и вѣра святая не токмо устами, но и дѣлами отвергается. Сего ради должно всякому себе искушать, имѣетъ ли вѣру, по увѣщанію Апостола: \textit{себе искушайте, аще есте въ вѣрѣ, себе искушайте}\footnote{2~Кор.~13,~5.}.

\paragraph*{§\:362.} Слово Божіе увѣряетъ насъ, что всѣ умершіе, благочестивые и нечестивые, воскреснутъ изъ мертвыхъ: \textit{яко грядетъ часъ и нынѣ есть, егда мертвіи услышатъ гласъ Сына Божія, и услышавше оживутъ}\footnote{Іоан.~5,~25.}. И всѣмъ, \textit{воскресшимъ} и живымъ оставльшимся, \textit{явитися подобаетъ предъ судищемъ Хрістовымъ, да пріиметъ кійждо, яже съ тѣломъ содѣла, или блага, или зла}\footnote{2~Кор.~5,~10.}. Тогда вѣру соблюдшіи услышатъ отъ праведнаго Судіи: \textit{пріидите благословенніи Отца Моего, наслѣдуйте уготованное вамъ царствіе отъ сложенія міра}. Тогда Всевѣдецъ и Сердцевѣдецъ Судія плоды живыя вѣры ихъ, какъ сокровища, которыя они не на земли, но на небеси вѣрою сокрывали, открыетъ и въ явленіе всему міру покажетъ: \textit{взалкахся, и дасте Ми ясти; возжадахся, и напоисте Мя; страненъ бѣхъ, и введосте Мене; нагъ, и одѣясте Мя; боленъ, и посѣтисте Мене; въ темницѣ бѣхъ, и пріидосте ко Мнѣ}. Тогда услышатъ и окаянніи грѣшники отъ разгнѣваннаго Судіи, Котораго, въ мірѣ живучи, не хотѣли слушать, услышатъ страшный и ужасный гласъ: \textit{идите отъ Мене проклятіи въ огнь вѣчный, уготованный діаволу и аггеломъ его}. Учинитъ имъ и выговоръ за неправду, неблагодарность и невѣріе ихъ: \textit{взалкахся, и не дасте Ми ясти; возжадахся, и не напоисте Мене; страненъ бѣхъ, и не введосте Мене; нагъ, и не одѣясте Мене; боленъ, и въ темницѣ, и не посѣтисте Мене. И идутъ сіи въ муку вѣчную: праведницы же въ животъ вѣчный}\footnote{Матѳ.~25,~34--46.}. Идутъ вси въ свои мѣста; идутъ, но не равно идутъ, "--- одни съ плачемъ и рыданіемъ неутѣшнымъ, другіи съ радостію и веселіемъ неизреченнымъ. Тогда \textit{пойдутъ избавленніи, и собранніи Господемъ возвратятся, и пріидутъ въ Сіонъ съ радостію, и радость вѣчная надъ главою ихъ: надъ главою бо ихъ хвала и веселіе, и радость пріиметъ ихъ: отбѣже болѣзнь и печаль и воздыханіе}\footnote{Исх.~35,~9 и 10.}. Тогда \textit{возвратятся и грѣшницы во адъ, вси языцы забывающіи Бога}\footnote{Пс.~9,~18.}\textit{}, совершенное по дѣламъ своимъ воспріяти воздаяніе; и постигнетъ ихъ печаль, плачь, воздыханіе и мученіе безконечное. Якоже бо праведніи во царствіи небесномъ безъ конца будутъ царствовать, и лицезрѣніемъ Божіимъ утѣшаться, радоваться и веселиться, и со ангелами святыми хвалить и пѣть Бога: тако грѣшніи во адѣ будутъ безъ конца мучиться и страдать съ діаволомъ и аггелами его злыми. Тогда \textit{поищутъ они смерти}, которой нынѣ убѣгаютъ, \textit{и бѣжитъ отъ нихъ}\footnote{Апок.~9,~6.}.

\paragraph*{§\:363.} Извѣстно знаемъ отъ святаго Писанія, что Господь нашъ Іисусъ Хрістосъ пріидетъ судити всему міру, и воздати всѣмъ по дѣломъ ихъ, и будетъ пришествіе Его и судъ Его праведный въ концѣ міра и послѣднемъ дни, но въ которомъ году, мѣсяцѣ, дни и часѣ, не знаемъ. Сокрылъ отъ насъ Богъ тотъ день и часъ, да всегда его ожидаемъ, и готови будемъ къ срѣтенію Господа нашего. \textit{О дни томъ и часѣ никтоже вѣсть}, глаголетъ Хрістосъ. \textit{Якоже бо бысть во дни Ноевы, тако будетъ и пришествіе Сына человѣческаго. Якоже бо бѣху во дни прежде потопа, ядуще и піюще, женящеся и посягающе, до негоже дне вниде Ное въ ковчегъ, и не увѣдѣша, дондеже пріиде вода и взятъ вся: тако будетъ и пришествіе Сына человѣческаго}\footnote{Матѳ.~24,~36--39.}. Будутъ и въ оное время подобныя дѣла человѣческія: иной тое, иной другое будетъ дѣлать. Иной будетъ ясти, пити, веселитися, банкеты строить, гостей принимать, лики производить, и прочія суеты заводить; иной каяться, плакаться за грѣхи и молиться Богу будетъ. Иной грабить, похищать, обижать, иной терпѣть будетъ. Иной старыя житницы и домы разорятъ и новыя созидать, иной иное начинать и совершать будетъ. Въ такомъ различіи начинаній и дѣлъ человѣческихъ нечаянно услышится гласъ: «Востаните мертвіи, идите вси на судъ: Хрістосъ Судія идетъ судити живымъ и мертвымъ!» \textit{И тогда явится знаменіе Сына человѣческаго на небеси: и тогда восплачутся вся колѣна земная, и узрятъ Сына человѣческаго, грядуща на облацѣхъ небесныхъ съ силою и славою многою}\footnote{Матѳ.~24,~30.}. Блаженни раби тіи, ихже Господь пришедъ обрящетъ бдящихъ. Блажени суть, которые, по подобію дому владыки, бдятъ и хранятъ душевный домъ свой. Блажени суть, которые имѣютъ \textit{чресла препоясана и свѣтильницы горящія, подобяся человѣкомъ чающимъ Господа своего, когда возвратится отъ брака, да пришедшу и толкнувшу Ему, абіе отверзутъ Ему}\footnote{Лук.~12,~36,~35,~36--44.}. Блажени суть, которые, какъ мудрыя дѣвы, кои не зная, когда пріидетъ Женихъ, взяли елей въ сосудѣхъ своихъ, дабы свѣтильники свои въ пришествіе Жениха украсить, и тако изыти въ срѣтеніе Ему, всегда носятъ въ сердцахъ своихъ, какъ въ сосудѣхъ, горящую вѣру и любовь, и ожидаютъ Его со бдѣніемъ, терпѣніемъ и желаніемъ. Блажени суть, яко готовы изыдутъ въ срѣтеніе безсмертному Жениху своему, и \textit{съ Нимъ внидутъ на браки}\footnote{Матѳ.~25,~1--10.}, и \textit{тако всегда съ Господомъ будутъ}\footnote{1~Сол.~4,~17.}. Напротивъ того, окаянніи раби тіи, которые, по подобію Евангельскаго онаго злаго раба, \textit{глаголютъ въ сердцѣ своемъ: коснитъ Господь нашъ пріити; и начнутъ бити клевреты своя, ясти же и пити съ піяницами. Пріидетъ же Господь рабовъ тѣхъ въ день, въ оньже не чаютъ, и въ часъ, въ оньже не вѣдятъ; и протешетъ ихъ полма, и часть ихъ съ невѣрными положитъ: ту будетъ плачь и скрежетъ зубомъ}\footnote{Мѳ.~24,~48--51.}. Наипаче окаяннѣйшіи суть, которые съ ругателями, отъ Петра святаго пророчески описанными, \textit{по своихъ похотехъ ходяще, глаголютъ: гдѣ есть обѣтованіе пришествія Его? Отнелѣже бо отцы успоша, вся тако пребываютъ отъ начала созданія}. Которымъ отвѣщаетъ Апостолъ, между прочіими, тако глаголя: \textit{не коснитъ Господь обѣтованія, якоже нѣцыи коснѣніе мнятъ; но долготерпитъ на насъ, не хотя, да кто погибнетъ, но да вси въ покаяніе пріидутъ. Пріидетъ же день Господень, яко тать въ нощи, въ оньже небеса убо съ шумомъ мимо пойдутъ, стихіи же сжигаемы разорятся, земля же, и яже на ней дѣла сгорятъ}. И мало спустя увѣщаваетъ насъ: \textit{тѣмже возлюбленніи, сихъ чающе, потщитеся нескверни и непорочни Тому обрѣстися въ мірѣ; и Господа нашего долготерпѣніе спасеніе непщуйте}; и ниже: \textit{вы убо, возлюбленніи, предвѣдяще хранитеся, да не лестію беззаконныхъ сведени бывше, отпадете своего утвержденія; но да растете во благодати и разумѣ Господа нашего и Спаса Іисуса Хріста}\footnote{2~Петр.~3,~3,~4,~9,~10,~14,~15,~17 и 18.}. Увѣщаваетъ насъ и Самъ Господь нашъ ко бдѣнію сему, и всегда готовыми быть къ срѣтенію Его: \textit{бдите, яко не вѣсте, въ кій часъ Господь вашъ пріидетъ}. И паки: \textit{сего ради и вы будите готови, яко, въ оньже часъ не мните, Сынъ человѣческій пріидетъ}\footnote{Матѳ.~24,~42 и 44.}. О чемъ чрезъ всю 24 и 25"~ю главы Матѳея спасительное Его простирается поощреніе. Сего ради препояшемъ и мы, хрістіанине, чресла наша, и свѣтильники наша возжемъ по увѣщанію Хрістову, и не попустимъ имъ съ помощію Божіею угаснути, да готовы будемъ къ срѣтенію безсмертнаго и небеснаго Жениха нашего, и тако съ мудрыми дѣвами внидемъ въ чертогъ Его.

\subsection[Глава 2-я. О слушаніи и чтеніи Божія слова.]{глава вторая.\\\bfseries О слушаніи и чтеніи Божія слова.}

\begin{quotation}\textit{Всяко писаніе богодухновенно и полезно есть ко ученію, ко обличенію, ко исправленію, къ наказанію, еже въ правдѣ, да совершенъ будетъ Божій человѣкъ, на всякое дѣло благое уготованъ}\footnote{2~Тим.~3,~16 и 17.}.\end{quotation}

\paragraph*{§\:364.} Писаніе святое, пророками и апостолами намъ преданное, есть истинное Божіе слово, которымъ Отецъ небесный къ намъ недостойнымъ благоутробно бесѣдуетъ. Оно есть благопріятное Создателя нашего посланіе, которое чрезъ святыхъ и вѣрныхъ Своихъ слугъ, яко посланниковъ, съ небесе послалъ намъ, и въ немъ благоволеніе воли Своея открылъ. Аще убо слышиши, хрістіанине, служителя Божія, читающаго Евангеліе или другое Писаніе, или самъ, вземши въ руки книгу Писанія святаго, благоговѣйно читаеши, помышляй и разумѣй, что Богъ великій и непостижимый тебѣ, подлому рабу Своему, чрезъ слово Свое бесѣдуетъ. Какъ бо въ молитвѣ нашей мы къ Богу бесѣдуемъ, когда молимся Ему настоящимъ образомъ: такъ, когда слышимъ и читаемъ слово Его съ усердіемъ, Его слышимъ къ намъ бесѣдующа и прошеніямъ и нуждамъ нашимъ отвѣщающа.

\paragraph*{§\:365.} Писаніе святое какъ всему роду человѣческому, такъ всякому человѣку, мнѣ и тебѣ, человѣче, отъ Бога предано. Богъ бо \textit{всѣмъ человѣкомъ}, слѣдственно и тебѣ, \textit{хощетъ спастися и въ разумъ истины пріити}\footnote{1~Тим.~2,~4.}. Сего ради и Писаніе Свое святое всякому подалъ, дабы читаючи или слушаючи его, спасеніе вѣчное моглъ получить. Писаніе бо святое предано намъ ради спасенія нашего, да познаемъ истиннаго Бога и Сына Его Іисуса Хріста, и тако получимъ вѣчный животъ. \textit{Се есть животъ вѣчный, да знаютъ Тебе единаго истиннаго Бога, и Егоже послалъ еси Іисуса Хріста}\footnote{Іоан.~17,~3.}. \textit{Сія писана быша, да вѣруете, яко Іисусъ есть Хрістосъ Сынъ Божій, и да вѣрующе, животъ имате во имя Его}\footnote{Іоан.~20,~31.}. Отсюду послѣдуетъ, что какъ всѣмъ людямъ, такъ мнѣ и тебѣ, хрістіанине, Богъ въ Писаніи Своемъ глаголетъ. Мнѣ и тебѣ глаголетъ о созданіи міра сего, о созданіи прародителей нашихъ, о паденіи ихъ, о незлобивомъ Авелѣ и враждебномъ Каинѣ, о патріархахъ и житіи ихъ, о нечестивыхъ и благочестивыхъ, о нечестіи и благочестіи ихъ, о казни нечестивыхъ и сохраненіи благочестивыхъ; глаголетъ о преселеніи Израильтянъ во Египетъ и о чудесномъ ихъ освобожденіи отъ работы его, о странствованіи ихъ по пустынѣ и пришествіи въ землю обѣтованную, и о прочемъ. Такожде глаголетъ о имѣвшемъ пріити Спасителѣ міра и пророками предвозвѣщенномъ; показуетъ Его мнѣ и тебѣ уже пришедшаго, отъ Дѣвы плотію рожденнаго, на землѣ пожившаго, учившаго пути истины, чудодѣйствовавшаго, отъ злыхъ людей отверженнаго, уничиженнаго, похуленнаго, поруганнаго, обезчещеннаго, оплеваннаго, уязвленнаго, на древѣ крестномъ распятаго, умершаго, погребеннаго, изъ мертвыхъ воскресшаго, на небо вознесшагося, прославленнаго и паки грядущаго со славою судити живымъ и мертвымъ. Объявляетъ намъ въ словѣ Своемъ, что Онъ не ради чего инаго пришелъ въ міръ, какъ насъ ради и нашего ради спасенія, и велитъ Его слушать во всемъ, когда хощемъ отъ Него вѣчный животъ получити. \textit{Сей есть Сынъ Мой возлюбленный, о Немже благоволихъ: Того послушайте}\footnote{Матѳ.~17,~5.}. Въ чемъ Его слушать намъ, научаетъ насъ святое Его Евангеліе, въ которомъ какъ святое и непорочное житіе Его, такъ и спасительное Его ученіе написано; такожде научаютъ посланія святыхъ Его Апостолъ, которые, Духомъ Святымъ умудрившися и укрѣпившися, пришествіе и ученіе Его, какъ неоцѣненное сокровище, во всю землю пронесли, якоже писано есть: \textit{во всю землю изыде вѣщаніе ихъ, и въ концы вселенныя глаголы ихъ}\footnote{Пс.~18,~5.}. Сего ради кто слову Божію не вникаетъ, тотъ не человѣку не внимаетъ, но Самому Богу; и кто слова Божія не слушаетъ и по слову Божію не живетъ, тотъ Бога не слушаетъ, Который чрезъ слово Свое глаголетъ. Напротивъ того, вѣрное сердце какъ Бога своимъ Богомъ вѣруетъ и исповѣдуетъ, такъ и святое Его слово себѣ присвояетъ и прикладываетъ, и, что ни открывается или обѣщается, запрещается или повелѣвается въ святомъ Писаніи, вѣруетъ и утверждаетъ, что тое все глаголется и объявляется, и того ради всему тому внимаетъ, что въ немъ открыто, обѣщано, запрещено и повелѣно, и такъ святое Писаніе за правило себѣ имѣетъ и по тому житіе свое исправляти тщится.

\paragraph*{§\:366.} Когда Богъ чрезъ слово Свое глаголетъ мнѣ и тебѣ, хрістіанине, то не къ тѣлу нашему глаголетъ, но къ душѣ и сердцу нашему. Не къ тѣлу глаголетъ: \textit{уклонися отъ зла, и твори благо}\footnote{Пс.~33,~15.}, но къ душѣ. Слѣдственно, хотя тѣломъ не твориши зла и внѣ являешися добръ, но внутрь не имѣеши добра, но паче зло помышляеши и поучаешися, золъ еси и Бога не слушаеши. Не имѣешь иныхъ боговъ, кромѣ Господа, не воздѣваешь къ нимъ рукъ, не преклоняешь колѣнъ твоихъ идоламъ, но сердцемъ работаешь страстямъ и похотѣніе ихъ исполняешь: ничего тебѣ не пользуетъ тое внѣшнее дѣло; яко не разнствуешь отъ язычника, который внѣшнихъ и чувственныхъ идоловъ почитаетъ. Богъ бо не къ тѣлу, но къ душѣ и сердцу нашему глаголетъ: \textit{да не будутъ тебѣ бози иніи, развѣ Мене}. Ищешь помощи въ напасти отъ князей, сыновъ человѣческихъ, отъ сребра, злата, отъ своей чести, хитрости и прочаго: тое тебѣ богъ твой, отъ чего ищеши помощи и защищенія; яко на то надѣешися, какъ на своего бога, отъ чего ищеши. Надежда бо отъ вѣры неотлучна; и такъ, когда ты надежду отъ Бога отлучилъ къ созданію, то и вѣрою отвратился отъ Бога къ тому, къ чему надежду свою привязалъ. Того ради, хотя устами Бога и исповѣдуешь и называешь Заступникомъ и Помощникомъ твоимъ; но сердцемъ твоимъ отступилъ отъ Него и оставилъ Его, и имѣеши тое себе за бога, къ чему сердцемъ и надеждою своею прилѣпился. Богъ бо не къ устамъ нашимъ глаголетъ, но къ сердцу: \textit{да не будутъ тебѣ бози иніи, развѣ Мене}. На сердцѣ убо должна быть надежда, какъ и вѣра; а яко на сердцѣ не имѣешь надежды на Бога, оттуда явно, что къ созданію ради помощи и защищенія прибѣгаеши; кто бо отъ кого помощи и защищенія ищетъ, на того и надѣется. Исканіе помощи и защищенія отъ созданія доказательствомъ есть надежды на созданіе. Не приходимъ бо къ тому съ прошеніемъ, на кого не надѣемся; такъ къ нищему и подлому не приходимъ ради защищенія, понеже знаемъ, что онъ не можетъ насъ защитить; но къ сильному и славному человѣку прибѣгаемъ подъ защищеніе. А когда на созданіе надежду полагаемъ, то уже отъ Бога отступаемъ надеждою. На Бога бо и созданіе надежду полагать купно невозможно, но надобно неотмѣнно единаго оставить и къ другому прилѣпиться. Едина бо есть душа и едино сердце у человѣка: когда къ Богу прилѣпляется, отъ созданія отстаетъ; и когда къ созданію прилагается, отъ Бога отступаетъ. Тоежде и о прочіихъ нашихъ дѣлахъ разумѣть должно, хрістіанине: яко чего въ сердцѣ не имѣется, того и въ самой вещи нѣтъ; и что въ сердцѣ есть, тое и въ самомъ дѣлѣ имѣется. Въ сердцѣ нашемъ нечестіе, или благочестіе есть. Молишься, поешь, славословишь и благодаришь Бога устами, но когда въ сердцѣ нѣтъ того, молитва, пѣснь и благодареніе твое ничтоже есть. Покланяешься Богу и колѣна преклоняешь, но когда сердца не преклоняешь, ничтоже есть поклоненіе твое. Смиряешься тѣломъ, но когда въ сердцѣ не имѣеши смиренія, смиреніе твое ничтоже есть. Богъ бо не къ тѣлу нашему, но къ душѣ глаголетъ: \textit{молися, пой, благодари, покланяйся и смиряйся}. А что въ душѣ и сердцѣ истинно имѣется, тое является и внѣ, на тѣлѣ: якоже какой сокъ внутрь древа имѣется, таковы его и плоды раждаются. \textit{Не можетъ бо древо добро плоды злы творити, ни древо зло плоды добры творити}\footnote{Мѳ.~7,~18.}. Такожде, хотя внѣ и не дѣлаеши зла, но въ сердцѣ зло сокрываеши: злодѣй еси. Не убиваеши ближняго твоего руками, или инымъ чѣмъ, но гнѣвъ и злобу въ сердцѣ твоемъ на него держиши: убійца еси предъ Богомъ. \textit{Всякъ бо, ненавидяй брата своего} "--- всякаго человѣка, \textit{человѣкоубійца есть}\footnote{1~Іоан.~3,~15.}. Богъ бо не къ рукамъ, но къ душѣ и сердцу глаголетъ: \textit{не убій}. Не крадешь руками, но желаешь чуждаго добра: тать еси. Не прелюбодѣйствуеши тѣломъ твоимъ, но въ сердцѣ вожделѣніе нечистое имѣеши: прелюбодѣйца еси, по ученію Хрістову: \textit{всякъ, иже воззритъ на жену, ко еже вожделѣти ея, уже прелюбодѣйствова съ нею въ сердцѣ своемъ}\footnote{Мѳ.~5,~28.}. Ибо не къ тѣлу глаголется: \textit{не прелюбы сотвори}, "--- не къ рукамъ: \textit{не укради}, но къ душѣ и сердцу нашему. И рука бо не будетъ красти, похищати, когда душа не похощетъ, и тѣло прелюбодѣйствовати, когда въ сердцѣ вожделѣнія не будетъ; языкъ не злословитъ, когда сердце не захощетъ. Видиши убо, хрістіанине, что все Божіе слово къ душѣ нашей простирается. Хощетъ Богъ, чтобы душа наша исправна была. А когда она исправится, то и внѣшнія дѣла и слова исправны будутъ. Якоже бо, когда сосудъ очистится отъ смрадной вони и наполнится благовоніемъ, издаетъ внѣ себе благую воню: тако душа и сердце человѣческое, когда очиститъ себе отъ зла и добрыми мыслями и нравами исполнится, добрыя дѣла и слова внѣ себе показуетъ. Отсюду правильно заключается, что и покаяніе истинное состоитъ не токмо въ оставленіи внѣшнихъ беззаконныхъ дѣлъ, но и въ премѣненіи сердца, которое когда премѣнится, и внѣшнія дѣла тому сообразныя послѣдуютъ. Тако и Хрістосъ глаголетъ фарисеомъ: \textit{фарисее слѣпый! очисти прежде внутреннее сткляницы и блюда, да будетъ и внѣшнее ихъ чисто}\footnote{Матѳ.~23,~26.}: то"=есть, очисти прежде сердце твое отъ злыхъ похотей, да будутъ и внѣшнія дѣла чисты. Богъ бо глаголетъ душѣ нашей: \textit{покайся}. А когда въ душѣ истинное покаяніе будетъ, то оттуду и достойные покаянія плоды, какъ отъ источника ручьи, проистекутъ.

\paragraph*{§\:367.} Когда Богъ намъ, хрістіанине, глаголетъ, то Божіи слова не пустыя и суетныя, но дѣйствительныя суть. Что повелѣваетъ намъ, хощетъ дѣйствительно, чтобы мы неотмѣнно тое исполняли; что запрещаетъ намъ, хощетъ усердно, чтобы мы того всего всячески не дѣлали: того воля Его святая хощетъ, и вѣчная Его правда требуетъ отъ насъ. Что открываетъ намъ, тое истинно есть и вѣрно, и неотмѣнно такъ есть, какъ открываетъ: \textit{свидѣтельство бо Господне вѣрно}\footnote{Пс.~18,~8.}. Что обѣщаетъ, или чѣмъ угрожаетъ намъ, тое неотмѣнно въ свое время будетъ. Будетъ воскресеніе мертвымъ; будетъ всемірный судъ Хрістовъ, на которомъ всякъ пріиметъ по дѣломъ своимъ; будетъ блаженство вѣчное святымъ, и мука вѣчная беззаконнымъ. Аще убо хощемъ Божіе слово читать, или слушать съ пользою нашею, то должно намъ вѣрить всему тому, что оно намъ глаголетъ. А когда сіе будетъ, то по слову Божію и житіе наше потщимся исправлять. Како бо сіе можетъ быть, дабы ты не съ охотою внималъ всему тому, что въ Писаніи святомъ сказано, когда истинно и сердечно вѣруешь, что той, Который въ Писаніи глаголетъ, Богъ, Господь и Создатель твой есть, и тебѣ точно, убогому Своему созданію, глаголетъ? Никакъ тому быть невозможно, когда въ сердцѣ твоемъ вообразится слово Его. Аще бо земнаго царя слову внимаемъ, когда что нибудь намъ, тебѣ или мнѣ, глаголетъ; и что приказуетъ намъ, съ охотою слушаемъ и исполняемъ; и что обѣщаетъ, вѣримъ, и чѣмъ грозитъ намъ, боимся, хотя и такой же человѣкъ есть, какъ и мы, кромѣ власти его; кольми паче Царь небесный, Богъ великій, святый, всемогущій и страшный, достоинъ вниманія, когда намъ, подлымъ Своимъ рабамъ, мнѣ и тебѣ въ словѣ Своемъ глаголетъ. Богъ бо не зритъ на лица, и лицъ не пріемлетъ, и слово Его святое никого не минуетъ; но, какъ всѣмъ хощетъ спастися, такъ всѣмъ и всякому слово Его равно простирается. Равно мнѣ и тебѣ глаголетъ: покайся, вѣруй, смиряйся, люби, терпи, буди кротокъ, милостивъ, уклоняйся отъ зла и твори благое, не отмщевай ближнему твоему, отпущай согрѣшенія его ему, и прочая. Такожде равно мнѣ и тебѣ обѣщаетъ будущая благая, когда вѣровать истинно и нелицемѣрно будемъ. Равно всякому вѣрующему, мнѣ и тебѣ, глаголетъ: востанешь въ силѣ, наслѣдишь вѣчный животъ, блаженство, царствіе небесное и некончаемую радость. Равно и невѣрующимъ и беззаконнующимъ грозитъ страшнымъ Своимъ и праведнымъ судомъ и вѣчною мукою, что неотмѣнно самымъ дѣломъ исполнится. Божіе бо слово вѣрно, истинно и твердо есть, и не мимоидетъ. \textit{Небо и земля мимоидетъ: словеса же Божія не мимоидутъ}\footnote{Матѳ.~24,~35.}. Отъ таковаго разсужденія можетъ съ помощію Божіею вселиться слово Божіе въ сердце наше, и оживотвориться, и плодъ свой имѣть; и хотя отъ плоти и діавола много искушенія претерпимъ, однакожь благодать Божія поможетъ намъ, когда въ подвигѣ семъ не будемъ ослабѣвать. Богъ бо рабамъ Своимъ помогаетъ и укрѣпляетъ ихъ; тойже Богъ и намъ поможетъ, когда будемъ слову Его, яко къ намъ глаголемому, внимать, и отъ Него помощи просить. Ибо Онъ на лице не зритъ, но всѣхъ равно милуетъ, и укрѣпляетъ молящихся и уповающихъ на Него.

\paragraph*{§\:368.} Писаніе святое есть духовная аптека, въ которой благоутробный Отецъ небесный различныя ради насъ сокрылъ врачеванія. Многоразличныя бо въ душахъ нашихъ имѣемъ немощи, недуги и болѣзни: того ради требуемъ многоразличнаго врачевства, "--- что все въ святомъ Писаніи находимъ. Тамо съ помощію Духа Святаго, глаголавшаго пророками и апостолами, сыщетъ всякъ врачеваніе немощамъ своимъ: тамо печальный утѣшеніе, сумнящійся разумъ и утвержденіе, невѣжда наставленіе и познаніе. Тамо бо сокровенъ совѣтъ недоумѣвающимся, и незнающимъ разумъ, и печальнымъ утѣшеніе. Ибо Богъ, по милосердію Своему, всякому тамо потребная показуетъ и предлагаетъ. Въ подвигѣ вѣры Божіею благодатію пребывающій найдетъ тамо ободреніе и подкрѣпленіе: \textit{буди вѣренъ даже до смерти, и дамъ ти вѣнецъ живота}\footnote{Ап.~2,~10.}, найдетъ предосторожность отъ супостата діавола: \textit{трезвитеся, бодрствуйте, зане супостатъ вашъ діаволъ, яко левъ рыкая, ходитъ, искій кого поглотити}\footnote{1~Петр.~5,~8.}; въ бѣдахъ и скорбяхъ \textit{(многи бо скорби праведнымъ)}\footnote{Пс.~28,~20.} сыщетъ утѣшеніе: \textit{яко многими скорбьми подобаетъ намъ внити въ царствіе Божіе}\footnote{Дѣян.~14,~22.}, "--- и паки: \textit{терпѣнія имате потребу, да, волю Божію сотворше, пріимете обѣтованіе}; "--- и: \textit{еще бо мало елико грядый, пріидетъ, и не закоснитъ}\footnote{Евр.~10,~36 и 37.}. Тамо различно изображается будущая слава, которая скорбь терпящимъ въ мірѣ семъ имѣетъ быть, и надеждою той, яко чистою и хладною водою, разгорѣвшееся прохладитъ сердце. Много хрістіанъ таковыхъ есть, которые блаженство свое полагаютъ въ чести, славѣ, богатствѣ и роскоши міра сего; но, когда со вниманіемъ въ святое Писаніе посмотрятъ, и Божія слова прилѣжно послушаютъ, въ иномъ мнѣніи находиться будутъ. Тамо бо Богъ, Судія праведный, блаженство наше поставляетъ не въ чести, славѣ, богатствѣ и прочіихъ веселостяхъ міра сего, но въ иномъ. Онъ, вопервыхъ, называетъ \textit{блаженнымъ} того, который \textit{не идетъ на совѣтъ нечестивыхъ, и на пути грѣшныхъ не стоитъ, и на сѣдалищи губителей не сѣдитъ, но въ законѣ Господни воля его, и въ законѣ Его поучится день и нощь}\footnote{Пс.~1,~1 и 2.}. И паки: \textit{блажени вси надѣющіися Нань} (на Господа)\footnote{2,~12.}. \textit{Блажени, ихже оставишася беззаконія, и ихже прикрышася грѣси. Блаженъ мужъ, емуже не вмѣнитъ Господь грѣха}\footnote{31,~1 и 2.}. \textit{Блаженъ разумѣваяй на нища и убога: въ день лютъ избавитъ Его Господь}\footnote{60,~2.}. \textit{Блажени хранящіи судъ, и творящіи правду во всякое время}\footnote{105,~3.}. \textit{Блаженъ мужъ бояйся Господа: въ заповѣдехъ Его восхощетъ зѣло}, и проч.\footnote{111,~1 и слѣд.} \textit{Блажени непорочніи въ пути, ходящіи въ законѣ Господни. Блажени испытающіи свидѣнія Его, всѣмъ сердцемъ взыщутъ Его} и проч.\footnote{Пс.~118,~1,~2 и слѣд.} Паки люди блаженными нарицаютъ тѣхъ, \textit{ихже сынове ихъ яко новосажденія водруженая въ юности своей; дщери ихъ удобрены, преукрашены, яко подобіе храма; хранилища ихъ исполнена, отрыгающая отъ сего въ сіе: овцы ихъ многоплодны, множащіяся во исходищахъ своихъ; волове ихъ толсти; нѣсть паденія оплоту, ниже прохода, ниже вопля въ стогнахъ ихъ: ублажиша людіе, имже сія суть}. Но Богъ, истинный Судитель всего и Источникъ всякаго блаженства, ублажаетъ тѣхъ, которые Господа за Бога \textit{своего} имѣютъ, и въ Немъ все блаженство свое полагаютъ: \textit{блажени людіе, имже Господь Богъ ихъ}\footnote{143,~12--15.}. Паки показуя намъ Богъ, что есть истинное блаженство, и хотя насъ къ тому привести, глаголетъ: \textit{блажени нищіи духомъ}, то"=есть, смиренніи сердцемъ; \textit{блажени плачущіи}, то"=есть, ради грѣховъ своихъ и окаянства своего душевнаго; \textit{блажени кротцыи}, которые прощаютъ согрѣшенія ближнимъ своимъ, не воздаютъ зла за зло, и не памятуютъ зла; \textit{блажени алчущіи и жаждущіи правды}, которые, не видя въ себѣ никакія правды своея, желаютъ правды о Хрістѣ Іисусѣ; \textit{блажени милостивіи}, которые ближнихъ своихъ, яко братію свою, требующихъ милости, снабдѣваютъ и помогаютъ имъ, и проч.\footnote{Мѳ.~5,~3--7 и слѣд.} Паки ублажаетъ Богъ не тѣхъ, которые на обѣдъ или вечерю зовутъ, яко воздаяніе другъ отъ друга пріемлютъ, но тѣхъ, которые зовутъ нищихъ, маломощныхъ и всякихъ бѣдныхъ, яко не имѣютъ что воздати звавшимъ: \textit{воздастся же имъ въ воскрешеніе праведныхъ}\footnote{Лук.~14,~13 и 14.}. Видиши ли, хрістіанине, како мнѣніе человѣческое погрѣшительное отъ святаго Божія слова исправляется и немощь душевная исцѣляется. Не тое бо истинное есть блаженство, которое люди мнятъ, но тое, которое есть согласно Божію слову. Ибо слово Божіе есть достовѣрное правило мнѣній, помышленій и дѣяній человѣческихъ, котораго неуклонно держаться должно намъ, когда не хощемъ погрѣшить и заблудить. Предлагаетъ намъ часто слѣпая плоть и чрезъ нее врагъ нашъ всегдашній діаволъ: како тѣло наше умершее и въ прахъ разсыпанное можетъ востати? Но сія немощь вѣрою, на святомъ Божіемъ Словѣ, яко на истинномъ и непоколебимомъ основаніи, утверждающегося, прогонится, которое насъ крѣпко увѣряетъ, что \textit{якоже о Адамѣ вси умираютъ, такожде и о Хрістѣ вси оживутъ}\footnote{1~Кор.~15,~22 и далѣе вся глава.}. Аще бо вѣруемъ, что все изъ ничего создалъ Богъ, кольми паче разсыпанный тѣлесъ нашихъ прахъ о Хрістѣ Іисусѣ соберетъ и оживитъ. Богу бо, яко всемогущему, все возможно и удобно; и, яко истинному и вѣрному въ словесѣхъ Своихъ, солгать невозможно. Будетъ неотмѣнно все тое, что въ словѣ Своемъ объявилъ. Симъ утверждай вѣру твою, хрістіанская душа, и въ подвигѣ благочестія не ослабѣвай. Часто благочестивымъ въ размышленіе и отъ того опасное поколебаніе въ вѣрѣ приходитъ: како люди добрые и богобоящіися въ бѣдахъ и напастяхъ, а злыи и безбожніи въ благоденствіи живутъ, что и Псаломникъ о себѣ призналъ: \textit{мои вмалѣ не подвижастѣся нозѣ, вмалѣ не проліяшася стопы моя: яко возревновахъ на беззаконныя, миръ грѣшниковъ зря}\footnote{Пс.~72,~2 и 3.}. Но когда прилѣжно вникнутъ въ святое Писаніе, и разумѣваютъ въ послѣдняя ихъ (беззаконныхъ), исправятъ сердце свое, благоденствіемъ злыхъ и злоденствіемъ добрыхъ колеблющееся, и, въ вѣрѣ утвердившися, воскликнутъ съ тѣмже Псаломникомъ: \textit{коль благъ Богъ Израилевъ правымъ сердцемъ!} Увидятъ тамо конецъ обоихъ: благочестивыхъ добрый, хотя и бѣдствуютъ въ мірѣ семъ; нечестивыхъ злый, хотя на мало время и возносятся. Отъ конца бо благополучіе и неблагополучіе истинное описуется и зависитъ. \textit{Обаче за льщенія ихъ положилъ еси имъ злая, низложилъ еси я, внегда разгордѣшася. Како быша въ запустѣніе? Внезапу исчезоша, погибоша за беззаконіе свое. Яко соніе востающаго, Господи, во градѣ Твоемъ образъ ихъ уничижиши}, поетъ тойже Псаломникъ о нечестивыхъ къ Богу\footnote{Пс.~72,~18--20.}. И паки о нечестивыхъ: \textit{яко трава скоро изсшутъ, и яко зеліе злака скоро отпадутъ}. И паки о праведныхъ и беззаконныхъ поетъ: \textit{лукавнующіи потребятся: терпящіи же Господа, тіи наслѣдятъ землю. И еще мало, и не будетъ грѣшника; и взыщеши мѣсто его, и не обрящеши. Кротцыи же наслѣдятъ землю, и насладятся о множествѣ мира. Грѣшницы погибнутъ: врази Господни, купно прославитися имъ и вознестися, исчезающе, яко дымъ исчезоша. Яко Господь любитъ судъ, и не оставитъ преподобныхъ Своихъ, во вѣкъ сохранятся: беззаконницы же изженутся, и сѣмя нечестивыхъ потребится. Праведницы же наслѣдятъ землю, и вселятся въ вѣкъ вѣка на ней}. И паки свидѣтельствуетъ: \textit{видѣхъ нечестиваго превозносящася и высящася, яко кедры ливанскія: и мимо идохъ, и се не бѣ: и взыскахъ его, и не обрѣтеся мѣсто его}. И заключаетъ псаломъ: \textit{спасеніе праведныхъ отъ Господа, и Защититель ихъ есть во время скорби: и поможетъ имъ Господь, и избавитъ ихъ, и изметъ ихъ отъ грѣшникъ, и спасетъ ихъ, яко уповаша на Него}\footnote{Пс.~36,~2,~9,~10,~11,~19,~20,~28,~29,~35,~36,~39 и 40.}. Видиши, хрістіанине, какій конецъ благочестивыхъ и нечестивыхъ слово Божіе намъ показуетъ. Высоко одни возносятся, но низко въ погибель свергаются; другіи низко ходятъ, но высоко наконецъ возносятся. Гдѣ богачъ оный евангельскій, который облачашеся \textit{въ порфиру и виссонъ, веселяся на вся дни свѣтло? Во адѣ возводитъ очи свои, сый въ мукахъ, и проситъ}, послѣ драгихъ винъ, \textit{капли воды}, и во вѣки просити будетъ, \textit{но не получитъ}. Гдѣ убогій Лазарь, который \textit{предъ вратами его лежалъ гноенъ}? По трудахъ и болѣзняхъ \textit{упокоевается на лонѣ Авраамли}\footnote{Лук.~16,~19--26.}. Какъ прочитаешь священную Исторію отъ начала міра, увидишь ясно, какъ блаженный конецъ есть благочестивыхъ, и коль злополученъ исходъ есть нечестивыхъ. Ибо правда Божія всякому свое возмѣряетъ, и потому какъ добра безъ награжденія, такъ и зла безъ казни не оставляетъ, хотя до времени и молчитъ. Хощеши знать, кто суть истинные хрістіане и вѣчнаго Сіона сынове? Во псалмѣ 14"~мъ объявляетъ тебѣ Богъ, и на вопросъ Псаломника: \textit{Господи! кто обитаетъ въ жилищи Твоемъ? или кто вселится во святую гору Твою?} "--- отвѣщаетъ тамо: \textit{ходяй непороченъ, и дѣлаяй правду, глаголяй истину въ сердцѣ своемъ, иже не ульсти языкомъ своимъ, и не сотвори искреннему своему зла}, и проч.\footnote{Пс.~14,~1--3 и слѣд.} Такожде во всемъ первомъ Іоанновомъ посланіи описуются примѣты истиннаго и вѣрнаго раба Божія, и на прочіихъ мѣстахъ. Тамо предлагается и грѣхами недугующимъ здравое врачевство "--- покаяніе и милость Божія о Хрістѣ Іисусѣ, яко цѣлительный и живительный пластырь, которымъ язвы грѣховныя обязуются и заглаждаются; словомъ, тамо всякъ приличное требованію своему сыщетъ, когда прилѣжно и подобающимъ образомъ поищетъ.

\paragraph*{§\:369.} Отъ вышеписанныхъ видиши, возлюбленный хрістіанине: 1)~Коль великая благость Бога нашего, Который въ словѣ Своемъ мнѣ и тебѣ, убогимъ Своимъ рабамъ, бесѣдуетъ. Который царь земный такъ съ рабами своими поступаетъ, якоже Царь Небесный, Богъ нашъ, съ нами непотребными рабами Своими? Читай святое слово Его, и увидишь, какъ милостиво, кротко и ласково открываетъ намъ о Себѣ, свойствахъ Своихъ, волѣ Своей и благоволеніи Своемъ къ намъ; какъ усердно хощетъ намъ спастися; какъ многоразлично призываетъ насъ на покаяніе; какъ изображаетъ вѣчнаго живота блаженство, и адскаго мученія горесть, и тѣмъ всѣмъ хощетъ насъ привлещи къ Себѣ, которые грѣхомъ отпали отъ Него. "--- 2)~Коль великій даръ Божій, человѣкамъ данный, есть святое Писаніе. Чувствуемъ и вкушаемъ благость Божію, что Онъ насъ питаетъ и одѣваетъ; но въ семъ большую Его примѣчаемъ къ намъ милость, что Писаніе Свое святое, какъ посланіе, послалъ къ намъ. За велико почитаетъ человѣкъ, когда отъ монарха своего получитъ письмо; коль несравненно за большее почитать долженъ хрістіанинъ, что отъ Бога, Царя Небеснаго и Создателя своего, пророковъ и апостоловъ Писаніе получилъ. Толико сей даръ большій онаго, сколько большій Царь Небесный отъ царя земнаго, и Богъ отъ человѣка, хотя человѣкъ ослѣпленный того и не познаетъ. Ибо, какъ сказано, всякому Богъ Писаніе Свое послалъ, и всякому въ немъ глаголетъ. "--- 3)~Какъ любить должно намъ Божіе Слово, такъ что ничимъ утѣшаться и услаждаться не должно намъ, хрістіанамъ, какъ словомъ Его святымъ. Слово Божіе есть слово устъ Его, якоже глаголетъ о себѣ Давидъ святый: \textit{благъ мнѣ законъ устъ Твоихъ, Господи, паче тысящъ злата и сребра}\footnote{Пс.~118,~72.}. Слѣдственно какое усердіе, охоту и ненасытное желаніе къ чтенію и слышанію имѣть должно! Сколько бо разъ или сами читаемъ, или служителя Божія читающа слышимъ, толико разъ слышимъ Бога нашего, къ намъ бесѣдующаго. Коль охотно слушаемъ, когда царь нашъ къ намъ бесѣдуетъ, всякъ сіе за немало почитаетъ: коль паче охотнѣе слушать должно Бога, Царя небеси и земли, бесѣдующаго. Почтилъ Онъ насъ недостойныхъ сею Своею милостію: не должно и намъ сію великую Его милость пренебрегать, но паче охотнѣйшимъ слушаніемъ, или чтеніемъ Слова Его благодарность къ Нему показывать, и что открываетъ и обѣщаетъ въ немъ, вѣрить, что повелѣваетъ, исполнять, и что запрещаетъ, отъ того уклоняться. Тако почтимъ и мы почетшаго насъ туне Господа нашего. "--- 4)~Сколь съ великимъ почтеніемъ и благоговѣинствомъ приступать къ слушанію или чтенію святаго Писанія, яко истиннаго Божія слова и небеснаго дара. Даръ бо Божій \textit{небесный} есть: сего ради требуетъ чистыхъ рукъ къ пріятію, чистаго ума и мыслей къ размышленію, чистыхъ устъ къ проповѣданію. Аще бо и человѣкъ проповѣдуеши, но слово Божіе и законъ устъ Его проповѣдуеши: уста Господня глаголали тое, что ты устами твоими воспріемлеши. Аще убо хощеши, человѣче, въ законѣ Господни поучатися и того проповѣдовати, должно тебѣ имѣть сердце очищенно отъ злыхъ похотей, умъ очистить отъ суетныхъ мыслей, уста заградить къ сквернословію, срамословію, празднословію, злословію, клеветѣ, и прочіихъ дѣлъ, слову Божію противныхъ, берещися, да не и тебѣ речется тое, что грѣшнику рече Богъ: \textit{Вскую ты повѣдаеши оправданія Моя, и воспріемлеши завѣтъ Мой усты твоими? Ты же возненавидѣлъ еси наказаніе, и отверглъ еси словеса Моя вспять. Аще видѣлъ еси татя, теклъ еси съ нимъ, и съ прелюбодѣемъ участіе твое полагалъ еси. Уста твоя умножиша злобу, и языкъ твой сплеташе льщенія; сѣдя на брата твоего клеветалъ еси, и на сына матере полагалъ еси соблазнъ}\footnote{Пс.~49,~16--20.}. Якоже убо въ молитвѣ, приступая къ Богу и Ему бесѣдовати хотя, тако приступая къ чтенію слова Божія, и отъ Него бесѣду съ пользою нашею слышати желая, должны себе прежде очистить отъ всего того, отъ чего Богъ, яко \textit{Святый}, отвращается, и что слову Его \textit{Святому} противно. Иначе какъ молитва наша, такъ и чтеніе и проповѣданіе слова Его не полезно намъ будетъ. Страшно бо есть приступать къ Богу тому, кто грѣхами оскверненъ, и бесѣдовати нечистому съ тѣмъ, Который самая есть Святыня, и слово Его, яко слово устъ Его, воспріимати во умъ и уста свои, мерзостьми грѣховными окаленныя. Примѣчайте сіе, служители Божіи и проповѣдники, которые слово устъ Божіихъ во уста ваша воспріемлете и проповѣдуете, и всякій хрістіанинъ, который имя Божіе исповѣдуетъ! Имя бо Его свято и страшно есть, которое устами твоими воспріемлеши, и отверзаеши уста твоя къ Богу святому и страшному въ молитвѣ и славословіи Божіи. \textit{Да отступитъ отъ неправды всякъ именуяй имя Господне}, глаголетъ Апостолъ\footnote{2~Тим.~2,~19.}. "--- 5)~Какъ нужно хрістіанамъ поучаться въ словѣ Божіемъ, въ которомъ Богъ и Хрістосъ Сынъ Божій открывается и познавается, которое познаніе необходимо къ полученію вѣчнаго живота требуется, по словеси самаго Сына Божія: \textit{се есть животъ вѣчный, да знаютъ Тебе}, Отче, \textit{единаго истиннаго Бога, и Егоже послалъ еси Іисусъ Хріста}\footnote{Іоан.~17,~3.}. Отъ истиннаго бо познанія Бога и Хріста Сына Божія зачинается истинная и живая вѣра, которая необходимо къ вѣчному спасенію нужна. Истиннаго же Божія познанія доказательство есть благочестивое хрістіанское житіе, безъ котораго суетная и пустая похвала только есть, а не познаніе. О семъ Апостолъ святый научаетъ насъ: \textit{о семъ разумѣемъ, яко познахомъ Его, аще заповѣди Его соблюдаемъ. Глаголяй, яко познахъ Его, и заповѣди Его не соблюдаетъ, ложь есть, и въ семъ истины нѣсть}\footnote{1~Іоан.~2,~3--4.}. Напрасно убо хвалятся познаніемъ Божіимъ тіи хрістіане, которые \textit{дѣлы отмещутся Его, мерзцы суще и непокориви, и на всякое дѣло благое неискусни}\footnote{Тит.~4,~16.}. Познаніе бо Божіе заключаетъ въ себѣ познаніе воли Его святыя, благоволенія и Божественныхъ Его свойствъ, что все въ святомъ Словѣ Его открывается. Аще убо кто дѣйствительно познаетъ сія, неотмѣнно въ немъ послѣдуетъ истинное премѣненіе, покаяніе и благочестивое житіе. "--- 6)~Какъ сладостно и благопріятно читающему или слушающему со вниманіемъ слово Божіе, вѣрою видѣть, яко зерцаломъ въ гаданіи, въ словѣ томъ Бога, Творца, Искупителя и Промыслителя своего, и Божественныя Его свойства; вѣрою и любовію лобызать непостижимую Его благость, изліянную на родъ человѣческій, удивляться непостижимой Его премудрости, въ созданіи міра и въ промыслѣ о немъ показуемой; хвалить непостижимую Его правду, которою рабовъ Своихъ защищалъ и защищаетъ, и враговъ Своихъ казнилъ и казнитъ; хвалить непостижимое величество славы Его, видѣть Божественная Его чудеса, отъ начала міра сотворшаяся; слышать милостивыя и отеческія Его обѣщанія, иная сбывшіяся, иная непремѣнно быть имѣющая; взирать вѣрою на славу будущую, различно въ Словѣ святомъ изображенную, и къ ней желаніемъ восхищаться! Коль увеселительно представлять себѣ образъ спасенія нашего, въ немъ описанный, которымъ мы избавилися отъ грѣха, смерти, діавола и ада, и въ семъ показанной къ намъ благости Божіей удивляться, и радостнымъ благодарить духомъ! Благопріятно намъ читать тую вѣдомость, въ которой описуется подвигъ воинства нашего, и изображается побѣда, надъ врагами нашими одержанная; далеко благопріятнѣе прочитывать и поминать тотъ подвигъ, которымъ Избавитель нашъ, Іисусъ Сынъ Божій, подвизался за насъ и славно побѣдилъ враговъ нашихъ. Сіе все въ святомъ Божіемъ Словѣ представляется. Читай самъ, хрістіанине, и увидишь. "--- 7)~Отсюду видно, какъ несмысленно или паче беззаконно дѣлаютъ тіи хрістіане, которые отъ Слова Божія удаляются. Многіи, оставивши сей живый и святый источникъ, забавляются въ непотребныхъ книжкахъ, которыя плоть ихъ увеселяютъ, но душу развращаютъ, и тако чтеніемъ тѣмъ погубляютъ себе, а не созидаютъ. Иные тщатся знать, что дѣлается въ Америкѣ, Африкѣ, Азіи и прочіихъ дальнихъ странахъ; а близъ себе и внутрь себе не хотятъ знать, что съ душею ихъ дѣлается, въ какомъ состояніи она находится, изъ слова Божія вѣдать не стараются. Иные натуру травъ, зелій, древесъ и прочіихъ вещей испытывать стараются; но своего естества, грѣхомъ растлѣннаго, изъ слова Божія разсматривать не хотятъ. Иные звѣзды считать, землю размѣрять учатся; но краткихъ житія своего дней и безчисленныхъ грѣхопаденій, которыми по вся дни предъ Богомъ виноватыми являемся, "--- \textit{грѣхопаденія бо кто разумѣетъ}\footnote{Пс.~18,~13.}? "--- изъ Писанія разсматривать не хотятъ. Иные день и нощь поучаются, какъ бы богатство собрать, которое съ тѣломъ въ семъ вѣкѣ оставить принуждены будутъ; но пользы душамъ изъ Писанія собирать не хотятъ, и о богатствѣ которое на оный вѣкъ съ душами отходитъ, небрегутъ. Другіе стараются тѣло, скоро въ прахъ обращаемое, врачевать и цѣло хранить; но о душѣ безсмертной нерадятъ, и немощи ея изъ слова Божія знать и врачевать не хотятъ. Иные иная замышляютъ и дѣлаютъ; но о томъ, что \textit{едино есть на потребу}, по ученію Хрістову\footnote{Лук.~10,~41.}, небрегутъ. Сіи вси и прочіи, симъ подобныи, дѣлаютъ подобно тому, который полушку хранилъ бы въ сундукѣ, но о тысящѣ червонныхъ нерадѣлъ бы; или тому, который о домѣ горящемъ и сгараемомъ нерадѣлъ бы, а выносилъ бы вонъ изъ него сметіе, чтобы не сгорѣло; или тому, который въ водѣ тонетъ, но, нерадя о себѣ, вещи свои сохранить старается. Аще бы кто сіи и симъ подобныи случаи увидѣлъ, неотмѣнно бы посмѣялся безумію таковыхъ людей; ибо должно берещи тое, что нужное и лучшее есть, а не тое, что подлѣйшее есть. Тако смѣха, или паче сожалѣнія достойны тіи люди, которыи о временной и скоро"=погибающей пользѣ пекутся, но о вѣчномъ спасеніи души, которое \textit{едино есть на потребу}, не пекутся. Бѣдное такихъ людей состояніе! Во тьмѣ и заблужденіи они ходятъ, хотя бы и за мудрецовъ отъ міра сего почиталися. Невозбранно и о временной пользѣ стараться, но далеко болѣе о спасеніи души, ради котораго Хрістосъ въ міръ пришелъ, пострадалъ и умеръ. Онъ велитъ намъ: \textit{ищите прежде царствія Божія и правды Его, и сія вся приложатся вамъ}\footnote{Мѳ.~6,~33.}. Какая тамо польза, гдѣ душѣ погибель? Безумный тотъ мудрецъ, который предъ міромъ симъ мудръ, но предъ Богомъ юродъ. Нищъ и убогъ тотъ богачь, который \textit{собираетъ себѣ, а не въ Бога богатѣетъ}\footnote{Лук.~12,~21.}. Что въ томъ здравіи, когда душа немоществуетъ и умираетъ? Какая польза отъ того знанія, когда умъ и сердце въ дѣлѣ спасенія слѣпотствуетъ? Такъ"=то слѣпъ отъ природы своей бѣдный человѣкъ! Оставляетъ лучшее и нужное, а ищетъ и гоняется за тѣмъ, что хуждшее и ненужное. И понеже толикое небреженіе у хрістіанъ о словѣ Божіи, то оттуду бываетъ, что ихъ житіе противно есть слову Божію. Слово бо Божіе есть правило хрістіанскаго житія; слѣдуетъ убо тому заблудить, кто правилу тому не внимаетъ. Слово Божіе есть свѣтильникъ, который свѣтитъ намъ, по пути міра сего идущимъ, якоже глаголетъ Давидъ: \textit{свѣтильникъ ногама моима законъ Твой, Господи, и свѣтъ стезямъ моимъ}\footnote{Пс.~118,~105.}: убо непремѣнно во тьмѣ ходятъ, которые свѣтильника сего не имѣютъ предъ собою и не держатся его. Отсюду"=то бываетъ, что многимъ хрістіанамъ нѣтъ грѣха, или за мало почитается ближняго обидѣть, обмануть, ругать, оклеветать, хитрымъ словомъ язвить; нѣтъ грѣха красть, похищать, зло за зло воздавать, злобиться и мстить; нѣтъ грѣха мамонѣ, вмѣсто Бога, работать; нѣтъ грѣха время въ праздности провождать, дни и нощи въ картежной игрѣ и прочіихъ забавахъ веселиться; нѣтъ грѣха имя Божіе великое, святое и страшное въ подлыхъ и непотребныхъ словахъ пріимать и злоупотреблять; нѣтъ грѣха на сребро, злато, на вельможъ, князей и прочіихъ сыновъ человѣческихъ, какъ божковъ своихъ, надѣяться, и тако отъ Бога, сотворшаго и искупившаго ихъ, отступать; нѣтъ грѣха правду попирать и неправду на мѣстѣ ея посаждать, слезы вдовицъ, сиротъ и убогихъ проливать, неповинныхъ осуждать, а виноватыхъ свобождать, и прочія страшныя беззаконія дѣлать. Сіи и симъ подобные плоды отъ небреженія святаго Божія слова послѣдуютъ. А хотя многіи, повидимому, и тщатся благочестивыми быть; но понеже не внимаютъ Божію слову, яко истинному и совершенному правилу, то не въ томъ полагаютъ благочестіе, въ чемъ оно состоитъ, но въ томъ, что слѣпому своему разуму и плоти угодно видятъ, и такъ заблуждаютъ. Многіи отъ нихъ тое, что человѣкъ написалъ, ненарушимо хранить тщатся; но что Богъ запретилъ, или повелѣлъ, о томъ небрегутъ. Слово и заповѣдь человѣческую соблюдаютъ, но слово и заповѣдь Божію оставляютъ и пренебрегаютъ. Много такого заблужденія вездѣ примѣчается. У многихъ въ среду, пятокъ и прочіе дни отъ нѣкіихъ снѣдей воздерживаться, а у иныхъ и совсѣмъ ничего не ясти въ тыя дни, за благочестіе поставляется; но отъ злобы, зависти, клеветы, злословія и прочіихъ хрістіанства язвъ, и единаго часа воздержаться не хотятъ. Многіи созидаютъ высокіе храмы Божія и украшаютъ, высокія колокольни и великіе колокола дѣлаютъ, богатое платье въ церкви шьютъ, книгу Евангелія и образа святые, и образъ креста Хрістова сребромъ и златомъ окладываютъ, и симъ веществомъ хотятъ \textit{невещественному} Богу угодить; но бѣдныхъ, нищихъ, въ тюрьмахъ за долги заключенныхъ, оставляютъ и презираютъ. И такъ дѣлаютъ тое, чего Богъ не повелѣлъ, но не дѣлаютъ того, что Богъ повелѣлъ. Я здѣсь не созиданіе храмовъ, но небреженіе Божія слова охуждаю. Нужны и храмы суть, но не великолѣпіе храмовъ. Во всякомъ храмѣ, только бы удобенъ былъ и чистъ, публичное Богослуженіе отправлять можно; но человѣкамъ, одушевленнымъ Божіимъ храмамъ, безъ пищи, одежды и покоя пребыть невозможно. На сихъ, которые требуютъ отъ насъ милости, излишество обращать должно. *Такуюжде имѣютъ плоть, которая хощетъ ясти, пити, одѣваться и упокоеваться, какую и богатіи; а книга Евангелія, образа святіи и крестъ безъ сребра могутъ быть.* Не тое бо дѣлать намъ должно, что нашимъ очамъ, но что Божіимъ угодно и слову Его святому согласно. Тако люди, оставивши слово Божіе и своему послѣдуя разуму слѣпому, не знаютъ, что дѣлаютъ, и свое за неблагодарность возмездіе воспріемлютъ. Богъ бо за небреженіе слова Своего отнимаетъ Свою благодать отъ таковыхъ, и тако ходятъ, какъ во тьмѣ, и осязаютъ, какъ слѣпые. Сія послѣдуетъ казнь за небреженіе Божія слова!

\paragraph*{§\:370.} Не пользуетъ слово Божіе читать, или слушать, но не творити. Сего ради Хрістосъ не тѣхъ, которые слово Божіе слушаютъ, но тѣхъ, которые слышатъ и хранятъ, ублажаетъ. \textit{Блажени слышащіи слово Божіе, и хранящіи е}\footnote{Лук.~11,~28.}. Что пользуетъ сѣмя посѣянное, которое никакого плода не приноситъ? ничего. Что пользуетъ пища желудку, которая въ сокъ и кровь не претворяется! весьма ничего. Такъ и слово Божіе, сѣмя Божественное, никакой пользы не приноситъ, когда слышащіи отъ него не созидаются. И какъ тѣло изнемогаетъ, а далѣе и умираетъ, когда пищу желудокъ пріемлетъ, но не варитъ, и въ сокъ и кровь не обращаетъ: тако и душа, хотя и слышитъ слово Божіе, которое есть душевная пища, но изнемогаетъ и погибаетъ, когда слышанное слово въ духовный ея сокъ и кровь не обращается, то"=есть, духовнѣ ее не укрѣпляетъ. Тогда же сіе бываетъ человѣку, когда онъ отъ слышаннаго Божія слова не исправляется въ житіи своемъ, не обращается къ Богу, не творитъ истиннаго покаянія, отъ грѣховъ не отстаетъ, и духовнаго утѣшенія не чувствуетъ въ сердцѣ своемъ. Слово бо Божіе дано намъ ради того, чтобы мы, слыша его, исправляли себе по правилу его; а когда того нѣтъ въ насъ, то оно намъ никакой пользы не приноситъ. Якоже бо хотя изрядное врачевство будетъ, но ничего не пользуетъ немощному, когда не хощетъ его себѣ ко врачеванію употреблять: тако и слово Божіе, которое дано намъ отъ Бога ко уврачеванію немощей нашихъ духовныхъ, какъ выше сказано, не воспользуетъ намъ, когда отъ него не будемъ исправлять немощей нашихъ. Такожде, какъ мастеру никакой пользы не приноситъ мастерства знаніе, когда мастерства своего въ дѣло не употребляетъ, но непремѣнно имѣетъ придти въ убожество: тако искусство въ Писаніи и знаніе воли Божіей не пользуетъ намъ, когда по слову Божію и волѣ Его житія нашего исправлять не тщимся. Паче же большему осужденію подлежатъ знающіи, но не творящіи, по словеси Хрістову: \textit{рабъ, вѣдѣвый волю господина своего, и не уготовавъ, ни сотворивъ по воли его, біенъ будетъ много}\footnote{Лук.~12,~47.}; и Апостолъ Петръ глаголетъ: \textit{лучше бѣ имъ не познати пути правды, нежели познавшимъ возвратитися вспять отъ преданныя имъ святыя заповѣди}\footnote{2~Петр.~2,~21.}. Не отъ единаго убо слышанія Божія слова познается хрістіанинъ, но наипаче отъ творенія. \textit{Аще сія вѣсте, блажени есте, аще творите я}, яже вѣсте, глаголетъ Хрістосъ\footnote{Іоан.~13,~17.}. Извѣстно знай, хрістіанине, что какъ отецъ не почитаетъ сына за сына своего, и наслѣдія его исключаетъ, который воли отчей не слушаетъ; и господинъ раба за вѣрнаго раба не вмѣняетъ, который приказу его не повинуется и вѣрно ему не работаетъ; и монархъ за вѣрноподданнаго не имѣетъ того, который указовъ его не исполняетъ, хотя и знаетъ ихъ: тако и Хрістосъ за своего не признаетъ, кто воли Его святой не тщится исполнять, хотя и слово Его слушаетъ. \textit{Что Мя зовете: Господи, Господи, и не творите, яже глаголю}\footnote{Лук.~6,~46.}? И паки: \textit{не вѣмъ васъ, откуду есте: отступите отъ Мене вси дѣлателіе неправды}\footnote{13,~27.}. Сіе есть извѣстное знаменіе, яко кто волю чію творитъ, тотъ того и рабъ есть, якоже глаголетъ Хрістосъ Іудеомъ: \textit{аще бы чада Авраамля бысте были, дѣла Авраамля бысте творили. Нынѣ же ищете Мене убити, человѣка, иже истину вамъ глаголахъ, юже слышахъ отъ Бога: сего Авраамъ нѣсть сотворилъ}; и ниже: \textit{вы отца вашего діавола есте, и похоти отца вашего хощете творити}\footnote{Іоан.~8,~39,~40 и 44.}. Кто волю Божію творитъ и вѣрою послѣдуетъ Аврааму, отцу вѣрныхъ, "--- чадо Авраамле есть, послѣдовательно Божіе чадо и рабъ Божій есть; а кто діавольскую волю въ дѣлахъ его злыхъ творитъ, діавольскій рабъ есть. Да не прельщаетъ убо тебе, хрістіанине, имя хрістіанское безъ житія хрістіанскаго. Ложное бо есть хрістіанство безъ дѣлъ хрістіанскихъ, и такому хрістіанину ничего не воспользуетъ, но паче большую принесетъ пагубу, какъ выше сказано.

\paragraph*{§\:371.} Что"=де надобно дѣлать хрістіанину, дабы сѣмя слова Божія въ немъ плодъ творило? 1)~Отвѣщаетъ Хрістосъ: \textit{просите, и дастся вамъ; ищите, и обрящете; толцыте, и отверзется вамъ. Всякъ бо просяй пріемлетъ, и ищай обрѣтаетъ, и толкущему отверзется}\footnote{Мѳ.~7,~7 и 8.}. Сами бо мы, яко немощни, никакого добра творити не можемъ безъ Хріста, по свидѣтельству Его: \textit{безъ Мене не можете творити ничесоже}\footnote{Іоан.~15,~5.}. Сего ради молитвою усердною всего искать у Него должно. О семъ поучаетъ насъ весь псаломъ 118"~й, въ которомъ Псаломникъ съ великимъ усердіемъ молится, чтобы его Самъ Богъ научилъ и наставилъ на путь святаго хрістіанскаго житія, утвердилъ и велъ по пути тому. Въ которомъ псалмѣ между прочимъ достойно примѣчанія усердіе его, воздыханіе и горячесть сердца: \textit{воззвахъ всѣмъ сердцемъ моимъ, услыши мя Господи: оправданія Твоя взыщу. Воззвахъ Ти, спаси мя: и сохраню свидѣнія Твоя}\footnote{Пс.~118,~145 и 146.}. Смотри, какъ молился пророкъ святый! Молился всѣмъ сердцемъ: \textit{воззвахъ всѣмъ сердцемъ моимъ}. Сей способъ къ полученію святаго житія хрістіанскаго и намъ примѣромъ своимъ показалъ. И намъ должно молиться о томъ, молитися не половиною сердца, но \textit{всѣмъ сердцемъ}: должно не токмо просить и искать, но и толкать, то"=есть, со всякимъ усердіемъ и неотступно просить. Кто бо чего усердно хощетъ, тотъ того со тщаніемъ и усердіемъ ищетъ; якоже сіе и въ сынахъ вѣка сего видимъ: съ коликимъ они прилѣжаніемъ и тщаніемъ ищутъ богатства, чести и прочаго, чего усердно желаютъ. Слово бо Божіе и законъ Божій духовенъ есть: \textit{мы же плотяны есмы, проданы подъ грѣхъ}\footnote{Римл.~7,~14.}; сего ради разумѣть его, много паче хранить, не можемъ безъ Духа Божія. Глубоко растлѣнно естество наше имѣемъ, которое слову Божію всегда противится; сего ради сами отъ себе исправлять его не можемъ; надобно бо себе преодолѣть и побѣдить, къ чему неотмѣнно вышеестественная сила требуется: надобно Духу Божію, Который пророками и апостолами глаголалъ, дѣйствовать въ насъ, просвѣщать, вразумлять, наставлять и руководить насъ, когда хощемъ по правилу Его жити. Два рода ученыхъ и мудрыхъ людей. \textit{Едины} учатся въ школахъ отъ книгъ, и множайшіи отъ нихъ суть безумнѣйшіи паче простыхъ и безграмотныхъ, яко и алфавита хрістіанскаго не знаютъ. Умъ острятъ, слова исправляютъ и красятъ, но сердца своего исправити не хотятъ. \textit{Другіи} учатся въ молитвѣ со смиреніемъ и усердіемъ, и просвѣщаются отъ Духа Святаго, и суть мудрѣйшіи паче философовъ вѣка сего, суть благочестивіи, святіи и Богу любезніи. Сіи хотя алфавита не знаютъ, но добро все разумѣютъ; просто, грубо говорятъ, но красно и благопріятно живутъ. Симъ, Хрістіанине, подражай, Божіе слово, яко Божіе, слушай, почитай и молись усердно Богу и Творцу слова, да тя умудритъ во спасеніе. "--- 2)~Когда хощемъ по правилу слова Божія провождать житіе наше, должно очи отвратить отъ суеты міра, то"=есть, славолюбія, честолюбія, сребролюбія, и сластолюбія, и о семъ такожде молиться Богу по примѣру Псаломника: \textit{отврати очи мои, еже не видѣти суеты}\footnote{Пс.~118,~37.}. Ибо чувства наша стремятся къ тому, что видятъ, и сердце наше растлѣнное похотствуетъ. Надобно убо молитвою противу сего пагубнаго склоненія подвизаться и отвращать очи сердечныя отъ мірской суеты, и обращать къ небеснымъ. Невозможно бо и слово Божіе любить, много паче по слову Божію жить, и міръ, и яже въ мірѣ, любить. Ибо міра любовь Божію слову противна. Слово бо Божіе отъ любви міра насъ отвращаетъ\footnote{1~Іоан.~2,~15--17.}, и привлекаетъ къ любви Божіей и любви ближняго. Слово Божіе велитъ намъ \textit{горняя мудрствовати, а не земная}\footnote{Кол.~3,~2.}, на что оно и дано намъ. Тѣмъ самымъ отвращается Божія слова, презираетъ и пренебрегаетъ его, кто земная мудрствуетъ, а не горняя. Какъ убо ни обращай мысль твою, человѣче, надобно едино изъ двухъ избрать: или слово Божіе любить, и міръ оставить, или слово Божіе оставить, и къ міру прилѣпиться, "--- а обоихъ вдругъ держаться невозможно, ибо противны суть. Слово Божіе отъ міра отвращаетъ сердце наше: міръ отъ слова Божія отвращаетъ его. Надобно убо единому коему изъ нихъ повинуться и держаться неотступно. Аще убо хощемъ слова Божія держаться и исправить себе, должно отъ любви міра отстать; иначе никакого успѣха не будетъ. "--- 3)~Должно читать, или слушать слово Божіе не ради того, чтобы быть остроумнымъ, быть мудрымъ предъ вѣкомъ симъ, за мудраго считатися отъ людей; сіе бо противно есть слову Божію. Ибо не на сей конецъ оно дано намъ, чтобы намъ славу отъ него въ мірѣ семъ имѣть. Таковый, который читаетъ и проповѣдуетъ слово Божіе, чтобы прославиться на землѣ, хощетъ даръ Божій, то"=есть, слово Божіе, не въ Божію славу, но въ свою суетную обратить, и тако Божію себѣ честь похитить отъ дара Божія, яко тать, "--- что есть превеликій и премерзкій грѣхъ. Но читать, или слушать должно слово Божіе, чтобы умудреннымъ быть во спасеніе, просвѣщенный умъ имѣть къ познанію истины, и сердце, преклонное ко злу, исправить къ добру и творенію воли Божіей; такожде и проповѣдывать его ради созиданія людей, а не ради показанія своея мудрости; на сей бо конецъ оно намъ отъ Бога и предано. Аще убо кто читаетъ, или слушаетъ, или проповѣдуетъ слово Божіе не на сей конецъ, тому слово Божіе не токмо не воспользуетъ, но и повредитъ. Якоже бо свѣтъ видимый больнымъ очамъ вредитъ, тако свѣтъ слова Божія вредить оку лукавому, которое намѣреваетъ и предпріемлетъ не то, что оно хощетъ и намѣреваетъ, но что развращенное его сердце. Отсюду"=то бываетъ, что многіи искуснѣйшіи въ писмени святаго Писанія далеко злѣйшіи бываютъ паче некнижныхъ; Богъ бо у таковыхъ отнимаетъ благодать за презрѣніе Свое и злоупотребленіе Своего Божественнаго дара, и тако безъ благодати Божіей, отъ грѣха въ грѣхъ падаютъ; ибо \textit{гордымъ Богъ противится}, которые не Божіей, но своей славы ищутъ. 4)~Кто хощетъ по правилу слова Божія жить, тому должно внутреннему совѣсти правленію повиноваться и тому послѣдовать. Ибо слово Божіе и совѣсть человѣческая согласны суть между собой. Что бо слово Божіе въ Писаніи положенное, тое внутрь человѣка совѣсть свидѣтельствуетъ; и какъ слово Божіе внѣ, такъ совѣсть внутрь человѣка отъ грѣха отводитъ; и какъ человѣкъ внѣ словомъ Божіимъ, такъ внутрь совѣстію за грѣхъ обличается. Сіе свидѣтельство совѣсти вси люди имѣютъ, ученыи и неученыи, и самыи грубыи народы, паче же и самый недорослый вѣкъ: познаютъ бо, что честно и нечестно есть, что хорошо и что худо; откуду стыдятся и боятся худаго предъ людьми дѣлать, и за худое дѣло, напр. за ложь, воровство и прочую обиду другъ друга обличаютъ, и жалуются другъ на друга. Отъ чего познается \textit{внутреннее совѣсти правленіе} и въ самыхъ отрокахъ и отроковицахъ; кольми паче въ возрастныхъ людяхъ имѣется оно, которыи совершенный возрастъ и разумъ имѣютъ. И сіе правленіе совѣсти есть законъ естественный, хотя то грѣхомъ и потемненъ. Аще убо хощемъ, хрістіанине, правилу совѣсти послѣдовать, то будемъ послѣдовать слову Божію. Слово Божіе велитъ намъ Бога почитать, бояться и слушать: тоежъ и въ совѣсти нашей сыщемъ. Слово Божіе запрещаетъ ближнему зло дѣлать: тоежъ и совѣсть творитъ. Слово Божіе увѣщаваетъ ближнему добро дѣлать: увѣщаваетъ и совѣсть. Грѣшишь противу совѣсти? "--- и противу слова Божія; творишь по совѣсти, "--- творишь и по правилу слова Божія. За что обличаетъ тебе совѣсть, за тое обличаетъ и слово Божіе. Чувствуешь утѣшеніе совѣсти за добродѣтель, въ томъ тебе и слово Божіе похваляетъ и утѣшаетъ. И сія"=то есть \textit{книга}, въ которой законныи и беззаконныи дѣла написуются, и открыется она всякому въ день суда Хрістова, и всякъ пріиметъ по дѣломъ своимъ.

\paragraph*{§\:372.} Послушай наконецъ, хрістіанине, терпѣливо, и пріими совѣтъ, душѣ твоей полезный и нужный. Хрістосъ глаголетъ: \textit{едино есть на потребу}\footnote{Лук.~10,~42.}. Что оно есть? Спасеніе вѣчное. Все, что въ мірѣ есть, минуется, какъ и самъ міръ. При смерти міръ, и все въ мірѣ "--- честь, богатство, славу, утѣхи, друговъ, братію, родителей, дѣтей и самое тѣло оставляемъ. Едиными душами отходимъ отсюду или къ вѣчной жизни, или вѣчной мукѣ. Болѣе всего міра пріобрящемъ, когда спасеніе души сыщемъ. Душа убо тебѣ всего міра дороже быть должна. Напротивъ того, какая тебѣ польза отъ того, что все міра сокровище сыщешь, когда спасеніе душевное потеряешь? \textit{Кая польза человѣку, аще міръ весь пріобрящетъ, душу же свою отщетитъ?} глаголетъ Спаситель міра\footnote{Матѳ.~16,~26.}. Ибо и весь міръ приобрѣтшему надобно его оставитъ, и отойти на оный свѣтъ безъ всего. Чтоже съ того, когда душа, которая здѣ все имѣла, тамо ничего, кромѣ погибели, не будетъ имѣть? О! какое раскаяніе, печаль, скорбь, тоска, воздыханіе, плачь, рыданіе и вопль обыметъ ее тогда! Тогда познаетъ истинно, но поздно свою прелесть. Воистину нѣтъ никакого пріобрѣтенія тамо, гдѣ душѣ пагуба. Истинно сіе есть безъ сумнѣнія. Едино убо спасеніе души такъ нужно намъ, что нѣтъ ничего нужнѣе его. Ради того Богъ пророковъ Своихъ послалъ; ради того слово Свое объявилъ намъ; ради того Сынъ Божій пришелъ въ міръ, пострадалъ, умеръ и воскресъ; ради того Тайны святыя установилъ. Сіе убо \textit{едино на потребу} намъ. Сіе когда сыщемъ, все имѣти будемъ; когда потеряемъ, ничего имѣть не будемъ. Сіе убо едино искать намъ должно болѣе, нежели пищи повседневныя, одежды, здравія, свободы тѣлесныя, покоя и паче всего, что до нынѣшняго житія касается. Ибо все сіе пріобрѣтенное при смерти оставимъ, а спасеніе души сысканное съ собою на оный свѣтъ отнесемъ, и во вѣки имѣти будемъ. "--- Знаеши ли, хрістіанине, къ чему я тебѣ сіе предлагаю? Вотъ къ тому, примѣчай и разсуждай! Сатана со злыми аггелами никогда не спитъ, но всегда бодрствуетъ на нашу погибель, и сѣмя Божія слова, посѣянное на сердцахъ нашихъ, похитить, или чрезъ нанесенную печаль и гоненіе словесъ ради, или чрезъ печаль вѣка сего и лесть богатства подавить, и безплодно сотворить тщится\footnote{Лук.~8,~12--14.}. Плоть наша на насъ со страстьми и похотьми всегда востаетъ, и духъ вѣры потребить хощетъ. Соблазны міра умножаются, и заченшуюся искру Божіей любви погасить нудятся. Другъ отъ друга научаются зла: юный на стараго, меньший на большаго, низшій на высшаго, подвластный на властелина, слуга на господина, сынъ на отца, овца на пастыря смотритъ и указуетъ. Что? "--- «Вотъ"=де и онъ тое и тое дѣлаетъ: для чего и мнѣ не дѣлать?» Что оно такое, что онъ дѣлаетъ? Зло, которое Богъ словомъ Своимъ запретилъ, и неугасающимъ огнемъ казнитъ. И такъ расширяется зло, и, по подобію моровой язвы, души человѣческія, за которыя Хрістосъ кровь Свою проліялъ, поядаетъ; а тако отъ часу умаляются сынове царствія. Откуду Хрістосъ о послѣднемъ времени глаголетъ: \textit{Сынъ человѣческій пришедъ убо обрящетъ ли Свою вѣру на земли}\footnote{18,~8.}? У многихъ вѣра на устахъ будетъ, которые глаголютъ Ему: \textit{Господи, Господи}; но не у многихъ на сердцахъ обрящется тая вѣра, о которой здѣ глаголетъ Хрістосъ, то"=есть, вѣра живая, вѣра любовію поспѣшествуемая. Въ сихъ, такъ бѣдственныхъ, обстоятельствахъ что намъ дѣлать? Чимъ себе отъ враговъ нашихъ защитить, которые не имѣнія нашего и тѣла, но души нашей ищутъ, и вѣчное спасеніе отнять (что ужасно есть) хотятъ? Чимъ, какъ только словомъ Божіимъ и молитвою; читать, или слушать Божіе слово, и внимать ему, и молитвою себе укрѣплять; не смотрѣть на тое, что злобою свирѣпѣющій міръ дѣлаетъ, но что слово Божіе научаетъ, и тщательное разсужденіе слова Божія молитвою усердною въ плодъ желаемый обращать и укрѣплять. Слово Божіе научаетъ, обличаетъ, исправляетъ, наказуетъ, утѣшаетъ. Молитва просвѣщаетъ, вразумляетъ, помоществуетъ, и содержитъ въ хрістіанскомъ дѣлѣ. Убо какъ свѣтильникъ требуетъ елея, чтобы не погаслъ: тако вѣра, душъ нашихъ свѣтильникъ, всегда требуетъ слышанія, поученія, размышленія слова Божія и благодати, дабы не угасла. И какъ свѣтильникъ безъ елея угасаетъ, тако вѣра безъ слова Божія и молитвы исчезаетъ. Якоже убо тѣло пищею на всякій день подкрѣпляемъ, тако душу пищею слова Божія и молитвы всегда питать и укрѣплять намъ должно. Безъ сего бо изнемогаетъ душа, а далѣе и умираетъ, то"=есть, лишается живыя и оживляющія вѣры. Якоже бо тѣлу животъ душа, тако душѣ животъ есть вѣра истинная. Сіе есть, что я хотѣлъ тебѣ здѣ предложить, хрістіанине! Сего ради глаголетъ: \textit{востани спяй, и воскресни отъ мертвыхъ, и освѣтитъ тя Хрістосъ}\footnote{Еф.~5,~14.}.

\subsection[Глава 3-я. О хрістіанскомъ званіи, и како его должно хранить.]{глава третія.\\\bfseries О хрістіанскомъ званіи, и како его должно хранить.}

\begin{quotation}\textit{Молю вы азъ юзникъ о Господѣ, достойно ходити званія, въ неже звани бысте}, и проч.\footnote{4,~1.}\end{quotation}

\paragraph*{§\:373.} Высокими и великолѣпными именами изображаетъ, и къ наставленію, утѣшенію и веселію нашему представляетъ намъ Духъ Святый въ Писаніи званіе хрістіанское, хрістіанине, хотя и ничто такъ, какъ истинные хрістіане, въ мірѣ семъ презираются. Уничижены они предъ міромъ симъ, но почтены о Хрістѣ Іисусѣ предъ Богомъ. Видимъ въ Писаніи святомъ, что они \textit{изъ тьмы призваны въ чудный Божій свѣтъ}\footnote{1~Петр.~2,~9.}, \textit{отъ области сатанины въ царство Хрістово}\footnote{Дѣян.~26,~18; Кол.~1,~13.}, призваны \textit{въ общеніе со Отцемъ и Сыномъ Его Іисусомъ Хрістомъ}\footnote{1~Іоан.~1,~3.}, въ \textit{сыновство живаго Бога}\footnote{Гал.~3,~26.}, въ \textit{обрученіе} Единородному \textit{Сыну Божію}\footnote{2~Кор.~11,~2.}, въ \textit{храмъ} Святому и Животворящему \textit{Духу}\footnote{1~Кор.~6,~19.}, въ \textit{наслѣдіе} вѣчнаго живота и блаженства\footnote{1~Петр.~1,~4; Кол.~1,~12.}, \textit{сожитіе} дружеское съ безплотными силами\footnote{Еф.~2,~19; Евр.~12,~22.}. Суть убо \textit{родъ избранъ, царское священіе, языкъ святъ, люди обновленія}\footnote{1~Петр.~2,~9.}. Высокое и совсѣмъ небесное званіе есть, любезный хрістіанине! Нѣтъ большаго, высшаго, почтеннѣйшаго, славнѣйшаго и благороднѣйшаго имени подъ небесемъ, какъ \textit{хрістіанинъ}. Сему высокопочтеннѣйшему титулу уступить должна всякая міра сего корона, знаменіемъ креста Хрістова не украшенная. На такъ высокое достоинство возводитъ милость и человѣколюбіе небеснаго Отца тлѣннаго человѣка! Возводитъ же вѣрою о Хрістѣ Іисусѣ. Благословенъ Богъ, благоволивый тако!

\paragraph*{§\:374.} Сіе званіе хрістіанамъ, живущимъ въ мірѣ семъ, какъ родѣ прелюбодѣйномъ, грѣшномъ, разсуждать и всегда помнить должно. Сіе да предостерегаетъ ихъ и отвращаетъ отъ грѣха, и поощряетъ ко всякой добродѣтели. Небесное званіе есть: небеснаго жительства и обращенія требуетъ. О семъ молитъ хрістіанъ юзникъ о Господѣ Павелъ, глаголя: \textit{молю вы азъ юзникъ о Господѣ, достойно ходити званія, въ неже звани бысте}\footnote{Еф.~4,~1.}, и чрезъ Павла, сосуда Своего святаго, увѣщаваетъ Духъ Святый: \textit{не призва бо насъ Богъ на нечистоту, но во святость}\footnote{1~Сол.~4,~7.}. \textit{Явися благодать Божія спасительная всѣмъ человѣкомъ, наказующи насъ, да отвергшеся нечестія и мірскихъ похотей, цѣломудренно и праведно и благочестно поживемъ въ нынѣшнемъ вѣцѣ, ждуще блаженнаго упованія и явленія славы великаго Бога и Спаса нашего Іисуса Хріста, Иже далъ есть Себе за ны, да избавитъ ны отъ всякаго беззаконія, и очиститъ Себе люди избранны, ревнители добрымъ дѣломъ}\footnote{Тит.~2,~11--14.}; да тако \textit{не по плоти ходимъ, но по духу}\footnote{Римл.~8,~1 и 4.}, яко \textit{пришельцы и странники} въ мірѣ семъ, \textit{огребающеся отъ плотскихъ похотей, яже воюютъ на душу, житіе свое имуще добро}\footnote{1~Петр.~2,~11 и 12.}; да яко \textit{чада свѣта ходимъ, искушающе, что есть благоугодно Богови, и не пріобщаемся къ дѣломъ неплоднымъ тмы, паче же и обличаемъ}\footnote{Еф.~5,~8,~10 и 11.}; но яко \textit{чада Божія непорочная, посредѣ рода строптива и развращенна} живуще, \textit{и являющеся яко свѣтила въ мірѣ}\footnote{Фил.~2,~15.}. Къ сему да подвигнетъ хрістіанъ \textit{небесное} ихъ званіе; къ сему да привлекаетъ сердца ихъ и помышленія. Хрістіанамъ убо не такъ должно обращаться въ мірѣ семъ, якоже языцы, незнающіи истиннаго Бога, обращаются. Они, какъ званіемъ отъ языковъ отдѣлены, яко \textit{родъ избранъ, царское священіе, языкъ святъ, люди обновленія}, такъ и житіемъ и нравами своими отъ нихъ отличны быть должны. Всякое беззаконіе и грѣхъ имъ мерзокъ долженъ быть, яко оскорбляетъ Отца ихъ небеснаго, и отлучаетъ отъ Него согрѣшающаго и лишаетъ его высокаго хрістіанскаго благородія, и дѣлаетъ подлѣйшимъ невольникомъ своимъ: \textit{всякъ бо творяй грѣхъ, рабъ есть грѣха}\footnote{Іоан.~8,~34.}. \textit{Блудъ и всякая нечистота и лихоимство ниже да именуется въ нихъ, якоже подобаетъ святымъ}\footnote{Еф.~5,~3.}. \textit{Ихъ бо тѣлеса суть храмъ живущаго въ нихъ Духа Святаго}\footnote{1~Кор.~6,~19 и 16.}. Какъ великое неистовство есть "--- изъ храма Божія храмомъ бѣсовскимъ дѣлать! \textit{Аще кто Божій храмъ растлитъ, растлитъ того Богъ}\footnote{3,~17.}. Ихъ тѣлеса суть удове Хрістовы. \textit{Вземъ ли убо уды Хрістовы, сотворю уды блудничи? да не будетъ! Прилѣпляяйся сквернодѣйцы, едино тѣло есть съ блудодѣйцею: будета бо, рече, оба въ плоть едину}\footnote{6,~15 и 16.}. О коль страшное и жалостное дѣло есть "--- изъ удовъ Хрістовыхъ удами блудничими дѣлать, сквернитъ душу и тѣло, которое кровію Хрістовою омовено и очищено, и удомъ Хрістовымъ и жилищемъ Святаго Духа содѣлалось! Многихъ и горячихъ слезъ, плача и рыданія достойно дѣло сіе! "--- Хрістіане уста свои отверзаютъ на молитву, славословіе, хвалу и пѣніе святаго имени Божія, и съ Богомъ Святымъ въ молитвѣ бесѣдуютъ: какъ опасно тыяжде уста отворять на хулу, клевету, злословіе, кощуны, осужденіе, пѣсни мерзкія, сквернословіе и прочія гнилыя слова! "--- Они устами пріемлютъ пречистое и животворящее тѣло и кровь Хрістову: какъ убо страшно тыяжде уста осквернять буесловіемъ и сквернословіемъ! "--- Они руки свои простираютъ къ Богу: какъ не беззаконно тыяжде руки простирать на похищеніе, лихоиманіе, біеніе и прочія беззаконныя дѣла! "--- Они должны помышлять, что они \textit{едино тѣло суть о Хрістѣ, а по единому другъ другу уди}\footnote{Рим.~12,~15.}. \textit{Едино тѣло, единъ духъ, якоже и звани суть во единѣмъ упованіи званія своего; единъ Господь, едина вѣра, едино крещеніе: единъ Богъ и Отецъ всѣхъ, Иже надъ всѣми и чрезъ всѣхъ, и во всѣхъ насъ}\footnote{Еф.~6,~4--6.}. Какъ убо любовно, согласно и мирно, яко едино тѣло и другъ другу уды, жити должны! Какъ убо неистовно есть устремляться на уды того же тѣла, и ихъ озлоблять! "--- Они должны и тому внимать, что они позваны къ вѣчному животу, славѣ и богатству; наслѣдіе ихъ не на землѣ, но на небеси, гдѣ и Отецъ ихъ Богъ и отечество ихъ: какъ несмысленно имъ прилѣпляться къ земнымъ, временнымъ и тлѣннымъ! Какъ безумно искать долгаго житія на землѣ, которымъ уготованъ вѣчный животъ на небеси! Какъ безразсудно гоняться за честію и славою земною и тлѣнною, которымъ обѣщана нетлѣнная слава на небеси! Какъ не суетно желать и собирать сокровища на землѣ тѣмъ, которые надѣются получить сокровище вѣчное на небеси! Сіе хрістіанамъ помнить должно, разсуждать, и сему внимать и своего небеснаго благородія, паче живота своего, берещи. \textit{Призваны они изъ тмы въ чудный Божій свѣтъ}\footnote{1~Петр.~2,~9.}, обращаются паки отъ свѣта во тьму, когда обращаются отъ Бога ко грѣху. Ибо грѣхъ есть тьма, и пребывающій во грѣхѣ, во тьмѣ пребываетъ. \textit{Общеніе ихъ со Отцемъ и Сыномъ Его Іисусъ Хрістомъ}\footnote{1~Іоан.~1,~3.}. Великая честь и умомъ непостижимая есть общеніе имѣть съ Богомъ святымъ и вѣчнымъ; но теряютъ общеніе сіе прелюбезное, когда во тьмѣ пребываютъ. Ибо \textit{Богъ свѣтъ есть, и тмы въ Немъ нѣсть ни единыя. Аще речемъ, яко общеніе имамы съ Нимъ и во тмѣ ходимъ, лжемъ и не творимъ истины}\footnote{1,~5 и 6.}. Имъ Апостолъ приписуетъ великолѣпный титулъ \textit{Божіихъ сыновъ: вси вы сынове Божіи есте вѣрою о Хрістѣ Іисусѣ}\footnote{Гал.~3,~26.}. И Іоаннъ святый Апостолъ со удивленіемъ на тойжде титулъ имъ указуетъ: \textit{видите, какову любовь далъ есть Отецъ намъ, да чада Божія наречемся, и будемъ}\footnote{1~Іоан.~3,~1.}. \textit{Аще же чада, и наслѣдницы: наслѣдницы убо Богу, снаслѣдницы же Хрісту}\footnote{Римл.~8,~17.}. Но лишаются (о жалости и неутѣшнаго плача достойное дѣло!), лишаются того небеснаго имени и дражайшаго наслѣдія, когда падаютъ въ грѣхи и погубляютъ истинную свою вѣру. \textit{Кое бо причастіе правдѣ къ беззаконію? или кое общеніе свѣту ко тмѣ? Кое же согласіе Хрісту съ веліаромъ}\footnote{2~Кор.~6,~14 и 15.}? О грѣхъ "--- горькое сѣмя и пагубное, который толикаго блаженства насъ лишаетъ, и въ толикое ввергаетъ бѣдствіе! \textit{Человѣкъ въ чести сый не разумѣ, приложися скотомъ несмысленнымъ и уподобися имъ}\footnote{Пс.~48,~13 и 21.}. И \textit{мужъ безуменъ не познаетъ, и неразуменъ не разумѣетъ сихъ}\footnote{91,~7.}. О когда бы бѣдный человѣкъ сію высокую хрістіанскую честь и славу хотя мало позналъ, и тщету ея усмотрѣлъ въ себѣ: рыдалъ бы неутѣшно день и нощь, и желалъ бы все терпѣть, что ни приключится ему, только бы той сыновъ Божіихъ славы не лишиться! Страшна и ужасна геенна; но не менѣе страшно есть лишеніе славы той. О Господи! отверзи сердечныя очи наши, да увидимъ сколько нибудь славу тую, которую обѣщалъ любящимъ Тя, и, познавши бѣдственную тщету нашу, со усердіемъ и вѣрою поищемъ ея у Тебе, Господа славы. Сего ради внимайте сему, хрістіане, которымъ нѣтъ страха душу и тѣло свое нечистотою осквернять, которымъ нѣтъ стыда безчинствовать и по идолопоклоннически жить, которымъ нѣтъ грѣха или за мало почитается ближняго обидѣть, похищать, красть, злословить, клеветать, лгать, обманывать, льстить, лукавновать и подобныхъ себѣ людей прельщать, на ближняго враждовать, злобиться и льстить, и прочія беззаконныя дѣла творить. Внимайте, глаголю, вышереченному званію, и разсуждайте, надлежите ли до того благословеннаго стада, которому пастырь есть Хрістосъ, до того святаго тѣла, которому преблагословенная Глава есть единородный Сынъ Божій; до тѣхъ братій и сыновъ, которымъ Богъ Отецъ есть. Непремѣнно въ сынахъ показуются нравы отчіи; и \textit{овцы Хрістовы гласа Его слушаютъ}\footnote{Іоан.~10,~27.}; и \textit{отъ плода древо познается}\footnote{Лук.~6,~44.}. Нарицаешися хрістіанинъ; но имя хрістіанское безъ житія хрістіанскаго лицемѣрство есть. Призываешь Бога въ молитвѣ Отцемъ; но тщишися ли Богу подобитися нравами, боишися ли Его, любиши ли Его, слушаеши ли Его, якоже сынамъ подобаетъ? Аще ли ни, то молитва и призываніе твое суетно есть, и гласъ только единъ безъ силы и дѣйствія, или паче мерзость есть. Онъ Отецъ есть, но сыновъ Своихъ; а враговъ Своихъ есть страшный Судія и мститель. Глаголеши: \textit{вѣрую во единаго Бога}; но \textit{покажи ми вѣру твою отъ дѣлъ твоихъ}, глаголетъ тебѣ Апостолъ Хрістовъ\footnote{Іак.~2,~18.}. Аще ли не твориши дѣлъ, вѣрѣ приличныхъ, вѣра твоя демонская есть, ибо \textit{и бѣси вѣруютъ, и трепещутъ}, какъ тойжде учитъ Апостолъ въ тойже главѣ 19. Счисляешися между вѣрными; но любиши ли ихъ, какъ братію твою? Аще же ни, внѣ числа ихъ находишися язычникомъ и безбожникомъ. Чаешь воскресенія мертвыхъ; но не противно ли житіе твое надеждѣ твоей? не живеши ли такъ, какъ скотъ, который издохнувши не востаетъ? Аще же сіе есть въ тебѣ, то надежда твоя суетна есть. Чаешь воскресенія, но отъ мертвыхъ дѣлъ востать не хощешь. И хотя воскреснутъ и беззаконники, но не въ животъ, а \textit{въ судъ и поношеніе безконечное}\footnote{Іоан.~5,~29; Дан.~12,~2.}. Видиши убо, бѣдный человѣче, что имя хрістіанское безъ хрістіанскаго житія есть едино лицемѣріе и въ самой вещи ничтоже. О, каковый страхъ, ужасъ, печаль и стыдъ обыметъ тебе въ день суда, что ты имя хрістіанское имѣлъ, но дѣлъ хрістіанскихъ не имѣлъ! Но уже избавиться тогда не можешь того, во что самъ себе здѣ вринулъ, и сыскать того, что здѣ потерялъ. Сего ради нынѣ, пока не ушло время, искать того должно.

\paragraph*{§\:375.} Како"=де мнѣ сыскать тое? \textit{Отвѣтъ}: 1)~Помни: чего ты при крещеніи святомъ отрицался, и что обѣщалъ Богу дѣлать. Ты тогда отрицался сатаны и всѣхъ дѣлъ его, и обѣщался служить Богу вѣрою и правдою. Сіе вспомнивши, сокруши сердце твое, покайся и жалѣй, что ты Бога, Которому обѣщался работать, оставилъ и паки къ сатанѣ, врагу Его, и мерзкимъ дѣламъ его обратился, и тако солгалъ Господу Богу твоему. 2)~Житіе беззаконное остави, и, наченши новое, хрістіанское, прибѣгни къ Богу съ сожалѣніемъ сердечнымъ и раскаяніемъ, что согрѣшилъ предъ Нимъ, и исповѣдай предъ Нимъ заблужденіе свое, и молися съ Давидомъ: \textit{заблудихъ, яко овча погибшее: взыщи раба Твоего}\footnote{Пс.~118,~176.}. 3)~Тако наченши, стой и утверждайся въ добромъ начатіи твоемъ, и молись Самому Богу, чтобы силою Своею укрѣпилъ тебе. Онъ труждающимся помогаетъ; поможетъ и тебѣ, яко \textit{хощетъ спастися тебѣ}\footnote{1~Тим.~2,~4.}. И хотя много отъ сатаны, міра и плоти искушеній и навѣтовъ претерпѣши; но когда будешь крѣпитися въ начатомъ дѣлѣ твоемъ и Бога на помощь призывати, Богъ, яко милосердъ и человѣколюбецъ, не оставитъ тебе. 4)~Тако обратившися, услышиши и ты отъ Бога въ сердцѣ твоемъ, что услышалъ Давидъ: \textit{услышу, что речетъ о мнѣ Господь Богъ: яко речетъ миръ на люди Своя, и на преподобныя Своя, и на обращающія сердца къ Нему}\footnote{Пс.~84,~9.}. Тогда благородіе хрістіанское возвратится къ тебѣ, котораго какъ себѣ, такъ и тебѣ отъ сердца желаю. Аминь.

\subsection[Глава 4-я. О повседневномъ хрістіанскомъ покаяніи.]{глава четвертая.\\\bfseries О повседневномъ хрістіанскомъ покаяніи.}

\begin{quotation}\textit{Аще речемъ, яко грѣха не имамы, себе прельщаемъ, и истины нѣсть въ насъ. Аще исповѣдаемъ грѣхи наша, вѣренъ есть и праведенъ Богъ, да оставитъ грѣхи наша, и очиститъ насъ отъ всякія скверны}\footnote{1~Іоан.~1,~8 и 9.}.\end{quotation}

Когда здѣ о повседневномъ покаяніи предлагается, то не такіи грѣхи разумѣются, которые съ произволенія и умысла дѣлаются, какъ то: блудъ, хищеніе и прочіи симъ подобныи\footnote{1~Кор.~6,~9 и 10; Гал.~5,~19--21 и на прочіихъ мѣстахъ.}. Таковыи грѣшники, понеже даръ крещенія и вѣру потеряли, внѣ числа истинныхъ хрістіанъ находятся. Того ради должно имъ оставить страсти и злыи обычаи свои и начать новое житіе о Хрістѣ Іисусѣ, какъ сказано выше. (О покаяніи такихъ грѣшниковъ смотри въ 1~кн. въ 1"~й ст.~2 части). Нѣтъ тамо и покаянія, гдѣ грѣхи не оставляются; но безполезное совѣсти умягченіе и прелесть діавольская, который такимъ ложнымъ покаяніемъ прельщаетъ людей, чтобы въ вѣчную съ собою погибель ихъ моглъ привлещи. Но разумѣется сіе разсужденіе о немощахъ человѣческихъ, которымъ и самыи благочестивіи и свято живущіи подлежатъ, какъ изъ слѣдующихъ увидишь. Такожде сіе разсужденіе и до тѣхъ грѣшниковъ надлежитъ, которые благодатію Божіею обратилися и, наченши новое житіе, тщатся Богу послушаніе показывать и угождать. Ибо должно и имъ по всякъ день до конца живота о томъ болѣзновать сердцемъ, что Бога безстрашнымъ житіемъ прогнѣвляли; должно поминать грѣхи свои и беззаконія прежнія, и тако смиряться; предъ Богомъ разсуждать, какъ въ бѣдственномъ состояніи были; и отъ того благодатію Божіею избавилися, и за тое Богу, не хотящему смерти грѣшника; усердно благодарить, и милости Божіей безъ сомнѣнія надѣяться. И понеже и сіи, хотя отстали отъ грѣховъ и въ числѣ благочестивыхъ находятся, однакожъ тѣмже немощамъ подлежатъ, какъ и прочіи благочестивіи, то и они повседневнаго покаянія требуютъ.

\paragraph*{§\:376.} Двоякій грѣхъ, по разсужденію учителей церковныхъ, примѣчается. Иной есть \textit{владѣющій} человѣкомъ, о каковомъ глаголетъ Апостолъ: \textit{да не царствуетъ грѣхъ въ мертвенномъ вашемъ тѣлѣ, во еже послушати его въ похотехъ его}\footnote{Римл.~6,~12.}. Сей грѣхъ иначе называется \textit{самопроизвольный}, яко съ хотѣніемъ, совѣтомъ, умысломъ и знаніемъ и предразсужденіемъ дѣлается, и противу совѣсти бываетъ, которую и уязвляетъ и безпокойствуетъ, "--- напр. соблудить, похитить, украсть чужое, обмануть, прельстить ближняго, злобиться на него, и проч. Сей грѣхъ называется \textit{смертный}, яко смерть вѣчную согрѣшающему, когда не покается, содѣловаетъ, по ученію апостольскому: \textit{оброцы грѣха смерть}\footnote{ст.~23.}; и паки: \textit{грѣхъ содѣянъ раждаетъ смерть}\footnote{Іак.~1,~15.}. Съ таковымъ грѣхомъ вѣра и благодать Святаго Духа въ человѣкѣ купно быть не можетъ, но изгоняется чрезъ него; и согрѣшающій человѣкъ тако лишается всѣхъ благъ, чрезъ Хріста пріобрѣтенныхъ; и Хрістосъ ему ничего не пользуетъ, пока тако грѣшить не перестанетъ. Отъ таковыхъ грѣховъ истинніи хрістіане берегутся, вѣрою и благодатію Святаго Духа подвизаются противу ихъ\footnote{О каковыхъ грѣхахъ поминается: 1~Кор.~6,~9 и 10; Гал.~5,~19--21 и на прочіихъ мѣстахъ Писанія.}. "--- Другій грѣхъ есть \textit{живущій} въ человѣкѣ, но \textit{не обладающій} человѣкомъ; и не иное что есть, какъ \textit{немощь} природная, \textit{плоть} со страстьми и похотьми востающая на духа, \textit{растлѣніе} и злость, съ человѣкомъ родившаяся, "--- которому грѣху силою вѣры и духа противятся благочестивіи, и не попускаютъ владѣти надъ собою. Напр. востаетъ похоть къ нечистотѣ, или ко отмщенію, или ненависти, или обидѣ ближняго, и проч., "--- они тую похоть укрощаютъ, усмиряютъ и побѣждаютъ. И хотя сему грѣху противятся благочестивіи, однакожъ, немощію убѣждаеми, часто \textit{творятъ, чего не хотятъ и не знаютъ}\footnote{Римл.~7,~19.}.

\paragraph*{§\:377.} Изъ святаго Божія Слова познаемъ, что и самыи благочестивыи и свято живущіи согрѣшаютъ, и согрѣшаютъ много. \textit{Много согрѣшаемъ вси}, глаголетъ Апостолъ\footnote{Іак.~3,~2.}. И Іоаннъ святый исповѣдуетъ: \textit{аще речемъ, яко грѣха не имамы, себе прельщаемъ, и истины нѣсть въ насъ}\footnote{1~Іоан.~1,~8.}. И \textit{за оставленіе грѣховъ помолится всякъ преподобный во время благопотребно}\footnote{Пс.~31,~6.}. И Хрістосъ вѣрнымъ, кровію Его оправданнымъ, и Бога Отца благодатію Его исповѣдающимъ, велитъ молитися: \textit{Отче! остави намъ долги наша} (грѣхи)\footnote{Мѳ.~6,~12.}, вѣдая, яко согрѣшаютъ вси. Велитъ брату оставляти грѣхи; \textit{внемлите себѣ: аще согрѣшитъ къ тебѣ братъ твой, запрети ему; и аще покается, остави ему. И аще седмищи на день согрѣшитъ къ тебѣ, и седмищи на день обратится, глаголя: каюся: остави ему}\footnote{Лук.~17,~3 и 4.}. А кто противу брата грѣшитъ, тотъ грѣшитъ и противу Бога, яко не хранитъ заповѣди Божіей: \textit{возлюбиши искренняго твоего, яко самъ себе}\footnote{Мѳ.~22,~39.}. Откуду и пророкъ святый вопіетъ къ Богу и воздыхаетъ: \textit{не вниди въ судъ съ рабомъ Твоимъ, яко не оправдится предъ Тобою всякъ живый}\footnote{Пс.~142,~2.}. Ибо Богъ не токмо внѣшнія наши дѣла и грѣхи видитъ и судитъ, но и внутреннія помышленія, склоненія, начинанія сердечныя, яко сердца и утробы испытуяй, ясно усматриваетъ и по тѣмъ судитъ. Многихъ грѣховъ своихъ человѣкъ не видитъ, ни познаетъ въ совѣсти своей; но Божіе око, свѣтлѣйшее солнца, все проницаетъ. Сего ради глаголетъ Псаломникъ: \textit{грѣхопаденія кто разумѣетъ}\footnote{18,~13.}? "--- то"=есть, никто не можетъ ихъ познать и исчислить. Откуду и молится къ Богу о очищеніи ихъ: \textit{отъ тайныхъ моихъ очисти мя}, показуя, что ничимъ инымъ оные не очищаются, какъ исповѣданіемъ, усердною молитвою и вѣрою о Хрістѣ Іисусѣ, Егоже кровь очищаетъ отъ всякаго грѣха вѣрующихъ въ Него и исповѣдующихъ грѣхи своя\footnote{1~Іоан.~1,~7.}. Тайные же грѣхи здѣ разумѣются не тыи, которые тайно съ произволеніемъ дѣлаются и отъ человѣкъ сокровенны суть; сіи бо хотя тайно отъ человѣка творятся, но отъ совѣсти его укрытися не могутъ, которая ихъ, паче всякаго свидѣтеля внѣшняго, обличаетъ, и на согрѣшившаго вопіетъ; но такіе разумѣются, которыхъ и самъ человѣкъ въ совѣсти не усматриваетъ, но Богъ ясно и чисто видитъ. Сердце бо человѣческое \textit{глубоко}\footnote{Іер.~17,~9; Пс.~63,~7.}, и смертоноснымъ зміина сѣмени ядомъ заражено такъ, что всякъ много имѣетъ труда разсматривать его, оплакивать, омывать вѣрою нечистоту его. \textit{Кто} бо \textit{похвалится чисто имѣти сердце? или кто дерзнетъ рещи чиста себе быти отъ грѣховъ?} глаголетъ Соломонъ\footnote{Притч.~20,~9.}. Сіе окаянство оплакиваетъ Апостолъ: \textit{окаяненъ азъ человѣкъ: кто мя избавитъ отъ тѣла смерти сея?} Причину тому выше полагаетъ: \textit{вижду}, рече, \textit{инъ законъ во удѣхъ моихъ, воюющъ противу закона ума моего, и плѣняющъ мя закономъ грѣховнымъ, сущимъ во удѣхъ моихъ}, и проч.\footnote{Римл.~7,~24 и 23.} Сего ради всѣмъ, благочестиво живущимъ, должно разсматривать сердце свое, и немощь свою разсуждать, и съ пророкомъ исповѣдатися предъ Богомъ: \textit{Тебѣ, Господи, правда, намъ же стыдѣніе лица}\footnote{Дан.~9,~7.}, "--- и отъ правды Божіей и праведнаго Его суда къ милосердію Его о имени Хрістовомъ прибѣгать: \textit{не вниди въ судъ съ рабомъ Твоимъ, яко не оправдится предъ Тобою всякъ живый}\footnote{Пс.~142,~2.}, "--- и молитися: \textit{Отче! остави намъ долги наша, якоже и мы оставляемъ должникомъ нашимъ}\footnote{Мѳ.~6,~12.}. \textit{Аще бо исповѣдаемъ грѣхи наша, вѣренъ есть и праведенъ, да оставитъ намъ грѣхи наша, и очиститъ насъ отъ всякія скверны}\footnote{1~Іоан.~1,~9.}.

\paragraph*{§\:378.} Источникъ, отъ котораго почерпаютъ воду веселія, то"=есть, утѣшеніе и оставленіе грѣховъ, вѣрніи, есть Хрістосъ Сынъ Божій. Къ сему живыя воды источнику прибѣгаютъ они, зноемъ гнѣва Божія, который въ совѣсти своей чувствуютъ, палимы, и тако прохлаждаются и воспріемлютъ утѣшительное вѣры подкрѣпленіе, восклицая съ богомудрымъ Павломъ: \textit{Иже своего Сына не пощадѣ, но за насъ всѣхъ предалъ есть Его: како убо не и съ Нимъ вся намъ дарствуетъ}\footnote{Римл.~8,~32.}? Аще Богъ Сына Своего ради насъ не пощадѣлъ (о глубина богатства благости Божіей!), то како ради Тогожде Сына намъ, вѣрующимъ въ Него и исповѣдающимъ грѣхи свои предъ Нимъ и отпущенія просящимъ, не отпуститъ грѣхи и не явитъ милости? Аще великое подалъ, когда благоволилъ послать Сына Своего ради насъ въ міръ, малаго ли не подастъ намъ? На то бо и пришелъ Хрістосъ въ міръ, чтобы насъ отъ грѣховъ избавить, отъ которыхъ мы сами избавиться не могли, и подать намъ правду Свою, которой мы отъ закона сыскать не могли, яко немощныи, и тако отворить намъ небесное царствіе, которое мы преслушаніемъ своимъ затворили. Сей конецъ есть благоволенія небеснаго Отца о насъ и пришествія Единороднаго Сына Его въ міръ. Данъ былъ намъ законъ Божій, въ которомъ вѣчная правда изображена. Тотъ должно было намъ или совершенно исполнить, и тако праведными предъ Богомъ быть, и животъ вѣчный получить, по Писанію: \textit{сотворивый та} (въ законѣ написанная) \textit{человѣкъ, живъ будетъ въ нихъ}\footnote{Гал.~3,~12.}; или неисполнившимъ подъ клятвою закона быть, по реченному: \textit{проклятъ всякъ человѣкъ, иже не пребудетъ во всѣхъ словесѣхъ закона сего, еже творити я}\footnote{Второз.~27,~26; Гал.~3,~10.}. Но грѣхъ, живущій въ человѣкѣ, большую воспріялъ отъ закона силу, и былъ какъ бы раздраженъ отъ него; и что законъ повелѣвалъ, или запрещалъ, большую въ немъ возжигалъ противу того похоть. Чего ради человѣкъ не моглъ того, что законъ отъ него хотѣлъ, исполнить, немощію и похотію побѣждаемый; и тако законъ обличалъ его, яко немощнаго и преступника, а не исцѣлялъ. И хотя человѣкъ не моглъ того дѣлать, что законъ ему повелѣвалъ; но правда Божія своего требовала отъ него; требовала того, что онъ долженъ Создателю и Богу своему, то"=есть, совершеннаго послушанія и почитанія; и тако опредѣлила дабы человѣкъ или весь законъ Божій исполнилъ и благословеніе Божіе получилъ, или неисполнившій лишился Божія благословенія и клятвою пораженъ былъ. Правда Божія нарушиться не можетъ, но требуетъ неотмѣнно законопреступнику наказаннымъ быть; ибо правды свойство есть всякому должное отдавать. Не сыскался человѣкъ, который бы законъ Божій совершенно исполнилъ: \textit{вси бо согрѣшиша, и лишени суть славы Божія}\footnote{Рим.~3,~23.}. Не учинилъ никто того, чего правда Божія требовала, и всякъ обличался отъ закона, яко преступникъ и непокоривый. Слѣдственно и остался всякъ по опредѣленію Божіей правды подъ клятвою, и подпалъ вѣчному осужденію и смерти. Милосердіе и человѣколюбіе Божіе не терпѣло человѣка оставить подъ клятвою и видѣть въ вѣчной погибели, правда Божія своего требовала: и милосердіе Божіе своего хотѣло. Надобно было и правдѣ Божіей удовольствіе свое получить: и милосердію Его не остаться безъ удовольствія своего. Богъ праведный и милосердный изобрѣлъ по Своей премудрости посредствіе, которымъ и правда Его святая и милосердіе Его удовольствовалось. Благоволилъ Сыну Своему, Иже есть Ѵпостасная Божія Премудрость, воплотитися. Явился во плоти Сынъ Божій, и за насъ, проклятыхъ отъ закона, учинился клятвою, единъ \textit{Благословенный во вѣки}\footnote{9,~5.}, и тако \textit{искупилъ насъ отъ клятвы законныя, бывъ по насъ клятва}\footnote{Гал.~3,~13.}; и, вмѣсто насъ, неисполнившихъ закона, совершенно исполнилъ законъ, \textit{да оправданіе закона исполнится въ насъ, не по плоти ходящихъ, но по духу}\footnote{Римл.~8,~14.}. Тако законъ Божій конецъ свой, то"=есть, исполненіе совершенное, во Хрістѣ получилъ, чего въ насъ не получилъ. Тако правдѣ Божіей и милосердію Его удовлетворила Ѵпостасная Отчая Премудрость "--- Хрістосъ. Правдѣ: яко \textit{Той язвенъ бысть за грѣхи наша, и мученъ бысть за беззаконія наша}\footnote{Ис.~53,~5; Римл.~4,~25; 1~Кор.~15,~3.}. Милосердію: яко человѣкъ проклятый и вѣчному осужденію подверженный, \textit{благословеніе вѣрою во Хріста получаетъ отъ Бога, и спасается}\footnote{1~Петр.~3,~9; Гал.~3,~9 и 14; Римл.~5,~9 и 10.}. И чего въ законѣ не моглъ сыскать, яко немощный, тое все вѣрою во Хрістѣ обрѣтаетъ. Тако \textit{милость и истина срѣтостѣся, правда и миръ облобызастася. Истина отъ земли возсія, и правда съ небесе приниче}\footnote{Пс.~84,~11 и 12.}. Отъ сего Источника живаго прохлаждаются вѣрніи; на Іисуса, вознесеннаго на крестѣ и грѣхи наша вознесшаго на тѣлѣ Своемъ на древо, вѣрою \textit{взираютъ и Его язвою исцѣляются}\footnote{Евр.~12,~2; 1~Петр.~2,~24.}. \textit{Якоже бо Моисей вознесе змію въ пустыни, тако подобало вознестися Сыну человѣческому: да всякъ, вѣруяй въ Онь, не погибнетъ, но имать животъ вѣчный}\footnote{Іоан.~3,~14 и 15.}. О семъ дерзаютъ со Апостоломъ: \textit{Ходатая имамы ко Отцу, Іисуса Хріста Праведника. И Той очищеніе есть о грѣсѣхъ нашихъ, не о нашихъ же точію, но и всего міра}\footnote{1~Іоан.~2,~1 и 2.}, "--- Который несовершенное ихъ послушаніе Своимъ совершеннымъ послушаніемъ дополняетъ, и Своимъ совершенствомъ недостатки и немощи ихъ прикрываетъ, и тако \textit{отъ исполненія Его вси пріемлютъ}\footnote{Іоан.~1,~16.}. Онъ \textit{бысть намъ Премудрость отъ Бога, правда же и освященіе и избавленіе}\footnote{1~Кор.~1,~30.}. Къ сему убо прибѣжищу нищихъ и убогихъ, и Врачу немощей человѣческихъ, и грѣховъ потребителю должно и намъ, хрістіанине, прибѣгать, и вѣрою взирать на Него, да, якоже сыны Израилевы, угрызаемыи отъ зміевъ въ пустынѣ, взирали на вознесенную отъ Моисеа змію, и исцѣлялися\footnote{Числ.~21,~9.}: тако мы, вѣрою взирая на вознесеннаго Сына Божія, исцѣлимся отъ угрызеній ядовитыхъ адскаго змія, діавола. Хрістосъ Сынъ Божій утѣшеніе, похвала и слава наша, хрістіане, \textit{умерый за насъ и воскресый, сѣдитъ одесную Бога, Иже и ходатайствуетъ о насъ}\footnote{Римл.~8,~34.}; Иже съ высоты славы Своея милостивно \textit{зритъ на кроткихъ и смиренныхъ и трепещущихъ словесъ Его}\footnote{Ис.~66,~2.}; \textit{Иже трости сокрушенны не преломитъ, и лена внемшася не угаситъ}\footnote{42,~3; Матѳ.~12,~20.}; Иже Самъ рече: \textit{не требуютъ здравіи врача, но болящіи. Не пріидохъ призвати праведники, но грѣшники на покаяніе}\footnote{Матѳ.~9,~12 и 13.}. "--- \textit{Да приступаемъ убо съ дерзновеніемъ ко престолу благодати Его, да пріимемъ милость, и благодать обрящемъ, во благовременну помощь}\footnote{Евр.~4,~16.}.

\paragraph*{§\:379.} Истинные хрістіане, когда исповѣдаютъ грѣхи свои предъ Богомъ, и отпущенія ихъ отъ Него просятъ, то и сами человѣкамъ согрѣшенія ихъ оставляютъ, якоже научилъ ихъ Хрістосъ молитися: \textit{Отче! остави намъ долги наша, якоже и мы оставляемъ должникомъ нашимъ}\footnote{Матѳ.~6,~12.}. Того ради, хрістіанине, намъ должно разсуждать тое, что придаетъ Господь по молитвѣ той: \textit{аще отпущаете человѣкомъ согрѣшенія ихъ, отпуститъ и вамъ Отецъ вашъ небесный. Аще ли не отпущаете человѣкомъ согрѣшенія ихъ, ни Отецъ вашъ отпуститъ вамъ согрѣшеній вашихъ}\footnote{Ст.~14 и 15.}; то"=есть, должны и мы оставлять ближнимъ нашимъ согрѣшенія ихъ. Откуду премудрый Сирахъ глаголетъ: \textit{отмщаяй, отъ Господа обрящетъ отмщеніе, и грѣхи своя соблюдаяй, соблюдетъ. Остави обиду искреннему твоему, и тогда, помольшуся тебѣ, грѣси твои разрѣшатся. Человѣкъ на человѣка сохраняетъ гнѣвъ; а отъ Господа ищетъ исцѣленія? Надъ человѣкомъ подобнымъ себѣ, не имать милости; а о грѣсѣхъ своихъ молится? Самъ сый плоть, хранитъ гнѣвъ: кто очиститъ грѣхи его}\footnote{Сир.~28,~1--5.}? Сего ради Хрістосъ Петру Апостолу, вопросившему Его: \textit{Господи, коль краты аще согрѣшитъ въ мя братъ мой, и отпущу ли ему до седмь кратъ?} отвѣщалъ: \textit{не глаголю тебѣ до седмь кратъ, но до седмьдесять кратъ седмерицею}\footnote{Мѳ.~18,~21 и 22.}. Чего ради и притчу приложилъ Господь о должникѣ, тьмою талантъ царю своему одолжившемся, который отъ царя своего прощеніе долга великаго по единой милости его получилъ, но клеврету своему ста пѣнязей оставить не хотѣлъ, и всадилъ его въ темницу, дондеже отдастъ должное, за что и самъ отъ царя своего прогнѣваннаго преданъ мучителямъ\footnote{ст.~23--34.}. Которая притча не иное что значитъ, какъ что содержащему гнѣвъ на ближняго своего, и согрѣшенія его не оставляющему, не токмо не оставятся грѣхи отъ Бога, но и прежніи его грѣхи оставленныи возвращаются и поминаются. Ибо царь оный милостивый простилъ было должнику тому долгъ, но за немилосердіе его, надъ братомъ своимъ показанное, паки возвратилъ на него долгъ, и предалъ мучителямъ во истязаніе долга. Откуду заключаетъ Господь притчу тую такъ: \textit{тако и Отецъ Мой небесный сотворитъ вамъ, аще не отпустите кійждо брату своему отъ сердецъ вашихъ прегрѣшенія ихъ}\footnote{Матѳ.~18,~35.}. Сего ради, когда получаемъ отъ Бога великихъ долговъ нашихъ прощеніе, то ради толикой къ намъ Божіей милости должны и сами долги малыи ближнимъ нашимъ прощать, да не и намъ тое приключится, что лукавому оному евангельскому рабу. Всего міра грѣхи потреблены на древѣ крестномъ, но отпущенія грѣховъ только сподобляются отъ Бога, которыи престаютъ грѣшить, каются за грѣхи и вѣруютъ во Хріста, умершаго за грѣхи наши и воскресшаго. Нѣтъ же тамо истиннаго покаянія, но притворное и ложное, гдѣ сердце злобою и гнѣвомъ исполненное. Самый гнѣвъ, держимый на ближняго, есть тяжкій грѣхъ, по свидѣтельству Апостола: \textit{всякъ, ненавидяй брата своего, человѣкоубійца есть}\footnote{1~Іоан.~3,~15.}. Онъ"=де мнѣ зло дѣлаетъ, и обиду чинитъ? Правда; но и ты равно зло дѣлаешь, когда злобу на него держиши; и обиду ему чинишь, когда словомъ или дѣломъ ему отмщеваешь. Онъ прежде тебѣ сдѣлалъ зло, но ты послѣ дѣлаешь ему зло. Ибо зло равно есть зло, и обида равно есть обида, прежде или послѣ сотворенная. Разсуждай сіе, возлюбленный хрістіанине, и не смотри на тое, что нынѣшняго вѣка хрістіане дѣлаютъ, которые за малое поносное или укорительное слово, или судебныя мѣста наполняютъ клеветами на ближняго, или хотятъ его погубить, не разсуждая того, что на всякій день и часъ безчестятъ и оскорбляютъ величество Божіе, и по дѣламъ своимъ не пріемлютъ. Но, великой милости отъ Бога сподобляяся, и самъ ближнему твоему милосердъ буди, да не и самъ лишишися Божіей милости, и во вѣки погибнеши. Сего ради, ежели хощеши Бога къ себѣ милостиваго имѣть, буди и самъ къ ближнему твоему милостивъ. Ближній нашъ положенъ намъ отъ Бога ко искушенію и познанію нашему о Бозѣ къ намъ. Любимъ мы ближняго: любимъ и Бога самаго, и Богъ насъ любитъ.

Благоволимъ о ближнемъ нашемъ, благоволитъ и Богъ о насъ. Милуемъ мы ближняго нашего, милуетъ и Богъ насъ. Прощаемъ согрѣшенія ближнему нашему, прощаетъ и Богъ намъ согрѣшенія наша. Ненавидимъ и гнѣваемся на ближняго нашего, находимся и сами во гнѣвѣ у Бога. Не отпущаемъ согрѣшеній ближнему нашему, не отпущаетъ и Богъ намъ согрѣшеній нашихъ. Мстимъ мы ближнему нашему, будетъ мстить и Богъ намъ, якоже глаголетъ Сирахъ: \textit{отмщаяй, отъ Господа обрящетъ отмщеніе}. Тако, каковы мы къ ближнему нашему, таковъ и Богъ къ намъ.

\subsection[Глава 5-я. Яко, хрістіанинъ долженъ весьма берещися отъ смертныхъ грѣховъ, которые противу совѣсти бываютъ, и уязвляютъ ее.]{глава пятая.\\\bfseries Яко, хрістіанинъ долженъ весьма берещися отъ смертныхъ грѣховъ, которые противу совѣсти бываютъ, и уязвляютъ ее.}

\begin{quotation}\textit{Да не царствуетъ грѣхъ въ мертвеннѣмъ вашемъ тѣлѣ, во еже послушати его въ похотехъ его; ниже представляйте уды ваша оружія неправды грѣху: но представляйте себе Богови, яко отъ мертвыхъ живыхъ, и уды ваша оружія правды Богови}\footnote{Римл.~6,~12 и 13.}.\end{quotation}

\paragraph*{§\:380.} Что внѣ, въ Словѣ своемъ Богъ запрещаетъ, тое внутрь, человѣку совѣсть его претитъ. Итакъ слово Божіе, написанное на хартіи, и совѣсть человѣческая въ едино сходятся, намѣреваютъ, и къ единому концу, то"=есть, чистому, святому и непорочному житію ведутъ человѣка. Сего ради кто противу совѣсти грѣшитъ, тотъ грѣшитъ и противу Божія слова; и который гнѣвъ Божій чувствуетъ человѣкъ внутрь себе, въ совѣсти, тойже гнѣвъ Божій изображается и въ словѣ Божіемъ противу его: такожде какую милость Божію и миръ чувствуетъ внутрь себе, таяжде милость и миръ Божій и въ словѣ Его святомъ означается и представляется человѣку. \textit{Елицы правиломъ симъ (вѣру, любовію поспѣшествуемую}, имѣютъ) \textit{жительствуютъ, миръ на нихъ и милость, и на Израили Божіи}\footnote{Гал.~5,~6; 6,~16.}. Такожде и пророкъ святый Давидъ какую милость и миръ почувствовалъ внутрь себе, тоежде и написалъ: \textit{услышу, что речетъ о мнѣ Господь Богъ: яко речетъ миръ на люди Своя, и на преподобныя Своя, и на обращающія сердца къ Нему}\footnote{Пс.~84,~9.}. Сего ради хрістіанамъ, когда не хотятъ противу слова Божія и самаго Бога согрѣшить, и тако милость Божію потерять и праведному Его гнѣву подпасть, должно берещи совѣсть свою чистую отъ такихъ дѣлъ, которыя ее уязвляютъ, безпокойствуютъ и лишаютъ благопріятнаго мира: все тое, за что въ словѣ Божіемъ возвѣщается гнѣвъ Божій, и совѣсть безпокойствуетъ и въ страхъ приводитъ, которая чувствуетъ въ себѣ тотъ грѣхъ, за который гнѣвъ Божій означается. Того ради должно намъ, хрістіанине, совѣсти нашей и слову Божію внимать, и удаляться отъ плотскихъ страстей, которыя \textit{воюютъ на душу}\footnote{1~Петр.~2,~11.}. \textit{Явлена суть дѣла плотская, яже суть прелюбодѣяніе, блудъ, нечистота, студодѣяніе, идолослуженіе, чародѣянія, вражды, рвенія, завиды, ярости, разжженія, распри, соблазны, ереси, зависти, убійства, піянства, безчинны кличи, и подобная симъ: яже предглаголю вамъ и предрекохъ, яко таковая творящіи царствія Божія не наслѣдятъ}\footnote{Гал.~5,~19--21.}. И паки: \textit{не льстите себе: ни блудницы, ни идолослужители, ни прелюбодѣи, ни сквернители, ни малакіи, ни мужеложницы, ни лихоимцы, ни татіе, ни піяницы, ни досадители, ни хищницы царствія Божія наслѣдятъ}\footnote{1~Кор.~6,~9 и 10.}. И паки: \textit{блудъ и всяка нечистота и лихоимство ниже да именуется въ васъ, якоже подобаетъ святымъ. И сквернословіе, и буесловіе, или кощуны, яже неподобная, но паче благодареніе. Сіе бо да вѣсте, яко всякъ блудникъ, или нечистъ, или лихоимецъ, иже есть идолослужитель, не имать достоянія въ царствіи Хріста и Бога. Никтоже да льститъ васъ суетными словесы: сихъ бо ради грядетъ гнѣвъ Божій на сыны непокоривыя. Не бывайте убо сопричастницы симъ}\footnote{Еф.~5,~3--7.}. Сему ученію и прещенію да внимаемъ, хрістіанине! Паки хрістіане, яко Божіи человѣцы, должны удаляться отъ сребролюбія: яко \textit{корень всѣмъ злымъ сребролюбіе есть, егоже нѣцыи желающе, заблудиша отъ вѣры}\footnote{1~Тим.~6,~10.}, яко глаголетъ Господь: \textit{не можете Богу работати и мамонѣ}\footnote{Мѳ.~6,~24.}. Должны уклоняться отъ лжи, лести: яко \textit{погубитъ Господь вся глаголющія лжу; мужа кровей и льстива гнушается Господь}\footnote{Пс.~5,~7.}. Должны изгонять изъ сердца злобу, яко имъ глаголетъ Хрістосъ: \textit{аще не отпущаете человѣкомъ согрѣшенія ихъ, ни Отецъ вашъ отпуститъ вамъ согрѣшеній вашихъ}\footnote{Матѳ.~6,~15.}. А кому нѣтъ отпущенія согрѣшеній, тому отъ Бога отмщеніе и вѣчная слѣдуетъ казнь. \textit{Тако и Отецъ Мой небесный сотворитъ вамъ, аще не отпустите кійждо брату своему отъ сердецъ вашихъ прегрѣшенія ихъ}, глаголетъ Хрістосъ\footnote{Матѳ.~18,~35.}. Сего ради симъ и прочіимъ святаго Писанія страшнымъ мѣстамъ, которыя возвѣщаютъ гнѣвъ Божій грѣшникамъ за грѣхи, и совѣсти своей внимати, и отъ такихъ дѣлъ, которыя ее уязвляютъ и безпокойствуютъ, должно опасно берещися, да не, уязвивши совѣсть, потеряютъ покой ея, и милости Божіей лишатся, и попадутъ праведному Его гнѣву.

\paragraph*{§\:381.} Совѣсть уязвляютъ и безпокойствуютъ не токмо внѣшніи грѣхи, которые внѣ, въ самое дѣло производятся, какъ"=то: блудъ, хищеніе, лесть, отмщеніе и прочіи, но и внутренніи, которые чрезъ злые и богопротивные помыслы совершаются. Сего ради хрістіане не токмо отъ внѣшнихъ, но и отъ внутреннихъ берещися должны грѣховъ. Понеже 1)~Богъ въ словѣ Своемъ къ душѣ нашей глаголетъ. Что убо пользуетъ тѣломъ не грѣшить, но душею и сердцемъ грѣхъ дѣлать; руками не убивать, но сердцемъ убивать; тѣломъ не смѣшаться съ женою, но сердцемъ услаждаться нечистою похотію? 2)~Какъ слово Божіе къ душѣ нашей, такъ и совѣсть наша къ душѣ глаголетъ: \textit{не дѣлай того и того}. Слѣдственно и внутренніи грѣхи какъ противу слова Божія, такъ и противу совѣсти, въ душѣ равно дѣлаются, какъ и внѣ, удами тѣлесными. 3)~Богъ какъ внѣшнее, такъ и внутреннее худое дѣло равно видитъ, о чемъ псалма 138"~го ст.~1--16~свидѣтельствуютъ, и на прочіихъ Писанія мѣстахъ глаголется. 4)~И совѣсть за внутреннее худое дѣло и помыслъ, тѣмъ уязвленная и раздраженная, обличаетъ человѣка. 5)~Какъ слово Божіе, грѣховнымъ помысломъ презрѣнное и неисполненное, такъ и совѣсть тѣмже раздраженная, судъ Божій и гнѣвъ Его возвѣщаютъ человѣку; и человѣкъ будетъ судиться не токмо за дѣла и слова, но и за помыслы злые, на страшномъ Хрістовомъ судѣ. 6)~Хрістосъ глаголетъ: \textit{речено бысть древнимъ: не прелюбы сотвориши. Азъ же глаголю вамъ: яко всякъ, иже воззритъ на жену, ко еже вожделѣти ея, уже прелюбодѣйствова съ нею въ сердцѣ своемъ}\footnote{Матѳ.~5,~27 и 28.}. Отсюду заключается, что и тотъ есть прелюбодѣй, который хощетъ прелюбодѣйствовать и нечистыя мысли въ сердцѣ питаетъ, хотя съ женою и не совокупляется тѣломъ; и тотъ \textit{убійца, который ненавидитъ ближняго}\footnote{1~Іоан.~3,~15.}, злобится на ближняго, и хощетъ его повредить или убить, хотя и не убиваетъ его дѣломъ; и тотъ есть тать и хищникъ, который желаетъ чужую вещь похитить, хотя и не похищаетъ дѣломъ; и тотъ піяница, который хощетъ упиваться, хотя и не упивается; и тотъ льстецъ и обманщикъ, который хощетъ прельстить и обмануть, хотя дѣломъ и не совершаетъ того. Тоежъ и о прочіихъ злыхъ помышленіяхъ разумѣть должно. Сей грѣхъ совершается противу заповѣди десятой: \textit{не пожелай}\footnote{Исх.~20,~17.}. И хотя такій человѣкъ не дѣлаетъ ближнему зла дѣломъ, но хощетъ дѣлать, и тако грѣшитъ противу Божія повелѣнія. Ибо Богъ какъ къ душѣ и сердцу нашему глаголетъ, такъ и смотритъ на сердце и волю души, какое кто сердце и волю имѣетъ, и къ чему воля и сердце клонится у человѣка, и потому судитъ ему. Хрістіанину убо не токмо не должно дѣлать, но и мыслить и хотѣть зла, когда не хощетъ противу слова Божія и совѣсти своей согрѣшить, и тако ее уязвить и суду Божію подпасть. "--- 7)~Понеже Богъ къ душѣ нашей глаголетъ, и велитъ или уклоняться отъ зла, или творить благое, то душа какъ благое, такъ и злое творитъ. А тѣло душѣ есть орудіе, которымъ или благое, или злое творитъ. Тѣло не будетъ зла дѣлать, рука похищать, убивать, языкъ злословить и клеветать, ухо клевету слушать, око къ вожделѣнію доброту личную смотрѣть, когда душа не захочетъ. Рукъ дѣло есть дѣлать; но добрѣ или злѣ дѣлать отъ произволенія души бываетъ. Ока есть смотрѣть; а злѣ, или добрѣ смотрѣть, воли нашей есть. Языка есть глаголати; а какъ и что глаголати, отъ воли бываетъ. Такъ и прочіе уды тѣла нашего бываютъ орудія правды, или неправды не \textit{отъ себе}, но отъ воли и произволенія сердечнаго. Душа убо и воля человѣческая грѣшитъ, или добро дѣлаетъ, и чрезъ уды тѣлесные хотѣнія и дѣйствія совершаетъ. Слѣдственно, хотя человѣкъ и не грѣшитъ чрезъ уды тѣлесные, напр. руками не крадетъ, не убиваетъ, тѣломъ не блудодѣйствуетъ, и проч.; но въ душѣ и сердцѣ мысли злыя имѣетъ и тѣмъ соизволяетъ, "--- такожде грѣшитъ предъ Богомъ и суду Его праведному подпадаетъ. Надобно убо намъ, хрістіанине, не токмо отъ внѣшнихъ дебѣлыхъ грѣховъ, но и отъ внутреннихъ злыхъ помышленій, которыя корень и начало суть внѣшнихъ, берещи душу свою, когда хощемъ по"=хрістіански жить. "--- 8)~Истинный хрістіанинъ подобенъ есть \textit{древу доброму}, у котораго какъ внѣ на вѣтвяхъ являются добрые и сладкіе плоды, а не горькіе, такъ и внутрь имѣется добрый и сладкій, а не горькій сокъ. Отъ сока бо и плоды происходятъ на древѣ; а каковый сокъ въ древѣ, таковыи и плоды его показуются на вѣтвяхъ. \textit{Всяко бо древо доброе плоды добры творитъ: а злое древо плоды злы творитъ. Не можетъ древо добро плоды злы творити}, глаголетъ Истина "--- Хрістосъ\footnote{Мѳ.~7,~17 и 18.}. Въ семъ и намъ должно древу подражать. И какъ внѣ не дѣлаемъ зла, такъ и внутрь не дѣлать зла; но какъ внѣ являемся добрыми, такъ и внутрь быть добрыми, по подобію добраго древа; внѣ не дѣлать зла, и внутрь не мыслить зла; внѣ не убивать ближняго, и внутрь не гнѣваться и не злобиться на него; руками не похищать, и внутрь не хотѣть того; языкомъ не злословить, и внутрь не мыслить того; тѣломъ не гордиться, и умомъ и сердцемъ не возноситься; тѣломъ не скверниться, и сердце отъ того чисто соблюдать, да будетъ истинное души съ тѣломъ согласіе въ благочестіи, якоже союзъ и согласіе имѣется въ составѣ ихъ естественномъ, да тако душу и тѣло наше въ послушаніе и жертву Богу нашему посвятимъ. Тако будемъ истинно доброе древо, которое и внѣ и внутрь доброе. Сего ради глаголетъ Апостолъ: \textit{прославите Бога въ тѣлесѣхъ вашихъ, и въ душахъ вашихъ, яже суть Божія}\footnote{1~Кор.~6,~20.}. Отсюду видишь, хрістіанине, кто есть истинный, и кто есть ложный хрістіанинъ и лицемѣръ. То"=есть, истинный хрістіанинъ, который и не дѣлаетъ и не хощетъ дѣлать зла, но паче хощетъ и тщится дѣлать добро ради Бога; якоже ложный есть и лицемѣръ тотъ, который отъ внѣшнихъ грѣховъ воздерживается и показываетъ себе внѣ добрымъ, но внутрь золъ есть, и подобенъ \textit{гробамъ повапленнымъ, которые внѣ являются красны, внутрь же полны суть костей мертвыхъ и всякія нечистоты}\footnote{Матѳ.~23,~27.}.

\subsection[Глава 6-я. Яко хрістіанинъ долженъ вожделѣніямъ и похотямъ плотскимъ противитися, и ихъ благодатію Божіею побѣждати и умерщвляти.]{глава шестая.\\\bfseries Яко хрістіанинъ долженъ вожделѣніямъ и похотямъ плотскимъ противитися, и ихъ благодатію Божіею побѣждати и умерщвляти.}

\begin{quotation}\textit{Духомъ ходите, и похоти плотскія не совершайте. Плоть бо похотствуетъ на духа, духъ же на плоть: сія же другъ другу противятся, да не яже хощете, сія творите}\footnote{Гал.~5,~16 и 17.}.\end{quotation}

\paragraph*{§\:382.} Понеже въ хрістіанинѣ двоякое имѣется рожденіе, ветхое и новое, или плотское и духовное, то въ томъ же хрістіанинѣ двоякій и человѣкъ есть, \textit{ветхій и новый}: двоякій не по существу (ибо по существу единъ есть человѣкъ, едино бо есть тѣло и душа, изъ которыхъ существо человѣка состоитъ), но по внутреннему устроенію, наклоненію и дѣйствію. Ветхій человѣкъ иначе называется \textit{внѣшній}, новый же \textit{внутренній}, якоже глаголетъ Апостолъ: \textit{аще и внѣшній нашъ человѣкъ тлѣетъ, но внутренній обновляется по вся дни}\footnote{Кор.~4,~16.}. Паки называется ветхій \textit{плоть}, новый же \textit{духъ}, якоже тойже Апостолъ учитъ: \textit{плоть похотствуетъ на духа}\footnote{Гал.~5,~17.}.

\paragraph*{§\:383.} Сіи два человѣка, въ единомъ хрістіанинѣ обрѣтающіеся, какъ противны суть между собою, такъ непрестанную между собою имѣютъ брань, и другъ другу противятся, какъ учитъ Апостолъ: \textit{плоть похотствуетъ на духа, духъ же на плоть: сія же другъ другу противятся}. Чего хощетъ плоть, того не хощетъ духъ; и чего хощетъ духъ, того не хощетъ плоть. Что плоть хощетъ дѣлать, того не хощетъ духъ, и противится тому; а что духъ начинаетъ и дѣлаетъ, тому плоть противится, и хощетъ тое разорить. Отъ сего сраженія и брани послѣдуетъ, что едино побѣждаетъ, другое побѣждается, брань бо безъ того не бываетъ. Когда побѣждаетъ внутренній человѣкъ, тогда побѣждается внѣшній; и когда побѣждаетъ внѣшній, тогда побѣждается внутренній, отъ чего бываетъ единаго животъ, другаго смерть. Якоже въ естествѣ бываетъ сіе: напр. когда находитъ свѣтъ, тогда исчезаетъ тьма; и когда наступаетъ холодъ, тогда погибаетъ теплота, и когда отходитъ отъ человѣка животъ, тогда наступаетъ смерть; растлѣніе бо единаго рожденіемъ бываетъ другаго. Тоежъ дѣлается и на брани сей духовной: когда умираетъ внутренній человѣкъ, тогда живетъ и обладаетъ внѣшній человѣкъ; а когда умираетъ внѣшній, тогда живетъ и обновляется внутренній, якоже глаголетъ: \textit{аще и внѣшній нашъ человѣкъ тлѣетъ, но внутренній обновляется по вся дни}.

\paragraph*{§\:384.} Сіе сраженіе или \textit{брань} между плотію и духомъ бываетъ въ двухъ: 1)~\textit{Въ догматахъ и тайнахъ святыя вѣры}. Плоть немощная чего не чувствуетъ и разумомъ не постигаетъ, того не пріемлетъ. Что Богъ единъ естествомъ, а троиченъ во Ѵпостасехъ, того она не пріемлетъ. Что міръ изъ ничего созданъ, того не пріемлетъ, разсуждая, что ничего изъ ничего не бываетъ, но все отъ инаго чего раждается. Дѣвѣ безъ мужа родити, и дѣвою пребывать; Богу воплотитися, и человѣкомъ быти; человѣку умершему и въ прахъ разсыпавшемуся востать, и прочія тайны святыя за буйство плоть вмѣняетъ. Откуду Никодимъ, плотская еще мудрствующій и неразумѣющій таинъ Божіихъ, не вѣритъ, и глаголетъ Хрісту, о новомъ и духовномъ свыше рожденіи поучающему: \textit{како можетъ человѣкъ родитися старъ сый? еда можетъ второе внити во утробу матере своея и родитися? како могутъ сія быти}\footnote{Іоан.~3,~4 и 9.}? И Апостолъ Павелъ съ прочіими Апостолами глаголетъ: \textit{мы проповѣдуемъ Хріста распята, Іудеемъ убо соблазнъ, Еллиномъ же безуміе}\footnote{1~Кор.~1,~23.}. Понеже \textit{душевенъ человѣкъ}, то"=есть, плотскій, не пріемлетъ яже Духа Божія: \textit{юродство бо ему есть}\footnote{2,~14.}. Отсюду произошли душепагубныя мнѣнія о созданіи міра, о Бозѣ, о Хрістѣ Сынѣ Божіемъ, о воскресеніи мертвыхъ, и о прочіихъ хрістіанскія вѣры догматахъ въ хрістіанахъ, которые о тѣхъ разсуждали по плотскому разума смыслу, который, яко слѣпъ самъ въ себѣ, постигнуть ихъ безъ помощи вѣры не можетъ, и такъ заблуждаетъ, а не плѣняли его въ послушаніе вѣры. Но духъ и вѣра, въ истинномъ хрістіанинѣ живущая, къ безопаснѣйшему пристанищу откровенія Божія, которое въ словѣ Божіемъ содержится, его отводитъ и указуетъ на истину и всемогущество Божіе, и симъ духовнымъ мечемъ \textit{всяко возношеніе, взимающееся на разумъ Божій, низлагаетъ, и плѣняетъ всякъ разумъ въ послушаніе Хрістово}\footnote{2~Кор.~10,~4--5.}. Немощной плоти и слѣпой много помогаетъ сатана, врагъ хрістіанскій, который такожде боретъ противу духа, и заченшуюся вѣры искру угасить тщится, глаголя человѣку въ сердце его о тайнахъ святыхъ: «\textit{како могутъ сія быти?} Како Богъ, Единъ естествомъ, можетъ быть въ тріехъ лицахъ? Како міръ изъ ничего можетъ быть созданъ? Како Дѣва могла родить Бога безначальнаго? Како Богъ человѣкомъ невидимый видимымъ быть, и Безначальный начаться, и Безстрастный пострадать, и Безсмертный умереть, и мертвый воскреснуть? Како тѣло умершее и согнившее востати паки можетъ? Како грѣшникъ, обремененный грѣхами, предъ праведнымъ судомъ Божіимъ вѣрою единою оправдатися можетъ?» Сія и прочая разженныя стрѣлы вражія понуждается терпѣть вѣрная душа. Но вѣра истинная твердо противится имъ, и, яко \textit{побѣда, побѣждшая міръ, побѣждаетъ ихъ}\footnote{1~Іоан.~5,~4 и 5.}, утверждаяся неложнымъ и вѣрнымъ свидѣтельствомъ слова Божія. \textit{Свидѣтельство бо Господне вѣрно}\footnote{Пс.~18,~8.}. "--- 2)~Бываетъ брань между плотію и духомъ \textit{въ склонностяхъ}, которыя до нравовъ надлежатъ. Плоть хощетъ дѣлать тое, что ей свойственно и пріятно, какъ"=то: величаться, гордиться, превозноситься, въ славѣ, чести и почтеніи у людей быть, сладострастію своему угождать, ближнему за обиду отмщевать, богатство собирать, уповать на свою силу, хитрость, богатство, на князей и вельможъ вѣка сего, и проч. Но духъ отъ всего того отвращается, яко суетнаго, и научаетъ Бога бояться, любить, почитать, и на Него единаго надѣяться, и призывать и помощи искать. И въ сей брани сатана плоти нашей помогаетъ на духъ нашъ, который какъ прародителей нашихъ въ раи, такъ и нынѣ хрістіанъ отвратить тщится отъ любви Божіей, послушанія и почитанія.

\paragraph*{§\:385.} Весь \textit{хрістіанскій подвигъ} въ томъ состоитъ, чтобы въ началѣ пресѣкать злые помыслы, которые востаютъ противу вѣры святыя и Божія закона, и хотятъ благочестивое сердце превратить, и не попускать имъ возрастать и усиливаться. О семъ вси святіи Божіи угодники тщились и подвизались. О семъ и нынѣ хотящіи благочестно о Хрістѣ Іисусѣ жити подвизаются. О семъ и намъ, хрістіанине, тщиться и подвизаться подобаетъ. Въ семъ дѣлѣ такъ важномъ, въ которомъ о спасеніи души своей подвизается хрістіанинъ, должно намъ подражать тѣмъ людямъ, которые или домъ свой отъ находящаго разбойника, или градъ защищаютъ отъ непріятеля. Тогда они все тщаніе полагаютъ, чтобы отъ враговъ своихъ непобѣжденными быть; противятся и отражаютъ ихъ, уязвляются и уязвляютъ. Тако должно и намъ поступить, когда на домъ души нашея наступаетъ врагъ нашъ діаволъ, и хощетъ его разорити: должно противитися ему, отражать его и не допущать. Или, якоже хозяинъ загарающійся домъ свой тщится угасить, такъ должно намъ, когда въ души наши злые помыслы, яко стрѣлы разженныя, бросаетъ лукавый, должно угашати и не допускать дому нашему загорѣтися. Якоже бо разбойникъ, вшедши въ домъ, разоряетъ его и хозяина умерщвляетъ; или якоже непріятель, доставши градъ, опустошаетъ его, и гражданъ или плѣняетъ, или умерщвляетъ: тако діаволъ, хрістіанскій непріятель, когда чрезъ злые помыслы достанетъ домъ душевный, подобное душѣ дѣлаетъ разореніе. Тогда душа подобна бываетъ опустошенному граду, о чемъ ангели и святіи Божіи сѣтуютъ. Якоже бо о \textit{грѣшникѣ кающемся радуются}\footnote{Лук.~15,~10.}, тако о праведникѣ совратившемся жалѣютъ и болѣзнуютъ. Сего ради должно намъ, хрістіанине, о томъ весь трудъ полагать, чтобы непріятеля сего въ домъ нашъ не допущать, затворять предъ нимъ двери, противиться, крѣпиться и отражать его. Не корысти какой ищетъ онъ у насъ, но насъ самихъ хощетъ плѣнить и погубить; не богатство какое тлѣнное, но сокровище нетлѣнное, вѣчное душъ нашихъ спасеніе старается отъять у насъ. Того ради послушаемъ Апостольскаго увѣщанія: \textit{трезвитеся, бодрствуйте: зане супостатъ вашъ діаволъ, яко левъ рыкая, ходитъ, искій кого поглотити. Емуже противитеся тверди вѣрою}\footnote{1~Петр.~5,~8 и 9.}.

\paragraph*{§\:386.} Примѣчай, хрістіанине, слѣдующее: 1)~Брань сія и подвигъ всѣмъ хрістіанамъ и всякаго чина и званія людямъ, имя свое записавшимъ Хрісту, належитъ. Ибо апостоли святіи ко всѣмъ хрістіанамъ безъ разбора о ней пишутъ, и пощряютъ къ подвигу. Сего ради всякому, кто хощетъ спастися, надобно подвизаться. Слѣдственно худо нѣкоторые отъ похотолюбивыхъ думаютъ и баснословятъ, аки бы до единыхъ монаховъ и прочіихъ безбрачныхъ надлежало умерщвленіе плоти. Апостоли бо святіи о семъ писали ко всѣмъ, брачнымъ и безбрачнымъ; и умерщвленіе плоти разумѣется не токмо умерщвленіе похоти блудныя, что наипаче до безбрачныхъ надлежитъ, но и прочіихъ стремленій страстныхъ, какъ"=то: гнѣва, злобы, зависти, сребролюбія, гордости, мщенія, злорѣчія, и проч. Все бо сіе зло плоти нашея плодъ есть, якоже изъ 5"~й главы къ Галатомъ видно, гдѣ плотскія дѣла отъ Апостола исчисляются. "--- 2)~Брань сія \textit{непрестанно} настоитъ намъ, на всякое время и на всякомъ мѣстѣ. Ибо вездѣ и всегда, во градѣ и пустынѣ, во дни и нощи страсти наши съ нами суть, яко домашніи враги; плоть всегда похотствуетъ на духа; и сатана съ своими злыми слугами, яко левъ рыкая, вездѣ и всегда ходитъ, искій кого поглотити. 3)~Не оставляютъ насъ до самой кончины, но тогда наипаче борютъ насъ и подвизаются противу насъ враги наши, хотя отъ насъ восхитити вѣнецъ нашъ. Тогда слѣдуетъ всякому хрістіанину или побѣдить и вѣчно жить и торжествовать, или побѣжденнымъ быть и вѣчной срамотѣ подпасть и вѣчно умирать. Отъ того бо часа всякъ или блаженную, или горькую начинаетъ вѣчность. О! спаси насъ, Господи, въ часъ оный. Хотя и всегда, наипаче въ часъ оный всякая хитрость и разумъ человѣческій оскудѣваетъ. Вѣра истинная и живая, упованіе крѣпкое на Іисуса, смерти и ада побѣдителя, торжествуетъ тогда. О! предстани мнѣ тогда, смиренному и недостойному рабу Твоему, благій Іисусе, Искупителю мой, егоже туне возлюбилъ еси, и предалъ еси Себе по мнѣ; и нынѣ не остави мене. \textit{Боже мой! не удалися отъ мене, Боже мой! въ помощь мою вонми. Да постыдятся и исчезнутъ оклеветающіи душу мою; да облекутся въ студъ и срамъ ищущіи злая мнѣ}\footnote{Пс.~70,~12 и 13.}. Тогда наипаче, любезный хрістіанине, нужна есть усердная молитва, и изъ глубины сердца воздыханіе, и надежда паче надежды, и упованіе паче упованія. Чего ради нынѣ, во время живота нашего, часъ оный поминать и молить Іисуса, чтобы тогда не оставилъ насъ; но должно и самимъ намъ нынѣ Его не оставлять и усердно работать Ему. "--- 4)~Понеже такъ лютая и всегдашняя брань належитъ намъ, то не должно намъ дремать. Не должно, говорю, дремать, когда враги наши на погибель нашу всегда и вездѣ бодрствуютъ; но, оттрясши сонъ лѣности, всегда бдѣть и на стражи противу нападенія ихъ стоять, по увѣщанію Хрістову: \textit{бдите, и молитеся, да не внидете въ напасть}\footnote{Мѳ.~26,~41.}. И паки: \textit{буди вѣренъ даже до смерти, и дамъ ти вѣнецъ живота}\footnote{Апок.~2,~10.}. И паки\textit{: се гряду скоро: держи, еже имаши, да никтоже пріиметъ вѣнца твоего}\footnote{3,~11.}. Тоежъ и апостоли Его предлагаютъ намъ: \textit{мняйся стояти, да блюдется, да не падетъ}\footnote{1~Кор.~10,~12.}. И паки: \textit{трезвитеся, бодрствуйте, зане супостатъ вашъ діаволъ, яко левъ рыкая, ходитъ, искій кого поглотити: ему же противитеся тверди вѣрою}\footnote{1~Петр.~5,~8 и 9.}. Тщаніе человѣческое само собою едва что можетъ. Надобно бо человѣку противу себе самого то"=есть, своего природнаго злонравія, вооружиться и стоять. Злонравію своему удобно и охотно послѣдовать, но противу его стоять и побѣдить его сами не можемъ. Якоже бо внизъ по рѣкѣ судно само собою и удобно пловетъ, но противу быстринъ рѣчныхъ и вѣтра плыть само не можетъ, "--- надобно къ тому сильнаго гребца и правителя: тако естество наше, яко растлѣнное, грѣшить удобно можетъ, и грѣшитъ, но противу грѣха стоять и побѣдить его никакъ не можетъ, "--- надобно къ тому искать помощи и силы отъ побѣдителя смерти и ада, Іисуса Господа, Который \textit{крѣпость людемъ Своимъ дастъ}\footnote{Пс.~28,~11.}. Сего ради должно намъ къ подвигу нашему помощи просить у Самаго Бога, Который велитъ просить, и обѣщается подать: \textit{просите, и дастся вамъ}\footnote{Мѳ.~7,~7.}; и \textit{во время пріятно слушаетъ насъ, и въ день спасенія помогаетъ намъ. Се нынѣ время благопріятно, се нынѣ день спасенія}\footnote{2~Кор.~6,~2.}. Нынѣ, то"=есть, въ вѣцѣ семъ, доколѣ живемъ здѣ, доколѣ на пути имѣемся, благопріятно время есть подвизатися, просить, искать и толкать, да въ оно время, когда только истязаніе и судъ будетъ, не посрамимся. "--- 6)~Понеже брань сія всѣмъ хрістіанамъ належитъ, яко плоть и врагъ діаволъ всякаго хрістіанина бороть не оставляетъ: то, ежели кто отъ хрістіанъ не чувствуетъ ея, должно ему осмотрѣться, не находится ли въ плѣну и власти вражіей. Ибо сіе бываетъ и на брани видимой, что кто непріятелю своему, востающему на него, не противится, но покоряется, того онъ не безпокойствуетъ, но оставляетъ въ покоѣ и мирѣ, яко своего подвластнаго. Тако и діаволъ, врагъ хрістіанскій, того не боретъ, кто злой его волѣ покоряется, но оставляетъ его, яко своего плѣнника въ покоѣ; который покой бѣдственнѣйшій есть паче всякаго безпокойствія, яко къ вѣчному безпокойствію и погибели ведетъ плѣнника такого. Сего ради всякому хрістіанину должно осматриваться, въ какомъ состояніи имѣется. Не на роскоши и веселости мірскія позваны хрістіане, но на брань, труды и подвиги. Не малый трудъ и подвигъ требуется "--- себе самого и растлѣнное свое естество, и врага діавола побѣдить. Обѣщанъ хрістіанамъ покой, веселіе и всякая сладость, но въ будущемъ вѣкѣ, хотя"=то и нынѣ истинные раби Божіи не лишаются своего утѣшенія и радости (о чемъ въ послѣдней статьѣ увидишь). Разсуждай сіе, хрістіанине, и не смотри, что нынѣшній свѣтъ дѣлаетъ, но что слово Божіе предлагаетъ: \textit{трезвитеся, бодрствуйте}, и проч. и внимай себѣ!

\subsection[Глава 7-я. Яко всякъ хрістіанинъ долженъ вѣры и новаго рожденія плоды показывать.]{глава седьмая.\\\bfseries Яко всякъ хрістіанинъ долженъ вѣры и новаго рожденія плоды показывать.}

\begin{quotation}\textit{Спогребохомся Ему} (Хрісту) \textit{крещеніемъ въ смерть, да якоже воста Хрістосъ отъ мертвыхъ славою Отчею, тако и мы во обновленіи жизни ходити начнемъ}, и проч\footnote{Римл.~6,~4 и сл.}.\end{quotation}

\paragraph*{§\:387.} Чтобы тебѣ, хрістіанине, лучше познать, что есть новое рожденіе, вопервыхъ, предлагаю тебѣ разсужденіе о смерти троякой. Смерть есть \textit{троякая}: тѣлесная, духовная и вѣчная. \textit{Тѣлесная смерть} состоитъ въ разлученіи души отъ тѣла. Сія смерть есть общая всѣмъ праведнымъ и грѣшнымъ, и никому неминуема, какъ видимъ. О сей смерти глаголетъ Божіе слово: \textit{лежитъ человѣкомъ единою умрети}\footnote{Евр.~9,~27.}. Вторая смерть есть \textit{вѣчная}, которою грѣшники осужденныи вѣчно умирати будутъ, но никогда не могутъ умереть; пожелаютъ обратитися въ ничто ради лютаго и нестерпимаго мученія, но не могутъ. О сей смерти глаголетъ Хрістосъ: \textit{страшливымъ и невѣрнымъ, и сквернымъ, и убійцамъ, и блудъ и чары творящимъ, идоложерцемъ и всѣмъ лживымъ, часть имъ въ езерѣ горящемъ огнемъ и жупеломъ, еже есть смерть вторая}\footnote{Апок.~21,~8.}. Третія смерть есть \textit{духовная}, которою всѣ, невѣрующіи во Хріста, истиннаго Живота и Источника живота, мертвы суть. Такожде хрістіане, исповѣдающіи Бога и Хріста Сына Божія, но беззаконно живущіи, сею смертію мертвы суть. Что тѣлу человѣческому душа, тое душѣ Божія благодать. Какъ убо тѣло мертво есть, когда души въ немъ не имѣется: тако душа мертва есть, когда не имѣетъ Божіей благодати, оживляющей ее. Якоже бо мертвый человѣкъ, хотя и имѣетъ ноги, уши, глаза, руки, языкъ и прочіе члены, однакожъ ими не дѣйствуетъ, яко не имѣетъ души, которая есть начало всѣхъ дѣйствій естественныхъ: тако духовный мертвый, то"=есть, во грѣхахъ живущій, и похотемъ плотскимъ работающій недѣйствителенъ къ духовнымъ дѣламъ, къ слову Божію глухъ; имѣетъ уши, но не слышитъ; Бога не боится, не почитаетъ, не слушаетъ слова Его, не любитъ Его и ближняго своего; сердцемъ или злое, или суетное только замышляетъ, и весь недѣйствителенъ къ дѣлу Божію; лежитъ во грѣхахъ мертвый, какъ трупъ бездушный во гробѣ. Откуду бываетъ, что въ таковомъ мертвецѣ слово Божіе ничего не дѣйствуетъ и, якоже въ землѣ изсохшей, никакого плода не творитъ, хотя и сѣется небесное тое сѣмя. Надобно бо тому во тьмѣ быть, кто удаляется отъ свѣта; въ смерти быть, кто отлучается отъ живота; изсохшей быть розгѣ, которая отторгнется отъ лозы. \textit{Хрістосъ есть Свѣтъ, Животъ и Лоза истинная}\footnote{Іоан.~8,~12; 12,~46; 11,~25; 14,~6; 15,~1 и 5.}. Слѣдственно во тьмѣ и смерти суть, которые отъ него отлучаются; и изсохшія розги суть, которыи отъ живой той Лозы отторгнулися. \textit{Аще кто во мнѣ не пребудетъ, извержется вонъ, яко розга, и изсышетъ}, глаголетъ Хрістосъ\footnote{Іоан.~15,~6.}. Изсохшая розга мертва есть, яко сока, которымъ оживляется, не имѣетъ и потому плода никакого не творитъ: тако человѣкъ мертвъ есть, который отъ Хріста, истиннаго Живота, отлучается, и потому плода богоугоднаго творити не можетъ, якоже паки глаголетъ Хрістосъ: \textit{безъ Мене не можете творити ничесоже}\footnote{ст.~5.}. Отлучается же отъ Хріста всякъ законопреступникъ и похотямъ міра сего прилѣпляющійся. \textit{Кое бо причастіе правдѣ къ беззаконію? или кое общеніе свѣту ко тмѣ}\footnote{2~Кор.~6,~14.}? Якоже бо вѣрою прицѣпляется человѣкъ Хрісту "--- истинной Лозѣ, и отъ Него сокъ благодати, живность духовную къ плодотворенію пріемлетъ, и тако живетъ духовно: такъ, когда вѣру живую погубляетъ, отсѣкается отъ благословенной Лозы той, и, якоже розга изсохшая, бываетъ мертвъ, и плода творити не можетъ. Такой мертвецъ тѣломъ живетъ, но душею умеръ, якоже о вдовицѣ, \textit{пространно питающейся}, глаголетъ Апостолъ: \textit{жива умерла}\footnote{1~Тим.~5,~6.}. Къ такому мертвецу Божія благодать, какъ чрезъ внѣшнее Божіе слово, такъ и внутрь, въ совѣсти вопія, ударяетъ его: \textit{востани спяй, и воскресни отъ мертвыхъ, и освѣтитъ тя Хрістосъ}\footnote{Еф.~5,~14.}. Сія смерть тѣмъ есть бѣдственна, что къ бѣдственнѣйшей вѣчной смерти ведетъ, которою пожертый человѣкъ во вѣки не увидитъ живота. \textit{Яже не вѣруетъ въ Сына, не узритъ живота, но гнѣвъ Божій пребываетъ на немъ}\footnote{Іоан.~3,~36.}. "--- Но какъ смерть, такъ и животъ троякій. Первый тѣлесный, когда душа съ тѣломъ совокупно пребываетъ. Сей животъ такожде, якоже и смерть тѣлесная, всѣмъ праведнымъ и грѣшнымъ общій есть. Вторый животъ \textit{духовный}, который содѣловаетъ Божія благодать, живущая въ человѣкѣ. Какъ грѣхъ оскверняетъ человѣка и отъ Святаго Бога отлучаетъ\footnote{Ис.~59,~2.}, и тако умерщвляетъ: тако вѣра и Божія благодать очищаетъ человѣка, съ Богомъ соединяетъ и оживляетъ. Якоже бо чувственныя вещи, къ свѣту, напр. солнцу или свѣтильнику приближившіяся, свѣтъ въ себе пріемлютъ и освѣщаются: тако душа, вѣрою къ Богу приближившаяся, свѣтъ отъ Него и животъ пріемлетъ, и тако просвѣщается и оживляется отъ присносущнаго Свѣта и Живота. Третія жизнь есть вѣчная, къ которой преддверіе есть вышеписанная духовная жизнь. Никто бо не внидетъ въ вѣчную жизнь, аще отъ мертвыхъ дѣлъ не воскреснетъ нынѣ. Ибо \textit{внѣ псы и чародѣи, и любодѣи и убійцы, и идолослужители, и всякъ любяй и творяй лжу}\footnote{Ап.~22,~15.}. Надобно убо неотмѣнно человѣку сперва здѣ востать, вѣрою Хрістовою оживиться, и тако отворится ему дверь къ вѣчному животу. \textit{Блаженъ и святъ, иже имать часть въ воскресеніи первѣмъ}\footnote{20,~6.}. \textit{Имѣяй Сына Божія, имать животъ; а не имѣяй Сына Божія, живота не имать}\footnote{1~Іоан.~5,~12.}.

\paragraph*{§\:388.} Какъ въ плотскомъ рожденіи \textit{духовнѣ мертвыми} раждаемся вси, яко \textit{въ беззаконіихъ зачинаемся и во грѣсѣхъ раждаемся}\footnote{Пс.~50,~7.}: тако въ духовномъ рожденіи или крещеніи, которое бываетъ водою и Духомъ, отъ мертвыхъ духовнѣ востаемъ. Крещеніемъ бо сообразуемся смерти и воскресенію Хрістову. Ибо якоже Хрістосъ умеръ за грѣхи наши, и восталъ плотію, тако мы въ крещеніи умираемъ грѣху, которымъ духовнѣ умерщвлены были, и востаемъ духовнѣ и начинаемъ жити Богу. Умеръ Хрістосъ: умертвился и грѣхъ нашъ. Погреблся Хрістосъ: погреблся и грѣхъ нашъ. Восталъ Хрістосъ отъ мертвыхъ: умертвилася и смерть наша, якоже поетъ Церковь въ день святыя пасхи: «Смерти празднуемъ умерщвленіе»\footnote{Пѣснь 7"~я.}; и подалося намъ востаніе \textit{нынѣ духовное} отъ смерти грѣховной, которое вѣрою въ воскресшаго Хріста и крещеніемъ бываетъ, \textit{въ послѣдній день тѣлесное и всесовершенное} всего состава нашего обновленіе. Тогда бо будетъ слово написанное: \textit{пожерта бысть смерть побѣдою. Гдѣ ти, смерте, жало? Гдѣ ти, аде, побѣда}\footnote{Кор.~15,~54--55; Осіи 13,~14.}? Когда убо вѣруемъ во Хріста умершаго и воскресшаго, и крещаемся, "--- умерщвляется и грѣхъ нашъ. Когда умерщвляется грѣхъ, то умерщвляется и смерть наша, смерть бо отъ грѣха бываетъ. Слѣдственно, когда грѣхъ умерщвленъ, то умерщвлена и смерть; а когда умерщвляется смерть, то мы востаемъ отъ мертвыхъ; а когда востаемъ отъ мертвыхъ, то вновь жити начинаемъ. Крещеніе убо наше есть духовное наше востаніе и воскресеніе. Крещеніе бо наше умертвило насъ грѣху и оживило Богу, и тако обновило, отродило и \textit{новою тварію} насъ содѣлало, по ученію Апостола: \textit{аще кто во Хрістѣ, нова тварь}\footnote{2~Кор.~5,~17.}. Откуду крещеніе святое отъ Апостола называется \textit{банею пакибытія: спасе насъ Богъ банею пакибытія}\footnote{Тит.~3,~5.}. Называется \textit{банею}, яко сею спасительною банею омывается человѣкъ крещаемый отъ сквернъ грѣховныхъ; \textit{пакибытія}, яко вновь начинаетъ жити умерщвленный грѣхомъ; оживляется благодатію Святаго Духа, и, грѣху умирая, начинаетъ Богу жити. Сего ради крещеніе святое есть новое, духовное, святое, второе рожденіе или пакирожденіе, яко въ немъ вновь раждаемся, духовно раждаемся, обновляемся \textit{обновленіемъ Духа Святаго}\footnote{3,~5.}; второе раждаемся или перераждаемся къ новому, лучшему, благочестивому, святому и небесному житію. Откуду при крещеніи бываетъ отрицаніе сатаны и всѣхъ дѣлъ его злыхъ, и обѣщаемся работать Богу. Сего ради внимать должно намъ, хрістіанине, сему новому рожденію, и ветхаго отрекшися, сему послѣдовати. 1)~Апостолъ глаголетъ: \textit{иже умрохомъ грѣху, како еще жити будемъ въ немъ}\footnote{Римл.~6,~2.}? Умрети грѣху есть не слушати его и не дѣлати, что онъ хощетъ. «Что есть мертвымъ грѣху быти? Тое, чтобы ни въ чемъ не послушати его прочее», глаголетъ святый Златоустъ на сіе слово Апостольское\footnote{Бес.~10"~я на посл. къ Римл.}. Якоже бо мертвый не слушаетъ и не дѣлаетъ ничего, что ему ни говоришь и повелѣваешь: тако и хрістіанамъ, яко умершимъ грѣху, не должно его ни въ чемъ слушать, къ чему онъ ни прельщаетъ. И сіе"=то есть \textit{плоти угодія не творити въ похоти}\footnote{Римл.~13,~14.}; \textit{похоти плотскія не совершать, плоть распинать со страстьми и похотьми}\footnote{Гал.~5,~16 и 24.}, и проч. А отсюду видимъ, что и жить грѣху не что иное есть, какъ дѣлать тое, чего онъ хощетъ, и повелѣніе его слушать и исполнять. Напр. грѣхъ склоняетъ и убѣждаетъ человѣка или къ блуду, или къ гнѣву и злобѣ, или къ хищенію, или другому какому злодѣянію: когда человѣкъ дѣлаетъ по тому, то и слушаетъ его, живетъ ему; и какъ рабъ повинуется и работаетъ ему, по ученію Хрістову: \textit{всякъ, творяй грѣхъ, рабъ есть грѣха}\footnote{Іоан.~8,~34.}. Отъ сего слѣдуетъ, что и жить Богу тотъ не можетъ, кто живетъ грѣху. Ибо жить Богу, или грѣху, есть слушать Бога, или грѣха, какъ сказано. А понеже Богъ и грѣхъ противное повелѣваютъ, то и не слушаетъ тотъ Бога, кто слушаетъ грѣха; слѣдственно и не живетъ тотъ Богу, кто живетъ грѣху. А тако всякъ грѣшникъ, прихоти грѣховныя исполняющій, есть мертвъ предъ очесами Божіими, яко не слушаетъ слова Божія, какъ мертвецъ тѣломъ умершій не слушаетъ гласа человѣческаго. Аще бо и живетъ тѣломъ, но духомъ мертвъ есть. Не напрасно бо глаголетъ: \textit{востани спяй, и воскресни отъ мертвыхъ}\footnote{Еф.~5,~14.}. Ибо не къ мертвецу, тѣломъ умершему, бездушному трупу, но къ грѣшнику, во гробѣ грѣховномъ лежащему, глаголется: \textit{воскресни отъ мертвыхъ}. О, колико живыхъ мертвецовъ предъ Богомъ имѣется, сіе апостольское слово учитъ насъ. Вси бо сіи живіи мертвецы суть, которыи истиннаго покаянія не творятъ, но по своимъ прихотямъ живутъ. Сего ради хрістіанамъ, которые въ крещеніи умерли грѣху, и Хрісту погребенному и воскресшему спогреблися и совостали, помнить должно сіе, и не грѣху уже, но Богу жити: да не, живучи грѣху, Богу и предъ Богомъ мертвы будутъ, и во вѣки въ смерти пребудутъ; но паче, умерши грѣху и Богу живучи, и здѣ предъ Нимъ живи будутъ, и во вѣки съ нимъ жити сподобятся. «Умертвило насъ крещеніе грѣху, глаголетъ Златоустъ: достоитъ прочее отъ нашего тщанія исправлятися сему выну, такъ, что аще и безчисленная повелитъ, уже не послушати его, но пребывати недвижиму, аки мертвецу»\footnote{Бес.~10"~я на посл. къ Римл.}. Хрістіанинъ убо по силѣ крещенія долженъ быть и мертвъ и живъ: \textit{мертвъ} грѣху и діаволу, грѣха изобрѣтателю, "--- \textit{живъ} же Богу. Сего ради хрістіанамъ спящимъ и во гробѣ грѣховномъ лежащимъ время есть востати и воскреснути отъ мертвыхъ, да и въ послѣдній день воскреснутъ въ вѣчный животъ. \textit{Востани спяй, и воскресни отъ мертвыхъ, и освѣтитъ тя Хрістосъ}. "--- 2)~Духовное и новое рожденіе, въ которомъ хрістіанинъ родился водою и Духомъ, требуетъ отъ него духовныхъ плодовъ. Какъ бо плотское, такъ и духовное рожденіе имѣетъ плоды, себѣ свойственные. Человѣкъ естественный или внѣшній, когда живетъ, не празденъ пребываетъ, но живности своей знаки оказываетъ, напр. чувствуетъ, видитъ, слышитъ, говоритъ, ходитъ, дѣлаетъ, ястъ, піетъ, и прочее: тако и новому человѣку внутреннему, когда живетъ въ насъ, должно оказывать знаки живности своея, которые новому духовному рожденію и новой духовной жизни приличествуютъ. Знаки и дѣйствія внутренняго, \textit{духовнѣ живущаго} человѣка, или новаго суть: истинное благочестіе и страхъ Божій, любовь къ Богу и ближнему, и оттуду богоугодныя дѣла, которыя безъ всякаго лицемѣрія творятся. Якоже бо знаки мертваго, ветхаго, или духовнѣ умершаго человѣка суть: нераскаянное житіе и мертвыя дѣла, грѣхи и беззаконія, дѣла, слова и помышленія, слову Божію противныя; яко таковый человѣкъ не Богу, но грѣху живетъ, какъ выше сказано: тако знаменія и дѣйствія новаго, отрожденнаго и духовнѣ живущаго человѣка, который Богу живетъ, суть истинныя хрістіанскія добродѣтели, которыми внутренній и новый человѣкъ, какъ доброе древо добрыми плодами, оказывается. И какъ знаки живаго человѣка суть дѣйствія его естественныя, слова, дѣла и прочее; не можетъ бо живый человѣкъ безъ дѣйствій быти: тако знаки духовно живущаго человѣка суть богоугодныя дѣйствія его, дѣла, слова и помышленія. Иначе не былъ бы живый, но мертвый, и было бы рожденіе не истинное, но мечтательное, безъ плодовъ себѣ приличныхъ. Откуду читаемъ и видимъ какъ въ книгѣ Дѣяній Апостольскихъ, такъ и въ церковной Исторіи, что истинно обратившіися къ Богу и вѣрою и Духомъ вновь отродившіися человѣки все иное, какъ прежде обращенія, на себѣ показывали: иные у нихъ замыслы, начинанія, дѣла и тщанія являлися, какъ доколѣ неотрожденными были. Какое усердіе, желаніе и алчба къ слышанію Божія слова, къ славословію и хвалѣ Божія имени! Какій былъ страхъ Божій, любовь къ Богу и ближнему! Какое братолюбіе, смиреніе, кротость, терпѣніе, милосердіе къ страждущимъ, щедроподаянія убогимъ, милость къ бѣдствующимъ и требующимъ милости! Какое попеченіе о спасеніи другъ друга! Какъ другъ друга поощряли къ подвигу въ благочестіи и проч.! И тѣмъ показывали, что они были истинные хрістіане, истинныя чада свѣта, истинные сынове небеснаго Отца, отъ Котораго родилися водою и духомъ, яко Ему добрыми нравами сообразовалися, "--- истинныя овцы Хрістовы, яко слушали гласа Пастыря своего. Дивимся мы, хрістіанине, такъ Богоугодному древнихъ хрістіанъ житію, дивимся; но еще должно болѣзновать и плакать, что нынѣ того невидно, но вмѣсто того \textit{за умноженіе беззаконія изсякла любы многихъ}\footnote{Мѳ.~24,~12.}. Сего ради должно намъ, хрістіанине, осмотрѣться, пребываемъ ли мы въ новомъ рожденіи, которое изъ плодовъ своихъ познается; имѣемъ ли духовный животъ, который изъ дѣйствій своихъ духовныхъ примѣчается; живетъ ли въ насъ новый, внутренній и духовный человѣкъ, который чрезъ страхъ Божій, любовь и прочія хрістіанскія добродѣтели себе оказываетъ. Всякая бо живая вещь живность свою чрезъ дѣйствія свои оказываетъ. Неотмѣнно убо и живому новому и внутреннему человѣку, когда онъ живетъ въ насъ, должно себе чрезъ дѣйствія и плоды свои оказывать. "--- 3)~Въ духовномъ семъ и новомъ рожденіи хрістіане раждаются отъ Бога, якоже Апостолъ глаголетъ о нихъ: \textit{иже не отъ крове, ни отъ похоти плотскія, ни отъ похоти мужескія, но отъ Бога родишася}\footnote{Іоан.~1,~13.}. Слѣдственно и суть чада Божія, и потому нравами и свойствами Отцу своему небесному подобитися должны. Якоже бо плотскій сынъ отца своего свойства въ себѣ изображаетъ, тако и сынамъ Божіимъ Отца своего небеснаго свойства и нравы въ себѣ изображать должно, якоже глаголетъ Хрістосъ: \textit{рожденное отъ плоти, плоть есть; и рожденное отъ духа, духъ есть}\footnote{3,~6.}. Откуду и Апостолъ учитъ и увѣщаваетъ хрістіанъ: \textit{бывайте подражатели Богу, якоже чада возлюбленная}\footnote{Еф.~5,~1.}. Въ чемъ подражателями быть? Не міръ созидать и прочія вышеестественныя дѣла творить, сіе бо единыя всемогущія силы Его дѣло есть. Но въ чемъ? \textit{Будите милосерди, якоже и Отецъ вашъ милосердъ есть}, глаголетъ Хрістосъ\footnote{Лук.~6,~36.}; и паки: \textit{любите враги ваша, благословите кленущія вы, добро творите ненавидящимъ васъ, и молитеся за творящихъ вамъ напасть и изгонящія вы: яко да будете сынове Отца вашего, Иже есть на небесѣхъ; яко солнце Свое сіяетъ на злыя и благія, и дождитъ на праведныя и на неправедныя}\footnote{Мѳ.~5,~44 и 45.}. И Самъ Богъ глаголетъ хрістіанамъ, чадамъ Своимъ: \textit{святи будите, яко Азъ святъ есмь}\footnote{1~Петр.~1,~16; Лев.~11,~44; 19,~2.}. И Апостолъ святый написалъ: \textit{бывайте другъ ко другу блази, милосерди, прощающе другъ другу, якоже и Богъ во Хрістѣ простилъ есть вамъ}\footnote{Еф.~4,~32.}, "--- и въ прочіихъ Божественныхъ Его свойствахъ. Ибо въ хрістіанахъ, яко чадахъ Божіихъ, долженъ быть наченшійся образъ Божій, которымъ должны подобитися Отцу своему небесному. "--- \textit{(О семъ смотри еще главу статьи третія книги сея)}.

\paragraph*{§\:389.} Отъ вышеписанныхъ видишь, хрістіанине, какихъ плодовъ отъ хрістіанина требуетъ новое его рожденіе или святое крещеніе. Сего ради всякому хрістіанину необходимо нужно есть осматриваться, пребываетъ ли въ новомъ духовномъ рожденіи (рожденіе же сіе изъ плодовъ своихъ познается, какъ выше сказано); не въ ветхомъ ли паче и плотскомъ только находится, котораго плоды суть страсти и похоти плотскія и грѣхи; не \textit{царствуетъ} ли надъ нимъ грѣхъ, \textit{во еже послушати его въ похотехъ его}\footnote{Римл.~6,~12.}; не \textit{творитъ ли плоти угодія въ похоти}\footnote{13,~14.}, и тако не имѣется ли \textit{духовно мертвъ} Богу. Не имя бо хрістіанское и исповѣданіе вѣры дѣлаетъ хрістіанина истиннымъ хрістіаниномъ, но вѣра живая. Вѣру же живую показуютъ плоды ея "--- добрыя дѣла, якоже Апостолъ глаголетъ: \textit{покажи ми вѣру твою отъ дѣлъ твоихъ}\footnote{Іак.~2,~18.}. И \textit{всяко древо отъ плода своего познается}\footnote{Лук.~6,~44.}. \textit{Не можетъ древо добро плоды злы творити, ни древо зло плоды добры творити}\footnote{Мѳ.~7,~18.}. Тако и хрістіанинъ, когда вѣру живую имѣетъ, и тако въ новомъ рожденіи пребываетъ, не можетъ злыхъ плодовъ творити, яко древо доброе. \textit{Всякъ} бо \textit{рожденный отъ Бога, грѣха не творитъ, яко сѣмя его въ немъ пребываетъ; и не можетъ согрѣшати, яко отъ Бога рожденъ есть}\footnote{1~Іоан.~3,~9.} (что о грѣхѣ, отъ произволенія и противу совѣсти творимомъ, разумѣется). Но яко древо доброе плоды добры, тако онъ добрыя дѣла живущею въ немъ благодатію Божіею творитъ. Сего ради нужно есть всякому искушать себе, имѣетъ ли вѣру, и находится ли въ спасительномъ новомъ рожденіи; нужно, глаголю, пока время не ушло и на пути міра сего имѣется, гдѣ спасеніе вѣчное или пріобрѣтается, или погубляется. Ибо нетворящимъ плода добра судъ Божій, какъ сѣкира, посѣкающій возвѣщается. \textit{Уже и сѣкира при корени древа лежитъ: всяко убо древо, еже не творитъ плода добра, посѣкаемо бываетъ, и во огнь вметаемо}\footnote{Мѳ.~3,~10.}.

\subsection[Глава 8-я. Хрістіанинъ, вѣрою во Хріста оправданный, долженъ правды плоды показывать, якоже древо доброе плоды добры творитъ.]{глава осьмая.\\\bfseries Хрістіанинъ, вѣрою во Хріста оправданный, долженъ правды плоды показывать, якоже древо доброе плоды добры творитъ.}

\begin{quotation}\textit{Иже} (Хрістосъ) \textit{грѣхи наша Самъ вознесе на тѣлѣ Своемъ на древо, да, отъ грѣхъ избывше, правдою поживемъ}\footnote{1~Петр.~2,~24.}.\end{quotation}
\begin{quotation}\textit{Аще вѣсте, яко Праведникъ есть} (Богъ), \textit{разумѣйте, яко всякъ, творяй правду, отъ Него родися}\footnote{1~Іоан.~2,~29.}.\end{quotation}
\begin{quotation}\textit{Всякъ не творяй правды, нѣсть отъ Бога, и не любяй брата своего}\footnote{3,~10.}.\end{quotation}

\paragraph*{§\:390.} Хрістіанинъ въ святомъ крещеніи вѣрою во Хріста оправдался и чистымъ предъ Богомъ учинился, такъ, какъ бы онъ никакого грѣха не сотворилъ. Откуду служитель Божій сею спасительною Божіею благодатію привѣтствуетъ крестившагося: \textit{оправдался еси, просвѣтился еси, освятился еси, омылся еси именемъ Господа нашего Іисуса Хріста, и Духомъ Бога нашего}. Тоежде и Апостолъ Хрістовъ глаголетъ хрістіанамъ: \textit{омыстеся, освятистеся, оправдистеся, именемъ Господа нашего Іисуса Хріста, и Духомъ Бога нашего}\footnote{1~Кор.~6,~11.}. Дивное, любезный хрістіанине, и умомъ нашимъ непостижимое перемѣненіе! Блаженная и всерадостная намъ грѣшнымъ перемѣна! Изъ мерзкаго грѣшника праведнымъ, и изъ нечистаго святымъ, и изъ сына погибели наслѣдникомъ царствія дѣлается человѣкъ! \textit{Заслужилъ} намъ сію непостижимую благодать Сынъ Божій Іисусъ Хрістосъ; \textit{подаетъ} милосердый Отецъ; \textit{совершаетъ} Духъ Святый. \textit{Благословенъ Господь Богъ Израилевъ, яко посѣти и сотвори избавленіе людемъ Своимъ}\footnote{Лук.~1,~68.}. Но оправдившимся и освятившимся туне должно плоды оправданія показывать, должно правду тую, вѣрою воспріятую, дѣлами праведными свидѣтельствовать, да покажутъ, что они истинно отъ Бога, Иже есть вѣчная правда, родилися, по реченному: \textit{аще вѣсте, яко праведникъ есть} (Богъ), \textit{разумѣйте, яко всякъ, творяй правду, отъ Него родися}\footnote{1~Іоан.~2,~29.}. То"=есть изъ творенія правды познается рожденный отъ Бога человѣкъ.

\paragraph*{§\:391.} Да разумѣеши, хрістіанине, что есть \textit{творить правду}, внимай слѣдующему разсужденію. \textit{Правда} есть такая добродѣтель, которая учитъ насъ всѣмъ должное отдавать, и потому всеобщая есть добродѣтель, всѣ добродѣтели въ себѣ заключающая. Она учитъ, что мы Богу и ближнему нашему отдавать должны, и какая ни есть истинная добродѣтель, въ правдѣ заключается. Она во всѣхъ десяти заповѣдяхъ Божіихъ изображена, какъ и всякая истинная добродѣтель; и потому заповѣди Божія исполнять и правду творить едино есть. А якоже правда всякую добродѣтель, тако неправда всякій грѣхъ въ себѣ заключаетъ; и когда человѣкъ какую заповѣдь Божію разоряетъ, тогда онъ дѣлаетъ неправду, и противу правды грѣшитъ, и тако всякій грѣхъ есть противу правды. Правда убо требуетъ отъ насъ, чтобы мы Бога \textit{единаго}, отъ Котораго созданы, знали, и кромѣ Его никакихъ боговъ не признавали, Его единаго почитали, любили, боялися, слушали, призывали, на Него единаго надѣялися, уповали, и имя Его святое съ почитаніемъ и благоговѣинствомъ поминали. Таяжде правда хощетъ отъ насъ, чтобы мы ближнему нашему, то"=есть, всякому человѣку, того не дѣлали, чего не хощемъ себѣ, и тое дѣлали, что хощемъ себѣ. Сіе Хрістосъ въ Евангеліи Своемъ изобразилъ тако: \textit{вся, елика аще хощете, да творятъ вамъ человѣцы, тако и вы творите имъ: се бо есть законъ и пророцы}\footnote{Матѳ.~7,~12.}. А понеже никто не хощетъ себѣ зла, то по силѣ правды и самъ не долженъ творить или хотѣть ближнему зла, и всякъ хощетъ себѣ добра, то и ближнему долженъ хотѣть и творить добро. Отсюду послѣдуетъ, что какъ тотъ грѣшитъ противу правды, кто ближнему зло творитъ, такъ и тотъ, кто добра не творитъ, когда можетъ. Не все бо человѣкъ, что хощетъ, можетъ. Отъ сего заключается правильно, что и врагамъ нашимъ не токмо зла не дѣлать, ни желать, но и добро желать и дѣлать должны мы. Ибо какъ не хощемъ мы, чтобы они намъ зло творили, такъ хощемъ, чтобы добро творили; потому и сами того, чего не хощемъ отъ нихъ, имъ не дѣлать, и что хощемъ отъ нихъ, имъ дѣлать должны. Отъ сего слѣдуетъ, что грѣшимъ, когда врагамъ нашимъ обиду за обиду и зло за зло воздаемъ и отмщеваемъ. Какъ бо они худо дѣлаютъ, что насъ обижаютъ, такъ и мы такожде худо дѣлаемъ, что обидою обиду награждаемъ. Ибо зло равно есть зло, прежде ли оно, или послѣ дѣлается. И потому, кто зло за зло воздаетъ дѣломъ или словомъ, хотя и не прежде, но послѣ, однакожъ дѣлаетъ зло. Сего ради Апостолъ глаголетъ: \textit{не побѣжденъ бывай отъ зла, но побѣждай благимъ злое}\footnote{Римл.~12,~21.}. И Хрістосъ повелѣваетъ хрістіанамъ: \textit{любите враги ваша, благословите кленущія вы, добро творите ненавидящимъ васъ, и молитеся за творящихъ вамъ напасть, и изгонящія вы}\footnote{Матѳ.~5,~44.}. Неспорно, что человѣку неотрожденному правду сію, какъ и всякую, творити невозможно, яко отъ природы всякъ золъ, и ко злу помышленіе его клонится; но благодатію Духа Святаго обновившемуся, и отъ Бога, \textit{Иже солнце Свое сіяетъ на злыя и благія, и дождитъ на праведныя и на неправедныя}\footnote{45.}, рожденному, должно растлѣніе, злость и немощь естества своего силою Духа Святаго побѣждать, и не тое, что страстная плоть, но что Хрістосъ хощетъ и долгъ правды требуетъ, дѣлать: да покажетъ, что онъ подлинно оправданъ, подлинно рожденъ отъ Бога, Который всѣмъ, добрымъ и злымъ, другамъ и врагамъ Своимъ, благотворити не престаетъ, подлинно облекся во Хріста, какъ учитъ Апостолъ: \textit{елицы во Хріста крестистеся, во Хріста облекостеся}\footnote{Гал.~3,~27.}, Который насъ, враговъ своихъ, возлюбилъ, и предалъ Себе за насъ, якоже писано есть: \textit{единъ за всѣхъ умре. Хрістосъ за всѣхъ умре, да живущіи не ктому себѣ живутъ, но умершему за нихъ и воскресшему}\footnote{2~Кор.~5,~15.}. Любящихъ бо себе любятъ и язычники, мытари и грѣшники, вѣрою непросвѣщенніи и неотрожденніии, якоже глаголетъ Хрістосъ: \textit{аще любите любящія вы, кая вамъ благодать есть? ибо и грѣшницы любящія ихъ любятъ. И аще благотворите благотворящимъ вамъ, кая вамъ благодать есть? ибо и грѣшницы тожде творятъ}\footnote{Лук.~6,~32 и 33.}. Хрістіанамъ же, яко свыше рожденнымъ, выше язычниковъ восходить, и плоды вышеестественнаго рожденія показывать должно, да тако засвидѣтельствуютъ о себѣ, что они истинно вещію, а не именемъ только, хрістіане суть, истинно новая о Хрістѣ тварь, истинныя Отца небеснаго чада, истинно одеждою оправданія Хрістова облечены пребываютъ, и хранятъ тую. Истинно и тое, что первѣйшая любовь должна быть другамъ и братіи о Хрістѣ. Ибо и Богъ болѣе любитъ любящихъ Себе, якоже глаголетъ Хрістосъ: \textit{любяй Мя, возлюбленъ будетъ Отцемъ Моимъ, и Азъ возлюблю его}\footnote{Іоан.~14,~21.}. Однакожъ не отнимаетъ и отъ враговъ Своихъ любви Своея, когда солнце Свое на нихъ сіяетъ, дождитъ на нихъ, терпитъ грѣхи ихъ, ожидаетъ обращенія ихъ, посылаетъ посредствія имъ къ обращенію, и проч. Тако подобаетъ и хрістіанамъ подражать Отцу своему небесному, и не отнимать любви своея отъ враговъ своихъ. Такихъ плодовъ правды желалъ Апостолъ въ хрістіанѣхъ: \textit{о семъ молюся, да любовь ваша еще паче и паче избыточествуетъ въ разумѣ и во всякомъ чувствіи, во еже искушати вамъ лучшая, да будете чисти и непреткновенни въ день Хрістовъ, исполнени плодовъ правды Іисусъ Хрістомъ, въ славу и похвалу Божію}\footnote{Филип.~1,~9--11.}. Видишь убо, хрістіанине, что есть творити правду. А именно: есть единаго истиннаго Бога знать, Его паче всего бояться, любить, почитать, слушать, на Него всю надежду полагать, со смиреніемъ и терпѣніемъ Ему покорять сердце свое, и отъ Него единаго всего добра просить и ожидать, и другъ къ другу взаимную и нелицемѣрную любовь имѣть и показывать. Къ сему насъ слово Божіе и совѣсть наша руководствуетъ; къ сему мы отрождены, обновлены и оправданы въ святомъ крещеніи, да творимъ плоды правды. \textit{Явися бо благодать Божія спасительная всѣмъ человѣкомъ, наказующи насъ, да, отвергшеся нечестія и мірскихъ похотей, цѣломудренно и праведно и благочестно поживемъ въ нынѣшнемъ вѣкѣ, ждуще блаженнаго упованія и явленія славы великаго Бога и Спаса нашего Іисуса Хріста, Иже далъ есть Себе за ны, да избавитъ ны отъ всякаго беззаконія, и очиститъ Себѣ люди избранны, ревнители добрымъ дѣломъ}\footnote{Тит.~2,~11--14.}. Внимай убо сему, возлюбленный хрістіанине, и примѣчай, какія суть примѣты истинныхъ хрістіанъ, которыи вѣру оправдающую въ сердцахъ своихъ, яко неоцѣненное сокровище, носятъ. Аще бо вѣра святая и есть даръ духовный, никакимъ чувствамъ неподлежащій, однакожъ оказываетъ себе внѣ плодами своими, якоже доброе древо, и якоже искра находящаяся въ пеплѣ теплотою своею, или якоже добрая и благовонная масть, находящаяся въ сосудѣ, благовоніемъ своимъ оказываетъ. Ибо гдѣ вѣра святая, истинная и живая, тамо и благодать Святаго Духа обитаетъ; гдѣ благодать Духа Святаго, тамо \textit{плодъ духовный: любы, радость, миръ, долготерпѣніе, благость, милосердіе, вѣра, кротость, воздержаніе}\footnote{Гал.~5,~22.}. Давидъ святый, когда вопросилъ Господа: \textit{Господи! кто обитаетъ въ жилищи Твоемъ? или кто вселится во святую гору Твою?} отвѣтъ чрезъ Духа Святаго таковый въ сердцѣ своемъ получилъ: \textit{ходяй непороченъ, и дѣлаяй правду, глаголяй истину въ сердцѣ своемъ}, и прочая\footnote{Пс.~14,~1,~2 и слѣд.}, "--- что и намъ въ наставленіе наше объявилъ и написалъ. Жилище Божіе есть \textit{церковь} святая, кровію Хрістовою омовенная и освященная\footnote{1~Тим.~3,~15.}; гора святая есть \textit{небо}; горній Сіонъ "--- вѣчное и блаженное избранныхъ Божіихъ жилище, якоже глаголетъ Апостолъ: \textit{о семъ воздыхаемъ, въ жилище наше небесное облещися желающе}\footnote{2~Кор.~5,~2.}. Совершенно безъ порока не сыскался никто отъ человѣкъ, яко, по свидѣтельству Писанія, \textit{вси согрѣшиша, и лишени суть славы Божія}\footnote{Римл.~3,~23.}. Единъ Хрістосъ \textit{грѣха не сотвори, ни обрѣтеся лесть во устѣхъ Его}\footnote{1~Петр.~2,~22.}. Кто истинно и сердечно вѣруетъ во Хріста, и, вѣрою во Хріста оправдавшися, творитъ плоды оправданія безъ всякаго лицемѣрія, тотъ ходитъ непороченъ и обитаетъ въ жилищи Божіи, то"=есть, въ церкви святой; и когда \textit{до смерти вѣренъ пребудетъ}\footnote{Апок.~2,~10.}, вселится во святую гору Божію. Таковый \textit{показуетъ вѣру свою отъ дѣлъ своихъ}\footnote{Іак.~2,~18.}; таковый \textit{бдитъ и блюдетъ ризы своя, да не нагъ ходитъ и узрятъ срамоту его}\footnote{Апок.~16,~15.}; таковыи суть \textit{земля благая}, пріемлющая благословенное Божія слова сѣмя, и \textit{творятъ плодъ въ терпѣніи ово сто, ово же шестьдесятъ, ово тридесять}\footnote{Мѳ.~13,~23; Лук.~8,~15.}. Сіи суть, \textit{иже насаждени въ дому Господни, во дворѣхъ Бога нашего процвѣтутъ}\footnote{Пс.~91,~14.}. Сіи истинныхъ хрістіанъ примѣты суть и доказательства, что они творятъ правду, по ученію Апостола: \textit{разумѣйте, яко всякъ, творяй правду, отъ Него родися}\footnote{1~Іоан.~2,~29.}, "--- якоже ложныхъ хрістіанъ доказательство есть, которые не творятъ правды: \textit{всякъ, не творяй правды, нѣсть отъ Бога}\footnote{1~Іоан.~3,~10.}. Сіи истинныхъ хрістіанъ примѣты разсудивше, осмотримся, хрістіанине, находимся ли мы между ими, и потщимся сотворить плоды достойны покаянія, да не услышимъ отъ Господа въ день Его: \textit{не вѣмъ васъ, откуду есте. Отступите отъ Мене вси дѣлателіе неправды}\footnote{Лук.~13,~27.}.

\subsection[Глава 9-я. О различныхъ хрістіанскихъ должностяхъ, которыя послѣдуютъ вѣрѣ святой, и отъ вышеписанныхъ разсужденій слѣдуютъ.]{глава девятая.\\\bfseries О различныхъ хрістіанскихъ должностяхъ, которыя послѣдуютъ вѣрѣ святой, и отъ вышеписанныхъ разсужденій слѣдуютъ.}

\begin{quotation}\textit{Прочее, братіе моя, елика суть истинна, елика честна, елика праведна, елика пречиста, елика прелюбезна, елика доброхвальна, аще кая добродѣтель и аще кая похвала, сія помышляйте}\footnote{Фил.~4,~8.}.\end{quotation}

\paragraph*{§\:392.} Хрістіанамъ должно горнихъ, то"=есть, небесныхъ и вѣчныхъ благъ искать со тщаніемъ, а не временныхъ и мірскихъ, по ученію Апостола: \textit{аще убо воскреснусте со Хрістомъ, вышнихъ ищите, идѣже есть Хрістосъ одесную Бога сѣдя: горняя мудрствуйте, а не земная}\footnote{Кол.~3,~1--2.}. Понеже 1)~хрістіане воскреснули духовно и вновь отродилися, и изъ плотскихъ духовными учинилися, то и искать должны не плотскихъ и мірскихъ, но духовныхъ и небесныхъ. 2)~Требуетъ того отъ нихъ вѣра ихъ, которая очищаетъ сердце отъ мірскихъ похотей, и обращаетъ къ Богу и желанію небесныхъ. Вѣра бо есть даръ Божій, духовный, то и научаетъ насъ \textit{духовныхъ} искать, когда ее имѣемъ; и понеже \textit{отъ небеси} приходитъ къ намъ, то къ небеси восхищаетъ и влечетъ сердце наше. 3)~Хрістіане въ мірѣ семъ суть \textit{пришельцы и странники}\footnote{Пс.~38,~13; 118,~19; Евр.~11,~13.}. Того ради не должны примѣшиваться сынамъ вѣка сего, \textit{имже богъ чрево ихъ, и слава въ студѣ ихъ, иже земная мудрствуютъ}\footnote{Филип.~3,~19.}. 4)~Хрістіане \textit{позваны въ жизнь вѣчную}\footnote{1~Тим.~6,~12.}, которая \textit{едино есть на потребу, и благая часть, яже не отымется отъ насъ}\footnote{Лук.~10,~42.}. Сію убо едину избирать и искать со всякимъ прилѣжаніемъ должно. 5)~\textit{Ничтоже внесохомъ въ міръ сей, явѣ, яко ниже что изнести можемъ. Имѣюще же пищу и одѣяніе, сими довольни да будемъ. А хотящіи богатитися впадаютъ въ напасти и сѣти и похоти многи}, и прочая\footnote{1~Тим.~6,~7--9 и слѣд.}. Сего ради не должно намъ сокровищъ скрывать на земли, но \textit{на небеси}\footnote{Матѳ.~6,~19 и 20.}. 6)~Жилище хрістіанъ \textit{есть на небесѣхъ}, по ученію Апостола\footnote{Филип.~3,~20.}. Сего ради, \textit{аще и течетъ} намъ \textit{богатство}, не должно намъ \textit{прилагать сердца}\footnote{Пс.~61,~11.}, но чрезъ нищихъ руки туды предпослать, да \textit{идѣже сокровище наше будетъ, ту будетъ и сердце наше}\footnote{Мѳ.~6,~21.}. \textit{(Смотри еще главу о презрѣніи міра въ первой книгѣ)}.

\paragraph*{§\:393.} Хрістіане не тѣло тлѣнное, но душу нетлѣнную украшать должны. Понеже 1)~хрістіане въ святомъ крещеніи отродилися и обновилися по внутреннему человѣку, который есть духовенъ; убо того новаго и внутренняго человѣка и украшать имъ должно. 2)~Хрістіанъ отъ суетнаго тѣлеснаго украшенія должно отвратить начало одѣянія. Читаемъ въ святой Библіи, что прародители наши прежде паденія не имѣли одѣянія, но \textit{наги были и нестыдилися}\footnote{Быт.~2,~25.}; а по паденіи уже дано имъ одѣяніе, ради прикрытія наготы и срама ихъ\footnote{3,~7--11,~21.}. Начало убо одѣянія нашего есть грѣхъ, почему одѣяніе наше обличаетъ грѣхъ нашъ, и показуетъ бѣдность нашу; а тако должно насъ смирять, а не возносить и пышность намъ придавать; должно намъ на память приводить, что мы грѣшники и закона Божія разорители есмы съ прародителями нашими. Сего ради и не должно намъ отъ того украшенія суетныя славы искать, но тѣмъ паче смиряться, и употреблять его къ прикрытію бѣднаго и нагаго нашего тѣла, а не къ украшенію, щегольству и пышности. 3)~Стыдъ и срамъ есть хрістіанину тѣло, въ землю имѣющее обратитися, украшать, а душу безсмертную и къ вѣчному животу благодатію Божію возобновленную пренебрегать и нагу оставлять. 4)~Украшеніе хрістіанское состоитъ не въ златѣ, сребрѣ, каменіяхъ драгихъ и одѣяніи многоцѣнномъ, яже суть сокровища міра сего и украшенія сыновъ вѣка сего, но въ вѣрѣ истинной и живой и въ хрістіанскихъ добродѣтеляхъ, въ которыя, яко во многоцѣнную утварь, хрістіане повседневно облекаться и ими украшаться должны, дабы не наги ходили предъ святѣйшими небеснаго Отца очами, якоже учитъ Апостолъ: \textit{облецытеся якоже избранніи Божіи святи и возлюбленни, во утробы щедротъ, благость, смиренномудріе, кротость и долготерпѣніе}, и проч.\footnote{Кол.~3,~12.} Сіе есть истинное и богатое хрістіанское украшеніе! Сія утварь, очамъ Божіимъ и святымъ ангеламъ Его благоугодна, самымъ хрістіанамъ душеспасительна. 5)~Тѣлесное украшеніе, какъ и всякая міра сего суета, на малое время, до смерти служитъ; яко тогда все міра сего сокровище отъ насъ отходитъ, такъ что и тѣло самое въ землѣ оставляемъ; но духовное украшеніе, вѣрою и плодами ея пріобрѣтенное, ни самая смерть не отлучаетъ отъ насъ, но на оный вѣкъ отходитъ съ нами, и на всемірномъ собраніи и судѣ Хрістовомъ явится съ нами и подастъ о насъ свидѣтельство Судіи праведному и всему міру, что мы не суетѣ, но истинѣ послѣдовали. 6)~Хрістіане къ небесной великой \textit{вечери}\footnote{Лук.~14,~16.} и преславному \textit{браку Агнца Хріста}\footnote{Апок.~19,~7 и 9.} позваны. Якоже убо хотящіи въ чертогъ царскій внити и на трапезѣ его сѣсти, украшаютъ себе различно, дабы не явиться предъ царемъ земнымъ гнусными, и тако не быть изгнанными: тако наипаче хрістіанамъ, хотящимъ внити въ чертогъ небеснаго Царя и возлещи на преславной оной и пресладкой вечери, на которой \textit{возлягутъ со Авраамомъ, Исаакомъ и Іаковомъ} избранніи Божіи\footnote{Матѳ.~8,~11.}, и \textit{упіются отъ тука дома Господня, и потокомъ сладости Его напоятся}\footnote{Пс.~35,~9.}, подобаетъ имѣть одѣяніе брачное, достойное брака онаго, и тѣмъ украсить себе, да не кто постраждетъ тое, что пострадалъ пришедшій на бракъ сына Царева безъ одѣянія брачнаго, которому Царь, учиня выговоръ: \textit{друже! како вшелъ еси сѣмо, не имый одѣянія брачна?} сказалъ слугамъ: \textit{связавше ему руцѣ и нозѣ, возмите его, и вверзите во тму кромѣшнюю: ту будетъ плачь и скрежетъ зубомъ. Мнози бо суть звани, мало же избранныхъ}\footnote{Мѳ.~22,~2--12,~13 и 14.}.

\paragraph*{§\:394.} Хрістіанамъ должно пищи и питія умѣренно употреблять, по повелѣнію Господню: \textit{внемлите себѣ, да не когда отягчаютъ сердца ваша объяденіемъ и піянствомъ и печальми житейскими}\footnote{Лук.~21,~34.}. 1)~Требуетъ отъ нихъ того новое рожденіе, которое, яко \textit{духовное}, духовною пищею, словомъ Божіимъ, молитвою и проч. укрѣпляется и растетъ; а всякимъ плотскимъ излишествомъ умерщвляется и исчезаетъ. 2)~Апостолъ глаголетъ: \textit{нѣсть царство Божіе брашно и питіе, но правда и миръ и радость о Дусѣ Святѣ}\footnote{Римл.~14,~17.}. 3)~Богатыхъ трапезъ и драгихъ винъ излишнее употребленіе не можетъ быть какъ безъ обиды ближняго, такъ безъ оскорбленія Божія величества. Ибо хотя они благочинно и безъ соблазна отправляются, что весьма рѣдко бываетъ, или безъ неправеднаго прибытка приготовляются, что такожде рѣдко случается; однакожъ излишность сія не допущаетъ бѣдныхъ и нищихъ людей снабдѣвать. И тако грѣшитъ роскошный хрістіанинъ противу правды, которая учитъ насъ \textit{дѣлать тое ближнему нашему, что хощемъ себѣ}\footnote{Лук.~6,~31; Мѳ.~7,~12.}; грѣшитъ противу Хріста, Который повелѣваетъ \textit{убогихъ снабдѣвать}\footnote{Марк.~10,~21; Лук.~14,~13.}; грѣшитъ противу любви хрістіанской, которая велитъ намъ брата нашего \textit{любить не словомъ, ниже языкомъ, но дѣломъ и истиною}\footnote{1~Іоан.~3,~18.}; грѣшитъ и противу естества своего, которое умѣренностію довольствуется и излишествомъ обременяется; грѣшитъ и противу заповѣди: \textit{внемлите себѣ, да не когда отягчаютъ сердца ваша объяденіемъ и піянствомъ и печальми житейскими}, и проч. Видиши, хрістіанине, колико грѣховъ и беззаконій роскошь влечетъ за собою. А что, когда она изъ неправедныхъ прибытковъ, хишеній, мздоиманій, излишнихъ съ подчиненныхъ оброковъ, сборовъ и работъ составляется, какъ и по большей части дѣлается? То не инымъ чимъ роскошный питается и утучняется, какъ явнымъ нечестіемъ. "--- 4)~Хрістіанамъ должно помнить, что богачу оному евангельскому, веселившемуся на вся дни свѣтло, и гнойнаго Лазаря презиравшему, по смерти судомъ Божіимъ опредѣлено. По маловременныхъ веселостяхъ на вѣчное мученіе преселился; возглашаетъ изъ ада, сый въ мукахъ: \textit{отче Аврааме! Помилуй мя, и посли Лазаря, да омочитъ конецъ перста своего въ водѣ, и устудитъ языкъ мой: яко стражду во пламени семъ}\footnote{Лук.~16,~19--24.}. По драгоцѣнныхъ винахъ и богатыхъ трапезахъ капли воды проситъ, но не получаетъ, и во вѣки не получитъ\footnote{Ст.~25--26.}. "--- 5)~Хрістіанамъ должно и тое разсуждать, что они пищи и питія просятъ отъ Бога: \textit{хлѣбъ нашъ насущный даждь намъ днесь}\footnote{Матѳ.~6,~11.}, не къ сладострастію и роскоши, и дается имъ пища не къ угожденію плоти въ похотяхъ ея, но къ подкрѣпленію ея немощи, дабы могла служить имъ къ благословеннымъ званія ихъ трудамъ, дабы, пищею подкрѣпившеся, могли трудиться во славу Божію, свою и ближняго пользу. Сей есть конецъ хрістіанскій пищи и питія употребленія. Къ сему Апостолъ увѣщаваетъ насъ: \textit{аще ясте, аще ли піете, аще ли ино что творите, вся въ славу Божію творите}\footnote{1~Кор.~10,~31.}. "--- 6)~Надобно хрістіанамъ и того опасаться, чтобы не попасться въ число тѣхъ, о которыхъ Апостолъ \textit{съ плачемъ} написалъ: \textit{имже богъ чрево}\footnote{Филип.~3,~18 и 19.}. Сей безчестный и душевредный титулъ тѣмъ людямъ приличествуетъ, которые пищи и питія или выше мѣры, или ради сластнаго плотскаго угодія, а не ради нужды и подкрѣпленія немощи плотскія, какъ выше сказано, употребляютъ, сласть мняще, какъ Апостолъ глаголетъ, \textit{вседневное насыщеніе}\footnote{2~Петр.~2,~13.}. "--- 7)~О хрістіанѣхъ первенствующія церкви пишется въ Дѣяніяхъ Апостольскихъ, что не тщалися о богатыхъ трапезахъ, но, \textit{ломяще по домомъ хлѣбъ, пріимаху пищу въ радости и простотѣ сердца, хваляще Бога}, и проч.\footnote{Дѣян.~2,~46 и 47.} Тако должно и нынѣшнимъ хрістіанамъ не о богатыхъ трапезахъ тщаться, но довольствоваться тѣмъ, чимъ немощь тѣлесная укрѣпляется, и не чреву, но Богу угождать, и не тѣло утучнять, но душу безсмертную питать словомъ Божіимъ, псалмы и пѣсньми духовными увеселять, и, пріемля пищу сію, поминать о пищи вѣчнаго живота, къ которому путь пресѣкаетъ невоздержное пищи и питія употребленіе.

\paragraph*{§\:395.} Хрістіанамъ должно съ молитвою и благодареніемъ пищу принимать. 1)~Хрістіанамъ должно знать, что все: пища, питіе, одѣяніе и прочее Божій есть даръ. Якоже убо, хотячи коснуться чужаго добра, просимъ у хозяина дозволенія, иначе за хищеніе и воровство поставляется, когда чужаго касаемся безъ воли хозяина: тако наипаче должно просить благословенія у Бога, когда хощемъ пищи и питія касаться, яко все Божіе есть, а не наше. Откуду повелѣно намъ, яко нищимъ и убогимъ, просить: \textit{хлѣбъ нашъ насущный даждь намъ днесь}; а безъ того надобно опасаться, чтобы и намъ не поставилося отъ Бога за хищеніе, когда Его добра безъ благословенія Его касаемся. "--- 2)~Апостолъ глаголетъ: \textit{освящается} пища \textit{словомъ Божіимъ и молитвою}\footnote{1~Тѳ.~4,~5.}. 3)~Съ благодареніемъ должно пищу принимать, яко сей даръ отъ Бога, всѣхъ благихъ Дателя, подается намъ, якоже поетъ Псаломникъ: \textit{очи всѣхъ на Тя, Господи, уповаютъ, и ты даеши имъ пищу во благовременіи; отверзаеши Ты руку Твою, и исполняеши всякое животно благоволенія}\footnote{Пс.~144,~15--16.}. "--- 4)~Хрістіанамъ не такъ должно пищи, питія и прочіихъ благихъ Божіихъ причащаться, какъ дѣлаютъ язычники, которые, не зная Бога, всѣхъ благъ Дателя, не просятъ у Него тѣхъ благихъ, ни благодарятъ Благодѣтеля, хотя ихъ и питаетъ, какъ Своихъ людей; но должно имъ вѣрою отъ Бога всего, и насущнаго хлѣба просить, и получая благодарить отъ сердца Благодѣтелю. "--- 5)~Хрістосъ Сынъ Божій и словомъ намъ тое, повелѣвая молиться: \textit{Отче! хлѣбъ нашъ насущный даждь намъ днесь}, и примѣромъ Своимъ показалъ, якоже пишется: \textit{воспѣвше} (по послѣдней вечери), \textit{изыдоша на гору Елеонску}\footnote{Мѳ.~26,~30. См. еще мѣста св. Писанія: Іоан.~6,~11; Мѳ.~14,~19; Марк.~6,~41; Лук.~9,~16.}. Такожде и о первыхъ хрістіанахъ повѣствуется, что \textit{ломяще по домомъ хлѣбъ, пріимаху пищу въ радости и простотѣ сердца, хваляще Бога}\footnote{Дѣян.~2,~46--47.}. Тоежде должно и нынѣшнимъ дѣлать, когда не хотятъ быть къ Богу неблагодарными и ложными хрістіанами. Примѣчай здѣ, хрістіанине, что здѣ о хлѣбѣ слово есть такомъ, который безъ обиды ближняго снискуется, а не о такомъ, который воровствомъ, хищеніемъ, лихоиманіемъ, лихвою или процентомъ, лестію, лукавствомъ и прочею неправдою ищется. Такому бо хлѣбу, яко беззаконно снисканному, безстыдно и беззаконно есть и благословенія просить у Бога, Который \textit{ненавидитъ неправды}\footnote{Евр.~1,~9.}. Отъ такого хлѣба и снискавшаго его удаляется благословеніе Божіе, но вмѣсто того приходитъ клятва, подвергающая его вѣчному осужденію, яко законопреступника; и таковый есть ложный хрістіанинъ, понеже явно противится заповѣди Божіей: \textit{не укради}, и заповѣдавшему Богу не покаряется. Такожде, когда съ молитвою и благодареніемъ пищу и питіе, аки отъ руки Божія подающія, хощемъ пріимать, то должно берещися въ употребленіи ихъ погрѣшить излишествомъ, какъ выше сказано; нѣтъ бо такожде благословенія Божія пищѣ и питію, ради обжирства, піянства и безчинной роскоши употребляемому, но единой умѣренности, когда отъ благословенныхъ трудовъ и во славу Божію бываетъ, посылается Божіе благословеніе.

\paragraph*{§\:396.} Хрістіане не должны въ \textit{праздности} жити, но въ благословенныхъ трудахъ упражняться. Понеже 1)~сама собою праздность есть грѣхъ, яко противна заповѣди Божіей есть: \textit{въ потѣ лица твоего снѣси хлѣбъ твой}, глаголалъ Богъ Адаму\footnote{Быт.~3,~19.}, которая заповѣдь и насъ, сыновъ Адамовыхъ, касается. 2)~Всякаго грѣха виною бываетъ праздность, что и самыи язычники признали. 3)~Апостолъ Павелъ, вселенныя учитель, представляетъ себе въ примѣръ хрістіанамъ, яко трудитися подобаетъ, и тако хлѣбъ ясти. \textit{Сами вѣсте, яко требованію моему и сущимъ со мною послужистѣ руцѣ мои сіи}\footnote{Дѣян.~20,~34.}; и паки: \textit{не туне хлѣбъ ядохомъ у кого, но въ трудѣ и подвизѣ, нощь и день дѣлающе, да не отягчимъ никогоже отъ васъ}\footnote{2~Сол.~3,~8.}. Отъ сего порока выключаются немощные, престарѣлые и содержащіеся въ узахъ, которыхъ одолжаются хрістіане сообща питать. \textit{(Смотри еще главу о праздности первыя книги)}.

\paragraph*{§\:397.} Хрістіане должны \textit{языкъ} свой \textit{хранить}, яко ничимъ такъ не грѣшитъ человѣкъ, какъ языкомъ необузданнымъ. Святый Апостолъ Іаковъ, описуя необузданность и отъ того вредъ языка человѣческаго, глаголетъ: \textit{языкъ малъ удъ есть, и вельми хвалится: се малъ огнь, и коль велики вещи сожигаетъ. И языкъ огнь, лѣпота неправды: сице и языкъ водворяется во удѣхъ нашихъ, скверня все тѣло, и паля коло рожденія нашего, и опаляяся отъ геенны. Всяко бо естество звѣрей же и птицъ, гадъ же и рыбъ укрощается и укротится естествомъ человѣческимъ: языка же никтоже можетъ отъ человѣкъ укротити, неудержимо бо зло, исполнь яда смертоносна. Тѣмъ благословляемъ Бога и Отца: и тѣмъ клянемъ человѣки, бывшія по подобію Божію}\footnote{Іак.~3,~5--9.}. Такъ страшно есть зло "--- языкъ, когда прилѣжно не хранится! Страшныи слѣдствія и премногая злая необузданнаго языка представляетъ намъ и Сирахъ: \textit{языкъ трегубый многи потрясе, и разлучи я отъ языка во языкъ, и грады тверды разори, и домы вельможей преврати; языкъ трегубый жены доблія изгна, и лиши я отъ трудовъ ихъ. Внемляй ему не имать обрѣсти покоя, ниже вселится съ безмолвіемъ. Язва бичная струпы творитъ: язва же язычная сокрушаетъ кости. Мнози падоша остріемъ меча, но не якоже падшіи языкомъ. Блаженъ, иже укрыется отъ него, иже не пройде въ ярости его, иже не повлече ига его, и узами его не связанъ бысть: иго бо его, иго желѣзно; и узы его, узы мѣдяны. Смерть люта, смерть его, и паче его лучше есть адъ}\footnote{Сир.~28,~16--24.}. Такъ много золъ языкъ необузданный дѣлаетъ! Сего ради должно хрістіанамъ весьма тщаться о томъ, чтобы его хранить во оградѣ своей, и не попускать ему свободно исходить, яко глаголетъ Апостолъ: \textit{аще кто мнится вѣренъ быти въ васъ, и не обуздаваетъ языка своего, но льститъ сердце свое, сего суетна есть вѣра}\footnote{Іак.~1,~26.}. И Хрістосъ судомъ грозитъ необузданному языку: \textit{всяко слово праздное, еже еще рекутъ человѣцы, воздадятъ о немъ слово въ день судный}\footnote{Мѳ.~12,~36.}. Аще за слово праздное, кольми паче за слово скверное, срамное, хульное, укорительное, поносительное, воздадутъ отвѣтъ. Сего ради такъ сильно слово Божіе увѣщаваетъ насъ къ храненію языка: \textit{да будетъ всякъ человѣкъ скоръ услышати, и косенъ глаголати, косенъ во гнѣвъ}\footnote{Іак.~1,~19.}. \textit{Кто есть человѣкъ, хотяй животъ, любяй дни видѣти благи? Удержи языкъ твой отъ зла, и устнѣ твои, еже не глаголати льсти}\footnote{Пс.~33,~13--14; 1~Петр.~3,~10 и на прочіихъ мѣстахъ.}. Помнить хрістіане должны, что они святое и страшное имя Божіе призываютъ, поютъ и хвалятъ: како убо имя Божіе великое и святое призовутъ тѣмъ языкомъ, который оскверняютъ срамословіемъ, сквернословіемъ, буесловіемъ, кощунствомъ, клятвою, хуленіемъ, злословіемъ и прочіимъ смрадомъ? И тое имъ помнить должно, что причащаются пречистаго Тѣла и Крове Хрістовыя: како убо пріимутъ во уста тую \textit{Святыню}, "--- во уста, глаголю, которыя гнилыми и смрадными словами оскверняютъ? Внимай сему прилѣжно, необузданный языкъ, да неосужденно имя Божіе призовешь, и Святыхъ Таинъ причастишися \textit{(Смотри еще главу о языкѣ въ книгѣ первой)}.

\paragraph*{§\:398.} Хрістіанамъ должно къ \textit{терпѣнію} себе пріуготовлять, и въ томъ себе содержать, хотя плоти и горестно. Ибо много хрістіанамъ бѣдствія и напастей въ мірѣ семъ бываетъ. \textit{Многи скорби праведнымъ}\footnote{Пс.~33,~20.} и: \textit{вси, хотящіи благочестно жити о Хрістѣ Іисусѣ, гоними будутъ}\footnote{2~Тѳ.~3,~12.}; \textit{въ мірѣ скорбни будете}\footnote{Іоан.~16,~33.}, и на прочіихъ Писанія святаго мѣстахъ свидѣтельствуется о томъ. Сатана бо со своими служителями какъ отъ начала міра озлоблялъ, такъ и нынѣ озлобляетъ, и до конца міра озлоблять не престанетъ, и чимъ ближе наступаетъ конецъ міра, тѣмъ болѣе озлоблять будетъ рабовъ Божіихъ, и напастей на нихъ посылать. \textit{Горе живущимъ на земли и мори: яко сниде діаволъ къ вамъ, имѣя ярость великую, вѣдый, яко время мало есть}\footnote{Ап.~12,~12.}. Противу сего лютаго врага и злыхъ его слугъ, искушеній ихъ и озлобленій, ничимъ инымъ, хрістіанине, не можемъ вооружаться, какъ терпѣніемъ и молитвою. На тое бо искушенія его предсказаны отъ Духа Святаго, и отъ рабовъ Его святыхъ написаны, да ихъ всегда, какъ стрѣлъ разженныхъ, ожидаемъ, и бросаемыхъ великодушно терпѣть и угашать силою терпѣнія и молитвы потщимся, взирая на Самаго Хріста Сына Божія, Который терпѣніемъ креста врага нашего побѣдилъ, и на святыхъ Его, которые терпѣніемъ послѣдовали Ему. Къ великодушному терпѣнію и до конца въ томъ пребыванію да ободряетъ насъ \textit{упованіе, отложенное намъ на небесѣхъ}\footnote{Кол.~1,~5.}. Въ скорбехъ и напастехъ помянемъ Апостольское слово, \textit{яко недостойны страсти нынѣшняго времене къ хотящей славѣ явитися въ насъ}\footnote{Римл.~8,~18.}. Ничто бо такъ не ободряетъ и не утѣшаетъ человѣческаго сердца въ трудахъ и напастехъ, какъ добрая надежда. Воина въ явную смерть и бѣдствіе опаснѣйшее, надежда побѣды и славы; купца по чужимъ странамъ и опаснымъ дорогамъ скитаться, надежда богатства, поселянина чрезъ цѣлое лѣто потѣть и трудиться, надежда плода поощряетъ; надежда свободы увеселяетъ сѣдящаго въ темницѣ, надежда возвращенія утѣшаетъ отлучившагося отъ дома и отечества; надежда здравія оживляетъ болящаго. Какъ несказанно радуются вси, получивше конецъ надежды своея! Забываютъ прежнюю скорбь и бѣдствіе, и утѣшаются тѣмъ, чего надѣялися и ожидали. Аще временныхъ благъ, которыя подобны дыму, вмалѣ являющемуся и скоро исчезающему, надежда человѣка такъ сильно ободряетъ къ понесенію трудовъ и бѣдствій: како не ободритъ насъ, о хрістіане, надежда вѣчныхъ благъ, \textit{ихже око не видѣ, и ухо не слыша, и на сердце человѣку не взыдоша, яже уготова Богъ любящимъ Его}\footnote{1~Кор.~2,~9.}! Когда вѣрою воззримъ на нихъ; когда помыслимъ тую чадъ Божіихъ славу, которою \textit{просвѣтятся, яко солнце, во царствіи Отца ихъ}\footnote{Мѳ.~13,~43.}; представимъ уму нашему тую преславную, великую и сладкую вечерю, на которую \textit{мнози отъ востокъ и западъ пріидутъ и возлягутъ со Авраамомъ, Исаакомъ и Іаковомъ во царствіи небеснѣмъ}\footnote{Мѳ.~8,~11.}\textit{}; и \textit{работающіи Господеви ясти будутъ, пити будутъ, радоватися будутъ и веселитися въ веселіи сердца будутъ}\footnote{Ис.~65,~13--14.}; узримъ Отца нашего небеснаго \textit{лицемъ къ лицу, Котораго нынѣ якоже зерцаломъ въ гаданіи видимъ}\footnote{1~Кор.~13,~12.}, и Котораго лицезрѣнія ангели святіи безъ сытости насыщаются, Того мы тогда насыщатися съ непрестаннымъ желаніемъ будемъ; въ Котораго Хріста, распятаго, умершаго и воскресшаго, нынѣ вѣруемъ, тогда Его уже въ Божественной Своей славѣ царствующаго увидимъ, и \textit{съ Нимъ воцаримся}\footnote{2~Тѳ.~2,~12; Ап.~22,~4 и 5.}; Которому нынѣ сообразуемся въ терпѣніи, Тому тогда сообразны будемъ въ славѣ, яко \textit{подобни Ему будемъ}\footnote{1~Іоан.~3,~2.}; съ Которымъ нынѣ страждемъ, съ Тѣмъ тогда и \textit{прославимся}\footnote{Римл.~8,~17.}; о которыхъ нынѣ въ святомъ Писаніи читаемъ и слышимъ святыхъ ангелахъ, патріархахъ, пророкахъ, апостолахъ и прочіихъ святыхъ, съ тѣми тогда блаженное и пресладкое возымѣемъ дружество; тогда вси сіи бѣдствія, напасти, печали, болѣзни, воздыханія и слезы забудемъ; но, вмѣсто того, \textit{возрадуется сердце наше, и радости нашея никтоже возметъ отъ насъ}\footnote{Іоан.~16,~23.}. \textit{Собранніи Господемъ обратятся, и пріидутъ въ Сіонъ съ радостію, и радость вѣчная надъ главою ихъ: надъ главою бо ихъ хвала и веселіе, и радость пріиметъ я; отбѣже болѣзнь, печаль и воздыханіе}\footnote{Ис.~35,~10.}. О всежеланнаго и радостнаго дня того, въ который, по обѣщанію Божію, пріимемъ сія отъ всещедрой десницы небеснаго Отца! Сею надеждою укрѣпляетъ Богъ рабовъ Своихъ въ подвигѣ ихъ, и утѣшаетъ въ печали ихъ, и увеселяетъ въ скорбехъ ихъ. На сей бо конецъ различно изображено и описано въ святомъ Писаніи блаженство оное, да на него вѣрою взирая, подкрѣпимся и укрѣпимся въ подвигѣ нашемъ, и въ скорбехъ утѣшеніе живое почувствуемъ. \textit{Елика бо преднаписана быша, въ наше наказаніе преднаписашася, да терпѣніемъ и утѣшеніемъ писаній упованіе имамы}\footnote{Римл.~16,~4.}. Сею пресладкою надеждою укрѣпляли и утѣшали себе святіи Божіи въ подвигѣ благочестія; сею должно и намъ укрѣпляться и не изнемогать въ злостраданіихъ, наносимыхъ отъ сатаны и служителей его. Всему временному будетъ конецъ; и благополучіе и злополучіе временное скоро мимоидетъ; и гонящихъ и гонимыхъ достойный ожидаетъ жребій, и озлобляющій и озлобляемый пріиметъ свое отъ правды Божіей. \textit{Праведно у Бога, воздати скорбь оскорбляющимъ васъ: а вамъ оскорбляемымъ отраду съ нами, въ откровеніи Господа Іисуса съ небесе, со Ангелы силы Своея, во огни пламеннѣ дающаго отмщеніе невѣдущимъ Бога, и не послушающимъ благовѣствованія Господа нашего Іисуса Хріста}, глаголетъ Апостолъ Хрістовъ\footnote{2~Сол.~1,~6--8.}. \textit{(Смотри еще главу о терпѣніи въ первой книгѣ)}.

\paragraph*{§\:399.} \textit{О чемъ должно радоватися хрістіанамъ?} "--- \textit{Отвѣтъ}. 1)~Что любитъ человѣкъ, о томъ и радуется. И какъ любовь есть двоякая, мірская или плотская, и духовная или Божія; такъ и радость есть двоякая, мірская или плотская, и духовная или хрістіанская. Любовь и радость познается изъ послѣдующія печали о потерянной вещи, которую человѣкъ любитъ, и радуется о ней. Такъ отецъ печалится о сынѣ, мужъ о женѣ, другъ о другѣ, братъ о братѣ умершемъ; печалится же не иной какой ради причины, какъ что разлучилися съ любимыми; и потому печалію своею показуютъ, что они тѣхъ, съ которыми разлучилися, любятъ, и которыми въ присутствіи ихъ утѣшалися, уже въ отсутствіи не утѣшаются, чего ради и оскорбляются тѣмъ. Сего не дѣлаетъ человѣкъ, когда лишается того, чего не любитъ. Такъ о чужомъ умершемъ не печалится, понеже не любилъ его; не любилъ же его потому, что имъ живущимъ не утѣшался; не утѣшался же потому, что не любилъ и не радовался о немъ живущемъ; а понеже о живущемъ не утѣшался, то и о умершемъ и отлучившемся не печалится. Знаменіе убо есть любве къ вещи какой радость о ней; знаменіе же радости есть послѣдующая печаль о той погибшей. "--- 2)~Сынове вѣка сего радуются о чести, славѣ, златѣ, сребрѣ, богатствѣ, роскошахъ и сластяхъ міра сего. Сіе "--- ихъ сокровище, утѣха и увеселеніе, а потому радуются о томъ сокровищѣ своемъ, яко любятъ его. Хрістіане сію радость должны изгонять изъ сердца своего, и не давать ей мѣста въ сердцѣ своемъ. Отъ сея радости отводитъ ихъ Духъ Святый: \textit{не любите міра, ни яже въ мірѣ}\footnote{1~Іоан.~2,~15.}. Что бо человѣкъ любитъ, о томъ и радуется. И потому, какъ любить міра, такъ и радоваться о немъ не должно хрістіанамъ. "--- 3)~Истинная хрістіанская радость состоитъ въ Бозѣ. Сіе ихъ сокровище, утѣшеніе, радость и увеселеніе есть: они о томъ радоватися должны, что вѣруютъ и работаютъ Богу не такому, каковы были и суть боги языческіи, \textit{бездушные, глухіе, нѣмые, слѣпые, сребро и злато, дѣла рукъ человѣческихъ}\footnote{Пс.~113,~12--15.}, но работаютъ Богу живому, безсмертному, присносущному, премудрому, всемогущему, вся въ руцѣ Своей содержащему, о всѣхъ промышляющему, благому и милостивому Создателю и Промыслителю своему. О семъ радоватися имъ велитъ Духъ Святый: \textit{радуйтеся праведніи о Господѣ}\footnote{Пс.~32,~1.}, "--- и чрезъ Апостола: \textit{радуйтеся о Господѣ}\footnote{Филип.~3,~1.}, и всегда радоватися повелѣваетъ: \textit{радуйтеся всегда о Господѣ: и паки реку, радуйтеся}\footnote{4,~4.}. "--- 4)~Радость имъ сія и отсюду проистекать наипаче должна, что они того великаго и живаго Бога Отцемъ своимъ нарицаютъ и имѣютъ, Которому и молятся тако: \textit{Отче нашъ, Иже еси на небесѣхъ!} и проч.\footnote{Мѳ.~6,~9.} Велико есть земнаго царя нарицать и имѣть за отца: несравненно большее есть достоинство "--- Царя небеснаго, Бога Вседержителя Отца имѣть и призывать. "--- 5)~Паки оттуду сію духовную утѣху почерпать должны, что они благодатію Божіею \textit{призваны изъ тмы въ чудный Божій свѣтъ}\footnote{1~Петр.~2,~9.}, и \textit{отъ области сатанины въ царство Хрістово}\footnote{Дѣян.~26,~18; Колос.~1,~13.}, что надъ ними не князь міра сего, якоже надъ языками, незнающими Бога и прочіими беззаконными, но Царь міра, Царь небесный, Царь праведный, Сынъ Божій, Ѵпостасное Отчее Слово со Отцемъ и Духомъ Святымъ царствуетъ, управляетъ, защищаетъ и сохраняетъ ихъ. О томъ имъ радоватися должно, якоже о семъ глаголетъ пророкъ, воспѣвая: \textit{и сынове Сіони}, то"=есть, сынове церкви, \textit{возрадуются о Царѣ своемъ}\footnote{Пс.~149,~2.}. "--- 6)~Наконецъ, радость сію весьма умножаетъ имъ надежда будущаго живота и блаженства, что благодатію Божіею отверзлося сокровище неизреченныхъ и вѣчныхъ благъ, которыхъ они получить благодатію Хрістовою надѣются, когда \textit{вѣрни до смерти будутъ}\footnote{Апок.~2,~10.}. Симъ утѣшаетъ ихъ Самъ радости сея Виновникъ, Іисусъ Хрістосъ: \textit{радуйтеся и веселитеся, яко мзда ваша многа на небесѣхъ}\footnote{Мѳ.~5,~12.}. "--- 7)~Радость сію духовную въ благополучіи и неблагополучіи имѣть должно хрістіанамъ, яко радость сія проистекаетъ отъ любви Божіей, которую они всегда имѣть должны. Богъ бо, яко непремѣняемая благостыня и любовь, всегда любленія достоинъ. Онъ и тогда намъ благъ и милостивъ есть и благодѣтельствуетъ, когда отнимаетъ отъ насъ благополучіе временное; тогда насъ милуетъ, когда біетъ насъ; тогда насъ щадитъ, когда наказуетъ; тогда насъ любитъ, когда опечаляетъ; тогда намъ благотворитъ, когда благая Своя отъемлетъ отъ насъ. Ибо \textit{егоже любитъ Господь, наказуетъ: біетъ же всякаго сына, егоже пріемлетъ}\footnote{Евр.~12,~6; Апок.~3,~19.}. Сего ради, какъ любовь къ Богу, такъ и послѣдующая ей радость духовная не токмо въ благополучіи, но и въ злополучіи должна быть въ хрістіанѣхъ. Ибо якоже отъ радости о Бозѣ истинная къ Богу любовь, такъ таяжде любовь въ неблагополучіи познается. Понеже кто только въ благополучіи радуется о Бозѣ, тотъ подаетъ о себѣ свидѣтельство, что онъ не Бога Самаго, но благополучіе, благая Божія временная любитъ. Иное же есть Бога любить, иное любить благая Его, какъ сіе всякъ можетъ признать. Тако, кто и человѣка дотолѣ любитъ, доколѣ ему благодѣтельствуетъ, тотъ не такъ самаго его, какъ благодѣяніе любитъ. Истинный бо человѣколюбецъ не токмо благодѣтеля и друга своего, но и всякаго человѣка любитъ и умилостивляется надъ нимъ. Тако и истинный боголюбецъ не токмо Бога любитъ тогда, когда увеселяетъ его временными благими, но и тогда, когда опечаляетъ его, отнимая временная благая. Отъ Бога бо, яко благаго и благихъ Источника, не что иное, какъ только благое, проистекаетъ и изливается на насъ. "--- 8)~Въ благополучіи и лицемѣры и злые люди радуются, любятъ Бога и благодарятъ Бога; но когда отымется отъ нихъ благополучіе, тогда ропщутъ, негодуютъ, печалятся, а часто и хулятъ; а тако показуютъ, что они устами, а не сердцемъ Бога любятъ, якоже пишется о Израильтянѣхъ: \textit{возлюбиша Его усты своими, и языкомъ своимъ солгаша Ему: сердце же ихъ не бѣ право съ Нимъ}\footnote{Пс.~77,~36--37.}, и тѣмъ свидѣтельствуютъ о себѣ, что они благая Божія любятъ, а не Самаго Бога. Не тако правое и боголюбивое сердце, которое Богу во всемъ покоряется и послѣдуетъ, не тако поступаетъ: оно и въ благоденствіи Бога, яко благодѣтеля своего, благодаритъ и поетъ; и въ злоденствіи, яко благодѣтеля своего, признаетъ и хвалитъ, и исповѣдуется съ Давидомъ: \textit{благо мнѣ, яко смирилъ мя еси}\footnote{118,~71.}; и на всякое время, веселое и печальное, съ тѣмжде Псаломникомъ благословитъ Его: \textit{благословлю Господа на всякое время, выну хвала Его во устѣхъ моихъ}\footnote{33,~2.}. И хотя плоть слѣпая и немощная начинаетъ волноваться и негодовать въ злоденствіи; но истинный боголюбецъ усмиряетъ ее и укрощаетъ, и тако всякую противность Божіею любовію побѣждаетъ, помышляя, что все отъ \textit{благаго} Бога посылается, что ни приключается намъ, и что лучше и полезнѣйше намъ въ злоденствіи быть, нежели въ благоденствіи. Въ благоденствіи можетъ человѣкъ вознестися и погибнуть; но въ злоденствіи смиряется, на что Богъ милосердо призираетъ, и \textit{даетъ благодать Свою}\footnote{Іак.~4,~6; 1~Петр.~5,~5.}. Примѣчай, хрістіанине, что можетъ быть хрістіанская радость и о временныхъ благихъ, напр. о побѣдѣ, надъ врагами церкви одержанной, о благораствореніи воздуха, о изобиліи плодовъ земныхъ, о здравіи своемъ и ближняго, и о прочемъ. Но и сія радость, какъ начало имѣетъ отъ Бога, всѣхъ благъ Дателя, такъ до Него относиться, и на Немъ единомъ почивать должна. Божія бо благая суть свидѣтельства благости Божіей; суть подобны ручьямъ, проистекающимъ отъ источника; подобны лучамъ, отъ солнца на землю спущаемымъ. Ручьи ведутъ насъ къ источнику и показуютъ намъ источникъ; и лучи солнечныи, которые насъ освѣщаютъ, возводятъ наши очи къ солнцу: тако временная благая отъ Бога, яко источника, проистекаютъ; и, яко лучи отъ присносущнаго солнца, на насъ низпущаются и согрѣваютъ, и возводятъ умъ и сердце наше къ Нему Самому, и увѣщаваютъ Его любити и радоватися о Немъ, яко Любителѣ и Благодѣтелѣ своемъ.

\paragraph*{§\:400.} Хрістіанамъ должно хранить себе отъ неблагодарности къ Богу. 1)~Человѣкъ неблагодарность Богу показуетъ, когда забываетъ благодѣянія Его, якоже неблагодарность Израильтянъ отсюду обличается, что \textit{забыша благодѣянія Его и чудеса Его, яже показа имъ}\footnote{Пс.~77,~21.}. Сіе бываетъ, когда человѣкъ благими Божіими насыщается, и не чувствуетъ; свѣтомъ освѣщается, воздухомъ сохраняется, огнемъ согрѣвается, хлѣбомъ насыщается, водою напояется, одеждою покрывается, отъ враговъ невидимыхъ, демоновъ, Божіею силою хранится, за грѣхи не тотчасъ казнится, и прочая благая отъ десницы Божіей пріемлетъ, и не познаетъ ихъ и не разумѣетъ, якоже пишется: \textit{мужъ безуменъ не познаетъ, и неразумивъ не разумѣетъ сихъ}\footnote{Пс.~91,~7.}. Что бы намъ воспослѣдовало, о хрістіанине, аще бы Богъ по правдѣ Своей за неблагодарствіе наше отнялъ хотя едину стихію, воздухъ? неотмѣнно бы слѣдовало всѣмъ тотчасъ умереть со скотами нашими. Или отнялъ бы отъ насъ воду, или якоже прежде во Египтѣ, \textit{въ кровь претворилъ}\footnote{Исх.~7,~20.}; или, вмѣсто свѣта дневнаго, послалъ тьму, якоже паки во Египтѣ учинилъ\footnote{10,~22.}: кто бы не захотѣлъ за великое сокровище того достать? Во тьмѣ познаемъ, коль великое добро есть свѣтъ, въ жаждѣ, воды драгость познаемъ; въ зимѣ и холодѣ огня нужду и пользу видимъ; въ голодѣ хлѣба цѣну узнаемъ; коль нужна одежда, въ наготѣ примѣчаемъ; какъ великое добро здравіе, въ болѣзни чувствуемъ; коль благопріятна и любезна намъ свобода, темница и неволя научаетъ. Что бы намъ паки было, когда бы Богъ по правдѣ Своей поступилъ съ нами, и тотчасъ за грѣхи наша казнилъ насъ: \textit{яко много согрѣшаемъ вси}\footnote{Іак.~3,~2.}, и \textit{грѣхопаденія кто разумѣетъ}\footnote{Пс.~18,~13.}? Или на малое бы время попустилъ врагу нашему діаволу, который, \textit{яко левъ рыкая ходитъ, искій кого поглотити}\footnote{1~Петр.~5,~8.}, котораго, яко пса на привязи, силою и властію Своею держитъ, да не вредитъ насъ, "--- что бы, глаголю, послѣдовало намъ, аще бы ему далъ свободу на насъ? Кто бы могъ цѣлъ быти? кто бы остался въ живыхъ? Убо что живемъ, движемся и есмы, Божіе есть превеликое благодѣяніе. Ежели человѣкъ сихъ и прочіихъ благодѣяній Божіихъ не чувствуетъ и не познаетъ, то онъ, по словеси Псаломника, \textit{мужъ безуменъ и неразумивъ} есть, и есть неблагодаренъ къ своему Благодѣтелю и Творцу, Богу. "--- 2)~Наипаче неблагодарность показуетъ Богу человѣкъ, когда забываетъ великое рода человѣческаго искупленія дѣло, что ради его Самъ Богъ на землю сошелъ, воплотился, на земли съ человѣками пожилъ, пострадалъ и умеръ, и тако спасеніе, которое онъ потерялъ, устроилъ ему. Но человѣкъ великую сію благодать Божію забываетъ, и, какъ бы ни во что вмѣняя, оставляетъ спасеніе свое, безцѣнною крови Хрістовой цѣною купленное, искать вѣрою; пренебрегаетъ вѣчнаго живота сокровище, которое \textit{едино есть на потребу}\footnote{Лук.~10,~42.}, и обращается къ міру сему, и яже въ немъ, и прилѣпляется къ суетной чести, славѣ, богатству и роскошамъ, и тако творитъ недостойнымъ себе вѣчнаго живота. Богъ хощетъ его помиловать, спасти, вѣчный животъ ему подать; но онъ тое благоволеніе Божіе презираетъ, и отвращается отъ Него, и такъ самовольно идетъ въ погибель, какъ слѣпый въ ровъ. Аще бы кто пришелъ къ утопающему въ водѣ, или погрязающему въ тинѣ, и, простерши руку, хотѣлъ бы его оттуду извлещи; или пришелъ бы кто къ плѣнному, и хотѣлъ бы его отъ плѣна избавить, но онъ самъ не хощетъ отъ бѣды своей избавиться, "--- не былъ ли бы безуменъ, и къ своему благодѣтелю неблагодаренъ? Неотмѣнно бы всякъ призналъ его за несмысленнаго и неблагодарнаго. Тако дѣлаетъ нераскаянный грѣшникъ, который отъ любви міра отстать, и къ Богу обратиться, и вѣчный животъ отъ Него получить не хощетъ. Хрістосъ, Сынъ Божій, пришелъ на землю, и простираетъ руку Свою святую ко всякому грѣшнику, погрязающему въ тинѣ грѣховной, и плѣненному отъ врага діавола; но грѣшникъ, яко \textit{мужъ безуменъ}, не хочетъ отъ бѣды своей избавиться, не соизволяетъ волѣ святой великаго своего Благодѣтеля и Спасителя, остается въ бѣдственномъ своемъ прежнемъ состояніи, погрязаетъ и утопаетъ въ грѣхахъ, какъ въ тинѣ, и остается, какъ былъ, въ горькомъ плѣнѣ и власти діавольской; и тако не къ иному чему, какъ къ вѣчной погибели, закрывши глаза идетъ. Тако къ Богу, Благодѣтелю своему, хотящему спасенія его, неблагодаренъ показуется, и самъ самовольно погибаетъ. "--- 3)~Неблагодарность и отсюду познается человѣческая къ Богу, что онъ Божію благость не токмо на всякъ день, но и на всякій часъ на себѣ дознаетъ, но Его, яко Благодѣтеля своего, не любитъ. Не любитъ же, яко не почитаетъ; не почитаетъ, яко не слушаетъ; не слушаетъ, яко заповѣдей Его не исполняетъ. Отъ сего всякій грѣхъ и беззаконіе послѣдуетъ. Аще бо и человѣка благодѣтеля любимъ, то и почитаемъ его; а когда почитаемъ, то и слушаемъ его; когда слушаемъ, то и волю его исполняемъ. Отсюду послѣдуетъ, что всякій грѣхъ, отъ произволенія и противу совѣсти содѣловаемый, есть знакъ неблагодарности къ Богу; яко, хотя и всякимъ, а наипаче таковымъ грѣхомъ Онъ прогнѣвляется, что съ любовію и благодарностію помѣститься не можетъ. Истинный бо благодѣтеля своего любитель, и къ нему благодарное сердце имѣющій, не хочетъ его прогнѣвать и оскорбить, яко любви истинной тяжко есть видѣть любимаго своего прогнѣваннаго и оскорбленнаго. "--- 4)~Неблагодарность человѣческая къ Богу и отъ того показуется, что онъ благими Божіими утѣшается, но имени Благодѣтеля своего хвалить, пѣть и прославлять не хочетъ. Благодарный бо къ благодѣтелю своему не токмо любитъ его и почитаетъ, но и хвалится имъ предъ другими, прославляетъ и превозноситъ его и благодѣяніе его. Тако, кто къ Богу благодарность имѣетъ, не скрываетъ въ сердцѣ своемъ благодѣянія Его, но устами исповѣдуетъ тое, и Самаго Его поетъ, хвалитъ и величаетъ. Отсюду толико благодарныхъ и похвальныхъ пѣсней произошло и написано въ книгахъ пророческихъ, во псалмахъ, въ новомъ Завѣтѣ и въ церковныхъ книгахъ, которыя святіи Божіи радостнымъ духомъ Богу, Благодѣтелю своему, пѣли, и намъ на тоежде оставили. Отсюду на многихъ мѣстахъ святаго Писанія, а наипаче во псалмахъ повелѣвается намъ хвалить и пѣть Господа: \textit{хвалите имя Господне, хвалите раби Господа}\footnote{Пс.~134,~1.}. И сіе"=то есть хрістіанская жертва, которую они по вся дни Богу приносить должны, якоже апостолъ глаголетъ: \textit{тѣмъ убо приносимъ жертву хваленія выну Богу, сирѣчь, плодъ устенъ исповѣдующихся имени Его}\footnote{Евр.~13,~15.}. Аще убо кто не творитъ сего, тотъ неблагодарное къ Богу имѣетъ сердце: напрасно созданіемъ Божіимъ называется, яко не хвалитъ Создателя Своего, Котораго \textit{все созданіе хвалитъ} и хвалить должно\footnote{Пс.~148,~1--13.}; напрасно благодѣянія Божія получаетъ, яко не прославляетъ Благодѣтеля своего; напрасно имя хрістіанское имѣетъ, яко не благодаритъ Хрісту, Искупителю и Спасителю своему и не прославляетъ Его. Сего ради обличилъ Хрістосъ неблагодарность прокаженныхъ, которые исцѣлились отъ проказы, но не пришли, кромѣ единаго, благодаренія Ему воздать и предъ Нимъ благодѣяніе Его исповѣдать: \textit{не десять ли очистишася? да девять гдѣ? како не обрѣтошася возвращшеся дати славу Богу}\footnote{Лук.~17,~17--18.}? Равнымъ образомъ и о хрістіанахъ сказаться можетъ, которые въ крещеніи святомъ очистилися отъ прокаженія грѣховнаго и прочіими благодѣяніями Божіими насыщаются, но не возвращаются Богу дать славу: не всѣ ли великое сіе благости Божіей дѣло на себѣ узнали, и прочая благая всегда дознаютъ? Да почто же не всѣ возвращаются съ благодарнымъ къ Богу сердцемъ и не даютъ славы Богу? \textit{Како не обрѣтошася возвращшеся дати славу Богу?} Напротивъ того, таможде пишется о благодарномъ самарянинѣ, что онъ, \textit{видѣвъ, яко исцѣлѣ, возвратися, со гласомъ веліимъ славя Бога, и паде ницъ при ногу Его} (Хріста), \textit{хвалу Ему воздая}\footnote{Лук.~17,~15 и 16.}. Тако и нынѣ благодарное сердце не молчитъ Божіяго благодѣянія, но возвращается къ Богу, отъ Котораго тое получаетъ, славитъ Его со гласомъ веліимъ и падаетъ ницъ предъ Нимъ, хвалу Ему воздая. "--- 5)~Отсюду видимъ, какъ ослѣпляетъ человѣка грѣхъ. Человѣка"=благодѣтеля своего, хотя, что ни получаетъ отъ него, Божіе есть, однакожъ любитъ, почитаетъ и прославляетъ человѣкъ; но Богу, Котораго благими \textit{живетъ, движется и есть}\footnote{Дѣян.~17,~28.}, не дѣлаетъ того. Всякое созданіе Божіе хвалитъ и прославляетъ Бога, Создателя своего. \textit{Небеса повѣдаютъ славу Божію}\footnote{Пс.~18,~2.}; солнце, луна и звѣзды свѣтомъ своимъ прославляютъ Бога; птицы на воздухѣ летаютъ, поютъ и прославляютъ Господа; земля съ плодами своими, и море съ живущими и движущимися въ немъ хвалитъ Господа; словомъ, все созданіе \textit{творитъ слово и повелѣніе Божіе}, и тако \textit{хвалитъ Господа своего}\footnote{48,~1--13.}. Человѣкъ, на котораго далеко большая, паче всего созданія, Божія изліялася благость, ради котораго небо и земля создана, ради котораго Самъ Богъ на землѣ явился и пожилъ, "--- человѣкъ, глаголю, разумная тварь, въ Божіихъ благодѣяніяхъ заключенный, не хочетъ хвалить и благодарить Бога, Господа, Создателя и Благодѣтеля своего. Такъ бѣдственно ослѣпляетъ грѣхъ человѣка! "--- 6)~Отсюду паки видимъ, коль великая благость Божія, что Онъ толико терпитъ неблагодарствію человѣческому, а не только терпитъ, но и благотворить не престаетъ врагамъ Своимъ: \textit{яко солнце Свое сіяетъ на злыя и благія, и дождитъ на праведныя и на неправедныя}\footnote{Матѳ.~5,~45.}. "--- 7)~Неблагодарность наша, хрістіанине, не Богу, но намъ вредитъ. Какъ бо солнце, хулится ли или хвалится, равно есть солнце, равно сіяетъ и лучи свои издаетъ, и тако ни хуленіемъ не убавляется, ни хваленіемъ не умножается свѣтъ его: тако и Богу, любимъ ли Его, почитаемъ, хвалимъ и благодаримъ Ему, славы не придаемъ, но намъ самимъ пользу дѣлаемъ. Такожде, когда не любимъ, не почитаемъ, не хвалимъ или хулимъ, славы у Него не отнимаемъ, но себѣ вредимъ. Божія бо слава ни умножиться, ни умалиться не можетъ. Ибо Богъ какъ въ свойствахъ Своихъ, такъ и въ славѣ Своей всесовершенъ и безконеченъ есть, и потому ни благодарностію нашею и хвалою славы Ему придать, ни неблагодарностію и хулою славы Его умалить не можемъ; но оною себе самихъ пользуемъ, сею же себе повреждаемъ и погубляемъ. Солнце всѣмъ свѣтитъ: кто хощетъ пользоватися свѣтомъ его, отворяетъ очи свои и пользуется свѣтомъ его; но солнцу оттуду ничего не придается: кто смежаетъ очи свои и свѣтомъ его не пользуется, себѣ, а не ему вредитъ. Тако благость Божія на всѣхъ изливается: кто ее чувствуетъ и благодаритъ Богу, самому ему благодарность его пользуетъ, а не Богу; кто не чувствуетъ и не благодаритъ Благодѣтелю, самому себѣ, а не Ему, вредъ дѣлаетъ. Таковаго бо человѣка, который къ Богу благодарности не показуетъ, лишаетъ Онъ праведнымъ судомъ благодати Своея; и тако человѣкъ, оставшися безъ помощи благодати Божіей, бываетъ какъ слѣпый, и ходитъ какъ во тьмѣ, грѣшитъ и не чувствуетъ, и отъ грѣха въ грѣхъ большій падаетъ и, какъ слѣпый рва, такъ онъ не видитъ погибели, въ которую имѣетъ впасти. Отсюду то бываетъ, что хрістіане неблагодарные бываютъ злѣйшіи паче самыхъ язычниковъ; яко язычники политичныи убѣгаютъ такихъ грѣховъ, на каковые дерзаютъ безстрашныи хрістіане, какъ"=то: разбой, хищеніе, лихоиманіе, неправду въ судахъ, лукавство, лесть, блудодѣяніе, и проч. Сей есть плодъ неблагодарности, плодъ нечувствія и погибели! Достойно и праведно презирается и оставляется отъ Бога тотъ, кто Его самъ презираетъ. Сего ради крайне берещися должно хрістіанамъ неблагодарности, яко они, паче всего созданія и паче прочіихъ человѣковъ, сподобилися и сподобляются милости Божіей.

\paragraph*{§\:401.} За что наипаче должно хрістіанамъ \textit{благодарить} Богу? "--- \textit{Отвѣтъ}: Хотя и за вся благая, однакожь наипаче отъ всего сердца благодарить Богу должны за сія: 1)~Что Сына Своего Единороднаго послалъ въ міръ на искупленіе и спасеніе рода человѣческаго, и Того предалъ на смерть. 2)~Что ихъ призвалъ изъ тьмы въ чудный Свой свѣтъ. \textit{Иже иногда не людіе, нынѣ же людіе Божіи, иже не помиловани, нынѣ же помиловани суть}\footnote{1~Петр.~2,~9--10.}. Сего ради глаголетъ апостолъ: \textit{благодаряще Бога и Отца, призвавшаго васъ въ причастіе наслѣдія святыхъ во свѣтѣ, Иже избави насъ отъ власти темныя, и престави въ царство Сына любве Своея, о Немже имамы избавленіе кровію Его, и оставленіе грѣховъ}\footnote{Кол.~1,~12--14.}. 3)~Что возжигаетъ въ нихъ свѣтильникъ вѣры елеемъ милости Своея, подаетъ имъ благодать Свою и отпущеніе грѣховъ. \textit{Благословенъ Богъ и Отецъ Господа нашего Іисуса Хріста, благословивый насъ всяцѣмъ благословеніемъ духовнымъ въ небесныхъ о Хрістѣ}, благодаритъ и хвалитъ Бога апостолъ Хрістовъ\footnote{Еф.~1,~3.}. 4)~Что подаетъ надежду вѣчнаго живота о Хрістѣ Іисусѣ, за что апостолъ Петръ благодаритъ и поетъ Бога, и хрістіанъ научаетъ: \textit{благословенъ Богъ и Отецъ Господа нашего Іисуса Хріста, Иже по мнозѣй Своей милости породилъ насъ во упованіе живо воскресеніемъ Іисусъ Хрістовымъ отъ мертвыхъ, въ наслѣдіе нетлѣнно и нескверно и неувядаемо, соблюдено на небесѣхъ васъ ради}\footnote{1~Петр.~1,~3--4.}. 5)~Что хранитъ отъ врага діавола, \textit{который, яко левъ рыкая, ходитъ, искій кого поглотити}\footnote{6,~8.}. 6)~Что предостерегаетъ отъ грѣха, въ который непрестанно бы падали, яко немощніи, аще бы милость Его не предваряла и не сохраняла, и проч. Аще убо кто хощетъ благодарнымъ быть Богу, Создателю и Искупителю своему, и не напрасно хрістіанское имя имѣть, и по окончаніи временнаго живота вѣчный Животъ получити, сія да помышляетъ и благодарнымъ сердцемъ да поетъ благость Божію. Вѣра бо хрістіанская сего отъ всякаго хрістіанина требуетъ.

\paragraph*{§\:402.} Что хрістіане въ мірѣ семъ за велико поставлять должны? "--- \textit{Отвѣтъ}: 1)~Сыны вѣка сего, или по плоти живущіи люди, за велико имѣютъ богатство, славу и роскошь міра сего, и сіе, какъ тройственнаго бога, почитаютъ, ищутъ и прилѣпляются тому. \textit{Идѣже бо сокровище ихъ, тамо и сердце ихъ есть}\footnote{Мѳ.~6,~21.}. Хрістіане сіе все за мало или паче за ничто почитать должны, яко они къ великимъ воистину вещамъ позваны, какъ сказано и ниже скажется; яко все сіе великое мимоходитъ: \textit{преходитъ бо міръ и похоть его}\footnote{1~Іоан.~2,~17.}, яко ихъ отъ того отводитъ апостолъ: \textit{не любите міра, ни яже въ мірѣ}\footnote{2,~15.}. При нашедшей же кончинѣ все сіе отступаетъ отъ человѣка, и такъ не остается ему ничего, и бываетъ онъ, какъ единъ отъ малыхъ, подлыхъ, нищихъ и убогихъ. И тогда самъ онъ познаетъ, что сіе великое есть весьма малое, или какъ ничто, и подобное тогда страждетъ, какъ и тотъ, который во снѣ нашелъ сокровище, но пробудившися ничего не имѣетъ; или какъ тотъ, который во снѣ отъ всѣхъ почитается какъ вельможа, но возбудившися видитъ себе въ прежнемъ подломъ состояніи и презрѣніи; или какъ тотъ, который во снѣ сладкое питіе піетъ, но воспрянувши, вмѣсто сладости, чувствуетъ жажду. Тако сіе великое воистину есть ничто! "--- 2)~Хрістіане должны почитать за велико "--- \textit{имѣть общеніе со Отцемъ и съ Сыномъ Его Іисусомъ Хрістомъ}\footnote{1,~3.}. Сіе воистину велико есть, и умомъ человѣческимъ непостижимо. За велико люди почитаютъ общеніе имѣть съ княземъ, царемъ, монархомъ и императоромъ земнымъ; но несравненно большее и высочайшее есть "--- съ Богомъ, Который есть Царь царствующихъ и Господь господствующихъ. Почему внимать должно хрістіанамъ и тому, что далѣе апостолъ написалъ: \textit{Богъ свѣтъ есть, и тьмы въ Немъ нѣсть ни единыя. Аще речемъ, яко общеніе имамы съ Нимъ, и во тьмѣ ходимъ, лжемъ и не творимъ истины}\footnote{1,~5--6.}. Аще убо за велико почитаютъ хрістіане съ Богомъ, Иже есть свѣтъ, общеніе имѣть, какъ и подлинно велико есть, и желаютъ того, "--- то должно имъ убѣгать тьмы, которая не иное что есть, какъ грѣхъ и діаволъ. \textit{Кое бо причастіе правдѣ къ беззаконію? или кое общеніе свѣту ко тьмѣ}\footnote{2~Кор.~6,~14.}? "--- 3)~Бога Отцемъ нарицать и имѣть и быть Его чадами, сіе паки воистину великое есть преимущество хрістіанское и умомъ непонятное. Кто бы всю міра сего славу и вся сокровища ни во что вмѣнить не захотѣлъ, чтобы сіе преимущество получить, аще здраво тое разсудитъ? Сіе преимущество представляетъ апостолъ вѣрнымъ и со удивленіемъ глаголетъ имъ: \textit{видите, какову любовь далъ есть Отецъ намъ, да чада Божія наречемся и есмы}\footnote{1~Іоан.~3,~1.}. Аще убо, хрістіанине, за велико сіе почитаемъ и желаемъ того, то должно намъ имѣть вѣру живую въ Господа Іисуса, Который \textit{даетъ область чадомъ Божіимъ быти, вѣрующимъ во имя Его}\footnote{1,~12.}, и тую вѣру показать отъ дѣлъ, какъ требуетъ апостолъ: \textit{покажи ми вѣру твою отъ дѣлъ твоихъ}\footnote{Іак.~2,~18.}, и засвидѣтельствовать богоподобными нравами, якоже учитъ апостолъ: \textit{бывайте подражатели Богу, якоже чада возлюбленная}\footnote{Еф.~5,~1.}. Онъ толико благъ, что \textit{солнце Свое сіяетъ на злыя и благія, и дождитъ на праведныя и на неправедныя}, "--- и намъ должно не токмо друговъ, но и \textit{враговъ нашихъ любить}\footnote{Мѳ.~5,~43--44.}. Онъ святъ есть, "--- и намъ должно святыми быть, якоже глаголетъ: \textit{святи будите, яко Азъ святъ есмь}\footnote{1~Петр.~1,~16.}. Онъ милосердъ есть, и мы да \textit{будемъ милосерди}, якоже глаголетъ Хрістосъ: \textit{будите милосерди, якоже и Отецъ вашъ милосердъ есть}\footnote{Лук.~6,~36.}. Онъ прощаетъ намъ грѣхи и мы другъ другу простимъ, якоже написано: \textit{бывайте другъ ко другу блази, милосерди, прощающе другъ другу, якоже и Богъ во Хрістѣ простилъ есть вамъ}\footnote{Еф.~4,~32.}, и проч. Таковая вѣра и таковые ея плоды когда будутъ въ насъ, то и Богъ Отецъ нашъ будетъ, и мы будемъ Его сыны и дщери, якоже глаголетъ Господь: \textit{и буду вамъ во Отца, и вы будете Мнѣ въ сыны и дщери}\footnote{2~Кор.~6,~18.}. \textit{Сицева убо имуще обѣтованія, о возлюбленніи, очистимъ себе отъ всякія скверны плоти и духа, творяще святыню въ страсѣ Божіи}, увѣщаваетъ насъ Апостолъ Хрістовъ\footnote{7,~1--2.}. "--- 4)~Быть не странными и пришельцами, но \textit{сожителями святымъ и присными Богу}\footnote{Еф.~2,~19.}; быть живымъ \textit{удомъ тѣла Хрістова}\footnote{Римл.~12,~5; 1~Кор.~12,~27.}; быть истиннымъ сыномъ святыя церкве: сіе за велико хрістіане должны почитать. За велико люди почитаютъ находиться въ фамиліи царя или князя земнаго, или жить во дворцѣ царскомъ и считаться близкимъ царю: какъ несравненно большее и высшее есть достоинство быть въ фамиліи небеснаго Царя, и жительство имѣть во святомъ домѣ Его, \textit{яже есть церковь Бога жива, столпъ и утвержденіе истины}\footnote{1~Тим.~3,~15.}. Аще убо за велико сіе почитаемъ, хрістіанине, и желаемъ того, то да покажемъ себе достойными сожитія святыхъ и присныхъ Богу. \textit{Приступихомъ бо къ Сіонстѣй горѣ и ко граду Бога живаго, Іерусалиму небесному, и тмамъ ангеловъ, торжеству, и церкви первородныхъ на небесѣхъ написанныхъ, и Судіи всѣхъ Богу, и духомъ праведникъ совершенныхъ, и къ Ходатаю завѣта новаго, Іисусу}, и проч.\footnote{Евр.~12,~22--24.} Въ семъ бо градѣ \textit{не проходитъ необрѣзанный} сердцемъ \textit{и нечистый}\footnote{Ис.~52,~1.}, но \textit{людіе, хранящіи правду и хранящіи истину, пріемлющіи истину и хранящіи миръ}\footnote{26,~2--3.}, \textit{родъ избранъ, царское священіе, языкъ святъ, люди обновленія}\footnote{1~Петр.~2,~9.}, которые \textit{и сами, яко каменіе живо, зиждутся въ храмъ духовенъ, святительство свято, возносити жертвы духовны благопріятны Богови Іисусомъ Хрістомъ}\footnote{ст.~5.}, и суть августѣйшій и великолѣпнѣйшій храмъ Бога живаго. \textit{Не вѣсте ли, яко храмъ Божій есте, и Духъ Божій живетъ въ васъ? Аще кто Божій храмъ растлитъ, растлитъ сего Богъ. Храмъ бо Божій святъ есть, иже есте вы}, глаголетъ апостолъ\footnote{1~Кор.~3,~16--17.}. "--- 5)~За великое и высокое у хрістіанъ почитаться должно таинственное тѣла и крове Хрістовой причащеніе. Но коль оно велико есть достойнымъ и познавшимъ истину, толь опасно и страшно недостойнымъ и неисправнымъ. \textit{Иже аще ястъ хлѣбъ сей, или піетъ чашу Господню недостойнѣ, повиненъ будетъ тѣлу и крови Господни. Да, искушаетъ же человѣкъ себе, и тако отъ хлѣба да ястъ, и отъ чаши да піетъ. Ядый бо и піяй недостойнѣ, судъ себѣ ястъ и піетъ, не разсуждая Тѣла Господня}\footnote{11,~28--29.}. "--- 6)~Молитва истинная. Молитва бо истинная есть бесѣда съ Богомъ: въ молитвѣ истинной человѣкъ къ Богу приходитъ и приближается, и предъ Богомъ предстоитъ и бесѣдуетъ съ Нимъ. Коль же сіе великое дѣло есть, кто не признаетъ? За велико люди имѣютъ съ земнымъ царемъ или княземъ, или инымъ какимъ высокимъ лицемъ бесѣдовать, и за честь тое себѣ поставляютъ: какъ хрістіанамъ за велико и за высочайшую честь не поставлять бесѣдованіе съ Богомъ, Царемъ небеснымъ, Царемъ царствующихъ и Господемъ господствующихъ? Столько сіе большее и честнѣйшее дѣло есть отъ онаго, сколько Царь небесный отстоитъ отъ земнаго, "--- отстоитъ же безконечно, яко Создатель отъ созданія Своего. Аще убо за велико сіе почитаемъ, человѣче, и хощемъ того, то да отступимъ отъ неправды, по ученію премудраго Павла: \textit{да отступитъ отъ неправды всякъ именуяй имя Господне}\footnote{2~Тим.~2,~19.}. Да отступимъ отъ неправды, грѣха и беззаконія, когда хощемъ приступать къ Богу неосужденно и съ пользою нашею. Ибо глаголетъ пророкъ: яко Богъ не хотяй беззаконія, Ты еси: \textit{не преселится къ Тебѣ лукавнуяй, ниже пребудутъ беззаконницы предъ очима Твоима: возненавидѣлъ еси вся дѣлающія беззаконіе}\footnote{Пс.~5,~5 и 6.}. Всякъ бо беззаконникъ мерзокъ есть предъ очима Божіима, и приносяй молитву свою, которая есть хрістіанская жертва, есть \textit{яко убиваяй пса}\footnote{Ис.~66,~3.}. Но \textit{зритъ Онъ милосердо токмо на кроткаго и молчаливаго и трепещущаго словесъ Его}\footnote{ст.~2.}. Разсуждай сіе, хрістіанине, который не хощеши и грѣха оставить и хощешь Богу молитися. "--- 7)~Хвала Божія. Ибо хвалити Бога есть ангельское дѣло. Ангели Божіи непрестанно хвалятъ и поютъ Бога и взываютъ: \textit{святъ, святъ, святъ Господь Саваоѳъ: исполнь вся земля славы Его}\footnote{6,~3.}. Ангеламъ убо уподобляется тотъ человѣкъ, который отъ сердца Бога хвалитъ и поетъ; и еще на земли вкушаетъ вѣчныя радости сладость. Ибо и въ будущемъ вѣкѣ не иное что избранніи Божіи будутъ дѣлать, какъ непрестанно Бога хвалить и пѣть, какъ видимъ въ Апокалипсисѣ. Высочайшая убо хрістіанская слава, честь и радость есть Бога хвалить, котораго ангели святіи хвалятъ. Какъ сіе за велико почиталъ Давидъ святый, Псалтирь его свидѣтельствуетъ; непрестанно желалъ хвалить Бога: \textit{благословлю Господа на всякое время, выну хвала Его во устѣхъ моихъ}\footnote{Пс.~33,~2.}, и другихъ къ тому созывалъ: \textit{возвеличите Господа со мною, и вознесемъ имя Его вкупѣ}\footnote{ст.~4.}. Но когда уста отверсти на похвалу Божію, хрістіанине, хощемъ, то должно и сердце имъ согласное имѣти. Ибо устная похвала, какъ и молитва, безъ сердца ничтоже есть, но есть только гласъ и шумъ словесный. Богъ бо не такъ на уста наши, какъ на сердце наше смотритъ, и слушаетъ только молитву и похвалу, отъ сердца происходящую. Сего ради глаголетъ апостолъ: \textit{воспѣвающе и поюще въ сердцахъ вашихъ Господеви}\footnote{Еф.~5,~19.}. Такожде, когда хощемъ ангеламъ святымъ подражать въ похвалѣ Божіей, то должны имъ подражать и житіемъ нашимъ, уклоняться отъ грѣха и творить волю Божію, якоже они творятъ, сколько можно человѣку въ немощной плоти. Иначе скажется и намъ: \textit{приближаются Мнѣ людіе сіи усты своими, и устнами чтутъ Мя: сердце же ихъ далече отстоитъ отъ Мене}\footnote{Матѳ.~15,~8.}. "--- 8)~Работа или служеніе Богу за велико у хрістіанъ почитаться должно. Люди за честь себѣ почитаютъ служить царю или князю, или высокому какому господину, нежели низкому и подлому человѣку; и доброму, праведному и милостивому, нежели злонравному, неправедному и немилостивому господину. Извѣстна сія истина и безпрекословна есть. Кто высшій паче Бога и кто лучшій, праведнѣйшій и милостивѣйшій паче Господа? Паче же единъ Онъ Вышній есть, единъ \textit{велій Господь, и Царь велій по всей земли; яко въ руцѣ Его вси концы земли, и высоты горъ Того суть; яко Того есть море, и Той сотвори е, и сушу руцѣ Его создастѣ}\footnote{Пс.~94,~3--5.}. Единъ Онъ благій, яко \textit{никтоже благъ, токмо единъ Богъ}\footnote{Мѳ.~19,~17.}. Единъ праведный и милостивый есть, и единъ Онъ Господь нашъ, Которому всякое колѣно покланяться должно. Сему работать увѣщаваетъ и пророкъ: \textit{работайте Господеви въ веселіи, внидите предъ Нимъ въ радости. Увѣдите, яко Господь той есть Богъ нашъ, Той сотвори насъ, а не мы: мы же людіе Его, и овцы пажити Его}, и прочая\footnote{Пс.~99,~2 и 3.}. Великая убо слава, честь и похвала хрістіанская есть работати единому Богу живому и безсмертному. Но когда хощемъ, хрістіанине, работати Богу, то не должно работати грѣху. Ибо Богу и грѣху работати невозможно, по ученію Хрістову: \textit{никтоже можетъ двѣма господинома работати}\footnote{Мѳ.~6,~24.}. А кто творитъ грѣхъ, тотъ работаетъ грѣху: яко \textit{всякъ, творяй грѣхъ рабъ есть грѣха}, какъ учитъ Хрістосъ\footnote{Іоан.~8,~34.}. Слѣдственно и Богу не можетъ работати. Что есть работати Богу, сказано выше, и ниже скажется. "--- 9)~Наконецъ велико есть побѣдить самого себе. Многіи побѣдили много тысящей людей, многіи покорили грады и государства; но себе самихъ побѣдить не могли, понеже съ самими собою не имѣли брани, безъ которой побѣда не бываетъ. Всякая бо побѣда безъ брани, а брань безъ враговъ не бываетъ. Брань между людьми бываетъ, когда одни возстаютъ, а другіе противятся: тако и внутрь у человѣка съ самимъ собою бываетъ брань, когда возстаетъ похоть и страсть, но онъ той противится; противится гордости и высокоумію смиреніемъ, гнѣву терпѣніемъ, злобѣ кротостію, зависти любовію, нечистотѣ цѣломудріемъ, и прочая. На брани между людьми тая сторона побѣждаетъ, которая гонитъ; а тая побѣждается, которая уступаетъ. Но въ брани у человѣка съ самимъ собою не тако бываетъ: онъ тогда побѣждаетъ, когда врагу своему уступаетъ, обидимъ не отмщеваетъ, ненавидимъ не ненавидитъ, лишаемъ не жалуется; \textit{любитъ}, по словеси Хрістову, \textit{враги своя, благословитъ кленущія себе, добро творитъ ненавидящимъ себе, и молится за творящихъ себѣ напасть и изгонящихъ себе}\footnote{Мѳ.~5,~44.}, \textit{не побѣжденъ бываетъ отъ зла, но побѣждаетъ благимъ злое}\footnote{Римл.~12,~21.}. Сія есть преславная побѣда, сіе есть благородіе, достоинство и преимущество хрістіанское "--- самого себе, то"=есть, растлѣнное свое естество, злонравіе и похоть свою злую побѣждать. Въ семъ состоитъ ихъ мужество, сила и храбрость. \textit{Лучше мужъ долготерпѣливъ паче крѣпкаго, удержаваяй же гнѣвъ паче вземлющаго градъ}\footnote{Прит.~16,~32.}. Сіе за велико почтимъ, хрістіанине, въ мірѣ семъ, а не честь, славу, богатство, роскошь и прочее, къ міру сему надлежащее. Сея побѣды пожелаемъ, поищемъ и имѣть съ помощію Божіею потщимся. Съ \textit{помощію}, глаголю, \textit{Божіею}, ибо безъ тоя не бываетъ. Самого бо себе побѣдить требуется сила не собственная и природная, но высшая, паче силы своея, то"=есть, Божія, которая изъ ветхаго новое, изъ злаго доброе, и изъ ничего все дѣлаетъ. За сію побѣду дается вѣнецъ не тлѣнный, но нетлѣнный, не въ царствіи земномъ, но небесномъ. (О сихъ пунктахъ и подобныхъ симъ смотри ниже \textit{статью послѣднюю о утѣшительныхъ плодахъ вѣры}).

Отъ сихъ пунктовъ видишь, хрістіанине, что тотъ хрістіанинъ, который грѣхъ творитъ, ни съ Богомъ общенія имѣть, ни Отцемъ Его нарицать, много паче имѣть, но паче во гнѣвѣ Его имѣется, ни внутрь церкви святой быть и сыномъ ея истиннымъ быть, ни таинъ Хрістовыхъ съ пользою души своея причащаться, ни молиться и хвалить Бога, ни работать Ему и прочей хрістіанской должности дѣлать не можетъ. Надобно убо всякому грѣхъ оставить, кто хощетъ истиннымъ, а не лицемѣрнымъ хрістіаниномъ быть, и спасеніе вѣчное получить.

\paragraph*{§\:403.} О чемъ хрістіане \textit{печалиться} должны? "--- \textit{Отвѣтъ}: 1)~Хрістіанская какъ радость, такъ и печаль иная должна быть, нежели сыновъ вѣка сего. Они не должны печалиться о томъ, что не имѣютъ въ мірѣ семъ благополучія, не имѣютъ богатства, славы, почитанія, что міръ ненавидитъ, гонитъ и озлобляетъ ихъ; сей печали они противитися, изгонять ее изъ сердца и не давать ей мѣста въ сердцѣ своемъ должны. Паче же радоватися о томъ, яко познаются отъ того не міра сего быти чада, но Божія, якоже глаголетъ Хрістосъ: \textit{аще отъ міра бысте были, міръ убо свое любилъ бы: якоже отъ міра нѣсте, но Азъ избрахъ вы отъ міра, сего ради ненавидитъ васъ міръ}\footnote{Іоан.~15,~19.}. Ибо кто печалится о томъ, что не имѣетъ отъ міра сего любви и почитанія, тотъ показуетъ о себѣ, что онъ любитъ міръ, земная мудрствуетъ, земнымъ прилѣпляется, что вѣрѣ хрістіанской противно. "--- 2)~Печаль хрістіанская истинная есть печалитися о томъ, что хрістіане высокое и небесное званіе имѣютъ, но того \textit{званія достойно ходити} не могутъ, немощію плоти воспящаеми; что Бога Отцемъ нарицаютъ, но Того такъ совершенно, какъ долгъ требуетъ, любить и угодить Тому не могутъ; отъ Него неизреченныя благодѣянія получаютъ, и надѣются наипаче въ будущемъ вѣкѣ получить, но Ему достойно возблагодарить не могутъ. \textit{Что} бо \textit{воздамы Господеви о всѣхъ, яже воздаде намъ}\footnote{Пс.~115,~3.}? Сія печаль имъ полезна и Богу благопріятна есть, яко \textit{жертва Богу духъ сокрушенъ, сердце сокрушенно и смиренно Богъ не уничижитъ}\footnote{50,~19.}. Таковая печаль нужна есть всякому хрістіанину, яко таковою печалію исправляется и обновляется растлѣнное естество. "--- 3)~Такую печаль, то"=есть, печаль по Бозѣ, имѣти должны вси тѣ, которыи согрѣшили предъ Господемъ, но обратившися каются. Печалитися, глаголю, должны, что Бога, Иже есть вѣчная любовь и благостыня, злыми дѣлами оскорбляли; и Котораго должны были почитать, беззаконнымъ житіемъ безчестили; Котораго должны были паче всего любить, не любили, Котораго должны были слушать, не слушали. Сія есть истинная по Бозѣ печаль, которую вѣрная душа имѣетъ не ради муки, слѣдующія за грѣхи, но ради того, что она Бога оскорбила. И сія"=то печаль есть печаль по Бозѣ, которая \textit{покаяніе нераскаянно во спасеніе содѣловаетъ}\footnote{2~Кор.~7,~10.}. За сію печаль похваляетъ апостолъ Коринѳянъ и радуется о ней: \textit{нынѣ радуюся, не яко скорбни бысте, но яко оскорбистеся въ покаяніе: оскорбѣсте бо по Бозѣ}\footnote{7,~9.}. Тако оскорбился Петръ апостолъ, когда отверглся Хріста, и \textit{изшедъ вонъ, плакася горько}\footnote{Матѳ.~26,~75.}. «Петръ, глаголетъ Златоустъ, егда отвержеся Хріста, не муки ради плакаше, но понеже любимаго отвержеся, еже всякія муки бѣ ему горчайше»\footnote{Бесѣда 5"=а на посл. къ Римл.}. "--- 4)~Такожде печаль имѣти хрістіане должны, когда ближняго согрѣшающаго слышатъ или видятъ; печалитися должны, а не осуждати, яко сами такомужде падежу подлежатъ: ибо таковый благаго Бога, Отца ихъ небеснаго, прогнѣвляетъ, что имъ, яко чадамъ Божіимъ, прискорбно быть должно, и что тако братъ ихъ погибаетъ. Такую печаль имѣлъ Давидъ святый: \textit{печаль}, рече, \textit{пріятъ мя отъ грѣшникъ, оставляющихъ законъ Твой}\footnote{Пс.~118,~53.}. Такую печаль имѣлъ Павелъ, вселенныя учитель, якоже изъ посланій его видно. Такую печаль, хотя всѣмъ хрістіанамъ, но наипаче пастырямъ, епископамъ и іереямъ должно имѣть, и болѣзновать о чадѣхъ своихъ духовныхъ, \textit{дондеже вообразится въ нихъ Хрістосъ}\footnote{Гал.~4,~19.}. "--- Какъ"=де мнѣ не печалитися, что я не имѣю богатства, славы и чести на земли, а прочіи имѣютъ? "--- \textit{Отвѣтъ}: 1)~Сіе многихъ немощныхъ смущаетъ и соблазняетъ, и, какъ пожаръ близкаго сосѣда, домы душъ ихъ запаляетъ и увѣщаваетъ ревновати тому, что въ другомъ видятъ. Обыкновенно бо люди тщатся о томъ, что въ другихъ видятъ; видятъ другихъ или богатыхъ, или славою и честію почтенныхъ, таковыми и сами быть хотятъ. И сія немощь плоти нашея есть, которая хощетъ и ищетъ въ мірѣ семъ прославитися и веселитися, и предъ другими въ благополучіи сего міра не остатися. 2)~Многіи находятся въ богатствѣ, чести и въ славѣ; но многіи и въ нищетѣ, подлости и презрѣніи живутъ: ты на сихъ обращай очи твои, и буди доволенъ тѣмъ, что имѣешь. 3)~Когда такую печаль имѣешь, то разсуди, находишися ли между хрістіанами, которые печалятся не о томъ, о чемъ ты печалишися. Кто бо о чемъ печалится, то знакъ есть, что онъ любитъ тое, чего не имѣя печалится. Хрістіанамъ же ничего, кромѣ Бога, и ради Бога человѣка любить не должно. 4)~Сія печаль тебѣ безполезна есть. Богатства бо, славы, чести и прочаго благополучія печалію тебѣ не сыскать, когда Богъ не подастъ. 5)~Когда печалишися, что находишися въ немощи, то тѣмъ самымъ немощь не умаляется, но умножается, какъ самъ сіе можешь чувствовать; и самая бо печаль есть немощь. И тако печаль въ немощи большую содѣлываетъ немощь, яко немощь съ помощію совокупляется. 6)~Когда лишился богатства, чести, славы, печалію того не можешь возвратить. Когда разлучился съ женою или отцемъ, или матерію, или братомъ, или другомъ, и о томъ печалишися, такожде того не возвратить печалію. О семъ ниже сказано будетъ пространнѣе. Видиши, что печаль міра сего безполезна есть; но едина только печаль по Бозѣ полезна, яко душеспасительна, яко душу отъ грѣховъ очищаетъ.

\paragraph*{§\:404.} Како хрістіане \textit{праздники} свои \textit{праздновать} должны? "--- \textit{Отвѣтъ}: 1)~Язычники, истиннаго Бога незнающіи, празднуютъ суетнымъ своимъ богамъ, или бѣсомъ, въ піянствахъ, нечистотахъ, похотѣхъ, козлогласованіяхъ, богомерзкихъ и блудническихъ пѣснѣхъ, въ кулачныхъ бояхъ, позорищахъ, конскихъ ристаніяхъ и въ прочіихъ безчинныхъ веселостяхъ, какъ читаемъ въ исторіяхъ о нихъ. Хрістіанамъ должно осмотрѣться: не празднуютъ ли и они, по подобію язычниковъ, праздниковъ, живому и страшному Богу посвященныхъ, Который въ тые дни преславныя содѣлалъ чудеса, и превеликая на родъ нашъ благодѣянія изліялъ? Когда тако празднуютъ, то безъ сумнѣнія празднуютъ не живому Богу, но бѣсомъ, врагамъ Его, хотя и имя истиннаго Бога нарицаютъ; и тако приносятъ жертву имъ, хотя того и не примѣчаютъ, которые при крещеніи обѣщалися работати Богу и Ему жертву правды и хвалы приносити. Коль беззаконное и страшное сіе дѣло есть, самимъ имъ оставляю на разсужденіе. Знатный былъ между языческими праздниками праздникъ, посвященный идолу \textit{Бахусу}, называемый \textit{Бакханалія}, а по россійскому нарѣчію можетъ называться \textit{масленица}. Въ сей неистовый праздникъ всякое піянство и всякая нечистота у нихъ разливалася. Сему безумію и неистовству языческому послѣдуя, премногіи хрістіане, по имени своему суще \textit{родъ избранъ, царское священіе, языкъ святъ, люди обновленія}\footnote{1~Петр.~2,~9.}, установили праздновать масленицу, въ которую всякое страстей и прихотей изліяніе совершаютъ, какъ всѣмъ о томъ извѣстно; котораго неистовства, яко всѣмъ извѣстнаго, не описую, но только воспоминаю хрістіанамъ таковымъ, что они званіе свое забыли, что они съ идолопоклонниками бѣсу жертвуютъ души свои, которыя Хрістосъ Сынъ Божій не сребромъ или златомъ, но \textit{честною Своею кровію искупилъ}\footnote{1,~18--19.}. Отъ таковыхъ празднованій, яко мерзостей языческихъ, на которыхъ бѣсовская воля совершается и имя Божіе хулится, хрістіанамъ должно удаляться, якоже апостолъ увѣщаваетъ: \textit{отложимъ дѣла темная, и облечемся во оружіе свѣта, яко во дни благообразно да ходимъ, не козлогласованіи и піянствы, не любодѣяніи и студодѣяніи}, и прочая\footnote{Римл.~12,~12--13.}. А которые тако празднуютъ, тѣмъ должно очувствоваться и покаяться и отстать отъ нихъ, да не съ язычниками во огнѣ вѣчномъ вмѣсто веселія и сладостей, вѣчно и безполезно восплачутся и услышатъ отвѣтъ: \textit{чадо! помяни, яко воспріялъ еси благая твоя въ животѣ твоемъ}\footnote{Лук.~16,~25.}. Тако, отъ языческаго неистовства удаляяся, должно хрістіанамъ праздники святые праздновать достойно званія своего, а именно: 2)~собираться въ церкви святыя на общую молитву, славословіе, хвалу Божію, слышаніе Божія слова, и собираться въ вечеръ, утро и полудне, то"=есть, на вечерню, утреню и литургію, но и въ домахъ того не оставлять. 3)~Поминать чудесныя и спасительныя дѣла Божія, въ тые дни милостивно во спасеніе наше сотворенныя; размышлять о нихъ, и за тыя радостнымъ духомъ Богу благодарить, которыя дѣла изрядно описаны и изъяснены въ церковныхъ стихахъ и пѣсняхъ, въ тые праздники поемыхъ. 4)~Отъ сихъ праздниковъ умъ и сердце свое вѣрою возводить къ вѣчному празднику, въ которомъ избранніи Божіи, упокоившися отъ трудовъ и болѣзней, въ семъ мірѣ подъятыхъ, будутъ непрестанно праздновать, и немолчно пѣть и хвалить благость Божію; и, возводя умъ, молиться человѣколюбцу Богу, да и самихъ тоя избранныхъ части сподобитъ. \textit{Помяни насъ Господи во благоволеніи людей Твоихъ; посѣти насъ спасеніемъ Твоимъ, видѣти во благости избранныя Твоя, возвеселитися въ веселіи языка Твоего, хвалитися съ достояніемъ Твоимъ}\footnote{Пс.~105,~4--5.}. 5)~Хотя и всегда, но наипаче въ тые дни нищихъ въ домы свои вводить и питать и давать милостыню: \textit{таковыми бо жертвами благоугождается Богъ}\footnote{Евр.~13,~16.}. 6)~О! коль свѣтелъ и благопріятенъ праздникъ былъ бы, когда бы хрістіане, отложивши всяку злобу и всяку лесть, и лицемѣріе, и зависть, и вся клеветы и прочіихъ прихотей отрицаяся, посѣщали другъ друга, и, лобызая лобзаніемъ святымъ, другъ другу повѣдали Божія чудеса, хвалилися о имени Его святомъ, бесѣдовали о вѣчной жизни и радости ея, и тѣмъ, какъ сладкою пищею, души свои питали: тогда почувствовали бы и здѣ сладости оныя частицу. Память бо и вѣрное разсужденіе о вѣчной жизни безъ утѣшенія не бываетъ душѣ, міра сего отрекшейся. 7)~Прекрасный праздникъ есть, и каковаго хрістіанское сердце всегда желаетъ, \textit{упраздняться отъ грѣха}: каковый праздникъ не токмо въ свѣтлые оные дни, но и всегда праздновать должны хрістіане; яко безъ сего никакій хрістіанскій праздникъ по"=хрістіански совершиться не можетъ. "--- \textit{(Смотри еще главу о почитаніи страстей Хрістовыхъ, книги сея)}.

\paragraph*{§\:405.} Како хрістіанамъ \textit{готовиться} должно \textit{къ исходу} отсюду на оный вѣкъ? "--- \textit{Отвѣтъ}: 1)~Отъ сего пункта начинаетъ всякъ или блаженную, или горестную и плачевную вѣчность, ибо преходитъ или въ вѣчный животъ, или въ вѣчную смерть\footnote{Лук.~16,~19--31.}. 2)~День и часъ, и образъ исхода никому неизвѣстенъ. Все сіе сокрылъ отъ насъ Божій Промыслъ. 3)~Сего ради хрістіанамъ должно въ семъ подражать вѣрнымъ и мудрымъ рабамъ, которые всегда ожидаютъ, когда ихъ позовутъ къ себѣ господа ихъ. Тако и хрістіане всегда должны готовыми быть, когда позоветъ ихъ Господь къ Себѣ. Зоветъ же всякаго Господь чрезъ смерть. Сего ради каковыми хощемъ быть при исходѣ, таковыми и прежде исхода и всегда должны быть, да не неготовыми явимся во время исхода. Никто при смерти не захощетъ грѣшить, чести искать, богатства собирать, и прочая земная мудрствовать; но паче всякъ думаетъ о грѣхахъ, жалѣетъ, что грѣшилъ, боится, чтобы не погибнуть. Сіе должно и нынѣ прежде смерти дѣлать, яко неизвѣстно, дождешися ли съ утра вечера, или съ вечера утра: не знаеши бо, когда позоветъ тебе Господь твой къ Себѣ. Того ради во всегдашнемъ покаяніии готовности хрістіанамъ должно быть, да не кто во вѣки погибнетъ. Великая вещь есть спасеніе вѣчное, и всего свѣта дражайшее; сего ради, паче всего сокровища и всего міра, и чрезъ все житіе искать его должно. \textit{Какая тебѣ польза съ того, что ты весь міръ сыщешь, а душу свою погубишь}\footnote{Мѳ.~16,~26.}? Какъ ничего не внесъ въ міръ, такъ ничего и не вынесешь отъ міра, кромѣ погибели, потерявши спасеніе, которое здѣ или обрѣтается или погубляется. О, како жалѣть, воздыхать, плакать и рыдать будешь, бѣдный грѣшникъ, потерявши спасеніе но безполезно! Пожелаешь тогда возвратиться въ міръ и нагимъ ходить, и отъ всѣхъ подлѣйшимъ быть, чтобы спасеніе, которое потерялъ, возвратить; но не дастся тебѣ. Дано было тебѣ время на тое, проповѣдывалося тебѣ Божіе слово, увѣщавали тебе пастыри и учители, обличала тебе и собственная твоя совѣсть; но ты не хотѣлъ исправиться. Случилось мнѣ слышать о единомъ, не послѣднемъ человѣкѣ, сіе, что онъ, находясь въ тяжкой болѣзни и при смерти, точно выговорилъ: «о, когда бы мнѣ возстать отъ сего одра, показалъ бы всякое исправленіе!» "--- но не возставши скончался. Пожелаешь и ты, бѣдный человѣче, отъ смертныхъ вратъ возвратитися и исправитися, но не получиши. Отъидеши, хотя и не хощеши, оставивши все мірское міру. Какъ вошелъ въ міръ, такъ и выйдешь отъ міра нагъ, и когда бы еще не обремененъ грѣхами! Ну"=жъ теперь поищи того, чего тогда безполезно будешь искать. Теперь ты еще здоровъ, еще не погибъ, еще можешь спастися: покажижъ исправленіе, котораго при смерти будешь желать, и тако съ надеждою спасенія отсюду отъидеши. Остави міръ, пока міръ не оставитъ тебе; свергни съ себе благодатію Божіею грѣхи твои, и тако работай Господу твоему свободно, и ожидай, когда Господь твой позоветъ тебе къ Себѣ, и явишися Ему одѣянъ одеждою спасенія. Чего тебѣ, какъ и себѣ, усердно желаю.

\paragraph*{§\:406.} Како хрістіанамъ должно къ \textit{причащенію Тѣла и Крови Хрістовой приступать? Отвѣтъ}: 1)~Апостолъ глаголетъ: да искушаетъ человѣкъ себе, и тако отъ хлѣба да ястъ, и отъ чаши да піетъ. \textit{Ядый бо и піяй недостойнѣ, судъ себѣ ястъ и піетъ, не разсуждая Тѣла Господня}\footnote{1~Кор.~11,~28--29.}. 2)~Многіи много читаютъ молитвъ, псалмовъ и акаѳистовъ, но сіе приготовленіе недовольно. Ибо молитва отъ неисправнаго и непокаявшагося сердца не пріемлется отъ Бога, какъ о томъ выше неоднократно сказано. 3)~Страшно приступать неочистившемуся истиннымъ покаяніемъ; страшно паки и приступившему и причастившемуся мірскими похотьми осквернятися: обое, и то и другое, страшно, какъ изъ апостольскаго слова и изъ разсужденія Тайны святыя видно. 4)~Сего ради должно и хотящему приступить разсуждать, кто и къ чему приступаетъ, "--- и причастившемуся, чего причастился. Откуду и прежде причастія нужно есть разсужденіе себе самаго и великаго того дара, и по причастіи нужно есть разсужденіе и память небеснаго того дара. \textit{Прежде} причастія нужно есть сердечное покаяніе, смиреніе, отложеніе злобы, гнѣва, прихотей плотскихъ, примиреніе съ ближнимъ, твердое предложеніе и изволеніе новаго и благочестиваго о Хрістѣ Іисусѣ житія. \textit{По} причастіи нужно есть показаніе исправленія, засвидѣтельствованіе любви къ Богу и ближнему, благодареніе, тщаніе усердное о новомъ, святомъ и непорочномъ житіи. Словомъ, прежде причастія истинное покаяніе и сокрушеніе сердечное нужно есть, по причастіи плоды покаянія, добрыя дѣла нужны суть, безъ которыхъ истинное покаяніе не можетъ быть. А оттуду слѣдственно заключается, что хрістіанамъ необходимо нужно есть житіе свое исправить и начать новое, богоугодное, дабы \textit{не въ судъ и во осужденіе было имъ причащеніе}. Хрістіанинъ бо безъ вѣры быть не можетъ, вѣра же безъ любви, а любовь безъ добрыхъ дѣлъ\footnote{1~Кор.~13,~1--8,~13.}. А тако хрістіанинъ остается не иное что, какъ язычникъ, хотя имя Божіе и Хрістово исповѣдуетъ. Коль же страшно таковому приступать къ Тайнѣ святой, всякъ видитъ. Отсюду бываетъ, что многіи хрістіане, причащающіися сей святой Тайнѣ, не въ лучшее, но въ горшее успѣваютъ, ослѣпляются болѣе, не просвѣщаются, ожесточаются, не исправляются, и тако отъ грѣха въ грѣхъ падаютъ, но грѣха своего не познаютъ, и грѣха за грѣхъ не поставляютъ, и дѣлаютъ тое, чего язычники политичніи не дѣлаютъ. Причина тому сія есть, что презираютъ Божію благодать, а тако и сами лишаются Божіей благодати, и \textit{предаются въ неискусенъ умъ, творити неподобная}\footnote{Рим.~1,~28.}. 5)~Сего ради должно себе грѣшникамъ исправить, да не въ судъ причащаются. Блудникамъ, прелюбодѣямъ и сквернителямъ перемѣнить сердце свое и возлюбить чистоту, яко къ \textit{Пречистому Агнцу} Хрісту приступаютъ. Гордымъ и высокоумнымъ отложить бѣсовскую гордость, и возлюбить смиреніе, яко къ \textit{Смиренному} приступаютъ Хрісту. Злобнымъ и огнемъ отмщенія дышущимъ злобу оставить, и возлюбить кротость и незлобіе, яко \textit{Кроткому и Незлобивому Агнцу} Божію приближаются. Ненавистливымъ и завистливымъ оставить ненависть и зависть, и возлюбить братолюбіе, яко \textit{Человѣколюбцу} Сыну Божію приближаются. Лестцамъ, лукавцамъ и обманщикамъ возненавидѣть діавольскую хитрость, и возлюбить простоту и истину, яко къ \textit{вѣчной и безлестной} приходятъ \textit{Истинѣ}. Клеветникамъ, злорѣчивымъ, хульникамъ исправить языкъ свой, яко устами воспріемлютъ \textit{Пречистое Тѣло Хрістово}. Сребролюбцамъ оставить мамону, и возлюбить Бога, яко къ Сыну Божію, \textit{Иже есть любы}, приступаютъ. Жестокосердечнымъ и немилостивымъ перемѣнить нравъ свой жестокій, и быть милостивыми тщаться, яко къ \textit{Милостивому и Милосердному} Господу приходятъ. Татямъ, хищникамъ, мздоимцамъ и лихоимцамъ оставить неправду сію, и возвратить или расточить похищенное, и поучаться правдѣ, яко къ Царю \textit{Праведному} приближаются. Богачамъ скупымъ оставить скупость и быть щедрыми, яко къ \textit{Щедрому} Господу приходятъ. Піяницамъ оставить піянство и начать трезвенное житіе, яко \textit{Распятому и Умершему} Хрісту пріобщаются. Словомъ, всякому должно исправить, очистить и обновить себе, да неосужденно приступитъ. «Должно, глаголетъ Василій Великій, очистить себе отъ всякія скверны плоти и духа приступающему къ Тѣлу и Крови Господней, да не въ судъ ястъ и піетъ, и память Того, Который за насъ умеръ и воскресъ, явно показать въ томъ, что онъ (приступающій) умеръ грѣху, и міру и самому себѣ; живетъ же Богу о Хрістѣ Іисусѣ Господѣ нашемъ»\footnote{Кн.~1"~я о Крещ. гл.~3"~я.}. (Что грѣху умереть и Богу жить, смотри § 388). Опасно убо и страшно приступать не оставляющему грѣха и совѣсть замаранную и раздраженную грѣхами имѣющему, \textit{да не въ судъ себѣ ястъ и піетъ}. Разсуждай сіе, хрістіанине, и берегись неисправнымъ приступать и, приступивши, паки мірскими похотьми окаляться. Ибо \textit{Господь нашъ огнь есть, недостойныя поядаяй}\footnote{Евр.~12,~29.}.

\subsection[Глава 10-я. Что можетъ и должно хрістіанина подвигнуть къ убѣжанію отъ грѣха, истинному покаянію, презрѣнію суетныхъ, и подвигу въ благочестіи?]{глава десятая.\\\bfseries Что можетъ и должно хрістіанина подвигнуть къ убѣжанію отъ грѣха, истинному покаянію, презрѣнію суетныхъ, и подвигу въ благочестіи?}

\begin{quotation}\textit{Яко чада свѣта ходите: плодъ бо духовный есть во всякой благостыни и правдѣ и истинѣ; искушающе, что есть благоугодно Богови}\footnote{Еф.~5,~8--10.}.\end{quotation}

\paragraph*{§\:407.} Вездѣсущіе Божіе и ясное видѣніе не токмо дѣлъ, но и тайныхъ помышленій и намѣреній нашихъ, можетъ насъ отвратить отъ грѣха, которое представляется намъ во псалмѣ 138"~мъ наипаче и на прочіихъ Писанія мѣстахъ. Не видимъ мы Бога сими очами, понеже и видѣть не можемъ, яко \textit{Богъ Духъ есть}\footnote{Іоан.~4,~24.}, никакимъ чувствамъ неподлежащій; но вѣра изъ святаго Писанія научаетъ насъ, что Онъ какъ вездѣ есть, такъ и съ нами неотлучно; и что мы ни дѣлаемъ, видитъ; и что ни помышляемъ, ясно проникаетъ; и что намѣреваемъ, замышляемъ, начинаемъ, хощемъ дѣлать, уже знаетъ, и прежде начинанія нашего зналъ; и что глаголемъ, слышитъ, и предъ Нимъ какъ день, такъ и нощь равно имѣется, и нѣтъ такого мѣста, гдѣ бы Его не было; слѣдственно всякое дѣло, и во дни и въ нощи творимое, и на всякомъ мѣстѣ совершаемое, не можетъ утаитися отъ Него. Злое ли убо дѣло творитъ человѣкъ, напр. блудодѣйствуетъ, убиваетъ, похищаетъ, крадетъ, и проч., или замышляетъ зло, "--- очи Его видятъ. Злословитъ, клевещетъ, осуждаетъ, кленетъ, лжетъ, обманываетъ брата своего, или имя Его святое хулитъ человѣкъ, "--- слышатъ святыя ушеса Его. И сколько на землѣ людей есть, тако всякаго назираетъ, и всякаго дѣло, начинаніе, умышленіе и намѣреніе видитъ, и слово слышитъ, и въ книзѣ Своей записываетъ, аки бы единаго только назиралъ, прочіихъ оставляя. Изрядно о семъ Августинъ глаголетъ къ Богу: «Тако хожденія моя и стези разсматриваешь, и во дни и въ нощи стрежешь мене бодренно, вси стези моя примѣчая, Зритель всегдашній: аки всю тварь небесную и земную забывши, мене только единаго смотришь, и ничего о прочіихъ небрежешь». И ниже: «Признаю, Господи, яко что ни дѣлаю, предъ Тобою дѣлаю; и тое, что дѣлаю, лучше Ты видишь, нежели я, который дѣлаю. Что бо ни дѣлаю, всегда, Ты равно всегда присутствуешь мнѣ, яко всегдашній Зритель всѣхъ моихъ помышленій, намѣреній, утѣшеній и дѣяній моихъ». И ниже: «Что намѣреваю въ дѣлѣ, что ни помышляю, и въ чемъ ни услаждаюся, Ты видишь, уши Твои слышатъ, очи Твои видятъ и назираютъ; означаешь, внимаешь, примѣчаешь и пишешь въ книгѣ Твоей, доброе ли или злое будетъ, да воздаси всѣмъ послѣжде за доброе "--- награжденіе, за злое "--- наказаніе, егда открыются книги»\footnote{Soliloquia (Бес. съ самимъ собою) гл.~14"~я.}. Оттуду видно: 1)~Что всякій грѣхъ, дѣломъ, или словомъ, или помысломъ творимый, безчестіе и досаду Богу дѣлаетъ. Якоже бо безчестіе и досада царскому величеству бываетъ, когда безчинникъ какій предъ нимъ безчинствуетъ, и не показуетъ предъ нимъ благоговѣинства, тако и предъ лицемъ Божіимъ и святѣйшими Его очами согрѣшающій безчинствуетъ, и потому безчестіе и досаду величеству Его дѣлаетъ. Коль сіе тяжко и страшно дѣлать есть, всякъ можетъ видѣть. "--- 2)~Хрістіанамъ должно отъ такаго безчинія берещися, и въ словахъ, дѣлахъ и помыслахъ благоговѣинство Богу показывать, да не оскорбятъ Бога и Создателя своего. Предъ царемъ земнымъ, или предъ господиномъ нашимъ, или властелиномъ, или отцемъ по плоти, стыдимся безчинія показывать; кольми паче сіе должно дѣлать предъ Отцемъ и Царемъ небеснымъ, Который \textit{есть Царь царей и Господь господей}\footnote{Апок.~17,~14.}. "--- 3)~Всякое безчинное дѣло, слово и помышленіе въ книзѣ Божіей записывается, и воздается безчиннику по дѣломъ его въ день суда: \textit{обличу тя, и представлю предъ лицемъ твоимъ грѣхи твоя}, грѣшнику глаголетъ Богъ\footnote{Пс.~49,~21.}, когда отъ того не отстанетъ и не покается грѣшникъ. 4)~И въ самомъ беззаконіи и безчиніи можетъ быть пораженъ человѣкъ праведнымъ судомъ Божіимъ, яко таковымъ образомъ не отдаетъ славы Богу. 5)~Беззаконнующимъ должно престать отъ злаго дѣла, да не дознаютъ на себѣ, и здѣсь и въ день суда, праведнаго суда Божія. 6)~Богу усердно благодарить, что имъ потерпѣлъ согрѣшенія ихъ, и не погубилъ ихъ, а впредь опасно поступать.

\paragraph*{§\:408.} Страданіе Хрістово сильно есть, и должно отвратить насъ отъ грѣха, которое двоякій позоръ душевнымъ нашимъ очесамъ представляетъ, то"=есть, непремѣняемую правду и милосердіе Божіе. 1)~Видимъ, что \textit{правда} Божія нарушиться не можетъ, но непремѣнно требуетъ своего удовлетворенія; видимъ, что грѣхъ безъ наказанія не бываетъ. Правда Божія требовала, чтобы намъ \textit{вѣчно казненными} быть, яко \textit{вѣчнаго и безконечнаго} Бога прогнѣвали. Кто могъ удовлетворить ей? Кто могъ цѣну и воздаяніе достойное дать ей? Кто могъ между безконечнымъ Богомъ прогнѣваннымъ и человѣкомъ прогнѣвавшимъ посредственникомъ, ходатаемъ и примирителемъ стать? Никто! ни ангелъ, ни иная какая тварь. Сталъ Іисусъ, Искупитель нашъ, Который глаголетъ: \textit{и воззрѣхъ, и не бѣ помощника; и помыслихъ, и никтоже заступи: и избави я мышца Моя}\footnote{Ис.~63,~5.}. И учинился \textit{Ходатаемъ Бога и человѣковъ}\footnote{1~Тѳ.~2,~5.}, и обратился праведный судъ Божій на Него: \textit{бысть по насъ клятва}\footnote{Гал.~3,~13.} единъ Благословенный. Судъ претерпѣлъ Неповинный за насъ повинныхъ; Своею казнію удовлетворилъ правдѣ Божіей за грѣхи наши: \textit{язвенъ бысть за грѣхи наша, и мученъ бысть за беззаконія наша}\footnote{Ис.~53,~5.}. Разсужденіе правды Божіей и праведнаго Его гнѣва противу грѣха, и жесточайшаго Хрістова мученія, которое за грѣхи наши претерпѣлъ, да устрашитъ насъ отъ грѣха. Когда Хрістосъ неповинный и единъ праведный за чужіе грѣхи такъ ужасно мученъ былъ, какъ мучени будутъ грѣшники въ гееннѣ за свои грѣхи! Когда симъ страшнымъ позорищемъ не умилятся люди и не покаются, какій уже гнѣвъ Божій возгорится на нихъ! Хрістосъ бо за всякаго человѣка пострадалъ и умеръ; но когда человѣкъ не хощетъ страданія Хрістова во спасеніе свое употребить, отстать отъ грѣха, покаяться, и тако спастися, тогда тое страданіе самое будетъ во обличеніе ему, что онъ толикую Божію благодать отвергнулъ, и такъ дознаетъ на себѣ правды Божіей силу и дѣйствіе. Тогда почувствуетъ бѣдный грѣшникъ, что есть грѣхъ, и коликая его горесть, чего нынѣ чувствовать не хотѣлъ. Безъ сумнѣнія заключается, что страшное и ужасное зло есть грѣхъ и тому послѣдующее вѣчное мученіе, отъ котораго ничимъ инымъ, какъ безцѣнною Сына Божія кровію, не могли мы избавиться. Отсюду почерпаетъ апостолъ святый поощреніе къ покаянію и предлагаетъ намъ: \textit{аще Отца называете нелицемѣрно судяща комуждо по дѣлу, со страхомъ житія вашего время жительствуйте, вѣдяще, яко не истлѣннымъ сребромъ или златомъ избавистеся отъ суетнаго вашего житія, отцы преданнаго, но честною кровію, яко Агнца непорочна и пречиста Хріста}\footnote{1~Петр.~1,~17--19.}. Разсуждай убо, хрістіанине, что есть грѣхъ: Хрістово страданіе научаетъ насъ, коль великая его тяжесть. Легко согрѣшить человѣкъ можетъ, но не легко свободиться отъ него можетъ, и безъ Хріста не можетъ, но слѣдуетъ тяжестію его погрузиться во дно адово и платить правдѣ Божіей безконечною казнію. Берегись же грѣха, который праведнаго Бога раздражаетъ. "--- 2)~Въ страданіи Хрістовомъ видимъ неизреченное Отца небеснаго къ намъ \textit{милосердіе}, Который, видя насъ въ погибели, и нехотя намъ въ той во вѣки быть, \textit{Сына Своего} ради нашего спасенія \textit{не пощадѣлъ, но за насъ предалъ Его}\footnote{Римл.~8,~32.}; о чемъ Самъ Сынъ Божій во утѣшеніе наше намъ глаголетъ: \textit{тако возлюби Богъ міръ, яко и Сына своего Единороднаго далъ есть, да всякъ, вѣруяй въ Онь, не погибнетъ, но имать животъ вѣчный}\footnote{Іоан.~3,~16.}. \textit{Иже во образѣ Божіи сый, не восхищеніемъ непщева быти равенъ Богу, но Себе умалилъ, зракъ раба пріимъ, въ подобіи человѣчестѣмъ бывъ, и образомъ обрѣтеся якоже человѣкъ; смирилъ Себе, послушливъ бывъ даже до смерти, смерти же крестныя}\footnote{Филип.~2,~6--8.}. О дивнаго благоволенія небеснаго Отца, Который, ради убогаго и погибшаго Своего созданія, Сына Своего не пощадѣлъ! О чуднаго послушанія Сына Божія, Который насъ ради, враговъ Своихъ, смирилъ Себе, послушливъ бывъ даже до смерти, смерти же крестныя! Его послушливый и намъ сладкій гласъ слышится во псалмѣ: \textit{се пріиду! въ главизнѣ книжнѣ писано есть о Мнѣ, еже сотворити волю Твою, Боже Мой, восхотѣхъ}\footnote{Пс.~39,~8--9; Евр.~10,~7.}; слышится во Евангеліи: \textit{снидохъ съ небесе, не да творю волю Мою, но волю пославшаго Мя Отца}\footnote{Іоан.~6,~38.}. Которое дѣло, какъ сладкое брашно, Себѣ вмѣнилъ Сынъ Божій: \textit{Мое брашно есть, да сотворю волю пославшаго Мя, и совершу дѣло Его}, то"=есть, дѣло спасенія нашего\footnote{4,~34.}. Смотри, хрістіанине, како восхощешь такъ милосерднаго Отца грѣхами оскорблять, и благоволеніе Его и послушаніе Сына Его, тебе ради учиненное, нераскаяннымъ житіемъ пренебрегать? Како будешь любить грѣхъ, за который Отецъ Сына Своего предалъ, и Сынъ Божій такъ горькое и тяжкое мученіе претерпѣлъ? Любишь же грѣхъ, яко не гнушаешься грѣхомъ; не гнушаешься, яко творишь его. Кто бо любитъ что, того не оставляетъ и держится; якоже, напротивъ того, чего не любимъ, ненавидимъ и гнушаемся, отъ того отвращаемся и удаляемся. Разсуждай убо Хрістово страданіе, въ которомъ видишь и правду Божію, грѣхи казнящую, и милосердіе Божіе, къ намъ явленное, и берегись грѣшить, да не подпадеши суду правды, и согрѣшивши покайся, да дѣломъ на себѣ дознаеши Божію милость; и отъ сего источника почерпай утѣшеніе, поминая апостольское утѣшительное слово: \textit{Иже Сына Своего не пощадѣ, но за насъ всѣхъ предалъ есть Его: како убо не и съ Нимъ вся намъ дарствуетъ}\footnote{Римл.~8,~32.}? Разсужденіе правды Божіей да отвратитъ тебе отъ грѣха; и разсужденіе милости Божіей да не попуститъ тебѣ отчаяться за содѣянный грѣхъ, но паче, воставши отъ грѣха, каяться. Поминай правду Божію во время искушенія ко грѣху; но милости Божіей не забывай во время искушенія за грѣхъ содѣянный. Тако разсужденіе страданія Хрістова учитъ насъ бояться грѣха, ненавидѣть его и удаляться отъ него, и за содѣянные грѣхи не отчаяваться, но возвратиться къ Отцу небесному, по подобію блуднаго, и просить отъ Него прощенія, и, получивши милость отъ Него, не отлучаться впредь отъ Него.

\paragraph*{§\:409.} Обѣщаніе, учиненное при крещеніи, должно насъ, хрістіанине, отвратить отъ грѣха и подвигнуть къ благочестію. Тогда между Богомъ и нами завѣтъ поставленъ: мы, отрекшися сатаны и всѣхъ дѣлъ его, обѣщались Богу \textit{служити преподобіемъ и правдою предъ Нимъ вся дни живота нашего}\footnote{Лук.~1,~75.}. Богъ насъ въ Свою высочайшую милость принялъ и въ той милости содержать насъ обѣщался. Богъ вѣренъ пребываетъ: работающихъ Себѣ содержитъ въ милости; должно убо и намъ обѣщаніе свое помнить и хранить, и вѣрными быть Господу Богу нашему, и работать Ему чистымъ сердцемъ. Аще же не сотворимъ сего, должными явимся предъ Нимъ; слѣдственно и Его милости лишимся, и въ праведномъ у Него гнѣвѣ будемъ, какъ и прежде крещенія были. Не сохраняетъ обѣщанія и вѣры своея Богу не токмо тотъ, который устами отрекается Его и обращается къ идоломъ, но и тотъ, который повелѣнія и заповѣдей Его отрицается и не работаетъ Ему, кто работаетъ грѣху, какъ выше сказано. Сего ради не токмо отрицанія устнаго, но и сердечнаго, которое чрезъ грѣхъ и беззаконіе бываетъ, берещися должно намъ. Не сохранившимъ же обѣщанія и отлучившимся отъ Бога не должно медлить въ пагубномъ томъ отлученіи, но, по подобію блуднаго сына, возвратитися къ небесному Отцу и со смиреніемъ просить прощенія у Него, и впредь отъ дому Его святаго, то"=есть, святой Его церкви, не отлучать себе беззаконнымъ житіемъ, да не праведному Его суду и гнѣву вѣчному Его подпадутъ.

\paragraph*{§\:410.} Высокое и великолѣпное именованіе святыя церкви, въ которой находятся хрістіане, поощряетъ ихъ уклоняться отъ грѣха и прилѣжать благочестію. Церковь святая нарицается отъ апостола \textit{домъ Бога живаго}\footnote{1~Тимоѳ.~3,~15.}. Сей святый и великолѣпный домъ имѣетъ основаніемъ Самого Хріста, по ученію апостола: \textit{основанія инаго никтоже можетъ положити паче лежащаго, еже есть Іисусъ Хрістосъ}\footnote{1~Кор.~3,~11.}. Сей домъ создали на лицѣ всея земли, Господу поспѣшествующу, духовныи архитекторы, апостоли святіи; очищенъ и освященъ онъ кровію Единороднаго Сына Божія\footnote{Еф.~5,~25--26.}; входятъ въ него вѣрою и крещеніемъ живущіи въ немъ. Въ сей домъ собрались и собираются языцы мнози, глаголюще: \textit{пріидите, и взыдемъ на гору Господню, и въ домъ Бога Iаковля, и возвѣститъ намъ путь Свой, и пойдемъ по нему}\footnote{Ис.~2,~3.}. Называется и \textit{градъ Божій}, о которомъ воспѣлъ пророкъ: \textit{основанія его на горахъ святыхъ}\footnote{Пс.~86,~1 и проч. до конца.}. \textit{Богъ основа его въ вѣкъ}\footnote{Пс.~47,~9.}; \textit{Богъ посредѣ его, и не подвижится: поможетъ ему Богъ утро заутра}\footnote{45,~6.}. Сего града граждане, и дому домашніи суть хрістіане, \textit{сожители святыхъ и присніи Богу}\footnote{Еф.~2,~19.}. Въ семъ градѣ нѣтъ мѣста необрѣзанному сердцемъ и нечистому\footnote{Ис.~52,~1.}, но ходятъ въ немъ \textit{людіе, хранящіи правду и хранящіи истину, пріемлющіи истину и хранящіи миръ}\footnote{26,~2--3.}.

\paragraph*{§\:411.} Можетъ и должно хрістіанъ подвигнуть къ благочестивому о Хрістѣ Іисусѣ житію и содержанію себе въ томъ высокое и небесное ихъ званіе: яко \textit{призваны отъ области сатанины въ царство Божіе, которое нѣсть брашно и питіе, но правда и миръ и радость о Дусѣ Святѣ}\footnote{Дѣян.~26,~18; Римл.~14,~17.}; призваны въ \textit{родъ избранъ, царское священіе, языкъ святъ, люди обновленія, яко да добродѣтели возвѣстятъ изъ тьмы ихъ призвавшаго въ чудный Его свѣтъ}\footnote{1~Петр.~2,~9.}. \textit{Не призва бо ихъ Богъ на нечистоту, но во святость}\footnote{1~Сол.~4,~7.}, да \textit{духомъ ходятъ, и похоти плотскія не совершаютъ}\footnote{Гал.~5,~16.}; да \textit{яко пришельцы и странники, огребаются отъ плотскихъ похотей, яже воюютъ на душу}\footnote{1~Петр.~2,~11.}; но \textit{да будутъ неповинни и цѣли, и чада Божія непорочна, посредѣ рода строптива и развращенна}\footnote{Филип.~2,~15.}, да \textit{якоже чада свѣта ходятъ, искушающе, что есть благоугодно Богови, и не пріобщаются къ дѣломъ неплоднымъ тьмы, паче же и обличаютъ}\footnote{Еф.~5,~8,~10 и 11.}. Словомъ, \textit{къ общенію со Отцемъ и съ Сыномъ Его Іисусомъ Хрістомъ} позваны\footnote{1~Іоан.~1,~3.}. \textit{Богъ же свѣтъ есть, и тьмы въ Немъ нѣсть ни единыя. Аще речемъ, яко общеніе имамы съ Нимъ, и во тьмѣ ходимъ, лжемъ и не творимъ истины}\footnote{5 и 6.}. Сего ради, хрістіанине, когда хощемъ блаженное сіе общеніе имѣти, и между истинными хрістіанами быть, а не между лицемѣрами счисляться, \textit{очистимъ себе отъ всякія скверны плоти и духа, творяще святыню въ страсѣ Божіи}\footnote{2~Кор.~7,~1.}. Откуду Павелъ святый, званіе сіе намъ представляя, глаголетъ: \textit{достойно ходите званія, въ неже звани бысте}\footnote{Еф.~4,~1.}.

\paragraph*{§\:412.} Причащеніе пречистыхъ и животворящихъ Таинъ Тѣла и Крови Хрістовыя должно подвигнуть насъ, хрістіанине, къ удаленію отъ грѣха и истинному покаянію. Како \textit{Пречистое} Тѣло Хрістово дерзнешь принять въ руки твои (къ пріемлющимъ въ руки рѣчь здѣсь), которыя хищеніемъ, грабленіемъ, мздоиманіемъ, лихоимствомъ, біеніемъ, нечистымъ прикосновеніемъ и прочіимъ грѣховнымъ каломъ оскверняешь? Како пріимешь во уста Тѣло и Кровь Его \textit{святую}, въ тыя уста, которыя злорѣчіемъ, сквернословіемъ, кощунствомъ, буесловіемъ, клеветою, осужденіемъ, лестію, лукавствомъ, лжею, язвительнымъ укореніемъ, руганіемъ, поношеніемъ и прочіимъ смрадомъ не престаеши наполнять? Како пріимешь Хріста въ сердце твое, которое злобою, лукавствомъ, ненавистію, нечистою похотію, сребролюбіемъ и лихоиманіемъ и прочіимъ зломъ преисполнено? Надобно таковому опасаться, чтобы не приличествовало ему Хрістово слово, сказанное единому книжнику, который и по Хрістѣ хотѣлъ ходить, и міра не оставить: \textit{лисы язвины имутъ, птицы небесныя гнѣзда: Сынъ же человѣческій не имать гдѣ главы подклонити}\footnote{Мѳ.~8,~19 и 20.}. Лесть, лукавство, лицемѣріе, сребролюбіе, лихоимство и всякое злое похотѣніе суть какъ \textit{лисы}, язвины своя въ сердцѣ человѣческомъ имѣющіи, и какъ \textit{птицы} гнѣздящіися: Сынъ убо человѣческій не имѣетъ гдѣ главы подклонити; но входитъ въ кроткія, смиренныя, сокрушенныя печалію за грѣхи сердца, и благодать Свою съ Собою вноситъ. \textit{Се стою при дверехъ и толку: аще кто услышитъ гласъ мой и отверзетъ двери, вниду къ нему, и вечеряю съ нимъ и той со Мною}\footnote{Ап.~3,~20.}. Стоитъ сей святый Гость, Который насъ ради странствовати на землѣ благоволилъ, стоитъ и ударяетъ въ двери сердецъ нашихъ, и хощетъ къ намъ внити, за которыхъ святую кровь Свою проліялъ. Но кто отверзаетъ Ему двери въ домъ свой? кто слышитъ гласъ Его?.. Кто же слышитъ гласъ Его?.. Неотмѣнно тотъ, кто слову Его повинуется, престаетъ отъ грѣховъ, кается, и жалѣетъ сердечно за содѣянные грѣхи, и содержитъ себе въ истинномъ покаяніи, якоже выше о семъ глаголетъ Хрістосъ: \textit{ревнуй и покайся}\footnote{Ст.~19.}. Къ такому Онъ входитъ и вечеряетъ съ нимъ, то"=есть утѣшаетъ его истиннымъ и живымъ утѣшеніемъ. Тоежъ и на другомъ мѣстѣ глаголетъ: \textit{аще кто любитъ Мя, слово Мое соблюдетъ: и Отецъ Мой возлюбитъ его, и къ нему пріидемъ, и обитель у него сотворимъ}\footnote{Іоан.~14,~23.}. А гдѣ Отецъ и Сынъ, тамо и Святый Духъ; гдѣ Богъ живетъ, тамо царствіе Божіе, тамо утѣшеніе и радость живая и истинная. Божіе бо благодатное присутствіе \textit{безъ утѣшенія и радости не бываетъ}\footnote{Римл.~14,~17.}. Надобно убо, хрістіанине, престать отъ грѣховъ, покаяться и начать новое о Хрістѣ житіе, когда хощемъ съ пользою нашею Таинъ Хрістовыхъ причащаться.

\paragraph*{§\:413.} Молитва да подвигнетъ насъ, хрістіанине, къ уклоненію отъ грѣха и творенію истиннаго покаянія. Понеже 1)~въ молитвѣ истинные хрістіане приступаютъ къ Богу и бесѣдуютъ съ Нимъ; 2)~возводятъ умъ и сердце свое къ Нему; 3)~воздѣваютъ руки и очи своя къ Нему; 4)~отверзаютъ уста своя и приносятъ Ему благодареніе, пѣніе и хвалу, якоже о томъ сказано нѣсколько и ниже скажется. Отъ сего послѣдуетъ, что должно намъ непремѣнно престать отъ грѣха, исправить себе и истинно каяться, когда хощемъ съ пользою нашею молиться. Како бо приступишь къ Богу и будешь просить Его: \textit{остави мнѣ грѣхи моя, Господи}, а самъ ихъ не оставляешь? Како будешь нарицать Его: \textit{Господи мой}, а самъ мамонѣ работаешь, и рабъ есть грѣха? \textit{яко всякъ, творящій грѣхъ, рабъ есть грѣха}\footnote{Іоан.~8,~34.}. Како скажешь: \textit{Царю мой и Боже мой}, а самъ попущаешь грѣху надъ тобою царствовати? Како скажешь: \textit{услыши мя Господи, и помилуй}, а самъ не хощешь Его слушать и прогнѣвлять не престаеши? Како можешь къ Нему звати: \textit{на Тя, Господи, уповахъ}, а самъ уповаешь на князей, своихъ защитниковъ, на силу, хитрость, санъ свой, на сребро, злато и прочее созданіе, яко въ нуждѣ къ тѣмъ ради помощи и защищенія прибѣгаешь? Како можешь пѣть: \textit{прилпѣ душа моя по Тебѣ}\footnote{Пс.~62,~9.}, или: \textit{мнѣ же прилѣплятися Богови благо есть}\footnote{72,~28.}, а самъ прилѣпляешися міру? Како можешь возвести сердце твое къ Нему, которое исполнено есть злобою и прочіими злыми похотьми? Како воздвигнешь къ Нему руки твоя, которыя оскверняешь хищеніемъ, грабленіемъ, лихоиманіемъ, біеніемъ, нечистымъ прикосновеніемъ, и проч.? Како отверзеши уста твои на славословіе и хвалу святаго имени Его, "--- уста, которыя оскверняешь злословіемъ, сквернословіемъ, лестію, лжею, клеветою, осужденіемъ и прочіимъ каломъ грѣховнымъ? Словомъ, вся Псалтирь и всѣ молитвы, въ церковныхъ книгахъ написанныя, не приличествуютъ тому грѣшнику, который грѣховъ оставить и творить истиннаго покаянія не хощетъ. Прочитай самъ, хрістіанине, со вниманіемъ, и уразумѣешь истину. \textit{Да отступитъ убо отъ неправды всякъ, именуяй имя Господне}\footnote{2~Тѳ.~2,~19.}. И чрезъ пророка глаголетъ Богъ: \textit{егда прострете руки ваша ко Мнѣ, отвращу очи Мои отъ васъ: и аще умножите моленіе, не услышу васъ; руки бо ваша исполнены крове}\footnote{Ис.~1,~15.}. Должно убо оставить грѣхъ, покаяться и берещися грѣха, который насъ разлучаетъ отъ Бога, и предъ Богомъ мерзкими и гнусными насъ дѣлаетъ, и молитву нашу безполезною творитъ, да не во вѣки отъ Бога и избраннаго Его стада отлучимся.

\paragraph*{§\:414.} Безконечная и неизреченная благость Божія да подвигнетъ насъ, хрістіанине, къ уклоненію отъ грѣха и покаянію, о которой Павелъ святый глаголетъ тако: \textit{о человѣче! или о богатствѣ благости Его и кротости, и долготерпѣніи нерадиши, не вѣдый, яко благость Божія на покаяніе тя ведетъ}\footnote{Рим.~2,~4.}? Благость Божія есть, что мы еще не погибли, что мы еще живемъ въ свѣтѣ и можемъ благодатію Его спастися, яко такъ \textit{благость Его на покаяніе насъ ведетъ}. Отъ сей благости Его происходитъ, что Онъ такъ любезно и благопріятно насъ на покаяніе зоветъ, и тако благость Свою на насъ изліять, и вѣчнаго блаженства участниками насъ сотворить хощетъ: \textit{обратися ко мнѣ, доме Израилевъ, рече Господь: и не утвержду лица Моего на васъ, яко милостивъ Азъ есмь, рече Господь, и не прогнѣваюся на вы во вѣки}\footnote{Іер.~3,~12.}; и паки: \textit{обратися, Израилю, ко Господу Богу твоему: зане изнемоглъ еси въ неправдахъ твоихъ}\footnote{Осіи 14,~2.}; и паки: \textit{обратитеся ко Мнѣ, глаголетъ Господь силъ, и обращуся къ вамъ}\footnote{Зах.~1,~3.}. И чрезъ апостоловъ, посланниковъ Своихъ, молитъ насъ, да примиримся съ Нимъ, да не во вѣки будемъ чувствовать на себѣ гнѣвъ Его праведный: \textit{по Хрістѣ молимъ, яко Богу молящу нами: молимъ по Хрістѣ, примиритеся съ Богомъ}\footnote{2~Кор.~5,~20.}. И Самъ Хрістосъ пресладкими Своими словами привлекаетъ къ Себѣ: \textit{пріидите ко Мнѣ вси труждающіися и обремененніи, и Азъ упокою вы}\footnote{Мѳ.~11,~28.}. И Духъ Святый увѣщаваетъ и глаголетъ: \textit{днесь аще гласъ Его услышите, не ожесточите сердецъ вашихъ}\footnote{Евр.~3,~7,~8 и 15; Пс.~94,~7--8.}. Не точію гласомъ словъ пророческихъ и апостольскихъ, но и самымъ долготерпѣніемъ Своимъ, что \textit{долготерпитъ къ намъ, не хотя, да кто погибнетъ, но да вси въ покаяніе пріидутъ}\footnote{2~Петр.~3,~9.}, призываетъ насъ преблагій Богъ на покаяніе, якоже апостолъ глаголетъ: \textit{благость Божія на покаяніе тя ведетъ}. Сколько есть идолопоклонниковъ, хульниковъ, разбойниковъ, грабителей, насильниковъ, хищниковъ, льстецевъ, лукавцевъ, блудниковъ, прелюбодѣевъ, сквернителей и прочіихъ беззаконниковъ; но Богъ ихъ не погубляетъ, хотя и враги Божіи суть. Что сіе есть, какъ, что \textit{благость Божія на покаяніе ихъ ведетъ}? И не токмо терпитъ и не погубляетъ ихъ, но и хранитъ ихъ отъ навѣта вражія, хотя они того и не чувствуютъ. Не попустилъ бы имъ и минуты жить \textit{супостатъ нашъ діаволъ}, который \textit{яко левъ рыкая, ходитъ, искій кого поглотити}\footnote{1~Петр.~5,~8.}, но тотчасъ восхитилъ бы души ихъ, и низвергнулъ бы во адъ; но благость Божія не даетъ ему того, яко \textit{хотѣніемъ не хощетъ Богъ смерти грѣшника, но обращенія ожидаетъ}\footnote{Іез.~33,~11; 18,~23 и 32.}. О непостижимыя ко грѣшникамъ благости Божія! Грѣшникъ оставляетъ Бога, но Богъ его не оставляетъ. Грѣшникъ, отступивъ отъ Бога, Творца и Промыслителя своего, присталъ ко врагу Божію, діаволу; но Богъ не велитъ ему погубить грѣшника, врага Своего. Кто сей благости Божіей довольно удивиться можетъ, не токмо словомъ изобразить! Тако \textit{благость Божія на покаяніе грѣшника ведетъ}. Не точію долготерпѣніемъ, но и самыми тѣми благими, которыя изливаетъ повседневно на всякую плоть, ведетъ насъ Богъ на покаяніе. Якоже бо чадолюбивый отецъ, хотячи сына своего участникомъ сотворити наслѣдія, не токмо словами наказуетъ его къ чинному и постоянному житію, но и любовію и ласканіями привлекаетъ къ тому, дабы всякимъ образомъ исправнымъ его сдѣлать, и тако сотворить его наслѣдникомъ своимъ: тако человѣколюбивый и милосердый Отецъ нашъ небесный, хотячи насъ вѣчнаго блаженства и небеснаго царствія участниками сотворити, не токмо словами Своими, но и благими чувственными привлекаетъ къ истинному покаянію и благочестію. Благая бо Божія, напр. солнце, луна, звѣзды, воздухъ, различные плоды земные, рыбы, птицы, скоты и прочая суть какъ ручьи, отъ источника проистекающіи. Якоже бо ручьи ведутъ и указуютъ намъ источникъ, да отъ него почерпаемъ воду и піемъ, тако созданная Божія благая ведутъ насъ къ Богу, Источнику благихъ, и указуютъ на Него, да Его любимъ и вкусимъ, \textit{коль благъ Господь}, Который сія сотворилъ. Солнце, луна и звѣзды указуютъ, и безъ гласа проповѣдуютъ намъ: смотрите, \textit{коль благъ Господь}, Который насъ во свѣтъ вамъ создалъ. Плоды земные, птицы, рыбы и скоты указуютъ и глаголютъ: \textit{вкусите и видите, яко благъ Господь}\footnote{Пс.~33,~9.}, Который насъ въ пищу вамъ сотворилъ, и проч. Но какъ Отецъ, видя сына неисправнаго, понуждается грозить ему наказаніемъ и отлученіемъ отъ наслѣдія, и самымъ дѣломъ налагаетъ ему раны, да исправитъ его: тако благоутробный Отецъ небесный поступаетъ съ сынами человѣческими: грозитъ имъ наказаніемъ временнымъ и вѣчнымъ: \textit{Азъ предамъ васъ подъ мечъ, вси закланіемъ падете: яко звахъ васъ, и не послушасте: глаголахъ, и преслушасте и сотвористе лукавое предо Мною; и яже не хотѣхъ, избрасте. Сего ради тако глаголетъ Господь: се работающіи Ми ясти будутъ, вы же взалчете; се работающіи Ми пити будутъ, вы же возжаждете; се работающіи Ми возрадуются, вы же посрамитеся; се работающіи Ми возвеселятся въ веселіи сердца, вы же возопіете въ болѣзни сердца вашего, и отъ сокрушенія духа восплачетеся}\footnote{Ис.~65,~12--14.}. И Хрістосъ глаголетъ: \textit{аще не покаетеся, вси такожде погибнете}\footnote{Лук.~12,~3.}. Грозитъ намъ милосердый Богъ наказаніемъ, да не постигнетъ насъ наказаніе; грозитъ геенною, да не впадемъ въ геенну, наказуетъ насъ, да исправимся. И сіе"=то значитъ напасти и бѣды, на насъ посылаемыя, которыя, какъ посланники, молча вопіютъ и безъ гласа увѣщаваютъ, да сотворимъ плоды достойны покаянія. Тако \textit{благость Божія} не токмо словами пророческими и апостольскими, но и долготерпѣніемъ, изліяніемъ временныхъ благихъ и насыланіемъ временныхъ злыхъ, то"=есть бѣдъ и напастей, \textit{ведетъ насъ на покаяніе}, призываетъ и ожидаетъ. А якоже любезно призываетъ и долготерпѣливо ожидаетъ грѣшника на покаяніе, тако любезно и благопріятно обратившагося его пріемлетъ, якоже въ притчѣ блуднаго сына изобразилъ Хрістосъ: \textit{еще далече ему} (блудному сыну) \textit{сущу, узрѣ его отецъ его, и милъ ему бысть, и, текъ нападе на выю его, и облобыза его}\footnote{Лук.~15,~20.}. Тако Отецъ небесный зритъ на грѣшника, возвратившагося отъ страны дальней, въ которую отъ Него удалился; зритъ приходящаго къ Нему, отъ Котораго беззаконнымъ житіемъ отлучился; зритъ любезно, и \textit{милъ Ему} бываетъ обратившійся и приходящій, и \textit{объемлетъ его} объятіями благости Своея, и \textit{лобызаетъ} лобзаніемъ Божественныя любви Своея. О премилосердыхъ очесъ небеснаго Отца, \textit{мило} зрящихъ на обращеніе грѣшника! О благоутробныхъ объятій отеческихъ, \textit{объемлющихъ} грѣшника! О святѣйшаго и сладчайшаго \textit{лобзанія}, котораго сподобляется грѣшникъ! Не гнушается и не отвращается грѣшника, грѣхами, какъ мерзкимъ рубищемъ, одѣтаго безконечная Святыня; не отгоняетъ отъ Себе пришедшаго, котораго отлучившагося не оставлялъ; пріемлетъ съ любовію, котораго обращенія усердно желалъ; не дѣлаетъ никакого выговора ему, что самовольно отлучился отъ Него, что богатство данное ему расточилъ. Нѣтъ ничего того! Но одѣвается въ \textit{первую одежду}, и пріемлетъ \textit{перстень на руку свою, и сапоги на нозѣ свои}, и тако причисляется къ святой фамиліи Его\footnote{Лук.~15,~22.}. Слышится отеческій гласъ радости: \textit{сынъ мой сей мертвъ бѣ, и оживе; изгиблъ бѣ, и обрѣтеся}\footnote{24 и 32.}. Веселятся о томъ небесніи Его ангели: \textit{яко радость бываетъ предъ ангелами Божіими о единомъ грѣшницѣ кающемся}\footnote{15,~10.}. Кто не почудится сей толикой благости Божіей, которой и во отвращеніи и отлученіи и обращеніи сподобляется грѣшникъ? Каменное воистину человѣческое сердце есть, которое сею благостію не умягчается! Сію благость Божію разсуждая Златоустъ святый, и человѣческое о ней нерадѣніе видя, плакалъ и рыдалъ: «Сіе есть, о чемъ наипаче плачуся и рыдаю. Чего не сотворилъ Богъ, да любимъ Его? чего не ухитрилъ? что оставилъ? Досадили мы Ему, ничимъ насъ не обидѣвшему, но содѣявшему безчисленная и неизреченная благая. Отвратилися призывающаго и отвсюду привлекающаго: и ни тако муки насъ не предалъ, но притекъ Самъ и удержалъ бѣжащихъ; но мы оттряслися, и къ діаволу отскочили. Однакожь и тако не отступилъ, но послалъ многихъ пророковъ, ангеловъ, патріарховъ, молити насъ. Мы же не приняли моленія, но и досадили пришедшимъ: а Онъ и за сіе не презрѣлъ насъ». И мало спустя: «По сихъ всѣхъ, глаголетъ отецъ святый, убили мы пророковъ, каменіемъ побили, и иная премногая злая и лютая содѣлали. Что убо? Вмѣсто сихъ уже не пророковъ, уже не ангеловъ, не патріарховъ, но Самаго послалъ Сына Своего. Убіенъ бысть и Сынъ пришедый. Но и тако не погасилъ любве Своея, но болѣе возжегъ; и не престаетъ увѣщавать по убіеніи Сына Своего, и молить, и все творить, да обратимся къ Нему. И вопіетъ Павелъ, глаголя: \textit{по Хрістѣ убо молимъ, аки Богу молящу нами, примиритеся Богу}\footnote{Римл.~6,~20.}. Но ничтоже отъ сихъ примирило насъ. А Онъ и тако насъ не оставилъ, но ждетъ, и геенною претитъ, и царство обѣщаетъ, да хотя тако насъ къ Себѣ привлечетъ»\footnote{Бес.~5"~я на посл. къ Римл.}. Тако благаго Бога и благоутробнаго Отца, или паче самую благостыню прогнѣвляетъ грѣхомъ, а паче нераскаяннымъ житіемъ человѣкъ! Богъ ему хощетъ спастися, и аки жаждетъ спасенія его; чего ради и Сына Своего не пощадѣлъ, но послалъ въ міръ, и на смерть предалъ за него; но грѣшникъ нераскаянный противится такъ великому Божіему хотѣнію и благоволенію. И сіе есть хитрость и злоба діавола, который какъ самъ отвратился отъ Создателя своего и въ ожесточеніи пребываетъ, и непрестанно прогнѣвляетъ Его, такъ и человѣка такожде научаетъ, который ослѣпившися не видитъ, что діавола слушаетъ, и съ нимъ Богу противится, и вѣчно погибаетъ, что Бога, яко человѣколюбца, весьма оскорбляетъ. Возненавидимъ убо, хрістіанине, грѣхъ, діавольское дѣло, которымъ благій и благоутробный Богъ нашъ оскорбляется, и покажемъ нашу благодарность Ему, Который нашего спасенія такъ усердно желаетъ. Человѣка благодѣтеля, который какое нибудь намъ добро дѣлаетъ, не хощемъ оскорбить; кольми паче Бога, Котораго безчисленная благая на себѣ дознаемъ, не должно оскорблять. Аще же пренебрежемъ тое, то дознаемъ на себѣ правды Его дѣло, которая \textit{всѣмъ воздастъ по дѣломъ ихъ}\footnote{Римл.~2,~6.}.

\paragraph*{§\:415.} Правда Божія да подвигнетъ насъ, хрістіанине, къ истинному покаянію и подвигу противу грѣха. Нынѣ время покаянія, когда велитъ каятися, призываетъ на покаяніе и пріемлетъ кающихся; но въ день воздаянія того не будетъ, хотя и со слезами поищемъ того. Нынѣ милуетъ Господь ищущихъ милости отъ Него, тамо будетъ судить. Нынѣ кающимся грѣхи оставляетъ, "--- тамо за всякій грѣхъ будетъ судить. Аще бо \textit{о всякомъ словѣ праздномъ воздадятъ человѣцы слово въ день судный, какъ учитъ Хрістосъ}\footnote{Мѳ.~12,~36.}, кольми паче о содѣянныхъ грѣхахъ и беззаконіяхъ. Нынѣ по большей части благость и милость Божія дѣйствуетъ: тогда правда Божія въ дѣло Свое вступитъ, и \textit{воздастъ всѣмъ по дѣломъ ихъ}. Нынѣ время милости и помилованія: тогда будетъ время суда. Нынѣ "--- просить, искать и толкать въ двери милосердія Божія, \textit{яко всякъ} нынѣ \textit{просяй пріемлетъ, и ищай обрѣтаетъ, и толкущему отверзается}\footnote{7,~8.}; \textit{се нынѣ время благопріятно, се нынѣ день спасенія}\footnote{2~Кор.~6,~1.}. Тогда все тое пресѣчется, прошенію и исканію не будетъ мѣста, заключатся двери милосердія, но едино строгое испытаніе будетъ. Нынѣ Богъ глаголетъ: \textit{обратитеся ко Мнѣ}; тогда возглаголетъ: \textit{отвѣщайте Мнѣ}. Нынѣ глаголетъ Хрістосъ: \textit{пріидите ко Мнѣ вси труждающіися и обремененніи, и Азъ упокою вы}\footnote{Мѳ.~11,~28.}; тогда возглаголетъ не хотѣвшимъ нынѣ пріити: \textit{идите отъ Мене проклятіи во огнь вѣчный, уготованный діаволу и аггеломъ его}\footnote{25,~41.}. Здѣ паки, хрістіанине, обрати умъ и размышленіе твое къ страданію и кресту Хрістову. Страданіе Хрістово кающимся и нынѣ утѣшеніе подаетъ, и тогда дерзновеніе подастъ: яко Хріста распятаго, яко Спасителя и Искупителя своего, истинною вѣрою приняли, почитали и благодарны Ему были; и увидятъ Его тогда въ Божественной Своей славѣ, и возрадуются о Немъ, яко несуетна была вѣра ихъ въ Него; и отъ Него за подвигъ вѣры получать обѣщанное имъ царство небесное: не кающимся же и нынѣ не пользуетъ, и тогда страхъ и ужасъ и несносный стыдъ содѣлаетъ. Тойжде бо Хрістосъ, который за насъ былъ оплеванъ, поруганъ, уязвленъ, мученъ, распятъ и умерщвленъ, явится въ страшной славѣ Своей и будетъ судити весь міръ; Который за грѣхи міра умеръ, Тойжде за грѣхи міръ судити будетъ. Страшно было позорище видѣти Сына Божія Господа славы оплеваема, ругаема, посмѣваема, уязвляема, обнажаема, крестъ носяща, на крестѣ протягаема и пригвождаема, на крестѣ висяща и со беззаконными вмѣненна. И мы то не безъ страха нынѣ воспоминаемъ, яко воистину страшное дѣло есть, что Богъ во плоти Своей святой тако отъ беззаконныхъ людей поруганъ и пострадалъ. Но далеко большій страхъ и ужасъ обыметъ грѣшниковъ, когда увидятъ тогожде Царя славы, въ Божественной своей Славѣ пришедшаго, безчисленнымъ ангелъ святыхъ множествомъ окружаемаго, пришедшаго уже не спасти, но судити грѣшниковъ непокаявшихся, и праведный Свой гнѣвъ на нихъ, яко неблагодарныхъ, явити, которыхъ хотя спасти, воплотился и распятіе и смерть претерпѣлъ, но отъ нихъ презрѣнъ и отверженъ былъ. \textit{И воззрятъ на Него, Егоже прободоша}, пишется о распинателяхъ Его\footnote{Ап.~1,~7.}. \textit{Воззрятъ} и прочіи вси, которые не хотѣли Его, за нихъ умершаго, слушать, почитать, но своимъ похотямъ послѣдовали и творили волю не Его, но противника Его "--- діавола, и тако Ему, хотѣвшему ихъ спасти, противились. Противится бо Хрісту не токмо тотъ, кто противно Ему учитъ, но и тотъ, кто противно слову Его живетъ. \textit{Тогда возглаголетъ къ нимъ гнѣвомъ Своимъ, и яростію Своею смятетъ ихъ}\footnote{Пс.~2,~5.}. \textit{Идите отъ Мене проклятіи во огнь вѣчный, уготованный діаволу и аггеломъ его}\footnote{Матѳ.~25,~41.}. «Благая Моя, которыя получали вы: солнце, луна и звѣзды, которыя свѣтили вамъ, облака, которыя кропили на васъ и нивы ваши; дождь, озера, рѣки и источники, которые напаяли васъ и скоты ваши; земля съ плодами, которая питала васъ; огнь, который согрѣвалъ васъ, и прочая благая Моя свидѣтельствовали о Мнѣ"=Благодѣтелѣ. Но вы, благими Моими довольствуяся, не хотѣли познать и почитать Благодѣтеля. Посылалъ Я къ вамъ пророки Моя, \textit{и звахъ васъ, и не послушасте: глаголахъ, и преслушасте}\footnote{Ис.~65,~12.}. Пришелъ и Самъ Я къ вамъ, отступившимъ отъ Мене, спасти васъ; пришелъ въ подобострастной вамъ плоти, и смерть за васъ вкусилъ, чтобы васъ умершихъ оживить. Сію благодать проповѣдать апостоламъ Моимъ повелѣлъ вамъ; воздвиглъ пастырей и учителей, которые о томжде вамъ проповѣдывали и научали; Евангеліе Мое, \textit{проповѣданное всей твари}\footnote{Марк.~16,~15.}, свидѣтельствовало о Мнѣ; чудеса, сотворенная и въ немъ написанная, показывали, что Я есмь Мессія, пророками проповѣданный и отъ всего міра ожиданный\footnote{Матѳ.~11,~4--5.}; ученіе, Мною проповѣданное, свидѣтельствовало о истинѣ Моей и великомъ Моемъ желаніи спасенія вашего; милость Моя, явленная грѣшникамъ кающимся, привлекала васъ ко Мнѣ; судъ Мой праведный, вамъ возвѣщенный, устрашалъ васъ и къ покаянію возбуждалъ; страданіе Мое, смерть Моя, и все Мое смотрѣніе, васъ ради воспріятое, какъ устами проповѣдывало, какъ Я васъ любилъ и хотѣлъ вамъ спастися, сотворити васъ блаженными и ничего не оставилъ, чтобы васъ привести къ вѣчному блаженству. Но вы любовь Мою сію, высокую сію Мою благодать и милость ни во что вмѣнили; послѣдовали волѣ плоти своея и Мене не хотѣли слушать. Оставили вы Мене въ мірѣ, оставляю и Я васъ нынѣ. Не знали вы Мене тогда, и Я нынѣ \textit{не вѣмъ васъ. Отступите отъ Мене вси дѣлателіе неправды}\footnote{Лук.~13,~27.}. Отреклися вы тогда Моихъ словесъ, отрицаюся и Я васъ нынѣ. Не хотѣли вы тогда пріити ко Мнѣ, не хощу и Я васъ нынѣ принять къ Себѣ. \textit{Идите отъ Мене проклятіи во огнь вѣчный, уготованный діаволу и аггеломъ его}». О какъ несносный страхъ и трепетъ "--- чувствовать гнѣвъ оный! Пожелаютъ бѣдные грѣшники въ тотъ часъ скрытися, но не могутъ. \textit{Тогда начнутъ глаголати горамъ: падите на ны, и холмамъ: покрыйте ны отъ лица Сѣдящаго на престолѣ, и отъ гнѣва Агнча: яко пріиде день великій гнѣва Его, и кто можетъ стати}\footnote{23,~30; Ап.~6,~16--17.}? По страшномъ гнѣвѣ и праведномъ выговорѣ услышанномъ потопляются грѣшники въ ужаснѣйшемъ праведнаго суда Божія потопѣ, и будутъ чрезъ всю вѣчность платить правдѣ Божіей \textit{безконечной}, которую грѣхами нынѣ огорчаютъ и раздражаютъ. Но и въ нынѣшнемъ вѣкѣ видимъ страшные правды Божіей суды. Читаемъ, что въ началѣ весь міръ потопомъ погиблъ; Содомъ и Гоморръ съ окрестными градами пожжены; Фараонъ со всѣмъ воинствомъ въ морѣ потопленъ; Израильтяне въ пустынѣ различно поражены за беззаконія, и проч. Слышимъ и видимъ нынѣ, коль много пожираетъ людей моровая язва, земли трясеніе; коликихъ городовъ поядаетъ огнь, опустошаетъ иноплеменническій мечь, и проч. Сколько праведнымъ судомъ восхищается беззаконниковъ, и въ самомъ беззаконія дѣйствіи низходитъ во адъ воспріяти по дѣломъ своимъ! Колико блудниковъ и блудницъ, въ скверномъ дѣлѣ связанныхъ, колико разбойниковъ въ самомъ разбоѣ, колико хульниковъ въ самомъ хуленіи, хищниковъ и воровъ въ хищеніи и воровствѣ, піяницъ въ нечувственномъ пьянствѣ погибаетъ! Но что единымъ беззаконникамъ сдѣлалося, или дѣлается, того ожидать должно и другимъ. Откуду Хрістосъ повѣдающимъ о погибели Галилеанъ сказалъ: \textit{мните ли, яко Галилеане сіи грѣшнѣйши паче всѣхъ Галилеанъ бяху, яко тако пострадаша? Ни, глаголю вамъ: но аще не покаетеся, вси такожде погибнете}\footnote{Лук.~13,~2--3.}. Слово сіе страшное какъ тогдашнимъ грѣшникамъ сказано, такъ и нынѣшнимъ, и въ слѣдующіе роды глаголется: \textit{аще не покаетеся, вси такожде погибнете}. И хотя не всѣ беззаконники, по недовѣдомымъ судьбамъ Божіимъ, здѣ казнь пріемлютъ, однакожь не избѣгнутъ страшнаго Божія суда въ будущемъ вѣкѣ, какъ выше сказано. Да убоимся убо, человѣче, праведнаго Божія суда и покаемся; обратимся, пока насъ не постигла месть Божія.

\paragraph*{§\:416.} Сильно есть подвигнуть хрістіанъ къ покаянію и осторожности написанное: \textit{полунощи вопль бысть: се Женихъ грядетъ, изыдите во срѣтеніе Его}\footnote{Матѳ.~25,~6.}. Когда сей вопль будетъ, неизвѣстно; когда Женихъ душъ нашихъ пріидетъ, не знаемъ: \textit{яко не знаемъ дне, ни часа, въ оньже Сынъ человѣческій пріидетъ}\footnote{ст.~13.}. \textit{Яко же бо бысть во дни Ноевы, тако будетъ и пришествіе Сына человѣческаго. Якоже бо бѣху во дни прежде потопа, ядуще и піюще, женящеся и посягающе, до негоже дне вниде Ное въ ковчегъ, и не увѣдѣша, дондеже пріиде вода и взятъ вся: тако будетъ и пришествіе Сына человѣческаго}\footnote{Мѳ.~24,~37--39.}. \textit{Ибо пріидетъ день Господень, яко тать въ нощи, въ оньже небеса убо съ шумомъ мимоидутъ, стихіи же сжигаемы розорятся, земля же, и яже на ней дѣла, сгорятъ}\footnote{2~Петр.~3,~10.}. \textit{Егда бо рекутъ: миръ и утвержденіе: тогда внезапу нападетъ на нихъ всегубительство, якоже болѣзнь во чревѣ имущей: и не имутъ избѣжати}\footnote{1~Сол.~5,~3.}. Сего ради всегда должно намъ, хрістіанине, готовыми быть къ срѣтенію Жениха нашего и Господа, и свѣтильники свои украшать, да не съ юродивыми дѣвами внѣ чертога останемся и къ Жениху безъ пользы будемъ молитися: \textit{Господи, Господи, отверзи намъ}\footnote{Мѳ.~25,~11.}! Но паче, съ мудрыми срѣтивши Его, внидемъ съ Нимъ на браки, \textit{и тако всегда съ Господемъ будемъ}\footnote{1~Сол.~4,~17.}.

\paragraph*{§\:417.} \textit{Упованіе, отложенное} хрістіанамъ \textit{на небесѣхъ}\footnote{Кол.~1,~5.}, сильно есть хрістіанъ подвигнуть и утвердить съ помощію Божіею въ благочестіи и терпѣніи, отвратить отъ грѣха, которымъ оно погубляется. Тамо домъ небеснаго Отца, и \textit{въ дому Его обители многи суть}\footnote{Іоан.~14,~2.}. Тамо \textit{градъ святый, Іерусалимъ новъ}, его же видѣлъ Іоаннъ святый\footnote{Ап.~21,~2.}; \textit{и нощи не будетъ тамо, и не потребуютъ свѣта отъ свѣтильника, ни свѣта солнечнаго, яко Господь Богъ просвѣщаетъ я}\footnote{22,~5.}. Тамо \textit{благая любящимъ Бога уготована суть, ихже око не видѣ, и ухо не слыша, и на сердце человѣку не взыдоша}\footnote{1~Кор.~2,~9.}. Тамо \textit{лицемъ къ лицу видятъ Бога} избранніи Его\footnote{1~Кор.~13,~12.}. Тамо сіяетъ Солнце правды, Хрістосъ, Сынъ Божій, въ славѣ Своей, и никогда не заходитъ, ни облаками покрывается, но всегда свѣтитъ, просвѣщаетъ, согрѣваетъ и веселитъ избранныхъ Своихъ, и сами \textit{яко солнце сіяютъ во царствіи Отца ихъ}\footnote{Мѳ.~13,~43.}. Тамо \textit{уготовася вечеря велія}\footnote{Лук.~14,~16.}, на которой \textit{возлежатъ со Авраамомъ, Исаакомъ и Іаковомъ пришедшіе отъ востокъ и западъ}\footnote{Мѳ.~8,~11.}. Тамо \textit{гласъ радованія и исповѣданія, шума празднующаго}\footnote{Пс.~41,~5.}, \textit{гласъ радости и спасенія въ селеніихъ праведныхъ}\footnote{117,~15.}. Тамо \textit{народъ многъ, егоже исчести никтоже можетъ, отъ всякаго языка и колѣна, и людей и племенъ, стояще предъ престоломъ и предъ Агнцемъ, облечены въ ризы бѣлы, и финицы въ рукахъ ихъ}, и проч.\footnote{Ап.~7,~9 и слѣд.}, и поютъ пресладкую пѣснь: \textit{Аллилуіа! яко воцарися Господь Богъ Вседержитель. Радуимся и веселимся, и дадимъ славу Ему: яко пріиде бракъ Агнчій, и жена Его уготовила есть себе. И дано бысть ей облещися въ виссонъ чистъ и свѣтелъ; виссонъ бо оправданія святыхъ есть}\footnote{19,~6--8.}. Тамо безчисленное множество ангелъ святыхъ, которые окружаютъ престолъ славы и поютъ Тріѵпостаснаго Бога: \textit{Святъ, Святъ, Святъ Господь Саваоѳъ}\footnote{Ис.~6,~3.}! Тамо ликъ патріарховъ, праотцевъ, пророковъ, апостоловъ, святителей, мучениковъ вселяется и хвалитъ Пресвятую Троицу. Тамо наконецъ \textit{будетъ Богъ всяческая во всѣхъ}\footnote{1~Кор.~15,~28.}, всѣхъ животворяяй, сохраняяй, упокоеваяй, ублажаяй, утѣшаяй, увеселяяй, радостотворяяй, просвѣщаяй и прославляяй. На сіе блаженство вѣрою взирая, \textit{Моисей отвержеся нарицатися сынъ дщере Фараоновы; паче же изволи страдати съ людьми Божіими, нежели имѣти временную грѣха сладость; большее богатство вмѣнивъ египетскихъ сокровищъ поношеніе Хрістово: взираше бо на мздовоздаяніе}\footnote{Евр.~11,~24--26.}. На сіе когда и мы, хрістіанине, почаще вѣрою взирать будемъ, безъ сумнѣнія восхощемъ лучше со Хрістомъ пострадавшимъ за насъ страдати и терпѣти, нежели со злымъ міромъ безумно утѣшаться.

\paragraph*{§\:418.} Праведное и вѣрное мірскихъ вещей разсужденіе можетъ сердце отвратити отъ суеты и обратити къ истинному благочестію. 1)~Злато, сребро, каменіе драгое, украшеніе одѣяній, доброта личная и все, что очесамъ человѣческимъ въ мірѣ показуется дорогимъ и пріятнымъ, что есть, аще не прелесть и похоть очесъ нашихъ? Во тьмѣ или слѣпотѣ познаемъ, что оно все ничимъ не разнствуетъ отъ другихъ вещей, по мнѣнію человѣческому подлыхъ; въ нощи не разпознаемъ злата и сребра отъ желѣза, мѣди и прочіихъ металловъ, каменія дорогаго отъ простаго, ибо и то и другое осязанію рукъ нашихъ равно показуется. Слѣпому какъ риза златотканная, такъ и одѣяніе многошвенное и рубищное; какъ хламида многоцѣнная, такъ и мантія монашеская одинакова кажется; какъ молодаго человѣка красное, такъ и стараго морщиноватое лице, и все, что очи видятъ и утѣшаются, и отъ чего отвращаются, "--- равно вмѣняетъ и не цѣнитъ одного отъ другаго лучшимъ. Тако и мы, хрістіанине, когда отвратимъ очи наши отъ суеты, и будемъ ихъ вперять въ вѣчное блаженство, тогда не будемъ цѣнить злата, сребра, каменія драгаго и прочаго мірскаго украшенія паче другихъ вещей, по мнѣнію людскому, некрасныхъ. Тогда, истину тебѣ говорю, все намъ равно будетъ, какъ злато и сребро, такъ желѣзо и олово, какъ каменіе дорогое, хранимое въ сундукахъ, такъ поверженное на дороги и попираемое ногами; какъ одѣяніе шелковое и разноцвѣтное, такъ и рубище, покрывающее наготу; какъ палаты украшенныя, такъ и хижина простая; какъ красота молодаго, такъ и морщина стараго. Тогда со святымъ Златоустомъ признаемъ и мы \textit{злато какъ блато, и сребро какъ плевелъ}\footnote{Бес.~24"~я на Матѳ.}. Тогда почтимъ милостыню, единожды во имя Хрістово сотворенную, обиду, великодушно претерпѣнную и ближнему отъ сердца оставленную, паче всѣхъ міра сего сокровищь. Тогда болѣе будемъ жалѣть и сокрушаться о единомъ и маломъ согрѣшеніи сотворенномъ, нежели о всемъ тлѣнномъ нашемъ сокровищѣ потерянномъ. Злато и сребро, какъ желѣзо, мѣдь, олово и прочее, не иное что, какъ земля, яко изъ земли дѣлается. Красота лица есть кровь червленѣющая и на верхъ тѣла исходящая. Одѣяніе разноцвѣтное различныя краски и пестроты дѣлаютъ. Тогда познаеши, что есть злато и сребро, когда изоржавѣетъ; что есть одѣяніе красное, когда моль тое поястъ; что есть доброта личная, когда на нее, во гробѣ лежащую, посмотришь. Тогда безъ сумнѣнія признаешь, что все сіе не иное что, какъ земля, изъ земли взято и въ землю обращается. А тако увидишь, что истинное и многоцѣнное добро наше есть хрістіанская добродѣтель, которая ни тлѣнію не подлежитъ, ни отъ насъ отлучается, съ нами въ мірѣ живетъ, и на оный вѣкъ отходитъ, и насъ къ Богу, высочайшему нашему добру, приводитъ. Что убо? Злато"=ли"=де и сребро и прочее временное добро охуждаешь? Никакъ! не охуждаю злата и сребра и прочаго богатства, ибо все есть Божіе созданіе и Божій есть даръ, человѣкамъ данный. Но охуждаю златолюбіе и сребролюбіе, которымъ сердце отъ любви и почитанія Бога живаго отводится. Не хулю злата и сребра, но не хвалю сребролюбца, который, оставивши Бога, къ сребру и злату сердцемъ прилѣпляется, и вмѣсто Бога, мамонѣ работаетъ. Негрѣшно имѣть злато и сребро, но грѣшно прилагать ему сердце. Оно должно служить намъ, а не мы ему. Служитъ же намъ какъ прочее созданіе, такъ злато и сребро, чтобы мы Богу служили. Но когда любимъ тое и прилагаемъ ему сердце, то уже не оно намъ, но мы ему работаемъ, и тако, работая мамонѣ, уже Богу не работаемъ. \textit{Никтоже бо можетъ двѣма господинома работати}\footnote{Мѳ.~6,~24.}. Ктомужъ все сіе сокровище, какое оно ни есть, какъ мірское, въ мірѣ останется. \textit{Ничтоже бо внесохомъ въ міръ сей, явѣ, яко ниже изнести что можемъ}\footnote{1~Тим.~6,~7.}. Хрістіанамъ же не на земли, но на небеси сокрывать себѣ сокровище повелѣлъ Хрістосъ, да \textit{идѣже сокровище ихъ, тамо будетъ и сердце ихъ}\footnote{Мѳ.~6,~19--21.}. Тамо бо отечество и домъ ихъ, яко въ мірѣ семъ \textit{пришельцевъ и странниковъ}\footnote{Пс.~38,~13; 118,~19; Евр.~11,~13; 1~Пар.~29,~15.}. Хрістіанское сокровище истинное суть добродѣтели, которое здѣ собирается, но на небеси сокрывается, и въ послѣдній день въ явленіе всему міру отъ Хріста Судіи произнесется\footnote{Мѳ.~25,~35--36.}. Такожде \textit{сребролюбіе} или богатства любовь \textit{корень всѣмъ злымъ} отъ апостола судится\footnote{1~Тим.~6,~10.}. Оно научаетъ похищать, красть, лихоимствовать, суды неправо судить, праваго обвинять и виноватаго оправдать, научаетъ лгать, напрасно клястися, имя Божіе всуе призывать и клятву нарушать, обманывать, лукавить; словомъ, всему злу научаетъ оно, что отъ хрістіанъ удалено быть должно. А хотя гдѣ и праведно собрано богатство, что весьма рѣдко бываетъ, но когда или въ сундукахъ сокрывается, или на излишніе расходы и непотребныя украшенія домовъ, садовъ, коней, каретъ, одѣяній, трапезъ и прочіихъ мірскихъ суетъ иждивается, "--- то и тое дѣлается противу воли Божіей, которая хощетъ, чтобы мы богатство, намъ отъ Него данное, не на оныя суеты, но какъ на свои домашнія нужды употребляли и благодарили Ему, такъ на убогихъ и нищихъ потребы раздѣляли. Богатый бо не хозяинъ есть богатства, ему отъ Бога даннаго, но прикащикъ и расходчикъ, который даровавшему Господу отвѣтъ долженъ воздати въ свое время. Ибо никто ничего съ собою не принесъ въ міръ, кромѣ тѣла нагаго, и потому ничего своего не имѣетъ, но все, что ни имѣетъ, Божіе есть добро. Тѣмъ сребромъ и златомъ, которое у тебе въ сундукахъ напрасно лежитъ, должно было тебѣ плѣнныхъ, въ темницѣ за долги и недоимки сѣдящихъ выкупить; тѣмъ излишествомъ; которое ты на созиданіе ненужныхъ строеній полагаешь, должно было тебѣ создать хижины неимущимъ гдѣ главы подклонить и упокоиться; что употребляешь на украшеніе одѣяній, тѣмъ должно было тебѣ одѣть полунагихъ и бѣдныхъ; что расточаешь на богатые столы, тѣмъ должно было питать алчущихъ, нищихъ и немощныхъ. Сего воля Божія хощетъ отъ насъ. Сего естественный законъ требуетъ: \textit{чего хощешь себѣ, тое и ближнему твори}. И такимъ образомъ не потерялъ бы ты богатства, отъ Бога даннаго, но предпослалъ бы оное на оный вѣкъ, и пріобрѣлъ его съ лихвою великою въ день Господень. Но злое самолюбіе, сребролюбіе и міролюбіе не допускаетъ до сего блаженства человѣка; но, вмѣсто того, приноситъ ему безполезную міра печаль, злую совѣсть, неполезное и поздное раскаяніе, всемірный въ день суда стыдъ и вѣчную погибель. Тогда ты самъ признаешь, что ты подлинно былъ нищь и убогъ, хотя и богатъ былъ. Тогда увидишь, что едино душевное добро есть истинное добро, а не тое, что очесамъ человѣческимъ кажется. Тогда добрѣ уразумѣеши, что богатство подобно есть \textit{тернію, которымъ подавляется сѣмя Божія слова}\footnote{Мѳ.~13,~22; Марк.~4,~18--19; Лук.~8,~14.}. "--- 2)~Честь и благородіе міра сего какъ за велико почитается отъ сыновъ вѣка сего, сказать невозможно. Но какъ истинное богатство, такъ и истинное благородіе не тое есть, которое судится и называется такъ отъ своихъ любителей (часто бо люди тое превозносятъ, что любятъ, хотя въ самой вещи и не похвально); но тое, которое называется и есть въ самой вещи и здравому разуму и слову Божію, которое есть истинное нашихъ помышленій правило, согласно. Смотри, человѣче, когда предковъ твоихъ благородіемъ хвалишься, какъ Іудеи хвалилися патріархомъ Авраамомъ: \textit{отца имамы Авраама}, "--- которыхъ Хрістосъ обличилъ: \textit{аще чада Авраамля бысте были, дѣла Авраамля бысте творили}\footnote{Іоан.~8,~33 и 39.}, "--- то дѣлаешь ли ты такъ достохвальныя дѣла, которыя предки твои дѣлали? Когда людямъ повелѣваешь, то не повелѣваютъ ли тебѣ страсти? Когда господиномъ называешься, не господствуетъ ли надъ тобою грѣхъ? Когда судишь людямъ, то не судитъ ли и не осуждаетъ ли тебе совѣсть твоя? Суетная похвала есть добрыхъ предковъ имѣти, но дѣламъ ихъ не подражати. Худо и несмысленно есть людямъ повелѣвать, но страстьми обладаему быть. Рабство и подлость есть не господство, работа не свобода, нарицаться господиномъ, но быть \textit{рабомъ грѣха}\footnote{Іоан.~8,~34.}, тѣломъ надъ людьми подобными себѣ господствовать, но благородною душею работать и покоряться грѣху. Бѣдный тотъ судія, который языкомъ и перомъ судитъ людямъ, но самъ непрестанно обличается и судится отъ совѣсти. Но и сей цвѣтокъ извнѣ красенъ, но внутрь гнилъ, или при нашедшей бѣды и напасти бурѣ, или при наступившей кончинѣ, которая никому необходима, спадаетъ и не является, и бываетъ, какъ бы его и не было. Тогда высокородный и благородный отъ подлаго и господинъ отъ раба ничимъ не разнствуетъ, ибо напасть, а паче смерть равными всѣхъ дѣлаетъ. Тогда всякъ познаетъ, что честь, благородіе и высокородіе міра сего есть пустое имя и титулъ, подобенъ мѣху надутому, но праздному. А что съ душею дѣлается, которая въ благородномъ тѣлѣ жила, надъ людьми господствовала, людямъ повелѣвала, но сама работала страстямъ и грѣху; людей обличала и судила, но сама совѣстію и закономъ Божіимъ обличаема и судима была? Знаемъ, что \textit{лица Богъ человѣча не пріемлетъ}\footnote{Гал.~2,~6.}. Онъ не зритъ на царя и его подданнаго, ни на господина и его раба, ни на благороднаго, ни на подлаго, но на вѣру и добрыя дѣла. Отъ Него \textit{кійждо пріиметъ, яже съ тѣломъ содѣла, или блага, или зла}\footnote{2~Кор.~5,~10.}. Паче же читаемъ въ словѣ Его святомъ, что \textit{судъ жесточайшій преимущимъ бываетъ. Ибо малый достоинъ есть милости: сильніи же сильнѣ истязани}, по другому переводу, мучены \textit{будутъ. Не щадитъ бо лица всѣхъ Владыка, ниже усрамится вельможи, яко мала и велика сей сотвори, подобнѣ же проразумѣваетъ о всѣхъ: державнымъ же крѣпко настоитъ испытаніе}\footnote{Прем.~6,~5--8.}. Честь убо и благородіе міра сего безъ хрістіанскаго истиннаго благородія большему суду и осужденію подлежитъ. Слѣдуетъ бо благородному лицу и въ чести находящемуся, но не по"=хрістіански живущему, не токмо за себе, но и за подчиненныхъ, которыхъ не токмо не исправлялъ, но и соблазнялъ, отвѣтъ дати праведному Судіи, котораго суда никто не избѣжитъ. Но люди, не разсуждая истины и прелести, настоящихъ и будущихъ, оставивши истину, за прелесть берутся, такъ, какъ малыя и неразумныя дѣти, оставивши злато, за горячее угліе руками хватаются, отъ котораго ожегшися, потомъ горько болѣзнуютъ и плачутъ. Я здѣ паки не лица благородныя охуждаю, "--- не буди тое, "--- но честолюбіе и славолюбіе, которое съ истиннымъ хрістіанскимъ благородіемъ и вѣрою помѣститься не можетъ, по реченному: \textit{како вы можете вѣровати, славу другъ отъ друга пріемлюще, и славы, яже отъ единаго Бога, не ищете}\footnote{Іоан.~5,~44.}? И признаю, что языческое дѣло есть за суетною славою гоняться, которые того только ищутъ, что чувствамъ ихъ подлежитъ, а чего не знаютъ, того и не ищутъ. Хрістіанамъ стыдно въ семъ язычникамъ подражать, которыхъ сердца къ иному и несравненно лучшему благородію вѣра возводить должна. Хрістіане таковыи такъ заблуждаютъ въ семъ, какъ тіи, которые, оставивши мѣшецъ, златомъ исполненный, хватаютъ и другій извнѣ красный, но внутрь пустый и воздухомъ только надмѣнный; или паче какъ тіи, которые, повергши царскую багряницу, рубищемъ гнуснымъ и смраднымъ хотятъ покрыться, и, оставивши высокій титулъ царскаго сыновства, хотятъ быть и называться подлымъ поселяниномъ и рабомъ, "--- чему бы всякъ достойно смѣялся. Тяжко они въ томъ грѣшатъ противу Бога, яко неблагодарніи, Который ихъ къ высокимъ и великимъ вещамъ, къ истинному и высокому благородію позвалъ, но они, оставивши тое, за подлымъ и мнимымъ гоняются. "--- \textit{Какое"=де есть истинное благородіе хрістіанское? Отвѣтъ}: быть истиннымъ хрістіаниномъ, сыномъ церкви святыя, живымъ удомъ тѣла Хрістова, имѣть общеніе со Отцемъ и Сыномъ Его Іисусомъ Хрістомъ, сыномъ и наслѣдникомъ Божіимъ быть, вѣрою и правдою работать живому и безсмертному Богу, и проч., о чемъ и сказано и ниже скажется. Сіе благородіе всякую славу міра сего несравненно превосходитъ. Вся слава царей и князей міра сего ничтоже предъ тѣмъ. Не видна сія нынѣ слава, но открыется въ послѣдній день. \textit{Нынѣ}, глаголетъ апостолъ, \textit{чада Божія есмы, и не у явися, что будемъ? Вѣмы же, яко, егда явится, подобни Ему будемъ: ибо узримъ Его, якоже есть}\footnote{1~Іоан.~3,~2.}. Но славолюбіе суетное не допущаетъ бѣднаго человѣка до истинной той славы. Славолюбіе, говорю, а не слава міра сего, не допущаетъ. Иное бо славу и честь въ мірѣ семъ имѣть; иное славу и честь желать и искать. Могутъ имѣть и имѣютъ многіи истинныи хрістіане и святіи славу въ мірѣ семъ, но не ищутъ ея, паче же и убѣгаютъ ея. Славолюбіе бо есть знакъ пристрастія къ міру, что отъ хрістіанскаго сердца удалено должно быть. "--- \textit{Что"=де мнѣ дѣлать, когда санъ и честь дается мнѣ? Отвѣтъ}: Когда не ищешь, а дается тебѣ, пріемли не какъ честь, но какъ иго, которое должно тебѣ въ славу Божію и пользу ближняго носить; буди на чести той, яко отецъ, который промышляетъ о дѣтяхъ своихъ; служи славѣ Божіей, а не своей, и пользѣ братіи твоея, а не своей; буди рабъ и слуга рабамъ Божіимъ, якоже глаголетъ Хрістосъ: \textit{иже аще хощетъ въ васъ вящшій быти, да будетъ вамъ слуга; и иже аще хощетъ въ васъ быти первый, буди вамъ рабъ: якоже Сынъ человѣческій не пріиде, да послужатъ Ему, но послужити и дати душу Свою избавленіе за многихъ}\footnote{Мѳ.~20,~26--28.}. Тогда и сія честь исходатайствуетъ тебѣ большую на небеси славу, егда милостивый Господь нашъ скажетъ тебѣ: \textit{добрѣ, рабе благій и вѣрный, о малѣ былъ еси вѣренъ, надъ многими тя поставлю: вниди въ радость Господа твоего}\footnote{25,~21--23.}. "--- 3)~Третія утѣха сыновъ вѣка сего состоитъ въ сластолюбіи и роскоши. О, коль много людей сія отъ Хріста и истиннаго богопочитанія отводитъ! Многіи презираютъ честь и славу суетную, не ищутъ богатства, расточаютъ и тое, что имѣютъ; но чреву угождать мало кто не хощетъ. Разсуди, хрістіанине, какъ и въ семъ много заблуждаемъ. Что есть сластопитаніе, какъ только вкуса и гортани услажденіе самократчайшее, которое дотолѣ услаждаетъ, доколѣ вкусъ и гортань переходитъ? Сладкая и различными приправами растворенная пища, вмѣстившися въ желудокъ, бываетъ какъ простая. Ибо желудокъ не разбираетъ, каковая будетъ пища "--- сладкая, или простая, только бы была здорова; онъ полезныя требуетъ пищи, а не сладкія. Равно насыщаются люди простою пищею, какъ и сладкою. Деревенскій мужикъ равно укрѣпляетъ себе хлѣбомъ съ солію и водою, какъ его помѣщикъ сладкими и различными снѣдьми: равно и сей и тотъ по обѣдѣ ходятъ, дѣлаютъ свои дѣла, и проголодавшися паки всякъ за свою пріемлется пищу; и кто вчера какую пищу вкушалъ, сегодня не познаетъ, но всякъ и кто многими и сладкими снѣдьми вчера насыщался, и кто простѣйшею довольствовался, паки сегодня алчетъ и хощетъ насытити чрево. Разсуди и слѣдствія отъ сладкія и простыя пищи и принадлежности ихъ, и увидишь, что сладкой и многоразличной снѣди всякое зло послѣдуетъ. Гдѣ болѣе нечистая похоть, смертоносная хрістіанскія души язва гнѣздится, какъ не въ роскошномъ и сластолюбивомъ сердцѣ? Кто къ молитвѣ и прочей хрістіанской должности лѣнивѣйшій, какъ не роскошникъ? Кто въ праздности, всему злу виновной, болѣе живетъ, какъ сластолюбецъ? Истина сія явна. Какъ въ гниломъ болотѣ всякія плодятся гадины, такъ въ сластолюбивомъ сердцѣ всякая родится грѣховная нечистота. Не тако умѣренно и простою пищею живущій: онъ хотя и чувствуетъ ражженіе похоти, но молитвою и помощію Божіею угашаетъ тое. Онъ въ дѣлахъ званія и прочіихъ благословенныхъ трудахъ всегда бодръ, къ молитвѣ нелѣностенъ, всегда ко всякому доброму дѣлу готовъ. Сладкія и различныя снѣди требуютъ много принадлежностей, требуютъ различныхъ составовъ, приправъ, искусныхъ поваровъ и приспѣшниковъ; но простая пища того не ищетъ. Различныя снѣди требуютъ не малыя суммы и иждивенія; и тако, что должно на нищую и убогую братію обратитися, тое сластопитаніе едино пожираетъ, что есть душевредное самолюбіе; простая пища отъ того свободна и не препятствуетъ рукѣ нашей быть щедрою къ подаянію. Молчу о томъ, что сластопитаніе часто бываетъ отъ неправдъ и обиды ближняго, отъ чего простая пища, яко малымъ иждивеніемъ составленная, свободна есть. Различной снѣди не вездѣ можно имѣть, какъ"=то: въ дорогѣ, на брани и въ прочіихъ случаяхъ, простая же вездѣ готова. Что различныя снѣди дѣлаютъ чреву, сколько раждаютъ болѣзней, "--- лекари о томъ знаютъ. Къ кому бо болѣе они приходятъ съ лѣкарствами, какъ къ сластолюбцамъ? Отъ простой и умѣренной нѣтъ такой опасности, и хотя приходитъ болѣзнь и къ воздержнымъ, но, мало сыскавши въ нихъ, чимъ бы замедлить могла, скоро и отходитъ, воздержаніемъ, какъ сильнѣйшимъ лѣкарствомъ, прогонима. Но чтобы тебѣ, хрістіанине, увѣриться въ томъ, что сластопитаніе нездравія и болѣзней причиною бываетъ, якоже простая пища и умѣренная есть матерь здравія, посмотри на благородныхъ дѣтей, которыя сладкою пищею питаются, и дѣтей крестьянскихъ, которыя по большей части хлѣбъ со щами или съ водою ядятъ, "--- и увидишь, какъ сіи отъ оныхъ разнятся въ крѣпости и цвѣтѣ лица. Какъ сіи крѣпки и красны лицемъ, а оныя дряхлы, блѣдны, и какъ бы безкровны! Сіе различіе не отъинуды, какъ отъ различности воспитанія происходитъ; что сіи просто, а оныя нѣжно и въ сластяхъ воспитуются. "--- \textit{Плоть"=де требуетъ утѣшенія?} Не плоть, но похоть плоти ищетъ сластопитанія; а натура плоти единаго требуетъ подкрѣпленія, которое подаетъ не различная, но здоровая и умѣренная пища. Хрістіанское же утѣшеніе не въ пищи и питіи, но въ духовной радости состоитъ. \textit{Нѣсть бо царство Божіе брашно и питіе, но правда и миръ и радость о Дусѣ Святѣ}\footnote{Римл.~14,~17.}. Требуетъ и плоть своего утѣшенія, какъ"=то: въ алчбѣ питанія, въ жаждѣ напоенія, въ зноѣ прохлажденія, въ холодѣ согрѣтія, по трудахъ упокоенія, и прочая; и такъ должно и плоти угождать, но \textit{не въ похоти}. Откуду и апостолъ не просто угодія плоти творить не велитъ, но въ похоти: \textit{и плоти}, рече, \textit{угодія не творите въ похоти}\footnote{Рим.~13,~14.}. "--- \textit{Я"=де не могу простой принимать пищи?} "--- Правда; но не можешь не отъ природы, но отъ пресыщенія и употребленія различныхъ снѣдей. Не можешь, понеже всегда сытъ имѣешься и трудовъ надлежащихъ не имѣешь. Чрево бо насыщенное и сладкія снѣди не пріемлетъ. А какъ чрезъ цѣлый день попостишися и потрудишися, то и хлѣбъ съ водою, какъ сладкая пища, пріятенъ будетъ. "--- \textit{Я"=де привыкъ такъ жить?} Правда и тое; привычка много можетъ. Но какъ къ сластопитанію привыкъ ты невоздержаніемъ, такъ къ простой пищѣ привыкнуть можешь воздержаніемъ: какъ единожды въ день и умѣренно приступишь къ пищѣ по трудахъ, то и къ простой пищѣ удобно привыкнешь, которая тебѣ какъ пріятная, такъ и здоровѣйшая паче сладкой и различной снѣди будетъ. "--- 4)~Въ заключеніе сего моего разсужденія полагаю Хрістово слово, которое всякаго можетъ подвигнуть къ презрѣнію не токмо богатства, чести и роскоши міра сего, но и живота временнаго ради спасенія души своея: \textit{кая польза человѣку, аще міръ весь пріобрящетъ, душу же свою отщетитъ? или что дастъ человѣкъ измѣну за душу свою? Пріити бо имать Сынъ человѣческій во славѣ Отца своего со ангелы своими: и тогда воздастъ комуждо по дѣяніямъ его}\footnote{Матѳ.~16,~26--27.}. Какая тебѣ польза съ того, хотя ты все богатство, всю славу и всѣ утѣхи міра сего будешь имѣть, но спасеніе вѣчное потеряешь? Нѣтъ никакой пользы тамо, гдѣ души погибель. Не токмо вѣчный, но и временный животъ дороже человѣку паче всего міра. Кто бы не назвалъ того безумнымъ, который бы захотѣлъ животъ сей временный и краткій потерять, чтобы міръ весь пріобрѣсти? Что бо тогда пользуетъ ему міръ, когда самъ погибаетъ? За славнымъ слава, за богатымъ богатство и за роскошникомъ роскошь въ слѣдъ не пойдетъ, но все отъ всякаго отстаетъ и отлучается при кончинѣ его. Аще временнаго живота никто не хощетъ погубить ради пріобрѣтенія всего міра, кольми паче не должно намъ, хрістіанине, вѣчнаго погублять, который есть несравненно лучшій, блаженнѣйшій и вожделѣннѣйшій паче временнаго, и \textit{единъ есть намъ на потребу}\footnote{Лук.~10,~42.}. Ради того не токмо міра, но и живота своего въ случаѣ отрещися должно намъ. Ради того и въ міръ сей раждаемся, чтобы къ вѣчному животу прейти. Не къ временному бо, но къ вѣчному животу создалъ насъ Создатель нашъ Богъ. Временный животъ не иное что есть и долженъ намъ быть, какъ путь къ вѣчному, которымъ путемъ должно намъ итить съ опасностію немалою, яко много враговъ окружаютъ его. Ради того Сынъ Божій въ міръ сей пришелъ, чтобъ намъ къ оному животу отворить двери, которыи мы грѣхами нашими затворили, и къ нему показать путь, съ котораго мы заблудили. Ради того Слово Свое святое, аки \textit{свѣтильникъ ногамъ нашимъ и свѣтъ стезямъ нашимъ}\footnote{Пс.~118,~105.}, подалъ, дабы намъ, по пути сему идущимъ, свѣтилъ. Ради того и Тайны Свои святыя установилъ, и проч. Такое ли сокровище вѣчное и неоцѣненное, ради котораго Самъ Богъ во плоти явился, захощемъ ради любви міра сего потерять? Безуменъ есть, кто временный животъ, который необходимо всякому слѣдуетъ оставить, ради богатства, или чести презираетъ; много паче безумнѣйшій есть, кто о вѣчномъ, ради любви міра сего, пренебрегаетъ. Всякъ же пренебрегаетъ о немъ, кто къ міру сему прилѣпился сердцемъ и любовію, якоже о томъ выше неединократно сказано. \textit{Пріити имать Сынъ Божій во славѣ Своей и воздати всякому по дѣломъ его}. Тогда не спроситъ у насъ, имѣли ли мы въ мірѣ семъ или честь, или славу, или богатство или иное что подобное симъ. Но что? взыщетъ отъ насъ того, чему Онъ и словомъ Своимъ, и образомъ житія Своего училъ. «Называлися вы хрістіанами и Мое имя исповѣдывали. Хорошо. Но гдѣ ваше хрістіанское житіе и плодъ вашего исповѣдованія? Называли вы Мене Господемъ: \textit{Господи, Господи!} Хорошо. Но гдѣ ваше послушаніе, которое Господу своему показывать должны были? Гдѣ ваше смиреніе, терпѣніе, кротость и любовь, чему васъ слово и житіе Мое учило? Обѣщалися вы Мнѣ при крещеніи вѣрою и правдою служить, но гдѣ ваша вѣрная Мнѣ служба? Обѣщалъ Я животъ вѣчный подать вѣрующимъ во имя Мое: вотъ даю его вѣровавшимъ въ Мя. Ваша гдѣ вѣра, которой Я отъ васъ въ словѣ Моемъ требовалъ? гдѣ плоды ея и добрыя дѣла? Стыдились вы тогда Мене и Моихъ словесъ, стыжуся и Я нынѣ васъ. Не хотѣли вы послѣдовать Моему смиренію, терпѣнію, кротости и любви, не хощу и Я васъ нынѣ принять въ славу Мою. \textit{Иже аще постыдится Мене и Моихъ словесъ въ родѣ семъ прелюбодѣйнѣмъ и грѣшнѣмъ, и Сынъ человѣческій постыдится его, егда пріидетъ во славѣ Отца Своего со ангелы святыми}\footnote{Марк.~8,~38. См. еще о семъ въ мѣстахъ, приличныхъ сему: Мѳ.~7,~13,~14,~21 и 23; Лук.~13,~24--27 и проч.}. О, какъ страшно и жалостно будетъ слышать отвѣтъ сей хрістіанамъ, которые нѣчто лучше быть показывалися отъ невѣрныхъ, но \textit{часть ихъ съ невѣрными полагается}\footnote{Мѳ.~24,~51.}, которыи слышали въ словѣ Божіемъ о вѣчномъ животѣ, блаженствѣ, царствіи, славѣ, радости и прочіихъ вѣчныхъ благихъ, но лишатся тѣхъ, яко не искали оныхъ истинною вѣрою, но, осуетившися помышленіями своими, въ мірѣ семъ хотѣли богатѣть, славиться, утѣшаться, и, аще бы возможно было, вѣчно въ немъ царствовати! Да убоимся убо, возлюбленный хрістіанине, страшнаго онаго отвѣта и плачевнаго лишенія вѣчнаго живота. Послушаемъ Духа Святаго, чрезъ апостола насъ увѣщавающаго: \textit{не любите міра, ни яже въ мірѣ. Аще кто любитъ міръ, нѣсть любве Отчи въ немъ: яко все, еже въ мірѣ, похоть плотская, и похоть очесъ, и гордость житейская, нѣсть отъ Отца, но отъ міра сего есть. И міръ преходитъ, и похоть его: а творяй волю Божію пребываетъ во вѣки}\footnote{1~Іоан.~2,~15--17.}.

\paragraph*{§\:419.} Разсужденіе грѣха можетъ человѣка съ помощію Божіею отвести отъ грѣха. 1)~Разсуди, хрістіанине, откуду начало свое имѣетъ грѣхъ, и увидишь, что начальникъ и изобрѣтатель его есть діаволъ. Онъ первый есть отступникъ, который отъ Бога и Создателя своего со злыми своими аггелами отступилъ\footnote{Іуд.~6.} и прародителей нашихъ въ раи лукавствомъ своимъ отвратилъ и за собою привлеклъ\footnote{Быт.~3,~1--6.}. Прилично ли убо хрістіанину, водою и Духомъ отрожденному, въ завѣтъ со святымъ Богомъ вступившему, и во общеніе съ Нимъ и дружество со святыми ангелами позванному, грѣхъ творить, и тако діаволу, начальнику грѣха и противнику Божію, послѣдовать? О, сохрани насъ, Господи, отъ такого безумія! Дѣлаетъ же тое человѣкъ, который противу совѣсти и отъ произволенія грѣшитъ. Якоже бо истинные хрістіане подражаютъ Богу, яко чада возлюбленная, когда сообразуются Ему правдою, истиною, любовію, терпѣніемъ, кротостію, милосердіемъ, и проч.: тако нераскаянные грѣшники подражаютъ діаволу, и сообразуются ему злонравіемъ своимъ, ненавистію, злобою, лукавствомъ и прочіими богопротивными дѣлами. Мерзкій предъ Богомъ начальникъ грѣха, мерзкіи суть и послѣдователи его, бѣдные и окаянные человѣки. "--- 2)~Разсуждай и сіе, хрістіанине, что есть грѣхъ, который человѣку сладокъ показуется. Онъ есть отступленіе отъ Бога живаго и животворящаго; онъ есть измѣна, которою присягу Богу, въ крещеніи учиненную, нарушаетъ человѣкъ. Якоже бо Государю своему измѣняетъ, кто присягу, ему учиненную, не сохраняетъ и невѣрно ему служитъ: тако человѣкъ измѣняетъ Богу, когда по обѣщанію своему невѣрно работаетъ Ему, но противится Ему грѣхомъ, и въ слѣдъ сатаны идетъ, якоже о нѣкіихъ вдовицахъ апостолъ написалъ: \textit{се бо нѣкія развратишася въ слѣдъ сатаны}\footnote{1~Тим.~5,~15.}. Паки грѣхъ есть нарушеніе святаго, праведнаго и вѣчнаго закона Божія, который на тое отъ Бога данъ намъ, дабы его дѣло и со всякою опасностію хранили мы, якоже глаголетъ къ Богу Псаломникъ: \textit{Ты заповѣдалъ еси заповѣди Твоя сохранити зѣло}\footnote{Пс.~118,~4.}. Но грѣшникъ безстрашно дерзаетъ нарушать тое, что должно быть нерушимо во вѣки. Паки грѣхъ есть сопротивленіе святой и благой волѣ Божіей. Богъ бо хощетъ, чтобы мы уклонялися отъ зла и творили благое; но окаянный грѣшникъ противно тому дѣлаетъ: уклоняется отъ добра, и творитъ злое, и такъ противится Божіей волѣ. Паки грѣхъ есть огорченіе и раздраженіе вѣчныя Божія правды, которая, грѣхомъ раздраженная, судитъ грѣшника повиннымъ временной и вѣчной казни, всякій бо грѣхъ противу правды Божіей бываетъ. Паки грѣхъ есть преслушаніе и уничиженіе великаго, безконечнаго, неописаннаго, страшнаго, святаго и вѣчнаго Бога, Отца и Сына и Святаго Духа, предъ Которымъ ангели святіи со страхомъ стоятъ и благовѣютъ\footnote{Ис.~6,~2--3.}, но человѣкъ "--- земля и пепелъ "--- не боится Его. И сіе есть превеликая человѣческая слѣпота, что онъ Того, у Котораго какъ весь свѣтъ, такъ и онъ въ руцѣ есть, не боится, не почитаетъ, но презираетъ и уничижаетъ. Сего ради и самъ отъ Него презирается и оставляется, и уничижается, якоже глаголетъ Господь: \textit{уничижаяй Мя, безчестенъ будетъ}\footnote{1~Цар.~2,~30.}. Грѣхъ есть превеликая человѣческая къ Богу неблагодарность. Ибо тако человѣкъ, наипаче хрістіанинъ, Бога Создателя, Промыслителя и Искупителя своего, Котораго долженъ любить, не любитъ, котораго долженъ почитать, не почитаетъ, Котораго долженъ прославлять, уничижаетъ. Грѣхъ бо любить и Бога любить, грѣшить и Бога почитать невозможно. Истина сія безспорна. Грѣхъ наконецъ есть душевная проказа, которая смрадъ издаетъ и другихъ заражаетъ, и ни отъ кого очиститися не можетъ, какъ только отъ Іисуса Хріста, Цѣлителя душъ; и есть злѣйшій самаго демона, по свидѣтельству Златоустаго\footnote{Бес.~41"~я на Дѣян.}. И демона бо грѣхъ сдѣлалъ демономъ, который отъ Создателя своего добрымъ ангеломъ созданъ былъ. Страшно убо, человѣче, грѣшить, страшно Богу измѣнять и отступать отъ Него, страшно нарушать ненарушимый законъ Божій, страшно волѣ Божіей противиться, страшно правду Божію раздражать, страшно Бога безконечнаго не слушать и уничижать, страшно и неблагодарнымъ къ Богу"=Благодѣтелю быть. \textit{Страшно бо есть впасти въ руцѣ Бога живаго}\footnote{Евр.~10,~31.}. \textit{Ибо Богъ нашъ огнь есть поядаяй}\footnote{12,~29.}. "--- 3)~Видишь, хрістіанине, что есть грѣхъ: но смотри еще, что есть работать грѣху, \textit{яко всякъ, творяй грѣхъ, рабъ есть грѣха}\footnote{Іоан.~8,~34.}. Златоустъ святый, изображая скаредность грѣха и мерзкую тому работу, уподобляетъ его женѣ звѣрообразной, варварской, огнь дышущей, некрасной, черной\footnote{Бес.~9"~я на 1"~е Посл. къ Кор.}. Сему мерзкому чудищу разумное Божіе и образомъ Его Божественнымъ почтенное созданіе "--- человѣкъ работаетъ, повинуется, прилѣпляется и дѣлаетъ такъ, какъ тотъ, который, будучи царевъ сынъ и царскою багряницею одѣянъ, презрѣвши высокую царскую честь и свергши одѣяніе многоцѣнное, одѣвается въ скаредное рубище, "--- и, оставивши красную и многоцѣнными утварьми одѣянную свою жену, прилѣпляется мерзкой и безобразной блудницѣ, любитъ ее и угождаетъ ей. Тако дѣлаетъ несмысленный хрістіанинъ, который въ святомъ крещеніи сотворился сыномъ небеснаго Царя и одѣялся въ чистѣйшую правды Хрістовой багряницу, но, не разсудивъ высокія сея чести, и свергши съ себе прекрасную тую порфиру, одѣвается паки въ смрадное грѣховное рубище, и добродѣтель, какъ прекрасную и цѣломудренную дѣвицу, оставивши, прилѣпляется грѣховной похоти, какъ скаредной и всякими пороками замаранной женщинѣ, или паче, оставивши Хріста Сына Божія, краснѣйшаго добротою паче сыновъ человѣческихъ, Которому вѣрою обрученъ былъ, прилѣпляется къ скаредному мучителю, діаволу, и изъ сына Божія сыномъ веліаровымъ дѣлается. Плачевное позорище и ужасная слѣпота! Человѣкъ такъ въ высокомъ достоинствѣ учиненъ, особливымъ Божіимъ совѣтомъ созданъ, образомъ Его почтенъ, кровію Хрістовою падшій искупленъ, въ крещеніи омовенъ и обновленъ, къ вѣчному блаженству позванъ, "--- благороднѣйшее созданіе человѣкъ, ради котораго весь прекрасный сей міръ созданъ, согражданинъ ангельскій, сынъ Божій, наслѣдникъ небеснаго царствія, удъ Хрістовъ, жилище Духа Святаго, въ такую подлость нисходитъ самовольно, и все свое блаженство погубляетъ. О! когда бы бѣдному грѣшнику открылись сердечныя очи, и увидѣлъ бы плѣненіе свое, въ которомъ находится, и тяжкую и мерзкую свою работу, которою не человѣку, созданію Божію, но грѣху, всѣхъ золъ виновному злу и грѣхомъ діаволу неистовно работаетъ, "--- неутѣшно бы плакалъ и рыдалъ, и блаженнѣйшимъ бы паче себе почиталъ тѣхъ, которые въ заточеніи, въ темницахъ, во узахъ, въ плѣненіи, посмѣяніи и всякомъ озлобленіи у варваровъ находятся. Ибо сія бѣда тѣлесная и временная, оная же душевная и вѣчная, когда отъ той благодатію Хріста, Искупителя міра, не избавится. Душевное бо и вѣчное бѣдствіе несравненно большее паче тѣлеснаго и временнаго есть, яко сіе смертію кончится; оное же смертію не прекращается, но паче большее быть начинаетъ и во вѣки безъ конца будетъ. "--- 4)~Грѣхъ совѣсть человѣческую сильно уязвляетъ и мучитъ, пока истиннымъ покаяніемъ не очистится. Доколѣ человѣкъ грѣшитъ, аки спитъ совѣсть его, хотя и тогда не престаетъ ударять его; но когда онъ очувствуется, и пробудится совѣсть его, тогда познаетъ, коль тяжкое мученіе ея. Тогда онъ почувствуетъ сильное и страшное ея удареніе: \textit{нѣсть спасеніе} тебѣ \textit{въ Бозѣ} твоемъ\footnote{Пс.~3,~3.}. А наипаче тогда бываетъ грѣшнику бѣдному напасть сія, когда онъ примѣтитъ себѣ приближившуюся кончину. Аще бо благочестиво и свято пожившимъ великій подвигъ бываетъ при кончинѣ, какъ читаемъ въ церковной исторіи и нынѣ примѣчаемъ, "--- что уже будетъ тѣмъ, которые до кончины грѣшили? Страхъ суда Божія, геенны, тоска и отчаяніе! Тутъ сатана, который грѣшника Божіимъ единымъ \textit{милосердіемъ} прельщалъ, и такъ грѣшить научалъ, и въ грѣхѣ содержалъ, и въ томъ до кончины его велъ, уже Божіе \textit{правосудіе} ему предлагаетъ въ совѣсти, а тако въ отчаяніе лукавый духъ грѣшника ввергаетъ. Горе бѣдному человѣку будетъ въ часъ тотъ, который уязвленную совѣсть и не очищенную истиннымъ покаяніемъ заблаговременно имѣетъ. Сего ради всякому, кто имя Хрістово нарицаетъ, совѣтую, ради своего спасенія, о семъ часѣ страшномъ помянуть и заблаговременно отъ грѣха отстать, и сокрушеніемъ сердца совѣсть свою очистить, и всегда тотъ часъ поминать, да не тогда безполезно будетъ раскаяваться, сокрушаться, ужасаться, смущаться, тосковать и время къ покаянію искать. Надобно оставить прихоти, доколѣ онѣ насъ не оставятъ; надобно оставить суету, доколѣ она насъ не оставитъ. "--- 5)~Гнусными и страшными именами называетъ святое Божіе слово тѣхъ людей, которые безстрашно на грѣхъ дерзаютъ, и въ непокаяніи находятся. Называетъ \textit{сынами діавольскими: вы отца вашего діавола есте, и похоти отца вашего хощете творити}, глаголетъ Хрістосъ беззаконнымъ Іудеомъ\footnote{Іоан.~8,~44.}. \textit{Творяй грѣхъ отъ діавола есть, яко исперва діаволъ согрѣшаетъ}, глаголетъ Іоаннъ святый; и ниже глаголетъ: \textit{явлена суть чада Божія и чада діаволя}, гдѣ чадамъ Божіимъ, которыя грѣха не творятъ, противуполагаетъ чадъ діавольскихъ, которыя творятъ грѣхъ\footnote{1~Іоан.~3,~8--10.}. И паки: называетъ \textit{врагами Божіими. Прелюбодѣи и прелюбодѣйцы! не вѣсте ли, яко любы міра сего вражда Богу есть? иже бо восхощетъ другъ быти міру, врагъ Божій бываетъ}\footnote{Іак.~4,~4.}. Тоежде и во Псалмахъ видишь, и читаешь и на прочіихъ Писанія мѣстахъ. Гнусно быть, хрістіанине, сыномъ діавольскимъ. Безстыдно и страшно быть врагомъ Божіимъ. \textit{Безстыдно}: яко беззаконникъ Создателю и Благодѣтелю своему, Которому долженъ быть благодарнымъ, враждуетъ, разоряя законъ Его святый. \textit{Страшно}: яко Богъ праведный есть Судія, и воздастъ месть врагамъ Своимъ. \textit{Вѣмы бо рекшаго: Мнѣ отмщеніе, Азъ воздамъ, глаголетъ Господь}\footnote{Евр.~10,~30.}. "--- 6)~Безстыдно дѣлаетъ человѣкъ, когда беззаконнуетъ. Тако бо \textit{не предлагаетъ Бога предъ собою}\footnote{Пс.~53,~5; 85,~14.}, Котораго долженъ предъ очесами своими имѣть, почитать и покланяться. Обращаетъ хребетъ свой, а не лице къ Нему, якоже о беззаконныхъ Іудеяхъ глаголетъ Господь: \textit{обратиша хребетъ ко Мнѣ, а не лице}\footnote{Іер.~32,~33; 2,~27.}. Отвергаетъ словеса Божія вспять, якоже Самъ Богъ грѣшнику глаголетъ: \textit{отверглъ еси словеса Моя вспять. Аще видѣлъ еси татя, теклъ еси съ нимъ, и съ прелюбодѣемъ участіе твое полагалъ еси. Уста твоя умножиша злобу, и языкъ твой сплеташе льщенія; сѣдя на брата твоего клеветалъ еси, и на сына матери твоея полагалъ еси соблазнъ}. И ниже придаетъ Псаломникъ: \textit{разумѣйте убо сія забывающіи Бога, да не когда похититъ, и не будетъ избавляяй}\footnote{Пс.~49,~16--20 и 22.}. Толикое безстыдство дѣлаетъ человѣкъ, когда грѣшитъ противу Бога, а паче хрістіанинъ, который мнится знать Бога, \textit{но не почитаетъ Его}; имѣетъ законъ Божій, но не хощетъ исполнять его, и повергаетъ его въ задъ себе, котораго долженъ имѣть предъ собою и, на него смотря, творить его. Сего ради должно сіе всякому беззаконнику разсуждать и внимать, что прилагаетъ пророкъ: \textit{разумѣйте сія, забывающіи Бога, да не когда похититъ, и не будетъ избавляяй}. Похищаетъ беззаконниковъ праведный Божій судъ, \textit{и нѣсть избавляяй. Уже и сѣкира при корени древа лежитъ: всяко убо древо, еже не творитъ плода добра, посѣкаемо бываетъ, и во огнь вметаемо}\footnote{Матѳ.~3,~10.}. "--- 7)~Грѣхъ, стыдъ и срамъ предъ человѣками содѣловаетъ. Откуду блудники, татіе и прочіи грѣшники сокровенныхъ мѣстъ ищутъ къ совершенію беззаконныхъ дѣлъ своихъ. Такожде сребролюбцы, злобныи, лестцы, лукавцы и хитрецы всякимъ образомъ тщатся сокрыть страсти своя. Никто бо явно не хощетъ грѣшить. Откуду Хрістосъ глаголетъ: \textit{всякъ дѣлаяй злая, ненавидитъ свѣта, и не приходитъ къ свѣту, да не обличатся дѣла его, яко лукава суть}\footnote{Іоан.~3,~20.}. Причина тому сія есть, что явленнымъ грѣшникомъ всякъ гнушается. А которые явно дерзаютъ грѣшить, тіи не токмо страхъ Божій, но стыдъ человѣческій потеряли. "--- 8)~Ради грѣха всякія въ свѣтѣ казни, наказанія, бѣды и напасти бываютъ, какъ"=то: болѣзни, моровыя язвы, войны, брани, недороды хлѣба, скотскіе падежи и прочая. Аще бы грѣха не было, не было бы и напастей. Прочитай, хрістіанине, со вниманіемъ книги Ветхаго Завѣта, и увидишь истину сію. "--- 9)~За грѣхъ вѣчная мука и огнь вѣчный уготованъ, въ который непокаявшихся грѣшниковъ Хрістосъ, Судія праведный, пошлетъ: \textit{идите отъ Мене, проклятіи, во огнь вѣчный, уготованный діаволу и аггеломъ его}\footnote{Матѳ.~25,~41.}. Чимъ болѣе грѣшитъ человѣкъ, тѣмъ большая, тягчайшая ожидаетъ его вѣчная мука. \textit{О человѣче! или о богатствѣ благости Его и кротости и долготерпѣніи нерадиши, не вѣдый, яко благость Божія на покаяніе тя ведетъ? По жестокости же твоей и непокаянному сердцу, собираеши себѣ гнѣвъ въ день гнѣва и откровенія праведнаго суда Божія, Иже воздастъ коемуждо по дѣломъ его}, и проч.\footnote{Рим.~2,~4--6, и слѣд.} Убоимся убо, хрістіанине, суда Божія, обратимся къ Богу, пока время не ушло, ибо пріемлетъ Богъ обращающихся, и сотворимъ плоды достойны покаянія. Возненавидимъ грѣхъ нынѣ, да не тогда плода его горестнаго вкусимъ. "--- \textit{(Смотри еще статью 2"~ю первыя книги)}.

\subsection[Глава 11-я. Како хрістіанинъ можетъ себе утѣшать въ приключающихся скорбехъ?]{глава перваянадесять.\\\bfseries Како хрістіанинъ можетъ себе утѣшать въ приключающихся скорбехъ?}

\begin{quotation}\textit{Сыне Мой! не пренемогай наказаніемъ Господнимъ, ниже ослабѣй отъ него обличаемь. Егоже бо любитъ Господь, наказуетъ; біетъ же всякаго сына, егоже пріемлетъ. Аще наказаніе терпите, якоже сыновомъ обрѣтается вамъ Богъ: который бо есть сынъ, егоже не наказуетъ отецъ?} и проч.\footnote{Евр.~12,~5--7 и слѣд.}\end{quotation}

\paragraph*{§\:420.} Да возможеши, хрістіанине, въ случившейся скорби получить утѣшеніе печальному сердцу твоему, внимай слѣдующему разсужденію: 1)~Безъ всякаго сумнѣнія извѣстно есть, что истиннымъ хрістіанамъ безъ скорби въ мірѣ семъ быти невозможно. Тако бо свидѣтельствуетъ Божіе слово: \textit{многи скорби праведнымъ}\footnote{Пс.~33,~20.}; \textit{въ мірѣ скорбни будете}\footnote{Іоан.~16,~33.}; \textit{вси хотящіи благочестно жити о Хрістѣ Іисусѣ, гоними будутъ}\footnote{2~Тим.~3,~2.}. Ибо путь, который ихъ вводитъ въ животъ, \textit{тѣсенъ есть}\footnote{Матѳ.~7,~14.}. Что убо? Ты ли единъ хощеши безъ скорби пробыти, и съ тѣснаго пути на пространный, который ведетъ \textit{въ пагубу}\footnote{7,~13.}, взыти, и тако отъ числа истинныхъ хрістіанъ себе выключить? Прочитай священную отъ начала міра исторію, и увидишь, что вси святіи чашу горести крестныя пили, и нынѣ въ мірѣ странствующіи піютъ, и до конца міра будутъ пити. Довлѣетъ тебѣ къ утѣшенію твоему сіе, что \textit{сообщникъ еси} имъ \textit{въ печали}\footnote{Апок.~1,~9.}, что \textit{пріобщаешися Хрістовымъ страстемъ}\footnote{1~Петр.~4,~13.}, что \textit{нѣси странный и пришлецъ, но сожитель святымъ и присный Богу}\footnote{Еф.~2,~19.}. Малое ли тебѣ сіе есть "--- быти сыномъ церкве святыя, быть живымъ удомъ Хрістовымъ, здѣ Ему сообразнымъ быть въ страданіи, и въ будущемъ вѣкѣ въ славѣ? Сіе всякую горесть усладить можетъ, когда тое пріимешь въ разсужденіе. Единаго только того смотри, тщишися ли достойно ходити толикаго званія, имѣешися ли истинный хрістіанинъ. О семъ должно печалитися, но не отчаяватися, но истинно исправиться, каяться и вѣрою во Хріста утѣшаться. А скорбь тебѣ не повредитъ, не отъиметъ у тебе истиннаго блаженства, но паче умножитъ, какъ ниже увидишь. "--- 2)~Въ прилучившейся скорби помяни, колико претерпѣлъ нашего ради спасенія Хрістосъ, Господь нашъ. Онъ неповинно и насъ ради терпѣлъ: намъ ли, рабамъ Его, и рабамъ неключимымъ и достойнымъ всякія скорби, не терпѣть того, чего мы достойны? Послушай, что глаголетъ Господь: \textit{поминайте слово, еже Азъ рѣхъ вамъ: нѣсть рабъ болій Господа своего. Аще Мене изгнаша, и васъ изжденутъ}\footnote{Іоан.~15,~20.}. Когда добрѣ разсудиши, кто есть Хрістосъ, что, и ради чего, и ради кого такъ страшное мученіе и безчестіе претерпѣлъ, "--- великое во всякой скорби утѣшеніе получиши и всякую скорбь съ благодареніемъ и радостію пріимеши. "--- 3)~Помяни, что скорбію истинніи хрістіане сообразуются образу страданія Хрістова, да и въ славѣ сообразны Ему будутъ: \textit{съ Нимъ страждемъ, да и съ Нимъ прославимся}\footnote{Римл.~8,~17.}. \textit{Иже преобразитъ тѣло смиренія нашего, яко быти сему сообразну тѣлу славы Его}\footnote{Филип.~3,~21.}. \textit{Возлюбленніи! нынѣ чада Божія есмы, и не у явися, что будемъ. Вѣмы же, яко егда явится, подобни Ему будемъ}\footnote{1~Іоан.~3,~2.}. Воскресе Хрістосъ пострадавшій и умершій, воскреснутъ и хрістіане, уды Его, съ Нимъ страждущіи. Вознесеся на небо Хрістосъ, вознесутся и раби Его. Прославился Хрістосъ, прославятся и раби Его. Воцарился Хрістосъ, воцарятся съ Нимъ и терпящіи раби Его. \textit{Аще съ Нимъ умрохомъ, то съ Нимъ и оживемъ. Аще терпимъ, съ Нимъ и воцаримся}\footnote{2~Тим.~2,~11--12.}. О, коль великое утѣшеніе отсюду проистекаетъ, что страждущіи истинные хрістіане не токмо въ страданіи Хрісту, Господу своему и Царю, сообразуются, но и въ славѣ Ему сообразны будутъ! Кого въ скорби не ободритъ и не утѣшитъ надежда славы тоя? Разсуждай сіе, возлюбленный хрістіанине, чадъ Божіихъ достоинство и славу, и находящую скорбь срѣтай съ радостію, и нашедшую терпи съ благодареніемъ, яко она тебѣ духовное утѣшеніе приводитъ съ собою; хотя плоть твою и оскорбитъ, но духъ твой увеселитъ. Послѣдуй нынѣ Хрісту, да и тамо съ Нимъ будеши. Не стыдися нынѣ \textit{поношеніе Его носити}\footnote{Евр.~12,~13.}, да и съ Нимъ прославишися. Пей нынѣ \textit{оцетъ съ желчію смѣшенъ} со Хрістомъ\footnote{Матѳ.~27,~34.}, да сподобишися вѣчныя радости вино \textit{пити на трапезѣ Его небесной}\footnote{Лук.~22,~30; Ис.~65,~13.}. Отъ сего источника почерпали святіи мученики утѣшеніе, и на всякое мученіе, какъ на сладкій пиръ, спѣшили; ничего не отреклися, но всякое ужасное мученіе претерпѣли. Разсуждали они, коль великая честь "--- за Хріста, пострадавшаго за весь міръ, страдать, и тако сообразоваться Пострадавшему; и разсуждали апостольское утѣшительное слово: \textit{понеже пріобщаетеся Хрістовымъ страстемъ, радуйтеся, яко да и въ явленіе славы Его возрадуетеся веселящеся}\footnote{1~Петр.~4,~13.}, "--- и, поминая оную славу, утѣшали себе въ страданіи тѣмъ, \textit{яко недостойны страсти нынѣшняго времене къ хотящей славѣ явитися въ насъ}\footnote{Римл.~8,~18.}; \textit{и еже нынѣ легкое печали нашея, по преумноженію въ преспѣяніе тяготу вѣчныя славы содѣловаетъ намъ}\footnote{2~Кор.~4,~17.}. Послѣ малой скорби великая и безконечная радость, послѣ временнаго безчестія вѣчная слава, и послѣ маловременныхъ злыхъ \textit{вѣчная и непостижимая благая послѣдуютъ}\footnote{1~Кор.~2,~9.}. \textit{Мученики"=де за Хріста страдали, и потому какъ страданіемъ тѣмъ, такъ и надеждою будущія славы утѣшалися}. Правда; но и ты когда хрістіанинъ еси и какую скорбь ни терпишь, и терпишь не яко убійца, или яко тать или яко злодѣй, и проч., но яко хрістіанинъ, "--- \textit{Хрістовымъ страстемъ пріобщаешися}, потому и надеждою вѣчныя славы утѣшайся. Не токмо бо мученики, но и вси святіи скорбнымъ путемъ вошли въ небо. Нѣтъ бо инаго пути къ отечеству оному, кромѣ пути крестнаго; ибо \textit{многими скорбьми подобаетъ намъ внити въ царствіе Божіе}\footnote{Дѣян.~14,~22.}, и \textit{узкая врата и тѣсный путь вводяй въ животъ}\footnote{Матѳ.~7,~14.}. Надобно и намъ тѣмъ же путемъ идти, когда съ ними участіе имѣти хощемъ. Не скорби убо, любезный хрістіанине, но радуйся паче и благодари въ скорбѣхъ, что тако блаженнѣйшему оному избранныхъ Божіихъ собору, которые подъ знаменіемъ креста здѣ Хрісту Царю служили, и нынѣ живущіи на земли служатъ, причисляешися и Самому Хрісту послѣдуеши и сообразуешися. Сія есть примѣта избранныхъ Божіихъ, что \textit{ихъ міръ ненавидитъ}, и ненавидитъ не за какія злыя дѣла, но за тое, что, удаляяся отъ злыхъ его дѣлъ, усердно Господу своему работаютъ. \textit{Аще отъ міра бысте были}, глаголетъ имъ Хрістосъ, \textit{міръ убо свое любилъ бы: якоже отъ міра нѣсте, но Азъ избрахъ вы отъ міра, сего ради ненавидитъ васъ міръ}\footnote{Іоан.~15,~19.}. "--- 4)~Всякая скорбь не по случаю бываетъ, но скорбь намъ посылается отъ Бога, и есть Его отеческое наказаніе, которымъ наказуетъ насъ не отъ гнѣва, но отъ любви, якоже учитъ апостолъ: \textit{егоже любитъ Господь, наказуетъ; біетъ же всякаго сына, егоже пріемлетъ}\footnote{Евр.~12,~6.}. Якоже бо на нечестивыхъ изливаетъ гнѣвъ Свой праведный, и чашу гнѣва Его и ярости во вѣки будутъ пить; тако благочестивыхъ не яростію обличаетъ, ниже гнѣвомъ наказуетъ, но отечески біетъ, во гнѣвѣ милости поминаетъ, наказуетъ, но смерти не предаетъ, якоже поетъ пророкъ: \textit{наказуя наказа мя Господь, смерти же не предаде мя}\footnote{Пс.~117,~18.}, и Апостолъ глаголетъ: \textit{судими, отъ Господа наказуемся, да не съ міромъ осудимся}\footnote{Евр.~11,~32.}. Отъ начала міра до Хріста, и отъ Хріста до нынѣ ни единъ благочестивый не былъ безъ наказанія Господня отеческаго; но вси тое приняли отъ руки Господни, и приняли съ благодареніемъ: \textit{Емуже причастницы быша вси}\footnote{Евр.~12,~8.}. Знаешь, что и отецъ плотскій сына своего наказуетъ не отъ гнѣва, но отъ любви; кольми паче Богъ, естествомъ благъ, наказуетъ чадъ Своихъ отъ единой любви и милости. "--- \textit{Я"=де такую, или такую"=то бѣду пріемлю отъ людей?} "--- Правда то, что люди, діаволомъ научаемые, наводятъ намъ бѣды и скорби, но попущеніемъ Божіимъ: люди бо и діаволъ ничего не могутъ намъ сдѣлать, когда имъ Богъ не попуститъ. Попущаетъ же имъ въ наше наказаніе, и сколько имъ попущаетъ Богъ, столько они намъ и досаждаютъ и озлобляютъ. Извѣстна сія истина отъ исторіи о Іовѣ и Давидѣ святомъ и прочіихъ. Помяни убо, хрістіанине, въ скорби твоей утѣшительное слово: \textit{егоже любитъ Господь, наказуетъ; біетъ же всякаго сына, егоже пріемлетъ. Аще наказаніе терпите, якоже сыновомъ обрѣтается вамъ Богъ}\footnote{12,~6--7.}. И поминая благодари Ему, что и отъ тебе отеческія Своея любви и милости не отнимаетъ. Держи въ памяти всегда сіе, что далеко лучше всякое наказаніе здѣ терпѣть отъ руки Господни и во вѣки въ будущей жизни утѣшаться, нежели здѣ безъ наказанія быть и по смерти вѣчный Божій гнѣвъ терпѣть, котораго никакъ не избѣгнутъ ненаказанныи и неисправныи грѣшники. Примѣчай и сіе, что написалъ апостолъ: \textit{аще безъ наказанія есте, емуже причастницы быша вси, убо прелюбодѣйчищи есте, а не сынове}\footnote{Евр.~12,~8.}. Когда отецъ сына безъ наказанія оставляетъ, и попущаетъ ему по своей волѣ жить, то видно, что отринулъ его отъ себе: такъ когда и Богъ оставляетъ человѣка безъ наказанія, знаменіе есть отверженія его отъ Божіей милости, и не иное что такому слѣдуетъ, какъ вѣчный Божій гнѣвъ на себѣ терпѣть. Вси бо избранніи Божіи и чада Его причастницы быша, и нынѣ суть отеческаго Его наказанія. Желать убо должно, хрістіанине, отеческаго Божія наказанія, а не убѣгать отъ него, и о нашедшемъ радоватися и благодарить за тое небесному Отцу. "--- 5)~Аще и посылаетъ Богъ рабамъ Своимъ скорби, но не оставляетъ ихъ въ скорбѣхъ, но съ нимъ есть въ скорбѣхъ ихъ, якоже глаголетъ: \textit{съ нимъ есмь въ скорби}\footnote{Пс.~90,~15.}. Тако читаемъ о святомъ Іосифѣ, сынѣ патріарха Іакова, яко \textit{бѣ Богъ съ нимъ}\footnote{Дѣян.~7,~9.}. Хотя въ раба проданъ бысть Іосифъ, \textit{смириша во оковахъ нозѣ его, желѣзо пройде душа его}\footnote{Пс.~104,~18.}, однакожь былъ Богъ съ нимъ. Былъ и со святыми патріархами Авраамомъ, Исаакомъ и Іаковомъ, хотя и многія бѣды окружали ихъ. Откуду пишется, что \textit{не остави человѣка обидѣти ихъ, и обличи о нихъ цари: не прикасайтеся помазаннымъ Моимъ и во пророцѣхъ Моихъ не лукавнуйте}\footnote{ст.~14 и 15.}. Былъ съ потомками ихъ, людьми Израилевыми во Египтѣ, хотя и великое порабощеніе и озлобленіе терпѣли отъ беззаконнаго народа. Откуду Самъ Богъ глаголалъ къ Моисею: \textit{видя видѣхъ озлобленіе людей Моихъ, иже во Египтѣ}\footnote{Исх.~3,~7.}. Былъ съ ними и во исходѣ ихъ изъ Египта. \textit{Его видѣ море, и побѣже; Іорданъ возвратися вспять; горы взыграшася яко овни, и холми яко агнцы овчіи}, и пр. \textit{Отъ лица Господня подвижеся земля, отъ лица Бога Іаковля}\footnote{Пс.~113,~3,~4 и 7.}. Былъ съ ними и въ пустынѣ, якоже поетъ Моисей въ пѣсни своей: \textit{удовли его} (Израиля) \textit{въ пустыни, въ жажди зноя въ безводнѣ; обыде его, и наказа его, и сохрани его яко зѣницу ока. Яко орелъ покры гнѣздо свое, и на птенцы своя возжелѣ: простеръ крилѣ свои, и пріятъ ихъ, и подъятъ ихъ на раму своею}, и проч.\footnote{Второз.~32,~10,~11 и слѣд.} Былъ съ ними и въ землѣ, которую далъ имъ въ наслѣдіе. И хотя многія посылалъ имъ напасти, но не оставлялъ ихъ въ напастехъ ихъ, якоже читаемъ въ книгахъ Ветхаго Завѣта. Былъ съ тремя отроками въ пещи разженнѣй\footnote{Дан.~3,~49 и 92.}, якоже церковь поетъ Ему: «Въ пещь огненную ко отрокомъ еврейскимъ снизшедшаго, и пламень въ росу преложшаго Бога пойте дѣла» и проч.\footnote{Пѣснь 8"~я гласа 2"~го.} Былъ съ мучениками посредѣ пещи скорбей, мученій и страданій, былъ и съ прочіими святыми. Есть и нынѣ неотступно съ рабами Своими, чтущими Его, и до скончанія вѣка будетъ: \textit{се Азъ съ вами есмь во вся дни до скончанія вѣка}\footnote{Матѳ.~28,~20.}. Но когда слышиши, возлюбленный хрістіанине, что Богъ съ вѣрными рабами Своими есть въ скорбѣхъ, не просто разумѣй сіе Божіе бытіе, но есть съ ними, и сохраняетъ ихъ, помогаетъ имъ, утѣшаетъ ихъ, растворяетъ горесть крестную сладостію любве Своея, подая ко вкушенію благость Свою, якоже матерь малое дитя свое скорбящее и плачущее различно утѣшаетъ; и тако или облегчеваетъ тягость скорбей, или, когда нужно, совсѣмъ изымаетъ ихъ отъ пещи скорбей. Откуду благость сія и человѣколюбіе Божіе удивительно изображается въ Писаніи святомъ. Уподобляется \textit{орлу}, покрывающему гнѣздо свое и птенцы своя согрѣвающему, какъ видѣли выше въ пѣсни Моисеовой; уподобляется \textit{крову крилъ, селенію}, и проч.\footnote{Пс.~16,~8; 30,~21; 35,~8; 60,~5; 62,~8; 90,~1.}, и \textit{плещамъ} осѣняющимъ, якоже глаголетъ: \textit{плещма Своима осѣнитъ тя}\footnote{90,~4 и проч.}. Видиши ли, какое присутствіе Божіе съ рабами Своими въ скорбѣхъ ихъ! Помяни убо, что и ты въ скорбѣхъ твоихъ, когда терпиши ихъ безъ роптанія, пріобщаешися благости сея Божія и человѣколюбія, которое раби Его на себѣ дознаютъ; и будь доволенъ тѣмъ, что и съ тобою есть Богъ въ скорби твоей таковымъ же образомъ: \textit{нѣсть бо на лица зрѣнія у Бога}\footnote{Римл.~2,~11.}, но \textit{всякаго, уповающаго на Господа, милость обыдетъ}\footnote{Пс.~31,~10.}. Вѣдай и сіе, что нѣтъ тамо Бога, гдѣ радость и веселіе міра сего есть, когда люди радуются о богатствѣ, о чести, о славѣ, о роскошахъ, когда ликуютъ, пиршествуютъ, смѣхи чинятъ, танцуютъ, піянствуютъ, поютъ недостойная хрістіанамъ, кричатъ, и прочія непотребныя веселости производятъ. Отходитъ отъ таковыхъ людей Богъ, яко безчиніемъ ихъ оскорбляется; но приходитъ туды лукавый міра сего духъ, яко угодная ему дѣла тамо содѣваются. Приходитъ же Богъ къ скорбнымъ и печалію сокрушеннымъ сердцамъ, и, какъ въ сосуды праздные, изливаетъ живую прохлажденія и утѣшенія воду. \textit{Близъ бо Господь сокрушенныхъ сердцемъ и смиренныя духомъ спасетъ}\footnote{33,~19.}. Близъ Господь \textit{исцѣляяй сокрушенныя сердцемъ, и обязуяй сокрушенія ихъ}\footnote{Пс.~146,~3.}. Ибо \textit{жертва Богу духъ сокрушенъ: сердце сокрушенно и смиренно Богъ не уничижитъ}\footnote{50,~19.}. "--- 6)~Коль полезно и нужно намъ сіе отеческое Божіе наказаніе есть, не можно сказать. Разсуди, колико крыется въ сердцѣ нашемъ духовныхъ немощей, которыя въ плотскомъ рожденіи родилися съ нами: какая гордость, гнѣвъ, злоба, зависть, сребролюбіе, нечистота, и прочая! Сіи вси немощи всевидящее Божіе око видитъ, и наведеніемъ скорбей и напастей, яко жестокимъ врачевствомъ, хощетъ ихъ исцѣлить, аще Ему въ волю себе отдадимъ. Всякая бо скорбь и напасть учитъ насъ смиренію, терпѣнію, кротости и прочему добру. Кто въ болѣзни захощетъ злобиться и отмщевать? Кто въ нищетѣ, въ темницѣ, въ изгнаніи будетъ гордиться? Бѣда учитъ смиряться и все терпѣти. Бѣда подобна уздѣ, которою свирѣпые кони воспящаются и укрощаются. Естество наше свирѣпѣетъ и какъ бы бѣснуется, которое бѣдами и напастьми, какъ уздою, востягается и укрощается. Аще убо бѣда и скорбь нашедшая плоти нашей и горестна есть, но душѣ нашей здорова. Утѣшаемся и радуемся, когда лѣкарь немощное тѣло наше цѣлитъ, хотя горькимъ и жестокимъ врачевствомъ тое цѣленіе бываетъ, и благодаримъ ему, и мзду даемъ; кольми паче радоваться и отъ сердца Богу благодарить должно намъ, когда Онъ душевныя наши немощи скорбьми и напастьми цѣлитъ. Сколько бо душа отъ тѣла честнѣйшая, столько немощь ея опаснѣйшая, и исцѣленіе ея дражайшее намъ должно быть. Зналъ сію пользу Давидъ святый, и благодарилъ Богу, яко смирилъ его: \textit{благо мнѣ}, рече, \textit{яко смирилъ мя еси, яко да научуся оправданіемъ Твоимъ}\footnote{118,~71.}. Сего пророка святаго, который Духомъ Святымъ зналъ, коль великая польза происходитъ отъ отеческаго Божія наказанія, и потому благодарилъ Ему за тое, и намъ образъ къ тому подаетъ; и царю Израилеву, который, въ такъ великой будучи славѣ, не оскорбился, ни изнемогалъ наказаніемъ Господнимъ, и намъ должно подражать, и благодарить смиряющему насъ Богу нашему и \textit{учиться оправданіямъ Его, учиться плоть распинать со страстьми и похотьми}\footnote{Гал.~5,~24.}, \textit{да не царствуетъ грѣхъ въ мертвеннѣмъ нашемъ тѣлѣ, во еже послушати его въ похотѣхъ его}\footnote{Римл.~6,~12.}. Сего ради Іаковъ, святый апостолъ, утѣшаетъ вѣрныхъ: \textit{всяку радость имѣйте, братіе моя, егда во искушенія впадаете различна, вѣдяще, яко искушеніе вашея вѣры содѣловаетъ терпѣніе}\footnote{Іак.~1,~2--3.}. Отсюда и Павелъ святый съ вѣрными хвалится въ скорбѣхъ и глаголетъ: \textit{хвалимся въ скорбѣхъ, вѣдяще, яко скорбь терпѣніе содѣловаетъ, терпѣніе же искусство, искусство же упованіе; упованіе же не посрамитъ}\footnote{Римл.~5,~3--5.}. Смотри и примѣчай, хрістіанине, къ чему приводитъ скорбь: къ терпѣнію, искусству, упованію, которое не посрамитъ. Тако вѣра святая, повинующаяся Божіей волѣ и послѣдующая Начальнику вѣры и Совершителю Іисусу, ведетъ вѣрное сердце многими скорбьми, какъ степенями, къ концу своему "--- спасенію вѣчному, и сбывается писанное слово: \textit{вѣруяй въ Онь не постыдится}\footnote{1~Петр.~2,~6.}. Должно убо сердечно признать, что милосердый Богъ великое намъ дѣлаетъ благодѣяніе, когда насъ скорбьми и напастьми, аки жезломъ отеческимъ, наказуетъ. Самъ разсуди, хрістіанине, сколько таковыхъ имѣется, которые, въ благополучіи и веселости живучи, развратилися и, какъ кони, на своей волѣ разгулялися и разсвирѣпѣли, и погибли, что повседневно предъ глазами нашими обращается. Весьма мало благочестія въ богатыхъ и славныхъ міра сего примѣчается: гордость, скупость, самолюбіе, роскошь и едино только почти плотоугодіе видится, "--- что не къ иному чему, какъ къ погибели ведетъ: \textit{яко пространная врата и широкій путь, вводяй въ пагубу, и мнози суть входящіи имъ}\footnote{Матѳ.~7,~13.}. Тебе же до того милосердіе Божіе не допущаетъ и хощетъ тебе симъ скорбнымъ путемъ въ вѣчную радость ввести. Примѣчай, что благополучіе и злополучіе дѣлаетъ: Адамъ, праотецъ нашъ, въ благополучіи райскомъ согрѣшилъ тяжко, и отпалъ отъ Бога, и погибъ; сего ради выгнанъ изъ рая и преданъ за труды, скорби, бѣды и напасти, да паки покаяніемъ взыщетъ Господа\footnote{Быт.~3"~я.}. И Израильтяне, когда во Египтѣ терпѣли озлобленіе, искали Господа и воздыхали къ Нему, и услышалъ ихъ Господь, якоже къ Моисею сказалъ Богъ: \textit{стенаніе ихъ услышахъ}\footnote{Дѣян.~7,~34.}. Но когда въ пустынѣ на свободѣ были, развратилися и забыли Бога: \textit{и яде Іаковъ и насытися, и отвержеся возлюбленный: уты, утолстѣ, разширѣ: и остави Бога, сотворшаго его, и отступи отъ Бога Спаса своего}\footnote{Второз.~32,~15.}. Но когда въ бѣды и напасти паки впали, живучи въ землѣ обѣтованной, "--- паки обращалися къ Богу и искали Его, какъ читаемъ въ книгѣ Судей и въ прочіихъ книгахъ. Вотъ какъ благополучіе отводитъ отъ Бога, злополучіе же, какъ узда, приводитъ и привлекаетъ насъ къ Богу. «Великое убо добро есть скорбь», глаголетъ Златоустъ\footnote{Бес.~16"~я на Дѣян.}. Великое добро: яко смиряетъ насъ, яко пресѣкаетъ путь къ плотоугодію и погибели, яко исправляетъ насъ, ведетъ къ покаянію и обращаетъ къ Богу, и тако отводитъ отъ погибели, и поставляетъ на пути къ вѣчному животу. Великую и милость дѣлаетъ съ нами Отецъ небесный, когда посылаетъ на насъ скорби и напасти. О скорбь "--- жезлъ Божія наказанія, горькое, но здравое наше врачевство, училище смиренія, терпѣнія и кротости, знамя Хрістовыхъ воиновъ, путь къ вѣчному животу! \textit{Накажи насъ, Господи, обаче въ судѣ, а не въ ярости}\footnote{Iер.~10,~24.}. "--- \textit{Тяжко"=де и люто терпѣть скорбь?} "--- Правда; но далеко тягчае вѣчную терпѣти скорбь, когда отсюду отъидеши ненаказанъ и неисправленъ. Но нынѣшняго времени скорбь плоти нашей горестна есть, но плодъ ея сладокъ и благопріятенъ. \textit{Всякое бо наказаніе въ настоящее время не мнится радость быти, но печаль: послѣди же плодъ миренъ наученымъ тѣмъ воздаетъ правды}\footnote{Евр.~12,~11.}. И самъ ты, когда разуменъ еси и любишь дѣтей своихъ, не даешь имъ по своей волѣ жить, но учишь, наказуешь и біешь, чтобы исправны были, хотя и знаешь, что наказаніе тое имъ непріятно и горестно есть. А можетъ быть, что и отъ отца своего наказуемъ былъ и терпѣлъ, и благодарно поминаешь нынѣ наказаніе оное. Почто же Божіяго отеческаго наказанія не хощешь благодарно терпѣть? Знаешь, что польза отъ наказанія бываетъ, и наказуешь дѣтей своихъ: ради чего же отъ руки Господни наказанія, яко полезнаго и душеспасительнаго, не хощешь съ благодареніемъ, а по крайней мѣрѣ съ молчаніемъ и безъ негодованія принять? Ибо Господь наказуетъ насъ на пользу, \textit{да причастимся святыни Его, да живи будемъ}\footnote{10 и 9.}. И наказаніе сіе Его краткое и легкое: краткое, яко временное, и, хотя до конца продолжится, кончиною прекратится; легкое, яко наказуетъ Онъ и милуетъ, опечаляетъ и утѣшаетъ, оскорбляетъ и увеселяетъ, уязвляетъ и врачуетъ, и біетъ, но не умерщвляетъ. Что убо? Хощеши ли вмалѣ наказанъ быти, и по"=хрістіански пожити здѣ, и по смерти во вѣки жити, царствовати, утѣшатися, радоватися и вѣчно блаженнымъ быти? или безъ наказанія маловременнаго и легкаго быти, \textit{его же причастницы быша вси} и неисправнымъ отъ вѣка отъити, и тамо вѣчно погибнути, терпѣти и страдати? Сіе бо неисправнымъ неотмѣнно послѣдуетъ. Избирай отъ сихъ двухъ едино, что тебѣ угоднѣйшее "--- сіе, или оное; избирай, что хощешь, человѣче! А я оное избираю, и хощу послѣдовать немощнымъ, которые отдаютъ себе въ волю искуснаго лѣкаря, хотячи исцѣлитися отъ болѣзни. Тако и мнѣ должно отдаться, яко немощному, въ волю премудраго Врача душъ нашихъ "--- Господа, да поступаетъ со мною якоже хощетъ. Знаю я совершенно, что какое Онъ ни подастъ мнѣ лѣкарство, мнѣ оно полезно будетъ. Хощу съ пророкомъ исповѣдатися Ему: \textit{благо мнѣ, яко смирилъ мя еси, яко да научуся оправданіемъ Твоимъ}\footnote{Пс.~118,~71.}. \textit{Вѣрую видѣти благая Господня на земли живыхъ. Потерпи Господа, мужайся, и да крѣпится сердце твое, и потерпи Господа}\footnote{26,~13 и 14.}. Якоже бо Хрістосъ глаголетъ о Себѣ: \textit{не сія ли подобаше пострадати Хрісту и внити въ славу Свою}\footnote{Лук.~24,~26.}: тако апостолъ Его о всѣхъ вѣрныхъ глаголетъ: \textit{многими скорбьми подобаетъ намъ внити въ царствіе Божіе}\footnote{Дѣян.~14,~22.}. Не широкимъ и пространнымъ, но узкимъ и тѣснымъ путемъ туды входятъ. "--- \textit{(Смотри еще о семъ глав. о терпѣніи, благодареніи и узкомъ пути въ 1"~й книгѣ)}.

\paragraph*{§\:421.} \textit{Утѣшеніе противу духовнаго искушенія, которое бываетъ чрезъ помыслы, къ отчаянію склоняющіе}. Хотя и всякому хрістіанину можетъ сіе искушеніе приключиться, яко сатана, который таковое искушеніе наводитъ, никого не оставляетъ; однакожъ наипаче тому приключается, который во грѣхахъ не мало времени прожилъ, и, Божіею благодатію исправившися, началъ каяться. Коликое же и коль тяжкое бѣдствіе сіе бываетъ, тіи только знаютъ, которые страждутъ тое. Они непрестанно чувствуютъ волны, ударяющія въ души ихъ: \textit{нѣсть спасенія ему въ Бозѣ его}\footnote{Пс.~3,~3.}. Что тягчае человѣку можетъ быть, когда сатана хощетъ у него отнять вѣчное спасеніе? Отсюду востаетъ въ сердцѣ тако бѣдствующаго страхъ, ужасъ, печаль, тоска, смущеніе и всякое безпокойствіе, такъ что никогда не можетъ веселымъ быти. Всякое утѣшеніе внѣшнее, какъ ни утѣшаются люди, ему ничего не пользуетъ. Ищетъ утѣшенія, какъ алчный хлѣба и жаждущій питія, но не находитъ. Отъ чего бываетъ, что въ такомъ душевномъ страданіи и тѣломъ изнемогаетъ, блѣднѣетъ, сохнетъ и истаеваетъ. Страхъ бо и печаль, какъ огнь внутренній, пищу и питіе поядаетъ. Однакожъ сколько можетъ человѣкъ человѣку помощи, предлагаю тако страждущей душѣ изъ святаго Писанія утѣшительныя разсужденія, которыя могутъ нѣсколько прохладить, съ помощію Божіею, печальное сердце. 1)~Когда о грѣхахъ твоихъ помышляешь, помышляй и о милосердіи Божіи. Сколько твоихъ грѣховъ ни есть, и какъ они ни велики, но у Бога болѣе милости. Слыши, что апостолъ Духомъ Святымъ глаголетъ намъ грѣшнымъ въ утѣшеніе: \textit{Богъ богатъ въ милости}\footnote{Еф.~2,~4.}. Не токмо, рече, милостивъ Богъ, но и \textit{богатъ въ милости}. Сіе богатство милости проповѣдуетъ намъ и Моисей Боговидецъ: \textit{Господь, Господь Богъ щедръ и милостивъ, долготерпѣливъ и многомилостивъ и истиненъ}\footnote{Исх.~34,~6.}. Проповѣдуетъ Давидъ святый, узнавши на себѣ тое богатство милости: \textit{возвеличися до небесъ милость Твоя, Господи}\footnote{Пс.~56,~11.}. И паки: \textit{щедръ и милостивъ Господь, долготерпѣливъ же и многомилостивъ}\footnote{102,~8.}, и на прочіихъ мѣстахъ. Словомъ, милость Его толика, колико величество Его. \textit{Яко величество Его, тако и милость Его}, глаголетъ премудрый\footnote{Сир.~2,~18.}. Величество же Его безконечно, убо и милость Его безконечна есть. Отъ сего источника проистекаетъ, что Онъ \textit{не по беззаконіямъ нашимъ} творитъ намъ, \textit{ниже по грѣхомъ нашимъ воздаетъ намъ}\footnote{Пс.~102,~10.}, но по безконечному Своему милосердію \textit{погрузитъ неправды наша, и ввержетъ въ глубины морскія вся грѣхи наша}\footnote{Мих.~7,~19.}. Источникъ сей ни изчерпается, ни изсыхаетъ, но всегда исполненъ пребываетъ, и вси къ нему чрезъ пророка призываются жаждущіи: \textit{жаждущіи, идите на воду}\footnote{Ис.~55,~1.}. Едино требуется, да \textit{оставитъ нечестивый пути своя, и мужъ беззаконенъ совѣты своя, и да обратится ко Господу, и помилованъ будетъ}\footnote{ст.~7.}. Открываетъ его намъ Единородный Сынъ Божій: \textit{тако возлюби Богъ міръ, яко и Сына Своего Единороднаго далъ есть, да всякъ, вѣруяй въ Онь, не погибнетъ, но имать животъ вѣчный. Не посла бо Богъ Сына Своего въ міръ, да судитъ мірови, но да спасется Имъ міръ}\footnote{Іоан.~3,~16 и 17.}. Хвалятъ, поютъ и прославляютъ ангели: \textit{Слава въ вышнихъ Богу, и на земли миръ, въ человѣцѣхъ благоволеніе}\footnote{Лук.~2,~14.}, и \textit{желаютъ приникнути}\footnote{1~Петр.~1,~13.}. Славятъ и превозносятъ во всѣхъ своихъ писаніяхъ апостоли; почерпаютъ и предлагаютъ всѣмъ пастыри и учители; вкушаетъ и прохлаждается вся святая церковь; прибѣгаютъ нищіи, убогіи и жаждущіи грѣшники и получаютъ прохлажденіе. Къ сему благости и милосердія Божія источнику и намъ должно прибѣгать во время зноя, и жаждущимъ душамъ почерпать \textit{воду веселія}. На сіе бо истое и представляетъ намъ его святое Божіе слово, да во время зноя искушеній прибѣгаемъ къ нему и оттуду почерпаемъ живую прохлажденія воду. \textit{Елика бо преднаписана быша, въ наше наказаніе преднаписашася, да терпѣніемъ и утѣшеніемъ писаній упованіе имамы}\footnote{Римл.~15,~4.}. \textit{Многи"=де и тяжки мои грѣхи?} "--- \textit{Отвѣтъ}: но благость и милосердіе Божіе \textit{безконечно} есть, предъ которымъ вси твои грѣхи, какіе бы они ни были, какъ капля воды предъ цѣлымъ моремъ, и \textit{идѣже умножися грѣхъ, преизбыточествова благодать Іисусомъ Хрістомъ Господемъ нашимъ}\footnote{5,~20 и 21.}, "--- которыя благодати \textit{широта и долгота и глубина и высота} неизмѣрима и непостижима\footnote{Еф.~3,~18.}. "--- \textit{Какъ"=де мнѣ со святыми участіе имѣть, которые въ толикихъ добродѣтеляхъ пожили?} "--- \textit{Отвѣтъ}: Ты молись Господу съ разбойникомъ имѣть участіе, который при самомъ исходѣ разбойникъ былъ, и, съ жалѣніемъ и вѣрою испустивши ко Хрісту гласъ: \textit{помяни мя, Господи, егда пріидеши во царствіи Твоемъ}\footnote{Лук.~23,~42.}, прежде апостоловъ въ рай взошелъ и нынѣ со Хрістомъ въ раи есть, якоже рече ему Хрістосъ: \textit{днесь со Мною будеши въ раи}\footnote{Лук.~23,~43.}. Молись съ мытаремъ, блуднымъ сыномъ, блудницею, Манассіею и прочіими грѣшниками покаявшимися вчиненъ быть, которые нынѣ вси со Хрістомъ въ раи суть: и ты, когда въ вѣрѣ утвердишися, съ ними будеши и съ Самимъ Хрістомъ, послѣдовательно и со всѣми святыми. Идѣже бо Хрістосъ, тамо и вси святіи Его. Помяни, что и апостоли и вси святіи благодатію Хрістовою спаслися. И отъ апостоловъ многіи и великіи грѣшники были: Матѳей мытарь былъ, Петръ трижды отреклся Хріста, Павелъ гналъ Хріста и церковь Его. Но Богъ, \textit{богатъ сый въ милости}, изъ сихъ грѣшниковъ сдѣлалъ Евангелія Своего проповѣдниками, да какъ словомъ, такъ дѣломъ самымъ проповѣдуютъ \textit{неизчерпаемую} Его благодать всему міру; и которую сами отъ Бога получили милость, тую и всѣмъ грѣшникамъ ко внушенію предлагаютъ. Откуду и грѣховъ своихъ въ писаніяхъ своихъ исповѣдывать, и себе предъ всѣмъ міромъ за грѣшниковъ признавать не стыдилися. Павелъ святый называетъ себе \textit{бывшимъ хульникомъ и гонителемъ и досадителемъ, но помилованъ}, рече, \textit{быхъ, "--- и первымъ грѣшникомъ: Хрістосъ Іисусъ пріиде въ міръ грѣшники спасти, отъ нихже первый есмь азъ}\footnote{1~Тим.~1,~13 и 15.}. Матѳей святый, исчисляя имена апостоловъ, глаголетъ о себѣ: \textit{Матѳей мытарь}\footnote{Матѳ.~10,~3.}. Отверженіе Петрово вси святіи евангелисты описали; но вси помилованы и апостолами учинены, и тѣмъ помилованіемъ, какъ живымъ гласомъ, всему міру проповѣдуютъ, что есть грѣшникамъ кающимся надежда, отверсты двери милосердія Божія обращающимся, пріемлетъ Отецъ небесный обращающихся блудныхъ сыновъ; глаголетъ нынѣ Хрістосъ грѣшнику кающемуся о грѣхахъ: \textit{отпущаются тебѣ грѣси твои}\footnote{Лук.~7,~48.}, и отверзаетъ двери райскія разбойникамъ кающимся и Его Господемъ исповѣдующимъ, \textit{яко пріиде Сынъ человѣческій взыскати и спасти погибшаго}\footnote{19,~10.}. Нѣсть бо лицепріятія у Бога, но всѣхъ равно милуетъ, пріемлетъ и спасаетъ кающихся: тебѣ ли единому отречетъ? Сею толикою благостію Божіею, всѣмъ грѣшникамъ отверзтою, печальная душа, утѣшайся и утверждайся въ вѣрѣ. "--- 2)~Когда ты во грѣхахъ жилъ и грѣхами Бога прогнѣвлялъ, Богъ не хотѣлъ тебе, ради благости Своей, погубить, но терпѣлъ тебе, яко такимъ образомъ \textit{благость Его на покаяніе тебе вела}: нынѣ ли восхощетъ погубить тебе, когда ты престалъ отъ грѣховъ? Когда ты волѣ Божіей противился, миловалъ тебе Богъ: нынѣ ли не помилуетъ тебе, когда хощеши и тщишися волю Его творить? Когда ты отвращался отъ Бога, но Богъ Своимъ милосердіемъ отъ тебе не отвращался: нынѣ ли отвратится отъ тебе, когда ты къ Нему обращаешися? Когда ты Его оставлялъ и удалялся, но Онъ тебе не оставлялъ, но на покаяніе призывалъ и ожидалъ: нынѣ ли оставитъ тебе и удалитъ щедроты Свои отъ тебе, когда съ покаяніемъ приближаешися къ Нему? Не презрѣлъ тебе преогорчевающаго грѣхами: презритъ ли умилостивляющаго воздыханіемъ и слезами? Щедрилъ тебе небоящагося Его, сохраняя животъ твой отъ козней врага діавола, погибели твоей ищущаго, и душу твою во адъ свести хотящаго: нынѣ ли не ущедритъ боящагося? Помяни, что пророкъ поетъ: \textit{якоже щедритъ отецъ сыны, ущедри Господь боящихся Его: яко Той позна созданіе наше, помяну яко персть есмы}\footnote{Пс.~102,~13 и 14.}. Помяни, яко Богъ Отецъ нашъ есть: \textit{Ты еси Отецъ нашъ, понеже Авраамъ не увѣдѣ насъ, и Израиль не позна насъ, но Ты, Господи, Отецъ нашъ}, глаголетъ къ нему пророкъ\footnote{Ис.~63,~16.}, "--- и болѣе любитъ и милосердствуетъ о насъ, нежели матерь наша. \textit{Еда забудетъ жена отроча свое, еже не помиловати исчадія чрева своего? Аще же и забудетъ сихъ жена, но Азъ не забуду тебе, глаголетъ Господь}\footnote{49,~15.}. И не токмо просто Отецъ есть, но \textit{Отецъ щедротъ и Богъ всякія утѣхи}\footnote{2~Кор.~1,~3.}. Помяни, что поетъ Давидъ святый: \textit{Богъ спасеній нашихъ, Богъ нашъ, Богъ еже спасати}\footnote{Пс.~67,~20 и 21.}. Противно тому погубляти, Которому собственно есть спасати. Чуждое дѣло тому зло творити, Котораго естество есть благо творити. \textit{Никтоже бо благъ, токмо единъ Богъ}\footnote{Матѳ.~19,~17.}, то есть, естественно и собственно. Послушай, что Богъ о Своемъ благоволеніи къ намъ глаголетъ: \textit{еда хотѣніемъ восхощу смерти грѣшника, глаголетъ Адонаи Господь, а не еже обратитися ему отъ пути зла, и живу быти ему}\footnote{Іез.~18,~23.}. Что же многіи погибаютъ, тое бываетъ не отъ Его благоволенія, но отъ своего произволенія, яко не хотятъ покаянія творити. Едино убо отъ твоей стороны требуется, чтобы ты утвердился въ томъ, что Богъ и тебѣ кающемуся есть милостивъ, есть Отецъ, и Отецъ щедротъ, есть Богъ спасеній, есть Богъ еже спасати. "--- 3)~Помяни, что Хрістосъ во утѣшеніе намъ грѣшнымъ глаголетъ: \textit{пріиде Сынъ человѣчь взыскати и спасти погибшаго}\footnote{Лук.~19,~10.}. И апостолъ его: \textit{вѣрно слово и всякаго пріятія достойно, яко Хрістосъ Іисусъ пріиде въ міръ грѣшники спасти}\footnote{1~Тим.~1,~15.}. Аще Хрістосъ пріиде взыскати погибшихъ, взыщетъ и тебе, понеже и ты единъ отъ погибшихъ. Аще грѣшники спасти пріиде, то и тебе пріиде спасти, понеже и ты единъ отъ грѣшниковъ, ради которыхъ пріиде въ міръ. Ради грѣшниковъ въ міръ пріиде, ради грѣшниковъ на земли пожилъ, пострадалъ, умеръ, воскресъ, на небо вознеслся, сѣдитъ одесную Бога и ходатайствуетъ о грѣшникахъ, якоже Іоаннъ святый во утѣшеніе намъ грѣшнымъ глаголетъ: \textit{чадца моя! аще кто согрѣшитъ, Ходатая имамы ко Отцу, Іисуса Хріста Праведника: и Той очищеніе есть о грѣсѣхъ нашихъ, не о нашихъ же точію, но и о всего міра}\footnote{1~Іоан.~2,~1 и 2.}. Сей всемилостивый Ходатай нашъ всякую милость заслужилъ намъ у небеснаго Своего Отца и дерзновеніе подалъ намъ къ Нему приступать. Избавилъ насъ отъ враговъ нашихъ, отъ которыхъ мы своею силою избавиться не могли. Враги наши суть: грѣхъ, діаволъ, смерть и адъ. Одолѣли они намъ и торжествовали надъ нами, какъ побѣжденными; и никто отъ сихъ насъ не могъ избавить, никто за насъ не вступился, не было намъ помощника противу ихъ; но сынъ Божій сталъ за насъ, якоже глаголетъ Самъ чрезъ пророка: \textit{и воззрѣхъ, и не бѣ помощника; и помыслихъ, и никтоже заступи: и избави я мышца Моя, и ярость Моя наста: и попрахъ я гнѣвомъ Моимъ, и сведохъ кровь ихъ въ землю}\footnote{Ис.~63,~5 и 6.}. Онъ діавола, страшнаго исполина, побѣдилъ и связалъ, \textit{дѣла его разрушилъ}\footnote{1~Іоан.~3,~8.}; грѣхи вѣрующихъ въ Него кровію Своею загладилъ, смерть смертію Своею и воскресеніемъ попралъ и умертвилъ; адъ разорилъ, и тако плѣненіе душъ нашихъ возвратилъ; и возшелъ на небо, и \textit{есть одесную Бога, Иже и ходатайствуетъ о насъ}\footnote{Римл.~8,~34.}. И поемъ ему, яко Побѣдителю враговъ нашихъ и нашему Избавителю: \textit{возшелъ еси на высоту, плѣнилъ еси плѣнъ}\footnote{Пс.~67,~19.}. И сами нѣкогда, когда \textit{тлѣнное сіе облечется въ нетлѣнное, и смертное сіе облечется въ безсмертіе}, будемъ надъ сими врагами нашими торжествовать: \textit{пожерта бысть смерть побѣдою! Гдѣ ти, смерте, жало? гдѣ ти, аде, побѣда}\footnote{1~Кор.~15,~54 и 55.}? \textit{Душа наша чаетъ Господа, яко помощникъ и защититель нашъ есть: яко о Немъ возвеселится сердце наше, и во имя святое Его уповахомъ. Буди Господи милость Твоя на насъ, якоже уповахомъ на Тя}\footnote{Пс.~32,~20--22.}. Разсуди, за кого Хрістосъ умеръ? за грѣшниковъ. За кого грѣхъ, діавола, смерть и адъ побѣдилъ? за грѣшниковъ. За грѣшниковъ скорбѣлъ до смерти, ужасался и тужилъ, да ихъ отъ вѣчныя скорби, ужаса и туги избавитъ. За грѣшниковъ на судѣ беззаконномъ судимъ былъ, да ихъ отъ вѣчнаго суда свободитъ. За грѣшниковъ связанъ былъ, да ихъ узы растерзаетъ. За грѣшниковъ оплеванъ, обезчещенъ, поруганъ, посмѣянъ былъ, да ихъ отъ діавольскаго поруганія исхититъ. За грѣшниковъ уязвленъ, біенъ и мученъ былъ, да ихъ отъ вѣчнаго мученія избавитъ. За грѣшниковъ на древѣ крестномъ умеръ, да ихъ смертію Своею оживитъ. За грѣшниковъ изъ мертвыхъ восталъ, да ихъ оправдаетъ. \textit{Иже преданъ бысть за прегрѣшенія наша, и воста за оправданіе наше}\footnote{Римл.~4,~25.}. Когда вся сія за грѣшниковъ учинилъ Хрістосъ, то и за тебе, понеже и ты единъ отъ грѣшниковъ. Почтожъ убо хощешь лишить себе толикихъ и толь высокихъ Его благодѣяній, которыя всѣмъ грѣшникамъ кающимся и вѣрующимъ въ Него туне подаются? Пострадалъ, умеръ и воскресъ Хрістосъ за насъ: съ нашей стороны требуется тое, чтобы грѣхи оставить, каяться и вѣровать въ Него, и тако Хрістосъ будетъ наша правда, избавленіе, освященіе, побѣда, торжество, вѣчный животъ, слава и блаженство. \textit{Иже бысть намъ премудрость отъ Бога, правда же и освященіе и избавленіе}\footnote{1~Кор.~1,~30.}. "--- 4)~Читаемъ въ книгахъ Моисеовыхъ\footnote{Числ.~21,~6--9.}, что, когда угрызаеми были отъ зміевъ сыны Израилевы въ пустыни, Моисей, по повелѣнію Божію, сотворилъ змію мѣдяну, и вознеслъ тую на высоту, дабы угрызаемые люди взирали на тую, и тако бы отъ угрызеній исцѣлѣвалися, что и дѣлалося. \textit{И бысть}, глаголетъ Писаніе, \textit{егда угрызаше зміи человѣка, и взираше на змію мѣдяну, и оживаше}. Вознесеніе зміи оныя прознаменованіемъ было вознесенія Хрістова на крестъ, якоже Самъ Хрістосъ утѣшительно намъ о семъ протолковалъ: \textit{якоже Моисей вознесе змію въ пустыни: тако подобаетъ вознестися Сыну человѣческому, да всякъ вѣруяй въ Онь не погибнетъ, но имать животъ вѣчный}\footnote{Іоан.~3,~14 и 15.}. Якоже убо тогда люди, угрызенныи отъ зміевъ, взирали на вознесенную змію и исцѣлѣвалися: тако нынѣ угрызенные отъ змія адскаго жаломъ грѣховнымъ, когда очами вѣры взираютъ на Хріста, вознесеннаго на древо крестное, отъ угрызенія того исцѣлѣваются и оживляются. Взирай убо и ты вѣрою на распятаго Хріста, и исцѣлишися отъ язвъ грѣховныхъ и оживеши. Взирающимъ на Него вѣрою всѣмъ подается исцѣленіе и вѣчное спасеніе: тебѣ ли единому откажется тое отъ нелицепріемнаго и милосердаго Бога? \textit{Се есть Агнецъ Божій, вземляй грѣхи міра}\footnote{Іоан.~1,~29.}, въ которомъ мірѣ я и ты заключаемся. Какій твой грѣхъ такъ великій, тяжкій и ужасный можетъ быть, который бы отъ тебе, съ вѣрою къ Нему пришедшаго, сей Агнецъ Божій не взялъ? Какая твоя язва такъ велика есть, которую бы Онъ не исцѣлилъ! Какое твое огорченіе такъ сильно есть, котораго бы Онъ тебѣ, со смиреніемъ и вѣрою просящему, не оставилъ, Который за распинателей и ругателей Своихъ молился: \textit{Отче отпусти имъ}\footnote{Лук.~23,~34.}? Прочитай евангельскую книгу: кому Онъ милость и человѣколюбіе Свое отказалъ, Который на тое пришелъ, чтобы всѣмъ милость Свою явить? Кого отъ Себе отгналъ, кого отринулъ, Который всѣхъ пришелъ къ Себѣ призвать? \textit{Пріидите ко Мнѣ вси труждающіися и обремененніи, и Азъ упокою вы}\footnote{Матѳ.~11,~28.}. Блудницы, разбойники, мытари и прочіи грѣшники приходили къ Нему и получали милость, яко \textit{не пріиде призвати праведники, но грѣшники на покаяніе}\footnote{9,~13.}. Слѣпіи прозрѣніе, хромыи хожденіе, прокаженніи очищеніе, разслабленніи крѣпость, бѣсноватіи свобожденіе, глухіи слышаніе отъ Него получали: твоей ли душѣ не подастъ исцѣленія? Тѣлеса смертныя исцѣлилъ Онъ: душу ли безсмертную не исцѣлитъ, Который на то пришелъ, чтобы души наши, грѣхомъ умершія, оживить? Чимъ бо душа лучшая отъ тѣла, тѣмъ и исцѣленіе ея нужнѣйшее паче тѣлеснаго. Сего ради и Хрістосъ болѣе печется о душахъ, нежели о тѣлесахъ; и болѣе печется о насъ, нежели мы о себѣ, яко Творецъ и Искупитель нашъ. \textit{Всю убо печаль нашу возверземъ на Него, яко Той печется о насъ}\footnote{1~Петр.~5,~7.}. "--- 5)~Помяни въ утѣсненіи совѣсти твоея, что апостолъ во утѣшеніе наше написалъ: \textit{Иже} (Богъ) \textit{Сына Своего не пощадѣ: но за насъ всѣхъ предалъ есть Его: како убо не и съ Нимъ вся намъ дарствуетъ}\footnote{Римл.~8,~32.}, \textit{Иже преданъ бысть за прегрѣшенія наша, и воста за оправданіе наше}\footnote{4,~25.}? Разсуждай сію непостижимую Божію благодать. Ежели Богъ Сына Своего ради насъ грѣшныхъ не пощадѣлъ, то не пощадитъ ли ради Его насъ кающихся? Когда ради насъ предалъ Его, то не пріиметъ ли насъ, ради Его, съ покаяніемъ къ Нему приходящихъ? Когда за грѣхи наша предалъ Его, како не отпуститъ намъ кающимся и вѣрующимъ въ Него грѣховъ нашихъ ради Его? Когда ради того въ міръ Его послалъ, чтобы насъ отъ грѣховъ избавить, то не избавитъ ли насъ, ради Его, вѣрующихъ въ Него? Когда на безчестіе, раны, распятіе и смерть насъ ради грѣшныхъ (о непостижимыя благости Божія!) предалъ Его Богъ: то како не отверзетъ намъ двери милосердія Своего ради Его, двери къ вѣчному животу, славѣ и блаженству вѣрующимъ въ Него? Великое подалъ Богъ намъ: малое ли не подастъ? Великое и непостижимое благости Божіей дѣло есть, что Онъ благоволилъ воплотитися Сыну своему ради насъ и умрети: меньшее есть спасти насъ ради Его. Отъ воплощенія бо и смерти Хрістовой, какъ отъ живаго и приснотекущаго источника, все наше блаженство, отпущеніе грѣховъ, оправданіе, благодать, избавленіе отъ смерти и ада, вѣчный животъ и все проистекаетъ. Аще убо Сына Своего не пощадѣлъ, но за насъ предалъ Его: како съ Нимъ не подастъ намъ всѣхъ Своихъ благъ? На сіе послалъ Его въ міръ, чтобы насъ кающихся и вѣрующихъ въ Него спасти: како убо не спасетъ? Конецъ бо пришествія Хрістова въ міръ спасеніе наше; \textit{пріиде бо Сынъ человѣческій взыскати и спасти погибшаго}\footnote{Лук.~19,~10.}. \textit{Вѣрно слово и всякаго пріятія достойно, яко Хрістосъ Іисусъ пріиде въ міръ грѣшники спасти}\footnote{1~Тим.~1,~15.}. Аще убо тогда насъ помиловалъ, когда мы были враги Его: кольми паче нынѣ помилуетъ, когда смертію Сына Его примирилися Ему. \textit{Аще бо врази бывше примирихомся Богу смертію Сына Его: много паче примирившеся спасемся въ животѣ Его}\footnote{Римл.~5,~10.}; \textit{зане Богъ бѣ во Хрістѣ міръ примиряя Себѣ, не вмѣняя имъ согрѣшеній ихъ}\footnote{2~Кор.~5,~19.}. Мы согрѣшили Ему, но Онъ не токмо не отмстилъ намъ, но и помиловалъ насъ. Мы досадили Ему, но онъ не токмо не возненавидѣлъ, но и предварилъ насъ любовію Своею. Мы враги Его были, но Онъ не токмо не враждовалъ намъ (Богъ бо, яко благъ, или паче самая благость, никогда не враждуетъ), но и послалъ Сына Своего примирити насъ Себѣ, "--- насъ, враговъ Своихъ, Себѣ примирити. Пришелъ Сынъ и примирилъ насъ Отцу Своему, \textit{послушливъ бывъ даже до смерти, смерти же крестныя}\footnote{Филип.~2,~8.}. \textit{Вознесеся на небо Сынъ, и сѣде одесную Бога}\footnote{Марк.~16,~19.}. Послалъ апостоловъ, аки посланниковъ Своихъ, къ намъ и чрезъ нихъ насъ увѣщаваетъ, что бы мы примирилися Ему, которые вси въ писаніяхъ своихъ увѣщаваютъ насъ: \textit{по Хрістѣ молимъ, яко Богу молящу нами; молимъ по Хрістѣ, примиритеся съ Богомъ. Не вѣдѣвшаго бо грѣха по насъ грѣхъ сотвори, да мы будемъ правда Божія о Немъ}\footnote{2~Кор.~5,~20 и 21.}. Смотри и разсуждай благость и человѣколюбіе къ намъ Божіе. Хотя мы и враждуемъ Ему, разоряя святый и вѣчный законъ Его, яко же глаголетъ: \textit{иже нѣсть со Мною, на Мя есть}\footnote{Матѳ.~12,~30.}, однакожь Онъ въ благости и человѣколюбіи Своемъ въ намъ пребываетъ непремѣненъ; и не токмо непремѣненъ пребываетъ, но и хощетъ, чтобы мы враждующіи примирилися Ему, и ожидаетъ насъ. На что и Единороднаго Сына Своего послалъ, и по Немъ апостоловъ во всю вселенную разослалъ, и нынѣ возставляетъ пастырей и учителей, которые Словомъ Его святымъ увѣщаваютъ насъ, да обратимся, покаемся и примиримся Ему. Како убо Сей, Который хощетъ, ожидаетъ и какъ бы жаждетъ нашего обращенія, не пріиметъ обращающихся? Непремѣнно сотворитъ съ нами, якоже съ блуднымъ сыномъ сотворилъ благоутробный отецъ его; простретъ и намъ отеческая Своя объятія, и руками Своими, которыми сотворилъ и создалъ насъ, обыметъ насъ, и отверзетъ домъ церкви Своея намъ, и сопричтетъ насъ святымъ, домашнимъ Своимъ, и въ первую одежду облечетъ насъ\footnote{Лук.~15,~20--24.}, и повелитъ ангеламъ Своимъ радоватися о насъ, яко \textit{радость бываетъ предъ ангелы Божіими о единомъ грѣшницѣ кающемся}\footnote{ст.~10.}. "--- 6)~Богъ научаетъ насъ, како намъ и каятися подобаетъ, какъ въ пророческихъ и апостольскихъ писаніяхъ видимъ, и примѣры покаявшихся написати повелѣлъ; како Ему молитися и просити у Него прощенія, якоже о томъ Псалтирь, молитва \textit{«Отче нашъ»} и прочія пророческія молитвы свидѣтельствуютъ. Како убо не услышитъ кающихся, просящихъ и молящихся, Который образъ подалъ, како каятися, просить и молитися? И возбуждаетъ Духомъ Своимъ Святымъ въ сердцахъ нашихъ воздыханія и святыя желанія: како убо не услышитъ тѣхъ желаній и воздыханій нашихъ, которыя Самъ возбуждаетъ въ насъ? Вѣру ими, христіанине, что всякое желаніе, воздыханіе и слеза кающихся у Него изочтена и записана. \textit{Забываетъ Онъ грѣхи кающихся}\footnote{Іезек.~18,~22.}, но воздыханія и желанія ихъ не забываетъ: \textit{желаніе убогихъ услышалъ еси, Господи, уготованію сердца ихъ внятъ ухо Твое}\footnote{Пс.~9,~38.}. Воистину убогіи суть, которые не находятъ въ себѣ никакія правды, кромѣ грѣховъ, но желаютъ тоя отъ Бога, каковыхъ услышать обѣщалъ Господь. \textit{Азъ Господь Богъ, Азъ услышу ихъ Богъ Израилевъ, и не оставлю ихъ}\footnote{Ис.~41,~17.}. "--- 7)~\textit{Хрістосъ Іисусъ пріиде въ міръ грѣшники спасти}\footnote{1~Тим.~1,~15.}. Конецъ убо пришествія и страданія Хрістова вѣчное наше спасеніе есть. Коль благопріятно небесному Отцу отпущати грѣхи кающемуся грѣшнику, яко тогда плодъ страданія Хрістова получается! Плодъ бо страданія Хрістова есть отпущеніе грѣховъ, оправданіе и вѣчное наше блаженство. Ибо и Богъ на сіе истое Сына Своего послалъ въ міръ, чтобы \textit{грѣшники спасти}, пріемлющіе Его за Спасителя своего: какъ убо не съ радостію отпуститъ грѣхи какъ всѣмъ грѣшникамъ кающимся, такъ и тебѣ Отецъ небесный, когда видитъ, что оттуду плодъ страданія Сына Его происходитъ? Отъ отпущенія бо грѣховъ оправданіе, отъ оправданія спасеніе вѣчное послѣдуетъ. Сугуба радость бываетъ Отцу небесному оттуду, что грѣшникъ кается, отпущеніе грѣховъ получаетъ и спасается; ибо тогда исполняется воля Его святая, которая \textit{всѣмъ хощетъ спастися и въ разумъ истины пріити}\footnote{2,~4.}; и кровь Единороднаго Сына Его, ради грѣшниковъ спасенія изліянная, плодъ и конецъ свой получаетъ. Како убо Сей не отпуститъ намъ грѣховъ, Который Самъ желаетъ отпустить, когда мы истинно каемся и о томъ молимся Ему? "--- 8)~Некающимся ничего не пользуетъ страданіе и смерть Хрістова, хотя они и исповѣдуютъ имя Хрістово; но слѣдуетъ имъ судъ Божій и вѣчное во адѣ осужденіе, яко отвергшимъ и презрѣвшимъ великую Божію благодать. Благодать бо, аще и благодать есть, \textit{хотящихъ} спасаетъ, а не нехотящихъ. Кающимся же и вѣрующимъ во Хріста должно сею благодатію утѣшаться, и неотмѣнно ожидать милости Божіей и вѣчнаго блаженства, смертію Хрістовою отверстаго. "--- 9)~Хотя и извѣстно, что сіи душепагубныи помыслы, страхъ, печаль, утѣсненіе совѣсти и отчаяніе движущіи, отъ діавола бываютъ, однакожъ попущеніемъ Божіимъ. Діаволъ бо безъ попущенія Божія ничего не можетъ намъ дѣлать. Попущаетъ же Богъ таковое искушеніе, какъ и всякое, на насъ не на пагубу, но на великую пользу нашу, какъ сказано выше. Отсюду бо познаемъ силу и горесть грѣха, который такъ сильно уязвляетъ совѣсть нашу. Якоже бо змій уязвляетъ тѣло, которое уязвленное страждетъ, мучится и умираетъ: тако содѣянный грѣхъ уязвляетъ душу, отъ чего душа мучится и страждетъ, совѣстію снѣдаема. Откуду грѣхъ называется въ Святомъ Писаніи \textit{жало смерти}\footnote{1~Кор.~15,~56.}. Паки познаемъ гнѣвъ Божій противу грѣха, клятву законную, которою законопреступники поражаются\footnote{Гал.~3,~10; Рим.~4,~15; Еф.~5. 6.}. Познаемъ великое мучительство діавола, который попущеніемъ Божіимъ тако смущаетъ и мучитъ душу. А отъ сего учимся и убѣждаемся отъ грѣха опаснѣе берещися, по увѣщанію премудраго Сираха: \textit{якоже отъ лица зміина, бѣжи отъ грѣха. Аще бо приступиши къ нему, угрызнетъ тя. Зубы бо львовы "--- зубы его, убивающіи души человѣчи. Яко мечь обоюду остръ "--- всякое беззаконіе}\footnote{21,~2--4.}. Опасно бо блюдемся отъ той вещи, отъ которой вредъ себѣ узнали. Коль же таковое искушеніе сильно есть ко исцѣленію злонравія сердечнаго! Якоже бо огнемъ искушается влага, мокрота и гной смрадный, тако огнемъ таковаго искушенія изсушается влага нечистоты, злости, гордости, высокоумія, самолюбія, славолюбія, сребролюбія и прочіихъ похотей, которыя не иначе предъ Богомъ имѣются, какъ предъ нами гнойный смрадъ и всякая скверна. И якоже въ пещи искушается сребро и злато и чистѣйшимъ дѣлается: тако въ сей искушенія пещи очищается человѣкъ отъ злонравія. И хотя о всякомъ искушеніи сіе истинно есть, но паче о семъ духовномъ. Тогда бо познаетъ человѣкъ, что онъ въ себѣ есть, какъ немощенъ, бѣденъ, окаяненъ, и тако узнавши смиряется. Паки познаетъ, коль суетная есть всякая міра сего утѣха, и тако отъ всякой міра сего суеты отвращается. Наконецъ, узнавши бѣдность и окаянство свое, и ниоткуду не получая помощи, къ единому источнику блаженства и утѣшенія "--- Богу прибѣгаетъ, и ищетъ отъ Него всего своего блаженства. "--- 10)~Когда таковое искушеніе минуется, тогда преизобильно утѣшится душа печальная. Якоже бо по великой бурѣ и мрачныхъ дняхъ яснѣе намъ сіяетъ солнце и болѣе увеселяетъ насъ, тако по бурномъ семъ искушеніи сладчае почувствуется милость Божія. Тогда таковое сердце самымъ дѣломъ дознаетъ, что есть \textit{законъ}, который обличаетъ грѣхъ, устрашаетъ и клятвою поражаетъ грѣшника; и что есть \textit{Евангеліе}, которое уязвленное страхомъ и печалію сердца врачуетъ и утѣшаетъ. И якоже по банѣ малое отроча, или человѣкъ утружденный по трудахъ, сладко почиваетъ и упокоевается, или, якоже корабль, переплывши море, тихо стоитъ въ гавани: тако имѣется душа, претерпѣвшая и окончившая таковую бурную непогоду. Ожидай убо и ты таковыя тишины и покоя: неотмѣнно будетъ тебѣ, когда потерпиши. \textit{И еще бо мало елико, елико грядый пріидетъ и не укоснитъ}\footnote{Евр.~10,~37.}.

\paragraph*{§\:422.} \textit{Утѣшеніе противу искушенія}, которое бываетъ отъ діавола чрезъ злые и хульные помыслы. Сколько пакости и таковыми помыслами дѣлаетъ намъ врагъ нашъ, сколько ихъ, какъ стрѣлъ разженныхъ, пускаетъ на насъ на всякое время, день и часъ, сказать не можно! Отъ чего убогая душа такъ уязвляется печалію, что лучше изволяетъ терпѣть всякое поношеніе и біеніе внѣшнее, нежели таковыя вражія стрѣлы. Откуду апостолъ увѣщаваетъ насъ какъ во всякомъ, такъ и въ семъ искушеніи, противу врага онаго осторожными быть: \textit{трезвитеся, бодрствуйте, зане супостатъ вашъ діаволъ, яко левъ рыкая ходитъ, искій кого поглотити: емуже противитеся тверди вѣрою}\footnote{1~Петр.~5,~8 и 9.}. И Павелъ святый: \textit{прочее, братіе моя, возмогайте во Господѣ и въ державѣ крѣпости Его. Облецытеся во вся оружія Божія, яко возмощи вамъ стати противу кознемъ діавольскимъ}, и проч.\footnote{Еф.~6,~10,~11 и слѣд.} "--- Противу тайнаго сего искушенія внимай, печальная душа, слѣдующему разсужденію: 1)~Таковое искушеніе бываетъ попущеніемъ Божіимъ \textit{въ пользу нашу}. Какъ бо, видя врага находящаго на градъ, затворяемъ врата, и со всякимъ опасеніемъ хранимъ градъ, чтобы вшедши не разорилъ града и насъ бы не погубилъ, или въ полонъ отвелъ: тако, чувствуя душевнаго врага чрезъ такъ злые помыслы къ намъ приходъ, должно затворить домы душъ нашихъ, укрѣплять и опасно хранить, дабы внутрь не ворвался и не разорилъ душевнаго дома, "--- что бываетъ страхомъ Божіимъ и усердною молитвою. И отсюду видимъ, что онъ всегда готовъ есть на насъ напасть и погубить насъ; но столько нападаетъ, сколько сила Божія попущаетъ ему. А тако, и нехотя, поощряетъ насъ ко всегдашнему трезвенію и бдѣнію противу его. Какъ бо частое непріятелей нападеніе опаснѣйшими дѣлаетъ гражданъ, тако частое діавольское искушеніе опаснѣйшимъ и искуснѣйшимъ творитъ хрістіанина. Ибо въ безстрастіи, покоѣ и безопасности обычай людямъ есть лѣниться и ослабѣвать, но въ страхѣ и бѣдѣ осторожными бываютъ. И діавольское убо таковое искушеніе дѣлаетъ человѣка осторожнымъ, да всегда его нападенія ожидаемъ, и къ отраженію его пріуготовляемъ себе, и нашедшее съ помощію Божіею отразимъ. И тако хотя намѣреніе его злое есть, но когда мы \textit{трезвимся и бодрствуемъ}, Божіею помощію обращается намъ въ добро. Онъ мыслитъ на зло и погибель нашу, но Богъ \textit{поспѣшествуетъ} намъ \textit{во благое}\footnote{Римл.~8,~28.} "--- ему въ стыдъ и поношеніе, намъ же въ похвалу, яко духъ лукавый и сильный отъ немощнаго человѣка отражается и посрамляется. "--- 2)~Таковые помыслы не по волѣ, но противу воли нашей бываютъ; того ради въ грѣхъ намъ они не вмѣняются. Тое бо только въ грѣхъ намъ вмѣняется, чему соизволяемъ. Откуду и вредити душѣ нашей не можетъ, что бы оно ни было, отъ врага нашего того наносимое, пока тому не соизволяемъ, но противимся. Якоже бо непріятель, хотя и мещетъ стрѣлы своя на градъ, но не вредитъ граду, пока не сдастся ему градъ: тако и враждебный душъ нашихъ непріятель, діаволъ, хотя и мещетъ разженныя стрѣлы своя на душевный градъ нашъ, но ничего не успѣваетъ, пока злой волѣ его не соизволяемъ, но отражаемъ отъ себе козненные навѣты его. "--- 3)~Какъ во всякомъ искушеніи, такъ въ семъ наипаче посрамляется и со стыдомъ отходитъ онъ отъ искушаемаго, когда твердо противу его стоитъ и оплеваетъ его со злыми его навѣтами. Чимъ бо большее отъ него искушеніе бываетъ, тѣмъ большій ему стыдъ и посрамленіе послѣдуетъ, а искушаемому большая похвала и слава, когда борется и подвизается противу искусителя: яко таковымъ образомъ гордый и высокоумный духъ отъ смиреннаго и немощнаго человѣка побѣждается и посрамляется. Якоже бо исполину большій стыдъ, срамъ и поношеніе бываетъ, когда побѣжденъ бываетъ отъ малаго отрока, а не отъ равнаго себѣ: тако діавола всякая отъ него на человѣка наносимая брань въ посрамленіе приводитъ, когда не по злому его хотѣнію бываетъ. Ибо всякій человѣкъ, самъ въ себѣ разсуждаемый, есть предъ нимъ, какъ малое отроча предъ исполиномъ; но силою Божію укрѣпленъ, побѣждаетъ его, яко малый отрокъ великаго исполина, на посрамленіе его. А когда діаволу стыдъ и посрамленіе бываетъ, что немощнаго человѣка не можетъ побѣдить, то человѣку неотмѣнно похвала и честь послѣдуетъ. Но человѣкъ всю похвалу и славу отдавать долженъ Богу, яко не собственною своею силою, но Божіею укрѣпляется и посрамляетъ борителя. "--- 4)~Въ житіи преподобнаго Нифонта читаемъ: «Братъ"=де нѣкій пришелъ къ сему святому, злыми и хульными помыслами смущенъ, и въ скорби сей просилъ совѣта и утѣшенія. Отвѣщалъ ему отецъ: «пріими, брате, утѣшеніе по разуму моему: море егда воспѣнится, коликія возноситъ волны и біетъ о камень, волны же паки возвращаются въ море. Такожде и помыслы злые, отъ діавола находящіе, человѣческому приближаются смыслу. Егда убо человѣкъ, послушавъ его совѣта и соизволивъ, тому послѣдуетъ, погибаетъ, якоже и мнози тому повинувшися погибоша. Аще же кто боримый помысломъ хульнымъ не соизволяетъ, и паче мужается и противится тѣмъ, оплевая бѣса, то злоба его возвращается ему на главу, человѣкъ же той вѣнецъ пріемлетъ. И ты, чадо, терпи, противися бѣсу молитвою и постомъ, и бѣжати имать отъ тебе скоро» и проч.\footnote{См. декабря 23 дня.}

\paragraph*{§\:423.} \textit{Утѣшеніе противу клеветъ и злорѣчій человѣческихъ}. Между прочіими скорбьми и напастьми, которыя претерпѣваютъ хрістіане, немалое есть поношеніе и злорѣчіе отъ необузданнаго языка беззаконныхъ людей, которые злорѣчивымъ и ядовитымъ своимъ языкомъ, какъ мечемъ острымъ и стрѣлами, уязвляютъ того, на кого нападутъ, по реченному: \textit{сынове человѣчестіи, зубы ихъ оружія и стрѣлы, и языкъ ихъ мечь остръ}\footnote{Пс.~56,~5.}. И когда единаго руганіемъ насытятся, нападаютъ на другаго, и тако то на томъ, то на другомъ, то на третіемъ языки своя, какъ оружіе и мечи, руганіемъ острятъ. "--- Противу такихъ ядовитыхъ стрѣляній и язвъ, которыми не тѣло, но душа наша уязвляется, пріими, претерпѣвающая душа, слѣдующее утѣшеніе: 1)~Хотя и люди злорѣчивымъ и ядовитымъ языкомъ поносятъ тебе, и имя твое хульными терзаютъ зубами, однакожъ суть то діавольскія попущеніемъ Божіимъ козни. Онъ, когда видитъ, что человѣкъ противится ему и злому его совѣту не соизволяетъ, свирѣпѣетъ на него, и чего самъ собою не можетъ человѣкъ сдѣлать, тое чрезъ злыхъ людей, яко служителей своихъ и свое истое орудіе, совершаетъ. Примѣчай самъ, злословитъ ли кого истинный хрістіанинъ? Никакъ; не видно того и не слышно. Престаетъ уже хрістіанинъ быть хрістіаниномъ, когда уста своя отверзаетъ на хулу, поношеніе и клевету ближняго; яко всякій истинный хрістіанинъ страхъ Божій и любовь къ ближнему имѣетъ, которыми воспящается отъ клеветы и злорѣчія. Злословіе же есть свойство и примѣта безстрашныхъ и беззаконныхъ людей, находящихся подъ властію діавола, который ихъ научаетъ нападать на хрістіанина богобоящагося, и языкъ ихъ изощряетъ на поношеніе и хуленіе его. Сіе извѣстно есть какъ отъ священной, такъ и церковной исторіи. Да и нынѣ видимъ, что человѣкъ, доколѣ со злымъ міромъ дружится, не терпитъ таковыхъ язвительныхъ угрызеній; а какъ только благодатію Божіею начнетъ отъ міра убѣгать и Хріста любить и искать, тутъ на него какъ прочія бѣды, такъ и хульныя уста и языки злорѣчивые, какъ мухи на медъ, нападаютъ. И удивиться тому довольно не можно, откуду что возмется. Какъ вѣтръ то съ той, то съ другой стороны подымается и обуреваетъ и колеблетъ дерево, такъ сей бурный лукаваго и злобнаго духа вѣтръ то изъ тѣхъ, то изъ другихъ злорѣчивыхъ устъ востаетъ, вѣетъ и колеблетъ хрістіанскую душу. Сказываютъ, что море все мертвое извергаетъ вонъ: тоежъ дѣлается и хрістіанину, который міру умираетъ. И міръ, какъ море, изгоняетъ вонъ \textit{умершаго міру}, то"=есть, похотямъ, славолюбію, сребролюбію, гордости, сластолюбію, мщенію и прочіимъ. И, что дивнѣе того, самыи други врагами его сдѣлаются, якоже глаголетъ пророкъ: \textit{друзи мои и искренніи мои прямо мнѣ приближишася и сташа, и ближніи мои отдалече мене сташа; и нуждахуся ищущіи душу мою, и ищущіи злая мнѣ глаголаху суетная, и льстивнымъ весь день поучахуся}\footnote{Пс.~37,~12 и 13.}. «Кто ищетъ дружество имѣти со Іисусомъ, глаголетъ Іеронимъ, да знаетъ, что отъ многихъ вражду претерпѣвать будетъ онъ. Ибо душа, прилѣпившаяся къ Божію слову, безъ сумнѣнія тотчасъ будетъ имѣть враговъ, такъ что и тѣ, которыхъ прежде за друговъ имѣла, обратятся во враговъ»\footnote{Бес.~1"~я на Іис. Нав.}. Причина тому сія есть: понеже таковый чуждается и отрекается и исходитъ отъ міра сердцемъ, и \textit{горняя мудрствуетъ, а не земная}\footnote{Колос.~3,~2.}, и \textit{помышляетъ житіе свое на небесѣхъ быти, отонудуже и Спасителя ждемъ Господа нашего Іисуса Хріста}\footnote{Филип.~3,~20.}, и \textit{о семъ воздыхаетъ, въ жилище свое небесное облещися желая}\footnote{2~Кор.~5,~2.}, и потому въ мірѣ семъ, яко странникъ и пришлецъ есть, \textit{не имѣя здѣ пребывающаго града, но грядущаго взыскуя}\footnote{Евр.~13,~14.}. Того ради и міръ его, яко не своего, чуждается, ненавидитъ и гонитъ, что и Хрістосъ назнаменуя, глаголетъ апостоламъ: \textit{аще отъ міра бысте были, міръ убо свое любилъ бы: якоже отъ міра нѣсте, но Азъ избрахъ вы отъ міра, сего ради ненавидитъ васъ міръ}\footnote{Іоан.~15,~19.}. Симъ убо утѣшайся, душа хрістіанская, что между Хрістовыми рабами имѣешися, хотя и терпишь злорѣчивыхъ языковъ клеветы и поношенія, а не между чадами міра, которыя, яко міролюбцы, міру суть любезны, яко свои истыя чада. "--- 2)~Помяни, что симъ поношенія и уничиженія путемъ предшелъ намъ Самъ Хрістосъ, никакого грѣха не сотворивый. Сколько и какъ тяжко хулили Его фарисейскія уста, и поношенія, какъ ядовитыя стрѣлы, бросали на Него, Евангеліе святое свидѣтельствуетъ. Мало имъ было называть Его ядцею и винопійцею, другомъ мытарей и грѣшниковъ, Самаряниномъ и бѣса имущимъ и неистовымъ, Который всякимъ образомъ искалъ погибшихъ; но и льстецемъ, развратникомъ нарицали: \textit{сего обрѣтохомъ развращающа языкъ нашъ и возбраняюща кесареви дань даяти}\footnote{Лук.~23,~2.}, Который училъ ихъ: \textit{воздадите убо кесарева кесареви и Божія Богови}\footnote{Матѳ.~22,~21.}. И глаголаху: \textit{о князѣ бѣсовстѣмъ изгонитъ бѣсы}\footnote{9,~34.}, Который силою Божества Своего запрещалъ и изгонялъ демоновъ. Симъ путемъ шли и святіи Его. Никто отъ нихъ клеветы и поношенія не избѣжалъ. Сыскали чада міра и въ непорочномъ житіи что хулить; выдумалъ лживый языкъ, чимъ и безпорочныхъ порочить. Моисей пророкъ, законодавецъ, вождь Израилевъ, другъ и собесѣдникъ Божій, отъ сонмища Кореова и Авиронова претерпѣлъ укореніе\footnote{Числ.~16.} и отъ прочіихъ людей своихъ. Сколько на Давида святаго, царя Израилева и пророка Божія, бросали враги его ядовитыхъ отъ лживаго языка своего стрѣлъ, псалмы его показуютъ, "--- который глаголетъ: \textit{весь день поношаху ми врази мои, и хвалящіи мя мною кленяхуся}\footnote{Пс.~101,~9 и слѣд.}. Даніила пророка лживый языкъ въ ровъ ко львамъ, какъ въ гробъ, ввергнулъ\footnote{Дан.~6,~16.}. Что пострадали апостоли отъ всего міра, которому милость Божію проповѣдывали! Какъ прелестники, развратники и возмутители вселенныя вмѣнялися и укорялися, которые отъ прелести къ истинѣ, и отъ тьмы къ свѣту, и отъ царства діавольскаго къ царствію Божію обращали. Тоежде дознали на себѣ и преемники ихъ, святители, мученики и прочіи святіи. Читай церковную исторію, и увидишь, како никто отъ нихъ не ушелъ клеветы. Тоежде и нынѣ святымъ, въ мірѣ живущимъ, отъ злаго міра случается. Міръ бо въ злобѣ своей постояненъ есть: не любитъ истины, которою и словомъ и житіемъ показуютъ святіи, и всегда держится лжи и неправды, которую они обличаютъ. Не первый убо ты терпишь поношеніе и безчестіе. Видишь, что и святіи терпѣли, и нынѣ терпятъ. Но и Самъ Богъ сколько хулится отъ безбожныхъ и беззаконныхъ людей Своихъ на всякій день, на которыхъ солнце Свое сіяетъ, и дождитъ, когда иніи присносущное бытіе Его, иніи промыслъ Его отъ созданій Его отъемлютъ, иніи неправеднымъ, иніи немилостивымъ Его дерзаютъ называть; иніи ропщутъ и негодуютъ на Него; другіи слову Его святому не вѣрятъ, и тако \textit{лжа его творятъ}\footnote{1~Іоан.~5,~10.}, иніи иныя хулы отрыгаютъ, на святое и страшное имя Его! Кто же мы, которые не хощемъ терпѣть хулы? раби неключимые и всякаго поношенія и безчестія достойные!.. "--- 3)~Всему будетъ конецъ. Злорѣчіе и терпѣніе кончится; хулящіи и хулы терпящіи вси свое отъ правды Божіей воспріимутъ; хулящимъ хула обратится въ вѣчное поношеніе и срамоту, и поношеніе терпящимъ въ вѣчную славу, когда человѣцы воздадятъ отвѣтъ не токмо за хулу, но и за всякое праздное слово. \textit{Праведно есть у Бога воздати скорбь оскорбляющимъ васъ, а вамъ оскорбляемымъ отраду съ нами, во откровеніи Господа Іисуса съ небесе}, и проч., глаголетъ апостолъ\footnote{2~Сол.~1,~6,~7 и слѣд.}. Болѣе бо себѣ вредятъ злорѣчивіи и клеветники, нежели тому, кого хулятъ; ибо того имя и славу временно помрачаютъ, свои же души погубляютъ, "--- въ чемъ они сожалѣнія достойны. Что имъ по хрістіанской должности чинить? Они, по Хрістову словеси, должны \textit{благословить кленущія ихъ, и молитися за творящихъ имъ напасть}\footnote{Матѳ.~5,~44.}. "--- 4)~Когда многія клеветы, поношенія и укоренія падаютъ на тебе, и изнеможеши отъ злорѣчивыхъ языковъ, яко елень отъ псовъ гонимый: бѣги ради прохлажденія къ живому святаго Писанія источнику, и ищи отъ него прохлажденія. Тамо Богъ не тѣхъ ублажаетъ, которыхъ вси хвалятъ, паче же глаголетъ имъ: \textit{горе, егда добрѣ рекутъ вамъ вси человѣцы}\footnote{Лук.~6,~26.}. Но ублажаетъ тѣхъ, которые поношеніе терпятъ отъ злыхъ: \textit{блажени есте, егда поносятъ вамъ, и ижденутъ и рекутъ всякъ золъ глаголъ на вы лжуще Мене ради. Радуйтеся и веселитеся, яко мзда ваша многа на небесѣхъ}\footnote{Матѳ.~5,~11 и 12.}. Кто не прохладится, отъ необузданныхъ языковъ гонимый, когда толь великую мзду, имѣющую быть на небесѣхъ терпящимъ, разсудитъ? Кто не утѣшится, толь богатое слыша обѣщаніе, и отречется всякое временное безчестіе и руганіе охотно терпѣть? Добрая бо надежда всякую умягчаетъ скорбь, а паче надежда вѣчныя жизни, славы и веселія. Всякой скорби и безчестію нынѣшнему, хотя и чрезъ все житіе продолжится, конецъ смерть будетъ; но будущему веселію и славѣ конца не будетъ. Тогда забудетъ человѣкъ всѣ бѣды и напасти; едино утѣшеніе, радость и веселіе непрестанное будетъ имѣть безъ конца. \textit{Якоже аще кого мати утѣшаетъ: тако и Азъ утѣшу вы, и во Іерусалимѣ утѣшитеся и узрите, и возрадуется сердце ваше}\footnote{Ис.~66,~13 и 12.}. "--- \textit{Тая"=де мзда обѣщается Хріста ради терпящимъ?} Правда; но кто отъ насъ страждетъ не яко убійца, ни яко тать, или яко злодѣй и проч., но \textit{яко хрістіанинъ, да не стыдится, но да прославляетъ Бога въ части сей}\footnote{1~Петр.~4,~15 и 16.}. Убо и утѣшенія сего пріобщается, яко \textit{общникъ со святыми въ печали и въ царствіи и въ терпѣніи Іисусъ Хрістовѣ}\footnote{Апок.~1,~9.}. "--- \textit{Любящимъ Бога вся поспѣшествуютъ во благое}, глаголетъ Апостолъ\footnote{Римл.~8,~28.}. Имъ клевета и поношеніе въ пользу милостію Божіею обращается. Цѣломудреннаго Іосифа женская клевета въ темницу ввергнула; но тако онъ на высокую честь вознесенъ, и всю страну оную отъ голода спаслъ\footnote{Быт.~39 и 41.}. Моисей отъ злорѣчивыхъ устъ бѣжалъ изъ Египта, и былъ пришлецъ въ землѣ Мадіамской\footnote{Исх.~2,~15 и 22.}. Но тамо сподобился видѣть купину, чудесно горящую въ пустыни, и Бога изъ купины бесѣдующаго къ нему слышать\footnote{3,~2--7 и слѣд.}. Давиду святому много навѣта дѣлалъ злорѣчивый языкъ; но тако онъ къ молитвѣ возбуждался, и много Богодухновенныхъ псалмовъ въ пользу церкви святой сочинилъ. Даніила клевета въ ровъ вринула на снѣденіе львамъ; но неповинность заградила уста звѣрей, и прославила его паче прежняго\footnote{6,~16--28.}. Мардохея Израильтянина Амановъ языкъ умыслилъ убити, но Божіимъ промысломъ противное учинилося. Мардохей прославился: Аманъ на древѣ, которое къ погибели Мардохею уготовилъ было, повѣшенъ, и тако въ ровъ, который неповинному уготовалъ, самъ впалъ\footnote{Есѳ.~7.}. Тыежде суды Божіи и нынѣ показуются. Тойжде Богъ милостію Своею и нынѣ призираетъ на терпящихъ и хульный языкъ смиряетъ; возноситъ и нынѣ мытарей смиряющихся и милости отъ Него ищущихъ, и смиряетъ фарисеевъ, хвалящихся о своихъ исправленіяхъ: \textit{яко всякъ возносяйся смирится, смиряяй же себе вознесется}\footnote{Лук.~18,~14.}. Сего ради, уязвляемая клеветою и злорѣчіемъ беззаконныхъ людей душа, \textit{потерпи Господа, мужайся, и да крѣпится сердце твое, и потерпи Господа}\footnote{Пс.~26,~14.}. \textit{Уповай на Господа, и Той сотворитъ; и изведетъ яко свѣтъ правду твою, и судьбу твою яко полудне}\footnote{36,~5 и 6.}. Молчи, яко нѣмый, якоже Давидъ дѣлалъ: \textit{азъ яко глухъ не слышахъ, и яко нѣмъ не отверзаяй устъ своихъ, и быхъ яко человѣкъ не слышай, и не имый во устѣхъ своихъ обличенія: яко на Тя, Господи, уповахъ; Ты услышиши, Господи Боже мой}\footnote{37,~14--16.}. Дѣлай и ты такожде и Богъ возглаголетъ вмѣсто тебе. Якоже бо отецъ плотскій, когда видитъ предъ собою дѣтей отъ какого безчинника ругаемыхъ и обиду терпящихъ, и въ молчаніи на отца своего взирающихъ, тогда, вмѣсто ихъ, отецъ ихъ отвѣщаваетъ, глаголетъ и защищаетъ ихъ: тако Богъ, Отецъ небесный, поступаетъ съ нами и обиждающими насъ. Всякая бо обида и поношеніе, намъ наносимое, предъ Богомъ, яко вездѣсущимъ и вся назирающимъ, дѣлается. Когда убо видитъ Онъ, что мы обиждаеми и поношаеми терпимъ, молчимъ и къ Нему единому взираемъ, и предаемъ тое дѣло суду Его праведному, глаголя со пророкомъ: \textit{Ты услышиши, Господи Боже мой}: тогда Онъ вмѣсто насъ возглаголетъ, заступитъ и защититъ насъ и смиритъ востающихъ на насъ. Тако дѣлалъ Давидъ святый, который во всякихъ напастяхъ къ единому Богу прибѣгалъ, и къ Нему взиралъ, и помощи и защищенія искалъ отъ Него, якоже изъ псалмовъ его можеши видѣти. Послѣдуй и ты пророку сему, и, затворивши уста, молчи, да Богъ самъ вмѣсто тебе возглаголетъ. Когда тако въ молчаніи пребудеши постоянно, то поношеніе и уничиженіе людей не иное что, какъ похвалу и славу у Бога исходатайствуетъ тебѣ. Весь свѣтъ какъ ничто предъ Богомъ, убо и уничиженіе всего свѣта, не токмо нѣкоторыхъ злорѣчивыхъ, какъ ничто предъ славою, которую Богъ вѣрному Своему подаетъ рабу. Не тотъ убо блаженъ, кого люди, неправедніи судіи, хвалятъ, но тотъ, кого Богъ святый и праведный похвалитъ; якоже и окаяненъ не тотъ, кого люди уничижаютъ, но кого Богъ уничижаетъ.

\paragraph*{§\:424.} \textit{Утѣшеніе изгнанному отъ отечества и дома своего}. Сатана всякимъ образомъ старается озлобить благочестивыя души, которыя противу его подвизаются и презираютъ мерзкіе и богопротивные совѣты его; и то сіи, то другія напасти наводитъ на нихъ чрезъ служителей своихъ, злобныхъ людей. И сіе убо его хитростію и пронырствомъ бываетъ, что душа благочестивая изъ дома и отечества своего изгоняется, и любимыхъ своихъ друговъ и пріятелей лишается. Ибо ему, яко духу злому и гордому, какъ нибудь озлобить душу, противящуюся ему, радостно. Въ такомъ случаѣ изгнанная душа можетъ себе слѣдующими разсужденіями утѣшать: 1)~Безъ Божія изволенія и промысла ничто не бываетъ, какъ выше сказано, убо и сіе твое изгнаніе. Злоба человѣческая изгоняетъ тебе, но попущеніемъ Божіимъ. Ибо ежели бы Богъ не попустилъ тому быть, сатана и служители его ничего бы не могли тебѣ учинить. \textit{Не двѣ ли птицы цѣнятся единымъ ассаріемъ? И ни едина отъ нихъ падетъ на земли безъ Отца вашего. Вамъ же и власи главніи вси изочтени суть}, глаголетъ Хрістосъ\footnote{Матѳ.~10,~29 и 30.}. Аще убо о малыхъ вещахъ Богъ имѣетъ промыслъ Свой, кольми паче о человѣкѣ, изрядномъ и паче прочіихъ созданій лучшемъ. Что убо можетъ человѣку приключиться, паче же вѣрному, безъ промысла и воли Божіей? Богъ видитъ все, кто и ради чего и кого изгоняютъ: поручи убо промыслу Божію дѣло изгнанія твоего, и Той будетъ пещися о тебѣ. "--- 2)~Хрістіане суть \textit{пришельцы и странники въ мірѣ семъ}\footnote{Евр.~11,~13.}, и \textit{не имѣютъ здѣ пребывающаго града, но грядущаго взыскуютъ}\footnote{13,~14.}, и \textit{ждутъ основанія имущаго града, емуже художникъ и содѣтель Богъ}\footnote{11,~10.}. Того ради въ мірѣ семъ житіе ихъ есть, какъ плѣненіе и странствованіе. Слѣдственно равно есть имъ какъ въ отечествѣ и домѣ, гдѣ родилися и воспиталися, такъ и на чужой землѣ, въ которую изгнаны, или плѣнены, странствованіе свое провождать, пока Богъ изволитъ имъ въ мірѣ семъ странствовать. Потому и тебѣ равно есть жить нынѣ на всякомъ мѣстѣ, въ домѣ, или гдѣ на чужой сторонѣ. Отвсюду равно можешь очи свои возводить къ небу "--- отечеству своему, и живущему на небеси "--- небесному Отцу, и Онъ на тебе равно взираетъ съ небесе Своего святаго, какъ въ домѣ твоемъ и отечествѣ, такъ и на чужой странѣ живущаго, паче же благопріятнѣе зритъ и призираетъ на тебе во изгнаніи, яко изгнаннаго отъ человѣкъ и озлобляемаго, и, яко Отецъ милосердый, состраждетъ тебѣ, и вѣнецъ славы вѣчныя готовитъ тебѣ. "--- 3)~Всякое мѣсто, страна и градъ есть Божій: \textit{Господня бо земля и исполненіе ея, вселенная и вси живущіи на ней}\footnote{Пс.~23,~1.}. И Богъ нашъ на всякомъ мѣстѣ есть, и \textit{есть упованіе всѣхъ концевъ земли, и сущихъ въ мори далече}\footnote{64,~6.}. Убо на всякомъ мѣстѣ \textit{царствіе и владычество Его}\footnote{102,~22.}; и ты на всякомъ мѣстѣ въ царствіи Божіи обрѣтаешися, и Богъ съ тобою есть вездѣ. Якоже бо нечестивіи хотя въ домахъ своихъ упокоеваются, прохлаждаются и веселятся въ царствіи діавольскомъ, и во власти его темной обрѣтаются, и Богъ отъ нихъ съ Своею благодатію далеко есть: тако благочестивіи, гдѣ ни находятся, въ царствіи Божіи суть, и Богъ съ ними неотступно есть, \textit{съ ними есть въ скорбѣхъ ихъ}\footnote{Пс.~90,~15.}. Съ Богомъ же быти, на всякомъ мѣстѣ отечество и домъ есть; якоже безъ Бога домъ и отечество есть ссылка и плѣненіе. Съ Богомъ въ бѣдахъ и страданіяхъ и въ самомъ адѣ добро; безъ Бога и самый рай и небо ничтоже есть. Ибо гдѣ Богъ, тамо и царствіе Божіе, рай, небо и все блаженство. Богъ бо есть сокровище и источникъ утѣшенія, радости, веселія и всякаго истиннаго блаженства, и безъ Него не можетъ быть истинное утѣшеніе, истинная радость и веселіе, истинный животъ и блаженство. Когда сіе добрѣ разсудишь, не будешь скучать, яко удалился ты отъ отечества и дома: но вездѣ будешь имѣть утѣшеніе. "--- 4)~Помяни, что святіи Божіи изгнаны были. Іосифъ святый завистію братіи своея изгнанъ былъ: \textit{въ раба проданъ бысть Іосифъ, смириша въ оковахъ нозѣ его, желѣзо пройде душа его}\footnote{104,~17 и 18.}. Давидъ отъ Саула\footnote{1~Цар.~8--26 и слѣд.} и сына своего Авессалома\footnote{2~Цар.~15--17.} изгнанъ былъ, и скитался по чужимъ странамъ, и во всякомъ бѣдствіи и злостраданіи былъ. Даніилъ и тріе отроки съ прочіими въ Вавилонѣ плѣнены были отъ Іерусалима\footnote{Дан.~1.}. Въ новой благодати апостоли, мученики, святители и прочіи святіи различныя изгнанія претерпѣли. Тоежде и нынѣ бываетъ. Міръ бо, яко море волнами, свирѣпостію своею изгоняетъ и отвергаетъ отъ себе святыхъ, яко \textit{мертвыхъ ему}, но живыхъ Богу. И ты, изгнанный, \textit{общникъ еси въ печали съ ними}\footnote{Апок.~1,~9.}, и лучше тебѣ съ ними скорби терпѣть, нежели съ неблагодарнымъ и злымъ міромъ веселитися, яко съ ними же утѣшеніе воспріимеши во царствіи небесномъ. Вотъ скоро избавленіе пріидетъ, и \textit{отыметъ Богъ всяку слезу отъ очей нашихъ}\footnote{7,~17.}. Помяни и сіе, что Самъ Сынъ Божій и Господь нашъ къ намъ, изгнаннымъ изъ рая и страннымъ и плѣненнымъ, самоизвольно пришелъ, и съ нами бѣдными бѣдствовать, и странными странствовать, и проданными подъ грѣхъ проданъ, преданъ, связанъ, веденъ быти на судъ и смерть и умрети благоволилъ. Смотри на сіе позорище, и получишь утѣшеніе. "--- 5)~\textit{Отъ отца"=де я и матери, отъ братіи, сродниковъ и друговъ любезныхъ удаляюся?!}. Правда; прискорбно отъ любимыхъ удаляться. Но отъ Бога удалиться нигдѣ не можешь, Который есть вмѣсто отца и матери вѣрнымъ, или паче болѣе отца и матери, Который глаголетъ тебѣ: \textit{аще и забудетъ исчадія жена, но Азъ тебе не забуду}\footnote{Ис.~49,~15.}. \textit{И аще проходиши сквозѣ воду, съ тобою есмь, и рѣки не покрыютъ тебе: и аще сквозѣ огнь пройдеши, и не сожжешися, и пламень не опалитъ тебе}\footnote{43,~2.}. Сіе узналъ на себѣ Давидъ святый: \textit{аще и пойду посредѣ сѣни смертныя, не убоюся зла; яко Ты}, Боже, \textit{со мною еси}\footnote{Пс.~22,~4.}. И паки: \textit{отецъ мой и мати моя остависта мя: Господь же воспріятъ мя}\footnote{26,~10.}. Други твои вѣрніи и любезніи суть ангели Божіи, которые невидимо \textit{ополчаются окрестъ боящихся Господа}\footnote{33,~8.}. Братія твоя суть святіи, которые прежде такожде изгнаніе претерпѣли, и нынѣ на подвигъ твой смотрятъ и радуются о терпѣніи твоемъ. Се колико любимыхъ имѣешь во изгнаніи твоемъ! Богъ помощникъ, защититель и утѣшитель твой съ тобою; ангели хранители тебѣ; братія твоя суть святіи, которымъ въ печали ихъ бывшей \textit{общникъ} еси. Ктомужъ совѣсть твоя увеселеніе твое, внутренній и всегдашній твой свидѣтель и утѣшитель. "--- 6)~Хрістосъ ублажаетъ изгнанныхъ правды ради: \textit{блажени изгнани правды ради: яко тѣхъ есть царствіе небесное}\footnote{Матѳ.~5,~10.}. Сія тебѣ надежда паче отца и матере, братіи и друговъ отечества и дома. Симъ утѣшайся, яко твое есть царствіе небесное; нынѣ вѣрою и надеждою, потомъ наслѣдіемъ безъ конца будетъ твой Богъ, Котораго \textit{нынѣ яко зерцаломъ въ гаданіи видиши, тогда лицемъ къ лицу увидиши}\footnote{1~Кор.~13. 12.}. Твои други святіи вси будутъ, и все блаженство вѣчное твое будетъ, когда вѣру до конца соблюдеши. Ничто бо такъ не утѣшаетъ человѣка въ напасти всякой, какъ надежда вѣчнаго живота. Внимай только себѣ, терпи съ благодареніемъ, и попусти волѣ Божіей на тебѣ совершитися, безъ которой ничто не можетъ намъ приключитися.

\paragraph*{§\:425.} \textit{Утѣшеніе страждущему въ болѣзни}. Многіи благочестивіи страждутъ въ болѣзняхъ различныхъ: иной въ ранахъ гніетъ, иной въ разслабленіи лежитъ, другой сохнетъ и плотію день отъ дня истаеваетъ, иной другою болѣзнію мучится. Всякому убо, въ какой бы ни находился болѣзни, слѣдующее, по моему убогому разуму, утѣшеніе предлагаю, которое, страждущая душа, прилагай къ язвамъ печальнаго сердца твоего. 1)~Различнымъ болѣзнямъ или недугамъ различныя причины полагаютъ лѣкари; но я, имъ тое въ разсужденіе оставляя, повторяю, что выше сказано, то"=есть, что и болѣзнь не безъ промысла бываетъ Божія, и есть отеческое наказаніе Божіе, которымъ Онъ смиряетъ насъ. Симъ убо утѣшайся, что болѣзнь твоя есть тебѣ отеческое Божіе наказаніе, которымъ тѣло твое біетъ, да душа твоя оздоровѣетъ и спасется. По тебѣ убо болѣзнь твоя, а не противу тебе, хотя плоти твоей и горестна; и всякое бо наказаніе плоти горестно, но душѣ полезно. Аще убо съ благодареніемъ терпишь болѣзнь твою, обратится тебѣ она во благое. "--- 2)~Многіи святіи были въ недугахъ и болѣзняхъ. \textit{Іовъ отъ главы до ногъ язвою пораженъ былъ} и страдалъ\footnote{Іов.~2,~7 и 8.}. Тимоѳей святый апостолъ \textit{частые недуги} имѣлъ\footnote{1~Тим.~5 и 23.}. Лазарь \textit{гноенъ былъ} и по смерти несенъ ангелы на лоно Авраамле\footnote{Лук.~18,~20 и 22.}. Такожде читаемъ и въ церковной исторіи, что многіи благочестивіи различными болѣзньми страдали. И ты убо терпи съ благодареніемъ болѣзнь, да съ ними въ будущемъ вѣкѣ участіе будеши имѣти. "--- 3)~Болѣзнь хотя тѣло и разслабляетъ, но душу укрѣпляетъ; тѣло умерщвляетъ, но душу животворитъ; внѣшняго человѣка растлѣваетъ, но внутренняго обновляетъ. \textit{Аще и внѣшній человѣкъ тлѣетъ, обаче внутренній обновляется по вся дни}\footnote{2~Кор.~4,~16.}. Какъ \textit{обновляется}? "--- Научается смиренію, терпѣнію, памяти смертной, и отъ той усердному покаянію, молитвѣ, презрѣнію міра и суеты мірской. Кто восхощетъ въ болѣзни гордиться? Кто, видя приближающуюся чрезъ недугъ кончину, пожелаетъ чести, славы, богатства? Кто безстрашно дерзнетъ грѣшить, видя наступающій страхъ суда Божія? Когда усерднѣе человѣкъ молится, какъ въ болѣзни? О, болѣзнь, горькое, но здоровое лѣкарство! Какъ соль гнилость отъ мяса и рыбы отвращаетъ и не попущаетъ зараждатися червямъ въ нихъ: тако всякая болѣзнь сохраняетъ духъ нашъ отъ гнилости и тлѣнія грѣховнаго, и не попущаетъ страстямъ, какъ червямъ душевнымъ, зараждатися въ насъ. Всякъ сію истину и на себѣ дознаетъ, и на другихъ видитъ. Видишь убо, что болѣзнь хотя плоть твою мучитъ, но духъ твой спасаетъ. "--- 4)~Помяни, коль многіи, въ цѣломъ здравіи живучи, развратилися и погибли. Богъ милосердый чрезъ недугъ не допущаетъ тебе до того, и хощетъ тебе спасти, когда благодарно претерпиши. Терпи убо и благодари Богу, ищущему спасенія твоего. "--- 5)~Смерть заключитъ страданіе твое, хотя и чрезъ все житіе болѣзнь продолжится, однакожъ кончится смертію. Всему бо бѣдствію нашему конецъ есть смерть. Симъ утѣшай себе въ болѣзни твоей, что она окончится, а оттуду къ вѣчному преселишися прохлажденію, гдѣ забудеши всякое злостражденіе. Мало потерпишь, но вѣчно утѣшаться будешь, когда съ благодареніемъ потерпишь. Всякое и долговременное страданіе нынѣшнее что есть, аще не единъ малѣйшій пунктъ въ сравненіи съ вѣчнымъ? И все бо время отъ созданія міра до конца не иное что есть, какъ малый пунктъ противу вѣчности, кольми паче единаго человѣка житіе и вѣкъ. Страшно вѣчное страданіе, яко лютое и безконечное; но временное не страшно, яко краткое и съ прохлажденіемъ бываетъ; нѣтъ бо здѣ никакого страданія, которому бы Богъ не подалъ какого либо утѣшенія. Ибо знаетъ Онъ немощь нашу, того ради и \textit{не попускаетъ намъ искуситися паче силы нашей}\footnote{1~Кор.~10,~13.}. Пріими убо въ разсужденіе время и вѣчность, временное и вѣчное страданіе, временное утѣшеніе и вѣчное страданіе, временное страданіе и вѣчное утѣшеніе, и облегчится болѣзнь твоя. Ничто бо такъ не облегчеваетъ скорби, какъ надежда избавленія отъ скорби и полученія вѣчнаго утѣшенія. Вотъ скоро будетъ конецъ всему: будетъ конецъ и утѣшенію временному, и временному страданію, и настанетъ или прохлажденіе, или страданіе вѣчное. Смерть всякому есть дверь къ вѣчности, или блаженной, или неблагополучной, и уже ближе тебѣ она сегодня, нежели вчера и третьяго дня была. А какъ придетъ, то и положитъ конецъ страданію твоему, и отъ временнаго страданія прейдеши къ вѣчному прохлажденію. "--- 6)~Послушай, наконецъ, что святый Златоустъ о благодарномъ въ болѣзни терпѣніи глаголетъ: «Нѣтъ ничего языка онаго святѣйше, иже въ злыхъ благодаритъ Бога: воистину отъ мучениковъ ничимъ не разнствуетъ; тако и сей, якоже они, вѣнчается. Ибо и сему предстоитъ мучитель, хуленіемъ понуждая, отрещися Бога, предстоитъ діаволъ, мучительными помыслы удручая, печальми помрачая. Аще убо кто терпитъ болѣзни и благодаритъ, мученическій вѣнецъ пріемлетъ»\footnote{Бесѣд.~8"~я на посл. къ Колос.}. Разсудивши убо святителя Хрістова златое ученіе, потерпи, да со святыми Божіими прохладишися во царствіи небесномъ. \textit{Потерпи Господа, мужайся, и да крѣпится сердце твое, и потерпи Господа}\footnote{Пс.~26,~14.}.

\paragraph*{§\:426.} \textit{Утѣшеніе противу нищеты}. Мало, весьма мало богатыхъ обрѣтается, которые бы были истинными хрістіанами. Не богатство тому виновно, но злое сердце человѣческое, которое или неправдою собираетъ богатство, или правдою собранное неправедно расточаетъ, или хранитъ его какъ сторожъ. Ктомужъ едва сыщется какій богачь, который бы о богатствѣ своемъ не гордился и не превозносился тѣмъ, хотя то не его оно есть, но Божіе. Все бо Божіе есть, что ни имѣемъ, кромѣ грѣховъ. Неправда же, роскошь, скупость и гордость много съ собою вводятъ беззаконій, и лишаютъ хрістіанина хрістіанства истиннаго. Истинное бо хрістіанство не можетъ быть безъ хрістіанскаго житія, которое должно отъ внутренности происходить. И такъ богатство, Божіе добро, безумному богачу, какъ безумному мечь, бываетъ, которымъ самъ себе закалаетъ и погубляетъ, что и о всякомъ Божіемъ дарованіи разумѣть должно. Всякое бо дарованіе, когда человѣкъ его не на добро, но на зло употребляетъ, обращается ему во зло и погибель, напр. ученіе, разумъ, краснорѣчіе, художество, знаніе языковъ и самый даръ чудотворенія, и проч.\footnote{1~Кор.~13,~1--3; Матѳ.~7,~22 и 23.} Можетъ бо человѣкъ злоупотреблять Божіими дарованіями. Тогда же то бываетъ, когда человѣкъ чрезъ дарованія не славы Божіей и пользы ближняго, на что они подаются, но славы и похвалы своея ищетъ. Тако и богатый, когда не славы Божіей и пользы ближняго чрезъ богатство ищетъ, но или хранитъ его какъ сторожъ, или какъ песъ сѣно, на которомъ лежитъ, и самъ не ястъ и скоту не даетъ, или расточаетъ на непотребные расходы, дабы или плоти своей угодить, или отъ человѣкъ славу и похвалу себѣ снискать, и проч., "--- злѣ употребляетъ дарованія Божія, и такъ злѣ употребляемое дарованіе Божіе во зло и ему обращается. А когда нѣтъ хрістіанскаго расхода въ богатомъ, то нѣтъ въ немъ и истиннаго хрістіанства. Очень же мало такихъ богачей, которые по"=хрістіански расходъ богатства своего чинятъ, какъ сіе примѣчается отъ дѣлъ ихъ; слѣдственно мало истинныхъ хрістіанъ сыскать можно между богатыми. По большей убо части истинные хрістіане въ нищетѣ и убожествѣ живутъ. Причина тому сія есть: понеже истинные хрістіане болѣе пекутся о богатствѣ душевномъ и вѣчномъ, нежели тѣлесномъ и временномъ; или того, которое отъ Бога имъ посылается добро, лишаются отъ злыхъ богачей, по Писанію: \textit{не богатіи ли насилуютъ вамъ, и тіи влекутъ вы на судища? не тіи ли хулятъ доброе имя нареченное на васъ}\footnote{Іак.~2,~6 и 7.}? Или сами расточаютъ богатство свое, якоже пишется: \textit{расточи, даде убогимъ}\footnote{Пс.~111,~9.}, "--- и прочія причины, Богу единому извѣстныя. "--- Которые убо хрістіане терпятъ нищету, и отъ той скучаютъ, слѣдующими разсужденіями могутъ себе утѣшать. 1)~Какъ прочее бѣдствіе, якоже выше сказано, такъ и нищета отъ Господа попущается на насъ. \textit{Благая и злая, животъ и смерть, нищета и богатство отъ Господа суть}\footnote{Сир.~11,~14.} и на добрый конецъ посылается намъ, какъ ниже увидишь. "--- 2)~Нищетою человѣкъ отъ всѣхъ тѣхъ золъ, которыми богатство окружается, удобно можетъ свободитися. Гордость, скупость, роскошь и прочая симъ подобная злая въ нищетѣ почти не имѣютъ себѣ мѣста; нищета бо смиряетъ всякаго, и щедрымъ и роскошнымъ быть нищему не отъ чего. При богатыхъ сіи хрістіанскія язвы находятся, и тѣхъ сердца поражаютъ. Отъ попеченія о храненіи богатства, которое за богатымъ неотлучно ходитъ и содержитъ сердце его, нищій свободенъ есть; нищій не боится татей и хищниковъ, которымъ богатство подвержено; тля нищеты не тлитъ, ни моліе растлѣваютъ; отъ зависти, которая на богатство непріятно смотритъ, удалена нищета; нищій отвѣта не готовитъ за расточеніе богатства, за что богатіи истязани будутъ. Нищета въ праздности и лѣности жить не попущаетъ, якоже обычай есть богатымъ, но понуждаетъ дѣлать. «Нищета, глаголетъ Златоустъ, не праздность, но трудолюбіе раждаетъ»\footnote{Бес.~2"~я на посл. къ Ефес.}. Видишь, коликихъ золъ и бѣдъ нищета свобождаетъ тебе! Да будетъ убо нищета твоя во утѣшеніе тебѣ, а не въ оскорбленіе и негодованіе, яко она отводитъ тебе отъ гордости, скупости, роскоши, пышности, тщеславія, печали, страха, зависти людской, праздности, лѣности, нѣжности и прочихъ душевныхъ золъ, и возводитъ тебе на путь хрістіанскій, тѣсный, и руководствуетъ къ небеси. «Руководство нѣкое, глаголетъ Златоустъ, пути ведущаго къ небеси нищета, помазаніе страдальческое, обученіе нѣкое великое и чудное, пристанище благоутишное»\footnote{Бес.~18"~я на посл. къ Евр.}. Что и Хрістосъ назнаменуя, глаголетъ: \textit{неудобно богату внити въ царствіе небесное}\footnote{Матѳ.~19,~23.}. Аще богату неудобно, удобно есть нищему благодатію Хрістовою внити въ царство небесное. "--- 3)~Нищій и богатый равно дни провождаютъ и живутъ: ни богатому богатство, ни нищему нищета ни отнимаетъ, ни придаетъ дней и житія; равно богатою и убогою пищею подкрѣпляется, равно богатымъ платьемъ и рубищемъ покрывается, равно въ хижинѣ и свѣтлыхъ чертогахъ упокоевается немощная плоть; равно и житіе свое и тотъ другій оканчиваетъ, но только съ тѣмъ разнствіемъ, что богатый какъ живетъ, такъ и кончится съ печалію и страхомъ, "--- нищій безъ печали и страха, равный наконецъ и богатаго и нищаго гробъ заключаетъ, и въ смерти и по смерти богатый съ нищимъ равняется. И когдабъ на всемірномъ судѣ равно судими были! Но нѣтъ, не будетъ того. \textit{Емуже бо дано много, много взыщется отъ него}\footnote{Лук.~12,~48.}. Слѣдуетъ богатому дать отвѣтъ Судіи о богатствѣ своемъ, како и на что богатство, данное ему отъ Бога, держалъ и расходъ чинилъ, какъ и \textit{всякъ о талантѣ своемъ воздастъ Господу отвѣтъ}\footnote{Матѳ.~25,~19--30.}. Нищій о томъ не печется: не имѣетъ богатства, не готовитъ и отвѣта о расходѣ богатства. Видишь, что ни богатство не придаетъ, ни нищета не отнимаетъ истиннаго блаженства отъ насъ, паче же дѣлаетъ насъ блаженнѣйшими отъ богатыхъ. Ибо истинное блаженство не состоитъ во внѣшнемъ видѣ, но въ покоѣ душевномъ, отъ котораго богатство отводитъ, но нищета къ тому руководствуетъ, какъ выше сказано. Что бо пользуетъ человѣку внѣ показываться, красоваться и блистать, но внутрь попеченіемъ, печалію и страхомъ, какъ яблоку красному червемъ, снѣдаться? Сіе есть богатыхъ блаженство! Напротивъ того, ничего не вредитъ быть внѣ смиреннымъ, умаленнымъ, презрѣннымъ и уничиженнымъ, но внутрь имѣть сокровище безопасности, покоя и мира. Сіе есть нищихъ состояніе! Не тотъ убо спокоенъ, слѣдственно и блаженъ, кто много имѣетъ, но кто ни о чемъ не печется: къ сему нищета ведетъ насъ. Сіе пріими въ разсужденіе, хрістіанине, и нищеты не бойся, яко она по тебѣ, а не противу тебе есть, побораетъ тебѣ, а не боретъ тебе. "--- 4)~Въ нищетѣ твоей помяни о вольной нищетѣ Хріста Сына Божія. \textit{Вѣсте благодать Господа нашего Іисуса Хріста, яко васъ ради обнища богатъ сый, да вы нищетою Его обогатитеся}\footnote{2~Кор.~8,~9.}, "--- Иже отъ убогія Матери Дѣвы родился, и \textit{положенъ былъ во яслѣхъ, зане не бѣ имъ мѣста во обители}\footnote{Лук.~2,~7.}; Иже \textit{во своя пріиде, и свои Его не пріяша}\footnote{Іоан.~1,~11.}; Иже \textit{не имѣлъ гдѣ главу подклонити}\footnote{Матѳ.~8,~20.}, и убогихъ учениковъ и Апостоловъ имѣлъ, и повелѣлъ имъ нищету любити: \textit{не стяжите злата, ни сребра, ни мѣди при поясѣхъ вашихъ}, и проч.\footnote{Матѳ.~10,~9 и слѣд.} Откуду Павелъ святый о себѣ и прочіихъ Апостолахъ глаголетъ: \textit{яко нищи, а многихъ богатяще; яко ничтоже имуще, а вся содержаще}\footnote{2~Кор.~6,~10.}. И Петръ святый къ нищему хромому глаголалъ: \textit{сребра и злата не имамъ}\footnote{Дѣян.~3,~6.}. Вольная убо Хрістова нищета и святыхъ Апостолъ твою отъ нищеты скорбь да умягчитъ, и утѣшитъ сердце твое. "--- 5)~Помяни, что странникъ еси и пришлецъ въ мірѣ семъ, путникъ еси на пути міра сего, и къ небесному отечеству идеши. Кто въ странствованіи обогащается! не собираетъ ли богатства, и въ отечество посылаетъ? Хрістіанское отечество есть небо: тамо сокрывать себѣ сокровище повелѣлъ намъ Хрістосъ: \textit{скрывайте себѣ сокровище на небеси, идѣже ни червь, ни тля тлитъ, идѣже татіе не подкопываютъ, ни крадутъ. Идѣже бо есть сокровище ваше, ту будетъ и сердце ваше}\footnote{Матѳ.~6,~20 и 21.}. Туды намъ предпосылать сокровища наша должно, сокровища не міра сего, но онаго отечества достойная: любовь, терпѣніе, кротость, милосердіе, и прочая; тамо ихъ безъ сумнѣнія обрящемъ. Путникъ еси: кто идучи по пути обременяется? не паче ли свергаетъ съ себе бремя, да легчая творитъ путь? Бремя человѣку, идущему по пути міра сего, есть богатство, которое воспящаетъ путь ему и удерживаетъ его, и не допущаетъ къ отечеству стремиться. Коль многихъ бремя сіе воспятило отъ неба и погрузило во дно адово! \textit{Узкая врата, и тѣсный путь вводяй въ животъ}, глаголетъ Хрістосъ\footnote{7,~14.}. Сими вратами и путемъ не проходятъ сѣдящіи на высокихъ гордости, надменія и пышности колесницахъ и расширяющіися широтою роскоши и попеченій мірскихъ, яко не вмѣщаются: смиренные, кроткіе и попеченія мірская отложившіи идутъ тѣмъ. Тебе отъ того бремени, тяжести и пространства избавилъ Господь и легкимъ учинилъ, да безъ задержанія идеши къ небесному отечеству. \textit{Пищу же даетъ} Богъ всякой плоти, не токмо человѣкамъ, но и \textit{скотомъ и птенцемъ врановымъ, призывающимъ Его}.\footnote{Пс.~146,~9.} \textit{Имуще же пищу и одѣяніе, сими довольни будемъ}\footnote{1~Тим.~5,~8.}. "--- 6)~Скоро окончится и богатыхъ довольство и утѣха, и нищихъ скудость и скорбь: всему будетъ конецъ скоро. Вотъ скоро будетъ перемѣна: вмѣсто нищеты временной, которую терпишь, вѣчная отверзутся тебѣ сокровища благости Божіей, и забудеши нищету твою и скорбь. Уже ты болѣе приближился къ концу тому нынѣ, нежели вчера и третьяго дня былъ. И какъ только пріидетъ онъ, то и нищетѣ твоей будетъ конецъ, якоже отъ богатыхъ богатство отыдетъ. \textit{Ничтоже бо внесохомъ въ міръ сей, явѣ, яко ниже изнести что можемъ}\footnote{6,~7.}. И тогда богатый съ нищимъ сравнится. Смерть бо всѣхъ равными дѣлаетъ, и богатыхъ съ нищими, и славныхъ съ подлыми, и господъ съ рабами сравняетъ. Открой гробъ умершихъ, и не узнаешь, гдѣ богатый и гдѣ нищій, гдѣ вельможа и слуга его, гдѣ господинъ и гдѣ рабъ его лежитъ: всѣхъ единъ видъ показуется "--- тлѣніе, земля и пепелъ. Едино неравенство въ душахъ вѣруется: яко благочестивая душа въ \textit{лоно Авраамле} ангелами несется, нечестивая же \textit{во адъ}, мѣсто, себѣ по дѣламъ ея уготованное, низвергается\footnote{Лук.~16,~22 и 23.}. Разсуждай убо вышеписанныя причины, и не скучай отъ нищеты, которая болѣе тебѣ добра дѣлаетъ, нежели богатымъ богатство \textit{(Смотри еще главу о терпѣніи и узкомъ пути въ первой книгѣ)}. Сіе утѣшеніе тѣмъ нищимъ не приличествуетъ, которые безъ страха Божія живутъ, пьянствуютъ, не хотятъ работать, но въ праздности дни свои провождаютъ: таковыи нищіи и тѣломъ и душею нищи суть, и здѣ и въ ономъ вѣкѣ нищи будутъ, когда не покаются.

\paragraph*{§\:427.} Что"=де о тѣхъ хрістіанахъ разсуждаешь, которые за злыя дѣла страждутъ, каковые суть разбойники, убійцы, татіе, хищники, лихоимцы, блудники и прочіи злодѣи? \textit{Отвѣтъ}. 1)~Отъ вышеписанныхъ можешь видѣть, что они хотя \textit{Бога и исповѣдуютъ, но дѣлами отмещутся Его}\footnote{Тит.~1,~16.}. Того ради внутрь церкви святой не находятся, и именемъ только хрістіане суть. 2)~Таковымъ должно покаяться, и Богу весьма благодарить, что ихъ въ таковое злостраданіе привелъ. Ибо тако ихъ преблагій Богъ, и злобою человѣческою непобѣдимый, приводитъ къ покаянію и хощетъ имъ \textit{спастися, и въ разумъ истины пріити}, да бѣдами и злостраданіемъ, яко гласомъ сильнымъ, отъ глубокаго сна грѣховнаго пробудившеся, исправятся и спасутся. Тако очувствовался и покаялся Манассія, царь Іудинъ, якоже читаемъ о немъ: \textit{и наведе Господь на нихъ начальники воевъ царя Ассирійска, и яша Манассію во узахъ, и связаша его оковы ножными, и отведоша въ Вавилонъ. И егда озлобленъ бысть, взыска лице Господа Бога своего и смирися зѣло предъ лицемъ Бога отецъ своихъ; и помолися къ нему, и послуша его, и услыша вопль его, и возврати его во Іерусалимъ на царство его: и позна Манассія, яко Господь той есть Богъ}\footnote{2~Пар.~33,~11--13.}. 3)~Когда злодѣи страждущіи признаютъ отъ сердца свои грѣхи, и смиряются предъ Богомъ, и каются чистосердечно, что грѣшили, и впредь на свои грѣхи возвратиться не хотятъ, хотя бы и свободу получили, чего истинное покаяніе требуетъ, то и имъ страданіе обращается во благое. Ибо человѣкъ, когда чистосердечно обращается отъ грѣховъ къ Богу, то и Богъ съ Своею благодатію обращается къ нему, и пріемлетъ его въ Свою милость.

\subsection[Глава 12-я. О успѣхѣ хрістіанскомъ.]{глава втораянадесять.\\\bfseries О успѣхѣ хрістіанскомъ.}

\begin{quotation}\textit{О семъ молюся, да любовь ваша еще паче и паче избыточествуетъ въ разумѣ и во всякомъ чувствіи, во еже искушати вамъ лучшая, да будете чисти и непреткновенни въ день Хрістовъ, исполнени плодовъ правды Іисусъ Хрістомъ въ славу Божію}\footnote{Филип.~1,~9--11.}.\end{quotation}

\paragraph*{§\:428.} Какъ въ естественномъ рожденіи бываетъ, что рожденный младенецъ не всегда младенецъ есть, но растетъ тѣломъ, и успѣваетъ разумомъ и разсужденіемъ: тако должно быть и въ духовномъ рожденіи, въ которомъ раждаемся свыше, отъ Бога. Не всегда должно быть младенцами о Хрістѣ и млекомъ питатися, но должно расти, по подобію естественнаго младенца, тщатися \textit{достигать въ мужа совершенна, въ мѣру возраста исполненія Хрістова}\footnote{Еф.~4,~13.}. Дѣлается тое: 1)~Тщательнымъ слышаніемъ или чтеніемъ, разсужденіемъ, размышленіемъ и вниманіемъ святаго Писанія. \textit{Всяко бо писаніе Богодухновенно и полезно есть ко ученію, ко обличенію, ко исправленію, къ наказанію, еже въ правдѣ: да совершенъ будетъ Божій человѣкъ, на всякое дѣло благое уготованъ}\footnote{2~Тим.~3,~16 и 17.}. Аще научитися чему хощемъ, отъ того научаемся; аще обличити ложь и неправду тщимся, тое подаетъ способъ; аще себе или другаго исправить желаемъ, оно показуетъ намъ правило; аще въ правдѣ и истинномъ благочестіи наставитися хощемъ, оно руководствуетъ. Откуду \textit{блаженнымъ} означается тотъ, который \textit{въ законѣ Господни поучается день и нощь}. Ибо \textit{будетъ яко древо, насажденое при исходищихъ водъ, еже плодъ свой дастъ во время свое, и листъ его не отпадетъ, и вся, елика аще творитъ, успѣетъ}\footnote{Пс.~1,~1--3.}. И \textit{Хрістосъ далъ есть овы убо апостолы, овы же пророки, овы же благовѣстники, овы же пастыри и учители, къ совершенію святыхъ въ дѣло служенія, въ созиданіе тѣла Хрістова, дондеже достигнемъ вси въ соединеніе вѣры и познанія Сына Божія, въ мужа совершенна, въ мѣру возраста исполненія Хрістова}\footnote{Еф.~4,~11--13.}. Аще убо хощеши успѣвати въ истинномъ благочестіи, читай или слушай святое Писаніе со всякимъ прилѣжаніемъ. "--- 2)~Къ успѣху въ хрістіанскомъ дѣлѣ руководствуетъ различный крестъ и искушеніе. \textit{Всяку радость имѣйте, братіе моя, егда во искушенія впадаете различна, вѣдяще, яко искушеніе вашея вѣры содѣловаетъ терпѣніе; терпѣніе же дѣло совершенно да имать, яко да будете совершенни и всецѣли, ни въ чемъ же лишени}\footnote{Іак.~1,~2--4.}. Крестъ бо и искушеніе различное есть училище духовное, въ которомъ обучаются хрістіане. Якоже бо во училищахъ обучаются отроки и юноши праворѣчію, краснорѣчію и мірской мудрости, тако во училищѣ креста и искушеній хрістіане, младенцы и отроки о Хрістѣ, обучаются правому, красному и богоугодному житію и духовной премудрости. И чимъ болѣе въ сей духовной школѣ обращаются, тѣмъ искуснѣйшими въ дѣлѣ хрістіанскомъ бываютъ. И какъ воины міра сего, чимъ болѣе обращаются въ брани и противу враговъ отечества своего подвизаются, тѣмъ искуснѣйшими и храбрѣйшими дѣлаются; тако хрістіане, воины Хрістовы, чимъ болѣе подвизаются подъ крестомъ, знаменемъ хрістіанскимъ, противу враговъ своихъ, плоти, міра и діавола, тѣмъ лучше успѣваютъ въ дѣлѣ хрістіанскаго благочестія. И якоже въ горнилѣ сребро и злато искушается и чистѣйшимъ бываетъ, тако въ горнилѣ искушеній, бѣдъ, напастей и креста хрістіане очищаются отъ злонравія и чистѣйшими дѣлаются. Крестъ убо и искушеніе изрядное и преславное есть училище хрістіанскія мудрости. "--- 3)~Образъ святаго и непорочнаго житія Хрістова есть совершенное и истинное правило, на которое почаще взирая хрістіане могутъ успѣвать въ дѣлѣ своемъ. Растлѣнное бо вси имѣемъ естество, котораго безъ правила исправить не можемъ. Сего ради должно взирать намъ на совершеннѣйшее правило житія Хрістова, когда хощемъ себе исправить и успѣть въ добродѣтельномъ житіи. Откуду Самъ Хрістосъ повелѣлъ намъ учитися отъ Себе сему искусству: \textit{научитеся отъ Мене, яко кротокъ есмь и смиренъ сердцемъ}\footnote{Мѳ.~11,~29.}. \textit{Образъ дахъ вамъ, да, якоже Азъ сотворихъ вамъ, и вы творите}\footnote{Іоан.~13,~15.}. Аще обучаемся художествамъ, краснорѣчію, иностраннымъ языкамъ и прочему, къ сему временному житію надлежащему: коль изрядно и полезно учитися отъ Хріста художеству хрістіанскому, красному житію, святымъ и божественнымъ нравамъ и небесной премудрости. Оное обученіе тѣло созидаетъ, и вмалѣ пользуетъ насъ, и при кончинѣ отходитъ отъ насъ: сіе душу созидаетъ и на оный вѣкъ отводитъ, и показуетъ на судѣ Хрістовомъ, въ чемъ душа, въ мірѣ живучи, обучалася и упражнялася. Поучимся убо, возлюбленный хрістіанине, отъ Хріста Господа нашего, якоже глаголетъ: \textit{научитеся отъ Мене}. Дѣти отъ матерей и пѣстуновъ своихъ говорить, ходить, обращаться между людьми и политично обходиться обучаются: мы отъ Хріста нашего поучимся по"=хрістіански говорить, по"=хрістіански жить, по"=хрістіански обращаться; поучимся, како другъ друга любить, Бога почитать, како смиряться, терпѣть, кроткими, милостивыми и щедрыми быть, и тогда будемъ истинными Его учениками. Всякъ бо хрістіанинъ есть ученикъ Хрістовъ, и долженъ быть ученикомъ Хрістовымъ, дабы не былъ именемъ только хрістіанинъ. "--- 4)~Нужна къ тому всегдашняя и усердная молитва. Понеже безъ помощи Божіей не токмо успѣвать, но и содержать себе не можемъ въ дѣлѣ хрістіанскаго благочестія, по словеси Хрістову: \textit{безъ Мене не можете творити ничесоже}\footnote{Іоан.~15,~5.}. Помощь же и благодать Божія получается молитвою, по реченному: \textit{просите, и дастся вамъ; ищите, и обрящете; толцыте, и отверзется вамъ}\footnote{Матѳ.~7,~7.}. Сего ради хотящему расти и успѣвать въ благочестіи должно усердно молитися и просить къ тому отъ Бога помощи. "--- 5)~Наконецъ хотящему хранить благочестіе и успѣвать въ томъ, должно отъ дружбъ и бесѣдъ мірскихъ удаляться. Въ собраніи бо сыновъ вѣка сего не иное что дѣлается, какъ слова и дѣла, волѣ и святому закону Божію противныя бываютъ. Въ томъ собраніи о томъ, въ другомъ о иномъ переговариваютъ и пересужаютъ. Тамо того поносятъ и проклинаютъ; индѣ на другаго жалуются, и зубами, какъ оружіемъ и мечемъ острымъ, имя его уязвляютъ, и когда того злословіемъ насытятся, на другаго нападаютъ. Въ томъ пиршества, піянства, безчинные кличи и прочая, пьянству послѣдующая; въ другомъ собраніи ссоры, свары и взаимная руганія совершаются, и тако законъ Божій то словомъ, то дѣломъ разоряется и соблазнъ подается. Все сіе ударяетъ и почти уязвляетъ совѣсть нашу, и лишаетъ ее мира и покоя своего. И уже не таковымъ, хрістіанине, возвратишися въ домъ свой, каковымъ вышелъ изъ дома. Какъ бо входящіи въ аптеку, благовонными мастьми исполненную, и нѣсколько времени медлящіи, выносятъ съ собою и благовоніе мастей оныхъ: тако благочестивая душа, вшедши въ собраніе людей, по плоти и міру живущихъ, и умедливши, выноситъ нѣсколько злонравія, яко зловонія, отъ нихъ прилѣпившагося. Чувства бо наши, наипаче слухъ и видѣніе, суть какъ двери, которыми всякое зло входитъ въ сердце наше, и хощемъ ли, или не хощемъ, ударяетъ тое, и влечетъ къ тому, что или ухо слышало, или око видѣло. Но какъ сердце человѣческое ко злу склонно: удобно превращается ко злу и прельщается, и тако зачинаетъ и раждаетъ беззаконіе. Тако Давидъ святый, изшедъ изъ дома своего, и ходя на кровѣ дома царскаго, увидѣлъ жену Уріину \textit{и прельстился, и палъ}\footnote{2~Цар.~11,~2--4.}. Откуду читаемъ въ исторіи церковной, что святіи отъ сожитія съ нечестивыми удалялися; многіи въ пустыняхъ изволяли жить, нежели во градѣхъ и селѣхъ, дабы соблазномъ дѣлъ и словъ беззаконныхъ не повредиться. Сего ради весь свѣтъ удивляется праведному Лоту, что между беззаконными Содомлянами живучи, \textit{цѣлъ сохранился}\footnote{Быт.~19"~я.}. Но святому мужу не малое было мученіе видѣть и слышать толикая беззаконія нечестивыхъ людей: \textit{видѣніемъ бо и слухомъ праведный живый въ нихъ, день отъ дне душу праведну беззаконными дѣлы мучаше}\footnote{2~Петр.~2,~8.}. Міръ сей есть Содомъ беззаконный и нечистый, и міролюбцы суть подобны Содомлянамъ. Аще убо не хощешь душу свою повредить и съ сими Содомлянами погибнуть, хрістіанине, бѣгай общенія и дружества ихъ, какъ огня, или какъ моровой язвы. И аще нужду имѣешь изыти изъ дома твоего и быть въ собраніи, не медли тамо; очи и уши береги, да не зло внидетъ въ сердце твое, и, по подобію разбойника, разоритъ душевный домъ твой, и въ томъ случаѣ молись Богу: \textit{отврати очи мои, еже не видѣти суеты}\footnote{Пс.~118,~37.}. \textit{Горе міру отъ соблазнъ! Нужда бо есть пріити соблазномъ}, глаголетъ Хрістосъ\footnote{Матѳ.~18,~7.}. Сатана вездѣ полагаетъ сѣти благочестивымъ, и тщится тѣми уловить ихъ: только"=что изъ дома выступишь, тутъ тебе и срѣтаетъ соблазнъ словъ или дѣлъ беззаконныхъ, которымъ или уязвится, или обезпокоится совѣсть твоя. Но язвы познать не можешь, развѣ въ домѣ уединившися. Напротивъ того, въ домѣ и уединеніи пребывающій свободенъ бываетъ отъ случаевъ тѣхъ, которые въ собраніяхъ и бесѣдахъ находящихся ко грѣху приводятъ. Домъ бо и стѣны соблазна подать не могутъ; ухо не слышитъ и око не видитъ зла, и тако человѣкъ отъ иныхъ не пріемлетъ, и самъ никому не подаетъ соблазна, но всему тому противное можетъ быть. Тамо можетъ быть разсмотрѣніе совѣсти, которая, какъ книга записная, представитъ тебѣ прешедшаго житія твоего грѣхи, представитъ судъ Божій за грѣхи, и геенну, грѣшникамъ уготованную; отъ чего послѣдуетъ истинное покаяніе и сердечное сокрушеніе. Тамо бываетъ праведное и вѣрное о мірскихъ и тлѣнныхъ вещахъ, богатствѣ, чести, славѣ и сласти разсужденіе, которое покажетъ тебѣ всю тую утѣху, какъ сонное видѣніе вмалѣ являющееся и исчезающее; и представитъ смерть приближающуюся, и неизвѣстно ко всякому приходящую, которая всему конецъ полагаетъ. Отъ смерти двоякій путь представится тебѣ "--- единъ къ благополучной, другій къ неблагополучной вѣчности ведущій. Сіе размышленіе отведетъ тебе отъ суеты, и возбудитъ готовитися къ блаженному исходу, "--- неблагополучной благодатію Божіею уклоняться, благополучной же желать и искать вѣчности. Уединеніе научаетъ истинной молитвѣ. Идѣже бо разсужденіе о вѣчности, тамо не можетъ не быть истинная молитва, которая не въ словахъ единыхъ, но въ сердечномъ, горячемъ и изъ глубины происходящимъ воздыханіи бываетъ. Ибо помышляющему о вѣчности невозможно не воздыхать и не сокрушаться сердцемъ. Блаженная бо вѣчность восхищаетъ размышляющаго сердце, но неблагополучная устрашаетъ и ужасаетъ. Тая влечетъ къ себѣ насъ: сія отвращаетъ самымъ поминаніемъ. Отсюду востаетъ воздыханіе къ Богу, у Котораго въ руцѣ смерть и животъ нашъ, "--- и истинное желаніе къ избавленію отъ горькой, и къ полученію радостной вѣчности. Уединеніе празднословія не знаетъ, за которое \textit{слѣдуетъ отвѣтъ дать въ день судный}\footnote{Матѳ.~12,~36.}, но есть матерь молчанія, которое есть корень безгрѣшія и добрыхъ дѣлъ. Въ уединеніи открыется книга, въ которой увидитъ вѣрная душа Создателя своего, и познаетъ всемогущество Его, Который все изъ ничего создалъ, и все созданіе силою Своею содержитъ и сохраняетъ; благость Его и премудрость, въ созданіи и правленіи показанную, увидитъ отеческую Его любовь къ роду человѣческому, которую въ возстановленіи падшаго и обновленіи обветшавшаго его чрезъ Единороднаго Сына Своего показалъ. Въ сей тишинѣ представится вѣрной душѣ святый, милостивый и отеческій Его о всѣхъ промыслъ, и благодарно воспоетъ съ пророкомъ Промыслителю: \textit{очи всѣхъ на Тя, Боже, уповаютъ, и Ты даеши имъ пищу во благовременіи; отверзаеши Ты руку Твою, и исполняеши всякое животно благоволенія}\footnote{Пс.~144,~15 и 16.} и проч. Въ сей тишинѣ услышиши, возлюбленный хрістіанине, гласъ Божій, внутрь тебе глаголющій: \textit{Азъ есмь Господь Богъ твой}, Который тебе создалъ, и образомъ Своимъ почтилъ, и о тебѣ печется, Котораго тебѣ, яко Создателя и Бога своего, должно усердно почитать, любить, слушать, призывать, и проч. Въ семъ безмолвіи можешь увидѣть Сына Божія, явившагося на землѣ во плоти и отъ мѣста на мѣсто переходящаго, проповѣдующаго слово Божіе, открывающаго тайну Евангелія Своего, показующаго путь истины, призывающаго грѣшники на покаяніе, проповѣдующаго царствіе небесное, знаменія и чудеса творящаго, "--- яко Агнца кроткаго, хулы, поношенія и безчестія терпящаго, страждущаго и умирающаго ради спасенія нашего, востающаго отъ мертвыхъ и миръ благовѣствующаго, со славою восходящаго на небо, сѣдящаго одесную Отца, и паки грядущаго судити живымъ и мертвымъ. Увидишь яко Судію, сѣдящаго на престолѣ славы Своея, и праведный судъ творящаго. Увидишь одесную Его вѣрныхъ Его, яко овецъ, послушавшихъ гласа Его, и яко чадъ Божіихъ, одеждою нетлѣнною одѣянныхъ, и яко свѣтила сіяющихъ, ублажаемыхъ и любезно призываемыхъ отъ Него въ наслѣдіе вѣчнаго и небеснаго царствія, "--- ошуюю грѣшниковъ, яко козлищъ, отлученныхъ отъ стада благословенныхъ Его овецъ, трепещущихъ отъ страха славы Его, покрытыхъ неизреченнымъ стыдомъ, воздыхающихъ и неутѣшно болѣзнующихъ ради лишенія и отверженія отъ вѣчнаго блаженства, и ради несумнѣннаго чаянія вѣчныхъ, въ которыя тотчасъ имѣютъ ввержены быть, адскихъ мученій. Тутъ немедленно представится внутреннимъ твоимъ очесамъ, что отверзается домъ небеснаго Отца, въ которомъ \textit{обители многи суть}\footnote{Іоан.~14,~2.}, и идутъ въ него избранніи сынове Его, идутъ съ радостію и веселіемъ и торжествомъ неизреченнымъ; идутъ \textit{собранніи Господемъ въ Сіонъ съ радостію, и радость вѣчная надъ главою ихъ; надъ главою бо ихъ хвала и веселіе, и радость пріиметъ ихъ, отбѣже болѣзнь и печаль и воздыханіе}\footnote{Ис.~35,~10.}. Отверзаетъ и адъ уста своя и пожираетъ осужденныхъ и отчаянныхъ грѣшниковъ: \textit{тамо будетъ плачъ и скрежетъ зубомъ}\footnote{Мѳ.~25,~30.}. \textit{Идутъ сіи въ муку вѣчную: праведницы же въ животъ вѣчный}\footnote{ст.~46.}. Въ семъ богоугодномъ субботствіи и сладкомъ покоѣ можешь обратить умъ твой къ началу временъ и вѣковъ и древнимъ родамъ, и умнымъ окомъ видѣти дивная дѣла Божія, и вѣрою съ удивленіемъ зрѣти: како Богъ изъ ничего небо и землю, и исполненія ихъ, такъ чудныя дѣла сотворилъ, и сотворилъ \textit{словомъ: рече, и быша; повелѣ создашася}\footnote{Пс.~148,~5.}; како послѣ человѣка особеннымъ совѣтомъ сотворилъ отъ земли, и Божественнымъ Своимъ образомъ и подобіемъ Его почтилъ, въ рай сладостей ввелъ, заповѣдь ему далъ, прельщеннаго совѣтомъ зміинымъ и преслушавшаго Господа своего изъ рая выгналъ, но безъ утѣшенія, по благости и милосердію Своему, не оставилъ его; како въ маломъ ковчегѣ праведнаго Ноя съ домомъ его, со всякими скотами, звѣрями, птицами и прочіими животными, посредѣ ужасныхъ водъ всемірнаго потопа сохранилъ; како гордое столпотворенія дерзновеніе нечестивыхъ уничтожилъ; како Авраама изъ земли, рода и отечества его вызвалъ, и показалъ того, яко вѣрнаго Своего послушника, отца вѣрнымъ Своимъ, и съ наслѣдниками его, Исаакомъ и Іаковомъ, благословилъ; како съ мѣста на мѣсто, яко странниковъ и пришельцевъ на земли чуждей, преселялъ, во Египетъ ввелъ, и тамо умножившихся и озлобленныхъ милостію Своею посѣтилъ и озлобившихъ показнилъ; извелъ изъ работы тоя людей Своихъ, отворилъ имъ путь посредѣ моря, и враговъ ихъ во глубинѣ, яко во гробѣ, заключилъ; како провелъ ихъ пустынею непроходимою, подалъ имъ манну съ небеси, и хлѣбомъ ангельскимъ напиталъ ихъ; извелъ имъ воду изъ камени, и изъ несѣкомаго источники водные и прочая славная чудеса содѣлалъ; како поразилъ предъ лицемъ ихъ враговъ ихъ, и ввелъ ихъ въ землю кипящую медомъ и млекомъ, и, истребивъ языкъ нечестивый, населилъ тамо избранныхъ людей Своихъ, и проч. Отъ сей земли обѣтованной, въ которой временно упокоилися люди Божіи, можеши умомъ и вѣрою взыти къ странѣ живыхъ, гдѣ вѣчно кроткіи живутъ, благоугождаютъ предъ Господемъ и сладко упокоеваются, "--- къ вышнему Іерусалиму и дому небеснаго Отца, куды не Іисусъ Навинъ, но Іисусъ Хрістосъ, Сынъ Божій, вѣрующихъ въ Него и послѣдующихъ Ему благодатію вводитъ. Тамо увидишь вѣрою Авраама, Исаака и Іакова, съ сѣменемъ ихъ праведнымъ, преселившихся и упокоевающихся во градѣ, \textit{емуже художникъ и содѣтель Богъ}\footnote{Евр.~11,~10.}. Увидишь пророковъ, апостоловъ, святителей, пастырей, учителей, мучениковъ, исповѣдниковъ, преподобныхъ и прочіихъ святыхъ безчисленное множество, по странствованіи и многобѣдственныхъ трудахъ и подвигахъ, упокоевающихся и празднующихъ вѣчную \textit{субботу}\footnote{Ис.~66,~25.}. Тамо увидишь Солнце праведное, Которое всегда сіяетъ, но никогда не заходитъ, и облаками не покрывается, просвѣщаетъ и согрѣваетъ, но не опаляетъ, "--- Хріста, Сына Божія, въ славѣ Своей сіяющаго, Которому безчисленное святыхъ ангелъ множество предстоитъ и хвалитъ Его со Отцемъ и Святымъ Духомъ, и проч. Сіе все уединеніе и во уединеніи вѣрное богомысліе представитъ тебѣ, хрістіанине, когда восхощеши, чимъ болѣе отъ людей будешь отлучаться, тѣмъ чаще къ Богу приближаться будеши; чимъ болѣе отъ людской молвы удалишися, тѣмъ тишайшій будеши; чимъ менѣе съ людьми, тѣмъ болѣе съ Богомъ бесѣдовати будеши, и Бога тебѣ бесѣдующаго, внутрь души твоея услышиши, и тако Его удобнѣе познаеши. Якоже бо въ чистой и тихой водѣ удобнѣе солнце видится, нежели въ возмущенной и волнующейся, тако вѣчное и духовное Солнце "--- Богъ удобнѣе въ тишинѣ и покоѣ вѣрою и умнымъ окомъ познается, нежели въ безпокойствіи и людской молвѣ. Отъ сего источника, какъ струи живыя, проистекаютъ хрістіанскія добродѣтели съ помощію Святаго Духа. Сіе духовное сокровище пресѣкается молвою житейскихъ бесѣдъ, и, когда не будетъ въ клѣти сердца хранимо, похищается. "--- \textit{Неужели"=де всѣмъ хрістіанамъ итить въ пустыню и отъ людей удаляться?} "--- Нѣтъ, о томъ здѣ рѣчи нѣтъ. Пустыня тебѣ буди домъ твой, какой Богъ тебѣ далъ ради упокоенія твоего. Не возбраняю и въ пустыню итить, когда хощешь. Лучше бо въ пустыни со звѣрьми жить, нежели въ градѣ и селѣ съ развращенными людьми. А когда съ людьми хощешь быть, то такихъ ищи, которыхъ слова и дѣла и поступки показуютъ хрістіанскія сердца; и не токмо отъ нихъ не удаляйся, но часто обходися съ ними. Тако бо не токмо въ домѣ твоемъ, но и въ бесѣдѣ ихъ день отъ дня пріобрѣтать будешь духовную пользу. А каковою пищею въ бесѣдѣ благочестивыхъ насытишися, тую въ домѣ въ сокъ и кровь обращать размышленіемъ тщися, и тако все тебѣ во благое будетъ поспѣшествовать: уединеніе и собраніе, безмолвіе и бесѣда будетъ тебе созидать. А отъ такихъ собраній, на которыхъ пиршества, піянства, кличи, танцы, смѣхи, личины безобразныя, оперы и маскерады, нынѣ въ обычай вшедшіе, игры и пѣсни непотребныя, переговоры о чужихъ порокахъ, согласія безсовѣстныя, такожде разговоры о вотчинахъ, земляхъ, судахъ, серебрѣ, златѣ, украшеніи суетномъ и проч., "--- берегись, да не инымъ, какъ вышелъ, въ домъ возвратишися. Помни и разсуждай, что Апостолъ написалъ: \textit{тлятъ обычаи благи бесѣды злы}\footnote{1~Кор.~15,~33.}.

\paragraph*{§\:429.} Знаки успѣвающаго въ благочестіи хрістіанина примѣчаются сіи: 1)~Чимъ болѣе кто Бога и Хріста Сына Божія познаетъ, тѣмъ болѣе въ благочестіи успѣваетъ. Познаніе же здѣ разумѣется дѣйствительное, о каковомъ вкратцѣ сказано выше, а не такое, о какомъ апостолъ написалъ: \textit{Бога} де \textit{исповѣдуютъ вѣдѣти, а дѣлы отмещутся Его, мерзцы суще и непокориви и на всякое дѣло благое неискусни}\footnote{Тит.~1,~16.}. Ибо не знаетъ тотъ Бога, кто Бога страхомъ, любовію и послушаніемъ не почитаетъ. Бога бо знать и Его не почитать, то есть, не бояться, не любить и не слушать, есть дѣло невозможное. "--- 2)~Знаменіе успѣвающаго есть, когда кто болѣе и болѣе познаетъ сердца своего немощь, слабость, растлѣніе и окаянство. Ибо отъ таковаго познанія послѣдуетъ сердечное смиреніе, самого себе уничтоженіе, на каковое смиреніе милостиво призираетъ Господь, и изливаетъ на него благодать Свою, \textit{смиреннымъ бо даетъ благодать} Богъ\footnote{1~Петр.~5. 5.}, "--- и самого себе въ лучшее и тщательное исправленіе. Якоже бо человѣкъ, узнавши тѣлесную немощь, старается излѣчить ее; тако хрістіанинъ, познавши души своея немощь и растлѣніе естества своего, будетъ искать посредствія, которымъ немощь тая исцѣляется и растлѣнное сердце исправляется. Къ испытанію же и познанію сердца нашего приводитъ насъ различный крестъ, искушеніе и молитва. Чимъ бо болѣе человѣкъ искушается, тѣмъ болѣе познаетъ себе; чимъ болѣе познаетъ, тѣмъ болѣе просвѣщается; чимъ болѣе просвѣщается, тѣмъ болѣе исправляетъ себе. Якоже бо чимъ болѣе просвѣщаемся естественнымъ, то"=есть, солнечнымъ, или свѣщнымъ свѣтомъ, тѣмъ болѣе видимъ и разпознаемъ едину вещь отъ другой, напр. бѣлое отъ чернаго, и проч.: тако чимъ болѣе просвѣщаемся вышеестественнымъ свѣтомъ, то"=есть, благодатію Божіею, тѣмъ лучше познаемъ состояніе сердца нашего, и разпознаемъ добро отъ зла, добродѣтель отъ порока, истину отъ лжи, правое отъ ложнаго ученія, и истинную вѣру отъ суевѣрія, и проч. Къ сему ведетъ насъ познаніе немощи и окаянства нашего. "--- 3)~Отсюду послѣдуетъ, что чимъ болѣе будешь познавать грѣхи и пороки твои, которыхъ прежде не познавалъ, тѣмъ лучше успѣвать будеши. Ибо познаваемые грѣхи покаяніемъ и вѣрою будешь очищать, и тако часъ отъ часу лучшимъ сотворишися благодатію Божіею. "--- 4)~Успѣхъ показуетъ нехудой, кто злонравіе свое природное, то"=есть гордость, высокоуміе, ненависть, зависть, гнѣвъ, ярость, злобу, мщеніе, нечистоту и всякую злость, въ сердцѣ своемъ крыющуюся, побѣждаетъ и умерщвляетъ силою помощи Божіей. "--- 5)~\textit{Исполненіе закона любы есть}, по свидѣтельству Апостола\footnote{Римл.~13,~10.}, \textit{конецъ завѣщанія есть любы отъ чиста сердца, и совѣсти благія, и вѣры нелицемѣрныя}\footnote{1~Тимоѳ.~1,~5.}. Аще убо кто въ хрістіанской любви успѣваетъ, сей успѣваетъ и въ благочестіи. Любовь здѣ разумѣется такая, о каковой Апостолъ написалъ выше, то"=есть \textit{отъ чиста сердца}, и проч.; и Іоаннъ святый: \textit{чадца моя, не любимъ словомъ, ниже языкомъ, но дѣломъ и истиною}\footnote{1~Іоан.~3,~18.}. \textit{(Смотри еще въ первой книгѣ главы о любви къ Богу и къ ближнему, и во второй главу о любви ко Хрісту)}. "--- 6)~Наконецъ, чимъ усерднѣе будемъ послѣдовать Хрісту Спасителю нашему любовію, терпѣніемъ, кротостію, смиреніемъ и прочіими добродѣтелями, тѣмъ лучше успѣемъ въ дѣлѣ хрістіанскаго благочестія.

\subsection[Глава 13-я. О благородіи, достоинствѣ и красотѣ души человѣческія.]{глава третьянадесять.\\\bfseries О благородіи, достоинствѣ и красотѣ души человѣческія.}

\paragraph*{§\:430.} Да возможемъ удобнѣе познать благородіе и красоту души человѣческія, хрістіанине, разсудимъ обстоятельства нѣкая, отъ которыхъ достоинство и доброта ея показуется. 1)~«Душа человѣческая, по описанію Макарія святаго Египетскаго, есть нѣкое созданіе Божіе, разумное, красное, великое, чудное и преизрядное подобіе и образъ Божій»\footnote{Бес.~1"~я гл.~7.}. "--- 2)~Богъ ради человѣка все, небо и землю съ исполненіемъ и украшеніемъ ихъ, сотворилъ, весь бо видимый міръ ради человѣка созданъ. Ибо прежде человѣка вси вещи созданы отъ Бога ради того, дабы было гдѣ человѣку жить и чимъ довольствоваться, и было бы кому служить человѣку имѣющему создатися, яко господину, а послѣ всего уже человѣкъ созданъ, дабы, какъ господинъ, созданными вещами обладалъ и довольствовался, и тако Бога, яко Создателя и Благодѣтеля своего, благодарилъ, почиталъ и хвалилъ. Сего ради человѣкъ есть конецъ всего Божія созданія видимаго. Оттуду заключается, что человѣкъ изряднѣйшее, краснѣйшее, благороднѣйшее и совершеннѣйшее Божіе созданіе. Все бо созданіе и украшеніе ради его отъ Бога сотворено, ибо естественный разумъ показуетъ, что тая лучшая и совершеннѣйшая есть вещь, ради которой прочія вещи бываютъ. Напр. лучшій и достойнѣйшій есть господинъ паче раба служащаго ему, понеже рабъ ради господина есть; лучшій человѣкъ паче пищи, одежды, дома, злата, сребра и прочаго, понеже все тое ради человѣка бываетъ. Сіе назнаменуя, Хрістосъ глаголетъ фарисеомъ: \textit{суббота человѣка ради бысть, а не человѣкъ субботы ради}\footnote{Марк.~2,~27.}, то"=есть человѣкъ лучшій есть субботы, понеже \textit{суббота бысть человѣка ради}. Аще убо все человѣка ради создано, убо человѣкъ лучшій и благороднѣйшій есть паче всего, паче неба, солнца, луны, звѣздъ, земли и исполненія и украшенія ихъ. Все же сіе приписуется человѣку ради души его разумныя, благородныя и красныя. "--- 3)~Красота души показуется и отъ тѣла человѣческаго, въ которомъ душа, какъ въ домѣ своемъ, обитаетъ. Посмотри всякъ на юнаго человѣка, который добротою лица и прочіими естественными дарованіями одаренъ: какъ красенъ! Какое животное съ нимъ сравнитися можетъ? Аще убо такъ красное тѣло "--- домъ и жилище души, какъ красная есть душа, которая въ томъ обитаетъ! "--- 4)~Красота и достоинство человѣческое и отъ мѣста прекраснаго, на которомъ въ началѣ человѣкъ былъ поселенъ, познается. Мѣсто тое было \textit{рай сладости}, утѣшенія и радости исполненъ, какъ царскій дворецъ прекрасный, всякими древами плодовитыми, благовонными и красными и прочіими украшеніями украшенъ, дабы и отъ мѣста того познавалъ своея души благородіе и красоту. Аще бо и чувства ударяетъ внѣшняя красота, но душа разумная красоту тую разсуждаетъ, похваляетъ, удивляется и утѣшается. "--- 5)~Человѣкъ особеннымъ Божественнымъ совѣтомъ созданъ. Вся тварь словомъ единымъ отъ Бога создана: \textit{рече, и быша}\footnote{Пс.~148,~5.}. Но когда до человѣка дошло дѣло, не такъ изволилося благому Богу и создателю нашему создать его, но особеннымъ нѣкіимъ совѣтомъ. Какимъ? Тако имѣя сотворить человѣка, глаголетъ Тріѵпостасный Богъ: \textit{сотворимъ человѣка}\footnote{Быт.~1,~26.}. Аки дивное нѣкое, преславнѣйшее, благороднѣйшее, достойнѣйшее, великолѣпнѣйшее хотя созданіе Свое въ дѣло произвести, глаголалъ Создатель: \textit{сотворимъ человѣка}. Сіе преимущество человѣку ради души его подалося. "--- 6)~Наипаче красота и достоинство души человѣческія познается отсюду, что въ ней образъ Свой Божественный напечаталъ Богъ и Создатель нашъ. \textit{И сотвори Богъ человѣка, по образу Божію сотвори его}\footnote{Быт.~1,~27.}. О коль дивная и великолѣпная красота и достоинство души! Образомъ Божіимъ почтена и украшена душа человѣческая! Вси вещи созданныя суть дивны, и суть свидѣтельства всемогущества и премудрости Божіей; но человѣкъ образъ и подобіе Создателя своего въ душѣ своей имѣетъ! Въ созданіи всѣхъ прочіихъ вещей всемогущество и премудрость Божія показалася, но въ душѣ человѣческой, кромѣ того, подобіе и образъ Божій написанъ явился. Какое можетъ быть большее украшеніе и преимущество созданію, какъ образъ Создателя своего въ себѣ имѣть? Кто можетъ сію красоту и преимущество понять и словомъ изобразить? Образъ подобенъ долженъ быть первообразному, якоже образъ смотрящаго въ зеркало подобенъ есть тому, кто смотритъ въ зеркало; таковый бо и образъ является въ зеркалѣ, каково есть лице, которое смотритъ въ зеркало. Разсуди, какую красоту и великолѣпіе имѣетъ несозданное Естество "--- Богъ: тогда и созданнаго отъ Него естества, по образу Его сотвореннаго, красоту и доброту нѣсколько познаешь. Богъ есть \textit{красота безконечная}, Которой ангели святіи отъ начала насытитися и похвалити не могутъ. И избранныхъ Божіихъ въ томъ наипаче вѣчное блаженство состоитъ, что сію безконечную доброту \textit{лицемъ къ лицу} безъ конца и ненасытно будутъ видѣти и наслаждатися\footnote{1~Кор.~13,~12; 1~Іоан.~3,~2.}. \textit{Господи Боже мой! возвеличился еси зѣло: во исповѣданіе и въ велелѣпоту облеклся еси, одѣяйся свѣтомъ, яко ризою}, поетъ Ему Давидъ святый\footnote{Пс.~103,~1 и 2.}. \textit{Богъ свѣтъ есть, и тмы въ Немъ нѣсть ни единыя}\footnote{1~Іоан.~1,~5.}. Аще убо разсудишь красоту и благолѣпіе \textit{безконечное} величества Божія, тогда познаешь, коль великая, дивная красота и благолѣпіе души человѣческія, въ которой образъ и подобіе Божіе имѣется. "--- 7)~Когда человѣкъ, хитростію зміиною прельщенъ, палъ, и душа образъ Божій погубила, и тако погибла, чего оплакать достойно не можемъ, тогда образъ души и самую ее взыскать не чрезъ ходатая и ангела, но Самому Собою Господу (о непостижимаго человѣколюбія!) благоизволилося. Самъ Богъ и Создатель, Сынъ Божій и Отчее Слово, въ сіе великое посольство вступилъ; Самъ Царь небесный, ради души человѣческой, на землѣ явился и съ человѣками пожилъ. Послалъ прежде пророки возвѣстити пришествіе Свое, которые вси возвѣстили и различно изобразили пришествіе тое небеснаго Царя. Якоже бо царь земный, хотя въ какій градъ внити, посылаетъ вѣстниковъ во градъ тотъ, дабы граждане уготовилися къ принятію его: тако Господь, Царь славы, небесе и земли Творецъ и обладатель, имѣя въ міръ сей, какъ какій превеликій градъ, отъ Него созданный, пріити, послалъ напредь вѣстниковъ, рабовъ Своихъ пророковъ, дабы возвѣстили душѣ человѣческой, что Самъ Господь и Создатель ея ради идетъ въ міръ, идетъ взыскати и спасти ее. \textit{Слыши дщи, и виждь и приклони ухо твое, и забуди люди твоя, и домъ отца твоего. И возжелаетъ Царь доброты твоея: зане Той есть Господь твой: и поклонишися Ему}\footnote{Пс.~44,~11 и 12.}. А потомъ и Самъ Господь, по обѣщанію Своему, пришелъ \textit{взыскати и спасти погибшаго}\footnote{Лук.~19,~10.}. Видишь, душа, коль великое твое благородіе, достоинство и преимущество! Самъ Господь твой ради тебе пришелъ въ міръ, \textit{взыскати и спасти тебе погибшую}. И коль великая Его любовь къ тебѣ, видишь. "--- 8)~Когда Господь благоволилъ въ міръ пріити и пожити на земли, не ангельское естество, но человѣческое соединилъ Себѣ: \textit{не отъ Ангелъ бо} воистину \textit{пріемлетъ, но отъ сѣмене Авраамова пріемлетъ}\footnote{Евр.~2,~16.}. Воистину великое чудо и удивленія достойное дѣло! Сынъ Божій сыномъ человѣческимъ сдѣлался, и Богъ Содѣтель и Вседержитель бысть плоть отъ плоти нашей, и кость отъ костей нашихъ; бысть братъ нашъ, Господь ангеловъ и всея твари, и насъ братіею Своею нарицати не стыдился, глаголя: \textit{возвѣщу имя Твое братіи Моей}\footnote{Пс.~21,~23; Евр.~2,~11 и 12.}; и паки: \textit{восхожду ко Отцу Моему и Отцу вашему, и Богу Моему и Богу вашему}\footnote{Іоан.~20,~17.}. Познай, человѣче, благородіе и преимущество души твоея! Сынъ Божій сыномъ человѣческимъ сдѣлался ради души твоея; плоть одушевленную, подобную нашей плоти, кромѣ грѣха, соединилъ Себѣ вѣчнымъ союзомъ, и тако почтилъ естество наше. Но большее, какъ достоинство и важность человѣческія души, такъ и человѣколюбіе Божіе къ намъ увидимъ, когда далѣе въ разсужденіе человѣколюбія Божія внидемъ. "--- 9)~Сынъ Божій, одѣявшися въ подобную нашей плоть, тридцать три года и болѣе на земли жилъ, какъ единъ отъ человѣковъ, и какъ рабъ и единъ отъ нищихъ и бѣдныхъ трудился, бѣдствовалъ, плакалъ, алкалъ, жаждалъ, безчестіе и хулу терпѣлъ ради души нашей, а наконецъ ужасное мученіе претерпѣлъ, и между злодѣями, яко единъ отъ злодѣевъ, единъ Праведный и Святый съ неизреченнымъ безчестіемъ на древѣ крестномъ повѣшенъ и умеръ. Сіе все учинилъ человѣколюбецъ Хрістосъ сего ради, дабы души наши отъ власти діавольской изхитить. Такъ дорога душа человѣческая Богу и Хрісту Сыну Божію есть! Безцѣнная \textit{цѣна Божія крове} за нее дана\footnote{Дѣян.~20,~28.}. Отъ сего видишь, что единыя души человѣческія драгость, важность и достоинство весь свѣтъ превосходитъ. О какъ зѣльно погрѣшаетъ человѣкъ, когда за малую и маловременную сласть, или за сребро и злато чувственное, или за мнимую честь и славу міра сего, или за иное нѣчто подобное симъ, такъ дорогую и благородную душу продаетъ и отдаетъ во власть діавольскую! Таковаго оплакиваетъ пророкъ святый: \textit{человѣкъ, въ чести сый, не разумѣ, приложися скотомъ несмысленнымъ, и уподобися имъ}\footnote{Пс.~48,~18 и 21.}. Вѣрная душа неизреченную красоту, благолѣпіе и благородіе имѣетъ. Ибо таковая душа есть невѣста Сыну Божію, Царю небесе и земли обрученная, по писанному: \textit{обручихъ васъ единому Мужу дѣву чисту представити Хрістови}\footnote{2~Кор.~11,~2.}, "--- дщерь небеснаго Отца, и храмъ Святаго Духа. Отсюду она получаетъ превеликую красоту и достоинство. Якоже бо неблагородная и подлая жена, благородному мужу сопряженная, отъ него славу, благородіе и украшеніе получаетъ: тако душа, грѣхомъ оскверненная, когда кровію Хрістовою очистится и вѣрою Ему сопряжется, благородною, преславною и прекрасною дѣлается. И якоже чистый, хрустальный сосудъ, когда внутрь въ себѣ огнь будетъ имѣти, дивное сіяніе и блистаніе издаетъ: тако душа, которая кровію Сына Божія очистилася и вѣрою Ему обручилася, и храмомъ Духа Святаго "--- Огня невещественнаго и чистительнаго, сдѣлалася, дивно просвѣщается и свѣта сіяніе издаетъ. Читаемъ, что когда Моисей святый на горѣ Синайской съ Богомъ бесѣдовалъ, такъ лице его просвѣтилося, что сыны Израилевы не могли на него смотрѣть\footnote{2~Кор.~3,~7; Исх.~34,~30 и слѣд.}. Сіяніе оное Моисеева лица не иное что значило, какъ красоту, доброту и славу святыя его души, благодатію Святаго Духа просвѣщаемыя. Тоежде должно разумѣть и о прочіихъ душахъ святыхъ, которыя вѣрою и любовію прилѣпляются безсмертному своему Жениху "--- Хрісту. Таковая душа называется \textit{царицею и дщерію царевою}, которыя \textit{вся слава есть внутрь: рясны златыми одѣяна и преиспещрена}\footnote{Пс.~44,~10 и 14.}. Не видна сія доброта души нынѣ, когда праведніи и грѣшніи единъ видъ внѣшній имѣютъ; но явится она въ воскресеніи мертвыхъ, когда красота и доброта ея явится на воскресшемъ тѣлѣ, въ которомъ красная обитала душа. Однакожъ изъ нѣкоторыхъ знаковъ познается, какъ отъ теплоты огнь, и отъ вкуса медъ, и отъ слова разумъ.

\paragraph*{§\:431.} \textit{Знаки доброты и красоты, въ душѣ} находящіеся и хранимые, сіи изъ святаго Писанія примѣчаются: 1)~\textit{Вѣра} сердечная, нелицемѣрная, истинная и живая во Іисуса Хріста Сына Божія, безъ которой какъ сыскать, такъ и хранить красоты своея душа не можетъ. Якоже бо \textit{вѣрою} душа \textit{обручается} въ невѣсту краснѣйшему Сыну Божію\footnote{Ос.~2,~19 и 20.}, и красоту себѣ отъ Него получаетъ: тако тоюжде вѣрою и сохраняетъ ее. Аще убо кто имѣетъ истинную и живую вѣру, сей имѣетъ доброту и красоту душевную. 2)~"--- \textit{Чистота совѣсти}, когда человѣкъ отъ грѣховъ и сквернъ міра сего хранить себе, душу и тѣло свое соблюдаетъ нескверно, безъ чего и вѣра истинная быть не можетъ въ сердцѣ. Истинный бо и вѣрный рабъ Хрістовъ не хощетъ согрѣшить противу совѣсти, и ее осквернить. Ибо вѣра противу грѣха и міра подвизается, и не попущаетъ душѣ оными обладаемою быть. "--- 3)~Знаменіе душевныя красоты есть свидѣтельство добрыхъ дѣлъ, нелицемѣрно отъ сердца происходящихъ, какъ"=то: чистоты, любви, смиренія, терпѣнія, кротости, истины, милосердія, молитвы "--- всякаго добра ходатаицы, и прочіихъ. Добрыми бо дѣлами, отъ чиста сердца и вѣры нелицемѣрныя происходящими, показуетъ въ себѣ подобіе свое, которымъ Богу сообразуется, яко образъ первообразному. Богъ есть святъ, любителенъ, истиненъ, щедръ, милостивъ, благъ, кротокъ, долготерпѣливъ, и проч.: и хрістіанинъ, когда Ему сообразуется добродѣтельми, подаетъ свидѣтельство имѣющагося въ немъ образа Божія, въ которомъ доброта и красота душевная состоитъ. Между образомъ бо и первообразнымъ имѣется сходство и подобіе. Истинныя убо и христіанскія добродѣтели свидѣтельствуютъ о доброй и красной душѣ не иначе, какъ \textit{добрые плоды о добромъ древѣ}\footnote{Мѳ.~7,~17.}. Понеже душа человѣческая всю доброту и красоту получаетъ отъ Хріста, какъ выше сказано, Которому вѣрою обручается, то несумнѣнный знакъ есть, что она тую доброту имѣетъ и хранитъ, когда отъ Него не отлучается къ мірской любви, но Ему, яко невѣста Жениху, вѣрою и любовію прилѣпляется, послѣдуетъ стопамъ Его, подражаетъ смиренію, терпѣнію, кротости и любви Его, и такъ хранитъ вѣрность свою къ Нему, да якоже въ смиреніи и страданіи Его не отлучается отъ Него, такъ и въ славѣ Его неотлучна отъ Него будетъ. Сего бо вѣра отъ души, Хрісту обрученной, требуетъ. Якоже бо вѣрность хранитъ мужу своему жена, когда отъ него ни въ безчестіи, ни въ изгнаніи не отлучается: такъ вѣрная Хрісту душа признается, которая отъ Него въ смиреніи и страданіи не отлучается. И какъ жена невѣрною мужу своему почитается, когда любитъ кого паче мужа своего, или равно мужу своему, или въ несчастіи его отлучается отъ него: тако душа не хранитъ вѣры Хрісту, Сыну Божію, которая къ міру прилѣпляется и не хощетъ со Хрістомъ въ смерти, терпѣніи и страданіи быть. И Хрістосъ такую душу за невѣрную Себѣ почитаетъ. \textit{Иже бо аще постыдится Мене и Моихъ словесъ въ родѣ семъ прелюбодѣйнѣмъ и грѣшнѣмъ: и Сынъ человѣческій постыдится его, егда пріидетъ во славѣ Отца Своего со ангелы святыми}\footnote{Марк.~8,~38.}. \textit{И иже не пріиметъ креста своего, и въ слѣдъ Мене грядетъ, нѣсть Мене достоинъ}\footnote{Мѳ.~10,~38.}. Сіе"=то значитъ обѣщаніе и присяга Хрісту, при крещеніи отъ хрістіанина учиненная, то"=есть, что онъ какъ имя свое Хрісту записалъ и отдалъ, такъ вѣрою и любовію Ему чрезъ все житіе свое будетъ служить и послѣдовать, якоже Самъ Хрістосъ отъ хрістіанина сего требуетъ: \textit{аще кто Мнѣ служитъ, Мнѣ да послѣдствуетъ: и идѣже есмь Азъ, ту и слуга Мой будетъ}\footnote{Іоан.~12,~26.}. Которое обѣщаніе когда хранитъ душа, то сохраняетъ и вѣру ко Хрісту "--- Жениху своему и съ вѣрою красоту и доброту, ей отъ Хріста данную.

\paragraph*{§\:432.} Когда хрістіанинъ отвращается сердцемъ и любовію отъ Хріста, и обращается ко грѣху и міру, тогда вѣры ко Хрісту не сохраняетъ; слѣдственно погубляетъ доброту и красоту души своея, и воспріемлетъ скаредное безобразіе на душу свою. Сего ради хотящему паки сыскать доброту души своея должно учинить слѣдующая: 1)~Якоже отвращеніемъ отъ Хріста потерялъ онъ доброту души своея, тако обращеніемъ паки къ Нему искать ему тую должно. Иначе сіе сокровище сыскать невозможно. 2)~Должно прежде отвратиться отъ грѣха и любви міра, и жалѣть о томъ, что любовь тую, которую долженъ отдавать Хрісту, Любителю и Искупителю своему, обращалъ къ міру и грѣху, и такъ неблагодарнымъ показывался Любителю и Благодѣтелю своему и тѣмъ безумно оскорблялъ Его. 3)~Тако отвратившися отъ міра и грѣха ко Хрісту, взирать къ Нему вѣрою и усердіемъ, да Самъ Онъ отъиметъ безобразіе отъ души и подастъ благообразіе ей. Якоже бо хотящему образъ лица своего написанный имѣть, должно отъ прочіихъ вещей отвратиться и обратиться лицемъ къ живописцу, и на него смотрѣть: тако, кто хощетъ имѣти благолѣпіе образа Божія въ душѣ своей, должно тому неотмѣнно отвратитися отъ любви міра и грѣха, и душею своею обратитися ко Хрісту, небесному Живописцу, и на Него съ вѣрою, усердіемъ и воздыханіемъ смотрѣть, и молить Его со смиреніемъ. Безъ Него бо сіе дѣло сотвориться не можетъ. Ибо Онъ единъ скаредное безобразіе отъ душъ, къ Нему обратившихся и воздыхающихъ, отнимаетъ; заглаждаетъ тое Своею кровію, ради всякаго грѣшника изліянною, и Духомъ Своимъ Святымъ написуетъ живую образа Божія красоту и благолѣпіе. "--- 4)~Душа, хотящая принять себѣ живую сію доброту, не должна чинить препятіе Хрісту въ семъ важномъ и спасительномъ дѣлѣ, не отвращаться отъ Него, но всегда къ Нему взирать и воздыхать о томъ, и потерпѣть Ему, что ни благоволитъ ей дѣлать. Надобно таковой душѣ Его придержаться, и на волю Его сдаться, да творитъ съ нею якоже хощетъ. Якоже бо живописецъ не можетъ написать образа, когда лице, съ котораго хощетъ образъ писать, отъ него отвращается: тако Хрістосъ не можетъ написать въ душѣ образа Божія, когда отвращается отъ Него, и обращается ко грѣху и суетѣ міра и самолюбію. Все бо сіе безобразіе душѣ наносятъ, и помрачаетъ и погубляетъ доброту и красоту душевную, и такъ препятствіе чинитъ спасительному Хріста Сына Божія дѣлу. Все, что ни запрещаетъ Божіе слово, препятствуетъ написуемому образу Божію. Ибо и Адамъ, праотецъ нашъ, ослушаніемъ потерялъ Божій образъ. Не можетъ и въ той душѣ изобразиться доброта Божія образа, которая не тщится отъ всего того уклонятися, что запрещаетъ и отъ чего отводитъ Божіе слово. Надобно убо отъ всего того, что противно волѣ Божіей и святому Его слову, отвращаться, и нудить себе ко всему тому, что оно повелѣваетъ намъ. Ибо какъ въ томъ, что повелѣваетъ слово Божіе, состоитъ свойство образа Божія, какъ"=то: правда, святыня, любовь, смиреніе, терпѣніе, кротость, и проч.: такъ все тое, что запрещаетъ, противно образу Божію, и свойственно скаредному ветхаго человѣка образу. Якоже убо отъ всякаго зла уклоняться, такъ всякому добру прилѣжать должно душѣ, хотящей получить отъ Хріста себѣ доброту и красоту, по увѣщанію Псаломника: \textit{уклонися отъ зла и сотвори благо}\footnote{Пс.~33. 15.}, хотя сердцу и противно. Надобно бо намъ съ трудомъ, прилѣжаніемъ и подвигомъ искать нынѣ того, что туне и безъ труда нашего намъ данное отъ Бога потеряли. Тогда видя Хрістосъ таковое души тщаніе, попеченіе и трудъ, по милости Своей, отнимаетъ отъ ней безобразіе и подаетъ доброту и красоту образа Своего. На сіе бо въ міръ пришелъ, якоже поетъ церковь: «Хрістосъ раждается прежде падшій возставити образъ»\footnote{Троп. на предпраздн. Хрістова Рождества.}. И сіе"=то есть "--- \textit{отложити намъ, по первому житію, ветхаго человѣка, тлѣющаго въ похотехъ прелестныхъ; обновлятися же духомъ ума нашего, и облещися въ новаго человѣка, созданнаго по Богу въ правдѣ и въ преподобіи истины}\footnote{Еф.~4,~22--24.}, къ чему не малое прилѣжаніе, трудъ и подвигъ требуется. И сами отъ себе не можемъ того учинить, внутрь бо себе носимъ его. Ибо совлещися ветхаго человѣка есть самого себе совлещися, что всякому самому чрезъ себе неудобное или паче невозможное дѣло есть. Къ труду и подвигу нашему требуется неотмѣнно дѣйствіе Іисуса, Сына Божія, Котораго должно усердною молитвою къ тому великому дѣлу преклонять. "--- 5)~Чимъ болѣе человѣкъ \textit{совлекаться} будетъ \textit{ветхаго человѣка съ дѣяньми его}, тѣмъ болѣе \textit{облекаться въ новаго}\footnote{Кол.~3,~9 и 10.}; чимъ болѣе будетъ облекаться въ новаго, тѣмъ чистѣйшая будетъ душа его; чимъ чистѣйшая будетъ душа, тѣмъ большая въ ней доброта и красота образа Божія явится; якоже чимъ чистѣйшая вода, тѣмъ яснѣе въ ней изображается видъ солнца; или чимъ чистѣйшее зерцало, тѣмъ живѣе въ немъ изобразуется лица подобіе, смотрящаго въ тое.

\paragraph*{§\:433.} Поищемъ убо, о хрістіане, доброты и красоты нашея во Хрістѣ, которую во Адамѣ потеряли; поищемъ, пока обрѣтается, да и здѣ ее въ душахъ нашихъ возъимѣемъ, и въ пришествіи Хрістовомъ съ нею предъ Нимъ и всѣмъ міромъ явимся, которая тогда не токмо въ душахъ нашихъ будетъ, но и на тѣлесахъ явится; и Хрістосъ, праведный Судія, видя въ насъ образъ Божій и насъ Себѣ сообразныхъ, признаетъ насъ за Своихъ и съ Собою прославитъ. Прочіихъ бо, въ которыхъ доброты сея не усмотритъ, но паче безобразіемъ грѣховнымъ помраченныхъ увидитъ, не признаетъ за Своихъ, яко Ему несообразныхъ, но паче противныхъ, и скажетъ имъ: \textit{не вѣмъ васъ, откуду есте}\footnote{Лук.~13,~25 и 27.}. Якоже бо монета, неимущая царскаго образа и надписанія, не годится въ казну Государеву; и якоже не имѣющіи одѣянія брачнаго на бракъ не допущаются, и къ царю на вечерю не входятъ, какъ только одѣянніи одеждами красными: тако душа, неимѣющая прекраснаго небеснаго Царя образа, въ сообщеніе святыхъ, тою добротою знаменанныхъ и украшенныхъ, не годится и не пріемлется; и ризою оправданія Хрістова не одѣянная, въ чертогъ небеснаго царствія не допущается, и отъ брака Агнча изгонится. Сего ради принуждена будетъ таковая душа, внѣ дверей оставшися, безполезно сѣтовать, плакать и рыдать. Поищемъ убо, о хрістіанине, да не и насъ оный безполезный плачь и рыданіе постигнетъ. Почто, о хрістіане, внѣ насъ ищемъ красоты отъ тлѣннаго міра? Внутрь насъ можемъ сыскать такую красоту, которая всю красоту міра сего превосходитъ. Образъ Божій есть украшеніе наше. Что сего краснѣйшее, славнѣйшее, достойнѣйшее, великолѣпнѣйшее, дивнѣйшее и любезнѣйшее быть и помыслитися можетъ? Сей красотѣ вся красота солнца, луны, звѣздъ, красныхъ птицъ, травъ, цвѣтовъ, древесъ и всего созданнаго міра, и вся красота и великолѣпіе царей, князей и вельможъ и прочіихъ славныхъ земныхъ, и всякая красота лица человѣческаго, и прочая, какая можетъ быть красота, уступить должна; паче же ничтоже предъ тою есть, ибо Божественная красота и доброта есть, подобная несозданной и безконечной Красотѣ. Почто желаемъ и ищемъ внѣшняго благородія, высокородія, чести, славы, достоинства, преимущества? Внутрь насъ благороднѣйшую душу имѣемъ, созданную по образу Божію, ради которой Царь небесный на землю пришелъ, жилъ, трудился, терпѣлъ, страдалъ и умеръ, и обѣщалъ и хощетъ насъ сынами Божіими сдѣлать, вѣчное и небесное намъ царствіе подать, и славы Своея участниками сотворить. Какая слава и благородіе съ сею славою и благородіемъ сравнитися можетъ? Всякая слава царей и князей земныхъ, во едино собранная, ничтоже есть противу сея славы. Ежели праведно дѣлаютъ люди, которые, оставивши меньшее и худшее, ищутъ большаго и лучшаго: то почто хрістіане не дѣлаютъ правды сея? Почто пренебрегаютъ истинную, лучшую и большую славу, и гоняются за меньшею и худшею или паче ложною? Что бо міра сего титулъ и слава? имя пустое, на языкахъ человѣческихъ носимое, и въ ушахъ звенящее. Отними имя "--- и будетъ славный, какъ подлый. Ибо вси люди по естеству равны, суть человѣцы, суть земля и пепелъ и снѣдь червей. Но кто отъ хрістіанъ не желаетъ прославитися на земли? Кто не хощетъ благороднымъ и высокороднымъ называтися? Кто не похощетъ княземъ, вельможею и другимъ знаменитымъ быти? Кто не съ великою радостію пріемлетъ подаваемую честь и рангъ мірскій? Паче же, кто не ищетъ съ прилѣжаніемъ сего, чтобы въ чести и славѣ міра сего быть? Сколько на тое хрістіане полагаютъ трудовъ и иждивеній! Не хочу много говорить о томъ, что всѣмъ извѣстно. Но благороднѣйшую свою душу почтить и прославить, тое ей благородіе и достоинство, которое она въ началѣ имѣла и потеряла, о Хрістѣ возвратить, и тую честь и славу ей сыскать, которую Хрістосъ ей пришелъ подать, едва кто старается. Весьма мало такихъ есть. О имя "--- хрістіанинъ! Коль ты великое и славное, толь отъ хрістіанъ, а паче въ нынѣшнемъ вѣкѣ, презираемое и попираемое! Сего ради вси тіи хрістіане заблуждаютъ, которые чести и славы на земли ищутъ, но честь и славу небесную, къ которой позваны, оставляютъ; красотою мірскихъ вещей, какъ Ева яблокомъ, отъ древняго змія предлагаемою, прельщаются и любуются, но красоты душевныя не разсуждаютъ и не ищутъ о Хрістѣ. Надобно всѣмъ таковымъ хрістіанамъ опасаться, чтобы до нихъ не надлежало тое страшное слово, которое сказано о званныхъ на вечерю велію, но отрекшихся: \textit{ни единъ мужей тѣхъ званныхъ вкуситъ Моея вечери}\footnote{Лук.~14,~24.}.

\subsection[Глава 14-я. О истинной печали или жалѣніи.]{глава четвертаянадесять.\\\bfseries О истинной печали или жалѣніи.}

\begin{quotation}\textit{Печаль, яже по Бозѣ, покаяніе нераскаянно во спасеніе содѣловаетъ}\footnote{2~Кор.~7,~10.}.\end{quotation}

\paragraph*{§\:434.} Что есть \textit{печаль по Бозѣ} и истинное жалѣніе, отъ слѣдующаго примѣра познати можемъ. Два человѣка единому своему благодѣтелю досадили и оскорбили, и оба жалѣютъ и скорбятъ о томъ, и просятъ у него прощенія за тое. И хотя внѣшнее дѣло ихъ равно кажется, но внутренняя скорбь и печаль неравна есть. Единъ скорбитъ и проситъ прощенія у оскорбленнаго, боясь какого зла отъ него, или чтобы впредь его милости не лишиться; другій, все прочее оставляя, того ради жалѣетъ и проситъ прощенія, что благодѣтеля своего, котораго долженъ любить и почитать, несмысленно опечалилъ. Видиши, хрістіанине, что двухъ сихъ человѣковъ дѣло внѣшнее равно есть, но внутреннее неравно, яко отъ неравнаго сердца происходитъ. У сего сердца любовію къ благодѣтелю исполнено, а у того самолюбіемъ недугуетъ. У сего правое, у того неправое сердце. Сего печаль и смиренное прошеніе отъ любви происходитъ, того "--- отъ самолюбія. Сей печалуетъ по благодѣтелѣ своемъ оскорбленномъ, той скорбитъ о себѣ и своей корысти. Сей хотя бы и не хотѣлъ отмщевать ему оскорбленный, или хотѣлъ и отмщевалъ, однакожъ жалѣлъ бы о немъ и стыдился, и жалѣть не преставалъ; сію печаль содѣловаетъ въ немъ любовь, живущая въ сердцѣ его: оный не печалился бы и не скорбѣлъ, когда бы не надѣялся какого отъ него зла пострадать. Сей примѣръ научаетъ насъ, что есть истинное жалѣніе и печаль по Бозѣ. Многіи хрістіане, пришедше въ чувство, грѣховъ своихъ, которыми величество Божіе прогнѣвали, жалѣютъ и сокрушаются не ради иной какой причины, какъ только того ради, что имъ слѣдуетъ мука, грѣшникамъ уготованная. Сія печаль происходитъ отъ самолюбія, какъ сіе всякъ можетъ видѣть; яко жалѣютъ о слѣдующей погибели своей, а не о Бозѣ, грѣхами ихъ разгнѣванномъ и оскорбленномъ. Таковыи, когда бы не надѣялися за грѣхи послѣдующія казни, и ежели бы возможно было во вѣки въ мірѣ жить и всегда грѣшить, никогда бы грѣшить не преставали. Ибо престаютъ они отъ грѣховъ не ради Бога, но ради страха своей погибели. И тако неправость сердца, самолюбіе и лукавство сердца познается. Все бо мы ради Бога творить, отъ злыхъ уклоняться и добрая творить должны. Якоже бо Богъ все въ нашу пользу творитъ, тако мы все во славу Его творить должны. Сіе есть правость сердечная. Не есть убо истинное и правое жалѣніе "--- жалѣть и сокрушаться ради страха геенны; но требуется отъ хрістіанина лучшее и совершеннѣйшее. Можетъ и сія печаль началомъ быть истинныя печали по Бозѣ, яко таковымъ страхомъ можетъ человѣкъ возбудитися и познати свое заблужденіе, и тако пріитить къ печали по Бозѣ, якоже таковыхъ примѣровъ довольно читаемъ въ исторіи церковной, но назвать того истинною по Бозѣ печалію невозможно. Иное бо печалиться о себѣ и своей пагубѣ, иное по Бозѣ, грѣхами обезчещенномъ и разгнѣванномъ, какъ сіе всякому видно, и примѣръ приведенный показуетъ. Истинное убо жалѣніе и \textit{печаль по Бозѣ} въ томъ \textit{состоитъ}, чтобы хрістіанину сокрушаться и жалѣть не ради лишенія вѣчнаго живота и послѣдующія во адѣ казни, но ради того, что онъ Бога, Создателя, Искупителя и Промыслителя своего, Котораго долженъ паче всего почитать, любить и слушать, не почиталъ, не любилъ и не слушалъ паче всего, но паче себе, міръ и грѣхъ паче Его любилъ. Сія есть истинная печаль по Бозѣ. О семъ печалиться хрістіанину должно, что онъ не отдавалъ Богу должнаго. Таковую печаль имѣющій, хотя бы и вѣчнаго живота и геенны не было, будетъ печалиться, плакать и стыдиться, и самъ себе окаевать. Такою печалію сокрушенная душа всякую обиду, отъ ближняго нанесенную, съ радостію проститъ; все противное и горькое, въ мірѣ семъ приключающееся, усердно претерпитъ; и всего, какъ временнаго, такъ и вѣчнаго наказанія достойною себе признаетъ, вѣдая, что какъ нѣтъ ничего большаго и лучшаго паче Бога, такъ нѣтъ такой казни, которой бы недостойна была за оскорбленіе и безчестіе величества Божія. Таковая печаль отъ любви происходитъ, и есть истинная, хрістіанская, праведная по Бозѣ печаль. Таковая печаль безъ сумнѣнія получаетъ отпущеніе грѣховъ, сколько бы ихъ ни имѣлъ человѣкъ; получаетъ же ради неизреченнаго человѣколюбія Божія и изліянной пресвятой крови Хрістовой. Кто кающійся сію печаль имѣетъ, тотъ причисляется благодатію Хрістовою къ сынамъ Божіимъ, которымъ наслѣдіе вѣчнаго живота уготовано. Сынамъ бо собственно есть печалитися за оскорбленіе отца своего, а не за страхъ наказанія. Разсуждай убо, хрістіанине, какую печаль въ сердцѣ твоемъ имѣеши "--- сію или оную? А каковъ въ сердцѣ твоемъ имѣешься, таковъ и предъ Богомъ находишися, Который, какъ всякаго сердце испытуетъ и видитъ, такъ всякому по сердцу и судитъ; и что нынѣ въ сердцѣ у кого имѣется, тое внѣ въ день судный явится, по словеси апостола: \textit{Господь во свѣтѣ приведетъ тайная тьмы, и объявитъ совѣты сердечныя}\footnote{1~Кор.~4,~5.}. Полезна печаль и ради страха геенны, чтобы не впасть въ геенну. На то бо и въ Писаніи святомъ какъ вѣчный животъ, такъ и геенна отъ человѣколюбиваго Бога объявлена намъ, да сея убѣжимъ, а оный получимъ. Имѣютъ и истинныи хрістіане и святіи страхъ суда Божія и геенны, но сей страхъ вѣрою и молитвою побѣждаютъ: \textit{яко Хрістосъ смертію Своею упразднилъ имущаго державу смерти, сирѣчь, діавола; и избавилъ сихъ, елицы страхомъ смерти чрезъ все житіе повинни бѣша работѣ}\footnote{Евр.~2,~14 и 15.}. Но печалятся и жалѣютъ наипаче ради того, что ради немощи плоти не могутъ угодить волѣ Божіей такъ, какъ слово Божіе требуетъ. Откуду воздыхаютъ и молятся небесному Отцу: \textit{остави намъ долги наша}\footnote{Мѳ.~6,~12.}. Нѣтъ бо въ семъ вѣкѣ совершенства. Сію печаль и намъ, хрістіанине, имѣть и искать ее въ сердцѣ нашемъ должно, зажегше свѣщу ума и размышленія нашего (о чемъ въ слѣдующихъ).

\paragraph*{§\:435.} Ко взысканію истинныя по Бозѣ печали пользуетъ сія помышлять и содержать: 1)~Какъ создалъ насъ Богъ ради единой благости Своей, а не ради нужды Своей, или корысти какой, такъ и нынѣ благотворитъ намъ отъ единаго человѣколюбія Своего, слѣдственно туне благотворитъ намъ, безъ всякихъ нашихъ заслугъ. "--- 2)~Богъ не требуетъ отъ насъ ради Себе Самого никакого служенія. Ибо, какъ прежде міра, такъ и нынѣ во всесовершеннѣйшемъ блаженствѣ пребываетъ. Но мы Ему, яко Создателю, Богу и Господу своему, по силѣ вѣры нашей, служить и работать должны. "--- 3)~Когда служимъ и работаемъ Ему, то намъ, а не Ему служеніе наше пользуетъ: когда не служимъ, то намъ оттуду вредъ бываетъ, а не Ему. "--- 4)~Послушаніе и почитаніе Богу отъ насъ должное есть. Созданіе бо Создателю своему, и рабъ Господу своему долженъ послушаніе и почитаніе показывать. Сего ради, когда послушаніе и почитаніе показуемъ Ему, должное Ему отдаемъ, и тѣмъ ничего не заслуживаемъ отъ Него. \textit{Егда сотворите вся повелѣнная вамъ, глаголите, яко раби неключими есмы: яко, еже должни бѣхомъ сотворити, сотворихомъ}, глаголетъ Хрістосъ\footnote{Лук.~17,~10.}. "--- 5)~Отцамъ, господамъ и властямъ нашимъ должно намъ послушаніе и почитаніе показывать ради Господа Бога нашего. Аще убо человѣкамъ послушаніе и почитаніе показывать должно, кольми паче Самому Тому, ради Котораго и человѣкамъ повинуемся. Аще бо посланнымъ отъ царя повинуемся, кольми паче Самому Царю, пославшему ихъ. И аще Царю, отъ Бога учрежденному, повинуемся, кольми паче Самому Богу, Который его учредилъ и повиноватися ему повелѣлъ, повиноватися должно. Самый естественный разумъ сему научаетъ. "--- 6)~Какое добро человѣкъ ни дѣлаетъ человѣку, не собственное свое даетъ, но Божіе. Ничего бо, кромѣ растлѣнія и грѣховъ, собственнаго не имѣемъ: все и всякое добро Божіе есть, а не наше. И душу бо и тѣло Богу, Создателю нашему, должны мы, кольми паче прочее добро. Сего ради, когда кто человѣку благодѣтельствуетъ, отъ Божія добра ему даннаго удѣляетъ. Единъ Богъ собственное свое добро подаетъ намъ, яко все добро, или душевное, или тѣлесное, Его добро есть, и отъ Него единаго, яко Источника добра, происходитъ. "--- 7)~Богъ лицепріятія не имѣетъ, но всякаго человѣка равно, мене и тебе и другаго, любовію предваряетъ; и всякаго, мене и тебе и прочіихъ равно хощетъ блаженными сдѣлать. 8)~Кто блаженства лишается и погибаетъ, тое его собственной волѣ приписать должно, а не Божіей, которая \textit{всѣмъ хощетъ спастися, и въ разумъ истины пріити}\footnote{1~Тим.~2,~4.}. "--- 9)~Какъ Богъ хощетъ, и аки жаждетъ спасенія нашего, хотя и все слово Его святое, и въ немъ увѣщаніе Его, званіе, угроженіе и вѣчныхъ благъ обѣщаніе прописанное, но наипаче воплощеніе, страданіе и смерть Единороднаго Сына Его, Іисуса Хріста, показуетъ, Который волею Отчею и Своимъ охотнымъ послушаніемъ за всѣхъ, то"=есть за мене, тебе и всякаго пострадалъ и умеръ, какъ о томъ святое Писаніе свидѣтельствуетъ. И Богъ лицепріятія не имѣетъ, и потому какъ ради тебе и мене, такъ и ради всякаго человѣка Сына Своего въ міръ послалъ, кто только ни восхощетъ Его сердечною вѣрою принять и о Немъ спастися. "--- 10)~Богъ, яко святый и праведный, что ни дѣлаетъ, свято и праведно дѣлаетъ. Сего ради никого ничимъ не обидитъ. "--- 11)~Когда не даетъ намъ добра, или отнимаетъ отъ насъ добро, то наша вина есть, а не Его. "--- 12)~Богъ на всякомъ мѣстѣ существенно есть, и никакимъ мѣстомъ не заключается, и всякое наше дѣло и помышленіе видитъ, и всякое слово слышитъ. Сего ради все, что ни дѣлается, или помышляется, или говорится, доброе или злое, согласное или противное закону Божію, все предъ лицемъ Его святымъ дѣлается. "--- 13)~Пока грѣшникъ не находится въ истинномъ покаяніи, но пребываетъ въ нераскаянномъ житіи, дотолѣ онъ не боится, ни почитаетъ Бога, не имѣетъ къ Нему благодарности, любви, вѣры, надежды и проч.

\paragraph*{§\:436.} Печаль по Бозѣ и истинное жалѣніе можетъ при помощи Божіей родитися отъ разсужденія и размышленія, когда человѣкъ возвратится въ себе и разсуждать будетъ о своемъ ничтожествѣ и Божіемъ величествѣ, которое грѣхами оскорблялъ. 1)~Отъ разсужденія высочайшія власти Божіей, которой грѣшникъ не покаряется. Богъ есть самоверховнѣйшій властелинъ и владѣтель надъ всѣми тварьми, видимыми и невидимыми, въ числѣ которыхъ и всякій грѣшникъ заключается, и власти Его \textit{всякое колѣно небесныхъ и земныхъ и преисподнихъ покланяется}\footnote{Исх.~45,~23; Римл.~14,~11; Филип.~2,~10.}, Ангели святіи со всякимъ усердіемъ и послушаніемъ волю Его исполняютъ; святіи человѣки велѣнію Его со усердіемъ, страхомъ и любовію повинуются; демони страшные власти Его трепещутъ, и вся тварь одушевленная, какъ"=то: скоты, звѣри, птицы, рыбы и прочая движущаяся, и бездушная, какъ"=то: солнце, луна, звѣзды, огнь, и прочая \textit{творитъ слово Его}\footnote{Пс.~148,~8.} и \textit{всяческая работна Ему}\footnote{118,~91.}. Но единъ грѣшникъ высочайшей Его власти не хощетъ покарятися, яко не хощетъ слушати Его и святыхъ Его повелѣній творити. Богъ глаголетъ всякому: \textit{уклонися отъ зла, и твори благое}\footnote{33,~15.}; но грѣшникъ безчувственный не внимаетъ тому, но противно тому творитъ, уклоняется добра и творитъ злое, и такъ безстыдно и беззаконно презираетъ страшную Его власть. "--- 2)~Отъ разсужденія Божія величества, предъ которымъ несмысленный грѣшникъ не хощетъ смиритися. Богъ нашъ есть Богъ вѣчный, безначальный, безконечный, живый, безсмертный и есть величества безконечнаго и неописаннаго, такъ что предъ Нимъ весь свѣтъ какъ ничто есть, якоже видимъ въ Писаніи святомъ. Аще же весь міръ, небо и земля и вси языцы, какъ ничто предъ Богомъ: кольми паче ты единъ, или я, который противу всего міра, какъ капля едина противу океана. Но грѣшникъ предъ безконечнымъ Его величествомъ не смиряется, и такъ не отдаетъ величеству Его почитанія, но паче презираетъ Его, "--- что есть страшная діавольская гордыня. Не смиряется же, яко не слушаетъ Его, не слушаетъ же, яко не исполняетъ Его святыхъ заповѣдей. Всякое бо преслушаніе, и отъ того грѣхъ, отъ гордыни происходитъ. Якоже бо смиреніе покорно и послушливо, тако гордость есть корень непослушанія. "--- 3)~Отъ разсужденія вездѣсущія Божія, которому грѣшникъ не отдаетъ достойныя чести. Богъ небо и землю исполняетъ, и на всякомъ мѣстѣ есть. И всякій человѣкъ, что ни дѣлаетъ, все предъ Нимъ дѣлаетъ. Онъ на всякое наше дѣло, начинаніе и помышленіе смотритъ и всякое слово слышитъ. Но когда грѣшникъ беззаконнуетъ, напр. блудодѣйствуетъ, похищаетъ, крадетъ, хулитъ, сквернословитъ, кощунствуетъ, льститъ, лжетъ, обманываетъ, ссорится, гнѣвается, ярится и прочія беззаконія творитъ, "--- безчинствуетъ предъ святѣйшимъ Его лицемъ, и такъ не отдаетъ Ему достойнаго почитанія. Якоже бо земному царю, или другому какому властелину, или высокому лицу, или отцу, не отдаемъ почтенія, когда предъ ними безчинствуемъ, то есть дѣлаемъ что непристойное лицу ихъ: тако наипаче не отдаемъ почитанія Богу, Творцу, Царю небесному и Отцу щедротъ, когда предъ Нимъ беззаконнуемъ; всякое бо беззаконіе предъ Нимъ и всевидящимъ Его окомъ творится какъ и всякое другое дѣло; и тѣмъ беззаконникъ досаждаетъ Ему, какъ и безчинникъ присутствующему царскому или другому какому почтенному и высокому лицу. "--- 4)~Причину печали по Бозѣ подаетъ безстыдное отвращеніе человѣка отъ Бога ко грѣху. Человѣкъ когда отвращается отъ Бога ко грѣху, тогда обращается лицемъ ко грѣху, а хребтомъ къ Богу: якоже обращающійся отъ востока къ западу, обращаетъ лице свое къ западу, а хребетъ къ востоку. Грѣхъ бо и Богъ суть противныя вещи, и когда къ одному обращаемся, отъ другаго отвращаемся, и къ тому хребетъ обращаемъ. Тако грѣшникъ безстыдный, отвращаяся отъ Бога ко грѣху, хребетъ обращаетъ къ Богу, якоже о беззаконныхъ Израильтянахъ глаголетъ Богъ чрезъ пророка: \textit{обратиша хребетъ ко Мнѣ, а не лице}\footnote{Іерем.~32,~33.}. Якоже бо Израильтяне оные, обращаяся отъ Бога живаго ко идоламъ, обращали хребетъ свой, по словеси Божію, къ Богу: тако нынѣ несмысленные хрістіане, когда обращаются ко грѣху, впадаютъ въ идолопоклонство, не чувственное, но духовное, и тако \textit{обращаютъ хребетъ къ Богу}, а не лице, что есть безстыдство великое. Ибо таковый грѣхъ, какъ идола, почитаютъ; и сколько разъ съ произволеніемъ его совершаютъ, столько разъ руки къ нему простираютъ, и колѣно ему преклоняютъ. Человѣкъ прежде согрѣшенія стоитъ между двумя противными вещами "--- Богомъ и сатаною; и имѣетъ свободное произволеніе къ тому или другому обратитися. Богъ зоветъ его къ добру, и отзываетъ отъ зла: сатана прельщаетъ и отзываетъ отъ добра, и склоняетъ ко злу и грѣху "--- дѣлу своему. И такъ, когда слушаетъ Бога и добро творитъ, "--- къ Богу обращается лицемъ: а когда слушаетъ сатаны и зло творитъ, "--- къ сатанѣ обращается лицемъ, а хребетъ обращаетъ къ Богу; и тако, отвратившися отъ Бога, въ слѣдъ сатаны идетъ. Отъ сего видѣти можешь, хрістіанине, какъ тяжко согрѣшаетъ человѣкъ предъ Богомъ, когда на грѣхъ, діавольское дѣло, обращается. "--- 5)~Разсужденіе вседержительства Божія, въ которомъ вси заключаемся, подаетъ вину и матерію грѣшнику къ печали по Бозѣ. Богъ вся въ руцѣ Своей содержитъ, якоже поетъ пророкъ: \textit{въ руцѣ Его вси концы земли}\footnote{Пс.~94,~4.}. Откуду называется \textit{Вседержитель}; въ которой вседержительной руцѣ и грѣшникъ содержится, сохраняется, живетъ, движется и есть: яко \textit{въ Немъ живемъ, движемся и есмы}\footnote{Дѣян.~17,~28.}. Но когда человѣкъ грѣшитъ и беззаконнуетъ: Бога, Содержителя своего, безчеститъ и оскорбляетъ, и дѣлаетъ подобно тому несмысленному младенцу, который, будучи во объятіяхъ матери и отца своего, хулитъ, біетъ и плюетъ на него. Подобно сему дѣлаетъ и безстыдный грѣшникъ Богу, Господу и Отцу своему, у Котораго во объятіяхъ сохраняется, хотя того и не примѣчаетъ ослѣпленный. "--- 6)~Отъ разсужденія благости Божіей, которая сохраняетъ насъ отъ козней діавольскихъ можемъ жалѣніе сіе имѣть. Человѣкъ когда грѣшитъ, сатана присѣдитъ ему и хощетъ душу восхитить и во дно адово низвергнуть; но Богъ милосердый не дозволяетъ ему. И пока въ нераскаяніи и развращеніи пребываетъ, тойжде душевный врагъ вси пути его наблюдаетъ и хощетъ его погубить; алчетъ бо и жаждетъ погибели нашей, яко завистливый и злобный духъ, какъ песъ крови: но Богъ запрещаетъ ему, и держитъ его силою Своею. Видишь, человѣче, какая благость и человѣколюбіе Божіе ко грѣшнику! Грѣшникъ прогнѣвляетъ и огорчеваетъ Бога; но Богъ не оставляетъ его, и не предаетъ въ руки врагу его. Грѣшникъ отступаетъ отъ Бога, и предаетъ себе врагу своему діаволу; но Богъ запрещаетъ ему погубить его. О благости Божія, непобѣдимыя грѣхами нашими! О неразумія и безчувствія грѣшникова!.. Кто можетъ сію благость Божію къ намъ грѣшнымъ постигнуть? Грѣшная и заблуждшая душа сего не разумѣетъ и ни во что вмѣняетъ! "--- 7)~Разсужденіе о безчисленныхъ Божіихъ благодѣяніяхъ, показанныхъ и показуемыхъ намъ, но отъ всякаго грѣшника презрѣнныхъ, можетъ его привести въ сокрушеніе и жалѣніе истинное. Чего Богъ, Создатель нашъ, не сдѣлалъ ради насъ? какого добра Своего не явилъ намъ и не являетъ? Заключены мы въ любви Его и благодѣяніяхъ, отъ человѣколюбія происходящихъ. Особеннымъ совѣтомъ и дивнымъ создалъ насъ; сотворилъ насъ по образу Своему и по подобію; весь свѣтъ въ служеніе наше подалъ намъ: солнце, луна и звѣзды Его свѣтятъ намъ; облака Его, яко мѣхи, разносятъ воду надъ нами и кропятъ на насъ и нивы наша; земля съ плодами, скотами и звѣрьми служитъ намъ; моря, озера и рѣки съ рыбами работаютъ намъ, и проч. Самъ Онъ, Господь нашъ, пришелъ на землю ради насъ, взыскати и спасти погибшихъ насъ. Но грѣшникъ окаянный все тое ни во что вмѣняетъ, и яко \textit{мужъ безуменъ не познаетъ, и неразумивъ не разумѣетъ сихъ}\footnote{Пс.~91,~7.}. Хощетъ насъ блаженными, праведными и святыми сдѣлать Господь нашъ, хощетъ любитися отъ насъ, какъ и Самъ любитъ насъ, и такъ дружество съ нами имѣти; хощетъ пріити къ намъ и обитель у насъ сотворити; хощетъ Отцемъ нашимъ быти, и насъ сынами Своими сотворити; хощетъ подати намъ наслѣдіе небеснаго царствія и вѣчныхъ благъ. Что можетъ быти большее и достойнѣйшее сего? Но грѣшникъ все тое Божіе благоволеніе (о неблагодарности и неистовства!) пренебрегаетъ. И такъ хощетъ Богъ грѣшника омыть, очистить, освятить, спасти и прославить вѣчно (о человѣколюбія непостижимаго!); но грѣшникъ отвращается, убѣгаетъ такъ великаго Любителя своего и Благодѣтеля, и самовольно въ вѣчную стремится погибель, діаволу и аггеломъ его уготованную! Богъ посылаетъ въ слѣдъ его рабовъ Своихъ, служителей и проповѣдниковъ слова Своего, и чрезъ нихъ зоветъ его къ Себѣ, обращаетъ и привлекаетъ; но грѣшникъ не хощетъ слышати гласа Его, обратитися и пріити ко Господу своему! Такъ злоба діавольская ожесточаетъ сердце человѣческое къ огорченію и прогнѣванію Бога, Который есть едина любовь, благость, святыня и кротость: чего оплакать по достоинству не можемъ, хрістіанине! "--- 8)~Грѣшникъ когда тое почитаніе, честь, славу и любовь, которую долженъ Богу, яко Создателю, Искупителю и Промыслителю своему, отдавать, къ себѣ обращаетъ: тогда почитаетъ себе паче Бога, Котораго долженъ паче всего, паче себе самого и воли своея, паче всего созданія, паче отца и матери, и всего, что кромѣ Бога есть, почитать; и такъ на томъ мѣстѣ, на которомъ долженъ Бога имѣть и почитать, себе какъ идола поставляетъ и боготворитъ. Тогда же сіе бываетъ, когда онъ волю свою волѣ Божіей предлагаетъ и предпочитаетъ; что Богъ хощетъ, того онъ не дѣлаетъ, и что Богъ повелѣваетъ, того онъ не исполняетъ; и такъ презрѣвши волю Божію, своей волѣ послѣдуетъ; и такъ самовольно дѣлается какъ верховный господинъ и неподвластный Богу, "--- что есть богомерзкое отступство отъ Бога, и самого себя боготвореніе. Подвластный бо власти своей покаряется и повинуется, какъ"=то: подданный царю, рабъ господину, сынъ отцу и матери своей покаряется. Грѣшникъ неисправный, когда покоренія и повиновенія Богу не показуетъ, тѣмъ показуетъ себе Богу неподвластнымъ и неподчиненнымъ, какъ сіе всякъ можетъ видѣть и признать. Иное бо есть грѣхъ отъ \textit{немощи и невѣдѣнія}, которому и богобоящіися люди подлежатъ; иное есть отъ \textit{произволенія}, продерзости и противу совѣсти грѣхъ, который не можетъ быть, какъ отъ сердца непокориваго и безстрашнаго.

\paragraph*{§\:437.} Примѣчается во многихъ грѣшникахъ и беззаконникахъ, что они властямъ, господамъ, родителямъ и благодѣтелямъ своимъ по закону естественному усердствуютъ, но Богу того показывать не хотятъ. 1)~Многіи царей, господъ и родителей своихъ повелѣнія и приказы слушаютъ и исполняютъ, какъ и должно; но Божіимъ повелѣніямъ, которыя по вся дни въ храмахъ святыхъ проповѣдуются и въ книгахъ святыхъ ради вѣдѣнія и исполненія написаны, никто отъ нихъ не хощетъ исполнять. Человѣку "--- царю, господину и отцу покаряются и повинуются; но Богу, Котораго власти цари, господа и отцы подлежатъ и повиноваться должны, "--- Богу, Который есть Царь царей и Господь господей и Отецъ всѣхъ, не хотятъ повиноваться; слышатъ слово Его, и не внимаютъ ему, что есть великая беззаконнаго сердца неправость. Человѣка слушать, и Бога, ради Котораго и всякаго властелина слушать должно, не слушать, "--- кто не признаетъ за превеликую неправость? Сіе всякій беззаконникъ творитъ, ибо не хощетъ заповѣдей Божіихъ творить. И сіе"=то есть, что глаголетъ Господь чрезъ пророка Своего: \textit{утвердиша слово сынове Іанадава, сына Рихавля, еже заповѣда сыномъ своимъ, еже не пити вина, и не пиша даже до дне сего, яко послушаша заповѣди отца своего: Азъ же глаголахъ вамъ заутра, и не послушасте Мене; и посылахъ къ вамъ рабы Моя пророки, утреннюя и глаголя: обратитеся кійждо отъ пути своего злаго}, и проч., "--- \textit{и не преклонисте ушесъ вашихъ, и не послушасте. И уставиша сынове Іонадавли, сына Рихавля, заповѣдь отца своего, юже заповѣда имъ: а людіе Мои не послушаша Мене}\footnote{Іерем.~35,~14--16.}. Сіе слово Божіе и жалоба Его до всякаго грѣшника неисправнаго касается, который заповѣди властелина, господина и отца своего слушаетъ, но Бога, Господа Вседержителя не слушаетъ. Что властелинъ или господинъ, или отецъ прикажетъ, тое исполняетъ и дѣлаетъ, но что Богъ приказываетъ, тому не внимаетъ окаянный грѣшникъ; человѣка слушаетъ, но Бога, Создателя и Вседержителя не слушаетъ: что сего безумнѣе и хуже можетъ быть? "--- 2)~Кто дерзнетъ предъ царемъ земнымъ безчинствовать, и законы его нарушать; или паче, предъ низшимъ властелиномъ, или господиномъ, или отцемъ своимъ кто безчинствуетъ? Самыи беззаконники сего творить или боятся, или стыдятся. Но предъ Богомъ, вездѣсущимъ и на вся смотрящимъ, грѣшникъ безчинствовать, законъ Его святый и вѣчный нарушать и часто таковая творить, о каковыхъ срамно есть писать и глаголати, не стыдится, не боится, не ужасается. И неотмѣнно таковыи или, по слову Псаломника, глаголютъ въ сердцѣ своемъ: \textit{нѣсть Богъ}\footnote{Пс.~13,~1.}; или думаютъ, что Богъ не вездѣ есть, и беззаконій ихъ не видитъ, или неправосудливъ есть; или иныя величеству Божію противныя мысли питаютъ въ сердцахъ своихъ; или въ глубокомъ о Бозѣ забвеніи имѣются. Иначе бы не дерзали беззаконій безстрашно дѣлать. Однакожъ кажется, что совѣсть, хотя и помраченная, не престаетъ всякаго беззаконника обличать и устрашать Божіимъ судомъ. Но грѣшникъ окаянный, презрѣвъ тую и, какъ конь свирѣпый, перервавши узду, стремится на всякое беззаконіе; и чего предъ человѣкомъ честнымъ стыдится, того предъ величествомъ Божіимъ не стыдится, ни боится дѣлать. "--- 3)~Многіи грѣшники любителей своихъ любятъ, и за добро, какое отъ нихъ получили или получаютъ, благодарятъ имъ, почитаютъ ихъ, услуживаютъ имъ, и имя ихъ вездѣ съ похвалою поминаютъ, и хвалятся ими, яко благодѣтелями своими, хотя то человѣкъ человѣку не свое собственное, но Божіе добро даетъ (все бо Божіе добро есть, какое ни имѣемъ), однакожъ и за тое благодарнымъ себе благодѣтелямъ своимъ показуютъ: но тыежде грѣшники Богу, отъ Котораго безчисленныхъ благодѣяній, отъ человѣколюбія единаго происходящихъ, Который собственная Своя благая всѣмъ подаетъ, на всякій день сподобляются, "--- однакожъ не благодарятъ. Не благодарятъ Ему, что созданы отъ Него, что питаются и одѣваются отъ Него, что сохраняются отъ Него, что ради ихъ Сына Своего въ міръ послалъ Онъ, и ради всѣхъ на смерть предалъ Его, да вси спасутся. Сего не разумѣютъ неисправные грѣшники, и за сіе такъ великое и чудное дѣло не благодарятъ любителю и благодѣтелю своему, Богу: яко не любятъ Его, не почитаютъ и не слушаютъ Его, безъ чего благодарность быть не можетъ. Не слушать бо, и тѣмъ прогнѣвлять и раздражать благодѣтеля, и благодарнымъ ему быть, дѣло невозможное есть. Къ благодарности требуется любовь, и отъ любви почитаніе. Любовь же не хощетъ любимаго оскорбить и раздражить, чего беззаконники не дѣлаютъ, слѣдственно и Богу не благодарни суть. "--- 4)~Съ какимъ усердіемъ, любовію, почтеніемъ и благодарностію читаетъ грѣшникъ письмо, которое царь къ нему напишетъ: часто въ руки беретъ его, прочитываетъ, любуется тѣмъ, и предъ прочіими хвалится тѣмъ; и какъ его, яко сокровище нѣкое, бережетъ и сохраняетъ, сказать невозможно. Богъ, Царь небесный и Господь Вседержитель, написалъ письмо "--- святое Свое слово, и послалъ его къ намъ всѣмъ безъ разбору, то"=есть благороднымъ и подлымъ, богатымъ и нищимъ, мнѣ, тебѣ и другому; послалъ чрезъ пророковъ и апостоловъ, рабовъ Своихъ, "--- въ которомъ открылъ намъ волю и благоволеніе Свое къ намъ. Но грѣшникъ Божественное тое письмо, аки не къ нему посланное, пренебрегаетъ. Книжками, духомъ міра сего наполненными, забавляется и утѣшается и время свое въ нихъ проводитъ; а Божія слова прочитать или послушать не хощетъ, и держать у себя сокровища того не хощетъ; а хотя и держитъ, то или на столѣ валяется, или въ шкапу мѣсто наполняетъ, но въ сердцѣ его не имѣетъ мѣста; и хотя вздумается когда ему взять въ руки тое, то такъ приступаетъ къ тому, какъ лихорадкою одержимый къ пищѣ. А отъ сего видно, какое почитаніе и къ Творцу того "--- Богу Самому имѣетъ! "--- 5)~Видимъ, какихъ трудовъ и подвиговъ не подъемлютъ грѣшники, когда царь обѣщаетъ имъ рангъ и честь, или иное какое награжденіе временное. Не токмо внутрь отечества служатъ ему со всякою охотою и усердіемъ, со всякимъ тщаніемъ и прилѣжаніемъ творятъ дѣло свое; но и внѣ отечества исходятъ на брань, ввергаютъ себе въ опасность смерти. Все сіе творятъ, чтобы отъ царя обѣщанную честь и отъ людей славу временную получить: сія ихъ надежда поощряетъ и подвигаетъ къ тому. Но тыежде тщательные слуги ради Бога, Который обѣщаетъ не временный, но вѣчный вѣнецъ, не временную честь и славу, но вѣчную слугамъ Своимъ, и такого служенія, какое ради человѣка и временныя корысти подъемлютъ, поднять не хотятъ. Что сіе есть, хрістіанине, какъ что они царю земному, обѣщающему рангъ и честь, вѣрятъ, и ради того подвизаются; а Царю небесному, обѣщающему вѣчную честь и славу, не вѣрятъ, и потому не хотятъ служить Ему, и въ подвигѣ хрістіанскаго благочестія стоять? Кто не пожелаетъ дружество съ царемъ земнымъ имѣти, къ нему на трапезу идти и съ нимъ веселитися? Кто не хощетъ сыномъ царскимъ быти и наслѣдствіе земнаго царства имѣти? Едва не всякъ сего Хрістовымъ именемъ знаменующійся, не отречется. Но смотри и примѣчай, желаетъ ли и ищетъ ли того, что Богъ неложный и истинный обѣщаетъ. Богъ обѣщается дружество съ нами имѣти: \textit{имѣяй заповѣди Моя и соблюдаяй ихъ, той есть любяй Мя; а любяй Мя, возлюбленъ будетъ Отцемъ Моимъ, и Азъ возлюблю его}\footnote{Іоан.~14,~21.}. Любити Бога и любитися отъ Бога, что есть, аще не дружество? Дружество бо есть взаимная любовь. Богъ обѣщается намъ Богомъ и Отцемъ нашимъ быти, и насъ за сыновъ Своихъ имѣти. \textit{И буду вамъ во Отца, и вы будете Мнѣ въ сыны и дщери, глаголетъ Господь Вседержитель}\footnote{2~Кор.~6,~18.}. Богъ зоветъ на велію вечерю въ царствіи небесномъ, и обѣщаетъ намъ наслѣдіе небеснаго царствія. Но кто сему великому и небесному званію внимаетъ? Кто ищетъ того, что неложный Богъ по милости Своей обѣщаетъ? Вси окаянніи грѣшники отрицаются. Возлюбили честь, славу, богатство и сласть міра сего паче, нежели честь, славу, богатство и утѣшеніе небесное и вѣчное\footnote{Іоан.~12,~43.}. Съ царемъ земнымъ дружества ищутъ, но съ Богомъ безсмертнымъ и Царемъ небеснымъ не ищутъ; на трапезу земнаго царя со тщаніемъ спѣшатъ, но къ трапезѣ небеснаго Царя не хотятъ идти; на земли царствовать временно желаютъ, но на небеси вѣчно царствовать не хотятъ. О прелести ума и сердца человѣческаго! \textit{Начаша вкупѣ отрицатися вси}\footnote{Лук.~14,~18.}. Чего? идти на уготованную небесную вечерю, и послушати званія Божія. Ради чего? понеже избрали временными и земными сластьми утѣшатися паче, нежели небесными и вѣчными. И тако званіе Божіе, которымъ чрезъ рабовъ Своихъ зоветъ ихъ, и Самого зовущаго Бога презираютъ. "--- Кто"=де отрицается и не хощетъ обѣщанныхъ оныхъ благъ получить? Не всякъ ли тѣхъ желаетъ? "--- Воистину мало такихъ есть, которые истиннымъ желаютъ сердцемъ! Кто бо чего искренно желаетъ, тотъ, все оставивши прочее, того единаго со всякимъ тщаніемъ и прилѣжаніемъ ищетъ. Какихъ трудовъ не подъемлетъ богатства любитель ради богатства, славолюбецъ и честолюбецъ ради славы и чести, сластолюбецъ ради желанной своей утѣхи! Колико трудится и потѣетъ мужикъ земледѣлецъ ради плода, ученикъ ради разума и премудрости внѣшнія! Кто сего не видитъ, что всегда предъ глазами нашими обращается? Тако разумѣй и о обѣщанныхъ отъ Бога благихъ. Якоже бо великая оная и неизреченная благая суть: тако великихъ трудовъ и подвиговъ отъ насъ къ полученію оныхъ требуютъ. Читай святое Евангеліе, и познаешь сію истину. Надобно отрещися не токмо мірскихъ похотей, но и \textit{себе самого и взяти крестъ свой, и послѣдовати Хрісту}\footnote{Матѳ.~16,~24.}, послѣдовати Ему не ногами, но сердцемъ, вѣрою, любовію, терпѣніемъ, кротостію, смиреніемъ, злостраданіемъ. И тако, гдѣ Онъ нынѣ Самъ, туды и послѣдователя Своего приведетъ, по неложному обѣщанію: \textit{аще кто мнѣ служитъ, Мнѣ да послѣдствуетъ; и идѣже есмь Азъ, ту и слуга Мой будетъ}\footnote{Іоан.~12,~26.}. Чимъ бо большее добро, тѣмъ большихъ трудовъ требуетъ къ полученію онаго. А какъ нѣтъ большаго добра паче вѣчнаго добра, и едино оно есть истинное добро: такъ большихъ трудовъ и подвиговъ отъ насъ требуетъ, чтобы тое постигнуть. \textit{Тако тецыте, да постигнете} вѣчное оное добро, глаголетъ Павелъ святый\footnote{1~Кор.~9,~24.}. Кто тое \textit{постигнетъ}? Не тотъ, кто лежитъ, но кто \textit{течетъ}. Кто \textit{течетъ}? Тотъ, кто, все прочее оставивши позади, къ тому единому, какъ своему центру, стремится, спѣшитъ и подвизается. А у кого нѣтъ труда и подвига къ тому, у того нѣтъ и желанія истиннаго къ тому. Желаніе истинное подвигаетъ желающаго къ труду, и научаетъ способъ искать, чтобы получить желаемое. Кто желаетъ въ Москву, или въ Петербургъ, или въ Кіевъ пріити, тою дорогою идетъ, которая въ тые грады ведетъ, а не иною. Тако къ Богу и вѣчному Его царствію хотящему пріитить должно \textit{узкими вратами и тѣснымъ путемъ идти, а не пространными вратами и широкимъ путемъ}, который \textit{вводитъ въ пагубу}\footnote{Матѳ.~7,~13 и 14.}. \textit{Многими бо скорбьми подобаетъ намъ внити въ царствіе Божіе}\footnote{Дѣян.~14,~22.}. И Господь глаголетъ: \textit{въ терпѣніи вашемъ стяжите души ваша}\footnote{Лук.~21,~19.}. И паки: \textit{въ мірѣ скорбни будете}\footnote{Іоан.~16,~33.}. Всякое бо зло и бѣдствіе срѣтаетъ тѣхъ, которые къ Богу и царствію Его стремятся. Должно все тое терпѣть, и противу враговъ "--- плоти, міра и діавола подвизаться, да трудомъ и подвигомъ съ помощію Божіею получимъ тое, что безъ труда въ началѣ имѣли и потеряли. "--- 6)~Кто прогнѣвать и оскорбить царя земнаго, или паче нижняго властелина, или высокое какое лице не боится? Но Бога, Иже \textit{велій Господь и Царь велій по всей земли, у Негоже въ руцѣ вси концы земли, и высоты горъ Того суть, Егоже есть море, и Той сотвори е, и сушу руцѣ Его создасте}\footnote{Пс.~94,~3--5.}, "--- такъ великаго Господа и Царя, Который небомъ и землею обладаетъ, безумный и окаянный грѣшникъ прогнѣвать и оскорбить, паче же на всякій день и часъ прогнѣвлять и оскорблять не стыдится, ни боится, ни ужасается. О долготерпѣнія Твоего Господи! О слѣпоты человѣческой! Такъ грѣхъ у хрістіанъ усилился и въ обычай вошелъ, что какъ шуточную какую вещь того поставляютъ. Воистину таковыи хрістіане въ превеликой тьмѣ ходятъ, хотя и мудрыми себѣ быть кажутся; и хотя звѣзды считаютъ и другихъ научаютъ, но хрістіанскаго алфавита не знаютъ; и не токмо безчувственны, но и мертвы суть, живутъ же токмо грѣху. "--- 7)~Во многихъ грѣшникахъ сіе примѣчается, что они отца, или властелина, или друга или благодѣтеля своего, или иное какое почтенное лице оскорбивше, каются, жалѣютъ и смиряются, и просятъ прощенія у него, признавая свою винность, что похвально есть. Но Бога, Творца и высочайшаго своего Благодѣтеля, Который всѣхъ создалъ, питаетъ, одѣваетъ и сохраняетъ, всегда оскорбляютъ; Бога, глаголю, Который есть едина любовь и благостыня, оскорбляютъ всегда, и не каются, ни скорбятъ, ни жалѣютъ, ни смиряются предъ Нимъ. И хотя многіи согрѣшивше каются и исповѣдуются предъ отцемъ духовнымъ, но, возвратившеся, паки на тыежде грѣхи обращаются: и такъ церемонію только отправляютъ, а не истинное покаяніе творятъ. Нѣтъ бо тамо истиннаго покаянія и жалѣнія, гдѣ грѣхи повторяются. Како бо кто можетъ каятися и жалѣть, что тое и тое дѣлалъ, но оставить того не хощетъ, тѣмъ и тѣмъ грѣхомъ Бога прогнѣвалъ, но прогнѣвлять Его не престаетъ? Ложное убо, неистинное покаяніе есть, когда кающійся грѣховъ не оставляетъ. Надобно убо жалѣть за грѣхъ, но и оставить грѣхъ; надобно и за тое жалѣть, когда мы ближняго обидѣли и оскорбили, но наипаче за тое, что тѣмъ Богъ прогнѣвали. Ибо кто противу ближняго согрѣшитъ, тотъ и противу Бога согрѣшитъ, заповѣдавшаго любить ближняго, и ничимъ не обиждать его. И человѣкъ, какъ ни высокъ, или какъ ни великій нашъ благодѣтель есть, ничто есть противу Бога; потому и согрѣшеніе противу человѣка всякаго есть ничто противу согрѣшенія, которое бываетъ противу Бога. Откуду Давидъ царь, хотя и противу Уріи тяжко согрѣшилъ, однакожъ кающійся глаголетъ Богу: \textit{Тебѣ единому согрѣшихъ, и лукавое предъ Тобою сотворихъ}\footnote{Пс.~50,~6.}, разсуждая, что грѣхъ тотъ и обида, Уріи сотворенная, касается Самого Бога, Который есть верховный Законодавецъ, Судія, Господь, Царь, Создатель и Благодѣтель всѣхъ, и есть безконечнаго величества, такъ что весь міръ какъ ничто противу Его есть, "--- кольми паче единъ человѣкъ, кто бы онъ ни былъ. Сего ради согрѣшившему противу ближняго должно не токмо жалѣть, что оскорбилъ его, но наипаче, что Самого Бога и Законодавца, Который не велѣлъ никого обиждать, тѣмъ грѣхомъ прогнѣвалъ и оскорбилъ; и, смиряяся предъ ближнимъ и прося у него прощенія, смиряться наипаче предъ Богомъ, и у Него просить прощенія, что заповѣдь Его святую нарушилъ. "--- 8)~Многіи хрістіане какъ сильно по славѣ и чести отца, или господина, или благодѣтеля своего ревнуютъ и защищаютъ имя ихъ, когда хулится! Но тыежде ревнители молчатъ, стыдятся или боятся защищать славу имени Божія, когда оно предъ ними отъ безбожныхъ хулится. О хрістіане! такъ мы вѣрны Господу Богу, создателю, искупителю и промыслителю нашему, Который насъ и отцевъ, господъ и благодѣтелей нашихъ создалъ, питаетъ, одѣваетъ и сохраняетъ, что имя Его святое страшное и такого почитанія не сподобляется отъ насъ, какое отдаемъ человѣкамъ, созданію Его, отцамъ, господамъ и благодѣтелямъ нашимъ! "--- Такожде многіи поступаютъ, когда кто имъ самимъ досадитъ и имя похулитъ. О властяхъ рѣчь есть, которымъ наипаче долгъ належитъ славу и честь имени Божія защищать. Какъ ревнуютъ, ярятся, гнѣваются и защищаютъ честь и славу свою! Но тыежде защитники тихо и кротко поступаютъ съ подчиненными, которые непрестанно честь имени Божія преступленіемъ закона Его умаляютъ, или явно на имя Его отрыгаютъ хулы. Что сіе есть, какъ имя и честь свою, которая ничто есть, паче имени и чести Божія безконечныя почитать? Увидишь, что есть имя и честь твоя, когда равно съ подлыми умирать будеши и въ землю отъидеши; и познаеши, что есть имя и честь Божія, которую нынѣ пренебрегаеши, когда предстанешь страшному Его суду, гдѣ \textit{царіе земстіи и вельможи и богатіи, тысящницы и сильніи, и всякъ рабъ и всякъ свободь, пожелаютъ сокрытися въ пещерахъ и каменіи горстѣмъ}\footnote{Апок.~6,~15.}. "--- 9)~Видимъ, что многіи хрістіане, называющіи Бога помощникомъ и защитникомъ своимъ, Который есть подлинно помощникъ и защитникъ хрістіанамъ, когда въ бѣду впадутъ, ко княземъ, вельможамъ и прочіимъ высокимъ по міру сему лицамъ ради защищенія прибѣгаютъ, и слыша отъ нихъ обѣщаемое о себѣ ходатайство, надѣются на нихъ, хотя и сами, какъ человѣки, всякой бѣдѣ подлежатъ. Богъ всемогущій и преблагій и премудрый, Который единъ защищаетъ и избавляетъ, обѣщаетъ помощь всякому призывающему Его и глаголетъ: \textit{призови Мя въ день скорби твоея, и изму тя}\footnote{Пс.~49,~15.}. Но они оставивше Бога, къ человѣкамъ въ день скорби своея прибѣгаютъ, и тако дотолѣ Бога за помощника своего имѣютъ, доколѣ скорби не имѣютъ; а въ день скорби другихъ помощниковъ и защитниковъ ищутъ. Суть такіи, которые въ сребрѣ и златѣ, или хитрости, или силѣ, или санѣ своемъ помощи и защищенія своего ищутъ; и тако вси таковіи надѣются на человѣка и на прочее созданіе, а не на Бога и Создателя своего, а безъ надежды, и вѣры въ Него не имѣютъ. Надежда бо отъ вѣры неотлучна, какъ и вѣра отъ надежды; но едина съ другою совокупно пребываетъ. Отсюду заключается, что языкомъ только называютъ Бога помощникомъ, а въ сердцѣ другихъ помощниковъ имѣютъ, а тако устами только чтутъ Бога, а не сердцемъ. Безъ вѣры бо и надежды чтить Бога невозможно. Якоже бо человѣкъ за непочтеніе себѣ вмѣняетъ, когда его обѣщанію не вѣримъ: тако наипаче неложному Богу и истинному не отдаютъ почитанія грѣшники, которые милостивымъ Его обѣщаніямъ не вѣруютъ, но, оставивше Его, къ немощному созданію Его прибѣгаютъ ради помощи и защищенія; и дѣлаютъ они подобно тому несмысленному рабу, который отвращается отъ царя своего, который обѣщается ему милость явить, и въ руцѣ своей содержитъ животъ и смерть и все его благополучіе и неблагополучіе, обращается къ подлому и немощному его рабу, и милости отъ него ищетъ, чимъ и чести царской досаждаетъ, и себѣ болѣе вредитъ, нежели пользуетъ. Тако бѣдныи грѣшники и Божію величеству досаждаютъ, что не вѣрятъ милостивымъ Его обѣщаніямъ: \textit{не вѣруяй бо Богови, лжа сотворилъ есть Его}\footnote{1~Іоан.~5,~10.}, "--- и себѣ тѣмъ вредитъ, когда отъ Создателя своего, у Котораго въ руцѣ животъ нашъ и смерть, благополучіе и неблагополучіе наше есть, отвращаются, и обращаются къ созданію Его, которое само безъ помощи, подкрѣпленія и содержанія всемогущія Божія руки падаетъ. "--- 10)~Видимъ паки, что многіи хрістіане отцамъ своимъ по плоти тщатся угождать, хотя и злымъ, дабы наслѣдія по отцахъ своихъ не лишиться: но тыежде человѣкоугодники Богу, Отцу небесному, Который даетъ чадамъ Своимъ, угождающимъ Ему, вѣчное наслѣдіе и неизреченное, угождать не тщатся, дабы того не лишиться. Что сіе есть, хрістіанине, какъ почитать временное и сновидѣнію подобное блаженство паче истиннаго и вѣчнаго? а тако и уготовавшаго тое и обѣщавшаго не почитать и презирать Бога? О какъ горько возрыдаютъ сіи наслѣдники, когда въ рукахъ своихъ ничего не увидятъ, кромѣ единаго лишенія и пагубы; когда Божіи угодники \textit{наслѣдятъ царствіе, уготованное имъ отъ сложенія міра, а они изгнани будутъ во тму кромѣшнюю, гдѣ будетъ плачъ и скрежетъ зубомъ}\footnote{Матѳ.~25,~34,~30 и 41.}. "--- 11)~Видимъ паки, что многіи хрістіане, когда отцы ихъ, или власти, или господа, за преступленіе не накажутъ ихъ, радуются, утѣшаются и благодарятъ имъ; но Богу, которому на всякій день и часъ согрѣшаютъ, и безконечнаго милосердія Его сподобляются, такъ что оная милость, съ сею Божіею милостію сравненная, ничтоже есть, однакожъ Ему не благодарятъ. Судъ бо, гнѣвъ и наказаніе человѣческое тѣла единаго касается, души же повредити и погубити не можетъ: но Божій судъ и гнѣвъ можетъ и душу и тѣло въ геенну огненную низринути\footnote{10,~28.}. Но грѣшникъ ослѣпленный того не примѣчаетъ, и человѣку, малую ему милость являющему, благодаритъ; но Богу, Который на каждый день и часъ ему безконечную являетъ милость, не благодаритъ. О когда бы всякому грѣшнику открылося внутреннее око, и позналъ бы, сколько онъ предъ величествомъ Божіимъ на всякій день то словомъ, то дѣломъ, то помышленіемъ согрѣшаетъ, досаждаетъ и оскорбляетъ, и коликой отъ Него въ семъ милости сподобляется: непрестанно бы падалъ предъ Нимъ со смиреннымъ и благодарнымъ сердцемъ!.. Не токмо бо словомъ изобразить, но и умомъ понять невозможно, сколько милости Своея въ семъ являетъ намъ Господь! Который Царь такъ милостивъ и кротокъ есть, который, видя законъ свой предъ собою разоряемый, разорителю можетъ стерпѣть? Скоро человѣческая кротость въ гнѣвъ и ярость претворяется. Богъ видитъ на всякое время законъ Свой святый, непремѣняемый и вѣчный, предъ Собою отъ грѣшника разоряемый, видитъ и терпитъ; и не токмо терпитъ, но и хранитъ грѣшника отъ врага его діавола, который согрѣшающему присѣдитъ и хощетъ душу его восхитить и во адъ низринуть. О человѣколюбія Твоего къ намъ грѣшнымъ, Боже нашъ! Кто можетъ сіе изслѣдити, или умомъ поняти? Но грѣшная душа сего не разумѣетъ, и Богу человѣколюбцу не благодаритъ. Сіе же долготерпѣніе Божіе спасенія грѣшникова хощетъ и ищетъ, якоже учитъ Апостолъ: \textit{Господа нашего долготерпеніе, спасеніе непщуйте}\footnote{2~Петр.~3,~15.}. Яко благость Божія на покаяніе его ведетъ\footnote{Рим.~2,~4.}. Которую благость, долготерпѣніе Божіе, когда грѣшникъ нераскаянный сердцемъ презритъ, тогда вси его худые поступки, грѣхи и беззаконія обличатся и представятся ему въ день суда, якоже глаголетъ Богъ грѣшнику: \textit{обличу тя, и представлю предъ лицемъ твоимъ грѣхи твоя}\footnote{49,~21.}. И тогда судъ и гнѣвъ Божій на себѣ почувствуетъ; и чимъ болѣе грѣшилъ и долготерпѣнія Божія на себѣ дозналъ, тѣмъ болѣе уже гнѣвъ Божій дознаетъ, якоже учитъ апостолъ: \textit{по жестокости твоей}, грѣшниче, \textit{и непокаянному сердцу, собираеши себѣ гнѣвъ въ день гнѣва и откровенія праведнаго суда Божія}\footnote{2,~5.}. \textit{Разумѣйте убо сія забывающіи Бога, да не когда похититъ, и не будетъ избавляяй}, глаголетъ пророкъ грѣшникамъ\footnote{Пс.~49,~22.}. "--- 12)~Примѣчается и тое въ хрістіанахъ, что многіи съ царемъ, или высокимъ какимъ лицемъ бесѣдовать съ радостію желаютъ, яко за честь себѣ тое поставляютъ. Что честнѣе, какъ съ Богомъ безсмертнымъ и живымъ молитвою бесѣдовать? Къ чему и Самъ Богъ по Своему человѣколюбію призываетъ насъ, какъ выше о томъ неоднократно сказано. Но грѣшная душа не разумѣетъ и пренебрегаетъ тое. Царь земный зоветъ къ бесѣдѣ, и какъ спѣшится, сказать неможно; но Царь небесный всегда зоветъ, и не хощемъ. Однакожъ хотящему съ Богомъ въ молитвѣ бесѣдовать должно прежде очистить себе покаяніемъ. Царь земный не терпитъ смрадныхъ рубищъ: тако Царь небесный отвращается души, которая смрадными грѣховъ рубищами замарана. Должно убо сіи рубища души отъ себя отринуть, и тако къ небесному Царю приступать. "--- 13)~Паки примѣчаемъ, что хрістіане имя царево почитаютъ, и при воспоминаніи его покровъ съ головы снимаютъ, или голову приклоняютъ, какъ и должно: но святое и страшное имя Божіе того не удостояется отъ многихъ хрістіанъ. Иныи въ неправдѣ и лжи тое поминаютъ; иныи въ ложной присягѣ воспріемлютъ тое, и, учинивше присягу и тую великимъ именемъ Божіимъ утвердивше, тотчасъ на дѣла, противныя присягѣ, устремляются. Другіе въ непотребныхъ и подлыхъ дѣлахъ и словахъ безпрестанно вносятъ тое, повторяя обыкновенныя себѣ слова: \textit{ей Богу, на то Богъ, Свидѣтель Богъ!} и проч. Суть такіе, которые въ шуткахъ и кощунствѣ (о долготерпѣнія Твоего, Боже!) дерзаютъ принимать имя Его, ангеламъ святымъ и душамъ сладкое, демонамъ страшное и всей твари честное. Тако хрістіане таковые, отъемля имени Божію достойное почитаніе, и Самому Богу, въ имени Своемъ святомъ почитаемому, не воздаютъ того. О хрістіанине несмысленный! Тако ли ты почитаеши имя Создателя и Искупителя твоего, что и такого почтенія, какое человѣку смертному показуешь, не удостояеши Его?.. 14)~Многіи, разсуждая честность и благородіе отецъ своихъ, имя ихъ берегутся опорочить худыми поступками; удаляются отъ такихъ дѣлъ, которыми тое безчестится: но имя небеснаго Отца, къ Которому мнятся молитися: \textit{Отче нашъ, Иже еси на небесѣхъ}, непрестанно безчестятъ, дѣлая такія дѣла, которыя хрістіанскому званію противны. Таковыи суть блудники, прелюбодѣи, пьяницы, хищники, грабители, клятвопреступники, разорители и прочіи, симъ подобныи. Якоже бо добрыми хрістіанами имя Божіе славится, по свидѣтельству Господню: \textit{тако да просвѣтится свѣтъ вашъ предъ человѣки яко да видятъ ваша добрая дѣла, и прославятъ Отца вашего, Иже есть на небѣсѣхъ}\footnote{Матѳ.~5,~16.}: тако злыми дѣлами безчестится. А отсюду видно, что таковые честные люди болѣе любятъ и почитаютъ отца своего "--- человѣка, нежели Бога, небеснаго Отца, или паче никакой любви и почитанія къ Нему не имѣютъ. Злый бо и неисправный хрістіанинъ Бога любить и почитать не можетъ, пока неисправенъ пребываетъ. "--- 15)~Многіи неисправно живущіи отцы на сыновъ, господа на рабовъ, власти на подчиненныхъ своихъ негодуютъ и гнѣваются, что отъ нихъ должнаго себѣ не видятъ послушанія. И есть правильная причина гнѣва и негодованія ихъ. Всякъ бо сынъ отцу, рабъ господину и подчиненный власти своей долженъ показывать послушаніе. Такожде, сотворившіи какое благодѣяніе человѣкамъ, не терпятъ, когда отъ нихъ не видятъ благодарности. Но сами сіи негодователи и ревнители Богу, Верховному Господу, не показуютъ послушанія и не отдаютъ благодарности. И почто на себе не гнѣваются? Развѣ велико есть, что ихъ человѣковъ человѣки не слушаютъ; а мало, что они сами Бога, Которому и они сами и подчиненніи ихъ подчинены, не слушаютъ? Или, развѣ добро, отъ человѣка человѣку сотворенное, достойно есть благодарности, а отъ Бога сотворенное не достойно? Какая душа, хотя малую разума искру имѣющая, стерпитъ сіе? Кому болѣе и послушаніе показывать намъ, какъ Богу, ради Котораго и отцамъ, господамъ и властямъ нашимъ хрістіане повиноватися должны? И кому болѣе благодарить намъ, какъ Богу, отъ Котораго едино добро происходитъ, и Котораго добро и благотворящіи намъ подаютъ, а не свое? Все бо, что ни имѣетъ человѣкъ, кромѣ немощи, растлѣнія и грѣховъ своихъ, Божіе есть, а не его собственное: тѣло и душа Божія есть. Однакожъ негодуетъ человѣкъ, когда сотворитъ кому добро и не чувствуетъ отъ него благодарности: а самъ на всякій день и часъ отъ Бога получаетъ добро, и неблагодаренъ пребываетъ къ Нему, но не примѣчаетъ того. Такожде злыи отцы на сыновъ, господа на рабовъ и власти на подчиненныхъ неисправныхъ гнѣваются; но сами всегда неисправны предъ Богомъ пребываютъ, и мнятъ какъ бы тое ничто есть. Аще убо правильно есть неисправнымъ властямъ, господамъ и отцамъ на подчиненныхъ гнѣваться за непослушаніе ихъ: то на себе прежде гнѣваться имъ должно, что сами предъ Богомъ, яко Ему подчиненніи, неисправны суть. И аще благотворящіи грѣшники негодуютъ за неблагодарность людей, подобныхъ себѣ: то наипаче на себе негодовать должны, что къ Богу, Творцу, и высочайшему Благодѣтелю своему, благодарности не имѣютъ. Всякій бо грѣшникъ какъ послушанія, такъ и благодарности не показываетъ Богу, доколѣ въ званіи и должности хрістіанской неисправенъ пребываетъ. Ибо какъ послушаніе, такъ и благодарность Богу безъ соблюденія заповѣдей Его, соблюденіе же заповѣдей безъ любви Его не бываетъ, по словеси Господню: \textit{имѣяй заповѣди Моя, и соблюдаяй ихъ, той есть любяй Мя}. И паки: \textit{не любяй Мя, словесъ Моихъ не соблюдаетъ}\footnote{Іоан.~14,~21 и 24.}. Видишь, хрістіанине, каковъ предъ Богомъ имѣется грѣшникъ неисправный: таковаго послушанія, почитанія, любве и благодарности не имѣетъ къ Богу, Создателю своему, какое послушаніе, почитаніе, любовь и благодарность человѣку показуетъ!..

\paragraph*{§\:438.} Какую неблагодарность Богу, такуюжде и Хрісту Сыну Божію, показуетъ грѣшникъ окаянный. И какъ къ Богу, такъ и къ Сыну Божію не имѣетъ почитанія, кромѣ устнаго и лицемѣрнаго, грѣшная душа. 1)~Видимъ какъ радуются граждане, когда царь во градъ пріидетъ; и съ какимъ желаніемъ, благоговѣніемъ и восклицаніемъ срѣтаютъ его вси, сказать невозможно. Царь небесный "--- Христосъ пришелъ въ міръ сей къ намъ, какъ во градъ нашъ, рукою Его сотворенный, \textit{и посѣтилъ насъ Востокъ свыше}\footnote{Лук.~1,~78.}. Поютъ Его пришествіе \textit{небесныя силы}\footnote{2,~13.}; показуетъ \textit{звѣзда}\footnote{Матѳ.~2,~9.}; проповѣдуютъ славы Божія проповѣдники\footnote{1~Кор.~1,~23; 2~Кор.~4,~5; Кол.~1,~28.}. Но грѣшная душа стоитъ недвижима, и не срѣтаетъ Его съ подобающею честію, не поклоняется Ему съ вѣрою и любовію; и хотя не глаголетъ устами съ противниками Его: \textit{не хощемъ Сему, да царствуетъ надъ нами}\footnote{Лук.~19,~14.}, но не хощетъ покорятися Ему, и взяти ига Его на себе, которое должно всѣмъ, пріемлющимъ Его за Царя своего, взять и носить, по повелѣнію Его: \textit{возмите иго Мое на себе}\footnote{Матѳ.~11,~29.}. А не пріемля ига Его, и самаго Его не пріемлетъ. Надобно бо пріемлющимъ Его за Царя своего и главу свою подклонить благому игу Его. Ибо невозможное дѣло есть быть подданнымъ царю, и закона его не пріимать и не покоряться ему; подданный бо царю ради того подданный нарицается, что власти его и закону подлежитъ и покоряется: тако и Царю небесному "--- Хрісту тіи только поддаются и пріемлютъ Его, яко своего Царя, которые пріемлютъ иго Его, и вѣрно работаютъ Ему. Прочіи же хотя и нарицаютъ Его Царемъ своимъ, но не пріемлютъ ига Его, и не покоряются Ему, не пріемлютъ Его за Царя своего. Надобно таковымъ хрістіанамъ опасаться того страшнаго суда Божія, которымъ грозитъ Хрістосъ Царь Іудеомъ, непріемшимъ Его: \textit{обаче враги Моя, оны, иже не хотѣша Мене, дабы Царь быхъ былъ надъ ними: приведите сѣмо, и изсѣцыте предо Мною}\footnote{Лук.~19,~27.}. Приведутся бо на судъ Его праведный вси, которые не пріемлютъ ига Его благаго, и не хотятъ Ему, яко истинному Царю, работать; изсѣкутся \textit{сѣкирою} праведнаго Его суда, и \textit{во огнь ввергнутся}\footnote{Матѳ.~3,~10; 25,~41.}. Тогда они познаютъ власть Его, которой не хотѣли нынѣ охотно и усердно повиноваться и слушать Его. \textit{Хрістосъ бо есть Царь царствующихъ и Господь господствующихъ}\footnote{1~Тим.~6,~15; Апок.~17,~14.}; и Его власти небо и земля, земная и преисподняя подлежатъ: однакожъ не принуждаетъ никого къ принятію ига и работѣ Своей; но призываетъ только чрезъ проповѣдниковъ Своихъ, и обѣщаетъ вѣчный покой работающимъ Ему, хотя и вси власти Его подданы суть. Пріемлющимъ убо Его, яко Царя своего, и работающимъ Ему вѣрою и любовію подастъ вѣчный покой въ небесномъ Своемъ царствіи, которому не будетъ конца; отвергающихъ же иго Его благое и не хотящихъ Ему, яко истинному своему Царю, покорятися и работати, властію и праведнымъ Своимъ судомъ предастъ вѣчному наказанію; яко Ему, яко Царю своему, Которому должны повиноватися и работати, не повиновалися и не работали. "--- 2)~Како паки радуется человѣкъ, когда пріидетъ и посѣтитъ домъ его царь, или хотя высокій какій и почтенный господинъ: съ какою охотою, усердіемъ и желаніемъ срѣтаетъ, принимаетъ и учреждаетъ его; съ какимъ благодареніемъ провождаетъ его, всѣмъ тое извѣстно. Царь небесный, Сынъ Божій хощетъ, яко Душелюбецъ, домы душъ нашихъ посѣтить, и стоитъ у всякаго при дверехъ (о человѣколюбія непостижимаго!), и толкаетъ: \textit{се стою при дверехъ и толку: аще кто услышитъ гласъ Мой, и отверзетъ двери, вниду къ нему, и вечеряю съ нимъ, и той со Мною}\footnote{Апок.~3,~20.}. Но грѣшная душа, безстыдно затворивши двери, не хощетъ сего высокаго, великаго и вожделѣннаго Гостя пустити и приняти. Увы намъ, грѣшники! Что мы это дѣлаемъ? Гдѣ наша вѣра, смыслъ и разумъ? Царь небесный стоитъ у домовъ нашихъ, и толкаетъ въ двери, и хощетъ съ царствіемъ Своимъ благодатнымъ пожити въ домахъ нашихъ, и со Отцемъ Своимъ небеснымъ обитель у насъ сотворити, якоже глаголетъ: \textit{аще кто любитъ Мя, слово Мое соблюдетъ: и Отецъ Мой возлюбитъ его, и къ нему пріидемъ, и обитель у него сотворимъ}\footnote{Іоан.~14,~23.}. Но мы не отверзаемъ Ему дверей нашихъ, и не хощемъ Его съ подобающею честію приняти. О крѣпкій и сильный Господи! Господи силъ! Удари крѣпко, и сокруши крѣпостію Твоею двери затворенныя, и воспріими домъ Твой, которымъ неправедно завладѣлъ чуждый, противникъ Твой. Надобно намъ, хрістіанине, нынѣ Его неотмѣнно въ домы наши принять, когда хощемъ съ Нимъ во вѣки жить. Нынѣ Онъ стоитъ и толкаетъ въ двери грѣшниковъ; но будетъ время, когда грѣшники, непослушавшіи и непріемшіи Его, \textit{начнутъ внѣ стояти и ударяти въ двери, глаголюще: Господи, Господи, отверзи намъ! И отвѣщавъ речетъ имъ Господь: не вѣмъ васъ, откуду есте. Отступите отъ Мене вси дѣлателіе неправды}\footnote{Лук.~13,~25 и 27.}. "--- 3)~Какъ паки радуется человѣкъ, когда къ нему, или въ плѣненіи находящемуся пріидетъ кто свободити его; или во глубинѣ утопающему руку помощи подать; или лютою болѣзнію одержимому здравіе подать, и какое тому благодѣтелю благодареніе приноситъ, кто сего не знаетъ? Кому бо свобода, животъ и здравіе не желаемо и не пріятно? Кто сего добра паче всего сокровища не почитаетъ? Хрістосъ пришелъ къ намъ, плѣненнымъ отъ діавола, обнаженнымъ и ураненымъ отъ тогожде врага, и лежащимъ на пути міра сего полумертвымъ\footnote{10,~30.}, и во глубинѣ погибели утопающимъ; и хощетъ всѣмъ подати свободу, здравіе и животъ. \textit{Пріиде бо Сынъ человѣческій взыскати и спасти погибшаго}\footnote{19,~10.}. Но о коль многіи, мнящіися быть хрістіане, не хотятъ Сего великаго Благодѣтеля и Избавителя разумѣти, и Его великою сею милостію пользоватися; изволяютъ лучше работати грѣху, и тѣмъ діаволу, нежели отъ его ига свободитися: и остатися въ погибели, нежели въ животъ внити. И дѣлаютъ подобно тому несмысленному плѣннику, который пришедшему его отъ плѣна свободити, не хощетъ послѣдовати, и отъ той бѣды свободитися. Тако Сынъ Божій, Господь нашъ, пришелъ къ намъ и вызываетъ отъ плѣна, и зоветъ въ слѣдъ Себе, да свободимся отъ того, и получимъ вѣчную небеснаго царствія свободу; но многіи несмысленніи, купно и неблагодарніи, а паче записавшіи Ему имя свое, сіе высокое Его благодѣяніе пренебрегаютъ, не хотятъ послѣдовати Ему, да свободни будутъ, и тако въ прежнемъ своемъ бѣдствіи и злополучіи остаются. Надобно бо неотмѣнно плѣннику за пришедшимъ его свободити, когда свободитися хощетъ, послѣдовати, и немощному слушати цѣлителя своего, и утопающему въ водѣ исполнити хотѣніе простирающаго ему руку: иначе никто отъ нихъ не можетъ отъ бѣдствія своего свободитися. Тако всякому, кто хощетъ свободитися отъ діавола, грѣха и смерти вѣчныя, должно неотмѣнно хотѣніе Хріста избавителя исполнити и предатися въ волю Его, да творитъ съ нимъ, яко лѣкарь съ больнымъ, и яко наставникъ съ заблуждшимъ, и яко свободитель съ плѣннымъ, и яко спаситель съ погибшимъ. Иначе свободитися и спастися невозможно. Хрістосъ бо, аще и человѣколюбецъ есть, \textit{хотящихъ} спасаетъ, а не нехотящихъ. "--- Кто"=де не хощетъ спастися? кто сего великаго добра себѣ не хощетъ? \textit{Отвѣтъ}. Хрістосъ глаголетъ: \textit{коль краты восхотѣхъ собрати чада твоя, Іерусалиме, якоже собираетъ кокошъ птенцы своя подъ крылѣ, и не восхотѣсте}\footnote{Матѳ.~23,~37.}! Видишь, что хощетъ Хрістосъ, но не хотятъ люди. Истинному хотѣнію послѣдуетъ соизволеніе, дѣло и трудъ. Какихъ трудовъ не подъемлютъ купцы ради желаемаго богатства, земледѣльцы ради плодовъ, воины ради побѣды и отъ той славы, и прочіи желающіи временныхъ вещей? Понеже вси они хотятъ получить желаемое, того ради и трудятся ради того, и чимъ большее добро, тѣмъ большій прилагается трудъ къ полученію добра того. Видишь, что хотѣнію дѣло и трудъ послѣдуютъ. А какъ нѣтъ большаго добра паче вѣчнаго, и едино оное добро есть истинное добро, и \textit{едино есть на потребу}, по словеси Господню\footnote{Лук.~10,~42.}: то и большій трудъ прилагать намъ должно къ полученію онаго, нежели трудимся ради временнаго. \textit{Многими бо скорбьми подобаетъ намъ внити въ царствіе Божіе}, по ученію Апостола\footnote{Дѣян.~14,~22.}. Видишь, что не токмо \textit{скорбьми}, но и \textit{многими скорбьми} входятъ въ царствіе Божіе. А гдѣ многія скорби, тамо много трудовъ требуется, чтобы преодолѣть оныя. Аще бо и отворилъ Хрістосъ смертію Своею дверь къ вѣчному животу, и подаетъ вѣрующимъ во имя Его; но нѣтъ къ нему инаго пути, кромѣ пути тѣснаго, якоже Самъ Господь глаголетъ: \textit{внидите узкими враты; яко пространная врата и широкій путь вводяй въ пагубу и мнози суть входящіи имъ. Что узкая врата, и тѣсный путь вводяй въ животъ, и мало ихъ есть, иже обрѣтаютъ его}\footnote{Матѳ.~7,~13 и 14.}. Плоть, или растлѣніе плоти, міръ и діаволъ суть враги хрістіанскіи. Сіи препятствіе чинятъ намъ, идущимъ къ оному животу; сіи запинаютъ ноги текущихъ къ полученію неистлѣннаго вѣнца: противу сихъ враговъ подвизаться надобно душѣ, хотящей спастися, и тако внити въ животъ, который подвигъ недремлющимъ окомъ отправляется, и то со всесильною Божіею помощію, которая \textit{просящимъ, ищущимъ и толкущимъ} посылается\footnote{7,~7 и 8.}. "--- 4)~Видимъ паки, какъ человѣкъ того своего благодѣтеля, который гнѣвающагося на него царя умолитъ и его въ милость царскую приведетъ, любитъ, благодаритъ ему, и незабвенно дѣло тое въ памяти своей содержитъ, а паче, когда отъ смерти, которой по законамъ и суду царскому подлежалъ, ходатайствомъ своимъ его избавитъ: какъ сіе дѣло за велико почитаетъ всякъ, всѣмъ извѣстно. Сынъ Божій, Іисусъ Хрістосъ, Бога, вѣчнаго и небеснаго Царя, разгнѣваннаго на насъ за грѣхи наши, яко вси мы согрѣшили Ему, умолилъ; и умолилъ не токмо горячею Своею молитвою и слезами, но и глубочайшимъ смиреніемъ, страданіемъ и смертію, смертію же крестною; благоволилъ Онъ за насъ отверженныхъ и изгнанныхъ странствовать, за бѣдныхъ, бѣдствовать, за проданныхъ подъ грѣхъ проданъ и преданъ быть, за ругаемыхъ и посмѣваемыхъ отъ сатаны поруганъ, посмѣянъ и оплеванъ быть, за проклятыхъ отъ закона \textit{клятвою быть}\footnote{Гал.~3,~13.}, за осужденныхъ осужденъ быть, за умершихъ умереть. О человѣколюбія Твоего, Іисусе! Видиши ли, хрістіанине, какъ дорого стало Сыну Божію исходатайствовать намъ милость у Бога, раздраженнаго нашимъ законопреступленіемъ? Учинилъ тое Заступникъ и Ходатай нашъ, и тако привелъ насъ въ милость Отцу Своему небесному; разрушилъ смертію крестною смерть, и отворилъ путь къ вѣчному животу, и входятъ въ него благодарнымъ сердцемъ пріемлющіи Его, яко Ходатая и Избавителя своего. Но грѣшникъ слѣпый и нераскаянный не чувствуетъ сего высокаго Его благодѣянія, и не благодаритъ Ему. Что временный животъ, который вси по долгу естества оставить должны, въ сравненіи съ вѣчнымъ есть? ничто есть!.. Однакожъ грѣшникъ за велико почитаетъ его и благодаритъ ходатаю того, но Хрісту, вѣчный животъ заслужившему, не благодаритъ. Увы намъ, бѣдныи грѣшники! Какъ великою тьмою объято сердце наше! "--- \textit{Какъ"=де хрістіанамъ не благодарить Хрісту? Отвѣтъ}. Безъ любви благодарность быть не можетъ, развѣ устная и лицемѣрная, какъ сіе всякъ довольно знаетъ. Сердце бо благодарное чувствуетъ, помнитъ и всегда въ памяти содержитъ благодѣяніе, и любитъ отъ чиста сердца благодѣтеля своего. Любителей же и нелюбителей Своихъ знаки показуетъ Самъ Господь: \textit{имѣяй заповѣди Моя, и соблюдаяй ихъ, той есть любяй Мя. Аще кто любитъ Мя, слово Мое соблюдетъ. Не любяй Мя, словесъ Моихъ не соблюдаетъ}\footnote{Іоан.~14,~21,~23 и 24.}. Чего не дѣлаетъ человѣкъ, чтобы любимому угодить и чимъ его не оскорбить? Не токмо богатства, но часто и чести, и, что болѣе, здравія своего не щадитъ? И, что ни хощетъ любимый, дѣлаетъ ради его любитель. Словомъ, любовь плѣняетъ сердце любителя, и въ слѣдъ влечетъ любимаго, и связуетъ съ любимымъ, и едино съ нимъ дѣлаетъ, что примѣчается въ плотской и скверной любви. "--- Ради Хріста, Которому благодарить и любовь имѣть хвалишься, что таковое дѣлаешь? Посмотри на сердце твое, чимъ оно плѣнено? Не любовію ли міра сего, которая отвращаетъ человѣка отъ любви Хрістовой? Когда славы, чести, почитанія, богатства и плотоугодія ищешь, то извѣстно знай, что міръ любишь, а не Хріста, и тѣмъ плѣнено сердце твое; той влечетъ тебе въ слѣдъ себе, и съ нимъ любовію связанъ пребываешь. Сыну своему богатое наслѣдіе и дочери изобильное приданое, которыхъ любишь, приготовляешь; богатые расширяешь домы, кареты и кони украшаешь ради гордости и тщеславія; драгими одеждами одѣваешься ради щегольства и пышности; изобильными трапезами и винами друговъ и гостей учреждаешь ради плотоугодія и человѣкоугодія; и на прочія любимыя твои утѣхи не щадишь имѣнія своего; но ради Хріста, ради Котораго и души не должны мы щадить, что таковое дѣлаешь? Съ чимъ, съ какою милостію отходятъ отъ тебе бѣдныя сироты и вдовицы, которыхъ велѣлъ Хрістосъ снабдѣвать, и которыя имя Хрістово тебѣ предлагаютъ, и просятъ отъ тебе милости? Праздны и съ тѣмижде слезами, съ какими приходятъ! Такъ"=то ты любишь Хріста, что и между человѣками, тобою любимыми и почитаемыми, не почитаешь Его! Такая твоя къ Нему и благодарность! Онъ ради тебе души Своея не пощадѣлъ, чтобы тебе отъ смерти избавить, смерти же вѣчныя: а ты ради Его и мѣди не хочешь подать! А гдѣ крестъ твой, который любящимъ Хріста должно носить, и въ слѣдъ Его ходить? Попранъ на земли лежитъ, и далеко отскочилъ ты отъ него; стыдишися и ужасаешися его, а съ нимъ и Хріста, крестъ понесшаго и на крестѣ распятаго. А что, когда нужда будетъ или отрещися Хріста, или смерти животъ свой пожрети, какъ то при гонителяхъ церкви было? Себе ли и животъ свой отдати восхощешь ради Хріста, чего вѣра въ Него и любовь къ нему отъ насъ требуетъ, когда ради Его денегъ, мѣди, сребра и злата не хощешь дать, поноснаго слова и скорби претерпѣть? Малаго не хощешь сдѣлать: великое ли сдѣлаешь? Денегъ, славы и чести не хощешь ради Его презрѣть: душу ли, которая человѣку всего дороже, презрѣти восхощешь? Кто сему повѣритъ? "--- 5)~Какъ паки человѣкъ человѣка любитъ, и благодарить ему, котораго ходатайствомъ честь и благородіе временное себѣ получитъ, всему міру тое извѣстно. Хрістосъ вѣчное, высокое, небесное и неизреченное благородіе, то"=есть \textit{сынами быти Божіими}, намъ исходатайствовалъ, и даетъ тое пріемлющимъ Его вѣрою: \textit{елицы бо пріяша Его, даде имъ область чадами Божіими быти, вѣрующимъ во имя Его}\footnote{Іоан.~1,~12.}. Но таяжде душа, чести и благородія любительница, не любитъ Хріста и не благодаритъ Ему за высочайшее тое благодѣяніе. Что честь и благородіе всякое міра сего противу оныя чести? Всякая и царская честь, которыя нѣтъ выше въ мірѣ семъ, какъ блато противу злата и тьма противу свѣта, или паче ничто. Признаешь сіе за истину, хрістіанине, когда разсудишь вѣрно славу Божіихъ чадъ. Однакожъ и за сіе малое благодаритъ ходатаю своему; но Хрісту, Который вѣчную славу крестомъ Своимъ отворилъ намъ, не благодаритъ. А тако и всего того добра, которое Онъ заслужилъ, лишается, яко неблагодарный.

\paragraph*{§\:439.} Отъ сихъ и подобныхъ симъ размышленій можетъ въ сердцѣ хрістіанина кающагося родитися \textit{печаль по Бозѣ}, то"=есть, что онъ Богу и Создателю своему и таковаго послушанія, почитанія, любви и благодарности не показывалъ, какое послушаніе и почитаніе показуетъ человѣку, подобному или паче равному себѣ: вси бо, начальники и подчиненніи, по естеству равны суть, яко вси человѣки суть единаго естества. Аще убо достойно есть почитать властелина "--- человѣка, кольми паче Бога, высочайшаго и вѣчнаго Владѣтеля, Который и надъ властями нашими господствуетъ и царствуетъ. И аще праведно есть благодарить благодѣтелю "--- человѣку, кольми паче Богу, Который благодѣтелей нашихъ и нашъ Благодѣтель есть высочайшій и первѣйшій. И аще должно жалѣть за оскорбленіе человѣка, который по естеству своему золъ: \textit{никтоже благъ, токмо единъ Богъ}\footnote{Матѳ.~19,~17.}, и который самъ согрѣшаетъ и оскорбляетъ: кольми паче сокрушаться и жалѣть должно, что согрѣшаемъ и оскорбляемъ Бога, Который есть \textit{единъ благъ} по естеству Своему, есть вѣчная любовь и благостыня, и никого не обидитъ и не оскорбляетъ, но паче всѣмъ отъ человѣколюбія Своего благотворитъ и благотворить не престаетъ. И аще вѣримъ человѣку, который есть лживъ: \textit{всякъ бо человѣкъ ложь}\footnote{Пс.~105,~2.}, кольми паче подобаетъ вѣрить Богу, Который есть истинный и есть самая истина. И аще любимъ и почитаемъ искренно и нелицемѣрно отца нашего по плоти, кольми паче Бога, Который насъ и отца нашего создалъ, питаетъ и прочіими благими снабдѣваетъ, любить и почитать должно намъ, что безъ послушанія быть не можетъ. И аще безчинствовать предъ царемъ и низшимъ властелиномъ или предъ честнымъ человѣкомъ или опасаемся, или стыдимся: кольми паче опасаться должно тое безчиніе показывать предъ величествомъ Божіимъ, Которое вездѣ есть и на всякія наши дѣла смотритъ. И аще защищаемъ честь монарха нашего, отца и господина нашего, кольми паче честь и славу Бога Вседержителя, Царя царей и Господа господей, защищать и хранить должно. Всякій бо человѣкъ, какъ ни великъ есть, почти ничто есть предъ Богомъ; и слава и честь его ничто есть предъ честію и славою Божіею. Словомъ, Бога, паче всего созданія и себе самого и всякаго имени именуемаго, любить, почитать, слушать и благодарить самая совѣсть научаетъ и убѣждаетъ насъ. Чего когда человѣкъ не отдавалъ Богу, и познаетъ сей предъ Нимъ свой грѣхъ и свое ничтожество, не можетъ не сокрушаться сердечно, болѣзновать и \textit{печалиться по Бозѣ}, которая печаль есть истинная и душеспасительная, каковыя печали Самъ Богъ отъ насъ въ покаяніи нашемъ требуетъ: \textit{расторгните сердца ваша, а не ризы ваша}\footnote{Іоил.~2,~13.}. Тако опечалился Петръ апостолъ, когда Хріста отреклся \textit{и изшедъ вонъ, плакася горько}\footnote{Матѳ.~26,~75.}. Тако оскорбился по Хрістѣ разбойникъ на крестѣ. \textit{Мы убо}, къ другому разбойнику хулящему отвѣщалъ, \textit{въ правду достойная по дѣломъ нашимъ воспріемлемъ}: вотъ его признаніе есть своего грѣха и винности! \textit{Сей же} (Хрістосъ) \textit{ни единаго зла сотвори}, но неповинно страждетъ, страждетъ же тое, что мы: вотъ сожалѣніе его есть Хрісту, неповинно страждущему! \textit{И глаголаше Іисусови: помяни мя Господи, егда пріидеши во царствіи Твоемъ}: здѣ исповѣданіе, и чрезъ исповѣданіе вѣра во Хріста изображается! Вѣрѣ его воспослѣдовало утѣшеніе отъ Хріста: \textit{днесь со Мною будеши въ раи}\footnote{Лук.~23,~41--43.}. Тако и намъ, хрістіанине, должно въ покаяніи нашемъ печалитися, что Богу и Создателю подобающія чести не отдавали, не слушали Его и не любили Его паче всего: каковой печали неотмѣнно послѣдуетъ утѣшеніе отъ Самаго Бога, Который \textit{исцѣляетъ сокрушенныя сердцемъ, и обязуетъ сокрушенія ихъ}\footnote{Пс.~146,~3.}.

\subsection[Глава 15-я. Истинному хрістіанину всѣ дѣла свои во славу Божію творити должно.]{глава пятаянадесять.\\\bfseries Истинному хрістіанину всѣ дѣла свои во славу Божію творити должно.}

\begin{quotation}\textit{Не намъ, Господи, не намъ, но имени Твоему даждь славу о милости Твоей и истинѣ Твоей}\footnote{113,~9.}.\end{quotation}

\paragraph*{§\:440.} Когда примѣчаетъ сатана, врагъ хрістіанскій, что человѣкъ тщится Богу угождать, и тако угодныя Ему добрыя дѣла творить: то всякимъ образомъ старается тыя его дѣла опорочить, дабы, что внѣ похвально дѣлается, внутрь въ сердцѣ его непохвально было. Не всякое бо дѣло доброе есть истинно доброе, но только тое, которое добрѣ и ради добраго конца дѣлается. Можетъ бо дѣло внѣ показываться добрымъ; но, когда на злый конецъ бываетъ, внутрь въ самомъ себѣ порочно есть. Тако милостыня, терпѣніе и прочее порочится, когда ради тщеславія бываетъ, и потому Богу неугодно таковое дѣло бываетъ. Како бо тое Богу угодно можетъ быть, что не ради Бога бываетъ? Сего ради діаволъ, примѣчая сіе, тщится доброе дѣло въ порокъ обратить, чтобы тое, что внѣ добрымъ показывается, внутрь и въ самомъ себѣ не было доброе, и тако злый и лукавый духъ и отъ добрыхъ нашихъ дѣлъ корысть и оброкъ себѣ тщится получить. Ктомужъ плоть наша славолюбивая все хощетъ во славу и похвалу свою обращать. Намъ, когда хощемъ Богу угождать, то не должно себѣ, то"=есть плоти, угождать добрыми дѣлами, и козни врага усматривать, который ищетъ дѣло наше опорочить, и тое, повидимому духовное и богоугодное, внутрь плотскимъ и небогоугоднымъ учинить.

\paragraph*{§\:441.} Истинные хрістіане, которые нелицемѣрно добрыя дѣла творятъ, въ святомъ Писаніи уподобляются \textit{древесамъ, добрые плоды приносящимъ. Всяко древо доброе плоды добры творитъ}\footnote{Мѳ.~7,~17.}. Древо доброе здѣ разумѣется вѣрное и богобоящееся сердце. Подобно и во Псалмѣ поется: \textit{и будетъ, яко древо насажденое при исходищихъ водъ, еже плодъ свой дастъ во время свое}, и проч.\footnote{Пс.~1,~3.}, "--- и на иномъ мѣстѣ: \textit{праведникъ, яко финиксъ процвѣтетъ, и, яко кедръ, иже въ Ливанѣ, умножится}\footnote{91,~13.}. Откуду добрыя ихъ дѣла называются \textit{плодами, плодами правды}\footnote{Филип.~1,~11; Іак.~3,~18.}, \textit{плодами духа}\footnote{Еф.~5,~9; Гал.~5,~22.}, \textit{плодами покаянія достойными}\footnote{Матѳ.~3,~3.}, и на прочіихъ мѣстахъ. Якоже убо на древѣ плоды извнутрь его, отъ сока въ немъ имѣющагося, раждаются, и доброе древо добрые плоды показываетъ: тако хрістіанскимъ добродѣтелямъ извнутрь отъ сердца должно происходить, и тако хрістіанское благочестивое показывать сердце. Вѣра, смиреніе, любовь, милость и прочее "--- лицемѣріе есть, когда на сердцѣ не имѣютъ мѣста и отъ сердца не происходятъ. И Богъ къ душѣ глаголетъ, а не къ тѣлу. Сего ради душа должна вѣровать, надѣяться, любить, милосердовать, смиряться и прочее благочестіе имѣть.

\paragraph*{§\:442.} Истинно добрыя дѣла отъ Бога происходятъ; или, какъ простѣе сказать, къ творенію добрыхъ дѣлъ возбуждаются, и силу и крѣпость отъ Бога получаютъ къ тому, и \textit{содѣйствующею Его благодатію} въ томъ дѣлѣ трудятся хрістіане. Тако бо свидѣтельствуетъ Божіе слово: \textit{Богъ есть дѣйствуяй въ насъ, и еже хотѣти, и еже дѣяти, о благоволеніи}\footnote{Фил.~2,~13.}. И \textit{безъ Него не можемъ творити ничесоже, и никакого плода сотворити, аще не будемъ насаждени на Лозѣ истинной} "--- Іисусѣ Хрістѣ\footnote{Іоан.~15,~5,~4 и 1.}. \textit{Того бо есмы твореніе, создани о Хрістѣ Іисусѣ на дѣла благая, яже прежде уготова Богъ, да въ нихъ ходимъ}\footnote{Еф.~2,~10.}. \textit{Иже} (Хрістосъ) \textit{далъ есть Себе за ны, да избавитъ ны отъ всякаго беззаконія, и очиститъ Себѣ люди избранны, ревнители добрымъ дѣломъ}\footnote{Тит.~2,~14.}.

\paragraph*{§\:443.} Сего ради къ тому Началу относить и восписывать добрыя дѣла хрістіанамъ должно, отъ Котораго они происходятъ. А понеже отъ Бога происходятъ, какъ сказано, то какъ въ славу и похвалу Божію творить ихъ, такъ Его благодати и приписывать должно. Тогда же сіе будетъ, когда будемъ ихъ творить не ради нашей славы и похвалы, да видимы будемъ отъ человѣкъ, и почитатися за святыхъ, и тако отъ нихъ славитися, но о семъ будемъ тщатися, да славится и святится имя небеснаго Отца, якоже учитъ Хрістосъ: \textit{тако да просвѣтится свѣтъ вашъ предъ человѣки, яко да видятъ ваша добрая дѣла, и прославятъ Отца вашего, Иже есть на небесѣхъ}\footnote{Матѳ.~5,~16.}, "--- и сотворенными не будемъ хвалитися и возноситися; но, помышляя о немощи нашей, признавать будемъ, что отъ себя и помыслить и хотѣть, не токмо творить, ничего богоугоднаго не можемъ. Надобно Самому Богу и помыслъ, и хотѣніе, и начинаніе доброе въ насъ дѣйствовать, какъ выше сказано. О семъ Ему буди слава, аминь!

\subsection[Заключеніе статьи сея. О побужденіи грѣшника къ покаянію.]{Заключеніе статьи сея.\\\bfseries О побужденіи грѣшника къ покаянію.}

\begin{quotation}\textit{Или о богатствѣ благости Его и кротости и долготерпѣніи нерадиши, не вѣдый, яко благость Божія на покаяніе тя ведетъ}\footnote{Римл.~2,~4.}?\end{quotation}

Сіе увѣщательное апостола Хрістова слово всякаго неисправнаго хрістіанина касается: ударяетъ въ совѣсть блудника, хищника, татя, клеветника, злорѣчиваго, злобнаго, пьяницу и всякаго, нерадящаго о благости Божіей и своемъ спасеніи; и представляетъ оно всякому грѣшнику тое, что онъ того ради на свѣтѣ живетъ, и еще по дѣламъ своимъ не пріемлетъ, яко Богъ ожидаетъ его обращенія: \textit{яко благость Божія на покаяніе его ведетъ}. Слава богатству благости Твоея и кротости и долготерпѣнію, человѣколюбивый Боже нашъ, яко такъ милосердо поступаеши съ нами грѣшными! Намъ же, о грѣшники, не подобаетъ о богатствѣ благости Его и кротости и долготерпѣніи нерадѣти, да не впадемъ въ судъ правды Его. Ибо кто нерадитъ о благости Божіей, тотъ правду Божію на себѣ узнаетъ. Боимся суда человѣческаго, который только тѣла нашего касается: почто не боимся суда Божія, который и тѣло и душу погубляетъ? Боимся смерти временной: почто не боимся смерти вѣчной? Убѣгаемъ временной бѣды, безчестія, укоризны, безславія, нищеты, плѣненія, страданія, печали, болѣзни и прочей противности: ради чего паче не тщимся убѣгать вѣчнаго бѣдствія, въ которомъ болѣзнь, печаль, плѣненіе, безчестіе, укоризны, нищета и всякое мученіе и страданіе заключается? Омываемъ тѣло смертное водою: почто покаяніемъ не омываемъ души нашея безсмертныя? Жалѣемъ о потеряномъ богатствѣ, чести, славѣ, которое все съ животомъ симъ оставить принуждены будемъ: почто не жалѣемъ о потеряніи богатства, чести и славы, которое сокровище никогда не отлучится отъ избранныхъ Божіихъ? Временнаго блаженства ищемъ и бѣдствія уклоняемся, которое все, какъ сновидѣніе, якоже скоро приходитъ, тако скоро и отходитъ отъ насъ; но вѣчнаго блаженства, которое непоколебимо есть, почто не ищемъ, и бѣдствія онаго почто не тщимся уклоняться? О прелести, о суеты, въ которой запутался ослѣпленный грѣшникъ! \textit{О человѣче! или о богатствѣ благости Его и кротости и долготерпѣніи нерадиши, невѣдый, яко благость Божія на покаяніе тя ведетъ?} Послушаемъ убо далѣе, хрістіанине, апостола святаго, что онъ нерадивому грѣшнику возвѣщаетъ. \textit{По жестокости твоей}, рече, \textit{и непокаянному сердцу собираеши себѣ гнѣвъ въ день гнѣва и откровенія праведнаго суда Божія, Иже воздастъ комуждо по дѣломъ его}\footnote{Римл.~2,~5 и 6.}. Когда въ небреженіи и нераскаяніи грѣшникъ пребудетъ и о благости и долготерпѣніи Божіи вознерадитъ, и грѣхи ко грѣхамъ будетъ прилагать, не примѣчая, что благость Божія долготерпѣніемъ на покаяніе его ведетъ: тогда его \textit{судъ Божій} постигаетъ; и чимъ болѣе грѣшитъ и дознаетъ на себѣ благости и долготерпѣнія Божія, тѣмъ болѣе дознаетъ гнѣва Божія. Сколько терпѣлъ Богъ Израильтянамъ, \textit{ихже всыновленіе и слава, и завѣти и законоположеніе, и служеніе и обѣтованія, ихже отцы, и отъ нихже Хрістосъ по плоти}\footnote{9,~4 и 5.}; сколько согрѣшали, отвращалися отъ Него, преогорчевали, прогнѣвляли и раздражали Его, и различно наказуеми и увѣщаваеми чрезъ пророковъ и обращаеми и исправляеми были, "--- Ветхаго Завѣта Писаніе свидѣтельствуетъ. Но когда все тое долготерпѣніе Божіе презрѣли и въ такое дерзновеніе пришли, что \textit{Господа славы}\footnote{1~Кор.~2,~8.}, Который ихъ пришелъ спасти, распявше убили, "--- конечному Божію гнѣву и мщенію подпали, такъ что какъ мѣсто ихъ и храмъ \textit{разоренъ до основанія}\footnote{Матѳ.~24,~2.}, такъ и сами во всѣ языки разсѣяны и отъ Бога отвержены, и \textit{отнялося отъ нихъ царствіе Божіе, и далося языку творящему плоды его}\footnote{21,~43.}. Подобно мстительную руку Божію дознали на себѣ Содомляне, которые дознавали благость Божію и долготерпѣніе, и не хотѣли обратитися и покаятися отъ беззаконій своихъ; исполнили бо мѣру беззаконій, паче же и прешли; сего ради огнемъ съ небесе пожжены, и сошли во адъ пити вѣчнаго гнѣва Божія чашу\footnote{Быт.~19,~24 и 25.}. Такойжде судъ постигнулъ Египтянъ, которые людей Божіихъ озлобляли, и по повелѣнію Божію не хотѣли ихъ отпустить, и хотя наказуеми были за то отъ Бога, однакожъ наказаніями тѣми болѣе ожесточилися; чего ради, когда за изшедшими Израильтянами погналися, и хотѣли ихъ паки себѣ поработить, которыхъ видѣли чудесною Божіею рукою покрываемыхъ, защищаемыхъ и избавляемыхъ, вси въ морѣ Чермномъ погрязнули, и водою, какъ землею во гробѣ, покрылися и тако погибли\footnote{Исх.~14,~27 и 28.}. Такожде читаемъ, что всѣхъ жителей всея земли, которые были при Ноѣ, праведный Божій гнѣвъ погубилъ: яко не хотѣли отъ нечестій своихъ обратитися и покаятися, чего \textit{ожидаше Божіе долготерпѣніе во дни Ноевы, дѣлаему ковчегу, въ немже мало, сирѣчь, осьмь душъ спасошася отъ воды}\footnote{1~Петр.~3,~20.}, которое долготерпѣніе когда презрѣли, то уже дознали на себѣ ужасный Божій гнѣвъ, и всемірнымъ потопомъ со скотами и звѣрьми покрылися и погибли нечестивіи\footnote{Быт.~7,~17--24.}. Сіи примѣры, изъ Писанія святаго приведенные, показуютъ намъ, что благость Божія и долготерпѣніе Его, отъ грѣшниковъ презрѣнное, обращается имъ въ большую погибель. Тыяжде судьбы Божія и нынѣ видимъ. Тойжде Богъ и нынѣ наказуетъ нечестіе, котораго ненавидитъ, и казнитъ нечестивыхъ, которые \textit{о богатствѣ благости Его и кротости и долготерпѣніи нерадятъ}. Видимъ, сколько тысящей поядаетъ моровая язва, поражаетъ мечь воинскій; сколько беззаконниковъ въ самомъ беззаконномъ дѣлѣ поражается и нисходитъ во адъ пріяти по дѣломъ своимъ! Слово Божіе проповѣдуется, въ которомъ слышитъ всякъ, что Богъ его зоветъ, отзываетъ отъ грѣха и прелести міра, и обращаетъ и призываетъ къ Себѣ, и ожидаетъ его обращенія и покаянія, и тако \textit{благость Божія на покаяніе всякаго ведетъ}; но когда беззаконники долготерпѣніе Божіе, въ нераскаяніи и прежнемъ нечестіи живучи, презираютъ, тогда Божія правда вступаетъ въ дѣло Свое, и мечемъ правосудія \textit{посѣкаетъ ихъ, яко древеса, не творящая плода добра, и во огнь вметаетъ}\footnote{Матѳ.~3,~10.}. Убоимся убо, грѣшники, суда Божія и отвратимся отъ беззаконій нашихъ, да не и насъ постигнетъ праведный Его гнѣвъ. Опасно, весьма опасно о благости и кротости и долготерпѣніи Божіи нерадѣти, какъ выше сказано. Вмѣсто благости и кротости Его, гнѣвъ Его возгорится на нерадивыхъ. \textit{Вѣмы бо рекшаго: Мнѣ отмщеніе, Азъ воздамъ, глаголетъ Господь}. И паки: \textit{яко судитъ Господь людемъ Своимъ. Страшно есть еже впасти въ руцѣ Бога живаго}\footnote{Евр.~10,~30 и 31.}! Ибо \textit{Богъ нашъ огнь поядаяй есть}\footnote{12,~29.}. Обратимся убо къ Нему, пока долготерпитъ и ожидаетъ насъ, да не впадемъ въ руцѣ Его. Нынѣ Онъ ожидаетъ насъ и хощетъ приняти, но утрешняго дня не обѣщаетъ. Нынѣ убо обратимся къ Нему, и помилуетъ насъ. Благъ бо есть и милостивъ обратившимся: не поминаетъ имъ прешедшихъ грѣховъ; не дѣлаетъ имъ выговора, что оставили Его и врагу Его работали, но только всякаго приходящаго пріемлетъ, и радуется о немъ, и отпущаетъ ему вся согрѣшенія его. Таковаго убо, такъ незлобиваго, милостиваго, щедраго, кроткаго, благоутробнаго имѣя Господа, о грѣшники, не усумнимся пріити къ Нему съ покаяніемъ и слезами, и спасемся отъ гнѣва Его, который падетъ на нераскаянныхъ, и поястъ ихъ яко огнь сѣно. О хрістіане, воспомянемъ имя, званіе и обѣщаніе наше. Хрістосъ Царь нашъ есть, и мы подданніи раби Его есмы; ибо отъ Него и имя наше сіе, \textit{хрістіанинъ}, имѣемъ. Онъ насъ создалъ, Онъ насъ кровію Своею искупилъ, Онъ о насъ промышляетъ. Ему мы при крещеніи обѣщалися вѣрою и правдою служить; Ему и имена свои записали, и называемся хрістіане. Гдѣ убо обѣщаніе наше? гдѣ вѣра? гдѣ правда? гдѣ вѣрная Ему работа? Почто имени не согласуетъ дѣло? Почто называемся \textit{хрістіане} и не работаемъ Хрісту, но міру и грѣху? Какая намъ польза отъ имени безъ вещи, которую означаетъ имя? Почто рабъ называется рабомъ господина, которому не повинуется и не работаетъ? Почто хрістіанинъ называется хрістіаниномъ, когда не работаетъ Хрісту, Царю своему и Господу? \textit{Аще кто Духа Хрістова не имать, сей нѣсть Его}\footnote{Римл.~8,~9.}. \textit{Иже бо Хрістовы суть, плоть распяша со страстьми и похотьми}\footnote{Гал.~5,~24.}. Великолѣпнѣйшее есть имя "--- \textit{хрістіанинъ}! Хрістіане суть \textit{сынове Бога Вышняго}\footnote{3,~26; 4,~7.}, и \textit{общеніе ихъ со Отцемъ и съ Сыномъ Его Іисусомъ Хрістомъ}\footnote{1~Іоан.~1,~3.}. Сей имъ титулъ приписуетъ Божіе слово. \textit{Вѣрно слово и всякаго пріятія достойно!} Какъ убо возможно оскверняться грѣхами и со святѣйшимъ Богомъ общеніе имѣти? \textit{Богъ свѣтъ есть, и тмы въ немъ нѣсть ни единыя. Аще речемъ, яко общеніе имамы съ Нимъ, и во тмѣ ходимъ, лжемъ, и не творимъ истины}\footnote{1,~5 и 6.}. Како возможно и чадомъ міра сего и чадомъ Божіимъ быть? Никакъ сіе быть не можетъ. Надобно неотмѣнно или тѣмъ или другимъ быть: надобно или чадомъ Божіимъ быть, и во свѣтѣ ходить, якоже \textit{Самъ Той есть во свѣтѣ}\footnote{1~Іоан.~1,~7.}, и свойство Отца своего нравами изображать, и Ему подражать; или чадомъ міра быть, и, во тьмѣ ходя, дѣла темная творить. \textit{Очистимъ} убо \textit{себе}, о хрістіане, \textit{отъ всякія скверны плоти и духа, творяще святыню въ страсѣ Божіи}\footnote{2~Кор.~7,~1.}: да будемъ Хрістовыми и сынами Вышняго, и возъимѣемъ общеніе со Отцемъ и съ Сыномъ Его Іисусомъ Хрістомъ. Какая намъ польза отъ мірской сласти, которая дотолѣ чувствуется, доколѣ совершается, и совершившися, вмѣсто утѣшенія ложнаго, истинную вноситъ въ душу горесть? Что намъ пользуетъ міра сего богатство, честь, слава и похвала? Честь и богатство не иное что приноситъ намъ, какъ большее безпокойство. Кто бо болѣе печется, какъ богачь и въ чести находящійся? Слава и похвала что есть, какъ только гласъ людей, отъ единаго къ другому приходящій и въ уши ударяющій, которые часто хвалятъ и славятъ тое, что въ себѣ порочно, и часто кого хвалятъ, того потомъ проклинаютъ? Почтожъ намъ, намъ "--- хрістіанамъ, того искать, что намъ запрещено и отнимаетъ у насъ покой нашъ; и тѣмъ утѣшаться, что въ себѣ никакого блаженства не заключаетъ, но только видится быть нѣчто, яко тѣнь? Все міра сего блаженство есть, какъ мѣхъ, надутый воздухомъ, который извнѣ видится нѣчто, но въ себѣ ничего не имѣетъ, и такъ только прельщаетъ насъ, и есть какъ пузырь на водѣ, который нѣчто показавшись исчезаетъ. Тако все міра сего сокровище только показуетъ нѣчто, и тако прельщаетъ человѣковъ, какъ несмысленныхъ дѣтей угліе огнемъ раскаленное, за которое руками хватившеся обжигаются и болѣзнуютъ и плачутъ. Тоежде и міръ любителямъ своимъ дѣлаетъ. Красное нѣчто имъ себе показуетъ; но сія красота послѣ обращается имъ въ болѣзнь и плачь неутѣшный, когда при кончинѣ живота своего ничего болѣе, какъ только уязвленную совѣсть и пагубу души, отсюду относятъ. Примѣчайте сія и разумѣйте, любители и чада міра! Ничего вы отъ любви міра не получите, кромѣ пагубы. \textit{Аще бо кто любитъ міръ, нѣсть любве Отчи въ немъ}\footnote{1~Іоан.~2,~15.}. Ибо \textit{никто не можетъ двѣма господинома работати: любо единаго возлюбитъ, а другаго возненавидитъ; или единаго держится, о друзѣмъ же нерадити начнетъ}\footnote{Матѳ.~6,~24.}. Не можетъ и хрістіанинъ Хрісту работати и міру; надобно единаго держаться, о другомъ же нерадѣти. Вознерадимъ убо, о хрістіане, о мірѣ, и поработаемъ Хрісту, Который насъ искупилъ Себѣ, и призвалъ къ Себѣ, да будемъ истинно, а не лицемѣрно Хрістовыми. \textit{Хрістосъ за всѣхъ умре, да живущіи не ктому себѣ живутъ, но умершему за нихъ и воскресшему}\footnote{2~Кор.~5,~15.}. Оставимъ міръ міру, не знающему Бога истиннаго и Сына Его Іисуса Хріста. Пусть язычники и идолопоклонники, не имѣющіи упованія, притворною его красотою и сладостію утѣшаются: мы къ великимъ и неисповѣдимымъ благимъ, къ вѣчному животу и небесному царствію позваны и Духомъ отрождены. Пусть они въ блатѣ и тинѣ его валяются, яко не омовенніи: мы банею крещенія омыты, и чистою правды Хрістовой одеждою одѣты, и \textit{приступили}, по слову апостольскому, \textit{къ Сіонстѣй горѣ, и ко граду Бога живаго, Іерусалиму небесному, и тмамъ ангеловъ, торжеству, и Церкви первородныхъ на небесѣхъ написанныхъ, и Судіи всѣхъ Богу, и духомъ праведникъ совершенныхъ, и къ Ходатаю завѣта новаго Іисусу, и крови кропленія, лучше глаголющей, нежели Авелева}\footnote{Евр.~12,~22--24.}. О коль дорогое и красное хрістіанъ одѣяніе "--- одѣяніе, кровію Хрістовою купленное и преиспещренное, несравненно дражайшее порфиры царскія! Коль преславное гражданство, которое имѣютъ на небеси! \textit{Наше бо житіе на небесѣхъ есть, отонудуже и Спасителя ждемъ, Господа нашего Іисуса Хріста}\footnote{Фил.~3,~20.}. Потщимся убо, о хрістіане, сіе святое одѣяніе хранить, удаляяся отъ мірскихъ нечистотъ, и, какъ учитъ апостолъ, \textit{огребаяся отъ плотскихъ похотей, яже воюютъ на душу}\footnote{1~Петр.~1,~11.}, а потерявшіи тое покаяніемъ и слезами сыскать, да и жители преславнаго онаго града будемъ, и получимъ оная \textit{благая, яже уготова Богъ любящимъ Его}\footnote{1~Кор.~2,~9.}. Аминь.


\section[Статья 6-я. О должности хрістіанской къ ближнему.]{статья шестая.\\\bfseries О должности хрістіанской къ ближнему.}

О сей должности сказано въ первой книгѣ въ главѣ о любви, милости и проч. Здѣ таяжде должность изъясняется отъ подобія, взятаго отъ членовъ тѣла человѣческаго: понеже хрістіанство есть духовное тѣло, по ученію апостола Павла.

\begin{quotation}\textit{Якоже во единомъ тѣлеси многи уды имамы, уды же вси не тожде имутъ дѣланіе: такожде мнози едино тѣло есмы о Хрістѣ, а по единому другъ другу уди}\footnote{Римл.~12,~4 и 5.}.\end{quotation}

\paragraph*{§\:444.} Истинніи хрістіане со Хрістомъ, яко тѣло съ главою, духовно соединены суть, о которомъ соединеніи прекрасно и утѣшительно святый Златоустъ бесѣдуетъ тако: «Смотри, Хрістосъ есть Глава; мы же тѣло. Можетъ ли убо быть какое между тѣломъ и главою разстояніе? Той Основаніе, мы же зданіе. Той Лоза, мы же рожденіе. Той Женихъ, мы же невѣста. Той Пастырь, мы же овцы. Той Путь, мы же идущіи. Мы церковь, Той Всельникъ. Той Первенецъ, мы же братія. Той Наслѣдникъ, мы же снаслѣдницы. Той Животъ, мы же живущіи. Той Воскресеніе, мы востающіи. Той Свѣтъ, мы просвѣщаеміи. Сія вся соединеніе являютъ»\footnote{Бес.~8"~я на 1"~е посл. къ Кор.}. Но какъ со Хрістомъ, такъ и другъ съ другомъ, яко единаго тѣла уды и едину Главу "--- Хріста имущіи, соединены суть хрістіане. Отсюду послѣдственно такая должность требуется отъ хрістіанъ другъ къ другу.

\paragraph*{§\:445.} Хрістіане должны между собою миръ и согласіе о Хрістѣ имѣть. Въ вещественнѣмъ тѣлѣ всѣ уды согласны и мирны пребываютъ: рука на руку, нога на ногу, око на око и на прочіи уды не враждуетъ; но что единъ удъ дѣлаетъ, тое и другіе; куды одно око смотритъ, туды и другое, куды едина нога ступаетъ, туды за нею и другая ступаетъ. Тако и въ прочіихъ тѣлесныхъ удахъ великое согласіе между собою и во всемъ составѣ имѣется. Тако бо Творецъ устроилъ: иначе бы не моглъ цѣлъ быть составъ. Такое согласіе должно быть и въ духовномъ тѣлѣ, то есть въ хрістіанствѣ. Союза и мира сего причину полагаетъ апостолъ: \textit{едино тѣло единъ Духъ, якоже и звани бысте во единомъ упованіи званія вашего; единъ Господь, едина вѣра, едино крещеніе: единъ Богъ и Отецъ всѣхъ, Иже надъ всѣми и чрезъ всѣхъ и во всѣхъ насъ}\footnote{Еф.~4,~4--6.}. И на другомъ мѣстѣ: \textit{единѣмъ Духомъ мы вси во едино тѣло крестихомся, аще Іудеи, аще Еллини или раби, или свободни; и вси единѣмъ Духомъ напоихомся}\footnote{1~Кор.~12,~13.}. Къ сему миру и увѣщаваетъ хрістіанъ: \textit{миръ имѣйте и святыню со всѣми, ихже кромѣ никтоже узритъ Господа}\footnote{Евр.~12,~14.}. Такій мира союзъ былъ у хрістіанъ первенствующія церкви, якоже Лука святый о семъ свидѣтельствуетъ: \textit{народу вѣровавшему бѣ сердце и душа едина}\footnote{Дѣян.~4,~32.}. Убо погрѣшаютъ хрістіане и отъ должности своей заблуждаютъ, которые другъ съ другомъ ссорятся и другъ на друга враждуютъ; а тако и съ Богомъ имѣти мира не могутъ, пока во враждѣ пребываютъ. Хотящему бо миръ имѣти съ Богомъ должно съ братомъ прежде примиритися, по словеси Господню: \textit{аще принесеши даръ твой ко олтарю, и ту помянеши, яко братъ твой имать нѣчто на тя: остави ту даръ твой предъ олтаремъ, и шедъ прежде смирися съ братомъ твоимъ, и тогда пришедъ принеси даръ твой}, и проч.\footnote{Матѳ.~5,~22,~23,~24 и слѣд.} И \textit{аще не отпущаете человѣкомъ согрѣшенія ихъ, ни Отецъ вашъ отпуститъ вамъ согрѣшеній вашихъ}, глаголетъ Господь\footnote{6,~15.}. Тожде показуется и притчею о царѣ и должникѣ его и клевретѣ того должника\footnote{18,~23--35.}. А кому отпущенія грѣховъ нѣтъ, тотъ и мира съ Богомъ имѣть не можетъ: слѣдовательно и гнѣву Его подлежитъ, и всему временному и вѣчному бѣдствію, которое Божію гнѣву послѣдуетъ.

\paragraph*{§\:446.} Между хрістіанами должна быть любовь нелицемѣрная. Въ вещественномъ тѣлѣ уды вси связаны жилами: тако въ духовномъ хрістіанства тѣлѣ хрістіане, яко другъ другу уды, должны быть связаны союзомъ любве. Сіе заповѣдуетъ хрістіанамъ Своимъ Хрістосъ: \textit{сіе заповѣдую вамъ, да любите другъ друга}\footnote{Іоан.~15,~17.}. Къ сему увѣщаваютъ апостоли: \textit{отъ чиста сердца другъ друга любите прилѣжно}\footnote{1~Петр.~1,~22.}. \textit{Братолюбіе да пребываетъ}\footnote{Евр.~13,~1.}. \textit{Возлюбленніи! аще сице возлюбилъ есть насъ Богъ, и мы должни есмы другъ друга любити}\footnote{1~Іоан.~4,~11.}. Плоды любве суть: \textit{любы долготерпитъ, милосердствуетъ, любы не завидитъ; любы не превозносится, не гордится, не безчинствуетъ, не ищетъ своихъ си, не раздражается, не мыслитъ зла, не радуется о неправдѣ, радуется же о истинѣ, вся любитъ, всему вѣру емлетъ, вся уповаетъ, вся терпитъ. Любы николиже отпадаетъ}\footnote{1~Кор.~13,~4--8.}. Аще убо хотятъ хрістіане истинными хрістіанами быть, а не лицемѣрами, должны показать хрістіанство свое отъ любви, по ученію Господню: \textit{о семъ разумѣютъ вси, яко Мои ученицы есте, аще любовь имате между собою}\footnote{Іоан.~13,~35.}. Ибо \textit{не любяй брата пребываетъ въ смерти. Всякъ ненавидяй брата своего, человѣкоубійца есть: и вѣсте, яко всякъ человѣкоубійца не имать живота вѣчнаго въ себѣ пребывающа}\footnote{1~Іоан.~3,~14 и 15.}.

\paragraph*{§\:447.} Хрістіане должни быть другъ другу милосердыми. Въ вещественномъ тѣлѣ, \textit{аще страждетъ единъ удъ, съ нимъ страждутъ вси уди}\footnote{1~Кор.~12,~26.}. Когда болитъ рука, или нога, или голова, или глазъ, или какій другій удъ, соболѣзнуютъ ему и прочіи вси уды: тако и въ тѣлѣ духовномъ хрістіане должны другъ другу соболѣзновать и сострадать. Когда единъ хрістіанинъ бѣдствуетъ и страждетъ, его бѣдствіемъ и страданіемъ подвигнуться должни и прочіи хрістіане и ему сострадать. Къ сему увѣщаваетъ хрістіанъ апостолъ: \textit{поминайте юзники, аки съ ними связани; озлобляемыя, аки и сами суще въ тѣлѣ}\footnote{Евр.~13,~3.}. И паки: \textit{бывайте другъ другу блази, милосерди, прощающе другъ другу, якоже и Богъ во Хрістѣ простилъ есть вамъ}\footnote{Еф.~4,~32.}. И Хрістосъ глаголетъ: \textit{будите милосерди, якоже и Отецъ вашъ милосердъ есть}\footnote{Лук.~6,~36.}. Милосердымъ же быть не ино что есть, какъ сердцемъ соболѣзновать болѣзнующему, сострадать бѣдствующему и страждущему, и \textit{плакать съ плачущимъ}\footnote{Римл.~12,~15.}. Аще убо какій хрістіанинъ надъ бѣдствіемъ брата своего милосердіемъ не подвигается: свидѣтельствуетъ о себѣ, что хрістіанскаго духа не имѣетъ. Таковіи суть, которые за обиды на другихъ гнѣваются и отмщеваютъ имъ; которые братіи своей въ нуждахъ не помогаютъ, а могутъ помощи; которые нищихъ и бѣдныхъ, могучи снабдить, отсылаютъ отъ себя безъ удовольствія ихъ, и прочіихъ милосердія дѣлъ, о которыхъ поминается Матѳея главы XXV въ 35 и 36, не дѣлаютъ имъ. Хрістіанинъ бо не можетъ быть безъ вѣры, вѣра безъ любви, любовь же безъ милосердія. Ибо любовь не токмо надъ братомъ своимъ единовѣрнымъ, но и надъ невѣрнымъ и врагомъ своимъ милосердствуетъ, и не можетъ не благотворити всякому, кого ни видитъ требующаго благотворенія.

\paragraph*{§\:448.} Въ вещественномъ тѣлѣ единъ удъ другому помогаетъ. Рука рукѣ, нога ногѣ, глазъ глазу, ухо уху, и прочіи уди другъ другу помогаютъ. Нога потыкающаяся отъ другой ноги подкрѣпляется; рука другую руку и прочіи уды моетъ, утираетъ, исправляетъ и въ прочемъ служитъ; око видѣніемъ, ухо слышаніемъ, ноздри обоняніемъ, ноги движеніемъ и бѣгствомъ прочіимъ удамъ и всему тѣлу служатъ; желудокъ и чрѣво вареніемъ пищи и питія работаетъ всему тѣлу. Тако и въ духовномъ тѣлѣ хрістіане другъ другу должни помогать и работать, \textit{любовію работать другъ другу}\footnote{Гал.~5,~13.}. Когда единъ немоществуетъ, другій ему послужить долженъ. Когда единъ печалится, другій "--- утѣшати его; когда единъ изнемогаетъ въ вѣрѣ, другій "--- его подкрѣплять; когда единъ совращается съ пути истиннаго, другій "--- направляетъ его; и всякъ, богатый убогому, разумный скудоумному, здоровый больному, свободный заключенному, молодый старому, сильный немощному, и прочій чего себѣ хощетъ, тое и другому дѣлать, и чего не хощетъ себѣ, того и другому не дѣлать долженъ. Къ сему на премногихъ мѣстахъ слово Божіе увѣщаваетъ насъ. А Хрістосъ Господь нашъ все, что ни сотворится хрістіанину доброе, яко хрістіанину, Себѣ сотворенное вмѣняетъ, якоже глаголетъ: \textit{взалкахся, и дасте Ми ясти, возжадахся, и напоисте Мя; страненъ бѣхъ, и введосте Мене; нагъ, и одѣясте Мя; боленъ, и посѣтисте Мене; въ темницѣ бѣхъ, и пріидосте ко Мнѣ. И понеже сотвористе единому сихъ братій Моихъ меншихъ, Мнѣ сотвористе}\footnote{Матѳ.~25,~35,~36 и 40.}. А которые братіи своей "--- хрістіанамъ не помогаютъ, тіи да внимаютъ, что далѣе тойжде праведный Судія имъ возглаголетъ: \textit{идите отъ Мене проклятіи во огнь вѣчный, уготованный діаволу и аггеломъ его. Взалкахся бо, и не дасте Ми ясти; возжадахся, и не напоисте Мене; страненъ бѣхъ, и не введосте Мене; нагъ, и не одѣясте Мене; боленъ, и въ темницѣ, и не посѣтисте Мене. Понеже"=де не сотвористе единому сихъ меншихъ, ни Мнѣ сотвористе}\footnote{ст.~41--45.}. Аще убо не благотворящимъ такъ страшный отвѣтъ будетъ: какій уже отвѣтъ будетъ тѣмъ, которыи у хрістіанъ и тое, что имѣютъ, отнимаютъ, и зло творятъ имъ? Разсуждай сіе, хрістіанине, который не только не творишь добра, но и зло творишь братіи своей. Отъ сего видишь, хрістіанине, что какъ отъ благотворенія познавается хрістіанинъ истинный и живый удъ духовнаго тѣла Хрістова, такъ отъ неблаготворенія, много паче отъ злотворенія, примѣчается лицемѣрный, гнилый и мертвый, или паче отсѣченный удъ. Единаго бо тѣла уды другъ друга чувствуютъ болѣзнь и нужду, и потому другъ другу помогаютъ. А когда хрістіанинъ другому не помогаетъ, то знаменіе есть, что онъ не чувствуетъ его болѣзни, яко несоединеннаго себѣ уда; и потому не чувствуя не соболѣзнуетъ и не помогаетъ ему. Иначе бы не оставилъ его безъ помощи въ нуждѣ его.

\paragraph*{§\:449.} Въ вещественномъ тѣлѣ единъ удъ другаго предостерегаетъ и защищаетъ отъ находящихъ бѣдствій. Рука предохраняетъ главу, очи и прочіе уды, когда приходитъ бѣдствіе отъ ударенія жезла, меча и прочаго; очи предостерегаютъ видѣніемъ другіе уды, уши слухомъ и внятнымъ слышаніемъ при находящей напасти отъ слуха зла; языкъ вкушеніемъ и испытаніемъ вреднаго желудку и всему тѣлу; обоняніе чувствованіемъ пагубнаго всему составу смрада; ноги бѣгомъ спасаютъ отъ приближающагося зла: и тако вси уды другъ о другѣ и о цѣлости всего состава тѣлеснаго пекутся. Тако должны и хрістіане другъ о другѣ, яко другъ друга уды, пещися и промышлять, когда видятъ другъ другу наступающую бѣду, "--- иной словомъ и совѣтомъ, иной молитвою, иной самымъ дѣломъ отъ находящаго бѣдствія предостерегать и охранять. А которые того не дѣлаютъ, но паче путь къ бѣдствію коварно братіи своей устрояютъ, тіи до сего, святаго и союзомъ любве связаннаго, общества не надлежатъ; но суть какъ \textit{древеса, не творящая плода добра, которыя посѣкаются и во огнь вметаются}\footnote{Матѳ.~3,~10.}; или паче, какъ \textit{розги изсохшія и изверженныя вонъ, которыя собираютъ и во огнь влагаютъ, и сгораютъ}\footnote{Іоан.~15,~6.}.

\paragraph*{§\:450.} Въ тѣлѣ вещественномъ чрево ради всего тѣла работаетъ, варитъ пищу, сочиняетъ соки и обращаетъ въ кровь, и тако кровь по всѣмъ удамъ раздѣляетъ, и симъ образомъ всѣ уды и все тѣло подкрѣпляетъ, умножаетъ и въ возрастъ приводитъ. Тако въ духовномъ тѣлѣ пастыри должны быть, какъ чрево, въ дѣлѣ своемъ: пищу слова Божія первѣе внутрь себя чтеніемъ, поученіемъ, размышленіемъ и вниманіемъ, единаго мѣста съ другимъ сношеніемъ, какъ бы разжевать и сварить, и въ сокъ и кровь духовную обратить, и тако братіи своей, духовнымъ удамъ, раздѣлять, и тѣмъ разслабленныхъ подкрѣплять, младенствующихъ въ возрастъ приводить, \textit{дондеже достигнемъ вси въ соединеніе вѣры и познанія Сына Божія, въ мужа совершенна, въ мѣру возраста исполненія Хрістова}\footnote{Еф.~4,~13.}. А которые пастыри сего не дѣлаютъ, погрѣшаютъ противу должности своей, и напрасно имя пастыря на себѣ носятъ. Пастырь бо не себѣ пастырь есть, но овцамъ пастырь. Почему и воздадятъ отвѣтъ Начальнику пастырей, Іисусу Хрісту въ день суда, Который тогда отъ руки ихъ взыщетъ Своихъ овецъ.

\paragraph*{§\:451.} Видимъ, что въ тѣлѣ вещественномъ вси уды ради чрева работаютъ, вси ему служатъ, пищу и питіе промышляютъ ему, чтобы укрѣплялося: понеже оно ради общей пользы удовъ работаетъ и варитъ пищу. Ибо ежели чрево, не пріемля пищи, ослабѣваетъ, то и вси уды ослабѣютъ, и такъ весь человѣкъ слабымъ сдѣлается, что примѣчаемъ на немощныхъ, которые пищи не пріемлютъ, и отъ глада истаевающихъ, и всякъ на себѣ во время глада и жажды узнаетъ. Тако и въ духовномъ тѣлѣ, то"=есть хрістіанствѣ, должно быть. Понеже пастыри и учители ради общей пользы трудятся, то всѣмъ хрістіанамъ обще питать ихъ должно. Должно, глаголю, питать ихъ, дабы имѣли время ради общей хрістіанской пользы трудиться, въ чтеніи и размышленіи Божія слова упражняться, и тако какъ бы сваренную пищу слова Божія на пользу слышателей своихъ раздѣлять. \textit{Достоинъ бо дѣлатель мзды своея}\footnote{1~Тим.~5,~18.}. Аще бо пастыри \textit{духовная сѣютъ: велико ли аще} слышателей \textit{тѣлесная пожинаютъ}\footnote{1~Кор.~9,~11.}? Аще они души слышателей питаютъ словомъ Божіимъ, и тако о вѣчномъ ихъ спасеніи промышляютъ: велико ли есть, когда отъ нихъ тѣлесную пищу къ содержанію временнаго живота получаютъ? Сколько душа отъ тѣла, и вѣчный животъ отъ временнаго разнствуетъ, столько пища Божія слова, которою питаютъ души слышателей пастыри, дражайша есть отъ пищи тѣлесной, которую отъ слышателей своихъ пріемлютъ. \textit{Да общается} убо \textit{учайся словеси учащему во всѣхъ благихъ}\footnote{Гал.~6,~6.}. \textit{Кто бо воинствуетъ своими оброки когда? или кто насаждаетъ виноградъ, и отъ плода его не ястъ? Или кто пасетъ стада, и отъ млека стада не ястъ}\footnote{1~Кор.~9,~7.}?

\paragraph*{§\:452.} Отъ вышеписанныхъ видимъ, хрістіанине: 1)~Коль великое есть соединеніе вѣрныхъ со Хрістомъ, Сыномъ Божіимъ, "--- а именно толикое, коликое всего тѣла и членовъ съ главою, "--- въ чемъ состоитъ превеликая честь, слава и радость ихъ. 2)~Какъ должны хрістіане Хрісту, яко Главѣ своей, повиноватися, и ничего противу воли Его не творити. Видимъ, что естественнаго тѣла уды вси повинуются главѣ своей, и ничего противу ея не творятъ. Руки ничего не дѣлаютъ, ноги не ходятъ, чрево не пріемлетъ пищи, языкъ не говоритъ, и все тѣло не движется, когда голова не помыслитъ и не захочетъ; а когда захочетъ голова, и что задумаетъ и захочетъ, тое и дѣлаютъ уды ея, и всякъ удъ свое дѣло отправляетъ: очи смотрятъ и, куды хочетъ, обращаются; языкъ говоритъ тое и столько, что хощетъ и сколько хощетъ она; ноги идутъ, куды хощетъ; руки дѣлаютъ тое, что хощетъ. Такъ и прочіи уды волѣ главы своея повинуются. Такое повиновеніе и послушаніе хрістіане обязуются оказывать Хрісту, Главѣ своей святой: не дѣлать ничего, чего Онъ не хощетъ; и дѣлать все, что Онъ хощетъ. Тако покажутъ, что они подлинно вѣрою и любовію съ Нимъ соединены, яко съ главою уды. "--- 3)~Отъ сего видно, что хрістіане отсѣчены и отторжены суть отъ Хріста и церкви Его святой, которые Ему послушанія не показуютъ, но по своимъ прихотямъ живутъ. Уды бо своей главѣ повинуются, но другаго тѣла главѣ не повинуются. Напр. Петрова тѣла уды не дѣлаютъ того, что Павлова голова задумаетъ и захочетъ: яко Петровы уды не соединены съ Павловою головою. Тако который хрістіанинъ не чинитъ послушанія Хрісту, свидѣтельствуетъ о себѣ, что онъ отторгнулся отъ Хріста, и не есть истинный и живый удъ церкве Его святыя, но есть удъ мертвый, и какъ \textit{розга изсохшая и отсѣченная отъ лозы}\footnote{Іоан.~15,~6.}, хотя и имя Хрістово исповѣдуетъ, и въ церковь съ вѣрными ходитъ, и прочія дѣла творитъ. "--- 4)~Паки видишь, какъ хрістіанинъ съ братомъ своимъ хрістіаниномъ обходиться долженъ, "--- а именно, какъ самъ съ собою, то есть, не лестно, не коварно, не лицемѣрно, но просто и искренно; чего себѣ не хощетъ, того и ему не дѣлать; и чего себѣ хощетъ, тое и ему дѣлать; что языкомъ ему говоритъ, тое и въ сердцѣ имѣть; и какъ внѣ является ему благосклоннымъ, ласковымъ, любительнымъ и благотворительнымъ, таковымъ и внутрь себе быть. Сего бо всего соединеніе удовъ, въ единомъ тѣлеси соединенныхъ, требуетъ. "--- 5)~Не есть истинный хрістіанинъ, какъ выше сказано, но есть лицемѣръ, кто съ ближнимъ не любовно, но лукаво и лицемѣрно и немилостивно обходится. 6)~Таковыя хрістіане да внимаютъ, что имъ пророкъ возвѣщаетъ: \textit{уже и сѣкира при корени древа лежитъ: всяко убо древо, не творящее плода добра, посѣкается и во огнь вметается}\footnote{Лук.~3,~9.}.

\section[Статья 7-я. О взаимной должности хрістіанской.]{статья седьмая.\\\bfseries О взаимной должности хрістіанской.}

Въ прешедшей статьѣ вкратцѣ предложено о должности хрістіанской другъ къ другу общей; какую должность имѣетъ хрістіанинъ къ хрістіанину. Въ сей предлагается разсужденіе о той должности, каковую имѣютъ начальники къ начальникамъ, и подчиненніи къ начальникамъ, родители къ дѣтямъ, и дѣти къ родителямъ, и проч.

\subsection[Глава 1-я. О власти хрістіанской вообще и подвластныхъ.]{глава первая.\\\bfseries О власти хрістіанской вообще и подвластныхъ.}

\begin{quotation}\textit{Всяка душа властемъ предержащимъ да повинуется. Нѣсть бо власть, аще не отъ Бога: сущія же власти отъ Бога учинены суть. Тѣмже противляяйся власти, Божію повелѣнію противляется: противляющіися же, себѣ грѣхъ пріемлютъ}\footnote{Римл.~13,~1,~2 и слѣд.; 1~Петр.~2,~13--19; Премудр.~6,~1--8 и проч.}.\end{quotation}

\paragraph*{§\:453.} Видимъ, что въ хрістіанскомъ обществѣ двоякая власть: гражданская и духовная. Гражданская власть управляетъ внѣшняя дѣла; духовная власть назираетъ тое, что касается до внутренняго, душевнаго состоянія. Гражданская законами гражданскими управляетъ; духовная словомъ Божіимъ исправляетъ. Гражданская безчиніе и соблазны подвластныхъ мечемъ пресѣкаетъ, духовная словомъ Божіимъ, страхомъ и судомъ Божіимъ, и отлученіемъ временнымъ отъ хрістіанскаго общества неблагообразно ходящихъ смиряетъ. Обѣихъ властей, гражданской и духовной, конецъ долженъ быть не иной, какъ благое состояніе подвластныхъ и слава имене Божія.

\paragraph*{§\:454.} Всякая власть, великая или малая, отъ Бога есть, по писанному: \textit{нѣсть власть, аще не отъ Бога}. Учинена же власть отъ Бога, какъ видимъ, къ доброму концу: 1)~къ отвращенію всякаго неблагополучія, которое въ обществѣ можетъ произыти. 2)~Къ усмиренію безчинія, нахальства и насилія продерзыхъ людей; къ защищенію неповинности и незлобія добрыхъ людей, якоже апостолъ Петръ учитъ, что \textit{князи посланы отъ царя во отмщеніе злодѣемъ, въ похвалу же благотворцемъ}\footnote{1~Петр.~2,~14.}. И Павелъ святый глаголетъ: \textit{князи не суть боязнь добрымъ дѣломъ, но злымъ. Хощещи ли не боятися власти? благое твори, и имѣти будеши похвалу отъ него. Божій бо слуга есть тебѣ во благое. Аще ли злое твориши, бойся: не бо всуе мечь носитъ; Божій бо слуга есть, отмститель въ гнѣвъ злое творящему}\footnote{Римл.~13,~3--6.}. 3)~Къ снисканію и храненію всякаго благополучія какъ тѣлеснаго, такъ и душевнаго. «Богъ, глаголетъ святый Златоустъ, во общую пользу начальства учинилъ»\footnote{Бес.~6"~я на 1"~е посл. къ Тим.}.

\paragraph*{§\:455.} Сего ради всякая власть, какъ голова въ тѣлѣ человѣческомъ, въ обществѣ своемъ. Всякое бо общество подобно есть человѣку живому. Что въ человѣкѣ голова, тое въ обществѣ власть. Голова въ человѣкѣ, яко ума и разсужденія сѣдалище, управляетъ всѣми удами, и о всѣхъ ихъ промышляетъ: тако и власти должно всѣхъ себѣ подчиненныхъ управлять, и о нихъ промышлять. И какъ голова печется о добромъ состояніи всего состава тѣлеснаго, тако и власти подобаетъ пещися и стараться о всякомъ благополучіи подчиненнаго себѣ общества; \textit{яко на сіе истое и отъ Бога учинена}, какъ выше сказано. Всякій бо властелинъ избирается и поставляется не ради его самого, но ради подчиненныхъ пользы; того ради о пользѣ ихъ и пещися долженъ. Сего ради всякій благочестивый властелинъ, гражданскій и духовный, долженъ быть своимъ подчиненнымъ, какъ отецъ своимъ дѣтямъ. Какъ отецъ благоразумный дѣтей своихъ въ страхѣ содержитъ и милуетъ ихъ: тако и властямъ должно съ своими подвластными поступать. Они въ томъ высочайшей и безначальной Власти, Самому небесному Отцу, подражать должны, Который насъ согрѣшающихъ наказуетъ и милуетъ, біетъ и милости не отнимаетъ, уязвляетъ и исцѣляетъ, опечаляетъ и утѣшаетъ. Среднимъ убо путемъ ити должно властямъ, и съ подчиненными своими поступать: не давать имъ по своей волѣ жить, согрѣшающихъ наказывать, страхомъ обуздовать и къ добронравію приводить, и мучительства надъ ними не дѣлать.

\paragraph*{§\:456.} Отъ вышеписанныхъ видно: 1)~Не можетъ быть истинный и полезный властелинъ, какъ только добросовѣстный и разумный. Ибо злый властелинъ, хотя и разумный, будетъ безсовѣстно съ подчиненными поступать, о нихъ нерадѣть; о себѣ, а не о нихъ пещися, своей корысти, а не ихъ пользы искать будетъ, и такъ вмѣсто общаго благополучія злополучіе обществу учинитъ. Такожде и добросовѣстный, но безъ разума, властелинъ мало что успѣетъ. Ибо хотя и будетъ пещися о общей пользѣ, но отъ оскудѣнія ума часто не тое будетъ дѣлать, что должно, и когда должно, кому и что должно; и тако хотя на добрый конецъ производить будетъ дѣла своя, но часто во вредъ общества своего. Однакожъ полезнѣйшій обществу добрый властелинъ безъ мудрости, нежели злый съ мудростію и хитростію своею. "--- 2)~Весьма худо дѣлаютъ тыи власти, которыя о своей корысти, а не о общемъ благополучіи пекутся. Тако бо они противятся Божіему установленію, Который властей учиняетъ къ сохраненію общаго благополучія. Всякая бо власть не ради своей корысти, но ради цѣлости общества поставляется, какъ сказано: того убо и искать обществу должно, ради чего поставляется. Но когда превратно, вмѣсто общей пользы, своей ищетъ, то и приноситъ обществу вредъ, а не пользу, и тако не созидаетъ, но разоряетъ общество и, вмѣсто благодѣтеля, дѣлается того врагомъ. "--- 3)~Какъ тыя власти худо дѣлаютъ, которыя своихъ подчиненныхъ неумѣренно и не по дѣламъ наказуютъ и мучатъ: такъ и тыя, которыя оставляютъ преступниковъ закона безъ наказанія. Тако бо подаютъ имъ поводъ къ другимъ грѣхамъ и беззаконіямъ. Воръ, не наказанный за беззаконіе его, на тоежде беззаконное дѣло паки обращается. Ибо беззаконіе, безъ наказанія оставленное, не иное что есть, какъ дверь къ прочіимъ беззаконіямъ. Оттуду послѣдуетъ привычка ко грѣху и ожесточеніе во грѣхѣ, чему явная погибель слѣдуетъ. Преступника убо безъ наказанія оставлять есть не милость, но безчеловѣчіе и безуміе, которое и милующаго къ погибели ведетъ. Умѣренность вездѣ похвальна: и наказаніе съ милостію, и милость съ правосудіемъ и наказаніемъ требуется отъ благочестиваго властелина. Сіе показуется и отъ естества, въ которомъ и морозъ неумѣренный и зной чрезвычайный вредителенъ; но теплота прохлажденіемъ и ведро дождемъ растворенное полезно есть: тако и общество умѣренностію созидается и процвѣтаетъ. "--- 4)~Худо и не по"=хрістіански дѣлаютъ, которые всякими происками или ходатайствомъ другихъ, или мздою, или инымъ какимъ способомъ на честь восходятъ. Симъ они показуютъ, что или славолюбіемъ суетнымъ, что хрістіанамъ неприлично, или желаніемъ сквернаго прибытка, что имъ самимъ и обществу вредно, недугуютъ: и тако тѣмъ самымъ исканіемъ показуютъ себе недостойныхъ власти. Всякая бо власть хрістіанину не покой и честь, но большій есть крестъ, большими и множайшими трудами, попеченіемъ и всегдашнимъ терпѣніемъ обремененный, чего никто не пожелаетъ. Надобно убо всякому прежде сдѣлать и уготовать себе достойнымъ чести, и званія Божія ожидать, которое бываетъ избираніемъ высшихъ и преимущихъ властей.

\paragraph*{§\:457.} Должность къ властямъ подчиненныхъ имъ: 1)~Подчиненніи своимъ властямъ должны повиноватися во всемъ, и повиноватися Господа ради, какъ учитъ апостолъ Петръ: \textit{повинитеся всякому человѣчу начальству Господа ради}\footnote{1~Петр.~2,~13.}! Богъ ихъ учинилъ: Бога ради должно имъ и повиноватися. Богъ ихъ повелѣваетъ слушать: ради Бога повелѣвающаго и слушать ихъ должно. Кто убо власти повинуется, тотъ Богу повинуется, якоже \textit{противляяйся власти, Божію повелѣнію противляется}\footnote{Римл.~13,~2.}. Аще бо что повелѣваютъ согласно Божію закону, Божіе есть повелѣніе, а не человѣческое, хотя и человѣки повелѣваютъ. "--- 2)~Должно ихъ съ любовію почитать, яко отцевъ и промыслителей своихъ. Самая бо совѣсть убѣждаетъ насъ любить и почитать тѣхъ, которые о пользѣ нашей пекутся и промышляютъ. Благочестивіи же власти не иное что суть, какъ общаго благополучія искатели и хранители. Отцевъ нашихъ по плоти любимъ и почитаемъ, яко родили насъ и промышляютъ о насъ: кольми паче властей благочестивыхъ, которыя и о насъ и о нашихъ отцахъ пекутся и промышляютъ, "--- промышляютъ, глаголю, о насъ, яко о обществѣ, въ которомъ и мы заключаемся, промышляютъ. Ибо, кто пользуетъ все тѣло, тотъ пользуетъ и всѣ уды тѣлесные, изъ которыхъ тѣло состоитъ. Они суть Божіи служители, Божіи посланники и какъ бы Божіи намѣстники, которые должность, порученную имъ отъ Бога, живущаго на небеси, отправляютъ на земли; сего ради какъ Божіихъ посланниковъ и служителей, въ пользу нашу посланныхъ, принимать и почитать должно. "--- 3)~Должно о нихъ молитися Богу, дабы сохранилъ ихъ животъ въ цѣлости и укрѣпилъ, соблюлъ ихъ отъ всякихъ вражіихъ навѣтовъ, вразумилъ и умудрилъ ихъ въ общую пользу, чтобы они могли отвращать все тое, что обществу вредъ наноситъ, и дѣлать тое, чимъ общее благополучіе созидается и умножается. Отъ добрыхъ бо и разумныхъ властей общее благополучіе зависитъ. Домъ тотъ въ добромъ состояніи находится, въ которомъ добрый и разумный хозяинъ имѣется: тако благополученъ тотъ градъ, въ которомъ добрый и премудрый градодержитель живетъ; благополучны тіи люди, которыи добраго, мудраго и тщательнаго пастыря имѣютъ; благополучно тое воинство, которымъ добрый, искусный и тщательный полководецъ управляетъ; благополучно тое государство, надъ которымъ добрый, премудрый и тщательный царствуетъ монархъ. Сего ради нужно есть всѣмъ подвластнымъ о властяхъ своихъ со усердіемъ молитися Богу, дабы онѣ исправны въ дѣлѣ своемъ были, и тако при благополучіи ихъ и сами были благополучными. Къ сему хрістіанъ апостолъ святый увѣщаваетъ: \textit{молю прежде всѣхъ творити молитвы, моленія, прошенія, благодаренія, за вся человѣки, за царя и за вся, иже во власти суть, да тихое и безмолвное житіе поживемъ, во всякомъ благочестіи и чистотѣ. Сіе бо добро и пріятно предъ Спасителемъ нашимъ Богомъ}\footnote{Тим.~2,~1--4.}.

\paragraph*{§\:458.} Хрістіане, которые вышеписанныхъ должностей не исправляютъ, тяжко грѣшатъ. Видно, что таковыи не знаютъ, что суть и отъ кого учинены власти, и къ какому концу учинены; и откуду общее благополучіе, въ которомъ и ихъ заключается, проистекаетъ, ни мало не чувствуютъ. А паче тіи безстыдно и беззаконно дѣлаютъ, которые злые умыслы сочиняютъ и возстаютъ противу своихъ властей, законно учиненныхъ, и не иное что суть, какъ сыны погибельные и враги отечества, и общаго благополучія. Надобно имъ опасаться онаго страшнаго суда Божія, который постигъ Корея, Даѳана и Авирона, возставшихъ на Моисея и Аарона, которыхъ \textit{живыхъ земля пожрала}\footnote{Числ.~16,~25--32.}.

\paragraph*{§\:459.} \textit{Что"=де тѣмъ властямъ дѣлать, которыя неправду дѣлаютъ? Отвѣтъ}. 1)~Что неправду дѣлаютъ, что къ тебѣ? ты свое дѣлай, и по должности своей поступай. За неправду свою они воздадятъ отвѣтъ праведному Богу, \textit{Иже истяжетъ дѣла ихъ, и помышленія испытаетъ: яко слузи суще царства Его, не судили право, ни сохранили закона, ниже по воли Божіей ходили}, и проч.\footnote{Премуд.~6,~3,~4 и слѣд.} 2)~Что не противное закону Божію приказуютъ, слушай и исполняй: въ противномъ не слушай, яко \textit{повиноватися подобаетъ Богови паче нежели человѣкомъ}\footnote{Дѣян.~5,~29.}. Тако поступали мученики святіи. Повелѣвали имъ нечестивыя власти землю копать, копали; каменіе носить, носили; въ темницу и ссылку итить, шли; подъ мечь главы подклонять, подклоняли; имѣнія у нихъ отнимали, отдавали, "--- и прочая, закону Божію не противная, дѣлать приказывали, дѣлали. Повелѣвали отрещися Хріста и прочая богопротивная дѣлать, не слушали. Такъ и ты поступай. Велитъ тебѣ господинъ твой всякую работу отправлять, отправляй: велитъ неправду дѣлать, обидѣть, украсть, солгать и прочая, не слушай. Грозитъ за то казнію, не бойся: яко \textit{Бога боятися подобаетъ паче, имущаго власть и душу и тѣло погубити въ гееннѣ, нежели человѣка, едино тѣло убивающаго, души же не могущаго убити}\footnote{Матѳ.~10,~28.}. Лишаетъ живота, не противься: пожирай животъ за правду, да въ грядущемъ вѣкѣ обрящешь его.

\subsection[Глава 2-я. О должности благочестивыхъ монарховъ и подданныхъ ихъ.]{глава вторая.\\\bfseries О должности благочестивыхъ монарховъ и подданныхъ ихъ.}

\begin{quotation}\textit{И приближишася Давиду дніе умрети ему, и заповѣда Соломону сыну своему, глаголя: азъ отхожду въ путь всея земли: ты же крѣпися, и буди мужъ совершенъ. И сохрани завѣтъ Господа Бога твоего, еже ходити во всѣхъ путехъ Его, хранити заповѣди Его, и оправданія Его, и судьбы Его, и свидѣнія Его, писанная въ законѣ Моѵсеовѣ}\footnote{3~Царст.~2,~1,~2 и 3.}.\end{quotation}

\paragraph*{§\:460.} 1)~Монархъ не имѣетъ надъ собою обладателя на земли, но имѣетъ на небеси, Который небомъ и землею владѣетъ: \textit{владѣетъ царствомъ человѣческимъ, и емуже, хощетъ, даетъ е}\footnote{Дан.~4,~22.}. Не имѣетъ повелителя на земли, но имѣетъ на небеси; не имѣетъ надзирателя и судіи "--- человѣка, но имѣетъ Бога, Который \textit{вся дѣла, слова и помышленія человѣческая назираетъ}\footnote{Пс.~32,~13--15; 138,~1--16.}; и \textit{судитъ всякому по дѣломъ его}\footnote{Римл.~2,~6.}, \textit{не щадитъ лица, ниже усрамляется вельможи}\footnote{Премудр.~6,~7.}. Сего ради долгъ имѣетъ благочестивый монархъ, что подданнымъ своимъ повелѣваетъ, тое самъ вопервыхъ дѣлать, и что имъ запрещаетъ, отъ того самъ прежде удаляться, дабы, за что прочіихъ судитъ, въ томъ самого совѣсть не судила предъ Богомъ. "--- 2)~Царь на высокомъ мѣстѣ сѣдитъ, и потому всѣмъ виденъ; вси на него обращаютъ глаза свои, не токмо присутствующіи, но и далеко отъ него живущіи, дѣла и слова его примѣчаютъ, и вси хотятъ знать, что дѣлаетъ въ палатахъ своихъ, и образъ житія своего отъ него пріемлютъ, и по примѣру его обращаться хотятъ. Того ради хотя и всѣмъ хрістіанамъ, но наипаче на высокомъ свѣщницѣ поставленнымъ, нужно есть примѣчать Хрістово слово: \textit{тако да просвѣтится свѣтъ вашъ предъ человѣки, яко да видятъ ваша добрая дѣла, и прославятъ Отца вашего, Иже есть на небесѣхъ}\footnote{Матѳ.~5,~16.}. "--- 3)~Царскій престолъ много окружаетъ ласкателей, какъ довольно того видимъ въ исторіяхъ, отъ которыхъ иніи въ милость монарху вкрасться, иніи злобу на другаго наченшуюся совершить желая, много могутъ лестнымъ и ядовитымъ своимъ языкомъ успѣть, недостойныхъ похваляя, достойныхъ минуя или пороча, и тако человѣка, незнающаго умысловъ сердечныхъ, обмануть, и тѣмъ обществу вредъ нанести. Того ради благочестивая корона одолжается не скоро таковымъ вѣрить, но испытовать; а клеветниковъ изгонять, по примѣру святаго царя Израилева Давида: \textit{оклеветающаго тай искренняго, сего изгоняхъ: гордымъ окомъ и несытымъ сердцемъ, съ симъ не ядяхъ}\footnote{Пс.~100,~5.}. "--- 4)~Царскую корону украшаютъ премудрость, правда и милость: \textit{съ правдою уготовляется престолъ начальства}\footnote{Притч.~16,~12.}. \textit{Милость и истина сохраненіе царю, и обыдутъ престолъ его въ правдѣ}\footnote{20,~28.}. Правда требуетъ, чтобы всѣмъ воздавать должная, добрымъ и вѣрноподданнымъ за вѣрность ихъ награжденіе, злымъ и неисправнымъ наказаніе, да и прочіи хотящіи нечествовати страхъ имѣютъ. Милость хощетъ объятіями своими всѣхъ согрѣвати, но, понеже многіи являются милости недостойны, премудрость растворяетъ тую правдою, и научаетъ добрыхъ жаловать, неисправныхъ наказывать, но съ милостію и надеждою исправленія, ожесточенныхъ отсѣкать отъ общества добрыхъ, да не и тіи злостію ихъ повредятся. Тако поступаетъ съ нами праведный, милостивый и премудрый Царь небесный, Царь царей и Господь господей, \textit{гордымъ противится, смиреннымъ же даетъ благодать}\footnote{1~Петр.~5,~5.}; добрыхъ любитъ и благословеніемъ Своимъ обогащаетъ ихъ; неисправныхъ наказуетъ, но наказанныхъ и смирившихся милуетъ; ожесточенныхъ отвергаетъ отъ лица Своего и предаетъ вѣчному наказанію. Сему небеснаго Царя примѣру подражать имѣютъ долгъ учиненніи отъ Него и подчиненніи Ему цари земніи и съ своими подданными поступать, и съ царствующимъ пророкомъ пѣть Господу: \textit{милость и судъ воспою Тебѣ, Господи}\footnote{Пс.~100,~1 и слѣд.}. Строгость бо излишняя вредитъ, и милость безразсудная не полезна. Оная въ негодованіе, огорченіе и уныніе людей приводитъ; сія къ разслабленію, своевольству, безчинію и развращенію дверь отворяетъ: правда же, съ милостію сопряженная, все тое отвращаетъ. Сіе"=то есть премудрость властелина, сіе есть царскій путь "--- правды держаться, и милости не забывать. Худо тамо являть милость, гдѣ наказаніе требуется; худо прощать и щадить того, кто не оставляетъ грѣха и не кается. Грѣхъ ненаказанный къ инымъ грѣхамъ подаетъ поводъ; и единъ злодѣй прощенный многимъ къ подобному злодѣянію поощреніе подаетъ. Ибо злыи, видя ненаказаннаго злодѣя, на тоежде злодѣяніе обращаются съ такою надеждою, что и имъ такожде прощено будетъ въ случаѣ обличенія. Отсюду послѣдуетъ умноженіе злодѣяній, и отъ тѣхъ отечеству немалая утрата. Напротивъ того, не хорошо тамо мечь употреблять, гдѣ пластырь милости и обязанія требуется, гдѣ грѣхъ отъ немощи, каковому и добрые люди подлежатъ, а не отъ умысла и предразсужденія содѣянъ; безполезно и того наказывать, который самъ себе добровольно наказуетъ. Отъ мудрости удаленное есть тое наказаніе, которое не по мѣрѣ преступленія налагается. Вездѣ умѣренность, какъ въ натурѣ, такъ и въ гражданствѣ, похвальна и полезна, якоже выше упомянуто. Чего ради вездѣ нужно есть здравое разсужденіе и нескорое опредѣленіе. "--- 5)~Истинное правосудіе лица человѣческаго не пріемлетъ: съ богатымъ и нищимъ равно поступаетъ, сановитаго не стыдится, и, подлаго не презираетъ, не смотритъ лицъ, но дѣла ихъ разсуждаетъ, и по тѣмъ судъ издаетъ. Подлаго, неисправнаго наказуетъ, и сановитому, не по чести сана и присягѣ своей поступившему, не спущаетъ: но паче отъ большаго и высшаго лица большій грѣхъ, и грѣху послѣдующее большее наказаніе заключаетъ, разсуждая, что, ежели зло есть подлому, большее зло есть сановитому неправду дѣлать, который на то учрежденъ, чтобы неправду искоренять и правду насаждать. Сего ради благочестивый царь, когда хощетъ правосудіе хранить, хранить же необходимо нужно, должность имѣетъ лицъ не пріимать, но на дѣла ихъ смотрѣть, и по тѣмъ судить, по примѣру небеснаго Царя, Который \textit{лица человѣча не пріемлетъ}\footnote{Гал.~2,~6.}, \textit{не щадитъ лица, ниже усрамится вельможи}\footnote{Премудр.~6,~7.}. Великіи и малыи, сановитыи и подлыи равно суть отечества сыны, равно отечество составляютъ, равнаго убо отеческаго о себѣ промысла отъ монарха требуютъ, и въ покойномъ отечества домѣ согрѣваться и питаться хотятъ. Паче же мнящіися быть подлѣйшіи большую приносятъ пользу. Что городъ безъ деревни можетъ? Откуду хлѣбъ и прочая, къ содержанію житія нужная? Деревня приноситъ все во градъ; поселянинъ питаетъ гражданина, и земледѣлецъ господина: запустѣетъ градъ безъ деревни, и господинъ оскудѣетъ безъ владѣльца. Полководецъ безъ солдатъ какъ тѣло или голова безъ рукъ и ногъ. Якоже убо равно отечество вси составляютъ и пользуютъ: такъ равно и монархъ благочестивый, яко глава о своихъ удахъ, о всѣхъ промышлять отечески обязуется; и какъ великаго гордаго смирять, такъ и малаго насилуемаго защищать и избавлять. "--- 6)~Хотя на высокомъ мѣстѣ сидитъ царь, однакожъ помнить ему необходимо нужно, что онъ такойжде человѣкъ, какъ и прочіи люди, такожде умираетъ и землѣ предается, какъ и подданніи его, гдѣ царя отъ подданнаго, и господина отъ раба, и вельможи отъ простолюдина нѣтъ никакого разнствія: смерть бо всѣхъ равными дѣлаетъ, и въ землю возвращаетъ и обращаетъ. Сего ради, по Писанію, \textit{елико великъ есть, толико ему смирятися долгъ надлежитъ, да отъ Господа обрящетъ благодать}\footnote{Сирах.~3,~18.}. Аще бо Хрістосъ, Царь царей и Господь господей, вѣрныхъ Своихъ не стыдится \textit{братіею нарицати}\footnote{Іоан.~20,~17; Евр.~2,~11; Пс.~21,~23.}: кольми паче человѣку подобныхъ себѣ человѣковъ за братію имѣти достойно. Отсюду проистекутъ хрістіанскія добродѣтели, какъ ручьи отъ источника; отсюду терпѣніе, хрістіанская побѣда, которая не надъ побѣжденными врагами, но Надъ своимъ торжествуетъ сердцемъ; отсюду кротость и незлобіе, которыми сердце въ тишинѣ сохраняется. Блаженна тая корона, которую сіи многоцѣнные и не міра сего каменіе украшаютъ. Болѣе она прославится, нежели побѣдами надъ врагами внѣшними: преславнѣе бо себе побѣдить, нежели людей. Болѣе она сохранитъ грады свои, нежели оружіе воинское: яко Бога Защитника своего имѣти будетъ. Болѣе научитъ людей своихъ, нежели законы и указы: болѣе бо научаемся дѣломъ, нежели словомъ, и примѣромъ добрымъ, нежели повелѣніемъ высшихъ. "--- 7)~Благочестивымъ царямъ долгъ належитъ защищать и умножать церковь святую, по пророчеству святаго Пророка: \textit{и будутъ царіе кормители твои, княгини ихъ кормилицы твои}\footnote{Ис.~49,~23.}. Сего ради должно имъ отъ гонителей и ругателей тую защищать, о добрыхъ пастыряхъ промышлять; ученія полезныя, которыми она утверждается, умножается, расширяется и процвѣтаетъ, расширять; училища, въ которыхъ сѣются сѣмена ученій, насаждать, распространять, и ихъ правильно содержать. "--- 8)~Благочестивыхъ государей должность о тишинѣ и благополучіи отечества своего промышлять: того ради нужно имъ избирать и посылать на власти добрыхъ и разумныхъ людей, которые бы правосудіе, какъ зѣницу ока, хранили, и общей, а не своей пользы, на власти будучи, искали, и тако бы подданныхъ его цѣлость и благополучіе хранили и умножали. Отъ добрыхъ бо и разумныхъ властей цѣлость и благополучіе общества зависитъ, якоже отъ злыхъ и неисправныхъ оно изнемогаетъ, разоряется и падаетъ. "--- 9)~Должность есть благочестивыхъ монарховъ отечество свое отъ иноплеменниковъ защищать. Того ради, когда начинаютъ войну, должно имъ крайне берещися отъ неправильной войны, и брани безъ правильной причины не начинать; свое хранить, и чужаго не касаться, да не и своего лишатся, и подданныхъ своихъ напрасно кровь проліютъ, и предъ Богомъ тяжко согрѣшатъ. Единаго человѣка напрасно кровь пролить страшно, кольми паче многихъ. Единаго, напрасно убіеннаго, кровь вопіетъ на небо, и отмщеніе низводитъ, кольми паче многихъ. За всѣхъ, на неправильной войнѣ убіенныхъ, слѣдуетъ отвѣтъ дать Богу, Который сердца человѣческія испытуетъ, и замыслы и начинанія знаетъ, въ книзѣ своей записываетъ все\footnote{Пс.~138,~16.}. Понеже не человѣческою, но Божіею мудростію управляется царство и не человѣческою, но Божіею всемогущею силою защищается и сохраняется: того ради благочестивый царь долгъ имѣетъ молитися Царю небесному, да самъ Онъ наставитъ и умудритъ его управляти царствомъ, ему порученнымъ. "--- 11)~Цари великое бремя трудовъ несутъ, всегдашнимъ попеченіемъ о цѣлости отечества отягченное, и отъ того не мало скуки, печали и изнемоганія имъ временемъ находитъ. Сего ради благочестивое монаршее сердце въ семъ случаѣ можетъ себе утѣшать надеждою будущаго въ вѣчномъ животѣ воздаянія: яко всѣ труды, скорби и печали скоро минуются, и за многіе труды, вкратцѣ о Хрістѣ подъятые, многое и безконечное будетъ воздаяніе. Праведенъ бо есть Господь: всѣмъ воздастъ по дѣломъ ихъ.

\paragraph*{§\:461.} \textit{Должность вѣрноподданныхъ} къ своему благочестивому монарху: 1)~Должно его отъ чистаго сердца любить, яко перваго по Бозѣ отца, промыслителя и попечителя, о цѣлости отечества и общемъ благополучіи неусыпно тщащагося. 2)~Высокое ему, яко отъ Бога посланному, и любовное отдавать почтеніе. 3)~Повелѣнія и указы его, яко въ пользу общества учиненные, безъ роптанія и съ усердіемъ исполнить. 4)~Дань, отъ него требуемую, давать доброхотно и безъ задержанія. 5)~О здравіи его, яко отечеству нужномъ, мирномъ царствованіи и мудромъ правленіи молить усердно Бога, яко \textit{Его совѣтъ и утвержденіе, Его разумъ, Его крѣпость; Имъ царіе царствуютъ и сильніи пишутъ правду; Имъ вельможи величаются, и властелини Имъ держатъ землю}\footnote{Притч.~8,~14--16.}. 6)~Въ случаѣ за здравіе его и своего здравія не щадить. Главу въ естественномъ тѣлѣ, когда слѣдуетъ ей опасность, вси уды защищаютъ и хранятъ, хотя и сами страждутъ: яко отъ бѣдствія главы и всему тѣлу слѣдуетъ бѣдствіе. Въ обществѣ глава есть царь, котораго цѣлость и здравіе всѣмъ хранить должно: яко отъ его цѣлости и общества цѣлость зависитъ, и отъ его бѣдствія всему обществу бѣдствіе бываетъ.

\subsection[Глава 3-я. О должности судей и къ суду приходящихъ.]{глава третія.\\\bfseries О должности судей и къ суду приходящихъ.}

\begin{quotation}\textit{Разслушайте посредѣ братій вашихъ, и судите праведно посредѣ мужа, и посредѣ брата его, и посредѣ пришельца его. Да не познаете лица въ судѣ, малому и великому судиши, и не устыдишися лица человѣча, яко судъ Божій есть}\footnote{Второз.~1,~16 и 17.}.\end{quotation}

\paragraph*{§\:462.} Истинное правосудіе необходимо въ судѣ нужно, такъ что безъ него праведный судъ быть не можетъ, но только видъ суда есть. Оно въ томъ состоитъ, въ чемъ и дѣло правды. Оно требуетъ, чтобы праваго оправдать, а виноватаго осудить; чтобы насилованный былъ защищенъ, насиловавшій смиренъ, озлобленный удовольствованъ, озлобившій наказанъ, да и прочіи страхъ имутъ. Слѣдственно правосудіе въ дѣлѣ суда лица не знаетъ, якоже повелѣваетъ Богъ: \textit{да не познаете лица въ судѣ}; но только дѣла единаго смотритъ и слушаетъ. Ему богатый и нищій, сановитый и подлый, одноземецъ и иноземецъ, сродникъ и несродникъ, другъ и врагъ, знаемый и незнаемый въ дѣлѣ суда равны суть. Оно богатаго и нищаго, сановитаго и подлаго, друга и врага, сродника и чуждаго означаетъ виноватымъ, когда находитъ въ немъ неправду; и тогожде правымъ судитъ, когда дѣло его оправдаетъ. И сіе"=то есть \textit{лица въ судѣ не принимати}. Откуду правосудіе не неприлично уподобляется человѣку, имѣющему закрытые глаза, потому что въ судѣ не смотритъ на лица, но дѣлъ единыхъ слушаетъ; уподобляется имѣющему одно отверстое, другое закрытое ухо, яко однимъ донощика слушаетъ, другое отвѣтчику бережетъ; уподобляется вѣсамъ, яко дѣло на разсужденіи, какъ на вѣскахъ, положенное, мѣритъ и разсматриваетъ, и на которую сторону правда переважитъ, къ той и оно склоняется, пристаетъ, и тую похваляетъ и оправдаетъ.

\paragraph*{§\:463.} Четыре вещи суть наипаче, которыя правосудію быть не попущаютъ и судъ праведный превращаютъ: страсть лихоиманія, страхъ человѣческій, любовь плотская и ненависть враговъ. Лихоиманіе привлекаетъ руку судіи къ пріятію даровъ, и ради тѣхъ оправдать богатаго, хотя бы и виноватъ былъ, и осудить нищаго, хотя бы и правъ былъ, научаетъ. Страхъ человѣческій можетъ судіино сердце превратить, когда высшія лица, которымъ самъ судія подчиненъ, по виноватомъ побораютъ и грозятъ ему лишеніемъ сана, или инымъ чимъ, когда по ихъ желанію не сдѣлаетъ. Любовь плотская, напр. любовь братій, сродниковъ, друзей и прочіихъ любимыхъ, которые или ходатайствуютъ о виноватомъ, или сами въ судѣ являются виноватыми, ударяетъ судейское сердце и хощетъ вѣсъ правосудія превратить. Ненависть враговъ, которые въ дѣлѣ суда бываютъ съ другами судіи или иными какими лицами не врагами его, и являются по суду враги его правы, а други и прочіи виноваты, нудится судію развратить, и друга, яко любимаго, оправдать, и врага, яко навѣтника, осудить побуждаетъ. Сіи суть наипаче враги правосудія! сіи хотятъ правду свергнуть съ престола своего и посадить неправду! Судіямъ благочестивымъ должно не лицъ, но дѣлъ смотрѣть, и дарами не прельщаться, и страха человѣческаго не бояться. Аще бо судія руку простретъ на дары, и по тѣмъ будетъ судить: на мѣстѣ правды мамону поставитъ, и вмѣсто судебнаго мѣста торжище учинитъ. Аще ради страха человѣческаго неправду сдѣлаетъ: человѣка, а не Бога, Который правду повелѣлъ хранить, бояться будетъ, и человѣческій страхъ Божію страху предпочтетъ. Аще любовію друговъ преклонится: Бога, возлюбившаго его, оставитъ и оскорбитъ. Аще ради вражды врага осудитъ: отмщевать будетъ, а не судить, и злобу совершать, а не судъ производить. И тако, когда что нибудь отъ сихъ учинитъ, престанетъ быть праведнымъ судіею, слѣдственно и истиннымъ хрістіаниномъ. Безъ правды бо и совѣсти доброй какъ судія праведный, такъ и истинный хрістіанинъ быть не можетъ. Аще убо судія хощетъ правду и благочестіе хранить, долженъ неподвижимо ни на ту, ни на другую сторону стоять, и противу всего выше прописаннаго подвизатися, да работаетъ правдою Богу, а не мамонѣ неправдою. \textit{Никтоже бо можетъ двѣма господинома работати: и Богу работать и мамонѣ невозможно}\footnote{Матѳ.~6,~24.}. Страхъ человѣческій страхомъ Божіимъ да прогонитъ, и да укрѣпитъ себе упованіемъ на Бога, Котораго суду и высшія и низшія власти и судіи подлежатъ. Любовь плотскую да погаситъ любовію Божіею, и ненависть вражію да побѣдитъ терпѣніемъ и кротостію. И тако правосудіе сохранится, и правда надъ неправдою торжествовать будетъ.

\paragraph*{§\:464.} Къ храненію правосудія поощряютъ судей слѣдующая: 1)~Богъ "--- Вѣчная Правда, Который правдою благоугождается и неправдою раздражается, Который въ судѣ невидимо присутствуетъ, и судейскіе замыслы, намѣренія и дѣла видитъ, и слова слышитъ, и въ книзѣ Своей записываетъ, и \textit{во свѣтѣ приведетъ тайная тмы, и объявитъ совѣты сердечныя}\footnote{1~Кор.~4,~5.}, и къ судіямъ чрезъ пророка Своего глаголетъ: \textit{не сотворите неправды въ судѣ}\footnote{Лев.~19,~15.}. "--- 2)~Хрістіанская должность, которая требуетъ, чтобы оправданія плоды показывали, и правду творили, и совѣсть свою хранили. Аще всѣмъ хрістіанамъ необходимо нужно правду хранить, когда хотятъ хрістіанами быть; кольми паче судіямъ, которые яко лучшіи и разумнѣйшіи отъ всѣхъ, избраны, и, какъ свѣтила, на высокомъ мѣстѣ поставлены, и потому должны свѣтъ правды издавать, и прочіихъ къ правдѣ приводить. "--- 3)~Клятвенное обѣщаніе, которое судіи учинили предъ Богомъ и святымъ Его Евангеліемъ, и обѣщались правду хранить, защищать, и неправду искоренять, и въ томъ совѣсти своей свидѣтеля Бога призывали. Сія ихъ такъ важная присяга да подвигнетъ ихъ къ храненію правды, которая нарушиться не можетъ безъ хуленія и безчестія имени Божія и ихъ явныя погибели. Богъ бо поруганъ не бываетъ, Который обѣщаніе ихъ и присягу, и призываніе Своего имени во свидѣтельство совѣсти ихъ слышалъ. "--- 4)~Цѣлость и благополучіе общества, въ которомъ и сами они заключаются, которое отъ правды хранимой созидается и процвѣтаетъ, отъ неправды разоряется и падаетъ. Откуду разбой, насилія, хищенія и всякая неправда въ злыхъ? отъ неправды судейской. Правда судейская искореняетъ зло, но неправда научаетъ злу. Единъ и другій злый человѣкъ избылъ мздою отъ казни; съ сею надеждою и многіи злу поучаются, и тако растетъ и умножается зло. Отсюду гнѣвъ Божій и страшные суды посылаются на отечество. Общая неправда общею казнію казнится. Нынѣшній вѣкъ видитъ зло сіе, въ которомъ попрана правда, и неправда мѣсто свое имѣетъ. Видимъ зло сіе, и оплакиваемъ и воздыхаемъ, дабы призрѣлъ Господь на слезы бѣдныхъ вдовицъ и сиротъ, и отвратилъ зло тое. Да подвигнетъ убо судей общее благополучіе къ храненію правды, и общее неблагополучіе да отвратитъ отъ неправды. "--- 5)~Читаемъ въ исторіяхъ, что язычники, незнающіи истиннаго Бога и закона Его писаннаго, но естественнымъ закономъ управляемыи, хранили правду въ судѣ. Хрістіанамъ стыдно въ сѣмъ дѣлѣ низшими оставаться отъ язычниковъ. А которые низшіи отъ нихъ бываютъ, то тѣмъ показываютъ, что и горшими отъ нихъ дѣлаются. Сего ради и страшнѣйшему суду Божію паче язычниковъ подлежатъ. "--- 6)~Видимъ въ Писаніи святомъ, что страшный судъ Божій на неправедныхъ судей есть. \textit{Проклятъ, иже уклонитъ судъ нищему и вдовѣ и сиротѣ!} И паки: \textit{проклятъ, иже возметъ дары поразити душу крови неповинныя}\footnote{Второз.~27,~19 и 25.}! \textit{Горе пишущимъ лукавство! Пишущіе бо, лукавство пишутъ, уклоняюще судъ убогихъ, восхищающе судъ нищихъ людей Моихъ, яко быти имъ вдовицѣ въ расхищеніе и сиротѣ въ разграбленіе}\footnote{Ис.~10,~1 и 2.}. \textit{Утыша, утолстѣша и преступиша словеса Моя злѣйшіи; прю вдовицы не судиша, суда сира не управиша, и суда убогимъ не судиша. Еда сихъ ради не посѣщу, рече Господь, или языку сицевому не отмститъ душа Моя}\footnote{Іер.~5,~28 и 29.}. \textit{Иже} (Вышній) \textit{истяжетъ дѣла ваша и помышленія испытаетъ: яко слузи суще царства Его, не судисте право, ни сохранисте закона, ниже по воли Божіей ходисте. Страшно и скоро явится вамъ! яко судъ жесточайшій преимущимъ бываетъ. Ибо малый достоинъ есть милости: сильніи же сильнѣ истязани будутъ}, по иному же переводу, \textit{сильно мучены будутъ}, глаголетъ Соломонъ\footnote{Премудр.~6,~3--6.}. Отъ сихъ и прочіихъ Писанія святаго мѣстъ видно, что судія чимъ болѣе неправды дѣлаетъ, тѣмъ болѣе погибели себѣ собираетъ. Отвсюду тѣсно будетъ неправедному судіи, когда станетъ на судѣ Хрістовомъ. Обличитъ его совѣсть, общее всѣхъ человѣковъ правило и законъ, что противу ея поступалъ; обличитъ обѣщаніе, при крещеніи учиненное, но оставленное; обличитъ слово Божіе, столько разъ слышанное, но презрѣнное; обличитъ присяга, съ клятвою и призываніемъ всевѣдца Бога во свидѣтельство правосудія его сотворенная, но попранная; обличатъ слезы вдовъ, сиротъ и прочіихъ бѣдныхъ пролитыя, но ради мздоиманія ничего не успѣвшія; явятся дѣла, слова и помышленія, неправедно, хитро и лукаво умышленныя и сотворенныя. Столько свидѣтелей неправды его предстанетъ, сколько онъ неправедно и противу совѣсти дѣлъ производилъ и рѣшилъ. Отсюду всемірный стыдъ и отъ праведнаго Судіи столько гнѣва, сколько неправды сдѣлалъ, почувствуетъ. Тогда познаетъ, что хотя и много въ мірѣ собиралъ, но ничего отъ міра не вынесъ, кромѣ превеликой пагубы; хотя и богатымъ, знаменитымъ и славнымъ почитался здѣ, но явился нищимъ, убогимъ, подлымъ и безчестнымъ, и окаяннѣйшимъ паче прочіихъ. \textit{Сильніи сильно мучены будутъ}, глаголетъ Божіе слово, истинное и непремѣняемое\footnote{ст.~6.}. Разсуждай сіе, судія, и неправды паче огня берегися.

\paragraph*{§\:465.} Къ суду часто бываетъ призываніе свидѣтелей ради лучшаго истины сысканія и доказательства. И понеже на свидѣтельствѣ свидѣтелей все дѣло утверждается, и по ихъ, яко слышавшихъ или видѣвшихъ, словамъ и показаніямъ дѣло рѣшится, и единъ правымъ, другій виноватымъ изъ судимыхъ опредѣляется: того ради и свидѣтелямъ въ семъ дѣлѣ должно вѣрными быть, не уклоняться ни на тую, ни на другую сторону, не знать ни дружества, ни вражды, ни любви, ни злобы судимыхъ, ни страха человѣческаго не бояться, и руку отъ пріятія даровъ отвращать. Едина истина да будетъ имъ другиня, которая, что добро видитъ во врагѣ, похваляетъ, и что въ другѣ видитъ злое, свидѣтельствуетъ; и она такъ, какъ и правда, лица человѣческаго не знаетъ, но показуетъ тое, что видитъ и слышитъ, и къ той сторонѣ склоняется, къ которой правость находитъ, и отъ той отвращается, въ которой противную себѣ неправду видитъ. Сего ради свидѣтелямъ должно истины держаться и противу лжи подвизаться, якоже повелѣваетъ Господь: \textit{не послушествуй на друга твоего свидѣтельства ложна}\footnote{Исх.~20,~16.}, да не како лжесвидѣтельствомъ своимъ неповинныхъ въ большую бѣду, виновныхъ въ большую продерзость и своевольство, судящихъ въ грѣхъ приведутъ и себе отъ хрістіанскаго общества, слѣдственно и отъ Хріста Самаго отлучатъ. Все бо сіе отъ ложнаго свидѣтельства слѣдуетъ. Ибо на свидѣтеляхъ все судебнаго дѣла основаніе полагается.

\paragraph*{§\:466.} О должности подчиненныхъ судіямъ смотри въ своемъ мѣстѣ. А хрістіане, которые хотятъ на обидѣвшихъ суду доносить, и обиду свою судомъ отвращать, да внимаютъ нижеписаннымъ святаго Писанія мѣстамъ\footnote{Исх.~20,~16; Матѳ.~5,~39,~40,~44 и слѣд; 6,~14 и 15; 7,~12--14; Марк.~8,~36,~38; Іак.~5,~7--11; Римл.~12,~17 и слѣд.; 1~Кор.~6,~7 и 8; Евр.~12 и проч.}.

\subsection[Глава 4-я. О должности пастырей и подчиненныхъ имъ.]{глава четвертая.\\\bfseries О должности пастырей и подчиненныхъ имъ.}

\begin{quotation}\textit{Старцы, иже въ васъ, молю, яко старецъ сый и свидѣтель Хрістовымъ страстемъ, иже и хотящей славѣ явитися общникъ, пасите еже въ васъ стадо Божіе, посѣщающе не нуждею, но волею и по Бозѣ; ниже неправедными прибытки, но усердно, ни яко обладающе причту, но образи бывайте стаду: и явльшуся Пастыреначальнику пріимете неувядаемый славы вѣнецъ}\footnote{1~Петр.~5,~1--4.}.\end{quotation}
\begin{quotation}\textit{Внимайте себѣ и всему стаду, въ немже васъ Духъ Святый постави епископы, пасти церковь Господа и Бога, юже стяжа кровію Своею}\footnote{Дѣян.~20,~28.}.\end{quotation}

\paragraph*{§\:467.} Всякій хрістіанинъ, который ни имѣетъ себѣ подчиненныхъ, пастырь имъ быть долженъ: хозяинъ домашнимъ своимъ, отецъ дѣтямъ и прочіимъ домашнимъ, господинъ рабамъ и крестьянамъ своимъ, настоятель монастырскій братіи своей, начальникъ подчиненнымъ своимъ пастырь есть. Ибо вси сіи по должности хрістіанской обязуются подчиненныхъ на путь спасенія наставлять, и себе во образъ благочестиваго житія подавать, и имѣютъ отвѣтъ дать за нихъ въ день суда Хрістова. Однакожъ по общему хрістіанскому гласу епископы наипаче и пресвитеры нарицаются пастыри, и Божіе слово имъ титулъ сей приписуетъ, и поручаетъ въ руки ихъ стадо словесныхъ Хрістовыхъ овецъ, кровію Хрістовою купленныхъ.

\paragraph*{§\:468.} Самое имя \textit{пастырь} показуетъ, что есть пастырь, и пастырямъ званіе и должность ихъ представляетъ предъ глазами ихъ, и увѣщаваетъ и доказываетъ, что они избраны отъ людей, Хрістовымъ именемъ нарицающихся, яко лучшіи, разумнѣйшіи и совершеннѣйшіи, и поставлены пасти стадо Хрістово, кровію Его стяжанное, какъ учитъ апостолъ. Пастухъ безсловесныхъ животныхъ на то избирается, чтобы стадо изгонялъ и пригонялъ цѣло, пасъ на пажити и сохранялъ отъ звѣрей: тако пастырь словесныхъ овецъ Хрістовыхъ на то избирается, дабы стадо тое пасъ и питалъ Божіимъ словомъ, и сохранялъ отъ волковъ, то"=есть, невидимыхъ демоновъ, окружающихъ стадо Хрістово, и хотящихъ \textit{поглотити кого} отъ того стада\footnote{1~Петр.~5,~8.}. Отсюда всякій пастырь можетъ видѣть, каковъ долженъ быть. То есть, какъ разумомъ, такъ и житіемъ добрымъ совершеннѣйшій паче прочіихъ хрістіанъ; онъ долженъ тое на себѣ показывать дѣломъ, что Павелъ святый написалъ хрістіанамъ о себѣ: \textit{подражатели мнѣ бывайте, якоже и азъ Хрісту}\footnote{1~Кор.~11,~1.}.

\paragraph*{§\:469.} Безъ званія и избранія правильнаго никто не долженъ въ пастырское служеніе вступать, якоже апостолъ глаголетъ: \textit{никтоже самъ о себѣ пріемлетъ честь, но званный отъ Бога, якоже и Ааронъ: тако и Хрістосъ не Себе прослави быти Первосвященника, но глаголавый къ Нему: Сынъ мой еси Ты, Азъ днесь родихъ Тя}\footnote{Евр.~5,~4 и 5.}. И святый Предтеча поучаетъ: \textit{не можетъ человѣкъ пріимати ничесоже, аще не будетъ дано ему съ небесе}\footnote{Іоан.~3,~27.}. Якоже убо пророки и апостоли безпосредственно гласомъ Божіимъ избраны и позваны въ великое служенія ихъ дѣло, какъ читаемъ въ книгахъ ихъ: тако хотящимъ вступити въ дѣло ихъ надобно ожидать званія и избранія церкве и первосѣдателей церковныхъ, которымъ поручено церковное правленіе: а безъ того не должно никому въ великое тое дѣло мѣшаться. Премудро Сирахъ увѣщаваетъ: \textit{высшихъ себе не ищи}\footnote{3,~21.}. Слѣдственно худо дѣлаютъ, которые безъ таковаго избранія вступаютъ въ тое; хуже того, которые чрезъ мзду или ходатайство вельможъ и князей того домогаются. Не знаютъ таковіи, чего ищутъ, какъ ниже увидишь. Тѣмъ бо показываютъ, что или мнятъ себе довольныхъ быть къ такъ великому и трудному дѣлу, то"=есть стадо Хрістово пасти и хранити отъ діавольскихъ ловленій, что есть высокоуміе, "--- или пастися паче, нежели пасти, хотятъ. Надобно имъ опасаться, чтобы имъ не приличествовало Хрістово оное слово: \textit{не входяй дверми во дворъ овчій, но прелазя инудѣ, той тать есть и разбойникъ: тать не приходитъ, развѣ да украдетъ и убіетъ и погубитъ}\footnote{Іоан.~10,~1 и 10.}.

\paragraph*{§\:470.} Пастырей избирать подобаетъ: 1)~Разумныхъ и учительныхъ, да возмогутъ наставлять на путь спасенія, вразумлять, исправлять и утѣшать подчиненныхъ себѣ людей. 2)~Добродѣтельныхъ, да чему учатъ людей, тое сами на себѣ дѣломъ показываютъ. Тако ученіе ихъ, добродѣтельнымъ ихъ житіемъ подтвержденное, сильно, дѣйствительно и полезно будетъ людямъ. Сихъ въ пастырѣ свойствъ требуетъ апостолъ: \textit{подобаетъ епископу быти непорочну, единыя жены мужу, трезвенну, цѣломудру, благоговѣйну, честну, страннолюбиву, учительну, не піяницѣ, не бійцѣ, не сварливу, не мшелоимцу, но кротку, не завистливу, не сребролюбцу, свой домъ добрѣ правящу, чада имущу въ послушаніи со всякою честностію}\footnote{1~Тим.~3,~2--4.}. И на другомъ мѣстѣ: \textit{подобаетъ епископу быти безъ порока, якоже Божію строителю: не себѣ угождающу, не дерзу, не напрасливу, не гнѣвливу, не піяницѣ, не бійцѣ, не скверностяжательну; но страннолюбиву, благолюбцу, цѣломудренну, праведну, преподобну, воздержательну, держащемуся вѣрнаго словесе по ученію, да силенъ будетъ и утѣшати въ здравомъ ученіи, и противящіяся обличати}\footnote{Тит.~1,~7--9.}. Сіи свойства пастырскія изъ слѣдующей статьи лучше познаются. Примѣчай здѣ, читатель, что во времена апостольскія епископы пресвитерами, и пресвитеры епископами называлися. Епископъ бо есть имя греческое и значить тое, что по нашему нарѣчію \textit{надзиратель}, и пресвитеръ такожде есть имя греческое, и значитъ то что \textit{старшій} или \textit{старецъ}. Отъ сего видишь, что и званіе ихъ общее есть. Надзирателемъ бо быть есть надъ другими быть надзирателемъ, и старшимъ такожде надъ другими старшимъ. Видимъ сіе общее ихъ именованіе въ Писаніяхъ Апостольскихъ. Павелъ святый, пиша къ Филипписіямъ, глаголетъ: \textit{сущимъ въ Филиппѣхъ со епископы и діаконы}\footnote{ст.~1.}. Видишь, что здѣ епископами пресвитеровъ называетъ потому, что градъ Филиппы единъ есть градъ, въ которомъ единъ обыкновенно поставлялся епископъ, а не многіи; а онъ пишетъ ко многимъ, а не единому: убо многихъ въ единомъ градѣ пресвитеровъ епископами называетъ. Паки читаемъ въ Дѣяніяхъ Апостольскихъ, что тойжде апостолъ, отъ Милита пославъ во Ефесъ, \textit{призва пресвѵтеры церковныи}, которымъ между прочіими глаголетъ: \textit{внимайте себѣ и всему стаду, въ немже васъ Духъ Святый постави епископы пасти церковь Господа и Бога}\footnote{Дѣян.~20,~17 и 28.}. Вотъ и здѣ пресвитеры нарицаются епископами. Паки, когда Титу святому повелѣлъ, чтобы по градамъ устроилъ пресвитеры, тотчасъ, каковыми подобаетъ быть тѣмъ пресвитерамъ, приложилъ: \textit{подобаетъ епископу быти безъ порока}\footnote{Тит.~1,~7.}. И здѣ пресвитера называетъ епископомъ. О семъ и Златоустъ святый повѣствуетъ, что тогда пресвитеры епископами называлися. «Что сіе! единаго ли града мнози епископы быша? Никако; но пресвитеры тако нарицаше». И ниже: «И пресвитерами древле нарицахуся епископы, и служителями Хрістовыми и епископами пресвитеры»\footnote{Бес.~1"~я на посл. къ Филип. на оныя слова апостольскія: \textit{сущимъ въ Филиппѣхъ со епископы и діаконы}.}. Видимъ убо, что во дни апостольскіе епископъ и пресвитеръ общее было имя; и на тое кромѣ прочіихъ причинъ наипаче поставлялися, чтобъ пасти церковь Хрістову. Но въ слѣдующія времена и нынѣ епископы только епископами и пресвитеры такожде пресвитерами называются, и что они суть, и какое обоихъ званіе, всѣмъ хрістіанамъ извѣстно. Дѣло же пастырское есть попеченіе, тщаніе и прилѣжаніе о стадѣ Хрістовыхъ овецъ, изъ самыхъ именъ ихъ, какъ сказано, познается быть равно: и тіи и другіи равно пасти церковь Хрістову должни.

\paragraph*{§\:471.} \textit{Должность пастырей} есть: 1)~Внимать имъ самимъ должно чтенію священныхъ Писаній\textit{, могущихъ ихъ умудрити во спасеніе вѣрою, яже о Хрістѣ Іисусѣ}\footnote{2~Тим.~3,~15.}. Къ сему увѣщаваетъ пастырей апостолъ святый въ лицѣ Тимоѳея святаго: \textit{внемли чтенію, утѣшенію, ученію}, "--- и мало спустя: \textit{внимай себѣ и ученію, и пребывай въ нихъ: сія бо творя, и самъ спасешися, и послушающіи тебе}\footnote{1~Тим.~4,~13 и 16.}. \textit{Всяко бо писаніе Богодухновенно и полезно есть ко ученію, ко обличенію, ко исправленію, къ наказанію, еже въ правдѣ, да совершенъ будетъ Божій человѣкъ, на всякое дѣло благое уготованъ}\footnote{2,~3,~16 и 17.}. Апостоли и пророки святіи безпосредственно \textit{просвѣщаеми были Духомъ Святымъ}\footnote{2~Петр.~1,~21.}, и учили людей, и ученіе свое святое написали въ пользу церкве. Пастырямъ, въ должность ихъ вступившимъ, надобно ихъ писанія держаться, читать, поучаться въ немъ день и нощь, и изъ того почерпать ученіе, и людямъ своимъ предлагать, и тѣмъ то учить, то обличать, то исправлять, то утѣшать ихъ, и молиться Хрісту Богу, \textit{дабы отверзлъ имъ умъ разумѣти Писанія}\footnote{Лук.~24,~45.}. Безъ того бо разумъ и сила Писаній не постигается, и отъ чтенія никакой пользы не будетъ, хотя бы кто и все Писаніе наизусть зналъ. Слѣдовательно худо дѣлаютъ тіи пастыри, которыи оставляютъ чтеніе и вниманіе божественнаго Писанія, и время провождаютъ въ иныхъ дѣлахъ, или къ званію ихъ не надлежащихъ, или, что горше того, противныхъ. Таковымъ приличествуетъ оное Хрістово слово: \textit{вожди суть слѣпи слѣпцемъ; слѣпецъ же слѣпца аще водитъ, оба въ яму впадутъ}\footnote{Матѳ.~15,~14.}. "--- 2)~Обдолжаются тщательно поучать людей порученныхъ себѣ, якоже увѣщаваетъ ихъ Петръ святый: \textit{пасите еже въ васъ стадо Божіе}\footnote{1~Петр.~5,~2.}. Пасется же стадо Божіе словомъ Божіимъ и святыми тайнами Его. И Павелъ святый къ томужде ихъ возбуждаетъ: \textit{внимайте себѣ и всему стаду, въ немже васъ Духъ Святый постави епископы пасти церковь Господа и Бога, юже стяжа кровію Своею}\footnote{Дѣян.~20,~28.}. И паки къ Тимоѳею написалъ, и въ лицѣ его ко святому пастырю: \textit{проповѣдуй слово, настой благовременнѣ и безвременнѣ, обличи, запрети, умоли со всякимъ долготерпѣніемъ и ученіемъ}\footnote{2~Тим.~4,~2.}. Пастыри бо вступили въ должность апостольскую, и неотмѣнно должны имъ послѣдовать. Они по всей вселенной проповѣдывали слово Божіе; а пастыри въ своихъ мѣстахъ, гдѣ учреждены пастырями, должны тое сокровище духовное открывать людямъ. Убо погрѣшаютъ пастыри, которые сію должность оставляютъ. Таковыи пастыри уподобляются псамъ нѣмымъ: \textit{вси пси нѣміи не возмогутъ лаяти}\footnote{Ис.~56,~10.}. Грозитъ имъ чрезъ пророка Господь: \textit{се млеко ядите, и волною одѣваетеся, и тучное закалаете, а овецъ Моихъ не пасете: изнемогшаго не подъясте, и болящаго не уврачевасте, и сокрушеннаго не обязасте, и заблуждающаго не обратисте, и погибшаго не взыскасте}, и проч. И далѣе глаголетъ Господь\textit{: се Азъ на пастыри, и взыщу овецъ Моихъ отъ рукъ ихъ}\footnote{Іезек.~34,~3--5 и 10.}. Взыщетъ Господь отъ таковыхъ пастырей овецъ своихъ, егда пріидетъ судити міру. Хозяинъ у пастуха овецъ своихъ спрашиваетъ и взыскуетъ, когда не пригонитъ ихъ, и истязуетъ его за нихъ: тако взыщетъ Хрістосъ Господь словесныхъ Своихъ овецъ отъ пастырей, овецъ, которыхъ \textit{не сребромъ или златомъ, но честною Своею кровію стяжалъ}\footnote{1~Петр.~1,~19; Іоан.~10,~15.}. Горе тогда будетъ пастырю нерадивому, пастырю погубившему овцы Хрістовы, которыя такъ дорогою цѣною куплены! Сего ради надобно таковымъ пастырямъ осмотрѣться и исправиться, да не и овецъ Хрістовыхъ, порученныхъ себѣ погубятъ, и сами страшному суду Его и истязанію подпадутъ. Помнить имъ нужно, что избираются пасти стадо Божіе, а не пастися. "--- 3)~Тотъ есть пастырь добрый, который учитъ, и чему учитъ, тое дѣломъ на себѣ показываетъ. Сего ради пастырямъ не токмо должно учить, но и образъ добрыхъ дѣлъ на себѣ показывать. Имъ Пастыреначальникъ Хрістосъ глаголетъ: \textit{тако да просвѣтится свѣтъ вашъ предъ человѣки, яко да видятъ ваша добрая дѣла, и прославятъ Отца вашего, Иже есть на небесѣхъ}\footnote{Матѳ.~5,~16.}. И апостолъ святый: \textit{образъ буди вѣрнымъ словомъ, житіемъ, любовію, духомъ, вѣрою, чистотою}\footnote{1~Тим.~4,~12.}. И на другомъ мѣстѣ: \textit{о всемъ самъ себе подавай образъ добрыхъ дѣлъ}\footnote{Тит.~2,~7.}. Пастырю убо нужно имѣть предъ Богомъ чистую и незазрительную совѣсть, и предъ людьми добрыхъ дѣлъ примѣръ, дабы совѣсть его самого въ томъ не обличала, что въ людяхъ обличаетъ, и въ томъ ихъ оправдала, чему другихъ научаетъ. Тогда пастырь свободно будетъ учить, когда ученію его совѣсть его согласуетъ. Слѣдовательно пастыри учащіи, но не творящіи, мало что успѣютъ: понеже чему учатъ словомъ, тое житіемъ своимъ разоряютъ; истину, которую словомъ проповѣдуютъ, примѣромъ своимъ въ сумнѣніе и подозрѣніе приводятъ. Таковыи пастыри услышатъ слово: \textit{врачу, исцѣлися самъ}\footnote{Лук.~4,~23.}. \textit{Научая инаго, себе ли не учиши? проповѣдая не красти, крадеши; глаголяй "--- не прелюбы творити, прелюбы твориши; гнушаяся идоловъ, святая крадеши; иже въ законѣ хвалишися, преступленіемъ закона Бога безчествуеши}\footnote{Римл.~2,~21--23.}. Къ таковымъ пастырямъ глаголетъ Богъ: \textit{вскую ты повѣдаеши оправданія Моя, и воспріемлеши завѣтъ Мой усты твоими? Ты же возненавидѣлъ еси наказаніе, и отверглъ еси словеса Моя вспять. Аще видѣлъ еси татя, теклъ еси съ нимъ, и съ прелюбодѣемъ участіе твое полагалъ еси. Уста твоя умножиша злобу, и языкъ твой сплеташе льщенія. Сѣдя на брата твоего клеветалъ еси, и на сына матери твоея полагалъ еси соблазнъ}\footnote{Пс.~49,~16--20.}. Прилично таковіи пастыри уподобляются столпамъ, на дорогѣ поставленнымъ, которые путь указуютъ во градъ, но сами съ мѣста не двигаются; уподобляются колоколамъ, которые людей въ церковь созываютъ, но сами не входятъ. Напротивъ того, пастырь добрый подобенъ есть вождю, который и путь указуетъ людямъ, и самъ напредъ идетъ, якоже о начальникѣ пастырей, Спасителѣ нашемъ, написалъ Лука святый: \textit{Іисусъ начатъ творити же и учити}\footnote{Дѣян.~1,~1.}. Двояко учитъ, кто учитъ и живетъ благочестиво: словомъ и житіемъ своимъ учитъ, яко ученіе свое подтверждаетъ дѣломъ и житіемъ своимъ. "--- 4)~Пастыри не должны ласкательствовать или угождать людямъ, что наипаче бываетъ предъ вельможами, князьями, славными и богатыми, "--- но истину и правду вездѣ свидѣтельствовать. Подаетъ имъ во образъ себе апостолъ Хрістовъ: \textit{никогдаже въ словеси ласканія быхомъ къ вамъ, якоже вѣсте}\footnote{1~Солун.~2,~5.}. И не хотѣлъ никому угождать: \textit{аще быхъ еще человѣкомъ угождалъ, Хрістовъ рабъ не быхъ убо былъ}\footnote{Гал.~1,~10.}. Сему примѣру подражать обдолжаются пастыри, когда хотятъ быть раби Хрістовы, и не человѣкамъ, но Богу, Котораго служители суть, угождать да потщатся; что порочно, неправедно и беззаконно, вездѣ и предъ всякимъ лицемъ обличать; не молчать, гдѣ должно говорить; не хвалить, что непохвально, и не порочить того, что достойно похвалы. Ласкательство бо не иное что дѣлаетъ, какъ пагубу душевную и имъ самимъ, и кому ласкательствуютъ. Тако бо не отвращаютъ отъ грѣховъ, но утверждаютъ во грѣхахъ, отъ которыхъ должны отвращать, и не Богу, но человѣкамъ угождаютъ: и потому противу званія своего дѣлаютъ таковіи. Сего ради пастырямъ должно всегда и вездѣ истину свидѣтельствовать, и, аще нужда будетъ, за истину все претерпѣть, что ни вымыслитъ и учинитъ имъ міръ, ненавидящій истины. Тако за истину подвизалися пророки святіи, апостоли и святители Хрістови, подвизалися до крове "--- и Самъ Хрістосъ, пророками и апостолами проповѣданный, \textit{свидѣтельствовавшій при Понтійстѣмъ Пилатѣ доброе исповѣданіе}\footnote{1~Тим.~6,~13.}. На сей образъ и нынѣшніи пастыри да взираютъ, и истину да свидѣтельствуютъ со всякимъ дерзновеніемъ, памятуя слово Хрістово: \textit{не убойтеся отъ убивающихъ тѣло, душу же немогущихъ убити, убойтеся же паче Могущаго и душу и тѣло погубити въ гееннѣ}\footnote{Матѳ.~10,~28.}. И паки: \textit{всякъ, иже исповѣсть Мя предъ человѣки, исповѣмъ его и Азъ предъ Отцемъ Моимъ, Иже на небесѣхъ. А иже отвержется Мене предъ человѣки, отвергуся его и Азъ предъ Отцемъ Моимъ, Иже на небесѣхъ}\footnote{ст.~32 и 33.}. "--- 5)~Пастыри долгъ имѣютъ разумно въ своемъ ученіи поступать, смотрѣть и разсуждать, гдѣ \textit{евангельское} утѣшительное, гдѣ \textit{законное} обличительное и устрашительное ученіе предлагать. Худо и безполезно, паче же и вредно тамо предлагать утѣшительное слово, гдѣ требуется страхъ, и тамо гремѣть, гдѣ утѣшеніе нужно. Въ народѣ двоякіи люди имѣются, нерадивыи и безстрашно живущіи, и страхомъ сокрушенніи и печальніи. Ради \textit{безстрашныхъ} страхъ Божій нуженъ, да, отъ сна грѣховнаго пробудившеся, покаются: ради \textit{сокрушенныхъ} и печальныхъ пластырь утѣшенія евангельскаго потребенъ, да въ вѣрѣ утвердятся, и живое въ сердцахъ своихъ воспріимутъ утѣшеніе. Сего ради пастыри тако должни ученіе предлагать, дабы всякъ отъ ученія ихъ приличное себѣ сыскалъ лѣкарство, и, тое къ язвамъ своимъ прилагая, исцѣлялся, что видимъ въ писаніяхъ пророческихъ и апостольскихъ и ихъ послѣдователяхъ святыхъ отцахъ, которымъ послѣдовать и нынѣшніи пастыри обдолжаются. "--- 6)~Пастыри часто должны ученіе сіе повторять, дабы у иныхъ страхъ Божій, у иныхъ вѣра и отъ ней утѣшеніе истинное въ сердцахъ насадилося и утвердилося. Павелъ святый усмотрѣлъ пользу сію и написалъ: \textit{таяжде писати вамъ, мнѣ убо нелѣностно, вамъ же твердо}\footnote{Фил.~3,~1.}. Въ книгѣ Бытія читаемъ, что и Самъ Богъ толико разъ повторялъ Аврааму милостивое Свое обѣщаніе. Тоежде видимъ и въ святыхъ учителяхъ церковныхъ, которые неусыпно порученныхъ себѣ людей учили, наставляли, увѣщавали, устрашали и утѣшали. Непостоянно бо сердце человѣческое и удобопреклонно есть, и скоро совращается съ истиннаго пути. Сего ради всегдашняго требуетъ напоминанія, увѣщанія, укрѣпленія, наставленія и поощренія. "--- 7)~Пастырямъ, понеже безъ помощи Божіей ничего успѣть не могутъ, должно усердно Богу молиться какъ о себѣ, чтобы помоглъ имъ въ семъ немалотрудномъ дѣлѣ, такъ и о людяхъ, себѣ порученныхъ, дабы просвѣтилъ душевныя ихъ очи къ познанію истины, и управилъ сердца ихъ къ творенію воли Своея, и сохранилъ ихъ отъ козней діавольскихъ. Творили сіе апостоли святіи, какъ читаемъ въ Посланіяхъ ихъ; творили пастыри, преемники ихъ и учители церковныи. Тоежде и нынѣшнимъ пастырямъ творить должно. "--- 8)~Хотя и вси хрістіане всякому гоненію подвержены, яко по словеси апостольскому, \textit{вси, хотящіи благочестно жити о Хрістѣ Іисусѣ, гоними будутъ}\footnote{2~Тим.~3,~12.}, но наипаче пастыри тому подлежатъ. Причина тому сія есть, что они царство діавольское проповѣдію слова Божія разрушаютъ, прелесть его и тьму открываютъ, души человѣческія отъ рукъ его восхищаютъ и приводятъ въ царство Хрістово: того ради діаволъ на нихъ паче ярится и возстаетъ, и всякое гоненіе возставляетъ чрезъ злыхъ людей, яко свое истое орудіе. Сего ради добрымъ пастырямъ надобно къ терпѣнію пріуготовляться, и всегда, какъ кораблю на морѣ бури бѣдствія находящаго ожидать, и нашедшее терпѣливымъ сердцемъ нести. Да будетъ имъ во образъ Самъ Пастыре"=Начальникъ Іисусъ, Который отъ Своихъ людей претерпѣлъ крестъ, "--- и святіи апостоли, послѣдователи Его, которые \textit{укоряеми были, гоними были, хулими были, яко отреби міру были, всѣмъ попраніе были}\footnote{1~Кор.~4,~12 и 13.}, "--- и святители Хрістовы преждебывшіи, которые столько укореній, клеветъ, озлобленій, гоненій и заключеній претерпѣли отъ неблагодарныхъ людей, которыхъ учили истинѣ и на путь спасенія наставить тщалися. Къ терпѣнію да подвигнетъ ихъ и въ томъ да утвердитъ многая мзда сокровенная имъ на небесѣхъ. \textit{Радуйтеся}, глаголетъ имъ Хрістосъ, \textit{и веселитеся, яко мзда ваша многа на небесѣхъ}\footnote{Матѳ.~5,~12.}. "--- 9)~Какъ учить и чему, и какъ съ своими людьми поступать, должно имъ учитися изъ книгъ пророческихъ, апостольскихъ и богоносныхъ отецъ. И отъ сего видно, какъ нужно имъ чтеніе святаго Писанія и прочіихъ хрістіанскихъ книгъ. "--- 10)~Какъ тщательными быть въ своемъ званіи пастырямъ должно, и какъ опасно держаться Божія слова, и тѣмъ людямъ наставлять, и себѣ и всему стаду внимать, отъ Павловыхъ увѣщаній могутъ примѣтить: \textit{внимайте}, рече, \textit{себѣ и всему стаду, въ немже васъ Духъ Святый постави епископы пасти церковь Господа и Бога, юже стяжа кровію Своею}\footnote{Дѣян.~20,~28.}. Какъ бы сказалъ: поставлены вы не отъ человѣка, но отъ Бога "--- Духа Святаго, пасти не стадо безсловесное, которое видиміи окружаютъ звѣри, но пасти стадо овецъ Хрістовыхъ, которыхъ окружаютъ невидиміи звѣри, діаволъ, яко левъ, и злобніи духи его; и сихъ овецъ, которыхъ пасете, стяжалъ Хрістосъ, Господь и Богъ, не сребромъ и златомъ, но кровію Своею: берегите убо и себе и стадо сіе, вамъ въ сохраненіе порученное отъ Бога. Такожде къ Тимоѳею святому въ двухъ посланіяхъ написалъ, и къ тщанію, усердію и прилѣжанію о себѣ и о людяхъ поощрялъ его, и въ лицѣ его всякаго пастыря: \textit{засвидѣтельствую предъ Богомъ, и Господемъ Іисусъ Хрістомъ, и избранными Его ангелы, да сія сохраниши безъ лицемѣрія, ничесоже творя по уклоненію}\footnote{1~Тим.~5,~21.}. И паки: \textit{завѣщаваю тебѣ предъ Богомъ оживляющимъ всяческая, и Хрістомъ Іисусомъ, свидѣтельствовавшимъ при Понтійстѣмъ Пилатѣ доброе исповѣданіе: соблюсти тебѣ заповѣдь нескверну и незазорну, даже до явленія Господа нашего Іисуса Хріста}\footnote{1~Тим.~6,~13 и 14.}. И паки взываетъ: \textit{о Тимоѳее! преданіе сохрани}\footnote{ст.~20.}. И паки: \textit{засвидѣтельствую азъ предъ Богомъ, и Господемъ нашимъ Іисусомъ Хрістомъ, хотящимъ судити живымъ и мертвымъ въ явленіи Его и царствіи Его: проповѣдуй слово, настой благовременнѣ и безвременнѣ, обличи, запрети, умоли со всякимъ долготерпѣніемъ и ученіемъ}\footnote{2~Тим.~4,~1 и 2.}. Видимъ отъ сихъ святаго апостола словъ къ Тимоѳею святому написанныхъ, какимъ желаніемъ, усердіемъ и ревностію горѣло святое тое сердце, дабы Тимоѳей святый усердно и бодренно паслъ стадо Хрістово, и по ученію отъ него преданному поступалъ; Бога Самаго въ свидѣтеля призываетъ, Который все знаетъ, и сердца человѣческая испытуетъ, "--- и Господа Іисуса Хріста, къ Которому на судъ слѣдуетъ ему явитися, "--- и святыхъ ангеловъ, съ которыми Судія оный пріидетъ судити всему міру. \textit{Засвидѣтельствую}, рече, \textit{предъ Богомъ, и Господемъ Іисусъ Хрістомъ, и избранными Его ангелы}. Въ чемъ? \textit{Да сія}, то"=есть, преданныя заповѣди, \textit{сохраниши безъ лицемѣрія, ничесоже творя по уклоненію}. И взываетъ: \textit{О Тимоѳее! преданіе сохрани}. Реченныя къ Тимоѳею святому слова апостольскія касаются и всѣхъ пастырей, увѣщаваютъ ихъ въ званіи семъ святомъ свято и непорочно пребывати, и преданіе его хранити, \textit{проповѣдати слово, настояти благовременнѣ и безвременнѣ, обличати, запрещати, умоляти со всякимъ долготерпѣніемъ и ученіемъ, и во образъ быти вѣрнымъ словомъ, житіемъ, любовію, духомъ, вѣрою, чистотою}. Слѣдуетъ имъ со всѣми овцами, имъ отъ Хріста порученными, явиться на судѣ Его, и о всѣхъ ихъ дати Ему отвѣтъ. Блаженъ есть, кто скажетъ тогда: \textit{се Азъ, и дѣти, яже далъ ми еси, Господи}\footnote{Евр.~2,~13.}! Окаяненъ и бѣденъ, кто не можетъ того сказать тогда, но, стыдомъ и страхомъ объятый, услышитъ: \textit{неключимаго раба вверзите во тьму кромѣшнюю: ту будетъ плачь и скрежетъ зубомъ}\footnote{Матѳ.~25,~30.}. Въ Апокалипсисѣ читаемъ, что Хрістосъ всякому ангелу, (епископу) седми церквей глаголетъ: \textit{вѣмъ твоя дѣла}\footnote{Апок.~2,~2,~9,~13 и 19; 3,~1,~8 и 15.}. Что сіе значитъ? Не иное что, какъ что всякаго пастыря назираетъ Хрістосъ, и дѣла его, замыслы и начинанія примѣчаетъ, и въ книзѣ Своей записуетъ, и всякому воздаетъ по дѣламъ его, какъ таможде видимъ, хотя то и о всякомъ человѣкѣ такоежде всевѣдѣніе имѣетъ Онъ. Аще убо непостыдны быть хотятъ пастыри въ онъ день, да разсуждаютъ, что Хрістосъ на вся ихъ дѣла смотритъ; и потому какъ предъ очесами Его да обращаются, и внимаютъ себѣ и всему стаду своему. Паки читаемъ у Іоанна Евангелиста, что Хрістосъ, когда по востаніи отъ мертвыхъ вопрошалъ Петра троекратно: \textit{Симоне Іонинъ! любиши ли Мя?} и Петръ отвѣтствовалъ: \textit{ей Господи! Ты вѣси, яко люблю Тя}: сказалъ ему: \textit{паси овцы Моя}\footnote{Іоан.~21,~15--17.}. Отъ сего видимъ, что пастыри, когда хотятъ любовь показать ко Хрісту, должны показать тую дѣломъ "--- \textit{пасти овцы Его}. Хрістосъ бо овецъ Своихъ любитъ: яко ихъ стяжалъ кровію Своею. Аще убо кто Хріста любить хощетъ, какъ и всякъ долженъ, долженъ любить овецъ Его, и пасти ихъ, и представить ихъ Ему цѣлыхъ, не нерадѣть о нихъ, якоже наемники дѣлаютъ.

\paragraph*{§\:472.} \textit{Должность людей къ пастырямъ своимъ}: 1)~Повиноватися имъ и ученія ихъ предлагаемаго слушать, и по тому житіе свое исправлять, якоже увѣщаваетъ апостолъ: \textit{повинуйтеся наставникомъ вашимъ, и покаряйтеся; тіи бо бдятъ о душахъ вашихъ, яко слово воздати хотяще: да съ радостію сіе творятъ, а не воздыхающе}\footnote{Евр.~13,~17.}. Сѣмя того ради на землѣ сѣется, дабы плодъ приносило земледѣльцу: тако сѣмя Божія слова того ради сѣется на нивахъ сердецъ, дабы прорастило плодъ себѣ приличный "--- хрістіанское и благочестивое житіе. Иначе напрасно и сѣется, когда безплодно бываетъ. Сего ради людямъ должно ученію пастырскому внимать, и углублять тое въ сердцахъ своихъ, и плодъ показывать, да пастырь, видя плодъ ученія своего, возрадуется и охотнѣйшій будетъ къ спасительному своему труду. "--- 2)~Какъ ученіе свое предлагаетъ пастырь, красно или некрасно, равно тое пріимать должно, и пріимать съ радостію, яко слово Божіе, а не человѣческое, только бы ученіе его было согласно святому Писанію. Пастырская бо должность есть учить и научать, а не слухи услаждать: такъ и людей должность есть учиться и научаться, а не услаждаться и безплодными быть. Сего ради хрістіанамъ, желающимъ научитися, равно пріимать должно ученіе простое и украшенное. "--- 3)~Аще когда пастырь обличительное слово говоритъ, не должно за то людямъ на него негодовать, много паче поносить имя его. Понеже онъ должность свою исполняетъ: \textit{обличи}, глаголетъ ему апостолъ, какъ выше сказано. Обличаетъ онъ грѣхи, а не лица: яко никакого лица въ словѣ своемъ не именуетъ, но токмо грѣхи. Обличительное слово подобно зеркалу, которое показуетъ намъ пороки на лицѣ, да очистимъ ихъ: такъ и слово обличительное представляетъ грѣхи, да, усмотрѣвше ихъ въ совѣсти нашей, потщимся покаяніемъ изгладить. Оставь грѣхъ, и покайся, и тогда спокоенъ будеши. Аще же въ совѣсти твоей не видишь грѣха, который обличается и порочится отъ пастыря, то и слово его не тебе касается. Совѣсть бо и слово обличительное во едино сходятся; и что слово обличаетъ, тое и совѣсть; и чего слово въ тебѣ не обличаетъ, того ниже совѣсть. Якоже убо, въ зерцало смотрячи и не видя пороковъ на лицѣ, спокойны отходимъ отъ него: тако, слыша слово обличительное и не находя въ совѣсти нашей грѣховъ обличаемыхъ, должны быть спокойны. И якоже на зеркало не гнѣваемся, что показуетъ пороки наши: тако не должно гнѣваться на пастыря, что обличаетъ грѣхи наши. Какъ бо зеркало показуетъ тое, что имѣется на лицѣ, и не показуетъ, чего не имѣется: такъ и слово обличаетъ тое, что противу закона Божія и совѣсти сотворено, и не обличаетъ того, что не сдѣлано. Видишь убо, что слово обличительное стремится на грѣхи, а не на лица: якоже жестокое лѣкарство дается больному не ради больнаго, но ради болѣзни его, дабы болѣзнь изгналась изъ него, а не больный страдалъ и мучился. Лютая бо болѣзнь есть грѣхъ, и требуетъ жестокаго лѣкарства ради изгнанія. Якоже убо благодаришь лѣкарю за лѣкарство, хотя и жестокое, и мзду даешь ему, что тѣло твое немощное и смертное хощетъ излѣчить: кольми паче долженъ еси благодарить пастырю, который душу твою безсмертную тщится исцѣлить отъ грѣха. Чимъ бо душа честнѣйшая паче тѣла, тѣмъ болѣзнь ея опаснѣйшая паче тѣлесныя, и исцѣленіе ея дражайшее паче тѣлеснаго. "--- 4)~Люди пастырей своихъ должны любить и отъ любви ихъ почитать, яко отцевъ своихъ, которые не къ временному, но къ вѣчному животу словомъ живаго Бога раждаютъ, и отеческимъ сердцемъ \textit{болѣзнуютъ, дондеже вообразится въ людяхъ Хрістосъ}\footnote{Гал.~4,~19.}. "--- 5)~Снабдѣвать ихъ отъ имѣній своихъ. \textit{Достоинъ бо дѣлатель мзды своея}\footnote{1~Тим.~5,~18.}. "--- 6)~Защищать ихъ отъ злорѣчивыхъ языковъ, которые славу и имя ихъ терзаютъ и помрачаютъ. Никто бо болѣе злорѣчію и клеветѣ не подлежитъ, какъ исправный пастырь. И сія есть кознь и хитрость сатанинская! Онъ тщится чрезъ злый языкъ хулу и клевету имени ихъ нанести, дабы тако люди ими гнушалися, и ученія ихъ не слушали и не спаслися, "--- и злоба злыхъ людей, которые отъ нихъ обличаются за беззаконія ихъ, вымышляетъ на нихъ различныя клеветы, что церковная исторія показуетъ. Сатана и злый міръ истины не любитъ, которую пастыри насадить стараются; лжи и неправды держится, которую они искоренить тщатся. Что оттуду послѣдуетъ имъ, какъ не ненависть и злоба отъ міролюбцевъ? Сего ради люди благочестивыи не должны вѣрить клеветѣ, о нихъ разсѣваемой, но и клевещущимъ уста заграждать. "--- 7)~Аще какія немощи въ нихъ сами примѣтятъ, то имъ снисходить должны, яко и пастыри такіежде люди, какъ и прочіи и тѣмижде немощьми обложени, какъ и прочіи человѣки. "--- 8)~Молитися о нихъ Богу, да умудритъ ихъ и укрѣпитъ въ немалотрудномъ и немаловажномъ дѣлѣ ихъ. Пастырь за людей, а люди за пастыря молитися должны, и такъ другъ другу помоществовать будутъ въ дѣлѣ спасенія. "--- 9)~Хотя и неисправны будутъ пастыри житіемъ, то не должно ихъ осуждать. Аще бо простаго человѣка запрещено осуждать, кольми паче пастыря. Они за себе и за тебе отвѣтъ дадутъ Судіи, Которому падаютъ, а не ты за нихъ. Отврати убо очи отъ неисправности ихъ, и внимай токмо ученію ихъ, когда согласно вѣрѣ учатъ, якоже Хрістосъ о фарисеяхъ глаголетъ: \textit{на Моисеовѣ сѣдалищи сѣдоша книжницы и фарисее. Вся убо, елика аще рекутъ вамъ блюсти, соблюдайте и творите: по дѣломъ же ихъ не творите, глаголютъ бо и не творятъ}\footnote{Матѳ.~23,~2 и 3.}.

\subsection[Глава 5-я. О должности родителей и дѣтей ихъ.]{глава пятая.\\\bfseries О должности родителей и дѣтей ихъ.}

\begin{quotation}\textit{Чада! послушайте своихъ родителей о Господѣ: сіе бо есть праведно. Чти отца твоего и матерь, яже есть заповѣдь первая во обѣтованіи: да благо ти будетъ, и будеши долголѣтенъ на земли}. И: \textit{отцы! не раздражайте чадъ своихъ, но воспитовайте ихъ въ наказаніи и ученіи Господни}\footnote{Ефес.~6,~1--5.}.\end{quotation}

\paragraph*{§\:473.} Нѣкоторые родители такъ нѣжно и слабо дѣтей своихъ воспитываютъ и содержатъ, что не хотятъ ихъ за преступленія наказывать, и такъ безстрашно и своевольно имъ попущаютъ жить; другіи безмѣрную строгость употребляютъ, и болѣе гнѣвъ и ярость свою надъ ними совершаютъ, нежели наказуютъ ихъ. Обои, и тіи и другіи, погрѣшаютъ. Вездѣ бо излишество порочно; строгость и милость безразсудная во всякомъ чинѣ охуждается. Сія въ разслабленіе, своевольство, развращеніе и явную погибель приводитъ юныхъ, отъ природы ко злу всякому склонныхъ: оная огорченіе, раздраженіе и уныніе въ нихъ содѣваетъ. Вездѣ бо умѣренность и средній путь похваляется. Сего ради родителямъ благочестивымъ средняго пути держаться должно.

\paragraph*{§\:474.} \textit{Должность родителей къ дѣтямъ своимъ}. 1)~Какъ только начнутъ дѣти приходить въ разумъ и понимать ученіе, тотчасъ должно имъ вливать млеко благочестія, и въ познаніе приводить Бога и Хріста Сына Божія: кто есть Богъ Тотъ, въ Котораго вѣруемъ, и поминаемъ имя Его, и исповѣдуемъ и молимся Ему? и кто есть Хрістосъ, и какъ Его должно почитать? Ради чего на сей свѣтъ вси раждаемся и крещаемся, и чего по смерти чаемъ? Яко нынѣшнее житіе наше не ино что есть, какъ путь, которымъ идемъ къ вѣчности, добрый къ благополучной, злый къ неблагополучной. Раждаемся на сей свѣтъ не ради чести, богатства, пищи сладкой, одѣянія краснаго, богатыхъ домовъ и прочаго: все бо сіе при смерти оставляемъ. Но раждаемся, чтобы здѣ благочестиво пожить, и Богу угодить, и по смерти къ Нему прейтить и въ вѣчномъ Его блаженствѣ пребывать. Иначе ежели бы къ сему житію раждалися мы, то бы во вѣки намъ должно пребывати здѣ; но противное видимъ. Убо къ иному житію раждаемся, и на путь міра сего вступаемъ, чтобы къ оному дойтить. Сего ради и крещаемся, и вѣруемъ въ Бога и Хріста Сына Божія, и призываемъ имя Его, и въ церковь ходимъ и молимся, да получимъ отъ Него оное будущее блаженство. Все сіе и прочее нужно въ началѣ предлагать дѣтямъ юнымъ, да, въ возрастъ приходя, приходятъ и въ познаніе Божіе, и должности и надежды хрістіанской. Отсюду чаять можно доброй надежды въ юномъ сердцѣ, когда такъ воспитоваться начнетъ. Какъ бо зло, такъ и добро въ юношескомъ сердцѣ крѣпко вкореняется; и чему въ юности научаемся, въ томъ и въ совершенный возрастъ пришедше пребываемъ: якоже молодое дерево, къ которой сторонѣ приклонится, такъ и стоитъ до конца. Сего ради нужно есть такое благочестивое воспитаніе юности. И когда сами родители или не могутъ или не допускаются званіемъ того дѣлать, нужно есть имъ таковыхъ наставниковъ искать и обучать дѣтей. Многіи дѣтей своихъ учатъ иностраннымъ языкамъ и художествамъ, но въ дѣлѣ благочестія не обучаютъ; отъ чего видно, что и сами того не знаютъ, хотя и хрістіанами называются. Полезно ради общества и коммерціи и языкамъ иностраннымъ обучать; но таинствамъ вѣры обучать нужно, и нужно неотмѣнно, и \textit{едино есть на потребу}\footnote{Лук.~10,~42.}. Что во французскомъ, или другомъ какомъ языкѣ, когда языкъ наученъ будетъ, а сердце не научено добру? Языкъ хорошо и красно витійствуетъ, но сердце есть праздно вѣры, и издаетъ смрадъ невѣрія, что и родителямъ небрежливымъ и дѣтямъ бѣдственно. "--- 2)~Понеже, по Писанію, \textit{начало премудрости "--- страхъ Господень}\footnote{Пс.~110,~10.}, того ради въ началѣ въ юныя сердца должно всаждать страхъ Божій, яко юность, отъ природы ко злу склонная, ничимъ болѣе, какъ симъ страхомъ, отъ того отвращается, какъ и всякій человѣкъ. А чтобы страхъ Божій всадить въ сердца ихъ, нужно имъ часто напоминать, что Богъ вездѣ есть и со всякимъ человѣкомъ присутствуетъ и, что человѣкъ ни дѣлаетъ или мыслитъ, видитъ, и что ни говоритъ, слышитъ, и за всякое слово, дѣло и помышленіе худое гнѣвается, и будетъ судить, и грѣшниковъ вѣчному мученію предастъ, якоже праведнымъ и добрымъ людямъ воздастъ мзду за добрыя ихъ дѣла, и согрѣшающаго, или зло творящаго можетъ въ самомъ дѣлѣ показнить, какъ тое читаемъ въ исторіяхъ, и нынѣ тоежде бываетъ. Сіе имъ въ началѣ нужно вкоренять, дабы не токмо явно, но и тайно отъ всякаго зла уклонялися; дабы, какъ дѣти предъ родителями своими, раби предъ господами своими, подвластніи предъ властями, ничего непристойнаго не дѣлаютъ, но благоговѣйно поступаютъ тако бы они предъ Богомъ, вся назирающимъ, поступали и ходили со страхомъ, и думали бы, что Богъ и съ ними есть, и вси ихъ поступки видитъ, и можетъ ихъ показнить, когда худо будутъ дѣлать. Хотя случается, что человѣкъ худаго дѣла не видитъ; но Богъ, большій всего свѣта и Судія всѣхъ, все видитъ. Отъ таковаго ученія и разсужденія о Богѣ можетъ въ юныхъ страхъ Божій насадиться, что и пожилымъ и старымъ помнить нужно. «Како, глаголетъ святый Златоустъ, страхъ Божій будетъ? Аще помыслимъ, яко вездѣ предстоитъ Богъ, яко все слышитъ, яко все видитъ, не токмо тое, что дѣлается и глаголется, но и тое все, что въ сердцѣ и что во глубинѣ помышленія есть: судителенъ бо есть помышленій и мыслей сердечныхъ. Аще сице устроимъ себе, ничтоже лукаво сотворимъ, ничтоже речемъ, ничтоже помыслимъ»\footnote{Бес.~8"~я на посл. къ Филип.}. "--- 3)~Дѣтей неисправныхъ должно наказывать родителямъ. Тако слово Божіе повелѣваетъ имъ: \textit{не преставай младенца наказовати: аще бо жезломъ біеши его, не умретъ отъ него; ты бо побіеши его жезломъ, душу его избавиши отъ смерти}\footnote{Притч.~23,~13 и 14.}. Видимъ, что Самъ Богъ любитъ чадъ Своихъ, но отъ любви ихъ наказуетъ: \textit{егоже бо любитъ Господь, наказуетъ: біетъ же всякаго сына, егоже пріемлетъ}\footnote{Евр.~12,~6.}. Тако и плотскимъ родителямъ должно послѣдовать Богу, и дѣтей своихъ отъ любви наказывать. Слѣпая убо тая есть любовь отчая, которая оставляетъ дѣтей неисправныхъ безъ наказанія: якоже истинная и мудрая есть тая любовь, которая своевольство ихъ смиряетъ наказаніемъ. \textit{Иже щадитъ жезлъ, ненавидитъ сына своего: любяй же наказуетъ прилѣжно}\footnote{Притч.~13,~25. См. о семъ еще: Притч.~29,~17; Сир.~7,~25; 30,~1--13.}. "--- 4)~Не должно въ наказаніи безмѣрной строгости употреблять, якоже апостолъ повелѣваетъ: \textit{отцы, не раздражайте чадъ своихъ, да не унываютъ}\footnote{Кол.~3,~21.}, но среднимъ путемъ поступать, какъ выше сказано. "--- 5)~Во образъ добрыхъ дѣлъ самихъ себе имъ представлять. Юность бо, да и всякій возрастъ, лучше наставляется къ добродѣтели житіемъ добрымъ, нежели словомъ: ибо юныя наипаче дѣти за правило себѣ имѣютъ житіе родителей своихъ; такъ, что въ нихъ примѣчаютъ, тое и сами дѣлаютъ, доброе ли тое будетъ, или худое, что видятъ. Чего ради родителямъ, какъ отъ соблазновъ берещися, такъ примѣръ добродѣтельнаго житія подавать дѣтямъ своимъ должно, когда хотятъ ихъ къ добродѣтели наставить. Иначе ничего не могутъ успѣть. Болѣе бо они смотрятъ на житіе родителей своихъ, и тое воображаютъ въ юныхъ своихъ душахъ, нежели слова ихъ слушаютъ. Всякаго наставника слово, съ житіемъ сопряженное, изрядное и сильное есть наставленіе: кольми паче родительское наставленіе. "--- 6)~Къ любви дѣтей и самое естество родителей влечетъ и убѣждаетъ: и самыя безсловесныя исчадія своя любятъ. Чего ради не потребно о томъ и предлагать, только бы не была безразсудная любовь, какъ выше сказано. "--- 7)~Родителямъ о дѣтяхъ своихъ должно молитися Богу, дабы ихъ Самъ Онъ наставилъ на страхъ Свой и умудрилъ во спасеніе. "--- 8)~Коль вредно небреженіе о правильномъ воспитаніи и наказаніи дѣтей и самымъ родителямъ, и дѣтямъ ихъ, "--- отъ вышеписанныхъ видитъ всякъ, и исторія показуетъ о Иліи, жрецѣ Израильскомъ, который, что правильно не воспитовалъ и не наказывалъ сыновъ своихъ за продерзости ихъ, и самъ и дѣти его наказаны отъ Бога\footnote{1~Цар.~4,~17 и 18.}. О чемъ святый Златоустъ поучаетъ тако: «Родители"=де не токмо за свои грѣхи наказаны будутъ, но и за дѣтей своихъ, ежели ихъ въ благочестіи не воспитуютъ». И паки: «Родители, которые дѣтей по"=хрістіански воспитывать пренебрегаютъ, дѣтоубійцъ беззаконнѣйшіи суть. Ибо дѣтоубійцы тѣло отъ души разлучаютъ, а они и душу и тѣло въ геенну огненную ввергаютъ. Оной смерти по естественному закону никакъ невозможно избѣжать; а сей возможно было бы, ежели бы нерадѣніе родителей не было ея виновно. Ктомужъ смерть сію воскресеніе будущее упразднитъ: душевной же погибели никто возвратить не можетъ»\footnote{Слово 3"~е противу хулителей монашескаго житія.}.

\paragraph*{§\:475.} \textit{Должность дѣтей} къ своимъ родителямъ. 1)~Должны дѣти родителей своихъ любить отъ чистаго сердца, и къ нимъ благодарными быть: яко отъ нихъ на свѣтъ родилися и воспиталися. Къ сему естественный законъ и совѣсть всякаго убѣждаетъ. 2)~Должное имъ отдавать послушаніе, якоже апостолъ повелѣваетъ имъ: \textit{чада! послушайте своихъ родителей о Господѣ}. 3)~Словомъ и дѣломъ любовное имъ почтеніе показывать. "--- 4)~Когда отъ нихъ наказуются словомъ или дѣломъ, за благо тое ихъ наказаніе принимать, ни негодовать, ни роптать на нихъ, но, признавая свое преступленіе, со смиреніемъ, любовію и жалѣніемъ просить прощенія у нихъ. 5)~Въ нуждѣ и старости ихъ питать, и со всякимъ усердіемъ служить имъ. 6)~Беззаконнымъ дѣтямъ, которыя родителей своихъ не почитаютъ, страшныя казни въ законѣ Господни опредѣлены. \textit{Иже біетъ отца своего, или матерь свою, смертію да умретъ}\footnote{Исх.~21,~21.}. \textit{Человѣкъ, иже аще зло речетъ отцу своему, или матери своей, смертію да умретъ}\footnote{Лев.~20,~9.}.

\subsection[Глава 6-я. О должности мужей и женъ.]{глава шестая.\\\bfseries О должности мужей и женъ.}

\begin{quotation}\textit{Мужіе! любите своя жены, якоже и Хрістосъ возлюби церковь, и Себе предаде за ню}, и проч.\footnote{Еф.~5,~25 и слѣд.}\end{quotation}
\begin{quotation}\textit{Жены! повинуйтеся своимъ мужемъ, якоже подобаетъ, о Господѣ}\footnote{Кол.~3,~18.}.\end{quotation}

\paragraph*{§\:476.} Сатана, всегдашній хрістіанъ врагъ, какъ между прочіими хрістіанами, такъ между женами и мужами тщится любовь и миръ отъять и посѣять вражду. Удивленія или паче сожалѣнія достойная вещь! Гдѣ большія любве надѣяться, какъ между мужемъ и женою? Естественною любовію любитъ человѣкъ отца и матерь свою; но Писаніе святое глаголетъ: \textit{оставитъ человѣкъ отца и матерь свою, и прилѣпится къ женѣ своей и будутъ оба въ плоть едину}\footnote{Быт.~2,~24.}. Въ \textit{едину} убо \textit{плоть} бываютъ мужъ и жена, по Писанію. Крѣпкій союзъ въ единой плоти есть! И кто на свою плоть враждуетъ? \textit{Никтоже бо когда свою плоть возненавидѣ, но питаетъ и грѣетъ ю}\footnote{Еф.~5,~29.}. Но сколько вражды и между сими лицами, такъ тѣсно связанными, видимъ! Мало который домъ сыскать можно, въ которомъ бы сіи плевелы діавольскія не были. Такъ"=то хитрость діавольская возмогаетъ, что гдѣ примѣчаетъ большую любовь, тамо онъ большее тщаніе полагаетъ разорвать тую и положить вражду свою. Сколько же сія язва, какъ прочіимъ хрістіанамъ, такъ наипаче бракомъ сопрягшимся вредитъ, изъяснить невозможно! Сего ради противу сего врага должно вооружиться вѣрою, и внимать тому, что Богъ въ словѣ Своемъ повелѣваетъ другъ другу чинить.

\paragraph*{§\:477.} \textit{Должность мужей къ женамъ}. 1)~Мужи должны любить своихъ женъ, по увѣщанію апостольскому: \textit{мужіе! любите своя жены}. Причину полагаетъ: понеже мужъ и жена \textit{едина плоть суть}. Откуду глаголетъ ниже: должни суть мужіе любити своя жены, якоже своя тѣлеса: \textit{никтоже бо когда свою плоть возненавидѣ, но питаетъ и грѣетъ ю}. И во образъ имъ представляетъ Самаго Хріста, Сына Божія, Который церковь Свою любитъ: \textit{якоже}, рече, \textit{и Хрістосъ возлюби церковь, и Себе предаде за ню}. "--- 2)~Не должны съ ними жестоко поступать, якоже учитъ апостолъ: \textit{мужіе! любите жены ваша, и не огорчайтеся къ нимъ}\footnote{Кол.~3,~19.}; но якоже Петръ святый увѣщаваетъ: \textit{вкупѣ живуще съ своими женами по разуму, яко немощнѣйшему сосуду женскому воздающе честь, яко и наслѣдницы благодатныя жизни, во еже не прекращатися молитвамъ вашимъ}\footnote{1~Петр.~3,~7.}. "--- 3)~Должны хранить вѣрность къ своимъ женамъ, супружеское ложе непорочно соблюдать, чужаго ложа не касаться, \textit{яко честна женитва во всѣхъ, и ложе нескверно: блудникомъ же и прелюбодѣемъ судитъ Богъ}\footnote{Евр.~13,~4.}. "--- 4)~Хотя любить и должны женъ своихъ мужи, но плѣняться любовію ихъ и безъ разума имъ угождать не должны. Видимъ, коль вредно угождать женамъ безъ разума. Адамъ безмѣрною любовію жены плѣнился, и палъ въ тяжкій грѣхъ, и себе и родъ человѣческій погубилъ\footnote{Быт.~3,~17.}. Соломонъ премудрый, излишнимъ женъ любленіемъ плѣненный, въ грѣхъ богоотступства и идолопоклоненія палъ\footnote{3~Цар.~11,~4 и слѣд.}. Іезавель нечестивая научила мужа своего Ахава, царя Іерусалимскаго, беззаконновать. Иродъ беззаконный не устрашился неповинно обезглавить великаго Предтечу ради любви женскія. Тоежде читаемъ и въ церковной исторіи. Подобныя бѣды и напасти и нынѣ бываютъ отъ женъ, когда имъ мужи не по разуму угодіе творятъ. Жена неразумная и безстрашная обычая своего не премѣняетъ, ласкательствуетъ и прельщаетъ мужа своего. Мужъ яко \textit{глава жены}, не долженъ ласкательствомъ ея прельщатися, а паче на власти какой учиненный, да не и самъ погибнетъ, и инымъ много бѣдъ учинитъ.

\paragraph*{§\:478.} \textit{Должность женъ къ своимъ мужамъ}. 1)~Такожде и жены должны своихъ мужей любить, и угождать имъ во благое и по волѣ Божіей. 2)~Хранить вѣрность супружескаго ложа, и не знать никого, кромѣ своихъ законныхъ мужей. 3)~Должны къ своимъ мужамъ благоговѣйны быть, по апостольскому велѣнію: \textit{жена да боится своего мужа}\footnote{Еф.~5,~33.}. Страхъ сей съ любовію сопряженъ долженъ быть. Жены мужамъ должны повиноватися, якоже повелѣваетъ имъ апостолъ: \textit{жены! повинуйтеся своимъ мужемъ, якоже подобаетъ о Господѣ}\footnote{Кол.~3,~18.}. И паки: \textit{женѣ учити не повелѣваю, ни владѣти мужемъ, но быти въ повиновеніи}. Причину тому полагаетъ апостолъ: \textit{Адамъ бо прежде созданъ бысть, потомъ же Ева. И Адамъ не прельстися: жена же, прельстившися, въ преступленіи бысть}\footnote{1~Тим.~2,~12--14.}. Однакожъ мужамъ должно помнитъ, что апостолъ написалъ на другомъ мѣстѣ: \textit{нѣсть мужескій полъ, ни женскій: вси бо вы едино есте о Хрістѣ Іисусѣ}\footnote{Гал.~3,~18.}; и женъ своихъ не имѣть яко рабынь, но яко помощницъ, и \textit{яко тѣлеса своя любить}, по апостольскому увѣщанію, какъ выше сказано. Женамъ да будетъ образъ въ повиновеніи мужамъ своимъ святая Сарра, которая \textit{послушаше Авраама, господина того зовущи}\footnote{1~Петр.~3,~6.}. Да не стыдится и нынѣ жена мужа своего господиномъ называти, когда хощетъ быть дщерію святыя Сарры.

\paragraph*{§\:479.} Союзъ брака долженъ быть твердъ и ненарушимъ безъ благословныхъ причинъ. Тако повелѣваетъ Божіе слово: \textit{всякъ, пущая жену свою развѣ словесе прелюбодѣйнаго, творитъ ю прелюбодѣйствовати, и иже пущеницу пойметъ, прелюбодѣйствуетъ}\footnote{Матѳ.~5,~32.}. \textit{Всякъ, пущаяй жену свою и приводя ину, прелюбы дѣетъ, и женяйся пущеною отъ мужа, прелюбы творитъ}\footnote{Лук.~16,~18.}. \textit{Привязался ли женѣ? не ищи разрѣшенія. И жена привязана есть закономъ, въ елико время живетъ мужъ ея}\footnote{1~Кор.~7,~27 и 39.}.

\subsection[Глава 7-я. О должности господъ и рабовъ ихъ.]{глава седьмая.\\\bfseries О должности господъ и рабовъ ихъ.}

\begin{quotation}\textit{Раби! послушайте господій своихъ по плоти, со страхомъ и трепетомъ, въ простотѣ сердца вашего, якоже и Хріста, не предъ очима точію работающе, яко человѣкоугодницы, но якоже раби Хрістовы, творяще волю Божію отъ души: со благоразуміемъ служаще якоже Господу, а не яко человѣкомъ, вѣдяще, яко кійждо, еже аще сотворитъ благое, сіе пріиметъ отъ Господа, аще рабъ, аще свободь. И господіе таяжде творите къ нимъ, послабляюще имъ прещенія, вѣдуще, яко и вамъ самѣмъ и тѣмъ Господь есть на небесѣхъ и обиновенія лица нѣсть у Него}\footnote{Еф.~6,~5--9.}.\end{quotation}

\paragraph*{§\:480.} Въ церкви Хрістовой или хрістіанскомъ обществѣ нѣтъ господина и раба, но вси суть братія о Хрістѣ, вси суть раби Господа Вышняго, и сынове небеснаго Отца; и потому вси едино суть, по апостольскому ученію: \textit{нѣсть Іудей, ни Еллинъ: нѣсть рабъ, ни свободь, нѣсть мужескій полъ, ни женскій; вси бо вы едино есте о Хрістѣ Іисусѣ}\footnote{Гал.~3,~28.}. Что же иные называются господа, иные раби, сіе бываетъ по образу вѣка сего, котораго разнствія въ обществѣ не отнимаетъ Божіе слово. Проповѣдуетъ оно свободу хрістіанамъ, и повелѣваетъ въ ней пребывати. \textit{Свободою, еюже Хрістосъ насъ свободи, стойте, и не паки подъ игомъ работы держитеся}\footnote{5,~1.}. Но сія свобода разумѣется духовная, то"=есть, свобода отъ грѣха, клятвы законныя, діавола, ада, и ветхозаконныхъ обрядовъ. Сію свободу и раби благочестивіи имѣютъ. Напротивъ того, и господа, хотя тѣломъ свободни суть, но когда грѣху и страстямъ работаютъ, оныя свободы не имѣютъ, но суть раби, по словеси Хрістову: \textit{всякъ, творяй грѣхъ, рабъ есть грѣха}\footnote{Іоан.~8,~34.}. И сія свобода, то"=есть, свобода духовная, хрістіанская, внутренняя, обща всѣмъ есть, господамъ и рабамъ благочестивымъ; и дотолѣ ее имѣютъ, доколѣ имѣютъ въ сердцахъ своихъ вѣру живую въ Господа нашего Іисуса Хріста. Безъ вѣры бо сея имѣть свободу духовную никому невозможно: яко безъ Хріста никто не можетъ быть духовно свободенъ. Онъ единъ сію свободу намъ даетъ\textit{: аще сынъ вы свободитъ, воистинну свободни будете}\footnote{ст.~36.}. Видимъ убо, что въ святой церкви находящіися вси свободни суть духовно, и нѣтъ разнствія между рабомъ и господиномъ, но вси суть братія между собою и раби Хріста Господа. \textit{Призванный бо о Господѣ рабъ, свободникъ Господень есть: такожде и призванный свободникъ, рабъ есть Хрістовъ}\footnote{1~Кор.~7,~22.}. По внѣшнему же міра сего виду, иніи свободни суть, иніи раби, котораго внѣшняго разнствія не отнимаетъ Господь, но повелѣваетъ всякому въ какомъ званіи имѣется, въ томъ пребывати: \textit{кійждо въ званіи, въ немже призванъ бысть, въ томъ да пребываетъ}\footnote{24.}. Какая же обоихъ, то"=есть, господъ и рабовъ, должность есть, вкратцѣ представляется.

\paragraph*{§\:481.} \textit{Должность господъ къ своимъ рабамъ}. 1)~Господа должни помнить, что они сами и раби ихъ суть братія о Хрістѣ, и единаго Господа имѣютъ на небесѣхъ. Слѣдовательно хотя и повелѣваютъ рабамъ своимъ, однакожъ не должни ихъ презирать и за подножіе свое имѣть. Хрістосъ, Господь славы и Царь небесе и земли, не стыдится хрістіанъ \textit{братіею Своею называти}\footnote{Пс.~21,~23; Евр.~2,~12.}: кольми паче человѣку простому подобнаго себѣ человѣка должно за брата имѣть. Да послѣдуютъ убо господа Хрісту, яко раби Господу своему, и съ рабами своими яко съ братіею своею да поступаютъ. И хотя должни господа рабовъ своихъ въ страсѣ содержать, однакожъ внутрь въ сердцѣ имѣть ихъ и почитать за братію, и думать о себѣ, что ничимъ не лучшіи отъ нихъ суть. "--- 2)~Хотя и наказывать неисправныхъ нужно будетъ, однакожъ должно имъ тое чинить умѣренно, и поступать духомъ кротости, и \textit{послаблять прещенія, вѣдая, что и имъ самимъ и тѣмъ Господь есть на небесѣхъ, и лицепріятія нѣсть у Него}, по ученію апостольскому\footnote{Еф.~6,~9.}. Да разсуждаютъ убо господа сіе, что они сами подчинены суть Господу господствующихъ, Который на лица не смотритъ, праведно судитъ между господами и рабами ихъ, предъ Котораго судомъ и господъ не защититъ свобода, и рабовъ не постыдитъ рабская подлость, но вси по своимъ дѣламъ воспріимутъ\footnote{Еф.~6,~8.}. "--- 3)~Не обременять ихъ излишними работами и оброками. Яко, какъ господа, такъ и раби такуюжде плоть имѣютъ: такожде хотятъ питаться, одѣваться и упокоеваться съ женами и дѣтьми своими; такожде, дабы имѣли время къ молитвѣ, славословію Божію и слушанію Божія слова, съ прочіими хрістіанами въ храмы святые собираться. "--- 4)~Любить ихъ, яко братію свою о Хрістѣ, и защищать ихъ и промышлять отечески о нихъ. Всякъ бо господинъ благочестивый долженъ быть отецъ своимъ рабамъ. "--- 5)~Словомъ и увѣщаніемъ обращать и наставлять ихъ на путь истины, и страхомъ Божіимъ обуздовать безчинныхъ. "--- 6)~Самихъ себе во образъ благочестиваго хрістіанскаго житія подавать имъ, дабы раби, видя господина своего благочестиваго и добронравнаго, и сами были добронравными. Ничто бо тако не приводитъ рабовъ къ добронравію, какъ примѣръ добронравнаго господина. Убѣждается рабъ благочестно жити благочестивымъ господина своего житіемъ. Сильное и дѣйствительное поощреніе рабамъ къ трезвому и благопостоянному житію трезвое и благопостоянное господина житіе. Тогда господинъ правильно и сильно увѣщаваетъ раба не обиждати, не красти, не сквернословити, не ругати, не прелюбодѣйствовати, не пьянствовати, не безчинствовати, когда самъ того не дѣлаетъ. Иначе и ему тое слово прилично: \textit{врачу, исцѣлися самъ}\footnote{Лук.~4,~23.}!

\paragraph*{§\:482.} Погрѣшаютъ господа противу должности хрістіанской: 1)~Которые рабовъ своихъ презираютъ, и не снисходительно съ ними поступаютъ. Сіи преступаютъ ученіе Евангельское. Ибо Спаситель нашъ обязываетъ \textit{всякаго любить}\footnote{Матѳ.~22,~39; 5,~44--47.}. И святый Апостолъ Павелъ увѣщаваетъ, чтобъ \textit{правду и уравненіе имъ подавать}\footnote{Кол.~4,~1.}. "--- 2)~Которые ихъ утѣсняютъ и поступаютъ немилосердо. Наказывать ихъ не иначе, какъ только чтобъ они исправлялись, но наказаніе растворять гнѣвомъ, яростію и мщеніемъ есть беззаконное дѣло. Должно помнить, что они отъ единаго Творца и тоежде человѣчество имѣютъ. Они обязаны оброкомъ и работою заслуживать попеченіе, каковое господинъ объ нихъ прилагаетъ; но должно оставлять ему время доставить пропитаніе себѣ и своимъ домашнимъ, и давать время къ покою. Должно всѣмъ помнить, что вси имѣемъ Господа на небесѣхъ, Который на всѣхъ взираетъ, и всѣхъ дѣла видитъ, и воздыханія убогихъ слышитъ, и на проливаемыя слезы смотритъ, и лицъ человѣческихъ не пріемлетъ, но равно господину и рабу судитъ, и всѣмъ воздаетъ по дѣламъ ихъ. А забывше о Господѣ забыли и тое, что они хрістіане суть: забыли, что Господь оный равно промышляетъ о всѣхъ, равно сіяетъ солнце Свое, равно дождитъ, равно пищу и одѣяніе даетъ всѣмъ, равно сіяетъ день къ дѣланію, и нощь опредѣляетъ къ упокоенію, равно проливаетъ щедроты Свои всѣмъ, и хощетъ, чтобы всѣ благости Его насыщалися, и насыщаяся благодарили Благодѣтелю. Но несытое человѣческое лакомство пресѣкаетъ тое; едино пожираетъ все, чимъ довольствоваться многіи бы могли, и довольствуяся хвалить Господа, \textit{отверзающаго руку Свою, и исполняющаго всякое животно благоволенія}\footnote{Пс.~144,~16.}. Надобно таковымъ очувствоваться и премѣнить свои нравы, и поступать по"=хрістіански, когда вѣруютъ, что есть Господь на небесѣхъ, Который лица человѣческаго не пріемлетъ и судитъ всѣмъ по дѣламъ ихъ. А когда нынѣ не очувствуются, то тогда очувствуются, когда вси вмѣстѣ станутъ предъ праведнымъ Судіею, и многихъ, которыхъ нынѣ за подножіе свое и какъ скотовъ имѣютъ, увидятъ въ дивной перемѣнѣ, увидятъ въ честь и славу вѣчную одѣянныхъ, себе же стыдомъ и срамомъ вѣчнымъ покрытыхъ; и пожелаютъ участіе имѣть съ тѣми, которыми нынѣ гнушаются, но не получатъ того; услышатъ слово реченное: \textit{чадо, помяни, яко воспріялъ еси благая твоя въ животѣ твоемъ, и Лазарь такожде злая: нынѣ же здѣ утѣшается, ты же страждеши}\footnote{Лук.~16,~25.}.

\paragraph*{§\:483.} \textit{Должность рабовъ къ своимъ господамъ}. Сей Божіе слово научаетъ ихъ: \textit{Раби, послушайте господій своихъ по плоти, со страхомъ и трепетомъ, въ простотѣ сердца вашего, якоже Хріста; не предъ очима точію работающе, яко человѣкоугодницы, но якоже раби Хрістовы, творяще волю Божію отъ души, со благоразуміемъ служаще якоже Господу, а не яко человѣкомъ, вѣдяще, яко кійждо, еже аще сотворитъ благое, сіе пріиметъ отъ Господа, аще рабъ, аще свободь}\footnote{Ефес.~6,~5--8.}. И паки: \textit{Раби, послушайте по всему плотскихъ господій вашихъ, не предъ очима точію работающе, аки человѣкоугодницы, но въ простотѣ сердца, боящеся Бога. И всяко, еже аще что творите, отъ души дѣлайте, якоже Господу, а не человѣкомъ, вѣдяще, яко отъ Господа пріимете воздаяніе достоянія: Господу бо Хрісту работаете}\footnote{Кол.~3,~22--24.}. И паки: \textit{Рабы, своимъ господемъ повиноватися, во всемъ благоугоднымъ быти, не прекословнымъ, не крадущимъ, но вѣру всяку являющимъ благу; да ученіе Спасителя нашего Бога украшаютъ во всемъ}\footnote{Тит.~2,~9 и 10.}. И паки: \textit{Раби, повинуйтеся во всякомъ страхѣ владыкамъ, не токмо благимъ и кроткимъ, но и строптивымъ. Се бо есть угодно предъ Богомъ, аще совѣсти ради Божія терпитъ кто скорби, стражда безъ правды. Кая бо похвала, аще согрѣшающе мучими терпите? Но аще добро творяще и страждуще терпите, сіе угодно предъ Богомъ. На сіе бо и звани бысте}\footnote{1~Петр.~2,~18--21.}.

\paragraph*{§\:484.} Раби благочестивіи могутъ себе утѣшить въ подломъ и бѣдномъ своемъ состояніи слѣдующими: 1)~Хотя и работаютъ господамъ своимъ, однакожъ суть раби Хрістовы: \textit{Господу бо Хрісту работаютъ}, по словеси апостольскому\footnote{Кол.~3,~24.}. Тѣломъ раби человѣческіи, но духомъ свободни суть. Духа бо никто не можетъ поработить. И \textit{суть свободники Господни}, по ученію апостольскому\footnote{1~Кор.~7,~22.}. Симъ великимъ титуломъ утѣшали себе и хвалилися святіи мученики; и хотя нечестивымъ царямъ и князямъ и господамъ работали, однако же отвѣтствовали имъ: «Хрістовы раби есмы; Хрісту, небесному Царю, работаемъ». Да почерпаютъ убо отсюду благочестивіи раби утѣшеніе, что они раби Хріста Царя небеснаго суть, хотя и человѣкамъ работаютъ. Напротивъ того, благородный и никакому человѣку не работающій лишается сладкаго сего титула, когда онъ неблагочестиво живетъ, то"=есть прихотямъ своимъ работаетъ. Хрісту бо и прихотямъ работать купно невозможно. "--- 2)~Которые страждутъ отъ своихъ господъ неповинно, или лишаются нужныхъ къ житію сему, да помышляютъ, что богатство и нищета, озлобленіе и терпѣніе, все минется, и минется скоро. Смерть бо все заключитъ, и тогда будетъ дивная перемѣна. Нынѣ только свирѣпѣетъ неистовство и гордость человѣческая; но тогда того не будетъ. Тогда смиреніе и терпѣніе вознесется, но своевольство и неистовство смирится и низпадетъ; и какъ терпѣніе, такъ и неистовство довольно наградится отъ правды Божіей. Ибо \textit{праведно у Бога, воздати скорбь оскорбляющимъ васъ; а вамъ, оскорбляемымъ, отраду съ нами, во откровеніи Господа Іисуса съ небеси, со ангелы силы Своея}, и проч.\footnote{2~Сол.~1,~7 и слѣд.} "--- 3)~Хотя и лишаются раби довольствія, однакожъ дни провождаютъ, какъ и лишающіи ихъ господа. Равно бо вчерашній день прошелъ у раба, какъ и у господина ихъ, съ тѣмъ только различіемъ, что сей въ роскоши, сластопитаніи и множайшихъ грѣхахъ, а оный въ скудости, нищетѣ и меньшихъ грѣхахъ прожили. И равно обои, и господинъ и раби его, приближаются къ концу, который и рабовъ и господъ ихъ равными учинитъ, и въ землю, отъ которой взятъ человѣкъ, пошлетъ, гдѣ нѣтъ разнствія раба отъ господина.

\subsection{Заключеніе статьи сея.}

1) Что о должностяхъ, выше кратко прописанныхъ, предложено, тое должно и о прочіихъ подобныхъ имъ разсуждать, какъ"=то: о должности учителей, которые наставляютъ юность, и о ученикахъ ихъ, мастеровыхъ людей и учащихся у нихъ, благодѣтелей и благодѣянія получающихъ, щедроподатливыхъ богачей и нищихъ, отъ нихъ снабдѣваемыхъ, и проч. "--- 2)~Вездѣ должно имѣть мѣсто свое увѣщательное апостольское оное слово, которое всѣмъ хрістіанамъ предлагаетъ: \textit{любовію работайте другъ другу}\footnote{Гал.~5,~13.}. Безъ любви бо ни въ какомъ званіи не можетъ отправиться хрістіанская должность, ибо всякое дѣло безъ любви мертво есть. Живность всякаго дѣла любовь есть. Якоже тѣло душею, тако всякое дѣло любовію оживляется. Хрістіанинъ безъ любви, какъ древо неплодное, или паче изсохшее. "--- 3)~Всякаго человѣка, кто бы и каковъ онъ ни былъ, напр. родителямъ дѣтей, и дѣтямъ родителей, мужу жену и женѣ мужа, и проч., дотолѣ должно любить, доколѣ тая любовь съ любовію Божіею согласна есть. А въ противномъ случаѣ любовь къ человѣку должна уступить любви Божіей. Паче же любовь живота своего не должна быть предпочтена любви Божіей: \textit{иже любитъ отца или матерь паче Мене, нѣсть Мене достоинъ, и иже любитъ сына или дщерь паче Мене, нѣсть Мене достоинъ}\footnote{Матѳ.~10,~37.}. И паки: \textit{аще кто грядетъ ко Мнѣ, и не возненавидитъ отца своего, и матерь, и жену, и чадъ, и братію, и сестръ, еще же и душу свою, не можетъ Мой быти ученикъ}, глаголетъ Хрістосъ\footnote{Лук.~14,~26.}. Бога бо должно любить паче всякаго созданія, паче всякаго человѣка и паче себе самого. "--- 4)~Понеже случается, что люди часто приказуютъ противу заповѣди Божіей, то всякъ подчиненный, напр. подвластный власть свою, сынъ и дочь родителей своихъ, жена мужа своего, рабъ господина своего, ученикъ учителя своего и проч., дотолѣ обдолжаются слушать, доколѣ согласное закону Божію или непротивное повелѣваютъ; а въ противномъ случаѣ заповѣдь Божію должно предпочитать заповѣди человѣческой, по ученію апостольскому: \textit{повиноватися подобаетъ Богови паче, нежели человѣкомъ}\footnote{Дѣян.~5,~29; 4,~19.}. "--- Хотя и всякъ ближнему, но наипаче власти подвластнымъ, а паче пастыри людямъ, родители дѣтямъ малымъ и юнымъ, учители ученикамъ своимъ подаютъ соблазнъ: того ради должно имъ крайне берещися, чтобы не подать соблазна. Понеже, какъ міру горе отъ соблазна, такъ горе и человѣку тому, имже соблазнъ приходитъ. \textit{Иже аще соблазнитъ единаго малыхъ сихъ вѣрующихъ въ Мя, уне есть ему, да обѣсится жерновъ осельскій на выи его, и потонетъ въ пучинѣ морстѣй. Горе міру отъ соблазнъ: нужда бо есть пріити соблазномъ. Обаче горе человѣку тому, имже соблазнъ приходитъ}, глаголетъ Господь\footnote{Матѳ.~18,~6 и 7.}. Соблазнъ приходитъ или словомъ гнилымъ, развращеннымъ, противнымъ святому Писанію и еретическимъ, или дѣломъ беззаконнымъ. Отъ всего того должно берещися высшимъ, да не соблазнятъ низшихъ.


\section[Статья 8-я. О утѣшительныхъ плодахъ святыя вѣры.]{статья осмая.\\\bfseries О утѣшительныхъ плодахъ святыя вѣры.}

\subsection{Предисловіе.}

Хрістіанина, что касается до внѣшняго его состоянія, нѣтъ ничего смиреннѣе и бѣднѣе въ мірѣ семъ. Сіе какъ въ святомъ Божіемъ словѣ видимъ, которое представляетъ его хулима, ненавидима, гонима, озлобляема и всякія скорби претерпѣвающа, такъ и въ исторіи церковной читаемъ *и нынѣ примѣчаемъ*. Тако благоволилъ небесный Отецъ, Богъ нашъ, да якоже Единородный Сынъ Его Іисусъ Хрістосъ, Господь нашъ, крестомъ устроилъ спасеніе намъ и крестомъ вошелъ въ славу Свою, якоже глаголетъ Самъ: \textit{не сія ли подобаше пострадати Хрісту и внити во славу свою}\footnote{Лук.~24,~26.}? "--- тако и вѣрныи раби Его, хрістіане, тѣмже путемъ крестнымъ идутъ и получатъ славу оную, страданіемъ и крестомъ Хрістовымъ пріобрѣтенную, якоже писано есть: \textit{многими скорбьми подобаетъ намъ внити въ царствіе Божіе}\footnote{Дѣян.~14,~22.}. \textit{Яко вси, хотящіи благочестно жити о Хрістѣ Іисусѣ, гоними будутъ}\footnote{2~Тим.~3,~12.}. Вѣра бо многоразличному искушенію подлежитъ. Но когда внутрь хрістіанина приникнуть, внутренняго человѣка разсудить, и на будущее по смерти и всеобщемъ воскресеніи состояніе обратить размышленіе, которое вѣрою единою постигается и неподвижною надеждою ожидается: нѣтъ ничего его блаженнѣе. \textit{Плоть и кровь} сего какъ \textit{не разумѣетъ}\footnote{1~Кор.~2,~14.}, такъ и \textit{наслѣдити не можетъ, ниже тлѣніе нетлѣнія наслѣдствуетъ}\footnote{15,~60.}. Сего ради благоутробный Отецъ небесный, промышляя о насъ, мноразличное противу горести крестной, которую чада Его здѣ вкушаютъ, положилъ утѣшеніе, и въ божественномъ Своемъ Писаніи различно изобразилъ, дабы тѣмъ утѣшеніемъ горесть крестную, яко пищу горькую вкушеніемъ меда, растворяли, и заченшуюся въ сердцахъ вѣру возгрѣвали, и противу сатанинскихъ козней, который тщится тую угасить, невредиму благодатію Святаго Духа сохраняли. Того ради и мнѣ заблагоразсудилось изъ того святаго и живаго божественныхъ Писаній источника почерпнуть, и въ утѣшеніе братіи моей, которымъ или время не достаетъ, или разума не далося къ тому, предложить на такій конецъ, дабы хранящіи вѣру, хранили тую благодатію Божіею до конца, "--- погубившіи, помощію Духа Святаго, сыскали тую, яко многоцѣнный бисеръ.

\subsection[Глава 1-я. О духовномъ хрістіанскомъ рожденіи или святомъ крещеніи.]{глава первая.\\\bfseries О духовномъ хрістіанскомъ рожденіи или святомъ крещеніи.}

\begin{quotation}\textit{Елицы пріяша Его, даде имъ область чадомъ Божіимъ быти, вѣрующимъ во имя Его, иже не отъ крове, ни отъ похоти плотскія, ни отъ похоти мужескія, но отъ Бога родишася}\footnote{Іоан.~1,~12 и 13.}.\end{quotation}
\begin{quotation}\textit{Всякъ вѣруяй, яко Іисусъ есть Хрістосъ, отъ Бога рожденъ есть}\footnote{1~Іоан.~5,~1.}.\end{quotation}
\begin{quotation}\textit{Елицы во Хріста крестистеся, во Хріста облекостеся}\footnote{Гал.~3,~27.}.\end{quotation}

О плотскомъ и духовномъ рожденіи, и силѣ и дѣйствіи ихъ сказано. Здѣ только полагается разсужденіе о утѣшительныхъ плодахъ духовнаго рожденія. А понеже лучше познается добро отъ сравненія со зломъ и плодами его, то и здѣ полагается сравненіе вожделѣнныхъ, утѣшительныхъ и радостныхъ плодовъ духовнаго рожденія съ плодами печальными ветхаго и плотскаго рожденія: да уразумѣеши яснѣе, возлюбленный хрістіанине, какого мы бѣдствія благодатію Хрістовою и милосердіемъ небеснаго Отца и дѣйствіемъ Святаго Духа свобождаемся, и какое блаженство получаемъ въ новомъ рожденіи; и тако отъ сердца Богу, такому Благодѣтелю, возблагодариши.

\paragraph*{§\:485.} Хрістіанамъ должно знать, что въ плотскомъ рожденіи, которое бываетъ отъ родителей, отца и матери, не иное что раждаемся, какъ сынове человѣчестіи отъ человѣкъ, и плоть отъ плоти, по писанному: \textit{рожденное отъ плоти плоть есть}\footnote{Іоан.~3,~6.}. Но въ духовномъ рожденіи раждаемся свыше, человѣцы отъ Бога, и вмѣсто сыновъ человѣческихъ сынове Божіи благодатію Его отраждаемся, якоже глаголетъ апостолъ: \textit{елицы пріяша Его} (Хріста), \textit{даде имъ область чадомъ Божіимъ быти, вѣрующимъ во имя Его, иже не отъ крове, ни отъ похоти плотскія, ни отъ похоти мужескія, но отъ Бога родишася}\footnote{Іоан.~1,~12 и 13.}. Въ плотскомъ рожденіи облекаемся въ ветхаго Адама\footnote{1~Кор.~15,~49.}: въ духовномъ облекаемся въ новаго "--- Хріста, по Писанію: \textit{елицы во Хріста крестистеся, во Хріста облекостеся}\footnote{Гал.~3,~27.}. Въ плотскомъ рожденіи раждаемся во грѣхахъ, якоже глаголетъ пророкъ: \textit{въ беззаконіихъ зачатъ есмь, и во грѣсѣхъ роди мя мати моя}\footnote{Пс.~50,~7.}: въ духовномъ омываемся, очищаемся, освѣщаемся и оправдаемся отъ сквернъ грѣховныхъ, и дѣлаемся праведными благодатію Хрістовою, якоже глаголетъ апостолъ вѣрнымъ: \textit{омыстеся, освятистеся, оправдистеся именемъ Господа нашего Іисуса Хріста, и Духомъ Бога нашего}\footnote{1~Кор.~6,~11.}, и служитель Божій новокрещенному глаголетъ: \textit{оправдался еси, просвѣтился еси, освятился еси, омылся еси именемъ Господа нашего Іисуса Хріста, и Духомъ Бога нашего}. Въ плотскомъ раждаемся чада тьмы, во области сатанинѣ: въ духовномъ раждаемся \textit{чада свѣта, и отъ тмы приводимся въ чудный Божій свѣтъ, и отъ области сатанины въ царство Хріста, Сына Божія}\footnote{1~Петр.~2,~9; Дѣян.~26,~18; Еф.~5,~8.}. Въ плотскомъ раждаемся \textit{естествомъ чада гнѣва}\footnote{2,~3.}: въ духовномъ чада благодати, чада благословенія Божія\footnote{1~Петр.~3,~6.}. Въ плотскомъ раждаемся къ временному животу: въ духовномъ къ \textit{вѣчному животу}\footnote{1~Тим.~6,~12.}. И тако въ плотскомъ рожденіи раждаемся ко всякому духовному бѣдствію, злополучію, какого умомъ понять и языкомъ изъяснить довольно невозможно: въ духовномъ отъ всего того зла свобождаемся, и приходимъ во всякое духовное блаженство, которое такожде ни умъ нашъ постигнуть, ни слово изъяснить довольно не можетъ. Отъ сего разумѣй, возлюбленный хрістіанине, что то есть вѣра и плоды ея, и что есть святое крещеніе, которымъ вновь раждаемся и вновь начинаемъ жить. Яко въ плотскомъ рожденіи раждаемся скаредными, скверными, проклятыми, осужденными, чадами гнѣва, тьмы, геенны и духовно мертвыми: въ духовномъ раждаемся чистыми, святыми, праведными, благословенными, чадами Божіими, чадами свѣта, наслѣдниками небеснаго царствія, и духовно живыми, и тако сотворяемся \textit{новою тварію}, по реченному: \textit{аще кто во Хрістѣ, нова тварь}\footnote{2~Кор.~5,~17.}.

\paragraph*{§\:486.} Здѣ разсуждай и удивляйся благости и человѣколюбію Божію, хрістіанине. Человѣка, отступника Своего, отверженнаго, грѣхомъ оскверненнаго, и къ противнику Его и врагу "--- сатанѣ уклонившагося, злаго его совѣта послушавшаго, и повелѣнія Создателя своего презрѣвшаго, въ такую достоинства высоту милосердо и человѣколюбно возвелъ! Удивляться, а не разумѣть можемъ сей Божіей къ намъ недостойнымъ благости, и съ пророкомъ глаголати: \textit{Господи! что есть человѣкъ, яко познался еси ему? или сынъ человѣчь, яко вмѣняеши его}\footnote{Пс.~143,~3.}? Аще бы царь земный злодѣя какого, бунтовщика, врага своего и отечества, заключеннаго въ темницѣ и къ смерти по законамъ приговореннаго, помиловалъ, грѣхъ простилъ и отъ смерти свободилъ, сверхъ того высокою почтилъ бы честію, и въ сына себѣ воспріялъ: сіе бы дѣло всему свѣту чудесно было, и во всѣхъ вѣкахъ со удивленіемъ слышалося бы. Человѣкъ, который возжелалъ божескія чести и Богомъ быти, учинился Божіимъ врагомъ, бунтовщикомъ и отступникомъ: почему судомъ правды Божія, непремѣняемыя, приговоренъ былъ въ вѣчную темницу и на смерть вѣчную осужденъ. Сію горести и страданія вѣчнаго чашу пить слѣдовало ему; яко правда Божія, грѣхомъ огорченная, того неотмѣнно требовала. Ибо правда Божія того требуетъ, чтобы и малѣйшій грѣхъ безъ наказанія не оставленъ былъ: яко всякій грѣхъ безконечное величество Божіе оскорбляетъ, "--- кольми паче такъ мерзкій и тяжкій богоотступства грѣхъ, который прародители наши въ раи учинили, и тако себе и насъ въ гнѣвъ и клятву Богу подвергли. Но Богъ, богатъ Сый въ милости, помиловалъ насъ, отъ такъ великаго бѣдствія избавилъ и въ такъ высокое достоинство, какого умъ нашъ постигнути не можетъ, благодатію Своею привелъ. Сію великую милость и благодать \textit{заслужилъ} намъ недостойнымъ и отверженнымъ Единородный Сынъ Божій, Іисусъ Хрістосъ, Господь нашъ; \textit{подаетъ} Отецъ небесный вѣрующимъ во имя Единороднаго Сына Его; \textit{совершаетъ} Духъ Святый. "--- Намъ, любезный хрістіанине, сподобившимся толикой милости Божіей должно: 1)~Всегда поминать великую сію благость, и отъ сердца искренняго благодарить Благодѣтелю; 2)~жити въ новомъ семъ и духовномъ рожденіи, то"=есть жити не по плоти, но по духу; помнить обѣщанія свои, учиненныя при крещеніи святомъ, которыми обязалися мы Богу вѣрою и правдою служить, святое послушаніе Ему показывать, безъ чего истинное благодареніе быть не можетъ, и крещеніе ничего не пользуетъ; 3)~которые безстрашіемъ и нерадѣніемъ погубили даръ святаго крещенія, обратитися всѣмъ сердцемъ къ Богу, каяться и молить Его, чтобы паки въ высокую Свою принялъ милость, да не во вѣки, въ діавольской власти оставльшеся, съ нимъ купно въ вѣчномъ огнѣ страдать будутъ.

\subsection[Глава 2-я. О томъ, что Богъ есть Отецъ хрістіанамъ истиннымъ.]{глава вторая.\\\bfseries О томъ, что Богъ есть Отецъ хрістіанамъ истиннымъ.}

\begin{quotation}\textit{И буду вамъ во Отца, и вы будете Мнѣ въ сыны и дщери, глаголетъ Господь Вседержитель}\footnote{2~Кор.~6,~18; Іер.~3,~19; Ос.~1,~10.}.\end{quotation}

\paragraph*{§\:487.} Хотя и всѣхъ людей, которыи и не знаютъ Бога, Богъ есть Отецъ, яко Создатель и Промыслитель всѣхъ, \textit{яко солнце Свое сіяетъ на злыя и благія, и дождитъ на праведныя и неправедныя}\footnote{Матѳ.~4,~45.}: однакожъ вѣрныхъ Своихъ, которые вѣруютъ въ Него, знаютъ Его единаго истиннаго Бога, почитаютъ Его, боятся и любятъ Его, и усердное Ему послушаніе тщатся показывать, особливымъ образомъ называется и есть Отецъ. Сіе человѣколюбіе Божіе, что Онъ называться и быть вѣрнымъ Своимъ Отцемъ благоволилъ, на многихъ святаго Писанія мѣстахъ во утѣшеніе наше изображается. Къ Нему Сынъ Божій молитися апостоловъ и чрезъ нихъ насъ научилъ, "--- \textit{сице молитеся: Отче нашъ, Иже еси на небесѣхъ! да святится имя Твое}, и проч.\footnote{6,~9 и слѣд.} Къ Нему Исаія молится: \textit{Ты еси Отецъ нашъ}\footnote{Ис.~63,~16.}. И самъ обѣщался вѣрнымъ быть во Отца: \textit{и буду вамъ во Отца}, и проч.\footnote{2~Кор.~6,~18.} И вѣрніи Ему вопіютъ: \textit{Авва Отче}\footnote{Римл.~8,~15.}! А когда Богъ Отецъ есть вѣрныхъ, то и вѣрныи сынове Его суть, якоже апостолъ глаголетъ: \textit{вси вы сынове Божіи есте вѣрою о Хрістѣ Іисусѣ. Елицы бо во Хріста крестистеся, во Хріста облекостеся}\footnote{Гал.~3,~26 и 27.}.

\paragraph*{§\:488.} Отъ сего человѣколюбія Божія источника проистекаетъ намъ грѣшнымъ, любезный хрістіанине, многоразличное утѣшеніе, которымъ, какъ хладного водою въ жаждѣ, въ приключающемся бѣдствіи прохлаждаемся. 1)~Отъ сего живаго и сладкаго источника проистекли столь многія и милостивыя Его обѣщанія, которыми пророческія и апостольскія книги преисполнены. 2)~Аще Богъ Отецъ нашъ есть, то кто на насъ? \textit{Аще Богъ по насъ, кто на ны?} глаголетъ богомудрый Павелъ\footnote{Римл.~8,~31.}. Аще бо Богъ Отецъ нашъ есть, то безъ сумнѣнія по насъ есть: который бо отецъ за сыновъ своихъ не стоитъ? Кто можетъ озлобити насъ, отъ руки Его восхитити, погубити, когда мы не восхощемъ? \textit{Мой еси ты: и аще преходиши сквозѣ воду, съ тобою есмь, и рѣки не покрыютъ тебе: и аще сквозѣ огнь прейдеши, не сожжешися, и пламень не опалитъ тебе}, глаголетъ чрезъ пророка къ вѣрному Своему Iакову\footnote{Ис.~43,~1 и 2.}. \textit{Съ нимъ есмь въ скорби, и изму его}\footnote{Пс.~90,~15.}. Аще Богъ Отецъ нашъ есть: чего не можемъ испросить у Него, когда по волѣ Его и съ вѣрою просити Его будемъ? \textit{Иже Сына Своего не пощадѣ, но за насъ всѣхъ предалъ Его: како не и съ Нимъ вся намъ даруетъ}\footnote{Римл.~8,~32.}? Отецъ плотскій, хотя бы и хотѣлъ просящимъ дѣтямъ просимое подать, но часто и не подаетъ; яко не все, что хощетъ, можетъ: Богъ Отецъ нашъ не тако. Какъ вся благая хощетъ, яко преблагій, тако и вся можетъ, яко всемогущій, подать намъ. Откуду повелѣно намъ всего просить у Него: \textit{просите, и дастся вамъ: ищите, и обрящете: толцыте, и отверзется вамъ}\footnote{Матѳ.~7,~7.}. \textit{Близъ бо Господь всѣмъ призывающимъ Его во истинѣ}\footnote{Пс.~144,~18.}. \textit{Волю боящихся Его сотворитъ, и молитву ихъ услышитъ, и спасетъ я}\footnote{Пс.~144,~19.}. "--- 4)~Сіе человѣколюбіе Его ободряетъ насъ за содѣянные грѣхи не отчаяватися, но съ вѣрою и надеждою толкать въ двери милосердія Его. Како бо отецъ надъ кающимися и съ сожалѣніемъ и слезами просящими прощенія не умилосердится? Откуду повелѣно къ Нему молитися: \textit{Отче! остави намъ долги наша, якоже и мы оставляемъ должникомъ нашимъ}\footnote{Матѳ.~6,~12.}. Отсюду и прочія утѣшенія въ нужномъ случаѣ можешь почерпати, хрістіанине. Сію благодать, умомъ нашимъ непостижимую, заслужилъ намъ Сынъ Божій, Іисусъ Хрістосъ, Который благоволилъ быть Сыномъ человѣческимъ, чтобы намъ Бога Отцемъ учинить, и Своего Отца нашимъ Отцемъ, и насъ, сыновъ человѣческихъ, сынами Божіими содѣлать, якоже пишется: \textit{елицы пріяша Его, даде имъ область чадомъ Божіимъ быти, вѣрующимъ во имя Его, иже не отъ крове, ни отъ похоти плотскія, ни отъ похоти мужескія, но отъ Бога родишася}\footnote{Іоан.~1,~12 и 13.}. "--- Отсюду, возлюбленный хрістіанине, поучайся и познавай: 1)~Коль великое есть къ намъ человѣколюбіе Его: яко Онъ, Царь царей и Господь господей, предъ Которымъ вси языцы какъ ничтоже, благоволилъ Отцемъ нарицатися и быть насъ бѣдныхъ, смиренныхъ, отверженныхъ и повинныхъ вѣчному осужденію, и принять Себѣ въ сыны и дщери. Сію неизреченную воистину любовь, которой и ангели святіи удивляются, святый апостолъ Іоаннъ разсуждая, тако со удивленіемъ глаголетъ: \textit{видите, какову любовь далъ есть Отецъ Намъ, да чада Божія наречемся, и есмы}\footnote{1~Іоан.~3,~1.}. "--- 2)~Какъ должно хрістіанамъ сей высокій титулъ и небесное достоинство помнить, и берещи паче всего міра сего сокровища и паче живота своего, и не прельщатися титулами міра сего и суетною славою, но довольствоваться тѣмъ единымъ, что Бога Отца имѣютъ, и суть сынове Божіи вѣрою о Хрістѣ Іисусѣ. "--- 3)~За сіе неизреченное человѣколюбіе Отцу небесному благодарить сердцемъ и устами, и прославлять и хвалить имя Его святое. "--- 4)~Любить Его отъ вѣры нелицемѣрныя, и усердное Ему отъ любви послушаніе показывать. "--- 5)~Любить другъ друга, яко братію, имѣющую единаго Отца"=Бога. "--- 6)~Удаляться отъ всякаго грѣха: яко тѣмъ благоутробный оный Отецъ оскорбляется, и оскорбляющій лишается отеческія Его милости, и праведному Его гнѣву подпадаетъ. "--- 7)~Отлучившимся и удалившимся отъ Него не должно медлить, но съ покаяніемъ и слезами возвратитися къ Нему, и съ глубокимъ смиреніемъ повергать себе предъ святѣйшими очами Его, и, признавая грѣхъ свой, исповѣдаться: \textit{Отче! согрѣшихъ на небо и предъ Тобою; и уже нѣсмь достоинъ нарещися сынъ Твой: сотвори мя яко единаго отъ наемникъ Твоихъ}, якоже блудный сынъ сотворилъ\footnote{Лук.~15,~19.}. "--- 8)~Возвратившимся не должно паки отлучаться, да не во вѣки отъ святѣйшаго Его лица и святой Его фамиліи "--- избранныхъ Божіихъ отвергнутся.

\subsection[Глава 3-я. О отечествѣ хрістіанскомъ.]{глава третія.\\\bfseries О отечествѣ хрістіанскомъ.}

\begin{quotation}\textit{Не имамы здѣ пребывающаго града, но грядущаго взыскуемъ}\footnote{Евр.~13,~14.}.\end{quotation}
\begin{quotation}\textit{Наше житіе на небесѣхъ есть, отонудуже и Спасителя ждемъ, Господа нашего Іисуса Хріста}\footnote{Филип.~3,~20.}.\end{quotation}

\paragraph*{§\:489.} Хрістіанъ въ мірѣ семъ житіе не ино что есть, какъ странствованіе и непрестанное шествіе и спѣшеніе къ отечеству своему, якоже глаголетъ апостолъ: \textit{не имамы здѣ пребывающаго града, но грядущаго взыскуемъ}. Откуду нарицаются и суть странники и пришельцы въ мірѣ семъ, якоже о Авраамѣ, Исаакѣ и Іаковѣ пишется: \textit{яко странніи и пришельцы быша на земли}\footnote{Евр.~11,~13.}. И Давидъ глаголетъ: \textit{яко пресельникъ азъ есмь у Тебе и пришлецъ, якоже вси отцы мои}\footnote{Пс.~38,~13.}. И на другомъ мѣстѣ: \textit{пришлецъ азъ есмь на земли}\footnote{118,~19.}. Аще убо здѣшнее житіе есть странствованіе наше, то непремѣнно есть отечество, отъ Бога Создателя нашего намъ уготованное, въ которомъ по странствованіи семъ упокоеваться должно.

\paragraph*{§\:490.} Хрістіанское отечество есть небо, гдѣ слава небеснаго Отца, Которому они молятся: \textit{Отче нашъ, Иже еси на небесѣхъ}; гдѣ преславный домъ Его, въ которомъ \textit{обители многи суть}\footnote{Іоан.~14,~2.}, гдѣ \textit{градъ великій святый Іерусалимъ, имущъ славу Божію, и градъ не требуя солнца, и луны, да свѣтятъ въ немъ: слава бо Божія просвѣти его, и свѣтильникъ ему "--- Агнецъ}\footnote{Апок.~21,~10 и 23.}. Тамо Отецъ небесный радостотворитъ зрѣніемъ святѣйшаго лица Своего сыновъ Своихъ, которыи отсюду по многобѣдственномъ странствованіи преселилися; тамо водворяется праотецъ ликъ съ праведнымъ своимъ сѣменемъ: тамо ликуютъ пророцы, апостоли и святители "--- проповѣдники истины; тамо торжествуетъ пресвѣтлое мученикъ воинство; тамо упокоевается преподобныхъ и подвижниковъ множество; тамо сіяетъ незнающее запада Солнце правды "--- Хрістосъ, и всѣхъ блистаніемъ славы Своея просвѣщаетъ, теплѣйшими любве Своея лучами согрѣваетъ, и благодатію Своею влечетъ къ Себѣ всѣхъ, въ мірѣ семъ странствующихъ и имя Его святое нарицающихъ и почитающихъ; тамо престолъ славы окружаютъ тысящи тысящей и тьмы темъ небесныхъ силъ безплотныхъ, и Господа, Царя славы, поютъ непрестанно: \textit{святъ, святъ, святъ Господь Саваоѳъ}\footnote{Ис.~6,~3.}. Въ ономъ отечествѣ нѣтъ страха отъ иноплеменниковъ, нѣтъ боязни отъ враговъ, нѣтъ опасности отъ болѣзни, смерти, глада, хлада, нищеты, вражды, ненависти, злобы и прочіихъ золъ: не слышится тамо жалоба; удалился оттуду плачь, печаль и воздыханіе; нѣтъ попеченія о пищѣ, питіи, одеждѣ, о домѣ и домашнихъ; нѣтъ тамо бури и непогоды, но всегда благопріятное ведро; нѣтъ утра, вечера, нощи, но всегда день; нѣтъ осени и зимы, но всегда красная и благорастворенная весна и лѣто; не слышится, ни видится, ни чувствуется тамо ничего, развѣ благопріятное, веселое, и любезное. Тамо жители всегда бдятъ, но никогда не утруждаются; всегда живутъ, но смерти не чаютъ: тамо житіе *безъ смерти, покой* безъ труда, радость безъ печали, здравіе безъ немощи, богатство безъ потерянія, честь безъ опасности и страха, довольствіе безъ оскудѣнія, блаженство безъ бѣдствія; тамо день безъ нощи, ведро безъ непогоды, солнце безъ облаковъ, свѣтъ безъ тьмы, сіяніе безъ мглы; нѣтъ тамо стара, хрома, слѣпа, разслабленнаго, безобразнаго, но вси въ цвѣтущей юности, красной добротѣ и возрастѣ мужа совершенна, \textit{въ мѣрѣ возраста исполненія Хрістова}. Тамо никто ни обидитъ, ни обиждается, ни ненавидитъ, ни ненавидимъ бываетъ, ни гнѣвается, ни досаждаетъ, не озлобляетъ, ни озлобляемъ бываетъ; никто никому не завидитъ, всякъ доволенъ тѣмъ, что имѣетъ: понеже болѣе того, что имѣетъ, не желаетъ; яко получилъ и довольствуется тѣмъ добромъ, котораго болѣе нѣтъ; тѣмъ утѣшается блаженствомъ, которое выше не восходитъ; тою славою и честію вѣнчается, которой болѣе не ищетъ. Въ ономъ преславномъ и блаженнѣйшемъ гражданствѣ совершеннѣйшая тишина, миръ, любовь между блаженными гражданами, другъ о другѣ радость, утѣха и веселіе; яко другъ друга любятъ какъ себе, и другъ о другѣ радуются какъ о себѣ, яко другъ друга видятъ въ блаженствѣ, какъ и себе. Въ сіе отечество, возлюбленный хрістіанине, отъ многомятежнаго міра сего преселяются хрістіане, которые вѣрою и любовію работаютъ небесному Отцу, Царю небесе и земли и всея твари. Сіе отечество прародители наши заключили себѣ и намъ преслушаніемъ своимъ: но Хрістосъ, Сынъ Божій, Иже насъ ради на земли благоволилъ странствовать, Своимъ вольнымъ послушаніемъ отворилъ, \textit{послушливъ бывъ даже до смерти, смерти же крестныя}\footnote{Филип.~2,~8.}. Мы, любезный хрістіанине, еще странники и пришельцы есмы на земли чуждей, еще на морѣ міра сего плаваемъ, и волнами житейскими обуреваемся; еще издалече, изъ юдоли сей плачевной, какъ изъ плѣненія Вавилонскаго на градъ Іерусалимъ, на отечество оное, горній оный Іерусалимъ, плачевнымъ вѣры окомъ взираемъ и \textit{воздыхаемъ, въ жилище наше небесное облещися желающе}\footnote{2~Кор.~5,~2.}. Того ради 1)~имѣемъ нужду трудитися, подвизатися, терпѣть различныя въ странствованіи бѣдствія; опасаться отъ враговъ, которые окружаютъ насъ, и тщатся заградить путь и къ отечеству оному не допустить, берещися козней діавольскихъ, прелестей міра и похотей плотскихъ, \textit{яже воюютъ на душу}\footnote{1~Петр.~2,~11.}. Откуду должно часто \textit{возводить очи наши къ Живущему на небеси}\footnote{Пс.~122,~1.}, и просить помощи отъ Него, да укрѣпитъ насъ въ подвигѣ нашемъ, и подастъ намъ \textit{теченіе скончать, вѣру соблюсти} и пріити въ оную \textit{страну живыхъ}\footnote{144,~8.}. "--- Аще отечество наше на небеси, хрістіанине, то должно намъ горняя мудрствовать, а не земная, по ученію апостольскому: \textit{горняя мудрствуйте, а не земная}\footnote{Кол.~3,~2.}. "--- Аще странствованіе есть житіе хрістіанъ на земли: почтожъ толико хрістіане припасаютъ здѣ стяжаній, аки бы во вѣки имѣли здѣ жить? почто столько созидаютъ домовъ и украшаютъ ихъ? почто столько притяживаютъ земель, вотчинъ, крестьянъ? почто столько собираютъ богатства и сокрываютъ сокровищъ на земли? почто умножаютъ злата, сребра, каменій драгихъ? почто толико отягощаютъ убогую душу свою въ странствованіи? Вся бо сія оставятъ на земли, и ничего съ собою не возмутъ, якоже апостолъ глаголетъ: \textit{ничтоже внесохомъ въ міръ сей, явѣ, яко ниже изнести что можемъ}\footnote{1~Тим.~6,~7.}, паче же и самое тѣло оставимъ въ землѣ погребенное, которое въ міръ сей внесохомъ. Едиными душами отъидемъ къ опредѣленному отъ Бога мѣсту. "--- 4)~Таковымъ хрістіанамъ, которыи хотятъ богатѣть въ мірѣ семъ, надобно внимать, что апостолъ глаголетъ: \textit{хотящіи богатитися, впадаютъ въ напасти и сѣть и въ похоти многи несмысленны и вреждающія, яже погружаютъ человѣки во всегубительство и погибель}\footnote{1~Тим.~6,~9.}, "--- и Хрістосъ Богъ съ высоты славы Своея отвѣщаваетъ: \textit{безумне! въ сію нощь душу твою истяжутъ отъ тебе, а яже уготовалъ еси, кому будутъ}\footnote{Лук.~12,~20.}? и опасаться, чтобы не лишиться небеснаго и блаженнѣйшаго онаго гражданства.

\subsection[Глава 4-я. О наслѣдіи хрістіанскомъ.]{глава четвертая.\\\bfseries О наслѣдіи хрістіанскомъ.}

\begin{quotation}\textit{Пріидите, благословенніи Отца Моего! наслѣдуйте уготованное вамъ царствіе отъ сложенія міра}\footnote{Матѳ.~25,~34.}.\end{quotation}
\begin{quotation}\textit{Аще чада и наслѣдницы: наслѣдницы убо Богу, снаслѣдницы же Хрісту}, и проч.\footnote{Римл.~8,~17 и слѣд.}\end{quotation}

\paragraph*{§\:491.} Какъ отечество хрістіанъ есть не отъ міра сего, яко они странники и пришельцы суть въ мірѣ семъ, якоже въ мимошедшей главѣ сказано: такъ и наслѣдіе ихъ не есть въ мірѣ семъ, но на иномъ мѣстѣ, гдѣ отечество и домъ ихъ имѣется. Якоже убо отечество хрістіанское есть небо, тамо и наслѣдіе ихъ есть, на небеси. Тамо они наслѣдятъ, по Писанію, \textit{жизнь вѣчную}\footnote{Матѳ.~19,~29.}, наслѣдятъ \textit{обители Отца небеснаго}\footnote{Іоан.~14,~2.}, наслѣдятъ \textit{покой}\footnote{Евр.~4,~1 и 11.}, наслѣдятъ \textit{пребываніе со Хрістомъ, да видятъ славу Его}\footnote{Іоан.~17,~24.}; наслѣдятъ \textit{лицезрѣніе Божіе, яко Бога узрятъ якоже есть}\footnote{1~Іоан.~3,~2.}; наслѣдятъ \textit{вѣнецъ неистлѣненъ}\footnote{1~Кор.~9,~25.}, наслѣдятъ "--- \textit{вѣнецъ правды}\footnote{2~Тим.~4,~8.}, "--- \textit{вѣнецъ жизни}\footnote{Іак.~1,~12; Апок.~2,~10.}, "--- \textit{вѣнецъ славы}\footnote{1~Петр.~5,~4.}; наслѣдятъ \textit{царство и славу Божію}\footnote{1~Сол.~2,~12.}, "--- \textit{уготованное имъ царствіе отъ сложенія міра}\footnote{Матѳ.~25,~34.}; наслѣдятъ тое, что \textit{будутъ Богу подобни}\footnote{1~Іоан.~3,~2.}; наслѣдятъ наконецъ \textit{благая, ихже око не видѣ, и ухо не слыша, и на сердце человѣку не взыдоша}\footnote{1~Кор.~2,~9.}. Откуду называются истинныи хрістіане \textit{наслѣдницы убо Богу, снаслѣдницы же Хрісту}\footnote{Римл.~8,~17.}. И тако видиши, возлюбленный хрістіанине, что \textit{наслѣдіе} хрістіанское есть \textit{нетлѣнно и нескверно и неувядаемо, соблюдено на небесѣхъ ихъ ради}\footnote{1~Петр.~1,~4.}. А коликое оно и какое, не токмо описать, но и умомъ понять не можемъ, но токмо вѣрою видимъ нынѣ, \textit{якоже зерцаломъ въ гаданіи}\footnote{1~Кор.~13,~12.}, и съ богомудрымъ Павломъ, который \textit{восхищенъ былъ до третіяго небесе и слышалъ неизреченные глаголы}\footnote{2~Кор.~12,~2 и 4.}, исповѣдуемъ и проповѣдуемъ, яко \textit{ихже око не видѣ, и ухо не слыша, и на сердце человѣку не взыдоша, яже уготова Богъ любящимъ Его}. Примѣчаемъ нѣсколько изъ настоящихъ будущая, и изъ видимыхъ невидимая, и изъ тлѣнныхъ нетлѣнная. Аще бо ради сего тѣла нашего тлѣннаго и смиреннаго сотворилъ человѣколюбивый Богъ такъ великая благая на небеси и на земли, на воздухѣ и морѣ, которыхъ нѣтъ числа: коль великая будутъ оная благая, яже наслѣдятъ избранніи Божіи въ небесномъ отечествѣ, \textit{гдѣ узрятъ Его лицемъ къ лицу}\footnote{1~Кор.~13,~12.}! Аще въ странствованіи и изгнаніи, юдоли плачевной, въ темницѣ, толикая подаетъ имъ: коликая подастъ въ дому Своемъ, гдѣ слава Его сіяетъ! Аще добрымъ и злымъ, праведнымъ и грѣшнымъ солнце Свое сіяетъ: коликая будутъ оная благая, яже единымъ добрымъ уготована суть! Аще толикихъ Его даровъ наслаждаются друзи Его и врази: коликая будутъ, коль великая, безчисленная, славная и ужасная, которая единымъ другомъ подастъ въ наслажденіе! Аще во днехъ плача толикое утѣшеніе: коликое будетъ во дни брака! \textit{Коль многое множество благости Твоея, Господи Боже, сокрылъ еси боящимся Тебе!} глаголетъ въ пѣсни своей къ Нему Псаломникъ\footnote{Пс.~30,~20.}. Тогда наслѣдятъ они всѣхъ желаній конецъ. Желаютъ люди свѣтлости: тогда \textit{праведницы просвѣтятся, яко солнце, во царствіи Отца ихъ}\footnote{Матѳ.~13,~43.}. Желаютъ люди легкости и крѣпости: тогда \textit{будутъ яко ангели Божіи}\footnote{22,~30.}. Желаютъ люди нетлѣнія и цѣлаго здравія: тогда будутъ \textit{нетлѣнни: яко сѣется въ тлѣніе, востаетъ въ нетлѣніи}\footnote{1~Кор.~5,~42 и 52.}. Желаютъ люди силы: тогда будутъ сильны: яко \textit{сѣется въ немощи, востаетъ въ силѣ; сѣется тѣло душевное, востаетъ тѣло духовное}\footnote{43 и 44.}. Желаютъ люди славы: тогда будутъ \textit{въ славѣ}: яко \textit{сѣется не въ честь, востаетъ въ славѣ}\footnote{43.}. Желаютъ люди торжества: тогда торжествовати будутъ надъ смертію и адомъ. \textit{Гдѣ ти, смерте, жало? Гдѣ ти, аде, побѣда}\footnote{55.}? Желаютъ люди долгаго живота: тогда будетъ \textit{вѣчный животъ}: яко \textit{праведницы во вѣки живутъ}\footnote{Прем.~5,~15.}. Желаютъ люди сытости: тогда \textit{насытятся, егда явится имъ слава Господня}\footnote{Пс.~16,~15.}. Желаютъ люди напоенія: тогда \textit{упіются отъ тука дому Божія, и потокомъ сладости Его напоятся}\footnote{35,~9.}. Желаютъ люди сладкаго пѣнія: тогда услышатъ ангельское пѣніе, и сами безъ конца и труда будутъ пѣть пресладкую пѣснь: \textit{Аллилуіа}\footnote{Апок.~19,~1.}. Желаютъ люди разума и премудрости: тогда премудрость Божія открываетъ имъ самую себе. Желаютъ люди дружества: тогда будутъ любить Бога паче себе самихъ, и другъ друга какъ себе самихъ, и Богъ ихъ будетъ любить какъ чадъ Своихъ. Желаютъ люди согласія и мира: тогда будетъ имъ воля и хотѣніе едино, яко вся будутъ хотѣть, что единъ Богъ хощетъ. Желаютъ люди радости и веселія: тогда \textit{радостію возрадуются о Господѣ}\footnote{Ис.~61,~9.}, \textit{возрадуется сердце ихъ, и радости ихъ никтоже возметъ отъ нихъ}\footnote{Іоан.~16,~22.}. Желаютъ люди богатства: тогда \textit{вѣрніи и благіи раби Господни надъ многими поставлены будутъ}\footnote{Матѳ.~25,~11.}. Желаютъ люди царства: тогда \textit{наслѣдятъ уготованное царствіе отъ сложенія міра}\footnote{ст.~34.}. Желаютъ люди Бога видѣти: тогда \textit{узрятъ Бога лицемъ къ лицу}\footnote{1~Кор.~13,~12.}, Который \textit{будетъ всяческая во всѣхъ}\footnote{15,~28.}. Желаютъ люди никогда не лишиться того добра, которое имѣютъ: тогда такъ о томъ извѣстны будутъ, что не лишатся того никогда, какъ извѣстны о томъ, что сами того не хотятъ. Сіе есть, возлюбленный хрістіанине, наслѣдіе хрістіанъ, которое имъ уготовалъ Отецъ ихъ небесный благодатію Единороднаго Сына Своего, Господа нашего Іисуса Хріста! Сіе наслѣдіе потеряли"=было мы грѣхами своими, но послушаніемъ и трудами Сына Божія паки возвратилося оно къ намъ. Входятъ въ тое хрістіане, \textit{подвигомъ добрымъ подвизающіися, теченіе скончавающіи и вѣру соблюдающіи}\footnote{2~Тим.~4,~7.}. "--- Примѣчай, хрістіанине: 1)~Коль великое, славное и воистину умомъ непостижимое наслѣдіе, истиннымъ хрістіанамъ на небесѣхъ уготованное. 2)~Какъ усердно тщаться и подвизаться намъ должно, да постигнемъ и мы оное благодатію Спаса нашего Іисуса Хріста, по увѣщанію богомудраго Павла: \textit{тако тецыте, да постигнете}\footnote{1~Кор.~9,~24.}. 3)~Какъ опасно берещися отъ любви мірскихъ вещей: чести, богатства, славы, сладострастія и грѣховъ, которые воспящаютъ и отвлекаютъ насъ отъ блаженнаго онаго наслѣдія. 4)~Какъ усердно благодарить Богу, уготовавшему намъ толикое блаженство, и молить Его, дабы Самъ Онъ въ подвигѣ вѣры помоглъ намъ, и укрѣпилъ до конца, и тако вѣру соблюдшимъ отворилъ оное.

\paragraph*{§\:492.} Когда"=де хрістіанамъ толикое наслѣдіе уготовано на небеси: почто имъ толико заготовлять сокровищъ на земли, столько собирать богатства, имѣній, стяжаній? Вѣдь имъ тое проповѣдуется словомъ Божіимъ, открываетъ святое Евангеліе, утверждаетъ вѣра. Да и оставить принуждены будутъ съ животомъ временнымъ и наслѣдіе временное. Хрістосъ имъ не велитъ того дѣлать: \textit{не скрывайте себѣ сокровищъ на земли, идѣже червь и тля тлитъ, идѣже татіе подкопываютъ и крадутъ}\footnote{Матѳ.~6. 19.}. И которые хрістіане слушаютъ Хріста, тѣ не дѣлаютъ того, но, оставивше земная, къ небеснымъ стремятся, \textit{горняя мудрствуютъ, а не земная}, по увѣщанію великаго Павла\footnote{Кол.~3,~2.}. Къ симъ и Хрістосъ глаголетъ: \textit{не бойся, малое стадо! яко благоизволи Отецъ вашъ дати вамъ царство}\footnote{Лук.~12,~32.}. Къ онымъ глаголетъ апостолъ: \textit{не любите міра, ни яже въ мірѣ. Аще кто любитъ міръ, нѣсть любве Отчи въ немъ, яко все, еже въ мірѣ, похоть плотская, и похоть очесъ, и гордость житейская, нѣсть отъ Отца, но отъ міра сего есть}\footnote{1~Іоан.~2,~15--16.}. Тако они, прилагая сердце къ наслѣдію земному, "--- \textit{идѣже бо сокровище ихъ, ту и сердце ихъ есть}, по словеси Хрістову\footnote{Матѳ.~6,~21.}, "--- не мудрствуютъ горнихъ, и не пекутся о нихъ, "--- и надобно опасаться, чтобы и не лишиться тѣхъ съ воздыханіемъ и плачемъ безполезнымъ. Къ собирающимъ бо, а не въ Бога богатѣющимъ, глаголетъ Богъ: \textit{безумне! въ сію нощь душу твою истяжутъ отъ тебе: а яже уготовалъ еси, кому будутъ}\footnote{Лук.~12,~20.}? Намъ же, любезный хрістіанине, по словеси Хрістову, должно \textit{искать прежде царствія Божія и правды Его, и сія вся приложатся намъ}\footnote{Матѳ.~6,~33.}, "--- и помнить апостольское слово: \textit{ничтоже внесохомъ въ міръ сей, явѣ, яко ниже изнести что можемъ, имѣюще же пищу и одѣяніе, сими довольни будемъ}\footnote{1~Тим.~6,~7 и 8.}.

\subsection[Глава 5-я. О достоинствѣ хрістіанскомъ.]{глава пятая.\\\bfseries О достоинствѣ хрістіанскомъ.}

\begin{quotation}\textit{Вы родъ избранъ, царское священіе, языкъ святъ, люди обновленія, яко да добродѣтели возвѣстите изъ тмы васъ призвавшаго въ чудный Свой свѣтъ}\footnote{1~Петр.~2,~9.}.\end{quotation}

\paragraph*{§\:493.} Якоже Хрістосъ, Сынъ Божій, во днехъ плоти Своея и смиренія умаленъ и безчестенъ былъ паче всѣхъ сыновъ человѣческихъ, якоже о Немъ пророкъ предвозвѣстилъ: \textit{видъ Его безчестенъ, умаленъ паче всѣхъ сыновъ человѣческихъ}\footnote{Ис.~53,~3.}, "--- и Евангеліе о томъ свидѣтельствуетъ: тако и хрістіане, которые сердцемъ вѣруютъ въ Него и любовію послѣдуютъ Ему, не ино что, какъ презрѣніе, поношеніе, уничиженіе злому міру суть. Но, что до внутренняго ихъ надлежитъ достоинства и чести, нѣтъ ихъ ничего выше. Нынѣ тое достоинство не видно, яко вѣрою, а не тѣлесными очами зрится; но открыется всему міру, \textit{егда тлѣнное сіе облечется въ нетлѣніе и смертное сіе облечется въ безсмертіе}\footnote{1~Кор.~15,~54.}. Святое Божіе слово достоинство ихъ открываетъ, и въ утѣшеніе представляетъ намъ: 1)~Богъ ихъ Отецъ, Которому молятся: \textit{Отче нашъ, Иже еси на небесѣхъ}\footnote{Матѳ.~6,~9.}, и они суть чада Божія вѣрою о Хрістѣ Іисусѣ, якоже глаголетъ апостолъ: \textit{вси вы сынове Божіи есте вѣрою о Хрістѣ Іисусѣ}\footnote{Гал.~3,~26.}. За великое почитаетъ міръ быть чадомъ царя земнаго, но коль несравненно высочайшее достоинство есть быть чадомъ Бога Вышняго, Царя небесе и земли; Отца имѣть Бога безначальнаго, безконечнаго, безсмертнаго, неописаннаго, всемогущаго, Царя царствующихъ и Господа господствующихъ, "--- умъ человѣческій сего титула не постигнетъ! Вся слава міра сего предъ симъ ничтоже. На сіе взирая, таинъ Божіихъ зритель, Іоаннъ святый удивляется и глаголетъ вѣрнымъ: \textit{видите, какову любовь далъ есть Отецъ намъ, да чада Божія наречемся, и есмы}\footnote{1~Іоан.~3,~1.}. "--- 2)~Братія Хрістовы суть истинные хрістіане, якоже Самъ \textit{не стыдится братіею нарицати ихъ, глаголя: возвѣщу имя Твое братіи Моей}\footnote{Евр.~2,~11 и 12; Пс.~21,~23.}. И къ мѵроносицамъ глаголетъ: \textit{идите возвѣстите братіи Моей}\footnote{Матѳ.~28,~10.}, "--- и къ Маріи Магдалинѣ: \textit{иди ко братіи Моей, и рцы имъ: восхожду ко Отцу Моему и Отцу вашему, и Богу Моему и Богу вашему}\footnote{Іоан.~20,~17.}. Смотри, любезный хрістіанине, въ какую высоту вознесены хрістіане: Господу небесе и земли братія суть! Высшее достоинство отъ сего быть не можетъ. Но и человѣколюбіе неисповѣдимо Сына Божія! Удивительно, когда царь земный подданныхъ своихъ братіею своею нарицаетъ, хотя и самъ такожде человѣкъ, какъ и подданныи. Хрістосъ, Царь небесный и Сынъ Божій, человѣковъ братіею Своею не стыдится нарицати: кто сіе человѣколюбіе Сына Божія и честь сію братіи постигнути можетъ? Отъ сего да постыдятся господа, которые не хотятъ рабовъ своихъ, подобныхъ себѣ людей, за братію себѣ имѣти, но вмѣсто подножія ихъ имѣютъ. "--- 3)~Общеніе имѣютъ со Отцемъ и Сыномъ Его Іисусомъ Хрістомъ, якоже глаголетъ апостолъ: \textit{общеніе наше со Отцемъ и съ Сыномъ Его Іисусомъ Хрістомъ}\footnote{1~Іоан.~1,~3.}. И паки Павелъ святый: \textit{вѣренъ Богъ, Имже звани бысте во общеніе Сына Его Іисуса Хріста, Господа нашего}\footnote{1~Кор.~1,~9.}. Сего ради и другами Своими вѣрныхъ Своихъ не постыдится нарицати: \textit{вы друзи Мои есте, аще творите, елика Азъ заповѣдаю вамъ. Не ктому васъ глаголю рабы: яко рабъ не вѣсть, что творитъ Господь его: васъ же рекохъ други: яко вся, яже слышахъ отъ Отца Моего, сказахъ вамъ}\footnote{Іоан.~15,~14 и 15.}. Все и вышереченное и сіе благородіе августѣйшее есть, высшее человѣческаго слова и понятія, высшее всякаго царскаго и монаршескаго міра сего титула. Не придаю болѣе къ сему словъ, хрістіанине любезный, яко умъ притупляется и слово оскудѣваетъ. Разсуждай сія вѣрою и утѣшайся благодатію и человѣколюбіемъ Бога нашего, Который тѣмъ болѣе почтилъ человѣка, чѣмъ болѣе обезчестился человѣкъ. "--- 4)~Хрістіане суть уды Хрістовы, якоже глаголетъ апостолъ: \textit{уди есмы тѣла Его, отъ плоти Его, и отъ костей Его}\footnote{Еф.~5,~30.}. И паки къ вѣрнымъ Коринѳяномъ глаголетъ: *\textit{не вѣсте ли, яко тѣлеса ваша удове Хрістовы суть?}* И паки: \textit{вы есте тѣло Хрістово, и уди отъ части}\footnote{1~Кор.~12,~27.}. Колико сіе есть быть живымъ удомъ тѣла Хрістова и имѣть себѣ Главу пренебесную "--- Хріста! Сіе же соединеніе вѣрныхъ со Хрістомъ, яко удовъ со главою своею, разумѣется таинственное и духовное. "--- Хрістіане суть жилища Божія, якоже о томъ свидѣтельствуетъ святое Писаніе. \textit{Аще кто любитъ Мя, слово Мое соблюдетъ: и Отецъ Мой возлюбитъ его, и къ нему пріидемъ, и обитель у него сотворимъ}, глаголетъ Хрістосъ\footnote{Іоан.~14,~23.}. И апостолъ: \textit{не вѣсте ли, яко храмъ Божій есте, и Духъ Божій живетъ въ васъ}\footnote{1~Кор.~3,~16.}? И паки: \textit{или не вѣсте, яко тѣлеса ваша храмъ живущаго въ васъ Святаго Духа суть, Егоже имате отъ Бога, и нѣсте свои}\footnote{6,~19.}. И паки: \textit{вы есте церкви Бога жива, якоже рече Богъ: яко вселюся въ нихъ, и буду имъ Богъ, и тіи будутъ Мнѣ людіе}\footnote{2~Кор.~6,~16.}. И на прочіихъ мѣстахъ о томъ свидѣтельствуется. О коль велико есть и сіе хрістіанъ преимущество, что они суть жилища Пресвятыя Троицы и храмъ Бога живаго! За велико почитается принять въ домъ монарха земнаго: какъ несравненно большее дѣло есть принять человѣку тлѣнному, землѣ и персти, Бога, небеснаго Царя, и не токмо принять, но и живущаго Его у себе имѣть! Сіе не ино что, какъ \textit{царствіе Божіе внутрь себѣ имѣти}\footnote{Лук.~17,~21.}. Блаженно и любезно тое сердце, которое сіе небесное сокровище сподобилося стяжать! Помышляй сіе, любезный хрістіанине, и удивляйся благости Бога нашего; Богъ въ человѣкѣ обитаетъ!.. и познавай, что то есть истинный хрістіанинъ. Сіе есть его благородіе, достоинство и преимущество! Какого тамо добра и блаженства не будетъ, гдѣ высочайшее и безконечное Добро?! Хрістіане таинственно причащаются тѣла и крови Хрістовой, и симъ духовнымъ брашномъ и питіемъ души свои питаютъ. Коль и сіе велико, хрістіанине, можешь познать, когда разсудишь, кто и Кому пріобщаешися! "--- 7)~Хрістіане въ молитвѣ бесѣдуютъ съ Богомъ. За немало почитается бесѣдовать съ царемъ земнымъ: малое ли дѣло бесѣдовать человѣку съ Богомъ, земному съ Небеснымъ, и созданію съ Создателемъ, и смертному съ Безсмертнымъ, и рабу съ Высочайшимъ Господемъ?! "--- 8)~Хрістіанамъ служить ангеламъ Богъ повелѣлъ: \textit{не вси ли суть служебніи дуси, въ служеніе посылаеми за хотящихъ наслѣдовати спасеніе}\footnote{Евр.~1,~14.}? "--- и хранятъ ихъ, и ополчаются окрестъ ихъ: \textit{ополчится ангелъ Господень окрестъ боящихся Его}\footnote{Пс.~33,~8.}. Аще бы какій монархъ, призвавши къ себѣ какого подданнаго, и учредивши довольно, отпустилъ въ домъ, и далъ бы ему слугъ своихъ ради провожденія и охраненія: какъ бы не за велико почиталъ сію милость монаршую подданный его? Богъ, Царь небесный, хрістіанъ призвавши въ вѣчное Свое царство, даетъ имъ слугъ Своихъ "--- ангеловъ святыхъ ради храненія, пока по пути міра сего странствуютъ, и велитъ имъ окрестъ ихъ ополчатися. Чудна и сія Божія милость къ человѣку! Велико и въ семъ хрістіанское достоинство есть, что ангели Божіи \textit{въ служеніе имъ посылаются}! "--- 9)~Высокое и совсѣмъ небесное хрістіанъ благородіе и отъ отечества, которое есть \textit{небо}\footnote{Филип.~3,~20.}, и отъ наслѣдія, которое есть \textit{нетлѣнно, неувядаемо и вѣчно на небесѣхъ}\footnote{1~Петр.~1,~4.}, познавается, о чемъ выше сказано, "--- Сіе есть благородіе хрістіанское, любезный хрістіанине! Сіе достоинство вѣрныхъ рабовъ Божіихъ! Сіе преимущество вѣрою и любовію почитающихъ Бога Вышняго! Сіе ихъ богатство, честь и слава и сокровище неоцѣненное, которое они \textit{внутрь себя} носятъ! Какое богатство, честь, слава, титулъ, благородіе, достоинство, преимущество міра сего съ симъ ихъ сокровищемъ сравнитися можетъ? Какъ блато противу злата, какъ гной противу благовонныхъ ароматъ, какъ тьма противу свѣта, или паче, какъ ничто противу всякаго добра вмѣняется. Признаешь сіе за истину, когда вѣрою вообразишь славу чадъ Божіихъ. Сію высочайшую благодать \textit{заслужилъ} человѣкамъ Сынъ Божій Іисусъ Хрістосъ, \textit{Иже даде область чадомъ Божіимъ быти вѣрующимъ во имя Его, иже не отъ крове, ни отъ похоти плотскія, ни отъ похоти мужескія, но отъ Бога родишася}\footnote{Іоан.~1,~12 и 13.}, "--- \textit{подаетъ} Отецъ небесный, \textit{совершаетъ} Духъ Святый. "--- Отсюду послѣдуютъ слѣдующая: 1)~Разсужденіе неизреченныя надъ нами явившіяся благости Божіей, которая показалася чрезъ воплощеніе Единороднаго Сына Его, Господа нашего Іисуса Хріста. Сѣдящихъ во тьмѣ и сѣни смертной, во власти діавольской и челюстехъ адовыхъ, и вѣчной погибели повинныхъ, Богъ, богатъ сый въ милости, отъ смерти къ животу, отъ бѣдственной тьмы къ блаженному свѣту, отъ вѣчной погибели къ вѣчному спасенію, и отъ толикаго окаянства къ толикому достоинству возвратилъ, не по заслугамъ нашимъ, но по единой Своей благодати, человѣколюбію и милосердію. "--- 2)~Какъ за сіе человѣколюбіе Ему мы обдолжены! коль усердное благодареніе воздавать должны не токмо устами, но и сердцемъ и усерднымъ послушаніемъ, безъ котораго не можетъ быть истинное благодареніе! Видишь, возлюбленный хрістіанине, хотя и за вся благая, которая подаетъ по благости Своей намъ, какъ"=то: за пищу, питіе, одежду и прочая, къ житію временному нужная, но паче за сіе, что намъ толикую явилъ Свою благодать, и позвалъ насъ изъ тьмы въ чудный Свой свѣтъ, "--- отъ всего сердца и непрестанно благодарить должно, что бываетъ отъ повседневнаго о той Его высокой милости размышленія, какъ о томъ выше сказано. "--- 3)~Какъ хрістіанамъ, которыи такъ въ чудный Божій свѣтъ позваны, должно берещися отъ всякаго грѣха! \textit{Не призва бо насъ Богъ на нечистоту, но во святость}\footnote{1~Сол.~4,~7.}. "--- 4)~Какъ должно усердными быть къ послушанію небеснаго Отца и къ творенію добрыхъ дѣлъ, къ чему такъ сильно увѣщаваютъ и поощряютъ насъ апостоли святіи, и отцы наши, святіи учители церковныи? "--- 5)~Видиши самъ, любезный хрістіанине, что худо и безумно дѣлаютъ тіи хрістіане, которые ищутъ прославитися на землѣ, ищутъ чести и славы отъ человѣкъ, ищутъ надъ другими господствовати и почитатися и покланятися отъ человѣкъ. Симъ они показуютъ, что не знаютъ общаго истиннымъ хрістіанамъ достоинства, которое со славолюбіемъ суетнымъ какъ огнь съ водою, помѣститися не можетъ; а не зная, и не пекутся о немъ, но его презрѣвше, къ мнимому благородію, которое какъ пузырь на водѣ, или пара, или дымъ, вмалѣ является и исчезаетъ, обращаются, и тако, \textit{гдѣ сокровище ихъ есть, тамо и сердце ихъ}\footnote{Матѳ.~6,~21.}. О! какъ горько возрыдаютъ сіи, по мнѣнію міра славніи, но въ самой вещи подлыи и невольники, когда увидятъ въ свое время чадъ Божіихъ, истинныхъ хрістіанъ, въ славѣ несказанной, себя же тоя со стыдомъ лишаемыхъ и отсылаемыхъ въ безчестіе и укоризну вѣчную! "--- 6)~Кто нерадѣніемъ высочайшаго хрістіанскаго благородія лишился, тому должно, пока время не ушло, того со тщаніемъ поискать истиннымъ покаяніемъ и возвратитися къ небесному Отцу, отъ Котораго удалилися, какъ блудный сынъ сотворилъ, да не во вѣки того лишится, и жалѣть, плакать и рыдать безполезно будетъ.

\paragraph*{§\:494.} Извѣстно сіе, что плоть и кровь ужасается нищеты, безчестія, руганія, поношенія и прочіихъ скорбей, которыя хрістіанамъ приключаются въ мірѣ семъ. И того ради многіи истиннаго хрістіанства удаляются. Таковыи, какъ видно, хотятъ и здѣ съ міромъ, и въ будущемъ вѣкѣ со Хрістомъ царствовать, чему быть никакъ невозможно, какъ они себе ни ласкаютъ. Должно плоти, то есть, плотскихъ похотей, непремѣнно отрещися, когда хотятъ Хрістовыми быть, и ненависти злаго міра, который хрістіанъ, яко не своихъ чадъ, ненавидитъ и гонитъ, не бояться. Крестъ и различныя скорби суть знамя хрістіанское, подъ которымъ хрістіане Царю своему, на крестѣ распятому, воинствуютъ. Крестъ Хрістовымъ ученикамъ, иже суть хрістіане, всѣмъ предлагается: \textit{аще кто хощетъ по Мнѣ ити, да отвержется себе, и возметъ крестъ свой, и послѣдуетъ Мнѣ}, глаголетъ Хрістосъ\footnote{Матѳ.~16,~24; Лук.~9,~23.}. И недостойнымъ Хріста означается отъ Самаго Хріста, кто креста Его не носитъ, и въ слѣдъ Его не ходитъ: \textit{иже не пріиметъ креста своего, и въ слѣдъ Мене грядетъ, нѣсть Мене достоинъ}\footnote{Матѳ.~10,~38.}. Отъ сего убо воинъ Хрістовъ познавается, что онъ отвергается себе, сирѣчь, плоть свою распинаетъ со страстьми и похотьми, якоже глаголетъ: \textit{иже Хрістовы суть, плоть распяша со страстьми и похотьми}\footnote{Гал.~5,~24.}; умираетъ сребролюбію, сладострастію, злобѣ, гнѣву, ненависти, зависти и всякой страсти, и не боится презрѣнія, ненависти, озлобленія, поношенія, изгнанія и всякаго бѣдствія, которое отъ міра рабамъ Хрістовымъ бываетъ. Подъ симъ знаменемъ подвизалися вси, которые нынѣ въ отечествѣ небесномъ съ Подвигоположникомъ Іисусомъ торжествуютъ. Подъ сіе знамя взыдемъ и мы, возлюбленный хрістіанине; не постыдимся смиренія, терпѣнія и кротости Хрістовой, да \textit{не и насъ постыдится Хрістосъ, егда пріидетъ во славѣ Своей}\footnote{Марк.~8,~38.}. Хрістосъ, Царь нашъ, такъ смирился глубоко: намъ ли, рабамъ Его, и рабамъ непотребнымъ, гордиться? Хрістосъ, Царь нашъ, терпѣлъ неповинно: намъ ли, рабамъ его повиннымъ, не терпѣть? Хрістосъ, Царь нашъ безгрѣшный, гонителямъ, хулителямъ и распинателямъ Своимъ прощалъ и молился за нихъ: \textit{Отче! остави имъ}\footnote{Лук.~23,~34.}: намъ ли, грѣшнымъ рабамъ Его, на подобныхъ себѣ грѣшниковъ злобиться? Хрістосъ, богатъ сый, обнищалъ: намъ ли, нищимъ и убогимъ, богатства ненасытно искать? Хрістосъ, Царь небесе и земли, \textit{не имѣлъ гдѣ главы подклонити}\footnote{Матѳ.~8,~20.}: намъ ли, убогой и бѣдной твари, разширять великолѣпныя зданія? Хрістосъ, Сынъ Божій, высшій всякія чести и славы, честь и славу презиралъ: намъ ли, отъ природы безчестнымъ, землѣ, пеплу и червямъ, искать чести и славы? Хрістосъ, Господь славы, отъ всѣхъ презрѣнъ былъ: намъ ли у всѣхъ почтенія искать? Хрістосъ, Царь нашъ, терновымъ вѣнцемъ увѣнчанъ былъ: намъ ли, главы свои украшать мастями и ароматами? Хрістосъ оцтомъ съ желчію напоенъ былъ: намъ ли ненасытно желать дражайшихъ винъ? Суетные замыслы, тщетныя начинанія, и не къ доброму концу идутъ! Скажи пожалуй, какъ намъ явиться предъ Нимъ, когда пріидетъ во славѣ Своей, спасти вѣрныя Своя рабы? Какими глазами воззримъ на Него, когда не хотѣли Ему послѣдовать? Какій вѣнецъ получимъ отъ Царя нашего, когда не хотѣли съ Нимъ подвизаться? Како съ Нимъ будемъ торжествовать, когда не хотѣли съ Нимъ побѣждать? Како съ Нимъ будемъ радоваться, когда не хотѣли съ Нимъ скорби терпѣть? Како съ Нимъ прославимся, когда не хотѣли съ Нимъ страдать? Како съ Нимъ будемъ жить, когда не хотѣли здѣ плоти и грѣху умереть? Како будемъ \textit{наслѣдницы Богу, снаслѣдницы же Хрісту}\footnote{Римл.~8,~17.}, когда не хотѣли съ Нимъ Отцу небесному послушанія имѣть? Какое знаменіе, какій видъ покажемъ Ему, что Его есмы, когда не хотѣли здѣ съ Нимъ быть, терпѣть и страдать? Надобно намъ опасаться, чтобы и намъ слово Его не приличествовало: \textit{иже нѣсть со Мною, на Мя есть; и иже не собираетъ со Мною, расточаетъ}\footnote{Матѳ.~12,~30.}. Кто съ міромъ дружится, на Него есть: \textit{иже бо восхощетъ другъ быти міру, врагъ Божій бываетъ}\footnote{Іак.~4,~4.}. И мы расточаемъ тое, что труды Его, болѣзни, скорби, страданія, святѣйшія заслуги собрали намъ. Онъ намъ пріобрѣлъ милость Божію, миръ, оправданіе, освященіе, благословеніе, жизнь и блаженство вѣчное, но, когда съ міромъ дружимся, все сіе расточаемъ. Не познаетъ убо и Онъ насъ, когда мы Его нынѣ, какъ должно, не знаемъ, хотя Ему и глаголемъ: \textit{Господи, Господи}\footnote{Матѳ.~7,~21--23.}! Услышимъ (о, не допусти Боже того!) отъ Него: \textit{не вѣмъ васъ}\footnote{Лук.~13,~27.}, когда не исправимся. Не познаютъ насъ и святіи Его, которые здѣ тѣснымъ и прискорбнымъ путемъ Ему послѣдовали, когда не хощемъ тѣмъ же путемъ за ними идти, но пространнымъ страстей, прихотей и роскошей шествуемъ. Сего ради совратимся, возлюбленне, отъ сего пагубнаго пространства, и обратимся къ пути хрістіанскому; прилѣпимся къ воинству Хрістову, и прибѣгнемъ подъ знаменіе хрістіанское, пока еще подвига время есть. Хрістосъ нашъ Царь, Хрістосъ нашъ вождь, Хрістосъ нашъ предводитель, Хрістосъ нашъ помощникъ, защитникъ, крѣпость, сила, утвержденіе, побѣда, торжество: только крѣпко держаться, послѣдовать и уповать на Него должно. Онъ, яко всесильный, малодушныхъ укрѣпитъ, изнемогающихъ утвердитъ, падающимъ руку помощи подастъ, сумнящихся наставитъ. Вступимъ убо въ подвигъ *съ помощію* Его; ибо побѣда не бываетъ безъ подвига, а вѣнецъ и торжество безъ побѣды. Съ Нимъ подвизаться, съ Нимъ побѣждать, съ Нимъ и торжествовать вѣчно во царствіи Его будемъ. Безъ сумнѣнія, подвигъ сей, какъ выше сказано, сначала намъ страшенъ. Якоже воинамъ міра сего сначала страшно вступать въ сраженіе съ непріятелемъ, но наченшимъ сраженіе, а паче къ тому привыкшимъ, уже нѣтъ страха, но едина ревность и надежда побѣды и торжества воспаляетъ сердца: тако и вступающимъ въ подвигъ противу міра, плоти и діавола, страшно и трудно кажется дѣло; но наченшимъ и тщащимся, съ помощію Хріста, Царя"=побѣдителя, все удобно. Ктомужъ хотя воини Хріста"=Царя извнѣ презрѣны отъ злаго міра, какъ выше сказано, но внутрь высоко почтены они: какъ имя свое "--- \textit{хрістіанинъ}, такъ благородіе ведутъ отъ Хріста, Царя своего, и Тому сообразуются. Царя Хріста во днехъ смиренія, умаленія не было ничего презрѣннѣе, но не было ничего и выше. Родился въ убогомъ вертепѣ; но ангели воспѣли рождество Его, и звѣзда приведе волхвовъ на поклоненіе Ему, яко царю Іудейскому\footnote{Лук.~2,~9--14; Матѳ.~2,~1,~2,~9 и 10.}. Во Іорданѣ крестился, но отъ Бога Отца свидѣтельство пріялъ: \textit{Сей есть Сынъ Мой возлюбленный}\footnote{Матѳ.~3,~17.}. Не имѣлъ гдѣ главы подклонить, но многи тысящи народа въ пустыни питалъ\footnote{14,~20 и 21.}. Отъ Іудеевъ гоненіе, презрѣніе, хуленіе терпѣлъ но Его трепетали демони, послушали вѣтры и море\footnote{8,~29,~26 и 27.}. Его гласомъ мертвыи воставали, прокаженніи очищалися, слѣпіи прозирали, глусіи слышали, разслабленніи крѣпость и силу воспріимали\footnote{11,~5; 9,~7.}. На крестѣ умеръ; но въ смерти Его \textit{завѣса церковная раздрася на двое, съ вышняго края до нижняго, и земля потрясеся, и каменіе распадеся, и гроби отверзошася, и многи тѣлеса усопшихъ святыхъ восташа}, и проч.\footnote{27,~51 и 52.} Во гробѣ, какъ мертвецъ, положенъ былъ; но восталъ Самъ силою Своего Божества, яко смерти и ада Побѣдитель. Царю своему Хрісту подобятся и воини Его, истинные хрістіане. Смиряются ниже всѣхъ, но Богъ ихъ выше всѣхъ возноситъ\footnote{1~Іоан.~4,~4.}; на землѣ странствуютъ, но отечество ихъ небо; на землѣ обращаются, но \textit{житіе ихъ на небесѣхъ, откуду и Спасителя ожидаютъ}\footnote{Филип.~3,~20.}. Въ рубищахъ ходятъ, но внутрь царскою багряницею покрываются. Человѣческаго наслѣдія лишаются, но Божіе имѣютъ: \textit{наслѣдницы убо Богу, снаслѣдницы же Хрісту}\footnote{Римл.~8,~16.}. На землѣ богатства не ищутъ, и не имѣютъ; но небесная сокровища отверзаются имъ\footnote{2~Кор.~4,~7.}. Отъ человѣкъ изгоняются, но Богъ ихъ пріемлетъ\footnote{6,~17.}. Люди ихъ оставляютъ, но съ Богомъ общеніе имѣютъ\footnote{1~Іоан.~1,~3.}. Міръ ихъ осуждаетъ, но Богъ оправдаетъ\footnote{Римл.~8,~30 и 33.}. Міръ ихъ проклинаетъ, но Богъ благословляетъ\footnote{Еф.~1,~3; 1~Петр.~3,~9; Гал.~3,~9.}. Міръ ихъ оскорбляетъ, но Богъ утѣшаетъ\footnote{2~Кор.~1,~4.}. Міръ ихъ безчеститъ но Богъ ихъ прославляетъ\footnote{Пс.~90,~15; 1~Петр.~5,~10.}. Смотри, возлюбленный хрістіанине, какъ сладкая хрістіанъ служба, и разсуди, коль высокое ихъ смиреніе, коль богатая нищета, коль славное безчестіе, коль благородная подлость, коль блаженное изгнаніе, коль желаемое оскорбленіе! Аще бо весь міръ какъ ничто предъ Богомъ есть, то и ненависть и презрѣніе, злоба, проклинаніе, изгнаніе его, рабамъ Божіимъ бываемое, яко ничтоже противу любви, милости, благословенія Божія, которыхъ они отъ Бога сподобляются. Ибо ежели земный царь кого въ милости и любви содержитъ, тому ненависть и злоба враговъ его ничтоже есть, хотя"=то царь земный, яко человѣкъ, не все, что хощетъ, можетъ: кольми паче кого Царь небесный любитъ, милуетъ и жалуетъ, тому ничтоже есть всего міра злоба и ненависть: яко Онъ болій есть несравненно паче всего міра, и Его милость, благословеніе, любовь большая и желательнѣйшая есть паче всего міра злобы и ненависти.

\subsection[Глава 6-я. О прибѣжищѣ и помощи хрістіанской.]{глава шестая.\\\bfseries О прибѣжищѣ и помощи хрістіанской.}

\begin{quotation}\textit{Господи, прибѣжище былъ еси намъ въ родъ и родъ. Прежде даже горамъ не быти, и создатися земли и вселеннѣй, отъ вѣка и до вѣка Ты еси}\footnote{Пс.~89,~2--3.}.\end{quotation}
\begin{quotation}\textit{Богъ намъ прибѣжище и сила, помощникъ въ скорбехъ, обрѣтшихъ ны зѣло}\footnote{45,~2.}.\end{quotation}

\paragraph*{§\:495.} Прибѣжище и помощникъ хрістіанамъ есть Богъ, якоже они со Псаломникомъ поютъ: \textit{Богъ намъ прибѣжище и сила, помощникъ въ скорбехъ, обрѣтшихъ ны зѣло}. И Самъ Онъ повелѣваетъ рабомъ Своимъ хрістіанамъ призывать Его въ день скорби ихъ: \textit{призови Мя въ день скорби твоея, и изму тя; и прославиши Мя}\footnote{Пс.~49,~15.}. И Богъ есть хрістіанамъ отецъ: того ради они къ Богу, какъ дѣти къ отцу своему, во всякихъ нуждахъ и скорбяхъ прибѣгаютъ. Сынове вѣка сего то къ князямъ и вельможамъ и прочіимъ своимъ защитникамъ въ день скорби прибѣгаютъ, то къ мамонѣ своей, которой работаютъ, простираютъ руки свои, искупуя себе сребромъ и златомъ отъ належащей напасти; то силу свою, хитрость и коварство свое полагаютъ въ защищеніе себѣ и помощь. Хрістіане истинніи не тако. Они все сіе прибѣжище, яко суетное и богопротивное, оставивше, къ единому Богу прибѣгаютъ сердцами, и сердцемъ и устами къ Нему вопіютъ: \textit{Господи, прибѣжище былъ еси намъ въ родъ и родъ}\footnote{89,~2.}. \textit{Ты еси прибѣжище отъ скорби, обдержащія мя: радосте моя, избави мя отъ обышедшихъ мя}\footnote{31,~7.}. Они въ томъ поступаютъ такъ, какъ дѣти малыя, которыя, или въ болѣзни страждуще, или видя какую находящую бѣду, къ отцу или матери своей простираютъ руки и прибѣгаютъ: тако и хрістіане, которые суть \textit{сынове Божіи вѣрою о Хрістѣ Іисусѣ}\footnote{Гал.~3,~26.}, ни откуду не ища и не надѣяся помощи, къ Богу, какъ Отцу своему, прибѣгаютъ и простираютъ руки, Который глаголетъ: \textit{не имамъ тебе оставити, ниже имамъ отъ тебе отступити}, "--- \textit{яко дерзающимъ намъ глаголати: Господь мнѣ помощникъ, и не убоюся, что сотворитъ мнѣ человѣкъ}\footnote{Евр.~13,~5 и 6.}. Дерзаютъ хрістіане и глаголютъ: \textit{сіи на колесницахъ, и сіи на конехъ: мы же во имя Господа Бога нашего призовемъ. Тіи спяти быша и падоша: мы же востахомъ и исправихомся}\footnote{Пс.~19,~8 и 9.}. Къ сему безопасному прибѣжищу прибѣгали праотцы, Авраамъ, Исаакъ и Іаковъ, и получили помощь. Къ сему сѣмя ихъ праведное и Израильтяне простерли руцѣ свои въ работѣ египетской, и избавились. Къ сему въ пещи вавилонской разженной тріе отроки возопили, и спаслись. Къ сему въ новой благодати апостоли, святители, мученики, пустынники, преподобніи отцы и прочіи святіи воздѣвали преподобныя свои руцѣ, и не постыдилися. Къ сему и нынѣ благочестивые люди вѣрою приступаютъ, и вопіютъ къ Нему со пророкомъ: \textit{на Тя уповаша отцы наши, уповаша, и избавилъ еси я; къ Тебѣ воззваша, и спасошася; на Тя уповаша, и не постыдѣшася}\footnote{21,~5--6.}. И молятся съ упованіемъ: \textit{буди Господи милость Твоя на насъ, якоже уповахомъ на Тя}\footnote{32,~22.}, "--- и не посрамляются. Смотри, возлюбленный хрістіанине, коль великое блаженство хрістіанъ и въ семъ вѣку! Аще Богъ прибѣжище и помощь хрістіанамъ, яко отецъ чадомъ своимъ: кто имъ что можетъ сдѣлать? \textit{Аще Богъ по насъ, кто на ны?} глаголетъ блаженнѣйшій Павелъ\footnote{Римл.~8,~32.}. Весь свѣтъ не можетъ ничего; яко весь свѣтъ какъ ничтоже противу Бога, Котораго манію вся повинуются. Огонь не жжетъ, вода не потопляетъ, мечь не сѣчетъ, ядъ смертоносный не умерщвляетъ, зубы звѣрей не касаются вѣрнаго и уповающаго на Господа; царіе и мучители страшные съ беззаконнымъ своимъ полчищемъ не могутъ побѣдить единаго хрістіанина: яко \textit{Господь силъ съ нимъ}. Откуду съ дерзновеніемъ поютъ: \textit{Господь силъ съ нами, заступникъ нашъ Богъ Іаковль: сего ради не убоимся, внегда смущается земля}\footnote{Пс.~45,~8,~12 и 3.}, "--- о чемъ и весь псаломъ сей свидѣтельствуетъ. Сіе безопасное прибѣжище и намъ отверсто: только должно берещися, чтобы не затворить его грѣхами. Отворилъ его намъ Сынъ Божій святѣйшимъ Своимъ послушаніемъ и смертію, смертію же крестною. Тому слава, честь и благодареніе со Отцемъ и Святымъ Духомъ во вѣки. Аминь.

\subsection[Глава 7-я. Пѣснь хрістіанская.]{глава седмая.\\\bfseries Пѣснь хрістіанская.}

\begin{quotation}\textit{Видѣна быша шествія Твоя, Боже, шествія Бога моего Царя, иже во святѣмъ}\footnote{Пс.~67,~25.}.\end{quotation}
\begin{quotation}\textit{Воспойте Господеви пѣснь нову, яко дивна сотвори Господь}\footnote{97,~1.}.\end{quotation}

\paragraph*{§\:496.} Видимъ, возлюбленный хрістіанине, преславная \textit{Бога нашего шествія}, въ сотвореніи міра сего прекраснаго, когда \textit{вѣрою разумѣваемъ совершитися вѣкомъ глаголомъ Божіимъ, во еже отъ не являемыхъ видимымъ быти}\footnote{Евр.~11,~3.}. \textit{Яко Словомъ Господнимъ небеса утвердишася, и Духомъ устъ Его вся сила ихъ. Яко Той рече, и быша; Той повелѣ, и создашася}\footnote{Пс.~32,~6 и 9.}. Видимъ и поемъ \textit{всемогущую силу Его}, яко вся изъ ничего создалъ; \textit{дивную премудрость Его}; яко \textit{вся премудростію сотворилъ}\footnote{103,~24.}; \textit{дивную благость Его}, яко вся сія не ради Себе, но насъ ради сотворилъ. "--- Видимъ \textit{шествіе Бога нашего} въ промыслѣ о созданіи Своемъ, и поемъ Ему со пророкомъ: \textit{вся къ Тебѣ чаютъ, дати пищу имъ во благо время. Давшу Тебѣ имъ, соберутъ; отверзшу Тебѣ руку, всяческая исполнятся благости. Отвращшу же Тебѣ лице, возмятутся: отъимеши духъ ихъ, и исчезнутъ, и въ персть свою возвратятся. Послеши Духа Твоего, и созиждутся, и обновиши лице земли. Буди слава Господня во вѣки}\footnote{Пс.~103,~27--31.}. \textit{Очи всѣхъ на Тя уповаютъ, и Ты даеши имъ пищу во благовременіи: отверзаеши Ты руку Твою, и исполняеши всяко животно благоволенія}\footnote{144,~15 и 16.}. И, послѣдуя Единородному Сыну Его, Господу нашему Іисусу Хрісту, исповѣдуемъ благость Его: \textit{яко солнце Свое сіяетъ на злыя и благія, и дождитъ на праведныя и на неправедныя}\footnote{Матѳ.~5,~45.}. "--- Видимъ преславная Его \textit{шествія} во всемірномъ потопѣ, въ которомъ беззаконники страшно показнены, и праведный Ной чудесно спасенъ былъ, и поемъ Его правду и милость: правду, беззаконныя наказующую; милость, благочестивыя обымающую. "--- Видимъ страшное Бога нашего \textit{шествіе}, егда, сошедъ видѣти вопль содомскій и гоморрскій, огнемъ свыше сожеглъ беззаконниковъ нераскаянныхъ, и изъ среды ихъ Лота праведнаго исхитилъ, и поемъ Его такожде правду и милость: \textit{яко праведенъ Господь, и правы суды Его}\footnote{Пс.~118,~137.}; "--- \textit{и уповающаго на Господа милость обыдетъ}\footnote{31,~10.}. "--- Видимъ преславное \textit{Бога нашего шествіе} во Египтѣ, егда \textit{посла знаменія и чудеса своя посредѣ Египта на Фараона и на вся рабы его}\footnote{134,~9.}; \textit{и изведе люди своя въ радости, и избранныя своя въ веселіи}\footnote{104,~43.}; \textit{и запрети Чермному морю, и изсяче, и настави я въ безднѣ яко въ пустыни; и спасе я изъ руки ненавидящихъ, и избави изъ руки враговъ; покры вода стужающія имъ, ни единъ отъ нихъ избысть}\footnote{105,~8--11.}. Его \textit{море видѣ и побѣже, Іорданъ возвратися вспять, горы взыграшася яко овни, и холми яко агнцы овчіи. Что ти есть море, яко побѣгло еси, и тебѣ Іордане, яко возвратился еси вспять? Горы, яко взыграстеся яко овни, и холми яко агнцы овчіи? Отъ лица Господня подвижеся земля, отъ лица Бога Іаковля, обращшаго камень во езера водная, и несѣкомый во источники водныя}\footnote{Пс.~113,~3--8.}. Видимъ сіе преславное Его шествіе, и поемъ со избранными Его людьми: \textit{десница Твоя, Господи, прославися въ крѣпости; десная Твоя рука, Господи, сокруши враги, и множествомъ славы Твоея стерлъ еси сопротивныхъ. Кто подобенъ Тебѣ въ бозѣхъ, Господи, кто подобенъ Тебѣ? Прославленъ во святыхъ, дивенъ въ славѣ, творяй чудеса}\footnote{Исх.~15,~6,~7 и 11.}! "--- Видимъ \textit{шествіе Бога нашего} въ пустынѣ непроходимой, въ которой предъ людьми Своими шелъ въ столпѣ облака и столпѣ огня: \textit{распростре облакъ въ покровъ имъ, и огнь еже просвѣтити имъ нощію}\footnote{Пс.~104,~39.}; \textit{настави я облакомъ во дни, и всю нощь просвѣщеніемъ огня}\footnote{77,~14.}, "--- и прочая преславная сотворилъ чудеса: и поемъ Ему съ Давидомъ: \textit{Боже! внегда исходити Тебѣ предъ людьми Твоими, внегда мимоходити Тебѣ въ пустыни, земля потрясеся, ибо небеса кануша отъ лица Бога Синаина, отъ лица Бога Израилева}\footnote{67,~8--9.}. "--- Видимъ преславная \textit{Бога нашего шествія, егда введе люди Своя въ гору святыни Своея, гору сію, юже стяжа десница Его, и изгна отъ лица ихъ языки, и по жребію даде имъ землю ужемъ жребодаянія, и всели въ селеніихъ ихъ колѣна Израилева}\footnote{77,~54--55.}. Видимъ, яко \textit{множицею избави я: тіиже преогорчиша Его совѣтомъ своимъ, и смиришася въ беззаконіихъ своихъ. И видѣ Господь, внегда скорбѣти имъ, внегда услышаше моленіе ихъ: и помяну завѣтъ Свой, и раскаяся по множеству милости Своея, и даде я въ щедроты предъ всѣми плѣнившими я}\footnote{Пс.~105,~43--46.}, "--- и поемъ непобѣдимое Его грѣхами человѣческими милосердіе: \textit{Благословенъ Господь Богъ Израилевъ отъ вѣка и до вѣка: и рекутъ вси людіе: буди, буди}\footnote{ст.~48.}. "--- Видимъ \textit{шествіе Божіе} въ Вавилонѣ, егда въ пещь огненную ко отрокомъ еврейскимъ снизшелъ и пламень въ росу преложилъ, любящія Его оросилъ и прохладилъ, и пламенемъ попалилъ враговъ Своихъ; и поемъ ему съ благословенными отроки Его: \textit{Благословенъ еси, Господи Боже отецъ нашихъ, и препѣтый и превозносимый во вѣки, и благословенно имя славы Твоея святое, и препѣтое и превозносимое во вѣки}\footnote{Дан.~3,~52.}! "--- Видимъ и въ прочіихъ преславныхъ чудесахъ преславная Бога нашего шествія. Видимъ наипаче \textit{шествія Бога нашего} въ плотскомъ Его рожденіи, на землѣ съ человѣки пожитіи, страданіи, смерти, воскресеніи, и живоносною плотію на небо возшествіи; и исповѣдуемъ со апостоломъ: \textit{исповѣдуемо велія есть благочестія тайна! Богъ явися во плоти; оправдася въ Дусѣ, показася ангеломъ, проповѣданъ бысть во языцѣхъ, вѣровася въ мірѣ, вознесеся во славѣ}\footnote{1~Тим.~3,~16.}. Поемъ благодарно со ангелами: \textit{слава въ вышнихъ Богу, и на земли миръ, въ человѣцѣхъ благоволеніе}\footnote{Лук.~2,~14.}! Хвалимъ и благословимъ со пророкомъ Захаріею: \textit{Благословенъ Господь Богъ Израилевъ! яко посѣти, и сотвори избавленіе людемъ Своимъ; и воздвиже рогъ спасенія намъ, въ дому Давида отрока Своего, якоже глагола усты святыхъ, сущихъ отъ вѣка пророкъ Его}\footnote{Лук.~1,~68--70.}. А понеже въ древнихъ родѣхъ неслыханное чудо таковое, каковое мы слышимъ и вѣруемъ, "--- чудо, яко Богъ отъ Дѣвы родился, Слово плоть бысть, Невидимый явися, Царь небесный на земли поживе, не оставивъ престола славы Своея, Господь славы на крестѣ распяся, "--- и такъ есть \textit{новое} сіе чудо, и чудо чудесъ; того ради новую поемъ пѣснь: «Новое чудо и боголѣпное! Дѣвическую бо дверь затворенную явѣ проходитъ Господь: нагъ во входѣ, и плотоносецъ явися во исходѣ Богъ, и пребываетъ дверь затворена»\footnote{Пѣснь 9"~я гласа 3"~го.}. Ради насъ сіе страшное сотворилъ Господь чудо. \textit{Вѣрно слово и всякаго пріятія достойно, яко Хрістосъ Іисусъ пріиде въ міръ грѣшники спасти}\footnote{1~Тим.~1,~15.}. Намъ, возлюбленный хрістіанине, должно радоватися о семъ и пѣти Ему благодарно новую пѣснь: видѣна быша шествія Твоя, Боже! Видѣна быша въ святомъ рожденіи Твоемъ отъ Дѣвы Богородицы, въ святомъ на земли пожитіи Твоемъ, въ святомъ и спасительномъ ученіи Твоемъ, въ страшныхъ и вышеестественныхъ чудесахъ Твоихъ, въ многоразличныхъ страданіяхъ Твоихъ, въ животворящей смерти Твоей, въ преславномъ и всерадостномъ воскресеніи Твоемъ, въ торжественномъ и восклицательномъ на небо восхожденіи Твоемъ, въ огненосномъ изліяніи Духа Твоего Святаго на апостолы Твоя, и тѣхъ посланіи на проповѣдь Евангелія твоего во всю вселенную. \textit{Видѣна быша шествія Твоя, Боже! Изшелъ еси на спасеніе людей Твоихъ, спасти помазанныя Твоя; вложилъ еси во главы беззаконныхъ смерть; воздвиглъ еси узы даже до выи въ конецъ. Разсѣклъ еси во изступленіи главы сильныхъ, сотрясутся въ ней; разверзутъ узды своя яко снѣдаяй нищій тай. И навелъ еси на море кони твоя, смущающія воды многи}\footnote{Авв.~3,~13--15.}. Каковыхъ пѣсней утѣшительныхъ въ святыхъ книгахъ довольно обрѣтается. Разсуди же, возлюбленный хрістіанине, коль великое утѣшеніе хрістіанской душѣ пѣть Бога, Котораго непрестанно и ненасытно хвалятъ и поютъ ангели. Откуду толь многія пѣсни какъ въ книгахъ пророческихъ, псалмахъ, такъ и въ церковныхъ книгахъ сложены въ утѣшеніе наше и духовную радость.

\subsection[Глава 8-я. Радость хрістіанская.]{глава осмая.\\\bfseries Радость хрістіанская.}

\begin{quotation}\textit{Сердце мое и плоть моя возрадовастася о Бозѣ живѣ}\footnote{Пс.~83,~3.}.\end{quotation}
\begin{quotation}\textit{Веселитеся о Господѣ, и радуйтеся праведніи}\footnote{31,~11.}.\end{quotation}
\begin{quotation}\textit{Сынове Сіони возрадуются о Царѣ своемъ}\footnote{149,~2.}.\end{quotation}

\paragraph*{§\:497.} Радость безъ любви не бываетъ, и гдѣ любовь, тамо и радость. А кто любитъ что, о томъ радуется. Кто любитъ сребро, злато, похоть, славу міра сего, сласть, о томъ утѣшается и веселится. Кто любитъ друга, о немъ радуется. И чимъ большая у кого любовь бываетъ къ чему, тотъ болѣе о томъ и радуется. Сія радость познается отъ послѣдующей печали. Аще бо кто какой лишился вещи и печалится о томъ, несумнѣнный знакъ, что онъ тую любилъ вещь. Тако другъ друга любимаго, мужъ жены, и жена мужа любимаго, братъ брата любимаго лишився печалится. Тако, кто любитъ честь, славу, богатство міра сего, когда лишится любимаго, скорбитъ и сѣтуетъ. И такъ къ чему сердце любовію прилѣпляется, о томъ сердце и радуется. Идѣже чье сокровище, ту и сердце его, какъ учитъ Хрістосъ: \textit{идѣже есть сокровище ваше, ту и сердце ваше будетъ}\footnote{Матѳ.~6,~21.}. Тако кто любитъ Бога и вѣчную славу, о Бозѣ и радуется, \textit{ту и сердце его есть}.

\paragraph*{§\:498.} \textit{Радость хрістіанская} есть: 1)~Что они вѣруютъ въ Бога истиннаго. Язычники ослѣпленніи вѣруютъ богамъ ложнымъ, бездушнымъ, мертвымъ, глухимъ, слѣпымъ: \textit{идоли бо языкъ сребро и злато, дѣла рукъ человѣческихъ: уста имутъ, и не возглаголютъ; очи имутъ, и не узрятъ; уши имутъ, и не услышатъ; ноздри имутъ, и не обоняютъ; руцѣ имутъ, и не осяжутъ; нозѣ имутъ, и не пойдутъ; не возгласятъ гортанемъ своимъ}\footnote{Пс.~113,~12--15.}. Хрістіане не тако; но вѣруютъ и покланяются Богу истинному, живому, вѣчному, безсмертному, премудрому, праведному, преблагому, молитвы и прошенія ихъ слушающему, и \textit{исполняющему во благихъ желанія ихъ}. "--- 2)~Того великаго Бога, Творца небесе и земли, живаго и безсмертнаго, нарицаютъ Отцемъ своимъ, и имѣютъ во Отца, якоже Самъ имъ обѣщалъ: \textit{и буду вамъ во Отца, и вы будете Мнѣ въ сыны и дщери}\footnote{2~Кор.~6,~18.}, "--- къ Которому и молятся: \textit{Отче нашъ, Иже еси на небесѣхъ}\footnote{Матѳ.~6,~9.}. Признаютъ хрістіане со отцемъ своимъ Авраамомъ, яко \textit{земля и пепелъ}\footnote{Быт.~18,~27.}, что они недостойны того, дабы Бога называть Отцемъ своимъ; но Онъ самъ ихъ милостивно того удостоеваетъ и \textit{посылаетъ Духа Сына Своего въ сердца ихъ, вопіюща: Авва Отче}\footnote{Гал.~4,~6.}! "--- 3)~Царя имѣютъ Сына Божія, \textit{Егоже царствію не будетъ конца}, Иже владѣетъ небомъ и землею, и вся покаряются подъ нозѣ Его; Иже разрушилъ грѣхъ, упразднилъ смерть, побѣдилъ діавола и посрамилъ, и, яко Побѣдитель, надъ всѣми нашими врагами восторжествовалъ, возшелъ на высоту, и сѣдитъ на престолѣ славы Своея со Отцемъ и Духомъ Святымъ, отъ всея твари видимыя и невидимыя покланяемый и почитаемый. Сему Царю великому, всесильному и въ славѣ страшному работаютъ хрістіане со страхомъ и радуются съ трепетомъ; охотно слушаютъ Его, повелѣнія Его исполняютъ. Аще бо сынове вѣка сего хвалятся и радуются о царѣ своемъ, который храбръ, силенъ, знаменитъ побѣдами и славенъ во всемъ мірѣ: кольми паче \textit{сынове Сіони}, то"=есть сынове церкве святыя, \textit{радуются} и хвалятся о \textit{Царѣ своемъ} Іисусѣ Хрістѣ, \textit{Иже есть Царь царей и Господь господей}\footnote{Апок.~17,~14.}, и воистину \textit{велій Господь, и хваленъ зѣло, и величію Его нѣсть конца, и царство Его царство всѣхъ вѣковъ, и владычество Его во всякомъ родѣ и родѣ}\footnote{Пс.~144,~3,~13.}, Иже \textit{владѣетъ царствомъ человѣческимъ, и емуже хощетъ, дастъ е}\footnote{Дан.~4,~29.}, "--- Егоже державы не токмо плоть и кровь, но и діаволъ со всѣмъ своимъ темнымъ полчищемъ \textit{трепещетъ}\footnote{Іак.~2,~19.}. О семъ великомъ Царѣ, крѣпкомъ, сильномъ, праведномъ, страшномъ врагамъ Его, кроткомъ и милостивомъ любящимъ Его и работающимъ Ему, хвалятся и радуются хрістіане, и поютъ со пророкомъ: \textit{Господь воцарися, въ лѣпоту облечеся; облечеся Господь въ силу и препоясася: ибо утверди вселенную, яже не подвижится. Готовъ престолъ Твой оттолѣ: отъ вѣка Ты еси!} и проч.\footnote{Пс.~92,~1,~2 и слѣд.} Къ Нему съ Давидомъ въ радости вопіютъ: \textit{стрѣлы Твоя изощрены, Сильне! Людіе подъ Тобою падутъ въ сердцы врагъ царевыхъ. Престолъ Твой, Боже, въ вѣкъ вѣка: жезлъ правости, жезлъ царствія Твоего. Возлюбилъ еси правду, и возненавидѣлъ еси беззаконіе: сего ради помаза Тя Боже Богъ Твой елеемъ радости паче причастникъ Твоихъ}, и проч.\footnote{44,~6--8.} О Немъ дерзаютъ и глаголютъ: \textit{Господь просвѣщеніе мое и Спаситель мой, кого убоюся? Господь защититель живота моего, отъ кого устрашуся}\footnote{26,~1 и слѣд.}? "--- 4)~Радость сія въ хрістіанахъ преизлишно растетъ и умножается, егда помышляютъ о томъ блаженствѣ, которое уготовано имъ вѣчное на небесѣхъ, егда чаютъ Того Бога, Котораго \textit{нынѣ видятъ яко зерцаломъ въ гаданіи, тогда лицемъ къ лицу видѣти}\footnote{1~Кор.~13,~12.}, "--- Тому предстоятъ безсмертному Царю и славою Его осіяватися, Которому нынѣ усердно и сладцѣ работаютъ, и подъ покровомъ Его нынѣ находятся, якоже написано: \textit{собранніи Господемъ обратятся, и пріидутъ въ Сіонъ съ радостію, и радость вѣчная надъ главою ихъ: надъ главою бо ихъ хвала и веселіе, и радость пріиметъ я, отбѣже болѣзнь и печаль и воздыханіе}\footnote{Ис.~35,~10.}. "--- 5)~Сія радость есть предвкушеніе вѣчныя радости, которой истинніи хрістіане нынѣ нѣкоторую частицу чувствуютъ. Тогда же совершенно \textit{упіются отъ тука дому Господня, и потокомъ сладости Его напоятся}\footnote{Пс.~35,~9.}, егда \textit{не взалчутъ ктому, ни вжаждутъ, не имать же пасти на нихъ солнце и всякъ зной, "--- яко Агнецъ, Иже посредѣ престола, упасетъ я, и наставитъ ихъ на животныя источники водъ, и отъиметъ Богъ всякую слезу отъ очію ихъ}\footnote{Апок.~7,~16--17.}, "--- егда \textit{работающіи Господу будутъ ясти, пити, радоватися, веселитися въ веселіи сердца}\footnote{65,~13 и 14.}; егда \textit{утѣшатся отъ Господа, якоже аще мати кого утѣшаетъ}\footnote{Ис.~66,~13.}; егда \textit{узрятъ Его, якоже есть}\footnote{1~Іоан.~3,~2.}; егда отверзутся имъ вся вѣчныхъ благъ сокровища, \textit{ихже око не видѣ, и ухо не слыша, и на сердце человѣку не взыдоша, яже уготова Богъ любящимъ Его}\footnote{1~Кор.~2,~9.}; егда \textit{возлягутъ со Авраамомъ и Исаакомъ и Іаковомъ} и прочіими патріархами, пророками, апостолами, мучениками и всѣми благословенными \textit{во царствіи Божіи}\footnote{Матѳ.~8,~11.}. Къ сему сладчайшему, радостнѣйшему и совершеннѣйшему пиру желаютъ съ Давидомъ: \textit{когда пріиду и явлюся лицу Божію}\footnote{Пс.~41,~3.}? "--- 6)~Какъ сыновъ вѣка сего, которые радуются о сребрѣ, златѣ, чести, славѣ и роскошахъ міра сего, есть ложная радость, и только видъ радости имѣетъ: тако истинныхъ хрістіанъ радость есть истинная и никогда не отъемлемая. Сыны вѣка сего подобны тому, который, нашедши мѣшецъ исполненный, думаетъ, что съ златомъ, или съ сребромъ нашелъ мѣшецъ; но, открывши и вмѣсто злата съ другимъ чимъ непріятнымъ увидѣвши его, весьма печалится, и самъ себе обличаетъ о пустой бывшей радости: тако они, нашедше богатство міра сего и роскоши, радуются о тѣхъ; но при смерти или нашедшей бѣдѣ, осмотрѣвше, что нѣтъ того ничего, что имѣли, и познавше, что радость ихъ пустая была, вмѣсто радости ложной истинную и безмѣрную въ сердцѣ чувствуютъ печаль. Сію перемѣну мірскія вещи любителямъ своимъ приносятъ, какъ самый случай научаетъ. Не тако истинные хрістіане. Они уподобляются нашедшему истинное сокровище и радующемуся о томъ, который, отверзши и увидѣвши истинное, а не притворное сокровище, болѣе и болѣе веселится: тако они нынѣ, на Бога и вѣчныя сокровища вѣрою взирая, радуются, но когда лицемъ къ лицу увидятъ Бога, и оныя сокровища имъ отверзутся, тогда \textit{радость ихъ исполнится, и радости ихъ никтоже возметъ отъ нихъ}\footnote{Іоан.~16,~22.}. И тако радость о созданіи вмалѣ является, какъ соніе, и, какъ соніе со сномъ, съ отъятіемъ мірскихъ любимыхъ вещей исчезаетъ: радость же духовная во времени начинается, но въ вѣчности совершится, и во вѣки пребываетъ, якоже Самъ Богъ, о Которомъ радуются любящіи Его, во вѣки пребываетъ.

\paragraph*{§\:499.} Хрістіанамъ"=де многія бываютъ скорби, якоже глаголетъ пророкъ: \textit{многи скорби праведнымъ}\footnote{Пс.~33,~20.}? "--- Правда; но сіи скорби извнѣ имъ бываютъ, и ради того душевныя имъ радости не отнимаютъ. Тѣло и плоть ихъ оскорбляется, но душа въ нихъ веселится: яко сіи скорби имъ бываютъ не яко злодѣямъ, но яко хрістіанамъ\footnote{1~Петр.~4,~15 и 16.}. И для того сими скорбьми не погружаются, но паче возносятся и \textit{хвалятся въ скорбехъ, вѣдяще, яко скорбь терпѣніе содѣловаетъ, терпѣніе же искусство, искусство же упованіе; упованіе же не посрамитъ}\footnote{Римл.~5,~3--5.}. Ктомужъ и скорбь терпѣть ради любимаго радостно, якоже читаемъ о святыхъ апостолахъ, которые \textit{идяху отъ лица собора радующеся, яко за имя Господа Іисуса сподобишася безчестіе пріяти}\footnote{Дѣян.~5,~41.}. Тоежъ читаемъ и о святыхъ мученикахъ. И хотя случается и хрістіанамъ въ духовномъ искушеніи печалиться; но они сію печаль препобѣждаютъ надеждою благости Божіей, и сія скорбь имъ обращается въ большее утѣшеніе, когда непогодное и бурное тое прейдетъ время. Ктомужъ \textit{отъ всѣхъ скорбей избавитъ ихъ Господь}, якоже *таможде* пророкъ поетъ\footnote{Пс.~33,~18.}.

\subsection[Глава 9-я. О мудрости хрістіанской.]{глава девятая.\\\bfseries О мудрости хрістіанской.}

\begin{quotation}\textit{Иже} (Хрістосъ) \textit{бысть намъ премудрость отъ Бога}\footnote{1~Кор.~1,~30.}.\end{quotation}

\paragraph*{§\:500.} Хрістіанская премудрость есть Хрістосъ, Сынъ Божій, якоже глаголетъ Павелъ святый: \textit{бысть намъ премудрость отъ Бога}. Онъ намъ открываетъ тайну познанія небеснаго Отца. \textit{Бога никтоже видѣ нигдѣже: Единородный Сынъ, Сый въ лонѣ Отчи, Той исповѣда}\footnote{Іоан.~1,~18.}. И \textit{никтоже знаетъ Отца, токмо Сынъ, и емуже волитъ Сынъ открыти}\footnote{Матѳ.~11,~27.}. \textit{Во Хрістѣ суть вся сокровища премудрости и разума сокровенна}, глаголетъ богомудрый Павелъ\footnote{Кол.~2,~3.}. Хрістіанинъ препростый, который наученъ догматамъ святыя вѣры, далеко мудрѣйшій есть паче Платона и Аристотеля и прочіихъ языческихъ мудрецовъ. Кто отъ нихъ уразумѣлъ тое, что простый хрістіанинъ вѣрою позналъ? \textit{Никтоже отъ князей вѣка сего разумѣ}\footnote{1~Кор.~2,~8.}. Буйство есть еллинскимъ философамъ слышать Бога "--- единаго естествомъ, но троична въ лицахъ; Бога и человѣка въ единомъ лицѣ; Слово, плоть бывшее\footnote{Іоан.~1,~14.}; Дѣву рождшую и Дѣвою пребывающую; тѣло умершее, согнившее и въ прахъ разсыпавшееся, но паки востающее и живущее. \textit{Душевенъ человѣкъ не пріемлетъ яже Духа Божія: юродство бо ему есть, и не можетъ разумѣти}\footnote{1~Кор.~2,~14.}. Но хрістіанинъ, вѣрою святою просвѣщенный, знаетъ и исповѣдуетъ тое, чего \textit{премудрость премудрыхъ и разумъ разумныхъ} и слышать не терпитъ: \textit{юродство бо имъ есть. Да не хвалится убо мудрый мудростію своею, и да не хвалится крѣпкій крѣпостію своею, и да не хвалится богатый богатствомъ своимъ, но о семъ да хвалится хваляйся, еже разумѣти и знати, яко Азъ есмь Господь, творяй милость и судъ и правду на земли: яко въ сихъ воля Моя, глаголетъ Господь}\footnote{Іер.~9,~23 и 24.}.

\paragraph*{§\:501.} Премудрость хрістіанская не въ единомъ только познаніи Бога состоитъ, но заключаетъ въ себѣ и непорочное житіе хрістіанское. Не знаетъ тотъ и Бога, кто воли Божіей не творитъ, хотя устами и исповѣдуетъ Его, и много о Немъ разглагольствуетъ. О семъ свидѣтельствуетъ святое Писаніе: \textit{о семъ разумѣемъ, яко познахомъ Его, аще заповѣди Его соблюдаемъ. Глаголяй, яко познахъ Его, и заповѣдей Его не соблюдаетъ, ложь есть, и въ семъ истины нѣсть}\footnote{1~Іоан.~2,~3 и 4.}. И паки: \textit{Бога исповѣдуютъ вѣдѣти, дѣлы же отмещутся Его, мерзцы суще и непокориви, и на всяко дѣло благое неискусни}\footnote{Тит.~1,~16.}. Откуду Богъ чрезъ пророка глаголетъ о таковыхъ: \textit{приближаются Мнѣ людіе сіи усты своими и устнами чтутъ Мя; сердце же ихъ далече отстоитъ отъ Мене}\footnote{Ис.~29,~13.}. И Хрістосъ глаголетъ: \textit{не всякъ глаголяй Ми: Господи, Господи, внидетъ въ царствіе небесное, но творяй волю Отца Моего, Иже есть на небесѣхъ}\footnote{Матѳ.~7,~21.}. Откуду въ святомъ Писаніи незнающіи Бога, или беззаконно живущіи называются \textit{безумніи, буіи, во тмѣ ходящіи} и проч., какъ читающему тое видѣть можно. Сего ради премудрость хрістіанская тѣмъ только хрістіанамъ приличествуетъ, которые о Хрістѣ Іисусѣ благочестно живутъ, а тѣмъ буйство и безуміе приписуется, которые на языкѣ Бога, а въ сердцѣ безбожіе имѣютъ, хотя бы и звѣзды считали, или все священное Писаніе наизусть умѣли. "--- (\textit{Смотри еще 1"~я книги первыя статьи главу вторую}).

\subsection[Глава 10-я. О свободѣ хрістіанской.]{глава десятая.\\\bfseries О свободѣ хрістіанской.}

\begin{quotation}\textit{Аще Сынъ вы свободитъ, воистинну свободни будете}, глаголетъ Хрістосъ\footnote{Іоан.~8,~36.}.\end{quotation}
\begin{quotation}\textit{Свободою, еюже Хрістосъ насъ свободи стойте}\footnote{Гал.~5,~1.}.\end{quotation}

\paragraph*{§\:502.} Ничто такъ людямъ не пріятно и не желательно, какъ свобода. Кто не желаетъ свободитися отъ работы, отъ темницы, отъ плѣненія? Вси свободу почитаютъ паче всякаго міра сего сокровища. Не пріятно намъ ни богатство, ни честь, ни пища сладкая, какъ свободы не имѣемъ. Птицы и прочія безсловесныя лучше изволяютъ на свободѣ быть, нежели въ неволѣ и питатися готовою пищею. У всѣхъ природное желаніе и любовь къ свободѣ. Почему никто ничѣмъ такъ не утѣшается, какъ свободою: истинныи хрістіане сіе сокровище у себя имѣютъ, "--- свободу, глаголю, свободу не временную, но вѣчную. Ихъ свобода ни мѣстомъ, ни временемъ не заключается. У нихъ свободы темница, плѣненіе, работа, желѣзная цѣпь и ничто симъ подобное отнять не можетъ; вездѣ и всегда сіе неоцѣненное сокровище съ собою имѣютъ, яко не тлѣнною міра сего цѣною, \textit{не истлѣннымъ сребромъ, или златомъ, но честною кровію, яко Агнца непорочна и пречиста Хріста, купленное}\footnote{Петр.~1,~18 и 19.}. Они работаютъ людямъ, работаютъ врагамъ, плѣнившимъ ихъ, заключаются въ темницахъ, связуются узами; но сладкія своея свободы не теряютъ. Такъ высокая, славная и сладкая хрістіанская свобода!

\paragraph*{§\:503.} Въ чемъ сія ихъ свобода состоитъ, святое Писаніе показуетъ. 1)~Свободны они отъ грѣха, смерти и вѣчнаго осужденія. \textit{Отъ руки адовы избавлю я, и отъ смерти искуплю я}, глаголетъ Господь чрезъ пророка\footnote{Ос.~13,~14.}. \textit{Той спасетъ люди своя отъ грѣхъ ихъ}, глаголется о Хрістѣ\footnote{Матѳ.~1,~21.}. \textit{Свобождшеся отъ грѣха, поработистеся правдѣ}, глаголетъ апостолъ вѣрнымъ\footnote{Римл.~6,~18.}. Сюда надлежитъ апостола Іоанна утѣшительное слово: \textit{Той очищеніе есть о грѣсѣхъ нашихъ}\footnote{1~Іоан.~2,~2.}. "--- Какъ"=де хрістіане отъ смерти свободни, а они умираютъ? "--- Умираютъ тѣлесно, но паки воскреснутъ въ послѣдній день, и во вѣки безсмертны будутъ. \textit{Тогда будетъ слово написанное: пожерта бысть смерть побѣдою. Гдѣ ти, смерте, жало? гдѣ ти, аде, побѣда}\footnote{1~Кор.~15,~54 и 55.}? "--- Неужели"=де хрістіане безгрѣшны? "--- Нѣтъ, имѣютъ хрістіане грѣхъ, но не царствуетъ надъ ними, и хотя отъ немощи грѣшатъ, но молитвою, покаяніемъ и вѣрою во Хріста очищаются отъ грѣха. \textit{Аще исповѣдаемъ грѣхи наша, вѣренъ есть и праведенъ, да оставитъ намъ грѣхи наша}, глаголетъ апостолъ\footnote{1~Іоан.~1,~9.}. А хрістіане, которые попущаютъ царствовати надъ собою грѣху, тіи сей свободы не имѣютъ: \textit{яко всякъ творяй грѣхъ, рабъ есть грѣха}, по словеси Хрістову\footnote{Іоан.~8,~84.}. "--- 2)~Свобода хрістіанская состоитъ въ томъ, что они свободни "--- отъ клятвы законныя, якоже глаголетъ апостолъ Павелъ: \textit{Хрістосъ ны искупилъ есть отъ клятвы законныя, бывъ по насъ клятва. Писано бо есть: проклятъ всякъ, висяй на древѣ: да во языцѣхъ благословеніе Авраамле будетъ о Хрістѣ Іисусѣ}\footnote{Гал.~3,~13--14.}. И паки: \textit{ни едино нынѣ осужденіе сущимъ о Хрістѣ Іисусѣ, не по плоти ходящимъ, но по духу}\footnote{Римл.~8,~1.}. Клятва законная въ томъ состоитъ, что законъ на всякаго, кто всего, закономъ Божіимъ повелѣннаго, не исполняетъ, гнѣвъ Божій означаетъ, и вѣчному осужденію подвергаетъ. Отъ сего хрістіане благодатію Хріста Сына Божія, Который весь законъ совершенно исполнилъ и за грѣхи наши преданъ бысть, и на древѣ, яко злодѣй, повѣшенъ былъ, и тако за насъ \textit{клятвою} былъ, свободни суть. И хотя закона совершенно не исполняютъ, однакожъ недостатки ихъ и немощи совершеннѣйшимъ Іисуса Хріста послушаніемъ дополняются, и немощи сіи благодатію Его оставляются имъ кающимся. "--- 3)~Свободни хрістіане отъ закона: яко \textit{не суть подъ закономъ, но подъ благодатію}, по ученію премудраго Павла\footnote{6,~14.}. Яко не по принужденію закона, но свободнымъ духомъ исполняютъ заповѣди Божія; не какъ раби, боящіися наказанія *господина* своего, но какъ сынове, показующіи охотное и любовное послушаніе отцу своему, приносятъ \textit{плодъ духовный, иже есть любы, радость, миръ, долготерпѣніе, благость, милосердіе, вѣра, кротость, воздержаніе: на таковыхъ нѣсть закона}\footnote{Гал.~5,~22--23.}. Въ семъ кажется разумѣ Павелъ написалъ: \textit{праведнику законъ не лежитъ}\footnote{1~Тим.~1,~9.}: яко Онъ благодатію Хрістовою дѣлаетъ тое, что законъ повелѣваетъ. Но \textit{лежитъ} кому? \textit{беззаконнымъ и не покоривымъ, нечестивымъ и грѣшникамъ, неправеднымъ и сквернымъ}, и проч., якоже тойжде апостолъ придаетъ\footnote{1~Тим.~6,~9--10.}. Сіи требуютъ закона, обуздающаго страсти ихъ, востягающаго отъ беззаконія, претящаго имъ казнію. На сіе слово Златоустъ святый глаголетъ: «Праведнику законъ не лежитъ. Почто? Понеже муки есть кромѣ, и яко не ожидаетъ дѣятельныхъ отъ него навыкнути, имѣя внутрь повелѣвающую Духа благодать. Законъ бо дадеся, да страхомъ и прещеніемъ мучатся. Не потребна есть прочее узда на благопокориваго коня, ниже пѣстунство на нетребующаго пѣстуна»\footnote{\textit{Бес.~2"~я на 1"~е посл. къ Тим}.}. "--- 4)~Хрістіане свободни суть отъ церемоній іудейскихъ, каковы были у нихъ, какъ"=то: обрѣзаніе, приношеніе животныхъ въ жертву, и проч.

\paragraph*{§\:504.} Когда"=де хрістіане не суть подъ закономъ, но подъ благодатію: почтожъ законъ проповѣдуется? "--- \textit{Отвѣтъ}. 1)~Проповѣдуется неисправнымъ, чтобы имъ устрашившеся и очувствовавшеся, въ покаяніе пришли. Тако Предтеча святый, хотя привести людей къ покаянію, и тако пріуготовати сердца ихъ къ воспріятію вѣры евангельскія, проповѣдалъ: \textit{уже и сѣкира при корени древа лежитъ: всяко убо древо, еже не творитъ плода добра, посѣкаемо бываетъ и во огнь вметаемо}\footnote{Матѳ.~3,~10.}. "--- 2)~Проповѣдуется и благочестивымъ истиннымъ хрістіанамъ, дабы по правилу его исправляли житіе свое, дабы на него, какъ зерцало, взирая, познавали немощь естества своего и очищали вѣрою во Хріста Сына Божія, и тако бы ветхаго человѣка совлекалися и облекалися въ новаго, успѣвали въ любви Божіей и ближняго, что есть конецъ закона. \textit{Конецъ} бо \textit{завѣщанія есть: любы отъ чиста сердца и совѣсти благія и вѣры нелицемѣрныя}\footnote{1~Тим.~1,~5.}.

\paragraph*{§\:505.} Когда"=де хрістіане на свободу позваны и свободу имѣютъ, убо имъ свободно, что хотятъ, дѣлать? "--- Никакъ; не въ томъ состоитъ хрістіанская свобода, чтобы имъ по волѣ своей жить и что хотятъ дѣлать. Сія свобода есть не хрістіанская, но плотская, и не такъ есть свобода, какъ работа истая и тяжкая: яко \textit{всякъ творяй грѣхъ, рабъ есть грѣха}, по ученію Спасителя\footnote{Іоан.~8,~34.}. Лишаются таковыи хрістіанской свободы, которые, плоти своей послѣдуя, \textit{угодіе ей творятъ въ похоти}\footnote{Римл.~13,~14.}; а вмѣсто того попадаются подъ тяжкое иго мучителя діавола и грѣха, и дѣлаются бѣднѣйшими плѣнниками страстей своихъ, и находятся подъ клятвою законною, гнѣвомъ Божіимъ, и чадами вѣчныя погибели. Хрістіане, когда сладчайшую свою свободу имѣютъ, между тѣмъ признаютъ себе рабами Бога вышняго, Которому должно \textit{работать со страхомъ}, яко Господу\footnote{Пс.~2,~11; Филип.~2,~12; 1~Петр.~1,~17.}, "--- изъ руки врагъ своихъ избавившеся, \textit{служити Ему преподобіемъ и правдою предъ Нимъ вся дни живота своего}\footnote{Лук.~1,~75.}, \textit{любовію работати другъ другу}\footnote{Гал.~5,~13 и пр.} и проч.

\subsection[Глава 11-я. О работѣ хрістіанской.]{глава перваянадесять.\\\bfseries О работѣ хрістіанской.}

\begin{quotation}\textit{Пріидите ко Мнѣ вси труждающіися и обремененніи, и Азъ упокою вы. Возмите иго Мое на себе, и научитеся отъ Мене, яко кротокъ есмь и смиренъ сердцемъ: и обрящете покой душамъ вашимъ. Иго бо Мое благо, и бремя Мое легко есть}, глаголетъ Хрістосъ\footnote{Матѳ.~11,~28--30.}.\end{quotation}

\paragraph*{§\:506.} И работа хрістіанская есть сладкая работа, возлюбленный хрістіанине! Понеже есть свободная работа, а не принужденная. Хрістіане бо, какъ позваны въ работу сію, такъ и работу тую отправляютъ безъ принужденія. Господа вѣка сего купуютъ себѣ рабовъ, хотя они къ симъ и не хотятъ, и опредѣляютъ ихъ къ работамъ, отъ которыхъ они отвращаются; и потому, какъ порабощаютъ ихъ себѣ нехотящихъ, такъ и работать ихъ заставляютъ нехотящихъ и потому работа сія есть печальная, яко противу воли и хотѣнія человѣческаго бываетъ. Богъ, Господь всѣхъ, не тако съ нами поступаетъ. Онъ зоветъ всѣхъ къ Своей работѣ, но никого не принуждаетъ; влечетъ Своею благодатію и убѣждаетъ\footnote{Матѳ.~20,~1--7; 22,~2--9; Лук.~14,~17--23.}, но не насилуетъ. Вотъ какъ сладко призываетъ къ Себѣ труждающихся и обремененныхъ Хрістосъ Господь: \textit{пріидите ко Мнѣ вси труждающіися и обремененніи, и Азъ упокою вы}. Но какъ безъ принужденія избираетъ рабовъ Своихъ Господь, такъ и работу ихъ, Себѣ должную, оставляетъ имъ на ихъ произволеніе. Хощетъ, чтобы вси работали Ему, яко Господу своему, Создателю и Искупителю своему; отворяетъ къ дѣланію виноградъ Свой, и созываетъ всѣхъ чрезъ рабовъ Своихъ, пророковъ и апостоловъ и святыхъ учителей, но не хощетъ принуждать никого.

\paragraph*{§\:507.} Утѣшительная есть хрістіанская работа. Ибо 1)~есть свободная и произвольная, а не принужденная, какъ выше сказано. 2)~Работаютъ хрістіане не человѣку, подобному себѣ, худому и немощному созданію, но Богу, Создателю и Господу всѣхъ. Вси человѣки лучше изволяютъ служить высокому лицу, нежели подлому; благому и милостивому, нежели жестокому; премудрому, нежели неразумному; праведному, нежели неправедному: ибо отъ чести лица, которому хотятъ работать, и свое блаженство хотятъ достать. Но какой бы господинъ человѣкъ ни былъ, сколь высокъ, великъ, добръ, премудръ и праведенъ ни былъ бы, однако такойжде есть, какъ и раби его, таяжъ \textit{земля и пепелъ}, какъ и слуги его. И человѣческое всякое блаженство, высочество, доброта, премудрость и правда скоро рушится отъ нашедшаго противнаго случая: и тако, при нашедшемъ господину несчастіи, тоежде претерпѣваютъ и служащіи ему раби его. "--- Господь, Которому хрістіане работаютъ, не таковъ есть, "--- не человѣкъ есть, но Богъ. Столько великъ есть, что весь свѣтъ предъ Нимъ есть какъ капля предъ моремъ, или какъ ничто; столько высокъ, что нѣтъ ничего Его выше: откуду есть и называется \textit{Вышній}. Онъ есть \textit{Царь царей и Господь господей}\footnote{Апок.~17,~14.}. \textit{Ему всяко колѣно небесныхъ, и земныхъ, и преисподнихъ} со страхомъ и трепетомъ \textit{преклоняется}\footnote{Римл.~14,~11; Филип.~2,~10; Ис.~45,~23.}. \textit{Его царство есть царство всѣхъ вѣковъ, и владычество Его во всякомъ родѣ и родѣ}\footnote{Пс.~144,~13.}; и \textit{единъ сильный Царь царствующихъ и Господь господствующихъ; единъ имѣяй безсмертіе, и во свѣтѣ живый неприступнѣмъ}\footnote{1~Тим.~6,~15 и 16.}. Благость, милость, премудрость, правда и истина Его безконечна и непостижима, какъ и Самъ Онъ. Такъ великому, высокому, благому, премудрому и милостивому Господу хрістіане преклоняютъ колѣно свое и покоряютъ главу свою съ любовію и охотнѣйшимъ послушаніемъ! Откуду симъ высокимъ титуломъ святіи мученики предъ царями и мучителями хвалилися: «Хрістовы есмы раби: Царю небесному работаемъ». И хотя имъ высокія чести и многія богатства мучители и прелестники обѣщали, чтобы ихъ богопротивной повинулись волѣ: все тое какъ сметіе вмѣняли. Смертію грозили, и какъ агнцевъ терзали: ничего не успѣли. Всегда отъ нихъ слышалась сладкая и красная пѣснь: «Богу живому и безсмертному служимъ, покланяемся и почитаемъ Его: Того слуги есмы и раби; Того повелѣнія слушаемъ». Откуду и \textit{не стыдится сими Богъ "--- Богъ нарицатися ихъ}\footnote{Евр.~11,~16.}. \textit{Азъ есмь Богъ Авраамовъ, и Богъ Исааковъ и Богъ Іаковль}\footnote{Исх.~3,~6.}. "--- 3)~Господа"=человѣки служеніе отъ своихъ рабовъ пріемлютъ; но имъ въ служеніи не помогаютъ, но отъ нихъ во всякихъ нуждахъ и случаяхъ себѣ помощи требуютъ. Хрістіанская работа не тако. Они Господу Богу своему работаютъ; но Богъ Самъ имъ помогаетъ, укрѣпляетъ и наставляетъ; усердіе и тщаніе имъ подаетъ; духъ ихъ ободряетъ, сердце возжигаетъ; умъ и разумъ ихъ просвѣщаетъ. Откуду имъ повелѣно помощи просить: \textit{просите и дастся вамъ}, "--- и просящіи пріемлютъ: \textit{всякъ бо просяй пріемлетъ}\footnote{Матѳ.~7,~7 и 8.}. "--- 4)~Человѣки"=господа, коль ни велики и сильни будутъ, не всегда могутъ отъ навѣтовъ вражіихъ защитить, въ нуждахъ ихъ помощи имъ, яко сами человѣки немощни: не все бо человѣкъ можетъ, что хощетъ. Богъ Господь не тако: все можетъ, что хощетъ, яко всесильный; и хощетъ, яко благъ, помощи и защитить рабовъ Своихъ; и знаетъ, какъ защитить, яко премудрый; и защищаетъ и сохраняетъ ихъ, якоже о томъ сказано выше. "--- 5)~Человѣки"=господа, коль ни добры и милостивы, часто рабовъ своихъ оставляютъ безъ награжденія за вѣрную ихъ службу, или не зная усердныя услуги ихъ, или не имѣя чѣмъ наградить, или хотя и награждаютъ, но какъ сами они и слуги временны, такъ и награжденіе ихъ временно: все бо принуждается человѣкъ оставить или при нашедшей бѣдѣ, или во время кончины. Богъ не тако съ Своими рабами поступаетъ: какъ знаетъ совершенно всякаго работу, тѣмъ имѣетъ чѣмъ наградить за вѣрную ихъ службу, и награждаетъ по мѣрѣ служенія ихъ; и за временную службу вѣчное даетъ имъ награжденіе, сотворяетъ ихъ участниками вѣчнаго Своего царствія: \textit{пріидите, благословенніи Отца Моего, наслѣдуйте уготованное вамъ царствіе отъ сложенія міра}\footnote{Матѳ.~25,~34.}. "--- 6)~Работа рабовъ Божіихъ состоитъ не въ ношеніи ига каковаго тяжкаго, каковая работа налагается людямъ отъ господъ своихъ, наипаче безчеловѣчныхъ, но въ ношеніи \textit{ига благаго и бремени легкаго Хрістова}. Не велитъ рабамъ Своимъ Богъ землю съ мѣста на мѣсто носити, *каменіе носити*, руды копати и прочія симъ подобныя тяжести дѣлати: нѣтъ ничего такого; не заповѣдаетъ имъ того Господь. Но что? \textit{Се}, глаголетъ, \textit{заповѣдаю вамъ, да любите другъ друга}\footnote{Іоан.~15,~17.}. И: \textit{конецъ завѣщанія есть любы отъ чиста сердца, и совѣсти благія, и вѣры нелицемѣрныя}\footnote{1~Тим.~1,~5.}. Что легчае, какъ любить? что сладчае, пріятнѣе и удобнѣе, какъ любить? Любви все легко, все не тяжестно, все сносно, все удобно. Что прочіимъ тяжко, несносно, неудобно, невозможно, "--- ей все удобно и возможно. Знаютъ о томъ истинные ея рачители. И хотя раби Божіи часто и по большей части въ тяжкія работы отъ немилостивыхъ властей и мучителей опредѣляются; однакожъ ради любви Божіей все благодарно и великодушно сносятъ: яко и тако Господу своему работаютъ, а не человѣкамъ. "--- 7)~Сладкая и утѣшительная сія хрістіанская работа познается отъ противной оной работы, которая есть работа страстей. Нѣтъ ничего тягчае, какъ работать страстямъ. \textit{Злобный} сколько мучится внутреннимъ своимъ мучителемъ, которымъ обладаемъ бываетъ, то"=есть, гнѣвомъ; сколько изобрѣтаетъ способовъ отмстить и обиду обидою наградить: знаютъ таковые сами, и свидѣтельствуютъ судебныя мѣста. \textit{Завистливый} сколько снѣдается внутрь ядовитымъ и смертоноснымъ тѣмъ червемъ, видя блаженство ближняго и соболѣзнуя о немъ, смотря на добро чуждое, и внутрь зло ощущая и сокрывая; блѣднѣетъ, сохнетъ и истаеваетъ, что ближній его благоденствуетъ: сами таковые лучше знаютъ, хотя и внѣ утаиться не можетъ внутренное зло. \textit{Сребролюбцу} тяжкая работа, которою служитъ мамонѣ, яко мерзкому господину, "--- какъ сердце его связуетъ и ожесточаетъ и держитъ при себѣ; руки отвращаетъ отъ подаянія милостыни и помощи требующимъ, и къ похищенію устрояетъ; и дѣлаетъ рабомъ богатства, а не господиномъ, врагомъ Божіимъ и людей Его, и самому себѣ мучителемъ. Что дѣлаетъ рачителямъ своимъ блудъ, піянство, всѣмъ явно: скотами изъ человѣковъ, и безумными изъ разумныхъ, и безсловесными изъ словесныхъ бываютъ. Тоежъ и о прочіихъ страстяхъ разумѣть должно. Ясно и твердо апостольское слово: \textit{имже кто побѣжденъ бываетъ, сему и работенъ есть}\footnote{2~Петр.~2,~19.}. И тако, хотя таковые кажутся себѣ быть свободными, хотя дѣлаютъ что хотятъ "--- раби суть, якоже писано есть: \textit{всякъ творяй грѣхъ, рабъ есть грѣха}\footnote{Іоан.~8,~34.}; и раби суть бѣднѣйшіи паче тѣхъ, которые варварамъ и мучителямъ работаютъ. Лучше бо человѣку, разумному созданію, работать, нежели страсти, какъ идолу глухому и нѣмому. \textit{Оброцы бо грѣха, смерть}\footnote{Римл.~6,~23.}. И паки: \textit{аще по плоти живете, имате умрети}\footnote{8,~13.}. Хрістіанская работа не такова. \textit{Иго Хрістово благо, и бремя Его легко есть}. И работа правдѣ преславная, благородная и величайшая свобода есть, \textit{плодъ} имѣющая \textit{святыню, конецъ "--- животъ вѣчный}\footnote{6,~22.}. Хрістіане, яко раби Бога кроткаго, долготерпѣливаго, милостиваго, *святаго*, истиннаго, человѣколюбиваго, и сами Господу своему подражаютъ: злобу злобныхъ благостію побѣждаютъ; озлобляеми не злобятся; хулими не хулятъ; лишаеми не плачутъ; о добрѣ ближняго, какъ о своемъ, радуются; съ плачущими плачутъ, и съ радующимися радуются; чести не ищутъ: когда дается, въ славу Божію и пользу общую "--- тую проходятъ: отнимается "--- не сѣтуютъ; течетъ имъ богатство, не прилагаютъ сердца, единому Господу прилѣпившеся, употребляютъ то на свои нужды и требующимъ удѣляютъ; нѣтъ того въ рукахъ ихъ, довольствуются тѣмъ, что имѣютъ; просятъ хлѣба насущнаго отъ Отца своего, "--- получаютъ "--- и благодарятъ Подателю. Пища и питіе имъ подкрѣпленіе едино тѣла изнемогшаго, а не сластопитаніе; столько ядятъ и піютъ, сколько немощная плоть подкрѣпленія требуетъ. Ястіе, питіе, покой и прочія дѣйствія къ единому концу "--- богоугодной работѣ намѣреваютъ. Пищу и питіе пріемля, поминаютъ о Подателѣ, яко отъ руки Его щедрой все пріемлютъ, и поминаютъ о пищѣ вѣчнаго живота. Веселятся, но о Господѣ Бозѣ своемъ; воздыхаютъ и плачутъ, но о томжде Бозѣ. Другъ друга братски любятъ, но и враговъ своихъ не ненавидятъ. Почитаются и хвалятся отъ людей, но они честь и хвалу единому Господу своему восписуютъ. Благоденствуютъ или злоденствуютъ, здравствуютъ или немоществуютъ, изобилуютъ или оскудѣваютъ, о всемъ благодарятъ, вѣдая, яко все отъ Господа происходитъ. Блаженное состояніе! Сладкая и любезная рабовъ Божіихъ работа! Земное, но небесному подобное житіе!..

\subsection[Глава 12-я. О мирѣ хрістіанскомъ.]{глава втораянадесять.\\\bfseries О мирѣ хрістіанскомъ.}

\begin{quotation}\textit{Оправдившеся вѣрою, миръ имамы къ Богу Господемъ нашимъ Іисусъ Хрістомъ}\footnote{Римл.~5,~1.}.\end{quotation}

\paragraph*{§\:508.} Коль тяжко подданному Государя своего гнѣвъ надъ собою чувствовати, "--- не можно словомъ изъяснить. Ничто ему тогда не мило "--- ни пища, ни питіе, ни жена, ни дѣти, ни други, ни иное что. Откуду бываетъ, что многіи сами себе, не стерпя гнѣва того, погубляютъ, и желаютъ умрети, нежели жити подъ гнѣвомъ тѣмъ. Какъ несравненно тяжестнѣе Божій гнѣвъ терпѣти. Сколько бо Богъ отъ человѣка отстоитъ, "--- отстоитъ же безконечно, "--- столько тягчайшій есть Божій гнѣвъ паче гнѣва царскаго. Откуду нечестивымъ день страшнаго Божія суда и праведнаго Его гнѣва пожелаютъ \textit{сокрытися въ пещерахъ и каменіи горстѣмъ, и возглаголютъ горамъ и каменію: падите на ны, и покрыйте ны отъ лица Сѣдящаго на престолѣ, и отъ гнѣва Агнча: яко пріиде день великій гнѣва Его и кто можетъ стати}\footnote{Апок.~6,~15--17.}. "--- Но сколь тяжко и горько Божій гнѣвъ чувствовати, столько утѣшительно и благопріятно благость и милосердіе Его въ сердцѣ своемъ ощущати. Сіе несравненно всякую радость и сладость міра сего превосходитъ. Якоже бо доброзрачная и благовонная роза природная изящнѣйшая и благопріятнѣйшая есть, неже сотворенная изъ какого другаго вещества, или художествомъ живописца написанная: тако малѣйшее вкушеніе благости Божіей далеко лучшее и увеселительнѣйшее есть паче всякаго міра сего увеселенія и утѣшенія. Сіе есть сладкая оная \textit{вечеря}, которую обѣщается Хрістосъ предложити послушающему Его гласа и отверзающему двери сердца своего: \textit{се стою при дверехъ, и толку: аще кто услышитъ гласъ Мой, и отверзетъ двери, вниду къ нему, и вечеряю съ нимъ, и той со Мною}\footnote{Апок.~3,~20.}. Сіе есть вкушеніе и ястіе манны сокровенныя, которое обѣщается отъ тогожде Господа побѣждающему: \textit{побѣждающему дамъ ясти отъ манны сокровенныя}\footnote{2,~17.}. Сіе есть вкушеніе вѣчныя радости и сладости, и есть малѣйшая частичка онаго \textit{увеселенія}, котораго вкусивше, раби Хрістовы забываютъ всю утѣху міра сего и все его увеселеніе, какъ блато мутное и смрадное, или какъ горесть вмѣняютъ, но онаго единаго желаютъ паче и паче. Сего ради глаголетъ премудрость Божія: \textit{память Моя сладка паче меда, и наслѣдіе Мое паче сота медвена. Ядущіи Мя еще взалчутъ и піющіи Мя еще вжаждутся}\footnote{Сир.~24,~22 и 23.}. Сіе сладчайшее \textit{увеселеніе} тіи только чувствуютъ и познаютъ, которыхъ сердца имъ услаждаются. Какъ бо сладость меда вкушающимъ токмо чувствительна, не вкушающимъ неизвѣстна: тако духовная сія сердечная сладость самою вещію дознающимъ только и вкушающимъ извѣстна; прочіи же тоя не знаютъ. И сія сладость есть \textit{миръ} оный \textit{Божій}, превосходяй всякъ умъ, котораго апостолъ Филипписіямъ желалъ: \textit{миръ Божій, превосходяй всякъ умъ, да соблюдетъ сердца ваша и *разумѣнія ваша* о Хрістѣ Іисусѣ}\footnote{Филип.~4,~7.}.

\paragraph*{§\:509.} Сладчайшій и дражайшій миръ сей между Богомъ и человѣкомъ пресѣклся было, когда сатана, змій лукавый, позавидѣлъ блаженству нашихъ прародителей, и ихъ злымъ и лукавымъ своимъ совѣтомъ ввергнулъ въ ровъ преслушанія, и отъ того послѣдующаго гнѣва Божія. Которымъ гнѣвомъ Божіимъ, какъ темнымъ мракомъ, вся вселенная покрылася; и слѣдовало всякому не инаго чего отъ разгнѣваннаго Бога ожидать, какъ достойныя клятвы и вѣчнаго осужденія. Яко \textit{вси согрѣшиша и лишени суть славы Божія, и весь міръ учинился повиненъ Богу}\footnote{Римл.~3,~23 и 19.}. Но Хрістосъ, Сынъ Божій Ходатай Богу и человѣкомъ учинился; средостѣніе, которое было между Богомъ и человѣки, отнялъ; Бога намъ умилостивилъ, и Богу насъ присвоилъ; ангеловъ человѣкомъ, и человѣковъ ангеломъ примирилъ; Іудеевъ съ языками, и языковъ съ Іудеями въ союзъ мира привелъ, и учинилъ едино благословенное стадо, то"=есть, изъ небесныхъ и земныхъ, Іудеовъ и Еллиновъ: и тако учинилося, по словеси Его, \textit{едино стадо, и единъ Пастырь}\footnote{Іоан.~10,~16.}, "--- Его съ Отцемъ и Духомъ, яко Начальника \textit{мира}, Пастыря и Посѣтителя душъ нашихъ, Начальника вѣры, Новаго Завѣта Ходатая и Архіереа великаго, прошедшаго небеса, начальника и совершителя спасенія, и всякаго вѣчнаго блаженства начало и конецъ, почитающее, славословящее пѣсньми, и похвалами величающее стадо. О семъ сладчайшемъ мирѣ проповѣдуютъ намъ апостоли Его: \textit{Той есть миръ нашъ, сотворивый обоя едино, и средостѣніе ограды разоривый: вражду плотію Своею, законъ заповѣдей ученми упразднивъ: да оба созиждетъ Собою во единаго новаго человѣка, творя миръ: и примиритъ обоихъ во единомъ тѣлѣ Богови крестомъ, убивъ вражду на немъ. И пришедъ благовѣсти миръ вамъ, дальнимъ и ближнимъ: зане Тѣмъ имамы приведеніе обои во единомъ Дусѣ ко Отцу}\footnote{Еф.~2,~14 и 18.}. И паки: \textit{оправдившеся вѣрою, миръ имамы къ Богу Господемъ нашимъ Іисусъ Хрістомъ}\footnote{Римл.~5,~1.}. "--- Благовѣствуютъ ангели Его: \textit{слава въ вышнихъ Богу, и на земли миръ, во человѣцѣхъ благоволеніе}\footnote{Лук.~2,~14.}. Пріемлетъ благодарно святая церковь, утѣшается имъ и, благодаря, поетъ Ходатаю того: «Ходатай Богу и человѣкомъ былъ еси, Хрісте Боже! Тобою бо, Владыко, къ Свѣтоначальнику Отцу Твоему отъ нощи невѣдѣнія приведеніе имамы»\footnote{Пѣснь 5"~я гласа 2"~го.}.

\paragraph*{§\:510.} Хотя миръ сей, дражайшій всѣмъ, на землю принеслъ отъ Отца небеснаго Сынъ Божій, Князь и Начальникъ мира, \textit{и пришедъ благовѣсти миръ дальнимъ и ближнимъ} чрезъ Себе Самого, отправляя на земли великое небеснаго Своего Отца посольство, то чрезъ апостоловъ Своихъ, Духомъ Святымъ умудренныхъ: однакожъ не сподобляются того небеснаго дара, которые Благовѣстителя того мира вѣрою не пріемлютъ; такожде хрістіане, которые хотя и мнятся вѣровати въ великаго того Посланника, но житіе вѣрѣ противное имѣютъ, \textit{Бога исповѣдуютъ вѣдѣти, дѣлы же отмещутся Его, мерзцы суще и непокориви, и на всяко дѣло благое неискусни}\footnote{Тит.~1,~16.}. Миръ бо сей въ совѣсти и сердцѣ имѣетъ мѣсто свое; но совѣсть, пороками оскверненная и беззаконіями раздраженная, не иное что грѣшнику, какъ Бога, законопреступленіемъ огорченнаго, и гнѣвъ Божій, клятву Божію и осужденіе вѣчное возвѣщаетъ: почему тамо и миръ сей мѣста не имѣетъ, но вмѣсто того червь злый, смущеніе и всякое безпокойствіе. А имѣютъ драгое сіе и небесное сокровище тіи только, которые, вѣрою очистивше сердца, \textit{творятъ плоды покаянія}\footnote{Матѳ.~3,~8.}, \textit{распинаютъ плоть со страстьми и похотьми}\footnote{Гал.~5,~24.}; яко Хрістовы раби, \textit{плотоугодія въ похоти не творятъ}\footnote{Римл.~13,~14.}, и \textit{работаютъ другъ другу любовію}\footnote{Гал.~5,~13.}, и \textit{миръ} другъ съ другомъ о Хрістѣ \textit{имѣютъ}\footnote{2~Кор.~13,~11.}. И хотя часто и сихъ колеблетъ страхъ и ужасъ праведнаго суда Божія и вѣчнаго осужденія, якоже случается въ великомъ духовномъ искушеніи; но надеждою благости Божіей и вѣрою въ Примирителя и Ходатая Іисуса Хріста, Сына Божія, побѣждаютъ тое волнованіе, и тако сладцѣ утѣшаются, памятуя апостольское утѣшительное слово: \textit{ни едино нынѣ осужденіе сущимъ о Хрістѣ Іисусѣ, не по плоти ходящимъ, но по духу}\footnote{Римл.~8,~1.}.

\subsection[Глава 13-я. О бесѣдѣ хрістіанъ съ Богомъ.]{глава третіянадесять.\\\bfseries О бесѣдѣ хрістіанъ съ Богомъ.}

\begin{quotation}\textit{Господи, кто обитаетъ въ жилищѣ Твоемъ? или кто вселится во святую гору Твою? Ходяй непороченъ, и дѣлаяй правду, глаголяй истину въ сердцѣ своемъ}, и проч.\footnote{Пс.~14,~1,~2 и слѣд.}\end{quotation}

\paragraph*{§\:511.} О коль великое утѣшеніе хрістіане вѣрою почерпаютъ отсюду, что съ Богомъ удостоеваются бесѣдовати, изъяснить невозможно! За великое счастіе почитаетъ человѣкъ съ монархомъ своимъ, или княземъ, или какимъ высокимъ лицемъ дружески разглагольствовать, хотя то царь, князь, вельможа и всякій *высокій* человѣкъ такойжде человѣкъ, какъ и прочіи, таяжъ земля и пепелъ, какъ и прочіи, и такіяжъ немощи имѣетъ, какъ и прочіи, такойжде грѣшникъ, какъ и прочіи, и ничимъ отъ подлыхъ\footnote{\textit{простыхъ}.} не разнствуетъ, кромѣ единаго временнаго и скоро отходящаго титула; однакожъ за блаженство отъ всѣхъ почитается съ такимъ лицомъ разглагольствіе дружеское. Какъ несравненно блаженнѣе, утѣшительнѣе и любезнѣе предъ Богомъ живымъ, безсмертнымъ, вѣчнымъ, великимъ и непостижимымъ, Творцемъ неба и земли, и вся въ руцѣ Своей содержащимъ, Иже есть Царь царей и Господь господей, Господь Вседержитель, вѣрою предстоять, вѣрою на пресвятое Его лице смотрѣть, и совѣты сердечные, которые Онъ и прежде предложенія нашего знаетъ, предлагать, и отъ Него взаимно отвѣтствующія милости ожидать! Хрістіане, истинные раби Хрістовы, такъ высочайшей милости и чести сподобляются! Сколько разъ въ молитвѣ обращаются, столько разъ съ Богомъ удостоеваются разглагольствовать, и Ему, какъ отцу чада, любезно и со страхомъ и благоговѣніемъ предстоя, о различныхъ вещахъ предлагать. Показуютъ Ему язвы свои грѣховныя, и отъ Него милостиваго исцѣленія просятъ: \textit{Отче, остави намъ долги наша}\footnote{Матѳ.~6,~12.}. Жалуются на врага своего, иже \textit{яко левъ рыкая, ходитъ искій кого поглотити}\footnote{1~Петр.~5,~8.}, и, прося отъ него защищенія: \textit{избави насъ отъ лукаваго}\footnote{Матѳ.~6,~13.}, просятъ помощи въ подвигѣ вѣры и благочестія. Благодарятъ Ему за оказанныя и оказуемыя Его благодѣянія, и впредь тѣхъ же благихъ не лишитися со смиреніемъ усердствуютъ. Представляютъ Ему дивная Его дѣла, въ славу Свою и нашу пользу сотворенная, и хвалятъ всемогущую силу Его, благость и человѣколюбіе непостижимое. И, что удивительнѣе и блаженнѣе, когда ни хотятъ и гдѣ хотятъ, всегда и вездѣ тое свободно совершаютъ. Ибо Богъ на всякомъ мѣстѣ есть, и всегда готовъ есть слушати бесѣды рабовъ Своихъ. Къ царямъ и прочіимъ высокимъ лицамъ не вездѣ и не всегда приступъ свободенъ есть имѣющимъ нужду. Къ Богу, Царю небесному и человѣколюбивому, благому и кроткому и вездѣсущему, не тако: вездѣ и всегда, когда хотятъ, раби Его приступаютъ къ Нему; всегда двери благодати къ Нему отверсты, и милость всѣмъ, съ вѣрою и благоговѣніемъ приступающимъ, безъ сумнѣнія обѣщается. И, аще бы вси въ едину минуту, сколько на свѣтѣ людей есть, приступили къ Нему подобающимъ образомъ, всѣхъ бы услышать готовъ былъ. Таковаго Господа, Царя и Отца своего имѣютъ хрістіане! Къ такому бесѣдуютъ, суще земля и пепелъ, еще же и грѣшніи! Безъ сумнѣнія сіе извѣстно, что недостоинъ человѣкъ съ Богомъ бесѣдовать; и хрістіане тое охотно признаютъ. Кто бо достоинъ съ Тѣмъ бесѣдовать, предъ которымъ и \textit{ангели лица своя покрываютъ, и не смѣютъ на Него взирати}\footnote{Ис.~6,~2.}. Но Богъ Самъ, благодатію Единороднаго Своего Сына, того удостоеваетъ ихъ. Хрістіане, разсуждая величество Божіе и Господа своего страшную славу, благоговѣютъ предъ Нимъ, падаютъ и покореніемъ духа глубоко смиряются; но, взирая на непостижимую Его благодать и милосердіе, и благодать всесильнаго Ходатая своего Іисуса Хріста, не отлагаютъ дерзновенія, якоже пророкъ глаголетъ: \textit{не на наша правды уповающе, повергаемъ моленіе наше предъ Тобою, но на щедроты Твоя многія, Господи}\footnote{Дан.~9,~18.}.

\paragraph*{§\:512.} Лишаются сего сладчайшаго утѣшенія и всякаго добра духовнаго: 1)~Которые оставляютъ молитвы, но упражняются въ мірскихъ бесѣдахъ и прочіихъ непотребностяхъ, отъ которыхъ не ино что относятъ съ собою, какъ уязвленную совѣсть. "--- 2)~Которые молятся, но безъ всякаго разсужденія и вниманія; много читаютъ молитвъ, или псалмовъ, но только искусствомъ языка, а что читаютъ, и сами не знаютъ; языкомъ молятся къ Богу, но умомъ отступаютъ отъ Него и обходятъ различныя мѣста, лица и житейскія попеченія. Къ Богу бо умомъ и сердцемъ приступаемъ, а не наружностію; умомъ и сердцемъ бесѣдуемъ съ Нимъ, а не единымъ языкомъ. Слово же не ино что есть, какъ вѣстникъ сердца и помышленія человѣческаго, которое, когда не можетъ вмѣститися въ сердцѣ, исходитъ чрезъ слово внѣ. А слово безъ ума и помышленія, согласнаго слову, не ино что есть, какъ гласъ трубный, или птичій и скотскій. И Богъ не слову, но сердцу нашему внимаетъ. "--- 3)~Которые хотятъ молитися, и грѣшить не престаютъ. Должно бо прежде примириться Богу покаяніемъ и вѣрою во Хріста Сына Божія, и тако во имя Хрістово приступать къ сердцевѣдцу Богу. Глаголетъ бо апостолъ: \textit{да отступитъ отъ неправды всякъ именуяй имя Господне}\footnote{2~Тим.~2,~19.}. И пророкъ глаголетъ: \textit{Богъ не хотяй беззаконія, Ты еси: не приселится къ Тебѣ лукавнуяй, ниже пребудутъ беззаконницы предъ очима Твоима. Возненавидѣлъ еси вся дѣлающія беззаконіе, погубиши вся глаголющія лжу: мужа кровей и льстива гнушается Господь}\footnote{Пс.~5,~5--7.}.

\subsection[Глава 14-я. О утѣшеніи отъ святаго Писанія.]{глава четвертаянадесять.\\\bfseries О утѣшеніи отъ святаго Писанія.}

\begin{quotation}\textit{Благъ мнѣ законъ устъ Твоихъ, паче тысящъ злата и сребра. Коль сладка гортани моему словеса Твоя, паче меда устомъ моимъ! Возрадуюся азъ о словесѣхъ Твоихъ, яко обрѣтаяй корысть многу}\footnote{118,~72,~103 и 162.}.\end{quotation}

\paragraph*{§\:513.} Не можно сказать, какую сладость, сладость же духовную, получаютъ раби Хрістовы и отъ сего источника, то"=есть, святаго Писанія! "--- 1)~Твердо они разумѣютъ и увѣрены въ томъ, что сіе святое Писаніе есть истиннѣйшее Бога своего, Егоже сердцемъ любятъ, слово, которымъ къ нимъ милостивно и отечески бесѣдуетъ, и любовное Царя небеснаго посланіе, которое чрезъ пророковъ и апостоловъ къ роду человѣческому послалъ, и въ немъ волю Свою святую открылъ. Какъ утѣшается человѣкъ, когда отъ царя своего посланіе получитъ! какъ тѣмъ любуется! Часто причитываетъ, и прочіимъ тое показуетъ, и хвалится тѣмъ, хотя и царь такойжде человѣкъ. Коль несравненно услаждаются раби Божіи о посланіи къ нимъ Царя своего небеснаго, когда разсуждаютъ, кто и кому Свое посланіе послалъ, и внимаютъ, что той есть Богъ ихъ, Создатель ихъ, Отецъ и благій Промыслитель ихъ, Царь и Господь славы, Которому они отъ сердца работаютъ, любятъ и со страхомъ покланяются. Таковый и толикій Господь благоволилъ къ подлымъ Своимъ рабомъ написать Духомъ Своимъ Святымъ, и послать чрезъ избранныхъ на тое рабовъ Своихъ, пророковъ и апостоловъ. "--- 2)~О, колико утѣшается сынъ, находящійся на чужой странѣ, когда получитъ отъ отца своего изъ дому отечества своего писаніе: читаючи бо его, думаетъ, какъ бы самого отца видѣлъ и слушалъ сладкую бесѣду его. Такъ точно, и паче еще болѣе того, утѣшаются сынове Божіи, истинные хрістіане, которые суть странники и пришельцы на земли, и издалека \textit{вѣрою взираютъ} плачевныма очима на отечество свое небесное, и \textit{воздыхаютъ, въ жилище свое небесное облещися желающе и, живуще въ тѣлѣ, отходятъ отъ Господа. Вѣрою бо ходятъ, а не видѣніемъ}\footnote{2~Кор.~5,~2,~6 и 7.}. Какъ, глаголю, въ сей плача юдоли утѣшаются, когда Отца своего небеснаго читаютъ любезное посланіе, и, взирая на него и читая тое, какъ бы Самого Его зрѣли, и слышали сладчайшую бесѣду Его, съ Давидомъ поя и глаголя: \textit{коль сладка гортани моему словеса Твоя, паче меда устомъ моимъ!} "--- 3)~Въ семъ святѣйшемъ и божественномъ посланіи читаютъ и видятъ Бога Отца своего, и божественнѣйшія Его свойства, никому кромѣ Его не приличныя: всемогущество, премудрость, благость, милосердіе, истину, вездѣсущіе, вѣчность, величество, страшную славу Его, и проч., и по премногу радуются о томъ, яко толикому Богу, Который единъ есть истинный Богъ, Богъ боговъ и Господь господей, Царь царствующихъ и Господь господствующихъ, во свѣтѣ живый неприступнѣмъ, имена съ сердцами своими отдали и вѣрою записали, и въ сей радости духовной восклицаютъ къ Нему съ пророкомъ: \textit{Господи Боже мой, возвеличился еси зѣло! во исповѣданіи и въ велелѣпоту облеклся еси, одѣяйся свѣтомъ яко ризою, простираяй небо яко кожу, покрываяй водами превыспренняя своя, полагаяй облаки на восхожденіе Свое, ходяй на крилу вѣтреню, творяй Ангелы Своя духи и слуги Своя, пламень огненный, основаяй землю на тверди ея}\footnote{Пс.~103,~1--5 и слѣд.}, и проч. "--- 4)~Въ семъ видятъ начало бытія всѣхъ вещей и своего: яко все изъ ничего единъ Тріѵпостасный Богъ, въ Котораго вѣруютъ, премудро сотворилъ. Видятъ, что человѣкъ особенное нѣкое, и паче прочіихъ вещей совершеннѣйшее есть созданіе, превосходнымъ образомъ, особливымъ совѣтомъ созданъ: \textit{сотворимъ человѣка}, глаголалъ Богъ; высокою и умомъ непонятною честію почтенъ: \textit{сотворимъ человѣка по образу Нашему и по подобію Нашему}\footnote{Быт.~1,~26.}. Видятъ и удивляются благости Божіей, которую въ семъ великомъ дѣлѣ на человѣка изліялъ. "--- 5)~Не безъ воздыханія и скорби читаютъ тое, что человѣкъ, который такъ высоко отъ Бога былъ почтенъ, въ такъ глубокую несчастія яму завистію діавольскою упалъ, и который тако отъ Бога возлюбленъ и вышеестественно одаренъ былъ, врагомъ Его сотворился, и со смиреніемъ и жалѣніемъ исповѣдаются: \textit{согрѣшихомъ со отцы нашими, беззаконновахомъ, неправдовахомъ}\footnote{Пс.~105,~6.}. \textit{Вси бо согрѣшиша, и лишени суть славы Божія}\footnote{Римл.~3,~23.}. Но, когда разсуждаютъ, что тойжде человѣкъ падшій такъ чуднымъ образомъ возставленъ и паче прежняго превознесенъ отъ Бога; тутъ находятъ большую и важнѣйшую причину непостижимой благости Создателя своего и Спасителя удивляться: яко отступника Своего паки такъ чудно любовію Своею къ Себѣ привлеклъ, "--- и въ семъ удивленіи достойно восклицаютъ съ пророкомъ: \textit{Господи! что есть человѣкъ, яко познался еси ему? или сынъ человѣчь, яко вмѣняеши его}\footnote{Пс.~143,~3.}? Читаютъ и видятъ въ томъ Божественномъ и святомъ Писаніи, что \textit{тако возлюби Богъ міръ, яко и Сына Своего Единороднаго далъ есть, да всякъ, вѣруяй въ Онь, не погибнетъ, но имать животъ вѣчный}\footnote{Іоан.~3,~16.}. И, чувствуя сію небеснаго Отца въ сердцахъ своихъ любовь, къ взаимной любви разжигаются, и глаголютъ съ пророкомъ: \textit{Возлюблю Тя, Господи, крѣпосте моя! Господь утвержденіе мое и прибѣжище мое, и избавитель Мой, Богъ Мой, помощникъ мой, и уповаю на Него: защититель мой, и рогъ спасенія моего, и заступникъ мой}\footnote{Пс.~17,~2 и 3.}! "--- 6)~Въ семъ святомъ посланіи видятъ описанное небесное званіе свое, яко \textit{отъ тьмы призваны въ чудный Божій свѣтъ, и отъ области сатанины въ царство Хріста Бога}\footnote{1~Петр.~2,~9; Дѣян.~26,~18; Кол.~1,~13.}, "--- яко \textit{суть родъ избранъ, царское священіе, языкъ святъ, люди обновленія}\footnote{1~Петр.~2,~9.}. Видятъ высокое и небесное свое, вѣрою во Хріста имъ данное благородіе, но еще не явленное, якоже имъ глаголетъ апостолъ во утѣшеніе: \textit{возлюбленніи, нынѣ чада Божія есмы, и не у явися, что будемъ? Вѣмы же, яко, егда явится, подобни Ему будемъ: ибо узримъ Его, якоже есть}\footnote{1~Іоан.~3,~2.}. \textit{И чаютъ откровенія сыновъ Божіихъ}\footnote{Римл.~8,~19.}, и \textit{хвалятся упованіемъ славы Божія}\footnote{5,~2.}. "--- 7)~Видятъ прекрасное, пресладкое, превожделѣнное отечество свое, различно отъ Отца небеснаго Духомъ Своимъ Святымъ описанное, и имъ во утѣшеніе и прохлажденіе во время скорби, какъ живую воду во время жажды, предложенное; и отъ многобѣдственнаго міра сего, какъ отъ земли чуждей и дальней страны, слезными вѣры очами взираютъ на него и о семъ \textit{воздыхаютъ, въ жилище свое небесное облещися желающе}\footnote{2~Кор.~5,~2.}. "--- 8)~Видятъ тамо спасительное Отца своего небеснаго наставленіе, како подобаетъ имъ безбѣдно въ мірѣ семъ, въ сей чуждей странѣ, обращатися, и посредѣ рода строптиваго и развращеннаго безсоблазненно жительствовати; какъ отъ супостата своего, діавола, который, какъ \textit{левъ рыкая, ходитъ искій кого поглотити}\footnote{1~Петр.~5,~8.}, "--- и отъ злыхъ слугъ его берещися, которыи вси пресѣкаютъ путь къ отечеству небесному; и, не надѣяся на силу свою, возводятъ плачевныя свои очи ко Отцу, живущему на небесѣхъ, моляся: \textit{Ты, Господи, сохраниши ны и соблюдеши ны отъ рода сего и во вѣкъ}\footnote{Пс.~11,~8.}. "--- 9)~Видятъ тамо, какъ въ чистомъ зерцалѣ, неисправности свои предъ Отцемъ своимъ, немощи и недостатки, въ которыя и нехотя, во плоти немощной странствующе, впадаютъ; но, смиренно призная немощь свою, молятся къ Нему: \textit{Отче! остави намъ долги наша, якоже и мы оставляемъ должникомъ нашимъ}\footnote{Матѳ.~6,~12.}, и получаютъ милостивное оставленіе. "--- 10)~Здѣ видятъ, что какъ всѣмъ, такъ и имъ, яко смертнымъ, слѣдуетъ отдати долгъ естеству смертному, единою умрети, и отъ сего міра къ будущему прейти вѣку, и явитися лицу Божію; и, помышляя часъ сей, готовятся къ исходу. "--- 11)~Паки видятъ тамо, что плоть погребается въ землѣ и истлѣваетъ, но въ послѣдній день, когда Богъ благоволитъ, паки съ душею соединится, и въ новый, прекрасный, нетлѣнный, безсмертный облечется видъ, и процвѣтетъ яко кринъ, и ожидаютъ безъ сумнѣнія прекрасныя тоя весны. "--- 12)~Видятъ наконецъ тамо, яко Господь ихъ, въ Котораго вѣруютъ и работаютъ Ему, \textit{пріидетъ въ славѣ Своей, и вси святіи ангели съ Нимъ, и сядетъ на престолѣ славы Своея, и соберутся предъ Нимъ вси языцы}\footnote{Матѳ.~25,~31 и 32.}, \textit{да пріиметъ кійждо, яже съ тѣломъ содѣла, или блага или зла}\footnote{2~Кор.~5,~10.}. Пришествіе сіе Господа, какъ прочіимъ, яко нечестивымъ и непослушавшимъ благовѣствованія Его, страшно и нестерпимо (кто бо стерпитъ гнѣвъ Господень?), такъ себѣ, яко вѣрующимъ въ Грядущаго судити живыхъ и мертвыхъ, утѣшительно и радостно быти ощущаютъ и познаютъ: яко тогда \textit{приближается избавленіе ихъ}\footnote{Лук.~21,~28.}. И хотя страхъ праведнаго того суда Божія ударяетъ въ сердца и ихъ; но вѣра, которою прилѣпились къ милостивому своему Избавителю, \textit{возлюбившему ихъ и предавшему Себе по нихъ}\footnote{Гал.~2,~20.}, живое смущаемымъ сердцамъ ихъ утѣшеніе подаетъ. Аще бо и вѣдаютъ, яко \textit{никто предъ Нимъ оправдитися не можетъ; и кто постоитъ, аще беззаконія назритъ}\footnote{Пс.~142,~2; 129,~3.}, и находятъ въ совѣсти своей немощи и согрѣшенія противу святаго закона Его; но покаяніемъ искреннимъ и надеждою на пострадавшаго за грѣхи людскіе и Агнца Божія, вземлющаго грѣхи міра, утверждаются въ непоколебимой милости Божія надеждѣ. И тако раби Хрістовы въ святомъ Писаніи обрѣтаютъ великое утѣшеніе. Которое писаніе имъ есть, яко чистое зеркало, въ которомъ, странствуя на земли чуждей, видятъ небеснаго своего Отца, и радуются о Немъ; видятъ божественный Его милостивый промыслъ, безприкладную милость Его, и лобызаютъ ю лобзаніемъ сердечныя любви; видятъ величество и страшную славу Его, и смиренно покланяются ей; видятъ премудрость, и удивляются той; видятъ святѣйшую правду Его, и прославляютъ ее; видятъ преславныя дѣла Его, и поютъ Его; видятъ божественное благородіе свое, отъ Отца небеснаго имъ вѣрою во Хріста данное, и благодарятъ Ему. Въ семъ чистѣйшемъ зерцалѣ видятъ и немощи своя, и слезами и вѣрою въ Сына Его отираютъ ихъ; видятъ и блаженство свое уготованное, и желаніемъ стремятся къ нему, видятъ наконецъ все, что до истиннаго ихъ касается блаженства, и противу всякихъ бѣдствій и скорбей, недуговъ и немощей, яко въ пребогатой аптекѣ, обрѣтаютъ живительныя врачевства.

\paragraph*{§\:514.} Сладчайшаго сего утѣшенія лишаются тіи хрістіане, которые 1)~оставляютъ божественное сіе сокровище и небеснаго Отца неоцѣненный даръ, но забавляются въ книжкахъ, которыя плоть увеселяютъ и къ суетѣ паче склоняютъ, нежели отводятъ отъ нея; и дѣлаютъ подобно тѣмъ, которые, оставивше живый чистыя воды источникъ, прибѣгаютъ къ мутнымъ кладезямъ и жажду свою хотятъ утолить. "--- 2)~Которые читаютъ или слушаютъ святое Писаніе ради пренія, а не ради утѣшенія, чтобы только знать, гдѣ что написано, чтобы остроумными быть и показаться предъ человѣками, а не того ради чтобы искусными быть хрістіанами, Бога познать и волю Его святую, начало и конецъ своего житія, блаженства и окаянства своего начало и состояніе. Таковыи мудрецы и хитрословцы сокровенныя въ святомъ Писаніи манны не находятъ, хотя бы и все тое въ памяти содержали: понеже не на такій конецъ читаютъ его, на какій отъ Духа Святаго "--- Творца Божія слова "--- сочинено, и чрезъ святыхъ Его рабовъ, пророковъ и апостоловъ, написано. "--- \textit{(Смотри еще главу 1"~ю первыя статьи въ 1"~й книгѣ и главу 2"~ю пятой статьи во 2"~й книгѣ)}.

\subsection[Глава 15-я. О присутствіи Божіемъ съ хрістіанами въ скорбехъ ихъ.]{глава пятаянадесять.\\\bfseries О присутствіи Божіемъ съ хрістіанами въ скорбехъ ихъ.}

\begin{quotation}\textit{Се Азъ съ вами есмь во вся дни до скончанія вѣка!} глаголетъ Хрістосъ\footnote{Матѳ.~28,~20.}.\end{quotation}

\paragraph*{§\:515.} Вѣдая Спаситель нашъ Господь Іисусъ Хрістосъ, что раби Его вѣрніи, истинные хрістіане, въ мірѣ скорбни будутъ, якоже и предреклъ имъ: \textit{въ мірѣ скорбни будете}\footnote{Іоан.~16,~33.}, "--- того ради обѣщался съ ними, по Своему человѣколюбію, всегда неотлученъ быти, аще видимо и отлучился; чего ради во утѣшеніе имъ сіе слово сказалъ: \textit{се Азъ съ вами есмь во вся дни до скончанія вѣка!} Сію надежду имѣя, вѣрніи съ церковію радостно къ Нему восклицаютъ: «О Божественнаго! о любезнаго! о сладчайшаго Твоего гласа! съ нами бо неложно обѣщался еси быти до скончанія вѣка, Хрісте. Егоже вѣрніи утвержденіе надежды имуще, радуемся»\footnote{\textit{Пѣснь 9"~я кан. Пасхи}.}. Сіе дражайшее и сладчайшее Свое съ вѣрными присутствіе и въ Ветхомъ Завѣтѣ объявлялъ Богъ. Іакову, святому патріарху, глагола Богъ: \textit{Азъ есмь Богъ Авраама отца твоего: не бойся! съ тобою бо есмь, и благословлю тя}\footnote{Быт.~26,~24.}. И паки: \textit{Азъ сниду съ тобою во Египетъ, и Азъ возведу тя до конца}\footnote{46,~4.}. И паки къ избранному Израилю глаголетъ: \textit{не бойся! съ тобою бо есмь, не прельщаю: Азъ бо есмь Богъ твой, укрѣпивый тя}, и проч.\footnote{Исх.~41,~10 и слѣд.} И паки: \textit{Мой еси ты: и аще преходиши сквозѣ воду, съ тобою есмь, и рѣки не покрыютъ тебе: и аще сквозѣ огнь пройдеши, не сожжешися, и пламень не опалитъ тебе}\footnote{Ис.~43,~1 и 2.}. Тако пишется о Іосифѣ святомъ, яко \textit{бѣ Богъ съ нимъ}\footnote{Дѣян.~7,~9.}, пишется о Израилѣ: \textit{яко Господь Богъ твой, Сей предъидый съ вами, и не отступитъ отъ тебе, ниже оставитъ тя}\footnote{Второз.~31,~6.}. Сіе же присутствіе Божіе съ вѣрными Своими рабами не такъ, какъ съ прочіими людьми и тварьми, поелику Богъ, яко вездѣсущій, вездѣ и во всякомъ мѣстѣ есть, разумѣется, "--- но благодатное и утѣшительное и хранящее ихъ, яко чадъ Своихъ, якоже о томъ Моисей святый въ пѣсни своей глаголетъ: \textit{обыде его (Израиля) и наказа его, и сохрани его, яко зѣницу ока; яко орелъ покры гнѣздо свое, и на птенцы своя возжелѣ}, и проч.\footnote{32,~10 и 11.} Симъ благодатнымъ Божіимъ присутствіемъ ободряеми, вѣрніи дерзаютъ и глаголютъ къ Богу со пророкомъ: \textit{аще и пойду посредѣ сѣни смертныя, не убоюся зла, яко Ты со мною еси}\footnote{Пс.~22,~4.}; дерзаютъ во узахъ, темницахъ, изгнаніяхъ, плѣненіяхъ, бранехъ, страданіяхъ и самыхъ смертехъ, яко Господь близъ ихъ есть, сохраняяй и спасаяй ихъ, и глаголютъ со пророкомъ: \textit{съ нами Богъ}\footnote{Ис.~8,~10.}; глаголютъ со Псаломникомъ: \textit{Господь пасетъ насъ, и ничтоже насъ лишитъ}\footnote{Пс.~22,~1.}; восклицаютъ со апостоломъ: \textit{аще Богъ по насъ, кто на ны}\footnote{Римл.~8,~31.}? Разсуди же, возлюбленный хрістіанине, какое большее можетъ быть утѣшеніе боголюбивой душѣ, какъ Бога присутствующаго имѣть и на Него очами вѣры смотрѣть? Идѣже бо Богъ съ Своимъ благодатнымъ присутствіемъ, тамо всякое утѣшеніе, радость, и веселіе. Почувствовалъ сіе Псаломникъ на себѣ: \textit{азъ выну съ Тобою; удержалъ еси руку десную мою, и совѣтомъ Твоимъ наставилъ мя еси, и со славою пріялъ мя еси. Что бо ми есть на небеси? и отъ Тебе что восхотѣхъ на земли? Исчезе сердце мое и плоть моя, Боже сердца моего и часть моя Боже во вѣкъ}\footnote{Пс.~72,~23--26.}. Чувствовали мученицы святіи, и въ самыхъ жесточайшихъ страданіяхъ веселилися, и надъ діаволомъ и слугами его торжествовали. Чувствовали и вси святіи, и всякія находящія скорби добльственно претерпѣвали. Чувствуютъ и нынѣ работающіи Ему со страхомъ, и вѣрою и любовію Ему прилѣпляющіися: откуду всякую сладость и утѣху міра сего, какъ гной, презираютъ и гнушаются, симъ единымъ внутреннимъ своимъ сокровищемъ довольствуяся и утѣшаяся.

\subsection[Глава 16-я. О побѣдѣ хрістіанской.]{глава шестаянадесять.\\\bfseries О побѣдѣ хрістіанской.}

\begin{quotation}\textit{Всякъ рожденный отъ Бога, побѣждаетъ міръ: и сія есть побѣда, побѣдившая міръ, вѣра наша. Кто есть побѣждаяй міръ, токмо вѣруяй, яко Іисусъ есть Сынъ Божій}\footnote{1~Іоан.~5,~4 и 5.}?\end{quotation}

\paragraph*{§\:516.} Побѣда безъ подвига, подвигъ безъ брани, брань безъ враговъ не бываетъ. Хрістіанъ, рабовъ Хрістовыхъ общій и всегдашній есть врагъ діаволъ съ темнымъ своимъ полчищемъ, демонами и служителями своими. Противу сего врага имъ непрестанная брань и подвигъ предлежитъ. Когда Израиль рукою крѣпкою и мышцею высокою избавился отъ работы египетскія Фараонъ, египетскій царь, съ воинствомъ своимъ гналъ въ слѣдъ его, хотя его паки себѣ поработить: тако хрістіанъ, новаго Израиля, иже суть \textit{чада Авраамля духовная}\footnote{Римл.~4,~11; Галат.~3,~17 и 29.}, гонитъ сатана съ своимъ мрачнымъ воинствомъ, гонитъ, яко отъ темныя власти и работы силою Хрістовою избавившихся, дабы ихъ паки себѣ покорить. Израилю ветхозаконному, идущему въ землю обѣтованную, многіи враги препятствіе чинили и не пущали до земли оной: тако новаго Израиля, хрістіанъ, тщится не допуститъ тойжде врагъ діаволъ до небеснаго отечества, которое обѣщалъ Богъ вѣрующимъ во имя Его. Съ симъ супостатомъ имѣютъ хрістіане брань и подвигъ всегдашній.

\paragraph*{§\:517.} Надобно признаться, что брань сія и подвигъ трудный есть, что слѣдующія обстоятельства показуютъ: 1)~Враги сіи хрістіанскіи не видимыи, но невидимыи суть; не кровь и плоть, но духи злобы поднебесныи, якоже апостолъ глаголетъ: \textit{нѣсть наша брань къ крови и плоти, но къ началомъ, и ко властемъ, и къ міродержителемъ тьмы вѣка сего, къ духовомъ злобы поднебеснымъ}\footnote{Еф.~6,~12.}. И потому на сей духовной брани не такъ бываетъ, какъ на брани міра сего, гдѣ другъ друга враги видятъ и опасаются. Но здѣ враги наши насъ видятъ, назираютъ что дѣлаемъ, что начинаемъ, къ чему дѣло, слово и начинаніе наше склоняемъ, примѣчаютъ; но мы ихъ, яко духовъ, не видимъ. "--- 2)~Суть безъ мѣры злобныи и враждебныи намъ, и всякимъ образомъ ищутъ насъ низложить и погубить. "--- 3)~Суть лукавыи и хитрыи и чрезъ толико вѣковъ самимъ искусствомъ научившіися; знаютъ добрѣ, какъ приступить, съ какой стороны напасть, какимъ оружіемъ кого уязвлять; знаютъ подъ видомъ добра и зло намъ предлагать, какъ подъ прикрытіемъ меда ядъ, и подъ видомъ вреда отъ добра отводить, и часто мнимымъ нашимъ добромъ борютъ насъ, приводя въ высокоуміе. "--- 4)~Никогда не престаютъ бороть насъ. Мы спимъ и почиваемъ, но они никогда не почиваютъ, но всегда и вездѣ на погибель нашу бодрствуютъ; и когда показуются насъ оставлять, тогда съ большимъ стремленіемъ хотятъ на насъ напасть, и болѣе уязвить. "--- 5)~Оружіе свое противу насъ въ насъ самихъ находятъ враги сіи, и нашими насъ орудіями борютъ. Сердце наше растлѣнное имъ есть какъ влагалище всякихъ орудій, которыми насъ уязвляютъ. \textit{Отъ сердца бо исходятъ помышленія злая}, глаголетъ Хрістосъ, Царь нашъ\footnote{Матѳ.~15,~19.}. Сими орудіями всегда на насъ стрѣляютъ. Умъ нашъ, языкъ, очи, уши и прочіи уды орудіями неправды тщатся учинить, и тако совѣсть нашу уязвить, и побѣду отъ насъ отнять. "--- 6)~Тако возстая противу насъ, не имѣній нашихъ, не земли, не градовъ, ни головы нашей, ни живота временнаго лишить насъ хотятъ, но, что дражайшее есть всего свѣта, вѣчнаго спасенія. Противу толикихъ и такихъ враговъ хрістіанамъ брань и подвигъ предлежитъ: который подвигъ не дремлющимъ окомъ, ни унывающимъ сердцемъ, но бодренно и неусыпно отправлять имъ должно, когда хотятъ непосрамленными быти предъ Подвигоположникомъ своимъ Іисусомъ Хрістомъ. Откуду увѣщаваетъ насъ Хрістосъ Царь нашъ и Господь: \textit{бдите и молитеся, да не внидете въ напасть}\footnote{26,~41.}. Увѣщаваютъ Его апостоли святіи, посланники Его и князи наши: \textit{трезвитеся, бодрствуйте, зане супостатъ вашъ діаволъ, яко левъ рыкая, ходитъ, искій кого поглотити}\footnote{1~Петр.~5,~8.}. \textit{Мняйся стояти, да блюдется, да не падетъ}\footnote{1~Кор.~10,~12.}. \textit{Братіе моя! возмогайте во Господѣ, и въ державѣ крѣпости Его: облецытеся во вся оружія Божія, яко возмощи вамъ стати противу кознемъ діавольскимъ}\footnote{Еф.~6,~10 и 11.}. И паки Хрістосъ: \textit{держи, еже имаши, да никтоже пріиметъ вѣнца твоего}\footnote{Апок.~3,~11.}. Отъ сихъ и прочіихъ въ святомъ Писаніи обрѣтающихся увѣщаній видно, коль тяжкая хрістіанамъ брань есть съ сими безплотными врагами. И сами вѣрніи по вся дни дознаютъ лютая ихъ нападенія, когда чрезъ злые помыслы, то вѣру ихъ, яко основаніе всего духовнаго блаженства, разрушить, то любовь погасить и свои беззаконій плевелы на нивѣ сердецъ ихъ посѣять тщатся.

\paragraph*{§\:518.} Хотя и страшная сія брань намъ есть, возлюбленный хрістіанине, однакожъ не должно намъ отчаяваться, яко имѣемъ Вождя и Царя таковаго, Котораго враги сіи наши и имени трепещутъ. Той за насъ стоитъ и побораетъ; Той знаетъ ихъ на насъ козни, замыслы, ухищренія ихъ, и, яко сѣти паутинныя, силенъ есть ихъ разорвати и уничтожити. Отнялъ"=было у насъ побѣду врагъ сей въ прародителѣхъ нашихъ Адамѣ и Евѣ, и подъ свою темную власть плѣнилъ насъ, и торжествовалъ надъ нами, яко страшный исполинъ и мучитель надъ побѣжденными; но Хрістосъ Сынъ Божій вступился за насъ, сразился со врагомъ нашимъ за насъ и побѣдилъ его со всѣмъ темнымъ полчищемъ его, отнялъ у него корысть похищенную, возвратилъ плѣнъ, и плѣнилъ плѣнившаго насъ, изъ темныя его власти, мучительства и горькія работы вывелъ насъ, и подалъ намъ свободу, да Ему, яко Царю праведному, \textit{Егоже царство и власть вѣчная}\footnote{Дан.~4,~31; Пс.~144,~13.}, свободнымъ духомъ воинствуемъ и служимъ, вѣрою и любовію послѣдуемъ Подвигоположнику, Князю, Царю и Вождю нашему. Сей нашъ Заступникъ милостивый, и всесильный враговъ нашихъ Побѣдитель ободряетъ насъ силою Своею, и унывающихъ оживляетъ: \textit{дерзайте! яко Азъ побѣдихъ міръ}\footnote{Іоан.~16,~33.}. Обѣщался до скончанія міра быти съ нами, *помогати намъ, ополчатися съ нами* противу враговъ нашихъ, и побѣждати ихъ: \textit{се Азъ съ вами есмь во вся дни до скончанія вѣка}\footnote{Матѳ.~28,~20.}. Не оставляетъ бо рабовъ и воиновъ Своихъ, которыхъ возлюбилъ и къ подвигу сему позвалъ; но помогаетъ имъ, укрѣпляетъ, заступаетъ въ нужныхъ случаяхъ, и Самъ Собою въ нихъ враговъ Своихъ побѣждаетъ и поражаетъ. Сего ради рабомъ Своимъ и воиномъ оружіе приличное подвигу сему даровалъ. Брань бо и подвигъ безъ оружія не отправляется, якоже отъ брани міра сего, гдѣ плоть съ плотію и человѣкъ съ человѣкомъ сражается, видѣти всякому можно.

\paragraph*{§\:519.} Приличное брани и подвигу сему оружіе не плотское, якоже и брань сія не плотская, но духовная. Духъ есть врагъ, и плотскихъ орудій, какъ"=то: меча, копія, стрѣлъ, луковъ и прочіихъ, не боится, яко не уязвляется ими, но паче посмѣвается имъ. Оружіе убо иное противу его подалъ намъ Царь и Вождь нашъ Хрістосъ, якоже апостолъ Павелъ, храбрый воинъ Хрістовъ, глаголетъ: \textit{оружія воинства нашего не плотская, но сильна Богомъ на разореніе твердемъ}\footnote{2~Кор.~10,~4.}. Оружія убо хрістіанскаго воинства суть: 1)~Вѣра, якоже апостолъ глаголетъ: \textit{сія есть побѣда, побѣдившая міръ, вѣра наша}\footnote{1~Іоан.~5,~4.}. Вѣра же всю крѣпость и силу свою получаетъ отъ Хріста Сына Божія, на Которомъ, яко на твердомъ и непоколебимомъ основаніи, утверждается и подвизается противу сатаны и слугъ его. Сего ради далѣе глаголетъ апостолъ: \textit{кто есть побѣждаяй міръ, токмо вѣруяй, яко Іисусъ есть Сынъ Божій}\footnote{1~Іоан.~5,~5.}. Въ сіе оружіе и Петръ святый учитъ насъ облекаться и противитися противу супостата нашего діавола: \textit{емуже}, рече, \textit{противитеся тверди вѣрою}\footnote{1~Петр.~5,~9.}. Вѣра же требуетъ того, чтобы мы, ополчаяся противу сего супостата нашего, не на свою силу и разумъ, крѣпость и искусство, но на того, въ кого вѣруемъ, надѣялися, и отъ него единаго помощи, наставленія и побѣды ожидали. Нѣтъ бо тамо и вѣры, гдѣ нѣтъ надежды на того, въ кого вѣруется. И потому должно въ своихъ силахъ отчаяться, чтобы благонадежными быть въ силѣ Хрістовой. Откуду Павелъ святый глаголетъ: \textit{сладцѣ похвалюся паче въ немощехъ моихъ, да вселится въ мя сила Хрістова}\footnote{2~Кор.~12,~9.}. И Псаломникъ на Того силу уповаетъ, и дерзая глаголетъ: \textit{о Бозѣ сотворимъ силу: и Той уничижитъ стужающія намъ}\footnote{Пс.~59,~14.}. \textit{Той} бо \textit{препоясуетъ} рабовъ Своихъ \textit{силою}\footnote{17,~33.}; \textit{научаетъ руцѣ ихъ на ополченіе, персты ихъ на брань}\footnote{143,~1.}, яко \textit{Господь крѣпокъ и силенъ, Господь силенъ въ брани}\footnote{23,~8.}. На сего всесильнаго поборника уповать и дерзать чадомъ своимъ велитъ святая церковь: «Поборника имамы изъ Нея (Богородицы Дѣвы) рождшагося Господа. Дерзайте убо, дерзайте, людіе Божіи! Ибо Той побѣдитъ враги, яко всесиленъ»\footnote{\textit{Гласа 1"~го Богородич}.}. И на другомъ мѣстѣ поетъ Ему: «Ты моя крѣпость, Господи, Ты моя и сила!» и проч.\footnote{\textit{Гласа 8"~го пѣснь 4"=я}.} Но при томъ должно и самимъ не унывать и не лѣнитися, но трезвится и бодрствовать, якоже Петръ святый увѣщаваетъ: \textit{трезвитеся, бодрствуйте}\footnote{1~Петр.~5,~8.}. Хрістосъ бо помогаетъ труждающимся, а не лежащимъ и спящимъ; и брань требуетъ подвига. Почему лѣнивый и сонливый всуе надѣется на Помощника Хріста. "--- 2)~Глаголъ Божій, "--- мечь духовный, по словеси апостола: \textit{и мечь духовный, иже есть глаголъ Божій}\footnote{Еф.~6,~17.}, "--- есть такожде оружіе на враговъ. Симъ оружіемъ поразилъ Самъ Хрістосъ сатану, егда отъ него искушаемъ былъ въ пустынѣ, и противу предложенныхъ его козней Писаніе приводилъ, которымъ посрамлялъ искусителя\footnote{Матѳ.~4,~1--11.}. Тако образъ подалъ и намъ въ нужныхъ случаяхъ тѣмжде оружіемъ противу его съ Божіею помощію боротися. Къ отчаянію ли насъ приводитъ, "--- милосердіе Божіе, грѣшникомъ кающимся обѣщанное, да предлагаемъ. Въ гордость ли возноситъ, "--- отвѣщаемъ ему, яко земля и пепелъ есмы, и \textit{яко всякъ возносяйся смирится}\footnote{Лук.~18,~14.}, и \textit{гордымъ Богъ противится}\footnote{1~Петр.~5,~5.}. Сребролюбіемъ ли, сластолюбіемъ и прочіими стрѣлами хощетъ уязвить насъ, приличныя противу стрѣлъ его изъ тогожде Писанія да взимаемъ пособія. "--- 3)~Противу гордаго духа, духа злобы, зависти и ненависти сильное и дѣйствительное оружіе есть смиреніе, кротость, любовь: сими оружіями онъ посрамляется и уступаетъ. Противу его гордиться, его ненавидѣть и на него единаго злобствовать, яко противника Божія и врага человѣческаго рода, похвально. Но предъ Богомъ и человѣками смиряться, обидѣвшему человѣку простить, и ненавидящаго брата любить, "--- язвительное есть ему оружіе. Къ сему Хрістосъ увѣщаваетъ: \textit{любите враги ваша, благословите кленущія вы, добро творите ненавидящимъ васъ, и молитеся за творящихъ вамъ напасть и изгонящія вы}\footnote{Матѳ.~5,~44.}. И апостолъ: \textit{не побѣжденъ бывай отъ зла, но побѣждай благимъ злое}\footnote{Римл.~12,~21.}. "--- 4)~Сильное и дѣйствительнѣйшее есть на супостата сего оружіе "--- молитва смиренная, усердная и чистая, Которою хотя и всегда должно ограждать себе и противу непріятельскаго нападенія приготовлять и вооружать, но паче тогда, когда онъ помыслами злыми нападаетъ на насъ, и тѣми, какъ стрѣлами разженными, хощетъ уязвить сердце и совѣсть нашу. Молитвою бо испрашивается всесильная Божія помощь, и человѣкъ, отчаявшися въ силахъ своихъ, на всемогущую Божію силу надѣется, которою укрѣпленъ стоитъ противу врага своего, "--- и въ своей силѣ немощенъ, но въ Божіей крѣпокъ, "--- посрамляетъ гордаго и высокоумнаго духа, плоть сый и кровь, что ему паче всякой муки язвительнѣйше. Откуду Хрістосъ къ молитвѣ такъ сильно увѣщаваетъ насъ: \textit{бдите и молитеся, да не внидете въ напасть}\footnote{Матѳ.~26,~41.}. И какъ молитися, показалъ: \textit{не введи насъ во искушеніе}, Отче, \textit{но избави насъ отъ лукаваго}\footnote{Лук.~11,~4.}. И апостолъ глаголетъ: \textit{непрестанно молитеся}\footnote{1~Сол.~5,~17.}. Сіе есть воинства хрістіанскаго оружіе! Симъ оружіемъ вси святіи ополчались противу общаго врага хрістіанскаго діавола; симъ противилися ему, и силою Царя и Господа своего побѣждали его. Симъ и намъ подобаетъ препоясатися и стати, да возможемъ \textit{противитися въ день лютъ}\footnote{Еф.~6,~13.}, то"=есть, въ часъ нападенія его на насъ.

\paragraph*{§\:520.} Побѣда хрістіанская въ мужественномъ, какъ видно, терпѣніи состоитъ. На брани міра сего тотъ побѣждаетъ, кто гонитъ непріятеля своего и поражаетъ, но на хрістіанской сей духовной брани не тако бываетъ, но все противно тому. Здѣ тотъ побѣждаетъ, кто гонимый терпитъ, обидимый не отмщеваетъ, злословимый благословляетъ, лишаемый не ищетъ, укоряемый противу не укоряетъ, и всякое находящее искушеніе и злостраданіе великодушно претерпѣваетъ. О семъ глаголетъ апостолъ: \textit{не побѣжденъ бывай отъ зла, но побѣждай благимъ злое}\footnote{Римл.~12,~21.}. Сія есть хрістіанская побѣда "--- побѣждать благимъ злое. Какъ бо брань здѣ духовная, яко духъ вѣры съ духомъ злобы сражается и подвизается: такъ и побѣда духовная есть, яко духомъ смиренія духъ гордости, духомъ кротости духъ злобы, духомъ терпѣнія духъ гнѣва и отмщенія побѣждается. Евангеліе святое научаетъ насъ, что Хрістосъ, Царь нашъ и Подвигоположникъ, когда въ подвигъ за насъ вступилъ, не инымъ чимъ побѣдилъ сихъ враговъ нашихъ, какъ терпѣніемъ креста, то"=есть, всякаго злостраданія. Съ терпѣніемъ началъ, съ терпѣніемъ провождалъ, съ терпѣніемъ и окончилъ святое Свое и многострадальное на землѣ житіе. Началъ въ самомъ младенчествѣ крестъ нести, гонимый отъ беззаконнаго Ирода: началось и темное врага нашего діавола царство разрушаться. Хулы и поношенія отъ беззаконныхъ за истину терпѣлъ: претерпѣвалъ и врагъ истины удары. Суду предсталъ неправедному, и Судія всѣхъ судимъ былъ неправедно: осужденъ и князь міра сего, якоже рече Самъ: \textit{яко князь міра сего осужденъ бысть}\footnote{Іоан.~16,~11.}. На крестѣ пригвожденъ былъ: уязвился и упалъ супостатъ нашъ. Тако низложенъ бысть клеветникъ нашъ, и поверженъ на землю, яко мертвецъ, и даровалася намъ побѣда надъ нимъ. Всесильному и страшному Побѣдителю Хрісту, Царю нашему, послѣдовали апостоли, мученицы и вси святіи; и побѣжденнаго врага силою побѣдителя мужественно попрали. Сей образъ побѣды подалъ намъ Избавитель нашъ, да и мы побѣждаемъ сего душъ нашихъ врага не инымъ чимъ, какъ терпѣніемъ. Откуду святый Златоустъ глаголетъ: «Научилъ насъ Хрістосъ, что діавола не знаменіями, но злостраданіемъ и долготерпѣніемъ побѣждати подобаетъ»\footnote{\textit{Бес.~13"~я на Матѳея}.}.

\paragraph*{§\:521.} Гонятъ"=де насъ и озлобляютъ люди злыи, а не діаволъ? "--- Правда. Озлобляютъ люди, но онъ, яко начальникъ злобы, людей злыхъ, яко орудіе свое, употребляетъ. Тако Хріста, Сына Божія, люди гнали, хулили, мучили и распинали; но его козньми тое дѣлалось. Онъ людей къ тому злохитрымъ своимъ совѣтомъ приводилъ. Откуду Хрістосъ за распинателей Своихъ, яко ослѣпленныхъ, сожалѣя о погибели ихъ, молился: \textit{Отче, отпусти имъ: не вѣдятъ бо, что творятъ}\footnote{Лук.~23,~34.}. Такожде мучениковъ святыхъ нечестивые цари и ихъ посланники мучили; но онъ всему тому злу начальникъ. Онъ тщался рабовъ и воиновъ Хрістовыхъ отъ вѣры святыя отвести и погубити, но какъ самъ чрезъ себе не моглъ тое учинитъ, изобрѣлъ угодныхъ себѣ служителей, нечестивыхъ людей, чрезъ которыхъ озлоблялъ и проливалъ кровь ихъ, терзалъ и раздроблялъ тѣлеса ихъ: но самъ отъ нихъ вѣрою и силою распятаго Хріста попирался съ своими слугами беззаконными. Тоежъ и нынѣ дѣлается съ вѣрными; тойжде супостатъ и нынѣ научаетъ людей гнать и озлоблять рабовъ Божіихъ. Онъ прежде самъ различными искушеніями чрезъ помыслы и прелесть міра нападаетъ на хрістіанъ, и хощетъ ихъ отвести отъ Хріста; но, когда видитъ ихъ противныхъ себѣ, обрѣтаетъ къ тому злому своему дѣлу другій способъ, находитъ людей волѣ своей повинующихся, и чрезъ тѣхъ гонитъ и озлобляетъ Хрістовыхъ людей, да не стерпѣвше злостраданія, отпадутъ отъ вѣры и Хріста, и въ его паки темную попадутся область. Отсюду"=то бываетъ, что \textit{многи скорби праведнымъ}\footnote{Пс.~33,~20.} и \textit{вси, хотящіи благочестно жити о Хрістѣ Іисусѣ, гоними будутъ: лукавіи же человѣцы и чародѣи преуспѣютъ, на горшее прельщающе и прельщаеми}\footnote{2~Тим.~3,~12 и 13.}. И которыи въ благочестіи неподвижно стоятъ, злохитрыя врага чрезъ помыслы нападенія вѣрою отражаютъ, и гоненія, чрезъ злыхъ людей наносимыя, великодушно претерпѣваютъ: побѣждаютъ его, и, когда до конца вѣрны будутъ, пріемлютъ отъ Подвигоположника Хріста вѣнецъ правды, по словеси Его: \textit{буди вѣренъ даже до смерти, и дамъ ти вѣнецъ живота}\footnote{Апок.~2,~10.}.

\paragraph*{§\:522.} Трудный поистинѣ подвигъ сей, какъ выше сказано! Много бо труда требуется противу невидимаго врага подвизатися, побѣждать помыслы, которые онъ въ сердцѣ нашемъ противу насъ возбуждаетъ, претерпѣвать великодушно бѣды и напасти, отъ него чрезъ злыхъ людей наносимыя, гордость надымающуюся и злобу возникающую и въ дѣйство произойти хотящую, удерживать и смирять, "--- и тако противу своего сердца стояти, природную злость побѣждать, и самого себе отрещися (всего бо сего подвигъ сей требуетъ); но славная есть побѣда! Что бо славнѣе, какъ плоти немощной сильнаго духа побѣдить; духъ гордости, злобы, зависти духу смиренія, кротости и любви попирать? Сіе ему, яко духу гордому, нестерпимая язва, и болѣе мучитъ его, нежели огнь геенскій, яко отъ немощной плоти побѣждается. На сію побѣду Подвигоположникъ Хрістосъ съ небесе благопріятно смотритъ, и побѣдителю вѣнецъ славы готовитъ. О сей ангели святіи радуются, и поютъ Господа своего, на таковый подвигъ укрѣпившаго рабовъ Своихъ. О сей весь свѣтъ удивляется. Кто бо сему не дивится, разсуждая немощь плоти и силу духа? Сами нечестивыи мучители терпѣнію мучениковъ, которыхъ плоть какъ звѣри терзали, удивлялися. Откуду многіи и въ познаніе истины приходили; и Котораго хулили и гнали, Хріста, Того съ страдальцами прославляли, и вѣровали въ Того. Многіи отъ нихъ всего свѣта побѣдители были; но отъ единаго хрістіанина, силою Хрістовою вооруженнаго, побѣждены. Такожде многіи были побѣдители вселенныя; но отъ сего душевнаго врага побѣждены и плѣнены, и вѣчными его невольниками сдѣлалися. И какая изъ того польза? какая то слава? Людей подобныхъ себѣ побѣждать, а отъ страстей, плоти, грѣха и діавола побѣждену быть; людямъ повелѣвать, а самому покаряться мерзкому мучителю, бѣсу? Не тако истинный хрістіанинъ. Онъ людямъ гонящимъ уступаетъ, но тѣмъ самымъ противу врага діавола стоитъ; язвы отъ человѣка великодушно терпитъ, но сатану врага невидимо уязвляетъ; отъ видимыхъ враговъ изгоняется, порабощается, ругается, но невидимыхъ гонитъ и попираетъ силою Того, ради Котораго вся сія терпитъ и страждетъ. Какъ подвигъ паче сего, такъ и побѣда паче славнѣйшая быть не можетъ. Признаеши сію истину, возлюбленный хрістіанине, ежели вѣрою вообразишь, кто и съ кѣмъ подвизается, и о чемъ подвигъ сей происходитъ. Смиренный человѣкъ съ веліаромъ и княземъ тьмы, и немощная плоть съ гордымъ духомъ сражается! "--- О чемъ? О чести и славѣ Царя своего Хріста, и о вѣчномъ своемъ спасеніи. Единъ (діаволъ) хощетъ обезчестить Царя славы, и воина Его покорить и погубить: другій (хрістіанинъ) стоитъ за то, не щадитъ плоти и крове своея, подвергаетъ животъ свой, не даетъ мѣста врагу, и тако побѣждаетъ. Како побѣждаетъ? Не даетъ ему, чего онъ хощетъ. Единою Хрістосъ побѣдилъ его, и побѣжденнаго отдалъ въ попраніе рабомъ Своимъ, глаголя: \textit{се даю вамъ власть наступати на змію и на скорпію, и на всю силу вражію}\footnote{Лук.~10,~19.}. Онъ сію побѣду хощетъ исторгнуть отъ рабовъ Хрістовыхъ. Кто крѣпко стоитъ и держитъ тое, что дано отъ Хріста, по словеси Его: \textit{держи, еже имаши, да никтоже пріиметъ вѣнца твоего}\footnote{Апок.~3,~11.}: тотъ побѣждаетъ его. Прекрасно святый Златоустъ поучаетъ и поощряетъ хрістіанъ къ подвигу сему. «Не тако бываетъ здѣ, якоже въ подвижникахъ міра сего. Тамо бо, аще не низложиши, нѣси побѣдилъ: а здѣ, аще нѣси низложенъ, побѣдилъ еси; аще не будеши низложенъ, низложилъ еси. И воистину! Тамо бо оба въ побѣдѣ тщатся; и аще единъ будетъ низложенъ, другій вѣнчается: здѣ же не тако, но діаволъ о нашемъ побѣжденіи (да насъ побѣдитъ) тщится. Егда убо отъ него отниму сіе, о чемъ тщится, побѣдихъ». И мало спустя: «Той бо уже низложился, и въ погибели есть: побѣда же не въ томъ, чтобы вѣнчатися, но чтобы мене погубити; того ради, аще не низложу, но не буду низложенъ, побѣдихъ»\footnote{\textit{Бес.~22"~я въ нрав. на посл. къ Ефес}.}.

\paragraph*{§\:523.} На брани міра сего бываетъ, что кто сначала побѣжденъ былъ, тотъ при концѣ войны побѣдителемъ дѣлается: такъ и на духовной сей брани многіи отъ врага побѣждены сначала, но при концѣ, благодатію Божіею справившеся и силою Хрістовою укрѣпившеся, добрѣ подвизалися и побѣдили его. Не начало бо, но конецъ есть похваленъ и всякое дѣло совершаетъ, и всякое блаженство не отъ начала, но отъ конца описуется. *Не тотъ блаженъ, кто добрѣ начинаетъ*, но тотъ, кто добрѣ кончаетъ подвигъ свой, и вѣнцемъ побѣды тотъ украшается, кто конечную надъ врагомъ побѣду получитъ, хотя бы прежде побѣжденъ и низложенъ былъ. Чего ради побѣжденнымъ отъ супостата діавола не должно отчаяваться и во всеконечную радость врагу себе отдавать; но, призвавъ въ помощь всемогущаго силу Хріста Господа, смерти и ада Побѣдителя, возстать, и, оттрясши сонъ лѣности и унынія, съ помощію Его подвизатися. Можемъ вси, пока въ вѣцѣ семъ находимся, начать добрѣ подвигъ и окончить силою всесильнаго Хріста: \textit{вѣрующему бо вся возможна}\footnote{Марк.~9,~23.}. Силенъ есть и крѣпокъ и вѣренъ, Который всѣмъ вѣрующимъ въ Него глаголетъ: \textit{дерзайте, яко Азъ побѣдихъ міръ}\footnote{Іоан.~16,~33.}, "--- и: \textit{се Азъ съ вами есмь во вся дни до скончанія вѣка}\footnote{Матѳ.~28,~20.}. \textit{Іисусъ} бо \textit{Хрістосъ вчера и днесь, тойжде и во вѣки}\footnote{Евр.~13,~8.}, Который какъ прежде рабомъ Своимъ помогалъ, такъ и нынѣ помогаетъ труждающимся и призывающимъ Его.

\paragraph*{§\:524.} Отъ вышеписанныхъ видѣть можешь, хрістіанине: 1)~Что хрістіанамъ чрезъ все житіе до самой смерти брань и подвигъ духовный предлежитъ, яко до конца боретъ ихъ сатана; и непрестанный подвигъ, яко непрестанно на нихъ возстаетъ врагъ различно. "--- 2)~Которые хрістіане брани сея не чувствуютъ, въ бѣдственномъ состояніи находятся: яко врагъ того не безпокойствуетъ, кто волѣ его злой покаряется, якоже сіе видѣть можемъ и на брани, на которой человѣкъ съ человѣкомъ сходится, гдѣ непріятель покаряющагося себѣ не боретъ. Иначе врагъ душевный противящагося ему и о цѣлости спасенія своего пекущагося никогда не оставляетъ. Ибо то ему желаемая корысть, чтобы человѣка"=хрістіанина низложить и погубить. "--- 3)~Работающіи страстемъ, яко"=то: блудники, піяницы, хищники, лихоимцы, злобники, грабители и прочіи симъ подобныи, суть явныи врага діавола плѣнники, и подъ игомъ его тяжкимъ и темною властію бѣдственно находятся. Откуду, когда не хотятъ вѣчно съ нимъ гнѣва и суда Божія чашу горькую пить, должно имъ очувствоваться, ко Хрісту, плѣнныхъ Свободителю, воздыхать и молиться, и такъ силою Его отъ тяжкаго того и погибельнаго ига свободиться, да не вѣчными его плѣнниками сотворившеся, вѣчно съ нимъ погибнутъ. "--- 4)~Вси таковыи плѣнники вѣры хрістіанской не имѣютъ, хотя и именуются хрістіанами. Ибо вѣра подвизается противу всякаго грѣха, и не даетъ мѣста діаволу, противится ему, якоже учитъ апостолъ: \textit{емуже} (діаволу) \textit{противитеся тверди вѣрою}\footnote{1~Петр.~5,~9.}. А когда вѣры не имѣютъ, то напрасно и именемъ хрістіанскимъ именуются. Что бо пользуетъ имя безъ самой вещи? Весьма ничтоже! Паче же въ большее осужденіе ведетъ, яко именовалися хрістіанами, а въ самой вещи язычниками были, паче же горшими и язычниковъ, отъ которыхъ многіи отъ таковыхъ дѣлъ удалялись по просвѣщенію естественнаго разума, на каковыя дерзаютъ заблуждшіи хрістіане. "--- 5)~Паки отъ сего видѣть можешь, любезный хрістіанине, чего ради благочестивыи истинныи хрістіане во всякихъ бѣдствіяхъ въ мірѣ семъ находятся. Понеже сатана на нихъ возстаетъ, и чего самъ не можетъ, чрезъ злыхъ людей, своихъ служителей, въ дѣйство приводитъ, и озлобляетъ ихъ. Ибо они противятся ему; того ради и онъ не престаетъ на нихъ возставать, чтобы ихъ низложить и подъ свою власть привести. "--- 6)~Явно и тое, чего ради люди злыи по большей части въ благополучіи живутъ. Ибо любезная суть вѣка сего чада, и князю вѣка сего угождаютъ: того ради оставляетъ ихъ въ покоѣ.

\subsection[Глава 17-я. О кончинѣ или смерти хрістіанъ.]{глава седмаянадесять.\\\bfseries О кончинѣ или смерти хрістіанъ.}

\begin{quotation}\textit{Честна предъ Господемъ смерть преподобныхъ Его}\footnote{Пс.~115,~6.}.\end{quotation}
\begin{quotation}\textit{Буди вѣренъ даже до смерти, и дамъ ти вѣнецъ живота}\footnote{Апок.~2,~10.}.\end{quotation}

\paragraph*{§\:525.} Что до внѣшняго вида смерти касается, "--- праведнаго и грѣшнаго, благочестиваго и нечестиваго равна есть кончина, и ничимъ въ семъ не разнствуютъ благочестивіи отъ нечестивыхъ. Равно бо благочестивіи и нечестивіи болѣзнуютъ, стонутъ и лишаются живота сего; равно мученику Хрістову отсѣкается глава, руки, ноги, какъ и злодѣю; равно мученикъ сожигается, распинается, копіемъ прободается, въ водѣ потопляется и отходитъ отъ живота сего, какъ душегубецъ, злодѣй и разбойникъ; такожде равно вси и въ землѣ погребаются, якоже о семъ святое Писаніе свидѣтельствуетъ: \textit{единъ входъ всѣмъ есть въ житіе, подобенъ же и исходъ}\footnote{Премудр.~7,~6.}. Паче же и самыи скоты не иначе, какъ люди, издыхаютъ и престаютъ жить. Но когда до внутренняго состоянія приникнуть, и разсудить тое, что вѣра святая представляетъ намъ, то увидимъ различіе кончины благочестивыхъ и нечестивыхъ такъ, какъ различно было ихъ въ семъ мірѣ житіе.

\paragraph*{§\:526.} \textit{Различіе смерти благочестивыхъ и нечестивыхъ}. 1)~Хотя благочестивіи и нечестивіи въ святомъ Писаніи называются спящими, якоже глаголется: \textit{мнози отъ спящихъ въ земнѣй персти востанутъ: сіи въ жизнь вѣчную, а оніи во укоризну и въ стыдѣніе вѣчное}\footnote{Дан.~12,~2.}; потому что вси архангельскою трубою, какъ отъ сна, отъ смерти возбудятся и оживутъ: однакожъ благочестивыхъ смерть особливо называется \textit{сонъ}, яко они \textit{почиваютъ отъ трудовъ своихъ}\footnote{Апок.~14,~13.}. Какъ бо трудившіися днемъ добрѣ почиваютъ въ ночи: тако благочестивіи, которые трудились въ подвигѣ вѣры и благочестія въ мірѣ семъ, сладцѣ по благословенныхъ трудахъ усыпаютъ, почиваютъ и упокоеваются, донелѣже архангельскою трубою возбудятся. "--- 2)~Благочестивіи лишаются временныхъ и земныхъ благихъ, но наслѣдствуютъ вѣчная и небесная: нечестивіи, лишившеся временныхъ, лишаются и вѣчныхъ. "--- 3)~Благочестивіи престаютъ временно жить, но начинаютъ вѣчно: нечестивіи, престая временно жить, престаютъ и вѣчно, но временную жизнь премѣняютъ на вѣчную смерть\footnote{21,~8.}. "--- 4)~Благочестивіи по трудахъ, скорбехъ и болѣзнехъ преселяются въ покой, прохлажденіе и радость, и \textit{переносятся ангелами на лоно Авраамле}\footnote{Лук.~16,~22.}: нечестивіи по веселостяхъ и роскошахъ своихъ обрѣтаются во адѣ въ мукахъ\footnote{ст.~23.}. "--- 5)~Благочестивіи, утѣшаяся на лонѣ Авраамли, ожидаютъ общаго воскресенія, въ которое, услышавше гласъ Сына Божія, оживутъ плотію, и, какъ трава прозябшая прекрасными цвѣтами одѣваются во время весеннее, исходя изъ гробовъ, облекутся въ новый, нетлѣнный и прекрасный видъ, и тако получатъ совершеннѣйшее отъ десницы Господни блаженство: нечестивіи, мучась во адѣ, ожидаютъ совершеннѣйшаго по дѣломъ своимъ воздаянія и мученія, по соединеніи съ тѣлами, съ которыми беззаконновали. И тако благочестивыхъ смерть есть \textit{пріобрѣтеніе}\footnote{Филип.~1,~21.}, яко чрезъ сію смерть пріобрѣтаютъ животъ вѣчный: нечестивымъ есть пагуба, яко по временной смерти вѣчною умираютъ.

\subsection[Глава 18-я. О воскресеніи хрістіанъ.]{глава осмаянадесять.\\\bfseries О воскресеніи хрістіанъ.}

\begin{quotation}\textit{Наше житіе на небесѣхъ есть, отонудуже и Спасителя ждемъ Господа нашего Іисуса Хріста, Иже преобразитъ тѣло смиренія нашего, яко быти сему сообразну тѣлу славы Его}\footnote{3,~20 и 21.}.\end{quotation}
\begin{quotation}\textit{Сѣется въ тлѣніе, востаетъ въ нетлѣніи; сѣется не въ честь, востаетъ въ славѣ}, и проч.\footnote{1~Кор.~15,~42 и слѣд.}\end{quotation}

\paragraph*{§\:527.} Разумъ человѣческій, вѣрою непросвѣщенный, не можетъ понять воскресенія мертвыхъ. Онъ мнитъ, что какъ скотскія тѣлеса, согнившія и разсыпавшіяся въ прахъ, не востаютъ, такъ и человѣческія. Отсюду многія душевредныя мнѣнія произошли о воскресеніи мертвыхъ, яко люди разуму своему, который безъ просвѣщенія вѣры слѣпъ, послѣдовали. Отсюду невѣрнымъ и плотскимъ людямъ буйство показуется слово о воскресеніи мертвыхъ. Отсюду Саддукеи глаголютъ \textit{воскресенію не быти}, и вопрошаютъ о семъ Хріста; но обличаются отъ Него, \textit{яко невѣдущіи писанія, ни силы Божія}\footnote{Марк.~12,~18--24 и слѣд.}. Отсюду Павла святаго апостола суесловнымъ называли Аѳиняне, яко воскресеніе благовѣствоваше имъ. \textit{Что убо хощетъ суесловный сей глаголати? И слышавше воскресеніе мертвыхъ, ругахуся}\footnote{Дѣян.~17,~18 и 32.}. Но недостатокъ и оскудѣніе ума вѣра святая дополняетъ, и чего самъ въ себѣ слѣпый разумъ не постигаетъ, въ томъ его просвѣщаетъ, и тако просвѣщенный вѣрою умъ видитъ тое такъ, какъ бы оно уже въ самомъ дѣлѣ было. Показуютъ тое и подобія, отъ естества взятыя, якоже Хрістосъ глаголетъ: \textit{аще зерно пшенично падъ на земли не умретъ, то едино пребываетъ: аще ли умретъ, многъ плодъ сотворитъ}\footnote{Іоан.~12,~24.}. И Павелъ святый: \textit{безумне! ты еже сѣеши, не оживетъ, аще не умретъ}\footnote{1~Кор.~15,~36.}. Но вѣра святая, словомъ Хріста Сына Божія, Иже есть вѣчная истина, утверждаема, всякое сумнѣніе отъемлетъ. Воскреснутъ убо мертвіи, и воскреснутъ тѣми тѣлесами, которыми почиваютъ во гробѣхъ; востанутъ убо тыя тѣлеса, которыя пали; пробудятся тіи, которые \textit{спятъ во гробѣхъ}\footnote{Дан.~12,~2.}; востанутъ тіи, которые посѣяны\footnote{1~Кор.~15,~37.}; востанутъ тыяжде тѣлеса, съ которыми или добро, или зло дѣлали, \textit{да пріиметъ кійждо, яже съ тѣломъ содѣла, или блага или зла}\footnote{2~Кор.~5,~10.}. \textit{Яко грядетъ часъ, воньже вси сущіи во гробѣхъ услышатъ гласъ Сына Божія, и изыдутъ сотворшіи благая въ воскрешеніе живота, а сотворшіи злая въ воскрешеніе суда}\footnote{Іоан.~5,~28 и 29.}. Восталъ Хрістосъ отъ мертвыхъ, и восталъ тѣмъ тѣломъ, которымъ пострадалъ и умеръ, и \textit{начатокъ умершимъ бысть}\footnote{1~Кор.~15,~20--23; Кол.~1,~18.}: востанутъ и мертвіи тѣми, которыми умерли, тѣлесами.

\paragraph*{§\:528.} Какъ нечестивымъ воскресеніе оное будетъ печальное и горькое, яко совокупившеся съ тѣлесами, съ которыми беззаконновали и свирѣпѣли въ мірѣ семъ, низринутся во адъ пріяти по дѣломъ своимъ, по писанному: \textit{да возвратятся грѣшницы во адъ, вси языцы забывающіи Бога}\footnote{Пс.~9,~18.}: тако благочестивымъ Хрістовымъ рабамъ утѣшительное, радостное и желаемое такъ, какъ по зимѣ прекрасная весна. Яко бо во время весны исходитъ трава изъ нѣдръ земныхъ и одѣвается благовонными различными цвѣтами: тако тѣлеса благочестивыхъ изъ гробовъ, въ которыхъ, мразомъ смерти сокрушены, крылись, въ оное время изыдутъ одушевленны, и въ новый прекрасный безсмертія и вѣчныя славы видъ облекутся. \textit{Сѣется бо}, учитъ апостолъ Хрістовъ, \textit{въ тлѣніе, востаетъ въ нетлѣніи; сѣется не въ честь, востаетъ въ славѣ; сѣется въ немощи, востаетъ въ силѣ}\footnote{1~Кор.~15,~42 и 43.}; яко Хрістосъ \textit{преобразитъ тѣло смиренія нашего, яко быти сему сообразну тѣлу славы Его}, какъ тойжде проповѣдуетъ апостолъ\footnote{Филипп.~3,~21.}. И апостолъ Іоаннъ: \textit{возлюбленніи, нынѣ чада Божіи есмы, и не у явися, что будемъ? Вѣмы же, яко егда явится, подобни Ему будемъ}\footnote{1~Іоан.~3,~2.}, такъ что \textit{тогда праведницы просвѣтятся, яко солнце, во царствіи Отца ихъ}, по словеси Хрістову\footnote{Матѳ.~13,~43.}. Прекрасная весна, въ которой таковы цвѣты явятся на земли! Блаженны и благословенны суть, которыхъ смиренная тѣлеса тако прозябнутъ! Какая радость и сладость въ сердцахъ ихъ будетъ, когда тѣлеса ихъ тако украсятся! Тогда исполнится слово написанное: \textit{пожерта бысть смерть побѣдою: гдѣ ти, смерте, жало? гдѣ ти, аде, побѣда}\footnote{1~Кор.~15,~54 и 55; Ос.~13,~14.}? Разсуждай сію прекрасную весну, возлюбленный хрістіанине, и тщись благодатію Хрістовою достигнуть въ воскресеніе праведныхъ.

\subsection[Глава 19-я. О второмъ Хрістовомъ пришествіи.]{глава девятаянадесять.\\\bfseries О второмъ Хрістовомъ пришествіи.}

\paragraph*{§\:529.} По утѣшеніи утѣшеніе, и по радости радости неизреченная слѣдуетъ избраннымъ Божіимъ. Облекшеся въ ризу безсмертія и славы, предстанутъ въ радости сердца Царю славы, Которому здѣ въ мірѣ семъ вѣрою и правдою служили, и вѣрою за честь имене Его подвизалися; услышатъ отъ Него вожделѣнный гласъ: \textit{пріидите, благословенніи Отца Моего, наслѣдуйте уготованное вамъ царствіе отъ сложенія міра}\footnote{Матѳ.~25,~34.}; услышатъ отъ праведнаго Судіи и Царя своего похвалу за вѣрную ихъ службу, которую Ему показывали, подвизаяся противу грѣха; увидятъ свои добродѣтели, какъ сокровища духовная, въ явленіе всему міру произнесенныя, которыя втайнѣ предъ Отцемъ небеснымъ во славу Его творили: \textit{увидятъ домъ Отца} своего \textit{небеснаго}, отверстый имъ, и \textit{обители многи}\footnote{Іоан.~14,~2.}; и \textit{пріидутъ въ Сіонъ съ радостію, и радость вѣчная надъ главою ихъ: надъ главою бо ихъ хвала и веселіе, и радость пріиметъ я, отбѣже болѣзнь и печаль и воздыханіе}\footnote{Ис.~35,~10.}. И тако вѣрніи раби Хрістовы \textit{внидутъ въ радость Господа своего}\footnote{Матѳ.~25,~21 и 23.}; начнутъ блаженную вѣчность съ торжествомъ и восклицаніемъ духовныя радости.

\paragraph*{§\:530.} Истинно то, что и благочестивыхъ страхъ суда Божія смущаетъ, яко никтоже есть безъ грѣха. Откуду со смиреніемъ молятся къ Богу: \textit{не вниди въ судъ съ рабомъ Твоимъ, яко не оправдится предъ Тобою всякъ живый}\footnote{Пс.~142,~2.}. Однакожъ вѣра святая, которою работаютъ Царю своему Іисусу Хрісту и на Его богатую благодать взираютъ, умягчаетъ страхъ оный, и подаетъ утѣшительное извѣстіе, яко \textit{всякъ вѣруяй въ Онь, не постыдится}\footnote{Римл.~9,~33.}. Откуду святый Златоустъ, разсуждая о дни ономъ, глаголетъ: «День оный правоживущимъ желаемый есть, якоже беззаконнымъ страшный»\footnote{\textit{Бес.~90"~я на Матѳ}.}. Чего ради, чтобы намъ, любезный хрістіанине, не страшенъ былъ, но желаемый день оный, должно о немъ почаще разсуждать, и усерднымъ покаяніемъ праведнаго Судію умилостивлять, а впредь отъ грѣховъ, которые онаго страхъ въ совѣсти содѣловаютъ, берещися.

\subsection[Глава 20-я. О торжествѣ и славѣ вѣчной избранныхъ Божіихъ.]{глава двадесятая.\\\bfseries О торжествѣ и славѣ вѣчной избранныхъ Божіихъ.}

\begin{quotation}\textit{И слышахъ яко гласъ народа многа, и яко гласъ водъ многихъ, и яко гласъ громовъ крѣпкихъ, глаголющихъ: аллилуіа! яко воцарися Господь Богъ Вседержитель. Радуимся и веселимся, и дадимъ славу Ему: яко пріиде бракъ Агнчій, и жена Его уготовила есть себе. И дано бысть ей облещися въ вѵссонъ чистъ и свѣтелъ: вѵссонъ бо оправданія святыхъ есть}\footnote{Апок.~19,~6--8.}.\end{quotation}

\paragraph*{§\:531.} По окончаніи праведнаго суда Хрістова, пойдутъ избранніи Божіи съ торжествомъ и восклицаніемъ въ животъ вѣчный: \textit{внидутъ въ радость Господа своего вѣрніи раби}\footnote{Матѳ.~25,~21 и 23.}; воспріимутъ \textit{наслѣдіе нетлѣнно и нескверно и неувядаемо, соблюдено на небесѣхъ ихъ ради}\footnote{1~Петр.~1,~4.}, \textit{царствіе пріимутъ, и вѣнецъ отъ руки Господни}\footnote{Премудр.~5,~16.}; \textit{не взалчутъ ктому, ниже вжаждутъ, не имать же пасти на нихъ солнце и всякъ зной: яко Агнецъ, Иже посредѣ престола, упасетъ я, и наставитъ ихъ на животныя источники водъ, и отъиметъ Богъ всяку слезу отъ очію ихъ}\footnote{Апок.~7,~16 и 17.}; вселятся во святомъ горнемъ Іерусалимѣ, \textit{имущемъ славу Божію}\footnote{21,~11.}, \textit{гдѣ нощи не будетъ, и не потребуютъ свѣта отъ свѣтильника, ни свѣта солнечнаго, яко Господь Богъ просвѣщаетъ я: и воцарятся во вѣки вѣковъ}\footnote{22,~5.}. Тогда работающіи Господеви ясти будутъ, пити будутъ, возрадуются и возвеселятся въ веселіи сердца\footnote{Ис.~65,~13 и 14.}. \textit{Якоже аще кого мати утѣшаетъ, тако и Азъ утѣшу вы, и во Іерусалимѣ утѣшитеся: и узрите, и возрадуется сердце ваше}, глаголетъ Господь\footnote{66,~13 и 14.}. \textit{Радостію возрадуются они о Господѣ}, глаголя: \textit{да возрадуется душа моя о Господѣ: облече бо мя въ ризу спасенія, и одеждою веселія одѣя мя; яко на жениха возложи на мя вѣнецъ, и яко невѣсту украси мя красотою}\footnote{61,~10.}. "--- \textit{Возрадуется и Господь о нихъ: и будетъ, якоже радуется женихъ о невѣстѣ, тако возрадуется Господь о тебѣ}\footnote{62,~5.}. Будутъ видѣть Бога Отца своего \textit{лицемъ къ лицу}\footnote{1~Кор.~13,~12.} и узрятъ Его \textit{якоже есть}\footnote{1~Іоан.~3,~2.}: будетъ и Богъ на нихъ смотрѣть, яко Отецъ чадолюбивый на возлюбленныхъ сыновъ Своихъ. Будутъ видѣть Хріста, Сына Божія, искупившаго ихъ кровію Своею; будутъ видѣть въ божественной славѣ Его, Которому здѣ смиреніемъ, любовію и терпѣніемъ послѣдовали, якоже Самъ о семъ молился: \textit{Отче, ихже далъ еси Мнѣ, хощу, да, идѣже есмь Азъ, и тіи будутъ со Мною: да видятъ славу Мою, юже далъ еси Мнѣ}\footnote{Іоан.~17,~24.}. Исполнятся всякихъ даровъ, утѣшенія, радости и сладости Духа Святаго, яко живаго и животворящаго источника. Будутъ имѣть любовную дружбу со святыми ангелами, и съ ними купно во вѣки вѣковъ восхвалять Господа. Тогда будутъ славно и весело въ радости духа торжествовать надъ смертію, адомъ и діаволомъ: \textit{гдѣ ти, смерте, жало? гдѣ ти, аде, побѣда}\footnote{1~Кор.~15,~55.}? Будутъ пѣть пресладкую пѣснь: \textit{аллилуіа! Радуимся и веселимся, и дадимъ славу Ему!} "--- Тому слава, честь и благодареніе буди и отъ насъ недостойныхъ, нынѣ и присно, и во вѣки вѣковъ, аминь!

\begin{center}\textbf{* * *}\end{center}

О Боже Отче, Сыне Божій Единородный, и Святый Душе Тріѵпостасный Боже, Творче, Спасителю и Промыслителю мой, немерцающій Свѣте, неложная Истино, Любы вѣчная, Благосте безконечная, непостижимое Всемогущество, непостижимая Мудросте, всегда сущая Вѣчносте, Начало безначальное, Сило вся дѣйствующая, Животе всѣхъ живущихъ! живи мя, просвѣти мя и вразуми мя, да узрю любовь и благость Твою, которую Ты явилъ роду человѣческому. Аминь!

\begin{center}\small\textsc{Конецъ третьяго тома.}\end{center}

\newpage
\section{Примѣчанія къ II – III тому.}

Сочиненіе «Объ истинномъ Христіанствѣ», помѣщенное въ настоящемъ изданіи во II и III томахъ, написано святителемъ Тихономъ въ первые годы пребыванія его въ Задонскомъ монастырѣ, именно въ 1770--1771 гг. "--- Все сочиненіе писано было святителемъ собственноручно и раздѣлено было \textit{переплетомъ} на 6"=ть томовъ. Къ сожалѣнію, до настоящаго времени изъ 6"~ти томовъ въ подлинникахъ сохранились (по крайней мѣрѣ въ нижеозначенныхъ библіотекахъ) только два: одинъ хранится въ библіотекѣ \textit{Софійскаго, близъ Усмани, женскаго монастыря}, доставшійся обители, вмѣстѣ съ другими подлинными сочиненіями святителя, отъ его келейника Іоанна; другой "--- въ библіотекѣ \textit{Саровской пустыни}. Гдѣ остальные томы, осталось пока, не смотря на всѣ, произведенныя издателями, розысканія, неизвѣстнымъ. Сохранившіеся подлинники заключаютъ въ себѣ, одинъ (\textit{Софійскій}, въ 4"=ку, на 325 л.) \textit{Статью пятую}, обнимающую, по настоящему изданію, стран.~172--337; другой (\textit{Саровскій}, въ 4"=ку, на 63 л.) \textit{Статью шестую и седьмую}, обнимающую, по настоящему изданію, стран.~337--370. Посему съ подлинниками (по тип. сч. № 3 и 4)~свѣрены только означенныя \textit{Статьи}; остальныя же сличены съ 1"~мъ изданіемъ.

Замѣчательный фактъ изъ исторіи написанія сочиненія «Объ истинномъ Христіанствѣ», передаетъ келейникъ святителя. "--- Фактъ характерный, показывающій, подъ какимъ настроеніемъ писалъ свое сочиненіе авторъ, и до какой степени онъ проникнутъ былъ своимъ предметомъ. «Однажды, говоритъ очевидецъ, когда святитель былъ занятъ сочиненіемъ книги \textit{Объ истинномъ Христіанствѣ}, размышлялъ онъ о страданіи Хріста Сына Божія, сидя на кровати, противъ коей на стѣнѣ прибита была картина, представлявшая распятаго Хріста, всего ураненнаго, уязвленнаго, умученнаго, окровавленнаго. Отъ великой радости и сердечнаго соболѣзнованія случилось ему чудное видѣніе: вдругъ всталъ съ кровати и бросившись къ Спасителевымъ ногамъ, чтобы облобызать ихъ, выговорилъ таковыя слова: и Ты ли, Спасителю мой, ко мнѣ грядеши? "--- чувствуя себя аки у ногъ Спасителевыхъ. Отъ того часа онъ еще болѣе сталъ углубляться въ размышленіе о страданіяхъ Его и объ искупленіи рода человѣческаго» \textit{(Записки о жизни св. Тихона его келейника Ив. Еѳим.)}.

Въ печати сочиненіе «Объ истинномъ Хрістіанствѣ» явилось въ первый разъ въ 1785 г. \textit{(Спб. у Шнора)}. Въ изданіи этомъ, какъ и во всѣхъ позднѣйшихъ 4"~хъ \textit{(1803 г. Спб., 1826 г. Спб., 1836 г. М. и 1860 г. М.)} сочиненіе это озаглавливалось такъ: \textit{«Объ истинномъ Хрістіанствѣ, сочиненіе, содержащее въ себѣ ученіе о истинной вѣрѣ, о святомъ житіи, о спасительномъ покаяніи, о сердечномъ умиленіи, о болѣзнованіи грѣховъ и о пребываніи истинныхъ и неложныхъ Хрістіанъ, такожде како истинный хрістіанинъ можетъ побѣдити грѣхъ, смерть, діавола, міръ и всякое бѣдствіе»..}. Но еще въ 1821 году составитель \textit{Опыта Россійской библіографіи} замѣтилъ, что «сочинитель на заглавіи сей своей книги поставилъ только два начальныя рѣченія: о истинномъ Хрістіанствѣ; книгопродавецъ же \textit{(московскій купецъ Тимоѳей Полежаевъ)}, иждивеніемъ коего она была напечатана, почитая сіе заглавіе \textit{для торговли} недостаточнымъ, дополнилъ оное, изъ слова въ слово съ книги, подъ симъ же заглавіемъ писанной Іоанномъ Арндтомъ, хотя сіи два сочиненія, кромѣ заглавій, не имѣютъ между собою ни малѣйшаго сходства» \textit{(Опытъ Россійской библіографіи № 11,~792--3)}. Соч. Іоанна Арндта «Объ Истинномъ Христіанствѣ» слич. по изданію \textit{1735 г. въ Галлѣ} (\textit{рѣдкій}, по замѣчанію того же Сопикова, экз. \textit{этого именно} изданія см. \textit{въ собраніи старопеч. книгъ при Румянц. Муз.~157}): потому что въ позднѣйшихъ изданіяхъ соч. Арндта заглавіе измѣнено; оно осталось, и доселѣ, во всѣхъ 6"~ти томахъ, было удерживаемо только въ изданіи сочиненій святителя Тихона, несмотря на то, что замѣтка Сопикова объ измышленномъ дополненіи къ заглавію почти буквально повторялась въ жизнеописаніи святителя, которое прилагалось тутъ же, при самыхъ его сочиненіяхъ. "--- Такъ какъ замѣчаніе Сопикова объ этомъ произвольномъ, "--- и неудачномъ притомъ, далеко не исчерпывающемъ содержанія книги, "--- дополненіи вполнѣ подтвердилось, и подлинникомъ, въ виду этого въ настоящемъ изданіи дополненіе это изъ заглавія устранено.

Сочиненіе «Объ истинномъ Хрістіанствѣ» въ бывшихъ доселѣ изданіяхъ (начиная съ 3"=го) дѣлилось на \textit{6"=ть частей}. Въ настоящемъ изданіи оно раздѣлено \textit{на двѣ книги}, съ подраздѣленіемъ \textit{первой} книги на \textit{двѣ части} и послѣднихъ: 1"~й на 4 \textit{Статьи} и 2"~й на 3; \textit{второй} книги на 8 \textit{Статей}. Такъ раздѣлилъ, соотвѣтственно содержанію, свое сочиненіе самъ авторъ (см. §§388,~392,~397,~398,~404,~419,~420,~426,~429 и 514.)

Что касается до самаго текста, то въ немъ при сличеніи съ подлинниками и частію съ 1"~мъ изданіемъ, оказалось кромѣ другихъ общихъ изданіямъ сочиненій святителя, уклоненій отъ подлинника, особенно много \textit{пропусковъ}, и преимущественно въ тѣхъ Статьяхъ, которыя сличены были съ подлинниками. Въ настоящихъ двухъ томахъ, какъ и въ другихъ, выдѣлены \textit{(звѣздочками)} только тѣ мѣста, которыя внесены въ изданіе не изъ подлинника, а изъ 1"~го изданія.

Сочиненіе «Объ Истинномъ Хрістіанствѣ», всегда пользовалось особеннымъ уваженіемъ у сыновъ Православной церкви. «Тихоново Истинное Хрістіанство», отзывался объ этой книгѣ московскій митрополитъ Платонъ, «есть сокровище духовное: оно должно быть предъ глазами у каждаго». \textit{(Москвит.~1843 г. № 4~стр.~470, и Преосвящ. Тихонъ 1"~й, епископъ Воронеж. М.~1844 года, стр.~32)}. Другой, знаменитый въ свое время, книговѣдъ \textit{(покойный митрополитъ Кіевскій Евгеній въ жизнеописаніи святителя Тихона)} такъ отозвался объ этомъ сочиненіи: «сочиненіе это сравнить можно со всѣми наилучшими въ Хрістіанствѣ нравственными книгами, а на россійскомъ языкѣ оно есть первое и образцовое. Оно есть не только достаточная библіотека для всѣхъ, желающихъ заниматься здравымъ хрістіанскимъ нравоученіемъ, но и для тѣхъ, кои хотятъ быть подлинными христіанскими, а не метафизическими проповѣдниками Евангельскаго ученія». Извѣстно также, что Михаилъ, митрополитъ Новгородскій, бывши московскимъ священникомъ, читалъ это сочиненіе въ своей церкви (Іоанна Воина, что близъ Калужскихъ воротъ) въ продолженіе великаго поста, во время утрени и часовъ, и столько привлекалъ слушателей, что «въ эти дни, по словамъ очевидца, церковь наполнялась народомъ, приходившимъ даже изъ другихъ, отдаленныхъ, приходовъ, и не смотря на то, что служба съ поученіями была довольна продолжительна, никто не выходилъ изъ церкви, никто не рѣшался нарушить тишину въ храмѣ». \textit{(Очеркъ жизни Михаила, митрополита Петербургскаго и Новгородскаго. М.~1857 г. стр.~16; Москвит.~1843 г. №4, стр.~470)}. «Прекрасное сочиненіе \textit{Объ истинномъ хрістіанствѣ}, замѣчаетъ авторъ \textit{Исторіи русской церкви (Филар.~1848 г., пер.~5, стран.~68)}, хотя не имѣетъ систематической строгости, но зато проникнуто истиннымъ хрістіанскимъ духомъ и оригинально отъ начала до конца».

\begin{center}\textbf{Къ стр.~158~т. II, гл.~4:}

\textit{О суетномъ и прелестномъ украшеніи}.

\end{center}

Въ духовномъ журналѣ «Странникъ» напечатана была, съ именемъ святителя Тихона, статья подъ заглавіемъ: \textit{глава о украшеніи суетномъ}, и сопровождена слѣдующимъ замѣчаніемъ отъ редакціи: «помѣщаемая здѣсь статья (вмѣстѣ съ другими, тутъ же помѣщенными), составляетъ рукописную тетрадь, которая была написана святителемъ Тихономъ для одной духовной его дочери и сообщена намъ преосвященнымъ Агаѳангеломъ, епископомъ вятскимъ и слободскимъ. Въ главѣ «О суетномъ украшеніи» находимъ поправки, сдѣланныя собственною рукою сочинителя, и имъ же \textit{собственноручно приписаны полторы страницы}. Эта «глава» въ \textit{пространнѣйшемъ} видѣ вошла въ составъ большаго сочиненія святителя «Объ истинномъ Хрістіанствѣ» \textit{(авг.~1891 г.)}. "--- Послѣднее указаніе относится къ 4"~й главѣ, напечатанной въ настоящемъ изданіи на означенныхъ выше страницахъ. Что касается до краткой редакціи, то въ дополненіе къ вышесказанному, не излишне замѣтить, что въ этомъ видѣ глава о украшеніи суетномъ составляетъ отрывокъ изъ уроковъ, преподанныхъ святителемъ Тихономъ, въ бытность его ректоромъ и учителемъ богословія въ тверской семинаріи, ученикамъ этой семинаріи. Уроки эти, какъ извѣстно, послужили впослѣдствіи основою для обширнаго сочиненія святителя: \textit{Объ истинномъ Хрістіанствѣ}: нѣсколько \textit{главъ} изъ семинарскихъ уроковъ внесены въ это сочиненіе почти безъ измѣненій; но нѣкоторыя, въ виду назначенія сочиненія не для школы, а для народнаго чтенія, подвергнуты были авторомъ тщательному пересмотру и частію измѣнены, частію дополнены (слич. по отрывку изъ семин. уроковъ, напечатанному въ концѣ 1"~го т. настоящ. изд., ст. \textit{о грѣхѣ}). Съ такими измѣненіями и дополненіями вошла въ составъ соч. \textit{Объ истинномъ Хрістіанствѣ}, между другими, и глава \textit{о украшеніи суетномъ}. "--- Разсылая въ спискахъ свои сочиненія, святитель иногда дѣлалъ на нихъ, по особеннымъ нуждамъ лица, къ которому относился, особыя \textit{приписки}. Такова, по своему происхожденію, и вышеозначенная \textit{приписка}, сдѣланная святителемъ для \textit{своей духовной дочери}. "--- Такъ какъ образчикъ краткой редакціи сочиненія «Объ истинномъ Хрістіанствѣ» уже представленъ въ вышепомянутомъ отрывкѣ изъ семинарскихъ уроковъ, то въ настоящее изданіе признано достаточнымъ занести только приписанныя въ концѣ главы строки: ихъ нѣтъ въ \textit{пространной} главѣ. "--- Послѣ словъ, которыми въ сочиненіи «Объ Истинномъ Хрістіанствѣ» оканчивается эта глава, именно: \textit{и имъ безъ грѣха тое украшеніе не бываетъ} (по наст. изд. стр.~166, кол. 2), святитель собственноручно приписываетъ: «Стыдъ и срамъ женамъ украшати себе суетно, пудрить волосы, мазать лица своя, облагать шею каменьями, надѣвать разноцвѣтныя, долгія и широкія платья, выставлять груди почти нагія, руки почти до локтей обнажать. Такія жены безчестятъ хрістіанскую вѣру и показываютъ на себѣ знакъ худой. Совѣтую всякой женѣ сію бездѣлицу оставить, и не слушать, что кто въ семъ ни приказуетъ и не смотрѣть, что нынѣшняго вѣка люди дѣлаютъ. Весьма мало спасающихся есть. Читай Евангеліе, Апостолъ и прочія книги, и увидишь, что, ей, правду пишу. Хрістіанскій путь тѣсный и скорбное житіе. Нынѣ люди пространство любятъ: въ пищи, въ питіи, въ посудѣхъ, платьяхъ, коняхъ, каретахъ, въ обращеніи всякомъ, въ ябедахъ, клеветахъ, въ исканіи чести, славы, богатства и проч. О тщета, бѣда, страхъ, вѣчная погибель!»

«Знай, хрістіанине, что нѣтъ никакой, и малѣйшей, пользы, гдѣ душѣ пагуба. Что во всемъ мірѣ, когда душа погибнетъ? Пропадай вся слава, богатство и роскошь, только бы душа спаслась. \textit{Отврати очи мои, Господи, еже не видѣти суеты} (Пс.~118)».

Спасайтеся о Хрістѣ.

Вашъ слуга N N.
