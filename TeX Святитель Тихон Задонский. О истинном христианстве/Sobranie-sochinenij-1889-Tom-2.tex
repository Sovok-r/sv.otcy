
\chapter*{часть первая.\\О ГРѢХАХЪ.}
\addcontentsline{toc}{chapter}{Часть 1-я. О грѣхахъ.}
\section[Статья 1-я. О нужныхъ къ вѣдѣнію.]{статья первая.\\\bfseries О нужныхъ къ вѣдѣнію.}
\subsection[Глава 1-я. О Словѣ Божіи.]{глава первая.\\\bfseries О Словѣ Божіи.}

\begin{quotation}\textit{Испытайте писаній, яко вы мните въ нихъ имѣти животъ вѣчный}, глаголетъ Хрістосъ\footnote{Іоан.~5,~39.}.\end{quotation}

\paragraph*{§\:1.} Слово Божіе содержится въ книгахъ пророческихъ и апостольскихъ, которыя по греческому нарѣчію называются: \textit{Библія}, то"=есть, книги. Сіи книги иначе называются \textit{писанія}, якоже глаголетъ Хрістосъ: \textit{испытайте писаній}, и проч. И Павелъ апостолъ къ Тимоѳею написалъ: \textit{изъ млада священная писанія умѣеши}, и проч.\footnote{2~Тим.~3,~15.} Называются же и \textit{книги закона Божія}, яко въ нихъ законъ Божій содержится, и проч.

\paragraph*{§\:2.} Слово Божіе ради того называется и есть \textit{Божіе}, что отъ Бога чрезъ пророковъ и апостоловъ, аки посланниковъ, намъ объявлено и предано, якоже Петръ святый апостолъ написалъ: \textit{не волею человѣческою бысть когда пророчество, но отъ Святаго Духа просвѣщаеми глаголаша святіи Божіи человѣцы}\footnote{2~Петр.~1,~21.}. И апостолъ Павелъ: \textit{пріемше}, рече, \textit{слово слышанія Божія отъ насъ, пріясте не аки слово человѣческо, но, якоже есть воистинну, Слово Божіе}\footnote{1~Сол.~2,~13.}. И къ Тимоѳею глаголетъ: \textit{всяко писаніе богодухновенно}\footnote{2~Тим.~3,~16.}. Писаніе бо святое есть какъ посланіе небеснаго Царя къ подданнымъ Своимъ рабамъ, намъ недостойнымъ, которое чрезъ вѣрныхъ Своихъ рабовъ и Духомъ Своимъ Святымъ умудренныхъ благоволилъ послать, и волю Свою святую и милостивое Свое благоволеніе открыть. Откуду дивную нѣкую и божественную силу и дѣйствіе въ себѣ имѣетъ, такъ, что въ маломъ времени, какъ видимъ въ проповѣди апостольской, не чрезъ многихъ, но чрезъ дванадесять человѣковъ, не мудрыхъ, но простыхъ и безкнижныхъ, во всю вселенную пронеслося, якоже пишется: \textit{во всю землю изыде вѣщаніе ихъ, и въ концы вселенныя глаголы ихъ}\footnote{Пс.~18,~5; Римл.~10,~18.}. И не токмо пронеслося, но и толикій плодъ принесло, что языки, въ прелести идолопоклонничестѣй увязшіи и застарѣвшіися, бросивше суевѣріе идолобѣсія, ко Хрісту распятому, такъ безчестною смертію умершему, пристали, и Того за Спасителя и Бога своего отъ сердца признали. И хотя діавольская хитрость и кознь дѣйствовала сіе, что мудрецы и сильніи вѣка сего проповѣди ихъ сильно противилися, но ничего не успѣли; и какъ солнечнымъ лучамъ вѣтръ, тако апостоламъ, яко лучамъ отъ Хріста, Солнца правды, въ поднебесную посланнымъ, бурный діавольскій вѣтръ ничего не моглъ воспрепятствовать. И что дивнѣе, которые сихъ проповѣдниковъ за буеслововъ и неистовыхъ вмѣняли, многіе у нихъ учениками быть возжелали; и которому ученію, какъ буйству, смѣялися, тое за истиннѣйшую премудрость признали. Откуду слово Божіе \textit{плодотворительно} называется\footnote{Кол.~1,~6.}, и есть, которое, въ сердцахъ человѣческихъ посѣянное, \textit{плодъ приноситъ, и творитъ ово сто, ово шестьдесятъ, ово тридесять}\footnote{Матѳ.~13,~23.}. Вѣрно и истинно есть слово Божіе, чрезъ пророка сказанное: \textit{якоже аще снидетъ дождь или снѣгъ съ небесе, и не возвратится, дондеже напоитъ землю, и родитъ, и прозябнетъ, и дастъ сѣмя сѣющему, и хлѣбъ въ снѣдь: тако будетъ глаголъ Мой, иже аще изыдетъ изъ устъ Моихъ, не возвратится ко Мнѣ тощь, дондеже совершитъ вся, елика восхотѣхъ}\footnote{Ис.~55,~10 и 11.}.

Отсюду послѣдуетъ, хрістіанине: 1)~Слово Божіе, яко великій и небесный даръ, должно намъ почитать и любить, якоже творилъ Давидъ святый\footnote{Пс.~118.}. "--- 2)~За сей великій небеснаго Отца даръ отъ сердца благодарить. "--- 3)~Поучаться въ немъ день и нощь. "--- 4)~По правилу Его житіе свое и нравы исправлять. "--- 5)~Неблагодарни и несмысленни суть, которые оставляютъ его читать или слушать; которымъ нѣтъ времени читать Божія слова, а есть время читать забавныя книжки; которые тщатся знать, что въ Италіи, въ Римѣ, въ Азіи, въ Африкѣ и прочіихъ мѣстахъ дѣлается, а не хотятъ изъ слова Божія познавать, что въ душахъ ихъ дѣлается и къ какому они концу идутъ, къ вѣчному животу, или мученію вѣчному, что безъ сумнѣнія есть знаменіе заблуждшія души.

\paragraph*{§\:3.} Священная Библія, или книги, въ которыхъ содержится Божіе слово, раздѣляются на книги Ветхаго и Новаго Завѣта. \textit{Ветхаго Завѣта} книги вси тіи разумѣются, которыя написаны до пришествія Хрістова, какъ"=то: книги Моисеовы, Пророческія, и проч. \textit{Новаго завѣта} книги тіи суть, которыя написаны отъ апостоловъ по пришествіи Хрістовомъ, и содержится въ нихъ ученіе и проповѣдь святая о спасительномъ Хрістовомъ пришествіи, какъ"=то: четыре Евангелія, Апостольская Посланія, и проч.

\paragraph*{§\:4.} Слово Божіе, въ священныхъ Ветхаго и Новаго Завѣта книгахъ содержащееся, раздѣляется на двѣ главнѣйшія части, то"=есть, \textit{Законъ и Евангеліе}. Подъ именемъ закона Божія разумѣются заповѣди Божія, въ которыхъ Богъ или повелѣваетъ, или запрещаетъ намъ что дѣлать, какъ"=то: \textit{уклонися отъ зла, и сотвори благо}\footnote{Пс.~33,~15.}. Подъ именемъ Евангелія разумѣются вси милостивыя и радостныя Его намъ обѣщанія, или уже исполнившіяся, какъ"=то: о пришествіи Хріста Сына Божія, Который уже пришелъ и великое спасенія нашего дѣло совершилъ, "--- или еще исполнитися имѣющіяся, какъ"=то: общее воскресеніе и животъ вѣчный вѣрующимъ во Хріста. Евангелія вся сила и сокращеніе означается отъ Хріста въ сихъ словахъ: \textit{Тако возлюби Богъ міръ, яко и Сына Своего Единороднаго далъ есть, да всякъ вѣруяй въ Онь, не погибнетъ, но имать животъ вѣчный}\footnote{Іоан.~3,~16.}.

\paragraph*{§\:5.} Законъ Божій, хотя прародителямъ нашимъ Адаму и Евѣ въ раи данъ, и написанъ былъ на сердцахъ ихъ; но потомъ Моисею святому на горѣ Синайской, на двухъ скрижаляхъ отъ Бога написанный преданъ, и чрезъ Него всѣмъ людямъ объявленъ, якоже о томъ пишется въ книгѣ Исхода\footnote{Исх.~32 и 34.}. Евангеліе же Хрістосъ Сынъ Божій отъ нѣдръ Отца Своего небеснаго на землю принеслъ и проповѣдалъ: \textit{и пришедъ благовѣсти миръ намъ дальнимъ и ближнимъ}\footnote{Ефес.~2,~17.}, якоже Самъ о Себѣ глаголетъ: \textit{Духъ Господень на Мнѣ, егоже ради помаза Мя благовѣстити нищимъ, посла Мя исцѣлити сокрушенныя сердцемъ, проповѣдати плѣненнымъ отпущеніе и слѣпымъ прозрѣніе, отпустити сокрушенныя во отраду: проповѣдати лѣто Господне пріятно}, и проч.\footnote{Лук.~4,~18 и 19.} О законѣ и Евангеліи Іоаннъ святый Евангелистъ вкратцѣ написалъ: \textit{законъ Моисеомъ данъ бысть: благодать же и истина Іисусъ Хрістомъ бысть}\footnote{Іоан.~1,~17.}.

\paragraph*{§\:6.} Что Слово Божіе открываетъ намъ, тому такъ несумнѣнно должно вѣрить, какъ бы глазами нашими видѣли тое, или далеко болѣе. Понеже чувство наше удобнѣе насъ обмануть можетъ, нежели Слово Божіе, яко отъ неложнаго Бога сказанное. Ему болѣе должны мы вѣрить, нежели отъ мертвыхъ воскресшему и повѣдающему; писано бо есть: \textit{аще Моисеа и пророки не послушаютъ, ни аще кто отъ мертвыхъ воскреснетъ, не имутъ вѣры}\footnote{Лук.~16,~31.}. Божіе бо свидѣтельство, которое во святомъ Писаніи объявлено и знаменіями и чудесами утверждено, достовѣрнѣйшее есть паче гласа всего свѣта. Ибо \textit{свидѣтельство Господне вѣрно}\footnote{Пс.~18,~8.}. Всякъ бо человѣкъ солгать можетъ; но Богъ, яко вѣчная истина, солгать не можетъ. Яко \textit{вѣренъ Господь во всѣхъ словесѣхъ Своихъ}\footnote{144,~13.}. И \textit{небо и земля мимоидутъ, словеса же Божія не имутъ прейти}\footnote{Матѳ.~24,~35; Лук.~21,~33.}. И потому, что открылъ намъ Богъ, тое неотмѣнно тако есть; и что предсказалъ имѣющее быть, тое непремѣнно въ свое время будетъ. Будетъ непремѣнно воскресеніе мертвыхъ, будетъ второе пришествіе Хрістово, какъ и первое было; будетъ судъ Его праведный, будетъ праведнымъ и грѣшнымъ воздаяніе, и проч.

\paragraph*{§\:7.} Слово Божіе отъ Бога предано намъ, и отъ богодухновенныхъ мужей написано на такій конецъ, дабы мы, держась его, спасеніе вѣчное получили. \textit{Сія писана быша}, глаголетъ Іоаннъ святый, \textit{да вѣруете, яко Іисусъ есть Хрістосъ Сынъ Божій, и да вѣрующе животъ имате во имя Его}\footnote{Іоан.~20,~31.}. И Павелъ святый: \textit{елика преднаписана быша, въ наше наказаніе преднаписашася, да терпѣніемъ и утѣшеніемъ писаній упованіе имамы}\footnote{Римл.~15,~4.}. Откуду и Хрістосъ Іудеевъ къ испытанію Писаній отсылаетъ: \textit{испытайте писаній, яко вы мните въ нихъ имѣти животъ вѣчный}\footnote{Іоан.~5,~39.}. И хотя праотцы и патріархи святіи спаслися безъ святаго Писанія; но они \textit{живымъ} Божіимъ гласомъ учими, наставляеми и утѣшаеми были, якоже читаемъ въ книзѣ Бытія. Намъ же таковаго гласа и наставленія Божія не слѣдуетъ ожидать, но отъ святаго Его слова написаннаго должно искать совѣта и наставленія, по словеси Хрістову: \textit{испытайте писаній}. Слово убо Божіе есть несумнѣнное, истинное, твердое и непоколебимое, яко отъ Бога данное, вѣры святыя и богоугоднаго хрістіанскаго житія правило. Ибо есть свѣтильникъ ногамъ нашимъ, какъ исповѣдуетъ пророкъ: \textit{свѣтильникъ ногама моима законъ твой, Господи, и свѣтъ стезямъ моимъ}\footnote{Пс.~118,~105.}. И \textit{пророческое слово есть, яко свѣтило, сіяющее въ темномъ мѣстѣ, емуже} должно \textit{внимать, дондеже день озаритъ и денница возсіяетъ въ сердцахъ нашихъ}\footnote{2~Петр.~1,~19.}. И \textit{законъ Господень непороченъ, обращаяй души; свидѣтельство Господне вѣрно, умудряющее младенцы; оправданія Господня права, веселящая сердце; заповѣдь Господня свѣтла, просвѣщающая очи сердечныя}\footnote{Пс.~18,~8 и 9.}. \textit{Всяко бо писаніе богодухновенно, и полезно есть ко ученію, ко обличенію, ко исправленію, къ наказанію, еже въ правдѣ: да совершенъ будетъ Божій человѣкъ, на всякое дѣло благое уготованъ}\footnote{2~Тим.~3,~16.}. Откуду благовѣствованіе Хрістово называется и \textit{есть сила Божія во спасеніе всякому вѣрующему}\footnote{Римл.~1,~16.}, яко \textit{вѣра отъ слуха, слухъ же глаголомъ Божіимъ}\footnote{10,~17.}, котораго посредствіемъ \textit{вѣруемъ} во Хріста\footnote{Іоан.~17,~20.}; яко \textit{буйствомъ проповѣди благоизволилъ Богъ спасти вѣрующихъ}\footnote{1~Кор.~1,~21.}. Сего ради всѣмъ хрістіанамъ, которые истинную и живую вѣру имѣть, и тую хранить до конца и тако спастися хотятъ, прилежное его чтеніе или слушаніе нужно есть. Якоже бо ходящимъ по пути, или дѣлающимъ что потребенъ есть свѣтъ чувственный: тако идущимъ къ вѣчному животу и труждающимся въ подвигѣ вѣры и благочестія, потребенъ есть свѣтильникъ Божія слова, да не заблудятъ на путь нечестивыхъ. И якоже тѣло по вся дни укрѣпляется пищею, да не ослабѣетъ, и ослабѣвши исчезнетъ: тако подобаетъ по вся дни укрѣплять душу духовною слова Божія пищею, да не гладомъ истаявши ослабѣетъ, и тако погибнетъ. Слѣпъ бо есть человѣкъ самъ въ себѣ, и потому требуетъ просвѣщенія; слабъ есть, и потому нужно есть ему подкрѣпленіе; лѣнивъ и унылъ, и потому потребно ему поощреніе и утѣшеніе, "--- что все изъ слова Божія получается. Мнози бо козни діавольскія и прелести міра, которыя вси тщатся совратить душу съ пути благочестія, отъ которыхъ слово Божіе предостерегаетъ насъ. Слѣдственно заблуждаютъ тые хрістіане, которые отъ божественнаго сего правила удаляются, и не иначе, какъ слѣпые, или во тмѣ находящіеся, ходятъ; и наконецъ, когда въ томъ до конца пребудутъ, имѣютъ впасти въ ровъ погибели.

\paragraph*{§\:8.} Слово Божіе, яко истиннѣйшее и совершеннѣйшее благочестія правило, всѣмъ хрістіанамъ нужно, какъ сказано; но пастырямъ, то"=есть, епископамъ и іереямъ, наипаче: ибо они взяли \textit{ключъ разумѣнія}\footnote{Лук.~11,~52.}, иже есть слово Божіе, и тѣмъ должны какъ себѣ, такъ и другимъ отворять дверь ко Хрісту Живому Богу и живота источнику, и вѣчному блаженству, смертію Его отверстому. Апостолъ святый къ Тимоѳею, а въ лицѣ его всякому пастырю написалъ: \textit{внимай себѣ и ученію и пребывай въ нихъ; сія бо творя и самъ спасешися, и послушающіи тебе}. И мало выше: \textit{внемли чтенію, утѣшенію, ученію}\footnote{1~Тим.~4,~16 и 13.}. Слѣдственно неисправны въ своемъ званіи бываютъ пастыри, которые не внимаютъ, по увѣщанію апостольскому, чтенію святаго Писанія. Како бо таковый пастырь другихъ научитъ, а самъ невѣжда? како другихъ просвѣтитъ, а самъ во тмѣ и слѣпотѣ? како другихъ наставитъ, а самъ заблуждаетъ? Таковые пастыри уподобляются отъ Хріста вождямъ слѣпымъ: \textit{вожди суть слѣпи слѣпцемъ; слѣпецъ же слѣпца аще водитъ, оба въ яму впадутъ}\footnote{Матѳ.~15,~14.}.

\paragraph*{§\:9.} Слово Божіе, какъ всѣмъ вообще, такъ всякому особенно, мнѣ и тебѣ, и другому равно предано и написано. Ибо Богъ лица человѣческаго не пріемлетъ, но \textit{всѣмъ хощетъ спастися, и въ разумъ истины пріити}\footnote{1~Тим.~2,~4.}. Потому и слово Свое святое ради всѣхъ и всякаго повелѣлъ написать дабы всякъ, читаючи или слушаючи его, моглъ спасеніе вѣчное получить. Якоже убо мнѣ глаголетъ Богъ въ словѣ Своемъ: \textit{Азъ есмь Господь Богъ твой; возлюбиши искренняго твоего, якоже себе самаго}, и проч.: такъ тоежде и тебѣ глаголетъ. И Хрістосъ всѣмъ испытовать Писаній повелѣваетъ: \textit{испытайте Писаній}. И надписи Апостольскихъ Посланій ко всѣмъ хрістіанамъ означаются, какъ изъ тѣхъ посланій всякъ можетъ видѣть. И святый апостолъ Іоаннъ въ своемъ посланіи написалъ: \textit{пишу вамъ, отцы, пишу вамъ, юноши, пишу вамъ, дѣти}, и проч.\footnote{Іоан.~2,~13 и 14.} Къ отцамъ, юношамъ и дѣтямъ написалъ: слѣдственно 1)~всѣмъ и всякому, и всякаго званія и чина людямъ, то"=есть, освященнымъ и неосвященнымъ, благороднымъ и простымъ, мужамъ и женамъ читать и слушать его можно и должно; 2)~всѣхъ и всякаго званія и чина людей обдолжаетъ къ послушанію, то"=есть къ уклоненію отъ зла и творенію добра; 3)~погрѣшаютъ тѣ, которые мнятъ и научаютъ, что священнаго Писанія не должно читать людямъ простымъ, но токмо іереямъ и прочіимъ освященнымъ лицамъ. И точно, мнѣніе сіе есть вымыслъ и кознь діавола, который отводитъ отъ душеполезнаго сего чтенія людей, дабы, не читаючи святаго Писанія, истинныя и живыя не имѣли вѣры, и тако бы не спаслися.

\paragraph*{§\:10.} Понеже не слышащіи токмо слово Божіе ублажаются, но слышащіи и хранящіи его, якоже глаголетъ Хрістосъ: \textit{блаженни слышащіи слово Божіе, и хранящіи его}\footnote{Лук.~11,~28.}: того ради должно тщатися, чтобы слышать и слышанное хранить съ Божіею помощію. Откуду апостолъ увѣщаваетъ хрістіанъ: \textit{бывайте творцы слова, а не точію слышателіе, прельщающе себе самихъ: зане аще кто есть слышатель, а не творецъ, таковый уподобися мужу, смотряющу лице бытія своего въ зерцалѣ: усмотрѣ бо себе, и отъиде, и абіе забы, каковъ бѣ}\footnote{Іак.~1,~23 и 24.}. Ибо не того ради Богъ слово Свое объявилъ, чтобъ оно токмо внѣ на хартіи, аки мертвое нѣкое начертаніе, лежало, но чтобъ внутрь въ сердцахъ нашихъ плодъ свой имѣло. Слово бо Божіе есть сѣмя \textit{живое} божественное, которое должно на землѣ сердецъ нашихъ духовные плоды проращать. Къ чему сѣмя на землѣ посѣянное пользуетъ, когда плода не приноситъ? Такъ и слово Божіе проповѣданное и слышанное ничего не пользуетъ, когда плода въ сердцахъ нашихъ не приноситъ, то есть когда по правилу его не тщимся житія нашего исправлять. Монаршій указъ того ради публикуется, чтобъ подданные его волю знали, и по тому исполняли: тако и слово Божіе того ради публиковано, чтобъ мы по правилу его житіе наше исправляли. Ничего убо не пользуетъ слышать слово Божіе и по правилу его не жить: паче же въ горшее осужденіе будетъ слышанное слово Божіе и несохраненное, какъ ниже сказано будетъ.

\paragraph*{§\:11.} Богъ въ святомъ словѣ своемъ къ душѣ человѣческой глаголетъ: \textit{Азъ есмь Господь Богъ} твой, и проч.; къ душѣ глаголетъ: \textit{уклонися отъ зла, и твори благое; покайся, вѣруй, смиряйся, люби, терпи}, и проч. Убо душа должна слушать гласъ Божій, повиноватися, каятися, вѣровать, любить, терпѣть и проч. И когда помыслы злые возникаютъ, не принимать ихъ; а когда добрый совѣтъ внутрь чувствуется, послѣдовать ему. Слѣдственно: 1)~Ничего не пользуетъ внѣ являться исправнымъ, а внутрь быть неисправнымъ; тѣломъ смиряться, а душею гордиться; на языкѣ вѣру и любовь имѣть, а на сердцѣ невѣріе и плоды его, "--- что есть лицемѣріе. "--- 2)~Всякій грѣхъ въ душѣ есть, который потомъ внѣ на удахъ является, и чрезъ тые совершается; напр. убійство, хищеніе, блудъ и проч., въ душѣ есть. Не будетъ бо рука убивать, похищать, не будетъ языкъ злословить, клеветать, не будетъ око смотрѣть, ухо слышать непристойнаго, чрево объядаться, ноги на зло ходить, когда душа не захочетъ. Такожде всякая добродѣтель должна быть въ душѣ, и въ случаѣ внѣ оказываться и дѣйствіе свое являть. Напр. вѣра должна быть въ душѣ, и въ случаѣ оказывать себе исповѣданіемъ; любовь въ душѣ, и въ случаѣ являть себе чрезъ дѣла милости, и проч. Кто истинную любовь имѣетъ къ ближнему, не откажетъ просящему. И Богъ, когда глаголетъ человѣку: \textit{не убій, не укради}, не къ рукамъ глаголетъ, но къ душѣ, отъ которой убійство и воровство происходитъ, и чрезъ руки совершается; такожде не къ языку глаголетъ: \textit{не лжесвидѣтельствуй}, но къ душѣ, которая языкъ злоупотребляетъ къ лжесвидѣтельству, и проч. Откуду отъ апостола уды называются \textit{оружія правды и неправды}\footnote{Римл.~6,~13.}. \textit{Оружія правды} бываютъ, когда ими душа дѣлаетъ правду; \textit{оружія неправды}, когда ихъ употребляетъ душа къ творенію неправды. "--- 3)~Не токмо тотъ есть убійца, тотъ хищникъ, прелюбодѣй и проч., кто самымъ дѣломъ дѣлаетъ зло, но и тотъ, кто хощетъ дѣлать зло. \textit{Не можетъ бо древо добро плоды злы творити, ни древо зло плоды добры творити}, глаголетъ Хрістосъ\footnote{Матѳ.~7,~18.}. Откуду въ Божіемъ словѣ приписуется убійство не токмо тому, кто дѣломъ убиваетъ человѣка, но и тому, кто ненавидитъ человѣка. \textit{Всякъ ненавидяй брата своего, человѣкоубійца есть}, глаголетъ Іоаннъ святый\footnote{1~Іоан.~3,~15.}. Такожде вожделѣніе блудное любодѣяніемъ называется отъ Хріста. \textit{Всякъ, иже воззритъ на жену, ко еже вожделѣти ея, уже любодѣйствова съ нею въ сердцѣ своемъ}\footnote{Матѳ.~5,~28.}. И хотя не казнитъ судъ гражданскій злыхъ вожделѣній, но Божій будетъ казнить, яко противу святаго и вѣчнаго закона Его содѣянныя: \textit{не пожелай}. Богъ бо не токмо внѣшніе грѣхи судитъ, но и внутренніе, хотя внѣ въ явленіе людямъ и не приходятъ.

\paragraph*{§\:12.} Богъ, Который повелѣваетъ уклоняться отъ зла, Тойжде велитъ и творить благое: \textit{уклонися отъ зла, и сотвори благо}\footnote{Пс.~33,~15.}; Который запрещаетъ красть, похищать, Тойжде повелѣваетъ давать: \textit{просящему у тебе дай}\footnote{Матѳ.~5,~42.}. Слѣдственно: 1)~Какъ къ убѣжанію отъ зла, такъ и къ творенію добра равно обдолжаемся. 2)~Отсюду слѣдуетъ, что какъ запрещенное дѣлать, такъ и повелѣннаго не дѣлать, противно есть святому Божію закону, и потому есть грѣхъ. Все бо, что противу закона Божія дѣлается, грѣхъ. \textit{Грѣхъ бо есть беззаконіе}! по свидѣтельству апостола\footnote{1~Іоанн.~3,~4.}. Не токмо бо за злыя дѣла, но и за небреженіе добрыхъ дѣлъ отсылаетъ Хрістосъ въ огнь вѣчный. \textit{Идите отъ Мене проклятіи во огнь вѣчный, уготованный діаволу и аггеломъ его}, возглаголетъ сущимъ ошуюю въ день суда Своего. А ради чего? \textit{Взалкахся бо}, рече, \textit{и не дасте Ми ясти}, и проч.\footnote{Матѳ.~25,~41 и 42.} О семъ поучаетъ и святый Василій великій въ словѣ о судѣ Божіи, и святый Златоустъ въ бесѣдѣ 13"~й къ Римл. на гл.~3"~ю. Отсюду послѣдуетъ, что богачи грѣшатъ, которые отъ имѣнія своего не удѣляютъ бѣднымъ, просящимъ у нихъ, такожде и прочіи грѣшатъ, которые дарованія Божія сокрываютъ у себе и не пользуютъ ими ближнихъ своихъ. Сіи бо суть \textit{таланты оные}, за которые слѣдуетъ намъ отвѣтъ дать Господу нашему въ день праведнаго Его суда\footnote{25,~14--30.}.

\paragraph*{§\:13.} Хотящему читать или слушать слово Божіе съ пользою духовною должно примѣчать слѣдующее: 1)~Понеже оно есть дражайшій Божій даръ, должно читать, или слушать со благоговѣніемъ, охотою и усердіемъ. Аще бо царя земнаго, или высокаго какого мужа, глаголющаго къ намъ, слушаемъ со вниманіемъ, усердіемъ и благоговѣніемъ: кольми паче Бога, иже есть Царь царей и Государь государей, Который намъ въ словѣ Своемъ глаголетъ и бесѣдуетъ. Какъ бо \textit{истиною и духомъ} моляся Ему, мы къ Нему бесѣдуемъ, такъ, когда по надлежащему читаемъ слово Его или слушаемъ служителя Его читающаго, Его къ намъ бесѣдующаго слышимъ. "--- 2)~Должно слушать или читать слово Божіе не ради того, чтобъ только остроумнымъ быть или словеснымъ, но чтобы Бога и Хріста Сына Божія, и волю Его святую познать, и тако вѣчное спасеніе получить. Сей есть надлежащій конецъ чтенія или слышанія Божія слова. Какъ бо дано оно намъ для спасенія души, такъ ради того и читать или слушать его должно. "--- 4)~Скрыть его въ сердцѣ, яко драгое духовное сокровище, якоже Давидъ пророкъ чинилъ: \textit{въ сердцѣ моемъ скрыхъ словеса Твои}, Господи, \textit{яко да не согрѣшу Тебѣ}\footnote{Пс.~118,~11.}; хранить и \textit{поучаться день и нощь}\footnote{1,~2.}, и тако имъ душу питать, какъ хлѣбомъ питается тѣло или паче болѣе. Ибо какъ тѣло безъ пищи ослабѣваетъ и исчезаетъ, тако вѣра безъ пищи Божія слова ослабѣваетъ, а далѣе и исчезаетъ. Или какъ свѣтильникъ безъ елея угасаетъ, тако вѣра и все благочестіе безъ слова Божія оскудѣваетъ и угасаетъ. \textit{Вѣра бо отъ слуха, слухъ же глаголомъ Божіимъ}\footnote{Римл.~10,~17.}. Того ради, какъ свѣтильнику приливаемъ елей, дабы не угасалъ, тако подобаетъ возжигать и возгрѣвать отъ слова Божія вѣру, дабы не угасла, и тако лишимся всего духовнаго блаженства, которое въ вѣрѣ состоитъ. "--- 4)~Думать и держать такъ всякому безъ сумнѣнія, что какъ всѣмъ людямъ, такъ мнѣ и тебѣ, подлому, нищему, убогому, окаянному и грѣшному, Богъ великій, всемогущій, святый и страшный глаголетъ въ словѣ Своемъ: \textit{Азъ есмь Господь Богъ Твой}: покайся, вѣруй, смиряйся, люби, терпи, буди кротокъ и проч., Котораго, яко праведнаго, не слушать страшно; яко благаго и благоутробнаго Отца, оскорблять жалостно; яко благодѣтеля, не почитать безстыдно; яко вездѣсущаго и всевѣдущаго, утаиться невозможно. Напротивъ того, \textit{трепещущему словесъ Его милость многа у Него}\footnote{Ис.~66,~2; Пс.~102,~11.}. "--- 5)~Не смотрѣть на другихъ, что дѣлаютъ, кто бы они ни были, но держаться единаго Божія слова, и разсуждать, чему оно поучаетъ. Ибо день отъ дне вѣра, и съ вѣрою любовь умаляется въ людяхъ, и умножаются соблазны, которые колеблютъ сердце наше, и заченшуюся вѣру хотятъ угасить. "--- 6)~Понеже разумъ нашъ безъ просвѣщенія Божія слѣпъ, воля безъ благодати Божіей зла, и хотѣніе и тщаніе безъ помощи Божіей не сильно: того ради должны усердно Богу молиться, чтобы Самъ разумъ нашъ просвѣтилъ, волю исправилъ, хотѣнію и тщанію нашему помоглъ, "--- подражая богомудрому Псаломнику, который чрезъ весь СXVIII псаломъ усердно молился, чтобы Богъ его вразумилъ и наставилъ на путь заповѣдей Своихъ, помогалъ и велъ по пути истины. Слово убо Божіе должно съ молитвою начинать, съ молитвою читать или слушать, съ молитвою и благодареніемъ кончать. Откуду и въ собраніи церковномъ предъ начатіемъ чтомаго Божія слова, то"=есть, Апостола и Евангелія, молимся, и по окончаніи благодаримъ Богу, за сей великій Его даръ. Начинающему читать или слушать Божіе слово можно тако молитися со Псаломникомъ: \textit{открый очи мои} душевныя, \textit{и уразумѣю чудеса отъ закона Твоего}, Господи. \textit{Пришлецъ азъ есмь на земли: не скрый отъ мене заповѣди Твоя}\footnote{Пс.~118,~18 и 19.}. Или такъ: Господи Іисусе Хрісте Сыне Божій! отверзи умъ мой разумѣти святое Твое слово, якоже апостоламъ Твоимъ отверзлъ еси\footnote{Лук.~24,~31.}. Или какъ кому благодать Божія подастъ. Читающему и, слушающему Божіе слово должно такожде молитися, и молитву разсужденіемъ такожде, что читается, возбуждать. Напр. читающему или слушающему о \textit{блаженствахъ} слово\footnote{Матѳ.~5,~3--12.} молиться Хрісту, истиннаго блаженства Виновнику, чтобы корень блаженствъ, истинную и живую вѣру, насадилъ и укрѣпилъ въ сердцѣ его, отъ которой сладкіе проистекаютъ плоды, то есть, нищета духовная, умиленіе, кротость, алчба и жажда правды, любовь съ плодами, чистосердечіе и простосердечіе, и проч. "--- Читающему или слушающему оное Хрістово слово: \textit{не всякъ глаголяй Ми: Господи, Господи, внидетъ въ царствіе небесное, но творяй волю Отца Моего, Иже есть на небесѣхъ}\footnote{7,~21.}, взявъ въ разсужденіе, что какъ истинно и по надлежащему призывать Бога, такъ и волю Божію творить безъ Бога не можемъ, "--- \textit{безъ Мене не можете творити ничесоже}, глаголетъ Самъ Богъ\footnote{Іоан.~15,~5.}, "--- можно такую молитву къ тому присоединить: Господи! сподоби мене грѣшнаго призывати Тя \textit{духомъ и истиною}\footnote{4,~23.}; \textit{научи мя творити волю Твою, яко ты еси Богъ мой}\footnote{Пс.~142,~10.}, да, тако призывая Тя и волѣ Твоей святой послѣдуя, спасуся по неложному Твоему обѣщанію: \textit{всякъ, иже призоветъ имя Господне, спасется}\footnote{Римл.~10,~13.}. "--- Читающему или слушающему слово о томъ, како Хрістосъ отверзлъ очи слѣпымъ можно такую приложить молитву: Сыне Божій, просвѣтивый слѣпыя словомъ, яко Богъ! просвѣти и мои душевныя очи, да увижду Тя, Свѣта вѣчнаго, и послѣдую Тебѣ вѣрою и любовію. "--- Читающему или слушающему оное апостольское слово: \textit{Хрістосъ за всѣхъ умре, да живущіи не ктому себѣ живутъ, но умершему за нихъ и воскресшему}\footnote{Кор.~5,~15.}, можно тако молитися: Іисусе Сыне Божій, за всѣхъ купно и за мене грѣшнаго умерый! умертви плоти моея мудрованіе, помози мнѣ умрети грѣху и міру, да поживу Тебѣ, Искупителю моему, умершему за мене и воскресшему. Тако и въ прочемъ можно поступать, и молитву разсужденіемъ, разсужденіе молитвою возбуждать и подкрѣплять. Окончивъ чтеніе или слушаніе Божія слова, должно благодарить милосердому Богу, что сей свѣтильникъ благоволилъ намъ, сущимъ во тмѣ, возсіять къ просвѣщенію разума нашего, и молитися такожде, дабы, яко денница, возсіялъ въ сердцахъ нашихъ благодатію Святаго Духа, пророками и апостолами глаголавшаго.

\paragraph*{§\:14.} Хрістіане, безстрашно живущіи и законъ Божій нарушающія, собираютъ себѣ гнѣвъ Божій, по словеси апостола: \textit{человѣче! или о богатствѣ благости и кротости и долготерпѣніи Божіи нерадиши, невѣдый, яко благость Божія на покаяніе тя ведетъ? По жестокости же твоей и нераскаянному сердцу собираеши себѣ гнѣвъ въ день гнѣва и откровенія праведнаго суда Божія}\footnote{Римл.~2,~4 и 5.}. Собираютъ, глаголю, болѣе, нежели язычники, незнающіи истиннаго Бога и Его святаго закона. Ибо язычниковъ естественный только, а хрістіанъ неисправныхъ и ожесточенныхъ естественный и написанный законъ обличитъ и осудитъ въ день суда Хрістова. \textit{Отметаяйся Мене, и не пріемляй глаголъ Моихъ, имать судящаго ему. Слово, еже глаголахъ, то судитъ ему въ послѣдній день}\footnote{Іоанн.~12,~48.}, глаголетъ Хрістосъ. Словеса, глаголетъ Василій великій, священнаго Писанія на судѣ Хрістовомъ предстанутъ. Той бо рече: \textit{обличу тя, и представлю предъ лицемъ твоимъ грѣхи твоя}\footnote{Пс.~49,~21.}. Тогда предъ хрістіаниномъ ожесточеннымъ предстанутъ заповѣди, отъ него слышанныя и нарушенныя, его обличающія. Предстанутъ прещенія и обѣщанія Божія, самымъ дѣломъ исполнившіяся, въ святомъ Писаніи писанныя, которымъ онъ не вѣрилъ, и обличатъ его. Тогда возстанетъ въ немъ совѣсть, которая будетъ ему воспоминать благодѣянія Божія и благодатныя причины, отъ него презрѣнныя, какъ"=то: Хрістово ради грѣшниковъ въ міръ пришествіе, страданіе, смерть, и прочая; "--- слово Божіе, столько кратъ отъ него слышанное и презрѣнное, столько увѣщаваній проповѣдническихъ бывшихъ, и отъ него оставленныхъ; и симъ воспоминаніемъ, или паче грызеніемъ совѣсти, будетъ весьма мучитися, наипаче ради того, что никакой надежды уже не будетъ имѣть тое возвратить, что нерадѣніемъ потерялъ. Отъ сего воспослѣдуетъ, что самъ себе осуждать, укорять безполезно, *ненавидѣть*, самъ собою гнушаться и проклинать начнетъ, и пожелаетъ въ ничто обратиться, но не можетъ. Къ сему страшному мученію и тоскѣ присовокупится тягчайшее гнѣва Божія чувствованіе, и пламень палящій, но не снѣдающій, и прочая. Сего мученія презиратели закона Божія не избѣгнутъ, какъ себе ни ласкаютъ и умягчаютъ злую свою совѣсть. Ибо \textit{рабъ, вѣдѣвый волю господина своего, и не уготовавъ, ни сотворивъ по воли его, біенъ будетъ много}, глаголетъ Хрістосъ\footnote{Лук.~12,~47.}. Откуду и Хрістосъ \textit{горе} означаетъ слышащимъ слово Божіе и не кающимся. Тирянамъ, Сидонянамъ и Содомлянамъ отраднѣе будетъ въ день судный, нежели хрістіанамъ беззаконнующимъ, аще не покаются\footnote{Матѳ.~11,~21--24.}. \textit{Богъ бо поругаемъ не бываетъ. Еже человѣкъ посѣетъ, тое и пожнетъ}\footnote{Гал.~6,~7.}. Сего ради неотмѣнно должно хрістіанамъ обратитися всѣмъ сердцемъ къ Богу, которые беззаконнымъ житіемъ отъ Него отступили, и молить Его благость со слезами, чтобы паки въ высочайшую Свою принялъ милость, и творить плоды достойны покаянія, когда не хотятъ паче язычниковъ на себѣ судъ Божій дознати. \textit{Всякое бо древо, не творящее плода добра, посѣкаемо бываетъ и во огнь вметаемо}\footnote{Матѳ.~3,~10.}.

\paragraph*{§\:15.} Знаменіе есть гнѣва Божія, когда въ какомъ мѣстѣ слово Божіе не проповѣдуется, якоже глаголетъ Богъ чрезъ пророка: \textit{се дніе грядутъ, глаголетъ Господь, и послю гладъ на землю, не гладъ хлѣба, ни жажду воды, но гладъ слышанія слова Господня; и поколеблются воды отъ моря до моря, и отъ сѣвера до востокъ, и обтекутъ ищуще слова Господня, и не обрящутъ}\footnote{Амос.~8,~11 и 12.}. Понеже что тѣлу нашему хлѣбъ, тое душѣ слово Божіе. Какъ бо тѣло питается и укрѣпляется пищею, такъ душа питается и укрѣпляется въ вѣрѣ словомъ Божіимъ. Слѣдственно, какъ тѣлу гладъ бываетъ, когда, облаки не кропятъ, а земля изсохшая не даетъ плода: тако гладъ душамъ послѣдуетъ, когда слышанія слова Божія лишаются. Вѣра бо, которая словомъ Божіимъ питается и укрѣпляется, оскудѣваетъ и исчезаетъ тогда, а вмѣсто того невѣріе и суевѣріе послѣдуетъ. Оттуду бываетъ, что люди за грѣхъ почитаютъ тое, въ чемъ грѣха нѣтъ; напротивъ того, за грѣхъ не вмѣняютъ того, въ чемъ великій грѣхъ есть: добродѣтель называютъ порокомъ, порокъ добродѣтелію. Отсюду многіи поставляютъ за грѣхъ касаться нѣкоторыхъ снѣдей, чего Богъ не запретилъ; но поядаютъ домы вдовицъ, сиротъ и прочіихъ беззаступныхъ людей, что Богъ съ угроженіемъ временныя и вѣчныя казни запретилъ. Отсюда и прочіе беззаконные плоды послѣдуютъ, которые безбожное житіе показуютъ, какъ"=то: клятвопреступленіе, лихоимство, хищеніе, лесть, обманъ, лукавство, злоба непримирительная, всякая нечистота и всякое беззаконіе. Человѣкъ бо самъ въ себѣ есть слѣпъ, и потому требуетъ просвѣщенія; есть забытливъ, и потому требуетъ частаго воспоминанія; есть лѣнивъ, и ради того нужно ему поощреніе; есть дряхлъ и унылъ, и потому потребно ему утѣшеніе; есть поползновененъ, и для того требуетъ подкрѣпленія; есть сумнителенъ, и для того требуетъ наставленія, "--- что все изъ святаго Писанія почерпается. \textit{Всяко бо писаніе богодухновенно, и полезно есть ко ученію, ко обличенію, ко исправленію, къ наказанію, еже въ правдѣ: да совершенъ будетъ Божій человѣкъ, на всякое дѣло благое уготованъ}\footnote{2~Тим.~3,~16 и 17.}. Но когда всего того лишается, то не ино что оттуду послѣдуетъ, какъ бѣдственное и плачевное души состояніе. Лютъ гладъ тѣлесный, но лютѣйшій есть душевный; гладъ бо тѣлесный дѣлаетъ смерть тѣлесную, но гладъ душевный дѣлаетъ смерть души, которая далеко лютѣйшая смерти тѣлесной. Чимъ бо душа честнѣйшая есть отъ тѣла, тѣмъ бѣдствіе души лютѣйшее отъ тѣлеснаго. Смерти тѣлесной никому, ни праведнику, ни грѣшнику, не миновать; но душевной съ помощію Божіею избѣжимъ, когда слова Его святаго держаться будемъ. И тѣло, хотя умретъ, однакожъ, когда душа жива будетъ вѣрою, паки въ общее воскресеніе оживетъ, соединившися съ душею: а когда душа умретъ, то и тѣло и душа во вѣки погибнутъ. Горе душамъ бѣднымъ, которыя симъ смертоноснымъ гладомъ истаеваютъ! Но сугубое горе тѣмъ, которыхъ должность есть питать души пищею слова Божія, но не питаютъ ихъ, истаевающихъ! Тако бо они сами не входятъ въ царствіе Божіе, и хотящихъ внити не допускаютъ. Но еще большее горе, когда путь ко всякому беззаконію отворяютъ соблазнами своими. На нихъ бо, яко на предводителей своихъ, взирая, простые люди нравамъ ихъ и житію ихъ подражаютъ. Вотъ"=де тые"=то и тые дѣлаютъ тое: намъ ради чего не дѣлать? Тако безумно и слѣпо разсуждаютъ, и тако день отъ дне умаляется вѣра, а съ вѣрою любовь угасаетъ, оскудѣваетъ благочестіе, и умножается нечестіе. Надобно въ такомъ случаѣ, по словеси Хрістову, \textit{молитися Господину жатвы, да изведетъ дѣлателей на жатву свою}\footnote{Матѳ.~9,~38.}. О семъ великоважномъ дѣлѣ, въ которомъ спасеніе вѣчное состоитъ, 1)~воздыхать должно и молитися тѣмъ, у которыхъ въ сердцахъ благочестія искра не угасла; 2)~попеченіе и тщаніе приложить тѣмъ, которымъ кормило церквей поручено.

\paragraph*{§\:16.} Худой и то знакъ, когда слово Божіе проповѣдуется, но люди неохотно и лѣниво слушаютъ, и не углубляютъ въ сердцѣ своемъ, или, что горше того, удаляются отъ него. Въ притчѣ о сѣмени написано: \textit{иная сѣмена падоша при пути и пріидоша птицы и позобаша я. Другая же падоша на каменныхъ, идѣже не имѣяху земли многи: и абіе прозябоша, зане не имѣяху глубины земли. Солнцу же возсіявшу, присвянуша, и, зане не имѣяху коренія, изсхоша. Другая же падоша въ терніи, и взыде терніе, и подави ихъ. Другая же падоша на земли добрѣй, и даяху плодъ: ово убо сто, ово же шестьдесятъ, ово же тридесять}\footnote{Матѳ.~13,~4--8.}. Сѣмя есть слово Божіе. А чрезъ землю, какъ добрую, сотворшую плодъ отъ посѣяннаго сѣмене, разумѣются люди добрые, творящіи плоды покаянія: такъ чрезъ несотворшую плода три рода людей нерадивыхъ означаются, въ которыхъ слово евангельскаго ученія никакого плода не приноситъ. Перваго рода люди уподобляются \textit{пути}, которые слышачи не слышатъ слова Божія: то есть, слышатъ ушесами плотскими только проповѣдуемое слово, но внутрь въ сердце не допускаютъ его; не помышляютъ о слышанномъ, но въ иныхъ суетныхъ мысляхъ забавляются, и потому гласъ только слова Божія ударяетъ въ ушеса ихъ, но до сердецъ ихъ не доходитъ. И такъ діаволъ восхищаетъ отъ нихъ слово Божіе, \textit{да вѣровавше, не спасутся}. Втораго рода люди слышатъ слово ученія, и \textit{съ радостію пріемлютъ его}, какъ учитъ Хрістосъ; но понеже \textit{не имѣютъ корене въ себѣ}, то во время печали или гоненія погубляютъ тое, и \textit{безплодны} бываютъ. Третія часть людей есть таковыхъ, которые такожде слышатъ слово Божіе, но, какъ терніе сѣмя, тако \textit{печаль вѣка сего и прелесть богатства}, слышанное слово въ нихъ \textit{подавляетъ, и безъ плода бываетъ}, якоже таможде самъ Хрістосъ изъясняетъ\footnote{19--22; Марк.~4,~14--19; Лук.~8,~11--14.}. "--- Отъ сего видно: 1)~Что и слышать слово Божіе, но съ нерадѣніемъ, ничего не пользуетъ. "--- 2)~Слышанное слово Божіе, и съ радостію пріятое, но въ сердцѣ не углубленное, печалію вѣка сего и напастьми, или суетами мірскими, славолюбіемъ, сребролюбіемъ, и сластолюбіемъ, какъ терніемъ, подавляется, и такожде безплодно бываетъ. "--- 3)~Аще и слышащіи слово Божіе, но съ нерадѣніемъ, и въ сердцѣ не скрывающіи его, никакого плода не приносятъ: какій уже плодъ принесутъ тѣ, которые удаляются отъ слова Божія, и не хотятъ его ни слышать, ни читать? "--- 4)~Отъ сего паки явствуетъ, коль малое есть число спасающихся. Четвертая часть только отъ слышащихъ слово Божіе спасается, какъ видно изъ притчи. Выключи удаляющихся отъ слова Божія, выключи идолопоклонниковъ, іудеевъ, магометанъ, еретиковъ и прочіихъ суевѣровъ, и такъ малое стадо Хрістовыхъ овецъ останется, якоже Самъ Хрістосъ глаголетъ: \textit{не бойся, малое стадо}\footnote{Лук.~12,~32.}; понеже мало кто хощетъ тѣснымъ путемъ итить, который \textit{единъ вводитъ въ животъ}\footnote{Матѳ.~7,~14.}. "--- 5)~Мы, хрістіанине, разсмотримъ себе и потщимся быть не только слышатели, но и творцы закона; сотворимъ плоды достойны покаянія, да не и насъ, яко безплодныхъ, сѣкира суда Божія посѣчетъ, и отошлетъ въ огнь. Писано бо есть: \textit{всяко древо, не творящее плода добра, посѣкаемо бываетъ и во огнь вметаемо}\footnote{3,~10.}.

\subsection[Глава 2-я. О духовной мудрости.]{глава вторая.\\\bfseries О духовной мудрости.}

\begin{quotation}\textit{Аще кто отъ васъ лишенъ есть премудрости, да проситъ отъ дающаго Бога всѣмъ нелицепріемнѣ, и не поношающаго, и дастся ему}\footnote{Іак.~1,~6, и проч.}.\end{quotation}


\paragraph*{§\:17.} Начало духовныя премудрости есть страхъ Господень, якоже глаголетъ Давидъ святый: \textit{начало премудрости страхъ Господень}\footnote{Пс.~110,~10.}. О чемъ тако святый Василій поучаетъ: «страхъ Господень есть очищеніе души, якоже пророкъ молится: \textit{пригвозди страху Твоему плоти моя}. Яко, идѣже страхъ есть, тамо души всякая чистота живетъ; тамо всякое зло и неподобное дѣйствіе отбѣгаетъ. Ибо уды тѣлесные, страхомъ пригвождаемые, не могутъ двигнуться къ неподобнымъ дѣламъ. Какъ бо имѣяй гвозди, тѣлу своему вонзенные, недѣйствителенъ есть ради болѣзни той: такъ страхомъ Божіимъ содержимый человѣкъ ни глазъ обратить къ непотребнымъ, ни рукъ двигнуть къ злымъ дѣламъ и ни малаго чего, ни великаго противъ званія своего сдѣлать не можетъ, яко болѣзнію нѣкоею, ожиданіемъ угроженій объятъ»\footnote{\textit{Въ сл. на нач. притч}.}. Тако страхъ Господень есть начало премудрости: ибо очищаетъ душу отъ скверны грѣховной, и уготовляетъ мѣсто духовной мудрости. Какъ бо начало тѣлеснаго здравія есть, когда тѣло очищается отъ вредныхъ соковъ: тако начало здравія душевнаго есть, когда душа страхомъ Божіимъ свобождается отъ злыхъ похотей и налоговъ, которые ее, какъ вредные соки, въ немощь и безсиліе приводятъ.

\paragraph*{§\:18.} Страхомъ Божіимъ, какъ бы жестокимъ врачевствомъ, очистившаяся и утвердившаяся душа бываетъ пріятелищемъ истиннаго любомудрія, то"=есть, хрістіанскія любве. Ибо къ духовной мудрости требуется не токмо уклоняться отъ зла, но и творить благое. Глаголетъ бо: \textit{уклонися отъ зла, и сотвори благо}\footnote{Пс.~33,~15.}. Но какъ отъ зла уклоняемся страхомъ Божіимъ, якоже выше сказано, тако любовію Божіею привлекаемся къ добру. И тако, якоже страхомъ Божіимъ полагаемъ начало премудрости духовныя, такъ любовію къ совершенству возводимся, поколику въ семъ вѣцѣ возможно.

\paragraph*{§\:19.} \textit{Посредствія}, при которыхъ къ духовной мудрости приходимъ, суть: Святое Божіе слово, съ помощію Духа Святаго. Ибо \textit{священная Писанія}, по словеси апостола, \textit{могутъ умудрить во спасеніе вѣрою, яже о Хрістѣ Іисусѣ}\footnote{2~Тим.~3,~15.}. 2)~Искренняя молитва. Тако глаголетъ апостолъ: \textit{аще кто отъ васъ лишенъ есть премудрости, да проситъ отъ дающаго Бога всѣмъ нелицепріемнѣ и не поношающаго, и дастся ему. Да проситъ же вѣрою, ничтоже сумняся}\footnote{Іак.~1,~5,~6 и проч.}. Откуда Соломонъ глаголетъ: \textit{познавъ, яко не инако одержу} (премудрость), \textit{аще не Богъ дастъ, и сіе еже бѣ разума, еже вѣдѣти, чія есть благодать: пріидохъ ко Господу, и молихся Ему}\footnote{Прем.~8,~21.}. Духовныя убо мудрости догматамъ научаемся не въ книгахъ мудрецовъ вѣка сего, но въ книгахъ пророческихъ и апостольскихъ, и навыкаемъ ея въ школѣ Духа Святаго, Который посредствомъ слова Своего святаго умудряетъ вѣрныхъ Своихъ сердца. Того ради, когда хощемъ, хрістіанине, духовную мудрость сыскать, должно прилѣжно поучаться день и нощь въ словѣ Божіи, и усердно молиться Подателю премудрости "--- Богу.

\paragraph*{§\:20.} \textit{Мѣсто и сѣдалище} свое имѣетъ духовная мудрость въ сердцѣ, а не на языкѣ, внутрь, а не внѣ, въ силѣ, а не въ словеси. Не всякъ бо тотъ мудръ, кто внѣ и предъ людьми является таковымъ, но кто въ самой вещи таковъ; и не всякъ, кто красную рѣчь говорить можетъ, но кто и сердце таковое имѣетъ. Часто бо бываетъ, что житіе несходно есть съ краснорѣчіемъ; и часто подъ видомъ краснорѣчія крыется злонравіе, которое съ духовною мудростію купно быть не можетъ. Отсюда бываетъ, что грубый поселянинъ, алфавита незнающій, но въ страсѣ Божіи живущій, далеко искуснѣйшій и мудрѣйшій есть въ дѣлѣ хрістіанскомъ, нежели словесникъ и мудрецъ вѣка сего безъ страха Божія.

\paragraph*{§\:21.} \textit{Знаки} духовнаго любомудрія суть: 1)~тщательное поученіе въ словѣ Божіи; 2)~исканіе совѣта у благочестивыхъ и разумныхъ людей; 3)~любленіе наставленія, наказанія, полезнаго совѣта, обличенія; 4)~разсужденіе о дивномъ промыслѣ Божіи, о правдѣ и милосердіи Его; 5)~частое воспоминаніе о смерти и послѣдующихъ, то"=есть, о страшномъ судѣ Хрістовомъ, о блаженной и неблагополучной вѣчности; 6)~презрѣніе міра; 7)~паче всего молитва, безъ которой духовная мудрость обрѣстися не можетъ, какъ выше сказано и ниже довольно скажется.

\paragraph*{§\:22.} \textit{Плоды} духовныя мудрости суть добродѣтели хрістіанскія, то есть: смиреніе, терпѣніе, кротость, правда, цѣломудріе, милость, и прочая. Тако научаетъ Соломонъ: \textit{труды ея} (премудрости) \textit{суть добродѣтели; цѣломудрію бо и разуму учитъ, правдѣ и мужеству, ихже потребнѣе ничтоже есть въ житіи человѣкомъ}\footnote{Прем.~8,~7.}. И Іаковъ апостолъ глаголетъ: \textit{яже свыше премудрость, первѣе убо чиста есть, потомъ же мирна, кротка, благопокорлива, исполнь милости и плодовъ благихъ, несумнѣнна, нелицемѣрна}\footnote{Іак.~3,~17.}. Она боится Бога и любитъ Его; отъ суеты міра отвращается и къ единому прилѣпляется Богу; всего позволеннаго, какъ"=то пищи, питія, одежды и прочаго со страхомъ употребляетъ, ради нужды, а не ради роскоши; приключающіяся бѣды и напасти великодушно претерпѣваетъ. Послушливу себе Богу и человѣкамъ ради Бога показываетъ. Ей не тяжестно все тое дѣлать, что Богу угодно, и отъ всего уклоняться, что Ему противно. Гордости, зависти, злобѣ, враждѣ, нечистотѣ, сребролюбію и прочіимъ душевнымъ язвамъ нѣтъ въ ней мѣста. Она любитъ всѣхъ безъ разбору, со всѣми чистосердечно и простосердечно обходится. Что словомъ объявляетъ, тое и внутрь себе имѣетъ; что обѣщаетъ словомъ, тое исполняетъ дѣломъ, когда нѣтъ какого препятствія. Ей записи не нужны въ договорахъ; слово, отъ ней сказанное и отъ другаго слышанное, то ей твердая запись. Видитъ страждущаго брата, и сама съ нимъ сердцемъ состраждетъ; требующему помощи помогаетъ; о всякомъ злополучіи ближняго болѣзнуетъ, и благополучію радуется; съ плачущими плачетъ, и съ радующимися радуется; друговъ объятіями своими объемлетъ, и отъ враговъ любве своея не отнимаетъ. Откуду есть мирна, тиха, спокойна, радостна, весела, хотя отъ злаго духа и злыхъ его служителей безпокоится.

\paragraph*{§\:23.} \textit{Слѣдствія} духовныя мудрости, которыя ей послѣдуютъ, суть поношеніе, презрѣніе, изгнаніе и всякія бѣды. Понеже она не отъ міра сего есть, но свыше приходитъ, какъ учитъ апостолъ Іаковъ\footnote{Іак.~3,~12.}, то міръ безумный и во злѣ лежащій за буйство ее имѣетъ, ругается ей, ненавидитъ и гонитъ ее. Ибо \textit{вси, хотящіи благочестно жити о Хрістѣ Іисусѣ}, въ чемъ состоитъ духовная мудрость, \textit{гоними будутъ}, глаголетъ богомудрый Павелъ\footnote{2~Тим.~3,~12.}. И хотя она внутрь честна, великолѣпна и предъ Богомъ благопріятна, но внѣ презрѣнна и умаленна пребываетъ.

\paragraph*{§\:24.} \textit{Конецъ}, къ которому мудрость сія человѣка ведетъ, есть Богъ и вѣчное блаженство. Ибо какъ она сама отъ Бога происходитъ, тако къ Богу и обращается, и любителей своихъ приводитъ.

\paragraph*{§\:25.} Духовная мудрость во всемъ \textit{разнится} отъ плотской, или мірской. Плотская мудрость горда: духовная смиренна. Плотская мудрость самолюбива: духовная боголюбива. Плотская мудрость нетерпѣлива, злобна: духовная терпѣлива, кротка. Плотская мудрость непримирительна: духовная мирна. Плотская мудрость немилостива: духовная милостива и исполнена благихъ дѣлъ. Плотская мудрость ненавистлива, завистлива: духовная любительна. Плотская мудрость лукава, лестна, хитра: духовная простосердечна, истинна, чистосердечна. Плотская мудрость неправдива: духовная правдива. Плотской мудрости смиреніе, поношеніе, страданіе и крестъ Хрістовъ есть буйство: духовной есть великая премудрость. Такъ и въ прочемъ духовная мудрость плотской противится.

\paragraph*{§\:26.} Духовная убо мудрость, какъ изъ вышеписаннаго видѣть можно, не въ иномъ чемъ \textit{состоитъ}, какъ въ истинномъ Бога и Хріста сына Божія познаніи и почитаніи. Отъ познанія бо сего послѣдуетъ страхъ Божій, иже есть начало духовныя мудрости, "--- и любовь Божія, которая человѣка къ духовному совершенству приводитъ, елико въ семъ вѣцѣ постигнуть можно. Въ сихъ же двухъ, \textit{страсѣ Божіи и любви Божіей}, истинное Богопочитаніе заключается. Всему же сему начало и корень есть вѣра святая, какъ ниже въ слѣдующихъ увидишь.

\subsection*{\bfseries * * *}

\paragraph*{§\:27.} Въ семъ параграфѣ представляется, како отъ случая духовное разсужденіе, которому подаетъ основаніе слово Божіе, можно пріимать ищущему духовной мудрости.

\subsection[Случай и духовное отъ того разсужденіе.]{\bfseries Случай и духовное отъ того разсужденіе.}

\paragraph*{I.} Смотришь на небо, такъ чудное, высокое и обширное, различными звѣздами украшенное; на солнце и луну сіяющія и всю поднебесную осіявающія; на облака, въ воздухѣ туды и сюды преходящія, и, какъ мѣхи, дождь точащія и напаяющія нивы наши. Представляетъ тебѣ чувство и разумъ землю съ исполненіемъ ея, съ древесами, травами, скотами, звѣрями, морями, озерами, рѣками, источниками и прочіимъ украшеніемъ. Отъ видимыхъ прейди умомъ къ невидимымъ, отъ созданія міра къ Создателю. Да будетъ тебѣ случай сей удивлятися: 1)~\textit{всемогуществу} Бога нашего, Который вся сія изъ ничего словомъ единымъ сотворилъ\footnote{Быт.~1"~я.}; 2)~\textit{премудрости} Его, Который такъ премудро сотворилъ; 3)~\textit{благости} Его, Который вся сія насъ ради сотворилъ. Тако размышляющему, и отъ созданія Создателя силу, премудрость и благость познавающему предлежитъ съ Псаломникомъ радостно духомъ восхититися и воспѣвать: \textit{яко возвеличишася дѣла Твоя Господи: вся премудростію сотворилъ еси}\footnote{Пс.~103,~24.}.

\paragraph*{II.} Видишь, что всякое созданіе всѣмъ равно служитъ. Солнце, луна, звѣзды всѣхъ равно освѣщаютъ; воздухъ всѣхъ равно жизнь сохраняетъ; земля всѣхъ равно содержитъ и питаетъ; вода всѣхъ равно напояетъ "--- всѣхъ, глаголю, равно, то"=есть, богатаго и нищаго, славнаго и неславнаго, господина и раба. Симъ научаемся, что и благая наша, намъ отъ Бога данная, намъ и ближнимъ нашимъ \textit{обща} должна быть. Хлѣбъ нашъ, одежда наша, разумъ нашъ, домы наши ближнимъ нашимъ должны общи быть.

\paragraph*{III.} Видишь паки, что едино созданіе, высшія стихіи низшимъ помогаютъ. Солнце воздухъ и землю просвѣщаетъ и согрѣваетъ; облака воздухъ и землю орошаютъ; воздухъ къ дыханію животнымъ служитъ; земля подаетъ пищу намъ и скотамъ нашимъ. Все сіе научаетъ насъ, что наипаче мы, яко разумная тварь, должны другъ другу \textit{помогать}: богатый нищему, разумный неразумному, здоровый немощному, свободный въ темницѣ заключенному, сильный обидимому, полномочный беззаступному, и проч. Откуду слово сіе: \textit{что мнѣ до него нужды}? "--- отъ предѣловъ хрістіанскихъ должно прогнано быть.

\paragraph*{IV.} Видишь раба, предстояща господину своему, и со всякимъ послушаніемъ повелѣніе его исполняюща. Сей случай подаетъ тебѣ разсудить, съ коликимъ, какъ несравненно большимъ почтеніемъ и благоговѣніемъ предстоять и обращаться предъ Богомъ, Господемъ нашимъ, должно намъ (ибо Богъ на всякомъ мѣстѣ есть, и гдѣ ни находимся, и что ни дѣлаемъ, предъ Божіимъ лицемъ, на насъ взирающимъ, дѣлаемъ), и Его повелѣнія съ усерднѣйшимъ послушаніемъ исполнять. Понеже господинъ, какъ бы высокъ ни былъ: человѣкъ есть, и человѣкъ такой же, каковъ и рабъ его, "--- но такое ему благоговѣинство рабъ его показуетъ, "--- а Богъ есть Господь господей и рабовъ, и Царь царей и подданныхъ ихъ; и Ему какъ господа, такъ и раби ихъ, царіе и подданные ихъ, раби суть; и Господь велій и страшенъ, Емуже и силы небесныя со страхомъ предстоятъ и благоговѣютъ, и повелѣнія Его исполняютъ. Разсуди же и тое, какъ безстыдно дѣлаемъ и тяжко грѣшимъ мы, когда и такого почтенія и благоговѣнія не показуемъ Ему, какое раби господину своему, подобному человѣку, показуютъ, но безчинствуемъ предъ Нимъ; и такого послушанія не дѣлаемъ, какое отъ рабовъ дѣлается человѣку"=господину, но смѣло и безстрашно дерзаемъ заповѣди Его нарушать. Въ чемъ насъ Самъ Господь чрезъ пророка обличаетъ: \textit{сынъ славитъ отца, и рабъ господина своего убоится. И аще Отецъ есмь Азъ: то гдѣ слава Моя? И аще Господь есмь Азъ: то гдѣ есть страхъ Мой? глаголетъ Господь Вседержитель}\footnote{Мал.~1,~6.}. Отсюду научись, коль великая наша слѣпота и нечувствіе, купно и удивляйся благости и долготерпѣнію Бога нашего, Который такъ кротко и терпѣливо съ нами, непотребными рабами, поступаетъ. Сіе научаетъ тебе не только не дѣлать и не говорить нигдѣ, но и не мыслить худаго и непристойнаго. Предъ Богомъ бо, яко вся проницающимъ, какъ слово и дѣло, такъ и помышленіе равно есть, ибо Онъ и помышленіе наше такъ зритъ, какъ самое дѣло; и Его помышленіе оскорбляетъ, какъ дѣло худое; и не тотъ только безчинствуетъ предъ Нимъ, кто худо дѣлаетъ, но и кто худо помышляетъ.

\paragraph*{V.} Видишь или въ женѣ "--- матери "--- къ своимъ дѣтямъ, или въ птицѣ къ своимъ птенцамъ, или въ скотѣ и звѣрѣ къ своимъ дѣтищамъ горячую любовь. Отъ сего случая возведи умъ твой къ любви Божіей, которую къ разумному Своему созданію "--- человѣку, по образу Своему созданному, имѣетъ. Ежели въ созданіяхъ Своихъ, напр. матеряхъ къ своимъ исчадіямъ, толикую насадилъ любовь: коль несравненно большую Самъ Онъ имѣетъ любовь къ человѣку, котораго по образу Своему и по подобію сотворилъ. Отъ сего источника проистекаетъ, что Онъ толико благодѣяній человѣку показуетъ, которыхъ не токмо исчислить, но и умомъ понять невозможно. Любовь Его несказанная дѣлаетъ, что Онъ такъ жаждетъ спасенія нашего, такъ много терпитъ согрѣшающимъ намъ, съ толикимъ желаніемъ ожидаетъ на покаяніе насъ, съ такимъ благопріятіемъ и радостію кающихся пріемлетъ, съ толикимъ неблаговоленіемъ некающихся наказуетъ. \textit{Тако возлюби Богъ міръ, яко и Сына Своего Единороднаго далъ есть, да всякъ вѣруяй въ Онь не погибнетъ, но имать животъ вѣчный}\footnote{Іоан.~3,~16.}. Откуду Самъ чрезъ пророка глаголетъ: \textit{еда забудетъ жена отроча свое, еже не помиловати исчадія чрева своего? Аще же и забудетъ сихъ жена, но Азъ не забуду тебе, глаголетъ Господь}\footnote{Ис.~49,~15.}. Сія отеческая небеснаго Отца любовь праведника увеселяетъ, грѣшника утѣшаетъ и увѣщаваетъ не отчаятися милости Его, но паче поощряетъ прибѣгать къ Нему съ покаяніемъ, и несумнительно милости Его ожидать.

\paragraph*{VI.} Видишь солнце пріятно сіяющее и всю поднебесную просвѣщающее, и насъ увеселяющее. Отъ сего солнца чувственнаго возведи умъ твой къ Солнцу правды мысленному, вѣчному Хрісту, Сыну Божію, Который вѣрныхъ чудесно просвѣщаетъ, и во вѣки будетъ просвѣщать и увеселять такъ, что и сами они \textit{просвѣтятся, яко солнце во царствіи Отца ихъ}, по неложному Его обѣщанію\footnote{Матѳ.~13,~43.}. И отъ видимаго сего міра, который свѣтится блистаніемъ солнечныхъ лучей, прейди умомъ ко граду оному, въ Откровеніи описанному, вышнему Іерусалиму, который \textit{не требуетъ солнца и луны, да свѣтятъ въ немъ: слава бо Божія просвѣти его и свѣтильникъ ему Агнецъ}\footnote{Апок.~21,~23.}, "--- гдѣ \textit{нощи не будетъ, и не потребуютъ свѣта отъ свѣтильника, ни свѣта солнечнаго, яко Господь Богъ просвѣщаетъ я: и воцарятся во вѣки вѣковъ}\footnote{Апок.~22,~5.}.

\paragraph*{VII.} Видишь, что во время весны все аки оживляется и вновь отраждается: земля издаетъ изъ нѣдръ своихъ плоды, древеса листвіемъ одѣваются, травы зеленѣютъ и цвѣтами украшаются, которое все во время зимы, какъ мертвое было; воды мразомъ замерзшія, растаяваютъ и протекаютъ, и вся поднебесная тварь въ новый облекается видъ, и аки вновь созидается. Отъ сей чувственной весны да возведетъ вѣра умъ твой къ прекрасной и желаемой веснѣ, юже обѣща преблагій Богъ въ Писаніи Своемъ святомъ, въ которой тѣлеса отъ начала міра усопшихъ вѣрныхъ изъ земли, яко сѣмена, силою Божіею прозябнувше, востанутъ и облекутся въ новый, прекрасный видъ, въ ризу безсмертія одѣются, пріимутъ вѣнецъ доброты отъ руки Господни, \textit{процвѣтутъ яко финиксъ, и яко кедръ, иже въ Ливанѣ, умножатся, насаждени въ дому Господни и во дворѣхъ Бога нашего}\footnote{Пс.~91,~13 и 14.}, яко невѣста украсятся красотою, и яко земля, растящая цвѣтъ свой, и яко вертоградъ сѣмена своя, прозябнутъ; и веселіе вѣчное надъ главою ихъ, и радостію возрадуются о Господѣ\footnote{Ис.~61.}. Тогда \textit{тлѣнное сіе облечется въ нетлѣніе, и смертное сіе облечется въ безсмертіе. Тогда будетъ слово написанное: пожерта бысть смерть побѣдою. Гдѣ ти, смерте, жало? гдѣ ти, аде, побѣда}\footnote{1~Кор.~15,~54 и 55.}? Тако восхищаяся духомъ къ желаемой оной веснѣ, сѣй нынѣ съ вѣрою и надеждою, Божіею помощію, благихъ трудовъ сѣмена, да тогда пожнеши съ радостію.

\paragraph*{VIII.} Видишь, что наполненный сосудъ, или губа напоенная, ничего другаго въ себѣ не вмѣщаетъ; а напротивъ того праздный сосудъ все вмѣщаетъ. Разумѣй отъ сего, что и душа, наполненная любовію міра сего, то"=есть, славолюбіемъ, честолюбіемъ, сребролюбіемъ, и сластолюбіемъ, ничего духовнаго и небеснаго вмѣстить не можетъ: Бога любить и небесныхъ благъ желать по надлежащему, и радости духовныя въ себѣ имѣть не можетъ. Отъ сего учись, хрістіанине, мірскую любовь изгонять, чтобъ Божія вошла въ сердце твое. Подобное бо съ подобнымъ помѣщается, какъ"=то вода съ водою, и проч., якоже противное съ противнымъ не помѣщается, какъ"=то огнь съ водою. Нѣтъ въ томъ сердцѣ Божіей любви, гдѣ любовь сребра, славы, чести и сласти имѣется, якоже въ благолюбивомъ сердцѣ нѣсть любви мірской. Ибо Божія и мірская любовь суть противныя вещи, и потому вмѣстѣ быть не могутъ, но едина другую изгоняетъ. \textit{Никтоже можетъ двѣма господинома работати: любо единаго возлюбитъ, а другаго возненавидитъ; или единаго держится, о друзѣмъ же нерадити начнетъ. Не можете Богу работати и мамонѣ}, глаголетъ Хрістосъ\footnote{Матѳ.~6,~24 и 25.}.

\paragraph*{IX.} Видишь поутру ясно восходящее солнце и всѣхъ увеселяющее. Помысли, коль великое веселіе ощущаютъ души, въ которыхъ Хрістосъ, Сынъ Божій, вѣчное правды Солнце, возсіяетъ. Сей случай да научитъ тебе усердно молитися Ему, да и въ твоемъ сердцѣ блеснетъ благопріятный благодати Его свѣтъ.

\paragraph*{X.} Видишь, что во дни все ясно видится, и всякъ познаетъ вредъ и пользу, ровъ и путь, бѣлое отъ чернаго раздѣляетъ, и едину вещь отъ другой различаетъ. Разумѣй, что такъ просвѣщенный свѣтомъ благодати Хрістовой видитъ душевными очами все ясно, распознаетъ добро отъ зла, добродѣтель отъ порока: видитъ вредъ душевный, и уклоняется отъ него; познаетъ пороки на душѣ, и очищаетъ ихъ покаяніемъ и слезами съ помощію Божіею.

\paragraph*{XI.} Видишь, что тьма и мракъ въ поднебесной, когда солнце зайдетъ, и ничего не видно, бѣлаго отъ чернаго распознать, и гдѣ ровъ, гдѣ ровно, гдѣ опасно, гдѣ безопасно "--- познать невозможно. Разсуди, какая то тьма бываетъ въ душѣ, и какій мракъ окружаетъ ее, которая благодати Хрістовой въ себѣ не имѣетъ. Такая душа зла отъ добра, вреда отъ пользы своей не распознаетъ; ходитъ, и осязаетъ какъ слѣпый. Сей случай научаетъ тя молитися Хрісту, истинному Свѣту, чтобъ и въ твоей душѣ возсіялъ свѣтъ благодати Своея.

\paragraph*{XII.} Видишь судъ производимый, на которомъ стоя виноватый, обличаемый, трепещетъ отъ страха и весь стыдомъ покрывается; слышитъ выговоръ за преступленіе закона, лишается чести и имѣнія, выключается изъ числа добрыхъ и честныхъ людей, означается злымъ и непотребнымъ человѣкомъ; потомъ со стыдомъ отводится въ темницу, и по законамъ пріемлетъ казнь. Отъ сего суда да восхититъ вѣра духъ твой къ страшному оному и вселенскому суду, на которомъ Судія самъ Богъ, сердца наша испытуяй. Представь себѣ въ умѣ, что на судѣ томъ стоятъ грѣшники, отлученніи отъ числа праведныхъ; стоятъ со стыдомъ, страхомъ и трепетомъ несказаннымъ; обличаются за преступленіе закона Божія, обличаются предъ всѣмъ свѣтомъ, предъ ангелами и человѣками; видятъ свои грѣхи, которые словомъ, дѣломъ и помышленіемъ втайнѣ дѣлали, въ познаніе всему свѣту являемые; видятъ уготованную по дѣломъ своимъ себѣ казнь; видятъ тѣхъ, которыхъ здѣсь поносили, ругали, смѣялись, гонили, озлобляли и какъ сметіе попирали, въ славѣ и радости великой; видятъ неправедные праведнаго Судіи гнѣвъ, слышатъ страшный выговоръ и опредѣленіе: \textit{идите отъ Мене проклятіи въ огнь вѣчный, уготованный діаволу и аггеломъ его. Взалкахся бо, и не дасте Ми ясти}\footnote{Матѳ.~25,~41 и 42.} и проч. Наконецъ, съ превеликимъ стыдомъ, страхомъ, печалію, плачемъ, рыданіемъ, воплемъ и отчаяніемъ отвергаются отъ лица Божія; влекутся лютыми аггелами, и заключаются въ темницѣ, \textit{идѣже червь ихъ не умираетъ и огнь ихъ не угасаетъ}\footnote{Марк.~9,~48.}. Сей случай, и того разсужденіе, да научитъ тя каятися, плакати за грѣхи, и милости нынѣ искать у Судіи, чтобъ и тебѣ на вселенскомъ ономъ позорищѣ не посрамиться, не услышать страшнаго онаго выговора и опредѣленія.

\paragraph*{XIII.} Видишь, что огнь всегда къ верху идетъ, и сколько ни воспящается отъ того, однакожъ никакъ не измѣняетъ дѣйствія своего, но всегда въ высоту стремится, естество бо его такое есть. Сей случай показуетъ тебѣ, что истинная къ Богу любовь такоежъ дѣйствіе имѣетъ. Сею бо возгорѣвшееся сердце всегда къ центру своему, прелюбезному Существу, стремится, и чѣмъ ни воспящается отъ того, удержатися не можетъ, ни красота, ни сласть, ни слава, ни страхъ, ни мечь, ни смерть, не сильны тому учинить препятствія. Дозналъ на себѣ избранный сосудъ Хрістовъ, Павелъ, силу ея, который извѣстившися сказалъ: \textit{ни смерть, ни животъ, ни ангели, ни начала, ниже силы, ни настоящая, ни грядущая, ни высота, ни глубина, ни ина тварь кая возможетъ насъ разлучити отъ любве Божія, яже о Хрістѣ Іисусѣ Господѣ нашемъ}\footnote{Римл.~8,~39 и 40.}. Такой любви, сласть міра сего есть горесть, красота смрадъ, слава ничтоже, царство неволя и плѣненіе. Такой человѣкъ на землѣ ногами, а на небеси сердцемъ, на землѣ тѣломъ, а на небеси духомъ обращается; съ человѣками живетъ, но духомъ Богу \textit{любезному} предстоитъ и покланяется Ему; въ вѣрѣ яко \textit{зерцаломъ въ гаданіи}, зритъ Его; тѣломъ ястъ и піетъ, но духомъ сея пищи непрестанно алчетъ съ Давидомъ глаголя: \textit{имже образомъ желаетъ елень на источники водныя, сице желаетъ душа моя къ Тебѣ, Боже}\footnote{Пс.~41,~2.}. Сіе разсужденіе научаетъ тебе тщаться вкусить и видѣть, коль благъ Господь, и молить Его, чтобъ Онъ самъ возжеглъ искру любве Своея въ сердцѣ твоемъ. Любовь бо истинная, какъ отъ Бога происходить, такъ къ Богу и обращается.

\paragraph*{XIV.} Видишь паки, что камень, желѣзо, олово и прочая тяжелая вещь, сколько ни подымается и бросается вверхъ, однакожъ къ землѣ, своему центру, съ великимъ стремленіемъ обращается и падаетъ. Отъ сего случая научаешься, что такое состояніе пристрастившихся къ міру, которые сколько ни возбуждаются проповѣдію Божія слова, увѣщаніями и устрашеніями, и какъ бы подымаются, чтобъ, свободившеся отъ суеты, къ небу стремилися, однакожъ отъ обычая и пристрастія отстать не могутъ; и хотя временемъ тщатся и нудятся отъ сего зла отторгнуться, благодатію Божіею побуждаеми и къ верху, то есть, любленію небесныхъ благъ подняться, однакожъ паки внизъ, то есть, къ страсти обращаются съ немалымъ стремленіемъ. Страсть бо пристрастившагося не иначе, какъ магнитъ желѣзо, къ себѣ обращаетъ и привлекаетъ. Сіе разсужденіе увѣщаваетъ тебе отъ всякаго пристрастія не иначе, какъ отъ огня или какъ яда смертоноснаго, берещися, и всякими силами нудить себе къ желанію и исканію небесныхъ благъ, смертію Хрістовою пріобрѣтенныхъ. А тщащемуся Богъ помогаетъ.

\paragraph*{XV.} Видишь человѣка странствующаго, неимущаго дома своего, но въ чужомъ домѣ временно упокоевающагося. Сей случай да воспомянетъ тебѣ, что мы странники и пришельцы есмы въ мірѣ семъ: \textit{яко не имамы здѣ пребывающаю града, но грядущаго взыскуемъ}, по словеси апостола\footnote{Евр.~13,~14.}. Сіе показуетъ, что слѣдуетъ намъ скоро или нескоро странствованіе сіе окончить и пріити въ отечество. Почему и обращаться въ мірѣ семъ должны мы, какъ странники и пришельцы, и помнить апостольское слово: \textit{ничтоже внесохомъ въ міръ сей явѣ, яко ниже изнести что можемъ; имѣюще же пищу и одѣяніе, сими довольни будемъ}\footnote{1~Тим.~6,~7 и 8.}.

\paragraph*{XVI.} Когда находишься на чужой странѣ, или на дорогѣ, всегда мысль твоя къ отечеству и къ дому твоему клонится, и влечетъ тя. Отъ сего случая научись, что подобно и намъ, странствующимъ въ мірѣ семъ, яко на чужой странѣ, должно мысль свою обращать къ небесному отечеству, идѣже Отецъ нашъ, Которому молимся: \textit{Отче нашъ, иже еси на небесѣхъ}; идѣже домъ нашъ, идѣже безопасный покой нашъ, идѣже \textit{предтеча о насъ вниде Іисусъ}\footnote{Евр.~6,~20.}, идѣже наслѣдіе наше, отъ преблагаго и милостиваго нашего Ходатая Хріста намъ уготованное; идѣже велія вечеря и бракъ Агнчій уготовася; идѣже безчисленное безплотныхъ силъ множество предстоитъ престолу величествія; идѣже лица святыхъ, отъ начала міра пожившихъ, по странствованіи, трудѣхъ, подвизѣхъ и скорбехъ упокоеваются, вѣнчаются и ждутъ насъ съ желаніемъ, дондеже и мы внидемъ въ покой оный.

\paragraph*{XVII.} Видишь, что купецъ товаръ свой везетъ, или предпосылаетъ изъ чужой стороны въ отечество свое и домъ свой. Сей случай научаетъ тебе не сокровиществовать въ мірѣ семъ, идѣже странствуеши, \textit{не скрывать сокровища своего на земли, идѣже червь и тля тлитъ, и идѣже татіе подкопываютъ и крадутъ}: но сокровище свое предпосылать въ отечество \textit{небесное, идѣже ни червь, ни тля тлитъ, и идѣже татіе не подкопываютъ, ни крадутъ}\footnote{Матѳ.~6,~19 и 20.}. И тако, \textit{идѣже будетъ сокровище твое, ту будетъ и сердце твое}\footnote{21.}. Куды будешь сокровище свое предпосылать, туды и духъ твой, мысль твоя, желаніе твое, тщаніе и попеченіе твое всегда клониться и стремиться будетъ. Сокровище же хрістіанское, которое въ отечество оное предпосылается, суть добродѣтели хрістіанскія: вѣра, любовь, терпѣніе, кротость, цѣломудріе, милосердіе, и проч.

\paragraph*{XVIII.} Когда находишься на чужой странѣ, или на дорогѣ, опасаешися всякихъ навѣтовъ отъ злыхъ людей. Сей случай такое подаетъ разсужденіе, что тако, или много болѣе, должно намъ, по пути міра сего идущимъ, опасаться козней діавольскихъ, которые, какъ разбойники на путешествующихъ, а паче обремененныхъ духовными товарами, нападаютъ, окружаютъ и обнажаютъ. Къ сей осторожности и тщанію увѣщаваетъ насъ Петръ, святый апостолъ Хрістовъ: \textit{трезвитеся, бодрствуйте; зане супостатъ вашъ діаволъ, яко левъ рыкая, ходитъ, искій кого поглотити. Емуже противитеся тверди вѣрою}\footnote{1~Петр.~5,~8,~9 и проч.}.

\paragraph*{XIX.} Видишь птицу сѣдящую, и туды и сюды осматривающуюся и опасающуюся, чтобъ или не уловилъ ея кто, или не устрѣлилъ. Сей примѣръ паки научаетъ тя подобнымъ образомъ осматриваться въ мірѣ семъ сѣтей діавольскихъ и стрѣлъ его. Стрѣлы же сего врага нашего суть: гордость, зависть, сребролюбіе, лихоиманіе, злоба, нечистота, и проч. Сіи стрѣлы мещетъ онъ на насъ, и уязвляетъ неосторожныхъ.

\paragraph*{XX.} Видишь корабль, на морѣ волнующійся и бѣдствующій. Отъ сего разумѣй, что тако житіе наше на морѣ міра сего бѣдствуетъ, и многоразличными напастьми, какъ волнами, обуревается. А симъ научаешися часто и усердно молитися, и воздыхать къ премудрому Управителю Хрісту, чтобъ не далъ твоему кораблецу въ волнахъ бѣдствій потопиться, но наставилъ бы въ тихое и безопасное вѣчнаго живота пристанище достигнуть.

\paragraph*{XXI.} Видишь, что свѣща горитъ, и далѣе сгараетъ. Сей примѣръ подаетъ тебѣ знать, что житіе наше, пока живемъ, какъ свѣща горитъ, а потомъ и угаснетъ, "--- и помнить кончину его, которая неизвѣстна, въ кій день и часъ будетъ; а потому всегда ожидать ея должно.

\paragraph*{XXII.} Видишь, что иной изъ славнаго дѣлается безчестнымъ, иной изъ богатаго нищимъ, иной изъ здороваго немощнымъ, иной изъ свободнаго невольникомъ, иной изъ господина рабомъ. Сіи перемѣны ясно показуютъ намъ, что житіе наше на земли есть непостоянное, суетное, бѣдное и плачевное, и знакъ есть, что мы отъ Создателя нашего созданы не къ сему, но къ иному лучшему и блаженнѣйшему житію. А тѣмъ самымъ научаемся, и аки убѣждаемся искать постояннаго онаго и блаженнаго житія; и въ такомъ случаѣ не только не должно роптать, какъ слѣпая наша плоть хочетъ, но и весьма благодарить Богу, хотя и противно намъ кажется, что тако аки убѣждаетъ насъ благодать Его внити въ животъ вѣчный.

\paragraph*{XXIII.} Видишь человѣка огневицею или горячкою мучима; или вшедши въ баню, чувствуешь жаръ великій; или видишь пещь, огнемъ распаленную. Отъ сего огня прейди умомъ до огня геенскаго, въ которомъ не кающіися грѣшники безъ конца будутъ страдать. Сіе разсужденіе научаетъ тя покаяніемъ и слезами умилостивлять благость Божію, чтобъ отъ той бѣды избавиться.

\paragraph*{XXIV.} Видишь человѣка въ темницѣ заключенна, узами окованна, всего блѣдна, изсохша. Отъ сей темницы прейди умомъ такожде къ вѣчной оной темницѣ адской, въ которой осужденники за беззаконія заключатся и свѣта во вѣки не увидятъ. Чего ради кайся за грѣхи своя, чтобъ и тебѣ не пріити на оное мученія мѣсто.

\paragraph*{XXV.} Видишь, или слышишь, что отецъ сына преслушнаго и неисправнаго отрекается и наслѣдія лишаетъ. Отъ сего случая научись, что Хрістосъ хрістіанъ неисправныхъ и нераскаянныхъ отречется, и наслѣдія вѣчнаго лишитъ. Скажетъ имъ, хотя имя Его и нарицали: \textit{не вѣмъ васъ, откуду есте; отступите отъ Мене вси дѣлателіе неправды}\footnote{Лук.~13,~25 и 27.}. Сіе увѣщаваетъ тебе, чтобы ты не токмо имя хрістіанское имѣлъ, но и житіе, согласное имени и исповѣданію, тщался имѣть.

\paragraph*{XXVI.} Видишь чертоги царскіе украшенные, и славу ихъ. Тутъ да восхититъ вѣра сердце твое къ вѣчнымъ чертогамъ небеснымъ, праведнымъ уготованнымъ, и къ обителямъ дому небеснаго Отца и славѣ Хріста, небеснаго и вѣчнаго Царя, которою вѣрные будутъ безъ сытости насыщаться въ безконечные вѣки; купно и очи твои душевныя возведи къ Сыну Божію, чтобы тебе части избранныхъ Своихъ участникомъ сотворилъ.

\paragraph*{XXVII.} Слышишь согласную и сладкую музыку или пѣніе. Да восхититъ вѣра духъ твой въ небесная селенія, въ домъ Бога Іаковля; и тамо душевными ушесами внуши, како ангельскій соборъ и святыхъ лики согласное возсылаютъ пѣніе Тріѵпостасному Богу. Сей случай подаетъ тебѣ причину усердно молитися Ему, да сподобитъ тя здѣ и въ будущемъ вѣцѣ славити Его со всѣми избранными Его.

\paragraph*{XXVIII.} Видишь, или слышишь, что монархъ земный раба своего за вѣрную услугу награждаетъ изобильно. При семъ размысли, какъ"=то Хрістосъ, царь небесный и вѣчный, вѣрныхъ Своихъ рабовъ наградитъ и прославитъ въ царствіи Своемъ, по неложному обѣщанію: \textit{ихже око не видѣ, и ухо не слыша, и на сердце человѣку не взыдоша, яже уготова Богъ любящимъ Его}\footnote{1~Кор.~2,~9.}, "--- и паки: \textit{тогда праведницы просвѣтятся, яко солнце во царствіи Отца ихъ}\footnote{Матѳ.~13,~43.}. Тогда они подлинно увидятъ и вознепщуютъ, \textit{яко недостойны страсти нынѣшняго времени къ хотящей явитися славѣ}\footnote{Римл.~8,~18.}. Тогда они увидятъ, чего здѣ око не могло видѣть, и услышатъ, чего здѣ ухо не могло слышати, и почувствуютъ, что здѣ на сердце не могло взойти. Тогда \textit{возрадуется сердце ихъ, и радости ихъ никтоже возметъ отъ нихъ}\footnote{Іоан.~16,~22.}. Тогда утѣшатся нынѣ плачущіи; тогда насытятся нынѣ алчущіи и жаждущіи правды; тогда обогатятся нынѣ лишенніи; тогда воспріимутъ царствіе нынѣ изгнанніи правды ради; тогда воспріимутъ отъ руки Господни все, на что нынѣ вѣрою, яко \textit{зерцаломъ въ гаданіи}, взирали, и къ чему надеждою воскриляеми стремились. Тогда возрадуется всякъ о Господѣ, и въ радости духа воскликнетъ: \textit{облече мя въ ризу спасенія, и одеждою веселія одѣя мя, яко на жениха, возложи на мя вѣнецъ, и яко невѣсту, украси мя красотою}\footnote{Ис.~61,~10.}. Сіе разсужденіе учитъ тя славу міра сего презирать, и къ оной славѣ всегда стремиться; Царю небесному, Хрісту, вѣрно служить, да со избранными Его славу оную благодатію Его сподобишися наслѣдити.

\paragraph*{XXIX.} Видишь древо, листвіемъ зеленѣющее и плодами исполненное, котораго взоръ всякаго увеселяетъ. Отъ сего случая обрати умъ твой ко внутреннему состоянію благолюбиваго человѣка, котораго душа тако или паче болѣе украшается и плодами Духа Святаго исполнена есть. Таковый, \textit{яко финиксъ, цвѣтетъ, и, яко кедръ, иже въ Ливанѣ, умножается}\footnote{Пс.~91,~13.}. Сей случай научаетъ внутреннему, а не внѣшнему прилѣжать украшенію.

\paragraph*{XXX.} Видишь паки древо изсохшее, и ни къ чему, какъ только къ сожженію годное. Разумѣй, что тако имѣется грѣшнаго душа, которая живыя вѣры и плодовъ ея не имѣетъ, которыя кончина "--- посѣченіе и вѣчнаго огня адъ. \textit{Всяко бо древо, еже не творитъ плода добра посѣкаемо бываетъ и въ огнь вметаемо}\footnote{Матѳ.~3,~10.}. Сей случай подаетъ тебѣ причину обратить очи твои на душу твою, не находится ли и она въ такомъ бѣдномъ состояніи, и молиться Хрісту Богу, Который и мертвыхъ оживляетъ, чтобъ оживотворилъ ее, и росою благодати Своея напоилъ къ прозябенію плодовъ духовныхъ.

\paragraph*{XXXI.} Видишь вѣтвь, отсѣченную отъ древа и изсохшую, которая ни на что не потребна, какъ только на сожженіе. Разумѣй отъ сего, что тако всякъ, кто отъ Хріста, \textit{Иже есть лоза истинная}, нерадѣніемъ и злымъ житіемъ отсѣчется, не только никакого плода не приноситъ, но, якоже розга, изшетъ, и собираютъ ю, и во огнь влагаютъ, и сгараетъ, какъ Самъ Онъ о томъ поучаетъ\footnote{Іоан.~15,~1,~5 и 6.}. Сіе увѣщаваетъ тебе крѣпко держаться Его вѣрою и любовію, удаляться отъ всякаго грѣха, и Хрісту единому прилѣпляться, смиреніемъ и терпѣніемъ Ему послѣдовать.

\paragraph*{XXXII.} Видишь, что земля во время бездождія, изсохши, не даетъ плода. Разумѣй, что тако душа, не напаяемая росою Божія слова и благодати, аки изсыхаетъ и безплодна бываетъ. Тѣло наше изнемогаетъ отъ алчбы и жажды, а далѣе, когда не подкрѣпится пищею, умираетъ: тако душа, когда не питается и не напаяется словомъ Божіимъ, изнемогаетъ, а потомъ и умираетъ. Сіе увѣщаваетъ тебе, какъ тѣлу на всякій день ищешь пищи и питія, такъ душѣ отъ слова Божія непрестанно искать пищи, питія и утѣшенія, чтобъ, изнемогши отъ глада и жажды, не умерла на вѣки.

\paragraph*{XXXIII.} Видишь, или слышишь ближняго твоего согрѣшающа, или согрѣшивша. Сей случай служитъ тебѣ не къ осужденію брата твоего, "--- якоже нѣкіихъ есть злый обычай, "--- но къ познанію немощи твоея; "--- не къ посмѣянію онаго, но къ сожалѣнію и исправленію твоему. Отъ него обрати очи твои на тебе, не былъ ли ты самъ въ такомъ же или подобномъ грѣхѣ, или нынѣ не находишься ли? Когда нѣтъ того, то можешь еще горше согрѣшити. Общая бо немощь и окаянство наше внутрь насъ: врази наши суть страсти наши; плоть наша порабощаетъ насъ; и сатана, врагъ нашъ, непрестанно ищетъ поглотити насъ. Вси всякому бѣдствію и паденію подлежимъ и падаемъ; и падемъ, когда благодать Божія не поддержитъ насъ. И тако отъ такого случая самъ себе разсмотри, и отъ братняго падежа тщись осторожнѣе съ помощію Божіею поступать.

\paragraph*{XXXIV.} Видишь, что сѣмя падшее на землю умираетъ, и, прозябши, выходитъ изъ земли. Сей примѣръ ясно показуетъ тебѣ, что тако тѣлеса наша, хотя умираютъ и погребаются, однакожъ Божіею силою во свое время паки оживутъ и облекутся одеждою безсмертія. О чемъ апостолъ тако проповѣдуетъ: \textit{сѣется въ тлѣніе, востаетъ въ нетлѣніи; сѣется не въ честь, востаетъ въ славѣ; сѣется въ немощи, востаетъ въ силѣ; сѣется тѣло душевное, востаетъ тѣло духовное}\footnote{1~Кор.~15,~42,~43 и 44.}. И тако стой, утверждайся въ вѣрѣ, чая воскресенія мертвыхъ и жизни будущаго вѣка.

\paragraph*{ХХХV.} Видишь земледѣльца, сѣюща сѣмена; но не всякое приноситъ плодъ. Ибо иное падаетъ при пути, какъ притча научаетъ, и птицы небесныя поядаютъ, иное на камени, иное въ терніе. Такъ и спасительное сѣмя, Божіе слово, равно сѣется на нивахъ сердецъ человѣческихъ, но токмо въ добрыхъ сердцахъ плодъ прорастаетъ\footnote{Матѳ.~13,~4--23.}. Сей примѣръ научаетъ тя усердно молитися Богу, чтобъ далъ тебѣ сердце плотяное, духъ новъ и сердце новое: и тако сѣмя слова Божія, посѣянное на землѣ сердца твоего, не будетъ безплодно.

\paragraph*{XXXVI.} Видишь паки земледѣльца вѣюща сѣмена своя, и сѣмена въ житницу собирающа, а плевелы вонъ изметающа. Тако будетъ и въ скончаніе вѣка сего, какъ пишется: тогда вѣрніи и благочестивіи люди, какъ пшеница въ житницу, въ небесное царствіе внидутъ, а злые, какъ плевелы, извергнутся \textit{и ввергнутся въ пещь огненную; ту будетъ плачь и скрежетъ зубомъ}\footnote{Матѳ.~13,~40--42.}. Сей примѣръ поощряетъ и тя весьма пещися о спасеніи своемъ, и не плевеломъ непотребнымъ, но пшеницею быть.

\paragraph*{XXXVII.} Видишь паки земледѣльца, съ радостію собирающа плоды трудовъ своихъ. Помысли, съ какою радостію наслаждатися будутъ плодами трудовъ своихъ, которые нынѣ въ подвигѣ вѣры и терпѣнія труждаются. \textit{Сѣющіи слезами, радостію пожнутъ. Ходящіи хождаху и плакахуся, метающе сѣмена своя: грядуще пріидутъ радостію, вземлюще рукояти своя}\footnote{Пс.~125,~5 и 6.}. На сіе взирая, не ослабѣвай въ подвигѣ благочестія.

\paragraph*{XXXVIII.} Видишь краснаго человѣка, ангелоподобнаго. Отъ сего возведи умъ твой ко внутреннему человѣку, то"=есть, душѣ, и помысли, коль краснѣйшая душа "--- образъ и подобіе Божіе, которою Богъ въ такомъ и такъ красномъ обиталищѣ поселилъ. Отъ сего научись познавать и почитать души благородіе, красоту и великолѣпіе, и сію, яко безсмертную, добрыми дѣлами, вѣрою, любовію, терпѣніемъ, цѣломудріемъ украшать, нежели тѣло, прахъ и пепелъ.

\paragraph*{ХХXIХ.} Видишь человѣка въ рубища одѣяна, полунага, всего пороками замарана. Отъ сего обрати умъ твой къ душѣ грѣшной, которая благодати Божіей, какъ прекрасной одежды, обнажилася, и вмѣсто того вся пороками грѣховными, какъ рубищами, обложилася. Гнусенъ человѣкъ замаранный предъ очесами человѣческими; но душа сквернами грѣховными замаранная, гнуснѣйшая есть предъ очесами Божіими. Сей случай увѣщаваетъ тебе уклоняться отъ всякаго грѣха, которымъ душа порочится и зрѣнія Божія недостойна является.

\paragraph*{ХL.} Видишь паки человѣка въ ранахъ, въ гноѣ смердяща. Обрати очи твои къ душѣ грѣшной, которая далеко большій смрадъ издаетъ, нежели смердящее тѣло. Что бо на тѣлѣ гной и раны, тое на душѣ грѣхи и беззаконія. Слыши, что пророкъ глаголетъ: \textit{беззаконія моя превзыдоша главу мою, и яко бремя тяжкое отяготѣша на мнѣ; возсмердѣша и согниша раны моя отъ лица безумія моего}\footnote{Пс.~37,~5 и 6.}. Смрадъ тѣлесный несносенъ намъ: смрадъ душевный несносенъ Богу. Отъ смрада тѣлеснаго отвращаемся мы и убѣгаемъ: отъ смрада душевнаго отвращается и отходитъ Духъ Божій. Сіе паки увѣщаваетъ тя бѣгать грѣха. Аще же душу проказою грѣховною зараженную чувствуешь, хотя издалеча ставши, вознеси гласъ твой, какъ нѣкогда прокаженные учинили, ко Іисусу, Сыну Божію, всякихъ болѣзней душевныхъ и тѣлесныхъ Прогонителю, и вознеси сердцемъ, не токмо устами, и молись: \textit{Іисусе наставниче, помилуй мя}\footnote{Лук.~17,~13.}! Тако Очистивый прокаженныя оныя, очиститъ и твою душевную проказу. Милосердъ бо есть; слушаетъ гласы бѣдныхъ насъ. На сіе бо въ міръ пришелъ, да избавитъ насъ отъ грѣховъ нашихъ. Не можетъ же быть такъ великая проказа грѣховная, которой бы Врачь нашъ премудрый и всесильный не хотѣлъ, или не моглъ милостиво очистить.

\paragraph*{ХLІ.} Видишь слѣпаго человѣка, который не видитъ пути, не знаетъ, куды идетъ, не видитъ ничего предъ собою, не видитъ рва, въ который имѣетъ пасти. Отъ сего случая обрати разсужденіе твое къ слѣпотѣ душевной, которою пораженный грѣшникъ такожде не видитъ добра и зла, не знаетъ куды идетъ, не видитъ своей погибели, въ которую имѣетъ пасти. Бѣдственна тѣлесная слѣпота, но душевная бѣдственнѣйшая! Лучше тѣлеснаго, нежели душевнаго зрѣнія не имѣть. Сей случай и разсужденіе увѣщаваетъ насъ молитися Хрісту, дающему слѣпымъ прозрѣніе. \textit{Призри, услыши мя, Господи Боже мой: просвѣти очи мои} душевныя, \textit{да не когда усну въ смерть}\footnote{Пс.~12,~4.}.

\paragraph*{ХLІІ.} Видишь глухаго человѣка, который, что ни говорится или запрещается, не внимаетъ и не дѣлаетъ тако: понеже не слышитъ рѣчей, и не разумѣетъ приказанія или запрещенія. Разумѣй, что тако имѣется грѣшникъ, которому сатана заткнулъ уши душевныя, чтобъ не слышалъ гласа словесъ Божіихъ. Таковый имѣетъ уши, но не имѣетъ ушесъ \textit{слышати}. Слышитъ тѣлесными, но глухъ душевными ушами. Гласъ слова ударяетъ въ слухъ тѣлесный, но душевнаго не касается; и такъ бываетъ, какъ глухій: почему и бываетъ недѣйствительно слово. Что сказуетъ Ему Божіе слово, тому не внимаетъ; что приказуетъ, или запрещаетъ, того не слушаетъ, какъ глухій. Такому слово, увѣщаніе, совѣтъ, какъ глухому музыка, не пользуетъ. Почему и Хрістосъ во Евангеліи глаголетъ: \textit{имѣяй уши слышати, да слышитъ}\footnote{Лук.~8,~15.}. Вси уши имѣютъ, но не вси имѣютъ \textit{уши слышати}. Откуду бываетъ, что вси, собравшіися во храмѣ, слышатъ слово Божіе, но не вси пользуются: понеже не вси имѣютъ уши \textit{слышати}. Плачевна тѣлесная глухота, но большаго плача достойна душевная. Понеже душа, не слышащая и не слушающая Божія гласа, подлежитъ всякому заблужденію и вѣчной пагубѣ. Сіе увѣщаваетъ тебе молитися Хрісту, чтобъ отверзлъ тебѣ слухъ душевный, какъ нѣкогда глухому и гугнивому отверзлъ слухъ и языкъ, и подалъ уши \textit{слышати}\footnote{Марк.~7,~34 и 35.}.

\paragraph*{ХLІІІ.} Слышишь, что попавшись въ плѣнъ непріятелю злому такой"=то человѣкъ чего не страждетъ въ бѣдствіи томъ? Облагается желѣзомъ, понуждается по волѣ плѣнившаго жить, дѣлать, что мучителю угодно; терпитъ насмѣяніе, поруганіе, раны; словомъ, житіе горькое и не лучшее самой смерти проводитъ. Отъ сего плѣненія чувственнаго обрати умъ твой къ плѣненію невидимому. Коль горькое плѣненіе страждетъ грѣшная душа, которую плѣнилъ князь міра сего, діаволъ! Связалъ ее ужемъ похотей міра сего, совершаетъ надъ нею злую волю свою, насмѣвается, ругается ей, по ранахъ раны налагаетъ ей, отъ бѣды въ бѣду, отъ беззаконія въ беззаконіе влечетъ ее. Горько плѣненіе тѣлесное, но горчайшее душевное! Тѣломъ бо плѣненный хотя и работаетъ мучителю, но духомъ свободенъ есть, "--- духа бо никто плѣнить и связать не можетъ; къ томужъ смертію отъ того бѣдствія свободится: душею же плѣненный, хотя тѣломъ и свободенъ можетъ быть, бѣднѣйшій всякаго невольника; ибо во вѣки связанъ будетъ, когда благодатію Хрістовою не избавится. И тѣмъ сіе бѣднѣйшее плѣненіе, что грѣшникъ не видитъ его; думаетъ о себѣ, что онъ свободенъ, что никому не работаетъ, но самою вещію бѣднѣйшій того, который окованное имѣетъ тѣло. Что сіе истинно слово есть, слыши, что Господь ко Іудеомъ глаголетъ. Іудеи глаголютъ ему: \textit{сѣмя Авраамле есмы, и никомуже работахомъ николиже: како Ты глаголеши, яко свободни будете}? Отвѣща имъ Іисусъ: \textit{аминь, аминь глаголю вамъ, яко всякъ творяй грѣхъ, рабъ есть грѣха}\footnote{Іоан.~8,~33 и 34.}. Сіе разсужденіе подаетъ тебѣ, возлюбленне, причину осмотрѣться: не работаешь ли ты мерзкому сему тирану въ похотехъ его, си есть, въ сребролюбіи, лихоиманіи, сластолюбіи, нечистотѣ, гордости, злобѣ и прочіихъ симъ подобныхъ. Аще усмотришь чѣмъ плѣненное сердце твое, воздыхай ко Хрісту Сыну Божію, Который единъ отъ сего плѣна избавляетъ и свободу даруетъ; молись Ему со Псаломникомъ: \textit{возврати, Господи, плѣненіе мое, яко потоки югомъ}\footnote{Пс.~125,~4.}. "--- \textit{Аще Сынъ Божій свободитъ насъ, воистинну свободни будемъ}\footnote{Іоан.~8,~36.}.

\paragraph*{ХLІV.} Видишь мертваго погребаемаго, или самъ погребаешь отца, или матерь, или брата, или друга. Сей случай подаетъ тебѣ причину плакать не о томъ, о чемъ многіе безполезно плачутъ; но прослезись, разсуждая о причинѣ смерти, то"=есть, о грѣхѣ, который въ такое и такъ бѣдное состояніе привелъ человѣка, который по образу Божію созданъ и безсмертіемъ почтенъ былъ. Сія есть праведная причина плача, что мы грѣхомъ прогнѣвали Бога, и такъ смерти подпали. А видя ближняго мертва, помяни, что и самъ ты \textit{земля еси, и въ землю пойдеши}\footnote{Быт.~3,~19.}. Но потомъ воздай Богу благодареніе, что Онъ въ Сынѣ Своемъ, Господѣ нашемъ Іисусѣ Хрістѣ, подалъ намъ благую воскресенія и жизни будущаго вѣка надежду.

\paragraph*{ХLV.} Видишь, что древо во время зимы листвіе свое отметаетъ; но во время весны, когда солнце согрѣваетъ, паки листвіе отъ себе испущаетъ. Тако многіе люди дѣлаютъ: когда ихъ зима и хладъ бѣдъ и неблагополучія поражаетъ, тогда прихоти свои отбрасываютъ, и мнятся покаяніе творити и Богу угождати; а когда теплота благополучія паки согрѣетъ ихъ, тогда прихоти свои, аки древо листвіе, по прежнему отъ себе оказываютъ, паки зеленѣютъ, украшаются и мірскими сластьми утѣшаются. Таковые суть, которые хладомъ болѣзни поражены, каются, отъ грѣховъ престаютъ, и обѣщаются благочестно жити и Богу работати: но когда въ здравіе пріидутъ, паки на прежнія прихоти и дѣла обращаются. Знай, что таковые люди не суть истинные хрістіане, но лицемѣры. Понеже \textit{истинные хрістіане} въ благополучіи и злополучіи, въ болѣзни и здравіи, въ нищетѣ и богатствѣ, въ презрѣніи и чести, въ славѣ и уничиженіи, въ неволѣ и свободѣ постоянны суть; всегда и вездѣ находятся въ покаяніи, плоды покаянія творятъ, Бога любятъ и Богу работаютъ. Сіе разсужденіе увѣщаваетъ тебе не подражать вышепомянутымъ людямъ, но быть постояннымъ всегда, во время счастія и несчастія, прихотей своихъ отрицаться и Богу угождать; а когда самъ таковъ же находишися, то увѣщаваетъ исправитися, да не съ лицемѣрами услышиши отъ Хріста: \textit{не вѣмъ васъ, откуду есте}\footnote{Лук.~13,~25.}.

\paragraph*{ХLVІ.} Видишь мертваго, который на плачь сродниковъ не отзывается, и хвалимъ не утѣшается, и поношаемъ не оскорбляется. Сей случай научаетъ насъ, что кто хощетъ \textit{истиннымъ} быть \textit{хрістіаниномъ}, тому должно ни похвалою міра сего не превозноситься, ни безчестіемъ и руганіемъ не раздражаться, и похотей плоти своея, аки искреннихъ своихъ сродниковъ, не слушать, но быть какъ мертвому. И сіе"=то есть \textit{умрети міру, умрети себѣ, и себе отрещися}. Сего ради насъ требуетъ Господь нашъ, когда хощемъ быть Его рабами и Его учениками, и Ему послѣдовать: \textit{аще кто хощетъ по Мнѣ ити, да отвержется себе, и возметъ крестъ свой, и послѣдуетъ Мнѣ}\footnote{Матѳ.~16,~24; Лук.~9,~23.}. А тѣмъ научаешися не смотрѣть на тѣхъ хрістіанъ, которые хотятъ и Хрісту угождать и міру и своимъ прихотямъ; и чести и славы и богатства въ мірѣ семъ искать, и Хрісту работать; отъ поруганнаго, обезчещеннаго и умаленнаго Хріста удаляются, срамляются Его, но съ воскресшимъ и прославленнымъ хотятъ быти, "--- что едино есть отъ невозможныхъ, якоже глаголетъ Самъ: \textit{иже аще постыдится Мене и Моихъ словесъ въ родѣ семъ прелюбодѣйнѣмъ и грѣшнѣмъ, и Сынъ человѣческій постыдится его, егда пріидетъ во славѣ Отца Своего со ангелы святыми}\footnote{Марк.~8,~38.}. Но паче надлежитъ внимать святому Божію слову, которое учитъ насъ у Хріста учитися смиренію, любви, терпѣнію, кротости и прочему добру; но прежде своего злонравія отрещися, и тако благимъ Хрістовымъ нравамъ послѣдовати.

\paragraph*{XLVII.} Видишь, что потерявшіи путь надлежащій и блудящіи ищутъ вождя, который бы ихъ наставилъ на путь и довелъ до намѣреннаго и желаемаго имъ мѣста, какъ то часто случается живущимъ въ мірѣ семъ. Сей случай научаетъ тя, что мы тако вси заблудили, когда совѣтомъ лукаваго духа преслушали Бога, и блудимъ по пустынѣ міра сего, удалившеся отъ отечества нашего "--- неба, къ которому созданы отъ Бога нашего, якоже пророкъ глаголетъ: \textit{вси, яко овцы, заблудихомъ; человѣкъ отъ пути своего заблуди}\footnote{Ис.~53,~6.}. И Давидъ святый глаголетъ и молится: \textit{заблудихъ яко овча погибшее: взыщи раба Твоего}\footnote{Пс.~118,~176.}. А понеже сами, яко заблуждшіе, къ отечеству оному никакъ не можемъ дойти: того ради должно намъ искать искуснаго вождя въ такъ важномъ и нужномъ дѣлѣ, которое \textit{едино} есть намъ \textit{на потребу}\footnote{Лук.~10,~42.}. Таковый искусный и премудрый вождь указуется намъ во Евангеліи Іисусъ Сынъ Божій, о Которомъ Отецъ съ небесе глаголетъ намъ: \textit{Сей есть Сынъ мой возлюбленный, о Немже благоволихъ: Того послушайте}\footnote{Матѳ.~17,~5.}. То есть: Я вамъ послалъ Его Учителя, Наставника и Вождя. Когда хощете ко Мнѣ пріити, и потерянное вами небесное царствіе получити: Того послушайте, чему Онъ учитъ, слушайте Его. Но и Самъ о Себѣ глаголетъ Господь: \textit{Азъ есмь путь и истина, и животъ: никтоже пріидетъ къ Отцу, токмо Мною}\footnote{Іоан.~14,~6.}. Аще убо не хощемъ, любезный хрістіанине, конечно заблудить и быть вѣчными діавола плѣнниками, но паче къ Богу пріити и вѣчный животъ получити, къ чему позваны мы и банею крещенія отрождены: то неотмѣнно должны мы Ему себе ввѣрить, вѣрою и любовію держаться Его, слушать святаго и истиннаго ученія Его, послѣдовать стопамъ Его, подражать чистому непорочнаго житія Его примѣру. Смиреніе Его да низлагаетъ нашу гордость; терпѣніе Его да укрощаетъ гнѣвъ нашъ; кротость Его да изгоняетъ злобу нашу и желаніе мщенія; нищета Его да отвращаетъ насъ отъ сребролюбія, лихоиманія и хищенія; любовь Его да истребляетъ зависть и ненависть нашу; святыня Его да научитъ насъ любить чистоту души и тѣла. Все святое и божественное Его житіе да будетъ намъ во образъ и исправленіе злыхъ нашихъ нравовъ, съ которыми мы отъ ветхаго Адама родилися. Тако будетъ Онъ намъ путь, истина и животъ вѣчный! Тако Ему послѣдуя, не заблудимъ отъ пути праваго, но пріидемъ къ желаемому отечеству и дому небеснаго Отца, въ которомъ \textit{обители многи суть}\footnote{2.}. Сей путь есть смиренный и низкій, возлюбленный хрістіанине, ко къ высокому небу идущихъ ведетъ. Симъ путемъ иди, когда того отечества достигнуть хощешь: и не заблудишь въ пропасть адову.

\paragraph*{ХLVIII.} Случается тебѣ видѣть монарха, входящаго во градъ какій. Примѣчай, съ какимъ почтеніемъ и радостію срѣтаютъ и пріемлютъ его граждане; очищаютъ путь и улицу, которою имѣетъ быть шествіе, и себе различно украшаютъ, и шествующему преклоняютъ главу и колѣна, и прочая. Отъ сего видимаго позорища обрати умъ твой вѣрою къ невидимому, и отъ тѣлеснаго къ духовному. Хрістосъ Сынъ Божій, Царь небесный, на землѣ явился, егда отъ пресвятой Дѣвы родился ради насъ, и въ міръ сей къ намъ, какъ въ какій градъ, пришелъ. Мы града сего обширнаго и нерукотвореннаго граждане есмы; къ намъ Онъ, подлымъ рабамъ, пришелъ, и \textit{посѣтилъ есть насъ Востокъ свыше, просвѣтити во тмѣ и сѣни смертнѣй сѣдящія, направити ноги наша на путь миренъ}, якоже поетъ святый Захарія пророкъ\footnote{Лук.~1,~78 и 79.}. И пришелъ въ смиреніи и кротости, во образѣ раба: \textit{зракъ раба пріимъ, въ подобіи человѣчестѣмъ бывъ}\footnote{Филип.~2,~7.}. Какъ намъ должно Его срѣтать и пріимать? съ какимъ почтеніемъ и радостію? Стыдно намъ съ пышностію и гордостію выходить въ срѣтеніе Царю нашему, когда онъ со смиреніемъ пришелъ къ намъ! Не нравится Ему таковое срѣтеніе. Ревекка, когда ѣхала на велблюдѣ къ обручнику своему Исааку, и увидѣла его предъ собою, соскочила съ велблюда: \textit{и воззрѣвши Ревекка очима своима, видѣ Исаака, и изскочи съ велблюда}, глаголетъ Писаніе\footnote{Быт.~24,~64.}. Тако намъ должно съ гордости, какъ высокаго велблюда, соскочить, и пѣшими итить, и срѣтать Жениха нашего съ великимъ смиреніемъ, и колѣна съ сердцами преклонять. Къ сему пророкъ призываетъ насъ: \textit{пріидите, поклонимся и припадемъ Ему}\footnote{Пс.~94,~6.}!

\paragraph*{XLIX.} Видишь, или слышишь, что жена честная и вѣрная мужу своему, никого не любитъ, кромѣ мужа своего, и никому не тщится угодить, только мужу своему. Отъ сего супружества плотскаго обрати умъ твой къ супружеству духовному, въ которомъ хрістіане вѣрою сопряглися небесному Жениху Хрісту. О чемъ Павелъ святый тако глаголетъ къ вѣрнымъ Коринѳянамъ: \textit{обручихъ васъ единому мужу, дѣву чисту представити Хрістовѣ}\footnote{2~Кор.~11,~2.}. Разсудижъ, какъ должно хрістіанамъ любовь хранить къ небесному оному Жениху! Какъ должно удаляться отъ нечистой міра сего любви, единому Хрісту прилѣпляться, понравляться, угождать, любезными быть и волю Его святую исполнять, и никого, кромѣ Его и ради Его, не любить, когда вѣрность къ нему хотятъ сохранить! Любочестіе, славолюбіе, сладострастіе, сребролюбіе, и прочая міра сего похоть есть какъ любодѣйца, къ которой душею прилѣпляется хрістіанинъ, когда честь, славу, сребро, злато, сласть и прочую похоть любитъ, и тако оставляетъ Хріста, безсмертнаго Жениха, и любви міра, какъ нечистой блудницѣ, душу свою отдаетъ. О! колико, коль тяжко согрѣшаютъ хрістіане Хрісту, возлюбившему ихъ и предавшему Себе по нихъ, которые не сохраняютъ вѣрности сея къ Нему, которую, вступая въ хрістіанство, обѣщались до конца хранить; а тако и всѣхъ благъ духовныхъ, которыхъ въ крещеніи сподобились"=было, "--- самовольно, къ крайнему своему бѣдствію, лишаются. Не напрасно апостоли святіи отъ любви міра сего отводятъ насъ. \textit{Не любите міра, ни яже въ мірѣ}, глаголетъ Іоаннъ святый\footnote{1~Іоан.~2,~15.}. \textit{Яко любы міра сего}, глаголетъ Іаковъ святый, \textit{вражда Богу есть: иже бо восхощетъ другъ быти міру, врагъ Божій бываетъ}\footnote{Іак.~4.}. Такъ вредительна есть любовь міра сего, что любящій его врагомъ Божіимъ бываетъ, "--- что страшно и помыслить, хотя слѣпый человѣкъ того и не разсуждаетъ!

\paragraph*{L.} Видишь, или слышишь, что мужъ отрекается жены, когда не хранитъ вѣрности къ нему, но, оставивши его, съ другими беззаконно смѣшается и прелюбодѣйствуетъ. Знай точно, что тако и Хрістосъ, небесный и пречистый Женихъ, отрекается таковыя души, которая, прилѣпляется къ міру сему, похоти плотской, похоти очесъ и гордости житейской: понеже не хранитъ вѣрности къ Нему, которую при вступленіи въ хрістіанство обѣщалася до конца жизни своея хранить. Къ таковымъ Онъ возглаголетъ: \textit{не вѣмъ васъ, откуду есте}, егда пріидетъ во славѣ Своей\footnote{Лук.~13,~25.}. Разсужденіе сіе увѣщаваетъ тя, хрістіанине, осмотрѣться, не оставилъ ли и ты Хріста, прилѣпившися сердцемъ къ міру, то есть, богатству, чести, славѣ и сласти міра сего. Ежели примѣтишь сіе, не медля обратися ко Хрісту съ покаяніемъ и слезами, пока время еще не ушло; ибо пріемлетъ Онъ обращающихся: да не и ты въ числѣ тѣхъ будеши, къ которымъ возглаголетъ: \textit{не вѣмъ васъ}, хотя съ ними и глаголеши къ Нему: \textit{Господи, Господи! Не всякъ бо}, рече Онъ, \textit{глаголяй Ми: Господи, Господи, внидетъ въ царствіе небесное, но творяй волю Отца Моего, Иже есть на небесѣхъ}\footnote{Матѳ.~7,~21.}.

\paragraph*{LI.} Видиши, что всему свое время опредѣлено, якоже глаголетъ Екклесіастъ: \textit{всѣмъ время, и время всякой вещи подъ небесемъ}\footnote{Еккл.~3,~1.}. Птицы въ свое время гнѣзда сочиняютъ и плодятъ дѣтей; скоты и звѣри въ свое время сходятся, и раждаютъ; рыбы такожде въ свое время; древеса и травы въ свое время одѣваются листвіемъ, цвѣтутъ и прозябаютъ плоды; земледѣльцы въ свое время орутъ, сѣютъ и собираютъ плоды; словомъ, всякой вещи свое время опредѣлилъ Создатель Богъ нашъ. Симъ научаемся, что и намъ Богъ время свое опредѣлилъ къ исканію вѣчнаго спасенія. Время сіе есть настоящее время, \textit{дондеже днесь нарицается}, дондеже въ мірѣ семъ находимся, дондеже живемъ и путники есмы. Сіе время опредѣлено намъ къ исканію спасенія, которое Хрістосъ Сынъ Божій кровію Своею заслужилъ. Откуду сіе время уподобляется въ писаніи времени сѣянія: \textit{еже сѣетъ человѣкъ, тожде и пожнетъ. Яко сѣяй въ плоть свою, отъ плоти пожнетъ истлѣніе: а сѣяй въ духъ, отъ духа пожнетъ животъ вѣчный}\footnote{Гал.~6,~7 и 8.}. \textit{Сѣющіи слезами, радостію пожнутъ. Ходящіи хождаху и плакахуся, метающе сѣмена своя: грядуще же пріидутъ радостію, вземлюще рукояти своя}\footnote{Пс.~125,~5 и 6.}. Будетъ время, когда никто не получитъ того, хотя и со слезами и воздыханіемъ поищетъ. Какъ и земледѣльцы несмысленные, пропустивше время удобное, хотя и сѣютъ, но погубляютъ сѣмя, яко не въ тое время, когда должно, сѣютъ: такъ и несмысленные грѣшники будутъ нѣкогда искать спасенія, но не получатъ того, яко тогда будетъ время суда, а не покаянія. Нынѣ время сѣяти, искати, просити, толкати въ двери Божія милосердія, когда Богъ обѣщалъ услышати и помогати, и слушаетъ и помогаетъ. Глаголетъ бо: \textit{во время благопріятно послушахъ тебе, и въ день спасенія помогохъ ти. Се нынѣ время благопріятно, се нынѣ день спасенія}\footnote{2~Кор.~6,~2.}. Нынѣ \textit{всякъ просяй пріемлетъ и ищай обрѣтаетъ, и толкущему отверзется}, глаголетъ Хрістосъ\footnote{Матѳ.~7,~8.}. Тогда того не будетъ. "--- Я"=де при смерти могу покаяться, отвѣщаетъ иный? "--- Можешь съ такимъ покаяніемъ и во адъ пойти. О человѣче, тогда ли хочешь каятися, когда время престаетъ покаянія, и наступаетъ время суда и истязанія? Тогда ли хочешь къ Богу обращатися, когда Богъ тебе къ отвѣту зоветъ? Тогда ли хочешь искать, когда время уже уходитъ? Безспорно, что должно и при смерти каятися, наипаче и молитися, и Бога со усердіемъ призывати, яко тогда наипаче подвигъ великій душѣ бываетъ; но до смерти отлагать покаяніе есть прелесть ума, нерадѣніе о спасеніи и кознь діавола, который таковую мысль въ сердце человѣку влагаетъ, и научаетъ законъ Божій безстрашіемъ разорять, Бога прогнѣвлять и покаяніе день отъ дня отлагать, чтобы тако человѣка лестно погубить. Сего ради, когда хощешь блаженно умереть, таковъ нынѣ буди, каковъ быть при смерти желаешь. Разсуждай о мірскихъ вещахъ, чести, славѣ, богатствѣ и роскоши нынѣ такъ, какъ умирающіи разсуждаютъ, которые все тогда оставляютъ. Берегись грѣха, и о содѣянномъ жалѣй такъ нынѣ, какъ тогда берегутся и жалѣютъ. Хощешь тогда отпущеніе грѣховъ отъ Бога во имя Хрістово получить? "--- нынѣ о томъ старайся. Хощешь милость у Бога тогда получить? "--- нынѣ тоя ищи. Но кто милость у милостиваго Бога хощетъ получить, тотъ престаетъ Его прогнѣвлять и о прежнемъ безстрашіи жалѣетъ. Тотъ же престаетъ Его прогнѣвлять, кто престаетъ грѣшить, и плоды покаянія творитъ. Разсуждай сіе, грѣшная душа, и внимай, что Предтеча сказалъ: \textit{уже и сѣкира при корени древа лежитъ: всяко убо древо, еже не творитъ плода добра, посѣкаемо бываетъ, и во огнь вметаемо}\footnote{Матѳ.~3,~10.}. Видишь, куды грѣшники, не творящіи плодовъ покаянія, опредѣляются: посѣкаются, какъ древеса безплодная, сѣкирою суда Божія, и въ огнь вѣчный, какъ дрова, ввергаются.

\paragraph*{LII.} Когда воины міра сего набираются, тогда поемлемые въ воинство оставляютъ домы свои, женъ, дѣтей, сродниковъ своихъ и имѣнія своя. Видишь, хрістіанине, что дѣлаютъ люди, которые записываются въ службу земному царю. Знай подлинно, что такъ должно дѣлать и хрістіанамъ, которые записалися въ службу небесному Царю. И имъ повелѣвается тое учинить, что нѣкогда Аврааму, всѣхъ вѣрныхъ отцу повелѣлъ Богъ: \textit{изыди отъ земли твоея, отъ рода твоего, и отъ дому отца твоего}\footnote{Быт.~12,~1.}. Ветхій Адамъ, отъ котораго мы родилися и котораго внутрь носимъ, есть отецъ нашъ; самолюбіе есть благопріятный домъ плоти нашей; сродники и имѣнія наша суть страсти наши, похоть плотская, похоть очесъ и гордость житейская. Сія вся оставить должны хрістіане, которые при святомъ крещеніи записалися въ воинство Хрісту, небесному Царю: \textit{ветхаго Адама отрицаться}, по вся дни и \textit{совлекаться его}\footnote{Еф.~4,~22; Кол.~3,~9.}; \textit{распинать плоть со страстьми и похотьми}\footnote{Гал.~5,~24.}; самолюбіе, любочестіе, славолюбіе, любострастіе, и сребролюбіе, которыми плоть утѣшается, какъ сродники своя, оставить, и тако служить Хрісту Царю. Вси хрістіане ради того хрістіанами нарицаются, что во Хріста вѣруютъ, Хрісту въ работу и службу записалися и тако воины Хрістовы именуются. Прилично ли убо воинамъ свою волю исполнять, а не царя своего? Срамно хрістіанамъ въ роскошахъ и веселостяхъ міра сего дни свои провождать, когда Хрістосъ, Царь ихъ, крестъ носилъ и крестомъ противу враговъ подвизался. Стыдно воинамъ лежать, а паче взадъ возвращаться, когда самъ Царь до изліянія крове подвизается: срамно хрістіанамъ унывать, когда видятъ Царя своего Хріста такъ сильно подвизавшагося, что и крове Своея за нихъ не пощадѣлъ; много паче взадъ въ домы своя возвращаться, и оставлять брань! Бѣдственно воинамъ, записавшимся въ службу Государю своему, оставлять его и къ другому, противнику его, отходить, и сіе не ино что, какъ явная государю измѣна: бѣдственно и явная погибель есть хрістіанамъ, именующимся воинами Хріста Царя небеснаго, оставлять Его, и работать міру; именоваться хрістіанами, но не хотѣть слушать Хріста; рабами Его называться, но не работать Ему! И сіе"=то значитъ, что Петръ святый написалъ: \textit{аще отбѣгше сквернъ міра познаніемъ Господа и Спаса нашего Іисуса Хріста, сими же паки сплетшеся побѣждаеми бываютъ, быша имъ послѣдняя горша первыхъ. Лучше бо бѣ имъ не познати пути правды, нежели познавшимъ возвратитися вспять отъ преданныя имъ святыя заповѣди. Случися бо имъ истинная притча: песъ возвращся на свою блевотину, и свинія, омывшися, въ калъ тинный}\footnote{2~Петр.~2,~20--22.}. Подобно дѣлаютъ и хрістіане, которые, очистившися въ крещеніе отъ сквернъ грѣховныхъ и обѣщавшися работать вѣрою и правдою Хрісту, уклоняются въ развращеніе и работаютъ похоти плотской и похоти очесъ, гордости житейской, и тако хотятъ Хрістовыми быть и міра сего чадами, и Хрісту угождать и міру работать, и хрістіанами быть, и съ міромъ веселиться, "--- чему быть невозможно такъ какъ и впередъ смотрѣть и взадъ. Воины міра сего, вѣрные государю своему, какъ цѣло и нерушимо хранятъ присягу государю своему, какъ за честь его и цѣлость отечества подвизаются! Идутъ въ далекія страны, вдаются въ смертоносные случаи, не боятся оружія вражія, смертію имъ грозящаго, пріемлютъ раны и падаютъ уязвленные. Сего бо отъ нихъ долгъ присяги требуетъ. И сихъ едина временная надежда и присяга, человѣку учиненная, такъ сильно поощряетъ. Хрістіане, Царю небесному и Богу учинившіи присягу, не показуютъ того Ему; за честь Царя небеснаго, Который за нихъ до изліянія Своея крове и подъятія смерти, смерти же крестныя, подвизался, и обѣщалъ имъ не временную, но вѣчную на небеси славу за подвигъ дать, подвизаться не хотятъ, но, бросивше оружіе, уступаютъ мѣсто врагу своему, или бѣгутъ взадъ. Сего ради тѣ (воины міра сего) будутъ хрістіанамъ судіи въ день суда. Подвигъ же хрістіанскій, который бываетъ не противу плоти и крови, но противу духовъ злобы, состоитъ въ смиреніи, въ презрѣніи славы суетныя, терпѣніи и кротости, отрицаніи себе самаго, распинаніи плоти со страстьми и похотьми, и мужественномъ искушеній терпѣніи. Когда сіе показуютъ хрістіане, тогда не уступаютъ мѣста на брани врагу своему, діаволу. Тако подвизалися вси святіи, и получили вѣнецъ правды отъ Подвигоположника Іисуса Хріста, которымъ и мы послѣдовать должны, возлюбленный хрістіанине, когда съ ними хощемъ участіе имѣть во царствіи Хрістовомъ. Иначе они не узнаютъ насъ, что мы хрістіане; не признаетъ и Хрістосъ за Своихъ, яко не имѣемъ знаменія Его, то есть, терпѣнія крестнаго, подъ которымъ вѣрные Его раби подвизаются. О! не допусти Боже слышать насъ отъ Него страшнаго онаго гласа: \textit{не вѣмъ васъ}\footnote{Лук.~13,~27.}, который услышатъ глаголющіи Ему: \textit{Господи, Господи! но не творящіи, яже глаголетъ}\footnote{Матѳ.~7,~21; Лук.~6,~46.}.

\paragraph*{LIII.} Видишь, или слышишь, что рабъ, котораго господинъ у другаго купилъ себѣ, или инымъ какимъ образомъ присвоилъ, уже не знаетъ прежняго господина; но того единаго, который его купилъ, знаетъ, тому единому служитъ, работаетъ и того волю исполняетъ. Примѣчай, что между сынами вѣка сего бываетъ, и что человѣкъ человѣку показуетъ, по силѣ общихъ всѣмъ народамъ правъ. Сатана лукавый нами лукаво и неправедно завладѣлъ"=было, и плѣнилъ въ свою темную власть; но Хрістосъ Сынъ Божій искупилъ насъ отъ власти его и присвоилъ Себѣ; и искупилъ не сребромъ или златомъ, но честною Своею кровію, якоже глаголетъ апостолъ: \textit{не истлѣннымъ сребромъ или златомъ избавистеся отъ суетнаго вашего житія, отцы преданнаго, но честною кровію, яко Агнца непорочна и пречиста Хріста}\footnote{1~Петр.~1,~18 и 19.}; и на другомъ мѣстѣ глаголетъ: \textit{цѣною куплени есте}\footnote{1~Кор.~7,~23.}. Сего ради, хрістіанине, должны мы неотмѣнно работать Хрісту Сыну Божію, Который насъ безцѣнною крове Своея цѣною купилъ, якоже апостолъ глаголетъ: \textit{цѣною куплени есте}; должны волю Его святую исполнять, повиноваться Ему и всякое послушаніе Ему отъ сердца показывать, какъ вѣрные раби господамъ своимъ показуютъ; словомъ, должны Ему жить, а не себѣ, какъ учитъ апостолъ: \textit{Хрістосъ за всѣхъ умре, да живущіи не ктому себѣ живутъ, но умершему за нихъ и воскресшему}\footnote{2~Кор.~6,~15.}. Самый бо долгъ правды того требуетъ, да тому господину работаетъ рабъ, чій рабъ есть. Слѣдственно коль тяжко согрѣшаютъ противу Господа Іисуса Хріста тѣ хрістіане, которые называются рабами Хрістовыми, но Хрісту не работаютъ, повелѣній Его не слушаютъ, но по своей волѣ живутъ. Что"=де таковымъ Хрістіанамъ имѣетъ быть? "--- Тое, что Хрістосъ о неключимомъ рабѣ глаголетъ: \textit{неключимаго раба вверзите во тму кромѣшнюю; ту будетъ плачь и скрежетъ зубомъ}\footnote{Матѳ.~25,~30.}. Ибо и господинъ "--- человѣкъ раба своего непокориваго и непослушливаго правильно наказуетъ. Но, какъ рабъ очувствуется и покается, признаетъ свою винность и смирится предъ господиномъ своимъ, и поступать будетъ такъ, какъ рабу должно, то есть, вѣрно господину своему служить, "--- милости его сподобляется: такъ и Хрістосъ, человѣколюбивый Господь, въ милость Свою пріемлетъ хрістіанина, когда онъ, чистосердечно грѣхи свои признавъ, покается, и покаянія плоды, безъ которыхъ истинное покаяніе быть не можетъ, покажетъ.

\paragraph*{LIV.} Видишь, что заимодавецъ должника истязуетъ, посаждаетъ въ темницу, дабы долгъ отдалъ. Тако Богъ грѣшниковъ за долги грѣховные, которыми Ему обдолжилися, истяжетъ. И чимъ кто болѣе грѣшитъ и грѣхи ко грѣхамъ прилагаетъ, тѣмъ болѣе симъ долгомъ обременяется: откуду большему истязанію и суду Божію подлежитъ. \textit{По жестокости твоей и нераскаянному сердцу, собираеши себѣ гнѣвъ въ день гнѣва и откровенія праведнаго суда Божія}\footnote{Римл.~2,~5.}. Заимодавецъ заключаетъ въ темницу должника, да отдастъ долгъ: тако Богъ заключитъ въ темницу вѣчную должниковъ Своихъ, гдѣ правдѣ Его вѣчно будутъ платить мученіемъ, но никогда не заплатятъ. "--- Какъ же убо отъ сихъ тяжкихъ долговъ, которые должниковъ во адъ погружаютъ, свободиться грѣшникамъ? Отвѣщаетъ Господь: \textit{аще Сынъ вы свободитъ, воистинну свободни будете}\footnote{Іоан.~8,~36.}. Милосердый Богъ ради Сына Своего, Господа нашего Іисуса Хріста, Который за всѣ наши долги грѣховные кровію и смертію Своею заплатилъ, прощаетъ кающимся усердно и вѣрующимъ въ Него: и тако, свободившеся отъ долговъ, свобождаются и казни послѣдующія. Сего ради всякому, кто хощетъ свободиться отъ долговъ сихъ, которые въ вѣчную заключаютъ темницу должниковъ, должно покаяться и къ Ходатаю милостивому Іисусу Хрісту прибѣгнуть. А кающемуся должно отъ грѣховъ отстать неотмѣнно. Должно прежде самому оставить ихъ, чтобы и Богъ оставилъ ихъ. Како бо можетъ тотъ просить о оставленіи долговъ, которыхъ не престаетъ умножать и обременять? Не можетъ убо, возлюбленный хрістіанине, тамо быть истинное покаяніе, гдѣ грѣхи не оставляются; но есть притворное и ложное, каковымъ покаяніемъ, о коль многіи, наипаче нынѣшняго вѣка люди, прельщаются, которые хотятъ и каяться и не хотятъ отъ грѣховъ отстать. Сего ради таковые и отъ долговъ своихъ грѣховныхъ свободиться не могутъ, пока чистосердечно грѣховъ не оставятъ. Притомъ должны и сами оставить долги должникамъ нашимъ, когда хощемъ оставленіе долгамъ нашимъ получить отъ Бога; иначе и намъ не оставятся, якоже глаголетъ Господь: \textit{аще отпущаете человѣкомъ согрѣшенія ихъ, отпуститъ и вамъ Отецъ вашъ небесный: аще ли не отпущаете человѣкомъ согрѣшенія ихъ, ни Отецъ вашъ отпуститъ вамъ согрѣшеній вашихъ}\footnote{Матѳ.~6,~14 и 15.}.

\paragraph*{LV.} Видишь, что вода съ высокихъ горъ на низкія мѣста стекаетъ. Отъ сего научись, что Богъ Вышній съ высоты Своея на смиренныхъ призираетъ, и рѣки дарованій Своихъ на сердца ихъ изливаетъ, якоже читаемъ: \textit{Богъ гордымъ противится, смиреннымъ же даетъ благодать}\footnote{1~Петр.~5,~5.}. Сего ради должно намъ смирятися, да милости Его сподобимся. «Како, глаголетъ святый Златоустъ, кто обрящетъ благодать предъ Богомъ? како же инако, аще не смиренномудріемъ? \textit{Богъ бо гордымъ противится, смиреннымъ же даетъ благодать}; и жертва Богу духъ сокрушенъ, и \textit{смиренно сердце Богъ не уничижитъ}»\footnote{Бес.~1"~я на 1"~е посл. къ Кор. въ нравоуч.}. А въ чемъ смиреніе состоитъ, таможде мало ниже учитъ отецъ: «аще, рече, досадиши смиренному, аще уничижиши, и аще что"=либо речеши, умолкнетъ, и кротцѣ претерпитъ. И толикъ имать миръ ко всѣмъ, каковаго и изрещи невозможно».

\paragraph*{LVI.} Когда видишь подданнаго, своему царю предстоящаго и просящаго у него милости, примѣчай, что онъ тогда дѣлаетъ? Стоитъ со благоговѣніемъ, главу преклоняетъ и на колѣна падаетъ; умъ и сердце его все тутъ присутствуетъ; не помышляетъ ни о чемъ другомъ, но въ единомъ томъ прошеніи умъ углубленъ имѣетъ. Отъ сего случая учиться должно намъ, христіанамъ, како къ Царю небесному приступать съ прошеніемъ; помышлять, кто мы, и къ кому приступаемъ. \textit{Азъ земля и пепелъ}, глаголалъ Авраамъ къ Богу въ молитвѣ\footnote{Быт.~18,~27.}. Кольми паче намъ тое о себѣ помышлять должно, которые далеко отстоимъ отъ Авраама, съ какимъ убо смиреніемъ, благоговѣніемъ, уничиженіемъ себе самихъ, предъ величествомъ Его падать, главу и сердце преклонять и весь составъ души и тѣла предъ Нимъ повергать: яко на смиренныхъ только призираетъ Богъ. \textit{Призрѣ на молитву смиренныхъ, и не уничижи моленія ихъ}, глаголетъ Псаломникъ\footnote{Пс.~101,~18.}. Но какъ о величествѣ Его, такъ и о великой Его къ намъ милости тогда помышлять должно: \textit{якоже бо величество Его, тако и милость Его}, глаголетъ Писаніе\footnote{Сир.~2,~18.}. Притомъ, когда къ праведному и святому Царю нашему и Богу приступаемъ, должно отступить отъ неправды, по ученію апостола святаго: \textit{да отступитъ отъ неправды всякъ именуяй имя Господне}\footnote{2~Тим.~2,~19.}. Яко \textit{не преселится}, рече, \textit{къ Тебѣ лукавнуяй, ниже пребудутъ беззаконницы предъ очима Твоима. Возненавидѣлъ еси вся дѣлающія беззаконіе}\footnote{Пс.~5,~5 и 6.}. И когда оставленія грѣховъ просимъ, должны и сами простить ближнимъ нашимъ, по ученію Хрістову: \textit{оставите, и оставится вамъ}. "--- Отъ сего видно: 1)~Чтомыя молитвы и пѣнія со скоростію, какъ языкъ можетъ исправляться, или, что горше того, въ два голоса отправляемыя отъ несмысленныхъ поповъ и клириковъ и прочіихъ, не иное что суть, какъ только шумъ въ церквахъ и воздуха біеніе и гласы гортаней, а не молитвы; а отсюду послѣдуетъ, что и въ житіи нѣтъ никакого исправленія у таковыхъ, но едино развращеніе и соблазнъ; что хотя и на всякій день мнятся молитися Богу, но въ самой вещи никогда не молятся: безъ чего хрістіанину исправить житія своего никакъ невозможно, яко не мощному и отъ діавола непрестанно навѣтуемому. "--- 2)~Грѣшникъ, къ Богу въ молитвѣ приступающій и житія нехотящій исправить, ничего не успѣетъ; но, когда хощетъ милость отъ Бога получить, долженъ себе исправить. "--- 3)~Злобующимъ на ближнихъ своихъ, и отмщеніе на нихъ помышляющимъ, такожде заключенъ къ Богу приступъ. Какой милости надѣяться тому у Бога, который самъ надъ подобнымъ себѣ человѣкомъ не хощетъ сдѣлать милости, и грѣшникъ грѣшнику не хощетъ простити? \textit{Аще не отпущаете человѣкомъ согрѣшеній ихъ, ни Отецъ вашъ отпуститъ вамъ согрѣшеній вашихъ}, глаголетъ Господь\footnote{Матѳ.~6,~15.}. "--- 4)~Безъ смиренія такожде тщетно приближеніе и приступъ къ Богу бываетъ. Аще бо предъ человѣкомъ смиряемся, желая отъ него что либо получити: кольми паче предъ Богомъ, Который \textit{гордымъ противится}, смиряться, падать и усердіе показывать должно.

\paragraph*{LVII.} Видишь, что сосудъ глиняный или скудельный такую воню издаетъ, каковою сначала, какъ свѣжимъ былъ, напоенъ; и какъ ни моется и чистится, не теряетъ тоя вони. Разумѣй отъ сего, что тако имѣется и человѣкъ, который есть какъ сосудъ глиняный. Чему съ молоду научится, того и въ слѣдующемъ житіи держаться будетъ. Когда добрѣ и въ страсѣ Господни воспитанъ будетъ, таково и житіе будетъ провождать: когда злу научится, злую злыхъ нравовъ воню и въ житіи будетъ издавать. Но какъ человѣкъ есть къ злу склоненъ, то удобно всякому злу смолоду научается, когда уздою страха и наказанія отъ того не воспящается, и свирѣпѣющая плоть не укрощается. Сей примѣръ научаетъ тебе дѣтей своихъ добрѣ воспитовать, и въ страсѣ Господни и наказаніи содержать, якоже апостолъ учитъ: \textit{воспитовайте ихъ} (чадъ отцы) \textit{въ наказаніи и ученіи Господни}\footnote{Еф.~6,~4.}.

\paragraph*{LVІІІ.} Когда свѣща въ храминѣ свѣтитъ, или свѣтъ дневной просвѣщаетъ, тогда все видишь ясно, и едино отъ другаго распознаешь, что вредно и что невредно. Подобно бываетъ и тому, которому свѣтильникъ закона Божія всегда сіяетъ, и умныя его просвѣщаетъ очи, якоже глаголетъ Псаломникъ: \textit{свѣтильникъ ногама моима законъ Твой, и свѣтъ стезямъ моимъ}\footnote{Пс.~118,~105.}. Когда сія божественная свѣща всегда сіяетъ предъ нами, и просвѣщаетъ насъ день и нощь: всегда какъ во дни, ходити будемъ, и добро отъ зла, пользу отъ вреда, и добродѣтель отъ порока распознавать, и тако ходяще, съ помощію Божіею, не повредимся. Сіе увѣщаваетъ тебе, хрістіанине, \textit{во дни и въ нощи поучатися въ законѣ Господни}\footnote{1,~2.} и \textit{въ сердцѣ своемъ скрывать словеса Божія, яко да не согрѣшити Ему}\footnote{118,~11.}. Къ сему увѣщаваетъ Богъ: \textit{да не отступитъ книга закона сего отъ устъ твоихъ, и да поучаешися въ ней день и нощь, да уразумѣеши творити вся писанная: тогда благоуспѣеши, и исправиши пути твоя, и тогда уразумѣеши}\footnote{Іис. Нав.~1,~8.}.

\paragraph*{LIX.} Видишь, что сосудъ, чимъ напоенъ, такую и воню изъ себе издаетъ: тако имѣется и человѣческое сердце. Когда любовь къ Богу и ближнему имѣетъ, такіе показуетъ и знаки, такіе у него замыслы, начинанія, слова, дѣла и помышленія. И какъ яблоко доброе отъ вкуса, тако оно отъ дѣлъ, словъ и обхожденія познается. И какъ огнемъ согрѣтая пещь теплоту издаетъ, и отъ теплоты познается: такъ огнемъ любве Божія и ближняго согрѣтое сердце, теплоту милости, милосердія, терпѣнія и кротости издаетъ, и отъ тѣхъ познается. Напротивъ того, кто имѣетъ любовію міра сего напоенное сердце, подобные тому и знаки оказываетъ, о томъ помышляетъ, въ томъ поучается, о томъ бесѣдуетъ, тое и дѣлаетъ. Кто богатство, или честь, или славу, или сласть и роскошь любитъ, о томъ и думаетъ, того и ищетъ. Что и Господь глаголетъ: \textit{идѣже есть сокровище ваше, ту будетъ и сердце ваше}\footnote{Матѳ.~6,~21.}. И хотя многіе тщатся сокрыть начинанія и дѣла своя, однакожъ, что внутрь крыется, утаиться не можетъ, но внѣ выходитъ и является, не иначе, какъ пузырь изъ воды, который показуетъ подъ водою крыющійся воздухъ, или какъ отрыжка отъ желудка. \textit{Не можетъ бо древо добро плоди злы творити, ни древо зло плоды добры творити}, глаголетъ Господь\footnote{7,~18.}.

\paragraph*{LХ.} Видишь, что зеркало таковый образъ въ себе воспріемлетъ, къ чему обращается. Къ небу ли обратится, неба образъ въ немъ изображается; къ землѣ обратится, земный образъ въ себе воспріемлетъ. Тако и душа человѣческая имѣется: къ чему любовію обращается и прилѣпляется, тое въ ней и видится. Когда къ Богу обращается, Божій образъ силою Духа Святаго въ ней и изображается: когда къ земнымъ и мірскимъ вещамъ склоняется, земный образъ и начертывается. А какій образъ въ себѣ носитъ, таковое и мудрованіе имѣетъ. Когда Божій образъ имѣетъ, "--- Бога любитъ и человѣка, по образу Божію созданнаго; небесная мудрствуетъ, горняя мудрствуетъ, а не земная: когда земный и скотскій образъ носитъ, "--- земная и мудрствуетъ, дѣлаетъ тое и послѣдуетъ тому, что чувствамъ ея пріятно, что скотамъ несмысленнымъ свойственно есть. Сіе разсужденіе научаетъ тя покаяніемъ и вѣрою о Хрістѣ Іисусѣ, \textit{совлекаться ветхаго человѣка} и скотскаго его образа, и искать и облекаться во образъ Божій, который есть преизряднѣйшее и превосходнѣйшее души украшеніе. А что нынѣ въ душѣ твоей изображается, тое явится и въ день судный; и тако тебе тогда, видя въ тебѣ образъ Свой, \textit{за Своего} признаетъ Господь: прочіихъ же, которые благолѣпія того не имѣютъ, но скотскій образъ и грѣховное безобразіе, какъ ужасное страшилище, носятъ въ себѣ, отречется и речетъ имъ: \textit{не вѣмъ васъ}, хотя и глаголютъ Ему: \textit{Господи, Господи}\footnote{Лук.~13,~27.}!

\paragraph*{LХІ.} Видишь, что древо и хозяину своему, который отребляетъ его, и чужимъ даетъ плодъ свой; или кладезь, и очищающихъ его, и засыпающихъ и плюющихъ въ него напаяетъ; или солнце и удивляющимся ему, и хулящимъ его свѣтитъ. Тако имѣются истинніи хрістіане, которые, какъ древо доброе, \textit{добрые плоды творятъ}\footnote{Матѳ.~7,~17.}; или \textit{яко свѣтила являются въ мірѣ}\footnote{Филип.~2,~15.}, всѣмъ благо творятъ, добрымъ и злымъ, другамъ и врагамъ, подражая Отцу своему небесному, \textit{Иже солнце Свое сіяетъ на злыя и благія, и дождитъ на праведныя и на неправедныя}\footnote{Матѳ.~5,~45.}. Сіе научаетъ тя, хрістіанине, сообразоваться \textit{истиннымъ хрістіанамъ} и съ ними небесному Отцу, а не сынамъ вѣка сего, которые никого не любятъ, токмо себе и ради себе.

\paragraph*{LХІІ.} Видишь, что немощный человѣкъ, который хощетъ исцѣлитися, признаетъ и являетъ лѣкарю немощь свою, и со усердіемъ проситъ его о излѣченіи. Сей случай научаетъ тебе, что и мы, когда хощемъ отъ душевной нашей немощи излѣчитися, должны признать отъ сердца тую, и объявить Хрісту Сыну Божію, Иже есть душъ и тѣлесъ Врачъ, и просить Его со смиреніемъ, дабы помоглъ. Вси бо мы отъ природы немощны, бѣдны и окаянны. Читаемъ во Евангеліи святомъ, что ко Хрісту приходили слѣпіи, глухіи, нѣмыи, прокаженныи, и объявляли Ему свои немощи, и со смиреніемъ и вѣрою просили милости: и исцѣлялись. Слѣпы мы отъ рожденія, слѣпы въ божественныхъ, духовныхъ и къ спасенію душевному надлежащихъ, и потому со слѣпцами должны Его молить: \textit{Господи, да отверзутся очи наши}\footnote{Матѳ.~20,~33.}. Глухи мы отъ природы къ слышанію слова Божія и заповѣдей Его святыхъ; и ради того должны просить у Него ушесъ, о которыхъ Онъ глаголетъ: \textit{имѣяй уши слышати, да слышитъ}\footnote{13,~9.}. Гугнивы и нѣмы мы отъ природы къ призыванію, славословію, хваленію и пѣнію имени Его святаго; и сего ради должно къ Нему воздыхать, чтобы отверзлъ языкъ и уста наша, и молитися Ему: \textit{Господи, научи насъ молитися}\footnote{Лук.~11,~1.}. Прокаженны мы отъ естества неисцѣльною грѣха проказою; и потому должны съ прокаженными возносить изъ глубины сердца гласъ, и глаголати Ему: \textit{Іисусе, Наставниче, помилуй насъ}\footnote{17,~13.}! Разслабленны мы, и не можемъ встать и пріитить къ Нему, и ради того должно молить Его, дабы простеръ къ намъ всемогущую десницу Свою, и возставилъ отъ одра грѣховнаго. Плоть наша злѣ бѣснуется, и ради того должно вопить съ женою хананейскою: \textit{помилуй мя, Господи, Сыне Давидовъ}! и повторять прошеніе свое: \textit{Господи, помози мнѣ}! и вмѣнить себе не достойныхъ части чадъ, но какъ псовъ, которые ядятъ отъ крупицъ, падающихъ отъ трапезы господей своихъ\footnote{Матѳ.~15,~22--27.}. Получили тогда бѣдные, какъ Евангеліе свидѣтельствуетъ, отъ Врача"=Хріста милость и исцѣленіе: получимъ и мы, когда съ признаніемъ немощей нашихъ и вѣрою будемъ у Него просить. Ибо на сіе и въ міръ пришелъ, чтобы немощи наши исцѣлить, и окаянство отъ насъ отнять, и подать намъ блаженство. Аще тѣлеса тлѣнная и смертная исцѣлилъ просящимъ: кольми паче души безсмертныя исцѣлитъ, яко душелюбецъ.

\paragraph*{LХІІІ.} Видишь, что зерцало запыленное или закопченное ничего не показуетъ, хотя въ него и смотришь; но, когда хощешь въ немъ усмотрѣть лице свое и на лицѣ пороки, должно отереть его. Тако совѣсть человѣческая имѣется: когда многими пороками грѣховными и беззаконнымъ житіемъ замарана будетъ, человѣкъ въ ней не усматриваетъ пороковъ, на душѣ своей прилипшихъ, и такъ весь замаранъ, какъ еѳіопъ, ходитъ, и отъ грѣха въ грѣхъ безопасно падаетъ. Бѣдственно и плачевно таковое состояніе есть! Не видитъ таковый скверны и мерзости своея, но увидитъ тогда, когда книги открыются въ позоръ всему міру, ангеломъ и человѣкомъ, и представятся предъ лицемъ его грѣхи его, якоже глаголетъ Богъ: \textit{обличу тя, и представлю предъ лицемъ твоимъ грѣхи твоя}\footnote{Пс.~49,~21.}. Тогда онъ увидитъ себе, какъ еѳіопа чернаго, и какъ ужасное страшилище. Ибо мерзость сія душевная, которая нынѣ внутрь крыется, тогда внѣ явится; якоже праведныхъ душъ доброта явится, \textit{и просвѣтятся, яко солнце во царствіи Отца ихъ}\footnote{Матѳ.~13,~43.}. Тогда бѣдный грѣшникъ начнетъ самъ собою гнушаться, самого себе ненавидѣть, отъ самого себе убѣгать, и въ ничто обратиться пожелаетъ, но не возможетъ: яко во вѣки безъ конца будетъ на немъ скаредность сія во обличеніе его, что въ такой мерзости житіе свое на земли провождалъ. О, ежели бы сію мерзость, нынѣ разсмотрѣвъ, увидѣлъ: непрестанно бы плакалъ, и искалъ бы отъ ней благодатію Хрістовою избавиться; но къ бѣдствію своему ослѣпленъ, не видитъ тоя. Но когда зерцало чистое будетъ, тогда все ясно показуетъ, что предъ нимъ ни обращается; и пороки на лицѣ, не токмо великіе, но и малые усматриваемъ въ немъ. Тако имѣется чистая совѣсть: и самые малые пороки усматриваетъ, и покаяніемъ, слезами и вѣрою омываетъ ихъ человѣкъ. Сіе разсужденіе научаетъ тебе очувствоваться, вникнуть въ законъ Божій, который грѣхи наши показываетъ и обличаетъ, и покаяться и очистить совѣсть свою прилѣжнымъ его разсмотрѣніемъ, покаяніемъ и вѣрою, и тако начать новое житіе, да не явятся тогда грѣхи твои всему міру въ познаніе, которыхъ нынѣ самъ единъ не хощешь разсмотрѣть и познать, и тако очистити ихъ.

\paragraph*{LXIV.} Ходячи въ саду, познаешь, что древеса не иначе узнаются, какъ отъ вкуса: тако разумѣй о людяхъ. Многіе называются хрістіанами: но \textit{истинный хрістіанинъ} быть и познаться не можетъ, какъ отъ истинной вѣры и любви. Сіи примѣты \textit{истиннаго хрістіанина} суть, что онъ любителенъ, смиренъ, терпѣливъ, милостивъ и благосклоненъ. И какъ въ саду многіи древа показуются своими плодами на взоръ пріятными, но, какъ вкусишь отъ нихъ, познаешь, что они негодны суть: такъ многіи показуются добрыми, гладко, ласково и тихо говорятъ, много постятся и много молятся; но, какъ коснешься ихъ, тогда отъ плодовъ своихъ познаются горькими, яко ненавистію, злобою, завистію и немилосердіемъ наполнены, и тако суть древеса злая, злые плоды творящая. И сіе"=то есть, что Господь глаголетъ: \textit{нѣсть древо добро, творя плода зла, ниже древо зло творя плода добра. Всяко бо древо отъ плода своего познается}\footnote{Лук.~6,~43 и 44.}. Ниже охуждается здѣ постъ, молитва, но безъ любви хрістіанской ничтоже суть; ниже сладкое, ласковое и тихое слово само въ себѣ порочится, но безъ плода любве есть единое лицемѣріе и лукавство. Не единыя бо слова, но дѣла, съ словами гладкими, ласковыми и тихими сопряженныя, показываютъ добраго хрістіанина. Чего ради внутрь себе должно намъ имѣть, и стараться имѣть хрістіанство: и тако, когда внутрь будетъ добро, то и внѣ будетъ являться добро. Тако, по словеси Хрістову, \textit{благій человѣкъ отъ благаго сокровища сердца своего износить будетъ благое}\footnote{Лук.~6,~45.}.

\paragraph*{LXV.} Видишь, что древо даетъ плодъ всякому, но молчитъ; кладязь напаяетъ, но молчитъ; тако и прочія созданія Божія пользу намъ подаютъ, но молчатъ. Тако бо отъ Создателя устроены, и, молча, проповѣдуютъ славу и хвалу Божію, подая намъ случай славити и хвалити благость Божію, намъ, которые ихъ къ пользѣ нашей употребляемъ. Симъ научаемся и мы, разумное созданіе, другъ другу добро дѣлать, но молчать, не трубить предъ собою, не искать славы своей, но Создателя нашего, отъ Котораго мы сами и всякое добро происходитъ. Тако научаетъ насъ Хрістосъ Спаситель нашъ: \textit{тако да просвѣтится свѣтъ вашъ предъ человѣки, яко да видятъ ваша добрая дѣла, и прославятъ Отца вашего, Иже на небесѣхъ}\footnote{Матѳ.~5,~16.}.

\paragraph*{LXVI.} Видишь, что въ единомъ составѣ тѣла вси члены между собою суть любовны, мирны, другъ другу помогаютъ: рука рукѣ и прочіимъ удамъ помогаетъ; нога ногѣ и прочіимъ, глазъ глазу и прочіимъ, ухо уху и прочіимъ помогаютъ. Тако имѣются \textit{истинные хрістіане}, которыми Хрістосъ, Сынъ Божій, Глава ихъ, управляетъ. Они между собою любовны, мирны, кротки, другъ другу помогаютъ, и другъ друга пользуютъ всяко, то"=есть, духовно и тѣлесно, когда единъ бѣдствуетъ, другій ему тщится помощи; когда единъ нищетствуетъ, другій нищету его *дополняетъ*; когда единъ неможетъ, другій тому служитъ; когда единъ страждетъ, другій ему состраждетъ и соболѣзнуетъ; когда единъ колеблется и бѣдствуетъ въ вѣрѣ, другій его подкрѣпляетъ; когда единъ малодушествуетъ и изнемогаетъ, другій его утѣшаетъ и руку помощи подаетъ и прочая. Ибо и они другъ другу суть уды, едину преблагословенную Главу имущіи Хріста, по словеси апостола: \textit{мнози едино тѣло есмы о Хрістѣ, а по единому, другъ другу уди}\footnote{Римл.~12,~5.}. Сіе научаетъ тя любовь имѣть ко всѣмъ, по увѣщанію апостола: \textit{любовію работайте другъ другу}\footnote{Гал.~5,~13.}, "--- и миръ со всѣми: \textit{миръ имѣйте и святыню со всѣми, ихже кромѣ никтоже узритъ Господа}\footnote{Евр.~12,~14.}, когда хощешь быть хрістіаниномъ. Нѣтъ бо въ тѣхъ людяхъ хрістіанства, слѣдственно и выключаются отъ числа \textit{хрістіанъ истинныхъ}, въ которыхъ любви къ ближнимъ нѣтъ, и съ ними мирно не хотятъ жить.

\paragraph*{LXVII.} Слышишь, что единъ у другаго спрашиваетъ, кто той"=то домъ или той градъ создалъ; и видишь, что ничто собою не дѣлается, но отъ инаго начало и бытіе свое имѣетъ. Отъ сего случая обрати разсужденіе твое къ міру и всему, что въ мірѣ имѣется, и вопроси, отъ кого онъ со всѣмъ своимъ исполненіемъ и украшеніемъ имѣетъ начало. И молча, отвѣщаетъ тебѣ: Той мя создалъ. Небеса отвѣщаютъ тебѣ: мы \textit{повѣдаемъ славу Божію}\footnote{Пс.~18,~2.}. Солнце, луна и звѣзды отвѣщаютъ тебѣ: Той насъ создалъ, и далъ намъ свѣтъ къ просвѣщенію вашему, Той, Иже \textit{одѣвается свѣтомъ яко ризою}\footnote{Пс.~103,~2.}. Облака отвѣщаютъ тебѣ: Той насъ создалъ, и повелѣлъ дождити на васъ и нивы ваши. Земля отвѣщаетъ тебѣ: Той мене создалъ, и повелѣлъ плоды творити вамъ. Вода отвѣщаетъ тебѣ: Той мене создалъ, и повелѣлъ напаяти васъ и скоты ваши. Рыбы отвѣщаютъ тебѣ: Той насъ создалъ, и подалъ вамъ въ пищу. Древеса, травы и вся прозябающая отвѣщаютъ тебѣ: Той насъ создалъ, и повелѣлъ служити нуждамъ вашимъ. Скоты и звѣри отвѣщаютъ: Той насъ создалъ, и повелѣлъ работати вамъ. Огнь отвѣщаетъ тебѣ: Той мене создалъ, и повелѣлъ согрѣвати васъ, свѣтити вамъ и варити пищу вашу. Люди Его отвѣщаютъ тебѣ: \textit{Той сотворилъ насъ, а не мы: мы же людіе Его, и овцы пажити Его}\footnote{99,~3.}, "--- и повелѣлъ намъ \textit{другъ другу любовію работати}\footnote{Гал.~5,~13.}, и \textit{хвалити имя Его, яко благъ Господь, въ вѣкъ милость Его и даже до рода и рода истина Его}\footnote{Пс.~99,~5.}. И Духъ Святый чрезъ пророка научаетъ тебе: \textit{Той рече, и быша; Той повелѣ, и создашася. Постави я въ вѣкъ, и въ вѣкъ вѣка. Повелѣніе положи, и не мимо идетъ}\footnote{148,~5 и 6.}. "--- Отъ сего научаешься, хрістіанине: 1)~Что есть Богъ Создатель, Котораго созданія показуютъ, какъ мастера мастерство. 2)~Что все отъ Него начало и бытіе свое имѣетъ. 3)~Что все повелѣніемъ Божіимъ человѣку работаетъ. 4)~Что человѣкъ, пользуясь Божіимъ созданіемъ, неотмѣнно обдолжается Богу благодарить, имя Его хвалить, и усердно работать Ему. 5)~Что заблуждаютъ тѣ, яко слѣпые, которые сея должности не исполняютъ, и отъ своей совѣсти изобличаются яко неблагодарные. 6)~Что сіе самое имъ во обличеніе будетъ на судѣ Хрістовомъ, что добромъ Божіимъ пользовались, живучи въ мірѣ, но Богу, добра Дателю, не хотѣли благодарить и работать. 7)~Отъ сего можешь и себе разсмотрѣть, тщишися ли ты сію должность исполнять, да не неблагодарнымъ къ Создателю твоему явишися, и съ неблагодарными рабами въ день суда Хрістова осудишься.

\paragraph*{LXVIII.} Видишь, что ни солнце не помрачается, ни сладкая вода въ горькую не обращается, ни древо доброе злымъ не дѣлается отъ хулителей своихъ; такожде ни солнце свѣтлѣйшимъ, ни сладкая вода сладчайшею, ни древо доброе лучшимъ сотворяется отъ похвалы человѣческой. Тако имѣются люди добрые: ни отъ хулы человѣческой худшими, ни отъ похвалы лучшими не бываютъ; ни похвала человѣческая не придаетъ имъ доброты, ни хула отнимаетъ, но каковы въ сердцѣ имѣются, таковыми всегда пребываютъ. Симъ научаешься ни хуленіемъ огорчеваться, ни похвалою возноситься, много паче не искать ея: понеже хуленіе не отыметъ, а похвала не придаетъ тебѣ внутреннія доброты. Не тотъ бо добръ, кого хвалятъ люди, ни тотъ золъ, кого хулятъ люди, но доброта или злость у всякаго внутрь есть, а не отъ человѣкъ происходитъ; и кого Богъ похвалитъ, или осудитъ. Что въ той похвалѣ, которая внѣ въ ушахъ звенитъ, но внутрь совѣсть осуждаетъ? И что тая хула сдѣлаетъ, которая внѣ ударяетъ, когда внутрь совѣсть оправдаетъ? Какъ внутрь у человѣка или добро или зло, блаженство или окаянство: такъ внутрь должно быть утѣшеніе или огорченіе, увеселеніе или печаль. Какъ у добраго человѣка внѣшняя хула не отымаетъ внутренняго блаженства и потому не опечаляетъ его: такъ и злому не придаетъ внутренняго блаженства внѣшняя похвала. Понеже сего внутрь злая совѣсть безпокойствуетъ и опечаляетъ, хотя отвнѣ и ублажается: онаго внутрь добрая совѣсть похваляетъ и увеселяетъ, хотя отвнѣ и охуждается. Совѣсть бо всякому человѣку есть какъ вѣрный свидѣтель, который или оправдаетъ его, или осуждаетъ; оправдаетъ за неповинность, осуждаетъ за вину: что въ ней написано будетъ, тое она и показуетъ вѣрно. Сей свидѣтель солгать не можетъ, какъ его ни умягчай.

\paragraph*{LXIX.} Видишь, что виноградъ приноситъ плодъ тому, кто его насадилъ, якоже апостолъ глаголетъ: \textit{кто насаждаетъ виноградъ, и отъ плода его не ястъ}\footnote{1~Кор.~9,~7.}? Хрістіанинъ всякъ есть лоза въ виноградѣ церкве святыя, оплотомъ закона Божія огражденномъ насажденная, который словомъ Божіимъ напаяется и тайнами святыми утучняется: долженъ убо плодъ добрыхъ дѣлъ приносити Богу, Насадителю своему и Питателю, то"=есть, плодъ любве, терпѣнія, кротости, \textit{плодъ устенъ исповѣдающихся имени Его}\footnote{Евр.~13,~15.}. Аще убо кто не приноситъ плода сего, надобно тому опасаться посѣченія, о которомъ Предтеча глаголетъ: \textit{уже и сѣкира при корени древа лежитъ: всяко убо древо, еже не творитъ плода добра, посѣкаемо бываетъ, и во огнь вметаемо}\footnote{Матѳ.~3,~10.}. Сей случай и разсужденіе его увѣщаваетъ тебе къ истинному покаянію и плодамъ покаянія, безъ которыхъ и покаяніе истинное быть не можетъ.

\paragraph*{LXX.} Видишь яблоко внѣ красно, но внутрь гнило и смрадно. Тако имѣются многіи люди: внѣ красуются благородіемъ и хвалятся, но внутрь въ душахъ своихъ гнилы и смрадны; тѣломъ надъ другими господствуютъ, повелѣваютъ инымъ, но душею своею работаютъ страстямъ и прихотямъ своимъ, и суть подлѣйшіи раби своихъ рабовъ. Лучше бо человѣку работать, нежели прихотямъ своимъ и грѣху: ибо человѣкъ есть созданіе Божіе, а грѣхъ есть дѣло діавольское. Таковые суть, которые господами называются, но попущаютъ надъ собою господствовать гнѣву, злобѣ, мамонѣ, сребролюбію, гордости, неправдѣ, зависти, сладострастію, нечистотѣ, чревоугодію и прочіимъ мерзкимъ мучителямъ. Откуду Златоустъ святый глаголетъ: «аще кто и богатъ и благороденъ, отъ худыхъ есть худѣйшій, егда отъ грѣха плѣненъ будетъ»\footnote{Бес.~9"~я на 1"~е посл. къ Кор.}. Такожде хрістіане вси подобны суть яблоку, внѣ красному, но внутрь гнилому, которые внѣ являются добрыми, но внутрь ненавистію и злобою и прочею гнилостію и смрадомъ исполнены суть, что есть лицемѣріе и лукавство.

\paragraph*{LXXI.} Видишь, что отъ теплоты иныя вещи умягчаются и растопляются, какъ"=то: воскъ, сало, масло и прочая; иныя ожесточаются, какъ"=то: глина, древо, трава и прочая. И сіе дѣлается не отъ стороны теплоты, но отъ стороны вещей, теплоту пріемлющихъ: теплота бо все, и тое и другое, равно согрѣваетъ. Тако имѣются люди: иные теплотою благости и долготерпѣнія Божія умягчаются, растаеваютъ, умаляются и оскорбляются въ покаяніе и печаль по Бозѣ; другіе ожесточаются и въ большая и множайшая успѣваютъ злая, къ каковымъ апостолъ глаголетъ: \textit{о человѣче! или о богатствѣ благости Его и кротости и долготерпѣніи нерадиши, невѣдый, яко благость Божія на покаяніе тя ведетъ? По жестокости же твоей и непокаянному сердцу собираеши себѣ гнѣвъ въ день гнѣва и откровенія праведнаго суда Божія}\footnote{Римл.~2,~3 и 4.}. И сіе бываетъ такожде не отъ стороны благости Божія, но отъ стороны сердца человѣческаго, которое злоупотребляетъ благостію Божіею и не хощетъ каятися: благость бо Божія всѣхъ равно на покаяніе ведетъ. Тако ожесточился Фараонъ, который толико видѣлъ чудесъ Божіихъ, и не исправился, и погиблъ. Тако и нынѣ многіи ожесточаются, которые день отъ дня отлагаютъ покаяніе, которыхъ постигаетъ нечаянно смерть, и на оный вѣкъ безъ надежды спасенія отсылаетъ. Симъ научаешися, хрістіанине, не отлагать покаянія, но тотчасъ, воставши, исправитися и каятися, по увѣщанію Духа Святаго: \textit{днесь аще гласъ Его услышите, не ожесточите сердецъ вашихъ}\footnote{Пс.~94,~7 и 8.}.

\paragraph*{LХХІІ.} Видишь, что стекло замаранное или закопченное не вмѣщаетъ свѣта естественнаго. Тако имѣется человѣческая душа, замаранная и очерненная грѣхами, которая дотолѣ не вмѣщаетъ свѣта Божія благодати, доколѣ во грѣхахъ пребываетъ. Грѣхи бо суть какъ мгла темная, или паче какъ препона, которая свѣта Божія не допущаетъ до души, и раздѣляетъ между Богомъ и душею, якоже глаголетъ пророкъ: \textit{грѣси ваши разлучаютъ между вами и между Богомъ, и грѣхъ ради вашихъ отврати лице Свое отъ васъ}\footnote{Ис.~59,~2.}. "--- Стекло, какъ начнетъ очищаться, начнетъ и свѣтъ въ себе впущать, и чимъ болѣе очищается, тѣмъ болѣе свѣта вмѣщаетъ: тако душа человѣческая, какъ, обратившися отъ грѣховъ къ Богу, начнетъ каятися и покаяніемъ и вѣрою во Хріста очищать грѣхи своя, "--- будетъ и благодатію Божіею просвѣщатися; и чимъ усерднѣе кается, тѣмъ болѣе просвѣщается. Сіе увѣщаваетъ насъ обратитися къ Богу и каятися, когда хощемъ милость Его получити.

\paragraph*{LXXIII.} Видишь, что земля, хотя и сѣется на ней сѣмя, ничего не прозябаетъ, когда не будетъ имѣть влаги. Тако имѣется и сердце человѣческое: хотя и слышитъ слово Божіе, но не можетъ прозябати плода, достойнаго слова Божія, когда влаги Божія благодати не будетъ имѣти. Откуду намъ повелѣно такъ усердно о семъ молитися: \textit{просите, ищите, толцыте}, глаголетъ Господь\footnote{Матѳ.~7,~7.}. Сіе убѣждаетъ насъ усердно молитися Богу и просить благодати у него, да подастъ намъ \textit{уши слышати} слово Его, и \textit{сердце творити}. Безъ сего бо ничего не успѣемъ, яко глубоко растлѣнно сердце имѣемъ; яко, слышавше слово Божіе, не можемъ разумѣти и въ сердце углубить; разумѣвше, не можемъ творить, такъ, что на всякую минуту требуемъ благодати Божіей, помоществующей намъ въ дѣлѣ благочестія.

\paragraph*{LXXIV.} Видишь, что стекло не токмо само пріемлетъ свѣтъ, но и въ храмину пущаетъ; печь не токмо ради себе, но и ради покоя согрѣвается. Тоежде примѣчается въ составѣ тѣлесномъ: очи не ради себе токмо видятъ, но и ради всего тѣла; уши не ради себе только слышатъ, но и ради всего тѣла; ноздри не ради себе только обоняютъ, но и ради всего тѣла; желудокъ не ради себе токмо пищу пріемлетъ, но и ради всего тѣла; руки не ради себе токмо дѣлаютъ, но и ради всего тѣла; ноги не ради себе токмо ходятъ, но и ради прочіихъ членовъ. Такожде и въ прочемъ естествѣ примѣчаемъ: солнце на небѣ свѣтитъ, но прочей твари; облака не держатъ въ себѣ воды, но на землю изливаютъ; земля не удерживаетъ въ себѣ плодовъ, но вонъ испущаетъ. Тоежде и о прочемъ разумѣй. Симъ научаемся мы, что дарованія отъ Бога намъ данная, должны быть намъ и ближнимъ нашимъ \textit{общи}. Разумный не токмо себе, но и ближняго разумомъ пользовати долженъ; богатство не токмо самому богатому, но и прочіимъ служить должно; такъ и въ прочемъ: сіи бо суть \textit{таланты Господа нашего}, за которые мы отвѣтъ Ему отдадимъ въ день пришествія Его\footnote{Матѳ.~25,~14--30.}. Все бо, что ни имѣемъ, Божіе есть, а не наше добро; которое намъ дано отъ Бога не ради насъ самихъ токмо, но и ради ближнихъ нашихъ. Аще убо кто отъ добра Божія, даннаго себѣ, не удѣляетъ другимъ, сей грѣшитъ противу Бога, даровавшаго ему благая Своя, за что имѣетъ истязанъ быть въ день суда Хрістова.

\paragraph*{LXXV.} Видишь, что стомахъ немощный пріемлетъ въ себе пищу, но не варитъ по надлежащему, и тако не раздѣляетъ сока и крови прочіимъ удамъ, а отъ того и прочіе уды и себе въ большее безсиліе приводитъ: якоже здоровый стомахъ, который добрѣ пищу варитъ, и прочіимъ членамъ сокъ и кровь сообщаетъ; тѣмъ самымъ, какъ прочіе члены, такъ и себе крѣпчайшимъ дѣлаетъ. Тако разумѣется богатый, когда онъ богатство, которое отъ Бога пріемлетъ, или сокрываетъ, или на непотребные расходы истощаетъ, и какъ никому, такъ и себѣ не дѣлаетъ пользы, но паче себѣ пагубу устрояетъ, не подая ближнимъ того, что подавать долженъ, по Божію велѣнію: \textit{просящему у тебе дай}\footnote{5,~42.}. Тако бо, лишая ближняго добра тѣлеснаго, лишаетъ себе плода духовнаго, заповѣди Божія не исполняетъ, самолюбіемъ и немилосердіемъ недугуетъ, и тако, не дѣлая милосердія ближнему, заключаетъ и себѣ двери милости Божія. Напротивъ, щедрый богачъ, какъ крѣпкій желудокъ, и самъ довольствуется богатствомъ во славу Божію, благодаря Бога, и прочіихъ довольствуетъ во славу Тогожде Бога, и тако себе духовно пользуетъ, другихъ тѣлесно пользуя. Сіе разумѣй и о прочіихъ дарованіяхъ, отъ Бога намъ данныхъ, которыя, когда ихъ сообщаемъ ближнимъ нашимъ, и намъ пользу приносятъ, а когда не сообщаемъ, и намъ вредятъ, не отъ себе, но отъ стороны нашей, когда ихъ не такъ, какъ должно и на что должно, употребляемъ. Якоже злоупотребляемъ огонь на сожженіе вещей ближняго, который данъ намъ ради варенія пищи, согрѣванія покоевъ и проч.; злоупотребляемъ мечь на убіеніе брата нашего, который устроенъ на защищеніе отечества; злоупотребляемъ языкъ на оклеветаніе, злословіе, сквернословіе, который данъ намъ ради сообщенія совѣтовъ сердечныхъ и на хваленіе Божіе; злоупотребляемъ руки на хищеніе, біеніе, которыя даны намъ къ дѣланію и помоществованію себѣ и ближнимъ нашимъ, и проч.: тако и дарованія Божія можетъ человѣкъ на зло, и по тому себѣ и ближнему во вредъ, употреблять.

\paragraph*{LXXVI.} Видишь, что кривое дерево ни къ какому дѣлу не годится. Тако имѣется злый человѣкъ, у котораго сердце неправое есть. Онъ, по подобію древа криваго, никуды не годится. Рабомъ ли будетъ? не вѣрно, но лестно, лѣниво и лукаво работаетъ. Біется ли, или наказуется инако? ропщетъ и негодуетъ. Жалуется ли, какъ и награждается? ни во что тое вмѣняетъ. Въ счастіи своевольствуетъ и свирѣпѣетъ; въ несчастіи негодуетъ и хулитъ. Словомъ, ни милостію не преклоняется, ни наказаніемъ не исправляется, какъ древо кривое, которое ни теплотою, ни морозомъ не исправляется. Причина тому сія, что неправое сердце имѣетъ, отъ котораго все зависитъ. Такожде и свободникъ злый тоежде показываетъ; тое только замышляетъ, что себѣ видитъ угодно, хотя бы Богу противно, себѣ и прочіимъ вредно было. Въ чести ли, или санѣ какомъ будетъ? своея только, а не подкомандныхъ корысти ищетъ. На чести свирѣпѣетъ; лишився чести, негодуетъ, ропщетъ и хулитъ, ничимъ умягчиться не можетъ: понеже кривое, развращенное сердце имѣетъ. Напротивъ того, какъ прямое древо ко всякому дѣлу годится, тако добрый человѣкъ имѣется, у котораго правое сердце. Правое же сердце есть то, которое Богу, Творцу своему, и святой волѣ Его послѣдуетъ. Богъ все дѣлаетъ въ пользу нашу: такъ и правое сердце тщится все дѣлать во славу Божію и пользу братіи своея. Чего хощетъ Богъ, того хощетъ и оно. Богъ хощетъ и дѣлаетъ всѣмъ только добро, безъ всякой Своей корысти: тако и оно хощетъ и тщится всякому дѣлать добро безъ своей пользы. \textit{Любовь бо не ищетъ своихъ}, глаголетъ апостолъ\footnote{1~Кор.~13,~5.}. Идѣже бо любовь истинная, тамо и правое сердце. Таковое сердце, въ какомъ званіи и чинѣ ни будетъ, всегда и вездѣ право дѣлаетъ, и ко всякому дѣлу и званію годится. Симъ научаемся, любезный хрістіанине, искать и просить у Бога сердца чистаго и духа праваго. \textit{Сердце чисто созижди во мнѣ, Боже, и духъ правъ обнови во утробѣ моей}\footnote{Пс.~50,~12.}! А когда сердце право будетъ, то и дѣла наши, слова и помышленія права будутъ. "--- Отъ сего видно: 1)~Что отъ сердца все внѣшнее дѣло зависитъ. Каковое сердце кто имѣетъ, таковое и внѣшнее дѣло его. 2)~И самое внѣшнее доброе дѣло (по видимому доброе) порочится, когда отъ неправаго сердца происходитъ. Напр. милостыня ближнему, даемая ради своей корысти или тщеславія, "--- правда отъ судіи творимая не ради Бога, но ради избѣжанія гражданскаго суда, суть по видимому добрыя дѣла, но порочатся, яко не отъ праваго сердца происходятъ. Доброе бо дѣло есть тое истинно доброе, которое такъ дѣлается, какъ повелѣно. \textit{Злое бо древо не можетъ добра плода творити}\footnote{Матѳ.~7,~18.}.

\paragraph*{LXXVII.} Видишь, что или сынъ предъ отцемъ, или рабъ предъ господиномъ своимъ безчинствуетъ и не отдаетъ достойныя чести: непристойно тебѣ кажется таковыхъ безчинниковъ дѣло. Разумѣй, что тако хрістіане безчинствуютъ предъ Богомъ, вся назирающимъ и на всякое дѣло и помышленіе наше смотрящимъ, когда тайно или явно грѣшатъ. Опечаляется отецъ безчинствомъ сына своего, и господинъ раба своего: много паче Богъ, Отецъ и Господь всѣхъ, опечаляется безчинствомъ хрістіанъ неисправныхъ. Сей случай и разсужденіе научаетъ тя: 1)~Въ словахъ, дѣлахъ и помышленіяхъ опасно поступать и храниться отъ богопротивныхъ дѣлъ, словъ и помышленій, которыми Духъ Его Святый оскорбляется. 2)~Удивляться благости Его и долготерпѣнію Божію, Который столько терпитъ намъ, предъ Нимъ словами, дѣлами и помышленіями безчинствующимъ. 3)~Прощать ближнему твоему всякую обиду, нанесенную отъ него, ради толикой благости Божіей, которую на всякій день и часъ на себѣ дознаешь. 4)~О прежнихъ грѣхахъ и безчиніяхъ сердечно жалѣть и смиренно предъ Нимъ падать, стыдиться и просить прощенія. 5)~Усердно благодарить Ему, что въ такихъ безчинствіяхъ и худыхъ поступкахъ праведнымъ судомъ не поразилъ тебе.

\paragraph*{LXXVIII.} Видишь, что цвѣтокъ или отъ зноя или отъ мраза скоро увядаетъ, или дымъ и пузырь на водѣ скоро исчезаетъ. Тако имѣется богатство, честь и слава міра сего: не успѣютъ явиться, и показуются нѣчто, но при нашедшей или бурѣ бѣдъ, или кончинѣ, все исчезаетъ, и бываетъ, что какъ бы ихъ и не было. Посмотри во гробы богатыхъ и славныхъ, и увидишь, что ничимъ отъ нищихъ и подлыхъ не разнствуютъ. Вси, какъ земля, въ землю возвратилися и обратилися. Оставили они міръ: оставила ихъ и слава, и богатство отступило; и учинилися богатые какъ нищіе, и славные какъ подлые, и сравнилися господа съ рабами своими, богатые съ убогими такъ, что не можно познать, гдѣ господинъ и гдѣ рабъ его, гдѣ богатый и гдѣ нищій лежитъ. Сей случай и разсужденіе его учитъ насъ искать истиннаго душевнаго богатства и славы, которыя отъ насъ не отступятъ, когда ихъ сыщемъ; а за утѣхами міра сего не гоняться, которыя какъ дымъ и пузырь исчезаютъ, и, вмѣсто мнимаго веселія, истинную намъ оставляютъ скорбь.

\paragraph*{LXXIX.} Видишь, что желѣзо или камень, хотя и горячіе, дотолѣ не издаютъ голоса, доколѣ не изліется на нихъ вода, или иное что подобное; когда изліется, тогда показываютъ свою горячесть изданіемъ шума. Тако имѣются многіи люди: дотолѣ кажутся себѣ смиренны и кротки быти, доколѣ обиды не пріемлютъ; а какъ почувствуютъ обиду нанесенную, тогда гнѣваются и ярятся и шумятъ, какъ желѣзо горячее, облитое водою. Обида бо нанесенная показуетъ, что въ сердцѣ крыется "--- гнѣвъ, или кротость. И сія"=то есть между прочіими причина, чего ради Богъ попущаетъ на насъ напасти и злорѣчивыя уста: тако бо показуетъ намъ самимъ, что у насъ въ сердцѣ крыется "--- гнѣвъ, или кротость, да, тако познавше немощь нашу, потщимся благодатію Его врачевать. Сей случай и разсужденіе учитъ тебе познавать сердце твое, которое \textit{глубоко} есть, и глубоко растлѣнно\footnote{Іер.~17,~9.}.

\paragraph*{LХХХ.} Видишь, что теплая печь всякаго согрѣваетъ, кто ни приступаетъ къ ней. Тако имѣется любительный человѣкъ, котораго сердце любовію къ Богу и ближнему согрѣто: онъ всякаго теплотою любве своея согрѣваетъ, кто къ нему ни прикасается. И какъ печь удѣляетъ теплоты своея требующимъ, такъ онъ отъ дарованій Божіихъ удѣляетъ требующей братіи: инаго наготу прикрываетъ, инаго въ домъ вводитъ и упокоеваетъ, иному въ болѣзни, иному въ темницѣ служитъ, инаго въ печали утѣшаетъ; словомъ, всякому во всемъ, въ чемъ можетъ, руку помощи простираетъ. Все же сіе дѣйствуетъ въ немъ теплота хрістіанскія любве. Напротивъ того, холодная печь никакой никому не дѣлаетъ пользы, и хотя кто и прикасается ей, обманувшися, отходитъ не согрѣтъ: тако имѣются нелюбительные люди, которыхъ сердца не согрѣты хрістіанскою любовію. Многіи мнятся быть нѣчто, много и часто читаютъ молитвы, часто постятся, созидаютъ храмы Божіи и украшаютъ (что само собою похвально и видъ благочестія показуетъ); но когда ихъ коснется кто бѣдный и требующій помощи, обманывается, какъ и тотъ, который къ холодной печи прикасается, хотя отъ ней согрѣтися, и такъ отходитъ съ тоюжде печалію и бѣдою, съ которою приходилъ. Сей случай и разсужденіе учитъ тебе обучатися любви хрістіанской, безъ которой всякое дѣло мертво бываетъ, хотя бы нѣчто и казалося предъ очесами человѣческими. Истинная бо любовь есть живность всѣхъ добрыхъ дѣлъ и хрістіанскихъ добродѣтелей. Какъ бо животное безъ естественной теплоты не можетъ жить, такъ всякое доброе дѣло безъ истинной любви не можетъ живо быть, но видъ только нѣкій ложный показуетъ. О ложныхъ любителяхъ глаголетъ апостолъ святый: \textit{аще братъ или сестра наги будутъ, и лишени будутъ дневныя пищи; речетъ же имъ кто отъ васъ: идите съ миромъ, грѣйтеся и насыщайтеся; не дастъ же имъ требованія тѣлеснаго: кая польза}\footnote{Іак.~2,~15 и 16.}? Каковыхъ любителей въ нынѣшнихъ хрістіанахъ много имѣется, которые ласково привѣтствуютъ, но въ дѣлѣ все противно. А многіи и не стыдятся, видя нужду братню, такія слова произносити: \textit{что мнѣ до его нужды}? А иные требующихъ къ Богу отсылаютъ: \textit{Богъ"=де дастъ}. И когда бы симъ словомъ желали, чтобы Богъ подалъ требующему, яко сами не имѣютъ что подать!.. Но отсылаютъ отъ себе къ Богу, яко сами \textit{не хотятъ} подать отъ дарованій Божіихъ просящему. Такъ"=то, любезный читатель, чимъ болѣе міръ приближается къ концу, тѣмъ паче оскудѣваетъ истинная любовь, а съ любовію и истинное благочестіе умаляется, вмѣсто того лицемѣріе и ложь умножается.

\paragraph*{LXXXI.} Видишь яблоко, или цвѣтъ извнѣ красенъ, но внутрь ядовитымъ червемъ поврежденъ, который не только человѣку неполезенъ, но и вреденъ есть. Тако имѣется человѣческое дѣло: хотя извнѣ и доброе кажется быть, но когда отъ сердца, самолюбіемъ, тщеславіемъ и гордостію напоеннаго, происходитъ, не токмо ему не пользуетъ, но и вредитъ. Ибо таковый не отдаетъ славы Богу, Отъ Котораго все добро происходитъ; и что Богу \textit{единому} должно, тое онъ себѣ привлекаетъ; Божіи дарованія не къ Божіей славѣ, но къ своей злѣ употребляетъ, и тако на томъ мѣстѣ, на которомъ Бога поставлять долженъ, себе, какъ идола одушевленнаго, поставляетъ; а тако отъ Бога отпадаетъ и отступаетъ сердцемъ, и впадаетъ въ богомерзкій идолопоклонства духовнаго порокъ. Таковые суть, которые обильныя даютъ милостыни, созидаютъ Божіи храмы, богадѣльни, но оттуду славы и похвалы человѣческія ищутъ; которые людей учатъ и наставляютъ ради того, чтобъ мудрецами и разумными слыли, и проч. И сіе есть діавольская кознь и несмысленнаго и слѣпаго сердца самолюбіе. Что Богъ даетъ человѣку къ славѣ Своей и ближняго пользѣ, тое онъ, самолюбіемъ ослѣпленъ, въ собственную свою славу обращаетъ. Что бо человѣкъ имѣетъ свое собственное, кромѣ грѣха? Воистину ничего. Имѣніе, богатство, разумъ, мудрость наша и проч., Божія суть дарованія. Какъ убо не безстыдно Божіе добро къ своей обращати, а не къ Божіей славѣ? О колико согрѣшаемъ предъ Богомъ и въ мнимыхъ нашихъ добрыхъ *дѣлѣхъ*! Гдѣ мнимъ быти добро, тамо истинное зло крыется, такъ, что \textit{грѣхопаденія кто разумѣетъ}\footnote{Пс.~18,~13.}? Сей случай и разсужденіе учитъ тебе не искать въ добрыхъ дѣлахъ славы и похвалы своея, но \textit{Божіе} Богу отдавать, а не себѣ привлекать. Аще же ищешь того себѣ, то знай, что, какъ цвѣтокъ внѣ красенъ, но внутрь смраденъ, ни къ чему негоденъ: такъ дѣла твоя внѣ нѣчто быть показуются, но внутрь ничтоже суть, паче же мерзость суть предъ Богомъ, Который на сердце твое и намѣреніе смотритъ, а не на дѣла.

\paragraph*{LXXXII.} Видишь, что человѣкъ, имѣющій чужое, у себе положенное, богатство, не гордится тѣмъ, не называетъ себе богатымъ, не хвалится тѣмъ. Тако должно и намъ поступать, когда имѣемъ дарованія Божія; вмѣнять ихъ не какъ наша, но какъ чужая; не превозноситься тѣми, но во славу Божію, откуду происходятъ, и въ пользу нашу и ближняго, ради чего поручены онѣ намъ, употреблять. Богатство, разумъ, мудрость и прочее да будетъ намъ въ пользу нашу и ближняго, а не въ похвалу нашу, но \textit{единому} Тому, отъ Кого произошли, да будетъ слава и похвала, Который, какъ отъ милости Своей подалъ намъ, такъ ради неблагодарности нашей можетъ и отнять отъ насъ, яко Свое. Тогда подлинно узнаемъ, что мы бѣдны, нищи, убоги, наги и ничего своего собственнаго не имѣемъ, кромѣ грѣха.

\paragraph*{LXXXIII.} Видишь на древѣ сухую вѣтвь, которая ради того суха есть, что не имѣетъ сока, оживляющаго ее. Разумѣй, что тако имѣется хрістіанинъ, какъ вѣтвь изсохшая, который не имѣетъ живыя вѣры, любовію и прочіими плодами живность свою оказывающія. Таковый никогда участія не имѣетъ со Хрістомъ, \textit{иже есть лоза истинная}\footnote{Іоан.~15,~1.}, съ истинными хрістіанами, \textit{иже суть духовные уды Его}\footnote{Ефес.~5,~30.}; чуждъ надежды вѣчнаго живота, доколѣ тако пребываетъ. Таковому судомъ Своимъ грозитъ Богъ. \textit{Уже и сѣкира при корени древа лежитъ: всяко убо древо, еже не творитъ плода добра, посѣкаемо бываетъ, и во огнь вметаемо}\footnote{Матѳ.~3,~10.}. Сей случай и разсужденіе учитъ тя, хрістіанине, осматриваться, имѣешь ли вѣру, которая плодами своими, си есть, добрыми дѣлами, себе оказываетъ.

\paragraph*{LXXXIV.} Видишь, что когда сынъ ко отцу своему въ домъ возвратится изъ чужой страны, отецъ о сынѣ и сынъ о отцѣ радуются преизобильно, яко другъ съ другомъ видятся и взаимно утѣшаются. Отъ сего случая вѣрою возведи умъ твой къ дому Отца небеснаго, въ который истинные хрістіане, чада Божія, возвращаются изъ страны міра сего, въ которую грѣхъ ради нашихъ изгнаны мы. О! како возрадуется вѣрная душа, которая нынѣ любовію къ Богу, своему Отцу небесному, горитъ и со пророкомъ глаголетъ: \textit{когда пріиду и явлюся лицу Божію}? "--- когда, по бѣдственномъ въ мірѣ семъ, яко странѣ чуждей, странствованіи, возвратится въ домъ небеснаго Отца и явится лицу Божію, увидитъ Его \textit{лицемъ къ лицу}\footnote{1~Кор.~13,~12.}, \textit{узритъ Его, якоже есть}\footnote{1~Іоан.~3,~2.}, и поклонится престолу славы Его! Узритъ и Отецъ небесный возвратившагося изъ юдоли сей плачевной сына Своего, и воззритъ на него милосердыма Своима очима, и возрадуется о немъ со всею Своею небесною фамиліею, увеселитъ его благопріятнымъ лица Своего зрѣніемъ, упокоитъ его по трудахъ и подвигѣ, утѣшитъ по плачѣ и слезахъ. Сей случай научаетъ насъ \textit{горняя мудрствовать, а не земная} и вѣрою возводить сердце наше къ Богу, да и мы истинныя оныя радости сподобимся благодатію Искупителя нашего Іисуса Хріста.

\paragraph*{LXXXV.} Видишь два яблока извнѣ равно красны и пріятны, но внутрь неравны; но едино внутрь гнило и смрадно, другое равно какъ и извнѣ. Разумѣй, что тако имѣются и дѣла человѣческія. У многихъ дѣла внѣ равно похвальны показуются, но внутрь разнствуютъ; яко не отъ равнаго сердца и намѣренія происходятъ. Напр. единъ судія не пріемлетъ мзды и правду дѣлаетъ, и другій такожде отъ мздоиманія удаляется и праведно поступаетъ; и тако внѣшнія ихъ дѣла одинаково похвальны, но внутри могутъ быть неравны, когда единъ дѣлаетъ тое, боячись гнѣва Божія, который ради неправды послѣдуетъ, "--- другій же, опасаясь гражданскаго суда и страха человѣческаго, или стыдясь людей честныхъ, отъ зла удаляется; сей политикъ есть и лицемѣръ, а оный истинный хрістіанинъ. Единъ даетъ милостыню ради любви Божіей и ближняго, "--- другій ради самолюбія и тщеславія: сей тщеславецъ, и самолюбецъ, а оный истинный милостивецъ. Единъ у оскорбленнаго проситъ прощенія, жалѣя, что ближняго опечалилъ; и другій такожде проситъ прощенія у обиженнаго, но ради того проситъ, чтобъ на него не искалъ суда, и тако бы бѣды ему какой не учинилъ: сей самолюбіемъ недугуетъ, а оный истинный братолюбецъ есть. Единъ ласково и учтиво съ ближнимъ своимъ обходится, не хотя его оскорбить, "--- и другій такожде, но въ сердцѣ обмануть его думаетъ: оный истинный и чистосердечный есть любитель; сей лукавецъ и съ діаволомъ единаго нраву есть, который всячески тщится человѣка обмануть. Единъ плачется, что или богатство, или честь потерялъ, или не можетъ отмстить врагу своему; другій плачется, что Бога, Творца и Благодѣтеля своего, прогнѣвалъ: сего хрістіанская есть печаль, и спасительная; онаго міра сего есть печаль, и пагубная. Тако и въ прочемъ могутъ быть равны внѣ дѣла человѣческія, но внутрь, въ сердцѣ, неравны могутъ быть. Мы ихъ, по мнѣнію нашему, равными судимъ, яко на внѣшнее только смотримъ; но Богъ, Который глубину сердца зритъ и испытуетъ, иначе судитъ, якоже глаголетъ Господь къ Самуилу, когда пришелъ въ Виѳлеемъ помазати единаго отъ сыновъ Іессеовыхъ на царство, и хотѣлъ то учиняти надъ Еліавомъ: \textit{Не зри на лице его, ниже на возрастъ величества его, яко уничижихъ его. Понеже не тако зритъ человѣкъ, яко зритъ Богъ: яко человѣкъ зритъ на лице, Богъ же зритъ на сердце}\footnote{1~Цар.~16,~7.}. Сей случай и разсужденіе учитъ тя тому, дабы дѣла твоя и внутрь въ сердцѣ добрыми были, которыя внѣ похвальными показуются, и всякое бы дѣло внѣшнее доброе отъ сердца добраго происходило, когда хочешь Богу угождать.

\paragraph*{LXXXVI.} Видишь, что немощный, который хощетъ исцѣлитися, отдается въ волю лѣкарю, и написанныя его правила хранитъ, дабы не было препятствія дѣйствію лѣкарствъ. Мы вси отъ природы немощны духовно, паче же смертельно уязвлены отъ разбойника діавола, и на пути міра сего лежимъ уязвленны, и сами себѣ помощи никакъ не можемъ, якоже сіе показуется отъ притчи о впадшемъ въ разбойники и отъ нихъ уязвленномъ\footnote{Лук.~10,~30--35.}. Аще убо хощемъ исцѣлитися отъ язвъ грѣховныхъ, должны отдать себе въ волю милостивому оному Самарянину, не отъ Самаріи, но отъ Маріи Дѣвы Богородицы пришедшему, Іисусу Хрісту, да поступаетъ съ нами, якоже хощетъ; должны и регулы Его, въ святомъ Евангеліи написанныя, хранить, хотя плоти нашей и горестно быти кажется. Немощные все за благо пріемлютъ и терпятъ, что отъ лѣкарей предписано ни будетъ, когда хотятъ исцѣлитися: тако должно и намъ за благо пріимать и терпѣть все, что хощетъ и повелѣваетъ намъ Врачъ нашъ Хрістосъ. Иначе спасительное Его намъ врачевство ничего не поможетъ, какъ и немощному ничего не поможетъ врачеваніе лѣкарское, когда написанныхъ правилъ не хранитъ; паче въ горшее успѣвать будетъ немощь его. Горестно плоти нашей самолюбіе свое оставить и волѣ Хрістовой послѣдовать, но сего неотмѣнно требуетъ дѣло исцѣленія и спасенія нашего. Отъ сего заключается, что нѣтъ истиннаго обращенія и покаянія въ тѣхъ, которые не хотятъ самолюбія, любочестія, славолюбія и прочіихъ плотоугодій оставить, а тако остаются и пребываютъ духовно немощны, неисцѣленны, хотя и предлагается всѣмъ спасительное Хрістово врачевство.

\paragraph*{LХХХVII.} Видишь, что двѣ вещи всегда предъ глазами обращаются, \textit{тьма} въ нощи, \textit{свѣтъ} во дни; и отъ природы вси убѣгаемъ тьмы (никто бо не хощетъ во тьмѣ сидѣти), и ищемъ свѣта: тако двѣ вещи предъ душевными нашими глазами обращаться должны: \textit{грѣхъ}, тьма душевная; и \textit{Богъ}, Свѣтъ вѣчный. И какъ отъ тьмы чувственной уклоняемся къ свѣту чувственному, тако отъ тьмы грѣховной должны отвращатися и обращатися къ Свѣту, искать Его и просвѣщатися. Какъ бо удаляющіися отъ естественнаго свѣта во тьмѣ находятся, и претыкаются, и ничего не распознаютъ, и всего опасаются: тако удаляющіися отъ свѣта присносущнаго, Бога, тьмою душевною объяты суть, претыкаются и падаютъ отъ грѣха въ грѣхъ, и тако погибаютъ, когда не очувствуются, по писанію: \textit{се удаляющіи себе отъ Тебе погибнутъ}\footnote{Пс.~72,~27.}. Ибо отъ тьмы сея душевныя въ вѣчную и \textit{кромѣшнюю тьму}, гдѣ будетъ плачъ и скрежетъ зубомъ, яко неключимые раби, \textit{ввергнутся}\footnote{Матѳ.~25,~30.}, якоже ищущіи Бога и держащіися Его, яко сынове свѣта и дне, наслѣдятъ вѣчный свѣтъ; и который Свѣтъ нынѣ вѣрою зрятъ, тогда узрятъ \textit{лицемъ къ лицу}\footnote{1~Кор.~13,~12.}, и \textit{просвѣтятся, яко солнце во царствіи небесномъ}\footnote{Матѳ.~13,~43.}. Сіе разсужденіе учитъ тя уклоняться отъ грѣха, какъ уклоняешься отъ тьмы, и искать Бога, истиннаго и духовнаго Свѣта, всѣмъ сердцемъ, пока взыскуется и обрѣтается, якоже увѣщаваетъ насъ пророкъ: \textit{взыщите Господа и утвердитеся, взыщите лица Его выну}\footnote{Пс.~104,~4.}!

\paragraph*{LXXXVIII.} Видишь, что часы заведенные непрестанно идутъ, и хотя спимъ или бодрствуемъ, дѣлаемъ или не дѣлаемъ, непрестанное теченіе имѣютъ, и приближаются къ термину своему. Тако житіе наше имѣется: отъ рожденія до смерти непрестанно течетъ и убавляется. Упокоеваемся или трудимся, бодрствуемъ или спимъ, бесѣдуемъ съ кѣмъ или молчимъ, непрестанно теченіе свое совершаетъ, и къ концу своему приближается, и уже ближе учинилося къ концу нынѣ, нежели вчера и третіяго дня, "--- сего часа, нежели прошедшаго. Тако, нечувствительно намъ житіе наше сокращается! тако преходятъ часы и минуты! А когда окончится вервь и престанетъ ударять маятникъ, не знаемъ того. Промыслъ Божій скрылъ отъ насъ тое, да всегда готовы будемъ ко исходу, когда ни позоветъ насъ къ себѣ Владыка нашъ Богъ. \textit{Блаженъ тотъ, егоже Господь обрящетъ бдяща}\footnote{Лук.~12,~37.}! Окаяненъ, егоже въ грѣховномъ снѣ погружена обрящетъ! Сей случай и разсужденіе учитъ тя, хрістіанине: 1)~Что время житія нашего безпрестанно уходитъ. 2)~Что прошедшаго времени возвратить невозможно. 3)~Что прешедшее и будущее не наше, но только тое, которое теперь имѣемъ. 4)~Что кончина наша намъ неизвѣстна. 5)~Слѣдственно всегда, на всякій часъ, быть намъ готовыми къ исходу должно, когда хощемъ блаженно умереть. 6)~Отсюду заключается, что хрістіанинъ въ непрестанномъ покаяніи, подвигѣ вѣры и благочестія находиться долженъ. 7)~Каковымъ хощетъ быть при исходѣ, таковымъ быть тщиться долженъ на всякое время житія своего: яко не знаетъ, отъ утра дождется ли вечера, и отъ вечера дождется ли утра. Видимъ бо, что которые съ утра ходили здоровы, тѣ къ вечеру на одрѣ смертномъ бездыханны лежать; и которые съ вечера засыпаютъ, поутру не востаютъ и спать будутъ до трубы архангельской. А что другимъ случается, тое тебѣ и мнѣ случиться можетъ: вси бо всякимъ случаямъ подлежимъ.

\paragraph*{LXXXIX.} Видишь спящаго человѣка. Сей случай два образа представляетъ тебѣ: 1)~\textit{Подобіе смерти}. Какъ мертвый не видитъ, ни слышитъ, ни бесѣдуетъ и ничего не дѣйствуетъ, тако спящій "--- кромѣ того, что у спящаго живность дыханіемъ показуется. И сколько разъ засыпаемъ, столько разъ подобными мертвецу дѣлаемся. "--- 2)~Въ спящемъ видишь \textit{образъ будущаго живота}. Какъ бо спящій свобождается отъ всѣхъ дѣлъ, трудовъ, попеченій; забываетъ тогда бѣды, напасти, скорби, болѣзни и ничего отъ тѣхъ не чувствуетъ, такъ, какъ бы ихъ и не имѣлъ: тако сподобляющіися улучить вѣчную жизнь отъ всѣхъ трудовъ, бѣдъ, напастей и скорбей избавляются. Откуду вѣчный животъ въ святомъ Писаніи \textit{покой} называется\footnote{Евр.~4,~1,~3 и проч.}; и въ Апокалипсисѣ глаголется: \textit{блажени мертвіи, умирающіи о Господѣ отнынѣ! Ей, глаголетъ Духъ, да почіютъ отъ трудовъ своихъ}\footnote{14,~13.}. Сей случай научаетъ тя: 1)~Сколько разъ отходишь ко сну, столько разъ вспомнить о смерти, которой подобенъ сонъ, и вѣрою умъ свой возвести къ вѣчному животу, сладкому оному и некончаемому \textit{покою}, въ которомъ избранные Божіи упокоеваются, и разбойничій съ вѣрою испустить ко Хрісту гласъ: \textit{помяни мя, Господи, егда пріидеши во царствіи Твоемъ}\footnote{Лук.~23,~42.}! 2)~Воставши отъ сна, Богу благодарить. 3)~Въ семъ вѣцѣ трудитися въ подвигѣ вѣры, да по смерти упокоишися.

\paragraph*{ХС.} Видишь, что сосудъ исполненный ничего другаго не вмѣщаетъ. Тако имѣется сердце человѣческое, которое подобно есть сосуду. Когда оно исполнено будетъ любовію міра сего и попеченіемъ мірскихъ вещей, не вмѣщаетъ слова Божія въ себѣ и такъ безплодно бываетъ. Откуду печаль вѣка сего и лесть богатства уподобляется \textit{тернію}, которымъ подавляется сѣмя Божія слова и безъ плода бываетъ\footnote{Матѳ.~13,~22.}. Откуду читаемъ, что званніи на вечерю велію небеснаго царствія начали \textit{вкупѣ отрицатися вси. Первый рече: село купихъ, и имамъ нужду изыти и видѣти е: молютися, имѣй мя отреченна. И другій рече: супругъ воловныхъ купихъ пять: и гряду искусити ихъ: молю тя, имѣй мя отреченна. И другій рече: жену пояхъ, и сего ради не могу пріити}\footnote{Лук.~14,~18--20.}. Что не ино значитъ, какъ пристрастіе къ временнымъ и мірскимъ вещамъ, которымъ наполненное сердце не пріемлетъ и не вмѣщаетъ въ себѣ слова Божія. И тако человѣкъ, исполненный мірскими суетами, хотя и слышитъ проповѣдуемое и зовущее его Божіе слово, но не слушаетъ его, не повинуется ему, и такъ недостойна себе творитъ великія оныя вечери, то"=есть, царствія небеснаго, къ которой вечери раби Божіи, пророки и апостоли и проповѣдники зовутъ. Сей случай и разсужденіе учитъ тя берещися любви міра сего, которою плѣняется сердце человѣческое, и молити о семъ Бога съ Давидомъ: \textit{отврати очи мои, еже не видѣти суеты}\footnote{Пс.~118,~37.}! а когда запутался въ суетѣ той, тщаться всякимъ образомъ отъ ней избыть благодатію Божіею, да не и ты съ отрекшимися явишися недостоинъ вечери оныя пресладкія. \textit{Мнози бо суть звани, мало же избранныхъ}, заключаетъ Господь притчу оную\footnote{Лук.~14,~24.}.

\paragraph*{ХСІ.} Видишь, что и праздный сосудъ ничего не можетъ принять въ себе, пока не открыется. Тако и сердце человѣческое, хотя и испраздненное будетъ отъ суеты міра сего, но связанное жестокостію, небреженіемъ, уныніемъ, не можетъ принять слова Божія. Откуду глаголетъ Хрістосъ, Ѵпостасное Божіе Слово: \textit{Се стою при дверехъ, и толку; аще кто услышитъ гласъ Мой, и отверзетъ двери, вниду къ нему, и вечеряю съ нимъ, и той со Мною}\footnote{Апок.~3,~20.}. Всегда Его божественный и пресладкій гласъ ударяетъ въ сердца наша чрезъ слово святое, но не всякъ слышитъ его; не слышитъ же ради того, что ушесъ сердечныхъ не имѣетъ отверстыхъ, о которыхъ Онъ глаголетъ: \textit{имѣяй уши слышати, да слышитъ}\footnote{Матѳ.~13,~9 и 45.}. Должно убо хотящему слово Божіе въ сердце свое принять, и отъ того плодъ сотворити, открыти сердце свое, и, якоже изсохшая земля жаждетъ дождя, да сотворитъ плодъ, тако сердцу нашему отворить уста своя, и жаждать Божія слова, то есть, ненасытно хотѣть и желать слово Божіе слышать, не ради иной какой причины, но того ради, чтобы душа духовное пріобрѣтала созиданіе. И якоже повседневно желаетъ стомахъ нашъ пищи и питія; тако подобаетъ сердцу нашему, какъ духовному стомаху, всегда желать Божія слова, которое есть душевная пища. О семъ Давидъ святый глаголетъ: \textit{уста моя отверзохъ, и привлекохъ духъ, яко заповѣдей Твоихъ желахъ}\footnote{Пс.~118,~131.}. И о семъ должно Самаго Бога молить, чтобы отверзлъ уста сердца нашего, и подалъ намъ сію спасительную алчбу и жажду.

\paragraph*{ХСІІ.} Видишь, что кто чего усердно желаетъ, тотъ того съ прилѣжаніемъ и ищетъ. Напримѣръ: кто желаетъ чести или богатство міра сего получить, тотъ о томъ и старается, тое въ сердцѣ имѣетъ и думаетъ о томъ. Такъ кто желаетъ благочестиваго хрістіанскаго житія, о томъ и попеченіе имѣетъ, поучается день и нощь въ словѣ Божіемъ, и молится Богу о томъ. Ибо благочестиваго житія зерцало есть святое Божіе слово, въ которомъ представляется истинное благочестіе и нечестіе, путь нечестивыхъ и благочестивыхъ. Симъ научаешься испытовать себе, желаешь ли ты усердно \textit{о Хрістѣ Іисусѣ благочестно жити}, или нѣтъ въ тебѣ сего спасительнаго желанія. Когда нѣтъ того и не тщишися о томъ, то непремѣнно никакого благочестія въ сердцѣ своемъ не имѣеши, и находишися внѣ числа вѣрныхъ, не имѣеши никакого участія со Хрістомъ, и имя хрістіанское ничего не поможетъ.

\paragraph*{ХСІІІ.} Видишь, что хотящіи дойти до нѣкоего града, или другаго какого мѣста, не стоятъ, но идутъ, и идутъ тѣмъ путемъ, который ведетъ до того мѣста: тако хотящіи дойти до града небеснаго, горняго Іерусалима, не стоятъ, но идутъ, и идутъ тѣмъ путемъ, который до того града ведетъ, то"=есть, путемъ благочестивымъ, на которомъ пути Вождь есть Хрістосъ, Которому послѣдуютъ вѣрою и любовію вѣрніи Его раби. Ради чего же и ты, грѣшниче неисправный, когда хочешь спастися и до отечества онаго небеснаго дойти, стоиши и не послѣдуешь Ему; или, что горше того, съ того благочестиваго пути сошедъ и ставши на пути нечестивыхъ, идешь по нему, который до ада и вѣчной погибели ведетъ? "--- Какой"=де путь благочестивыхъ и нечестивыхъ? "--- Господь нашъ указуетъ тебѣ во Евангеліи Своемъ, велитъ сего уклониться, а тѣмъ ити. \textit{Внидите узкими враты: яко пространная врата и широкій путь вводяй въ пагубу, и мнози суть входящіи имъ. Что узкая врата и тѣсный путь вводяй въ животъ, и мало ихъ есть, иже обрѣтаютъ его}\footnote{Матѳ.~7,~13 и 14.}. Узкій путь есть путь смиренія, послушанія, терпѣнія, кротости, и самого себе, то есть, воли своея отверженіе: широкій путь есть путь гордости, непослушанія, нетерпѣнія и самолюбія. Сей къ аду, оный къ небу ведетъ. Избирай убо "--- которымъ хочешь ити путемъ, или узкимъ и низкимъ къ небу, или широкимъ и высокимъ ко аду, "--- а третьяго въ святомъ Писаніи не видимъ. Понеже и въ будущемъ вѣкѣ, куды вси идемъ, только два мѣста видимъ открытыя отъ Бога: вѣчный животъ, который будетъ въ небѣ, и вѣчная мука, которая будетъ во адѣ. Сіе ради того, любезный читатель, предлагается, что многіи, паче же вси хотятъ спастися, но немногіи дѣлаютъ тое, чего спасенія дѣло отъ насъ требуетъ. Самые душегубцы, хищники, татіе, блудники, грабители, лукавцы, лживые, піяницы, клеветники, хульники, ругатели и прочіи хотятъ спастися; никто отъ нихъ не отречется спасенія: но, понеже не идутъ тѣмъ путемъ, который къ вѣчному спасенію ведетъ, а идутъ путемъ погибельнымъ, который до ада приводитъ, то и хотѣнія къ спасенію дѣйствительнаго въ нихъ нѣтъ. Изрядно и твердо философы научаютъ: \textit{кто хощетъ конецъ постигнуть, тотъ и посредствія къ тому концу приличнаго хощетъ и ищетъ}. Напр. хощешь, какъ выше сказано, дойти до какого града, ищешь того пути, который къ тому граду ведетъ; хощешь въ немощи здравіе получить, призываешь лѣкаря, и проч. Кто бо чего дѣйствительно и усердно желаетъ, и какъ выше такожде сказано, тотъ усердно и дѣйствительно того и ищетъ. Тако кто хощетъ усердно и дѣйствительно спастися, тотъ и посредствія къ спасенію ищетъ. Посредствіе къ спасенію вѣчному едино представляется намъ въ святомъ словѣ "--- вѣра святая, евангельская. Но вѣра тая должна быть живая, а не мертвая. Живность же вѣры познается отъ дѣлъ добрыхъ. Какъ бо вѣра сія есть великое добро и сокровище въ человѣцѣ, такъ, яко добро, добрые и плоды раждаетъ, якоже древо доброе плоды добрые творитъ. Отсюду заключается, что гдѣ нѣтъ добрыхъ дѣлъ, но вмѣсто того злыя, тамо нѣтъ и вѣры истинныя, но прелестная и ложная. Таковую ложную вѣру имѣющіи подобно обманываются, какъ и тѣ, которые въ ночи идутъ, и думаютъ, что они тѣмъ путемъ идутъ, который къ намѣренному мѣсту ведетъ, но, какъ возсіяетъ денница, узнаютъ свое заблужденіе, и жалѣютъ: такъ и ложную вѣру, мертвую и безплодную имѣющіи думаютъ, что они хрістіане суть, путемъ благочестивымъ и спасительнымъ идутъ, но послѣ познаютъ свое заблужденіе, и каются. И блажени суть, которымъ еще на пути денница благодати сея возсіяетъ въ сердцахъ; блажени, которые прежде конца своего усматриваютъ свое заблужденіе, и такъ каются. Таковое бо познаніе есть начало спасенія: ибо тако могутъ совратитися съ пути погибельнаго, и взыти вѣрою на путь спасенія, который въ вѣчный животъ ведетъ.

\paragraph*{XСIV.} Видишь, что хотящіи смотрѣти на солнце обращаются лицемъ къ солнцу, что и о всякой вещи разумѣется. Напр. человѣка ли, или какую другую вещь хощемъ видѣти, къ тому лицемъ своимъ обращаемся. Тако хотящіи видѣти Солнце вѣчное и духовное "--- Бога, видѣти нынѣ вѣрою, въ ономъ вѣкѣ лицемъ къ лицу, должны неотмѣнно обратитися къ Нему лицемъ сердца своего. Богъ на всякомъ мѣстѣ есть, и потому, гдѣ ни ходимъ, предъ лицемъ Его ходимъ. Но люди, когда не слушаютъ Божіихъ повелѣній и беззаконнуютъ, "--- отвращаются отъ Него, и какъ бы, вмѣсто лица, хребетъ Ему обращаютъ, якоже глаголетъ Богъ о беззаконникахъ: \textit{обратиша хребетъ ко Мнѣ, а не лице}\footnote{Іер.~32,~33.}; и во Псалмахъ читаемъ: \textit{не предложиша Бога предъ собою}\footnote{Пс.~53,~5.}; и паки: \textit{не предложиша, Тебе}, Боже, \textit{предъ собою}\footnote{85,~14.}, что не иное значитъ, какъ сердце отъ Бога отвратившееся, Бога не слушающее, не почитающее и презирающее. Какъ бо, хотя кого почтить, обращаемся къ нему лицемъ, а отвращеніемъ лица непочтеніе и презрѣніе ему показуемъ: такъ, отвращаяся отъ послушанія къ Богу, великое Ему непочтеніе и презрѣніе являемъ, хотя того и не примѣчаемъ, ослѣпленніи; тако бо, вмѣсто лица, хребетъ Ему обращаемъ. Но когда человѣкъ, познавъ свою беззаконную сію грубость, которую Богу и Создателю своему показывалъ, очувствуется и, преставъ отъ грѣховъ, кается и сердечно жалѣетъ, что такъ безчинно и беззаконно противу Создателя своего и Господа поступалъ, тогда обращается къ Богу; и когда въ томъ стоитъ, и послушанія плоды тщится Ему показывать, тогда какъ бы лицемъ предъ Нимъ стоитъ, и что Онъ въ словѣ святомъ повелѣваетъ, слушаетъ и тщится исполнять. Тако обратившійся и здѣ \textit{вѣрою}, и въ будущемъ вѣкѣ \textit{лицемъ къ лицу} сподобится видѣти Бога. "--- Отъ сего видно: 1)~Какъ беззаконно и безчинно дѣлаемъ, когда отъ произволенія грѣшимъ. Тако бо, вмѣсто лица, хребетъ Богу обращаемъ, что знакъ есть презрѣнія великаго и страшнаго Бога. 2)~Какъ страшное и богомерзкое есть заблужденіе "--- нераскаянное и безбожное житіе! Тако бо люди, какъ бы обративше хребетъ къ Богу, а не лице, ходятъ и беззаконнуютъ, "--- что страшно и писать и думать. Но воистинну тако есть, якоже Богъ самъ чрезъ пророка глаголетъ: \textit{обратиша хребетъ ко Мнѣ, а не лице}. 3)~Какъ великая есть Божія благость ко грѣшникамъ, которая такое презрѣніе и непочтеніе грѣшникамъ терпитъ и ожидаетъ ихъ покаянія! 4)~Покаяніе истинное есть премѣненіе сердца и нравовъ злыхъ въ добрые нравы. 5)~Истинное покаяніе непремѣнно стыдѣніе и жалѣніе въ сердцѣ кающагося содѣловаетъ, что онъ такъ безстыдно и безчинно съ Богомъ поступалъ. 6)~Истинное покаяніе быть не можетъ, когда человѣкъ отъ грѣховъ отстать и Богу послушаніе показывать не хощетъ, но есть прелестное и ложное: того ради тако кающемуся ничего и не пользуетъ. Каяться бо и грѣшить отъ произволенія суть противная себѣ, и купно быть не могутъ.

\paragraph*{XСV.} Видишь восхожденіе солнца и начинающійся день. Здѣ имѣешь довольную матерію помыслить о благости Божіей, и удивиться ей. 1)~Когда зайдетъ солнце подъ землю и наступитъ нощь, такъ кажется, что какъ бы его уже и нѣтъ: когда восходитъ паки, аки вновь отъ Создателя создалося, и тѣмъ приводитъ тебѣ на память великое сіе свѣтило свое и твое отъ Создателя бытіе, что какъ оно, такъ и ты отъ Него созданъ еси. "--- 2)~Приводитъ тебѣ на память Солнце праведное, Хріста Сына Божія, умершаго и воскресшаго изъ мертвыхъ. Какъ бо солнце сіе чувственное заходитъ и востаетъ: такъ Хрістосъ, явившися на земли и совершивши великое спасенія нашего дѣло, зашелъ во гробъ, и паки восталъ. И какъ солнце, воставши, всю поднебесную просвѣщаетъ, всѣхъ радостотворитъ и согрѣваетъ: тако Хрістосъ, воставши отъ мертвыхъ, всю вселенную просвѣтилъ, и свѣтъ Свой чрезъ апостолы, какъ лучи духовные, распустилъ, всѣхъ сердца согрѣлъ и возвеселилъ. "--- 3)~Показуетъ тебѣ имѣющее быть воскресеніе мертвыхъ. Какъ бо солнце, зашедши подъ землю, паки востаетъ: тако умершихъ тѣлеса, въ земли погребенная, востанутъ, и \textit{праведницы просвѣтятся, яко солнце, во царствіи Отца ихъ}\footnote{Матѳ.~13,~43.}. "--- 4)~Восходитъ солнце, и обновляется утро: обновляются и щедроты Божіи, и изливаются на насъ; благость и милость Божія есть, что мы не погибли. Восходитъ солнце, и просвѣщаетъ и согрѣваетъ все; является и благость Божія, праведника радостотворитъ, веселитъ, просвѣщаетъ, влечетъ и согрѣваетъ; грѣшника лучами любве Своея ударяетъ и \textit{ведетъ на покаяніе}\footnote{Римл.~2,~4.}; и \textit{просвѣщаетъ всякаго человѣка грядущаго въ міръ}\footnote{Іоан.~1,~9.}, отверзаетъ руку Свою, и \textit{исполняетъ всякое животно благоволенія}\footnote{Пс.~144,~16.}. Отъ сего случая учимся, любезный хрістіанине, Богу благодарить отъ сердца за три знатныя Его къ намъ благодѣянія "--- за созданіе, искупленіе и дивные Его о насъ промыслы, и глаголати Ему радостнымъ духомъ: Слава Тебѣ, Боже, создавшему насъ! слава Тебѣ, искупившему насъ! Слава Тебѣ, промышляющему о насъ, недостойныхъ! Благословенъ еси во вѣки, аминь!

\paragraph*{XСVI.} При восходѣ солнца исходятъ люди на дѣло свое, якоже поетъ Псаломникъ: \textit{изыдетъ человѣкъ на дѣло свое, и на дѣланіе свое до вечера}\footnote{Пс.~103,~23.}. Симъ научаются и хрістіане исходить на дѣла своя, Богу угодная, \textit{отлагать дѣла темная, и облекаться во оружіе свѣта, творить дѣла благая}\footnote{Рим.~13,~12.}: яко \textit{Того есмы твореніе, создани о Хрістѣ Іисусѣ на дѣла благая, яже прежде уготова Богъ, да въ нихъ ходимъ}\footnote{Еф.~2,~10.}.

\paragraph*{ХСVІІ.} Видишь, что дѣлается, когда солнце отъ насъ отыдетъ: наступаетъ тьма и покрываетъ всю поднебесную; человѣкъ ничего не можетъ дѣлати, осязаетъ какъ слѣпый, блудитъ, спотыкается и падаетъ и всего опасается. Тако дѣлается съ душею человѣческою, которую Хрістосъ, Солнце праведное, оставляетъ; никакихъ богоугодныхъ дѣлъ не можетъ творити, якоже Самъ рече: \textit{безъ Мене не можете творити ничесоже}\footnote{Іоан.~15,~5.}. Понеже покрываетъ ее тьма духовная, и тако заблуждаетъ, осязаетъ какъ слѣпый, спотыкается и падаетъ отъ грѣха въ грѣхъ, пока въ ровъ вѣчныя погибели не впадетъ, аще покаяніемъ не исправится. Оставляетъ же Хрістосъ тую душу, которая сама прежде оставитъ Его: иначе Хрістосъ никого не оставляетъ, яко \textit{просвѣщаетъ всякаго человѣка грядущаго въ міръ}\footnote{1,~9.}. Сей случай научаетъ насъ держаться Хріста и послѣдовать Ему вѣрою и любовію, да не и насъ духовная тьма и бѣдствіе тое обыметъ.

\paragraph*{ХСVІІІ.} Видишь, что человѣкъ, получившій благодѣяніе отъ кого, хотя и показываетъ учтивость предъ нимъ и тако предъ глазами показуется благодарить благодѣтелю своему, но въ сердцѣ истинныя благодарности и любви къ нему не имѣетъ. Учтивость тая его не ино что есть какъ лицемѣрство, что всякъ признать можетъ. Тако имѣются лживые хрістіане: мнятся Богу благодарить, когда въ церковь ходятъ, поютъ и хвалятъ Его; но, понеже страха Его не имѣютъ, и не любятъ Его (что показуется отъ злыхъ дѣлъ ихъ и преступленія святыхъ Его заповѣдей, что съ страхомъ Божіимъ и любовію помѣститься не можетъ), то хожденіе въ церковь, пѣсни и славословія ихъ не ино что суть какъ лицемѣрство, которое Богъ \textit{сердца и утробы испытуяй}, ясно видитъ, и тако пѣсней, славословія и молитвъ ихъ не пріемлетъ и отвращается, пока не исправятъ сердца своего. Таковые хрістіане подобны суть онымъ Израильтянамъ, о которыхъ написалъ Псаломникъ: \textit{возлюбиша Его усты своими, и языкомъ своимъ солгаша Ему; сердце же ихъ не бѣ право съ Нимъ}\footnote{Пс.~77,~36 и 37.}; и Богъ чрезъ пророка глаголетъ: \textit{приближаются Мнѣ людіе сіи усты своими, и устнами своими чтутъ Мя; сердце же ихъ далече отстоитъ отъ Мене}\footnote{Ис.~29,~13.}. А тако поступая, никакой Богу благодарности и почитанія не показуютъ. Сіе научаетъ тя искать благочестія внутрь, въ сердцѣ, а не въ наружности, и почитать Бога не токмо устами и церемоніями, но и сердцемъ, "--- что бываетъ вѣрою, страхомъ Божіимъ, любовію къ Богу и ближнему, безъ чего истинное богопочитаніе быть не можетъ.

\paragraph*{ХСІХ.} Знаеши ли, хрістіанине, что дѣлаютъ мореплаватели во время великія бури? Тогда они, не имѣя надежды къ спасенію, бросаютъ котву или якорь во глубину морскую, и утверждаютъ его въ землѣ, дабы тако могли корабль и себе съ кораблемъ спасти, "--- чимъ и спасаются отъ потопленія. Тако подобаетъ и хрістіанамъ, которые на морѣ міра сего въ кораблѣ церкви святыя плаваютъ, дѣлати, и подражать корабельщикамъ. Когда сатана воздвигаетъ на нихъ бурю искушеній, бѣдъ и напастей, должны всю человѣческую надежду оставлять и прибѣгать къ Богу, сердца свои колеблющіяся и бѣдствующія утверждать въ любви Божіей. Отъ любви ли Божія хощетъ ихъ врагъ отторгнуть и потопить во глубинѣ грѣховъ и законопреступленій? въ любви Божіей утверждать кораблецъ сердца своего должны. Въ отчаяніе ли хощетъ вринуть? въ той же любви котву упованія своего и спасенія укрѣплять должны, памятуя написанное: \textit{тако возлюби Богъ міръ, яко и Сына Своего Единороднаго далъ есть, да всякъ, вѣруяй въ Онь, не погибнетъ, но имать животъ вѣчный; не посла бо Богъ Сына Своего въ міръ, да судитъ мірови, но да спасется Имъ міръ}\footnote{Іоан.~3,~16 и 17.}, "--- и милостивыхъ Божіихъ обѣщаній, клятвою Его въ укрѣпленіе колеблющагося сердца нашего утвержденныхъ держася: \textit{живу Азъ, глаголетъ Адонаи Господь, яко не хощу смерти грѣшника, но еже обратитися нечестивому отъ пути своего, и живу быти ему}\footnote{Іез.~33,~11.}; и паки: \textit{аминь аминь глаголю вамъ, яко слушаяй словесе Моего, и вѣруяй пославшему Мя, имать животъ вѣчный, и на судъ не пріидетъ, но прейдетъ отъ смерти въ животъ}\footnote{Іоан.~5,~24.}; и паки: \textit{аминь аминь глаголю вамъ: вѣруяй въ Мя, имать животъ вѣчный}\footnote{6,~47.}; и паки: \textit{аминь аминь глаголю вамъ: аще кто слово Мое соблюдетъ, смерти не имать видѣти во вѣки}\footnote{8,~51.}. Сими \textit{двѣма вещьми непреложными, въ нихже невозможно солгати Богу, крѣпкое утѣшеніе да имамы прибѣгшіи ятися за предлежащее упованіе. Еже аки котву имамы души тверду же и извѣстну, и входящую во внутреннее завѣсы}\footnote{Евр.~6,~18 и 19.}. Аще убо на тя, возлюбленный хрістіанине, востаетъ буря бѣсовскаго искушенія, подражай въ семъ дѣлѣ мореходцамъ, и во время напасти колеблющееся сердце утверждай въ любви Божіей, еюже возлюби насъ въ возлюбленномъ Сынѣ Своемъ: сія любы да содержитъ насъ при любви Божіей, когда насъ смущаетъ сатана любовію міра сего, и отторгнуть отъ любви Божіей, привлещи же къ любви похоти плотской, похоти очесъ и гордости житейской замышляетъ. Сія любы да содержитъ и утвердитъ насъ въ надеждѣ, когда тойжде врагъ начнетъ колебать сердце наше бурею отчаянія, и чрезъ помыслы злые глаголати намъ: \textit{нѣсть спасенія тебѣ въ Бозѣ твоемъ}.

\paragraph*{С.} Видишь овцу, отлучившуюся отъ стада своего, которая дотолѣ бѣгаетъ, кричитъ и ищетъ стада, доколѣ паки сыщетъ его. Тако подобаетъ хрістіанину дѣлать, который нерадѣніемъ и непослушаніемъ отлучился отъ стада овецъ Хрістовыхъ: дотолѣ искать, просить, толкать и плакать, доколѣ паки присоединится оному; просить и молить Бога съ Давидомъ: \textit{заблудихъ, яко овча погибшее: взыщи раба Твоего}\footnote{Пс.~118,~176.}! Отлучается же отъ стада Хрістова всякъ законопреступникъ, блудникъ, прелюбодѣй, хищникъ, воръ, лихоимецъ, процентщикъ, досадитель, лукавецъ, клеветникъ, піяница, сластолюбецъ, злорѣчивый, злобный, и проч., есть заблуждшая овца, отлучившаяся отъ стада Хрістова. Ибо овцы Хрістовы гласа Хрістова слушаютъ, якоже Самъ глаголетъ: \textit{овцы Моя гласа Моего слушаютъ}\footnote{Іоан.~10,~3.}. И хотя таковые мнятся быти въ стадѣ Хрістовомъ, однакожъ самою вещію чужди суть его, чужди *Хріста Самаго, чужди* живота вѣчнаго, пока истинно не обратятся и не присоединятся ему. Сей случай и разсужденіе увѣщаваетъ тя осмотрѣться, не отлучился ли и ты отъ Хрістова стада своимъ небреженіемъ; и когда примѣтишь въ совѣсти твоей сіе бѣдствіе, не медли обратитися покаяніемъ, плачемъ и воздыханіемъ, пока еще время не ушло, и отверсты двери въ благословенный овецъ Хрістовыхъ дворъ, въ который покаяніемъ истиннымъ и вѣрою входимъ.

\paragraph*{СІ.} Видишь, что юноша и дѣва въ супружество, или человѣкъ съ человѣкомъ въ дружество вступая, постановляютъ между собою завѣтъ и обѣщаются другъ другу вѣрность хранить до смерти. При семъ случаѣ помяни о завѣтѣ томъ, который постановленъ между Богомъ и тобою въ святомъ крещеніи. Тогда Богъ милостивый принялъ тебе въ высочайшую Свою милость вѣрою о Хрістѣ; и ты Ему обѣщался вѣрою и правдою служить до кончины жизни твоея. Отъ сего случая можешь: 1)~Помыслить и удивиться неизреченной благости Божіей, что человѣка отверженнаго, законопреступника и врага Своего, въ такъ высокую милость пріемлетъ, что о Хрістѣ, Господѣ нашемъ, сыномъ Своимъ дѣлаетъ и вѣчнаго царствія наслѣдникомъ его означаетъ о томжде Хрістѣ, когда вѣру до конца соблюдетъ, якоже глаголетъ Апостолъ: \textit{вси вы сынове Божіи есте вѣрою о Хрістѣ Іисусѣ. Елицы бо во Хріста крестистеся, во Хріста облекостеся}\footnote{Гал.~3,~26 и 27.}. 2)~Осмотрѣться, вѣрно ли ты работаешь Господу твоему, пріявшему тебе въ милость Свою? Онъ вѣренъ есть и всегда пребываетъ: \textit{вѣренъ Господь во всѣхъ словесѣхъ Своихъ и преподобенъ во всѣхъ дѣлѣхъ Своихъ}\footnote{Пc.~144,~13.}. Обѣщалъ вѣрныхъ Своихъ въ любви и милости содержать, и содержитъ; обѣщалъ царствіе небесное имъ подать, и подаетъ. Ты сохранилъ ли, и храниши ли Ему вѣрность свою? Не оставилъ ли ты Его, хотя имя Его исповѣдуеши? Не отвратилъ ли сердца своего отъ Него, и не прилѣпился ли къ любви міра сего? Не любиши ли богатства, злата, сребра, чести, славы, сласти міра сего вмѣсто Господа твоего? Показуеши ли Ему послушаніе, котораго Онъ отъ тебе требуетъ? \textit{Оправдался ты} въ святомъ крещеніи именемъ Господа нашего Іисуса Хріста и Духомъ Бога нашего\footnote{1~Кор.~6,~11.}: показуеши ли того оправданія плоды? дѣлаешь ли правду? \textit{Освятился тѣмже великимъ именемъ и Духомъ}: храниши ли себе отъ всякія скверны плоти и духа? Когда примѣтишь въ себѣ невѣрность къ Богу, обратися всѣмъ сердцемъ къ Нему съ покаяніемъ, жалѣніемъ и вѣрою, пока время не ушло; ибо обращающихся пріемлетъ Богъ: да не во вѣки отлучишися отъ Него къ крайнему бѣдствію своему. Аще же обратишися къ Нему съ истиннымъ жалѣніемъ и вѣрою, обратится и Онъ къ тебѣ съ милостію Своею, якоже глаголетъ чрезъ пророка: \textit{обратитеся ко Мнѣ, глаголетъ Господь силъ, и обращуся къ вамъ}\footnote{Зах.~1,~3.}. Самая бо животворящая кровь Хрістова, за грѣшниковъ изліянная, вопіетъ къ Богу за таковаго грѣшника, который всѣмъ сердцемъ отъ грѣховъ отвращается и обращается къ Богу; якоже нераскаяннымъ ничего не пользуетъ, но паче въ большее имъ осужденіе будетъ, яко такъ высокую и великую благодать презрѣли.

\paragraph*{СІІ.} Видишь, что всякое сѣмя такой плодъ раждаетъ, каково само есть: отъ ржи рожь, отъ пшеницы пшеница, отъ ячменя ячмень родится. Такожде и въ животныхъ: отъ человѣка человѣкъ, отъ волка волкъ, отъ медвѣдя медвѣдь, отъ овцы овца, отъ гуся гусь раждается. И неотмѣнно въ плодѣ, отъ сѣмени рожденномъ, подобіе показуется; напримѣръ: волкъ подобнаго себѣ волка въ хищеніи, лисица подобнаго себѣ лиса въ лукавствѣ, свинія подобную себѣ свинію въ обжирствѣ и нечистотѣ раждаетъ. Тако имѣется и духовное рожденіе, которымъ хрістіане раждаются свыше. Раждаются отъ Бога, и въ нихъ неотмѣнно долженъ плодъ показываться подобенъ Богу, отъ Котораго духовно родилися; должны нравами своими подобитися Отцу своему небесному, отъ Котораго водою и Духомъ родилися: \textit{быть подражателями Богу, якоже чада возлюбленная}\footnote{Еф.~5,~1.}; святыми быть, якоже Онъ святъ есть, якоже глаголетъ Самъ: \textit{святи будите, яко Азъ святъ есмь}\footnote{1~Петр.~1,~16.}; быть милосердыми, любительными, терпѣливыми, кроткими, праведными, и проч. Ибо отъ святаго, праведнаго, милосердаго, терпѣливаго, кроткаго Бога таковыя и чада раждаются. Нравы бо отцовскіе въ дѣтей преходятъ. Тако въ плотскомъ рожденіи раждаемся отъ грѣшника Адама грѣшныя чада; отъ грѣхолюбиваго, гордаго, немилостиваго, нечистаго, неправеднаго, злобнаго, гнѣвливаго, таковыя и дѣти раждаемся; дѣти отца по плоти нравами изобразуемъ. Такъ кто вновь отродился отъ Бога, тотъ Бога, яко отца чадо, нравами добрыми изображаетъ: \textit{яко сѣмя Его} (то есть благодать) \textit{въ немъ пребываетъ}\footnote{1~Іоан.~3,~9.}, "--- творитъ новаго рожденія плоды, плоды правды, святыни, любве, милосердія, терпѣнія, кротости и проч., и тако \textit{есть о Хрістѣ нова тварь}\footnote{2~Кор.~5,~17.}. Отъ сего видно, что хрістіане, беззаконно живущіи, въ ветхомъ рожденіи пребываютъ; и хотя крещены, но крещенія святаго даръ потеряли, и тако пребываютъ удаленные отъ Хріста, хотя имя Его и исповѣдуютъ; ибо \textit{дѣлами отмещутся Его}\footnote{Тит.~1,~16.}, "--- пребываютъ безъ надежды вѣчнаго живота, пока въ такомъ состояніи находятся. Ибо новое рожденіе два знатнѣйшіе плода приноситъ: первый "--- \textit{отпущеніе грѣховъ}, якоже глаголетъ Апостолъ: \textit{омыстеся, освятистеся, оправдастеся именемъ Господа нашего Іисуса Хріста, и Духомъ Бога нашего}\footnote{1~Кор.~6,~11.}; вторый "--- \textit{обновленіе сердца и духа} къ послушанію, къ творенію добрыхъ дѣлъ: яко отрожденный съ охотою и усердіемъ тщится показывать послушаніе Богу, подвизается противу грѣха, не попускаетъ ему царствовати надъ собою, плоть распинаетъ со страстьми и похотьми, яко Хрістосъ. Якоже бо отъ зла зміина сѣмене, которое излиялъ въ сердца прародителей нашихъ (и въ плотскомъ рожденіи вси мы съ тѣмжде раждаемся ядомъ), злые плоды происходятъ "--- сребролюбіе, славолюбіе, самолюбіе, любосластіе и тѣмъ послѣдующее бѣдствіе, клятва и вѣчное осужденіе, когда человѣкъ вновь не отродится свыше: тако отъ Божія духовнаго сѣмене, которое зачинается и дѣйствуется Божіимъ словомъ чрезъ Духа Святаго и въ духовномъ новомъ рожденіи отрождаетъ человѣка, то"=есть изъ злаго доброхотнымъ, и изъ сына гнѣва сыномъ благословенія Божія дѣлаетъ, послѣдуютъ духовные плоды, то"=есть, плоды добронравія. Отъ подобнаго бо подобное раждается, и злое сѣмя не можетъ родити, какъ злый плодъ, и доброе сѣмя не раждаетъ, какъ добрый плодъ. Каково сѣмя, отъ котораго древо зачалося и возрасло, таковый отъ него и плодъ раждается, якоже глаголетъ Хрістосъ: \textit{не можетъ древо добро плоды злы творити, ни древо зло плоды добры творити}\footnote{Матѳ.~7,~18.}. Отъ сего заключается правильно, что нѣтъ въ томъ хрістіанинѣ хрістіанства, вѣры, покаянія, благочестія, въ которомъ нѣтъ плодовъ хрістіанства, плодовъ вѣры, покаянія и благочестія; но хрістіанство его, вѣра, покаяніе и благочестіе ложное, прелестное и мечтательное, и въ самой вещи языческое состояніе, или паче скотское и звѣриное. Какіе бо нравы въ комъ являются, таково и состояніе внутрь сердца его есть. Какое въ томъ хрістіанство, который хищеніемъ показуется какъ волкъ, хитростію какъ лисъ, гордый какъ павлинъ, обжирливый какъ свинія, похотливый какъ конь, ядовитый какъ ехидна, лютый какъ левъ? Нѣтъ добронравія: убо нѣтъ и кореня добронравія. Нѣтъ плодовъ добрыхъ: убо нѣтъ сѣмени добраго. Нѣтъ хрістіанскихъ дѣлъ: убо нѣтъ и хрістіанства. Нѣтъ плодовъ достойныхъ покаянія: убо нѣтъ и покаянія. Симъ случаемъ и разсужденіемъ научаешися, хрістіанине, испытовать себе, находишися ли ты въ вѣрѣ и числѣ вѣрныхъ? Изъ сего можешь познать, что многіи, въ нынѣшнемъ наипаче вѣкѣ, подъ именемъ хрістіанъ язычниками имѣются, какъ подъ овчіими кожами волки.

\paragraph*{СІІІ.} Видишь приговореннаго и осужденнаго на смерть, или больнаго при смерти. Разсуждай и смотри, что онъ тогда дѣлаетъ? Нѣтъ попеченія о богатствѣ, чести, славѣ; не ищетъ ни на кого суда, всѣмъ прощаетъ, чимъ ни обижденъ; не помышляетъ о роскоши и ни о чемъ, къ міру сему надлежащемъ. Единая смерть предъ душевными его глазами обращается, и тоя страхъ сердце его колеблетъ. Сей случай, любезный хрістіанине, насъ увѣщаваетъ о нашей смерти. Что на братіи нашей видимъ, тое себѣ съ часа на часъ ожидаемъ, и не знаемъ, они ли прежде, которыхъ при вратахъ смерти видимъ, отъидутъ, или мы, которые надѣемся еще жить на свѣтѣ. Часто бо бываетъ, что надѣющіися продолжить свой животъ тотчасъ падаютъ бездыханными и, прежде находящихся на одрѣ смертномъ, полагаются во гробѣ. Вси бо мы къ смерти приговорены; всякому Богъ въ словѣ Своемъ святомъ глаголетъ, что праотцу нашему сказалъ: \textit{земля еси, и въ землю отъидеши}\footnote{Быт.~3,~19.}. Почтожъ не имѣемъ того приговора въ памяти? Почто не готовимся къ исходу? Почто обѣщаваемъ себѣ долго дней, "--- что не въ нашей власти, но въ Божіей только состоитъ? Почто столько богатства собираемъ, столько домовъ строимъ, столько утѣхъ заводимъ, аки вѣчно въ мірѣ семъ жити надѣяся? Почто за честію и славою міра сего гоняемся, которая сегодня или утро отъ насъ отъидетъ, и оставитъ насъ подлѣйшихъ и худшихъ паче пса живаго? Почто язвительными и мстительными челобитными другъ на друга наполняемъ судебныя мѣста? Почто другъ на друга злобимся и клянемъ? Почто другъ у друга похищаемъ добро, другъ друга оклеветаемъ, прельщаемъ? Почто беззаконно чужаго касаемся ложа, и душу и тѣло наше сквернимъ? Сегодня или утро позоветъ насъ гласъ Божій отсюду, который всегда гремитъ всякому: \textit{земля еси, и въ землю отъидеши}. О горе намъ, когда съ симъ тяжкимъ бременемъ отъидемъ отсюду! Неотмѣнно погрузитъ оно обремененныхъ во дно адово: \textit{ту будетъ плачь и скрежетъ зубомъ}\footnote{Матѳ.~25,~30.}. Сей случай и разсужденіе увѣщаваетъ тя всегдашнюю память о смерти имѣть. Она научитъ тя во всегдашнемъ быть покаяніи; она не попуститъ тя собирать богатства, чести и славы искать и сладострастіемъ утѣшаться; угаситъ пламень нечистыя похоти; укротитъ гнѣвъ, въ сердцѣ твоемъ курящійся; увѣщаетъ тя оставить ближнему всякіе долги согрѣшеній, отвратить руки твоя отъ хищенія и языкъ отъ сквернословія, злословія, клеветы и осужденія, и устамъ твоимъ положить храненіе; подвигнетъ къ усердной молитвѣ, воздыханію и теплымъ слезамъ. Страхъ бо будущаго суда и боязнь мученія сердце связуетъ и не попущаетъ хотѣть того, что Богу противно, и къ вѣчному суду приводитъ, и колеблющуюся и падающую душу удерживаетъ и возставляетъ: яко въ чемъ застанетъ насъ Богъ при кончинѣ, въ томъ и судитъ\footnote{Іезек.~18,~20 и проч., 33,~20 и проч.}. Блаженъ и мудръ, кто всегда помнитъ смерть!

\paragraph*{СІV.} Видишь на судъ позываемыхъ людей: примѣчай, что они дѣлаютъ? Готовятся, думаютъ, стараются, спрашиваютъ и совѣтуются у искусныхъ въ судебныхъ дѣлахъ; не щадятъ часто и денегъ, чтобы на судѣ не постыдиться и не быть осужденными. Видишь, любезный хрістіанине, что сынове вѣка сего дѣлаютъ; что стыдъ и страхъ временнаго осужденія въ нихъ дѣлаетъ, какъ сердца ихъ трогаетъ, къ какому попеченію, тщанію и труду ихъ побуждаетъ! Сей случай да подвигнетъ тебе помянуть о будущемъ Хрістовомъ судѣ. Всѣхъ насъ позоветъ Хрістосъ на судъ Свой: \textit{зане уставилъ есть день, въ оньже хощетъ судити вселеннѣй въ правдѣ}\footnote{Дѣян.~17,~31.}; \textit{уготова на судъ престолъ Свой}\footnote{Пс.~9,~8.}. \textit{Всѣмъ бо явитися намъ подобаетъ предъ судищемъ Хрістовымъ, да пріиметъ кійждо, яже съ тѣломъ содѣла, или блага, или зла}\footnote{2~Кор.~5,~10.}. А когда тое судище будетъ "--- сокрылъ Богъ отъ насъ, да всегда готовы будемъ и на всякій день ожидаемъ. \textit{Уже близъ есть, близъ день Господень}\footnote{Апок.~1,~3; 22,~10; Ис.~13,~6; Іоил.~1,~15; Іак.~5,~8 и 9.}. \textit{Пріидетъ же день Господень, яко тать въ нощи}\footnote{2~Петр.~3,~10.}. \textit{Егда бо рекутъ миръ и утвержденіе: тогда внезапу нападетъ на нихъ всегубительство}\footnote{1~Сол.~5,~3.}. \textit{Якоже бо бысть во дни Ноевы, тако будетъ и пришествіе Сына человѣческаго. Якоже бо бѣху во дни прежде потопа, ядуще и піюще, женящеся и посягающе, до негоже дне вниде Ное въ ковчегъ; и не увѣдѣша, дондеже пріиде вода и взятъ вся: тако будетъ и пришествіе Сына человѣческаго}\footnote{Матѳ.~24,~37--39.}. Аще убо къ человѣческому суду люди такъ приготовляются, какъ видиши: како намъ должно приготовляться, которые съ часа на часъ чаемъ явитися предъ судищемъ Божіимъ, и не знаемъ дне и часа, въ который Судія оный пріидетъ и позоветъ насъ на судъ. Аще стыдъ, предъ немногими имѣющій быть, такъ смущаетъ человѣка: кольми паче стыдъ, который быть имѣетъ предъ всѣмъ свѣтомъ, предъ всѣми ангелами и человѣками! Аще страхъ временныя казни такъ колеблетъ сердца людей: кольми паче страхъ вѣчныя муки долженъ колебать! Всякая бо временная казнь, какъ ни велика и продолжительна, минуется; вѣчная же не имѣетъ конца, но единожды наченшися, никогда не кончается. Видишь, какъ хитро люди приготовляютъ себе къ суду человѣческому: къ суду Божію не тако. Судія "--- человѣкъ не знаетъ дѣлъ человѣческихъ, и что видитъ и слышитъ, по тому и судитъ; часто и лице пріемля, производитъ судъ, и дары у многихъ судей много могутъ: Богъ "--- Судія не тако. Онъ не токмо внѣшнія наши дѣла и слова, но и сердечные помыслы знаетъ. Почему не требуетъ свидѣтелей и обличителей грѣхамъ человѣческимъ, но Самъ грѣшника обличитъ: \textit{обличу тя, и представлю предъ лицемъ твоимъ грѣхи твоя}\footnote{Пс.~49,~21.}. Лица никакого не пріемлетъ, не смотритъ ни на богата, ни на нища, ни на господина, ни на раба, по дѣламъ, а не по лицамъ судитъ всѣмъ. Того ради приготовленіе къ суду Его не такое должно быть, какое къ человѣческому суду бываетъ. Но какое? Должно заранѣе, пока еще Онъ не зоветъ, смирить себе, грѣхи и винность свою предъ Нимъ признать, самого себе предъ Нимъ обличить и означить отъ сердца достойнымъ себе всякія казни, обѣщаться впредь Ему вѣрою и правдою служить, и въ томъ неподвижно стоять, и тако милости Его безъ сумнѣнія ожидать. На сіе бо истое и объявилъ премилостивый Богъ имѣющій быть праведный Свой судъ, да покаемся и, покаявшеся, убѣжимъ осужденія. Не несогрѣшившихъ, но согрѣшившихъ и непокаявшихся осудитъ праведный оный Судія. Сей случай и разсужденіе увѣщаваетъ тебе, хрістіанине, отъ грѣховъ отстать, начать новое, богоугодное житіе, и въ томъ постоянну съ помощію Божіею быть, да сподобишися съ избранными Его овцами одесную Его стать въ пришествіи Его славномъ, а на нынѣшній свѣтъ, который въ безстрашіи безумномъ и пагубномъ въ безопаствѣ живетъ, не смотрѣть, да не съ нимъ погибнеши.

\paragraph*{СV.} Видишь, что когда идемъ къ солнцу, тѣнь за нами идетъ; и чимъ болѣе къ солнцу отъ тѣни убѣгаемъ, тѣмъ болѣе тѣнь за нами бѣжитъ: такъ имѣются люди благочестивые. Приближаются они къ Богу, послѣдуетъ за ними и слава; и чимъ болѣе къ Богу приближаются, тѣмъ болѣе и слава за ними слѣдуетъ. И хотя они отъ славы убѣгаютъ, но слава за ними бѣжитъ, и отъ ней уйти не могутъ. Напротивъ того, когда отходимъ отъ солнца, тѣнь отъ насъ отходитъ, и чимъ далѣе отъ солнца отходимъ, тѣмъ болѣе тѣнь отъ насъ отступаетъ; и хотя бѣжимъ за тѣнію, но достигнуть ея не можемъ: всегда гонящихся насъ убѣгаетъ, какъ бы за нею ни гналися. Тако имѣются люди міролюбивые: чимъ они болѣе отъ Бога удаляются, тѣмъ болѣе отъ нихъ удаляется слава; и чимъ болѣе гоняются за славою, тѣмъ болѣе бѣжитъ отъ нихъ слава. Сіе научаетъ тя искать и приближаться къ Богу смиреніемъ, молитвою, обученіемъ хрістіанскихъ добродѣтелей, и презирать міра сего славу; и тогда будеши имѣть славу, хотя ея и не желаеши. Не тотъ бо славенъ, кого міръ славитъ; но тотъ, кого Богъ прославитъ. Прославляетъ же Богъ того, кто Его прославляетъ, якоже глаголетъ Господь: \textit{прославляющія Мя прославлю, и уничижаяй Мя безчестенъ будетъ}\footnote{1~Цар.~2,~30.}.

\paragraph*{СVІ.} Видишь паки, что тѣнь не имѣетъ ничего въ себѣ, ни жизни, ни тѣла, ни движенія, но только подобіе тѣла; и тотчасъ исчезаетъ, когда сокрывается солнце. Тако имѣется человѣкъ; и онъ ничто есть безъ Бога. Пока Богъ его содержитъ, управляетъ и просвѣщаетъ, нѣчто есть и быти показуется; но когда Богъ свѣтъ Свой сокрыетъ и жизненную силу отниметъ, тотчасъ исчезаетъ. О семъ глаголетъ пророкъ: \textit{убо образомъ ходитъ человѣкъ}\footnote{Пс.~38,~7.}, \textit{и дніе его яко сѣнь преходятъ}\footnote{143,~4.}. Сей случай научаетъ тя познавать свое ничтожество и смиряться, что начало есть хрістіанскія премудрости.

\paragraph*{СVII.} Видишь паки, что чимъ болѣе приближается солнце, тѣмъ болѣе умаляется тѣнь. Тако имѣется благочестивое сердце: чимъ ближе приходитъ къ нему Богъ съ Своимъ свѣтомъ и дарованіями, тѣмъ болѣе онъ познаетъ свое ничтожество и смиряется предъ Богомъ и человѣками, себе малѣйшимъ вмѣняетъ. Напротивъ того, чимъ далѣе отходитъ солнце, тѣмъ большая бываетъ тѣнь; какъ заходитъ солнце, великая тѣнь дѣлается; зайдетъ солнце, исчезнетъ и тѣнь великая. Тако имѣется и человѣкъ: колико удаляется отъ него Богъ, толико въ сердцѣ своемъ возносится и величается; колико же человѣкъ возносится, толико отъ Бога удаляется, и Богъ отъ него; и, какъ тѣнь исчезаетъ, когда солнце сокрывается, тако исчезаютъ и сіи, которые мнятся быти великими, когда Богъ сокрыется. Отъ сего случая научаемся, хрістіанине: 1)~Истинное смиреніе отъ истиннаго благочестія неотлучно, но всегда съ нимъ сопряжено. 2)~Чимъ благочестивѣйшее сердце есть, тѣмъ смиреннѣйшее есть. Яко чимъ болѣе просвѣщается человѣкъ благодатію Божіею, тѣмъ болѣе видитъ свое недостоинство: якоже чимъ болѣе просвѣщаемся естественнымъ свѣтомъ, тѣмъ болѣе усматриваемъ пороки на лицѣ, рукахъ, и соръ и сметіе въ покояхъ. Откуду, видя человѣкъ свое недостоинство, смиряется. 3)~Въ которомъ сердцѣ нѣтъ смиренія, но паче гордость имѣется, въ томъ нѣтъ и благочестія. Ибо таковый человѣкъ не познаетъ своего недостоинства, того ради и благочестіе въ немъ не вмѣщается. Благочестіе бо истинное безъ Бога быть не можетъ: \textit{гордымъ Богъ противится}, по Писанію\footnote{1~Петр.~5,~5.}. 4)~Таковое сердце, которое не тщится познавать своего недостоинства и въ гордости своей пребываетъ наконецъ исчезаетъ и погибаетъ, якоже тѣнь по захожденіи солнца исчезаетъ. 5)~Научаемся симъ истиннаго благочестія искать отъ познанія нашего недостоинства, то"=есть, когда хощемъ истинное благочестіе въ сердцѣ имѣть, должны всячески познавать свое недостоинство и ничтожество, и тако смиряться, да и на насъ призритъ Господь, \textit{на смиренныя призираяй}, и подастъ благодать Свою. 6)~Понеже глубоко растлѣнно сердце имѣемъ, и отъ природы имѣемъ слѣпоту душевныхъ очей, и того ради не можемъ видѣть бѣдности нашей и окаянства: то должно воздыхать къ Богу, чтобы Онъ Самъ насъ просвѣтилъ, и показалъ намъ наше окаянство. Тогда познаемъ, что мы въ себѣ есмы тѣнь движущагося тѣла, которая тогда движется, когда тѣло движется; тогда стоитъ, когда тѣло стоитъ; отступаетъ тѣло, отступаетъ и она.

\paragraph*{СVІІІ.} Видишь, что сынъ, поступившій съ отцемъ своимъ грубо и его оскорбившій, горько плачетъ и окаеваетъ себе, жалѣя, что отца своего, отъ котораго рожденъ и воспитанъ, прогнѣвалъ и оскорбилъ. Отъ сего случая должны мы, хрістіанине, учитися истинному покаянію, \textit{печаль по Бозѣ} имѣти, \textit{яже покаяніе нераскаянно во спасеніе содѣловаетъ}\footnote{2~Кор.~6,~10.}. Аще бо жалѣемъ и плачемъ, что отца по плоти оскорбили, отъ котораго только родилися и воспитаны мы, и праведно тое дѣлаемъ: кольми паче жалѣть, печалиться и неутѣшно плакать намъ должно, что прогнѣвляемъ Бога, небеснаго Отца, отъ Котораго отцы наши и мы созданы, образомъ Его почтены, падши возставлены, и безчисленными благими одарены и одаряемы. Великая воистину слѣпота и злоба оскорблять Того, Иже есть единая \textit{Любовь}! Сатанинская кознь есть, злобою на Создателя и образъ Его дышущая, который, не могучи ничего Богу сдѣлать, на образъ Его устремляется, и тѣмъ хощетъ оскорбить Его! Человѣкъ слѣпый, какъ рыба за удицу, за прелесть вражію хватается, и ему соизволяетъ, и тако, слушая совѣта вражія, Бога, Создателя своего, не слушаетъ и оскорбляетъ. Чего Богъ не сдѣлалъ намъ? какого добра не показалъ? Создалъ насъ, образомъ Своимъ почтилъ насъ, міръ сей красный создалъ ради насъ; солнце Свое сіяетъ на насъ, и дождитъ на насъ, питаетъ, одѣваетъ, сохраняетъ насъ. Но мы оскорбляемъ Его, не слушаемъ Его, не любимъ такъ великаго Благодѣтеля. Не великая ли се есть бѣсовская злоба и прелесть ума нашего? Погибли мы бѣсовскимъ коварствомъ, и лишилися Его милости; но Онъ и тако не оставилъ насъ, не потерпѣлъ насъ видѣти въ погибели лежащихъ, не отринулъ насъ на вѣки отъ Себе, яко непотребныхъ; но умилосердился надъ нами, послалъ намъ спасеніе погибшимъ и избавленіе плѣненнымъ, и возстановленіе падшимъ. Какъ? не чрезъ ангела, ни чрезъ человѣка, но чрезъ кого? Чрезъ \textit{Сына Своего Единороднаго}. Того подалъ намъ на спасеніе наше и избавленіе; Того насъ ради \textit{не пощадѣлъ, но за насъ предалъ Его}, вопіетъ всей вселеннѣй Апостолъ святый\footnote{Римл.~8,~32.}. \textit{Духа} Своего Святаго хощетъ дати намъ, и \textit{подаетъ} вѣрующимъ и \textit{просящимъ у Него}\footnote{Лук.~11,~13.}. Что болѣе можетъ Богъ сотворити намъ паче сего? какое добро и благодѣяніе подати намъ? Но мы и тако не любимъ Его, не любимъ такъ возлюбившаго насъ; не почитаемъ Его, такъ высоко почтившаго насъ: воистину не можетъ быть большая злоба и ослѣпленіе! Вси ангели святіи со страхомъ покланяются Ему, почитаютъ Его и поютъ Его\footnote{Пс.~6,~2 и 3.}: человѣкъ ослѣпленный, \textit{земля и пепелъ}, толико благодѣяній и милостей Божіихъ дознавшій на себѣ, не дѣлаетъ того, но раздражаетъ Его безстыдно!.. Такъ бѣсовская хитрость ослѣпила умъ нашъ! Ежели бы кто тебе, хрістіанине, питалъ, одѣвалъ, согрѣвалъ, берегъ, сохранялъ, и какъ дитя свое любилъ; и ты бы его не почиталъ, презиралъ и всякую грубость показывалъ ему: не великое ли бы было безуміе и безстыдство твое? Воистину бы вси видящіи удивлялися безумію твоему. Знай точно, что такое безстудство показываемъ Богу нашему, когда Его не слушаемъ, грѣшимъ и раздражаемъ Его. О, воистину лучше свѣта лишиться, во гноищѣ и ранахъ лежать, въ темницѣ заключену быть, оковану быть, сто кратъ умереть, нежели Бога, вѣчную любовь, благодать, благостыню, святыню, истину, правду, милосердіе, начало и источникъ всѣхъ благъ, хотя мало прогнѣвать!..

\paragraph*{СІХ.} Знаешь, хрістіанине, что когда хощемъ чужое добро употребить, напримѣръ: или пищи вкусить, или питія напиться, или инаго чего на нужду нашу взять, у хозяина спрашиваемся и просимъ дозволенія, и пріявши добро его, благодаримъ ему: такъ обыкновенно вездѣ люди честные поступаютъ, и такъ сему быть должно. Тако подобаетъ хрістіанамъ поступать, когда хотятъ пищи, питія и другихъ благихъ міра сего употребить. Все Божіе есть: пища, питіе, одежда и прочее есть Божіе добро. Богъ есть властелинъ и хозяинъ міра, и всего, что ни имѣется въ мірѣ. Слѣдственно съ благословенія Его должно намъ все, чего хощемъ употребить, принимать, и молитвою просить, когда не хощемъ противу Его согрѣшить. Чуждое бо употреблять безъ дозволенія хозяйскаго грѣшно, какъ сіе всякъ можетъ признать. Безспорно сіе, что Богъ никому не отрицаетъ благихъ міра сего, и самымъ нечестивымъ подаетъ; но хрістіанамъ должно Подателя знать, и у Него благословенія просить къ употребленію, и за тое Ему, яко Благодѣтелю, отъ сердца благодарить. Отсюду послѣдуетъ: 1)~Что худо тѣ хрістіане дѣлаютъ, которые безъ молитвы и благословенія Божія начинаютъ ясти, пити, и прочая благая употребляти. Знать, такіе не знаютъ, откуду всякое добро происходитъ, и Дателя не зная, не благодарятъ Ему. Тако они подобятся язычникамъ, не знающимъ истиннаго Бога и Творца всѣхъ благихъ, или паче безсловеснымъ скотамъ, которые безъ всякаго разсужденія ядятъ и пьютъ, что предъ собою видятъ. "--- 2)~Которые довольствуются благими Божіими, и не благодарятъ Благодѣтелю: чѣмъ они показуютъ свою къ Благодѣтелю Богу неблагодарность, что есть великій грѣхъ. "--- 3)~Худо просить благословенія къ объяденію и піянству, такожде одежды къ щегольству и величанію: Богъ бо дозволяетъ пищу и питіе употреблять намъ, но запрещаетъ объяденіе и піянство. Такожде подаетъ одежду намъ къ нуждѣ нашей, а не къ славолюбію, "--- къ прикрытію наготы и согрѣянію убогаго тѣла нашего, а не къ самохвальству и величанію; словомъ, дозволяетъ угодіе творить плоти, дозволяетъ алчущую питать, нагую одѣвать, согрѣвать, упокоевать утружденную, но не въ похоти ея, якоже апостолъ учитъ: \textit{плоти угодія не творите въ похоти}\footnote{Римл.~13,~14.}. "--- 4)~Беззаконно и безстыдно и на тое просить благословенія у Бога, что отъ неправды и хищенія собрано. Богъ бо не велитъ намъ неправды дѣлать и хищеніе запрещаетъ. Нѣтъ убо тамо и благословенія Божія, гдѣ неправда и хищеніе есть, но вмѣсто того клятва: яко законъ Божій разоряется и Законодавецъ презирается.

\paragraph*{СХ.} Видишь, что хотящіи научитися художеству, наукамъ и мірской мудрости, входятъ въ училища и обучаются отъ учителей искусныхъ, слушаютъ отъ нихъ наставленія и правила, и внимаютъ имъ, поступаютъ по нихъ: иначе не могутъ научитися, когда по наставленіямъ и правиламъ ихъ не будутъ поступать. Тако подобаетъ хрістіанамъ, которые суть ученики Хрістовы, учитися у Хріста, иже есть Ѵпостасная Божія Премудрость, \textit{Иже бысть намъ Премудрость отъ Бога}\footnote{1~Кор.~1,~30.}, и внимать святѣйшимъ и истиннѣйшимъ догматамъ Его, наставленіямъ и правиламъ Его, которыя представляются намъ во святомъ Евангеліи Его, какъ въ божественномъ и небесномъ училищѣ, когда хотятъ духовной и небесной мудрости научитися, якоже Самъ глаголетъ: \textit{научитеся отъ Мене}\footnote{Матѳ.~11,~29.}. Догматы и правила Его святые научаютъ насъ истинному покаянію, вѣрѣ, смиренію, любви, терпѣнію, кротости, и проч. Въ семъ бо состоитъ хрістіанская духовная мудрость, которая предъ міромъ презрѣнна и умаленна, но предъ Богомъ честна и возвеличена. Искусные учители, преподая правила наукъ, показуютъ и образъ, како по правиламъ тѣмъ поступать; ибо правило само безъ показанія образа и дѣла немного и почти ничего не пользуетъ; хотячи убо научить учениковъ своихъ, и дѣломъ самымъ показуютъ, чему правила учатъ, да на правила и дѣло взирая, лучше научатся по правиламъ поступать. Тако Хрістосъ, небесный и премудрый нашъ Учитель, хрістіанине, сотворилъ. Не токмо догматы небесныя Своея премудрости предложилъ намъ, но и образъ подалъ, якоже глаголетъ: \textit{образъ дахъ вамъ, да якоже Азъ сотворихъ вамъ, и вы творите}\footnote{Іоан.~13,~15.}. Училъ смиренію, терпѣнію, любви, кротости и прочіимъ добродѣтелемъ: и Самъ тое самымъ дѣломъ показалъ, когда насъ ради \textit{смиритися даже до умовенія ногъ}\footnote{13,~5.}, "--- терпѣти, и \textit{послушливъ быти даже до смерти, смерти же крестныя}\footnote{Филип.~2,~8.}, "--- возлюбити враговъ и \textit{за нихъ молитися: Отче, отпусти имъ}\footnote{Лук.~23,~34.}, благоволилъ и тако совершенный и живый образъ богоугоднаго житія, въ чемъ истинная есть мудрость, показалъ. Сей намъ премудрости подобаетъ учитися у Учителя нашего Хріста, когда хощемъ учениками Его, то есть, хрістіанами быти. Хрістіанинъ бо не иное что, какъ ученикъ Хрістовъ есть. Откуду въ началѣ хрістіанства хрістіане называлися \textit{учениками}, какъ читаемъ въ Дѣяніяхъ Апостольскихъ\footnote{6,~1 и 2; 9,~1; 11,~26 и 29 и проч.}. Когда убо учениками Хрістовыми называемся, то неотмѣнно должны мы у Него учитися, "--- чемужъ? тому, чему Онъ и словомъ и дѣломъ учитъ: иначе напрасно и учениками называемся, когда не учимся. Ученикъ бо у учителя своего учится тому, чему онъ учитъ. Смотри, хрістіанине, учишися ли у Хріста хрістіанской мудрости, безъ которой всякая мудрость буйство есть? Учишься право и красно говорить: но учишься ли отъ Хріста право и красно жить? Учишься натуру вещей познавать, землю размѣрять, звѣзды считать: но учишься ли немощь твою и бѣдность естества своего познавать, краткость дней твоихъ мѣрять и грѣхомъ растлѣнное сердце твое испытовать? Учишься по"=французски, по"=нѣмецки, по"=италіянски говорить: но учишься ли по"=хрістіански жить? Когда нѣтъ того, то знай, что все твое тщаніе суетно есть, и мудрость твоя безуміе есть, и ничего тебѣ не воспользуетъ, ниже сотворитъ тя блаженнымъ, хотя тѣмъ и утѣшаешися. Паче же не противишися ли Хрісту и ученію Его святому, хотя и ученикомъ Его нарицаешися? Что бо имя пользуетъ безъ самой вещи? весьма ничего. Что имя хрістіанское безъ обученія хрістіанскаго? какъ образъ безъ самыя вещи, еже не ино что, какъ прелесть и обманъ. Хрістосъ смирился ради тебе: смотри, не гордишися ли ты? Хрістосъ обнищалъ ради тебе: смотри, не ищеши ли ненасытно богатства міра сего, не хощеши ли прославитися на земли? Хрістосъ Себе ради тебе не пощадѣлъ: смотри, не жалѣеши ли ты подать укруха хлѣба или одѣянія ближнему своему? Хрістосъ за враговъ Своихъ молился: смотри, не злобишься ли на оскорбившихъ тебе, и любишь ли тѣхъ, которые тебе ничимъ не обидѣли? Хрістосъ заушенъ и оплеванъ былъ ради тебе: смотри, не мстиши ли ты кому за противное и укорительное слово? Хрістосъ скорбь и болѣзнь имѣлъ: смотри, не живеши ли въ сладострастіи? Хрістосъ терновый вѣнецъ носилъ: смотри, не хощеши ли ты вѣнчаться славою міра сего? Хрістосъ нашихъ ради грѣховъ плакалъ: смотри, ты ради своихъ плачеши ли? О, коль далеко отстоишь отъ общества хрістіанскаго и Самаго Хріста, когда не токмо не учишися у Хріста, чему Онъ и словомъ и примѣромъ Своимъ учитъ, но и противишися Ему! Не токмо бо тотъ противится Хрісту, кто противно слову Его святому учитъ, но и тотъ, кто противно ученію Его, житію Его живетъ. Не учишь противу слова Его "--- хорошо; но смотри, не поступаеши ли въ житіи твоемъ противу Хріста? Ибо Онъ не токмо догматамъ небеснымъ научаетъ насъ, но и небесному и святому житію, и во образъ Себе въ томъ подаетъ намъ. Итакъ, хотя не учимъ противно Ему, но не хочемъ по правиламъ Его и образу святаго житія Его поступать, противимся Ему. Осмотрись убо, возлюбленный хрістіанине, пока время не ушло, и воздыхай къ Нему, чтобы Самъ Онъ исправилъ сердце твое и наставилъ на путь Свой. Не смотри, что нынѣшній свѣтъ дѣлаетъ, но чему Хрістосъ учитъ и словомъ и дѣломъ, когда не хощешь вѣчно заблудить и погибнуть. Повѣрь, что день отъ дня умаляются сынове царствія Божія, и умножаются сынове непріязненніи; грѣхъ часъ отъ часу усиливается, и соблазнъ болѣе и болѣе умножается. Помни, что Хрістосъ во Евангеліи всѣмъ глаголетъ: \textit{Азъ есмь свѣтъ міру: ходяй по Мнѣ, не имать ходити во тмѣ, но имать свѣтъ животный}\footnote{Іоан.~8,~12.}. Отсюду, безъ сумнѣнія, заключается, что неотмѣнно во тьмѣ ходятъ, хотя бы и мудрыми казалися, которые не ходятъ въ слѣдъ Хріста, то есть, не послѣдуютъ Его смиренію, любви, терпѣнію и кротости Его. Слово бо Хрістово не есть ложно, но истинно, и какъ сказано, такъ и есть, яко Онъ Самъ \textit{есть истина}\footnote{14,~6.}.

\paragraph*{СХІ.} Видишь, что впадшій въ погрѣшность предъ монархомъ, и осужденію смертному по законамъ подлежащій, не имѣя надежды ни откуду, прибѣгаетъ къ любезному другу монаршему, котораго ходатайству себе ввѣряя, надѣется и отъ монарха милость получить; а какъ надѣется, такъ и получаетъ милость отъ него чрезъ ходатая своего: отсюду въ получившемъ милость послѣдуетъ: 1)~утѣшеніе и радость въ сердцѣ; 2)~благодареніе и любовь къ благодѣтелю своему; 3)~надежда на благодѣтеля, такъ сильнаго и милостиваго; 4)~страхъ, чтобы не оскорбить и не прогнѣвать его, и тако бы милости его не отпасть. Примѣчай, возлюбленный хрістіанине, что сынове вѣка сего дѣлаютъ, когда хотятъ отъ временной бѣды избавиться и временную милость у подобныхъ себѣ людей получить. Отъ сего случая учимся мы, хрістіанине, что есть евангельская святая вѣра, которую имѣютъ истинные хрістіане въ Господа нашего Іисуса Хріста. Мы вси, то есть, весь родъ человѣческій, согрѣшили предъ Царемъ нашимъ Богомъ, по Писанію: \textit{вси согрѣшиша и лишени суть славы Божія}, и тако \textit{всяка уста заградилися, повиненъ бысть весь міръ Богови}\footnote{Римл.~3,~23,~19.}. Тако по закону правды Божія непремѣняемыя подлежали вси вѣчному осужденію и вѣчной казни за свою вину. Но благость Божія, вся премудро строящая, изобрѣла намъ путь спасенія чуднымъ образомъ. Сынъ Божій, благоволеніемъ небеснаго Своего Отца, человѣкомъ ради человѣковъ учинился, и смертію правдѣ Божіей вмѣсто насъ удовлетворилъ, и тако между Богомъ праведнымъ и нами согрѣшившими Ходатаемъ сдѣлался. Аще убо кто сему всесильному и милостивому Ходатаю себе ввѣряетъ, имя свое съ сердцемъ своимъ Ему отдаетъ и записываетъ: той отъ Бога милость получаетъ, грѣховъ и казни, грѣхамъ послѣдующія, свобождается, означается праведнымъ такъ, какъ бы онъ никакого грѣха не сотворилъ: \textit{правдою} бо \textit{Хрістовою}, которую вѣрующимъ во имя Его неповинною смертію и страданіемъ заслужилъ, какъ царскою багряницею, одѣвается, и тако чистымъ, праведнымъ и святымъ предъ чистѣйшими Божіими очесами бываетъ, якоже о семъ богомудрый Павелъ въ лицѣ всѣхъ вѣрныхъ глаголетъ: \textit{Иже} (Хрістосъ) \textit{бысть намъ Премудрость отъ Бога, правда же и освященіе и избавленіе}\footnote{1~Кор.~1,~30.}. Получаетъ же великую сію и непостижимую отъ Бога милость \textit{туне}, вѣрою единою, безъ всякихъ заслугъ своихъ, но ради единыхъ заслугъ Хрістовыхъ. Что бо согрѣшившій и праведно осужденный заслужити можетъ, кромѣ гнѣва и клятвы? Отсюду въ томъ, который тако во Хріста сердечно вѣруетъ, и толикую милость Божію въ сердцѣ своемъ чувствуетъ, послѣдуютъ неотмѣнно: 1)~Утѣшеніе и радость, ибо вѣра безъ того не бываетъ. Какъ бо не утѣшаться и не радоваться тому, который вѣчному гнѣву Божію и осужденію подлежалъ, но толикую милость \textit{туне} получилъ отъ Него? 2)~Благодарность и любовь сердечная. Какъ бо не благодарить толикому Благодѣтелю, и не любить такъ великую благость? 3)~Смиреніе. Яко такой милости сподобляется отъ Бога не по своему достоинству, но по Его единой благости и заслугамъ милостиваго Ходатая. 4)~Страхъ Божій, чтобы Бога паки не прогнѣвать, и тако бы милости Его не лишиться. 5)~Молитва, чтобы Богъ въ благодати Своей содержалъ его, и благодатію Своею сохранялъ отъ грѣха, діавола и прочаго. 6)~Высокое почтеніе къ спасительному Хрістову смотрѣнію, воплощенію, страданію и смерти: яко симъ единымъ отъ бѣдствія великаго избавился и великія милости Божіей сподобился. 7)~Несумнѣнная надежда вѣчнаго живота: яко \textit{Иже сына Своего не пощадѣ, но за насъ всѣхъ предалъ есть Его, како не и съ Нимъ вся намъ дарствуетъ}\footnote{Римл.~8,~32.}? Тако съ апостоломъ разсуждаетъ и заключаетъ вѣрная душа: что"=де когда согрѣшу, толикую милость получивши отъ Бога и туне оправдавшеся, паки потеряю тое оправданіе? Отвѣщаетъ таковому апостолъ святый Іоаннъ: \textit{аще кто согрѣшитъ, Ходатая имамы ко Отцу, Іисуса Хріста Праведника. И Той очищеніе есть о грѣсѣхъ нашихъ, не о нашихъ же точію, но и о всего міра}\footnote{1~Іоан.~2,~1 и 2.}. То"=есть: согрѣшившему не должно отчаятися, но тотчасъ обратитися и прибѣгнуть къ милостивому Ходатаю Іисусу Хрісту, признать грѣхъ свой со смиреніемъ и просить у Него прощенія и милости, чтобы паки пріялъ въ общество вѣрныхъ Своихъ. Не сократилися бо щедроты Его, и кающимся и толкущимъ двери милосердія Его отверзаются. Однакожъ должно крайне берещися грѣха, хотя и толикая есть грѣшникамъ Божія милость, да не согрѣшимъ противу милости Божія, и, вмѣсто милости, правду Его на себѣ узнаемъ. Безъ сумнѣнія есть милостивъ Богъ, но есть и праведенъ: и кто, уповая на милость Божію, не престаетъ грѣшить, тому опасаться надобно, чтобы не почувствовалъ на себѣ праведнаго суда Его. Богъ бо \textit{поруганъ не бываетъ}\footnote{Гал.~6,~7.}. Отсюду заключается, что въ комъ нѣтъ плодовъ вѣры, то"=есть, добрыхъ дѣлъ, много паче въ комъ злыя дѣла, безстрашное житіе и закону Божію противное, въ томъ нѣтъ истинныя сердечныя вѣры, но есть только устная, ложная и прелестная, хотя бы проповѣдывалъ и училъ вѣрѣ; каковою вѣрою премногихъ хрістіанъ исполнены уста въ нынѣшнемъ наипаче вѣкѣ, которые устами исповѣдуютъ и проповѣдуютъ Бога, но въ сердцѣ безбожіе имѣютъ и \textit{дѣлы отмещутся Его}\footnote{Тит.~1,~16.}.

\paragraph*{СХІІ.} Видишь образъ распятаго на древѣ крестномъ Хріста. Здѣ довольную имѣешь, хрістіанине, матерію поучитися въ дѣлѣ хрістіанскаго благочестія. На сіе бо и представляетъ намъ святая Церковь страшное и спасительное сіе позорище, да, на тое взирая, вѣрою взираемъ на Хріста, насъ ради распятаго, и поминаемъ, что мы были, что силою распятаго Хріста сдѣлалися. Представляетъ оно намъ: 1)~Непремѣняемую Божию \textit{правду}. Видимъ здѣ, что правда Божія нарушитися не можетъ, но требуетъ неотмѣнно, чтобы вѣчный и святый законъ Божій цѣлъ и ненарушенъ былъ отъ насъ, или законопреступникъ достойно наказанъ былъ. "--- 2)~\textit{Тяжесть грѣха и гнѣвъ Божій} за грѣхи наши. Легко человѣкъ можетъ согрѣшити, но чрезъ всю вѣчность въ геенскомъ огнѣ очищать грѣхъ свой будетъ, когда здѣ покаяніемъ и вѣрою во Хріста не очистится отъ него. Тяжекъ грѣхъ есть: яко никто его отъять отъ насъ не моглъ, кромѣ неповиннаго Хріста, Сына Божія. Неумолимый гнѣвъ Божій есть за грѣхи: яко къ умилостивленію Его потребно было умрети за насъ Хрісту Сыну Божію, и за насъ правдѣ Божіей наградити, которую мы грѣхами нашими раздражили; чего никто, кромѣ Его, учинить не моглъ. Отсюду, хрістіанине, учись познавать гнѣвъ Божій за грѣхи, что Хрістосъ, Сынъ Божій, \textit{неповинный}, такъ ужасно мученъ былъ за грѣхи наши, якоже слышиши изъ исторіи евангельской, и видишь въ образѣ распятія Хрістова и прочіихъ Его спасительныхъ страстей, и тако убѣгати отъ грѣха. Аще же въ какомъ грѣхѣ находишься, потщися отстать отъ него, да не вѣчному суду и гнѣву Божію подпадеши, и за всѣ твои грѣхи въ геенскомъ огнѣ безъ ослабленія страдати будеши. Отъ ужаснаго страданія Хрістова познавай, коль великій гнѣвъ Божій возгорится на нераскаянныхъ грѣшниковъ, которые толикую Божію благодать презрѣли, и не хотѣли каятися, престать отъ грѣховъ и къ Богу обратитися, "--- наипаче на тѣхъ, которые Бога исповѣдуютъ, законъ Его слышатъ, и волю вѣдаютъ, но дѣлами своими отмещутся Его. Не токмо слово изобразить, но и умъ понять не можетъ бѣдствія того, возлюбленный хрістіанине! Неизреченная и непостижимая милость Божія явилася къ намъ, что Сына Своего Единороднаго ради спасенія нашего не пощадѣлъ; но милость сія въ ужасный гнѣвъ обратится тѣмъ, которые не хотѣли милости тоя принять и употребить во спасеніе: \textit{Богъ бо поруганъ не бываетъ}, учитъ Апостолъ. Хрістосъ, Сынъ Божій и Богъ во плоти Своей такъ поруганъ, посмѣянъ, обезчещенъ, уязвленъ и мученъ былъ за грѣхи наши, что и помыслити о томъ ужасно: какъ уже презрѣвшіе сіе великое смотрѣнія Его дѣло, за свои грѣхи, которыми правду Божію раздражили, мучимы будутъ въ геенскомъ огнѣ! Милостивъ Богъ есть, и вездѣ въ святомъ Писаніи милость Его проповѣдуется, но кающимся и обращающимся отъ грѣховъ къ Нему; а некающіися правду Его и праведный Его гнѣвъ на себѣ неотмѣнно дознаютъ. И чимъ кто болѣе грѣшитъ, тѣмъ болѣе на себѣ узнаетъ гнѣвъ Божій, по неложному ученію Апостола: \textit{по жестокости твоей и непокаянному сердцу, собираеши себѣ гнѣвъ въ день гнѣва и откровенія праведнаго суда Божія, иже воздастъ коемуждо по дѣломъ его}\footnote{Римл.~2,~6 и 7.}. "--- 3)~Представляетъ сіе позорище намъ безмѣрную благость Божію, и безприкладное Его къ намъ \textit{милосердіе}; яко ради спасенія нашего Сына Своего Единороднаго не пощадѣлъ, но на смерть предалъ Его. Понеже бо сами мы, согрѣшивше, не могли спастися, не могли раздраженной грѣхами нашими правдѣ Божіей никакою силою удовлетворить, грѣховъ нашихъ очистить и оправдитися: того ради на сіе великое дѣло Сына Своего послалъ въ міръ, дабы мы благодатію Его отъ грѣховъ нашихъ очистилися, вѣруя въ Него, и тако бы спасеніе получили, которое во Адамѣ потеряли. \textit{Тако бо возлюби Богъ міръ, яко и Сына Своего Единороднаго далъ есть, да всякъ, вѣруяй въ Онь, не погибнетъ, но имать животъ вѣчный}\footnote{Іоан.~3,~16.}. Взирая убо на образъ распятія Хрістова, хрістіанине, помяни непостижимую благость и человѣколюбіе небеснаго Отца, и отъ сердца чиста благодари Ему за сіе великое Его къ намъ смотрѣніе. Взирай вѣрою и на распятаго Хріста, воскресшаго и сѣдящаго одесную Отца, и проси отъ Него исцѣленія грѣховныхъ язвъ, якоже нѣкогда взирали Израильтяне на вознесенную змію въ пустыни, и исцѣлѣвалися отъ угрызеній зміиныхъ. \textit{Якоже бо Моѵсей вознесъ змію въ пустыни, тако подобало вознестися Сыну человѣческому: да всякъ, вѣруяй въ Онь, не погибнетъ, но имать животъ вѣчный}\footnote{14 и 15.}. Отсюду почерпай прохлажденіе печальному сердцу твоему, которое совѣстію грѣховъ и страхомъ сокрушенно есть. Аще бо Богъ Сына Своего не пощадѣлъ ради нашего спасенія, како насъ, согрѣшившихъ и кающихся, и просящихъ прощенія ради Его, не пощадитъ и не помилуетъ? Симъ утверждай вѣру твою, и безъ сумнѣнія милости Божія надѣйся, когда истинно, обратившися отъ грѣховъ, каешися. "--- 4)~Въ распятомъ Хрістѣ видимъ, хрістіанине, безмѣрную Божію \textit{премудрость}: яко тамо путь къ спасенію нашему изобрѣла, гдѣ казалося никакому не быть спасенію. Ибо неотмѣнно слѣдовало человѣку, согрѣшившему противу \textit{Бога вѣчнаго, вѣчно казнену быть}, по силѣ правды Божіей. Но Божія премудрость изобрѣла такое посредствіе, которымъ и правда Божія свое удовольствіе получила, и милосердіе Его надъ человѣкомъ согрѣшившимъ исполнилося. Хрістосъ Сынъ Божій \textit{безгрѣшный}, и Богъ во плоти, за грѣшниковъ пострадалъ и умерлъ, и тако раздраженной правдѣ Божіей за нихъ удовлетворилъ; а тако и грѣшникамъ къ милосердію Божію дверь отворилася. Правда Божія сею святѣйшею жертвою удовольствовалася, и милосердіе Его мѣсто возъимѣло дѣйствовать въ спасеніе человѣческое. Человѣкъ согрѣшившій благодатію Божіею и милосердіемъ Его спасается и правда Божія непремѣнна пребываетъ. Тако премудростію Божіею, и правда Божія и милосердіе Его, въ страданіи и распятіи Хрістовомъ исполнилися. \textit{Древомъ запрещеннымъ} клятва вселилася въ насъ\footnote{Быт.~3.}, и \textit{древомъ крестнымъ} клятва отнялася отъ насъ, и благословеніе подалося намъ\footnote{Гал.~3,~13 и 14.}: \textit{язвою Хрістовою} мы вси исцѣлѣли\footnote{1~Петр.~2,~24.}, и смертію Его ожили; скорбію Его мы утѣшилися, и Его безчестіемъ славу получили; страданіемъ Его наше страданіе отнялося, и смертію Его наша смерть умертвилася; кровію Его наши долги загладилися, и Его совершенною правдою мы оправдалися. Тако вся возможна суть всемогущему, премудрому и благому Богу нашему. Благословенный Господи во вся вѣки, слава Тебѣ! "--- 5)~Въ семъ спасительномъ изображеніи видимъ, возлюбленный хрістіанине, \textit{вольное} Сына Божія насъ ради \textit{послушаніе}, смиреніе глубочайшее, превеликое терпѣніе и кротость. \textit{Волею} ради насъ благоволилъ быти \textit{послушливъ даже до смерти, смерти же крестныя; волею} ради насъ смирился Господь славы такъ, что большее смиреніе быти не можетъ; \textit{волею} ради насъ претерпѣлъ такое безчестіе и страданіе, каковаго большее не можетъ быти, и не токмо претерпѣлъ, но и молился за враговъ своихъ: \textit{Отче, отпусти имъ}\footnote{Лук.~23,~34.}. Вся же сія сотворилъ отъ любви къ небесному Своему Отцу и къ намъ, подлымъ рабамъ Своимъ, чтобы Его, грѣхами нашими раздраженнаго и огорченнаго, вольнымъ своимъ страданіемъ умилостивить, и насъ раздражившихъ и огорчившихъ въ милость Ему привести, правдѣ Его за насъ удовольствовать и намъ милость заслужить, клятву отъ насъ потребить и благословеніе намъ подать. Тако учинился Онъ \textit{Ходатаемъ} между Богомъ разгнѣваннымъ и нами прогнѣвавшими. Отъ Его послушанія, смиренія, терпѣнія и кротости должно и намъ учитися усердному послушанію, смиренію, терпѣнію и кротости, когда хощемъ не напрасно хрістіанское имя носити и Хрістовыми быти\footnote{Гал.~5,~22--26.}.

\paragraph*{СХIII.} Видишь, что человѣкъ, потерявши сокровище свое, злато, сребро, каменіе драгое, или честь, которымъ сокровищемъ утѣшался, ищетъ его со всякимъ тщаніемъ и желаніемъ, чтобы паки сыскать потерянное. Примѣчай, что сынове вѣка сего дѣлаютъ, когда хотятъ сокровища своя сыскать, которыя съ великимъ трудомъ ищутъ, съ великимъ попеченіемъ хранятъ, не долго содержатъ, и вскорѣ съ неизреченною печалію оставляютъ, оставляя свѣтъ сей. Примѣчай, глаголю, возлюбленный хрістіанине, какъ люди временное, тлѣнное и, какъ истину сказать, прелестное сокровище ищутъ, идѣже сердце ихъ есть. Отъ сего случая и примѣра сыновъ вѣка сего учиться мы, хрістіане, должны искать сокровища хрістіанскаго, которое есть Богъ со всѣми благими Своими. А гдѣ Его искать намъ? во градѣ, или на селѣ или въ пустыни? Нѣтъ! Богъ, какъ ни какимъ мѣстомъ не заключается, такъ и ни отъ какого мѣста не отлучается, но вездѣ есть. Гдѣ же Его искать? Хрістосъ глаголетъ: \textit{се царствіе Божіе внутрь васъ есть}\footnote{Лук.~17,~21.}. Идѣже бо царствіе Божіе, тамо и Богъ со Своими благими. Какъ сіе неоцѣненное сокровище искать? Многіи многая изобрѣтаютъ посредствія къ тому. Мы да внимаемъ слѣдующимъ: 1)~Отрещися самолюбія; яко тое отъ Бога удаляетъ всякаго. Чрезъ самолюбіе бо и прародители наши отъ Бога отпали, когда заповѣдь Божію преслушали, волѣ своей послѣдовали, и восхотѣли чести, Богу единому подобающія, и тако потеряли тую честь которою отъ Бога высоко превознесены были. "--- 2)~Должно волю свою волѣ Божіей покорить, и Его святому и премудрому промыслу отдать себе, да дѣлаетъ съ нами, якоже хощетъ; да, какимъ хощетъ, насъ путемъ ведетъ "--- гладкимъ или острымъ, веселымъ или печальнымъ, "--- якоже поступаетъ искусный лѣкарь съ немощнымъ, который на волю его отдается и тако исцѣляется. "--- 3)~Когда хощемъ Бога искать, то не должно искать въ мірѣ семъ чести, богатства и славы. Понеже Бога искать, и чести или славы или богатства мірскаго искать вдругъ невозможно. Сердце бо человѣческое раздвоено быть не можетъ, но или къ созданію, или къ Создателю только прилѣпляется; и когда къ созданію прилѣпляется, то отпадаетъ отъ Создателя; а когда къ Создателю пристаетъ, то отлучается отъ созданія. Якоже, хотя два глаза имѣемъ, не можемъ смотрѣть вдругъ и на небо и на землю, и взадъ и впередъ: много паче, едино сердце имѣя, не можемъ и къ Богу прилѣпляться и къ созданію. Надобно неотмѣнно или Бога оставить и искать созданія, или созданіе оставить и Бога искать; или къ Богу на небо смотрѣть, или къ созданію на землю; или Бога предъ собою имѣть и смотрѣть на Него вѣрою, или отвратиться отъ Него и смотрѣть къ созданію: едино изъ двухъ надобно избрать и искать того. Вѣрно слово Хрістово и истинно есть: \textit{идѣже есть сокровище ваше, ту будетъ и сердце ваше}\footnote{Матѳ.~6,~21.}. Кто честь міра сего, или славу, или богатство, или что иное отъ міра сего за сокровище свое вмѣняетъ, къ тому и сердце свое прилагаетъ, тое замышляетъ, о томъ печется, старается, думаетъ, желаетъ и ищетъ со тщаніемъ. Кто Бога за сокровище свое едино имѣетъ, тотъ все, что кромѣ Бога есть, позади оставивши, къ Нему единому сердцемъ, мыслію, тщаніемъ, попеченіемъ и желаніемъ стремится, и хотя многія препятствія отъ міра, плоти и діавола ему чинятся, однакожъ, по подобію огня, къ верху всегда стремится и къ желаемому своему сокровищу силуется, нудится и восходитъ. Духа бо воспятить, удержать и отвратить никто и ничто не можетъ. "--- 4)~Нѣтъ удобнѣйшаго пути ко взысканію и обрѣтенію великаго и высочайшаго Бога, какъ истинное и сердечное смиреніе. Богъ бо, яко благъ и милосердъ, ни на что такъ не преклоняется и не призираетъ, какъ на смиренное и сокрушенное сердце. \textit{Жертва Богу духъ сокрушенъ: сердце сокрушенно и смиренно Богъ не уничижитъ}\footnote{Пс.~50,~19.}. О семъ на многихъ святаго Писанія мѣстахъ свидѣтельствуется, яко Бога смиренное сердце привлекаетъ къ себѣ съ Своею благодатію. Отъ смиренія же сердечнаго неотлучна бываетъ духовная нищета, то"=есть, познаніе и признаніе своего недостоинства, ничтожества, подлости и окаянства во всякихъ вещахъ, тѣлесныхъ и духовныхъ: яко нищій духомъ всего себе недостойнымъ признаетъ за свое ничтожество. Такожде неотлучно терпѣніе истинное и кротость: яко всякаго злополучія достойна себе быти познаетъ, и злотворящимъ не отмщеваетъ, вмѣняя себе быти того достойнымъ. Къ таковому смиренному сердцу Богъ приходитъ, и исполняетъ его Своею благодатію. "--- 5)~Къ сему посредствію принадлежитъ молитва усердная, которая не въ единыхъ словахъ и чтеніи молитвъ, но \textit{въ дусѣ и истинѣ} состоитъ. Поищемъ убо, любезный хрістіанине, сего вѣчнаго и неоцѣненнаго сокровища со всякимъ нашимъ усердіемъ и тщаніемъ, да здѣ, въ семъ вѣкѣ его сыскавше, и въ будущемъ имѣти его будемъ безъ конца. Аще бо временное, скорогиблющее, прелестное и бездушное сокровище люди такъ усердно ищутъ; кольми паче намъ Бога, Иже есть животъ вѣчный, радость и веселіе вѣчное, прилѣжно и всякимъ образомъ искать должно. Аще бо взыщемъ Его, обрящемъ; аще же оставимъ Его, оставитъ насъ въ конецъ, якоже царь Давидъ къ Соломону, сыну своему, глаголетъ: \textit{аще взыщеши Его, обрящется: аще же оставиши Его, оставитъ тя въ конецъ}\footnote{1~Парал.~28,~9.}. Здѣ Онъ токмо ищется, и обрѣтается отъ ищущихъ Его, а по смерти уже нѣтъ никакого способа къ тому. Кто здѣ не сыщетъ, тотъ во вѣки пробудетъ безъ Бога, безъ Котораго, яко живота и живота Источника, не иное что будетъ таковому, какъ вѣчная пагуба и смерть.

\paragraph*{СХІV.} Видишь, что кто въ дружбу съ царемъ земнымъ пріити удостоится, все имѣетъ, кромѣ короны, скиптра и порфиры его; ибо все у него получаетъ, что по волѣ его проситъ: никто его обидѣти, озлобити и обезчестити не можетъ, яко друга царева, но паче всякъ почитаетъ его, и ищетъ у него и чрезъ него у царя милости, якоже много такихъ видимъ въ исторіяхъ. Разумѣй, что тако, или паче несравненно блаженнѣйшимъ бываетъ человѣкъ, который вѣрою сыщетъ Царя небеснаго, и въ святѣйшее Его внити дружество удостоится. Таковому будетъ Богъ Богомъ его, крѣпостію его, утвержденіемъ его, прибѣжищемъ его, избавителемъ его, помощникомъ его, защитителемъ его и заступникомъ его. И съ Богомъ все его будетъ. Оставляетъ честь, славу, богатство, утѣшеніе міра сего; но въ Бозѣ несравненно лучшую честь, славу, богатство и утѣшеніе вѣрою обрѣтаетъ. Убѣгаетъ отъ чести, славы, богатства и роскоши; но за нимъ все превосходнымъ образомъ слѣдуетъ такъ, какъ къ солнцу идущимъ тѣнь послѣдуетъ. Причина тому сія есть: понеже Богъ есть начало и источникъ, отъ Котораго вся благая, видимая и невидимая, проистекаютъ, и какое въ созданіяхъ ни содержится добро, отъ Бога произошло, и тое въ Бозѣ превосходнымъ образомъ имѣется. Никто бо не можетъ подати, чего самъ не имѣетъ, какъ философы истинно научаютъ. Богъ подаетъ всѣмъ бытіе, красоту, смыслъ, разумъ и прочая благая: убо Самъ превосходнымъ образомъ имѣетъ тая. Богъ подаетъ солнцу, лунѣ, звѣздамъ и огню свѣтъ: убо Самъ въ Себѣ превосходнѣйшій есть свѣтъ. Подаетъ всѣмъ животнымъ жизнь: убо Самъ есть превосходнѣйшій животъ. Насаждаетъ въ сердцахъ человѣческихъ любовь: убо Самъ по превосходству въ Себѣ имѣетъ любовь. Тако въ Бозѣ все сокровенно есть, что въ созданіяхъ по малѣйшей части разсѣяно; и сокровенно превосходно и несравненно лучше и болѣе. Такъ какъ отъ солнца просвѣщается воздухъ; убо въ солнцѣ самомъ большій и превосходнѣйшій есть свѣтъ. Когда огнь дѣлаетъ теплоту намъ; убо въ себѣ далеко большую теплоту имѣетъ. Такъ и о прочіихъ естественныхъ вещахъ разумѣй. Тако въ Бозѣ вся благая имѣются, которая въ созданіяхъ суть. И такъ кто Бога обрѣтаетъ, тотъ вся благая съ Богомъ и въ Бозѣ обрѣтаетъ превосходнѣйшимъ образомъ. Блаженнѣйшее и прелюбезное есть сердце тое, въ которомъ сіе всѣхъ благъ сокровище пребываетъ! Весь свѣтъ ему со всѣмъ своимъ сокровищемъ, увеселеніемъ, радостію, утѣхою, славою и честію ничтоже есть: единымъ тѣмъ, что имѣетъ внутрь, довольствуется. Что бо тому въ богатствѣ, чести, славѣ, утѣхѣ мірской, который всѣхъ благъ Источника имѣетъ внутрь себе! На что ему искать утѣшенія внѣ, когда внутрь себе имѣетъ Источника всякія утѣхи? Подобно сіе тому, какъ кто отъ колодезя живыя воды напаяется, оставляетъ ручейки отъ него проистекающіе; кто отъ солнца просвѣщается, не требуетъ отъ свѣчи свѣта; кто всякихъ вещей довольствіе и изобиліе у себе имѣетъ, не имѣетъ нужды отъ инуды искать потребныхъ; кто порфирою царскою одѣвается, не требуетъ другаго подлѣйшаго одѣянія; кто царскую славу имѣетъ, не ищетъ нижайшаго какого чина славы; кто здравіе цѣлое имѣетъ, не требуетъ помощи отъ лѣкарства: тако кто Бога, всѣхъ благъ Источника имѣетъ, какая ему нужда въ прочіихъ міра сего благихъ? И понеже Бога имѣетъ, у Котораго въ руцѣ весь свѣтъ, то и имѣющему Бога все покоряется. Таковый поистинѣ есть \textit{ничтоже имущь, а вся содержащь}\footnote{2~Кор.~6,~10.}. Не имѣетъ богатства; но весь міръ работаетъ ему, якоже читаемъ въ священной и церковной исторіи о таковыхъ, которые домъ и жилище Божіе были. И хотя \textit{многи скорби праведнымъ}\footnote{Пс.~33,~20.}, но \textit{вся имъ поспѣшествуютъ во благое}\footnote{Римл.~8,~28.}, ибо скорби ихъ умножаютъ имъ блаженство, а не отнимаютъ. Лишаются внѣшняго имѣнія; но внутренняго сокровища лишить никто не можетъ. Помрачаютъ славу ихъ; но она, какъ свѣтъ въ темномъ мѣстѣ, паче сіяетъ. Оскорбляютъ тѣло, но они душею радуются; біютъ тѣло ихъ, но они большій любве и ревности по Бозѣ своемъ издаютъ огонь, по подобію кремня, который чимъ болѣе сѣчется, тѣмъ множайшіи испущаетъ изъ себе искры. О блаженное таковаго сердца состояніе! Оно еще на землѣ чувствуетъ небесныя радости частицу. Сего богатства и утѣхи подобаетъ намъ, хрістіанамъ, искать, любезный хрістіанине, чего отъ насъ ни самая смерть не отъиметъ, а, не міра сего сокровища, которое, какъ только пріидетъ, такъ скоро и отъидетъ отъ насъ.

\paragraph*{СХV.} Видишь, что, когда огнь коснется ладона или инаго какого благоуханнаго порошка, тотчасъ куреніе дыма бываетъ, и пріятное благоуханіе восходитъ. Такъ точно бываетъ, когда сердца человѣческаго коснется благодать Святаго Духа: тогда востанетъ въ таковомъ сердцѣ воздыханіе и молитва истинная, аки благоуханіе, отъ огня возбужденное, и восходитъ въ высоту къ небесному Отцу, и обрѣтаетъ у Него благодать и милость. Сей случай насъ учитъ просить у Бога Духа Святаго, дабы Онъ возбуждалъ въ сердцахъ нашихъ истинную молитву, \textit{о Немже вопіемъ: Авва Отче}\footnote{Римл.~8,~15.}!

\paragraph*{СХVІ.} Видишь, что всякая вещь созданная добра есть, и удѣляетъ свою доброту хотящимъ тою пользоватися. Солнце, луна, звѣзды, огнь, имѣютъ свѣтъ, и намъ свѣтъ свой сообщаютъ; вода прохладительна есть, и утоляетъ нашу жажду; земля плодородна, и подаетъ намъ хлѣбъ и насыщаетъ плоть нашу; звѣри, скоты, птицы и рыбы различную намъ приносятъ пользу, и нѣтъ такой вещи во всемъ естествѣ, которая бы доброты въ себѣ не имѣла и не подавала пользы. Отсюду заключается, хрістіанине, что когда созданія суть добра и \textit{добра зѣло}, по свидѣтельству святаго Писанія\footnote{Быт.~1,~31.}, кольми паче Создатель Самъ. Вси бо созданія суть слѣды благости Божіей, которая въ нихъ является; и свидѣтели суть тоя благости, которую безъ гласа намъ свидѣтельствуютъ и проповѣдуютъ, да \textit{вкусимъ и видимъ, коль благъ Господь Самъ}\footnote{Пс.~33,~9.}, Который сотворилъ благая сія. Ибо \textit{небеса повѣдаютъ славу Божію}\footnote{18,~2.}, не гласомъ, но представленіемъ и свидѣтельствомъ славнаго Творца своего, Который такъ дивная и славная дѣла сотворилъ изъ ничего. Тако вся тварь повѣдаетъ благость Божію доказательствомъ благаго Создателя своего, Который такъ благая создалъ. Аще убо дѣло благое есть, кольми паче Самъ Содѣтель. Аще воздухъ отъ солнца свѣтелъ бываетъ, кольми паче само солнце свѣтло; аще медъ вещи сладкими дѣлаетъ, кольми паче самъ медъ сладокъ есть вкушающимъ его. Неспорна сія истина есть здравому разуму. Аще убо созданныя вещи благи суть, и \textit{благи зѣло}, кольми паче Самъ Создатель ихъ. Оттуду слѣдуетъ, любезный хрістіанине: 1)~Созданія намъ, разумной твари, какъ перстомъ, указуютъ на Создателя, и, какъ бы за руку вземше, ведутъ къ познанію благости Божіей, такъ точно, какъ ручьи къ самому источнику живыя воды; и подаютъ намъ случай \textit{вкусить и видѣть, яко благъ Господь}. Какъ бы тако къ намъ слово простирали: смотрите на насъ, мы вси добры отъ Создателя сотворены, мы вси вамъ пользу приносимъ и подаемъ: \textit{вкусите и видите, коль благъ Самъ Господь}, сотворивый насъ! "--- 2)~Тѣмъ самымъ научаютъ насъ Его, яко Источника благихъ, любити, Его единаго искать, отъ Котораго все благое происходитъ, яко существеннаго и вѣчнаго Блага. "--- 3)~Тѣмъ самымъ, что благость Божію проповѣдуютъ, отводятъ насъ отъ себе, то"=есть, отъ любви своей, и приводятъ къ любви Божіей; отвращаютъ сердце наше отъ себе, и привлекаютъ къ любленію высочайшаго того Блага, отъ Котораго сами начало и бытіе свое воспріяли: якоже истекающіе ручьи указуютъ на самый источникъ водный, да изъ него паче самаго піемъ воду, а не отъ ручьевъ. "--- 4)~Научаютъ насъ благодарить Богу, что Его созданными благими пользуемся, довольствуемся и тѣми животъ нашъ сохраняемъ: безъ созданія бо и единой минуты жить не можемъ. "--- 5)~Отсюду познаемъ, что мы сами въ себѣ скудны, нищи, бѣдны и убоги: все бо, что не имѣемъ, Божіе есть добро, а не наше, какъ и сами не свои, но Божіи. "--- 6)~Отсюду учимся познавать наше ничтожество, бѣдность и окаянство, и тако смирятися. "--- 7)~Когда все чуждое добро имѣемъ, а не наше, то есть Божіе, то съ благословеніемъ Божіимъ и молитвою всего касаться и употреблять должно намъ, да не грѣшимъ противу Создателя, созданія Его употребляя безъ Его благословенія. "--- 8)~Употреблять не къ плотоугодію и роскоши, но къ нуждѣ "--- пищу ради укрѣпленія немощныя плоти, питіе ради утоленія жажды, одежды ради согрѣянія и прикрытія наготы, и проч. "--- 9)~Понеже не свое добро имѣемъ, но Божіе, какое ни имѣемъ, то по волѣ и велѣнію Его должно тое сообщать требующей братіи нашей во славу Его, да тому \textit{Единому} слава будетъ, отъ Котораго всякое добро происходитъ. "--- 10)~Понеже мы и сами не свои, но Божіи, того ради и жить не такъ должны, какъ мы хочемъ, но какъ Богъ велитъ, и съ прочіими созданіями, Ему, яко раби Его; работать неослабно: якоже вся созданія Его работаютъ Ему безотступно, такъ какъ Онъ повелѣлъ и учредилъ. "--- 11)~Отсюду послѣдуетъ, что безотвѣтны суть, которые сихъ должностей не исполняютъ: самая бо ихъ совѣсть и законъ естественный осудитъ ихъ въ томъ.

\paragraph*{СХVІІ.} Видишь, что вода во время весны съ великимъ стремленіемъ и шумомъ, течетъ, но вскорѣ и протекаетъ; или дымъ и огнь высоко подымается, но тотъ часъ исчезаетъ. Такое точно состояніе гордыхъ и нечестивыхъ людей: свирѣпѣютъ и шумятъ и они въ гордости своей, но уничтожаются, яко вода мимотекущая; возносятся, какъ пламень и дымъ, высоко, но исчезаютъ такожде, какъ пламень и дымъ, такъ, что и слѣда ихъ не видно бываетъ. О семъ свидѣтельствуетъ пророкъ Божій, очевидный и вѣрный истины свидѣтель: \textit{видѣхъ нечестиваго превозносящася и высящася, яко кедры Ливанскія, и мимоидохъ, и се не бѣ: и взыскахъ его, и не обрѣтеся мѣсто его}\footnote{Пс.~36,~35 и 36.}. Напротивъ того, о богобоящемся глаголется: \textit{яко въ вѣкъ не подвижится. Въ память вѣчную будетъ праведникъ: отъ слуха зла не убоится. Готово сердце его уповати на Господа; утвердися сердце его, не убоится, дондеже воззритъ на враги своя}\footnote{111,~5--8.}, "--- о чемъ и въ псалмѣ 36"~мъ свидѣтельствуется. Сей случай и разсужденіе научаетъ тя, хрістіанине, не послѣдовать примѣру безстрашныхъ и гордящихся богатствомъ, славою и честію міра сего, да не съ ними погибнеши; но смириться подъ крѣпкую руку Божію, и боятися имени Его, да со смиренными и богобоящимися отъ Него \textit{вознесешися}\footnote{1~Петр.~5,~6.}, \textit{яко всякъ возносяйся смирится: смиряяй же себе вознесется}, по ученію вѣчныя Истины "--- Хріста\footnote{Лук.~18,~14.}.

\paragraph*{СХVІІІ.} Видишь слѣды или человѣчьи, или скотскіе, или звѣриные, и отъ тѣхъ познаешь хожденіе ихъ, и самихъ ихъ, чьи слѣды видишь; напр. отъ слѣда человѣческаго человѣческое хожденіе познаешь. При семъ случаѣ помяни, что пророкъ Божій поетъ: \textit{видѣна быша шествія Твоя, Боже, шествія Бога моего Царя}\footnote{Пс.~67,~25.}. Дѣла Бога нашего суть аки слѣды Его, изъ которыхъ познаемъ божественная Его шествія, и Самаго Бога. Егда взираемъ на небо и красоту его, на солнце, луну, звѣзды и свѣтъ ихъ, на землю и исполненіе ея: познаемъ въ нихъ шествіе Бога нашего, и Самаго Бога, шествіе въ нихъ показавшаго, признаемъ. И тако убѣждаемся прославлять всемогущую \textit{силу} Его: яко вся сія изъ ничего словомъ единымъ сотворилъ; "--- \textit{премудрость} Его: яко такъ дивно устроилъ и въ порядкѣ Своемъ все постановилъ; "--- \textit{благость} Его: яко вся сія насъ ради сотворилъ. Самъ бо сихъ Себе ради не требуетъ: ибо какъ прежде вѣкъ, такъ и нынѣ Самъ въ Себѣ доволенъ и блаженъ. Созданія бо суть аки слѣды Бога нашего, изъ которыхъ убѣждаемся познавать Самаго Бога, и яко дивная показуютъ славу Божію, по оному: \textit{небеса повѣдаютъ славу Божію}\footnote{18,~2.}. Не можно бо не удивиться всемогуществу, премудрости и благости Божіей, когда со вниманіемъ разсудить, како солнце ради пользы поднебесной восходитъ и заходитъ, и паки восходитъ и всю безмѣрную небесную широту въ двадесять четыре часа обтекаетъ, "--- и когда приближается къ намъ, приводитъ весну и лѣто, "--- и когда отходитъ отъ насъ, оставляетъ намъ осень и зиму; како на воздухѣ, тончайшей стихіи, облака, аки мѣхи, наполнены водою, держатся и не падаютъ, съ мѣста на мѣсто переходятъ, воду разносятъ и на землю низпущаютъ; како земля, великая и тягчайшая стихія; посредѣ воздуха утверждена и не отступаетъ съ своего мѣста, како различные и безчисленные плоды изъ нѣдръ, аки сокровищъ своихъ, издаетъ; како единъ огнь иныя умягчаетъ, иныя ожесточаетъ вещи, и все какъ бы поядаетъ; како рыба въ водѣ живетъ, но не заливается, ни умираетъ, чего прочія животныя терпѣть не могутъ, како въ составѣ человѣческомъ едина душа, но различныя дѣйствія показуетъ "--- разумѣетъ, хощетъ, помнитъ, удивляется, ужасается и проч.; видитъ, слышитъ, обоняетъ и проч.; како чувства суть едино тѣло, но одно дѣлаетъ тое, чего не можетъ дѣлать другое, "--- напр.: око видитъ, но ухо не видитъ, а только слышитъ; како въ маломъ сѣмени толь великій и многій плодъ сокровенъ и проч. И воистину не ино что сіе есть, какъ слѣды всемогущія Божія премудрости, которая все изобрѣтаетъ, совершаетъ и въ порядкѣ своемъ устрояетъ. О сихъ премудрыхъ и дивныхъ Божіихъ дѣлахъ преизрядно намъ представляетъ псаломъ 103"~й, въ которомъ пророкъ Божій удивляется силѣ и премудрости Божіей, покланяется, поетъ и прославляетъ Создателя въ восхищеніи духа, повторяя: \textit{благослови, душе моя Господа}! Но когда въ священную Писанія исторію посмотримъ, увидимъ преславная Бога нашего шествія и слѣды. Чудесныя Его дѣла, тамо написанная, суть аки слѣды и свидѣтели Его, яко Тойже Богъ, Который сотворилъ чудесно міръ изъ ничего, творитъ и въ сотворенномъ мірѣ дѣла, Ему единому собственная. Казни страшныя, нечестивымъ явленныя, суть слѣды и свидѣтели правды Его, которая дѣйствіе свое являетъ на беззаконныхъ и врагахъ Его. Милости и благодѣянія Его изліянныя суть слѣды и свидѣтели благости Его, которая подаетъ себе ко вкушенію достойнымъ и ищущимъ ея рабамъ Его. При всемірномъ потопѣ на нечестивыхъ, водами погибшихъ, правда Божія; но на Ноѣ, праведномъ, посредѣ такъ ужасныхъ водъ сохраненномъ съ домомъ его, благость Божія показала себе намъ\footnote{Быт. гл.~7"~я.}. Содомъ и Гоморръ, съ окрестными градами сожженные, что, аще не правду Божію, свидѣтельствуютъ? Но Лотъ праведный, отъ того сохраненный, благость и милость Его намъ проповѣдуетъ\footnote{Быт. гл.~19"~я.}. Во исходѣ Израилевѣ изъ Египта видимъ такожде слѣды правды, премудрости и милости Божія. Гордый и ожесточенный Фараонъ съ воинствомъ своимъ правды Божія дѣло на себѣ узналъ, который не хотѣлъ милости Его, столько разъ явленныя, разумѣть\footnote{Исх. гл.~14"~я.}; вѣрный Божій Израиль, озлобленный отъ него, милость Божію поетъ и проповѣдуетъ намъ; всемогущая Божія мудрость тамо путь ему къ спасенію изобрѣла, гдѣ казалося не быть пути; и гдѣ человѣкъ отчаявался спасенія, тамо указала ему способъ спасенія, такъ что Израиль, которому, по мнѣнію человѣческому, слѣдовало погибнуть, чудесно спаслся, и отъ котораго надѣялся погибели, того увидѣлъ погибшаго въ морѣ. Таковыя же Божія \textit{шествія} и слѣды Его видимъ и въ слѣдующихъ временахъ и мѣстахъ. Видимъ въ пустыни, якоже поетъ Псаломникъ: \textit{Боже! внегда исходити Тебѣ предъ людьми Твоими, внегда мимоходити Тебѣ въ пустыни: земля потрясеся}\footnote{Пс.~67,~8.}. И паки: \textit{Отъ лица Господня подвижеся земля, отъ лица Бога Іаковля, обращшаго камень во езера водная, несѣкомый во источники водныя}\footnote{113,~7 и 8.}. Видимъ страшные праведнаго Его гнѣва слѣды, когда \textit{отверзеся земля и пожре Даѳана, и покры на сонмищи Авирона: и ражжеся огнь въ сонмѣ ихъ, пламень попали грѣшники}\footnote{105,~17 и 18.}. Видимъ милостиваго Его о людяхъ промысла слѣды, когда люди, въ пустыни угрызаемые отъ зміевъ, взирали на змію, по Его повелѣнію вознесенную, и тѣмъ исцѣлѣвалися: не змія бо ихъ исцѣлѣвала, но Его божественная промыслительная сила. Видимъ при вхожденіи людей Его въ землю обѣтованную, когда рѣку Іорданъ, яко сушу прешли; когда путь имъ открылся, гдѣ пути не было, когда стѣны іерихонскія отъ трубнаго гласа пали предъ ними, и проч. Въ Новомъ Завѣтѣ, егда Богъ во плоти явился, таковыяжде, паче же чуднѣйшія Божія \textit{шествія} видимъ. Вся церковь, по вселенной разсѣянная, но единымъ Духомъ и вѣрою соединенная, знаетъ и признаетъ Хріста обезчещеннаго, посмѣяннаго, уязвленнаго, умученнаго, распятаго и умершаго между злодѣями; но Тогожде воскресшаго и прославленнаго Царя своего съ радостію исповѣдуетъ, дерзаетъ о Немъ, хвалится о Немъ; въ похвалу себѣ поставляетъ, что вѣруетъ въ Него; надежду, утѣшеніе, радость, веселіе и животъ вѣчный полагаетъ въ Немъ; наконецъ Его Бога своего исповѣдуетъ, почитаетъ, прославляетъ и покланяется Ему. Что сіе есть, аще не слѣдъ всемогущія Божія силы? Что, аще не божественная сила на крестѣ распятаго Хріста дѣйствуетъ? \textit{Сила бо Божія есть во спасеніе всякому вѣрующему}\footnote{Римл.~1,~16.}. Кто можетъ надѣятися отъ умученнаго и на крестѣ повѣшеннаго спасенія и избавленія своего? Но сила Божія, которая есть во спасеніе всякому вѣрующему, дѣлаетъ, что во Хрістѣ распятомъ свое спасеніе и животъ вѣчный полагаемъ, вѣрующіи во имя Его; и инаго Избавителя и Спасителя не знаемъ и не признаемъ, кромѣ Его \textit{единаго}: и хотя умираемъ и полагаемся во гробѣхъ и растлѣваемъ, и яко земля въ землю возвращаемся и обращаемся, однакожъ твердо и безъ сумнѣнія всякаго надѣемся, о Немъ \textit{умершемъ и воскресшемъ} востати, ожити, новый, лучшій и вѣчный животъ получити. И, что умножаетъ чудо, не чрезъ мудрецовъ и риторовъ вѣка сего, не чрезъ славныхъ и великородныхъ вселенная отдала и записала сердца свои и имена Распятому, но чрезъ простыхъ, некнижныхъ, худородныхъ, скинотворцевъ, рыбарей, мытарей. Сколько ихъ было, которые такъ великое, неслыханное и умомъ непонятное совершили дѣло? Дванадесять. Сему, такъ малому числу надобно было проитить всю поднебесную, по реченному: \textit{шедше научите вся языки}\footnote{Матѳ.~28,~19.}; и паки: \textit{шедше въ міръ весь, проповѣдите Евангеліе всей твари}\footnote{Марк.~16,~13.}. Симъ немногимъ, по человѣческому разуму невѣждамъ, надобно было стать противу премногихъ вѣка сего мудрецовъ, и заградить уста ихъ; симъ немногимъ простымъ, худороднымъ и безоружнымъ, слѣдовало вооружиться противу безчисленныхъ славныхъ, великородныхъ вельможъ, князей и царей всея земли, и побѣдить ихъ. Учинили тое: \textit{они же изшедше, проповѣдаша всюду}\footnote{20.}, \textit{во всю землю изыде вѣщаніе ихъ}\footnote{Пс.~18,~5.}; стали некнижные противу премудрыхъ книжниковъ и совопросниковъ вѣка сего, и вооруженныхъ безъ оружія одолѣли. Надобно было имъ погруженный и застарѣлый въ идолобѣсіи міръ изъ глубины погибели изторгнуть, и насадить истинное богопочитаніе; прелесть діавольскую доказать, и научить истинѣ; злобою обветшавшій народъ обновить, и привести къ истинному покаянію; уставы, законы, обычаи застарѣлые опровергнути, и новые поставить и утвердить; невѣріе отъ человѣческихъ сердецъ, глубоко вкоренившееся, отнять, и насадить вѣру; прелестное многобожіе уничтожить, и единаго Бога, но въ тріехъ лицахъ покланяемаго, въ познаніе привести. Все сіе умъ человѣческій и силу превосходитъ. Они тое въ дѣло и совершенство произвели. Надобно было имъ проповѣдывать Хріста Іисуса \textit{распятаго}, и доказать и увѣрить невѣрныя сердца, что Сей Іисусъ, Егоже проповѣдуютъ, есть Хрістосъ, Мессія, пророками проповѣданный, отъ Бога обѣщанный, есть Искупитель, Избавитель и Спаситель міра; увѣрить, что распятый Іисусъ не за Свои грѣхи, но за наши распятъ, пострадалъ и умеръ, и Своею волею, и благоволеніемъ Божіимъ; увѣрить, что въ \textit{распятомъ} Хрістѣ животъ нашъ и блаженство и слава вѣчная сокровенна есть; что Онъ какъ пострадалъ, распялся и умеръ \textit{Своею волею}, такъ изъ мертвыхъ восталъ \textit{Своею силою}; что Онъ есть Единородный Божій сынъ, и есть истинный Богъ, плотію отъ Дѣвы неискусобрачныя рожденный, и безъ Него и кромѣ Его нѣтъ никакого посредствія къ полученію вѣчныя жизни и блаженства; увѣрить, что тѣлеса умершихъ, согнившія и въ землю обратившіяся, возстанутъ съ душами совокупившеся, и въ нетлѣніе, безсмертіе и прекрасный вѣчныя славы видъ умершіе о Господѣ облекутся. Все сіе какъ умъ человѣческій не вмѣщаетъ, такъ и учинить выше силы человѣческія есть. Но дванадесять некнижные и препростые совершили тое, и доказали, что \textit{нѣтъ инаго имене подъ небесемъ, даннаго въ человѣцѣхъ, о немже подобаетъ спастися намъ, кромѣ Іисуса Хріста Назореа}\footnote{Дѣян.~4,~10--12.}. Надобно было имъ, наконецъ, отвратить сердца людей отъ міра сего, къ которому прилѣпилися и пристрастилися, и обратить ко Хрісту умаленному, обезчещенному, уязвленному, ко кресту пригвожденному, между злодѣями повѣшенному, и безчестною смертію умершему, дабы тѣмъ не стыдилися, что въ распятаго вѣруютъ, почитаютъ Его, Царя и Бога своего исповѣдуютъ, но паче тѣмъ хвалилися, радовалися, и предъ врагами истины исповѣдывали и свидѣтельствовали небоязненно; увѣщать всѣхъ, чтобъ ради Его не боялися не токмо нищеты, безчестія, поношенія, изгнанія, ранъ, узъ, темницъ, но и самыя смерти; отвратить отъ любленія временныхъ и земныхъ благъ, которыя видятъ и чувствуютъ, и обратить къ желанію и исканію вѣчныхъ, которыхъ \textit{око не видѣ, ухо не слыша и на сердце человѣку не взыдоша}; доказать и увѣрить, что къ сему не иный путь есть, какъ путь крестный, прискорбный и тѣсный, и на томъ пути поставить ихъ, утвердить и къ теченію ободрить. Все сіе выше предпріятія и силы человѣческія было. Но дванадесять апостоли Хрістовы содѣлали, чего сила человѣческая предпріять и содѣлать не могла и не можетъ. Чего не дѣлали Іудеи, враги Хрістовы? какого тщанія не полагали въ началѣ, чтобъ славу распятаго и воскресшаго Хріста помрачить? Воинамъ стрегущимъ руки исполнили мздою, дабы лжою покрыли воскресеніе Его. Апостоламъ какихъ запрещеній, угроженій, озлобленій не дѣлали, дабы не учили о имени распятаго Хріста, какъ читаемъ въ Дѣяніяхъ? но однакожъ ничего не успѣли. Пронеслося имя Іисусово во вси концы земли, и какъ солнца восходящаго свѣтъ, воскресшаго Хріста слава всю поднебесную наполнила. Разсѣявшимся по вселенной некнижнымъ симъ и простымъ какъ ни сопротивлялися книжники, совопросники и мудрецы вѣка сего, какихъ прекословій ни соплетали противу ихъ; но наконецъ признали за истину слово ихъ, и, что за буйство вмѣняли, тое за истинную мудрость приняли: отвергли свою мнимую и хитроплетенную мудрость, и пристали къ простѣйшей, не въ словахъ, но въ силѣ состоящей, простыхъ и безкнижныхъ мудрости; и которымъ, какъ невѣждамъ, смѣялися, у тѣхъ учитися истинной премудрости возжелали. Какъ смутилися и возшумѣли вельможи и князи съ людьми своими? чего не дѣлали? какихъ ни изобрѣтали орудій противу сихъ безоружныхъ? Изгоняли, озлобляли, били, уязвляли, мучили, въ темницахъ и узахъ заключали, и всякій видъ мученія имъ изобрѣтали; но ничего не успѣли. Побѣждены отъ безсильныхъ сильные, отъ безоружныхъ вооруженные, отъ подлыхъ славные, отъ простыхъ великородные, которыхъ взяли, озлобляли, били, уязвляли и мучили, тѣмъ или удивлялися, или покорялися со всѣми людьми своими, отдавали себе имъ сердцами своими, паче же чрезъ ихъ "--- Хрісту распятому. Какая сила человѣческая сіе учинить могла? Воистину \textit{сія измѣна есть десницы Вышняго}\footnote{Пс.~76,~11.}. Какой огнь любве возгорѣлся къ Распятому въ сердцахъ всякаго пола и чина, мужей и женъ, старыхъ и младыхъ, юношъ и дѣвъ, господъ и рабовъ, "--- въ сердцахъ, глаголю, мразомъ идолобѣсія и невѣрія оледенѣвшихъ, такъ что \textit{ничто не сильно было ихъ отвратить отъ любве Божія, яже о Хрістѣ Іисусѣ}\footnote{Римл.~8,~38 и 39.}. Какихъ орудій не изобрѣтали ожесточенные князи и царіе міра сего, и прочіе слуги діавольскіе, чтобъ огнь сей божественный угасить, и отъ исповѣданія и любви распятаго Іисуса отвратить! Мечами уязвляли и посѣкали, въ темницахъ и узахъ заключали, огнемъ жгли и сожигали, въ волахъ мѣдяныхъ разжженныхъ повергали, на сковородахъ и рѣшеткахъ желѣзныхъ раскаленныхъ, какъ мясо въ снѣдь пекли; къ колесамъ привязывали и обращали, трезубцами и прочіими орудіями плоть терзали, кожу сдирали, до костей обнажали, въ сапоги, гвоздьми набитые, обували и какъ скотъ гоняли; звѣрямъ лютымъ повергали, въ водахъ потопляли, ко крестамъ пригвождали, и прочая ужасная вымышляли мученія, но ничего не успѣли. Не могли побѣдить вѣры, которая \textit{есть побѣда побѣждшая міръ}\footnote{1~Іоан.~5,~4.}; не могли погасить пламене любве \textit{воды многи}. Что бо могла успѣть злоба человѣческая, гдѣ сила Божія дѣйствовала? И, что дивнѣе, самый недорослый возрастъ, отроки и отроковицы, оставляя идоловъ, оставляли и родителей своихъ, идолопоклонниковъ, и лучше изволяли все терпѣть и животъ оставить, нежели Хріста отрещися. Какъ вси премудро и дерзновенно противу князей и царей, которыхъ взоръ единъ устрашити ихъ моглъ, витійствовали, "--- надивиться не можно! Когда монархъ, а паче мучитель, гнѣвомъ и яростію пылаетъ, "--- кто можетъ противу его что дерзати, а паче подлый, простый и никогда его невидавшій? Но мученицы святіи, иные безкнижные, иные подлые, въ селахъ и деревняхъ рожденные и воспитанные, иные въ недоросломъ и отроческомъ возрастѣ находящіися, такъ дерзали противу мучителей страшныхъ, которыхъ высокая царская власть, гнѣвъ, ярость и звѣрообразный видъ моглъ устрашить и безгласными учинить; и обличали и укоряли ихъ, которыхъ весь свѣтъ трепеталъ. Воистину Божіе дѣло сіе есть! воистину сила Божія укрѣпляла ихъ! воистину Духъ Божій какъ дѣйствовалъ, такъ и глаголалъ въ нихъ, по писанному: \textit{не вы будете глаголющіи, но Духъ Отца вашего, глаголяй въ васъ}\footnote{Матѳ.~10,~20.}! "--- Тако, возлюбленный хрістіанине, дванадесять препростые и безкнижные буйствомъ проповѣди своея успѣли, что весь свѣтъ Хрісту распятому, яко праведному Царю и Царю славы, подъ сладкое подклонили иго; учинили тое рыбари, мытари и скинотворцы, чего никакимъ образомъ учинить не могли мудрецы вѣка сего. Силу бо человѣческую предпріятіе и дѣло сіе превосходитъ. Оттуду заключается, что всемогущая сила Божія дѣйствовала въ нихъ, якоже пишется: \textit{они же изшедше, проповѣдаху всюду, Господу поспѣшствующу, и слово утверждающу послѣдствующими знаменьми}\footnote{Марк.~16,~20.}. И тако исполнилося Хрістово слово, которое прежде страсти Своей изрече: \textit{аще Азъ вознесенъ буду отъ земли, вся привлеку къ Себѣ}\footnote{Іоан.~12,~32.}. Воистину привлекъ къ Себѣ отъ востока и запада, и сѣвера и юга разсѣянныя овцы Своя, и привелъ во дворъ церкве Своея, якоже паки рече: \textit{ины овцы имамъ, яже не суть отъ двора сего, и тыя Ми подобаетъ привести, и гласъ Мой услышатъ: и будетъ едино стадо, и единъ Пастырь}\footnote{10,~16.}. А тако видимъ \textit{шествія Божія} слѣды, и отъ тѣхъ познаемъ Самаго Бога, Который изшелъ на спасеніе людей Своихъ, якоже поетъ Ему пророкъ: \textit{изшелъ еси на спасеніе людей Твоихъ, спасти помазанныя Твоя} пришелъ еси\footnote{Авв.~3,~13.}. И Псаломникъ: \textit{видѣна быша шествія твоя, Боже, шествія Бога моего царя}\footnote{Пс.~67,~25.}. Сколько еретиковъ и хульниковъ возставало, и славу распятаго и воскресшаго Хріста помрачать тщалися, и церковь Его одолѣть; но не могли того силы и тщанія человѣческія разорить, что всемогущая Божія сила и премудрость создала. Утверждена стоитъ церковь, и до конца стояти будетъ на камени, яко \textit{и врата адова не одолѣютъ сей}\footnote{Матѳ.~16,~18.}. Камень же есть Хрістосъ, сынъ Бога живаго, отъ Отца небеснаго во основаніе Сіону положенный, по Писанію: \textit{се полагаю въ Сіонѣ камень краеуголенъ, избранъ, честенъ, и вѣруяй въ онь не постыдится}\footnote{1~Петр.~2,~6.}. А отсюду видимъ, что какъ создать міръ, такъ и возобновить, единаго Бога дѣло есть; и Который создалъ міръ, Тотъ падшій и обветшавшій міръ возставилъ и возобновилъ. И какъ изъ видимаго созданія познаемъ Бога и Его всемогущество, премудрость и благость, такъ изъ обращенія тмою заблужденія помраченныхъ языковъ ко Хрісту, и вѣры ихъ во Хріста, познаемъ Тогожде всемогущую силу; и какъ въ созданіи, такъ и въ искупленіи міра видимъ \textit{слѣды} Бога нашего, и \textit{самаго Бога} видимъ. "--- Случай убо сей и разсужденіе его научаетъ тя, возлюбленный хрістіанине, 1)~заграждать уста безбожныхъ человѣковъ, которые или присносущное бытіе Божіе, или Промыслъ Его святый и премудрый отъ созданій отнимаютъ хульно, которые слѣпой нѣкакой натурѣ все приписуютъ, и такъ вѣру свою защищать отъ нечестивыхъ и хульныхъ устъ ихъ, которыхъ тѣмъ болѣе умножается, чимъ болѣе міръ къ концу приближается, и дерзновенно имъ отвѣщать, что есть Богъ, Который какъ міръ изъ ничего создалъ, такъ и падшій возстановилъ, исправилъ, и обновилъ чудесно, обновилъ силою страданія и креста Сына Своего Іисуса Хріста, Господа нашего; привлеклъ къ Себѣ отпадшихъ языковъ не чрезъ хитрецовъ и мудрецовъ вѣка сего многихъ, но чрезъ дванадесять простыхъ и безкнижныхъ, силою Его вооруженныхъ и Святымъ Его Духомъ умудренныхъ; Которому, какъ удобно было міръ создать, такъ удобно есть содержать, хранить и промышлять о немъ, невѣрія тмою помраченныхъ просвѣщать, изъ зла добро дѣлать, погибшихъ и повинующихся благости Его спасать. "--- 2)~Хранить и утверждать вѣру свою, которая многоразличному искушенію подлежитъ. "--- 3)~Удивляться благости Божіей, которая никакою злобою побѣждена быть не могла и не можетъ. Человѣкъ врагомъ Божіимъ учинился, но благость Его не отринула его во вѣки отъ себе; но такъ чудесный къ спасенію его способъ премудро изобрѣла, и въ самое дѣйствіе произвела, что прежде вѣкъ въ божественномъ Его совѣтѣ было поставлено и опредѣлено. "--- 4)~Удивляться силѣ, всемогуществу и премудрости Божіей, что Ему какъ малое, такъ и великое и намъ непонятное дѣло сдѣлать и совершить весьма удобно. Ему удобнѣе міръ создать изъ ничего, содержать и хранить, нежели намъ что нибудь малое изъ чего либо сдѣлать. Ему удобно тамо показать намъ путь и способъ къ спасенію, гдѣ намъ отнюдь не видится спасеніе; тамо подать помощь, гдѣ никакой помощи не надѣялися, тамо подать избавленіе, гдѣ по нашему мнѣнію не можетъ быть избавленіе, "--- чимъ мы надежду нашу укрѣплять должны. "--- 5)~Видимъ отсюду, что волѣ и совѣту Божію ничто противиться не можетъ. Какихъ сопротивленій ни чинилося отъ діавола и слугъ его проповѣди евангельской, и отъ той спасенію человѣческому, но всѣ ничего не успѣли; паче же въ лучшее обратилися Божіею силою и премудростію, такъ, что чимъ болѣе умножалось гоненій и прочіихъ бѣдъ на вѣрующихъ, тѣмъ болѣе умножалося вѣрующихъ, какъ читаемъ о мученикахъ, которыхъ тѣмъ большее число умножалося, чимъ болѣе умерщвлялися. Вѣрно и твердо слово есть Псаломника: \textit{Господь разоряетъ совѣты языковъ, отметаетъ же мысли людей и отметаетъ совѣты князей: совѣтъ же Господень во вѣкъ пребываетъ, помышленія сердца Его въ родъ и родъ}\footnote{Пс.~32,~10 и 11.}. "--- 6)~Видимъ отсюду, что человѣческое дѣло есть глаголати, обличати, увѣщавати; но Божіе есть въ сердцахъ дѣйствовати: ибо слово Божіе проповѣдуемое есть какъ орудіе, чрезъ которое Духъ Святый въ сердцахъ слушающихъ оное дѣйствуетъ. "--- 7)~Отсюду послѣдуетъ, что истинное обращеніе и покаяніе есть дѣйствіе Божіе. Пасти сами можемъ, но востати сами не можемъ; погубити себе можемъ, но спасти себе не можемъ; заблудити можемъ, но обратитися на путь истины не можемъ безъ Бога: надобно пастырю искати овцу погибшую. Приписуется же человѣку обращеніе и покаяніе потому, что онъ зовущей и взыскующей благодати не противится; а которые противятся и не слушаютъ ея, тѣ сами отъ себе погибаютъ.

\paragraph*{СХІХ.} Видишь, что малое отроча всегда во всякихъ нуждахъ къ матери своей прибѣгаетъ; ясти ли, пити, или инаго чего хощетъ у матери проситъ: бѣду ли какую видитъ, къ матери прибѣгаетъ. И хотя отъ матери біеніе или наказаніе какое пріемлетъ, однакожъ не отлучается отъ нея, но при ней держится и прилѣпляется ей, любитъ ее и всего отъ ней проситъ. Такова имѣется вѣра хрістіанская, которая Бога за Отца признаетъ и имѣетъ. Она все, что кромѣ Бога есть, оставляетъ позади, и къ Нему единому во всякихъ нуждахъ прибѣгаетъ, глаголя съ пророкомъ: \textit{Ты еси упованіе мое, часть моя еси на земли живыхъ}\footnote{Пс.~141,~6.}; и паки: \textit{прилпѣ душа моя по Тебѣ, мене же пріятъ десница Твоя}\footnote{62,~9.}. Таковая вѣрная душа всего у Бога, Отца своего, со смиреніемъ и страхомъ проситъ и ожидаетъ: \textit{яко всяко даяніе благо и всякъ даръ совершенъ свыше есть, сходяй отъ Отца свѣтовъ}\footnote{Іак.~1,~17.}. Получаетъ ли просимое, "--- благодаритъ: не получаетъ, "--- однакожъ не отчаевается, но съ терпѣніемъ ожидаетъ, съ пророкомъ вопія: \textit{къ Тебѣ возведохъ очи мои, живущему на небеси. Се, яко очи рабъ въ руку господій своихъ, яко очи рабыни въ руку госпожи своея: тако очи наши ко Господу Богу нашему, дондеже ущедритъ ны}\footnote{Пс.~122,~1 и 2.}. И якоже отроча, видя напасть и бѣду, къ матери бѣжитъ, и у ней помощи и защищенія ищетъ; и въ болѣзни и немощи находясь, къ матери воздыхаетъ и простираетъ руки, и немощь свою матери объявляетъ: тако вѣрное сердце, во всякомъ искушеніи и бѣдствіи находящемъ, къ Богу бѣжитъ, и у Него помощи и защищенія проситъ, съ прочіими вѣрными вопія къ Нему: \textit{Господи! Прибѣжище былъ еси намъ въ родъ и родъ}\footnote{Пс.~89,~2.}; и паки: \textit{Ты еси прибѣжище отъ скорби обдержащія мя. Радосте моя! избави мя отъ обышедшихъ мя}\footnote{31,~7.}. И хотя отъ Отца Небеснаго наказаніе пріемлетъ, однакожъ не отступаетъ отъ Него; за благо пріемлетъ все, что отъ благаго Бога посылается, поминая утѣшительное слово, которое сынамъ Своимъ глаголетъ: \textit{сыне Мой! не пренемогай наказаніемъ Господнимъ, ниже ослабѣй отъ него обличаемъ. Егоже бо любитъ Господь, наказуетъ: біетъ же всякаго сына, егоже пріемлетъ}\footnote{Евр.~12,~5 и 6.}. И якоже отроча незнакомыхъ и чужихъ людей удаляется и бѣжитъ отъ нихъ, опасаяся ихъ, хотя и опасности нѣтъ никакой: тако вѣрная душа, понеже не есть отъ міра сего, отъ него и отъ всего, что въ немъ имѣется, какъ чуждаго, удаляется и опасается, чтобъ чрезъ тое отъ Бога не удалиться и не лишиться Его; честь, славу, богатство и сласть міра сего за подозрѣніе имѣетъ; бережется отъ тѣхъ, которые и дѣломъ и словомъ земная и тлѣнная мудрствуютъ; бережется яко чужихъ, не обходится съ ними дружески, опасаяся, дабы чрезъ ихъ дружество отъ Бога своего не удалиться, и чрезъ любовь ихъ Божіей любви не отпасть. И якоже отроча, хотя и манятъ его чужіе и привлекаютъ къ себѣ яблокомъ или другимъ чимъ, подобнымъ тому, отъ матери своей не отстаетъ и не идетъ къ чужимъ; а хотя и случается, что приходитъ къ нимъ и пріемлетъ отъ нихъ подаемое, но со страхомъ, и тотчасъ, принявши, возвращается и бѣжитъ къ матери своей: тако вѣрная душа, хотя прельщаетъ ее міръ сей честію, славою, богатствомъ, утѣхою своею, не соизволяетъ ему, не отстаетъ отъ Бога своего; не держится его, опасаяся прелести міра, \textit{яко все, еже въ мірѣ, похоть плотская и похоть очесъ и гордость житейская, нѣсть отъ Отца, но отъ міра сего есть}\footnote{1~Іоан.~2,~16.}. И хотя случается, что подаемую честь пріемлетъ, но пріемлетъ со страхомъ, бояся, чтобы чрезъ честь, которая по большей части нравы добрые въ худые перемѣняетъ, отъ Бога не отпасти, и тую честь, какъ послушаніе, а не яко утѣшеніе свое, вмѣняетъ, и со страхомъ ее проходитъ въ славу Бога своего и пользу ближняго. Такожде, хотя богатство подается, не прилагаетъ сердца къ нему, но къ единому Богу прилѣпляется; пищу, питіе, одежды, покой и покровъ къ нуждѣ, яко безъ того невозможно прожити, а не къ сладострастію и роскоши употребляетъ; употребляетъ со страхомъ Божіимъ, чтобъ излишествомъ Бога не прогнѣвать, и тако Его милости не лишиться и отъ Него не отпасть. И якоже отроча, хотя его чужіе люди отъ матери отторгаютъ и оттаскиваютъ прочь, крѣпко держится при матери своей, обымаетъ ее, плачется и кричитъ, чтобъ не отторгнуться отъ ней: тако вѣрное сердце, хотя его тщится діаволъ и злый міръ наведеніемъ бѣдъ, скорбей и напастей, отторгнуть отъ Бога, однакожъ крѣпко держится Бога своего, усердно молится Ему и проситъ Его, чтобъ не оставилъ въ напасти и помоглъ противу враговъ Своихъ. Словомъ: якоже отроча все благополучіе свое вмѣняетъ быть при матери, и неблагополучіе поставляетъ удалиться отъ нея: тако вѣрная душа за блаженство и утѣшеніе имѣетъ единому Богу прилѣпляться, и за бѣду поставляетъ къ созданію пристать и тако отъ Создателя отпасть. Отъ сего случая учися, хрістіанине, что есть хрістіанская и живая вѣра, которая безъ страха Божія и любви не можетъ быть. Читай псалмы святые, писанія пророческая и апостольская и житія святыхъ, вѣрою угодившихъ Богу, и уразумѣеши истину, и тою вѣрою, которая только на устахъ носится, а не въ сердцѣ, не прельщайся, "--- каковою вѣрою многіи наполнены.

\paragraph*{СХХ.} Видишь, что люди по общему обычаю въ праздничные дни одѣваются въ одежды свѣтлыя и веселятся. Отъ сего временнаго и земнаго празднованія возведи умъ свой вѣрою къ празднику избранныхъ Божіихъ, которые вѣчно имѣютъ праздновати и веселитися. Нынѣшнее хрістіанъ время есть время трудовъ и подвига, плача, сѣтованія, печали и крестоношенія: въ ономъ вѣкѣ, егда время минется и настанетъ вѣчность, открыется истиннымъ хрістіанамъ пресвѣтлый праздникъ. Тогда они упокоятся отъ трудовъ, и будутъ праздновати субботу безпрестанно и вѣчно, не единъ день въ седмицѣ, но всю вѣчность; тогда совлекутся рубищнаго и плачевнаго одѣянія, и облекутся въ праздничную и брачную ризу, \textit{егда тлѣнное сіе облечется въ нетлѣніе, и смертное сіе облечется въ безсмертіе}\footnote{1~Кор.~15,~54.}. Тогда \textit{облекутся въ ризу спасенія, и одеждою веселія одѣются}\footnote{Пс.~61,~10.}; \textit{облекутся въ ризы бѣлыя}\footnote{Апок.~3,~5.}; \textit{облекутся въ вѵссонъ чистъ и свѣтелъ}\footnote{19,~8.}. Въ тые дни не увидятъ ничего печальнаго, скорбнаго; не будетъ тамо страха отъ враговъ, нищеты и болѣзни; не будетъ вопля, жалобы, сѣтованія, глада, жажды, холода, горячести, никакого неблагополучія, никакого бѣдствія, но все тихо, мирно, безопасно, весело, свѣтло, радостно, благопріятно. Въ семъ житіи почти все печально и плачевно; страхъ отъ смерти, страхъ отъ діавола и грѣха, страхъ отъ непріятелей и лжебратій; всего надобно опасаться и осматриваться по подобію птицы отъ ловцовъ и стрѣльцовъ. Здѣ по вся дни умираемъ чаяніемъ и ожиданіемъ смерти; и чимъ болѣе живемъ, тѣмъ болѣе чаемъ умрети: яко чимъ болѣе живемъ, тѣмъ паче уменьшается житіе наше и ближе наступаетъ смерть. И сіе горькое и смертное житіе многими преисполнено бѣдами; и чимъ болѣе продолжается оно, тѣмъ болѣе умножается бѣдъ. Но въ ономъ вѣкѣ ничего того не будетъ, но едина только жизнь, блаженство, радость, торжество, восклицаніе и веселіе вѣчное. "--- Нынѣ люди обыкли во время праздниковъ другъ къ другу приходити въ домъ, и другъ друга посѣщать и тако купно веселитися: тогда избранные Божіи соберутся въ домъ небеснаго Отца, въ которомъ \textit{обители многи суть}, и возъимѣютъ любезное дружество со святыми ангелами и всѣми небесными силами, и будутъ купно веселитися и радоватися другъ о другѣ предъ лицемъ небеснаго своего Отца. Тогда \textit{собранніи Господемъ обратятся, и пріидутъ въ Сіонъ съ радостію: и радость вѣчная надъ главою ихъ; надъ главою бо ихъ хвала и веселіе, и радость пріиметъ я, отбѣже болѣзнь, печаль и воздыханіе}\footnote{Ис.~35,~10.}. "--- Нынѣ обычай имѣютъ люди въ праздники купно ясти, пити и ликовати: тогда работающіи Господу \textit{возлягутъ со Авраамомъ, Исаакомъ и Іаковомъ во царствіи небеснѣмъ}\footnote{Матѳ.~8,~11.}, и \textit{будутъ ясти и пити на трапезѣ Господней}\footnote{Лук.~22,~30.}; \textit{упіются отъ тука дому Господня, и потокомъ сладости Его напоятся}\footnote{Ис.~35,~9.}, якоже о семъ чрезъ пророка Своего Богъ намъ въ утѣшеніе предвозвѣстилъ: \textit{се работающіи Ми ясти будутъ, се работающіи Ми пити будутъ, се работающіи Ми возрадуются, се работающіи Ми возвеселятся въ веселіи сердца}, и проч.\footnote{Ис.~65,~13 и 14.} "--- Паки обычай люди имѣютъ во дни праздничные радостныя пѣти пѣсни: тогда избранніи Божіи \textit{увидятъ Бога лицемъ къ лицу}\footnote{1~Кор.~13,~12.}, \textit{увидятъ Его въ славѣ Его}\footnote{1~Іоан.~3,~2.}, и будутъ единодушно хвалити, славити и пѣти Его безъ конца, сытости и труда. О коль свѣтелъ, веселъ и радостенъ праздникъ будетъ тогда, любезный хрістіанине! Коль вожделѣненъ день той, въ который увидимъ Бога лицемъ къ лицу, Котораго нынѣ видимъ, якоже зерцаломъ въ гаданіи! \textit{Блажени живущіи въ дому Твоемъ: въ вѣки вѣковъ восхвалятъ Тя, Царю мой и Боже мой}\footnote{Пс.~83,~5 и 4.}. Симъ случаемъ, хрістіанине, и подобными сему научаемся возводить умъ нашъ и сердце къ блаженной вѣчности, и искреннею вѣрою и усердіемъ искать того.

\paragraph*{СХХІ.} Видишь, что садовникъ очищаетъ дерево, напр., яблонь, или иное какое, отрѣзываетъ негодные сучки и отрасли, которые дерево въ безсиліе приводятъ и препятствуютъ къ проращенію плодовъ, а далѣе изсушаютъ тое, когда сначала не обрѣзываются. Всякій человѣкъ есть какъ древо, у котораго извнутрь отъ сердца исходятъ, какъ отрасли и сучки, помышленія злая, похоть нечистая, гнѣвъ, злоба, сребролюбіе, хищеніе, ложь, лукавство, лихоиманіе, роптаніе, хула и прочая, якоже глаголетъ Хрістосъ: \textit{отъ сердца исходятъ помышленія злая, убійства, прелюбодѣянія, любодѣянія, татьбы, лжесвидѣтельства, хулы}\footnote{Матѳ.~15,~19.}. Якоже убо садовникъ обрѣзываетъ древо, и исходящіе изъ него сучки негодные и отрасли обрѣзываетъ, дабы, возрастше, самаго дерева не повредили: тако подобаетъ хрістіанину всякому обрѣзывати сердце свое и обрѣзывати помышленія злая, какъ только начнутъ отъ сердца исходити и показыватися, дабы, возрастая, не повредили внутренняго духовнаго человѣка. Якоже бо возрастшіе на деревѣ сучки и отрасли негодные отъ дерева отнимаютъ сокъ, и тако безплоднымъ тое дѣлаютъ, а далѣе и изсушаютъ: тако злые помыслы, когда въ началѣ не пресѣкаются, но попущаются возрастать, внутренняго духовнаго человѣка въ изнеможеніе приводятъ, безплоднымъ дѣлаютъ, и дѣлаютъ, какъ розгу изсохшую, душу человѣческую, которая вонъ изъ винограда Хрістова, яко розга изсохшая, \textit{извергается}\footnote{Іоан.~15,~6.}. Каковой души конецъ есть сожженіе, по Писанію: \textit{уже и сѣкира при корени древа лежитъ: всяко убо древо, еже не творитъ плода добра, посѣкаемо бываетъ, и во огнь вметаемо}\footnote{Матѳ.~3,~10.}. "--- Отсѣкаются злые сердца помыслы: 1)~Глаголомъ Божіимъ, иже, по ученію премудраго Павла, есть \textit{мечь духовный}\footnote{Еф.~6,~17.}. Аще убо востаетъ помыслъ похоти нечистыя, или помыслъ татьбы, хищенія, лихоиманія и всякія неправды, или помыслъ ненависти, зависти, гнѣва, злобы, мщенія, или помыслъ гордости, киченія, возношенія, сомнѣнія, презорства, уничиженія ближняго, осужденія и оклеветанія, или помыслъ роптанія и хулы и прочее зло восходитъ отъ сердца твоего, хрістіанине: тотчасъ, вземши духовный глагола Божія мечь, которымъ низлагаются помышленія и всяко \textit{возношеніе, взимающееся на разумъ Божій}, отсѣкай, да не, возрастши, душу твою повредитъ и погубитъ. "--- 2)~\textit{Молитвою}, которою Божія помощь на тое трудное дѣло испрашивается. Узнавши убо отъ Божія слова, что востающій помыслъ есть духа злобнаго плодъ, тотчасъ обращай сердце твое къ помощнику Богу и проси отъ Него помощи въ брани, да Самъ Онъ тобою, немощнымъ, побѣдитъ врага. Изрядно Андрей святый Критскій въ семъ дѣлѣ наставляетъ насъ, глаголя: «Востани и побори, Господи, яко Іисусъ Амалика, плотскія страсти, и Гаваониты "--- лестные помыслы, присно побѣждающи»\footnote{\textit{Пѣснь 6"~я великаго канона Тріод. постн. въ четвергъ 5"~я седмицы}.}. Тако научаетъ насъ Давидъ святый своимъ примѣромъ побѣждать, который вездѣ въ псалмахъ своихъ къ Богу прибѣгалъ и просилъ помощи. Тако вси святіи отсѣкали помыслы отъ сердецъ своихъ, и тако плоды приносили Хрісту, яко древа плодовитая. Тако подобаетъ и намъ, хрістіанине, поступать, когда хощемъ Хрістовыми быть. И сіе"=то есть \textit{плоть распинати со страстьми и похотьми}\footnote{Гал.~5,~24.}, "--- не попущать грѣху \textit{царствовать въ мертвеннѣмъ тѣлѣ, во еже послушати его въ похотехъ его}\footnote{Римл.~6,~12.} "--- \textit{не по плоти жити, но духомъ дѣянія плотская умерщвляти}\footnote{8,~12 и 13.}, "--- \textit{плоти угодія не творити въ похоти}\footnote{13,~14.}, "--- \textit{духомъ ходити, и похоти плотскія не совершати}\footnote{Гал.~5,~16.}, "--- \textit{огребатися отъ плотскихъ похотей, яже воюютъ на душу}\footnote{1~Петр.~2,~11.}. Сей трудъ, брань и подвигъ всѣмъ хрістіанамъ, небрачнымъ и брачнымъ, женамъ и мужамъ, младымъ и старымъ, юношамъ и дѣвамъ, господамъ и рабамъ настоитъ, и настоитъ непрестанно и до кончины жизни сея. Сколько бо на каждый день и часъ таковыхъ діавольскихъ непотребныхъ и пагубныхъ плодовъ возникаетъ, и хотятъ душу умертвить, "--- исчислить невозможно! Хрістіанская душа долгъ имѣетъ всѣхъ ихъ отсѣкать, и не ослабѣвать въ подвигѣ семъ.

\paragraph*{СХХІІ.} Видишь, что къ богатому и щедрому человѣку прибѣгаютъ бѣдные и убогіе, и получаютъ вси по своему требованію потребное. Отъ сего случая возведи умъ твой къ Богу, богатому въ милости и щедротахъ. Аще человѣкъ отъ природы золъ, такъ милуетъ и щедритъ просящихъ, и помогаетъ имъ въ нуждахъ ихъ: кольми паче Богъ естествомъ благъ тако, яко \textit{никтоже благъ, токмо единъ Богъ}, по словеси Господню\footnote{Матѳ.~19,~17.}, подастъ просящимъ съ вѣрою, что просятъ по волѣ Его, и поможетъ въ нуждѣ всякому, съ вѣрою къ Нему приходящему. Аще въ человѣкахъ, созданіи Своемъ милосердіе насадилъ, кольми паче Самъ имѣетъ милосердіе къ бѣдствующимъ и воздыхающимъ къ Нему. Аще заблудшихъ и незнавшихъ Его насъ позналъ и помиловалъ, кольми паче познанныхъ отъ Него, и знающихъ и исповѣдующихъ имя Его, и призывающихъ Его, познаетъ, услышитъ и помилуетъ. Аще неищущихъ Его насъ взыскалъ, кольми паче взысканныхъ отъ Него и ищущихъ Его пріиметъ, и явитъ милость Свою. Едино отъ насъ требуется, чтобы мы познавали и признавали предъ Нимъ нищету свою, бѣдность и окаянство; а богатство благости Его и щедротъ всѣмъ готово и отверсто. Симъ случаемъ и разсужденіемъ научаемся мы, любѣзный хрістіанине, не отлагать надежды нашея, но къ престолу благодати съ вѣрою приступать и толкать въ двери милосердія Божія и, хотя медлитъ, неотмѣнно помилованія отъ Него ожидать. Аще бо человѣкъ щедритъ и милуетъ, кольми паче Богъ, всякаго человѣка милостиваго несравненно милостивѣйшій; аще человѣкъ на гласъ бѣдныхъ утробою мятется и милосердствуетъ, кольми паче Богъ, Отецъ щедротъ. Читай святое Писаніе, и увидишь сію истину.

\paragraph*{СХХІІІ.} Аще бы ты увидѣлъ, хрістіанине, что благородный, высокій и богатый какій князь хощетъ и ищетъ дружество имѣть съ какимъ подлымъ, убогимъ и нищимъ человѣкомъ, непремѣнно бы такому неслыханному дѣлу дивился; а паче, что тотъ простецъ бѣдный не хощетъ того дружества, неотмѣнно бы ты его назвалъ несмысленнымъ. Тако имѣется несмысленный, безумный и слѣпый всякъ грѣшникъ нераскаянный, превосходнѣйшимъ образомъ. Хрістосъ, Сынъ Божій, Царь небесный, хощетъ съ нимъ имѣти дружество, любитъ его, и желаетъ его привлещи къ Себѣ, и отъ него любимъ быти; хощетъ пріити къ нему и вечеряти съ нимъ, и тому съ Собою, якоже глаголетъ: \textit{се стою при дверехъ и толку; аще кто услышитъ гласъ Мой, и отверзетъ двери, вниду къ нему, и вечеряю съ нимъ, и той со Мною}\footnote{Апок.~3,~20.}. Дружество бо не иное что, какъ усердная и взаимная любовь "--- любити и любиму быти. Сего святаго и высокаго дружества грѣшникъ слѣпый и несмысленный безумно отвращается, и неистовно обращается къ нечистой и мерзкой міра сего любви. О колико удивленія, страха, плача и слезъ достойное безуміе и неистовство сіе! Не видитъ сего бѣдный грѣшникъ, но въ самой вещи тако есть. Такъ"=то прелесть міра суетнаго и хитрость врага діавола ослѣпляетъ бѣднымъ намъ глаза! О, когда бы грѣшнику открылися очи сердечныя, и увидѣлъ, чего лишается, и гдѣ находится: непрестанно бы воздыхалъ и неутѣшно плакалъ; непрестанно бы гласъ отъ сердца возносилъ къ Богу и глаголалъ: \textit{заблудихъ яко овча погибшее: взыщи раба Твоего}\footnote{Пс.~118,~176.}.

\paragraph*{СХХІV.} Аще бы паки ты увидѣлъ, что сынъ царскій, отвергши дорогую багряницу, которою одѣянъ былъ, одѣялся въ худое и смрадное рубище: непремѣнно бы тому непристойному дѣлу удивился. Тако имѣется хрістіанинъ, который, оставивши Бога и Хріста Сына Божія, грѣху себе отдаетъ. Всякъ бо крестящійся сыномъ Божіимъ вѣрою о Хрістѣ Іисусѣ сотворяется по писанному: \textit{вси вы сынове Божіи есте вѣрою о Хрістѣ Іисусѣ}\footnote{Гал.~3,~26.}, "--- и одѣвается прекрасною оправданія Хрістова ризою внутрь, яко царскою багряницею, якоже глаголется хрістіанамъ: \textit{омыстеся, освятистеся, оправдистеся именемъ Господа нашего Іисуса Хріста, и Духомъ Бога нашего}\footnote{1~Кор.~6,~11.}; и паки: \textit{елицы во Хріста крестистеся, во Хріста облекостеся}\footnote{Гал.~3,~27.}. Все мы отъ грѣшнаго Адама грѣшными и скверными раждаемся въ плотскомъ рожденіи, но въ крещеніи святомъ, духовномъ рожденіи, вѣрою во Хріста Сына Божія отъ сквернъ грѣховныхъ омываемся, раждаемся духовно отъ Бога, и сотворяемся чадами Божіими, по Писанію: \textit{елицы пріяша Его} (Іисуса Хріста), \textit{даде имъ область чадомъ Божіимъ быти, вѣрующимъ во имя Его}\footnote{Іоан.~1,~12.}; и паки: \textit{видите}, Іоаннъ, святый апостолъ, удивляяся глаголетъ, \textit{какову любовь далъ есть Отецъ намъ, да чада Божія наречемся, и есмы}\footnote{1~Іоан.~3,~1.}. Сію великую благодать заслужилъ намъ Іисусъ Хрістосъ Сынъ Божій вольнымъ Своимъ послушаніемъ, страданіемъ и смертію, \textit{заслужилъ} вѣрующимъ во имя Его; \textit{подаетъ} небесный Отецъ Своею благостію и человѣколюбіемъ; \textit{совершаетъ} Духъ Святый. Но хрістіанинъ, когда уклоняется ко грѣху и дѣлаетъ неправду, "--- великую сію благодать, такъ дорого, то"=есть, страданіемъ и смертію Сына Божія и Господа славы снисканную, и \textit{туне} ему отъ единыя милости отъ Бога данную, погубляетъ, совлекается прекрасныя царскія одежды, въ которую въ крещеніи благодатію Хрістовою облеклся"=было, и одѣвается паки въ смрадное и скверное грѣховное рубище, котораго въ томжде крещеніи совлеклся"=было, и тако дѣлается позоръ плачевный ангеламъ и всѣмъ святымъ: якоже \textit{бо радость бываетъ на небеси о единомъ грѣшницѣ кающемся}, глаголетъ Господь\footnote{Лук.~15,~10.}, тако печаль бываетъ о праведникѣ, который заблуждаетъ и отпадаетъ вѣчнаго живота. Не видитъ сего смрада и студа бѣдный грѣшникъ, но увидитъ его тогда, когда открыются сынове Божіи и подобни будутъ единородному Сыну Божію, явившемуся во славѣ Своей божественной, якоже глаголетъ апостолъ: \textit{возлюбленіи! нынѣ чада Божія есмы, и не у явися, что будемъ: вѣмы же, яко егда явится, подобни Ему будемъ}\footnote{1~Іоан.~3,~2.}. А отъ числа ихъ отлучатся грѣшники, и явится ихъ внутренняя скаредность въ позоръ всему міру. Сей случай и разсужденіе учитъ тя, хрістіанине, испытать себе и разсмотрѣть совѣсть свою, не находишься ли и ты въ такъ бѣдномъ и плачевномъ состояніи. И когда примѣтишь, не медли возвратитися къ небесному Отцу, по подобію блуднаго сына, и молися къ Нему со смиреніемъ: \textit{Отче, согрѣшихъ на небо и предъ Тобою; и уже нѣсмь достоинъ нарещися сынъ Твой: сотвори мя, яко единаго отъ наемникъ Твоихъ}\footnote{Лук.~15,~18 и 19.}. Онъ, яко милосердъ, готовъ есть пріяти тебе и объяти святѣйшими любве Своея объятіями: и тако радость будетъ на небеси и о тебѣ кающемся.

\paragraph*{СХХV.} Видишь, что отецъ, хотя обратити сына своего непостояннаго къ честному и постоянному житію, увѣщаваетъ его, и глаголетъ ему: тебѣ ли имѣніе мое, потомъ и трудами моими собранное, расточать съ піяницами и блудницами? Якоже таковыхъ случаевъ много бываетъ на свѣтѣ. Отъ сего случая возми разсужденіе о душевномъ имѣніи, данномъ всякому хрістіанину въ крещеніи, то"=есть, о благодати, правдѣ, освященіи, усыновленіи, наслѣдіи вѣчнаго живота; и отъ плотскаго отца обрати умъ твой ко Хрісту, Иже есть \textit{Отецъ вѣчный} вѣрнымъ Своимъ\footnote{Ис.~9,~6.}, и помяни, что Онъ глаголетъ во Евангеліи Своемъ: \textit{иже не собираетъ со Мною, расточаетъ}\footnote{Матѳ.~12,~30.}. Хрістосъ вольнымъ Своимъ на земли пожитіемъ, терпѣніемъ, страданіемъ и смертію заслужилъ намъ, грѣшнымъ и отверженнымъ, у небеснаго Своего Отца милость, отпущеніе грѣховъ, оправданіе, освященіе и избавленіе, якоже апостолъ учитъ: \textit{Иже бысть намъ премудрость отъ Бога, правда же и освященіе и избавленіе}\footnote{1~Кор.~1,~30.}. И сіе неоцѣненное сокровище подается вѣрою всякому крещающемуся; тогда человѣкъ сподобляется \textit{туне} сихъ небесныхъ дарованій. Якоже бо честный и богатый человѣкъ, хотячи наслѣдника имѣній своихъ имѣть, пріемлетъ со стороны бѣднаго въ сына себѣ, и дѣлаетъ его по своей милости наслѣдникомъ своимъ: тако Богъ, \textit{богатъ сый въ милости и щедротахъ}, человѣка, который хитростію діавольскою и своею волею отторгнулся отъ Него, и бѣднымъ, нищимъ, убогимъ и окаяннымъ учинился, "--- благодатію Единороднаго Своего Сына Іисуса Хріста пріемлетъ къ Себѣ, и Духомъ Своимъ Святымъ \textit{омываетъ} отъ сквернъ грѣховныхъ, \textit{освящаетъ и оправдаетъ его}\footnote{6,~11.}, дѣлаетъ \textit{сыномъ Своимъ и наслѣдникомъ Своимъ, снаслѣдникомъ же Хрісту}\footnote{Римл.~8,~16 и 17.}. Но когда человѣкъ, забывъ великую сію благодать, обращается на грѣхъ, яко \textit{песъ на своя блевотины, и свинія омывшися въ калъ тинный}\footnote{2~Петр.~2,~22.}, и расточаетъ неоцѣненное оное небесное сокровище, якоже блудный сынъ сотворилъ\footnote{Лук.~15,~13 и 14.}, и тако, Хріста опечаляя безмѣрно, и себе погубляетъ; слышитъ отъ Него въ совѣсти своей и святомъ Писаніи подобный сему гласъ: о человѣче! тебѣ ли расточать богатство Мое, данное тебѣ, которое Я толикимъ трудомъ, потомъ, болѣзнію, терпѣніемъ, страданіемъ и смертію собралъ? Помяни, кто Я, и коликъ, Который ради тебе такъ трудился, и какимъ образомъ сыскалъ тебѣ сокровище сіе! Богъ и Создатель всѣхъ, Которому \textit{покланяется всякое колѣно, небесныхъ и земныхъ и преисподнихъ}! Но тебе ради подобнымъ тебѣ человѣкомъ сотворился, \textit{Себе умалилъ, зракъ раба пріимъ, въ подобіи человѣчестѣмъ бывъ, и образомъ обрѣтохся яко человѣкъ, смирилъ Себе, послушливъ бывъ даже до смерти, смерти же крестныя}\footnote{Филип.~2,~10,~7 и 8.}, и тако тебѣ отпущеніе грѣховъ, благодать, благословеніе, правду, усыновленіе и наслѣдіе вѣчнаго живота пріобрѣлъ. Помяни, кто ты былъ? Беззаконникъ, отступникъ, отверженный отъ лица Божія, клятвы и гнѣва Божія сынъ, плѣнникъ и невольникъ діаволовъ, огня негасимаго и вѣчнаго, погибели чадо. Но отъ всего сего дѣйствія Я избавилъ тебе, не сребромъ или златомъ, но \textit{честною Моею кровію}\footnote{1~Петр.~1,~19.}; омылъ тебе отъ сквернъ грѣховныхъ, и одѣялъ тебе чистою правды Моея одеждою, и наслѣдіе вѣчнаго живота обѣщалъ тебѣ. \textit{И внидохъ въ завѣтъ съ тобою, глаголетъ Адонаи Господь: и была еси Мнѣ, и омыхъ тя водою, и ополоскахъ кровь твою отъ тебе, и помазахъ тя елеемъ. И облекохъ тя въ пестроты, и обухъ тя въ червлены, и препоясахъ тя вѵссономъ}\footnote{Іез.~16,~8--10, и проч.}. Ты же толикую Мою любовь и милость къ тебѣ явленную пренебреглъ; что Я трудами, болѣзньми, терпѣніемъ, страданіемъ, послушаніемъ до смерти, смерти же крестныя, собралъ, все тое расточилъ и расточаешь, послѣдуя волѣ твоей и страстнымъ вожделѣніямъ, и не хочешь оставить сего злаго и страстнаго нрава, очувствоватися и покаятися, да паки получиши у Мене милость. \textit{О человѣче! или о богатствѣ благости, и кротости и долготерпѣніи Моемъ нерадиши, невѣдый, яко благость Моя на покаяніе тя ведетъ}\footnote{Римл.~2,~4.}. "--- \textit{Помяни, откуду спалъ ты}\footnote{Апок.~2,~5.}, съ какого достоинства и высоты въ какую подлость и окаянство? Гдѣ нынѣ находишься, и къ чему воля твоя и страстолюбивое твое житіе ведетъ тебе? \textit{Помяни и покайся}! Аще ли ни, то имѣеши во вѣки о семъ воздыхати, тужити, скорбѣти и плакати, но безполезно, \textit{егда увидиши Авраама и Исаака и Іакова и прочіихъ праведниковъ во царствіи Моемъ, тебе же изгонима вонъ во тму кромѣшнюю, ту будетъ плачь и скрежетъ зубомъ}\footnote{Лук.~13,~28.}. Тако благость Хріста Бога нашего обратившагося хрістіанина на грѣхъ и неисправнаго обличаетъ внутрь и внѣ, "--- \textit{внутрь} "--- въ совѣсти, и \textit{внѣ} "--- въ Писаніи святомъ. Хрістіанину же должно не нерадити о толикой благости Божіей, на покаяніе его зовущей, да не большій себѣ гнѣвъ Божій соберетъ, по писанному: \textit{по жестокости твоей и непокаянному сердцу, собираеши себѣ гнѣвъ въ день гнѣва и откровенія праведнаго суда Божія}\footnote{Римл.~2,~5.}.

\paragraph*{СХХVІ.} Видишь въ естествѣ, что равное равному и подобное подобному пристаетъ и къ тому стремится. Напр. огнь не помѣщается съ водою, но съ огнемъ; воздухъ не держится въ водѣ, но къ воздуху вверхъ восходитъ. Такожде въ животныхъ бываетъ: овца къ овцамъ пристаетъ и ходитъ съ ними, воробей съ воробьями имѣется и летаетъ. Въ человѣкахъ: подлый съ подлымъ, честный и постоянный съ постояннымъ, богатый съ богатымъ, благородный съ благороднымъ, и развращенный съ развращеннымъ водятся. Тако имѣется и въ духовномъ хрістіанскомъ дѣлѣ. Вѣра святая, понеже есть небесный даръ и отъ Бога происходитъ, никакого сообщенія съ тлѣнными и мірскими вещами не можетъ имѣти, отвращается ихъ; честь, богатство, славу и сладострастіе міра сего, какъ ничто вмѣняти вѣрное сердце научаетъ, отъ всѣхъ сихъ отвращатися увѣщаваетъ и восхищаетъ къ небеси, откуду произошла, и Бога, отъ Котораго начало имѣетъ, усердно желать и искать непрестанно побуждаетъ; вѣчныхъ и небесныхъ благъ богатство и славу представляетъ и къ нимъ восхищаетъ сердце человѣческое, въ которомъ мѣсто свое имѣетъ. Сродное бо къ сродному стремится и отъ несроднаго удаляется, и подобнаго себѣ ищетъ и отъ неподобнаго отвращается, якоже выше сказано. Тако вѣра, подобнаго себѣ духовнаго естества и начала, отъ котораго произошла, ищетъ и спѣшитъ. И какъ огонь, хотя и воспящается, однакожъ, къ верху восходитъ; свойство бо его сіе есть: тако и вѣра, хотя отъ соблазновъ суеты, прелести и отъ духовъ злыхъ препятіе терпѣти понуждается; однакожъ, яко духовное и небесное дарованіе, къ горнимъ стремится, \textit{горняя мудрствуетъ, а не земная}. И понеже вѣрное сердце примѣчаетъ свою немощь, и недостатокъ силъ къ сопротивленію врагамъ своимъ: прибѣгаетъ къ всесильной Божіей помощи и усердно молится Богу, дабы Самъ противу враговъ ея въ подвигѣ помоглъ и укрѣпилъ. Тако вѣра, яко небесный даръ, отъ земныхъ отвращается и небеснаго ищетъ. Отсюду послѣдуетъ, что тѣ хрістіане, которые любовію сердца своего приложилися къ міру, то"=есть, богатству, чести, славѣ и сладострастію, не имѣютъ въ сердцахъ своихъ вѣры, яко вѣра, какъ духъ, сроднаго себѣ добра ищетъ и любитъ; а они, оставивше духовное добро, ищутъ того, что плоть и кровь любитъ, ищутъ того, отъ чего вѣра отвращаетъ вѣрное сердце, и не ищутъ того, къ чему вѣра ведетъ и возбуждаетъ. Якоже бо духъ духовными и невидимыми благими услаждается и ищетъ ихъ, тако плоть, духа не имѣющая, плотскими и видимыми утѣшается, довольствуется и ищетъ ихъ. Сродное бо къ сродному стремится, какъ выше сказано, а отъ несроднаго уклоняется. Сей случай и разсужденіе учитъ тя, хрістіанине, испытовать, имѣеши ли вѣру, которая сердце человѣческое очищаетъ и ведетъ къ Богу. Не вси бо, которые нарицаются хрістіанами, вѣрные суть, но которые вѣру исповѣдуютъ, и на сердцѣ тую имѣютъ. Вѣра бо есть даръ Божій \textit{духовный}, который неотмѣнно человѣка отвращаетъ отъ суетныхъ и движетъ къ любленію и исканію небесныхъ; и есть какъ огнь, очищающій сердце отъ налоговъ и сквернъ грѣховныхъ; и въ вѣрѣ все существо хрістіанскаго духовнаго блаженства состоитъ.

\paragraph*{СХХVІІ.} Видишь, что отецъ сына неисправнаго и непокориваго, по многихъ увѣщаніяхъ и отеческихъ наказаніяхъ, отрицается и отвергаетъ отъ себе, и наслѣдія лишаетъ. Тако дѣлаетъ съ хрістіанами Богъ: неисправныхъ увѣщаваетъ и \textit{внутрь} въ совѣсти, и \textit{внѣ} чрезъ писанія пророческая и апостольская и проповѣди учителей и пастырей церковныхъ, и чрезъ обѣщаніе будущихъ благихъ и угроженія казней, временныхъ и вѣчныхъ, и чрезъ наказанія посылаемая, и тѣмъ всѣмъ убѣждаетъ къ истинному обращенію и покаянію. Но кто не престаетъ грѣшить, и сію благость Его, кротость и долготерпѣніе пренебрегаетъ: таковаго отрицается и отвергаетъ, и наслѣдія вѣчнаго живота лишаетъ. Тако отверглъ отъ Себе непокоривыхъ и жестоковыйныхъ Іудеовъ, которые ни писаніями пророческими, ни наказаніями Божіими, ни увѣщаніями, чудесами и милостію Сына Божія, во плоти явившагося, не исправилися, но паче злобою на Него, пришедшаго ихъ спасти, возстали, умучили и умертвили. Откуду и исполнилося на нихъ Хрістово слово: \textit{отъимется отъ васъ царствіе Божіе, и дастся языку, творящему плоды его}\footnote{Матѳ.~21,~43.}. Тако, которые ни милостію, ни наказаніемъ Божіимъ не исправилися, отвержены суть отъ Бога. Тако дѣлается съ хрістіаниномъ неисправнымъ, котораго ни совѣсть угрызающая, ни милость Божія и блаженство вѣчное обѣщанное, ни геенны страхъ и ничто иное къ покаянію подвигнути не можетъ. Къ таковому апостольское слово глаголется: \textit{о человѣче! или о богатствѣ благости Его, и кротости и долготерпѣніи нерадиши, не вѣдый, яко благость Божія на покаяніе тя ведетъ? По жестокости же твоей и непокаянному сердцу, собираеши себѣ гнѣвъ въ день гнѣва и откровенія праведнаго суда Божія}\footnote{Римл.~2,~4 и 5.}. Да убоимся убо, возлюбленный хрістіанине, сего прещенія Божія, которое чрезъ апостола Своего возвѣщаетъ неисправнымъ, да не и мы себѣ соберемъ гнѣвъ Божій, и на насъ исполнится, что на Іудеяхъ непокоривыхъ, и отвержени явимся отъ лица Божія на вѣки. Да послушаемъ Бога призывающаго, увѣщавающаго и ведущаго насъ Своимъ долготерпѣніемъ на покаяніе. Да не устрашаетъ насъ трудность и неудобство дѣла сего: начнемъ только, и Онъ съ нами начнетъ дѣло наше; подымемся только отъ паденія и лежанія, и Онъ намъ подсобитъ; станемъ добрѣ, и Онъ насъ укрѣпитъ, и будетъ помогать и дѣлать дѣло спасенія нашего. Зоветъ, возбуждаетъ, толкаетъ и движетъ лежащихъ; но начинающимъ возставать и возстающимъ и дѣлающимъ помогаетъ. Милостивъ бо весьма есть и человѣколюбивъ, и несказанно всѣмъ намъ хощетъ спастися. О семъ все святое Его Писаніе свидѣтельствуетъ. Помянемъ и разсудимъ, како Онъ, въ мірѣ живучи, никому въ милости Своей не отказалъ: слѣпымъ видѣніе, прокаженнымъ очищеніе, немощнымъ здравіе, глухимъ слышаніе, бѣсноватымъ свободу и мертвымъ воскресеніе подалъ. О тѣлесномъ здравіи приходящихъ къ Нему и вопіющихъ услышалъ, помиловалъ и исцѣлилъ: кольми паче о душевномъ исцѣленіи просящихъ услышитъ и исцѣлитъ. Ибо Онъ какъ тѣло, такъ и душу создалъ; и Тойжде есть и нынѣ на небеси Онъ, какъ и на земли былъ; таяжде Его благость, милость, человѣколюбіе, кротость и милосердіе къ страждущимъ намъ и воздыхающимъ къ Нему: слышитъ и нынѣ бѣдствующихъ и вопіющихъ къ Нему, и помогаетъ и исцѣляетъ, какъ и на земли живучи слышалъ, помогалъ и исцѣлялъ. Не далече отъ насъ есть, хотя и не видимъ Его ходящаго, и не слышимъ глаголющаго. Отъ видѣнія нашего и слуха только отстоитъ, но призывающимъ Его близъ есть: \textit{близъ бо Господь всѣмъ призывающимъ Его}\footnote{Ис.~144,~18.}. И аще къ тѣлесамъ страждущимъ, которыя умираютъ и растлѣваются, толикое милосердіе показалъ, и всякому прошеніе свое охотно исполнилъ: кольми паче душѣ безсмертной, далеко отъ тѣла лучшей, болѣзнующей, бѣснующейся, прокаженной, разслабленной, глухой и слѣпой поможетъ и исцѣлитъ, когда приступитъ къ Нему и помощи и милости отъ Него будетъ просить, по неложному обѣщанію Его: \textit{просите и дастся вамъ; ищите, и обрящете; толцыте, и отверзется вамъ}\footnote{Матѳ.~7,~7.}. На сіе бо и въ міръ сей къ намъ отлученнымъ и удаленнымъ отъ Него пришелъ, чтобы слѣпыя души наши просвѣтить, заблуждшія возвратить, прокаженныя грѣхами очистить и мертвыя оживить, и на путь покаянія и истины наставить, и тако подать истинное и вѣчное блаженство, которое мы въ прародителяхъ нашихъ потеряли. И аще толикую милость показалъ намъ, что ради насъ бѣдствующихъ и погибающихъ въ міръ пришелъ, и въ плоть облеклся подобострастную намъ, и тою ради насъ пострадалъ и умеръ Царь славы и Господь нашъ и Создатель: кольми паче, пострадавше по насъ, помилуетъ насъ обращающихся, и услышитъ и поможетъ. И аще помянулъ насъ забывшихъ Его и незнающихъ, кольми паче помянетъ поминающихъ и призывающихъ. Отвратившихся отъ Него насъ помиловалъ, кольми паче обращающихся къ Нему помилуетъ. Нужъ, возстанемъ, и станемъ на пути Его: и Онъ, вземше насъ, поведетъ въ слѣдъ Себе, и приведетъ къ Отцу Своему; и тако Богъ будетъ Богъ нашъ и Отецъ, и мы будемъ людіе Его и сыны Его, еже есть несказанная слава и похвала наша. Что бо не токмо быть, но и помыслитися славнѣе можетъ, какъ Бога живаго, всемогущаго, великаго, вѣчнаго, и преблагаго за Отца имѣть, и сынами Его нарицатися и быть? Сіе всю славу, честь, достоинство, утѣху и радость міра сего несравненно превосходитъ. Аще же нынѣ Богъ будетъ намъ Богъ и Отецъ, то неотмѣнно наслѣдіе вѣчнаго живота \textit{наше}, блаженство и слава вѣчная \textit{наша} будетъ.

\paragraph*{СХХVІІІ.} Видишь, что находящійся въ горячкѣ всякую пищу отвергаетъ, хотя бы сладкая и пріятная была, и гнушается тою, яко внутрь огнемъ палится, и снѣдается. Тако имѣется боголюбивая душа: желаніемъ святымъ небесныхъ благъ объята, и любовію Божіею уязвленна, и Божественнымъ и небеснымъ \textit{огнемъ}, котораго Хрістосъ \textit{пришелъ воврещи на землю}\footnote{Лук.~12,~49.}, распаляема, все, что въ мірѣ семъ честное, дорогое, славное, пріятное и веселое, отвергаетъ, гнушается и чего прочіи съ трудомъ и усердіемъ ищутъ, отъ того она отвращается и убѣгаетъ; честь, славу, богатство и роскошь, какъ мертвечину гнилую вмѣняетъ, и отвращается ихъ. Едино внутрь желаніе имѣетъ сіе, дабы вѣчное благо, еже есть Богъ Самъ со всѣми Своими вѣчными благими, сыскать и не потерять. О семъ у ней попеченіе, тщаніе, мысль, труды, воздыханіе, молитва и бесѣда. Ибо кто чего желаетъ и ищетъ прилѣжно, о томъ всегда думаетъ и тщится\footnote{Макар. Егип. бес.~9"~й гл.~9"~я.}. Вѣчно слово Хрістово есть: \textit{идѣже есть сокровище ваше, ту будетъ и сердце ваше}\footnote{Матѳ.~6,~21.}. Сей случай и разсужденіе, которое взято изъ книги святаго Макарія Египетскаго, мудростію духовною исполненныя, учитъ тя испытовать сердце твое, чѣмъ оно объято, какимъ желаніемъ, и какою любовію "--- Божіею ли, или міра сего? Многіи мнятъ о себѣ, что любовь Божію имѣютъ. Спроси всякаго: любиши ли Бога? Непремѣнно отвѣщаетъ: какъ Бога не любить? и совѣсть бо къ тому убѣждаетъ. Но, вмѣсто Бога, себе и міръ любятъ. Любовь истинная познается отъ дѣлъ, а не отъ словъ, какъ и всякая добродѣтель. Сердце у человѣка одно есть, и раздвоено быть не можетъ; и такъ непремѣнно или къ Богу, или къ міру преклоняется, желаетъ, ищетъ и прилѣпляется; и когда къ одному пристаетъ, отъ другаго отстаетъ. Откуду глаголетъ Господь: \textit{не можете Богу работати и мамонѣ}\footnote{Матѳ.~6,~24.}. Сего ради и апостолъ увѣщаваетъ: \textit{не любите міра, ни яже въ мірѣ: аще кто любитъ міръ, нѣсть любве Отчи въ немъ}\footnote{1~Іоан.~2,~15.}. Аще убо хощемъ Бога любить, то изженемъ изъ сердца нашего любовь къ міру и самолюбіе, да не вмѣсто любителей врагами своими будемъ. Ибо \textit{любы міра сего вражда Богу есть: иже бо восхощетъ другъ быти міру, врагъ Божій бываетъ}\footnote{Іоан.~4,~4.}.

\paragraph*{СХХІХ.} Видишь, что когда отецъ нищь есть и убогъ: и дѣти его убогіи суть. И когда хозяинъ или дому владыка въ бѣдствіи и озлобленіи находится: и домашніи его съ нимъ бѣдствуютъ, и озлобленіе терпятъ. Тако имѣется въ церкви святой, яже есть домъ Бога живаго. Дому сего святаго Отецъ и владыка есть Хрістосъ. Якоже убо Онъ, на земли живучи, терпѣлъ всякое бѣдствіе и озлобленіе и страданіе, и тако вошелъ въ славу Свою, якоже глаголетъ Самъ: \textit{не сія ли подобаше пострадати Хрісту, и внити въ славу Свою}\footnote{Лук.~24,~25.}? тако и домашнимъ Его, истиннымъ хрістіанамъ, въ томъ Ему послѣдовать должно, и всякое приключающееся бѣдствіе и озлобленіе терпѣть великодушно, взирая на Начальника и Владыку своего. Господь бо, на земли живучи, образъ намъ показалъ, какъ намъ жить и Богу угождать, и къ вѣчному животу доходить. Сему образу послѣдовали вси святіи, присніи Богу и домашніи Его, и вошли въ славу, Его страданіемъ и болѣзньми обрѣтенную. Убо и намъ должно тѣмъ путемъ идти, которымъ Самъ Господь нашъ и святіи Его шли, когда хощемъ съ Нимъ и святыми Его прославитися. Надобно бо невѣстѣ съ женихомъ, и удамъ, членамъ съ главою страдати, да и прославятся купно съ главою. Нѣтъ бо инаго пути къ вѣчному животу и граду оному святыхъ, кромѣ пути тѣснаго, прискорбнаго и крестнаго\footnote{Матѳ.~7,~14.}. Надобно неотмѣнно терпѣть со Хрістомъ терпѣвшимъ, смириться съ смирившимся, и крестъ нести съ понесшимъ крестъ, да купно и прославишися съ прославленнымъ. «Аще убо славы отъ человѣкъ ищешь, глаголетъ великій Макарій вышепомянутый, и желаешь поклоненія и почитанія, и ищеши сластолюбиваго житія: сошелъ еси съ пути того»\footnote{Бес.~12"~я гл.~5"~я.}. Сей случай и разсужденіе научаетъ тебе испытовать себе, имѣешься ли ты отъ домашнихъ Хрістовыхъ и присныхъ Богу, то"=есть, сынъ ли еси церкве, и имѣешь ли съ Нимъ участіе здѣ, когда хочешь въ будущемъ вѣкѣ имѣти участіе съ Нимъ. А хотящему имѣть съ Нимъ часть въ будущемъ вѣкѣ, надобно нынѣ держаться Его вѣрою и любовію, и послѣдовать Ему терпѣніемъ, кротостію и смиреніемъ нынѣ, да и въ ономъ вѣкѣ участіе въ царствіи Его возъимѣетъ.

\paragraph*{СХХХ.} Видишь, что нагій человѣкъ самъ себе стыдится и срамляется, и другіе его срамляются и отвращаются. Тако имѣется душа, которая не имѣетъ благодати Божія, покрывающія ее. Сатана врагъ нашъ, какъ разбойникъ, обнажилъ насъ, совлекши съ насъ одежду благодати Божія и спасенія. Откуду прародители наши, лукавствомъ его обнажившеся, познали наготу свою, которыя прежде паденія своего не видали, и начали сами себе стыдитися и срамлятися, что было доказательствомъ и свидѣтельствомъ внутренняго ихъ обнаженія. Хрістосъ, Сынъ Божій, видя нашу наготу, отъ которыя Богъ и ангели Его святіи отвращалися, умилосердился надъ нами, пришелъ на землю къ намъ нагимъ, бѣднымъ, убогимъ и нищимъ, облеклся въ плоть нашу и уподобился намъ во всемъ, кромѣ грѣха, дабы насъ нагихъ, замаранныхъ и оскверненныхъ омыть кровію Своею, и одѣть чистою правды Своея одеждою, и тако представить чистѣйшимъ Отца Своего небеснаго очесамъ. Сія святая одежда дается вѣрою всякому въ крещеніи. Аще убо кто хранитъ ее, подвизаяся противу враговъ своихъ, которые невидимо ополчаются противу насъ, и хотятъ насъ паки обнажить, срамъ и стыдъ намъ предъ Богомъ и ангелами Его учинить, "--- блаженъ есть, якоже глаголетъ Господь: \textit{блаженъ бдяй и блюдый ризы своя, да не нагъ ходитъ, и узрятъ срамоту его}\footnote{Апок.~16,~15.}. Ибо таковаго Отецъ небесный, видя его облечена одеждою Единороднаго Сына Своего, познаетъ и признаетъ за Своего, и подастъ ему наслѣдіе со святыми избранными Своими, и введетъ въ чертогъ славы Своея. Аще же кто нерадѣніемъ и лѣностію потерялъ тую священную багряницу, и тако явится нагъ предъ Богомъ и Его избранными, не иное что ему воспослѣдуетъ, какъ стыдъ и срамъ нестерпимый, самъ себе будетъ срамлятися, яко нагъ; и Богъ со избранными Своими постыдятся наготы его и отвратитъ святѣйшія очи Свои отъ него, и отречется его, яко не Своего. Не видитъ сея наготы грѣшникъ ослѣпленный, яко не имѣетъ душевныхъ очесъ отверстыхъ, и того ради не срамляется: якоже спящій или слѣпый не видитъ тѣлесныя своея наготы, почему и срама не чувствуетъ. Но Божіе всевидящее око видитъ; видитъ, и отвращается таковыя души. Увидитъ и грѣшникъ бѣдный тогда, когда открыются книги совѣстныя, и тайная сердецъ человѣческихъ помышленія въ явленіе всему міру приведутся, "--- увидитъ, но къ крайнему своему стыду, сраму, безчестію и бѣдствію. Аще бо предъ очесами немногихъ людей стоять нагому тѣлеснѣ стыдъ и срамъ есть, и не терпимъ того, но или убѣгаемъ, или прикрываемся: кольми паче предъ всѣмъ свѣтомъ, предъ ангелами и человѣками, избранными Божіими и самимъ Судіею праведнымъ, стоять нагой душѣ, ктомужъ замаранной и оскверненной, стыдъ и срамъ нестерпимый будетъ. Неотмѣнно пожелаетъ таковая душа укрытися въ пещерѣ и тьмѣ и самомъ адѣ, не терпя всемірнаго стыда и срама и праведнаго Судіи гнѣва, но не можетъ. О, коль блаженъ есть, кто нынѣ сію бѣдственнѣйшую свою познаетъ наготу! Тако бо будетъ себе стыдитися и срамлятися, и искать вѣрою одежды себѣ отъ Хріста, Который всякаго, вѣрующаго въ Него и просящаго у Него, одеждою спасенія одѣваетъ. Сей случай научаетъ насъ испытовать нашу душу, не нага ли она есть, и тако искать и просить вѣрою одѣянія ей у Хріста, Сына Божія, да не явимся въ день оный наги предъ Богомъ и святыми Его, и покрыемся студомъ.

\paragraph*{СХХХІ.} Видишь, что сребро, или злато, или мѣдь, когда царскаго на себѣ образа не имѣетъ, не годится къ употребленію общему и куплѣ, и въ казну Государеву не пріемлется, но отвергается; но только знаменанная монета ходитъ и отъ всѣхъ пріемлется. Тако имѣется душа: аще не будетъ имѣть на себѣ печати небеснаго Царя, Хріста, Сына Божія, которая вѣрою и Духомъ Святымъ изображается, непотребна бываетъ къ обществу святыхъ и къ небесному сокровищу не годится, но извергается вонъ. Сей случай учитъ тебе вѣрою и тщаніемъ искать небеснаго божественнаго образа, который есть прекрасное душъ нашихъ украшеніе и лѣпота, да не безъ него постраждеши тое, что пострадалъ оный званный на бракъ и пришедшій, безъ одѣянія брачнаго, которому сказано отъ Царя: \textit{друже, како вшелъ еси сѣмо, не имый одѣянія брачна? Онъ же умолча. Тогда рече Царь слугамъ: связавше ему руцѣ и нозѣ, возмите его и вверзите во тму кромѣшнюю}\footnote{Матѳ.~22,~12 и 13.}.

\paragraph*{СХХХII.} Видишь, что мертвый между живыми не имѣется, и непотребенъ быть между живыми, но износится изъ дома и града, и въ землю закапывается. Тако имѣется душа, умершая грѣхомъ: износится, или паче извергается отъ сообщенія святыхъ, изъ дому и града живаго Бога, то"=есть, церкви святой, \textit{яже есть домъ Бога живаго}\footnote{1~Тим.~3,~15.}. Ибо и сія непотребна быть вмѣняется между живыми, которые вѣрою и любовію живутъ и работаютъ Богу. Якоже бо тѣлу жизнь есть душа, тако душѣ жизнь есть благодать Божія. И какъ тѣла животъ познается отъ дѣйствій тѣлесныхъ, напримѣръ, движенія, глаголанія, и проч.: тако душевный животъ примѣчается отъ дѣйствій духовныхъ, то"=есть, истиннаго покаянія, молитвы, любве и страха Божія. Испытай убо, хрістіанине, отъ сего случая душу твою, не мертва ли она. \textit{Якоже бо тѣло безъ духа мертво есть, тако и вѣра безъ дѣлъ мертва есть}, глаголетъ апостолъ\footnote{Іак.~2,~20 и 26.}.

\paragraph*{СХХХІІІ.} Видишь, что когда живописецъ хочетъ написать образъ князя, единъ предъ другимъ прямо долженъ стоять, и лицемъ къ лицу другъ друга смотрѣть, и тогда живописецъ написуетъ образъ смотрящаго на него князя; а когда князь отвращается лицемъ отъ живописца, тогда живописецъ не можетъ добрѣ изобразить лица его и образъ ему подобный написать. Тако имѣется и внутренняго человѣка состояніе, и образъ Божій изображается въ немъ. Хрістосъ, Сынъ Божій, небесный и премудрый живописецъ есть. Онъ хощетъ изобразить въ душахъ нашихъ образъ Божій, который мы потеряли, и живность душамъ навести нашимъ, и напечатать прекрасный небеснаго Царя портретъ. Но кто отвращается отъ Него душею и сердцемъ, не можетъ дѣла Своего въ немъ совершить: аще же кто обратится къ Нему, и очесами вѣры на Него будетъ взирать и просить того у Него, на душѣ таковаго начертываетъ и изображаетъ божественное оное благолѣпіе живое, нетлѣнное и во вѣки сіяющее. Подобаетъ убо намъ, возлюбленный хрістіанине, вперить душевныя наши очи на Него, и вѣрою взирать, да и въ насъ божественная сія изобразится отъ Него доброта. А когда хощемъ тое учинить и желаемое получить, то образъ міра сего позади себе оставить, и тако къ Нему единому обратиться, смотрѣть на Него, любить и прилѣпляться Ему должно. Сего съ болѣзнію сердечною желалъ божественный апостолъ Галатамъ, когда они отпали отъ Хріста, и глаголалъ имъ: \textit{чадца моя, имиже паки болѣзную, дондеже вообразится Хрістосъ въ васъ}\footnote{Гал.~4,~19.}. Сего желаетъ и Хрістосъ, дабы образъ Его святый начертался и изобразился въ душахъ нашихъ, дабы, божественнымъ Его образомъ знаменанныхъ насъ видя, Отецъ небесный призналъ сообразныхъ Ему, Единородному Сыну Своему, и за сыновъ Своихъ имѣлъ, и тако вѣчнаго живота наслѣдіе подалъ намъ. Сего ради толь многоразлично призываетъ насъ Онъ, чтобы всѣмъ сердцемъ обратилися мы къ Нему, дабы возмоглъ Духомъ Своимъ Святымъ на душахъ нашихъ изображать образъ Свой святый, и такъ дѣло спасенія нашего совершать. Отъ сего случая учимся, хрістіанине, что непремѣнно должно намъ отъ грѣха и міра сего отвратить сердца и души наши, и обратить ко Хрісту, истинному, небесному и премудрому живописцу, Который не мертвый какій и земный, но живый, небесный и безсмертный, не на дскахъ каменныхъ, деревянныхъ и полотняныхъ, но на скрижалехъ душевныхъ, не перстомъ и кистію, но дѣйствіемъ Святаго и Животворящаго Своего Духа изображаетъ образъ, и на Него съ вѣрою, надеждою и усердіемъ взирать, \textit{дондеже вообразится въ насъ Хрістосъ}, насъ ради распятый. И хотя святаго сего образа доброта и благолѣпіе не видится нынѣ, яко духовное, но явится въ воскресеніи, по Писанію: \textit{Иже преобразитъ тѣло смиренія нашего, яко быти сему сообразну тѣлу славы Его}\footnote{Филип.~3,~21.}. Поищемъ убо вѣрою и усердною молитвою, хрістіанине, божественныя сея доброты, да, и нынѣ на душахъ нашихъ изобразившися, явится и на смиренныхъ нашихъ тѣлесахъ, егда воздвигнутся силою Божіею отъ гробовъ. Обратимся всѣмъ сердцемъ и душею ко Хрісту, и неотступно окомъ вѣры взираемъ, да напишетъ и на нашихъ душахъ образъ Свой Духомъ Святымъ, живымъ и животворящимъ. На сіе бо и въ міръ пришелъ, и въ нашъ облеклся образъ, да Свой образъ въ насъ обновитъ и оживитъ, который сатана лукавствомъ своимъ обезчестилъ, растлилъ и погубилъ.

\paragraph*{СХХХІV.} Видишь, что когда непріятель во градъ, или разбойникъ въ домъ входитъ, и хощетъ разграбить и опустошить градъ или домъ, всякъ противится ему, какъ и чимъ можетъ. Злые помыслы суть, какъ непріятели наши и разбойники, которые входятъ въ домы душъ нашихъ, и хотятъ разграбити сокровище, въ нихъ сокровенное, и насъ самихъ умертвить и погубить. И кто симъ врагамъ не противится и ихъ не умерщвляетъ, тотъ неотмѣнно самъ отъ нихъ умерщвленъ и погубленъ будетъ. Надобно бо непремѣнно единому изъ противныхъ и сражающихся побѣждену быть и пасти, другому быть побѣдителемъ, ибо брань безъ того не бываетъ. Должно убо и намъ, любезный хрістіанине, въ началѣ самомъ, какъ только почувствуемъ приходъ сихъ нашихъ враговъ, противитися имъ, затворять крѣпко клѣть сердца нашего и хранити, и на помощь себѣ все могущаго Іисуса Хріста, Царя нашего, силу призывати, и симъ нашимъ соперникамъ и обидчикамъ суда и отмщенія просити отъ Него. \textit{Суди, Господи, обидящія ми, побори борющія мя. Пріими оружіе и щитъ, и востани въ помощь мою}\footnote{Пс.~34,~1,~2 и проч.}. Аще бо въ началѣ не воспротивимся и не отразимъ ихъ, то вшедше въ домъ сердца нашего, разорятъ и опустошатъ его и насъ самихъ погубятъ. Непріятель, вшедши во градъ, пустымъ его дѣлаетъ, и гражданамъ неутѣшный плачъ, или, что горше того, смерть содѣловаетъ: тоежде дѣлаютъ намъ злые помыслы, когда имъ не противимся мужественно. Откуду глаголетъ премудрость Божія: \textit{всяцѣмъ храненіемъ блюди твое сердце}. А чтобы удобнѣе сохранить намъ сердце свое, то велитъ намъ хранить уста, очи и ноги, какъ далѣе глаголетъ: \textit{отъими отъ себе строптива уста, и обидливы устнѣ далече отъ тебе отрини. Очи твои право да узрятъ, и вѣжди твои да помаваютъ праведная. Права теченія твори твоими ногами, и пути твоя исправляй. Ни уклонися ни на десно, ни на шуе; отврати же ногу твою отъ пути зла}\footnote{Притч.~4,~23--27.}. А якоже вредно и пагубно есть отворять сердце злымъ помысламъ, яко врагамъ душевнымъ: тако душеспасительно есть отверзать его и давать входъ въ него слову Божію, паче же и желать того должно и тщаться. Ибо слово Божіе, вшедши въ сердце, содѣлаетъ тое сѣдалищемъ премудрости духовныя, которая есть \textit{чиста, мирна, кротка, благопокорлива, исполнь милости и плодовъ благихъ, несумнѣнна и нелицемѣрна}\footnote{Іак.~3,~17.}.

\paragraph*{СХХХV.} Видишь, что желѣзо огнемъ умягчается, и чимъ болѣе жжется, тѣмъ мягчайшее бываетъ. Тако имѣется человѣкъ: отъ природы человѣкъ всякъ, какъ желѣзо, жестокъ есть; злость грѣха сердце ожесточила и, какъ желѣзо, твердымъ содѣлала. Откуду сердце, благодатію Духа Святаго необновленное, въ Писаніи святомъ называется \textit{каменное. Исторгну каменное сердце отъ плоти ихъ, и дамъ имъ сердце плотяно}, глаголетъ Господь\footnote{Іез.~11,~19.}. Того ради отъ природы всякъ человѣкъ весьма непослушливымъ и непокоривымъ Богу бываетъ, ничего не могутъ словеса, угроженія и увѣщанія, когда Божія благодать не подвигнетъ: какъ желѣзо холодное, молотомъ ударяемое, непреклонно и непремѣняемо бываетъ. Но когда огнь благодати Божіей коснется сердца и согрѣетъ, начнетъ умягчатися; и чимъ болѣе симъ небеснымъ огнемъ жжется, тѣмъ болѣе умягчается, краснѣетъ и чистѣйшимъ дѣлается. Тогда таковый отъ міра и всего, что въ немъ красное, пріятное, веселое и сладкое есть, отвращается, и за ни что имѣетъ, что прежде за великое и дорогое почиталъ, и къ единымъ небеснымъ и вѣчнымъ всѣмъ сердцемъ стремится, на все, что Богъ благоволитъ, охотнѣйшимъ послушникомъ себе показуетъ. Къ сему блаженному состоянію сердца о коль нужно есть искушеніе! Ибо искушеніе 1)~показуетъ намъ сердца нашего бѣдное и плачевное состояніе; показуетъ, что въ немъ, коль великая злость и жестокость крыется, и къ чему оно клонится; и тако убѣждаетъ насъ познавать свою бѣдность, окаянство и ничтожество духовное. 2)~Убѣждаетъ искать способъ, которымъ бы отъ того избыть: и не иный какій представляется способъ, какъ Божія благодать, которая вѣрою и усердною молитвою получается. Чимъ бо отъ природы своей злой избавиться можемъ, какъ только Божіею силою, которая какъ вся можетъ, такъ можетъ и злое естество премѣнить и, какъ огнь желѣзо, умягчить? 3)~Убѣждаетъ міръ оставить, и Бога единаго искать, у Котораго единаго есть истинное и непоколебимое блаженство. И сіе"=то значитъ, что на хрістіанъ толико бѣдъ и напастей въ мірѣ семъ находитъ! Ибо сими они, какъ желѣзо во огни, умягчаются, и, какъ злато и сребро въ горнилѣ, искушаются и благопріятными Богу дѣлаются. Не убѣгай убо, возлюбленный хрістіанине, искушенія и креста, отъ Бога посылаемаго. Когда безропотно и великодушно потерпишь и понесешь его, принесетъ тебѣ великую Божію милость. Плоти нашей горестно терпѣть; но тако внутренній человѣкъ исцѣляется и здравіе получаетъ. Надобно бо неотмѣнно \textit{совлекаться ветхаго человѣка}, то есть, природнаго злонравія, когда хощемъ \textit{облещися въ новаго человѣка}, что необходимо къ спасенію нужно.

\paragraph*{СХХХVІ.} Видишь, что человѣкъ, когда съ женою бракомъ соединится, оставляетъ отца, матерь, братію, сестеръ и всѣхъ друговъ, и къ единой женѣ искреннею любовію прилѣпляется. \textit{Сего ради оставитъ человѣкъ отца своего и матерь: и прилѣпится къ женѣ своей, и будета оба въ плоть едину}, глаголетъ Божіе слово\footnote{Матѳ.~19,~5.}. Видиши ли, хрістіанине, что естественная и плотская любовь дѣйствуетъ? Все прочее оставить убѣждаетъ любителя, и прилѣпляется единому любимому. Отъ сего познавать учись, что есть любовь Божія. Такъ точно любитель Божій поступаетъ, все позади себе оставляетъ: честь, славу, богатство и утѣху міра сего, какъ сметіе и гной, ради любви Божіей вмѣняетъ, все ему мерзѣетъ, и какъ смердящая мертвечина кажется. Самаго живота своего, котораго нѣтъ человѣку дороже ничего, не щадитъ. Къ единому любимому своему Сокровищу стремится, какъ пламень огненный въ высоту; туды воздыханія, желанія, помышленія и сердце свое непрестанно возводитъ. Тамо сердцемъ и умомъ обращается, медлитъ и живетъ, гдѣ любимое и неоцѣненное Сокровище его: \textit{идѣже бо есть сокровище его, тамо есть и сердце его}\footnote{Матѳ.~6,~21.}. Сей примѣръ научаетъ насъ, хрістіанине, что есть Божія любовь. Многіи мнятся любити Бога, но вмѣсто того любовь міра сего имѣютъ, и тако обманываются. Божія бо любовь и любовь міра сего купно въ единомъ сердцѣ быть не могутъ, но единая другую изгоняетъ. Какъ спросить всякаго: любиши ли Бога, "--- никто не отречется, но скажетъ: какъ не любить Бога? кого же и любить, какъ не Бога? Но въ самой вещи себе и міръ любитъ, а не Бога. Вѣрно бо слово апостольское и истинно: \textit{аще кто любитъ міръ, нѣсть любве Отчи въ немъ}\footnote{1~Іоан.~2,~15.}. Неотмѣнно, кто что любитъ, тамо и сердце его есть. Должно убо намъ испытовать себе, имѣемъ ли любовь Божію, да не прельстимся суетнымъ и ложнымъ мнѣніемъ; и молить Бога усердно, дабы Самъ Духомъ Своимъ возжеглъ любовь сію въ сердцахъ нашихъ да не вмѣсто любителей Божіихъ врагами Его будемъ, якоже глаголетъ апостолъ: \textit{иже восхощетъ другъ быти міру, врагъ Божій бываетъ}\footnote{Іак.~4,~4.}.

\paragraph*{СХХХVІІ.} Видишь, что человѣкъ сѣетъ пшеницу или иное что рѣшетомъ, и то къ тому, то къ другому краю бросаетъ зерно. Разумѣй, что такъ точно сатана сердца людей, благодатію Божіею свыше неотрожденныхъ, возмущаетъ и помышленіями житейскихъ попеченій движетъ, и то къ той, то другой, то къ третіей суетѣ бросаетъ ихъ и привязываетъ, дабы, тѣми попеченіями опутавшеся, не могли истиннымъ сердцемъ Бога искать. Откуду бываетъ, что замышляютъ въ сердцахъ своихъ то о снисканіи богатства, то о полученіи чести и славы, то о сладострастіи и угожденіи плоти, то объ отмщеніи врагомъ, то о строеніи богатыхъ домовъ, приготовленіи слугъ, каретъ, коней, о украшеніи одѣяній и о прочемъ, что на землѣ и въ мірѣ семъ дорого, пріятно и угодно плоти страстной. Сія вся сатана влагаетъ имъ въ сердца, и движетъ ихъ и связуетъ таковыми замыслами, дабы не могли осмотрѣться и Бога съ вѣчными благими искать. Сіе извѣстнѣйшее есть знаменіе человѣка, вѣры и любве Божія неимѣющаго. И которыи хрістіане такъ дѣлаютъ, внѣ только видъ хрістіанства показуютъ, но внутрь хрістіанства не имѣютъ; и суть подобни гробамъ, внѣ украшеннымъ, но внутрь исполненнымъ мертвыхъ костей и смрада. Сіи ихъ замыслы, попеченія и тщанія суетныя явятся внѣ, которыя нынѣ въ сердцѣ имѣютъ; явятся въ общее воскресеніе, когда пріидетъ Господь, \textit{Иже во свѣтъ приведетъ тайная тьмы, и объявитъ совѣты сердечныя}\footnote{1~Кор.~4,~5.}. И сіи самыя тщанія ихъ будутъ во обличеніе имъ предъ всѣмъ свѣтомъ, что они въ мірѣ семъ міръ и плоть свою любили, а не Бога, временныхъ искали, а не вѣчныхъ: якоже святыхъ доброта душевная на тѣлесахъ ихъ явится и прославитъ ихъ. И хотя въ благодати Божіей находящихся и просвѣщенныхъ искушеніе, отъ врага бываемое, касается, и то извнутрь чрезъ злые помыслы, то извнѣ чрезъ злыхъ людей сердца ихъ трогаетъ, возмущаетъ, обуреваетъ и по подобію вѣтра, волнуетъ, и кораблецъ ихъ тщится погрузить: но помощію всесильнаго Духа Божія укрѣпляются и твердо стоятъ надеждою, какъ корабль котвою во глубинѣ земли, въ человѣколюбіи Божіи утверждени. Сей случай и разсужденіе показуетъ тебѣ, чимъ разнствуютъ сынове вѣка сего, то"=есть, неимущіи живыя вѣры, отъ сыновъ свѣта или сыновъ царствія Божія; и учитъ тебе испытовать себе, имѣеши ли ты живую вѣру.

\paragraph*{СХХХVІІІ.} Видишь, что господинъ, который у себе имѣетъ дѣтей и рабовъ, иную пищу и одѣяніе дѣтямъ, иную рабамъ своимъ подаетъ. Тако дѣлаетъ Господь всѣхъ Богъ: иную пищу и одѣяніе подаетъ чадамъ Своимъ, иную невѣрнымъ и непочитающимъ Его. Что надлежитъ до плотской немощи, до \textit{внѣшняго} состоянія и видимаго и чувствамъ подлежащаго, пища, одѣяніе и питіе тоежде вѣрнымъ и праведнымъ людямъ, какое невѣрнымъ и грѣшнымъ есть; таковоюжде пищею и питіемъ укрѣпляются, и таковоюжде одеждою покрываются и чада міра сего, какъ и чада Божія. Ибо Богъ, яко щедръ и милостивъ, всѣмъ праведнымъ и грѣшнымъ, благая міра сего подаетъ, якоже глаголетъ Господь, \textit{Иже солнце Свое сіяетъ на злыя и благія, и дождитъ на праведныя и на неправедныя}\footnote{Матѳ.~5,~45.}. Солнце, луна, звѣзды, огнь, воздухъ, вода съ рыбами, земля со скотами и зеліями и всѣмъ исполненіемъ ея, обща суть праведнымъ и грѣшнымъ, знающимъ Бога и незнающимъ, почитающимъ и непочитающимъ, съ тѣмъ только разнствіемъ, что праведные пріемлютъ благая отъ Бога дающаго, и знаютъ Дателя, и благодарятъ Благодѣтеля; грѣшные и беззаконные насыщаются благими Божіими, но Дателя не знаютъ, не почитаютъ и не благодарятъ Благодѣтеля. Что же касается до внутренняго состоянія, праведные люди иную пищу, иное питіе, иную одежду имѣютъ, каковыхъ грѣшніи, міру сему работающіи, не имѣютъ. Благодать Божія, которую внутрь себе имѣютъ, пища святымъ есть, которою укрѣпляются; есть питіе, которымъ утѣшаются и услаждаются; есть одежда, которою вѣрою во Хріста одѣяни ходятъ. Хрістосъ Сынъ Божій, Ѵпостасная Божія Премудрость, вѣрою въ нихъ живущій, пища, питіе и одежда имъ есть. Той ихъ внутрь укрѣпляетъ, утѣшаетъ, услаждаетъ, радостотворитъ и одѣваетъ одеждою правды Своея, якоже писано есть: \textit{ядущіи Мя еще взалчутъ, и піющіи Мя еще вжаждутъ}\footnote{Сир.~24,~23.}; и паки: \textit{вода, юже Азъ дамъ ему, будетъ въ немъ источникъ воды, текущія въ животъ вѣчный}\footnote{Іоан.~4,~14.} и паки: \textit{елицы во Хріста крестистеся, во Хріста облекостеся}\footnote{Гал.~3,~27.}. Помолимся убо, возлюбленный хрістіанине, Хрісту Сыну Божію, да подастъ и намъ грѣшнымъ пищу сію, питіе и одежду: да, здѣ вкусивше отъ части пищи сея, во вѣки ея преизобильно наслаждатися будемъ; и, здѣ одѣявше души тою одеждою, въ пришествіи Его не наги обрящемся, и предъ всѣмъ свѣтомъ не посрамимся.

\paragraph*{СХХХІХ.} Видишь, что въ домѣ земнаго царя слуги ему предстоятъ и служатъ, не рубищами и худыми одеждами, но дорогими и цвѣтными одѣяни. Тако имѣется и въ церкви, яже есть домъ небеснаго Царя. Тые только въ семъ небеснаго Царя домѣ имѣются, и Царю своему небесному служатъ и работаютъ, которые, вѣрою очистившеся и рубищъ грѣховныхъ совлекшеся, украшаютъ себе добрыми дѣлами, \textit{облекаются во утробы щедротъ, благость, смиренномудріе, кротость и долготерпѣніе}\footnote{Кол.~3,~12 и проч.}. Прочіи, хотя и называются хрістіанами, до сего дому святаго не надлежатъ и Богу не работаютъ, которые грѣшить не престаютъ, и плодовъ покаянія творить не хотятъ. Аще бо земному царю рубище гнусно и царской палатѣ неприлично безобразіе тѣлесное, кольми паче очесамъ небеснаго Царя грѣховная скверна мерзка есть и гнусна, и святому дому Его безобразіе душевное непристойно есть. И аще отъ лица царскаго и службы отлучаются и удаляются подлые и рубищами одѣянные, и въ домѣ его таковые мѣста не имѣютъ: кольми паче отъ лица небеснаго Царя и службы отлучаются грѣшники, скверною грѣховною замаранніи, и отъ святаго дому Его изгоняются и никакого мѣста и части въ немъ не имѣютъ. \textit{Не преселится къ Тебѣ лукавнуяй, ниже пребудутъ беззаконницы предъ очима Твоима: возненавидѣлъ еси вся дѣлающія беззаконія, Господи}, глаголетъ Пророкъ\footnote{Пс.~5,~5 и 6.}. Сіе учитъ насъ, хрістіанине, испытовать себе самихъ, имѣемся ли мы истинныя церкве Хрістовой удами, и работаемъ ли Ему вѣрно; не работаемъ ли міру и прихотямъ своимъ, и такъ отъ благословеннаго Его дому отлучились, хотя и именуемся хрістіанами.

\paragraph*{СХL.} Видишь, что конь свирѣпый и неученый, чтобы угодный былъ къ дѣлу и употребленію хозяину, обучается и различно укрощается и смиряется, и тако природную свою свирѣпость помалу отлагаетъ и кроткимъ дѣлается, и удобенъ бываетъ ко всякому дѣлу хозяйскому. Всякъ человѣкъ отъ природы своей, какъ конь свирѣпый есть, жестокій и неукротимый, и на службу Господу своему Богу весьма неугоденъ; себѣ, своей волѣ и міру хощетъ и ищетъ угождать, а не Богу, Господу своему: весьма бо растлился по паденіи Адамовомъ. О! коль убо многаго требуетъ обученія, укрощенія, усмиренія, чтобы природную отложилъ свирѣпость, жестокость и бѣшенство, и тако угоднымъ сотворился на службу Господу Богу своему! Къ сему спасительному обученію предложено намъ, хрістіанине: 1)~Святое Божіе слово, которымъ свирѣпость растлѣннаго нашего естества показуется, обуздывается и востягается. \textit{Всяко бо писаніе богодухновенно и полезно есть ко ученію, ко обличенію, ко исправленію, къ наказанію, еже въ правдѣ: да совершенъ будетъ Божій человѣкъ, на всякое дѣло благое уготованъ}, глаголетъ апостолъ\footnote{2~Тим.~3,~16 и 17.}. "--- 2)~Тщаніе, трудъ и нужное себе самого ко всякому благому дѣлу силованіе, хотя бы того и не хотѣло растлѣнное сердце. Лѣнивъ человѣкъ есть отъ природы на дѣло благое и злопохотливъ, и потому должно себе самого убѣждать, нудить и силовать на доброе дѣло, себе побѣждать и отвращать отъ всякаго злаго дѣла. Напр. не хощетъ сердце твое ближняго любить, миловать, простить, смириться, цѣломудрствовать, и проч.: убѣждай себе къ тому. Хощетъ сердце твое ближняго ненавидѣть, гнѣваться, злобиться, мстить, злословить, и проч.: силою отвращай себе отъ того, и побѣждай себе въ томъ. Сія есть славная побѣда, и далеко лучшая, нежели людей побѣждать! Себе бо самого побѣдить "--- нѣтъ славнѣйшей побѣды. "--- 3)~Нужна есть молитва. Сами бо себе безъ помощи Божіей побѣдить не можемъ, и безъ благодати Божіей богоугоднаго дѣла творити не можемъ. \textit{Безъ Мене не можете творити ничесоже}, глаголетъ Хрістосъ Богъ\footnote{Іоан.~15,~5.}. Откуду повелѣно намъ часто и усердно молитися и помощи и благодати просить. Якоже бо чрезъ злобу діавольскую человѣкъ испортился, тако подобаетъ ему благодатію и помощію Божіею исправитися. Якоже чрезъ дѣйствіе врага онаго учинился ко всякому злому дѣлу склоненъ и охотенъ, и отъ всякаго дѣла добраго отвращенъ, яко внутрь сердца его злое похотѣніе и отвращеніе отъ добра крыется: тако подобаетъ ему силою Святаго Духа, вѣрою во Хріста, охотнымъ ко всякому дѣлу благому и тщательнымъ и прилѣжнымъ и отъ всякаго злаго дѣла отвращающимся учиниться, дабы какъ злоба діавольская дѣйствовала въ сердцѣ его, тако бы въ сердцѣ его дѣйствовала благодать Святаго Духа. И тако, когда сердце изъ злаго въ доброе перемѣнится и исправится, исправиться можетъ весь человѣкъ; къ чему неотмѣнно нужна молитва и воздыханіе. Но и къ молитвѣ должно себе нудить и обучать, дабы не въ словахъ единыхъ была молитва; должно разсѣвающіяся помышленія собирать и привязывать къ молитвѣ, и что глаголется словами, тое бы разсуждалъ умъ, и тако сердце съ воздыханіемъ къ Богу возводить. "--- 4)~Изрядное училище хрістіанамъ есть крестъ или страданіе, терпѣніе бѣдствій и всякихъ искушеній. \textit{Скорбь бо терпѣніе содѣловаетъ, терпѣніе же искусство}, глаголетъ Павелъ святый\footnote{Римл.~5,~3 и 4.}. Бѣдствіемъ и искушеніемъ показуется человѣку, что у него въ сердцѣ крыется "--- любовь Божія, или любовь міра и самолюбіе. Многіи мнятся Бога любить; но нашедшая бѣда показуетъ имъ, что они себе и міръ любятъ, а не Бога. Многіи мнятся быти терпѣливы, кротки, смиренны; но приспѣвшая досада и обида показуетъ имъ, что въ сердцѣ у нихъ нѣтъ такихъ добродѣтелей, и тако познаютъ себе быти убогихъ и ничтоже, которые прежде думали о себѣ нѣчто. Всякое убо искушеніе бываетъ намъ ко искушенію сердца нашего и познанію, что въ немъ крыется "--- терпѣніе ли или гнѣвъ, смиреніе или гордость, послушаніе или непослушаніе, "--- и есть какъ зерцало, чрезъ которое смотримъ въ сердце наше и разсматриваемъ, что въ немъ имѣется; иначе познать его не можемъ, яко глубоко. \textit{Глубоко бо сердце человѣку паче всѣхъ}, глаголетъ пророкъ\footnote{Іер.~17,~9.}. И тако познавше сердце наше, смиряемся и падаемъ предъ Богомъ, признаемъ себе виновными, и милости отъ Него ищемъ и просимъ и молимся съ пророкомъ: \textit{сердце чисто созижди во мнѣ, Боже, и духъ правъ обнови во утробѣ моей}\footnote{Пс.~50,~12.}! И для сего"=то, кромѣ прочіихъ причинъ, напасти на насъ посылаются отъ Бога, то есть, да познаемъ самихъ себе, бѣдность, убожество и окаянство наше душевное, и тако смиримся. Нашедшее бѣдствіе подобно есть лѣкарству, называемому рвотному. Какъ бо, рвотное принявши человѣкъ, изрыгаетъ вонъ изъ себе соки вредные, и тако видитъ и познаетъ, чѣмъ онъ немоществуетъ: тако при нашедшемъ бѣдствіи исходятъ помыслы злые, какъ соки вредные, крыющіися въ сердцѣ и повреждающіе душу, и тако узнаетъ человѣкъ, чѣмъ онъ духовно немоществуетъ; а тако убѣждается прибѣгать со смиреніемъ и прошеніемъ ко Хрісту, душъ человѣческихъ Врачу. "--- Отъ всѣхъ сихъ вышереченныхъ пунктовъ видно: 1)~коль нужно есть поученіе въ святомъ Божіемъ словѣ; 2)~коль нужна есть молитва; 3)~коль полезенъ есть крестъ или терпѣніе бѣдствій, хотя плоти нашей то и непріятно; 4)~коль нужно есть дѣтей въ страхѣ Божіи и прочей хрістіанской должности обучать, дабы они помалу привыкали къ работѣ хрістіанской, и тако бы въ возрастъ совершенный пришедше, могли сами себе содержать въ хрістіанской должности.

\paragraph*{СХLІ.} Видишь, что домъ, который не имѣетъ господина, живущаго въ себѣ, нечистъ бываетъ, преисполненъ сора, сметія и паутины; земля безъ дѣлателя запустѣваетъ, безплодна бываетъ и раждаетъ непотребныя зелія; садъ и виноградъ безъ дѣлателя пропадаетъ; корабль безъ кормчія добраго отъ бури и волнъ морскихъ потопляется, и проч. Тако имѣется душа человѣческая безъ Хріста, и есть какъ домъ запустѣлый, преисполненъ смрада, нечистоты и паутинъ, отъ діавола и демоновъ распростертыхъ, то"=есть, злыхъ похотѣній, замысловъ и начинаній; есть какъ земля недѣланная, терніе только и волчецъ прорастающая и проклятію подлежащая, есть какъ садъ и виноградъ, безъ дѣлателя запустѣвшій, плода нетворящій, и не иному чему, какъ посѣченію и сожженію подлежащій; есть какъ корабль безъ кормчія плавающій, отъ бури и волнъ волнующійся и близъ потоплепія находящійся. Что бо есть дому хозяинъ, землѣ и винограду дѣлатель, кораблю кормчій, тое душѣ нашей есть Хрістосъ. По паденіи отступилъ Хрістосъ отъ души человѣческой, и сдѣлалась какъ домъ безъ хозяина, земля и виноградъ безъ дѣлателя, корабль безъ кормчія. Нѣтъ и въ той душѣ Хріста, которая, хотя водою и Духомъ въ крещеніи отрождена была, но, первому Адаму послѣдуя, заповѣдь Божію отвергла, и лукавому врагу діаволу, древнему оному змію, волею своею послѣдовала; и когда въ томъ бѣдственномъ состояніи до конца пребудетъ, во вѣки будетъ безъ Хріста; и понеже Хрістосъ есть животъ души, то безъ Хріста пребывая, яко живота, во вѣки будетъ въ смерти. Покаемся убо, хрістіанине, \textit{очистимъ себе отъ всякія скверны плоти и духа}, да удостоимся принять Хріста въ домы душъ нашихъ. На сіе бо пришелъ въ міръ сей, чтобы мы благодарно срѣтили Его и, очистивше домы душъ нашихъ, приняли Его. Стоитъ Онъ и ударяетъ въ двери сердецъ нашихъ, да отверземъ Ему, да почіетъ въ душахъ нашихъ; и мы да умыемъ святыя ноги Его\footnote{Лук.~7,~37.} и \textit{обитель} у насъ \textit{сотворитъ}\footnote{Іоан.~14,~23.}. \textit{Се стою при дверехъ}, глаголетъ Самъ, \textit{и толку: аще кто услышитъ гласъ Мой, и отверзетъ двери, вниду къ нему}\footnote{Апок.~3,~20.}. Горе намъ, когда нынѣ не послушаемъ Его, и насъ ради пришедшаго въ міръ, и странствовати въ немъ благоволившаго, не пріимемъ въ домы своя! Слыши, что глаголетъ во второмъ пришествіи Своемъ сущимъ ошуюю Его: \textit{страненъ бѣхъ, и не введосте Мене; взалкахся, и не дасте Ми ясти; возжадахся, и не напоисте Мя}\footnote{Матѳ.~25,~42 и 43.}. Стыдно и страшно будетъ выговоръ сей терпѣть отъ Него. Страшнѣе того на вѣки отъ Него отлучиться, и съ діаволомъ во огнь вѣчный отъитить. \textit{Идите отъ Мене проклятіи во огнь вѣчный, уготованный діаволу и аггеломъ его}\footnote{41.}.

\paragraph*{СХLІІ.} Видишь, что солнце, хотя всю поднебесную освѣщаетъ и всѣмъ лице свое являетъ равно, однакожъ всякому кажется, что какъ бы на него единаго смотрѣло; и что ни дѣлаемъ, на всякаго изъ насъ, какъ на единаго только, смотритъ; и куды ни идемъ, съ нами идетъ, хотя всегда равно теченіе свое отъ восхода къ западу совершаетъ. Тако имѣется вѣчное и духовное солнце "--- Богъ. Хотя и на всѣхъ равно смотритъ, и всякаго дѣло и помышленіе, намѣреніе, начинаніе, умышленіе видитъ и слово слышитъ, и со всякимъ на всякомъ мѣстѣ есть; и дѣлаетъ ли кто или почиваетъ, стоитъ или идетъ, добро или худо дѣлаетъ, съ нимъ есть, и на дѣло доброе и худое смотритъ, и въ книзѣ своей записуетъ, да воздастъ всякому по дѣломъ его: однакожъ тако всякаго отъ насъ назираетъ и примѣчаетъ, что какъ бы, всѣ созданныя вещи оставивши, единаго только мене или тебе смотрѣлъ; и что я или ты хощемъ дѣлать, говорить, мыслить, начинать и намѣревать, доброе или худое, совершенно знаетъ и видитъ такъ, какъ бы оно уже въ самой вещи было и дѣлалося. И далеко лучше и совершеннѣйше все знаетъ и видитъ, нежели мы сами знаемъ и видимъ; и все тое видитъ и смотритъ такъ, что какъ бы, все прочее забывше, насъ единыхъ дѣло, начинаніе и намѣреніе смотрѣлъ. Тако о Немъ свидѣтельствуетъ Писаніе святое: \textit{вси путіе мои предъ Тобою, Господи}, глаголетъ Давидъ святый\footnote{Пс.~118,~168.}. И паки: \textit{Господи, искусилъ мя еси, и позналъ мя еси: Ты позналъ еси сѣданіе мое и востаніе мое: Ты разумѣлъ еси помышленія моя издалеча: стезю мою и уже мое Ты еси изслѣдовалъ, и вся пути моя провидѣлъ еси}\footnote{138,~1,~3.}. "--- Отъ сего научаемся, хрістіанине: 1)~Боятися Бога и берещися всякаго грѣха въ словѣ, дѣлѣ, помышленіи, начинаніи и намѣреніи: яко Онъ, какъ слово наше слышитъ, такъ дѣло, помышленіе и намѣреніе явно видитъ, и всякому по дѣломъ воздаетъ, и согрѣшающаго въ самомъ дѣлѣ можетъ праведнымъ судомъ поразить; и нѣтъ такъ сокровеннаго мѣста, въ которомъ сокрытися могли бы мы отъ сего всевидящаго ока, яко оно не только внѣшнее дѣло видитъ, но и тайное помышленіе, и въ глубинѣ сердца сокровенное проницаетъ. Дѣлаешь ли убо что противу совѣсти въ сокровенномъ мѣстѣ, или помышляешь въ сердцѣ твоемъ, или намѣреваешь что дѣлать: уже тое Богу явно есть. Берегись же грѣха, какъ смертнаго яда, да не дознаешь на себѣ праведный судъ Божій. "--- 2)~Учимся уповать на благость Божію, яко Онъ вездѣ съ нами есть. И какъ солнце ясно сіяющее вездѣ съ человѣкомъ есть, освѣщаетъ и согрѣваетъ его, гдѣ бы онъ ни былъ, и въ благоденствіи или злоденствіи былъ: тако благость и человѣколюбіе Божіе вездѣ и никогда не отступаетъ отъ насъ, но всегда согрѣваетъ и сохраняетъ насъ, и въ домѣ ли нашемъ или на пути, въ своей сторонѣ или чужой, въ благополучіи ли или злополучіи, въ здравіи ли или болѣзни, въ печали или радости находимся, съ нами есть. "--- 3)~Когда насъ враги наши гонятъ, поносятъ и злословятъ, молчать и не отвѣтствовать, и предавать все Ему, яко Онъ все слышитъ и знаетъ и достойно ли, или недостойно отъ нихъ гоненія и поношенія страждемъ, Ему явно есть. И якоже дѣти предъ отцемъ не отвѣтствуютъ противу хулителей своихъ, но на отца своего взираютъ, и отъ него ожидаютъ защищенія: тако должно намъ предъ Богомъ, Отцемъ нашимъ небеснымъ, Который съ нами есть, и мы предъ Нимъ находимся, со врагами нашими поступать, не отвѣтствовать имъ, но съ молчаніемъ очи душевныя возводить къ Нему, и глаголати со Псаломникомъ: \textit{Ты услыши, Господи Боже мой}\footnote{Пс.~37,~16.}! "--- 4)~Отсюду учимся молитися Ему не токмо явно, словами, но и въ тайнѣ сердца нашего; яко Онъ не токмо слова наши слышитъ, но и воздыханія, желанія и хотѣнія сердца нашего разумѣетъ, якоже глаголетъ пророкъ: \textit{желаніе убогихъ услышалъ еси, Господи, уготованію сердца ихъ внятъ ухо Твое}\footnote{Пс.~9,~38.}. \textit{И будетъ прежде неже воззвати имъ, Азъ услышу ихъ}, глаголетъ Господь\footnote{Ис.~65,~24.}. "--- 5)~Учимся наконецъ, что когда согрѣшимъ предъ Богомъ, отъ гнѣва Его никуды убѣжати не можемъ, ибо куды ни обратимся и побѣжимъ, вездѣ Онъ насъ предваряетъ и срѣтаетъ. \textit{Камо пойду отъ Духа Твоего? и отъ лица Твоего камо бѣжу? Аще взыду на небо, Ты тамо еси; аще сниду во адъ, тамо еси; аще возму крилѣ мои рано}, по другому переводу, \textit{денницы, и вселюся въ послѣднихъ моря, и тамо бо рука Твоя наставитъ мя, и удержитъ мя десница Твоя} и проч., глаголетъ Псаломникъ\footnote{Пс.~138,~7--10.}. Едино убѣжище есть намъ грѣшнымъ милосердіе Его, къ которому отъ гнѣва Его праведнаго убѣгать и прибѣгать должно намъ, а предъ Нимъ падать и съ сокрушеніемъ сердца просить прощенія.

\paragraph*{СХLIІІ.} Видишь, что во время весны извнутрь земли исходятъ зелія и издаютъ листвія, цвѣты и плоды свои; такожде древа издаютъ листвія, цвѣты и плоды, которые во время зимы не видны были и аки бы безплодны были, и одно отъ другаго почти не разнствовало. Весна образъ есть воскресенія мертвыхъ. 1)~Вѣруемъ, что тако возстанутъ и изыдутъ тѣлеса наша мертвая изъ гробовъ, въ которыхъ сокровенны были, какъ видимъ нынѣ исходящая зелія изъ земли, въ которой чрезъ зиму сокрывалися. "--- 2)~Какъ во время зимы не распознаемъ, какой цвѣтъ и плодъ въ какомъ зеліи и древѣ имѣется, но весна тое все показуетъ: такъ нынѣ не можемъ познать, что у кого внутрь, въ сердцѣ и душѣ крыется "--- добро или зло, благочестіе или нечестіе "--- и часто добраго за злаго, и злаго за добраго вмѣняемъ; но въ воскресеніе все открыется ясно. "--- 3)~Какъ во время весны листвія и цвѣты зелій и древесъ извнутрь ихъ исходятъ и являются внѣ, и тѣми листвіями и цвѣтами аки одѣваются зелія и древа, и тако прекрасный образъ и видъ представляютъ смотрящимъ: тако въ воскресеніи доброта святыхъ на тѣлесахъ ихъ явится, которая нынѣ въ душахъ ихъ сокровенна пребываетъ, и тою добротою, аки ризою прекрасною, одѣются тѣлеса ихъ оживотворившаяся. Будутъ бо сообразны прославленному тѣлу Спасителя нашего, по словеси апостола: \textit{Иже преобразитъ тѣло смиренія нашего, яко быти сему сообразну тѣлу славы Его}\footnote{Филип.~3,~21.}. «Оле, глаголетъ Златоустъ святый, Оному, Иже сѣдитъ одесную Отца, сообразно сіе тѣло бываетъ; Оному, Иже отъ ангелъ покланяется; Оному, Ему же предстоятъ безплотныя силы; Оному, Иже превыше есть всякаго начала и власти и силы»\footnote{Бес.~13"~я на вышепис. мѣсто Апостола.}. "--- 4)~Якоже во время весны изсохшая зелія и древа такогожде вида бываютъ, какого были во время зимы, то"=есть, непріятны и гнусны, и не иному чему, какъ сожженію подлежатъ: тако и грѣшныхъ людей тѣлеса наги будутъ и безобразны, какъ и нынѣ имѣются, паче же далеко большее на нихъ явится безобразіе; яко грѣхъ и тьма, которая нынѣ въ сердцахъ и душахъ ихъ крыется, тогда внѣ на нихъ явится во свидѣтельство имъ, что они не о Хрістѣ провождали на землѣ житіе, но въ своихъ похотехъ. "--- 5)~А отъ сего послѣдовательно научаемся, что у всякаго человѣка доброта или безобразіе, благочестіе или нечестіе внутрь имѣется; и чего нѣтъ внутрь, въ сердцѣ и душѣ, того и въ самой вещи нѣтъ. Вѣра, любовь, терпѣніе, смиреніе и прочая внутрь должны быть; и когда нѣтъ внутрь, въ сердцѣ, то не ино что, какъ лицемѣріе, которое внѣ является быть нѣчто, но въ самой вещи ничтоже есть. "--- 6)~Никого не должно осуждать и судить, такожде и хвалить безсмысленно, яко не знаемъ, что у кого въ сердцѣ крыется, и часто называемъ безумно того злымъ, кто внутрь и въ самой вещи добръ, и того добрымъ, кто внутрь золъ, и тако судіи неправедные бываемъ. Отъ чего отвращаетъ апостолъ: \textit{прежде времене ничтоже судите, дондеже пріидетъ Господь, иже во свѣтѣ приведетъ тайная тьмы, и объявитъ совѣты сердечныя: и тогда похвала будетъ комуждо отъ Бога}\footnote{1~Кор.~4,~5.}. "--- 7)~Искать вѣрою доброты и украшенія души, и просить того у Хріста, Который единъ можетъ очистить ее отъ сквернъ и украсить; а не гоняться за красотою міра сего: яко красота душевная вѣчна есть, какъ и сама душа, красота же тѣлесная временна есть.

\paragraph*{СХLIV.} Видишь, что слина и прочее, что отъ пищи и питія имѣется, доколѣ внутрь содержится, не гнушается тѣмъ человѣкъ; а когда извержется вонъ, тогда всякъ гнушается тѣмъ, что у себе внутрь имѣлъ. Тако имѣется грѣхъ: доколѣ внутрь человѣка есть и совершаетъ его, не гнушается имъ человѣкъ: но когда благодатію Божіею избавится отъ него, и аки блевотину извергаетъ отъ сердца своего, тогда ему мерзѣетъ. Откуду бываетъ, что всякимъ грѣхомъ, въ другихъ видимымъ, гнушается человѣкъ, хотя и самъ тойжде грѣхъ въ себѣ имѣетъ. Гнусенъ есть и блуднику блудъ, хищнику хищеніе, вору воровство, лукавцу лукавство, завистливому зависть, злобному злоба, клеветнику клевета, сквернословцу сквернословіе, въ ближнемъ своемъ видимое: но въ себѣ хотя и имѣетъ тое зло, однакожъ не гнушается тѣмъ; не гнушается же, понеже смертоноснаго яда того не познаетъ. О, когда"=бы чудовище сіе и страшилище ужасное увидѣлъ бѣдный грѣшникъ въ себѣ, и мерзость его обонялъ бы: болѣе бы гнушался тѣмъ, нежели гнушается каломъ, извнутрь чрева своего изверженнымъ! Но сія есть общая наша слѣпота, бѣдность и окаянство, что тоя мерзости не познаемъ и не тщимся познавать. Такъ"=то, возлюбленный хрістіанине, заразилъ ядомъ своимъ сердце наше сатана, и помрачилъ душевное наше око! "--- Сей случай и разсужденіе учитъ насъ: 1)~Что сердце наше глубоко растлѣнно есть, и безъ благодати Божіей не ино что, какъ зло замышляетъ. Ибо зміино на немъ посѣянное сѣмя приличные и плоды родитъ. 2)~Учитъ испытовать тое, и мерзость въ немъ крыющуюся познавать, и тою гнушаться. Сіе наипаче бываетъ, когда уединяемся во дни или въ нощи, и въ глубину сердца нашего проникаемъ и разсматриваемъ, какіе отъ него замыслы, начинанія и намѣренія родятся, и къ чему клонятся. Такожде отъ ближняго нашего себе познаваемъ, то"=есть, что видимъ въ ближнемъ, тоежде въ насъ есть: когда видимъ ближняго гнѣвающагося, или иную какую душевную немощь показывающаго, тутъ должно обратить и на себе очи, что и мы тѣмжде недугомъ немоществуемъ. Всякаго бо грѣха сѣмя въ сердцѣ нашемъ имѣется, и тако, что примѣчаемъ въ ближнемъ нашемъ дѣломъ совершаемое, тое можемъ и мы при случаѣ дѣлать, и злаго сѣмене злые плоды показывать. Природно бо намъ зло сіе, ибо съ тѣмъ вси родимся. "--- 3)~Очищать его тщатися нелѣностно, по подобію огородника или садодѣлателя, который сучки вредные отрѣзываетъ отъ древесъ, чтобы ихъ не вредили. Тако и намъ отъ помысловъ сердце наше очищать, которые вредятъ очищенію души нашей и не попускаютъ плодовъ приносить, вѣры хрістіанской достойныхъ. Напр. возстаетъ ли помыслъ гнѣва, или злобы, или мщенія или ненависти, или блуда, или зависти, или гордости, или клеветы, и проч.? "--- тотчасъ его отсѣкать, или подавлять, чтобы не возраслъ, и тако не повредилъ бы души; и въ семъ помощи просить отъ Хріста Господа нашего. "--- 4)~Съ вѣрою просить тогожде Избавителя, дабы Онъ силою Своею отъ сего мерзкаго мучителя насъ свободилъ, да, отъ него свободившеся, усердно и безпорочно поработаемъ Свободителю, по словеси Его: \textit{аще Сынъ вы свободитъ, воистинну свободни будете}\footnote{Іоан.~8,~36.}. Тогда будемъ гнушаться и мы мерзостію грѣха, какъ гноемъ изверженнымъ и смраднымъ. И сія"=то есть истинная хрістіанская свобода, о которой тщатися неусыпно должны мы, и предпочитать ее паче всѣхъ міра сего сокровищъ. Сія бо свобода далеко превосходитъ свободу князей, вельможъ и господъ вѣка сего, которые людямъ повелѣваютъ, но сами своимъ страстемъ повинуются; надъ людьми господствуютъ, но надъ ними господствуетъ грѣхъ. Хрістіанская свобода не тако: она Богу единому, и ради Бога людямъ повинуется и работаетъ, но грѣховнаго ига не носитъ. Поищемъ сего, возлюбленный хрістіанине, дражайшаго сыновъ Божіихъ бисера, который они въ \textit{скудѣльныхъ своихъ сосудѣхъ носятъ}\footnote{2~Кор.~4,~7.}.

\paragraph*{СХLV.} Видишь, что хотящіи воню смрадную изъ покоя выгнать, зажигаютъ благовонные порошки или иное что, къ тому угодное, и тако зловоніе благовоніемъ изгоняется. Душа человѣческая есть какъ храмина, которая до паденія праотеческаго была исполнена благолѣпіемъ и добронравіемъ, какъ благовонными ароматами, и была жилищемъ Святаго Духа; но по паденіи вошла въ нее злоба, какъ злосмрадная воня, и исполнила ее. Чувствуемъ и обоняемъ злосмрадную воню сію вси внутрь себе, когда извнутрь, \textit{отъ сердца исходятъ помышленія злая}, какъ смрадные запахи; чувствуемъ, любезный хрістіанине, и воздыхаемъ о семъ, \textit{совлещися ветхаго человѣка и облещися въ новаго желающе}. О, да сподобитъ насъ сего человѣколюбивый Господь, Который ради человѣка учинился человѣкомъ! "--- Какъ же можемъ зловоніе изъ души нашей выгнать, "--- какъ аще не противнымъ тому благоуханіемъ? Противное бо противнымъ прогоняется: такъ тьма свѣтомъ, холодъ зноемъ и проч. прогоняется. Отъ паденія Адамова вошла злоба и злонравіе въ души наши. Господь пришелъ на землю и показалъ намъ образъ святаго Своего добронравія и благостыни, показалъ образъ смиренія, терпѣнія, кротости, любве и прочіихъ святѣйшихъ нравовъ, въ божественномъ Его сердцѣ бывшихъ; и, какъ орелъ птенцовъ своихъ учитъ летати, такъ Онъ вѣрныхъ научилъ въ мірѣ семъ жити, и Богу угождати, якоже глаголетъ: \textit{научитеся отъ Мене}\footnote{Матѳ.~11,~29.}; и паки: \textit{образъ дахъ вамъ, да, якоже Азъ сотворихъ вамъ, и вы творите}\footnote{Іоан.~13,~15.}; и апостолъ глаголетъ: \textit{Хрістосъ пострада по насъ, намъ оставль образъ, да послѣдуемъ стопамъ его}\footnote{1~Петр.~2,~21.}. Его убо благостію, смиреніемъ, терпѣніемъ, кротостію и любовію да изгоняемъ изъ сердецъ нашихъ злобу, гордость, гнѣвъ, ненависть, зависть и всякую нечистоту; и тако, вводя въ души наши благонравіе Хрістово, своего злонравія потщимся избавляться, и, какъ злую воню, прибывшую намъ отъ врага, благовоніемъ непорочныхъ нравовъ Хрістовыхъ изгонять будемъ. И чимъ болѣе тое будемъ дѣлать, тѣмъ болѣе очистится душевный домъ нашъ: якоже чимъ болѣе курится благовонный порошокъ въ покоѣ, тѣмъ болѣе очищается отъ зловонія покой, и большее благоуханіе въ себѣ имѣетъ. Тяжко сіе есть плоти нашей, хрістіанине, чтобы своего злонравія отрещися; но сего должность вѣры хрістіанскія требуетъ отъ насъ. "--- Къ сему очищенію должно чинить слѣдующее: 1)~Образъ имѣть предъ собою всегда непорочнаго житія Хрістова, и на него душевныма очима взирать, "--- къ чему нужно чтеніе, или слушаніе святаго Евангелія, въ которомъ написано Хрістово житіе. Симъ не исключается образъ святыхъ Божіихъ житія, которые такожде взирали на сей живый божественный образъ, и подражали тому. "--- 2)~Всякимъ тщаніемъ и силою нудить себе къ тому, чтобы противиться возстающимъ въ сердцѣ страстямъ и пресѣкать ихъ въ началѣ, какъ только начнутъ изъ сердца показываться, и противная имъ воспріимать въ сердцѣ, напр. противу гордости смиреніе, противу гнѣва, злобы, мщенія, ненависти, зависти, нечистоты "--- терпѣніе, кротость, любовь, доброжелательство, чистоту отъ Хрістова житія воспріимать, и тѣмъ ихъ возстающихъ усмирять и укрощать. Удобно о семъ говорить, но не удобно тое дѣлать, хрістіанине! Ибо природно есть зло, и противу страстей бороться и ихъ побѣждать не ино что, какъ противу себе самого бороться и себе самого побѣждать: трудная брань и побѣда, но славная! "--- 3)~Понеже тщаніе наше не сильно безъ помощи Божіей, яко весьма растлѣнно имѣемъ сердце: того ради должно Хріста на помощь призывать, чтобъ Самъ намъ помоглъ, и вмѣсто злонравія нашего Свое благонравіе насадилъ Духомъ Своимъ Святымъ въ сердцахъ нашихъ. На сіе бо Онъ и въ міръ пришелъ, чтобъ злобу, какъ злую воню, въ душахъ нашихъ вселившуюся, выгнать, и Своихъ божественныхъ нравовъ благоуханіе въ нихъ насадить. Наше тщаніе, вѣра и молитва да будетъ прилѣжная, "--- что Онъ видя, преклонится на милость и подастъ намъ благодать Духа Своего Святаго, которая намъ во всемъ будетъ помогать, что надлежитъ до дѣла спасенія нашего.

\paragraph*{СХLVІ.} Видишь, что земледѣлецъ первѣе полагаетъ въ сердцѣ своемъ надежду о собраніи плодовъ, и потомъ, такую положивши надежду, принимается за дѣло, котораго посредствіемъ до такой надежды доходитъ, дѣлаетъ землю, пашетъ и сѣетъ. Тако и во всякой вещи надежда, воспріятая въ сердцѣ, къ труду и посредствію, чтобъ надежду получить, надѣющагося поощряетъ. Купца надежда богатства купечествовать, воина надежда побѣды и славы подвизаться, ученика надежда разума и мудрости учиться въ школахъ и проч. убѣждаетъ, и въ дѣлѣ начатомъ не ослабѣвать увѣщаваетъ. Такъ и хрістіанамъ въ дѣлѣ спасенія своего поступать должно. Надежда хрістіанская есть вѣчная жизнь въ будущемъ вѣкѣ, якоже въ святомъ Сѵмволѣ глаголется: \textit{чаю воскресенія мертвыхъ, и жизни будущаго вѣка}. Сію надежду потеряли мы во Адамѣ; но во Хрістѣ обрѣтаемъ ее, когда въ Него чистосердечно вѣруемъ, и вѣру нашу оказываемъ любовію и терпѣніемъ. Должно убо всякому вѣрующему во Хріста сію надежду въ сердцѣ своемъ положить и утвердить, какъ земледѣльцы дѣлаютъ, и тако въ подвигѣ вѣры и прочей хрістіанской должности трудиться, не ослабѣвать, взирая на вожделѣнный и неоцѣненный вѣчнаго живота плодъ. Земледѣльцы очищаютъ нивы своя, исторгаютъ плевелы, чтобъ не препятствовати расти пшеницѣ: тако должно намъ отсѣкать прихоти и страсти отъ сердецъ нашихъ, дабы не препятствовали расти сѣмени Божія слова, и тако бы безплоднымъ его не сотворили. "--- Земледѣльцы ожидаютъ ранняго и поздняго дождя на нивы своя, безъ котораго нивы безплодны бываютъ, сколько ни трудятся на нихъ: тако и нашъ трудъ тщетенъ бываетъ, аще не снидетъ на нивы сердецъ нашихъ дождь благодати Божія. Должно убо и намъ трудиться и ожидать дождя милости Божія, и молиться Богу, чтобъ послалъ намъ свыше росу благости Своея и напоилъ тою нивы сердецъ нашихъ; должно воздѣвать руцѣ и очи къ Живущему на небеси, якоже Псаломникъ дѣлалъ и подалъ образъ намъ. \textit{Къ Тебѣ возведохъ очи мои, живущему на небеси. Се яко очи рабъ въ руку господій своихъ, яко очи рабыни въ руку госпожи своея: тако очи наши ко Господу Богу нашему, дондеже ущедритъ ны}\footnote{Пс.~122,~1 и 2.}. "--- Земледѣльцы чрезъ все лѣто трудятся, потѣютъ и зной солнечный претерпѣваютъ, чтобъ вожделѣнный плодъ получить: тако хрістіанамъ должно чрезъ все житія своего время трудиться, подвизаться и приключающіяся напасти терпѣть, дабы не лишиться вѣчнаго живота плода, который Хрістосъ, Сынъ Божій, трудами Своими и кровію намъ пріобрѣлъ. "--- Земледѣльцы приспѣвшіе плоды съ радостію собираютъ: тако хрістіане, со страхомъ и трепетомъ спасеніе свое содѣловающіи, съ радостію воспріемлютъ тое отъ руки Господни, что обѣщалъ вѣрующимъ и любящимъ Его. \textit{Сѣющіи слезами, радостію пожнутъ. Ходящіи хождаху, и плакахуся метающе сѣмена своя: грядуще пріидутъ радостію вземлюще рукояти своя}\footnote{Пс.~125,~5 и 6.}. И: \textit{собранніи Господемъ обратятся, и пріидутъ въ Сіонъ съ радостію, и радость вѣчная надъ главою ихъ: надъ главою бо ихъ хвала и веселіе, и радость пріиметъ я; отбѣже болѣзнь и печаль и воздыханіе}\footnote{Ис.~35,~10.}. "--- Земледѣльцы лѣнивые и въ праздности живущіи, увидѣвше братію свою трудовъ своихъ плоды собирающихъ и радующихся, себе же ничего не имѣющихъ, яко не трудившихся, скорбятъ, тужатъ, печалуютъ и окаеваютъ себе, что въ лѣтѣ не трудилися, и тако плодовъ не имѣютъ: тако хрістіане небрежливые, увидѣвше прочіихъ за подвигъ вѣры и труды, въ благочестіи подъятые, ублажаемыхъ и прославляемыхъ отъ Господа, восплачутся и возрыдаютъ неутѣшно, и будутъ себе окаевать, что не хотѣли трудиться во временномъ житіи. Лазарь, упоминаемый во Евангеліи, упокоевается на лонѣ Авраамли по трудахъ и болѣзняхъ своихъ временныхъ; но богачъ несмысленный, который, \textit{облачашеся въ порфиру и виссонъ, веселяся по вся дни свѣтло}, по веселостяхъ и роскошахъ маловременныхъ, находится въ мукахъ и зритъ Авраама издалеча и Лазаря на лонѣ его, и возглашаетъ и глаголетъ: \textit{отче Аврааме! помилуй мя, и посли Лазаря, да омочитъ конецъ перста своего въ водѣ, и устудитъ языкъ мой, яко стражду въ пламени семъ}\footnote{Лук.~16,~24.}, "--- и во вѣки возглашать будетъ, но ничего не получитъ. "--- Отъ вышереченныхъ видишь, хрістіанине: 1)~Что настоящее время есть время трудовъ, подвига, скорбей и креста хрістіанамъ; а будущій вѣкъ есть время покоя, воздаянія, радости и увеселенія, яко \textit{многими скорбьми подобаетъ намъ внити въ царствіе Божіе}, по апостольскому словеси\footnote{Дѣян.~14,~22.}. "--- 2)~Видишь, какъ ослѣпленъ есть въ духовныхъ бѣдный человѣкъ! Чего не дѣлаютъ люди ради временнаго и скоро погибающаго прибытка? какихъ трудовъ и бѣдъ не пріемлютъ? Но ничто имъ не тягостно и не неудобно, чтобъ желаемое получить. Такъ сильно подвигаетъ, поощряетъ и одобряетъ ихъ несумнѣнная желаемаго добра надежда, хотя часто и обманываются. Не все бо получаетъ человѣкъ, чего хощетъ и надѣется получить, яко въ Божіей власти состоитъ и временное благополучіе, а не въ человѣческой, и кому хощетъ, подаетъ его: однакожъ люди и несовершенно извѣстнаго такъ тщательно, понеже видятъ тое тѣлесными очами, ищутъ добра. Но вѣчнаго блаженства, предъ которымъ вся слава, богатство, честь и царство земное, какъ ничтоже есть, или нерадиво ищетъ, или совсѣмъ оставилъ искать человѣкъ. О! когда бы внутреннее око открылося человѣку, и увидѣлъ бы въ единомъ мгновеніи ока будущую славу, избраннымъ Божіимъ уготованную: стремился бы къ ней такъ, какъ жаждою палимый къ источнику водному, и ничто бы его отъ того не могло отвратить. Повѣрь, хрістіанине, что честь, богатство, слава, сласть и утѣхи міра сего, какъ смрадная мертвечина, омерзѣютъ ему, и самое злостраданіе и мученіе не сильно воспятить и удержать стремленія его. Спасительное Хрістово воплощеніе и страданіе научаетъ насъ всему. Лишились мы во Адамѣ будущаго того блаженства и подпали вѣчному бѣдствію: Хрістосъ, Сынъ Божій, такъ великъ и высокъ, что никто Его большій и высшій есть, ибо Богъ есть. Ради того въ міръ пришелъ и пострадалъ, чтобъ отъ вѣчнаго бѣдствія избавить и къ вѣчному блаженству возвратить человѣка отпадшаго. Отъ сего единаго можемъ, хрістіанине, познать, что велико и неизреченно бѣдствіе есть оное, отъ котораго Хрістосъ, хотя насъ избавить, кровь Свою изліялъ; и велико и непостижимо есть блаженство будущее, которое смертію Своею исходатайствовалъ намъ, потерявшимъ оное. Слава непостижимой любви Его къ намъ и милосердію! Непостижимо убо какъ зло вѣчнаго мученія, такъ добро вѣчнаго блаженства, ради котораго Сынъ Божій и Богъ истинный Себе Самаго не пощадѣлъ, чтобъ отъ онаго насъ избавить, а сіе намъ возвратить. Но се есть общая наша слѣпота и окаянство, что или мало вѣруемъ, или совсѣмъ не вѣруемъ тому, хотя и различно Богъ открылъ въ святомъ Своемъ словѣ блаженство оное въ пользу и утѣшеніе наше. "--- 3)~Несмысленно дѣлаютъ тіи хрістіане, которые не хотятъ подвизаться, трудиться въ дѣлѣ званія хрістіанскаго и желаютъ вѣчный животъ получить; въ сластехъ, роскошахъ, чести, славѣ и богатствѣ міра сего веселиться, и со Хрістомъ въ будущемъ вѣкѣ царствовать хотятъ; креста со Хрістомъ носить отрицаются, стыдятся и ужасаются его и отбѣгаютъ его, но прославиться съ Нимъ хотятъ. Сіи подобно тѣмъ людямъ дѣлаютъ, которые хотятъ плодъ собрать, но землю пахать, сѣять и трудиться не хотятъ; которые славу побѣды имѣть и торжествовать хотятъ, но подвизаться не хотятъ; отъ царя милость получить желаютъ но вѣрно ему служить не хотятъ; которые до града какого дойти желаютъ, но путемъ, къ тому граду ведущимъ, итить не хотятъ, паче же другимъ путемъ идутъ. Тѣсный путь есть къ вѣчному животу, хрістіанине; а пространный въ вѣчную пагубу ведетъ, по неложному словеси Хріста Господа нашего\footnote{Матѳ.~7,~13 и 14.}. Вѣра, которая указуетъ вѣчный животъ и ведетъ къ нему, многоразличному подлежитъ искушенію. Діаволъ, міръ, съ суетою и \textit{плоть со страстьми и похотьми} тщатся ея цѣлость нарушить и истребить. Противу всѣхъ сихъ супостатовъ надобно подвизаться, и вѣру паче живота своего хранить. Чтожъ отсюду слѣдуетъ, какъ ни непрестанное озлобленіе, скорбь и тѣснота душѣ подвизающейся? То тотъ, то другій, то третій врагъ нападаетъ и хощетъ у души отнять неоцѣненное сокровище "--- вѣчное спасеніе. Недосугъ будетъ думать тому о чести, славѣ, богатствѣ и роскоши міра сего, объ отмщеніи обиды, отъ человѣка нанесенной, и о прочіихъ къ міру надлежащихъ, кто хощетъ съ оными врагами вступить въ брань. Едино будетъ у него тщаніе тое, чтобъ отъ тѣхъ враговъ не быть побѣжденну.

\subsection*{\bfseries * * *}

Тако отъ всякаго случая и отъ всякаго видимаго созданія къ невидимымъ можно разсужденіе обращать, напр. отъ видимаго міра къ невидимому, отъ видимаго свѣта къ невидимому и вѣчному, то"=есть, Богу, отъ видимой тмы къ невидимой, то"=есть, діаволу и грѣху, отъ видимаго зрѣнія къ невидимому, отъ тѣлесной слѣпоты къ душевной, отъ тѣлеснаго здравія къ душевному, отъ тѣлесной немощи къ душевной и проч., и духовно пользоваться. Ибо что видимъ въ видимыхъ вещахъ, тое имѣется и въ невидимыхъ, и что чувствуемъ въ тѣлѣ, тое примѣчается и въ душѣ, хотя неравнымъ образомъ, какъ сіе въ святомъ Писаніи, въ которомъ души нашея состояніе доброе и недоброе описуется, читаемъ. Имѣетъ тѣло наше животъ, смерть, здравіе, немощь, нищету, богатство, благообразіе, безобразіе, скорбь, утѣшеніе, плѣненіе, свободу, тьму, свѣтъ, алчбу, жажду, пищу, питіе, одежду, трудъ, покой и прочее: тоежде примѣчается и въ душѣ нашей. Тѣло оживляется душею, умираетъ, когда душа отъ него изыдетъ: душа живетъ благодатію Божіею, умираетъ, когда ее благодать оставляетъ. Тѣлу нашему свѣтитъ солнце, луна и свѣтильникъ; но душѣ нашей свѣтъ, которымъ просвѣщается, есть Хрістосъ, \textit{Иже просвѣщаетъ всякаго человѣка, грядущаго въ міръ}\footnote{Іоан.~1,~9.}, "--- и свѣтильникъ законъ Божій, по писанному: \textit{свѣтильникъ ногама моима законъ Твой}\footnote{Пс.~118,~105.}. Слѣпо есть тѣло наше, когда не имѣетъ очесъ отверстыхъ; глухо, когда не имѣетъ ушей: слѣпотствуетъ душа, когда не имѣетъ просвѣщенія Духа Святаго; и глуха, когда не имѣетъ надлежащаго слышанія гласа Божія, въ святомъ Писаніи простираемаго. Тѣло наше имѣетъ богатство "--- злато, сребро и прочія вещи, по мнѣнію міра сего драгія; нище и убого, когда не имѣетъ: имѣетъ и душа богатство свое "--- дарованія Божія духовная, вѣру, оправданіе, освященіе, мудрость и проч.; нища, убога и нага, когда тѣхъ не имѣетъ. Тѣлу нашему безобразіе дѣлаютъ струпы, раны, гной, морщины и прочая: душѣ безобразіе приносятъ грѣхъ и страсти безчинныя. Тѣло наше алчетъ и жаждетъ безъ пищи и питія, и наготу терпитъ безъ одѣянія: душа алчетъ и жаждетъ безъ слышанія Божія слова, и наготствуетъ безъ благодати Его. Тѣло утѣшается и укрѣпляется пищею и покрывается одеждою: душа утѣшается словомъ Божіимъ благодатію Его, живущею въ ней, и покрывается нагота ея Хрістовою правдою. Плѣняется тѣло наше, когда попадается въ плѣнъ непріятелю, или заключается въ темницѣ, или узами связуется; свободу получаетъ, когда сихъ бѣдствій избавляется: имѣетъ и душа свое горькое плѣненіе, когда человѣкъ попадается подъ власть діавольскую, работаетъ грѣху и страстемъ безчиннымъ, блудодѣйствуетъ, пьянствуетъ, похищаетъ, крадетъ, злобится и прочія страсти совершаетъ; свобождается отъ Хріста, когда обратится отъ грѣховъ къ Богу и вѣруетъ во Хріста, избавителя всѣхъ. Такъ и въ прочемъ невидимыхъ съ видимыми, и души съ тѣломъ сходство примѣчается, о чемъ святое Писаніе представитъ тебѣ, хрістіанине, когда съ прилѣжаніемъ и желаніемъ спасенія твоего будешь тое читать или слушать. А какъ намъ отъ бѣдствій духовныхъ свободиться и блаженство духовное получить, представляется въ томжде Писаніи образъ, то"=есть, вѣра во Хріста, Который вѣрующихъ въ Него избавляетъ и блаженными дѣлаетъ, якоже Самъ глаголетъ: \textit{Аще Сынъ вы свободитъ, воистинну свободни будете}\footnote{Іоан.~8,~36.}. И паки: \textit{Хрістосъ Іисусъ бысть намъ премудрость отъ Бога, правда же и освященіе и избавленіе}, апостолъ написалъ\footnote{1~Кор.~1,~30.}.

Примѣчай здѣ, хрістіанине: 1)~Сіи пункты служатъ къ изъясненію слѣдующихъ моихъ предложеній и разсужденій. 2)~Сіи подобія, взятыя отъ различныхъ вещей, чувствамъ подлежащихъ, и приложенныя къ невидимымъ, наипаче полагаются \textit{ради простаго народа}, который не можетъ понять ученія о духовныхъ вещахъ. Ничимъ бо такъ не изъясняется и понятнымъ дѣлается и въ памяти углубляется ученіе, какъ подобіями тѣхъ вещей, которыя чувствамъ нашимъ подлежатъ и предъ глазами нашими обращаются. Откуду видимъ, что и въ священныхъ книгахъ премного таковыхъ подобій имѣется, которыми спасительное ученіе изъясняется, и небесныя и духовныя вещи земными и видимыми доказываются, какъ"=то: слово Божіе "--- \textit{сѣмени}, грѣшники \textit{болящимъ}, непросвѣщенные благодатію Божіею \textit{слѣпымъ}, неизвѣстный втораго Хрістова пришествія день \textit{татю}, въ нощи приходящему, уподобляются, и проч. "--- 3)~Вещь едина другой подобна бываетъ не во всемъ, но въ нѣкоторыхъ только свойствахъ: того ради и о подобіяхъ, какъ здѣ, въ семъ §, такъ и на прочіихъ мѣстахъ приведенныхъ, разумѣть должно тоежде, что они въ нѣкіихъ только свойствахъ сходны съ тѣми вещами, которыя изъясняютъ "--- 4)~Чтобъ отъ видимыхъ вещей къ невидимымъ разсужденіе обращать и тако духовно пользоваться, требуется размышленіе вѣрою, которая за основаніе свое имѣетъ Божіе слово, управляемое и утверждаемое, безъ которой удобно погрѣшить можно.


\section[Статья 2-я. О сердцѣ и языкѣ человѣческомъ.]{статья вторая.\\\bfseries О сердцѣ и языкѣ человѣческомъ.}
\subsection[Глава 1-я. О сердцѣ человѣческомъ.]{глава первая.\\\bfseries О сердцѣ человѣческомъ.}

\begin{quotation}\textit{Благій человѣкъ отъ благаго сокровища сердца своего износитъ благое; и злый человѣкъ отъ злаго сокровища сердца своего износитъ злое: отъ избытка бо сердца уста глаголютъ}, глаголетъ Господь\footnote{Лук.~6,~45.}.\end{quotation}
\begin{quotation}\textit{Даждь ми, сыне, твое сердце}\footnote{Притч.~24,~26.}.\end{quotation}


\paragraph*{§\:28.} Сердце здѣ разумѣется не естественно, поелику есть начало жизни человѣческія, какъ философы разсуждаютъ, но нравоучительно, то"=есть, внутреннее человѣческое состояніе, расположеніе и наклоненіе. Тако разумѣется оное апостольское слово: \textit{сердцемъ вѣруется въ правду}\footnote{Римл.~10,~10.}, "--- и пророческое оное: \textit{рече безуменъ въ сердцѣ своемъ: нѣсть Богъ}\footnote{Пс.~13,~1.}, и на прочіихъ святаго Писанія мѣстахъ. Откуду сердце сіе уподобляется въ Писаніи сокровищу, въ которомъ или благое или злое сокрывается, якоже глаголетъ Господь: \textit{благій человѣкъ отъ благаго сокровища сердца своего износитъ благое; и злый человѣкъ отъ злаго сокровища сердца своего износитъ злое}. То"=есть: какое человѣкъ внутреннее состояніе имѣетъ, такія у него слова и дѣла внѣшнія являются, и о чемъ внутрь поучается, замышляетъ, тщится, тое и внѣ показуетъ. Слово и дѣло внѣшнее есть вѣстникъ и свидѣтель внутренняго человѣческаго состоянія. Сердце, \textit{естественно} разсуждаемое, поелику есть начало живота человѣческаго, у всѣхъ равно, то"=есть, у добрыхъ и злыхъ, якоже и прочіи естественные уды; но \textit{нравоучительно} разумѣемое не равно есть, но у инаго доброе, у инаго злое, и проч.

\paragraph*{§\:29.} Сердце сіе человѣческое само собою, безъ благодати Божіей, есть злое, яко не ино что, какъ только суетное и злое помышляетъ, якоже писано есть: \textit{прилѣжитъ помышленіе человѣку прилѣжно на злая отъ юности его}\footnote{Быт.~8,~21.}. Откуду сердце сіе называется въ Писаніи \textit{каменное}, какъ ниже увидишь: яко ни увѣщаніями, ни угроженіями, ни милостію, ни строгостію не преклоняется и не умягчается. Откуду Богъ обѣщается вѣрнымъ Своимъ дать \textit{сердце иное. И дамъ имъ сердце ино, и духъ новъ дамъ имъ, и исторгну каменное сердце отъ плоти ихъ, и дамъ имъ сердце плотяное}\footnote{Іез.~11,~19.}. Откуду о человѣкѣ, благодатію Божіею неотрожденномъ, глаголется: \textit{рече безуменъ въ сердцѣ своемъ: нѣсть Богъ}. И Господь глаголетъ: \textit{отъ сердца исходятъ помышленія злая, убійства, прелюбодѣянія, любодѣянія, татьбы, лжесвидѣтельства, хулы}\footnote{Матѳ.~15,~19.}. И хотя благодатію Божіею просвѣщенъ будетъ человѣкъ, однакожъ зло сіе, отъ сердца исходящее, чувствуетъ, и имѣетъ многа труда чрезъ все житіе противу природнаго того бѣдствія духовно подвизаться, непрестанная ему предлежитъ противу того брань. Откуду водою и Духомъ отрожденныхъ и святыхъ читаемъ паденія въ тяжкіе грѣхи, которые отъ сего растлѣннаго источника проистекаютъ. Сего ради повелѣно всѣмъ вѣрнымъ молитися и просить у небеснаго Отца благодати Святаго Духа, дабы возмогли противитися тому отъ сердца происходящему злу и умерщвляти тое. Растлѣнному сердцу нашему помогаетъ противу насъ діаволъ со злыми своими аггелами и соблазнами міра сего. Отсюду послѣдуетъ, что сердце доброе и богобоящееся не можетъ быть, какъ только отъ Бога. Самъ бо человѣкъ сердца своего злаго на доброе премѣнить не можетъ; откуду Псаломникъ молится: \textit{сердце чисто созижди во мнѣ, Боже, и духъ правъ обнови во утробѣ моей}\footnote{Пс.~50,~12.}! Сего ради всякому, хотящему имѣть сердце доброе, надобно со Псаломникомъ молиться о томъ Богу, дабы, по милостивому Своему обѣщанію, подалъ сердце ино и духъ новъ, и отъялъ сердце каменное, и подалъ сердце плотяное, дабы на немъ, яко на новыхъ скрижалехъ, возмоглъ написанъ быти законъ Евангелія Хрістова, \textit{не черниломъ, но духомъ Бога жива, не на скрижалехъ каменныхъ, но на скрижалехъ сердца плотяныхъ}\footnote{2~Кор.~3,~3.}, якоже писано есть: \textit{яко сей завѣтъ, егоже завѣщаю дому Израилеву, по днѣхъ онѣхъ}, глаголетъ Господь, \textit{дая законы Моя въ мысли ихъ, и на сердцахъ ихъ напишу я, и буду имъ въ Бога, и тіи будутъ Ми въ люди}\footnote{Іер.~31,~33.}.

\paragraph*{§\:30.} Сердце едино и душу едину имѣть значитъ согласіе великое имѣть въ ученіи и волѣ, якоже пишется о вѣрныхъ, бывшихъ во днехъ апостольскихъ: \textit{народу вѣровавшему бѣ сердце и душа едина}\footnote{Дѣян.~4,~32.}.

\paragraph*{§\:31.} Двоедушіе или двоесердечіе бываетъ въ человѣкѣ томъ, который иное языкомъ говоритъ, иное мыслитъ, каковаго обыкновенно называютъ вси двоедушнымъ, или явнѣе назвать можно льстецомъ, обманщикомъ и лукавцемъ, которому пороку противна добродѣтель есть простосердечіе.

\paragraph*{§\:32.} Сердце чисто и правое тое примѣчается въ святомъ Писаніи, которое 1)~тщится волѣ Божіей послѣдовать; 2)~ищетъ во всѣхъ своихъ дѣлахъ, словахъ и начинаніяхъ славы Божіей; 3)~печется о пользѣ ближняго своего, то"=есть, всякаго человѣка. О семъ яснѣе и пространнѣе ниже увидишь, читатель.

\paragraph*{§\:33.} Сердце сіе есть начало и корень всѣхъ дѣяній нашихъ. Что бо ни дѣлаемъ внутрь и внѣ насъ, сердцемъ дѣлаемъ "--- или добро, или зло. Сердцемъ вѣруемъ, или не вѣруемъ; сердцемъ любимъ, или ненавидимъ; сердцемъ смиряемся, или гордимся; сердцемъ терпимъ, или ропщемъ; сердцемъ прощаемъ, или злобимся; сердцемъ примиряемся, или враждуемъ; сердцемъ обращаемся къ Богу, или отвращаемся; сердцемъ приближаемся, приходимъ къ Богу, или отходимъ и удаляемся; сердцемъ благословимъ, или клянемъ; на сердцѣ радость или печаль, надежда или отчаяніе, покаяніе или нераскаянное житіе, страхъ или дерзновеніе; въ сердцѣ простота или лукавство; сердце воздыхаетъ, молится, уповаетъ, или противная дѣлаетъ, и проч. Слѣдственно, чего на сердцѣ нѣтъ, того и въ самой вещи нѣтъ. Вѣра не есть вѣра, любовь не есть любовь, когда на сердцѣ не имѣется, но есть лицемѣріе; смиреніе не есть смиреніе, но притворство, когда не въ сердцѣ; дружба не дружба, но горшая вражда, когда внѣ только является, а въ сердцѣ не имѣетъ мѣста. Откуду Богъ требуетъ отъ насъ сердца нашего: \textit{Даждь, Ми сыне, твое сердце}.

\paragraph*{§\:34.} Что въ сердцѣ зачинается, или доброе или злое, тое внѣ чрезъ уды тѣлесные является; и сердце человѣческое внѣшніе члены, какъ"=то: языкъ, руки, ноги и проч., какъ орудія употребляетъ къ произведенію замысловъ своихъ въ самое дѣло. Такъ языкомъ благословитъ, или клянетъ, руками похищаетъ, или подаетъ, убиваетъ, или сохраняетъ; ушами слушаетъ доброе, или злое; ногами ходитъ къ намѣренному мѣсту; другими членами другое намѣреніе свое совершаетъ. Кто доброе сердце и святою вѣрою Хрістовою очищенное имѣетъ, тотъ нелицемѣрно добрые и плоды внѣ показываетъ; а кто злое сердце имѣетъ, злые и плоды являетъ, якоже Господь глаголетъ: \textit{благій человѣкъ отъ благаго сокровища сердца своего износитъ благое: и злый человѣкъ отъ злаго сокровища сердца своего износитъ злое}.

\paragraph*{§\:35.} Всякое дѣло человѣческое не по внѣшнему, но по внутреннему сердца состоянію и намѣренію судится. Часто бо внѣшнее дѣло у того и у другаго едино бываетъ; но внутреннее расположеніе и намѣреніе различно можетъ быть. Напр. единъ судія не пріемлетъ мзды, и другой не пріемлетъ; первый ради того не пріемлетъ что указы монаршіе казнію грозятъ, а бралъ бы, когда бы того не было; другій не беретъ ради того, чтобы Бога не прогнѣвать преступленіемъ заповѣди Его святой: сей истинный немздоимецъ, доброе и богобоящееся сердце имѣетъ; а оный хотя руками не пріемлетъ мзды, но сердцемъ принимаетъ, и потому какъ мздоимецъ судится; ибо страхъ человѣческій имѣетъ, а не Божій; человѣка боится, а не Бога, и для того злое и невѣрное сердце имѣетъ. Единъ не крадетъ ради того, что не хочетъ; другій не крадетъ для того, что не можетъ, а кралъ бы, когда бы случай былъ: оный подлинно не тать есть, а сей всегда сердцемъ крадетъ. Единъ даетъ милостыню ради имени Хрістова, хотя въ бѣдности просящему помощи, а другій, чтобы славу добрую нажить: оный истинный милостивецъ, а сей тщеславецъ. Единъ входитъ въ церковь обще съ вѣрными молитися и славословити имя Божіе, другій посмотрѣть церемоніи: оному полезенъ, сему неполезенъ входъ церковный; ибо и язычники смотрятъ на церемонію хрістіанскую. Единъ воспріемлетъ рангъ ради того, чтобы богатство собрать и большую честь достать; а другій того ради, чтобы обществу и братіи своей послужить: сего доброе, а онаго злое и пагубное намѣреніе. Единъ проповѣдникъ говоритъ проповѣдь для того, чтобы показать себе, славу и похвалу отъ слышателей получить и высокій степень сыскать; другій проповѣдуетъ, чтобы людей пользовать: сего доброе намѣреніе, а того злое. Не убиваешь, не прелюбодѣйствуешь, не крадешь, не злословишь ближняго ради того, что боишься суда гражданскаго: политикъ еси, а не хрістіанинъ, человѣка боишься, а не Бога, и потому между невѣрными считаешися, хотя и Хрістово имя носишь; не дѣлаешь злая ради того, что Богъ запретилъ, и боишься Бога прогнѣвать: вѣры хрістіанской дѣло есть. Идешь на брань, чтобы корысть получить отъ непріятеля, обогатитися: не много и почти ничимъ же разнствуешь отъ разбойника, который для того на людей нападаетъ, чтобы обогатиться; идешь противу непріятеля, чтобы церковь и отечество свое защитить: похвальное намѣреніе есть. Смотришь на жену просто, безъ всякаго вожделѣнія, яко на созданіе Божіе: нѣтъ грѣха; смотришь съ нечистою похотію, \textit{прелюбодѣйствуешь въ сердцѣ своемъ}, како учитъ Господь\footnote{Матѳ.~5,~28.}. Одѣваешься въ платье, пристойное рангу твоему: безгрѣшно дѣлаешь; украшаешь себе одѣяніемъ, чтобы пышность показать и почтеніе отъ незнающихъ получить: міролюбецъ еси, и гордое сердце имѣешь. Просишь у ближняго твоего прощенія за оскорбленіе и обиду, боясь суда гражданскаго: мудрость человѣческая есть; смиряешься предъ братомъ твоимъ съ сожалѣніемъ, что ты оскорбилъ его: любви знаменіе есть. Не упиваешься, понеже не имѣешь чего пить: пьянствуешь въ сердцѣ твоемъ; не упиваешься, понеже грѣшно: воздержанія дѣло есть, истинный воздержникъ еси. Плачетъ кто, что потерялъ богатство, или лишился чести: печаль вѣка сего, и потому неполезна есть; плачетъ другій, что ближнему не можетъ отмстить: злобы дѣйствіе есть, и печаль пагубная; плачетъ третій, что Бога человѣколюбца прогнѣвалъ: \textit{печаль по Бозѣ есть}, и душеспасительная. Плачешь надъ мертвымъ или отцемъ, или братомъ, или другомъ, что съ любимымъ разлучился: печаль безполезная есть; плачешь надъ мертвымъ, помышляя, что грѣхъ насъ въ такъ бѣдное состояніе привелъ: плачь хрістіанскій есть. Наказуетъ командиръ подкоманднаго, что отъ него не почтенъ, или обезчещенъ: гнѣва и злобы исполненіе есть, а не наказаніе; наказуетъ, яко законопреступника, и чтобы впредь исправнѣе себе велъ: наказаніе правильное есть, и намѣреніе хрістіанское. Даешь просящему милостыню, чтобъ пользу отъ него какую получить: купля и торгъ, а не милостыня есть. Почитаешь царя, или отъ него посланныхъ, боясь наказанія за непочтеніе: человѣческая хитрость есть; почитаешь для того, что Богъ велѣлъ почитать: хрістіанское дѣло есть. Терпишь обиду, понеже не можешь отмстить: невольное терпѣніе; терпишь обиду добровольно, повинуяся заповѣди Хрістовой: истиннаго терпѣнія дѣло есть и спасительное. Такъ и въ прочемъ по состоянію сердечному судится всякое дѣло, или злое или доброе, о чемъ всякъ въ своей совѣсти извѣствуется, "--- якоже сіе и отъ святаго Писанія видимъ. Каинъ и Авель приносили жертву Богу; но \textit{призрѣ Богъ на Авеля и на дары его, на Каина же и на дары не внятъ}\footnote{Быт.~4,~4 и 5.}. Мытарь и фарисей молилися въ церкви Богу; но мытарь \textit{принятъ}, а фарисей \textit{отверженъ}\footnote{Лук.~18,~14.}. Радовался Закхей, увидѣвши Господа нашего Іисуса Хріста\footnote{Лук.~19,~3--9.}: радовался и Иродъ\footnote{23,~8.}, но Ироду въ \textit{пагубу}, а Закхею во \textit{спасеніе} обратилася радость. Тако по внутреннему состоянію сердечному судится предъ Богомъ всякое дѣло. И хотя дѣло внѣшнее кажется быть доброе, но отъ намѣренія недобраго и непотребнаго происходящее предъ Богомъ судится за непотребное. Намѣреніе бо есть, какъ основаніе, на которомъ дѣло назидается. И какое намѣреніе, такое и дѣло: ежели доброе намѣреніе, доброе дѣло; ежели злое намѣреніе, злое и дѣло есть совершаемое. Отсюду послѣдуетъ, что никого ни хвалить, ни хулить безразсудно не должно намъ. Понеже внутренняго человѣческаго состоянія и намѣренія никто, кромѣ единаго Бога, знать не можетъ. \textit{Кто бо вѣсть отъ человѣкъ, яже въ человѣцѣ, точію духъ человѣка живущій въ немъ}\footnote{1~Кор.~2,~11.}? Ибо часто бываетъ, что безумно хвалимъ того, кто предъ Богомъ окаяненъ и потому въ самой вещи мерзокъ есть; осуждаемъ того, котораго Богъ оправдаетъ, и оправдаемъ несмысленно того, кого Богъ осуждаетъ, и потому грѣшимъ.

\paragraph*{§\:36.} Понеже убо всякое дѣло человѣческое, какъ внутреннее, такъ и внѣшнее отъ сердца зависитъ, и каковое сердце, таково и дѣло "--- доброе или злое, и отъ природы имѣемъ сердце растлѣнное: того ради должно усердно и непрестанно молиться и просить у Сердцевѣдца Бога, Который изъ ничего все и изъ худаго доброе дѣлаетъ, просить сердца новаго и духа праваго, чтобы помышленія и дѣла, отъ сердца происходящія, правыя были и къ единой Божіей славѣ намѣреваемыя; должно, по примѣру Псаломника, всегда воздыхать: \textit{сердце чисто созижди во мнѣ, Боже, и духъ правъ обнови во утробѣ моей}\footnote{Пс.~50,~12.}. А когда сердце правое будетъ, то и дѣла правы будутъ.

\subsection[Глава 2-я. О языкѣ человѣческомъ.]{глава вторая.\\\bfseries О языкѣ человѣческомъ.}

\begin{quotation}\textit{Аще кто мнится вѣренъ быти въ васъ, и не обуздаваетъ языка своего, но льститъ сердце свое: сего суетна есть вѣра}\footnote{Іак.~1,~26.}.\end{quotation}


\paragraph*{§\:37.} Ничимъ такъ не грѣшитъ человѣкъ, какъ языкомъ, когда его не управляетъ по надлежащему. Языкъ клянетъ человѣки, бывшіе по подобію Божію; языкъ злословитъ отца и матерь; языкъ научаетъ убійству; языкъ совѣтуетъ и сговаривается о прелюбодѣяніи, нечистотѣ, татьбѣ, хищеніи, неправдѣ всякой; языкъ лжетъ, льститъ, обманываетъ; языкъ празднословитъ, буесловитъ, кощунствуетъ, сквернословитъ; языкъ виноватыхъ въ судѣ оправдаетъ и правыхъ обвиняетъ; языкъ въ купечествѣ худую вещь за добрую продаетъ; языкъ касается монарховъ, которые никакому земному суду не подлежатъ; языкъ терзаетъ и святыхъ мужей, которые незлобіемъ своимъ никому не вредятъ; отъ языка не избылъ Самъ Господь нашъ, Спаситель міра, Который \textit{грѣха не сотвори, ниже обрѣтеся лесть во устѣхъ Его}; языкъ отрыгаетъ хулы на великое, святое и страшное имя Божіе; словомъ, ничего языкъ не оставляетъ, но на все отрыгаетъ ядъ, который крыется въ сердцѣ человѣческомъ.

\paragraph*{§\:38.} Языкъ безчисленная злая въ свѣтѣ дѣлаетъ. 1)~Языкъ единому монарху на другаго клевещетъ; отъ чего вражды, ссоры, брани, кровопролитія происходятъ; столько тысящъ неповинныхъ людей падаетъ; столько вдовъ и сиротъ плачущихся остается; столько государствъ, градовъ, селъ запустѣваютъ; столько суммы денежной и всякаго иждивенія напрасно пропадаетъ. "--- 2)~Языкъ разсѣваетъ ереси, расколы, соблазны, и отъ того много церкви святой вреда, людямъ пагубы, церковнымъ правителямъ безпокойствія и труда дѣлается, какъ дѣло самое показуетъ. "--- 3)~Языкъ клеветою наполняетъ уши царей, князей и прочіихъ власть имѣющихъ, отъ чего много неповинныхъ погибаетъ. "--- 4)~Языкъ такожде отягчаетъ судебныя мѣста коварными и безсовѣстными клеветами, отъ чего присутствующимъ и служителямъ канцелярскимъ безполезныя безпокойствія и затрудненія. "--- 5)~Языкъ, касаяся злословіемъ властей, приводитъ ихъ въ подозрѣніе, отъ чего имъ бываетъ презрѣніе, непослушаніе отъ подвластныхъ, а въ нихъ самихъ безстрашіе, своевольство; въ обществѣ замѣшательство и всякое нестроеніе. "--- 6)~Языкъ немощныхъ и малодушныхъ приводитъ въ несносную печаль, уныніе и въ отчаяніе. "--- 7)~Языкъ въ единомъ градѣ и селѣ между сосѣдами, въ единомъ домѣ между женою и мужемъ, между братьями, сестрами, рабами, между другами любезными ссоры и драки дѣлаетъ. "--- 8)~Языкъ открываетъ секреты, которые по присяжной должности твердо хранить должно, отъ чего такожде многія затрудненія и бѣды происходятъ. "--- 9)~Языкъ часто и самаго клевещущаго приводитъ въ великое раскаяніе и печаль, что слово выпустилъ, котораго возвратить невозможно. Вкратцѣ сказать: сколько на свѣтѣ бѣдъ есть или было, всѣ языкъ или учинилъ, или умножилъ. О языкъ необузданный! малъ удъ, но велико зло, \textit{неудержимо зло, исполнь яда смертоносна}\footnote{Іак.~3,~8.}! Блаженъ и мудръ, обуздаваяй языкъ! блаженъ кто \textit{словесемъ своимъ творитъ вѣсъ и мѣру, и устамъ своимъ творитъ дверь и завору! Обуздаваяй бо языкъ, тихо, мирно поживетъ; и ненавидяй велерѣчія умалитъ порокъ}, глаголетъ Сирахъ\footnote{28,~29; 19,~6.}. Истинно Соломонъ написалъ: \textit{иже хранитъ своя уста, соблюдаетъ свою душу}\footnote{Притч.~13,~3.}; и паки: \textit{иже хранитъ своя уста и языкъ, соблюдаетъ отъ печали душу свою}\footnote{21,~23.}. Окаяненъ и безуменъ, кто не хранитъ своего языка, не знаетъ времене молчати и времене глаголати: \textit{яко мужъ языченъ не исправится на земли}\footnote{Пс.~139,~12.}; \textit{яко смерть и животъ въ руцѣ языка}\footnote{Притч.~18,~21.}; \textit{яко поползновеніе на земли лучше, нежели отъ языка}\footnote{Сир.~20,~18.}. \textit{Мнози бо падоша остріемъ меча, но не якоже падшіи языкомъ. Блаженъ, иже укрыется отъ него, иже не пройде въ ярости его, иже не повлече ига его, и узами его не связанъ бысть! Иго бо его, иго желѣзно; и узы его, узы мѣдяны; смерть люта, смерть его; и паче его лучше есть адъ}\footnote{Сир.~28,~21--23.}.

\paragraph*{§\:39.} Когда языкъ, по разуму не управляемый, толико грѣховъ и бѣдъ виновенъ бываетъ, "--- должно намъ тщаться его обуздовать. Но тщаніе безъ помощи Божіей мало что можетъ, \textit{ибо неудержимо зло}, глаголетъ апостолъ. И хотя двѣ ограды "--- зубы, то"=есть, и губы "--- имѣетъ, однакожъ прорывается. Сердце бо человѣческое, какъ сосудъ преисполненный, все, что не вмѣщается, вонъ извергаетъ, и такъ \textit{отъ избытка сердца уста глаголютъ}, по словеси Господню\footnote{Лук.~6,~45.}. Сего ради со Псаломникомъ должно смиренно молитися всемогущему Богу: \textit{положи, Господи, храненіе устомъ моимъ и дверь огражденія о устнахъ моихъ}\footnote{Пс.~140,~3.}, чтобы какъ сердце исправилъ, такъ и языкъ "--- орудіе сердца "--- направилъ, и научилъ благовременно и какъ должно говорить, а чего не должно говорить, о томъ бы и въ сердцѣ не помышлять.


\section[Статья 3-я. О грѣхѣ вообще и послѣдующихъ грѣху.]{статья третія.\\\bfseries О грѣхѣ вообще и послѣдующихъ грѣху.}
\subsection[Глава 1-я. Какъ великое зло есть грѣхъ.]{глава первая.\\\bfseries Какъ великое зло есть грѣхъ.}

\begin{quotation}\textit{Горе беззаконному! лукавая бо приключатся ему по дѣломъ рукъ его}.\quad Исаіи 3,~11.\end{quotation}


\paragraph*{§\:40.} Все тое есть грѣхъ, что ни дѣлается противу святаго и вѣчнаго Божія закона, якоже учитъ апостолъ: \textit{всякъ, творяй грѣхъ, и беззаконіе творитъ; и грѣхъ есть беззаконіе}\footnote{1~Іоан.~3,~5.}. Грѣхъ убо есть преступленіе и разрушеніе вѣчнаго и непремѣняемаго Божія закона, ослушаніе и противленіе святой Божіей волѣ.

\paragraph*{§\:41.} Грѣхъ бываетъ дѣломъ, словомъ, помышленіемъ, желаніемъ и намѣреніемъ. \textit{Дѣломъ}, какъ"=то: убійство, хищеніе и проч.; \textit{словомъ}, какъ"=то: хула, злословіе, клевета, сквернословіе и проч.; \textit{помышленіемъ}, какъ"=то: сквернословіе, услажденіе въ помыслѣ, отъ похоти блудной бываемое, тайная ненависть, злоба и проч.; \textit{желаніемъ и намѣреніемъ}, какъ"=то: похоть нечистоты, желаніе чуждаго добра, мщенія, и все, что ни хочетъ человѣкъ сдѣлать противу закона Божія, каковый грѣхъ есть противу десятой Божіей заповѣди: \textit{не пожелай}.

\paragraph*{§\:42.} Грѣхъ есть какъ тое, что закономъ Божіимъ запрещенное дѣлается, такъ и тое, что закономъ Божіимъ повелѣнное оставляется. Грѣшно убивать, но грѣшно и бѣдствующему, напр. утопающему въ водѣ и проч., руку помощи не подавать. Ибо Богъ въ законѣ Своемъ повелѣлъ какъ уклоняться отъ зла, такъ и творить благое: \textit{уклонися отъ зла, и сотвори благо}\footnote{Пс.~33,~15.}; и какъ сказалъ: \textit{не укради}, такъ повелѣлъ: \textit{просящему у тебе дай}\footnote{Матѳ.~5,~42.}. Откуду казнь отъ Бога опредѣлена не токмо за злыя дѣла, но и за оставленіе добрыхъ дѣлъ, якоже написано: \textit{всякое древо не творящее плода добра посѣкается и во огнь вметается}\footnote{Лук.~3,~9.}; и во огнь вѣчный отсылаетъ Господь не сотворшихъ дѣлъ милости, какъ читаемъ у Матѳея святаго въ главѣ 25"~й, гдѣ образъ суда Хрістова представляется. Сего ради какъ зло дѣлать, такъ и добра не дѣлать грѣхъ есть. Какъ бо твореніе зла, такъ и оставленіе добра противу закона Божія бываетъ, запрещающаго зло и повелѣвающаго добро.

\paragraph*{§\:43.} Дѣло, по виду доброе, въ порокъ и грѣхъ обращается, когда не на добрый, но на злый конецъ дѣлается. Тако милостыня порочится, когда ради тщеславія подается; грѣшитъ проповѣдникъ, когда слово Божіе ради похвалы своея проповѣдуетъ, и проч. Причина тому сія есть: понеже таковый отступаетъ отъ Бога сердцемъ своимъ, и на томъ мѣстѣ поставляетъ себе, какъ идола, на которомъ долженъ Бога имѣть и почитать; яко славу Его, которая Ему единому, яко всякаго добра Виновнику, подобаетъ, себѣ предвосхищаетъ. Таковый бо человѣкъ самолюбіе въ сердцѣ имѣетъ и вмѣсто Бога себе любитъ и почитаетъ. И дѣло его подобно яблоку, внѣ красному, но внутрь гнилому и смрадному, и самъ есть какъ гробъ повапленный, который внѣ красенъ, но внутрь смраденъ; сего ради есть какъ древо злое, которое добраго плода творити не можетъ; но какъ злое есть древо, такъ и злый плодъ творитъ\footnote{Матѳ.~7,~17 и 18.}. Ибо, чтобы дѣло истинно доброе было, должно быть и внѣ и внутрь доброе; тогда же доброе бываетъ, когда добрѣ дѣлается, то"=есть, когда отъ добраго сердца происходитъ, и къ доброму концу, то"=есть къ славѣ Божіей и пользѣ ближняго бываетъ.

\paragraph*{§\:44.} Грѣхъ есть великое зло, что слѣдующіе резоны доказываютъ. 1)~Всякимъ грѣхомъ величество Божіе оскорбляется. Человѣка проста оскорбить не мало, сановитаго болѣе, коль велико царя: а какъ тяжкое зло, оскорбить безконечнаго и неописаннаго Бога! Обида бо и оскорбленіе растетъ отъ лица оскорбляемаго; и чимъ большее и высшее есть лице оскорбляемо, тѣмъ большее оскорбленіе и большій грѣхъ. Но оскорбленіе всякаго человѣка, не токмо простаго, но и сановитаго и самаго царя, какъ ничто въ сравненіи съ оскорбленіемъ Бога: яко всякій человѣкъ, и самый царь, есть какъ ничто предъ величествомъ Божіимъ, предъ Которымъ весь свѣтъ и \textit{вси языцы, какъ капля едина} вмѣняются\footnote{Ис.~40,~15.}. Сего ради всякій грѣхъ есть весьма тяжкое зло, яко тѣмъ безконечное Божіе величество оскорбляется; которое зло чрезъ всю вѣчность огнемъ геенскимъ очищаться будетъ, когда здѣ сокрушеніемъ сердца, покаяніемъ и кровію Сына Божія не очистится. "--- 2)~Грѣхомъ законъ Божій, вѣчный и непремѣняемый, разоряется. Ибо Богъ далъ законъ Свой намъ того ради, дабы отъ насъ цѣло соблюдаемъ былъ. А когда грѣшитъ человѣкъ и законъ *Божій* преступаетъ, то онъ нарушаетъ тое Божіе установленіе и узаконеніе, которое вѣчно нерушимо и цѣло должно быть, и тако человѣкъ разоряетъ тое, что Богъ установилъ; и грѣшникъ окаянный премѣняетъ и разрушаетъ тое, что само въ себѣ непремѣняемо и нерушимо быть должно. Въ законѣ бо Божіи тое изображено, что такъ, а не иначе, какъ изображено, должно быть; напр. Бога почитать и любить паче всего, и ближняго любить какъ себе. Сего вѣчная и непремѣняемая правда отъ человѣка требуетъ. Но грѣшникъ ослѣпленный, не смотря на тое, дерзаетъ разорять неразоряемое, и нарушать ненарушаемое, къ безчестію вѣчнаго Бога и своей погибели. "--- 3)~Человѣкъ, когда грѣшить, болѣе почитаетъ себе, нежели Бога, болѣе слушаетъ похоть свою, нежели Бога, предпочитаетъ волю свою волю Божіей и закону Его святому, и дѣлается какъ бы самовластный и неподчиненный власти Божіей, и такъ съ діаволомъ упорно стоитъ противу Бога. Не можетъ бо сіе иначе быть, когда человѣкъ отъ произволенія, предразсужденія и противу совѣсти грѣшитъ. Діавольское бо дѣло есть противиться Богу и непокоряться Ему. Откуду о такихъ грѣшникахъ пишется: \textit{всякъ творяй грѣхъ, отъ діавола есть}\footnote{1~Іоан.~3,~8.}, и \textit{чадами діавольскими} называются\footnote{ст.~10 и Іоан.~8,~44.}. Коль же сіе тяжко и страшно съ діаволомъ противиться Богу и, Создателю своему, всякому видно. "--- 4)~Грѣхъ такъ великое и ужасное зло, что его никто не моглъ разрушить, кромѣ единороднаго Сына Божія. Надобно было пріити всемогущему Сыну Божію и Своимъ страданіемъ и смертію отнять его. \textit{Сего ради явися Сынъ Божій, да разрушитъ дѣла діаволя}\footnote{1~Іоан.~3,~8.}. Божій бо законъ есть непремѣняемый, какъ выше сказано: сего ради должно было человѣку или всего его исполнить, или преступившему и согрѣшившему \textit{вѣчную} претерпѣвать казнь; ибо сего правда Божія отъ человѣка требуетъ. Но какъ вси словомъ Божіимъ означились быть законопреступниками, то вси за то и подлежали вѣчному осужденію; и никто отъ того своею силою никакъ свободиться не моглъ. Того ради Сынъ Божій пришелъ, и за грѣхъ удовлетворилъ страданіемъ и смертію Своею и Своею жертвою на древѣ крестнѣмъ очистилъ его; слѣдственно и казнь отнялъ, послѣдующую грѣху, вѣрующихъ въ Него. Грѣхъ бо безъ казни не бываетъ; а когда грѣхъ отнимается, то и казнь не послѣдуетъ: казнь бо не бываетъ, какъ только за грѣхъ. Сего ради когда вѣрные молятся и просятъ отпущенія грѣховъ, то во имя Хрістово того просятъ, Который единъ отпущаетъ и отнимаетъ грѣхъ. Безъ Него бо отпущеніе и отъятіе грѣха не бываетъ. Онъ бо есть \textit{Агнецъ Божій, вземляй грѣхи міра}\footnote{Іоан.~1,~20.}. Не помышляй убо, человѣче, аки бы мало нѣчто былъ грѣхъ. Когда Агнецъ Божій не возметъ грѣха отъ тебе, то чрезъ всю безконечную вѣчность будешь за него платить въ гееннѣ правдѣ Божіей; вземлетъ же только отъ тѣхъ, которые престаютъ грѣшить, каются и вѣруютъ въ Него. А когда за мало почитаешь грѣхъ, то и не избѣжишь горести его; ибо знаменіе есть, что не имѣешь покаянія истиннаго. Истинно бо кающійся познаетъ тяжесть и горесть его, и того ради бережется его, какъ смертоноснаго яда. "--- 5)~Страшно язычникамъ, незнающимъ Бога и святаго Его закона, грѣшить: но страшнѣе хрістіанамъ, Бога исповѣдующимъ и свѣтомъ слова Его просвѣщаемымъ. Ибо язычники единою совѣстію, хрістіане совѣстію и закономъ Божіимъ писаннымъ обличаются за грѣхъ. Язычники какъ Бога не знаютъ, такъ и воли Его, въ законѣ объявленныя, не видятъ: хрістіане по вся дни слышатъ волю Божію проповѣдуемую, и преступникамъ закона Божія казнь слѣдующую, и тако, вѣдуще волю Божію, не исполняютъ ея. Язычники какъ не знаютъ Бога, такъ и обѣщаній Ему никакихъ не чинятъ: хрістіане, вступая въ хрістіанство, обѣщаются Богу работать, волю Его творить, отрицаются сатаны и дѣлъ его; но когда на грѣхи обращаются, все тое забываютъ и попираютъ, и Богу измѣняютъ, и дѣлаются лживыми. И тако случися имъ истинная притча; \textit{песъ возвращся на свою блевотину, и свинія, омывшися, въ калъ тинный}\footnote{2~Петр.~2,~22.}. Грѣхъ убо хрістіанамъ не иное что, какъ богоотступство, измѣна, которою не человѣку, но Богу измѣняютъ. Разсуждай убо, хрістіанине, что есть грѣхъ, которымъ услаждаешься, хотя исповѣдуешь имя Божіе, но дѣломъ отмещешися Его, когда заповѣдь Его преступаешь. Сколько бо разъ согрѣшаешь, напр. убиваешь, блудодѣйствуешь, крадешь, похищаешь, злословишь, хулишь, клевещешь, и проч., столько разъ совѣта діавольскаго слушаешь и заповѣдь Божію и повелѣніе Его отмещешь; сколько разъ врага діавола слушаешь и не слушаешь Бога, столько разъ отъ Него отступаешь и приступаешь къ діаволу. Истинно сіе такъ есть, хотя того и не примѣчаешь. "--- 6)~Грѣхъ разлучаетъ человѣка отъ Бога, якоже глаголетъ пророкъ: \textit{грѣси ваши разлучаютъ между вами и между Богомъ}\footnote{Ис.~59,~2.}. И чимъ болѣе грѣшитъ человѣкъ, тѣмъ болѣе удаляется отъ Бога; чимъ же болѣе удаляется, тѣмъ болѣе помрачается. \textit{Богъ бо свѣтъ есть}\footnote{1~Іоан.~1,~5.}. Сего ради чимъ кто далѣе отходитъ отъ Него, тѣмъ въ большую тьму вдается и болѣе умомъ слѣпотствуетъ. Откуду бываетъ, что таковый часто за грѣхъ того не вмѣняетъ, въ чемъ великий грѣхъ, и съ того радуется, за что плакать должно. Напр. отмстить ближнему, зло за зло воздать, оскорбить, обмануть, прельстить и словомъ, какъ стрѣлою, уязвить человѣка многіи за утѣшеніе себѣ поставляютъ, что есть превеликое безуміе и крайнее ума помраченіе. Сего ради таковый во всѣхъ своихъ поступкахъ, дѣлахъ и начинаніяхъ ходитъ какъ слѣпый или какъ во тьмѣ, и рва погибели, въ который имѣетъ пасти, не видитъ. О таковыхъ глаголется: \textit{не познаша, ниже уразумѣша, во тьмѣ ходятъ}\footnote{Пс.~81,~5.}; и паки: \textit{осяжутъ яко слѣпіи стѣну, яко суще безъ очесъ осязати будутъ, и падутся въ полудни, яко въ полунощи}\footnote{Ис.~59,~10.}. Къ сему бѣдствію грѣхъ приводитъ. "--- 7)~Глаголетъ Господь: \textit{всякъ, творяй грѣхъ, рабъ есть грѣха}\footnote{Іоан.~8,~34.}. Творяй блудъ, рабъ есть блудныя похоти; упивающійся и обжирающійся рабъ есть чрева: \textit{сему чрево богъ есть}\footnote{Филип.~3,~19.}. Любяй сребро и злато рабъ есть мамоны, \textit{мамонѣ работаетъ}\footnote{Матѳ.~6,~24.}. Тоежъ и о прочіихъ страстехъ разумѣй. \textit{Имже бо кто побѣжденъ бываетъ, сему и работенъ есть}\footnote{2~Петр.~2,~19.}. Коль же тяжкая и мерзкая сія работа есть! Тяжчайшая и подлѣйшая есть паче, нежели работать человѣку; лучше бо человѣку всякому, и самому мучителю, яко созданію Божію, работать, нежели грѣху и грѣхомъ діаволу. О, когда бы сію работу бѣдный грѣшникъ увидѣлъ! Призналъ бы себе бѣднѣйшимъ паче плѣнниковъ, каторжныхъ, заключенныхъ въ темницѣ, окованныхъ и прочіихъ бѣдныхъ и тѣлесно страждущихъ людей. Но тѣмъ паче бѣднѣйшій, что сей бѣдности своей не познаетъ. Страсть бо ослѣпляетъ разумное око. "--- 8)~Апостолъ глаголетъ: \textit{всякъ, творяй грѣхъ, отъ діавола есть, яко исперва діаволъ согрѣшаетъ}; и ниже грѣшниковъ \textit{чадами діаволими} называетъ\footnote{1~Іоан.~3,~8 и 10.}. И Господь къ злобнымъ Іудеемъ сказалъ: \textit{вы отца вашего діавола есте, и похоти отца вашего хощете творити}\footnote{8,~44.}. Весьма бѣдственно и страшно есть быть сыномъ діаволовымъ. Но грѣхъ, злое и діавольское сѣмя, къ сему страшному бѣдствію приводитъ человѣка. Ибо грѣшникъ, творящій грѣхъ и не хотящій каятися, сего князя тьмы, какъ сынъ отца своего, нравами въ себѣ изображаетъ, и самымъ дѣломъ показуетъ, что онъ отъ того сквернаго отца есть: яко злаго его сѣмене злые плоды творитъ, то"=есть, грѣхи. Отъ плода бо сѣмя познается, и каково сѣмя, таковъ и плодъ его. Діаволъ противится и не покоряется Богу, и грѣшникъ нераскаянный въ такомъ же непокореніи пребываетъ. Внимай сему всякъ и разсуждай, чій еси сынъ, хотя и имя Хрістово на себѣ носиши. Истинно и вѣрно слово апостольское: \textit{всякъ творяй грѣхъ, отъ діавола есть; и всяко древо отъ плода своего познается}, глаголетъ Господь\footnote{Лук.~6,~44.}. "--- 9)~Грѣхъ, понеже отъ Бога разлучаетъ, какъ сказано, у Котораго единаго животъ есть, или паче Онъ Самъ есть Источникъ жизни, отлучившуюся душу лишаетъ жизни и такъ умерщвляетъ ее. Таковый человѣкъ живъ, и мертвъ есть, живъ тѣломъ, но мертвъ душею. Такъ прародители наши въ раи, въ который день вкусили отъ заповѣданнаго древа и согрѣшили, умерли, по слову Господню: \textit{въ оньже аще день снѣсте отъ него, смертію умрете}\footnote{Быт.~2,~17.}. Тако о блудномъ сынѣ, который отлучился"=было отъ отца своего, но паки покаяніемъ обратился, сказано: \textit{сынъ мой сей мертвъ бѣ, и оживе, и изгиблъ бѣ, и обрѣтеся}\footnote{Лук.~15,~24 и 32.}. Не о тѣлесной смерти здѣ глаголется, потому что онъ и въ отлученіи отъ отца своего тѣломъ живъ былъ, какъ обстоятельства притчи показуютъ, но о душевной, отъ которой ожилъ, когда отъ заблужденія къ отцу чрезъ покаяніе возвратился. Притчею бо сею изображается грѣшникъ, отлучившійся отъ Бога, яко Отца своего, но чрезъ истинное покаяніе паки къ Нему обращающійся, который дотолѣ мертвъ душею, доколѣ отлучается отъ Бога. Надобно бо неотмѣнно мертвымъ быть отлучившемуся отъ живота. Якоже бо удалившемуся отъ свѣта слѣдуетъ во тьмѣ обращаться, тако удалившемуся отъ живота слѣдуетъ въ сѣни смертнѣй быть. Ибо гдѣ нѣтъ свѣта, тамо неотмѣнно есть тьма; и гдѣ нѣтъ живота, тамо неотмѣнно есть смерть. Богъ есть свѣтъ животный и животворящій: слѣдовательно во тьмѣ и смерти есть, кто отъ Него отлучился. Къ таковому мертвецу глаголетъ слово Божіе: \textit{воскресни отъ мертвыхъ, и освѣтитъ тя Хрістосъ}\footnote{Еф.~5,~14.}. "--- 10)~Грѣхомъ раздраженная совѣсть мучитъ человѣка, которое мученіе тяжко есть такъ, что часто и въ отчаяніе приводитъ, какъ"=то случилося Каину братоубійцѣ\footnote{Быт.~4.} и Іудѣ предателю Господню\footnote{Матѳ.~27,~5.}. Доколѣ грѣхъ дѣлается, не познается его зло; но когда по грѣхопаденіи пробудится совѣсть, тогда познаетъ грѣшникъ, коль горекъ есть грѣхъ. Якоже пьяный, пока пьетъ и пьянъ бываетъ, не чувствуетъ вреда своего; а когда проспится, тогда почувствуетъ, коль вредно есть пьянство: тако грѣшникъ, пока въ грѣховномъ пьянствѣ пребываетъ, не чувствуетъ грѣховнаго вреда; но когда отъ того пьянства пробудится, тогда познаетъ, коль великое, тяжкое и мучительное зло есть грѣхъ. Тогда бо пробудится совѣсть и паче всякаго мучителя терзаетъ его, и, какъ червь древо, душу его внутрь грызетъ и снѣдаетъ; тогда печаль, страхъ и ужасъ гнѣва Божія и суда, геенны и вѣчнаго мученія возстаетъ въ немъ; тогда помыслы, какъ волны, убогую душу ударяютъ: \textit{нѣсть спасенія тебѣ въ Бозѣ твоемъ}\footnote{Пс.~3,~3.}. Гдѣ бы ни былъ грѣшникъ, вездѣ мучитель сей неотлучно съ нимъ есть, вездѣ мучитъ и снѣдаетъ его; днемъ безпокойствуетъ, ночью безпокойствуетъ и отнимаетъ сонъ; мало заснетъ, и тогда устрашаетъ его; пробудится отъ сна, пробудится и злая совѣсть его. Съ сею бѣдою день начинаетъ и провождаетъ, ночь начинаетъ и провождаетъ, и тако никогда отъ того бѣдствія не можетъ свободиться, понеже внутрь себе всегда имѣетъ зло тое. О зло, мучительное зло "--- совѣсть злая! О тяжкое зло "--- грѣхъ, который такъ безпокойствуетъ совѣсть нашу! Лучше уязвленное тѣло имѣть, нежели грѣхами уязвленную совѣсть; лучше біеніе тѣла терпѣть, нежели біеніе совѣсти; лучше всякія внѣшнія бѣды принимать, нежели сію едину внутрь бѣду имѣть. Аще въ семъ вѣкѣ, гдѣ не всякій грѣхъ въ совѣсти усматриваемъ, "--- \textit{грѣхопаденія бо кто разумѣетъ}\footnote{Пс.~18,~13.}? "--- такъ тяжко мучитъ совѣсть за грѣхъ: какое, коль тяжкое и несносное ея мученіе будетъ въ будущемъ, гдѣ всѣ грѣхи въ ней открыются и представятся, и бѣдная душа весь гнѣвъ Божій на себѣ почувствуетъ, и во всеконечномъ милости Божіей отчаяніи во вѣки безконечные пребудетъ! Но кто по содѣяніи грѣха не чувствуетъ обличенія и мученія совѣсти, тотъ послѣдней надежды спасенія лишился, и знакъ, что никакого грѣха въ грѣхъ не ставитъ: только кажется, что сему быть невозможно. Ибо свидѣтель сей вѣренъ есть: какъ его ни утѣшай и ни умягчай "--- доносить, обличать и вопить не престаетъ; всегда ищетъ суда на грѣшника. Какъ бо законъ Божій написанный, такъ и естественный сей законъ, хотя въ грѣшникѣ и потемненный, всегда обличаетъ грѣхъ и судъ Божій грѣшнику означаетъ. Законъ бо Божій и совѣсть во едино сходятся, и что законъ Божій на хартіи глаголетъ, тое совѣсть внутрь. Законъ Божій глаголетъ: \textit{не укради}, тоежде и совѣсть твоя тебѣ глаголетъ. Откуду бываетъ, что и самые великіе беззаконники ищутъ сокровеннаго мѣста къ творенію беззаконныхъ дѣлъ. Никто бо не хощетъ явно грѣшити, и содѣянный грѣхъ всячески утаеваетъ. Сіе же не отъинуды, какъ отъ совѣсти происходитъ, которая, обличаетъ грѣшника, что онъ худо дѣлаетъ, и стыдъ въ немъ содѣлываетъ. "--- 11)~За грѣхъ геенна, адъ и вѣчное мученіе уготовано. Коль же великая и ужасная бѣда есть, вопервыхъ, лишиться вѣчно Божіей милости, и Его сладчайшаго лицезрѣнія, котораго вси святіи ангели и святіи человѣки во вѣки безъ сытости насыщатися будутъ, отчужденну и отриновенну быть отъ блаженнаго онаго собора и попасться въ мучительнѣйшее вѣчное состояніе, съ діаволомъ и злыми его аггелами безъ конца гнѣва Божія чашу пить, страдать въ пламени ономъ снѣдающемъ, но не умерщвляющемъ\footnote{Матѳ.~13,~42 и 25,~41; Марк.~9,~43--45 и проч.}, гдѣ едина капля воды пожелается, но никогда не получится, и услышится отвѣтъ: \textit{чадо! помяни, яко воспріялъ еси благая твоя въ животѣ твоемъ}\footnote{Лук.~16,~25.}. Коль, глаголю, тяжко въ такомъ бѣдствіи воздыхать, стенать, плакать, рыдать, снѣдаться, зубами скрежетать безъ конца, безпрестанно и безполезно! Въ сіе ужасное зло грѣхъ приводитъ\footnote{Апок.~21,~8.}. "--- 12)~Грѣхъ и временныя казни наводитъ, какъ"=то: кровавыя войны, моровыя язвы, болѣзни, скорби, глады, трусы, пожары и проч. Аще бы грѣха не было, ничего бы того не было. Грѣхъ причина всѣхъ золъ и бѣдствій есть. "--- 13)~Грѣхъ причиною есть, что и самое неповинное естество страждетъ. Человѣкъ грѣшитъ, но прочая тварь страждетъ: земля не даетъ плода, скоты и звѣри отъ голода пропадаютъ, воздухъ и вода растлѣвается, и проч. Всему тому грѣхъ человѣческій причиною есть\footnote{Второз.~28,~15 и проч.}. Видишь убо, человѣче, коликое зло есть грѣхъ. О грѣхъ "--- неизреченное зло, яко тѣмъ безконечный и преблагій Богъ оскорбляется и прогнѣвляется! Безстыдная измѣна "--- грѣхъ, которою Богу безсмертному и праведному измѣняемъ! Язва неисцѣльная "--- грѣхъ, которая совѣсть нашу уязвляетъ, мучитъ и снѣдаетъ! Проказа душевная "--- грѣхъ, которыя никто, кромѣ единороднаго Сына Божія, исцѣлити не можетъ! Временныхъ и вѣчныхъ бѣдствій источникъ, корень осужденія и смерти, превращеніе естества, помраченіе разума, растлѣніе душевныя доброты и всѣхъ золъ злѣйшее "--- грѣхъ! Праведно святый Златоустъ написалъ: нѣтъ ничего тягчайшаго отъ грѣха\footnote{На Пс.~121"~й.}; нѣтъ ничего сквернѣйшаго и нечистѣйшаго отъ грѣха\footnote{Бес.~25"~я на Іоан.}. Воистину злѣйшій есть грѣхъ паче демона, какъ тойжде учитъ отецъ\footnote{Бес.~41"~я на Дѣян.}. Ибо и демона грѣхъ сдѣлалъ демономъ, который отъ Создателя созданъ былъ ангеломъ свѣтлымъ. Воистину лучше нагому ходить, нежели обремененному грѣхами, тойжде отецъ бесѣдуетъ\footnote{Бес.~5"~я на Іоан.}. О грѣхолюбивая душа! смотри и разсуждай, какъ великое, ужасное и неисцѣльное зло любишь ты! какъ мерзкимъ чудовищемъ услаждаешися?! Очувствуйся и осмотрись, къ какому концу скверная любовь сія ведетъ тебе, и, отвратившися отъ страшилища сего, обратися къ началу и источнику всѣхъ благъ "--- Богу. \textit{Егда возвратився воздохнеши, тогда спасешися, и уразумѣеши, гдѣ еси была}\footnote{Ис.~30,~15.}.

\paragraph*{§\:45.} Помощь въ подвигѣ противу грѣха: 1)~Тщаться имѣть доброе въ младыхъ лѣтахъ воспитаніе, которое бываетъ \textit{въ наказаніи и ученіи Господни}\footnote{Еф.~6,~4.}. Отъ воспитанія бо все прочее житіе зависитъ, какъ бо свѣжій глиняный сосудъ, чимъ съ начала напоенъ будетъ, такій запахъ и чрезъ прочее время житія своего издавать будетъ. Всякъ раждается золъ; яко всякъ \textit{въ беззаконіяхъ зачинается и во грѣсѣхъ раждается}\footnote{Пс.~50,~7.}. Сего ради, чтобы добрымъ сдѣлаться, требуетъ исправленія, обученія: якоже конь свирѣпый и необученый обучается, чтобы удобенъ и угоденъ былъ къ ѣздѣ. Страсти, хотя и имѣются въ молодомъ сердцѣ, однакожъ еще не усилились, и потому, когда съ начала наказаніемъ и страхомъ обуздаются, усмирятся и укротятся: надобно ожидать доброй надежды въ юношѣ, тако воспитанномъ. Молодое деревцо, къ которой сторонѣ приклонено будетъ, такъ и до конца будетъ стоять: такъ и юное сердце, чему пріобучено будетъ, того и до смерти будетъ держаться. "--- 2)~Отъ злыхъ и развращенныхъ людей удаляться. Ибо хотя бы и добрѣ воспитанъ былъ человѣкъ, и благочестиво жилъ, но когда со злыми обходиться будетъ, развратиться можетъ, якоже прикасающійся сажѣ очерняется: \textit{тлятъ бо обычаи благи бесѣды злы}\footnote{1~Кор.~15,~33.}. Сего ради, какъ Лотъ отъ Содома, тако доброму отъ сожитія со злыми должно убѣгать, да не, беззаконнымъ ихъ житіемъ развратившися, погибнетъ. "--- 3)~Во святомъ Божіемъ словѣ поучаться, и прочія книги душеполезныя читать или слушать. Ибо Божіе слово и прочія книги, ему согласныя, обличаютъ грѣхъ и научаютъ добродѣтели, и такъ отводятъ насъ отъ грѣха, показуя мерзость его и отъ него послѣдующую погибель: чимъ человѣкъ можетъ отвратиться отъ грѣха и творить покаяніе и плоды его, то"=есть, добрыя дѣла. "--- 4)~Не гнушаться, но паче любить обличеніе и наставленіе добрыхъ и разумныхъ людей, которые сами познаютъ грѣхъ и гнушаются имъ, и другимъ совѣтъ добръ могутъ подать. "--- 5)~Часто на память приводить тое, что при крещеніи отрицалися сатаны и всѣхъ дѣлъ его, то"=есть, грѣховъ, и обѣщались служить Богу преподобіемъ и правдою, якоже въ формѣ крещенія написано. Прилично ли хрістіанину къ сатанѣ и злымъ его дѣламъ обращаться, которыхъ отреклся и поплевалъ, и отвращаться отъ Создателя и Искупителя своего и измѣнять Ему, Которому, яко Царю своему и Государю, обѣщался вѣрою и правдою служить? Воистину таковому приличествуетъ притча: \textit{песъ возвращься на свою блевотину, и свинія, омывшися, въ калъ тинный}\footnote{2~Петр.~2,~22.}. Ибо хрістіанинъ омывается при крещеніи отъ всѣхъ грѣховныхъ сквернъ, по реченному: \textit{омыстеся, освятистеся, оправдистеся именемъ Господа нашего Іисуса Хріста и Духомъ Бога Нашего}\footnote{1~Кор.~6,~11.}. Но когда хрістіанинъ ко грѣхамъ обращается и творитъ ихъ, какъ песъ возвращается на свою блевотину и свинія омытая въ калъ тинный. Всякій бо грѣхъ есть какъ мерзкая блевотина и какъ смрадный калъ, которымъ человѣкъ, благодатію Хрістовою въ крещеніи омовенный, освященный и оправданный, паки оскверняется, когда творитъ его и измѣняетъ Богу и Создателю своему, и такъ дѣлается Ему невѣрнымъ, Которому обѣщался вѣрно служить. О коль тяжка сія и бѣдственна измѣна! \textit{Тяжка}: ибо Богу, а не человѣку измѣняетъ грѣшникъ. Тяжко Государю своему, царю земному измѣнить, кольми паче Богу "--- Царю небесному? \textit{Бѣдственна}: ибо предается грѣшникъ паки во власть діавольскую, отъ которой благодатію Божіею избавился"=было, и тако паки дѣлается чадомъ гнѣва Божія, тьмы и вѣчнаго осужденія, который сотворился было сыномъ свѣта, благословенія Божія, и наслѣдникомъ вѣчнаго живота. Коль многихъ слезъ и плача потребно грѣшнику сію измѣну оплакать и омыть оскверненную свою душу! Помни убо, хрістіанине, баню крещенія, въ которой омылся отъ сквернъ грѣховныхъ, и не возвращайся въ тыя; и отрицанія и обѣщанія не забывай, и берегись нарушать тое, да не измѣниши Создателю твоему. Аще же измѣнилъ, обратися паки со слезами и плачемъ къ Отцу твоему, якоже блудный сынъ, и помилуетъ тя; и впредь не отступай отъ Него, да не во вѣки лишишься милости Его "--- 6)~Памятовать \textit{послѣдняя} четыре: смерть, судъ Хрістовъ, адъ и царство небесное. Сихъ всегдашняя память и вѣрное размышленіе отвращаетъ отъ грѣха. Смерть нечаянно приходитъ и восхищаетъ всякаго, праведника и грѣшника, и посылаетъ на оный вѣкъ. По смерти судъ Хрістовъ слѣдуетъ праведный, на которомъ словъ, дѣлъ и помышленій беззаконныхъ испытаніе будетъ, гдѣ слѣдуетъ или прославиться, или постыдиться. По судѣ двѣ дороги откроются: едина во адъ, которою поженутся грѣшники нераскаянные; другая въ царство небесное, и пойдутъ тою праведніи и святіи. \textit{Собранніи Господемъ обратятся, и пріидутъ въ Сіонъ съ радостію, и радость вѣчная надъ главою ихъ}\footnote{Ис.~35,~10.}. \textit{И идутъ тіи въ муку вѣчную, праведницы же въ животъ вѣчный}\footnote{Матѳ.~25,~46.}. Сихъ размышленіе есть истинная хрістіанская философія, которая научаетъ не натуру вещей испытовать, но суету міра, краткость времени и долготу вѣчности познавать, и сердце обращать отъ видимыхъ къ невидимымъ и отъ временныхъ къ вѣчнымъ. \textit{Помни} убо \textit{послѣдняя твоя, и во вѣки не согрѣшиши}\footnote{Сир.~7,~39.}. "--- 7)~Разсуждать, что краткая грѣха есть сладость, но вѣчная горесть послѣдуетъ. Грѣхъ, какъ совершается, нѣсколько услаждаетъ, но сотворенный горесть вводитъ, совѣсть уязвляетъ, душу оскверняетъ и къ вѣчной мукѣ ведетъ. "--- 8)~Подлость и мерзость грѣха да отвратитъ хрістіанина отъ грѣха: яко онъ есть сѣмя и плодъ діавольскій, \textit{яко исперва діаволъ согрѣшаетъ}\footnote{1~Іоан.~3,~8.}. "--- 9)~О хрістіанскомъ благородіи разсуждать, которое есть великое, высокое, превознесенное и непостижимое. \textit{Быть сыномъ Божіимъ}, имѣть \textit{общеніе со Отцемъ и съ Сыномъ Его Іисусъ Хрістомъ}\footnote{1,~3 и 3,~1.}, быть \textit{храмомъ Духа Святаго и удомъ Хрістовымъ}\footnote{1~Кор.~6,~15 и 19.}: колико есть! какая честь, слава и благородіе человѣку отъ сего большее можетъ быть! Вся слава и честь царей и князей уступаетъ тому, какъ тьма свѣту. Но того лишается хрістіанинъ чрезъ грѣхъ. \textit{Богъ свѣтъ есть, и тьмы въ Немъ нѣсть ни единыя. Аще речемъ, яко общеніе имамы съ Нимъ, и во тьмѣ ходимъ, лжемъ и не творимъ истины}\footnote{1~Іоан.~1,~5 и 6.}. "--- 10)~Смерть Хрістова и ужасное Его за грѣхи наши страданіе сильно есть отвратить хрістіанина отъ грѣха. Гордости нашей ради такъ глубоко смирился Онъ; сквернословій, злословій, хуленій, клеветъ нашихъ ради святѣйшія Его уши претерпѣли хулу и поношеніе; за хищенія наша руцѣ Его ко кресту пригвождены были; за піянство и сладострастіе наше оцта съ желчію смѣшена напоенъ былъ; за скверны и нечистоты наши такъ ужасно мученъ былъ, и всѣ наши грѣхи и беззаконія на древѣ крестномъ страданіемъ Своимъ очищалъ. \textit{Той бо язвенъ бысть за грѣхи наша, и мученъ бысть за беззаконія наша}\footnote{Ис.~53,~5.}. Тебѣ ли убо, хрістіанине, дѣлать тое, за что Сынъ Божій, Избавитель твой, толикое безчестіе и мученіе пострадалъ? Что Хрісту горесть и мученіе содѣлало, тѣмъ ли тебѣ услаждаться? да не будетъ! "--- 11)~Примѣры святыхъ Божіихъ, противу грѣха подвизавшихся, прочитывать и въ томъ имъ подражать. "--- 12)~Злые помыслы возстающіе, и ко грѣху влекущіе, тотчасъ отсѣкать "--- не иначе, какъ искру огненную отъ руки отрясаемъ, да не, усилившеся, какъ разбойники вшедше въ домъ, опустошатъ домъ нашъ душевный и погубятъ насъ. Въ томъ бо и подвигъ хрістіанскій состоитъ. Аще бо помыслы отсѣкать будемъ, отсѣчемъ и грѣхъ. Отъ помысловъ бо и грѣхъ бываетъ, какъ отъ корня дерево, или какъ отъ сѣмени плодъ. Отсѣки корень, и древа не будетъ; подави сѣмя, и плодъ не возрастетъ. Труденъ сей подвигъ, но хрістіанамъ нуженъ. Надобно бо всякому распинать плоть со страстьми и похотьми, кто хощетъ Хрістовымъ быть. Сіи бо только Хрістовы суть, которые \textit{распинаютъ плоть}\footnote{Гал.~5,~24.}. "--- 13)~Тщаться всегда имѣть предъ собою Бога, якоже глаголетъ пророкъ: \textit{предзрѣхъ Господа предо мною выну}\footnote{Пс.~15,~8.}, и на Того душевными очами смотрѣть, яко вездѣсущаго и всевѣдущаго, предъ Которымъ ничто не утаено; яко праведнаго, Который всѣмъ воздаетъ по дѣломъ; яко благаго и благоутробнаго Отца, Который грѣхомъ оскорбляется; яко великаго и святаго, предъ Которымъ грѣшить, безчинствовать и благоговѣинства не имѣть безстыдно и страшно. Писаніе святое свидѣтельствуетъ, что Богъ нашъ вездѣ есть, и нѣсть такого мѣста, гдѣ бы Его не было, и всякое дѣло наше и помышленіе видитъ, и слово слышитъ. Все, что ни дѣлается, предъ очами Его дѣлается. \textit{Очи Его на нищаго призираютъ, вѣжди Его испытаютъ сыны человѣческія. Господь испытаетъ праведнаго и нечестиваго}\footnote{Пс.~10,~4 и 5.}. \textit{Съ небесе призрѣ Господь, видѣ вся сыны человѣческія: отъ готоваго жилища своего призрѣ на вся живущія на земли, создавый наединѣ сердца ихъ, разумѣваяй на вся дѣла ихъ}\footnote{Пс.~32,~13--15.}: \textit{Господи, искусилъ мя еси, и позналъ мя еси; Ты позналъ еси сѣданіе мое и востаніе мое; Ты разумѣлъ еси помышленія моя издалеча: стезю мою и уже мое Ты еси изслѣдовалъ, и вся пути моя провидѣлъ еси}, и проч. И ниже: \textit{камо пойду отъ Духа Твоего? и отъ лица Твоего камо бѣжу? Аще взыду на небо, Ты тамо еси; аще сниду во адъ, тамо еси} и проч. И ниже: \textit{тьма не помрачится отъ Тебе, и нощь яко день просвѣтится: яко тьма ея, тако и свѣтъ ея}, и проч. И ниже: \textit{не содѣланное мое видѣстѣ очи Твои, и въ книзѣ Твоей вся напишутся}\footnote{138,~1--3; 7,~8,~12 и 16.}. И Іеремія пророкъ глаголетъ: \textit{очи Твои, Боже, отверсты на вся пути сыновъ человѣческихъ: дати комуждо по пути его и по плоду начинаній его}\footnote{Іер.~83,~19.}. Видишь, хрістіанине, како Богъ на всѣхъ насъ смотритъ, и дѣла наша и помышленія испытуетъ; и ничто предъ Нимъ не утаено, что ни дѣлается во дни и въ нощи, яко предъ Нимъ какъ день, такъ и нощь равно имѣется, но и тое, что имѣемъ дѣлать, помышлять и говорить, знаетъ, и всякое дѣло, слово и помышленіе наше въ книзѣ Своей записываетъ, и потому всякому воздастъ. Аще убо кто грѣхъ какій творитъ явно или тайно, аще крадетъ, убиваетъ, блудодѣйствуетъ и прочее беззаконіе дѣлаетъ, видятъ очи Его; аще кто злословитъ, хулитъ, ругаетъ, клевещетъ, сквернословитъ и прочія беззаконныя слова произноситъ, слышатъ святѣйшія уши Его; аще въ сердцѣ своемъ замышляетъ что худое и злое, уже Онъ знаетъ тое; аще ненавидитъ, презираетъ ближняго своего, аще злобится на него и отмстить ему хощетъ какъ нибудь, уже тое явно предъ Нимъ и за тое имѣетъ воздать ему. Разсуждай сіе, хрістіанине, и берегись предъ всевидящимъ окомъ Божіимъ грѣшить и беззаконновать. Беззаконно и безстыдно предъ царемъ земнымъ "--- человѣкомъ, предъ властелиномъ, предъ отцемъ плотскимъ и предъ всякимъ честнымъ человѣкомъ безчиніе показывать, яко таковымъ безчиніемъ присутствующему лицу досада и безчестіе дѣлается: кольми паче предъ величествомъ Божіимъ дѣлать тое беззаконно, безстыдно и страшно. Не видишь ты Его, но Онъ видитъ тебе, яко близъ тебе есть. Берегись же досажденіе и безчестіе величеству Его дѣлать, да не Его судъ праведный на себѣ дознаеши. "--- 14)~Грѣхи по большей части бываютъ отъ случаевъ и соблазновъ. Тако Давидъ царь вышелъ на кровъ дома царскаго проходитися, и увидѣлъ жену Уріину, и прелюбодѣйствова съ нею\footnote{2~Цар.~11,~2--4.}. Тако нынѣ многіи выходятъ изъ домовъ своихъ и бываютъ въ собраніяхъ, въ пиршествахъ и прочіихъ дѣлахъ человѣческихъ, и коль много согрѣшаютъ то словомъ, то дѣломъ, то соблазняютъ, то соблазняются: отъ чего сохранилися бы, когда бы внутрь дома своего пребывали, и въ трудахъ по своему званію находилися. Сего ради хотящему отъ грѣха убѣгать должно удаляться и отъ случаевъ, которые ко грѣху приводятъ, какъ"=то: отъ разговоровъ, отъ безчинныхъ собраній и пиршествъ: \textit{тлятъ бо обычаи благи бесѣды злы}\footnote{1~Кор.~15,~33.}. Въ таковыхъ бо собраніяхъ столько почти бываетъ соблазновъ, сколько лицъ; и столько грѣховъ и беззаконій, сколько словъ и дѣлъ. Бѣги убо, хрістіанине, отъ такихъ собраній, какъ Лотъ отъ Содома. "--- 15)~Понеже часъ отъ часу умножаются соблазны, и оскудѣваетъ благочестіе и нечестіе усиливается: того ради душѣ, благочестіе любящей, не должно смотрѣть, что люди дѣлаютъ, какіе бы они ни были, но внимать, чему слово Божіе учитъ. Не смотри, хрістіанине, на тѣхъ хрістіанъ, которые то и знаютъ, что или ѣздятъ въ гости, или срѣтаютъ и принимаютъ гостей; которые въ томъ и поучаются, чтобы обогатиться и прославиться на землѣ; которые за грѣхъ не поставляютъ чужое присвоить, ближняго осудить, оклеветать, обезчестить, обмануть и прельстить, на ближняго злобиться, отмстить ему, и проч.: нѣтъ бо въ такихъ и слѣда хрістіанства. \textit{Не ревнуй лукавнующимъ, ниже завиди творящимъ беззаконіе}\footnote{Пс.~36,~1.}, но паче внимай, что слово Божіе глаголетъ: \textit{не любите міра, ни яже въ мірѣ. Аще кто любитъ міръ, нѣсть любве Отчи въ немъ: яко все, еже въ мірѣ, похоть плотская, и похоть очесъ, и гордость житейская, нѣсть отъ Отца, но отъ міра сего есть. И міръ преходитъ и похоть его: а творяй волю Божію пребываетъ во вѣки}\footnote{1~Іоан.~2,~15--17.}. "--- 16)~Понеже чрезъ глаза и уши наши, какъ окна, въ храмину сердца нашего всякій соблазнъ входитъ, и, хощемъ ли, или не хощемъ, трогаетъ насъ и возставляетъ похотѣніе: того ради имъ не должно давать свободу, да не, чрезъ тыя соблазнъ вшедши, поколеблетъ и совратитъ домъ нашъ душевный. Тако Давидъ увидѣлъ жену и палъ съ нею, какъ выше сказано. Тако и нынѣ многіи не опасно видятъ и падаютъ, хотя не тѣломъ, но душею. Безопаснѣе врага въ домъ не допущать, нежели допустивши съ нимъ бороться. Тако паки многіи слышатъ клеветы и переговоры ближнихъ своихъ, и языкомъ съ ними падаютъ. Сего ради какъ очи отъ суеты, такъ уши отъ клеветы отвращать должно хотящему отъ грѣха убѣгать. "--- 17)~Тщаніе и подвигъ человѣческій безъ помощи Божіей ничего не можетъ, ибо растлѣнное естество имѣемъ и ко всякому злу склонное, и сатана всегда тщится запяти стопы наша, сокрываетѣ сѣти намъ и при стезяхъ нашихъ полагаетъ соблазны намъ, и тако тщится всякимъ образомъ насъ въ грѣхъ вринуть. Сего ради неотмѣнно нужна молитва всегдашняя хотящему противу грѣха подвизатися. На всякій часъ требуемъ помощи Божіей, яко на всякій часѣ боретъ насъ врагъ нашъ: убо всегда и помощи отъ Бога противу его просить должно.

\subsection[Глава 2-я. О пристрастіи или привычкѣ ко грѣху.]{глава вторая.\\\bfseries О пристрастіи или привычкѣ ко грѣху.}

\begin{quotation}\textit{Аще премѣнитъ Еѳіоплянинъ кожу, и рысь пестроты своя: и вы можете благотворити, научившеся злу}.\quad Іерем.~13,~23.\end{quotation}


\paragraph*{§\:46.} Доколѣ человѣкъ какимъ не искусится грѣхомъ, то не безъ страха къ тому приступаетъ, и по согрѣшеніи совѣсть сильно мучитъ его. Тако нѣкоторые сами изволяютъ лучше убіенными быть, нежели нападающихъ на нихъ убить. Тако къ воровству, хищенію, блудодѣянію и прочіимъ беззаконіямъ не безъ стыда и страха сперва приступъ бываетъ. Ибо совѣсть и разумъ, благодатію Божіею просвѣщенний, прежде согрѣшенія какъ свѣща въ человѣкѣ сіяетъ, и показуетъ ему грѣха мерзость и грѣху послѣдующій гнѣвъ Божій, почему человѣкъ и боится на грѣхъ дерзнуть: но какъ согрѣшитъ, и нѣсколько разъ на тоежъ дерзнетъ, уже удобно къ беззаконному тому дѣлу приступаетъ; и чимъ болѣе согрѣшаетъ, тѣмъ удобнѣе на тое дерзаетъ. «Грѣхъ, глаголетъ святый Златоустъ, доколѣ начинается, имѣетъ нѣкое стыдѣніе; егда же совершится, тогда безстуднѣйшими творитъ содѣвающихъ его»\footnote{Бес.~10"~я на Дѣян. Апост.}. Ибо разумъ, грѣхомъ какъ тьмою помраченный, и совѣсть частымъ грѣха повтореніемъ замаранная, въ обличительномъ своемъ дѣйствіи ослабѣваетъ, хотя то никогда не престаетъ и самыхъ безстыдниковъ обличать; и какъ огнь, сметіемъ насыпанный, дымъ, такъ она, хотя грѣхами засорена, однакожъ дымъ гнѣва Божія, на нихъ грядущій, показывать и освѣщать не престаетъ.

\paragraph*{§\:47.} Отъ замедленія во грѣхѣ какомъ, или отъ многократнаго повторенія грѣха дѣлается пристрастіе или привычка ко грѣху. Такъ дѣлается пристрастіе къ піянству, воровству, лихоиманію, блуду, оклеветанію, осужденію и прочіимъ беззаконіямъ.

\paragraph*{§\:48.} Пристрастіе сіе или привычка такъ сильна бываетъ, что какъ вторая природа человѣка къ тому, къ чему пристрастится, влечетъ. «Велико, глаголетъ святый Златоустъ, велико есть обычая мучительство, и толико, что въ нужду естества устрояется»\footnote{7"~я на 2"~е посл. къ Кор.}, понеже пристрастіе глубоко въ сердцѣ вкореняется. И чимъ болѣе человѣкъ грѣхъ какій творитъ и медлитъ въ томъ, тѣмъ сильнѣе въ сердцѣ его пристрастіе углубляется. Какъ бо древо, чимъ болѣе растетъ, тѣмъ болѣе въ землю корень свой пущаетъ, тако чимъ болѣе грѣховный обычай растетъ, тѣмъ глубочае въ сердцѣ человѣческомъ корень свой утверждаетъ. Но какъ древо, чимъ большее будетъ, тѣмъ съ большею трудностію изъ земли исторгается: тако чимъ болѣе усилится и утвердится обычай грѣховный, тѣмъ съ большею трудностію отъ того освобождается человѣкъ, и то не безъ помощи Божіей. «Жестокая брань есть побѣдить обычай», глаголетъ на Псал.~30"~й Августинъ. И хотя часто бываетъ, что человѣкъ нѣсколько времени и воздерживается отъ страсти; но обычаемъ, какъ веревкою нѣкоею, къ тойже блевотинѣ привлекается; и, какъ вѣтромъ огнь, такъ обычаемъ застарѣлымъ похоть въ немъ злая раздувается, возжигается и силу свою воспріемлетъ. Видимъ явно истину сію на тѣхъ, которые къ піянству привыкли. Какъ горько отъ сихъ бѣдныхъ людей, изтрезвившеся, плачутъ, рыдаютъ, окаеваютъ себе, видя свою бѣду и души пагубу; но при случаѣ, обычаемъ влекомы, паки на страсть обращаяся, многіи въ семъ бѣдственномъ, воистинну плача достойномъ, состояніи и житіе свое оканчиваютъ. Что о піянственной страсти, тое и о другихъ страстехъ разумѣть должно. Видимъ, что воровъ и лихоимцевъ ни стыдъ, ни страхъ человѣческій, ни страхъ Божій, ни страхъ суда и казни временныя, ни страхъ суда Божія и казни вѣчныя, отъ злодѣйства отвратити не можетъ; лучше они изволяютъ все терпѣть и погибать, нежели лакомство свое оставить. О таковыхъ глаголется: \textit{не уснутъ, аще зла не сотворятъ; отъимется сонъ отъ нихъ, и не спятъ}\footnote{Притч.~4,~10.}.

\paragraph*{§\:49.} Страстолюбіе есть внутреннее и душевное идолослуженіе; потому что работающіи страстямъ почитаютъ ихъ внутреннимъ сердца покореніемъ, какъ идоловъ. Тако угождающимъ и работающимъ чреву \textit{чрево богъ есть}\footnote{Филип.~3,~19.}; лихоимцу \textit{лихоиманіе есть идолослуженіе}, по ученію апостола\footnote{Кол.~3,~5.}; работающіи мамонѣ мамону за господа почитаютъ\footnote{Матѳ.~6,~24.}, \textit{и всякъ творяй грѣхъ, рабъ есть грѣха}\footnote{Іоан.~8,~34.}; и \textit{имже кто побѣжденъ бываетъ, сему и работенъ есть}\footnote{2~Петр.~2,~19.}. Грѣхолюбивому бо человѣку грѣхъ, которому работаетъ, есть какъ идолъ. Сердце его грѣхолюбное есть какъ капище мерзкое, въ которомъ мерзкому сему истукану жертву приноситъ: грѣхъ бо въ сердцѣ имѣется. Вмѣсто тельцевъ, барановъ и прочіихъ животныхъ, волю свою и охотнѣйшее послушаніе въ жертву приноситъ. И тако сколько разъ грѣшникъ соизволяетъ на грѣхъ, къ которому пристрастился, столько сердцемъ отрекается Хріста; и сколько разъ дѣломъ его исполняетъ, столько идолу тому жертву приноситъ. \textit{Никтоже бо можетъ двѣма господинома работати}, глаголетъ Господь\footnote{Матѳ.~6,~24.}. О семъ и Златоустъ святый тако поучаетъ: «да не речеши ми сего, яко не покланяешися идолу златому, но оно ми покажи, яко сего не твориши, еже повелѣваетъ тебѣ злато. Ибо различніи суть идолослуженія образы: иный мамону почитаетъ за господа, иный чрево за бога, иный другую похоть вселютѣйшую. Не пожираешь имъ воловъ, якоже Еллины, но много хуждше свою закалаеши душу; не преклоняеши колѣнъ, ни покланяешися, но съ большимъ послушаніемъ твориши вся, яже тебѣ повелѣваетъ чрево, злато и похоти мучительство. Понеже и Еллины сего ради мерзцы суть, яко страсти наша обоготвориша»\footnote{Бес.~6"~я на посл. къ Римл.}. Какъ же мерзко и бѣдственно есть пристрастіе грѣховное! Мерзко, ибо страсть вмѣсто бога, какъ идолъ, почитается. Бѣдственно, ибо почитающій оную "--- Хріста и Бога, въ Которомъ спасеніе наше есть, отрекается, и съ превеликою трудностію отъ мерзкія сея работы свобождается человѣкъ. А что бѣдственнѣе, въ томъ застарѣвшеся, многіи и на оный вѣкъ безъ покаянія и надежды спасенія отходятъ и вѣчными плѣнниками ада и смерти дѣлаются.

\paragraph*{§\:50.} Дабы отъ мерзкія и тяжкія сея работы человѣку избавиться, пользуетъ примѣчать и дѣлать: 1)~Должно человѣку всякому осмотрѣться, въ какомъ онъ состояніи находится, и не обладаетъ ли имъ какая страсть и грѣховный обычай; не плѣнилося ли сердце его сребролюбіемъ и лихоиманіемъ, или блудною похотію, или любленіемъ суетныя славы и чести; не обладаетъ ли имъ гнѣвъ и злопомнѣніе, или піянство; не имѣетъ ли привычки къ осужденію, оклеветанію, хуленію, руганію, злословію ближняго и проч. Ибо страсть и обычай ослѣпляютъ душевное око такъ, что человѣкъ бѣдствія и пагубы своея не видитъ. Къ чему нужно прилѣжное чтеніе или слушаніе святаго Писанія и прочіихъ хрістіанскихъ книгъ, яко изъ тѣхъ грѣхъ познается; обращеніе съ добрыми и благочестивыми людьми, отъ которыхъ добраго житія свои развращенные нравы можетъ познать человѣкъ. Къ томужъ призывать и молить Іисуса Хріста, просвѣтителя слѣпыхъ, да Самъ Онъ просвѣтитъ душевныя очи: безъ Него бо какъ просвѣщеніе, такъ и познаніе бѣдственнаго грѣховнаго состоянія не бываетъ; пока Онъ не коснется слѣпыхъ нашихъ очесъ, всегда будемъ слѣпы, и будемъ блудить, какъ слѣпые. Сіе испытаніе и познаніе неотмѣнно нужно всякому хотящему вѣчное спасеніе получить; ибо отъ сего есть начало спасенія. Како бо будешь искать исцѣленія, не познавши болѣзни? Надобно неотмѣнно познать прежде болѣзнь. Болящіи бо требуютъ врача, якоже глаголетъ Господь: \textit{не требуютъ здравіи врача, но болящіи}\footnote{Матѳ.~9,~12.}. Кто же суть болящіи? Неотмѣнно тѣ, которые познаютъ и признаютъ свою болѣзнь. Якоже убо испытуемъ тѣлесную болѣзнь, чтобы излѣчиться: тако должно испытовать душевную болѣзнь, да, познавше ее, исцѣленія поищемъ. Якоже бо начало здравія тѣлеснаго "--- познать болѣзнь, тако начало спасенія "--- познать бѣдственное души своея состояніе. Таковое бо познаніе подвигнетъ человѣка искать посредствія, чрезъ которое бы избавиться моглъ отъ бѣдствія своего. Испытуй убо себе и познавай, хрістіанине, въ какомъ состояніи находишься. Имѣешь ли богатство, или не имѣешь, здравіе или болѣзнь тѣлесную, славу или безславіе: что къ тебѣ? Все сіе въ мірѣ останется. Единаго того испытывай, какую болѣзнь имѣетъ душа, и имѣешь ли надежду спасенія, которое едино нужно. "--- 2)~Познавшему бѣдственное души своея состояніе не должно медлить, но скоро отъ злаго обычая отстать: ибо чимъ болѣе будешь медлить въ страстномъ обычаѣ, тѣмъ болѣе онъ усилится и труднѣе отъ него отстать; якоже чимъ болѣе продолжается болѣзнь тѣлесная, тѣмъ неудобнѣйшая бываетъ къ исцѣленію. И хотя сильно будетъ бороть и на прежнее состояніе привлекать страсть, твердо противу ея, какъ домашняго врага, стоять, не поддаваться похоти ея, призывать всемогущую Сына Божія помощь. Страсть бо подобна псу. Какъ песъ бѣжитъ за нами и гонитъ насъ, когда отъ него убѣгаемъ, а когда противу его стоимъ и гонимъ его, бѣжитъ отъ насъ: такъ и страсть гонитъ того, кто ей поддается и слушаетъ ее; уступаетъ тому, кто противится ей. Произволеніе, тщаніе и трудъ съ помощію Божіею все можетъ; и хотя много мучительства отъ ней претерпитъ подвижникъ, однакожъ наконецъ уступитъ ему, яко силою Божію укрѣпленному, которая помогаетъ труждающимся и молящимся. Таковыхъ примѣровъ много представляетъ намъ церковная исторія, въ которой читаемъ, что многіе разбойники, блудники, блудницы и прочіи грѣшники, обратившіеся отъ грѣховъ къ Богу, хотя и много претерпѣли отъ злаго обычая, однакожъ наконецъ съ помощію Божіею побѣдили его, и \textit{плоть} распяли со \textit{страстьми и похотьми}, и тако учинилися \textit{Хрістовыми} рабами\footnote{Гал.~5,~24.} которые были \textit{раби грѣха}\footnote{Іоан.~8,~34.}. Таковые бо примѣры на сіе истое и написаны, дабы грѣшники заблуждшіе не отчаявалися обращаться, и противу злаго обычая подвизались, яко съ помощію Божіею все возможно человѣку. \textit{Тойжде бо Іисусъ Хрістосъ вчера и днесь, Тойжде и во вѣки} Помощникъ и Спаситель есть обращающимся, труждающимся, подвизающимся, и призывающимъ Его\footnote{Евр.~13,~8.}. "--- 3)~Къ сему подвигу помогаетъ и укрѣпляетъ въ немъ частое размышленіе о смерти, которая всякаго и всякимъ образомъ восхищаетъ и въ вѣчности заключаетъ; размышленіе о праведномъ Хрістовомъ судѣ, о царствіи небесномъ и мукѣ безконечной. "--- 4)~Оставившему злый обычай не должно возвращатися на той, какъ \textit{псу на свою блевотину, и свиніи омывшейся въ калъ тинный}\footnote{2~Петр.~2,~22.}; но твердо противу его стоять, побѣждать благодатію Божіею, \textit{задняя забывать, и въ предняя простираться}\footnote{Филип.~3,~13.}, и, по увѣщанію Господню, \textit{поминать жену Лотову}\footnote{Лук.~17,~32.}, которая обозрѣвшись вспять, то"=есть, къ Содому, отъ котораго вышла, \textit{бысть столпъ сланъ}\footnote{Быт.~19,~26.}. Міръ съ похотьми своими есть Содомъ беззаконный, отъ котораго убѣгающимъ не должно обращаться къ нему, да не паки запутавшеся въ сѣтяхъ его, съ нимъ осуждены будутъ, и помнить Апостольское слово: \textit{не любите міра, ни яже въ мірѣ: аще кто любитъ міръ, нѣсть любве Отчи въ немъ}\footnote{1~Іоан.~2,~15.}: и \textit{иже восхощетъ другъ быти міру, врагъ Божій бываетъ}\footnote{Іак.~4,~4.}.

\subsection[Глава 3-я. О слѣпотѣ человѣческой.]{глава третія.\\\bfseries О слѣпотѣ человѣческой.}

\begin{quotation}\textit{Осяжутъ яко слѣпіи стѣну, и яко суще безъ очесъ осязати будутъ, и падутся въ полудни яко въ полунощи}.\quad Исаіи 59,~10.\end{quotation}


\paragraph*{§\:51.} Что свѣтъ чувственный глазамъ нашимъ, тое душѣ нашей есть Божія благость. Когда свѣтъ сіяетъ, человѣкъ все добрѣ видитъ, видитъ путь, ровъ, вредное, и бережется того, раздѣляетъ бѣлое отъ чернаго и едину вещь отъ другой: тако, когда Божія благодать просвѣщаетъ душу, душа все добрѣ познаетъ и видитъ; видитъ чудныя Божія дѣла, промыслъ и судьбы Его, распознаетъ добро отъ зла, добродѣтель отъ порока, и пользу душевную видитъ и ищетъ ея, видитъ вредъ и уклоняется отъ него.

\paragraph*{§\:52.} Что тьма глазамъ тѣлеснымъ, тое грѣхъ душѣ человѣческой. Тьма ослѣпляетъ глаза такъ, что человѣкъ, хотя и глаза имѣетъ, ничего не видитъ, и во тьмѣ ходящій во всемъ подобенъ слѣпому: пути, по которому идетъ, не видитъ, и вреда предъ собою не познаетъ; единой вещи отъ другой не раздѣляетъ, напр. злата отъ сребра, мѣди отъ желѣза, бѣлаго отъ чернаго, и краснаго отъ зеленаго и проч.; съ пути сбивается и заблуждаетъ, не знаетъ, куды идетъ, въ ровъ падаетъ, и прочая вредная приключенія претерпѣваетъ. Сія злая тьма и слѣпота вещественная человѣку дѣлается. Подобнымъ образомъ грѣхъ "--- тьма духовная, око душевное помрачаетъ и ослѣпляетъ такъ, что грѣшникъ подобная или паче горшая душевная злая претерпѣваетъ, и ходитъ какъ слѣпый; не знаетъ, куды путь ведетъ его; не видитъ предъ собою рва вѣчныя погибели, въ который имѣетъ пасти; порока отъ добродѣтели, зла отъ добра, истины отъ лжи, истиннаго благополучія отъ истиннаго злополучія не распознаетъ, и такъ видяще не видитъ, и осязаетъ какъ слѣпый. Въ благополучіи ли живетъ? Свирѣпѣетъ, какъ конь необученный и необузданный, и не видитъ, что тѣмъ благополучіемъ Богъ его, какъ отецъ малое отроча яблокомъ, къ Себѣ привлекаетъ. Въ противность ли попался? Ропщетъ, негодуетъ и хулитъ, аки бы неправда съ нимъ дѣлается; возноситъ жалобные, и паче хульные гласы: какая моя неправда? что я согрѣшилъ? неужели я паче прочіихъ грѣшнѣйшій? того ли я достоинъ? тое ли труды мои заслужили? Оправдываетъ себе, всякою неправдою наполненъ; очищаетъ себе, весь замаранъ; недостойнымъ себе временнаго наказанія судитъ, вѣчнаго достоинъ; заслуги свои выхваляетъ, которыя ничего не стоятъ. Лѣкаря въ болѣзни призываетъ, чтобы отъ болѣзни исцѣлиться, и всякое лѣкарство, и самое жестокое, пріемлетъ, и все терпитъ, что ему ни предлагаетъ онъ, и мзду цѣлителю даетъ. Богъ милосердый, \textit{Иже хощетъ всѣмъ человѣкомъ спастися и въ разумъ истины пріити}\footnote{1~Тим.~2,~4.}, хощетъ душу его болѣзнующую, бѣдою, какъ жестокимъ лѣкарствомъ, исцѣлить и въ здравіе привести. Лютая бо болѣзнь душевная есть сребролюбіе, самолюбіе, славолюбіе, гордость, гнѣвъ, зависть, ненависть, нечистота и прочія страсти; сими она недугуетъ, истаеваетъ и умерщвляется; но грѣшникъ слѣпый не познаетъ того Божія благодѣянія, и милостивому Благодѣтелю не токмо не благодаритъ, но ропщетъ на Него: какая моя неправда? Всякое бо бѣдствіе настоящаго времени, на грѣшника отъ Бога посылаемое, есть жезлъ отеческаго Его наказанія, которымъ грѣшника біетъ и отъ сна грѣховнаго возбуждаетъ; или есть лѣкарство жестокое, какъ выше сказано, которое ему посылаетъ, да душа его раслабленная исцѣлится: но грѣшникъ того не чувствуетъ. Здравіе ли паки и богатство имѣетъ? Сіе трудамъ и тщанію, оное мудрости и разуму своему приписуетъ, а не Богу, \textit{отъ Котораго животъ и смерть, нищета и богатство суть}\footnote{Сир.~11,~14.}. Одержалъ побѣду надъ непріятелемъ? Храбрости, хитрости, искусству своему, силѣ и множеству воиновъ, а не Богу побѣду причитаетъ. \textit{Не спасается бо царь многою силою, и исполинъ не спасется множествомъ крѣпости своея. Ложь конь во спасеніе, во множествѣ же силы своея не спасется}\footnote{Пс.~32,~16 и 17.}; яко \textit{не въ силѣ констѣй восхощетъ, ниже въ лыстѣхъ мужескихъ благоволитъ: благоволитъ Господь въ боящихся Его, и во уповающихъ на милость Его}\footnote{146,~10 и 11.}. Насыщается ли пищи, одѣвается одеждою, согрѣвается огнемъ, просвѣщается свѣтомъ и прочіими благими Божіими снабдѣвается грѣшникъ? Но Бога, благихъ Дателя, и въ благодѣяніяхъ не чувствуетъ, и Благодѣтелю не благодаритъ, яко не любитъ Его: безъ любви бо не можетъ быть благодарность. Вся тварь, небо, солнце, луна, звѣзды, земля и исполненіе ея, аки устами \textit{повѣдаютъ славу Божію}\footnote{18,~2.}; но грѣшникъ ослѣпленный величества славы Его не чувствуетъ и не трепещетъ. Градъ и палата архитектора своего, художество мастера своего, домъ хозяина своего показуетъ и въ познаніе всякому приводитъ. Всякъ бо отъ художества художника, и отъ строенія строителя познаетъ. Міръ сей видимый, какъ градъ прекрасный, не руками, но словомъ Божіимъ сотворенный, и яко художество премудрое и домъ пребогатый, въ которомъ вси живемъ, Бога, яко архитектора, строителя, промыслителя и хозяина своего, какъ перстомъ показуетъ, и яко чудное, премудрое, благое созданіе, чуднаго, благаго, премудраго и всесильнаго Создателя доказываетъ: но грѣшникъ слѣпый не познаетъ. Сынъ отца, отъ котораго рожденъ и воспитанъ, рабъ господина своего, отъ котораго стяжанъ, знаетъ: но грѣшникъ Бога, Господа и Отца своего, отъ Котораго сохраняется, питается, одѣвается, не познаетъ. Песъ господина своего, волъ стяжавшаго его знаетъ: но грѣшникъ, разумомъ одаренный, Создателя, Стяжателя и Хранителя своего не познаетъ, якоже глаголетъ Богъ чрезъ пророка: \textit{позна волъ стяжавшаго его, и оселъ ясли господина своего: Израиль же Мене не позна, и людіе Мои не разумѣша}\footnote{Ис.~1,~3.}; паче же глаголетъ, \textit{яко безуменъ, въ сердцѣ своемъ: нѣсть Богъ}\footnote{Пс.~13,~1.}. Такъ бѣдственною слѣпотою поражаетъ грѣхъ сердце человѣческое, отъ которой слѣпоты безстрашіе, отъ безстрашія всякое беззаконіе бываетъ!

\paragraph*{§\:53.} Аще бы кто сказалъ: какъ Бога не знать, Который всегда словомъ Его святымъ проповѣдуется? "--- \textit{Отвѣтъ}. Сіе то и дивно, или жалѣнія достойно, что Богъ какъ чрезъ созданіе, такъ и чрезъ слово Свое въ познаніе всѣмъ Себе подаетъ; но грѣшникъ, яко глухій, не слышитъ слова Его и Господа своего не познаетъ; имя Божіе слышитъ, но Бога не познаетъ; гласъ о Господѣ слышитъ ушами плотскими, но душевными не слышитъ, и такъ, слышащи не слышитъ, и видящи не видитъ. Когда Богъ проповѣдуется словомъ Его святымъ, то проповѣдуется и святая воля Его; но грѣшникъ не знаетъ и не творитъ ея; проповѣдуется всемогущество и величество Его, предъ которымъ грѣшникъ не смиряется; проповѣдуется правда Его, которой грѣшникъ не боится и не почитаетъ; проповѣдуется истина Его, которой грѣшникъ не вѣритъ; проповѣдуется вездѣсущіе Его, предъ которымъ грѣшникъ благоговѣинства не показуетъ, "--- не показуетъ же, яко не познаетъ Его; проповѣдуется премудрый промыслъ Его, котораго грѣшникъ не разсуждаетъ; проповѣдуется высочайшая святость Его, которой грѣшникъ не почитаетъ; проповѣдуется верховная власть Его, которой грѣшникъ не покаряется; проповѣдуется страшная слава Его, которой грѣшникъ не почитаетъ; проповѣдуется безконечная благость Его, которой участникомъ быть грѣшникъ не тщится и не ищетъ; проповѣдуются страшные суды Его, которыхъ грѣшникъ не трепещетъ, и проч. Тако грѣшникъ, яко \textit{мужъ безуменъ, не познаетъ, и неразумивъ не разумѣетъ Бога и дѣлъ Божіихъ}\footnote{Пс.~91,~7.}! Истинное бо познаніе Божіе праздно и безплодно быть не можетъ, но вноситъ истинное богопочитаніе. Сынъ отца своего знаетъ и почитаетъ, любитъ и слушаетъ его; рабъ господина своего знаетъ и боится его, повинуется ему и работаетъ ему: тако кто Бога знаетъ, не можетъ не почитать Его, не слушать, не повиноваться и не работать Ему, воли и повелѣній Его не исполнять. О семъ Богъ чрезъ пророка глаголетъ: \textit{сынъ славитъ отца, и рабъ господина своего боится: и аще Отецъ есмь Азъ, то гдѣ есть слава Моя? и аще Господь есмь Азъ, то гдѣ есть страхъ Мой}\footnote{Мал.~1,~6.}? Славитъ же сынъ отца потому, что за отца своего почитаетъ, и рабъ господина своего боится того ради, что за господина своего имѣетъ его. Аще убо отцу "--- человѣку отъ сына честь, и господину "--- человѣку отъ раба страхъ отдается, потому что оный отца за отца своего, сей господина за господина своего признаетъ: кольми паче Богъ, Который надъ всѣми Господь есть, достоинъ чести и страха, когда Его за Господа и Отца нашего признаемъ и почитаемъ. Видиши, что отъ познанія Божія слѣдуетъ богопочитаніе. Скажи, пожалуй, который познаніемъ Божіимъ хвалишися и не поистинѣ почитаеши Его, како можетъ быть, чтобы ты восхотѣлъ грѣшить, вѣдая, что Богъ вездѣ есть, и всякое дѣло видитъ и правосуденъ есть; Который за грѣхъ и временно и вѣчно казнитъ, и въ самомъ дѣлѣ грѣховномъ можетъ тя поразить? Предъ царемъ земнымъ, властію, господиномъ и отцемъ твоимъ не смѣешь безчинствовать: предъ Богомъ ли, великимъ и страшнымъ, дерзаешь худое и противное Ему дѣлать? Не можетъ сіе быть! Како не будешь Ему отъ сердца благодарить и любить Его, зная, что Онъ есть Создатель, Искупитель, Отецъ, Питатель, Хранитель и Благодѣтель высочайшій, каковаго большій не можетъ быть? Человѣка "--- благодѣтеля, который не своимъ, но Божіимъ добромъ снабдѣваетъ тебе, любишь и благодаришь ему: Богу ли, отъ Котораго бытіе свое имѣешь, и на всякій часъ и минуту получаешь благодѣянія, безъ которыхъ и жить не можемъ, не будешь благодарить? Како отъ сердца не повѣришь слову Его, когда познаешь, что Онъ истиненъ есть, и солгати не можетъ, и слово Его истина есть? Человѣку честному вѣришь, который солгати можетъ, \textit{яко всякъ человѣкъ ложь}\footnote{Пс.~115,~2.}: Богу ли, Который солгати не можетъ и не хощетъ, не повѣришь? Отъ сего все истинное благочестіе происходитъ: отъ вѣры бо все истинное богопочитаніе и благочестіе зависитъ. Како не будешь трепетать суда Его, когда познаешь правду Его, которая всякому воздаетъ по дѣломъ Его? Властелина праведнаго боишися и не дерзаешь закона его нарушать, который не все знаетъ, что подвластніи его дѣлаютъ: Бога ли праведнаго не убоишися, Который все, слово, дѣло и помышленіе твое знаетъ и въ книзѣ Своей записываетъ, и по всему тому тебѣ воздастъ? Како не покоришися и не послушаеши повелѣній Его, когда точно познаеши, что Онъ твой самоверховнѣйшій Господь, царь и властелинъ, Который небомъ и землею владѣетъ? Человѣку "--- царю, который только надъ тѣломъ твоимъ власть имѣетъ, покаряешися, повинуешися и всякое послушаніе показуеши: Богу ли, Который и тѣла и души твоей животъ и смерть въ руцѣ Своей содержитъ, не повинешися и не покажеши отъ сердца послушанія? Како не возлюбиши Его отъ сердца твоего, и не будешь искать Его и прилѣпляться Ему, и общеніе съ Нимъ имѣть, когда познаешь, что Онъ единъ есть верховное, существенное, непремѣняемое и безконечное добро, съ которымъ быть "--- истинное и совершенное есть блаженство, и отъ котораго отлучиться "--- истинное и совершенное есть злополучіе? Человѣка добраго, который милосердіемъ и другими добродѣтелями украшенъ; и какую доброту имѣетъ, отъ Бога имѣетъ, любишь: Бога ли не будешь любить, Который Самъ въ Себѣ благъ такъ, что \textit{никтоже благъ, токмо единъ Богъ}, по словеси Господню\footnote{Матѳ.~19,~17.}? Коль многіи при царяхъ земныхъ добрыхъ, которые и сами, какъ человѣки всякому бѣдствію подлежатъ, быть желаютъ, и съ великимъ усердіемъ ищутъ того, яко при нихъ быть за блаженство себѣ поставляютъ: съ Богомъ ли быть, Царемъ царствующихъ и Господемъ господствующихъ, благимъ и милостивымъ, не пожелаешь, когда познаешь благость Его и вкусишь и увидишь, \textit{яко благъ Господь}\footnote{Пс.~33,~9.}? Како не покажешь милосердія ближнему твоему "--- всякому человѣку, когда познаешь, что на всякій день и часъ отъ Бога получаешь неизреченное и непостижимое милосердіе? Его милосердіемъ живешь на земли, недостоинъ жити; Его милосердіемъ питаешися, одѣваешися и цѣлъ пребываеши. Како не подашь ближнему твоему отъ благихъ твоихъ, когда познаешь, что все, что ни имѣешь, кромѣ грѣховъ и немощей твоихъ, Божіе есть, а не твое собственное? Все бо отъ Бога пріемлемъ, душу и тѣло наше, жизнь и дыханіе, и бытіемъ нашимъ Ему мы должны. Како не отпустишь ближнему согрѣшеній малыхъ, когда познаешь, что \textit{Богъ во Хрістѣ простилъ} безчисленныя тебѣ согрѣшенія\footnote{Еф.~4,~32.}; и что твоихъ согрѣшеній долги паки возвратятся на тебе и удержатся, аще ближнему твоему не оставиши, и за нихъ во вѣки будешь страдать въ темницѣ адской, какъ притчею тое изображается\footnote{Матѳ.~18,~23--34.}? Откуду Господь заключаетъ притчу тую: \textit{тако и Отецъ небесный сотворитъ вамъ, аще не отпустите кійждо брату своему отъ сердецъ вашихъ согрѣшенія ихъ}\footnote{ст.~35.}. Како не презриши суету міра сего, и не пожелаеши блаженства онаго будущаго, и со всякимъ усердіемъ не будеши искать онаго, когда точно увѣруеши слову Божію, которое различно изображаетъ благая оная, и несказанная и умомъ непостижимая и вѣчная, и конца неимущая, якоже и Богъ Самъ проповѣдуетъ и обѣщаетъ вѣрующимъ во Хріста и вѣрно работающимъ Богу и любящимъ Его\footnote{1~Кор.~2,~9.}? Временнаго блаженства, которое ничто предъ онымъ, усердно ищешь, яко видишь его: како онаго безконечнаго не будешь искать, когда вѣрою увидишь его? Ради временныя чести и славы вѣрно работаешь земному царю и всякіе труды, бѣды и напасти пріемлеши, и на кровавыя войны и сраженія исходишь, яко надѣешися честь и славу отъ него получить: како не будеши вѣрно работать Царю небесному, Богу, Который не временную, но вѣчную и несказанную славу и честь обѣщаетъ вѣрнымъ рабамъ Своимъ? Повѣрь слову сему, что ежели бы человѣкъ частицу нѣкую будущія оныя славы во мгновеніи ока увидѣлъ, тако бы къ ней распалился желаніемъ и любовію, что всю бы міра сего славу, честь, утѣху и богатство какъ гной вмѣнялъ, и тоя бы единыя со всякимъ тщаніемъ искалъ, и не токмо бы похотьми міра сего, но и никакою бы бѣдою и злостраданіемъ отъ той отвратиться не моглъ, "--- якоже пишется о Моисеѣ: \textit{яко отвержеся нарицатися сынъ дщере Фараоновы: паче же изволи страдати съ людьми Божіими, нежели имѣти временную грѣха сладость, большее богатство вмѣнивъ египетскихъ сокровищъ поношеніе Хрістово: взираше бо на мздовоздаяніе}\footnote{Евр.~11,~25 и 26.}, что о мученикахъ святыхъ и прочіихъ угодникахъ Божіихъ разумѣть должно. Видишь убо, хрістіанине, что отъ познанія Божія, слѣдуетъ, а именно: отвращеніе отъ грѣха, міра, тщательное твореніе заповѣдей Его и истинное богослуженіе. Въ познаніи бо Божіемъ не токмо имени Его, но и божественныхъ Его свойствъ, то"=есть, вездѣсущія, правды, благости, милосердія, истины, всемогущества и прочаго познаніе заключается, отъ котораго свойствъ Божіихъ познанія неотмѣнно воспослѣдуетъ, какъ сказано, и нелицемѣрное Божіе богопочитаніе. Правда Его научитъ тебе отъ грѣха берещися; всемогущество Его и величество "--- боятися, трепетати Его и смирятися предъ Нимъ; истина Его "--- вѣрить Ему во всемъ; благость "--- любити Его, и всѣмъ сердцемъ искати Его и прилѣплятися Ему; милосердіе и щедроты Его "--- не отчаяватися за грѣхи, но каятися и ближнему своему являти милость; вездѣсущее всевѣдѣніе Его такожде научитъ тебе отвращаться отъ всякаго зла, грѣха и безчинія, но благоговѣинствовать предъ Нимъ, и проч. Отсюду видишь, что беззаконное и безстрашное житіе, каковое есть татей, хищниковъ, мздоимцевъ, лихоимцевъ, блудниковъ, хульниковъ, ругателей, злобныхъ, ненавистниковъ, піяницъ и прочіихъ, доказательствомъ есть безбожнаго ихъ сердца, и слѣпотою и тьмою невѣдѣнія Божія помраченнаго. И хотя таковые исповѣдуютъ Бога, но въ сердцахъ своихъ глаголютъ: \textit{нѣсть Богъ}, якоже псаломъ 13"~й показуетъ таковыхъ, и прочія въ Писаніи мѣста. О таковыхъ исповѣдникахъ апостолъ написалъ: \textit{Бога исповѣдуютъ вѣдѣти, а дѣлы отмещутся Его, мерзцы суще и непокориви, и на всякое дѣло благое неискусни}\footnote{Тит.~1,~16.}. И Господь глаголетъ: \textit{приближаются Мнѣ людіе сіи усты своими, и устнами чтутъ Мя: сердце же ихъ далече отстоитъ отъ Мене}\footnote{Матѳ.~15,~8.}. Якоже убо древо не отъ листвія, но отъ плодовъ, тако Божіимъ познаніемъ просвѣщенный человѣкъ и богочтецъ не отъ устнаго исповѣданія, но отъ дѣлъ, согласныхъ исповѣданію, познается. Сердце, Божіимъ познаніемъ просвѣщенное, Бога, какъ выше всего созданія на небеси и на земли именуемаго, познаетъ и вѣруетъ; такъ паче всего боится, почитаетъ, любитъ, служитъ и прочія должности, Ему единому собственныя, усердно отдавать тщится. Слѣдовательно бояться человѣка, но Бога не бояться; человѣку вѣрно работать и угождать, но Богу не угождать; человѣка"=царя, властелина, господина и отца по плоти слушать, но Бога, Который всѣми царями и господами владѣетъ, не слушать; человѣку, который Божіе добро подаетъ, а не свое, благодарить, но Богу, отъ Котораго всякое добро происходитъ и изливается на насъ, не благодарить; на человѣка подобнаго себѣ надѣяться и помощи отъ него въ скорби искать, но на Бога не надѣяться и отъ Него помощи не искать, Который единъ все можетъ и единъ помогаетъ, защищаетъ, избавляетъ и спасаетъ уповающихъ на Него; человѣческаго суда, который тѣла единаго касается, бояться, но Божія суда, который \textit{и душу и тѣло погубляетъ въ гееннѣ}\footnote{Матѳ.~10,~28.}, не бояться, и проч., "--- не явная ли слѣпота и безбожіе? Воистину и малѣйшія искры нѣтъ познанія Божія въ таковомъ сердцѣ, хотя бы и училъ и проповѣдывалъ о Бозѣ! Видимъ сіе во многихъ нынѣшняго вѣка хрістіанехъ; видимъ, что многіе боятся прогнѣвать господъ и властей своихъ, но Бога прогнѣвать не боятся, яко беззаконнуютъ; многіе людямъ угождаютъ, яко не хотятъ оскорбить ихъ, но Богу не угождаютъ, яко оскорбляютъ Его преступленіемъ закона Его святаго; многіе властей и господъ своихъ слушаютъ, повелѣнія ихъ исполняютъ, но Бога не слушаютъ, яко повелѣній Его не исполняютъ; многіе благодарятъ благодѣтелямъ своимъ, но Богу, высочайшему своему Благодѣтелю, не благодарятъ, яко не любятъ Его, "--- безъ любви бо благодарность быти не можетъ; многіе во время напасти къ человѣку прибѣгаютъ ради помощи и заступленія, яко надѣются на него, но къ Богу не прибѣгаютъ, не иныя ради причины, какъ что надежды на Него не имѣютъ; гражданскаго суда какъ многіе боятся, всѣмъ извѣстно, но Божія не боятся, яко не престаютъ грѣшить: чему судъ Божій неотмѣнно послѣдуетъ. Видиши, что сіи и того Богу не отдаютъ, что человѣку отдаютъ, и такъ, какъ человѣка, не почитаютъ Бога, Котораго почитать паче всего истинное Его познаніе учитъ. Сего ради таковые и подобные имъ не знаютъ Бога, хотя устами Его и исповѣдуютъ. Безъ богопочитанія бо и познаніе Божіе быть не можетъ. Якоже убо истинное богопочитаніе, которое состоитъ въ страхѣ, любви, покореніи, послушаніи, смиреніи, надеждѣ, терпѣніи, свидѣтельствомъ есть познанія Божія: тако неистинное, но едино устное и лицемѣрное богопочитаніе знаменіемъ есть незнанія Божія, якоже апостолъ учитъ: \textit{глаголяй, яко познахъ Его, и заповѣди Его не соблюдаетъ, ложь есть, и въ семъ истины нѣсть}\footnote{1~Іоан.~2,~4.}. Самъ разсуди, ради чего царя, властелина, отца и господина почитаешь и слушаешь? Не того ради, что они люди почтенные, "--- иначе бы всѣхъ людей почитать и слушать должно было: но того ради, что царя за твоего царя, властелина за твоего властелина, отца за твоего отца, господина за твоего господина признаешь. Чего ради благодѣтеля любишь и благодаришь? Понеже его за твоего благодѣтеля имѣешь. Чего ради милостиваго и добраго человѣка любишь? Понеже его за таковаго вмѣняешь. Чего ради къ богатому и щедрому ради милостыни приходишь? Понеже надѣешься отъ него получить. Чего ради отъ защитника твоего во время нужды помощи и защищенія ищешь? Понеже уповаешь на силу его. Чего ради праведнаго властелина боишься прогнѣвать? Понеже его за таковаго признаешь. Чего ради предъ начальникомъ, господиномъ и отцемъ твоимъ ничего непристойнаго не дѣлаешь? Понеже видишь ихъ предъ собою, и не смѣешь того дѣлать, да не оскорбишь и не прогнѣваешь ихъ. Ради чегожъ убо Богу того не показуешь, когда признаешь Его за Бога твоего, Царя твоего, Господа твоего, Отца твоего, Благодѣтеля твоего, Защитника и Помощника твоего? Или паче почто немного болѣе Богу тое отдаешь, нежели человѣкамъ? Яко Богъ, какъ несравненно высшій всѣхъ есть, такъ несравненно большаго почитанія достоинъ, нежели вси и всякаго званія человѣки. Исповѣдуешь Бога: но почто не почитаешь Его какъ Бога? Исповѣдуешь Его, Царя и Господа своего: но почто не повинуешься Ему, какъ Царю и Господу своему? Исповѣдуешь Его Отца своего: но почто не любишь Его, какъ отца своего, и не показуешь Ему сыновняго послушанія? Исповѣдуешь Его Благодѣтеля своего: но почто не благодаришь, какъ благодѣтелю? Исповѣдуешь Его Защитника, Помощника и Избавителя своего: но почто не отъ Него, но отъ человѣка ищешь помощи, защищенія и избавленія въ нуждѣ и скорби твоей? Исповѣдуешь, что Богъ вездѣ и на всякомъ мѣстѣ есть, и вся дѣла человѣческая назираетъ, и отъ слова Его слышишь тое: почтожъ не отдаешь Ему чести, и не благоговѣеши предъ Нимъ, якоже предъ отцемъ твоимъ, господиномъ и властію твоею чинишь? Исповѣдуешь Его праведнаго Судію: почто не боишься суда Его праведнаго, и не престаешь грѣшить? и проч. Видиши, что устамъ несогласное сердце и мысль о Богѣ имѣешь, и исповѣданію не соотвѣтствуютъ дѣла и житіе твое. Како убо можешь сказать, что знаю Бога, а не отдаешь того Богу, чего честь Его отъ тебе требуетъ? Знаешь Бога, какъ сказуешь, но не почитаешь Его, якоже Бога: убо не истинно и знаніе твое безъ почитанія, и тако въ слѣпотѣ, тьмѣ и незнаніи находишися.

\paragraph*{§\:54.} Не токмо въ должности къ Богу, но и въ должности къ ближнему, то"=есть, ко всякому человѣку, слѣпъ есть плотскій и непросвѣщенный человѣкъ. Видимъ, что человѣкъ дѣлаетъ ближнему зло, котораго себѣ не хощетъ, и не дѣлаетъ добра, котораго себѣ хощетъ. Какъ негодуетъ и гнѣвается на того, кто его обиждаетъ, злословитъ, клянетъ, хулитъ, порочитъ, касается ложа его, крадетъ, похищаетъ, отнимаетъ что у него, и прочія обиды дѣлаетъ ему "--- видимъ; но самъ тогожде зла дѣлать, или зла за зло воздавать не стыдится и не чувствуетъ. Напротивъ того хощетъ, чтобы его ближній его помиловалъ, въ нуждѣ не оставилъ, напр. алчущаго напиталъ, жаждущаго напоилъ, нагаго одѣлъ, страннаго въ домъ ввелъ и упокоилъ, больнаго и въ темницѣ сѣдящаго посѣтилъ, и прочія милости дѣла сдѣлалъ ему "--- хощетъ того; безспорна сія истина есть: но самъ того ближнимъ *своимъ* не хощетъ дѣлать. Видимъ сіе зло "--- самолюбіе, неправду и слѣпоту въ хрістіанехъ, которые, или молча мимоходятъ ближнихъ своихъ бѣдствующихъ и аки ихъ не видятъ, или не стыдятся о нихъ отвѣщавать: что мнѣ до него нужды? Примѣтно, что многіе преизобильными трапезами насыщаются, а о ближнихъ алчущихъ нерадятъ; иные различными и дорогими одеждами украшаются, а о ближнихъ нагихъ нерадятъ; другіе богатые, обширные и высокіе домы и прочія строенія созидаютъ и украшаютъ, а о ближнихъ, неимущихъ гдѣ главу подклонить и упокоиться, нерадятъ; у прочіихъ сребро, злато и прочее богатство, какъ душа или животъ, въ цѣлости блюдется и хранится, но о ближнемъ, или долгами обремененномъ и истязуемомъ, или въ темницѣ за недоимки или долги сидящемъ и мучимомъ, нѣтъ радѣнія. Сіе самолюбіе и неправду въ хрістіанехъ видимъ: не токмо бо зло дѣлать, но и добра не дѣлать ближнему, неправда есть. И, что горше есть, видимъ, что многіе хрістіане красть, похищать и лукавить, льстить, лгать, обманывать, злословить, клеветать, судить, ругать, прелюбодѣйствовать и прочія обиды ближнему дѣлать, чего себѣ крайне не хотятъ, не стыдятся и не боятся, какъ выше сказано. Сіе все отъ слѣпоты происходитъ.

\paragraph*{§\:55.} Въ сей должности, то"=есть къ ближнему, слѣпотствуетъ человѣкъ потому, что должности къ Богу не внимаетъ, и въ той слѣпотствуетъ и заблуждаетъ. Богъ бо повелѣлъ, какъ зла не дѣлать, такъ добро дѣлать ближнему: сего ради человѣкъ, когда или зло дѣлаетъ, или добра не дѣлаетъ ближнему, не слушаетъ повелѣнія Божія, и Бога повелѣвающаго не почитаетъ, не боится, не любитъ; и тако, не исполняя должности къ ближнему, и къ Богу должность оставляетъ. И отъ сего видно, что должность къ ближнему отъ должности къ Богу неотлучна бываетъ, и сія безъ оной исполнитися не можетъ. Не почитаетъ тотъ Бога, кто должность къ ближнему оставляетъ; не боится Бога, кто ближнему не боится зла дѣлать; и не любитъ Бога, кто ближняго не любитъ. О семъ апостолъ поучаетъ насъ: \textit{аще кто речетъ, яко люблю Бога, а брата своего ненавидитъ, ложь есть}\footnote{1~Іоан.~4,~20.}. Ибо Богъ повелѣлъ ближняго любить: \textit{возлюбиши искренняго своего яко самъ себе}\footnote{Матѳ.~22,~39.}; и паки: \textit{сія заповѣдаю вамъ, да любите другъ друга}\footnote{Іоан.~15,~17.}. Сего ради кто искренняго своего не любитъ, тотъ не слушаетъ Бога, Который повелѣваетъ искренняго любить, и такъ не почитаетъ Его. Корень убо и источникъ всего благочестія есть истинное богопознаніе; и якоже благочестивое житіе знанія Божія свидѣтельствомъ есть, такъ грѣховное и беззаконное житіе доказательствомъ есть невѣдѣнія Божія и безбожнаго сердца, хотябъ таковый Бога и Хріста исповѣдывалъ, въ церковь ходилъ, молился, таинъ причащался, и прочіе знаки хрістіанства показывалъ.

\paragraph*{§\:56.} Понеже Богъ, яко Духъ, никакимъ чувствамъ не подлежащій, не тѣлесными, но душевными очами видится: того ради и грѣшникъ не познаетъ Его, и не зная, не почитаетъ; яко тьма грѣховная ослѣпила душевныя очи его. Откуду бываетъ, что тое дѣлаетъ, о томъ тщится, того только ищетъ, что чувствамъ его внѣшнимъ подлежитъ. Понеже видитъ отца своего, видитъ господина своего, видитъ властелина своего, то и почитаетъ ихъ, слушаетъ ихъ, услужить и угодить имъ тщится, оскорбить и прогнѣвать ихъ бережется; видитъ судъ и казнь временную, и боится тѣхъ; видитъ честь, славу и богатство міра сего, того ради и желаетъ ихъ и ищетъ. Бога же и вѣчныя славы и чести и богатства не видитъ, потому какъ Богу угождать, такъ и вѣчнаго онаго сокровища искать не старается. Кто бо чего не видитъ, или не знаетъ, тотъ того не желаетъ и не ищетъ. Меда или плодовъ земныхъ и древесныхъ не желаетъ, пока не познаетъ вкусомъ сладости ихъ: надобно прежде ихъ вкусить, каковы они, сладки или горьки, полезны или вредны намъ, и тако ихъ желать или отвращаться. Купецъ въ тую страну не идетъ, гдѣ не надѣется сыскать прибытка себѣ; отрокъ неразумный не видитъ пользы въ наукѣ, того ради и нерадитъ о ученіи. Тако и въ духовныхъ вещахъ бываетъ: кто просвѣщенія духовнаго въ себѣ о нихъ не имѣетъ, тотъ о нихъ и не старается, ихъ не желаетъ и не ищетъ ихъ. Не вкусилъ грѣшникъ, \textit{коль благъ Господь}, то и не ищетъ Его; не чувствуетъ въ сердцѣ своемъ благодѣяній Божіихъ, то и не благодаритъ Богу; не знаетъ всемогущества и величества Божія, то и не трепещетъ Его; не знаетъ истины Его, и не вѣритъ Ему; не знаетъ правды, и не боится прогнѣвлять Его; не знаетъ, коль великая сладость и покой и миръ обрѣтается въ любви хрістіанской и во всемъ благочестіи, и не тщится о томъ; не видитъ вѣчнаго блаженства, то и не ищетъ того надлежащимъ образомъ, чего не видитъ. Надобно убо прежде познать, и тогда искать, что познаешь. Надобно познать Бога въ свойствахъ Его, и тогда почитать Его: познать, что Онъ какъ всѣхъ, такъ и твой есть Создатель, Искупитель, Промыслитель, Господь и Царь твой, и тако какъ Создателя, Искупителя, Промыслителя, Господа и Царя своего почитать, слушать и волю и повелѣнія Его исполнять, якоже отца твоего, родившаго тебе, господина и царя твоего почитаешь и слушаешь. Надобно познать, что Онъ какъ всѣхъ, такъ и твой есть Богъ, Помощникъ, Защитникъ и Заступникъ, и нѣтъ инаго, кромѣ Его; и тако на Него надежду возложить, и въ день скорби и напасти къ Нему прибѣгать и призывать, и помощи и заступленія отъ Него просить и проч. Надобно безъ сумнѣнія держать, что въ будущемъ вѣкѣ будетъ блаженство вѣчное и умомъ непостижимое почитающимъ Бога, и тако возжжется къ полученію онаго желаніе и тщаніе. Тако земледѣльцы взираютъ прежде на плоды земли, и тако, надѣяся плоды собрать, трудятся въ земледѣліи; тако рудокопатели познаютъ прежде, гдѣ руда въ землѣ, или мѣдная, или желѣзная, или серебряная имѣется, и тогда копаютъ ее. Тако окомъ вѣры должно прежде увидѣть будущую жизнь, и тако вѣрою искать ея, и ожидать и надѣяться ея, якоже пишется о Моисеѣ: \textit{взираше на мздовоздаяніе: сего ради отвержеся нарицатися сынъ дщере Фараоновы; паче же изволи страдати съ людьми Божіими, нежели имѣти временную грѣха сладость, большее богатство вмѣнивъ египетскихъ сокровищъ поношеніе Хрістово}\footnote{Евр.~11,~26 и 27.}. А ради чего всю египетскую славу и богатство презрѣлъ онъ? Понеже окомъ вѣры видѣлъ будущую славу и жизнь, и тако земную, яко ничто, презрѣлъ, и оныя искалъ, \textit{взираше бо на мздовоздаяніе}. Тако и нынѣ кто будущую жизнь и славу избранныхъ Божіихъ окомъ вѣры увидитъ, тотъ неотмѣнно славу, честь, богатство и всю суету міра сего презритъ, и къ оной единой стремиться будетъ, какъ своему концу и безцѣнному сокровищу. Видиши, что всякая вещь прежде познается, и тако желается и взыскуется, якоже чего не знаемъ, того и не желаемъ и не ищемъ. А отъ сего видно, что грѣшникъ всякій потому не почитаетъ Бога, не боится и не любитъ Его, яко не знаетъ Его; и будущія жизни и славы по надлежащему не желаетъ и не ищетъ, понеже не видитъ ея. Не знаетъ же Бога и славы оныя не видитъ потому, что не имѣетъ душевныхъ очесъ, вѣрою и благодатію Его просвѣщенныхъ, которыми только едиными Богъ и слава оная видится. У таковыхъ хрістіанъ на душевныхъ очахъ мгла и тьма, какъ покрывало, лежитъ, и не допущаетъ ихъ Бога и славу Его видѣти; и дотоль будетъ лежать, доколь не обратится всѣмъ сердцемъ ко Господу, якоже о Іудеяхъ апостолъ написалъ: \textit{даже доднесь, внегда чтется Моѵсей, покрывало на сердцѣ ихъ лежитъ: внегдаже обратятся ко Господу, взимается покрывало}\footnote{2~Кор.~3,~15 и 16.}. Симъ покрываломъ очи душевныя покрыты имѣя, хотя и читаютъ и слушаютъ святое Писаніе, но не разумѣютъ, якоже о Іудеяхъ таможде глаголетъ апостолъ: \textit{ослѣпишася помышленія ихъ}\footnote{ст.~14.}, и тако слѣпотствуютъ какъ въ познаніи Божіи, такъ и въ почитаніи Его. Сія слѣпота и тьма у всякаго грѣшника на сердцѣ лежитъ, который отъ міра и грѣха къ Богу не обратился.

\paragraph*{§\:57.} Страшныя слѣпоты образъ представляетъ намъ Писаніе на ветхомъ Израилѣ. Всякія чудеса сотворилъ Господь предъ ними въ землѣ Египетстѣй, какъ читаемъ въ книгѣ Исхода и въ Псалмахъ: \textit{посла тьму и помрачи, яко преогорчиша словеса Его; преложи воды ихъ въ кровь, и измори рыбы ихъ; воскипѣ земля ихъ жабами въ сокровищницахъ царей ихъ; рече, и пріидоша песіи мухи и скнипы во вся предѣлы ихъ; положи дожди ихъ грады, огнь попаляющь въ земли ихъ; и порази винограды ихъ и смоквы ихъ, и сотры всякое древо предѣлъ ихъ; рече, и пріидоиша прузи и гусеницы, имже не бѣ числа, и снѣдоша всяку траву въ земли ихъ, и поядоша всякъ плодъ земли ихъ; и порази всякаго первенца въ земли ихъ, начатокъ всякаго труда ихъ}\footnote{Пс.~104,~28--36.} и проч. Но пророкъ святый исповѣдуется Господеви: \textit{отцы наши во Египтѣ не разумѣша чудесъ Твоихъ}\footnote{Пс.~105,~7.}, когда страшныя оныя казни единыхъ Египтянъ, враговъ Израилевыхъ, поражали, а Израильтянъ не касалися, какъ читаемъ въ книгѣ Исхода. Сіе самое какъ бы перстомъ показывало имъ особливый о нихъ промыслъ Божій и милосердіе; \textit{но отцы наши не разумѣша чудесъ Божіихъ, ни помянуша милости Божія}. Изшедшимъ изъ Египта и бѣжащимъ отъ враговъ своихъ, открылъ Господь и тамо путь, гдѣ не было пути, \textit{разверзе море, и проведе ихъ}\footnote{77,~13.}\textit{, и запрети Чермному морю, и изсяче, и преведе я въ безднѣ, яко въ пустыни}\footnote{105,~9.}. Увидѣли и враговъ своихъ водою, яко единымъ гробомъ, покрытыхъ: \textit{покры вода стужающія имъ, ни единъ отъ нихъ избысть}; воспѣли и хвалу Божію\footnote{Ст.~11 и 12.}. Но вскорѣ забыли такъ великія и страшныя Божія дѣла: \textit{ускориша, забыша дѣла Его}. Сдѣлали тельца златаго; вмѣсто Бога Творца, Отца и Избавителя своего, почтили бездушнаго идола и поклонилися дѣлу рукъ своихъ: \textit{сотвориша тельца въ Хоривѣ, и поклонишася истуканному: и измѣниши славу Его въ подобіе тельца ядущаго траву. И забыша Бога, спасающаго ихъ, сотворшаго велія во Египтѣ, чудеса въ земли Хамовѣ, страшная въ морѣ Чермнѣмъ}\footnote{13,~19--22.}. Приписали такъ чудесное благодѣяніе Божіе бездушной вещи и творенію рукъ своихъ: \textit{сіи бози твои, Израилю, изведшіе тя изъ земли Египетскія}\footnote{Исх.~32,~4.}. Страшная хула и слѣпота! Бездушная вещь, сама собою недвижимая, какую помощь можетъ подать, не токмо такъ чудесное дѣло сдѣлать!.. Но ослѣпленный Израиль простираетъ руки и вопіетъ: \textit{сіи бози твои, Израилю, изведшіе тя изъ земли Египетскія!.}. Подобная слѣпота и нынѣ въ хрістіанехъ многихъ видится, которые Бога \textit{исповѣдуютъ вѣдѣти, дѣлы же отмещутся}, какъ выше сказано. Позваны къ вѣчному животу и небесному царствію, какъ земли обѣтованной, кипящей медомъ и млекомъ; но хотятъ и стараются въ мірѣ семъ богатѣть, прославиться, царствовать, и, аще бы возможно было, во вѣки пребывати, "--- что показуется отъ ненасытнаго ихъ желанія богатства, чести, славы и излишнихъ строеній. Разсуди всякъ, не великое ли безуміе было бы того человѣка, который, находясь въ чужой странѣ и скоро имѣя возвратитися въ домъ свой, началъ бы тамо созидать богатые домы, которые бы вскорѣ принужденъ былъ оставить, и тако бы безполезный только былъ трудъ его; или того, который, точно извѣстившися, что вскорѣ, по двухъ дняхъ или двухъ седмицахъ умретъ, однакожъ собираетъ богатство, ищетъ чести и славы? Воистину всякъ бы и того и другаго несмысленнымъ назвалъ! Тако несмысленно дѣлаютъ тѣ люди, которые въ мірѣ семъ заводятъ богатыя и излишнія строенія, собираютъ великія сокровища, ищутъ чести и славы, вѣдая, что все тое вскорѣ въ странѣ сей слѣдуетъ оставить и ити въ мѣсто свое безъ всего того. Никто ничего отъ міра сего съ собою не выноситъ, ибо \textit{ничего не внесохомъ въ міръ сей, явѣ, яко ниже изнести что можемъ}\footnote{1~Тим.~6,~7.}. И хотя всякому отъ таковыхъ, которые собираютъ, а не въ Бога богатѣютъ, Божій гласъ гремитъ: \textit{безумне! въ сію нощь душу твою истяжутъ отъ тебе: а яже уготовалъ еси, кому будутъ}\footnote{Лук.~12,~20.}? "--- и хотя вси по вся дни видятъ, что богатые и нищіе, славные и безславные, почтенные и подлые равно восхищаются и на оный вѣкъ преходятъ, и восхищаются различно, то"=есть, чаянно и нечаянно, и которые надѣются долго жить, внезапу престаютъ жить, и что на другихъ видятъ, того и себѣ надѣются: однакожъ такъ богатѣть и прославиться на сей чужой странѣ стараются, какъ бы многія тысящи лѣтъ, или безъ конца жить здѣ имѣли. У таковыхъ людей мгла и тьма лежитъ на душевномъ окѣ и ослѣпляетъ тое: хотя они и богаты, не видятъ нищеты своея; хотя они и славны, не видятъ подлости и бѣдности своея; хотя и мудры суть, не видятъ неразумія и слѣпоты своея; хотя они и блаженными себе почитаютъ, но не усматриваютъ своего окаянства.

\paragraph*{§\:58.} Доколѣ человѣкъ въ таковой слѣпотѣ находится, думаетъ о себѣ, что онъ все право и разумно дѣлаетъ; но въ самой вещи всѣ его поступки, дѣла, замыслы и начинанія значатъ едино заблужденіе. Ибо сердце, отъ котораго все происходитъ, суетою исполненное и мірскою любовію напоенное, что иное, какъ только суетное замышляетъ и въ дѣйство производитъ? И таковый во всемъ подобенъ слѣпому, или во тьмѣ находящемуся, который хотя и весь замаранъ, однакожъ думаетъ, что онъ чистъ; хотя и сбился съ пути и блудитъ, однакожъ думаетъ, что онъ надлежащимъ трактомъ идетъ. И тѣмъ бѣдственнѣйшая сія слѣпота, что человѣкъ ея не усматриваетъ, познаніе бо ея есть начало духовнаго блаженства. И сія слѣпота не токмо въ простыхъ и безграмотныхъ людяхъ примѣчается, но и въ мудрыхъ и разумныхъ вѣка сего, которые нѣчто высокое о себѣ мечтаютъ и отдѣляютъ себе отъ простыхъ, безкнижныхъ и невѣждъ; ибо гдѣ неумѣренное самолюбіе и любовь міра сего имѣется, тамо и слѣпота сія мѣсто свое имѣетъ. Самолюбіе бо и мірская любовь безъ тьмы той не бываетъ.

\paragraph*{§\:59.} Ежели человѣкъ отъ бѣдственной той слѣпоты не избавится, и свѣтомъ благодати Божія не просвѣтится; то она его приведетъ къ бѣдственному состоянію и поздному, но безполезному раскаянію, въ книгѣ Премудрости Соломоновой описанному, гдѣ нечестивые, поздно познавше свое заблужденіе, въ тѣснотѣ духа воздыхаютъ, каются и глаголютъ: \textit{заблудихомъ отъ пути истиннаго, и правды свѣтъ не облиста намъ, и солнце не возсія намъ; беззаконныхъ исполнихомся стезь и погибели, и ходихомъ въ пустыни непроходимыя: пути же Господни не увѣдохомъ}\footnote{Прем.~5,~6 и 7.}.

\paragraph*{§\:60.} Чтобы намъ, хрістіанине, отъ сей слѣпоты избавиться и не пріитить къ неблагополучному оному, поздному и безполезному раскаянію, нужно чинить слѣдующее: 1)~Внимать съ прилѣжаніемъ свѣтильнику Божія слова, который свѣтитъ всѣмъ, и просвѣщаетъ любящихъ его, и взирающихъ на него, и внимающихъ ему; внимать говорю, \textit{яко свѣтилу, сіяющему въ темномъ мѣстѣ, дондеже день озаритъ, и денница возсіяетъ въ сердцахъ нашихъ}\footnote{2~Петр.~1,~19.}. Ибо слово Божіе, яко духовное свѣтило, прогоняетъ тьму и озаряетъ душевныя очи, открываетъ суету и прелесть міра, обличаетъ самолюбіе и грѣхи наши: якоже зеркало показуетъ нечистоту и пороки лица смотрящаго въ тое. "--- 2)~Дѣлать не тое, что похотливое сердце хощетъ, но чему Божіе слово учитъ, хотя тое слѣпому разуму противно и сердцу горестно; и подражать въ такъ важномъ дѣлѣ немощнымъ, которые, хотя исцѣлитися, лѣкарство непріятное и горькое пріемлютъ, дабы только отъ болѣзни свободитися: тако хотящему свободитися отъ слѣпоты, должно нудить и убѣждать себе къ тому, что слово Божіе предписуетъ, и отвращаться отъ того, что оно запрещаетъ. Ибо растлѣнному и ослѣпленному естеству нашему слово Божіе противно кажется такъ, какъ немощнымъ глазамъ свѣтъ, и въ лихорадкѣ находящимся пища, и несмысленнымъ отрокамъ наука. \textit{Душевенъ бо человѣкъ не пріемлетъ яже Духа Божія: юродство бо ему есть}\footnote{1~Кор.~2,~14.}. Но оно есть истинное и недвижимое правило, и не можетъ прельстити насъ, яко плодъ \textit{истиннаго} Бога. Сего ради должно всякому себе ввѣрить ему, и что ни предлагаетъ, тое за истину имѣть, и держаться того, яко твердаго и непоколебимаго правила, и покорять ему сердце свое, хотя не хотящее и отвращающееся. "--- 3)~Понеже вси человѣки отъ пути истиннаго заблудили, по свидѣтельству святаго Писанія: \textit{вси яко овцы заблудихомъ, человѣкъ отъ пути своего заблуди}\footnote{Ис.~53,~6; Пс.~13,~1--3 \textit{и на проч. мѣстахъ}.}: сего ради Сынъ Божій, \textit{путь и истина и животъ}\footnote{Іоан.~14,~6.}, явился на земли, \textit{и посѣтилъ насъ Востокъ свыше, просвѣтити во тьмѣ и сѣни смертнѣй сѣдящія, направити ноги наша на путь миренъ}\footnote{Лук.~1,~78 и 79.}, и святое и непорочное житіе въ правило и образъ намъ подалъ. На тое убо смотрѣть, и Ему послѣдовать должно, да отъ природныя тьмы и слѣпоты свободимся, якоже Самъ глаголетъ: \textit{Азъ есмь свѣтъ міру: ходяй по Мнѣ, не имать ходити во тьмѣ, но имать свѣтъ животный}\footnote{Іоан.~8,~12.}. По Хрістѣ ходити не иное что есть, какъ житію Его послѣдовать. Аще убо послѣдуяй Хрісту не будетъ ходити во тьмѣ, какъ глаголетъ, то неотмѣнно во тьмѣ и заблужденіи будетъ, кто Ему не послѣдуетъ. Якоже бо солнце видимое сотворилъ Господь на небеси, дабы просвѣщало всю поднебесную, тако мысленное солнце, Хріста, Сына Своего, послалъ въ міръ, дабы имъ просвѣщалися и согрѣвалися души наши. \textit{Бѣ свѣтъ истинный, иже просвѣщаетъ всякаго человѣка грядущаго въ міръ}\footnote{Іоан.~1,~9.}. Якоже убо тьма бываетъ въ поднебесной, когда солнце отходитъ отъ насъ, и человѣкъ тогда слѣпотствуетъ и осязаетъ какъ слѣпый: тако тьма послѣдуетъ въ душѣ человѣческой, и душа бѣдственно слѣпотствуетъ, и заблуждаетъ, и претыкается, и падаетъ, и не знаетъ, куды идетъ, яко тьма ослѣпила очи ея, когда Хрістосъ, Солнце праведное, съ ученіемъ и житіемъ Своимъ, которое есть свѣтъ духовный, отъ ней отлучается, то"=есть, когда душа ученію и образу житія Хрістова не внимаетъ. И какъ тьма отступать начнетъ отъ поднебесной, то всякъ просвѣщается, и \textit{исходитъ человѣкъ на дѣло свое и на дѣланіе свое до вечера}\footnote{Пс.~103,~23.}, когда солнце приближается къ поднебесной: тако тьма отъ души начинаетъ отходить, и душа просвѣщаться, и на дѣло свое и на дѣланіе свое исходитъ, то"=есть, себе познавать, свою бѣдность, окаянство, суету и прелесть міра сего и истинное блаженство, и творить хрістіанскія добродѣтели, когда къ ней Хрістосъ съ ученіемъ и житіемъ Своимъ будетъ приближаться и въ нее вселяться. Сего ради какъ окна отворяемъ, дабы солнце лучи свои впустило въ храмину, и тую бы просвѣтило: тако должно намъ храмину сердца нашего отворить, и допустить свѣту ученія и житія Хрістова внити и просвѣщать тое, когда хощемъ отъ тьмы и слѣпоты сердечной избавиться. Отъ природы вси мы слѣпы и помрачены тьмою, сего ради свѣтомъ Хрістовымъ просвѣщаться должно. Противнымъ бо противное прогонится, какъ"=то: холодъ теплотою, горечь сладостію, тьма свѣтомъ, и проч. Аще убо и мы хощемъ слѣпоту отъ сердецъ нашихъ отгнать, должны внимать ученію и примѣру житія Хрістова, которое есть свѣтъ, прогоняющій тьму нашу; и чимъ болѣе будемъ внимать и послѣдовать тому, тѣмъ болѣе просвѣтимся: якоже чимъ кто болѣе приближается къ свѣтильнику, тѣмъ болѣе просвѣщается. Хрістосъ со святымъ ученіемъ и житіемъ Своимъ есть \textit{путь}: должно убо держаться Его, да не заблудимъ; есть \textit{истина}: должно убо намъ себе ввѣрить Ему, да не прельстимся; есть \textit{животъ}: должно убо Ему прилѣпиться, держаться и послѣдовать, да не во вѣки умремъ, но да съ Нимъ и въ Немъ и чрезъ Него оживемъ, и живи будемъ во вѣки. Всякъ бо заблуждаетъ, кто отъ пути сего удаляется; и прельщается, кто истинѣ сей не вѣритъ; и мертвъ есть, кто живота сего не держится. "--- 4)~Чтобы ученію и примѣру житія Хрістова внимать и тако просвѣтиться, должно оставить грѣхи и обратиться къ Богу, ибо грѣхи суть тьма, помрачающая очи душевныя: сими душа помрачается и слѣпотствуетъ. Якоже убо тьма и свѣтъ вмѣстѣ быть не могутъ, тако Хрістово просвѣщеніе не можетъ быть въ томъ сердцѣ, которое тьма грѣховная объяла. \textit{Кое} бо \textit{общеніе свѣту ко тмѣ}\footnote{2~Кор.~6,~14.}? \textit{Сего ради}, апостолъ глаголетъ, \textit{востани спяй, и воскресни отъ мертвыхъ, и освѣтитъ тя Хрістосъ}\footnote{Еф.~5,~14.}. Видиши, что первѣе должно востать отъ сна грѣховнаго и воскреснуть отъ мертвыхъ (всякъ бо нераскаянный грѣшникъ есть живый мертвецъ), и тогда просвѣтитъ Хрістосъ. Надобно убо спящему востать и умершему воскреснуть, и тогда просвѣтится отъ Хріста: иначе просвѣщеніе не бываетъ. Не спящій бо и умершій, но воставшій и живущій свѣта требуетъ и просвѣщается. О Хрісте, Божія сило и Божія премудросте! возбуди спящихъ, которыхъ ради на крестѣ уснулъ Ты, и воскреси умершихъ, за которыхъ смерть вкусилъ Ты, животе и воскресеніе наше, и тогда поищемъ свѣта Твоего, и просвѣтимся. \textit{Господи Боже силъ! обрати ны, и просвѣти лице Твое, и спасемся}\footnote{Пс.~79,~20.}. Надобно хотящему видѣть солнце обратиться къ солнцу: тако хотящему познать и видѣть Бога, обратиться сердцемъ къ Богу. Беззаконнующіе бо и нераскаянные грѣшники, понеже сердцами ко грѣху и міру приложилися и прилѣпилися, отъ Бога отвратилися и обратилися ко грѣху и міру: како убо могутъ видѣти Его, отвратившеся отъ Него? Неотмѣнно убо должно обратиться ко Господу, когда хощемъ просвѣтиться. Сего ради апостолъ глаголетъ о Іудеяхъ: \textit{внегда обратятся ко Господу, взимается покрывало}\footnote{2~Кор.~3,~16.}. Покрывало сіе не иное что есть, какъ тьма и слѣпота, помрачающая очи душевныя человѣка необратившагося и непросвѣщеннаго, какъ выше сказано. Человѣкъ бо необращенный и непросвѣщенный не знаетъ самъ, что дѣлаетъ, и блудитъ какъ слѣпый, "--- что не отъинуду, какъ отъ покрывала онаго и природной грѣховной тьмы происходитъ. Читаемъ, что Моисей отъ глаголанія съ Богомъ на горѣ Синайской такъ просвѣтился лицемъ, что сыны Израилевы не могли на него смотрѣть; и когда къ людямъ бесѣдовалъ, лице свое покрывалъ покрываломъ; а когда къ Богу обращался, отнималъ покрывало тое\footnote{Исх.~34,~33--36.}. Тако человѣкъ всякій, доколѣ міру работаетъ, имѣетъ на сердцѣ своемъ лежащее покрывало; а когда обратится отъ міра къ Богу, когда благодатію Хрістовою отнимается тое, и самъ просвѣщается, и Бога познаетъ, и Хріста Сына Его, и тайну смотренія Его, и силу закона и Евангелія Его начнетъ разумѣти; тогда начнетъ ему свѣтъ душевный, какъ заря послѣ нощи, возсіявати, и его въ дѣлѣ спасенія просвѣщати и вразумляти. \textit{Сего ради глаголетъ: востани спяй, и воскресни отъ мертвыхъ, и освѣтитъ тя Хрістосъ}. "--- 5)~Хотящему получить отъ Хріста просвѣщеніе, неотмѣнно должно признать свою слѣпоту отъ сердца: иначе какъ былъ, такъ и будетъ всегда слѣпъ, хотя и все святое Писаніе наизусть будетъ знать, то"=есть, письмена будетъ знать, а силы не уразумѣетъ. Ибо Богъ отъ тѣхъ, которые мнятъ себе быти мудрыми и разумными, тайны Своя утаеваетъ, и открываетъ младенцамъ, то"=есть, простосердечнымъ и признающимъ свое невѣжество, по свидѣтельству Господню: \textit{утаилъ еси сія отъ премудрыхъ и разумныхъ, Отче, и открылъ еси та младенцемъ}\footnote{Матѳ.~11,~25.}. Сихъ просвѣщаетъ, вразумляетъ, умудряетъ, и имъ тайны Своя открываетъ; прочіихъ же оставляетъ, которые мудростію, разумомъ и искусствомъ своимъ хвалятся. Откуду глаголетъ Господь: \textit{на судъ Азъ въ міръ сей пріидохъ, да невидящіи видятъ, и видящіи слѣпы будутъ}\footnote{Іоан.~9,~39.}, то"=есть, признающіи свою слѣпоту Хрістовымъ свѣтомъ просвѣтятся и увидятъ; а не признающіи и за мудрыхъ себе почитающіи въ прежней своей слѣпотѣ останутся, паче же болѣе помрачатся за неблагодарность и гордость, каковые были во время житія Хрістова на земли книжники и фарисеи, которые говорили Хрісту: \textit{еда и мы слѣпи есмы}\footnote{ст.~40.}? Гордость бо и высокоуміе, яко діавольское дѣло, какъ его самого помрачило и изъ ангела свѣтоносна тьмою учинила: такъ и людей, послѣдующихъ ему, подобными ему дѣлаетъ. Хрістосъ какъ оправдаетъ только тѣхъ, которые признаютъ и исповѣдуютъ себе грѣшниками, и исцѣляетъ тѣхъ, которые признаютъ свою болѣзнь, такъ и просвѣщаетъ только слѣпыхъ, какъ Евангеліе свидѣтельствуетъ: \textit{не требуютъ бо здравіи врача, но болящіи}\footnote{Матѳ.~9,~12.}. Аще убо хощемъ просвѣтитися, признаемъ нашу слѣпоту, и просвѣтитъ насъ Хрістосъ. Аще хощемъ оправдатися, и тако спастися, признаемъ насъ грѣшниками и погибшими, и оправдаетъ и спасетъ насъ Хрістосъ. Аще хощемъ исцѣлитися, признаемъ нашу немощь, и исцѣлитъ насъ Хрістосъ. "--- 6)~Какъ безъ обращенія отъ грѣха ко Господу и истиннаго покаянія, якоже выше сказано, такъ и безъ усердной молитвы просвѣщеніе истинное не бываетъ. Просвѣщеніе сіе отъ Бога происходитъ. Сего ради обратившемуся и кающемуся нужно молитися и воздыхати со Псаломникомъ: \textit{призри, услыши мя, Господи и Боже мой, просвѣти очи мои, да не когда усну въ смерть}\footnote{Пс.~12,~4.}; и паки: \textit{Ты просвѣтиши, свѣтильникъ мой, Господи, Боже мой, просвѣтиши тьму мою}\footnote{17,~29.}; и паки: \textit{посли свѣтъ Твой и истину Твою, Боже; та мя наставятъ, и введутъ мя въ гору святую Твою, и въ селенія Твоя}\footnote{42,~3.}; и паки: \textit{открый очи мои, Господи, и уразумѣю чудеса отъ закона Твоего}\footnote{118,~18.}; и со слѣпцами евангельскими: \textit{Господи, да отверзутся очи мои душевныя}\footnote{Матѳ.~20,~33.}; каковыхъ молитвъ и въ церковныхъ книгахъ, много имѣется. Тогда Господь, Который, есть свѣтъ міру, и просвѣтилъ тѣлесныя слѣпымъ очи, видя, наше обращеніе къ Нему и тщаніе и усердіе, просвѣтитъ наипаче душевныя очи ко спасенію, на что и въ міръ сей пришелъ. "--- 7)~Къ сему съ помощію Божіею пособствуетъ размышленіе о времени и вѣчности, о суетѣ міра сего, яко все, что ни имѣется въ мірѣ семъ дорогое, ничто есть противу вѣчнаго добра; и что все преходитъ и отъ любителя отступаетъ: отъ богатаго богатство, отъ честнаго честь, отъ славнаго слава, и сластолюбца сластолюбіе отходитъ. Кончина бо всѣхъ равными дѣлаетъ. Едина добродѣтель отъ человѣка неотлучна, и съ нимъ на оный свѣтъ сопутствуетъ и къ лицу Божію приводитъ его, и милость ходатайствуетъ. О чемъ ниже скажется, а паче въ § 67"~мъ.

\paragraph*{§\:61.} Просвѣщенію истинному послѣдуютъ: 1)~Когда сія спасительная луча блеснетъ и начнетъ сіяти въ храминѣ сердца, тогда просвѣщаемый увидитъ, коль великою тьмою объятъ былъ, коль слѣпъ и безуменъ былъ, хотя и мудрымъ себѣ казался; коль далеко отъ пути истиннаго блудилъ, хотя и право ступати мнился. На таковомъ слово Господне исполнится: \textit{егда возвратився воздохнеши, тогда спасешися, и уразумѣеши, гдѣ еси былъ}\footnote{Ис.~30,~15.}; то"=есть, обратившійся и просвѣщаемый познаетъ, какъ въ бѣдномъ, плачевномъ и погибельномъ состояніи былъ, пока луча благодати Божія не коснулася сердца его. Тогда таковый дознаетъ на себѣ тое, что человѣкъ тотъ, который, въ ночи идучи по дорогѣ не той, по которой должно было ему итить, думаетъ, что онъ надлежащимъ трактомъ идетъ; но когда денница возсіяетъ, и просвѣтитъ все, познаетъ, что онъ съ надлежащаго пути совратился и блудитъ; и такъ, видя свое заблужденіе, и время напрасно потерянное, и свой безполезный трудъ, весьма жалѣетъ и окаеваетъ себе. Тако, грѣшникъ, пока естественнымъ разумомъ и мнимою своею мудростію довольствуется, думаетъ, что онъ хорошо дѣлаетъ и путемъ правымъ идетъ; но когда денница Божія благодати возсіяетъ ему, "--- узнаетъ свою прелесть и заблужденіе, жалѣетъ, кается и сокрушается, что время напрасно потерялъ, весь трудъ его безполезенъ былъ, и плачется съ пророкомъ: \textit{исчезоша, яко дымъ, дніе мои}; и паки: \textit{дніе мои, яко сѣнь уклонишася}\footnote{Пс.~101,~4 и 12.}. Воистину достойна плача вина! воистину бо погибаетъ время, которое на суету проживается! воистину безполезенъ есть трудъ, который ради снисканія чести, славы и богатства міра сего воспріемлется! подлинно исчезаютъ дніе тѣ, которые въ сластолюбіи, въ роскошахъ, въ плотоугодіи и въ веселостяхъ міра сего провождаются! \textit{Кая бо польза человѣку, аще міръ весь пріобрящетъ, душу же свою отщетитъ! или что дастъ человѣкъ измѣну за душу свою}\footnote{Матѳ.~16,~26.}? \textit{Ничтоже бо внесохомъ въ міръ сей, явѣ, яко ниже изнести что можемъ}\footnote{1~Тим.~6,~7.}. \textit{Наги исходимъ изъ чрева матере своея, наги и отходимъ} отъ міра сего\footnote{Іова 1,~21.}. \textit{Не убойся, егда разбогатѣетъ человѣкъ, или егда умножится слава дому его; яко внегда умрети ему, не возметъ вся, ниже снидетъ съ нимъ слава его}\footnote{Пс.~43,~17 и 18.}. Гдѣ нынѣ славные и богатые преждебывшіе, которые ненасытно сихъ сокровищъ въ мірѣ семъ желали и искали? Имѣются на своихъ мѣстахъ и ожидаютъ общаго суда и по дѣломъ воздаянія: туды же идутъ и нынѣшніи славолюбцы и прочіи міролюбцы. Правильна убо плача вина, когда кто въ суетѣ провождалъ дни свои: надобно плакать тому, кто \textit{собиралъ себѣ, а не въ Бога богатѣлъ}\footnote{Лук.~12,~21.}. Паче же тому плакать и сѣтовать должно, кто во грѣхахъ и беззаконіяхъ дни своя провождалъ, ибо время сокращается, дни преходятъ, житіе кончится, смерть невидимо приближается и восхититъ, судъ праведный наступаетъ, гдѣ за слова, дѣла и помышленія человѣки судимы будутъ. Поплачемъ убо, грѣшниче, поплачемъ здѣ, да не тамъ безполезно плакати будемъ, и прочее время въ покаяніи и благочестіи поживемъ. Сего ради просвѣщаемый, позная, что все, что ни есть въ мірѣ семъ, есть какъ дымъ исчезающее и со днями и временемъ проходитъ, и поминая прежнее свое житіе, яко въ суетѣ, заблужденіи и грѣхахъ прожитое, жалѣетъ, кается, воздыхаетъ и плачетъ, какъ выше сказано, и желаетъ, чтобы тое возвратить; но уже, что сдѣлалось и прошло, иначе быть не можетъ. Время бо прошедшее не возвращается, и дѣло сдѣланное и слово сказанное иначе не бываетъ, какъ было. "--- 2)~Усердно Богу благодарить за милость Его, показанную ему, что потерпѣлъ окаянству его, что въ такомъ заблужденіи не восхитилъ его, и что подалъ ему просвѣщеніе и разумъ спасенія. "--- 3)~Вѣдая окаянство и слабость свою, что человѣкъ самъ въ себѣ удобно преклоненъ есть къ паденію, заблужденію и всякому злу, молится усердно Богу, чтобы подалъ ему помощь Свою, и не оставилъ его единаго безъ Своей благодати; и самъ опасно поступаетъ и бережется, чтобы въ прежнее не впасть заблужденіе. "--- 4)~Всякаго грѣха бережется, и подвизается противу всякаго грѣха, и какъ прежде легко было ему грѣшить, такъ уже въ семъ состояніи тяжко ему и въ маломъ согрѣшить и раздражить Бога и обезпокоить совѣсть. Вѣдаетъ бо, что всякимъ грѣхомъ Богъ прогнѣвляется, и согрѣшающій лишается милости Его. "--- 5)~Таковый человѣкъ обидѣть никого не хощетъ ни дѣломъ, ни словомъ, но паче всякаго нелицемѣрно любить тщится, и какъ себѣ, такъ и всякому всякаго добра желаетъ, и хощетъ всѣмъ, какъ и себѣ спастися, о чемъ и молится. "--- 6)~Со всякимъ человѣкомъ не лестно, ни лукаво, но просто обходится; что словомъ объявляетъ, тое на сердцѣ имѣетъ, и потому солгать и обмануть никого не хощетъ. "--- 7)~Міра сего честь, славу, богатство, сластопитаніе и все, что въ мірѣ семъ красное, дорогое и пріятное отъ сыновъ вѣка сего почитается, презираетъ, вѣдая, что все сіе какъ приходитъ, такъ скоро и отходитъ отъ насъ, якоже мечтаніе сонное. "--- 8)~Единое вѣчное блаженство за истинное блаженство почитаетъ, которое, сысканное единожды, никогда не потеряется, и единожды потерянное, никогда не сыщется. Сего ради о томъ думаетъ и тщится, какъ бы того не лишиться; туды мысли и воздыханія своя возводитъ, о томъ и Богу молится, да благодатію Своею сподобитъ его тое получити. "--- 9)~Понеже ничего дорогаго въ мірѣ семъ не признаетъ просвѣщаемый, то всѣ вещи, въ мірѣ семъ находящіяся, равно почитаетъ. Ему сребро, злато и прочіе, по мнѣнію человѣческому, дорогіе металлы суть какъ мѣдь, желѣзо, олово, свинецъ, и проч. Онъ каменіе дорогое, которое люди въ сундукахъ и сокровищницахъ своихъ хранятъ, равно вмѣняетъ какъ и тое, которое по дорогамъ валяется и ногами попирается. Его равно питаетъ какъ богатая, такъ и убогая пища, только бы была здоровая. Онъ равно какъ шелковою, такъ и суконною, какъ цвѣтною, такъ и темною одеждою, какъ лисьею, такъ и овчинною кожею покрывается и согрѣвается. Ему равно жить какъ въ каменномъ, такъ и въ деревянномъ покоѣ, какъ въ богатыхъ палатахъ, такъ и въ убогой хижинѣ и проч. Понеже онъ всего, что ни имѣется въ мірѣ семъ, ради нужды только употребляетъ, и ни къ чему сердца своего не прилагаетъ, но къ единому только Создателю своему. "--- 10)~Житіе сіе имѣетъ какъ путь, которымъ къ желанной идетъ вѣчности, и какъ странствованіе, которое скоро надобно окончить, глаголя: \textit{пресельникъ азъ есмь у Тебе, Боже, и пришлецъ, якоже вси отцы мои}\footnote{Пс.~38,~13.}. Сего ради, яко путникъ и странникъ, мірскими вещами не обременяетъ себе, но довольствуется тѣмъ, что имѣетъ, помышляя апостольское слово: \textit{ничтоже внесохомъ въ міръ сей, явѣ, яко ниже изнести что можемъ; имѣюще же пищу и одѣяніе, сими довольни будемъ}\footnote{1~Тим.~6,~7 и 8.}. "--- 11)~Чимъ болѣе просвѣщается, тѣмъ болѣе исправляетъ свои поступки, исправляетъ свои дѣла, слова и помышленія, усматриваетъ пороки и малые, и бережется. И якоже чимъ болѣе естественнымъ и чувственнымъ свѣтомъ просвѣщается человѣкъ, тѣмъ яснѣе усматриваетъ все, дорогу, ровъ, лежащія по дорогѣ и около вещи, и прочая; яснѣе бо видимъ при денницѣ возсіявшей, нежели при начинающей только восходити, и днемъ, нежели на зарѣ: тако яснѣе видѣть все начинаетъ, и самые малые пороки усматриваетъ, и уклоняется ихъ, кто вышеестественнымъ благодати Божія свѣтомъ болѣе и болѣе просвѣщается. И уже ему праздное слово сказать не безбѣдно, которому прежде хулить, ругать, злословить, сквернословить, лгать, льстить, обманывать, красть, похищать и прочіе тяжкіе грѣхи творить легко было. "--- 12)~Ежели въ чемъ проступится и отъ немощи согрѣшитъ, и въ совѣсти своей усмотритъ грѣхъ свой: весьма безпокойствуется и окаеваетъ себе, ужасается и болѣзнуетъ сердцемъ. Сего ради смиряется и повергаетъ себе предъ величествомъ Божіимъ, и со смиреніемъ проситъ оставленія, признавая свою винность, и въ надеждѣ милосердія Божія утверждается. "--- 13)~Вѣдая слѣпоту ума человѣческаго, которою часто прельщается человѣкъ и думаетъ, что онъ на добромъ пути находится, но обманывается въ мнѣніи своемъ, "--- сего ради усердно и часто молится болѣе и болѣе просвѣтитися, и, не вѣря ласкательству плоти своея, въ правительство и руководство Богу себе предаетъ и проситъ: \textit{искуси мя, Боже, и увѣждь сердце мое; истяжи мя, и разумѣй стези моя и виждь, аще, путь беззаконія во мнѣ; и настави мя на путь вѣченъ}\footnote{Пс.~138,~23 и 24.}.

\subsection[Глава 4-я. О суетномъ и прелестномъ украшеніи.]{глава четвертая.\\\bfseries О суетномъ и прелестномъ украшеніи.}

\begin{quotation}\textit{И сотвори Господь Богъ Адаму и женѣ его ризы кожаны, и облече ихъ}\footnote{Быт.~3,~21.}.\end{quotation}
\begin{quotation}\textit{Имже}, то есть, \textit{женамъ, да есть не внѣшняя плетенія власъ, и обложенія злата, или одѣянія ризъ лѣпота, но потаенный сердца человѣкъ, въ неистлѣніи кроткаго и молчаливаго духа, еже есть предъ Богомъ многоцѣнно}\footnote{1~Петр.~3,~3 и 4; 1~Тим.~2,~9 и 10.}.\end{quotation}


О семъ украшеніи здѣ полагается разсужденіе потому, что тое отъ ослѣпленнаго ума происходитъ, и что тѣмъ ослѣпленный человѣкъ ищетъ себѣ славы и чести суетной отъ одежды, которая должна его гордость низлагать и къ смиренію приводить, какъ ниже покажется; а паче личное женъ украшеніе безъ всякаго извиненія есть ослѣпленіе ума и развращеніе плотскаго и нехрістіанскаго сердца.

\paragraph*{§\:62.} Одежда началася по преступленіи заповѣди Божія, и отъ того начало свое воспріяла. Ибо доколѣ не согрѣшили прародители наши, не имѣли одежды, какъ пишется въ главѣ 2"~й Бытія; и не требовали ея, яко не видѣли наготы своея: а когда согрѣшили, увидали наготу свою и начали срамлятися; посему и искать принуждены были, чимъ бы срамоту свою прикрыть. \textit{И отверзошася очи обѣма, и разумѣша, яко нази бѣша; и сшиста листвіе смоковное, и сотвориста себѣ препоясанія}\footnote{Быт.~3,~7.}. Богъ же, милосердствуя о нихъ, \textit{сотворилъ имъ ризы кожаны, и облече ихъ}\footnote{ст.~21.}. Видиши убо, что начало одежды есть грѣхъ. Грѣхъ бо открылъ имъ наготу ихъ, и содѣлалъ срамоту и стыдѣніе, которое требуетъ прикрытія. Откуду Андрей святый Критскій сѣтуетъ о семъ: «сшиваше кожаны ризы грѣхъ мнѣ, обнаживый мя первыя боготканныя одежды»\footnote{\textit{Пѣснь 2"~я велик. канона}.}. Отсюду видишь, какъ худо дѣлаютъ люди, которые ищутъ украшенія въ одеждахъ, и дѣлаютъ одежду не ради прикрытія наготы и согрѣтія, но ради украшенія и щегольства; тое поставляютъ себѣ въ честь и украшеніе, что показуетъ безчестіе и безобразіе; отъ того ищутъ похвалы, что обличаетъ грѣхъ; съ того утѣшаются, что представляетъ печаль, то"=есть, знаменіе законопреступленія; тѣмъ гордятся, чѣмъ должно ихъ смирять. Всѣ животныя, какъ созданы отъ Бога, такъ и имѣются донынѣ, не требуютъ прикрытія, естественнымъ одѣяніемъ довольствуются. Единъ человѣкъ лишился того за грѣхъ, и отъ нихъ заимствуетъ и прикрывается. Но то чудно, или паче сожалительно, что чужею кожею прикрывается, и тѣмъ возносится, и тое въ честь себѣ поставляетъ, что должно было ему, яко бѣдному, скудному, нагому и лишенному, случай подавать къ смиренію. Но не довольно было страстной и похотливой плоти выдумать ради себе украшенія, и тѣмъ любоваться и утѣшаться. Чтожъ она вымышляетъ и замышляетъ? Надобно коней, кареты и прочій къ тому припасъ украшать, чтобы оттуду нѣкую себѣ похвалу и честь отъ незнающихъ и такой суетѣ удивляющихся сыскать. О разумная, но убогая и бѣдная тварь! не примѣчаеши ли, како міръ прельщаетъ тебе, какъ Еву яблоко заповѣданное, и суета его плѣняетъ несмысленное сердце твое? Аще видимая красота нравится тебѣ, обрати очи твои на красное солнце, луну и звѣзды, на поля, древа, травы и цвѣты, на птицы, рыбы и прочія животныя и прочая Божія созданія, и удивляйся имъ, и отъ тѣхъ познавай Создателя ихъ и твоего, и красоту Его. Аще Онъ такъ красныя созданія сотворилъ, какъ несравненно красенъ Самъ, Который создалъ сія! Сея красоты ищи, которая всѣхъ красотъ виновна, которыя наслаждающіися сытости не знаютъ, но чѣмъ болѣе видятъ ее, тѣмъ болѣе желаютъ ея. Но нѣтъ къ тому охоты, не нравится тое, что само въ себѣ красно, яко суетной похвалы и чести не приноситъ!.. Надобно тое дѣлать и о томъ тщиться, что міръ похваляетъ, почитаетъ, прославляетъ и удивляется, и отъ того самому похваляемымъ, почитаемымъ и прославляемымъ быть!.. Къ сему ослѣпленію страстная и слѣпая плоть и прелесть міра приводитъ бѣднаго человѣка.

\paragraph*{§\:63.} Украшеніе сіе, какъ есть суетное, такъ знаменіе есть нерадѣнія о души. Понеже 1)~кто печется о тѣлесномъ украшеніи, тому недосугъ пещися о душевномъ. Въ чемъ бо занята у кого мысль, о томъ и тщится, того и ищетъ, въ томъ и время проводитъ. \textit{Идѣже бо сокровище ваше, ту и сердце ваше будетъ}\footnote{Матѳ.~6,~21.}. "--- 2)~Самое сіе украшеніе щегольское показуетъ сердце, желающее суетныя чести и славы, что и закону Божію противно: \textit{не любите міра, ни яже въ мірѣ}\footnote{1~Іоан.~2,~15.}, и вѣрѣ хрістіанской не сходно, которая на вѣчную честь и славу взираетъ, и тоя ищетъ и ожидаетъ; и званію хрістіанъ неприлично, которые къ вѣчнымъ и небеснымъ позваны; и должности ихъ противно, которые душу, а не тѣло украшать должны, и \textit{горняя мудрствовать, а не земная}\footnote{Кол.~3,~2.}. "--- 3)~Украшеніе сіе безъ обиды ближняго, и потому безъ оскорбленія Божія быть не можетъ. Ибо не можетъ оно быть безъ иждивенія, которое не отъинуду, какъ отъ подобныхъ людей сбирается, или паче сдирается. "--- 4)~Хотя и праведное имѣніе будетъ, но понеже суть нагая братія, суть трясущіися отъ хладу и мразу уды Хрістовы, суть въ темницѣ за долги и оброки заключенніи, суть лишившіися домовъ отъ пожарныхъ случаевъ, суть и прочіи бѣдствующіи: на таковыхъ должны излишность и избытокъ употребляемы быть. Но все тое суетнымъ украшеніемъ и ненасытною роскошію пресѣкается. До нищихъ ли тому, у кого суета сія въ сердцѣ мѣсто свое имѣетъ? Надобно украшать себе, жену, дѣтей, слугъ, коней, кареты, уборъ ихъ, домъ свой, галлереи, пруды и прочія увеселенія созидать!..

\paragraph*{§\:64.} Большая еще суета и срамъ хрістіанству есть, что жены бѣлилами, красками и мастьми лица своя намазуютъ. Ибо ради чего онѣ дѣлаютъ сію бездѣлицу? Причины другой сыскать невозможно, какъ только чтобы людямъ показаться, или, что хуждше того, *понравиться* и въ любовь войтить. Въ ложницахъ онѣ о сей бездѣлицѣ не пекутся; краски здравія не придаютъ, развѣ только вредятъ. Едина причина сія есть, чтобы люди глаза свои на нихъ обращали. Правда, обращаютъ глаза свои люди, но многіи съ глазами и сердца обращаютъ, и отъ того око душевное весьма помрачаютъ. Не малая юнымъ и неутвержденнымъ сердцамъ брань отъ прелести сея належитъ. Но ежели онѣ, жены, глаголю, которыя Бога и Хріста Сына Божія исповѣдуютъ, осмотрятся, то увидятъ, что 1)~выставляютъ на торгъ тое, что непродаемое есть, что хранить должно такъ, какъ зѣницу ока. Ахъ, бѣдная тая хрістіаныня, которая лице свое румянитъ, но душу свою помрачаетъ; лице украшаетъ, но души своея благообразіе теряетъ, и, какъ чудовище, душею предъ Богомъ и ангелами Его святыми является. "--- 2)~Таковыхъ лицъ своихъ украшеніемъ, кромѣ того, что цѣломудріе потеряли, переправляютъ Божіе дѣло, которое есть совершенное и не требуетъ исправленія, и потому противу Создателя своего, Который возрастъ тѣла и доброту лица всякому свою даруетъ, весьма грѣшатъ, и обиду Ему дѣлаютъ, подобно тому невѣждѣ, который образъ, добрѣ отъ живописца написанный, переправилъ бы, и тѣмъ живописца не мало бы обидѣлъ: тако и жены, переправляя лица своя, немалую обиду и досажденіе дѣлаютъ Создателю своему. Приличнѣе хрістіанынямъ умывать лица своя слезами, нежели раскрашивать бѣлилами и красками. Хрістіанинъ бо во всегдашнемъ покаяніи находиться долженъ, яко всегда предъ Богомъ согрѣшаетъ, что тяжко и жалобно хрістолюбивой душѣ. "--- 3)~Худо онѣ дѣлаютъ, что такъ на обѣды, вечери и прочія мѣста ходятъ, но хуже того, что съ тоюжде бездѣлицею и въ храмы святые дерзаютъ входить, и себе входящимъ выставлять и показывать. Тако бо онѣ храмы святые позорищемъ, или безчестнѣе позорища, чего изображать слухъ не терпитъ, дѣлаютъ. Всякъ сіе удобно можетъ разумѣть, когда въ разсужденіе все возметъ. Таковымъ безстыднымъ женамъ приличествуетъ оное Божіе слово: \textit{храмъ Мой, храмъ молитвы наречется: вы же сотвористе его вертепъ разбойникомъ}\footnote{Матѳ.~21,~13.}. Храмъ Божій есть храмъ молитвы, и ради того въ храмъ Божій входимъ, чтобы молиться: не другъ на друга смотрѣть, но къ единому Богу сердечныя очи возводить; не себе другимъ показывать, но совѣсть свою, грѣхами обремененную, предъ Богомъ обнажать; не ликовать, но о грѣхахъ Бога умилостивлять; словомъ, молиться Богу приходимъ въ храмы святые. Молитвѣ же не такая, какую онѣ на себѣ имѣютъ, утварь приличествуетъ: а какая? Смиреніе, сокрушеніе сердечное, умиленіе, слезы, плачь. Сими молитва утварьми украшается, и къ престолу Божію восходитъ, и очесамъ Божіимъ благопріятна бываетъ, и желаніе свое получаетъ. Тако молилися святыя жены: \textit{Анна}, матерь Самуила пророка\footnote{1~Цар. гл.~1"~я.}, \textit{Іудиѳь}\footnote{Гл.~9"~я.}, \textit{Есѳирь}\footnote{4,~17.}, и услышаны отъ Бога. Симъ послѣдовать должны жены, которыя имя Хрістово исповѣдуютъ, когда хотятъ съ пользою душъ своихъ входить въ церковь и молитися. Оная же утварь, которая состоитъ изъ злата, бѣлилъ, красокъ, мастей и прочіихъ матерій, не ино что значитъ, какъ тщеславіе, пышность, гордость міра сего, надмѣніе, око лукаво, соблазнъ, разжженіе и прочія страстныя плоти прихоти. Съ сими ли убо предъ Богомъ являться? Къ Богу бо приходимъ, и предъ Богомъ стоять хощемъ, когда приходимъ на молитву. Тако ли Бога умилостивлять должно? тако ли смиряться предъ Нимъ? Сіе есть прелесть, а не молитва; гордость и пышность, а не смиреніе; умноженіе грѣховъ, а не умаленіе; большее оскорбленіе и прогнѣваніе, а не умилостивленіе милосердаго Бога, \textit{яко, еже есть въ человѣцѣхъ высоко, мерзость есть предъ Богомъ}\footnote{Лук.~16,~15.}. Аще бо въ сердца таковыхъ проникнуть, что, аще не гордость житейская, какъ высокій идолъ, крыется? «Что глаголеши, обличаетъ ихъ Златоустъ? Приходящи ли молитися Богу, облачаешися златомъ и плетеньми? еда ликовати пришла еси, еда браку пріобщитися? еда на явленіе пришла еси? Тамо злато, тамо плетеніе, тамо ризы многоцѣнныя имѣютъ время: а нынѣ ничего отъ сихъ непотребно. Пришла ты просити, молитися о грѣхахъ, молитву сотворити, о чемъ согрѣшила ты, просити Владыку, милостива Того сотворити: почто убо украшаеши себе? Не суть сія молящіяся образы. Како возможеши воздохнути, како возможеши прослезитися, како съ прилѣжаніемъ помолитися, сицевымъ образомъ обложившися? И аще слезы проліеши, смѣхъ будутъ слезы зрящимъ. Не златомъ бо одѣятися подобаетъ плачущей. Понеже сѣнь есть и лицемѣріе. Како бо не сѣнь, аще отъ тогожде сердца, отъ котораго толикое изнуреніе и любочестіе происходитъ, и слезы проливаются?» и проч.\footnote{\textit{Бес.~8"~я на} 1"~е къ Тим. посл.}

\paragraph*{§\:65.} Чтобы украшеніе суетное, которое хрістіанамъ крайне не приличествуетъ, и душепагубную роскошь, которая, какъ видно, часъ отъ часу въ хрістіанехъ усиливается и умножается, удобнѣе оставить, и умѣренно житіе провождать, пользуетъ примѣчать и разсуждать слѣдующая: 1)~Всякое излишество тѣлесное безъ оскорбленія Божія и обиды ближняго не бываетъ, какъ выше сказано. 2)~Всякое добро, какое ни имѣемъ, Божіе есть, а не наше: наги бо вышли изъ чрева матерня, и потому нищи и бѣдны въ себѣ есмы; а что ни имѣемъ, отъ Бога данное имѣемъ. Богатство наше Божіе есть добро, а не наше собственное: того ради должно расходъ его чинить по воли давшаго Бога, а не по прихотямъ нашимъ. Истязуешь ты прикащика своего, куды и на что онъ деньги, данныя ему отъ тебе, издержалъ: истяжетъ и тебе Господь о данномъ тебѣ отъ Него богатствѣ. Готови убо Ему отвѣтъ. Не спроситъ Онъ у тебе: имѣлъ ли ты богатое и красное платье и богатые домы, кони, кареты и прочая, часто ли принималъ гостей, и прочая? Нѣтъ, ничего того не видно въ Писаніи святомъ. Но что, чего Онъ у тебе спроситъ? Спроситъ: питалъ ли ты Его добромъ, тебѣ даннымъ, алчныхъ? одѣвалъ ли нагихъ? дѣлалъ ли домы неимущимъ гдѣ главы подклонить? принималъ ли въ домъ странныхъ и учреждалъ ли ихъ? выкупалъ ли сѣдящихъ въ темницахъ за долги и прочія требованія, и прочіимъ бѣднымъ и неимущимъ удѣлялъ ли отъ даннаго тебѣ добра\footnote{Матѳ.~25,~35 и 36.}? Дается бо намъ богатство не ради насъ единыхъ, но и ради ближнихъ нашихъ, которые того требуютъ. Видишь, какъ неправедно дѣлаютъ люди, когда тое добро, которое дано на общую пользу, на свои прихоти расточаютъ, и такъ воли Божіей противятся. Таковые богачи расточители суть богатства, а не строители; а которые хранятъ въ сундукахъ богатство свое, тѣ стражи его суть и раби, а не Божіи слуги и раби. Сего ради надобно разсмотрѣть всякому богачу, куды и на что богатство, ему отъ Бога данное, держитъ. Надобно всегда помнить оный судіи праведнаго страшный гласъ: \textit{идите отъ Мене, проклятіи, во огнь вѣчный, уготованный діаволу и аггеломъ его. Взалкахся бо, и не дасте Ми ясти}, и проч.\footnote{ст.~41--43.}; "--- и Авраамовъ отвѣтъ богачу, который облачашеся въ порфиру и вѵссонъ, веселяся на вся дни свѣтло, реченный: \textit{чадо, помяни, яко воспріялъ еси благая твоя въ животѣ твоемъ, и Лазарь такожде злая: нынѣ же здѣ утѣшается, ты же страждеши}\footnote{Лук.~16,~25.}. Алчетъ Хрістосъ, подающій, всѣмъ богатство, алчетъ въ удахъ Своихъ "--- нищихъ хрістіанехъ; но отъ богачей сластолюбивыхъ и скупыхъ безстыдно презирается! Лежатъ и нынѣ многіи Лазари предъ вратами богатыхъ, но пагубная роскошь ослѣпляетъ имъ глаза не видѣть оныхъ, и удерживаетъ руку ихъ отъ творенія милости съ ними. О, роскошь и скупость "--- противныя сестры, но обѣ смертоносно заражаютъ человѣческія сердца. Едина расточать, другая хранить и стрещи богатство учитъ, но обѣ на погибель человѣческую; едина разслабляетъ, другая связуетъ человѣка, но и тая и другая умерщвляетъ душу его. "--- 3)~Тѣло, которое вскорѣ въ прахъ и землю обратится, украшать, а о душѣ безсмертной нерадѣть, великое есть безуміе, какъ всякому сіе видно. Невозможно бо тому не нерадѣть о душѣ, который тѣло свое на показаніе и тщеславіе украшаетъ. Знакъ бо сердца міролюбительнаго, тщеславнаго и гордаго таковое украшеніе, которое душу потемняетъ. Едина бо есть добродѣтель украшеніе души. Сею она, какъ утварію себѣ приличною, украшается; а оною, яко тщеславною и гордостною, помрачается. "--- 4)~Женамъ таковымъ слѣдуетъ отвѣтъ дать праведному Судіи за соблазнъ. \textit{Горе бо человѣку тому, имже соблазнъ приходитъ}\footnote{Матѳ.~18,~7.}. Не малая бо брань юному сердцу бываетъ отъ лица женска, кольми паче отъ таковаго лица, которое на прельщеніе мастями и красками украшается. Ибо діаволу нѣтъ лучшаго и удобнѣйшаго орудія къ прельщенію юныхъ сердецъ и ловленію въ сѣть нечистоты, какъ лице женское, а паче искусно устроенное, и вонями и мастями намазанное. Таковое лице есть сильная стрѣла, которою онъ ударяетъ въ юныя сердца и уязвляетъ многихъ. Сего ради женамъ, которыя чаютъ на судъ Хрісту предстать и о всемъ отвѣтъ отдать Ему, должно внимать Хрістову оному страшному слову: \textit{горе человѣку тому, имже соблазнъ приходитъ}! Единаго человѣка соблазнить страшно, яко всякій человѣкъ кровію Хрістовою искупленъ: кольми паче многихъ, за которыхъ кровь Хрістова изліяна. А сколько таковыхъ имѣется, которые сею притворною красотою прельщаются и погибаютъ? Единъ сердцевѣдецъ Богъ знаетъ, Который какъ прельстившихся, такъ и прельстившихъ будетъ судить и никакого лица не пріиметъ. Внимай сему, жено, которая имя Хрістово исповѣдуешь, и чаешь на судъ Хрістовъ предстать; а хотя и не чаешь, однакожъ неотмѣнно предстанешь. \textit{Всѣмъ бо явитися намъ подобаетъ предъ судищемъ Хрістовымъ, да пріиметъ кійждо, яже съ тѣломъ содѣла, или блага или зла}\footnote{2~Кор.~5,~10.}. "--- 5)~О, когда бы жены, именемъ хрістіанскимъ называющіяся, часто смотрѣли умными глазами на простое лице Хрістово, которое ради грѣховъ и беззаконій нашихъ оплевано и біено было, и не имѣло вида и доброты: никогда бы не захотѣли мазать и красить лицъ своихъ на прельщеніе душъ человѣческихъ, которыхъ Онъ смертію Своею искупилъ! Но видно, что забыли онѣ великое оное и страшное дѣло, а только о единой суетѣ помышляютъ, и что въ сердцѣ имѣютъ, тое и вонъ выставляютъ. "--- 6)~Такожде, когда помнить будутъ, что по украшенному лицу нѣкогда черви будутъ ходить и ползать, и которое нынѣ мастями воняетъ, смрадъ несносный будетъ издавать, а потомъ въ прахъ и землю обратится со всѣмъ тѣломъ, "--- уповательно, что отъ сего душепагубнаго неистовства воздержатся. "--- 7)~Какъ сіе украшеніе женское Богу ненавистно, можетъ всякъ прочитать и видѣть третію главу Исаіи пророка, и отъ той усмотрѣть. "--- 8)~Единое истинное и хрістіанское украшеніе есть украшеніе душевное, къ которому и Божіе слово увѣщаваетъ насъ, о которомъ въ слѣдующемъ параграфѣ увидимъ.

\paragraph*{§\:66.} Хрістіанамъ, которые на святомъ крещеніи отреклися міра и суеты міра, и очистилися отъ сквернъ грѣховныхъ тою святою банею, и обѣщалися \textit{благочестно жити о Хрістѣ Іисусѣ, и чаютъ воскресенія мертвыхъ и жизни будущаго вѣка}, какъ и въ Сѵмволѣ значится, должно, по силѣ обѣтовъ своихъ, тогда учиненныхъ, плоды крещенія показывать, и потому душу свою, а не тѣло украшать. Понеже 1)~душа безсмертна, и красота ея: тѣло тлѣнно, и красота его, какъ сіе всѣмъ извѣстно. "--- 2)~Когда тѣло украшается, а душа пренебрегается, то и тѣло и душа погибнутъ. Понеже по общемъ воскресеніи когда тѣло съ душею совокупится, что душа, въ небреженіи пожавшая, постраждетъ, тоежде и тѣло страдати будетъ, и обоя купно смертію безсмертною умирати будутъ. "--- 3)~Когда душа украшается, то и тѣло, съ душею совокупившееся, въ свое время красоту воспріиметъ, \textit{егда тлѣнное сіе}, по свидѣтельству апостола, \textit{облечется въ нетлѣніе, и смертное сіе облечется въ безсмертіе}\footnote{1~Кор.~15,~54.}. "--- 4)~Чимъ болѣе душа украшается нынѣ, тѣмъ большія красоты сподобится и тѣло по воскресеніи. Тогда будетъ \textit{слава}, инымъ \textit{какъ солнцу}, инымъ \textit{какъ лунѣ}, инымъ \textit{какъ звѣздамъ}\footnote{41.}. Чимъ же болѣе украшается тѣло нынѣ, а душа пренебрегается, тѣмъ большее тогда и на душѣ и на тѣлѣ явится безобразіе. Скаредность бо небрежливыя души и на тѣлѣ тогда покажется, во обличеніе ея и въ показаніе всѣмъ, что она въ мірѣ не по Бозѣ жила, но по своимъ прихотямъ ходила, и не Хрісту, но міру работала. Тогда міролюбивый грѣшникъ, какъ чудовище какое страшное, предъ всѣмъ міромъ, ангелами и избранными Божіими явится, и самъ себе будетъ стыдиться, ужасаться, ненавидѣть, и восхощетъ въ ничто обратиться, но не можетъ; самъ себе будетъ окаевать, укорять и оплакивать, но поздно. Душа человѣческая подобна зеркалу, которое такой видъ въ себе воспріемлетъ, къ чему обратится; тако душа къ чему обратится; такой и образъ въ себе пріемлетъ. Къ Богу ли обратится, и Бога будетъ искать, любить и прилѣпляться: образъ Божій въ ней изображается и написуется Духомъ Святымъ. Къ міру ли обратится и суету его любить будетъ: такой и образъ въ ней является. И что нынѣ въ ней есть, тое тогда внѣ явится. Якоже убо боголюбящія души доброта, тако міролюбивыя скаредность и безобразіе тогда внѣ покажется. "--- 5)~Тѣло какъ ни укрывается, красоты не прибавится ему: рябый, черный, морщливый, щедрявый, косоглазый и проч., какъ ни украшается, таковъ же и непремѣненъ пребываетъ: душа же, чимъ болѣе удаляется отъ міра и мірскихъ похотей, и чимъ болѣе \textit{совлекается ветхаго человѣка съ дѣяньми его, и облекается въ новаго человѣка}\footnote{Кол.~3,~9 и 10.}, тѣмъ краснѣйшая дѣлается. Ибо тогда образъ Божій, который есть божественное и неизреченное ея украшеніе, въ ней, какъ очищаемомъ зерцалѣ смотрящаго въ тое, показуется. Нынѣ красота души не видна на тѣлѣ, когда праведніи и грѣшніи единъ внѣшній видъ имѣютъ, хотя то и святыя души благообразіе, и грѣшныя злообразіе изъ дѣлъ и словъ часто примѣчается, какъ отъ вкуса яблоко; но тогда, когда откроются сынове Божіи и сынове міра сего, покажется благолѣпіе ея, то"=есть, души, образъ Божій въ себѣ имѣющія. Тогда она, яко солнце, блистаніе красоты своея издастъ, и на тѣлѣ, съ которымъ здѣ работала Богу, доброту свою явитъ: \textit{тогда бо праведницы просвѣтятся, яко солнце, во царствіи Отца ихъ}\footnote{Матѳ.~13,~43.}. \textit{Возлюбленніи! нынѣ чада Божія есмы, и не у явися, что будемъ: вѣмы же, яко, егда явится, подобни Ему будемъ}\footnote{1~Іоан.~3,~21.}. О коликая будетъ слава чадъ Божіихъ, которые нынѣ, нерадя о тѣлѣ своемъ, о красотѣ души своей пекутся, и у лукаваго міра, какъ сметіе, въ презрѣніи и попраніи имѣются: \textit{Богу подобни будутъ! Богъ одѣвается свѣтомъ, яко ризою}\footnote{Пс.~103,~2.}: и чада Его просвѣтятся. Сея красоты желай и ищи, хрістіанине, пока обрѣтается. А что тебѣ въ тѣлесной красотѣ, которая нынѣ цвѣтетъ, а утро увядаетъ и въ ничто обращается? Сего ради какъ хотящіи внити въ чертогъ земнаго царя, и предъ лицемъ его явитися, и трапезы его пріобщитися, очищаютъ себе и убираются въ лучшее и пристойнѣйшее одѣяніе, чтобы негнусными очесамъ царскимъ показаться: тако наипаче хотящему внити въ чертогъ небеснаго Царя, и свѣтлѣйшему его лицу предстати, и великой оной вечери пріобщитися, должно очистить себе отъ скверны и порока: \textit{не имать бо въ онь внити всяко скверно}\footnote{Апок.~21,~27.}, "--- и достойною онаго чертога одеждою одѣятися, дабы не услышать отъ Царя страшнаго гласа: \textit{друже! како вшелъ еси сѣмо, не имый одѣянія брачна}\footnote{Матѳ.~22,~12.}? "--- и съ посрамленіемъ не быть изверженнымъ и вверженнымъ во тьму кромѣшнюю, \textit{гдѣ будетъ плачь и скрежетъ зубомъ}\footnote{Матѳ.~22,~13.}.

\paragraph*{§\:67.} Душу свою украшаетъ, кто, по апостольскому словеси, \textit{отлагаетъ ветхаго человѣка, тлѣющаго въ похотѣхъ прелестныхъ; обновляется же духомъ ума своего, и облекается въ новаго человѣка, созданнаго по Богу въ правдѣ и въ преподобіи истины}\footnote{Еф.~4,~22--24.}; то"=есть, кто противится плотскимъ вожделѣніямъ, и отсѣкаетъ ихъ, и прилежитъ добродѣтелямъ, отлагаетъ нечистоту, похоть блудную, гордость, зависть, гнѣвъ, злобу, ярость, сребролюбіе, невоздержаніе и прочее злонравіе, и облекается въ цѣломудріе, святыню, смиренномудріе, братолюбіе, терпѣніе, кротость и прочее добронравіе. Сія есть красная душевная утварь! Тая душа добра и красна есть, которая сообразуется небесному Отцу святостію, истиною, милосердіемъ, правдою, терпѣніемъ, кротостію и прочіими свойствами. Хрістіанинъ бо, яко отъ Бога рожденный, долженъ Ему, яко отцу сынъ, нравами сообразоваться. Всякъ бо сынъ отцу своему свойствами и нравами сообразуется, якоже сіе видимъ въ плотскомъ рожденіи, въ которомъ не иное что родится, какъ плоть отъ плоти, по свидѣтельству Хрістову: \textit{рожденное отъ плоти плоть есть}\footnote{Іоан.~3,~6.}. Сего ради и Божіе слово хрістіанамъ повелѣваетъ не плотскою, но духовною утварію украшаться. Какая сія утварь, въ чемъ состоитъ, слыши апостола, который утварь сію предлагаетъ намъ: \textit{облецытеся, якоже избранніи Божіи, святи и возлюбленни, во утробы щедротъ, благость, смиренномудріе, кротость и долготерпѣніе}, и проч.\footnote{Кол.~3,~12.} Сею утварію украсимся, хрістіанине, а не златомъ, сребромъ, каменіями и одеждами драгими. \textit{Облецемся во утробы щедротъ и милосердіе}, якоже и Господь нашъ повелѣваетъ намъ: \textit{будите милосерди, якоже и Отецъ вашъ милосердъ есть}\footnote{Лук.~6,~36.}, \textit{Иже Сына Своего не пощадѣ, но за насъ всѣхъ предалъ есть Его}\footnote{Римл.~8,~32.}, да насъ отъ вѣчныхъ бѣдствій избавитъ и приведетъ въ вѣчное блаженство. \textit{Облецемся въ благость}, и благотворить всѣмъ добрымъ и злымъ, другамъ и врагамъ потщимся, подражая небесному Отцу: \textit{яко Той благъ есть на неблагодарныя и злыя}\footnote{Лук.~6,~35.}; \textit{яко солнце Свое сіяетъ на злыя и благія, и дождитъ на праведныя и на неправедныя}\footnote{Матѳ.~5,~45.}. \textit{Облецемся въ смиренномудріе}, послѣдуя единородному Сыну Божію, который не устыдился \textit{умыти ноги апостоловъ, Господь и Учитель ихъ}\footnote{Іоан.~13,~14.}. \textit{Сіе да мудрствуется въ насъ, иже и во Хрістѣ Іисусѣ, Иже во образѣ Божіи сый не восхищеніемъ непщева быти равенъ Богу; но Себе умалилъ, зракъ раба пріимъ, въ подобіи человѣчестѣмъ бывъ, и образомъ обрѣтеся якоже человѣкъ; смирилъ Себе, послушливъ бывъ даже до смерти, смерти же крестныя}\footnote{Филип.~2,~5--8.}. \textit{Облецемся въ кротость и долготерпѣніе}, учася тому отъ кроткаго и смиреннаго Іисуса, якоже велитъ намъ: \textit{научитеся отъ Мене, яко кротокъ есмь и смиренъ сердцемъ}\footnote{Матѳ.~11,~20.}; и оставимъ долги должникамъ нашимъ, якоже и Богъ оставляетъ намъ. Будемъ святи, зане писано есть: \textit{святи будите, яко Азъ святъ есмь}\footnote{1~Петр.~1,~16.}. Сей уборъ, сія утварь, сіе украшеніе прилично хрістіанамъ есть; и угождаютъ тѣмъ не міру и очесамъ человѣческимъ, но очесамъ небеснаго Отца. Сею утварію потщимся украсить себе, хрістіанине, да угодимъ не міру, но Богу, Который на всѣхъ насъ призираетъ съ небесе Своего, якоже пророкъ поетъ: \textit{съ небесе призрѣ Господь, видѣ вся сыны человѣческія, отъ готоваго жилища Своего призрѣ на вся живущія на земли}\footnote{Пс.~32,~13 и 14.}.

\paragraph*{§\:68.} Извѣстно, что многіи одѣянія цвѣтнаго употребляютъ не ради снисканія суетныя славы и чести, но послѣдуя общему обычаю и примѣняяся къ званію и сану своему. Таковыхъ вышеписанное мое разсужденіе не касается, но только о таковыхъ слово здѣ, которые отъ платья ищутъ себѣ нѣкія славы и почитанія, каковые обыкновенно называются отъ всѣхъ щеголями, какъ изъ вышеписаннаго разсужденія видно. Ибо у такихъ людей, хотя и хрістіане называются, сердце суетою и любовію міра сего преисполнено, отъ чего апостолъ хрістіанскія души отвращаетъ: \textit{не любите міра, ни яже въ мірѣ: аще кто любитъ міръ, нѣсть любве Отчи въ немъ}\footnote{1~Іоан.~2,~15.}. И потому здѣ охуждается щегольство, пышность, гордость житейская, суета и слѣпота развращеннаго сердца, а не пристойность; охуждается роскошь и самолюбіе, которое о пользѣ ближняго небрежетъ и по большей части съ обидою ближняго бываетъ, а не умѣренность, которая въ гражданствѣ и общемъ житіи почти нужна; такожде излишнее попеченіе о смертномъ и тлѣнномъ тѣлѣ, а небреженіе о безсмертной души охуждается. Ибо, которымъ сердцемъ овладѣла суета міра сего, въ томъ нѣтъ попеченія о души и вѣчномъ ея спасеніи. Вѣрно слово Спасителя нашего: \textit{идѣже есть сокровище ваше, ту будетъ и сердце ваше}\footnote{Матѳ.~6,~21.}. Одѣяніе, какое ни есть, само собою есть вещь посредняя, ни добродѣтели, ни грѣха непричастная, какъ и всякая вещь; но только употребленіе его, конецъ и намѣреніе его можетъ быть, какъ и бываетъ, худое, и потому въ грѣхъ употребляющему обращается: ибо не на такій конецъ употребляется, на какій отъ Бога подано. Суета въ сердцѣ и у того имѣется, который ради того носитъ черное, манатейное и рубищное платье, чтобы его люди за презирателя міра и святаго почитали, какъ и у того, который того ради одѣвается цвѣтною и дорогою одеждою, чтобы въ собраніи первое мѣсто имѣть и отъ срѣтающихъ поклоны получать, и тѣмъ одѣяніемъ своимъ, какъ павлину перьемъ, любоваться. Всякаго бо одѣянія единъ долженъ быть конецъ, то есть, прикрытіе наготы и защищеніе немощнаго тѣла отъ стужи и непогоды, на что оно и дано намъ отъ Создателя нашего, какъ выше сказано. Однакожъ всякому хрістіанину помнить и держать должно, что всякое излишество, какъ въ пищи, въ строеніи и прочіихъ къ сему житію надлежащихъ, такъ и въ одѣяніи, безъ грѣха не бываетъ. Дается бо отъ Бога намъ имѣніе и богатство не только ради насъ самихъ, но и ради нищихъ, дабы тѣмъ и сами умѣренно довольствовалися, и неимущихъ снабдѣвали. А когда человѣкъ того не дѣлаетъ, но на излишество и роскошь тое расточаетъ, "--- дѣлаетъ противу воли Божіей, и потому грѣшитъ; все бо тое есть грѣхъ, что противу воли Божіей дѣлается: за что и истязанъ будетъ отъ Него, яко неправедный строитель и расточитель Божіяго добра. Что же надлежитъ до личнаго женскаго украшенія, оно никакого извиненія не можетъ имѣть, но есть точно едина прелесть, кознь, хитрость и вымыслъ злаго духа, на прельщеніе и развращеніе цѣломудрія учиненный; и хотя многія изъ женъ не ради худаго конца, но послѣдуя застарѣлому обычаю, украшаютъ лица своя, однакожъ безъ соблазна юнымъ и имъ самимъ безъ грѣха тое украшеніе не бываетъ.


\section[Статья 4-я. О грѣхахъ нѣкіихъ особенно.]{статья четвертая.\\\bfseries О грѣхахъ нѣкіихъ особенно.}

Въ началѣ сея статьи нужно мнѣ объявить нѣкая ради читателя.

1)~Я разсужденія своего не положилъ здѣ о грѣхахъ, всѣмъ извѣстныхъ, каковыми вси, и самые грубые люди и народы, гнушаются, какъ"=то: разбой, явное насиліе и грабленіе, убійство, блудъ, прелюбодѣяніе, и проч. Ибо всякъ естественнымъ закономъ и совѣстію убѣждается таковыхъ грѣховъ берещися, и по содѣяніи каковаго нибудь отъ таковыхъ грѣховъ не престаетъ совѣсть мучить согрѣшившаго, хотя"=то и всякимъ грѣхомъ она уязвляется, наипаче, когда человѣкъ ее хранить тщится неповрежденну. Совѣсть бо и естественный законъ согласны суть Божіему писанному закону, и потому, какъ сей законопреступника обличаетъ, такъ и оный согрѣшившаго не оставляетъ въ покоѣ, но непрестанно мучитъ его. Сіе же разумѣется о тѣхъ наипаче, которые по правилу ея и закона Божія тщатся житіе свое провождать. Ибо въ совѣсти, грѣхами замаранной, и великіе грѣхи едва усматриваются, такъ, какъ въ закопченномъ зеркалѣ и лица и пороковъ своихъ не видитъ смотрящій въ него. "--- 2)~Разсужденіе здѣ полагается наипаче о такихъ порокахъ, которые и другихъ пороковъ причиною бываютъ, и каковыхъ нынѣшній наипаче вѣкъ за пороки не почитаетъ, какъ увидишь. "--- 3)~Еще объявляю читателю, что я старался здѣ показать мерзость и тяжесть грѣховную въ разсужденіи о грѣхахъ на такой конецъ, дабы хранящіи должность хрістіанскую береглися грѣха, а согрѣшающіи отстали отъ того; и разсужденіемъ моимъ не лица хулятся, но пороки, которые тѣми лицами обладаютъ, охуждаются и обличаются. Грѣхъ всякій въ человѣкѣ есть, того ради и о человѣкѣ надобно помянуть, въ которомъ онъ живетъ и обладаетъ, хотя все слово на грѣхъ, а не на человѣка самаго должно стремиться. Какъ бо соль червей изъ вещества гонитъ, или до гнилости не допущаетъ, хотя тою и растворяется вещество; или какъ лѣкарство на болѣзнь стремится и намѣревается выгнать тую изъ человѣка, а не на самаго человѣка, хотя и внутрь человѣка пріемлется: такъ и обличеніе, хотя на человѣка нападаетъ и совѣсть его уязвляетъ, но намѣревается на отгнаніе грѣха, которымъ обладаемъ бываетъ человѣкъ, дабы, узнавши грѣхъ и тяжесть его пагубную, потщался благодатію Божіею отъ того свободиться. Надобно бо неотмѣнно познать грѣхъ и мерзость его, кто хощетъ истинно покаяться и тако спастися. Безъ познанія грѣха истинное покаяніе не бываетъ, какъ безъ познанія болѣзни исцѣленія не бываетъ. Какъ же можетъ познаться грѣхъ, когда мерзость и тяжесть его грѣшнику предъ глазами не представится? Сего ради не должно никому гнѣваться на обличительное слово, наипаче вообще бываемое: яко оно на грѣхи стремится, а не на самаго человѣка, какъ и лѣкарство на болѣзнь, а не на человѣка. Безъ болѣзни не нужно и лѣкарство, и гдѣ грѣха нѣтъ, тамо не потребно и обличеніе. Оставь грѣхъ, тогда и обличеніе касаться тебе не будетъ; яко не будетъ того, въ чемъ оно обличаетъ. Обличеніе подобно зеркалу, которое показуетъ пороки на лицѣ; а когда пороковъ на лицѣ не будетъ, то и не показуетъ ихъ. Не гнѣваешься на зеркало за показаніе пороковъ твоихъ: не гнѣвайся же и на обличительное слово, которымъ души твоей пороки показуются. Къ томужъ, когда гнѣваешься на обличеніе, вообще бываемое, то тѣмъ показуешь, что и совѣсть твоя тебе обличаетъ въ томъ, въ чемъ слово. Слово бо обличительное есть сходно съ совѣстію. Откуду бываетъ, что, когда въ собраніи говорится о какихъ грѣхахъ, свѣдущіе ихъ трогаются совѣстію и какіе нибудь на себѣ показуютъ знаки. Что внѣ слово, тое внутрь совѣсть обличаетъ; и что внѣ слово, тое внутрь совѣсть похваляетъ. Два сія свидѣтеля какъ сходны и согласны между собою, такъ и вѣрны суть, и солгать не могутъ, но всегда истину свидѣтельствуютъ, и что единъ, тое и другій показуетъ: единъ похваляетъ добродѣтель, и другій; единъ обличаетъ грѣхъ, и другій. И какъ здѣ въ вѣкѣ семъ, такъ и въ послѣднемъ дни на страшномъ судѣ достовѣрные свидѣтели и обличители будутъ грѣховъ человѣческихъ. Сего ради, дабы и слово и совѣсть твоя здѣ и на судѣ ономъ тебе не обличали, остави грѣхи и покайся истинно: тогда какъ внутренняго обличителя не будеши имѣть, такъ и словесное обличеніе касаться тебе не будетъ; тогда будеши покоенъ и миренъ, хотя вси тебе будутъ хулить и проклинать. Чистыя бо совѣсти свидѣтельство достовѣрнѣйшее есть паче свидѣтельства всѣхъ людей, и единъ есть истинный покой внутренній и совѣстный, безъ котораго человѣкъ покоенъ быть не можетъ, хотя отвнѣ и никакъ безпокоить ея не будетъ. "--- 4)~Лучше тебѣ здѣ въ мірѣ семъ, и тайно внутрь себе и предъ собою единымъ обличеніе претерпѣть и оставить грѣхъ и покаяться, и тако спастися, нежели на всемірномъ ономъ судѣ обличитися и постыждену отослатися во тьму кромѣшнюю, гдѣ будетъ плачь и скрежетъ зубомъ. Что бо здѣ въ совѣсти заглаждено будетъ, тое и тамо не явится. \textit{Примирися} убо \textit{здѣ съ соперникомъ твоимъ} симъ, \textit{дондеже еси на пути съ нимъ, да не предастъ тебе соперникъ судіе, и судія тя предастъ слузѣ, и въ темницу вверженъ будеши}\footnote{Матѳ.~5,~25.}; къ чему всякое обличительное слово намѣревается, то"=есть, дабы на ономъ всенародномъ позорищѣ грѣшникъ не обличался отъ праведнаго Судіи. А кто не учинитъ очищенія совѣсти и не загладитъ въ ней грѣховъ, тотъ непремѣнно обличится тамо, и увидитъ всѣ свои грѣхи предъ собою представлены, по неложному Божію слову: \textit{обличу тя, и представлю предъ лицемъ твоимъ грѣхи твоя}\footnote{Пс.~49,~21.}. Тогда всякій грѣшникъ нераскаянный услышитъ: \textit{се человѣкъ и дѣла его}! се человѣкъ, который хрістіаниномъ назывался и богочтецемъ, но хрістіанскихъ дѣлъ не творилъ; \textit{Бога исповѣдалъ вѣдѣти, дѣлами же отметался Его}\footnote{Тит.~1,~16.}; \textit{имѣлъ образъ благочестія, силы же его отвергался}\footnote{2~Тим.~3,~5.}! Услышитъ же предъ всѣмъ міромъ, ангелами и человѣками, и тако покрыется стыдомъ. Надобно убо оставить тое, за что внутрь совѣсть и внѣ слово Божіе обличаетъ, и сокрушеннымъ сердцемъ покаяться, да не тамо обличится, гдѣ покаяніе мѣста не имѣетъ. "--- 5)~Обличеніе, когда единому какому лицу бываетъ, не должно умягчать слова, но изъяснять важность грѣха, да не въ грѣхѣ утвердившися погибнетъ; кольми паче тое чинить должно, когда бываетъ обличеніе вообще безъ поминанія лицъ, да не на ласкательство, а на обличеніе будетъ простираемо слово. Почему не для чего и гнѣваться за обличеніе, хотя и за тое, которое прямо въ лице бываетъ, а паче за тое, которое вообще говорится. На лѣкаря не гнѣвается никто, что жестокое лѣкарство подаетъ ради излѣченія тѣла: почтожъ намъ гнѣваться на того, кто хощетъ душу нашу исцѣлить, и жестокимъ словомъ, какъ жестокимъ лѣкарствомъ, грѣхъ изъ насъ выгнать? Кому не тяжко было грѣшить, тому да не будетъ тяжко и обличеніе за грѣхъ слышать, и тако покаятися. Когда оставишь грѣхъ, въ которомъ обличаешися, тогда познаешь, коль полезно было тебѣ обличеніе; тогда не только гнѣваться не будеши на обличившаго, но и благодарить ему будеши, якоже больный, свободившійся отъ болѣзни, благодаритъ лѣкарю. Чего я тебѣ усердно желаю. "--- 6)~Иный грѣхъ отъ немощи, иный отъ произволенія, предразсужденія и противу совѣсти бываетъ. Отъ немощи грѣхъ, какому и благочестивые люди подлежатъ, легко и ласково обличать должно, но грѣхи, противу совѣсти и отъ произволенія творимые, а паче застарѣлые и въ обычай вшедшіе, требуютъ жестокаго и строгаго обличенія, какъ застарѣлая болѣзнь горькаго и жестокаго лѣкарства. Таковые бо грѣхи явно къ погибели ведутъ грѣшниковъ, и не могутъ иначе, какъ жестокимъ наказаніемъ съ помощію Божіею отгнатися. Строго убо должно таковыхъ обличать, дабы таковымъ гласомъ, какъ громомъ, пробудилися отъ сна грѣховнаго, и сотворили истинное покаяніе. Надобно бо вездѣ истину говорить и не молчать, что должно говорить. Грѣхъ, по надлежащему неоткровенный и необличенный, или за грѣхъ не почитается, или за легкій вмѣняется, который въ себѣ толикія есть тягости, что и малый можетъ погрузить грѣшника во дно адово, яко всякій грѣхъ противу величества Божія бываетъ. А когда за грѣхъ не почитается, или легкимъ быть судится, то и неисцѣленъ пребываетъ, что весьма опасно. "--- 7)~Отъ сей статьи, въ которой нѣкіе грѣхи исчисляются и изъясняются, нѣсколько можешь познать, читатель, какъ великое растлѣніе и зло носитъ внутрь себе человѣкъ, который по образу Божію и по подобію сотворенъ, и былъ святъ, чистъ, непороченъ, праведенъ и жилище Святаго Духа; но совѣтомъ злаго и лукаваго духа и ядомъ зміинымъ вліяннымъ тако растлился, что \textit{приложился скотомъ}\footnote{Пс.~48,~13.}, "--- чего довольно оплакать не можемъ. Что бо въ человѣкѣ, благодатію необновленномъ, видится, кромѣ скотскаго? Въ скотѣ видится нечистота, таяжде и въ человѣкѣ, паче же и большая и мерзостнѣйшая. Въ скотѣ обжирство: тоежде и въ человѣкѣ. Въ скотѣ хищеніе примѣчается: тоежде и въ человѣкѣ. Въ скотѣ лютость, свирѣпство, гнѣвъ, злоба и ярость: тоежде и въ человѣкѣ. Въ скотѣ зависть: таяжде имѣется и въ человѣкѣ. Въ скотѣ лукавство и хитрость видимъ: тоежде и въ человѣкѣ, и проч. Вотъ каковъ человѣкъ самъ въ себѣ! И, что жалостнѣе и плачевнѣе, какія страсти безчинныя во всѣхъ безсловесныхъ разсѣяны, тѣ въ единомъ человѣкѣ имѣются. Такъ обезобразился бѣдный человѣкъ, которому пожелалось имѣть Божію честь! Страшно описуетъ святое Писаніе неотрожденнаго человѣка, и предъ глазами нашими представляетъ, да отсюду познаемъ бѣдность, окаянство и ядъ смертный, крыющійся въ сердцѣ нашемъ: что бо видишь въ другомъ, тое и въ тебѣ имѣется, хотя и не является внѣ. Тамо, то"=есть въ Писаніи, видимъ, что человѣкъ неотрожденный то \textit{ехиднымъ порожденіемъ}, то \textit{лисомъ}, то \textit{свиніею} называется,то \textit{льву}, то прочіимъ животнымъ уподобляется ради злонравія своего, которымъ сходенъ и сообразенъ есть безсловеснымъ тѣмъ. Отсюду долженъ всякъ заградить уста и молчать, какимъ бы именемъ ни названъ былъ. Всякаго бо безчестнаго нарицанія достоинъ человѣкъ, яко, хотя всякаго зла не дѣлаетъ внѣ, но внутрь всякое зло носитъ, которое при случаѣ и на верхъ или внѣ является. Отъ сего скареднаго источника проистекаетъ, что и самые отрожденные, благочестивые и святые чувствуютъ брань во удесѣхъ своихъ, и хотя подвизаются противу зла того, однакожъ часто \textit{не еже хотятъ доброе, творятъ}\footnote{Римл.~7,~19.}; и молятся о оставленіи грѣховъ къ небесному Отцу: \textit{остави намъ долги наша}; да со всею святою церковію вопіютъ къ Богу: \textit{Господи, помилуй}! "--- и отъ суда Его праведнаго къ милосердію Его прибѣгаютъ: \textit{не вниди въ судъ съ рабомъ Твоимъ, яко не оправдится предъ тобою всякъ живый}\footnote{Пс.~142,~2.}; и оплакиваютъ окаянство сіе съ апостоломъ: \textit{окаяненъ азъ человѣкъ, кто мя избавитъ отъ тѣла смерти сея}\footnote{Римл.~7,~24.}? "--- Отсюду заключается, что оправданіе наше предъ Богомъ и вѣчное спасеніе состоитъ въ единомъ милосердіи небеснаго Отца, заслугахъ Хрістовыхъ и вѣрѣ, которая Его, яко Искупителя и Спасителя своего, съ радостію признаетъ, и святѣйшія Его заслуги съ любовнымъ и благодарнымъ, смиреннымъ и къ послушанію готовымъ сердцемъ, яко цѣлительное и спасительное врачевство, смертельной своей немощи прилагаетъ.

\subsection[Глава 1-я. О гордости.]{глава первая.\\\bfseries О гордости.}

\begin{quotation}\textit{Еже есть въ человѣцѣхъ высоко, мерзость есть предъ Богомъ}\footnote{Лук.~16,~15.}.\end{quotation}
\begin{quotation}\textit{Всякъ возносяйся смирится: смиряяй же себе вознесется}\footnote{18,~14.}.\end{quotation}
\begin{quotation}\textit{Гордымъ Богъ противится: смиреннымъ же даетъ благодать}\footnote{Iак.~4,~6.}.\end{quotation}


\paragraph*{§\:69.} Гордости \textit{начало} есть діаволъ, отступившій отъ создателя своего, и изъ ангела свѣтла княземъ тьмы содѣлавшійся. Симъ смертоноснымъ ядомъ какъ самъ зараженъ пребываетъ, такъ и наши сердца такъ сильно заразилъ, что чрезъ все житіе окаянства того нашего довольно оплакати не можемъ.

\paragraph*{§\:70.} Нѣтъ ничего опаснѣе, сокровеннѣе и труднѣе гордости. \textit{Опасна} гордость, ибо гордымъ заключается небо, и вмѣсто неба адъ опредѣляется. \textit{Гордымъ} бо \textit{противится Богъ}, глаголетъ Писаніе. \textit{Сокровенна} гордость, понеже такъ глубоко въ сердцѣ нашемъ крыется, что и усмотрѣть ее не можемъ безъ помощи кроткаго сердцемъ Іисуса Хріста, Сына Божія; и лучше ее узнаемъ на ближнихъ нашихъ нежели на себѣ. Прочіи пороки, какъ"=то: піянство, блудъ, воровство, хищеніе и прочая, видимъ; ибо часто ради ихъ жалѣемъ и стыдимся: но гордости не видимъ. Кто бо себе когда призналъ отъ сердца гордымъ? еще не случалось того видѣть. Многіи себе называютъ грѣшниками, но отъ другихъ называтися не терпятъ, и хотя многіи изъ нихъ языкомъ не отзываются, однакожъ не безъ негодованія и огорченія сердечнаго пріемлютъ тое. И тако отъ сего показуется, что языкомъ только называютъ себе грѣшниками, а не сердцемъ, на устахъ смиреніе показуютъ, а на сердцѣ не имѣютъ. Истинно бо смиренный огорчитися и гнѣватися отъ укоренія не можетъ, ибо всякаго уничтоженія достойна себе мнитъ. "--- Нѣтъ ничего \textit{труднѣе} гордости, ибо съ великою неудобностію и такожде не безъ помощи Божіей побѣждаемъ ее. Внутрь бо себе носимъ зло сіе. Въ благополучіи ли находимся? она съ вѣличаніемъ и пышностію, презрѣніемъ и уничиженіемъ ближнихъ нашихъ присѣдитъ намъ. Въ злополучіе ли попадемся? чрезъ негодованіе, роптаніе и хуленіе оказываетъ себе. Терпѣнію ли, кротости и прочіимъ добродѣтелямъ обучатися тщимся? киченіемъ фарисейскимъ востаетъ на насъ. И такъ нигдѣ и никакъ отъ нея избавиться не можемъ: всегда съ нами ходитъ, всегда хощетъ господствовати и владѣти нами.

\paragraph*{§\:71.} Какъ \textit{гордымъ Богъ противится}, показуютъ страшныя судьбы Божія, которыя намъ святое Божіе слово представляетъ, дабы мы, взирая на нихъ, всѣми силами береглися мерзкаго и душепагубнаго сего порока.

Вознеслись прародители наши въ раю, и возжелали божескія чести; но лишились и тоя чести, которую имѣли, и \textit{скотомъ несмысленнымъ приложились и уподобились имъ}, и всякому бѣдствію подпали\footnote{Быт.~3.}. Вознеслись потомки Ноевы, и сотвореніемъ столпа хотѣли себѣ сотворить имя славное: \textit{и смѣси Господь тамо языкъ ихъ, и не услыша кійждо гласа ближняго своего}\footnote{11,~3--9.}. Вознеслся Фараонъ, царь египетскій, противу Бога, и людей Его хотѣлъ озлобить и погубить: и погиблъ самъ со всѣмъ воинствомъ въ морѣ Чермнѣмъ, и \textit{погрязъ яко олово въ водѣ зѣльнѣй}\footnote{Исх.~14,~15 и 10.}. Вознеслися Корей, Даѳанъ и Авиронъ, и Моисею пророку Божію и вождю своему воспротивились: \textit{и разверзеся земля, и пожре я; и снидоша тіи, и вся, елика суть ихъ, живи во адъ, и покры ихъ земля, и погибоша отъ среды сонма}\footnote{Числ.~16.}. Возвысилъ гласъ хульный на Бога вышняго и на святый градъ Его Сеннахиримъ, царь ассирійскій: \textit{и изыде ангелъ Господень, и изби отъ полка его сто осмьдесятъ пять тысящъ, самъ же, пораженъ мечемъ отъ Сыновъ своихъ}\footnote{Ис.~37.}. Возвысился Олофернъ, вождь силы ассирійскія: и рука женская \textit{отсѣче}, гордую \textit{выю его}\footnote{Іудиѳ.~13,~8.}. Возвысился Аманъ, первый Артаксеркса царя Перскаго, совѣтникъ: \textit{и повѣшенъ на древѣ}, которое неповинному Израильтянину, Мардохею, уготовалъ"=было\footnote{Есѳ.~7.}. Возвысился Навуходоносоръ, царь вавилонскій, "--- и услышалъ гласъ съ небесе: \textit{тебѣ глаголется, Навуходоносоре царю: царство твое прейде отъ тебе, и отъ человѣкъ отженутъ тя, и со звѣрьми дивіими житіе твое, и травою аки вола напитаютъ тя}, и проч.\footnote{Дан.~4.} Вознеслся до небесъ Капернаумъ градъ, и слышитъ отъ Хріста: \textit{и ты, Капернауме, иже до небесъ вознесыйся, до ада снидеши}\footnote{Матѳ.~11,~23.}. Вознеслся Фарисей: \textit{и сниде осужденъ въ домъ свой}\footnote{Лук.~18,~4.}. Тако гордымъ противится Богъ! тако вознесенныхъ смиряетъ Господь! Нѣтъ бо ничего ненавистнѣе Богу въ человѣцѣ, паче гордости. \textit{Еже въ человецѣхъ высоко, мерзость есть предъ Богомъ}.

\paragraph*{§\:72.} Знаки гордости сіи суть: 1)~высшимъ не покаряется; 2)~равнымъ и нижнимъ не уступаетъ; 3)~гордость велерѣчива, высокорѣчива и многорѣчива; 4)~славы, чести и похвалы всякимъ образомъ ищетъ; 5)~высоко себе и дѣла свои превозноситъ; 6)~другихъ презираетъ и уничтожаетъ; 7)~ищетъ себе показать; 8)~безстыдно себе хвалитъ; 9)~какое добро имѣетъ, себѣ приписуетъ, а не Богу; 10)~хвалится и тѣмъ добромъ, котораго не имѣетъ; 11)~недостатки и пороки свои весьма тщится сокрывать; 12)~въ презрѣніи и уничтоженіи быть не терпитъ; 13)~увѣщанія, обличенія, совѣта не пріемлетъ; 14)~въ дѣла чужія самовольно мѣшается; 15)~сана или чести лишившись, и въ прочемъ несчастіи ропщетъ, негодуетъ, а часто и хулитъ; 16)~слѣдственно гордость гнѣвлива, 17)~гордость завистлива: не хощетъ бо, чтобы кто равенъ ей и выше ея былъ, равную или большую честь имѣлъ, но чтобы она всѣхъ во всемъ превышала. 18)~гордость нелюбительна, ненавистлива. "--- Самолюбіе же корень всѣхъ золъ. "--- И такъ гордость есть начало и корень всякаго грѣха.

\paragraph*{§\:73.} Гордость не токмо сама собою есть грѣхъ тяжкій и мерзкій, но и \textit{другихъ грѣховъ причиною} бываетъ. Ибо Богъ, Который гордымъ противится, праведнымъ судомъ отъ гордаго благодать Свою отъемлетъ; сатана же, яко духъ гордый и человѣкоубійца, къ такому \textit{дому}, яко \textit{пометенному и украшенному, удобно приступаетъ}\footnote{Матѳ.~12,~44.}. Потому человѣкъ, безъ благодати Божіей оставшійся, яко немощный и ко всякому злу удобопреклонный, во всякій грѣхъ удобно падаетъ. О чемъ многія какъ священнаго Писанія, такъ и церковныя исторіи свидѣтельствуютъ.

\paragraph*{§\:74.} \textit{Конецъ}, къ которому \textit{гордость} приводитъ, Самъ Хрістосъ означаетъ: \textit{всякъ возносяйся смирится}. Она высоко возносится, но весьма низко спадаетъ. \textit{Смотри § 71}.

\paragraph*{§\:75.} Многи причины суть, которыя нашу \textit{низлагаютъ} гордость, изъ которыхъ нѣкоторыя здѣ приводятся. 1)~Люди наипаче гордятся или ради чести и славы, или ради богатства, или ради разума, или ради крѣпости, или ради благородства. Но понеже вся сія перемѣнѣ подлежатъ, и какъ приходятъ къ намъ, такъ и отходятъ отъ насъ, яко не наша: убо возноситься ради того, что не наше, весьма несмысленно. Все бо, что ни имѣемъ, не наше, но Божіе есть; сосуды и влагалища есмы Божіихъ дарованій. Богу дарующему подобаетъ всякая хвала и честь и благодареніе, а человѣку смиряться, дабы, что имѣетъ, не отнято было ради гордости. "--- 2)~Низлагаетъ гордость нашу рожденіе и воспитате наше. Какое животное съ большимъ трудомъ и болѣзнію раждается, какъ человѣкъ? знаютъ о семъ матери и при такихъ случаяхъ бывающіи. Какое животное въ воспитаніи большаго требуетъ смотрѣнія, попеченія, питанія, береженія, какъ человѣкъ? Многія животныя тотчасъ по рожденіи сами себѣ достаютъ пищу; а человѣкъ сколько чужими руками носится, одѣвается, чужими трудами питается, согрѣвается, сохраняется! "--- 3)~Низлагаетъ гордость нашу нагота наша. Прочія животныя не требуютъ одежды, а человѣкъ такъ бѣденъ и убогъ, что и одежды своей не имѣетъ, но отъ животныхъ и прочаго созданія пріемлетъ ея. Овца, лисица, волкъ, заяцъ, рысь и прочія, одѣваютъ и грѣютъ насъ. "--- 4)~Низлагаютъ гордость нашу и бѣдствія наши. Кто большимъ страстямъ, тлѣнію, болѣзнямъ, немощамъ подлежитъ, какъ человѣкъ? Кто большему страху, печали, скорби подверженъ, какъ человѣкъ? Отвсюду окруженъ бѣдами: созади грѣхи, спереди смерть, сверху судъ Божій, снизу адъ, со сторонъ соблазны міра и козни бѣсовскія, внутрь плоть со страстьми и похотьми. Въ такомъ ли бѣдственномъ состояніи гордиться? "--- 5)~Низлагаетъ гордость нашу конецъ житія нашего, \textit{яко земля есмы, и въ землю пойдемъ}. Приникни во гробища, и распознай тамо царя отъ воина, славна отъ безчестна, богата отъ нища, крѣпкаго отъ немощнаго, благороднаго отъ худороднаго, мудраго отъ неразумнаго; тутъ смотря, хвались своимъ благородіемъ, тутъ превозносись разумомъ, тутъ величайся богатствомъ, тутъ надымайся честію, тутъ считай ранги, тутъ исчисляй титулы. О бѣдное созданіе, бѣдное по началу, бѣдное по срединѣ, бѣдное и по концу! Яко утлый и гнилый сосудъ, яко земля есть, и въ землю пойдетъ человѣкъ. "--- 6)~Чимъ болѣе Хріста будемъ познавать и поминать, тѣмъ болѣе свою подлость и окаянство узнавать, и такъ смиряться. Хрістосъ, Сынъ Божій, Господь твой, тебе ради смирился: тебѣ ли рабу гордиться? Господь твой тебе ради рабій пріялъ зракъ: тебѣ ли рабу искать господства? Господь твой тебе ради безчестіе пріялъ: тебѣ ли рабу честію возноситься? Господь твой не имѣлъ, гдѣ главы подклонити: тебѣ ли рабу расширять великолѣпныя зданія? Господь твой ноги умылъ Своимъ ученикамъ: тебѣ ли стыдно послужить братіи своей? Господь твой злословія, поношенія, руганія, посмѣянія, заплеванія претерпѣлъ: тебѣ ли рабу досаднаго слова не терпѣть? Онъ неповинно и ради тебе терпѣлъ: тебѣ ли виноватому и для себе не терпѣть? не заслужилиль грѣхи Твои того? Господь твой за распинателей Своихъ молился: \textit{Отче, остави имъ}\footnote{Лук.~23,~34.}: тебѣ ли рабу на оскорбившихъ гнѣваться, злобиться, искать мщенія? Но кто ты таковъ, что не терпятъ уши твои оскорбленія? Тварь убогая, немощная, нагая, страстная, заблуждшая, всякимъ злополучіямъ подверженная, всякими бѣдами окруженная, трава, сѣно, пара, вмалѣ являющаяся и исчезающая. Но смотри и берегись, чтобы и тебе Хрістосъ Господь твой не постыдился, когда ты смиренія и кротости Его стыдишися. Глаголетъ бо: \textit{иже аще постыдится Мене и Моихъ словесъ въ родѣ семъ прелюбодѣйнѣмъ и грѣшнѣмъ, и Сынъ человѣческій постыдится его, егда пріидетъ во славѣ Отца Своего со ангелы святыми}\footnote{Марк.~8,~38.}. Стыдится же Хріста и словесъ Его, кто не послѣдуетъ Его смиренію, кротости, терпѣнію, но съ міромъ хощетъ въ гордости царствовать. "--- 7) «Бойся, глаголетъ Василій великій къ гордящемуся, паденія подобнаго діаволу, который противу человѣка вознесшися, ниже человѣка паде, и котораго попиралъ, тому въ попраніе отданъ»\footnote{Въ сл. о смиреніи.}.

\paragraph*{§\:76.} Худо и мерзко предъ Богомъ гордиться всякому ради вышеписанныхъ причинъ; но хуже тому, котораго и состояніе несчастія должно приводить въ смиреніе. Худо гордиться сановитому, благородному, господину, богатому; но хуже того подлому, худородному, рабу, нищему. Такъ и о прочіихъ разумѣй, какъ всякъ сіе можетъ признать.

\paragraph*{§\:77.} Понеже, какъ сказано выше, такъ глубоко въ сердцѣ нашемъ смертоносный гордости ядъ укоренился: должно часто взирать на глубочайшее Сына Божія смиреніе, и тому учиться отъ Него, якоже Самъ глаголетъ: \textit{научитеся отъ Мене, яко кротокъ есмь и смиренъ сердцемъ}\footnote{Матѳ.~11,~29.}, "--- а притомъ и проситъ Его усердно, дабы, пагубный оный ядъ врачевствомъ Своея благодати выгнавъ, подалъ духъ смиренія, которому воспослѣдуютъ и прочія Его дарованія. Ибо \textit{смиреннымъ Богъ даетъ благодать}.

\subsection[Глава 2-я. О зависти.]{глава вторая.\\\bfseries О зависти.}

\begin{quotation}\textit{Аще зависть горьку имате и рвеніе въ сердцахъ вашихъ, не хвалитеся, ни лжите на истину. Нѣсть сія премудрость свыше низходящи, но земна, душевна, бѣсовска. Идѣже бо зависть и рвеніе, ту нестроеніе и всяка зла вещь}\footnote{Іак.~3,~14--16.}.\end{quotation}


\paragraph*{§\:78.} «Зависть есть печаль о благополучіи ближняго», глаголетъ Василій Великій въ словѣ о зависти. Опечалися Каинъ, и испаде лице его. А для чего? понеже видѣлъ, что братъ его Авель отъ Господа всѣхъ за приношеніе даровъ похваленъ, а онъ ради своея лѣности отринутъ. Большій блуднаго сына братъ, какъ идучи съ села услышалъ лики и пѣніе въ дому отца своего, и узналъ, что веселіе тое бываетъ ради брата его, котораго благоутробный отецъ оный здрава пріятъ, \textit{разгнѣвася и не хотяше внити}, то есть, въ домъ отца своего, и быть участникомъ веселія того\footnote{Лук.~16,~26--30.}. Праведна веселія причина; брать его \textit{мертвъ былъ и ожилъ, и изгиблъ было, и обрѣтеся}. Но онъ на тое не смотритъ, тое ему не любо, что ради брата его телецъ упитанный закланъ, и ради благополучія его отецъ его со всѣмъ домомъ ликуетъ. Тако зависть печалится о добрѣ ближняго, и братнее благополучіе за свое вмѣняетъ неблагополучіе! Сію печаль безполезную приписуетъ зависти и святый Златоустъ, «зависть, глаголя, благополучіе ближняго своего вмѣняетъ себѣ за неблагополучіе»\footnote{Бес.~62"~я на Быт.}.

\paragraph*{§\:79.} Зависти корень и \textit{начало} есть гордость. Гордый бо, понеже хощетъ выше прочіихъ вознестися, не можетъ терпѣть, кто бы ему равенъ, а паче высшій въ благополучіи былъ, потому и негодуетъ о возвышеніи его. Смиренный бо завидѣть не можетъ, ибо видитъ и признаетъ свое недостоинство, прочіихъ же достойнѣйшихъ быти судитъ паче себе; почему и о дарованіяхъ ихъ негодованія не имѣетъ. Страсть убо сія есть тѣхъ, которые мнятъ о себѣ, что они нѣчто въ мірѣ суть, и тако о себѣ высоко мечтая, прочіихъ ничтоже быти судятъ. Тако негодуетъ Саулъ гордый на кроткаго и смиреннаго Давида, что ему болѣе похвалы приписали ликующія жены, какъ самъ призналъ, и рече Саулъ: \textit{Давиду даша тьмы, мнѣ же даша тысящи}\footnote{1~Цар.~18,~8.}. Откуду и гнать началъ неповиннаго.

\paragraph*{§\:80.} \textit{Конецъ} зависти есть, чтобы того, кому завидитъ, видѣть въ неблагополучіи. Она тогда раждается, когда другаго начинается благополучіе; тогда престаетъ, когда престаетъ его благополучіе и начинается злополучіе. Тако завистію праотцы наши съ высокаго блаженства въ бѣдственное состояніе низринуты. Зависть Каина научила востать на брата своего Авеля и убить. Зависти есть дѣло, что Іосифъ проданъ во Египетъ. Зависти приписать должно, что Іудеи Хріста, Господа и Благодѣтеля своего на крестъ вознесли. Тако отъ гордости начинается зависть, отъ зависти ненависть, отъ ненависти злоба; злоба къ неблагополучнѣйшему приводитъ концу. Откуду святый Златоустъ глаголетъ: «корень убійства зависть»\footnote{На Быт. бес.~64"~я.}.

\paragraph*{§\:81.} \textit{Мучительная} страсть и смѣха или паче плача достойная зависть: яко такимъ ядомъ отъ діавола человѣческое заражено сердце. Прочія страсти нѣкое, хотя мнимое, услажденіе имѣютъ, а завидливый грѣшитъ, купно и мучится. Неугодна Аману, врагу іудейскому, какъ самъ признаетъ, ни слава, которою возвеличилъ его царь, ни богатство, которымъ изобиловалъ, ни тая часть, что къ царицѣ на пиръ съ царемъ позванъ былъ; сія, глаголетъ, \textit{не суть мнѣ угодна}. Что тому причиною? Не обидѣлъ ли тебе кто, Амане? не похитилъ ли кто твоихъ пожитковъ? или не оклеветалъ ли кто тебе къ царю? здоровъ ли самъ, или домашніи твои? или не смущаетъ ли тебе страхъ непріятельскій? Нѣтъ! нѣтъ ничего того. Чтожъ убо? что смущается сердце твое? чего скорбишь посреди толикой славы и богатства? чего ради преклоняешь главу твою? что дряхлъ и унылъ ходиши? что темнѣетъ лице твое? чего тебѣ, первому совѣтнику царскому, недостаетъ? Единаго только скипетра и державы царскія не имѣешь. Нѣтъ, глаголетъ Аманъ, вся мнѣ сія ничто: ибо вижу Мардохея въ благополучіи. Сія не суть мнѣ угодна, когда вижу Мардохея Іудеянина во дворѣ царевѣ. Ничто"=де мене, ни слава, ни честь, ни богатство не веселитъ; понеже вижу Мардохея у царя въ милости\footnote{Есѳ.~5"~я.}. Тако завистливый грѣшитъ и купно казнь пріемлетъ, беззаконнуетъ и мучится! И отъ сего видѣть можно, какъ скареденъ порокъ и посмѣянія достоинъ "--- зависть! Печалиться о томъ, что ближній веселится; сѣтовать и снѣдаться,что братъ въ благополучіи цвѣтетъ: не смѣха ли достойное дѣло? Всякъ сіе можетъ признать, точно свойственное есть дѣло діавольское: діаволу бо печально и, несносно, что хрістіане спасаются и вѣчную получаютъ славу, отъ которой онъ низринутъ въ вѣчную погибель и безчестіе.

\paragraph*{§\:82.} Изъ вышереченныхъ всякому видно, коль и \textit{пагубна} есть сія страсть. Понеже 1)~собственный грѣхъ есть діаволовъ, который сею язвою прародителей нашихъ, а съ ними и насъ умертвилъ. Діавольское бо дѣло есть радоватися о погибели человѣческой, и печалитися о спасеніи, что зависти свойственно. "--- 2)~Демонъ подобному демону не завидитъ, но человѣку; а человѣкъ человѣку, братъ брату, сродный сродному, подобный подобному завидитъ, который долженъ \textit{радоватися съ радующимися, и плакати съ плачущими}, какъ учитъ апостолъ\footnote{Римл.~12,~16.}; чему удивитися, или паче чего оплакать достойно неможно. Такъ"=то сатана ядомъ своимъ заразилъ сердце наше! "--- 3)~Зависть, какъ сказано въ § 78"~мъ, приводитъ ко всякому неблагополучнѣйшему концу тѣхъ, на которыхъ вооружается. "--- 4)~Зависть и тѣмъ завидитъ, отъ которыхъ благодѣяніе получаетъ; ей и доброхотъ не милъ, она и благодѣтеля гонитъ. Кто большій Іудеямъ благодѣтель былъ, какъ Хрістосъ, Спаситель міра? Мертвецевъ ихъ воскресилъ, слѣпцевъ просвѣтилъ, прокаженныхъ очистилъ, и прочая чудесная благодѣянія учинилъ. Но зависть на тое не смотритъ; она совѣтуетъ: \textit{что сотворимъ? яко человѣкъ сей многа знаменія творитъ: аще оставимъ Его тако, вси увѣруютъ въ Него}\footnote{Іоан.~11,~47 и 48.}. Она не разсуждаетъ, ни почитаетъ высокаго добра, но поучается убить благодѣтеля. Питай, одѣвай, береги, защищай, обогащай, утѣшай завистливаго, какъ хощешь: ему тое непріятно; понеже ты благополученъ, тебе вси любятъ, хвалятъ, почитаютъ. Сіе его уязвляетъ сердце, котораго любовь и благодѣяніе твое исцѣлити не можетъ; и дотолѣ уязвлять будетъ, доколѣ не увидитъ тебѣ въ бѣдствіи. "--- 5)~Зависть и тѣхъ, которыми обладаетъ, къ бѣдственному приводитъ концу. Ибо кромѣ того, что вѣчному мученію повинными ихъ творитъ, "--- и временному бѣдствію подвергаетъ. Тако Каинъ стенетъ и трясется; Аманъ на древо, которое Мардохею израильтянину готовилъ, вознесенъ погибаетъ. Откуду Златоустъ святый глаголетъ: «завистливый, хотячи погубить инаго, и самаго себе погубляетъ»\footnote{Бес.~55"~я на Іоан.}. Но, хотя бы не было какого другаго внѣшняго бѣдствія завистливому, довольно ему внутренняго, довольно того, что внутрь снѣдается, мучится и терзается собственнымъ мучителемъ.

\paragraph*{§\:83.} \textit{Посредствіе} противу злаго и пагубнаго сего недуга сіе примѣчается: 1)~Гордость, отъ которой зависть, какъ сказано, раждается, должно съ помощію Божіею отложить, и тако, безъ злаго кореня и злаго плода не будетъ. «Зависть, глаголетъ Августинъ, есть дщерь гордости: умертви матерь, и дщерь ея погибнетъ». "--- 2)~Поучаться въ любви ближняго: тако зависть упадать будетъ. Ибо \textit{любы не завидитъ}, глаголетъ апостолъ\footnote{1~Кор.~13,~4.}. И хотя въ сердце и будетъ ударять пагубная сія стрѣла, но духомъ любве дѣйствію ея станетъ противиться и побуждать себе и нехотящаго къ благодаренію Богу, что ближній въ благополучіи находится: тако бо всякое зло внутреннее исцѣляется и, якоже клинъ клиномъ, какъ говорятъ, изгоняется. Нудить бо себе должны мы ко всякому добру, и не что злое сердце хощетъ, но чего вѣра и совѣсть хрістіанская требуетъ, дѣлать: \textit{нуждницы бо восхищаютъ царствіе небесное}\footnote{Матѳ.~11,~12.}. Тако противимся злобѣ и мщенію, роптанію и хуленію и прочіимъ страстямъ, и нудимся къ терпѣнію и прочему благочестію. Что сначала не безъ трудности, но послѣ съ помощію Божіею удобно будетъ. "--- 3)~Думать и безъ сомнѣнія держать, что въ мірѣ семъ нѣтъ ничего великаго и удивленія достойнаго, и нѣтъ истиннаго блаженства, кромѣ вѣчнаго и небеснаго. А когда въ семъ мнѣніи находиться будемъ, то и зависть ослабѣетъ и недѣйствительна будетъ. Ибо зависть раждается отъ благополучія ближняго; но когда благополучія временнаго, то"=есть, чести, богатства и прочаго за благополучіе истинное поставлять не будемъ, то и завидовать въ томъ не будемъ. Аще убо, земная презрѣвши, будешь искать небесныхъ: ни въ чести, ни въ славѣ, ни въ похвалѣ, ни въ богатствѣ, ни въ благородіи не будешь завидовать, ибо несравненно лучшихъ желаешь. Князь и вельможа не завидитъ похвалѣ сапожника, портнаго, столяра и прочіихъ мастеровъ, понеже далеко лучшую имѣетъ: такъ и временному, подлому и, такъ сказать, мнимому не завидитъ, кто постояннаго и истиннаго, вѣчнаго блаженства ищетъ. Хощеши ли убо отъ мучительной сей не снѣдаться и не вредиться язвы? вмѣняй временная вся какъ ничто, и такъ не будетъ она въ тебѣ имѣть мѣста.

\subsection[Глава 3-я. О гнѣвѣ и злобѣ.]{глава третія.\\\bfseries О гнѣвѣ и злобѣ.}

\begin{quotation}\textit{Всякъ, гнѣваяйся на брата своего всуе, повиненъ есть суду: иже бо аще речетъ брату своему "--- рака, повиненъ сонмищу; а иже речетъ "--- уроде, повиненъ есть гееннѣ огненнѣй}\footnote{Матѳ.~5,~22.}.\end{quotation}
\begin{quotation}\textit{Нынѣ отложите и вы та вся, гнѣвъ, ярость, злобу, хуленіе, срамословіе отъ устъ вашихъ}\footnote{Кол.~3,~8.}.\end{quotation}


\paragraph*{§\:84.} Гнѣвъ есть чувствованіе сердечныя болѣзни, отъ обиды другаго родившіяся, которая или дѣломъ, или словомъ бываетъ.

\paragraph*{§\:85.} Гнѣвъ есть мучительная и лютая страсть, и сокровенна быть не можетъ. Прочія страсти сокрываются удобно, а гнѣвъ утаиться не можетъ. Сердце бо, гнѣвомъ исполненное, по подобію котла кипящаго, извергаетъ гнѣва различные знаки, которые являются на различныхъ членахъ. Отъ гнѣва краснѣютъ и сверкаютъ, какъ искры, очи; отъ гнѣва надымаются жилы, подымаются брови и власы; гнѣвъ дѣлаетъ, что скрежещемъ зубами, пѣну точимъ устами, главою киваемъ, главу крутимъ и вертимъ. Отъ гнѣва лице помрачается, гнѣва дѣйствіе есть, что руки сжимаемъ и плескъ издаемъ, ногами о землю ударяемъ; во гнѣвѣ ударяемъ въ груди, власы и одежду терзаемъ; гнѣвомъ исполненный кричитъ, вопитъ, слезитъ, жалится, хулитъ и часто тое изрыгаетъ, о чемъ послѣ жалѣетъ; словомъ сказать, весь человѣкъ въ гнѣвѣ измѣняется, весь видъ бѣсноватому подобенъ является. Ежели такъ скаредные внѣшніе показуются знаки, ежели такъ безобразно тѣло отъ гнѣва является: что уже внутрь, въ сердцѣ есть, которое такъ смрадный извергаетъ запахъ? Какъ мерзка и безобразна душа гнѣвающагося, какъ гнусна предъ очесами Божіими является, когда только знаки гнѣвающагося предъ вами, которые и сами тоежде зло внутрь носимъ, несносны показуются! Словомъ изобразить не можно скареднаго того бѣдныя души состоянія. И сіе не токмо на большомъ человѣкѣ, но и на маломъ отрочати и младенцѣ примѣтить можно: какъ онъ кричитъ, ярится, всего отвращается, пока гнѣвъ не утолится! Отъ сего видно, какъ великимъ ядомъ отъ діавола сердце человѣческое напоено, какъ великое зло внутрь насъ крыется, чего довольно оплакать не можемъ. А сіе научаетъ насъ непрестанно воздыхать и молиться Богу, чтобы сердце наше, такъ люто испорченное, исправилъ и обновилъ. \textit{Сердце чисто созижди во мнѣ, Боже, и духъ правъ обнови во утробѣ моей}\footnote{Пс.~60,~12.}. Когда же сіе исправится и доброе будетъ, то и плоды его добры будутъ, то"=есть, дѣла, слова и помышленія.

\paragraph*{§\:86.} Гнѣвъ обращается въ злобу и памятозлобіе, когда долго удерживается и питается въ сердцѣ. Чего ради повелѣно намъ скоро его пресѣкать, чтобы въ ненависть и въ злобу не возраслъ, и такъ къ злу большее зло не приложилося. \textit{Солнце да не зайдетъ въ гнѣвѣ вашемъ, ниже дадите мѣсто діаволу}, глаголетъ апостолъ\footnote{Еф.~4,~26 и 27.}. Якоже бо пожаръ, когда вскорѣ не потушится, многіе поядаетъ домы: тако гнѣвъ, когда вскорѣ не пресѣчется, много зла учинитъ и многихъ бѣдъ виною бываетъ. Чего ради, по увѣщанію апостола, должно вскорѣ гнѣвъ изъ сердца изгонять, когда начнется, чтобы, усилившися, и насъ самихъ гнѣвающихся, и тѣхъ, на кого гнѣваемся, болѣе не повредилъ и не погубилъ.

\paragraph*{§\:87.} Какъ гнѣвъ, такъ и злоба \textit{раждается} отъ безмѣрнаго самолюбія. Ибо самолюбецъ во всемъ ищетъ своея корысти, славы и чести; а когда видитъ въ чемъ препятствіе своему намѣренію и желанію, о томъ смущается, печалится и гнѣвается на того, кто препятствіе дѣлаетъ, почему и тщится гнѣвъ свой въ дѣло произвести, то"=есть, отмстить, что свойственно есть злобѣ.

\paragraph*{§\:88.} Злоба различнымъ образомъ совершается: 1)~Отъятіемъ живота, на кого зло бываетъ. Тако Каинъ, завистію и злобою подстрекаемъ, возсталъ на неповиннаго Авеля и умертвилъ единоутробнаго брата своего\footnote{Быт.~4,~8.}. 2)~Біеніемъ, отравою, отъятіемъ здравія, имѣній, поврежденіемъ скотовъ и прочаго. 3)~Ненавистію и гоненіемъ друговъ того, на кого злоба питается въ сердцѣ. Злобный бо, когда не можетъ повредить тому, на кого злобится, гнать и озлобить тщится пріятелей и любителей его. Тако сатана, злобясь на Хріста и ненавидя Его, гонитъ и озлобляетъ хрістіанъ, рабовъ Его, любящихъ Его и отъ Него любимыхъ. Сему злому и злобному духу послѣдуютъ люди, и по подобію его злобу совершаютъ. "--- 4)~Злословіемъ и клеветою. Отсюду"=то бываетъ, что такъ многіи неповинно тяжкія хулы и клеветы страждутъ, котораго зла какъ церковныя, такъ и свѣтскія исторіи преисполнены. Да и въ нынѣшнемъ вѣкѣ многіи на себѣ тоежде злобы дѣйствіе и богомерзкіе ея плоды дознаютъ. 5)~"--- Многіи такою злобою напоенное имѣютъ сердце, что, исполнивше ее, и хвалятся. Великое безуміе хвалиться тѣмъ, о чемъ должно жалѣть! Ближнему своему вредъ сдѣлали; законъ Божій, святый и вѣчный, нарушили; Бога, Вседержителя великаго и страшнаго, прогнѣвали; себе сатанѣ записали, и вѣчному мученію подвергли: и дѣломъ симъ беззаконнымъ хвалятся!.. Такъ злоба помрачаетъ око душевное, что бѣдный человѣкъ явной своей пагубы не видитъ. Дважды грѣшитъ: \textit{грѣшитъ и грѣхомъ хвалится}: вотъ"=де я ему далъ, пусть онъ мене знаетъ. Правда, знаетъ онъ тебе, знаетъ и твою злобу, знаетъ, что ты повредилъ его; но знаешь ли ты себе? гдѣ ты и въ какомъ находишься состояніи? знаешь ли, что ты болѣе себе, нежели его повредилъ? Ты тѣло его, а свою душу погубилъ. Его отъ твоей клеветы гнушаются люди и осуждаютъ, а тебе осуждаетъ Богъ. Ты его безславію и поношенію временному, а себе вѣчному подверглъ. Слыши, что Псаломникъ поетъ тебѣ: \textit{что хвалишися во злобѣ, сильне}? А что далѣе придаетъ, примѣчай: \textit{сего ради Богъ разрушитъ тя до конца; восторгнетъ тя, и преселитъ тя отъ селенія твоего, и корень твой отъ земли живыхъ}\footnote{Пс.~51,~3 и 7.}. Откуду святый Златоустъ глаголетъ: «мы когда на другихъ злобствуемъ и коварствуемъ, тогда противу самихъ себе изощряемъ мечь, и гораздо большими себе, нежели ихъ уязвляемъ ранами»\footnote{Бес.~51"~я на Іоан.}. "--- 6)~Многіе отъ злобы въ такое безуміе и ослѣпленіе приходятъ, что сами себе погубить лучше изволяютъ, нежели мщеніе оставить. Я"=де самъ погибну, а его доступлю!.. Что молвишь и яришься, ослѣпленная и бѣдная тварь? Самъ себе хочешь погубить, чтобы братъ твой погиблъ, за котораго Хрістосъ, Сынъ Божій, умеръ? Осмотрись, какой и чей духъ въ тебѣ сіе дѣйствуетъ? не того ли, который самъ погиблъ, и прародителей нашихъ, а съ ними и насъ погубилъ? Онъ всегда злобою дышетъ на погибель нашу. Хрістосъ хочетъ и тебе и его спасти: а ты хочешь и себе и его погубить. Но смотри, на кого ты возстаешь и вооружаешься, глаголетъ Онъ: \textit{иже нѣсть со Мною, на Мя есть}\footnote{Матѳ.~12,~30.}. Ты своею погибелію ближняго погибели ищешь, и потому не со Хрістомъ, но противу Хріста еси, Который и тебѣ и ближнему твоему спастися хощетъ такъ, что и кровь Свою за то изліялъ; а тѣмъ самымъ показуешь, что единомудрствуешь съ тѣмъ злобнымъ духомъ, который тебе и ближняго твоего и всѣхъ людей всякимъ образомъ хощетъ привлещи въ погибель. И, какъ кажется, горше дѣлаешь, нежели демонъ. Ибо демонъ на демона не возстаетъ, но вси они на единаго человѣка вооружаются и погубити ищутъ: а ты на подобнаго себѣ человѣка возстаешь, и брата твоего погубити хощешь, который тогожде Творца и Господа признаетъ, какъ и ты; тогожде отца по плоти Адама имѣетъ, какъ и ты; тогожде естества есть, какъ и ты; отъ тогожде Искупителя Хріста искупленъ отъ діавола и ада, какъ и ты; тоюжде банею крещенія омовенъ, какъ и ты; къ томужде вѣчному животу позванъ, какъ и ты; "--- однакожъ во огни гнѣва твоего кричишь: я самъ погибну, а его доступлю!.. Правда, самъ ты себе погубишь, когда того желаешь, а не ближняго твоего, когда онъ въ благодати и помощи Вышняго живетъ. Да развѣ ты не знаешь, что есть погибель, которыя какъ себѣ, такъ и ближнему твоему желаешь? Приложи руку свою къ огню, и узнаешь отчасти, что есть погибель. Когда сего огня малаго не стерпишь: како стерпишь вѣчный огнь, который жжетъ, а не свѣтитъ; палитъ, а не снѣдаетъ; мучитъ, а не умерщвляетъ? Симъ огнемъ злобныи, когда не покаются, какъ желѣзо, раскалены будутъ, и извнѣ и внутрь снѣдаться и мучиться безъ конца. Но злобою ослѣпленный не видитъ того, и въ ярости своей кричитъ: я"=де самъ погибну, а его не оставлю!..

О злоба, діавольская дщерь "--- злоба, какъ ты бѣднаго человѣка ослѣпляешь! Самъ себе хочетъ погубить злобный, чтобы ближній его погиблъ. Не можетъ большее быть безуміе, какъ своею погибелью ближняго погибели искать. Многіе отъ злобныхъ воздерживаются отъ рыбъ, молока, яицъ, мяса, чего Богъ не запретилъ, но паче съ благодареніемъ и молитвою употреблять позволилъ; но людей Божіихъ живыхъ хотятъ пожерти! Многіе въ среду и въ пятокъ ничего не ядятъ: но отъ злобы и на едину минуту попоститься не хотятъ: за что Богъ вѣчною мукою претитъ. Такъ великое и страшное зло "--- злоба! такъ бѣдственно плѣняетъ сердце и ослѣпляетъ умъ злоба! И точно порокъ есть, діаволу собственный, который и самъ погиблъ, и другихъ тщится привлещи въ туюжде погибель.

\paragraph*{§\:89.} Понеже у злобныхъ людей ядъ злобы сокровенъ въ сердцѣ, и различныя хитрости тайнымъ образомъ соплетаютъ на того, на кого злобятся, и то тѣмъ, то другимъ, то третьимъ образомъ злобу свою совершить умышляютъ; и потому уберещися ихъ невозможно. Благочестивой душѣ едино остается убѣжище "--- молитва святая, терпѣніе, непоколебимое упованіе на Промыслителя Бога, предъ Которымъ ничто не сокровенно, Который всѣмъ воздаетъ по дѣломъ.

\paragraph*{§\:90.} Злобнымъ людямъ, когда вѣрятъ, что Богъ есть, и всѣмъ воздаетъ по дѣломъ, должно отложить злобу, и ближнему оставить, по словеси Хрістову: \textit{оставите, и оставится вамъ}, "--- и внимать, что Сирахъ глаголетъ\textit{: отмщаяй, отъ Господа обрящетъ отмщеніе, и грѣхи своя соблюдаяй, соблюдетъ. Остави обиду искреннему твоему, и тогда помольшутися, грѣси твои разрѣшатся. Человѣкъ на человѣка сохраняетъ гнѣвъ, а отъ Господа ищетъ исцѣленія? Надъ человѣкомъ, подобнымъ себѣ, не имать милости; а о грѣсѣхъ своихъ молится! Самъ сый плоть, хранитъ гнѣвъ: кто очиститъ грѣхи его? Помяни послѣдняя, и престани враждовати; помяни истлѣніе и смерть, и пребывай въ заповѣдехъ; помяни заповѣди, и не гнѣвайся на ближняго}\footnote{Сир.~28,~1--7.}. Ничто бо человѣка не губитъ такъ, какъ злоба. Хотя бы какія ни имѣлъ добродѣтели, однакожъ всѣ погубляетъ злоба. Да и тое бываетъ, что въ ровъ, который ближнему изрываютъ злобные, сами нечаянно впадаютъ, и прежде времене сами себѣ погибель снискиваютъ. И такъ сбывается на такихъ псаломническое слово: \textit{ровъ изры, и ископа и, и падетъ въ яму, юже содѣла. Обратится болѣзнь его на главу его, и на верхъ его неправда его снидетъ}\footnote{Пс.~7,~16 и 17.}.

\subsection[Глава 4-я. О клеветѣ и осужденіи.]{глава четвертая.\\\bfseries О клеветѣ и осужденіи.}

\begin{quotation}\textit{Не судите, да несудими будете. Имже бо судомъ судите, судятъ вамъ, и въ нюже мѣру мѣрите, возмѣрится вамъ. Что же видиши сучецъ, иже во оцѣ брата твоего, бервна же, еже есть во оцѣ твоемъ, не чуеши? Или како речеши брату твоему "--- остави, да изму сучецъ изъ очесе твоего: и се бервно во оцѣ твоемъ? Лицемѣре, изми первѣе бервно изъ очесе твоего: и тогда узриши изъяти сучецъ изъ очесе брата твоего}\footnote{Матѳ.~7,~1--5.}.\end{quotation}


\paragraph*{§\:91.} Осужденіе, оклеветаніе, злословіе суть сродные пороки и суть плоды необузданнаго языка и сердца, страхомъ Божіимъ неогражденнаго. Осужденіе же не токмо языкомъ, но и мыслію, помаваніемъ, покиваніемъ главою, вздохомъ, смѣхомъ и прочіимъ образомъ бываетъ.

Различныя причины суть, отъ которыхъ сіи пороки происходятъ: 1)~Бываютъ отъ гордости: гордый бо, возносясь и не терпя, чтобы другій ему равенъ былъ, пересуждаетъ и уничтожаетъ его, или, хотя свои грѣхи укрыть, другихъ злословитъ и клевещетъ, чтобы слышащіи думали, что онъ такихъ грѣховъ не имѣетъ, въ какихъ ближняго осуждаетъ. "--- 2)~Иногда бываютъ отъ зависти: понеже завистливый не хощетъ видѣть въ почтеніи и славѣ ближняго своего; тщится честное имя его помрачить и для того безчестнымъ именемъ его порочитъ. "--- 3)~Иногда бываютъ отъ злобы: злобный бо, не имѣя чимъ отмстить тому, на кого злобится, злословіемъ и оклеветаніемъ славу его потемнить тщится. "--- 4)~Бываютъ еще отъ злой привычки, ревнованія, нетерпѣнія и прочаго.

\paragraph*{§\:92.} Причины нѣкоторыя, отводящія отъ сихъ пороковъ, полагаются здѣ: 1)~Богъ весьма запретилъ намъ судить и осуждать ближнихъ нашихъ: \textit{не судите}, глаголетъ Господь, \textit{да несудими будете. Ты почто осуждаеши брата твоего? или ты что уничижаеши брата твоего? Вси бо предстанемъ судищу Хрістову}, глаголетъ апостолъ\footnote{Римл.~14,~10.}. Слѣдственно осуждающіе ближняго заповѣди сея Божія не слушаютъ, и воли Божіей противятся, и потому тяжко грѣшатъ. "--- 2)~Единъ есть Судія Хрістосъ, Сынъ Божій: \textit{Тому судъ весь даде Отецъ небесный}\footnote{Іоан.~5,~22.}; Той всѣхъ будетъ судить безъ лицепріятія. Его праведнаго суда и судящіе и судимые не избѣгнутъ, и пріимутъ вси по дѣломъ своимъ. \textit{Всѣмъ бо явитися намъ подобаетъ предъ судищемъ Хрістовымъ, да пріиметъ кійждо, яже съ тѣломъ содѣла, или блага, или зла}, глаголетъ апостолъ святый\footnote{2~Кор.~5,~10.}. Слѣдственно, кто судить безумно ближняго, тотъ санъ Хрістовъ похищаетъ, "--- что такожде тяжко и страшно. «Не то повелѣно тебѣ, о человѣче, глаголетъ Златоустъ святый, чтобъ ты другихъ судилъ, но чтобы самого себе разсуждалъ: для чегожъ ты похищаешь должность Господа твоего? Ему принадлежитъ судить, а не тебѣ»\footnote{\textit{Бес.~11"~я на посл.~1~къ Кор}.}. И паки: «имѣетъ всякъ грѣшникъ своего судію: не долженъ ты похищать чести единороднаго Сына Божія; Ему единому престолъ суда опредѣленъ»\footnote{\textit{Бес.~42"~я на Матѳ}.}. "--- 3)~Несносно человѣку господину, когда рабъ его предъ нимъ отъ инаго безъ воли его судится и злословится. Всякъ человѣкъ есть Божій рабъ: какъ убо досадно Богу, когда раба Его предъ очесами Его судимъ и злословимъ! Предъ очесами Его, глаголю, судимъ, ибо Богъ вездѣ есть, и вся назираетъ. "--- 4)~Ближній нашъ есть рабъ Божій, своему убо Господеви стоитъ или падаетъ, и потому Божія раба человѣку судить и злословить такожде тяжко; да и нужды намъ до него въ сей части нѣтъ. \textit{Ты кто еси, судяй чуждему рабу? своему Господеви стоитъ или падаетъ. Станетъ же: силенъ бо есть Богъ поставити его}\footnote{Римл.~14,~4.}, глаголетъ апостолъ. "--- 5)~Вси грѣшники есмы предъ Богомъ. Когда въ какомъ грѣхѣ не находишься, то, можетъ быть, уже былъ; когда не былъ, то можешь быть; можешь тяжчайше согрѣшить, нежели ближній твой, котораго судишь за грѣхъ. Общее бо всѣхъ окаянство; внутрь насъ зло крыется; врази наши въ домѣхъ нашихъ суть страсти наши. Всѣ тѣмже случаямъ подлежимъ. Ближній твой сегодня, а ты заутра тоежъ дѣлаешь, хотя не дѣломъ, такъ или словомъ, или мыслію. Безъ Божіей бо благодати ничего не можемъ творити, кромѣ единаго зла. Случай и дѣло показуетъ тое. "--- 6)~Часто бываетъ, что многіе являются предъ нами грѣшными, но внутрь праведны суть, міру юроды, но Хрісту мудры: какъ, напротивъ того, многіе праведными являются, но внутрь грѣшные суть, и потому лицемѣры; міру мудры, но Хрісту юродивы. "--- 7)~Часто худой слухъ отъ злобныхъ, завистливыхъ, гордыхъ, напрасно проносится, и такъ осуждаемый часто напрасно терпитъ поношеніе. "--- 8)~Кто осуждаетъ, самъ будетъ осужденъ, по словеси Хрістову: \textit{не осуждайте, да не осуждени будете}\footnote{Лук.~6,~37.}. "--- 9)~Ближнему великая обида чрезъ то, не иначе какъ бы отъ кого біенъ былъ жезломъ, да еще и болѣе. Ибо раны тѣлесныя скорѣе заживутъ, нежели язвы душевныя. Тѣло жезломъ, а душа поноснымъ словомъ уязвляется. Отъ сего уязвленія печаль, а отъ печали немощь, отъ немощи смерть. И чимъ большее поношеніе, тѣмъ большее души уязвленіе и печаль послѣдуетъ. Человѣкъ бо, какъ честолюбивъ, лучше желаетъ лишиться богатства, нежели добраго имени: богатства бо лишившися, можетъ его паки сыскать, а имя доброе весьма трудно возвратить. "--- 10)~Чимъ честнѣйшее и высшее есть поносимое лице, тѣмъ большая ему язва, а поношающему большій грѣхъ. "--- 11)~Когда начальникъ какій осуждается и злословится, уменьшается ему почтеніе отъ подначальныхъ; а отъ непочтенія презрѣніе, отъ презрѣнія непослушаніе, отъ непослушанія всякое безстрашіе, безчиніе и нестроеніе послѣдуетъ въ нихъ, какъ всякому сіе удобно можно разумѣти, и потому хотя всякаго грѣхъ есть осуждать, и грѣхъ тяжкій, но начальника злословить далеко большій грѣхъ. "--- 12)~Часто случается, что хотя подлинно кто согрѣшилъ, но уже покаялся, а кающемуся Богъ прощаетъ, и для того страшно человѣку осуждать того, кого Богъ прощаетъ и оправдаетъ. "--- 13)~Какъ есть тяжкій и мерзкій осужденія и оклеветанія порокъ, въ подтвержденіе приводится здѣ святаго Златоустаго ученіе.

«Хотя бы мы никакого другаго не дѣлали беззаконія, сіе едино бы могло насъ ввергнуть въ геенскую бездну, что мы, тяжкихъ бревенъ въ очахъ нашихъ не чувствуя, строгій всегда производимъ судъ надъ погрѣшностьми чужими, и цѣлую жизнь нашу иждиваемъ въ любопытномъ пересматриваніи и пересуживаніи чужихъ дѣлъ, не разсуждая того, что грозитъ Хрістосъ: \textit{имже судомъ судите, судятъ вамъ; и въ нюже мѣру мѣрите, возмѣрится вамъ}»\footnote{Матѳ.~7,~2. \textit{Въ кн.~1"~й о умиленіи къ Димитрію монаху}.}.

«Не осуждай другаго, но старайся самаго себе исправить, дабы ты самъ не былъ достоинъ осужденія. Ибо большую еще пріимешь казнь, если будешь другихъ обвинять въ томъ, чего и самъ не дѣлаешь, и осуждать за тотъ грѣхъ, которому и самъ ты подверженъ. Ежели и праведнымъ человѣкамъ не дозволяется осуждать другихъ, то много паче грѣшные должны воздерживать себе отъ осужденія».

«Хотя бы ты, осуждая ближняго, и праведное что на него говорилъ, и хотя бы извѣстное его дѣло порочилъ, однако и за то не избѣжишь наказанія; ибо не по дѣламъ его, но по словамъ твоимъ будетъ судить тебе Богъ. Глаголетъ бо Онъ: \textit{отъ словесъ твоихъ осудишися}\footnote{Матѳ.~12,~37.}. Фарисей праведное и всѣмъ извѣстное говорилъ, однако и за то принялъ наказаніе. Ежелижъ извѣстныхъ дѣлъ на обличеніе другихъ не должно говорить, то много паче сумнительныхъ»\footnote{Бес.~42"~я на Матѳ.}.

«Какая польза съ того, что мы воздерживаемся отъ птицъ и рыбъ, когда другъ друга угрызаемъ и снѣдаемъ? Угрызаетъ же и снѣдаетъ ближняго плоть всякъ клеветникъ и поноситель»\footnote{Бес.~3"~я на отшествіе Флавіана епископа.}.

«Хотя бы мы пепелъ ѣли, то и такъ строгое житіе никакой пользы намъ не принесетъ, ежели не воздержимся отъ поношенія: \textit{ибо не входящее во уста сквернитъ человѣка, но исходящее изъ устъ}»\footnote{тамъ же.}.

«Какой есть зла родъ, который бы не происходилъ отъ поношенія? Отсюду раждаются ненависти, отсюду бываютъ вражды и несогласія, отсюду происходятъ худыя подозрѣнія, которыя къ безчисленнымъ злымъ подаютъ причину»\footnote{Бес. на пс.~100"~й.}.

«Кто любитъ клевету, тотъ служитъ діаволу».

«Какъ проповѣдникъ истины словомъ своимъ многихъ людей спасаетъ, такъ напротивъ клеветникъ тѣмъ же словомъ премногихъ погубляетъ, произнося слова погруженія и метая рѣчи, на погибель человѣческую употребляя. Притомъ же не токмо клеветникъ тѣлеса, но и самыя души убиваетъ, когда льстивымъ своимъ языкомъ ложныя и беззаконныя мнѣнія влагаетъ онымъ»\footnote{\textit{Бес. на пс.~61"~й}.}.

«Не токмо поносить ближнему, но и другихъ поносящихъ ему слушать не должно»\footnote{\textit{Бес.~3"~я на отшествіе Флавіана епископа}.}

До здѣ Златоустъ.

Разсуждайте сія, злорѣчивые, которые за грѣхъ не поставляете злословіемъ и клеветою ближняго уязвлять.

\paragraph*{§\:93.} Чтобы сихъ пороковъ избѣжать, пользуетъ слѣдующая примѣчать: 1)~Всякому на себе смотрѣть, и свои пороки и грѣхи предъ глаза положить, и ихъ очищать тщиться: ибо за нихъ имать быти истязанъ предъ судомъ Божіимъ, аще не покается. Такое собственныхъ грѣховъ разсмотрѣніе не попуститъ чужихъ пороковъ изыскивать. Какъ бо немощный, видя свою немощь, тщится о себѣ, а не о другихъ, такъ, кто свою душевную видитъ немощь, старается отъ ней свободиться, и первѣе тщится исправить себе, нежели другихъ, первѣе бервно изъ очесе своего изъяти, нежели сучецъ изъ очесе брата своего. "--- 2)~Помнить, что за осужденіе такое самъ будетъ судимъ. "--- 3)~Берещися отъ разговоровъ непотребныхъ, на которыхъ только люди пересуждаются, и то того, то другаго имя и честь терзается. "--- 4)~Удаляться отъ тѣхъ, которые привычку сію злую имѣютъ, какъ прокаженныхъ, которые злосмраднымъ своимъ запахомъ и другимъ вредятъ. "--- 5)~Брату падшему или падающему духомъ любве соболѣзновать, и отъ его падежа самому осторожно поступать, и за него молиться милосердному Богу, чтобы падшаго возставилъ, а себе въ тойжде грѣхъ пасти не попустилъ. "--- 6)~Злую привычку имѣющимъ молиться со Псаломникомъ: \textit{положи, Господи, храненіе устомъ моимъ}\footnote{Пс.~140,~3.}, и вышеписанныя причины помнить.

\paragraph*{§\:94.} Тѣмъ, которые терпятъ клевету и поношеніе, ради утѣшенія слѣдующая примѣчать должно: 1)~Поношеніе и оклеветаніе бываетъ или праведное, или ложное. \textit{Праведное}, когда мы подлинно въ томъ виноваты, въ чемъ поносятъ намъ, и потому достойная страждемъ: почему должно исправляться, чтобы поношеніе упразднилося, и ложное было. \textit{Ложное} есть поношеніе, когда мы не виноваты въ томъ, въ чемъ намъ поносятъ: и сіе поношеніе съ радостію терпѣть должно и утѣшаться надеждою вѣчныя Божія милости. Ктомужъ, хотя въ томъ не виноваты, въ чемъ поносятъ намъ, но въ другомъ согрѣшили; и для того должно терпѣть. "--- 2)~Можетъ быть, что сами кого оклеветали и осудили, и ради того, \textit{въ нюже мѣру мѣрили мы, мѣрится намъ}, по словеси Господню\footnote{Матѳ.~7,~2.}, и \textit{имиже кто согрѣшаетъ, сими и мучится}\footnote{Прем.~11,~17.}. Языкъ злорѣчивый злорѣчіемъ и наказуется. "--- 3)~Когда изсякла любовь, ненависть и злоба умножилась, другъ друга озлобляютъ, обманываютъ, поносятъ, клевещутъ: чего болѣе и тебѣ ожидать, кромѣ озлобленія, въ такъ лютое время? "--- 4)~Злорѣчіемъ и клеветою смиряемся; самомнѣніе, высокоуміе фарисейское и гордость, какъ высокій идолъ, въ сердцѣ нашемъ низлагается: тако бо дается намъ злорѣчивый языкъ, аки \textit{ангелъ сатанинъ, пакости} плоти нашей \textit{дѣющій, да не превозносится}, но да покаряется духу смиренія. "--- 5)~Къ святому Писанію, источнику утѣшенія, прибѣгай, которое вездѣ терпѣніе ублажаетъ и подкрѣпляетъ. "--- 6)~Злословіе и безчестіе бываетъ тебѣ ко искушенію сердца твоего, которымъ показуется, что въ сердцѣ твоемъ крыется "--- кротость или гнѣвъ: таковъ бо человѣкъ бываетъ пріемшій досаду, каковъ въ сердцѣ имѣется. Сей случай тебѣ представляетъ, что въ сердцѣ твоемъ крыется, "--- кротость или гнѣвъ. Ежели кротость въ сердцѣ имѣется, удобно стерпишь поношеніе: ежели въ сердцѣ имѣется злость, отъ поношенія огнь ярости и желаніе мщенія, или злословію злословіе послѣдуетъ. И такъ поношеніе и клевета учитъ тебе состояніе сердца твоего познавать и исправлять. "--- 7)~Хрістосъ, Сынъ Божій, безгрѣшный, неповинно терпѣлъ всякія поношенія, и намъ \textit{оставилъ образъ, да послѣдуемъ стопамъ Его}. Сему образу послѣдовали вси святіи. На Хріста убо и подражателей Его святыхъ взирай, и подкрѣпленіе терпѣнію пріемли. \textit{Смотри еще ниже}.

\subsection[Глава 5-я. О лжи, лести и лукавствѣ.]{глава пятая.\\\bfseries О лжи, лести и лукавствѣ.}

\begin{quotation}\textit{Не ревнуй лукавнующимъ, ниже завиди творящимъ беззаконіе. Зане яко трава скоро изсшутъ, и яко зеліе злака скоро отпадутъ. Не ревнуй, еже лукавновати; зане лукавнующіи потребятся}\footnote{Пс.~36,~12 и 8.}.\end{quotation}


\paragraph*{§\:95.} Лесть, ложь, лукавство суть пороки сродные, и суть такожде діаволу собственны, ибо діаволъ есть \textit{отецъ лжи} и лукавства\footnote{Іоан.~8,~44.}. Симъ обучаетъ и служителей своихъ, людей, которые волю его злую и нравы въ себѣ изобразуютъ. Сіи пороки тѣми людьми обладаютъ, которые иное языкомъ говорятъ, и иное на сердцѣ имѣютъ. Таковыхъ людей называютъ обыкновенно двоедушными; потому что какъ бы двѣ души имѣютъ, то есть, внутреннюю и внѣшнюю; внѣшнею душею съ людьми обходятся и людей обманываютъ, а внутреннею себе берегутъ. Сего рода люди обходятся съ ближними ласково, гладко, тихо, но лестно и коварно, чтобы въ сердца ихъ, по подобію татя, вкрасться могли. Они знаютъ и смиренными себе показывать, но внутрь все иное; они временемъ слезятъ, воздыхаютъ, но да сердцѣ иное. А ротитися и клятися первые, которая клятва ихъ только на языкѣ, и тѣмъ болѣе людей обманываютъ. Они въ томъ подражаютъ отцу своему, отцу лжи, діаволу, который иногда преобразуется въ ангела свѣтла, чтобы тако моглъ удобѣе человѣка обмануть и погубить. Сихъ людей слова какъ медъ, а дѣло самое какъ ядъ. Хотящіе отравить кого ядъ намазываютъ медомъ, чтобы злодѣйство не познано было; такъ сіи души, чтобы удобѣе уловить простосердечнаго, ядъ злобы своея, какъ медомъ, словами мягкими и ласковыми прикрываютъ. Такъ учинилъ Каинъ, который вызвалъ брата своего Авеля на поле, аки бы ради нѣкоей доброй причины, но въ сердцѣ замышлялъ убійственною на него рукою вооружиться, и пролить кровь неповинную\footnote{Быт.~4,~8.}. Такъ поступилъ Іуда предатель, который Хріста устами привѣтствовалъ и лобызалъ: \textit{радуйся, Равви! и облобыза Его}\footnote{Матѳ.~26,~49.}; но самою вещію въ руцѣ беззаконныхъ Господа и Учителя своего предавалъ. Такъ и нынѣ многіе привѣтствуютъ насъ: \textit{здравствуй, здравствуй}! а сердцемъ дышутъ на погибель нашу и тако \textit{умягчаютъ словеса паче елеа, но та суть стрѣлы}, по словеси Псаломника\footnote{Пс.~54,~22.}. Изрядно описалъ ихъ Сирахъ: \textit{устнама своима усладитъ врагъ, и много пошепчетъ и речетъ добро глаголя: очима своима прослезится, а сердцемъ своимъ усовѣтуетъ вринути тя въ ровъ, и егда обрящетъ время, не насытится крове. Аще срящутъ тя злая, ту обрящеши его первѣе себѣ, и яко помогаяй подсѣчетъ пяту твою: покиваетъ главою своея, и восплещетъ рукама своима и много пошепчетъ, и измѣнитъ лице свое}\footnote{Пс.~12,~15--18.}. И, какъ кажется, таковыхъ людей нравы крайне развращенны, и не ино что суть, какъ нравы діавольскіе, который лжетъ и обманываетъ, льститъ и лукавнуетъ, а тѣмъ тое только намѣреваетъ, чтобы погубить человѣка: такъ и они въ томъ и поучаются, чтобы прельстить ближняго и посмѣяться ему.

\paragraph*{§\:96.} Обществу нѣтъ большей язвы, какъ лестцы и лукавцы. Понеже 1)~вѣрность, безъ которой союзъ общества не стоитъ, быть тамо не можетъ, но вмѣсто того невѣріе вступитъ. Отъ того послѣдуетъ, что другъ другу ни въ чемъ не будутъ вѣрить, другъ друга опасаться, бояться, блюстися, сокрываться, другъ друга за врага имѣть: отъ чего вся злая въ обществѣ. "--- 2)~Не можетъ тамо быть миръ, но вмѣсто того взаимная ненависть, вражда, ссоры, и прочая; ибо лестцамъ и лукавцамъ и сіе свойственно есть, что они рѣчи единаго къ другому переносятъ и пересказываютъ не такъ, какъ слышали, и не въ такой силѣ толкуютъ, какъ говорены, но иначе, и такъ единому на другаго подозрѣнія влагаютъ: отъ чего дружбы разрушеніе, и отъ того взаимная вражда бываетъ. Тако они тщатся всѣмъ показаться приятелями, но внутрь суть истинные всѣхъ врази. "--- 3)~Лесть и лукавство, какъ кажется, ввело и умножило богопротивныя клятвы и напрасныя въ самыхъ подлыхъ вещахъ страшнаго Божія имени призыванія, какъ"=то: \textit{ей Богу! на то Богъ! видитъ Богъ! и свидѣтель Богъ}! и проч. Понеже многіе, отъ лживыхъ и лестныхъ обманувшеся, не вѣрятъ и тѣмъ, которые правду говорятъ: того ради люди принуждены вѣрность своихъ словъ призываніемъ имене Божія утверждать, что уже и въ самыхъ подлыхъ вещахъ дѣлаютъ. А дѣти отъ родителей, малые отъ старыхъ, низшіе отъ высшихъ тогожде научаются, что и въ обычай беззаконный и пагубный вошло, и за грѣхъ не почитается. Такъ ложь и лукавство плевелы свои непотребныя всѣваетъ, и не попускаетъ расти пшеницѣ благихъ дѣлъ!

\paragraph*{§\:97.} Двоедушные лукавцы, сколько ни хитрятъ, однакожъ лукавство свое и нехотя оказываютъ, наипаче тогда, когда или виномъ разгорячившеся, или гнѣвомъ сильнымъ распалившеся, извергаютъ тое, что въ сердцѣ сокровенно было: тогда \textit{отъ избытка сердца} ихъ \textit{уста} ихъ \textit{глаголютъ}; тогда они оказываютъ себе, что они такое. Піянство бо и сильная холера или гнѣвъ премогаютъ всякую хитрость, и какъ истинный и вѣрный свидѣтель, возвѣщаютъ тайну сердца; въ такомъ случаѣ временемъ и самъ человѣкъ не знаетъ, что дѣлаетъ и говоритъ. Ибо тогда не умъ, но сердце, виномъ или гнѣвомъ распалившеся, дѣйствуетъ.

\paragraph*{§\:98.} Простая душа, лести и лукавства незнающая, да внимаетъ Псаломника увѣщанію: \textit{не ревнуй лукавнующимъ, ниже завиди творящимъ беззаконіе: зане яко трава скоро изсшутъ, и яко зеліе злака скоро отпадутъ}. И съ Давидомъ да молится: \textit{изми мя, Господи, отъ человѣка лукава, отъ мужа неправедна избави мя, иже помыслиша неправду въ сердцѣ, весь день ополчаху брани, изостриша языкъ свой яко зміинъ: ядъ аспидовъ подъ устнама ихъ}, и проч.\footnote{Пс.~139,~1--3.} Да слышатъ и лукавцы тоежъ слово: \textit{яко трава скоро изсшутъ, и яко зеліе злака скоро отпадутъ}, и да внимаютъ, что тойжде пророкъ написалъ: \textit{погубиши вся глаголющія лжу: мужа кровей и льсти гнушается Господь}\footnote{Пс.~5,~7.}. И аще вѣруютъ, что есть Богъ, \textit{сердца и утробы испытуяй, и воздаяй комуждо по дѣломъ его}, чистосердно съ ближнимъ да обходятся; и что языкомъ говорятъ, на сердцѣ имѣть, и что словомъ объявляютъ, дѣломъ самымъ показывать да научатся.

\subsection[Глава 6-я. О праздности.]{глава шестая.\\\bfseries О праздности.}

\begin{quotation}\textit{Въ потѣ лица твоего снѣси хлѣбъ твой}\footnote{Быт.~3,~19.}.\end{quotation}
\begin{quotation}\textit{Изыдетъ человѣкъ на дѣло свое, и на дѣланіе свое до вечера}\footnote{Пс.~103,~23.}.\end{quotation}
\begin{quotation}\textit{Егда бѣхомъ у васъ, сіе завѣщавахомъ вамъ, яко аще не хощетъ кто дѣлати, ниже да ястъ}, глаголетъ апостолъ\footnote{2~Сол.~3,~10.}.\end{quotation}


\paragraph*{§\:99.} Праздность, или удаленіе отъ трудовъ, или лѣность, есть сама собою грѣхъ, ибо противна есть заповѣди Божіей, которая велитъ намъ въ потѣ лица нашего ясти хлѣбъ нашъ. Еще бо праотцу нашему Адаму сказано отъ Бога: \textit{въ потѣ лица твоего снѣси хлѣбъ твой, дондеже возвратишися въ землю, отъ неяже взятъ еси}\footnote{Быт.~3,~19.}, которое повелѣніе и насъ сыновъ его касается. И апостолъ святый и ясти тому запрещаетъ, кто не хощетъ дѣлати. Слѣдственно въ праздности живущіи и чужими трудами питающіися непрестанно грѣшатъ; и дотоль грѣшить не престанутъ, доколь въ благословенные труды не отдадутъ себе. Отъ сего выключаются немощные и престарѣлые, которые хотя бы и хотѣли трудиться, но не могутъ.

\paragraph*{§\:100.} Не токмо въ праздности быть и житіе безъ добрыхъ трудовъ провождать грѣшно, какъ сказано, но сіе и многихъ золъ причиною бываетъ. Ибо сердце человѣческое праздно быть не можетъ, но какими нибудь мыслями занято бываетъ. А понеже оно склонно ко всякому злу; и къ праздному сердцу, которое никакими полезными трудами не занято, не иначе какъ къ дому праздному, пометенному и украшенному, удобно приступаетъ душевный врагъ діаволъ, и мыслями злыми, какъ вредными плевелами, наполняетъ его, и въ самое дѣйствіе производить поучаетъ: оттуду бываетъ, что праздность много беззаконій раждаетъ, а ко всякому добру безплодна бываетъ. Отсюду пьянство, многихъ золъ и соблазновъ виновное; отсюду всякія блудныя дѣла; отсюду злыя бесѣды, пересужденія, осужденія, насмѣянія, злословія, хуленія; отсюду частыя пиршества и симъ послѣдующая злая, какъ"=то: хищенія, грабленія, клятвопреступленія, и проч., отсюду картежныя игры и съ ними совокупленные обманы, безчинія, ссоры, драки и прочія беззаконія. Праздность вымышляетъ излишнія роскоши, которыя безъ обиды ближняго и разоренія общества не могутъ быть; праздность замышляетъ непотребныя строенія и прочія симъ подобныя изобрѣтенія, которыя безъ кровавыхъ слезъ бѣдныхъ быть не могутъ; праздность научаетъ разбивать, воровать, похищать, насилія чинить, лгать, льстить, обманывать: ибо праздный, не имѣя чимъ питаться, устремляется на похищеніе чужихъ трудовъ, или явно, или тайно, или лестно. Тако \textit{праздность научаетъ многой злобѣ}, по словеси премудраго Сираха\footnote{Быт.~33,~28.}. Откуду и Соломонъ глаголетъ: \textit{въ похотѣхъ есть всякъ праздный}\footnote{Притч.~13,~4.}.

\paragraph*{§\:101.} Праздность не токмо душу погубляетъ, но и тѣлу вредъ наноситъ. 1)~Въ праздности живущіи всякимъ недугамъ и немощамъ подлежатъ. Какъ бо вода растлѣвается, которая теченія не имѣетъ, такъ тѣло человѣческое безъ движенія и трудовъ портится и ослабѣваетъ. Ибо кровь, отъ которой вся цѣлость тѣлесная зависитъ, въ неимущемъ движенія отъ трудовъ, загустѣваетъ, и такъ по малу согниваетъ. "--- 2)~Не трудящійся не можетъ въ сладость и пищи принимать. Труды бо суть, какъ бы нѣкая поджога къ воспріятію и варенію пищи въ желудкѣ, безъ которыхъ желудокъ отвращается пищи. "--- 3)~Какъ по трудахъ сонъ сладокъ, такъ безъ трудовъ безпокоенъ бываетъ. "--- 4)~Гуляки и волочаги подлежатъ посмѣянію и порицанію людей. "--- 5)~Понуждаются въ бѣдности и нищетѣ жить. Ибо \textit{находитъ на нихъ, аки золъ путникъ, убожество: скудость же аки благій течецъ}, глаголетъ Соломонъ\footnote{Притч.~6,~11.}.

\paragraph*{§\:102.} Чтобы праздности и оной послѣдующихъ золъ убѣжать, должно твердо держать: 1)~Что время дражайшее есть паче всякаго сокровища. "--- 2)~Какъ слова сказаннаго возвратить, такъ и времени потеряннаго сыскать невозможно. "--- 3)~Потерявшіе напрасно время будутъ жалѣть, и малѣйшаго времени къ покаянію поищутъ, не иначе, какъ жаждущіе студеныя воды; но не обрящутъ, когда время будетъ суда, а не покаянія; строгости, а не помилованія. "--- 4)~Слѣдуетъ непремѣнно отвѣтъ дать и за самое время, туне потерянное. Ибо настоящее время есть торгъ, на которомъ таланты, отъ Господа нашего намъ данные, должно трудами съ помощію Божіею умножать, чтобы съ лѣнивымъ рабомъ не услышать страшнаго онаго опредѣленія Господня: \textit{неключимаго раба вверзите во тьму кромѣшнюю}\footnote{Матѳ.~26,~30.}. Чего ради должно всякому послушать увѣщанія премудраго Соломона: \textit{иди ко мравію, о лѣниве, и поревнуй видѣвъ пути его, и буди онаго мудрѣйшій: онъ бо, не сущу ему земледѣльцу, ниже нудящаго его имущій, ниже подъ владыкою сый, готовитъ въ жатву пищу, и многое въ лѣто творитъ уготованіе. Или иди ко пчелѣ, и увѣждь, коль мудрая есть дѣлательница, дѣланіе же коль честное творитъ: еяже трудовъ царіе и простіи во здравіе употребляютъ, любима же есть всѣми и славна: аще силою и немощна сущи, но премудростію почтена произведеся. Доколѣ, о лѣниве, лежиши? когда же отъ сна востанеши? Мало убо спиши, мало же бдиши, мало же дремлеши, мало же объемлеши перси рукама: потомъ же найдетъ тебѣ аки золъ путникъ убожество: скудость же аки благій течецъ}\footnote{Прем.~6,~11.}.

\paragraph*{§\:103.} Какъ не всякій трудъ полезенъ, такъ не всякая праздность порочна. Злые тые и беззаконные труды, которые ради беззаконныхъ дѣлъ подъемлются. Злый трудъ и пагубный злобнаго, который ради ближняго козни и хитрости соплетаетъ, сѣти простираетъ и ровъ погибели копаетъ; пагубныя тщанія, хищниковъ, которые по путямъ и стезямъ скитаются, хотя обнажить путниковъ; беззаконный подвигъ есть лукавца, который тщится прельстить и обмануть брата своего; тяжкая и мерзкая работа мамонѣ; безполезная и душепагубная болѣзнь и трудъ завистливыхъ, которые ради добра ближняго снѣдаются и мучатся. Какъ сіи и прочіе симъ подобные труды не похвальны, но порочны, такъ отъ сихъ упраздненіе есть достохвальное. Коль блаженный покой есть, когда умъ отъ злыхъ и душевредныхъ мыслей, сердце отъ похотей лукавыхъ упокоевается, очи не смотрятъ ничего, уши не слушаютъ ничего, языкъ и уста не глаголютъ ничего, руки не дѣлаютъ ничего, что святому Божію закону противно! Блаженная сія праздность есть благословенный покой. Сего Творецъ нашъ и Господь отъ насъ требуетъ. Сію субботу не токмо на каждой седмицѣ, но и на каждый день, часъ и минуту праздновать должно намъ. Къ сему спасительному и тишайшему упокоенію, въ которомъ не токмо внѣшніе уды тѣлесные, но и внутреннія душевныя силы отъ вредныхъ дѣлъ упокоеваются, и совѣсть сладко почиваетъ, Самъ Хрістосъ призываетъ: \textit{пріидите ко Мнѣ вси труждающіися и обремененніи, и Азъ упокою вы}\footnote{Матѳ.~11,~28.}.


\subsection[Глава 7-я. О пьянствѣ.]{глава седмая.\\\bfseries О пьянствѣ.}

\begin{quotation}\textit{Горе востающимъ заутра, и сикеръ гонящимъ, ждущимъ вечера: вино бо сожжетъ я. Съ гусльми бо и пѣвницами и тимпаны и свирельми вино піютъ; на дѣла же Господня не взираютъ, и дѣлъ руку Его не помышляютъ. Убо плѣнени быша людіе мои, за еже не вѣдѣти имъ Господа}\footnote{Ис.~5,~11--13.} .\end{quotation}
\begin{quotation}\textit{Кому горе? кому молва? кому судове? кому горести и свары? кому сокрушенія вотще? кому сини очи? Не пребывающимъ ли въ винѣ, и не назирающимъ ли, гдѣ пирове бываютъ}\footnote{Притч.~23--29 и 30.}.\end{quotation}
\begin{quotation}\textit{Не упивайтеся виномъ, въ немже есть блудъ}\footnote{Еф.~5,~18.}.\end{quotation}


\paragraph*{§\:104.} Вино, какъ и всякая вещь созданная, есть добро, ибо на пользу нашу отъ Создателя нашего устроено намъ. \textit{Всякое бо созданіе Божіе есть добро}, глаголетъ апостолъ, \textit{и ничтоже отметно, со благодареніемъ пріемлемо}\footnote{1~Тим.~4,~4.}. И въ книзѣ Бытія пишется: \textit{и видѣ Богъ вся, елика сотвори: и се добра зѣло}\footnote{1,~31.}. \textit{Вино бо въ мѣру пріемлемо, полезно животу человѣчу, радованіе сердца и веселіе души, піемо во время прилично}, глаголетъ Сирахъ\footnote{31,~31 и 33.}; ибо тако употребляемо, печальнаго увеселяетъ, немощнаго подкрѣпляетъ. Сего ради и къ Тимоѳею святому написалъ апостолъ: \textit{ктому не пій воды: но мало вина пріемли: стомаха ради твоего и частыхъ твоихъ недуговъ}\footnote{1~Тим.~5,~23.}. Откуду и Павелъ святый, не пить вина, но упиваться виномъ запрещаетъ, глаголя: \textit{не упивайтеся виномъ}. Иное бо есть пить вино, иное упиваться виномъ. Слѣдственно грѣшатъ 1)~тѣ, которые гнушаются виномъ и не употребляютъ его, не ради воздержанія, но ради того, что аки бы грѣшно его употреблять. Тако бо, хуля созданіе, отъ Бога человѣку въ пользу человѣколюбно устроенное, и Создателя хулою касаются, не иначе, какъ кто хуля художество, напр., образъ написанный, и самаго художника, то"=есть, живописца хулитъ. "--- 2)~Которые виномъ гнушаются, и братію свою, которые употребляютъ его, презираютъ, а о себѣ высоко нѣчто мечтаютъ: я"=де отъ роду не пью вина!.. Будто въ томъ только и благочестіе состоитъ, что не пить вина? Сатана не піетъ вина и хлѣба не ястъ, яко безплотный; но души человѣческія, яко левъ, пожираетъ, и своея вѣчныя погибели участниками дѣлаетъ, и есть всѣхъ злѣйшій и всякія злобы начало: такъ и не піющіи вина могутъ быть подражатели его. Не піешь вина "--- хорошо; но не хуль вина, и братію піющую не презирай. Ибо когда не піешь вина, а высоко о себѣ мечтаешь, всегда высокоумія духомъ напоенъ еси, что предъ Богомъ мерзко. Не вино убо, но піянство, не употребленіе вина, но злое употребленіе порочно и вредно. Злѣ его употребляетъ человѣкъ, когда не во время и выше мѣры употребляетъ. Тако не токмо вино, но и хлѣбъ, воду, огнь и прочее созданіе Божіе злѣ употребляетъ человѣкъ. Тако злѣ употребляетъ и свои члены, какъ"=то: языкъ къ злословію, осужденію и оклеветанію; руки къ хищенію, грабленію и неправедному біенію; чрево къ объяденію и піянству, и тако сими, какъ бы какими орудіями, душу свою убиваетъ.

\paragraph*{§\:105.} Не вино причиною піянства бываетъ: ибо ежели бы вино того причиною было, то бы вси его употребляющіи піяницы были, но противное самая вещь показуетъ; многіе бо употребляютъ вино, но трезвы суть. Причиною убо піянства бываетъ: 1)~злое и невоздержное сердце, какъ и прочіихъ грѣховъ. «Не вино производитъ піянство, глаголетъ Златоустъ святый, но невоздержаніе»\footnote{\textit{Бес.~1"~я къ народу антіохійскому}.}. 2)~Праздность, какъ выше сказано. 3)~Частыя пиршества, компаніи, усильныя потчиванія. 4)~Со злыми и невоздержными обращеніе. "--- Отъ частыхъ же повтореній дѣлается страсть и злый обычай.

\paragraph*{§\:106.} Піянство, какъ само собою есть грѣхъ, яко \textit{піяницы царствія Божія не наслѣдятъ}, по ученію апостола\footnote{1~Кор.~6,~10.}; и Хрістосъ глаголетъ: \textit{внемлите себѣ, да не когда отягчаютъ сердца ваша объяденіемъ и піянствомъ}\footnote{Лук.~21,~34.}, такъ многихъ и тяжкихъ грѣховъ виновно бываетъ. Оно ссоры, драки, и отъ того послѣдующія кровопролитія и убійства производитъ; оно буесловія, кощунства, хуленія, оно ближнему досады и обиды дѣлаетъ. Оно научаетъ лгать, льстить, чужая грабить и похищать, чтобы было чимъ страсть довольствовать; оно возжигаетъ гнѣвъ и ярость; оно дѣлаетъ, что люди въ нечистотѣ, какъ свиніи въ блатѣ, валяются; словомъ сказать, изъ человѣка скотомъ, изъ словеснаго безсловеснымъ дѣлаетъ, такъ, что не токмо внутреннее состояніе, но и внѣшній человѣческій видъ часто перемѣняетъ. Откуду святый Златоустъ глаголетъ: «діаволъ"=де ничего такъ не любитъ, какъ роскоши и піянства, понеже никто такъ злой его воли не исполняетъ, какъ піяница»\footnote{Бес.~58"~я на Матѳ.}.

\paragraph*{§\:107.} Піянство не токмо душевныхъ, но и тѣлесныхъ временныхъ золъ причиною бываетъ. 1)~Тѣло разслабляетъ, и въ немощь приводитъ. Откуду пишется: \textit{въ винѣ не мужайся, многихъ бо погуби вино}\footnote{Сир.~31,~29.}. 2)~Въ убожество и нищету приводитъ. \textit{Дѣлатель піянивый не будетъ богатъ}, глаголетъ Сирахъ\footnote{19,~1.}. 3)~Славу и имя доброе отъемлетъ; напротивъ того, въ безславіе, презрѣніе и омерзеніе приводитъ: никѣмъ бо такъ люди не гнушаются, какъ піяницею. 4)~Домашнимъ, сродникамъ, друзьямъ скорбь и печаль, врагамъ посмѣяніе дѣлаетъ. 5)~Ни къ какому званію неспособнымъ рачителя своего устрояетъ. И хотя въ какомъ званіи будетъ піяница, болѣе бѣдъ и напастей строитъ, нежели пользы обществу. Святый Златоустъ, изображая бѣдствія и пагубы піянства, глаголетъ: «Піянство есть самоизвольное бѣснованіе, откровеніе помышленій, посмѣятельная бѣда, болѣзнь, смѣха достойная, демонъ добровольный, и проч.»\footnote{\textit{Сл. о воскресеніи}.}

\paragraph*{§\:108.} Чтобы отъ піянства остерещися, пользуетъ примѣчать слѣдующая: 1)~Юнымъ не дозволять піянаго питія пить, понеже юные удобно привыкаютъ, и чего въ юности научатся, того и чрезъ все житіе держаться будутъ. 2)~Не дозволять имъ съ піяницами и развращенными водиться. 3)~Взрослымъ и мужескаго вѣка достигшимъ безъ нужды не пить вина. 4)~Отъ злыхъ компаній и пиршествъ удаляться. 5)~Помнить имъ, что отъ сея страсти весьма трудно отстать. И многіе отъ той и въ той самой страсти погибаютъ душею и тѣломъ. 6)~Привыкшимъ къ сей страсти крѣпко противу мучительства ея вооружиться, стоять, не поддаваться, молить и призывать всесильную Божію помощь. 7)~На память приводить случающіяся въ піянствѣ бѣды, и сравнить трезваго житія состояніе съ состояніемъ піянаго. 8)~Помышлять имже, что многіе піяными во снѣ умираютъ, и отъ сего свѣта на оный переходятъ безъ всякаго чувства, и потому безъ покаянія, и проч.

\subsection[Глава 8-я. О сребролюбіи и лихоиманіи.]{глава осьмая.\\\bfseries О сребролюбіи и лихоиманіи.}

\begin{quotation}\textit{Блюдите и хранитеся отъ лихоимства: яко не отъ избытка кому животъ его есть, отъ имѣнія его}, глаголетъ Хрістосъ\footnote{Лук.~12,~15.}.\end{quotation}
\begin{quotation}\textit{Никтоже можетъ двѣма господинома работати: любо единаго возлюбитъ, а другаго возненавидитъ: или единаго держится, о друзѣмъ же нерадити начнетъ. Не можете Богу работати и мамонѣ}\footnote{Матѳ.~6,~24.}.\end{quotation}
\begin{quotation}\textit{Хотящіи богатитися впадаютъ въ напасти и сѣть, и въ похоти многи несмысленны и вреждающія, яже погружаютъ человѣка во всегубительства и погибель. Корень бо всѣмъ злымъ сребролюбіе есть: егоже нѣцыи желающе, заблудиша отъ вѣры, и себе пригвоздиша болѣзнемъ многимъ. Ты же, о человѣче Божій, сихъ бѣгай}\footnote{1~Тим.~6,~9 и 10.}.\end{quotation}


\paragraph*{§\:109.} Сребролюбіе здѣ разумѣется не токмо единаго сребра любленіе, но всякаго земнаго имѣнія, стяжанія и богатства ненасытное желаніе.

\paragraph*{§\:110.} Сребролюбіе, какъ и всякая страсть, въ сердцѣ у человѣка имѣетъ свое мѣсто и сердцемъ обладаетъ. Слѣдственно 1)~не токмо тотъ сребролюбецъ есть, который самою вещію всякимъ образомъ богатство собираетъ и хранитъ у себе, не удѣляя требующимъ; но и тотъ, кто хотя не собираетъ и не имѣетъ, но ненасытно желаетъ его. "--- 2)~Не токмо тотъ лихоимецъ есть и хищникъ, кто самымъ дѣломъ чуждое похищаетъ; но и тотъ, кто чуждаго желаетъ неправедно, который грѣхъ есть противу заповѣди десятыя: \textit{не пожелай}. Уже бо, что до воли его надлежитъ, лихоимствуетъ и похищаетъ чуждое; а что самымъ дѣломъ не исполняетъ того, тое дѣлается не отъ его стороны, но отъ внѣшняго препятствія, которое его къ похищенію чуждаго добра не допускаетъ. "--- 3)~Не всякъ богатый есть сребролюбецъ, но кто любитъ сребро, кто сердцемъ къ сребру приложился, кто \textit{къ богатству прилагаетъ сердце}\footnote{Пс.~61,~11.}. "--- 4)~Нищій, хотя бы и ничего у себе не имѣлъ, но имѣетъ къ богатству любовь, истинный есть сребролюбецъ. И потому не тотъ, кто много имѣетъ, но кто много желаетъ, и не тотъ, кто богатъ, но кто къ богатству сердцемъ прилѣпляется, порочится. Многіе были, какъ и нынѣ имѣются, богатые и славные, но Богу угодили, и потому сребролюбцы не были. Сребролюбецъ бо угодить Богу не можетъ, ибо не Богу, но своей страсти угождаетъ. «Не худая вещь есть богатство, глаголетъ Златоустъ святый, но худая вещь есть сребролюбіе, худая вещь есть лихоимство. Иное есть лихоимецъ, а иное богатый: лихоимецъ не есть богатъ, лихоимцу много недостаетъ; а кому много недостаетъ, тотъ никогда не можетъ быть богатымъ. Лихоимецъ есть стражъ денегъ, а не господинъ, рабъ, а не владѣтель. Скорѣе онъ можетъ удѣлить другому своея плоти, нежели сокровеннаго сребра»\footnote{\textit{Бес.~2"~я къ нар. антіох}.}.

\paragraph*{§\:111.} Сребролюбіе и лихоимство есть страсть ненасытная. Чимъ бо болѣе собираетъ сребролюбецъ богатства, тѣмъ болѣе желаніе и похоть къ богатству въ немъ умножается. «Сребролюбіе, глаголетъ святый Златоустъ, есть ненасытное піянство. Какъ піяные чимъ болѣе вина піютъ, тѣмъ большую чувствуютъ жажду: такъ сребролюбцы чрезмѣрнаго своего желанія утолить никакъ не могутъ; но сколько имѣніе свое умножаемое видятъ, столько желаніемъ горятъ и не прежде отъ злаго своего желанія отстанутъ, какъ развѣ уже въ самую бѣдственную попадутъ пропасть»\footnote{\textit{22"~я на кн. Быт}.}. На другомъ мѣстѣ глаголетъ: «Лихоимцы и хищники хуждшіи отъ звѣрей. Ибо звѣри, глаголетъ, насытившеся, престаютъ похищать; а сіи никогда насытиться не могутъ»\footnote{\textit{9"~я на 1"~е посл. къ Кор}.}. И паки: «аще желаеши богатитися, никогда не престанешь симъ желаніемъ уязвлятися; желаніе бо сіе безконечно есть»\footnote{\textit{12"~я на посл. къ Римл}.}.

\paragraph*{§\:112.} Сребролюбецъ и лихоимецъ, хотя много богатства имѣетъ, окаяннѣйшій паче всѣхъ есть, ибо покоя и безопасности имѣть никогда не можетъ; внутрь бо себе всегда носитъ мучителя, которому бѣдно работаетъ, и отъ сего непрестанно и вездѣ мучится. Добрѣ поучаетъ святый Златоустъ: «Лихоимцы и хищники нигдѣ покоя и безопасности не имѣютъ, понеже внутрь себе всегда имѣютъ брань и враговъ»\footnote{\textit{Бес. на пс.~4"~й}.}. «Какъ море, глаголетъ тойжде отецъ, никогда безъ волнъ быть не можетъ, такъ сребролюбиваго человѣка сердце не можетъ быть безъ печали, попеченія, страха и мятежа»\footnote{\textit{39"~я на Матѳ}.}. «И потому лихоимцы вездѣ пріемлютъ свои наказанія "--- въ жизни, въ смерти и по смерти», тойжде учитель святый разсуждаетъ\footnote{\textit{На пс.~42"~й}.}. И тако, во мнѣнію святаго отца, лихоимецъ мучится чрезъ все житіе, мучится при смерти, мучится и по смерти. Мучится \textit{чрезъ все житіе}: понеже въ печали и страсѣ всегда находится. Въ печали "--- ради того, чего еще не имѣетъ у себе, а желаетъ; сребролюбцу бо и лихоимцу много недостаетъ. Въ страсе "--- ради того, чтобы не лишиться того, что имѣетъ. Мучится \textit{при смерти}: понеже 1)~и не хотя лишается любимаго своего сокровища, идѣже сердце и увеселеніе свое имѣлъ; 2)~что оно инымъ въ руки достается; 3)~что достанется, можетъ быть, тѣмъ, коимъ не хощетъ; 4)~что слѣдуетъ предъ судомъ Божіимъ истязану быть, и за всякую неправедную копѣйку отвѣтъ дать. Мучится лихоимецъ и \textit{по смерти}: ибо \textit{хищницы царствія Божія не наслѣдятъ}\footnote{1~Кор.~6,~10.}, но яко должники Божіи и ближнихъ своихъ, яко преступники Божія закона и братіи своей обидчики, ввергнутся въ темницу, огнемъ и жупеломъ дымящуюся, гдѣ чрезъ всю вѣчность безвозвратно будутъ платить долгъ свой. Откуду правильно заключить можно, что бѣдное житіе имѣетъ лихоимецъ. «Онъ, по мнѣнію святаго Златоустаго, бѣднѣйшую жизнь проводитъ, нежели Каинъ, всегда пребывая въ печали и страхѣ: въ страхѣ ради того, что имѣетъ, а въ печали ради того, чего еще не имѣетъ»\footnote{\textit{Бес.~31"~я на 1"~е посл. къ Кор}.}.

\paragraph*{§\:113.} Сребролюбецъ и лихоимецъ хотя истиннаго Бога исповѣдуетъ, но самою вещію отступилъ отъ Него: устами бо только Его почитаетъ, но сердцемъ другому господину, то"=есть, мамонѣ работаетъ, служитъ, покаряется. \textit{Никтоже бо можетъ двѣма господинома работати: не возможно Богу работати и мамонѣ}, глаголетъ Хрістосъ. Богъ и мамона суть два господина, но весьма противные, и противное повелѣвающіе. Богъ повелѣваетъ: \textit{просящему у тебе дай}: мамона: что имѣетъ, возми. Богъ повелѣваетъ: \textit{алчущаго напитай}: мамона: оставь его. Богъ глаголетъ: \textit{нагаго одѣй}: мамона вопреки: и послѣднее совлеки. Богъ глаголетъ: \textit{продай имѣніе, и дай милостыню}, мамона говоритъ: собирай, и береги ради себе. Богъ глаголетъ: \textit{не укради}, мамона говоритъ: доставай, какъ можешь. Богъ глаголетъ: \textit{помилуй бѣднаго}, мамона: что тебѣ до него нужды? Лихоимецъ, оставляя Бога, мамону слушаетъ, убо тѣмъ самымъ, какъ заповѣди Божіи, такъ и Бога заповѣдавшаго отвергается. Почему и отъ апостола называется лихоиманіе \textit{идолослуженіемъ}\footnote{Кол.~3,~5.}, и лихоимецъ \textit{идолослужитель}\footnote{Еф.~6,~5.}. Коль же бѣдственно есть и безстыдно, ради страсти, живаго и безсмертнаго Бога, отъ Котораго животъ и все блаженство наше зависитъ, отвергаться!

\paragraph*{§\:114.} Сребролюбіе не токмо само въ себѣ грѣхъ тяжкій есть, но и есть корень всѣхъ золъ, по словеси апостола: \textit{корень всѣмъ злымъ сребролюбіе есть}. Отъ сего происходитъ всякая неправда, многоразличное хищеніе и воровство. И въ какомъ ни будетъ званіи и чинѣ язва сія, многоразличная злая наноситъ. 1)~Когда подлый человѣкъ будетъ сребролюбецъ и лихоимецъ, научаетъ его окрадывать лавки, клѣти, житницы, церкви, уводить коней, засады дѣлать по дорогамъ, разбой чинить, насиловать, убивать и прочая. "--- 2)~Въ купечествѣ научаетъ лгать, обманывать, всякую чинить неправду, худую вещь за добрую, дешевую за дорогую продавать, мзду наемничу удерживать, беззаконную лихву или чрезмѣрный процентъ брать, воду въ вино мѣшать, и такъ законъ Божій нарушать и ближняго обманывать, и проч. "--- 3)~Ежели помѣщикъ, имѣющій крестьянъ, будетъ сребролюбецъ, крестьянъ излишними оброками обременяетъ. Отъ чего послѣдуетъ, что иные крайнюю скудость въ пищѣ претерпѣвать, иные полунаги, безъ одѣянія скитаться понуждаются, "--- отъ чего слезы и плачевныя жалобы на него возносятся; а иные, не имѣя чимъ оброки оные платить и себе и домашнихъ пропитать, принуждены итить на разбой, хищенія, воровства, какъ самая вещь показываетъ; иные "--- оставлять церковное собраніе въ праздничные дни, отлучаться отъ общія молитвы и слушанія Божія слова, и доставать въ запрещенные тые дни работою себѣ и домашнимъ потребная. Всему тому безчеловѣчное господина лакомство виновно. "--- 4)~Когда язва сія заразитъ тѣхъ, которые въ судебныхъ засѣдаютъ мѣстахъ, послѣдуетъ ужасное и плача достойное попраніе правды, нарушеніе присяги, которую предъ страшнымъ и всевѣдцемъ Богомъ и святымъ Его Евангеліемъ учинили, и обѣщались свято и ненарушимо хранить правду, подъ опасеніемъ мстительнаго гнѣва Божія въ противномъ дѣлѣ. А отъ сего вся злая послѣдуютъ, и обществу нестерпимыя раны. Понеже злые люди и насильники, видя попраніе правды, въ большее безстрашіе приходятъ, и на большія злодѣянія устремляются, обнадеживая себе тѣмъ, что они, хотя и обличаются въ злодѣйствѣ, могутъ отъ законныя казни мздою свободиться, что и бываетъ. А отъ сего болѣе умножается разбоевъ, хищенія, воровства, грабленія, злодѣйства, наглостей, раззоренія, убійства и прочіихъ тяжкихъ и несносныхъ обществу золъ и беззаконій. Отъ сихъ своевольствій и наглостей добрымъ и бѣднымъ людямъ, сиротамъ, и вдовицамъ и прочіимъ безпомощнымъ большее утѣсненіе и бѣдность послѣдуетъ. Отъ чего болѣе умножается плача, слезъ, жалобъ, на небо восходящихъ, и всенародную казнь, какъ"=то: глады, пожары, трясенія земли, нашествія иноплеменниковъ, войны, страшныя кровопролитія и прочія казни наводящихъ. Нѣтъ большей обществу язвы паче судей, мздоиманіемъ и лихоиманіемъ растлѣнныхъ. Какъ черви внутрь древо грызутъ, такъ они общество снѣдаютъ, и въ погибель приводятъ, какъ отъ вышереченныхъ всякъ можетъ видѣть. И тѣмъ наипаче вредительнѣйшіи, что видъ добрыхъ и вѣрноподданныхъ на себѣ носятъ, но въ самой вещи инако имѣются. Хотя они о себѣ и мечтаютъ, что искореняютъ злодѣйство, но неправдою своею и лихоиманіемъ путь ко всякому злодѣйству открываютъ; и хотя мнятся служить обществу, но въ самомъ дѣлѣ его разоряютъ съ злодѣями и разбойниками, которыхъ искоренить мнятся. Добрѣ Златоустъ святый о нихъ глаголетъ: «лихоимцы суть разбойники, засѣдающіи при дорогахъ и вещи проходящихъ похищающіи, и въ домахъ своихъ, какъ въ земныхъ пещерахъ, имѣнія другихъ сокрывающіи»\footnote{Слово 1"~е о Лазарѣ.}. Паче же самыхъ злодѣевъ явныхъ тяжчае грѣшатъ, какъ изъ слѣдующихъ можешь видѣть. \textit{Первое}. Какъ злодѣи похищаютъ чуждое, такъ и они: злодѣи явно или тайно, а они лестно, въ чемъ не разнствуютъ отъ злодѣевъ. Хищеніе бо равно есть хищеніе, какое ни есть "--- явное или тайное, или лестное. \textit{Второе}. Злодѣи не дѣлали присяги быть судіями другихъ, а они присягали не токмо правду хранить, но насаждать и утверждать, въ чемъ превосходятъ злодѣевъ, яко присягою обязавшіися. \textit{Третіе}. Злодѣи суть люди грубые, невѣжды, а они Божій и монаршій законъ мнятся вѣдать, "--- и сіе ихъ зло умножаетъ, яко тяжчайшій грѣхъ отъ вѣдѣнія, нежели отъ невѣдѣнія. \textit{Четвертое}. Злодѣи люди подлые, а они чиновные, и ради того образъ добрыхъ дѣлъ и прочіимъ показывать должны; но они образъ злыхъ дѣлъ показываютъ, въ чемъ они какъ злодѣямъ, такъ и прочіимъ къ злодѣянію поводъ подаютъ, яко начальники и мнимыя свѣтила. \textit{Пятое}. Злодѣи часто скудостію понуждаются на злое дѣло, а они отъ единаго лакомства ненасытнаго, и въ семъ превосходятъ злодѣевъ: яко меньшее зло есть отъ скудости, нежели отъ довольствія похищать. Разсуди всякъ, не смѣха ли достойное было бы дѣло, ежели бы богачъ на воровствѣ пойманъ былъ? Воистину вси бы тому удивилися, ибо отъ довольствія на такъ безчестное и беззаконное дѣло дерзаетъ. Такъ они удивленія и смѣха достойное дѣло творятъ, когда, довольны будучи, на такъ безчестное и безбожное дѣло безстыдно устремляются. \textit{Шестое}. Злодѣи каковы суть, такими и отъ людей имѣются, а они за добрыхъ и вѣрныхъ отчества слугъ почитаются. Смотри и здѣ, не большее ли зло есть быть злымъ, и за добраго почитаться, быть злодѣемъ, и добрымъ слыть, неправду дѣлать, и за праведнаго имѣться, отчество разорять, и защитникомъ и хранителемъ его славиться? Воистину несказанное зло "--- и беззаконновать и похвалу имѣть. Явные злодѣи нѣкое наказаніе и здѣ въ вѣкѣ семъ пріемлютъ, потому что ихъ за злодѣевъ и люди имѣютъ, и ими гнушаются: а сихъ за добрыхъ, честныхъ и вѣрноподданныхъ почитаютъ, и тако они грѣшатъ, и отъ людей славятся; зло творятъ, и благодѣтелями почитаются. Тако они подобны яблокамъ, которыя внѣ красны, но внутрь гнилы, смрадны и вредны; подобны яду, медомъ намазанному, который внѣшнимъ видомъ услаждаетъ вкушающаго, но смертоносно заражаетъ. Ибо лице не умаляетъ, много паче не уничтожаетъ согрѣшенія; но паче чимъ высшее лице, тѣмъ большій его грѣхъ есть. Беззаконное дѣло подлому похищать, но беззаконнѣе чиновному; большій грѣхъ секретарю, большій и того судящему, нежели канцеляристу и писчику неправду дѣлать. Такожде и образъ хищенія не уничтожаетъ, ни умаляетъ хищенія; но явное ли, или тайное, или лестное хищеніе будетъ, равно есть хищеніе; равно заповѣдь Божія сія "--- \textit{не укради} "--- разоряется, равно Законодавецъ презирается, преслушается и отвергается. "--- 5)~Когда подъ митрою епископскою зло сіе крыется, то такожде всѣхъ тѣхъ золъ, о которыхъ выше (въ 4"~й статьѣ) помянуто, виновно бываетъ; сверхъ того еще къ злу зло прилагается. Ибо непродаемую Божію благодать продавать, и недостойныхъ удостоивать учитъ, отъ чего соблазны паче, нежели созиданіе церкви святой, послѣдуютъ, какъ всѣмъ сія истина явна. Словомъ: корчемствовать, а не стадо Хрістово пасти страсть сія научаетъ его, и со Іудою Хріста продавать, а не проповѣдывать. "--- 6)~Ежели военачальника сердцемъ мамона обладаетъ, то приводитъ его къ тому, что такожде, клятвенное свое обѣщаніе презрѣвши, измѣняетъ Государю своему; не стыдится, ни ужасается продавать отчества своего, въ которомъ родился, воспитался, живетъ и, какъ въ нѣдрѣ матернемъ, со всею своею фамиліею упокоевается; столько крови человѣческой проливать, столько тысящей неповинныхъ людей погублять не усумнѣвается. Тако во всякомъ чинѣ и званіи серебролюбіе и лихоиманіе безчисленная злая содѣловаетъ!.. И, какъ кажется, да и всякъ можетъ видѣть, что никто такъ не вредитъ обществу, какъ лихоимцы. Отъ сихъ оно болѣе стонетъ, воздыхаетъ и изнемогаетъ, нежели отъ иноплеменныхъ враговъ. Ибо отъ иноплеменныхъ, какъ явныхъ, бережется, и защищаетъ себе оружіемъ; а отъ сихъ уберещися не можетъ, понеже внутрь суть. Добрѣ Златоустъ святый глаголетъ, что сребролюбіе вси бѣды въ свѣтѣ производитъ: оно обагряетъ кровію море, и часто окровавляетъ неправедно судейскіе мечи; оно вооружаетъ разбойниковъ, оно дѣлаетъ убійцъ отцевъ и матерей\footnote{\textit{Бес.~23"~я на 1"~е посл. къ Кор}.}.

\paragraph*{§\:115.} Сребролюбіе и лихоимство не токмо другимъ злая дѣлаетъ, но и своихъ рачителей въ бѣдственные случаи ввергаетъ. Тако Гіезій, пророка Божія Елиссея отрокъ, который съ Неемана Сиріянина, Божіею благодатію исцѣлившагося и въ домъ возвращающагося, сребро и ризы тайно взявъ, тою проказою, которою онъ прокаженъ былъ, праведнымъ судомъ Божіимъ поражается\footnote{4~Цар.~5,~20--27.}. Тако Іуда предатель, который безцѣннаго Хріста, Сына Божія, за тридесять сребренниковъ продать не убоялся, сребролюбія достойную казнь пріемлетъ, отъ апостольскаго лика и числа вѣрныхъ извергается, и самъ себе удавленіемъ умерщвляетъ\footnote{Матѳ.~26,~15,~16,~47--49.}. Тако и нынѣ тойжде праведный Божій судъ постигаетъ воровъ, хищниковъ, разбойниковъ, клятвопреступниковъ, продающихъ за сребро правду Божію и Бога отвергающихся, измѣнниковъ, предателей и губителей отчества и прочіихъ лихоимцевъ. А хотя кто и избѣжитъ временныя казни, не вси бо беззаконники здѣ наказуются по недовѣдомымъ Божіимъ судьбамъ; но не избѣжитъ вѣчныя, которая непремѣнно, какъ прочіимъ беззаконникамъ, такъ и лихоимцамъ слѣдуетъ. Тогда они и за послѣдній кодрантъ въ геенскомъ огнѣ будутъ платить, но никогда заплатить не возмогутъ.

Отъ вышереченныхъ можешь видѣть, читатель, 1)~что страсть лихоиманія есть крайне развращенныхъ людей, у которыхъ безбожіе крыется въ сердцѣ, хотя устами и исповѣдуютъ Бога; и есть знакъ человѣка, преобразившагося въ хищнаго звѣря, который безъ разбора на всякое животное нападаетъ, чтобы насытиться плоти и крове его; или паче худшій самыхъ звѣрей, какъ учитъ святый Златоустъ. Ибо звѣри, насытившеся, болѣе не устремляются на животныхъ; а они никогда насытиться не могутъ, но всегда алчутъ и жаждутъ чуждаго добра; и чимъ болѣе собираютъ, тѣмъ болѣе желаютъ и похищаютъ. А тако видишь, что есмь лихоимецъ? Есть врагъ Божій, врагъ человѣческій, врагъ и самому себѣ. \textit{Врагъ Божій}: понеже безстрашно законъ Божій нарушаетъ и Законодавца презираетъ. \textit{Иже бо восхощетъ другъ быти міру, врагъ Божій бываетъ}, учитъ апостолъ\footnote{Іак.~4,~4.}. \textit{Врагъ человѣческій}: яко человѣковъ обнажаетъ и разоряетъ. \textit{Врагъ самому себѣ}: яко душу свою вѣчному огню и мученію предаетъ. "--- 2)~Лихоиманіе опаснѣйшее паче прочіихъ беззаконій. Блуднику, злобному, піяницѣ и прочіимъ нужно только отстать отъ грѣховъ и покаяться, чтобы спастися: а лихоимцу не токмо отстать должно отъ лихоиманія, но и восхищенное возвратить тому, у кого похитилъ, или, когда того невозможно учинить, расточить, что злѣ собралъ, и тако каятися; иначе бо ему каяться невозможно. Слыши, что чрезъ пророка Богъ глаголетъ: \textit{и егда реку нечестивому "--- смертію умреши; и обратится отъ грѣха своего, и сотворитъ судъ и правду, и залогъ отдастъ, и восхищенное возвратитъ, беззаконникъ въ заповѣдехъ жизни ходити будетъ, еже не сотворити неправды, жизнію живъ будетъ, и не умретъ}\footnote{Іез.~33,~14 и 15.}. Смотри, что похищенное должно возвратить; развѣ бы хищникъ въ такую скудость пришелъ, что крайне не имѣетъ чимъ возвратить похищеннаго, а въ чувство пришедъ, хощетъ каяться, и все бы отдать желалъ, что бы ни было, "--- въ такомъ случаѣ отъ милосердаго Бога желаніе вмѣсто истиннаго возвращенія пріемлется: иначе не истинное покаяніе есть, но притворное и ложное, и не иное что, какъ прельщеніе и умягченіе грызущія совѣсти. "--- 3)~Чимъ кто болѣе лихоимствуетъ и похищаетъ, тѣмъ болѣе беззаконія и погибели себѣ собираетъ. \textit{По жестокости твоей и непокаянному сердцу, собираеши себѣ гнѣвъ въ день гнѣва и откровенія праведнаго суда Божія, Иже воздастъ коемуждо по дѣломъ его}, глаголетъ апостолъ\footnote{Римл.~2,~5 и 6.}. "--- 4)~Какъ вредительно есть во всякомъ званіи, то"=есть, духовномъ и свѣтскомъ, лихоиманіе, яко многія бѣды и соблазны чинитъ, но паче въ духовномъ: яко духовные словомъ Божіимъ должны и другихъ отъ беззаконнаго того дѣла и прочіихъ отвращать; но они сами дѣлаютъ, безстрашно дерзаютъ и другимъ къ тому поводъ подаютъ. "--- 5)~Чимъ высшее лице будетъ лихоимецъ, тѣмъ большія и множайшія въ отчествѣ бѣды и соблазны дѣлаетъ. Отъ его бо и низшіе и подвластные его тогожде беззаконнаго дѣла научаются. Ради чего"=де намъ не брать (такъ они беззаконно поучаются и вносятъ), когда высшіе насъ и начальники берутъ? Неужели будутъ насъ судить въ томъ, что сами дѣлаютъ; за тое казнить, въ чемъ сами виноваты? Какъ"=де ихъ совѣсть не будетъ обличать и осуждать, когда насъ въ томъ, что сами дѣлаютъ, будутъ судить и наказывать? "--- Правильное, по мнѣнію твоему, заключеніе есть, лихоимче, но не хрістіанское, но беззаконное, но безбожное. Не будутъ они тебе судить, но будетъ Богъ судить и ихъ и тебе, беззаконнаго послѣдователя ихъ; избѣжишь человѣческой казни, но не избѣжишь Божіей, временнаго, но не вѣчнаго наказанія. Не укажешь тогда на нихъ, вотъ"=де и они дѣлали тое; но какъ они, такъ и ты по дѣломъ своимъ воспріимешь. Не заступитъ тебе человѣкъ, и ничто не поможетъ тебѣ, когда Богъ будетъ судить тебе, и \textit{представитъ предъ лицемъ твоимъ грѣхи твоя Тотъ, Котораго ты словеса отверглъ вспять, и видя татя, теклъ еси съ нимъ}\footnote{Пс.~49,~21,~17 и 18.}.

\paragraph*{§\:116.} Но чтобы пагубной сребролюбія и лихоиманія страсти не поработиться, должно всякому помнить апостольское слово: \textit{ничтоже внесохомъ въ міръ сей, явѣ, яко ниже изнести что можемъ; имѣюще же пищу и одѣяніе, сими довольни будемъ}, и проч.\footnote{1~Тим.~6,~7 и 8.}, "--- и Хрістово оное страшное реченіе: \textit{кая польза человѣку, аще міръ весь пріобрящетъ, душу же свою отщетитъ}\footnote{Матѳ.~16,~26.}? А плѣнившіися лихоиманіемъ и хотящіи каятися да внимаютъ, что Закхей мытарь, каяся, ко Хрісту, въ домъ его пришедшему, сказалъ: \textit{ставъ же Закхей рече ко Господу: се полъ имѣнія моего, Господи, дамъ нищимъ; и аще кого чимъ обидѣхъ, возвращу четверицею}. Почему и слышитъ отъ Хріста: \textit{яко днесь спасеніе дому сему бысть}\footnote{Лук.~19,~8 и 9.}.

\subsection[Глава 9-я. О лихвѣ и процентѣ.]{глава девятая.\\\bfseries О лихвѣ и процентѣ.}

\begin{quotation}\textit{Взаимъ дайте, ничесоже чающе}, глаголетъ Хрістосъ\footnote{6,~35.}.\end{quotation}
\begin{quotation}\textit{Хотящаго отъ тебе заяти, не отврати}\footnote{Матѳ.~5,~42.}.\end{quotation}


\paragraph*{§\:117.} Лихва или процентъ есть, когда заимодавецъ съ должника своего сверхъ даннаго ему взаимъ что нибудь беретъ. Изрядно простый народъ лихву называетъ ростомъ; понеже данное заимообразно денегъ число въ лихвѣ ростетъ. Напр. дается сто рублей взаимъ, а отбирается сто шесть, или сто десять, или болѣе рублей, и чимъ болѣе держитъ должникъ взаимъ данное сребро, тѣмъ болѣе ростетъ и сумма даннаго сребра.

\paragraph*{§\:118.} Хотя многіе лихвы или процента за грѣхъ не поставляютъ, однакожъ или есть то вымыслъ сребролюбивыхъ сердецъ, которыя совѣсть свою грызущую такимъ образомъ тщатся утѣшить и страсть сребролюбія оставить не хотятъ; или что не внимаютъ святому Божію слову, и знать о томъ не хотятъ; или что лихва вошла въ обычай, который такъ ослѣпляетъ душевные глаза, что человѣкъ грѣха своего и отъ того пагубы своея не видитъ. Но какъ кто хочетъ, да разсуждаетъ, однакожъ многіе и несумнительные доводы показуютъ, что лихва или процентъ есть дѣло беззаконное и грѣхъ тяжкій, какъ въ слѣдующемъ § покажется.

\paragraph*{§\:119.} Доводы, показующіе, что лихва есть грѣхъ: 1)~Святое Божіе слово запрещаетъ лихву брать. \textit{Взаимъ дайте, ничесоже чающе}, глаголетъ Хрістосъ\footnote{См. еще Исх.~22,~25; Второз.~23,~19.}. Все же сіе грѣхъ есть, что противно Божію слову. Убо лихва есть грѣхъ. "--- 2)~Лихву между великими грѣхами тоежде Божіе слово полагаетъ, какъ"=то въ псалмѣ 14"~мъ глаголется, въ которомъ между изчисляемыми беззаконіями и лихва полагается: \textit{сребра своего не даде въ лихву, и мзды на неповинныхъ не пріятъ}\footnote{ст.~5.}. И у пророка Іезекіиля въ главѣ 18"~й, стихѣ 13"~мъ, между великими грѣхами полагается. Убо лихва есть грѣхъ великій. "--- 3)~Богъ за лихву казнію грозитъ, какъ читаемъ у тогожде пророка Іезекіиля въ главѣ 22"~й, ст.~12,~13,~14 и проч. Убо лихва есть грѣхъ великій, за который казнь слѣдуетъ. "--- 4)~Лихва противится Божіей заповѣди не токмо десятой, которая учитъ \textit{чуждаго не желать}, но и осьмой, которая запрещаетъ самою вещію \textit{чуждаго касаться}. Понеже пріемлющій лихву не токмо желаетъ, но и самою вещію чужое похищаетъ. «Какъ рыболовъ, глаголетъ Василій великій, червячка къ удицѣ прицѣпляетъ, чтобы рыбу моглъ тако обмануть и поймать: тако лихвовзиматель ближняго деньгами своими привлекаетъ, который неосторожно съ червякомъ и удицу поглощаетъ, и тако якоже рыба умерщвляется»\footnote{Бес. на пс.~14"~й.}. "--- 5)~Отъ чужаго убытка прибыли себѣ искать есть грѣхъ великій, какъ всякъ сіе за истину признаетъ. Лихву пріемлющіе отъ чужаго убытка прибыли себѣ ищутъ; ибо что съ должниковъ сверхъ даннаго имъ взаимъ берутъ, въ томъ имъ убытокъ, а себѣ прибыль дѣлаютъ; ихъ въ большую нищету приводятъ, а себѣ большее богатство собираютъ, съ ихъ нищеты богатятся: убо лихву брать есть грѣхъ великій. "--- 6)~Отъ чужихъ трудовъ, безъ труда и безъ воли трудящихся, пользы себѣ искать есть грѣхъ. Лихву пріемлющіе безъ всякаго своего труда отъ чужихъ трудовъ хотятъ пользоваться и богатиться: убо грѣшатъ. "--- 7)~Лихва противна естественному закону, который учитъ не творить того ближнему, чего себѣ не хочешь. Но никто не хощетъ лихвы или процента давать, убо и самъ съ другаго не долженъ брать. Естественный же, непротивный Божію закону, но паче Божій законъ есть, законъ естественный, изображенный на скрижалѣхъ. Ибо и Хрістосъ глаголетъ: \textit{вся убо, елика аще хощете, да творятъ вамъ человѣцы, тако и вы творите имъ}. И придалъ: \textit{се бо есть законъ и пророцы}\footnote{Матѳ.~7,~12.}. Убо лихва, яко естественному и Божію, написанному пророками, апостолами и самимъ Хрістомъ протолкованному закону противная, есть дѣло беззаконное. "--- 8)~Лихва противится любви хрістіанской, которая требуетъ, чтобы мы не токмо чужаго не желали, но и отъ своихъ благихъ требующимъ удѣляли безъ возврата, по словеси Хрістову: \textit{просящему у тебе дай}\footnote{5,~42.}. Убо лихва есть грѣхъ. "--- 9)~Лихва знаменіе есть самолюбія, которое хощетъ, чтобы мы все только сами у себе имѣли, и себе самихъ только снабдѣвали, а не другихъ. Самолюбіе же есть грѣхъ и корень всякаго грѣха. Убо лихва, яко злый плодъ злаго кореня самолюбія, есть зло. "--- 10)~Лихва есть политичное или лестное хищеніе. Ибо что воры тайно, хищники нагло, тое лихву пріемлющіи лестно похищаютъ; тѣ руками, а сіи деньгами чужое добро берутъ; вмѣсто рукъ, деньги свои употребляютъ, которыми, какъ руками, чужое добро себѣ привлекаютъ. Чего сами стыдятся, или боятся закона гражданскаго, дѣлать, тое чрезъ посредство денегъ безчувственныхъ совершаютъ. Скажи, пожалуй, къ тебѣ говорю, который нищетою ближняго богатишься, и убыткомъ брата твоего пагубно пользуешься, "--- скажи, за что брать, когда ты не трудился? Вѣдь мзда за трудъ дается, а ты никакого труда не поднялъ ради ближняго твоего; за чтожъ убо берешь сверхъ того, что ему взаимъ далъ? Деньгамъ твоимъ равно какъ въ сундукахъ лежать, такъ и по рукамъ ходить; и тебѣ какъ при себѣ ихъ держать, такъ и людямъ взаимъ давать убытка нѣтъ, но паче и людямъ и тебѣ польза. Людямъ польза въ томъ, что ихъ нуждѣ и требованію послужишь; тебѣ польза душевная, что къ ближнему любовь покажешь, и заповѣдь Хрістову исполнишь: \textit{хотящаго отъ тебе заяти не отврати}.

\paragraph*{§\:120.} Не терпитъ лихвопріиматель, и глаголетъ: «я"=де не насильно беру, но по воли должника; онъ мнѣ самъ даетъ, и я его не принуждаю». Не насильно берешь, но подобно тому дѣлаешь, который утопающаго въ водѣ дотоль не извлекаетъ, доколь отъ него обѣщаемаго награжденія не слышитъ: и ты дотоль не даешь взаимъ денегъ въ нуждѣ и бѣдѣ утопающему, доколь лихвою себе не обяжетъ тебѣ, и тѣмъ самымъ насильно, принужденно отъ него отъемлешь; и онъ, когда безъ лихвы не даешь, принужденъ давать тебѣ лихву, хотя лакомство твое удовольствовать и немилосердіе умягчить, чтобы ты, когда милосердіемъ не преклоняешься, склонился корыстію. "--- Иный отзывается: «судъ"=де гражданскій не казнитъ за лихву: убо свободно брать». "--- Не казнитъ судъ гражданскій, но казнитъ судъ Хрістовъ, который не токмо за лихву, но и за непоказанное ближнему милосердіе отсылаетъ во огнь вѣчный: \textit{идите отъ Мене, проклятіи, во огнь вѣчный, уготованный діаволу и аггеломъ, его: взалкахся бо, и не дасте Ми ясти}\footnote{Матѳ.~25,~41 и 42.}, и проч. Убо и брать свободно лихву, свободно идолослужителю, а не хрістіанину, тому, который сребро, а не Хріста любитъ, мамонѣ работаетъ, а не Хрісту; свободно неимѣющему упованія вѣчнаго, а не чающему воскресенія мертвыхъ и жизни будущаго вѣка. Таковымъ и все свободно дѣлать: свободно похищать, насиловать, слезы проливать, убивать, вдовицу и сира умерщвлять свободно, пока благость Божія имъ терпитъ. "--- Иный жалится, «чимъ"=де мнѣ жить, когда лихвы не брать?» "--- Неправда: не жить тебѣ нечѣмъ, но нечѣмъ сребролюбію и роскоши угождать. Многіе богатства не имѣютъ и лихвы не пріемлютъ, но живутъ и питаются: а иные и богатствомъ изобилуютъ, но лихвы не оставляютъ. Сребролюбію бо и роскоши и многое не довольно, а умѣренность и нужда и малымъ довольствуется. Видишь, что не нужда, но страсть лихву брать научаетъ. "--- Иный на другихъ указуетъ: «тые"=де и тые берутъ, а мнѣ ради чего не брать»? "--- Худое слѣдствіе: тые"=то лихву берутъ: убо и тебѣ можно? Они грѣшатъ: убо и тебѣ вольно? Многіе разбой чинятъ, убиваютъ, грабятъ, насилуютъ, блудодѣйствуютъ и прочія беззаконія дѣлаютъ: убо и тебѣ вольно? Сіе слѣдуетъ ли дѣлать, когда на другихъ указуешь въ беззаконномъ дѣлѣ? Возможно и вольно и тебѣ; но смотри, что отъ сего слѣдуетъ? Они гнѣву Божію и казни подлежать за грѣхъ: убо и ты; сіе непремѣнно и тебѣ слѣдуетъ. Не во всемъ убо подражать другимъ должно, но въ томъ только, что похвально и съ закономъ Божіимъ согласно дѣлаютъ. Впрочемъ, хотя бы вси беззаконновали, но ты берегись того, и внимай тому, чего законъ Божій учитъ. Какъ бо ты за нихъ, такъ и они за тебе не будутъ отвѣщать предъ судомъ Божіимъ; но всякъ за себе отвѣтъ дастъ. "--- Иный отзывается: «я"=де съ богатыхъ беру». "--- Правда: не равно съ убогихъ и богатыхъ лихву брать, какъ неравный грѣхъ богатаго и бѣднаго обижать. Грѣшно богатаго обнажать, но безчеловѣчно нищаго, который и безъ обнаженія голъ; такъ и о лихвѣ разумѣй. Грѣхъ есть съ богатаго лихву брать, но большій грѣхъ есть и безчеловѣчіе съ нищаго, которому и туне помогать должно. "--- «Убо"=де лучше и не давать взаимъ? такъ иный думаетъ». Худое и тое мнѣніе: какъ бо сребро давать въ лихву, такъ и взаимъ не давать безъ лихвы грѣшно, когда имѣешь что дать. Ибо Хрістосъ учитъ: \textit{просящему у тебе дай, и хотящаго отъ тебе заяти не отврати}. Давать убо взаимъ просящему должно, но безъ лихвы, какъ паки Хрістосъ научаетъ: \textit{взаимъ дайте, ничесоже чающе}.

Святый Василій великій къ лихоимцамъ глаголетъ: «Должно было тебѣ человѣческую нужду умалить, ты же умножаешь, прибытокъ отъ убогаго пріемля, по подобію лѣкаря поступая, который, къ немощному пришедъ, вмѣсто исцѣленія и послѣднюю силу охъ него отнимаетъ. Убогихъ нужды за случай къ обогащенію себѣ поставляешь, и, якоже земледѣльцы желаютъ дождей, чтобы сѣмена ихъ расли и умножались, тако и ты нужды и убожества человѣческаго хочешь, чтобы у тебе умножались деньги. Или не знаешь, что большее приращеніе грѣха дѣлаешь, нежели умноженіе денегъ отъ лихвы бываемое?» "--- И ниже: «что отъ убогаго берешь, сіе всякое человѣконенавидѣніе превосходитъ, Отъ бѣдъ богатишися, отъ слезъ сребро собираешь, нагаго обнажаешь, алчнаго біеши; нѣтъ никакого милосердія, нѣтъ никакого разумѣнія о сродствѣ страждущемъ. И прибытокъ отъ того человѣколюбіемъ нарицаешь!.. Горе именующимъ горькое сладкимъ, и человѣконенавидѣніе человѣколюбіемъ нарицающимъ!» "--- И паки къ концу: «Безъ земли садиши, безъ сѣмени жнеши, и неизвѣстно, кому собираеши. Что тотъ, отъ кого лихву берешь, плачется "--- извѣстно: а кто плодомъ отъ лихвы собраннымъ будетъ питаться, неизвѣстно. Неизвѣстно бо, кому о богатствѣ утѣху оставиши: зло же отъ неправды себѣ сокровиществуеши. \textit{Убо хотящаго отъ тебе заяти не отврати, и сребра своего не даси въ лихву}, да, и отъ Новаго и Ветхаго Завѣта полезныхъ научившися, со благою надеждою ко Господу отъидеши, тамо лихву за благія дѣла имѣя воспріять»\footnote{Бес. на пс.~14"~й.}. О томжде и святый Златоустъ во многихъ своихъ бесѣдахъ поучаетъ.

\subsection[Глава 10-я. О обидѣ убогихъ.]{глава десятая.\\\bfseries О обидѣ убогихъ.}

\begin{quotation}\textit{Всякія вдовицы и сироты не озлобите: аще же злобою озлобите я, и возстенавше возопіютъ ко Мнѣ, слухомъ услышу гласъ ихъ, и разгнѣваюся яростію, и побію вы мечемъ, и будутъ жены ваша вдовы, и чада ваша сироты}\footnote{Исх.~22,~22--24.}.\end{quotation}
\begin{quotation}\textit{Проклятъ, иже уклонитъ судъ пришельцу и сиротѣ и вдовѣ; и рекутъ вси людіе: буди}\footnote{Второз.~27,~19.}.\end{quotation}
\begin{quotation}\textit{Сія глаголетъ Господь: сотворите судъ и правду, и избавите силою угнетена отъ руку обидящаго и пришельца и сира, и вдовицы не оскорбляйте, не угнетайте беззаконно, и крови неповинныя не изливайте на мѣстѣ семъ... Аще же не послушаете словесъ сихъ, Самъ Собою кляхся, рече Господь, яко въ пустыню будетъ домъ сей}\footnote{Іер.~22,~3 и 5.}.\end{quotation}


\paragraph*{§\:121.} Какъ благодѣяніе и милость, являемая человѣку двояка есть "--- душевная и тѣлесная: такъ и обида двоякая бываетъ "--- иная тѣло, иная душу озлобляетъ. Тѣлу обида дѣлается, когда біется, ранится, уязвляется, лишается пищи, одежды, и проч.: душа обиду терпитъ, когда опечаляется, поносится, злословится, безчестится, оклеветается, и проч., хотя то, ради совершеннаго между тѣломъ и душею союза, какъ тѣло душевную, такъ душа тѣлесную обиду чувствуетъ. Откуду бываетъ, что, когда страждетъ тѣло, и душа состраждетъ; а когда страждетъ и болѣзнуетъ душа, состраждетъ и соболѣзнуетъ тѣло: чего ради и недугуетъ, ослабѣваетъ, увядаетъ, блѣднѣетъ, изсыхаетъ, "--- что случается печалующимъ и сѣтующимъ.

\paragraph*{§\:122.} Какъ тяжко обиждать ближняго, слѣдующими доводами показуется. 1)~Какъ благодѣяніе, являемое ближнему, Хрістосъ Богъ вмѣняетъ Себѣ, по оному: \textit{еже сотвористе единому сихъ братій Моихъ меньшихъ, Мнѣ сотвористе}\footnote{Матѳ.~25,~40.}: такъ и обида, чинимая ближнему, Самаго Хріста касается: \textit{Савле, Савле, что Мя гониши}? глаголетъ Хрістосъ\footnote{Дѣян.~9,~4.}. Савлъ церковь Хрістову и вѣрныхъ Его озлоблялъ, но Хрістосъ Себѣ тое приписуетъ озлобленіе: \textit{Савле, Савле, что Мя гониши}? Обида бо самаго отца касается, когда сынъ обиждается, и господинъ себѣ за обиду вмѣняетъ, когда рабъ его озлобляется. Богъ же есть Отецъ и Господь всѣхъ: слѣдственно чинимая обида людямъ, яко рабамъ Его, и Самаго Его, яко Господа и Отца, касается. Коль же сіе страшно, всякъ можетъ видѣть. "--- 2)~Богъ, яко преблагій Промыслитель и Отецъ, всякое добро дѣлаетъ, пищу и одежду всѣмъ подаетъ. Слѣдственно, кто человѣку обиду дѣлаетъ, тотъ противится Богу благодѣтельствующему: ибо что Богъ подаетъ, тое онъ отнимаетъ; что Богъ созидаетъ, тое обидяй разоряетъ. Богъ благодѣяніемъ созидаетъ, а онъ злодѣяніемъ разоряетъ. \textit{Иже не созидаетъ со Мною, разоряетъ}, глаголетъ Хрістосъ. "--- 3)~Сатана, яко духъ злобы и непріязни, всѣхъ и всякимъ образомъ ищетъ озлобить и озлобляетъ. Убо кто озлобленіе и утѣсненіе ближнему дѣлаетъ, единаго съ сатаною духа есть, едино съ нимъ мыслитъ, едино согласіе имѣетъ съ нимъ. Обида бо умышленная не можетъ быть, какъ отъ злобнаго духа, и ядомъ древняго онаго змія напоеннаго. Коль же безстыдное и пагубное дѣло съ діаволомъ на разумное созданіе Божіе, на человѣка, созданнаго по подобію Божію, и на свое сродное естество обидою возставать!.. Діаволъ и аггели его злые вси на человѣка ополчаются, и озлобить хотятъ, а не другъ на друга вооружаются, яко сатана на сатану не возстаетъ: но человѣкъ ослѣпленный подобнаго себѣ человѣка гонитъ. Такъ"=то заразилъ сердце наше и ослѣпилъ умъ нашъ человѣкоубійца оный древній, чего довольно оплакать не можемъ. "--- 4)~Божій законъ, который велитъ ближняго любить, а не ненавидѣть, благодѣяніе, а не злодѣяніе ему показывать, обидою ближняго разоряется, и Законодавецъ безчестится и презирается. "--- 5)~За обиду ближняго и временною и вѣчною казнію Богъ грозитъ, какъ о томъ въ книгахъ святыхъ пророковъ и въ Новомъ Завѣтѣ читаемъ.

\paragraph*{§\:123.} Беззаконное дѣло есть, всякаго человѣка обиждать ради вышереченныхъ причинъ, но бѣднаго еще беззаконнѣе. Худо у богатаго что отнимать, но у убогаго еще хуже; худо всякаго озлоблять, но озлобленнаго хуже; худо всякаго опечалять, но печальнаго еще хуже; худо всякаго уязвлять, но уязвленнаго и того хуже. Ибо убогій помилованія, печальный утѣшенія, уязвленный врачеванія требуетъ. А кто ихъ озлобляетъ, тотъ и послѣднее, что имѣютъ, у нихъ отнимаетъ, и тако въ послѣднюю бѣду приводитъ, животъ сокращаетъ и истребляетъ. Обиждающій бо бѣднаго подобно дѣлаетъ тому, который утопающаго въ водѣ болѣе въ воду погружаетъ, и тако бѣду къ бѣдѣ, печаль къ печали, язвы къ язвамъ придаетъ, что не токмо беззаконное, но и безчеловѣчное дѣло есть, и знаменіе есть человѣка, нравомъ въ свирѣпаго звѣря преобразившагося.

\paragraph*{§\:124.} Сіе безчеловѣчное беззаконіе наипаче дѣлаютъ: 1)~Судіи неправедные, которые или по мздѣ, или по дружеству, или лицепріемля, суды производятъ, и не токмо не отираютъ слезъ у насилованныхъ и обиженныхъ, но ихъ и обвиняютъ, а насиловавшихъ оправдаютъ, и тако двоякое зло совершаютъ: обидчиковъ къ большему своевольствію, а обижденныхъ въ послѣднюю бѣдность приводятъ. "--- 2)~Сильные и высокіе люди, которые, ни суда гражданскаго, ни Божія не страшась, насильно или коварно отнимаютъ у вдовъ, сиротъ, убогихъ и прочіихъ беззаступныхъ людей землю, рощи, крестьянъ и прочее какое нибудь добро, отъ чего они себѣ пищу получаютъ, и тако ихъ послѣдняго пропитанія лишаютъ. "--- 3)~Богачи и прочіе, которые или недоплачиваютъ, или совсѣмъ не даютъ мзды должныя наемникамъ. Ибо вси таковые убогіе и бѣдные суть, которые нанимаются дѣлать, и чрезъ дневные труды дневную достаютъ пищу. "--- 4)~Лихоимцы или процентщики, которые убогимъ даютъ деньги взаимъ, и отъ нихъ съ лихвою воспріемлютъ; и чимъ большую лихву или процентъ съ убогихъ должниковъ своихъ берутъ, тѣмъ большее безчеловѣчіе дѣлаютъ. "--- 5)~Военачальники и прочія власти, которые подкоманднымъ своимъ опредѣленнаго жалованья не даютъ, или даютъ, но не сполна. Ибо тако ихъ лишаютъ нужнаго къ пропитанію, и изъ бѣдныхъ бѣднѣйшими дѣлаютъ. "--- 6)~Помѣщики, которые на крестьянъ излишніе оброки налагаютъ, и тако послѣднее съ нихъ дерутъ, и оставляютъ ихъ безъ нужнаго пропитанія, или несносными работами ихъ обременяютъ, и симъ не допущаютъ ихъ сыскивать себѣ къ житію потребныхъ. "--- 6)~Къ обидѣ убогихъ причислиться должно похищеніе съ пожара, а наипаче, когда что похищается у такихъ, которые и прежде пожара убоги были; и сіе безчеловѣчіе люто есть и несносно: тако бо бѣда къ бѣдѣ и болѣзнь къ болѣзни придается. Отъ сихъ пунктовъ и о прочіихъ подобныхъ симъ разумѣть можно, что то есть обида убогихъ.

\paragraph*{§\:125.} Толикое безчеловѣчіе являющіи ближнимъ своимъ, хотя устами и исповѣдуютъ Бога, но сердцемъ не признаютъ Его, или о промыслѣ Его не вѣруютъ, и наказанія за беззаконія свои не чаютъ, якоже о нихъ Духъ Святый, глубину сердца ихъ испытуяй, чрезъ Псаломника объявляетъ: \textit{вдовицу и сира, умориша, и пришельца убиша, и рѣша: не узритъ Господь, ниже уразумѣетъ Богъ Іаковль}. Но обличаетъ безуміе ихъ таможде чрезъ тогожде Псаломника: \textit{разумѣйте, безумніи, въ людехъ, и буіи нѣкогда умудритеся! Насаждей ухо, не слышитъ ли? или создавый око не сматряетъ ли? наказуяй языки, не обличитъ ли, учай человѣка разуму}\footnote{Пс.~93,~6--10.}?

\paragraph*{§\:126.} Обида убогихъ есть грѣхъ весьма тяжкій и на небо вопіющій, якоже можно изъ святаго Писанія видѣть, и самый разумъ тое показуетъ. \textit{Всякія вдовицы и сироты}, глаголетъ Богъ, \textit{не озлобите: аще злобою озлобите я, и возстенавше возопіютъ ко Мнѣ, слухомъ услышу гласъ ихъ, и разгнѣваюся яростію, и побію вы мечемъ, и будутъ жены ваша вдовы и чада ваша сироты}\footnote{Исх.~22,~22--24.}. И паки клятвою поражаетъ таковыхъ: \textit{проклятъ, иже уклонитъ судъ пришельцу и сиротѣ и вдовѣ; и рекутъ вси людіе: буди}\footnote{Второз.~27,~19.}. И паки съ клятвою грозитъ казнію такое безчеловѣчіе показующимъ: \textit{сотворите судъ, правду, и избавите силою угнетена отъ руку обидящаго и; и пришельца, и сира и вдовицы не оскорбляйте, не угнетайте беззаконно, и крове неповинныя не изливайте на мѣстѣ семъ. Аще же не послушаете словесъ сихъ, Самъ Собою кляхся, рече Господь, яко въ пустыню будетъ домъ сей}\footnote{Іер.~22,~3 и 5.}. И паки чрезъ пророка Амоса: \textit{сего ради, понеже пястьми біете убогихъ, и дары избранны пріясте отъ нихъ, домы украшены соградисте отъ нихъ, и не вселитеся въ нихъ; винограды вожделѣнны насадисте, и не имате пити вина отъ нихъ. Яко увѣдѣхъ многа нечестія ваша, и крѣпцы грѣси ваша: попирающе праведнаго, пріемлюще дары, и убогихъ отъ вратъ отрѣвающе}\footnote{5,~11 и 12.}. И Псаломникъ, Духомъ Святымъ движимый, жалится къ Богу и вопіетъ на такое безчеловѣчіе беззаконниковъ: \textit{вознесися Судяй земли, воздаждь воздаяніе гордымъ! Доколѣ грѣшницы, Господи, доколѣ грѣшницы восхвалятся? провѣщаютъ и возглаголютъ неправду, возглаголютъ вси дѣлающіи беззаконіе? Люди твоя, Господи, смириша, и достояніе Твое озлобиша; вдовицу и сира умориша, и пришельца убиша, и реша: не узритъ Господь, ниже уразумѣетъ Богъ Іаковль}. И придаетъ при концѣ: \textit{и воздастъ имъ Господь беззаконіе ихъ, и по лукавствію ихъ погубитъ я Господь Богъ}\footnote{Пс.~93,~2--7 и 23.}. А святый апостолъ плачъ и рыданіе имъ возвѣщаетъ ради лютыхъ скорбей, грядущихъ на нихъ, глаголя: \textit{пріидите, нынѣ богатіи, плачитеся и рыдайте о лютыхъ скорбехъ вашихъ, грядущихъ на вы! Богатство ваше изгни, и ризы ваша моліе поядоша! злато ваше и сребро изоржавѣ, и ржа ихъ въ послушество на васъ будетъ, и снѣсть плоти ваша, аки огнь, егоже снискасте въ послѣднія дни. Се мзда дѣлателей, дѣлавшихъ нивы ваша, удержанная отъ васъ, вопіетъ; и вопіенія жавшихъ во уши Господа Саваоѳа внидоша}\footnote{Іак.~5,~1--4.}! Откуду видимъ страшныя Божія казни за озлобленіе вдовъ, сиротъ и убогихъ бываемыя; видимъ домы и грады, то мечемъ и огнемъ поядаемые, то землею пожираемые; видимъ, что питающіися слезами убогихъ и сиротъ истребляются отъ земли живыхъ, и низходятъ во адъ воспріяти по дѣломъ рукъ своихъ. \textit{Богъ бо поругаемъ не бываетъ}: сколько милостивъ, столько и праведенъ есть и мститель ругателямъ ожесточеннымъ; и чимъ грозитъ грѣшникамъ, тое самымъ дѣломъ дознаетъ на себѣ нераскаянный беззаконникъ; а хотя въ семъ вѣкѣ не дознаетъ, то въ будущемъ не избѣжитъ страшнаго Божія суда. Ибо Богъ, хотя и всѣхъ есть Отецъ, однакожъ наипаче называется \textit{сирыхъ Отецъ, и Судія вдовицъ}\footnote{Пс.~67,~6.}. \textit{И сира и вдову пріемлетъ}\footnote{145,~9.}. Богатый богатому, сильный сильному, славный славному, вельможа вельможѣ помогаетъ; а нищій, сирый, вдовица отъ всѣхъ оставляется, нигдѣ помощи сыскать отъ человѣкъ не можетъ. Сего ради Богъ единъ имъ Помощникъ есть и готовый Покровитель. \textit{Тебе}, Господи, \textit{оставленъ есть нищій, сиру Ты буди помощникъ}, вопіетъ о нихъ Псаломникъ къ Богу\footnote{9,~35.}. Сей нелицепріятный Судія разсудитъ прю ихъ, Сей, какъ Отецъ милосердый и сострадательный, вступится за нихъ; Сей услышитъ воздыханія убогихъ; Сей отретъ горькія слезы ихъ; Сей сотворитъ судъ нищимъ и месть убогимъ. Сего ради разумѣйте сія питающіися слезами убогихъ, озлобляющіи вдовицъ и сиротъ, уклоняющіи судъ пришельцу, и сиротѣ, и вдовѣ, созидающіи и расширяющіи домы отъ имѣнія нищихъ; разумѣйте, что слезы ихъ не напрасно падаютъ, но, какъ кровь Авелева, вопіютъ отъ земли на небо, и вопіенія ихъ во уши Господа Саваоѳа входятъ. Разумѣйте, что Богъ всѣмъ равно благая Своя подаетъ, яко щедрый Отецъ: вся тварь повелѣніемъ Божіимъ равно всѣмъ служитъ, богатому и нищему, сильному и немощному, славному и неславному, вельможѣ и подлому, благородному и худородному, рабу и господину его. Солнце, луна и звѣзды всѣмъ равно служатъ свѣтомъ своимъ; облаки всѣхъ равно орошаютъ; воздухъ всѣхъ равно жизнь сохраняетъ; земля всѣмъ равно плоды подаетъ; вода всѣхъ равно напаяетъ; огнь всѣхъ равно согрѣваетъ; скоты, волы, овцы, кони всѣмъ равно работаютъ. Богъ бо тако устроилъ, дабы вси человѣки, имѣя услуженіе отъ твари, Божіей благости насыщалися, и Богу Благотворителю благодарили. Но вы, обнажая убогихъ, тую благодать Божію отъ нихъ отнимаете; что Богъ даетъ ради всѣхъ, тое вы себѣ единымъ тщитеся присвоить. Богъ, богатъ сый въ благости, обогащаетъ, а вы обнажаете; Богъ утѣшаетъ, а вы опечаляете; Богъ оживляетъ, а вы умерщвляете. Разумѣйте убо сія!.. А когда нынѣ не разумѣете, то будетъ время, когда злато, сребро, отчины, села, винограды, кони, кареты, трапезы богатыя, драгія вина, рабы и рабыни, шелковыя и виссонныя одежды и всю свою славу и великолѣпіе, отъ имѣнія бѣдныхъ собранное, и нехотя оставите, и нечаянно увидите слезы пролитыя и падшія предъ страшнымъ и праведнымъ Судіею, и увидѣвше смятетеся и ужаснетеся. Тогда уразумѣете, что есть Богъ и промыслъ Его, и судъ Его, есть \textit{Богъ отмщеній Господь}\footnote{Пс.~93,~1.}. Когда слезы бѣдныхъ, отъ васъ пролитыя, вамъ покажетъ, и за послѣднюю каплю ихъ отвѣта у васъ истяжетъ, "--- тогда увидите, что, которыхъ вы нынѣ озлобляете, насилуете, обнажаете, и самыхъ псовъ вашихъ, любезной вашей охоты, худшими вмѣняете, тые явятся въ сынѣхъ Божіихъ; и которыхъ вы недостойныхъ быть мнили крупицъ, падающихъ отъ трапезы вашей, отъ тѣхъ съ великимъ воздыханіемъ пожелаете капли водныя прохладить языкъ свой, геенскимъ пламенемъ жегомый, но не получите того.

\paragraph*{§\:127.} Тые, которые слезами омываютъ лица своя, которые воздыхаютъ отягчаеми, которые ищутъ помощи и не обрѣтаютъ, трудятся и не насыщаются плодами трудовъ своихъ, да внимаютъ утѣшенію апостола Іакова: \textit{долготерпите, братія моя, до пришествія Господня. Се земледѣлецъ ждетъ честнаго плода отъ земли, долготерпя о немъ, дондеже пріиметъ дождь ранъ и позденъ. Долготерпите убо и вы, утвердите сердца ваша: яко пришествіе Господне приближися}. \textit{Не воздыхайте другъ на друга, братіе, да не осуждени будете: се Судія предъ дверьми стоитъ! Образъ пріимите, братія моя, злостраданія и долготерпѣнія пророки, иже глаголаша именемъ Господнимъ. Се блажимъ терпящія: терпѣніе Іовле слышасте, и кончину Господню видѣсте, яко многомилостивъ Господь и щедръ}\footnote{Іак.~5,~7--11.}. \textit{Праведно} бо \textit{у Бога}, глаголетъ апостолъ Павелъ, \textit{воздати скорбь оскорбляющимъ васъ: а вамъ, оскорбляемымъ, отраду съ нами, во откровеніи Господа Іисуса съ небесе, со ангелы силы Своея, во огни пламеннѣ, дающаго отмщеніе невѣдущимъ Бога, и непослушающимъ благовѣствованія Господа нашего Іисуса Хріста: иже муку пріимутъ, погибель вѣчную отъ лица Господня, и отъ славы крѣпости Его, егда пріидетъ прославитися во святыхъ Своихъ, и дивенъ быти во всѣхъ вѣровавшихъ}\footnote{2~Сол.~1,~6--10.}.

\subsection[Глава 11-я. О славолюбіи и честолюбіи.]{глава перваянадесять.\\\bfseries О славолюбіи и честолюбіи.}

\begin{quotation}\textit{Како вы можете вѣровати, славу другъ отъ друга пріемлюще, и славы, яже отъ единаго Бога, не ищете}? глаголетъ Хрістосъ\footnote{Іоан.~5,~44.}.\end{quotation}


\paragraph*{§\:128.} Славолюбіе и честолюбіе происходитъ отъ гордости, которой свойство есть любить почтеніе, похвалу, поклоненіе и прославленіе. Корень же и начало его есть лестный оный и ядовитый древняго змія совѣтъ, который прародителемъ нашимъ тако совѣтовалъ: \textit{будете яко бози}\footnote{Быт.~3,~5.}. Тойжде коварный врагъ и нынѣ страстной нашей плоти, яко Евѣ, во уши шепчетъ, дабы искала владычества, господства, почитанія и славы суетныя.

\paragraph*{§\:129.} Славолюбіе есть знаменіе сердца, невѣріемъ недугующаго, какъ Хрістосъ сіе явно научаетъ: \textit{како вы можете вѣровати, славу другъ отъ друга пріемлюще}? Понеже 1)~кто славы своея въ семъ мірѣ ищетъ, тотъ не отдаетъ Богу славы, но себе, какъ идола, вмѣсто Бога поставляетъ, и что единому Богу должное есть, тое себѣ привлекаетъ. Богу бо единому \textit{подобаетъ всякая слава, честь и поклоненіе}. "--- 2)~Кто славы суетной ищетъ, тотъ о будущей вѣчной славѣ нерадитъ, ибо временной и вѣчной славы искать купно невозможно. Нерадитъ же не иныя ради причины, какъ что окомъ вѣры, которою она зрится, величества и превосходнаго изящества ея не видитъ; ибо, ежели бы видѣлъ ее, непремѣнно бы искалъ ея всѣми силами. Воинъ, усматривая славу, отъ побѣды имѣющую быть, храбро подвизается противу врага отчества: тако, кто вѣрою предусматриваетъ будущую славу, которую вѣрніи Хрістовы воини имѣютъ въ будущемъ вѣцѣ получить, подвизается противу враговъ міра и грѣха, о суетной сей славѣ нерадитъ, и яко сметіе вмѣняетъ ее; яко препятствіемъ бываетъ къ полученію истинныя оныя славы.

\paragraph*{§\:130.} Какъ сребролюбія и лихоиманія, якоже выше сказано, такъ славолюбія страсть есть ненасытна. Какъ бо сребролюбецъ, чимъ болѣе растетъ у него сребра, тѣмъ болѣе жаждетъ его: такъ славолюбецъ, чимъ выше восходитъ въ честь, тѣмъ выше еще подняться желаетъ. Истина сія извѣстна и явна и не требуетъ доказательства: повседневные сего примѣры предъ глазами являются. Откуду въ исторіяхъ видимъ, что многіе язычники, не удоволяяся высокимъ титломъ царскимъ, богами себе и желали и повелѣвали называть; а иные, недовольствуяся предѣлами своего владѣнія, другихъ себѣ покоряли, и тако свою славу расширять тщалися. О, когда бы язва сія въ хрістіанскихъ предѣлахъ не имѣла мѣста!..

\paragraph*{§\:131.} Славолюбіе многихъ золъ виновно бываетъ. 1)~Оно учитъ ласкательствовать, сообразоваться высшимъ, и нравамъ ихъ послѣдовать, хотя бы и развращенны были; пороки добродѣтелями, и добродѣтели пороками безстыдно называть, льстить, лгать, лукавновать, чтобы тако въ любовь и, милость имъ вкрасться, и такимъ образомъ честь желаемую получить. "--- 2)~Оно учитъ клеветать на ближняго, добрыхъ злыми, а злыхъ добрыми называть, чтобы тако угодить тому, у кого милость мнимая ищется; словомъ сказать, изъ человѣка псомъ ласкательнымъ дѣлаетъ, и что бы ни захотѣлъ мнимый его благодѣтель, на котораго надѣется, все дѣлать готовымъ себе показуетъ. "--- 3)~Оно учитъ лихоиманію: ибо, чтобы честь получить, надобно щедро наполнить руки ходатаевъ, многихъ дарить, угощать; но, чтобы на вся сія достатокъ былъ, понуждается славолюбецъ устремляться на неправду и хищеніе. "--- 4)~Славолюбіе научаетъ неповинную человѣческую проливать кровь. Тако умыслилъ Авессаломъ на кроткаго и неповиннаго Давида, царя Израилева, обнажить мечь и сынъ беззаконный праведнаго отца своего умертвить, чтобы царствомъ Израилевымъ завладѣть, котораго ни высокая монаршая честь, ни неудобность начинаемаго дѣла, ни страхъ опасности, ни тотъ долгъ, которымъ подданные къ своимъ Государямъ обязаны, ни сродная отческая кровь, отъ которой родился, ни тая любовь, которой отъ отца своего паче прочей братіи почтенъ былъ, ни кротость и неповинность святаго и благочестиваго царя, отъ беззаконнаго и пагубнаго намѣренія отвратить не могли; но едино славолюбіе все сіе превозмогло, и убѣдило беззаконнаго сына гоняться въ слѣдъ благочестиваго отца и Государя своего\footnote{2~Цар.~15 и проч.}. Тако учинилъ славолюбивый и беззаконный Иродъ: всѣхъ дѣтей, отъ двухъ лѣтъ и нижайше, въ Виѳлеемѣ и во всѣхъ предѣлѣхъ его избилъ, какъ святое Евангеліе свидѣтельствуетъ\footnote{Матѳ.~2,~5.}. Убити младенцевъ умыслилъ, чтобы тако съ младенцами и рождшагося Младенца Хріста, Царя Іудейскаго, убити, хитрый лисъ возмоглъ, и тако бы не потерялъ царскія чести. Тоежъ славолюбіе и нынѣ много неповинной крови проливаетъ; и самыхъ помазанниковъ Божіихъ касаться не устрашается; и ихъ то ядомъ раствореннымъ умерщвляетъ или сокращаетъ жизнь ихъ, то мечемъ окровавляетъ порфиру и престолъ ихъ, и тако отчество безъ главнаго правителя остается; отъ чего всякое нестроеніе и замѣшательство послѣдуетъ въ обществѣ. "--- 5)~Славолюбіе и по полученіи чести не покоится: и тогда многія бѣды производитъ. Ибо по чести (тако оно рачителей своихъ научаетъ) надобно имѣть приличный уборъ; надобно расширять и украшать покои; надобно набирать богатые столы; надобно немалое число имѣть слугъ въ убранствѣ; надобно ради проѣзда имѣть богатыхъ коней и кареты, ради увеселенія собачью охоту; надобно самому и фамиліи своей златомъ и сребромъ украшенныя имѣть одежды; надобно ради забавы поиграть въ карты; надобно и прочія утѣхи или паче прихоти производить. Но на вся сія требуется немалая сумма: чего"=для слѣдуетъ собрать не откуду, какъ отъ подчиненныхъ; слѣдуетъ нарушить присягу, которую учинилъ предъ Сердцевѣдцемъ Богомъ и Святымъ Его Евангеліемъ; слѣдуетъ попрать правду, которую съ клятвою обѣщался хранить; слѣдуетъ дѣлать насилія убогихъ, налагать на крестьянъ, когда имѣются, излишніе сборы, или работами излишними отягощать. Коль убо люто есть славолюбіе, хотя люди нынѣшняго наипаче вѣка за зло его не вмѣняютъ! Зло есть лихоиманіе, какъ сказано; но злѣйшее есть славолюбіе: ибо гдѣ славолюбіе, тамо гнѣздится и лихоиманіе; славолюбіе бо и лихоиманію учитъ, какъ изъ вышереченныхъ видно.

\paragraph*{§\:132.} Славолюбіе и честолюбіе и самыхъ тѣхъ, которыми обладаетъ, въ бѣдственные случаи приводитъ. Прародители наши, возжелавше божескія чести, лишились и того, что имѣли, и \textit{приложилися скотомъ несмысленнымъ и уподобилися имъ}. Похотѣлъ Авессаломъ царскія чести такъ, что и убити искалъ отца и Государя своего; но вмѣсто чести погибель себѣ сыскалъ, и который хотѣлъ сидѣть на царскомъ престолѣ, \textit{обѣсися на древѣ, повисъ между небомъ и землею}\footnote{2~Цар.~18,~9.}. Тойжде же судъ и нынѣ постигаетъ славолюбцевъ и похитителей чести, которые хотя и получаютъ желаемое, то ходатайствомъ своихъ заступниковъ, то посредствомъ мзды; то представленіемъ наслѣднаго благородства, однакожъ не въ пользу себѣ достаютъ тое, но въ пагубу, хотя не временную, а непремѣнно вѣчную. Понеже \textit{вѣрные} сіи отчества слуги скоро въ вѣрности своей оказываютъ себе, когда въ чести находясь, не Божіей чести и славы, чему честь служить должна, но своей ищутъ корысти; не ближнимъ, но своимъ страстямъ служатъ; не созидаютъ, но болѣе разоряютъ братію свою. Истина сія всѣмъ явна, и не требуетъ слова къ доказательству. Ибо самая вещь доказуетъ тое, когда на беззаконное мздоимство и лихоимство вездѣ жалобы слышатся, вездѣ слезы бѣдныхъ, беззаступныхъ и нигдѣ удовольствій себѣ сыскать не могущихъ, проливаются. Что бо успѣетъ нищій съ богатымъ, что подлый съ благороднымъ, вступая въ судъ, въ которомъ \textit{вѣрные} сіи отчества сыны и слуги засѣдаютъ? Весьма ничего! отъидутъ, какъ приходятъ, съ тымижде слезами, паче же съ горшими. Чего ради принуждены бываютъ всякое озлобленіе и насиліе терпѣть, нежели напрасно и съ большею своею бѣдою искать защищенія, и уже не къ человѣку, но къ Богу Судіи прибѣгать и вопить. Сіи же слезы и вопли не иное что, какъ Божій гнѣвъ и вѣчную погибель тѣмъ, которые по должности своей не отираютъ ихъ, ходатайствуютъ.

\paragraph*{§\:133.} Санъ или честь хрістіанину должна быть не ино что, какъ иго, отъ Бога наложенное, которое долженъ носить въ славу Божію и пользу братіи своея. Онъ, въ чести находясь, долженъ, какъ присягою обязался, обществу служить, неправду искоренять, обидимыхъ защищать, нахальныхъ и своевольныхъ продерзость удерживать, образъ добрыхъ дѣлъ подчиненнымъ быть, и проч. Ежели кто позванъ, и для сей причины въ честь идетъ, тотъ пороку славолюбія не подлежитъ; но таковый и искать не будетъ чести, но его сама честь искать будетъ. Онъ убѣгаетъ чести, но за нимъ честь гоняется. Таковый"=то и достоинъ чести, какъ вси мудрые люди мнятъ и утверждаютъ. Напротивъ того, недостоинъ тотъ чести, кто ищетъ чести. Ибо чего ради кто въ честь домогается, тое и дѣлать будетъ, получивъ честь. Но едва ли кто желаетъ чести ради поднятія трудовъ въ общую пользу, а не ради своей временной корысти? Скоро таковые искатели, получивъ желаемое, оказываютъ себе не иначе, какъ плевелы, между пшеницею посѣянные, какъ сказано.

\paragraph*{§\:134.} Но, чтобы отъ сего порока избавиться, должно помнить: 1)~что равно славные и неславные, господа и раби, почтенные и подлые умираютъ и предаются землѣ: никто съ собою славы сея не относитъ, но все мірское міру остается. 2)~Что \textit{судъ жесточайшій преимущимъ бываетъ}. \textit{Ибо малый достоинъ милости, сильніи же сильнѣ истязани будутъ}, глаголетъ Соломонъ\footnote{Прем.~6,~5 и 6.}. 3)~Что нѣтъ большей чести и славы, какъ быть \textit{истиннымъ хрістіаниномъ}, котораго достоинства славолюбіе лишаетъ, по словеси Хрістову: \textit{како вы можете вѣровати, славу другъ отъ друга пріемлюще, и славы, яже отъ единаго Бога, не ищете}?

\section[Заключеніе статьи сея. О удаленіи отъ злыхъ.]{заключеніе статьи сея.\\\bfseries О удаленіи отъ злыхъ.}

\begin{quotation}\textit{Повелѣваемъ вамъ, братіе, о имени Господа нашего Іисуса Хріста, отлучатися вамъ отъ всякаго брата, безчинно ходяща, а не по преданію, еже пріяша отъ насъ}\footnote{2~Сол.~3,~6.}.\end{quotation}
\begin{quotation}\textit{Тлятъ обычаи благи бесѣды злы}\footnote{1~Кор.~15,~33.}.\end{quotation}


\paragraph*{§\:135.} Ничто такъ не вредитъ человѣку, наипаче юному, какъ злая компанія. Якоже бо обхожденіе съ добрыми есть такая школа, въ которой безъ книгъ обучается человѣкъ философіи хрістіанской, то есть, честному житію, тако со злыми обращеніе бываетъ виновно крайняго развращенія. Хотя бы и природы доброй былъ кто, и добрѣ воспитанъ, но когда съ развращенными обходиться будетъ, трудно и почти невозможно тому не развратиться. Злоба бо такъ, какъ смола, прилипчива и нечувствительно въ добрые нравы входитъ и заражаетъ ихъ. Сказуютъ, что здоровые глаза, когда смотрятъ на больные, сами вредъ пріемлютъ, а имъ никакой не дѣлаютъ пользы: такъ добрый, живучи со злыми, самъ портится, а не ихъ исправляетъ. «Какъ больные глаза, глаголетъ святый Златоустъ, наносятъ вредъ здоровымъ, и какъ проказу имѣющій заражаетъ чистаго, такъ обращеніе со злыми портитъ и развращаетъ добрыхъ людей. И таковыхъ не токмо убѣгать, но и отсѣкать повелѣваетъ Хрістосъ, глаголя\textit{: аще око твое десное соблазняетъ тя, изми е, и верзи отъ себе}»\footnote{На пс.~4"~й.}. Разсуждай убо, хрістіанине, съ кѣмъ хочешь имѣть обращеніе; и, пока не познаешь совершенно, ни къ кому не приставай, да не въ пропасть погибели впадешь. Лучше со звѣрьми жити, нежели со злыми людьми. «Злые бо люди, глаголетъ тойжде Златоустъ, болѣе вредятъ, нежели ядовитые зміи: ибо сіи откровенно ядъ свой наносятъ, а тіи каждаго дни тайно и нечувствительно заражаютъ»\footnote{Въ тойжде бес.}.

\paragraph*{§\:136.} Горе юнымъ дѣтямъ, которыя развращенныхъ родителей имѣютъ! Куды имъ отъ такихъ родителей удалиться, когда принуждены съ ними въ единыхъ покояхъ жить, за единою трапезою сидѣть, слова, дѣла и всѣ поступки ихъ наблюдать? Юность бо, какъ сама собою ко всякому злу склонна, требуетъ прилежнаго наблюдательства, добраго воспитанія и наставленія; но вмѣсто того срѣтаетъ ее великое зло "--- ядовитый родительскихъ нравовъ соблазнъ. Нравы бо родительскіе юнымъ дѣтямъ, какъ регула, имѣются, на которые взирая, послѣдуютъ имъ, и что въ нихъ примѣчаютъ, тому и сами навыкаютъ. Откуду бываетъ, что дѣти злыхъ родителей злѣйшими ихъ бываютъ, а внучата и тѣхъ злѣйшими, и такъ растетъ зло, пока судомъ Божіимъ пресѣчется. Соблазнъ бо не иначе какъ пожаръ усилившійся далѣе и далѣе простирается и поядаетъ одушевленныя храмины. Горе бо юнымъ дѣтямъ отъ сего соблазна; но сугубое горе родителямъ, которые, вмѣсто полезнаго ученія, злымъ примѣромъ, какъ ядомъ, юныя сердца заражаютъ! \textit{Горе бо человѣку тому, имже соблазнъ приходитъ въ міръ}, глаголетъ Хрістосъ. \textit{Уне есть ему, да обѣсится жерновъ осельскій на выи его, и потонетъ въ пучинѣ морстѣй}\footnote{Матѳ.~18,~7 и 8.}. «Родители, глаголетъ святый Іоаннъ Златоустъ, которые дѣтей по"=хрістіански воспитывать пренебрегаютъ, дѣтоубійцъ беззаконнѣйшіи суть. Ибо дѣтоубійцы тѣло отъ души разлучаютъ, а они и душу и тѣло въ геенну огненную ввергаютъ. Оной смерти, по естественному закону, никакъ избѣжать невозможно; а сію возможно бы было, ежели бы родителей нерадѣніе не было ея виновно. Къ томужъ смерть тѣлесную воскресеніе приспѣвшее абіе упразднитъ; душевной же погибели никто не можетъ возвратить»\footnote{\textit{Бес. противъ хулит. житія монаш. гл.~3.}}. Ежели, какъ святый отецъ учитъ, тые родители хуждшіе дѣтоубійцъ, которые дѣтей по"=хрістіански не наставляютъ: что уже о тѣхъ родителяхъ сказать, которые не токмо не наставляютъ, но и развращаютъ нравами своими! А что о родителяхъ здѣ говорится, тое должно разумѣть и о тѣхъ, кои отческое званіе на себѣ имѣютъ, какъ"=то: учители, пастыри, начальники, и проч.

\section[Заключеніе 2-е. О гнѣвѣ Божіемъ противу грѣха.]{заключеніе второе.\\\bfseries О гнѣвѣ Божіемъ противу грѣха.}

\begin{quotation}\textit{О человѣче! или о богатствѣ благости Его и кротости и долготерпѣніи нерадиши, не вѣдый, яко благость Божія на покаяніе тя ведетъ? По жестокости же твоей и непокаянному сердцу собираеши себѣ гнѣвъ въ день гнѣва и откровенія праведнаго суда Божія, Иже воздастъ коемуждо по дѣломъ его}\footnote{Римл.~2,~4--6.}.\end{quotation}


\paragraph*{§\:137.} Гнѣвъ Божій противу грѣха чрезъ казни, на грѣшниковъ отъ Бога посылаемыя, показуется. Аще бо Богъ и благъ есть, такъ, что \textit{никтоже благъ, токмо единъ Богъ}, по свидѣтельству Хрістову\footnote{Матѳ.~19,~17.}, однакожъ, сколько благъ, столько и праведенъ есть, котораго правда требуетъ того, чтобы грѣшникъ, яко святаго и непремѣняемаго Его закона преступникъ, казненъ былъ. Что же грѣшникъ не тотчасъ по согрѣшеніи, или въ самомъ дѣйствіи грѣха казнится, тое благости Божіей приписать должно, которая терпитъ грѣшнику, и такъ на покаяніе его ожидаетъ, якоже апостолъ къ грѣшнику глаголетъ: \textit{о человѣче! или о богатствѣ благости Его и кротости и долготерпѣніи нерадиши, не вѣдый, яко благость Божія на покаяніе тя ведетъ}?

\paragraph*{§\:138.} Читаемъ въ святомъ Писаніи, что различныя казни Богъ на грѣшниковъ посылаетъ. Аггеловъ согрѣшившихъ съ небесе сверже. О семъ глаголетъ Хрістосъ: \textit{видѣхъ сатану, яко молнію съ небесе спадша}\footnote{Лук.~10,~18.}; и Петръ апостолъ глаголетъ: \textit{аггеловъ согрѣшившихъ не пощадѣ, но, пленицами мрака связавъ, предаде на судъ мучимы блюсти}\footnote{2~Петр.~2,~4.}. Прародители наши Адамъ и Ева изъ рая за преслушаніе изгнаны, и всякому бѣдствію, а съ ними и мы подвержены. За грѣхи первый міръ, кромѣ Ноя, \textit{правды проповѣдника}, ужаснымъ потопомъ погубленъ\footnote{Быт.~7"~я; 2~Петр.~2,~5.}. Содомъ и Гоморръ съ окрестными градами за мерзкую нечистоту огнемъ пожжены\footnote{Быт.~19"~я; 2~Петр.~2,~6.}. Гордый и ожесточенный Фараонъ со всѣмъ воинствомъ своимъ въ Чермномъ морѣ потопленъ\footnote{Исх.~14"~я.}. Даѳанъ и Авиронъ съ единомышленниками своими, возставшіи на Моѵсея и Аарона, въ пустынѣ живы землею пожерты\footnote{Числ.~16"~я.}. \textit{Отверзеся земля, и пожре Даѳана, и покры на сонмищи Авирона}\footnote{Пс.~105,~17.}. Прочіи люди Израильскіе таможде за различныя беззаконія различно поражены гнѣвомъ Божіимъ: иные огнемъ, иные ядовитыми зміями, иные мечемъ иноплеменническимъ, иные инымъ образомъ, якоже о томъ читаемъ въ книгахъ Моисеовыхъ, пророческихъ и въ 10"~й главѣ 1"~го посланія къ Коринѳ. Пришедшіи въ землю обѣтованную сколько разъ въ работу иноплеменникамъ предаваны, сколько въ плѣненіе отведены и инако казнены были, "--- книги Ветхаго Завѣта свидѣтельствуютъ. Всему тому грѣхи ихъ и беззаконія виною были, якоже о томъжде свидѣтельствуютъ святыя оныя книги. По пришествіи Хрістовомъ въ міръ такожде видимъ казни Божія, на грѣшниковъ посланныя. Читаемъ, что Ананія съ Сапфирою женою своею за ложь, что солгали Духу Святому, нечаянною поражены смертію\footnote{Дѣян.~5,~1--10.}. \textit{Ирода внезапу порази Ангелъ Господень, зане не даде славы Богу: и бывъ червми изъяденъ, издше}\footnote{12,~23 и 24.}. Градъ Іерусалимъ, избившій пророки и каменіемъ побившій посланныя къ нему, и кровію Хріста Сына Божія обагренный, совсѣмъ разоренъ, и чада его то"=есть, жители, остріемъ меча разбіены якоже о томъ Самъ Хрістосъ съ плачемъ пророчествовалъ: \textit{пріидутъ дніе на тя, Іерусалиме, и обложатъ врази твои острогъ о тебѣ, и обыдутъ тя, и разбіютъ тя и чада твоя въ тебѣ}\footnote{Лук.~19,~43 и 44.}. Но что въ книгахъ читаемъ, тое и въ наши времена видимъ. Тыежде страшные Божіи суды и нынѣ примѣчаемъ; тойжде праведный Его гнѣвъ и нынѣ на грѣшникахъ нераскаянныхъ является. Видимъ и слышимъ страшныя брани и ужасныя и плача достойныя кровопролитія; столько тысящей падающихъ на брани, столько остающихся вдовъ, сиротъ, плачущихъ отцевъ и матерей, столько разоряемыхъ градовъ отъ иноплеменническаго оружія, отъ моровой язвы, отъ глада, труса и огня, сколько внезапу восхищается беззаконниковъ и сходятъ отъ сего свѣта. Сія вся суть слѣды праведнаго Божія суда и гнѣва, который беззаконниковъ нераскаянныхъ яко огнь поядаетъ. Что онымъ приключилося праведнымъ Божіимъ судомъ, того надобно ожидать и прочіимъ нераскаяннымъ грѣшникамъ. Читаемъ въ благовѣстіи святаго Евангелиста Луки, что когда нѣкіи пришли ко Хрісту, и повѣдали Ему о Галилеянахъ, которыхъ \textit{кровь Пилатъ смѣси съ жертвами ихъ}, отвѣщалъ Хрістосъ и сказалъ: \textit{мните ли, яко Галилеане сіи грѣшнѣйши паче всѣхъ Галилеанъ бяху, яко тако пострадаша? Ни, глаголю вамъ: но аще не покаетеся, вси такожде погибнете}\footnote{13,~1--3.}. Отъ сихъ словъ Хрістовыхъ заключается, что и прочіимъ грѣшникамъ нераскаяннымъ надобно ожидать такойжде погибели, каковую на другихъ видятъ. \textit{Уже бо и сѣкира при корени древа лежитъ: всяко убо древо, еже не творитъ плода добра, посѣкаемо бываетъ, и во огнь вметаемо}\footnote{Матѳ.~3,~10.}. Отъ бывшихъ и бываемыхъ казней заключаемъ и о будущихъ, и отъ временныхъ о вѣчныхъ примѣчаемъ. Будетъ грѣшникамъ ожесточеннымъ вѣчная казнь, будетъ геенна огненная, будетъ адъ, тьма кромѣшная, скрежетъ зубовъ; отринутся отъ лица Божія и царствія Его беззаконники. \textit{Не льстите себе: ни блудницы, ни идолослужители, ни прелюбодѣи, ни сквернители, ни малакіи, ни мужеложницы, ни лихоимцы, ни татіе, ни піяницы, ни досадители, ни хищницы, царствія Божія не наслѣдятъ}, написалъ апостолъ святый\footnote{1~Кор.~6,~9 и 10; Гал.~5,~19--21.}. Хрістосъ возглаголетъ въ день праведнаго Своего суда грѣшникомъ: \textit{идите отъ Мене, проклятіи, во огнь вѣчный, уготованный діаволу и аггеломъ его}\footnote{Матѳ.~25,~41.}. \textit{Страшливымъ и невѣрнымъ, и сквернымъ, и убійцамъ, и блудъ творящимъ, и чары творящимъ, идоложерцемъ, и всѣмъ лживымъ, часть имъ въ езерѣ, горящемъ огнемъ и жупеломъ, еже есть смерть вторая}, паки глаголетъ Хрістосъ\footnote{Апок.~21,~8.}. Тоежде читаемъ и на прочіихъ святаго Писанія мѣстахъ.

\paragraph*{§\:139.} Божія слова и прещенія, хрістіанине, не суть суетныя и пустыя, но суть истинныя и достовѣрныя: что сказалъ, тое неотмѣнно такъ есть, какъ сказано; и что предсказалъ, тое непремѣнно будетъ. Предсказалъ прародителямъ нашимъ смерть, отъ заповѣданнаго древа имѣющую быть: и послѣдовала смерть преступившимъ заповѣдь Его. Предсказалъ Ною праведному всемірный потопъ на беззаконниковъ: и былъ потопъ. Предсказалъ Фараону ожесточенному погибель: и погиблъ Фараонъ. Предсказалъ Израильтянамъ плѣненіе: и были плѣнены. Предсказалъ разореніе и опустошеніе Іерусалима: и сбылося тое. Объявилъ въ Писаніи Своемъ, что будетъ вѣчная мука беззаконникамъ: неотмѣнно будетъ она, дознаютъ на себѣ вѣчный праведнаго суда Божія гнѣвъ, будутъ пить сію горести чашу, но никогда не выпьютъ.

Да убоимся убо, хрістіанине, суда Божія и праведнаго Его гнѣва, и покаемся, да не впадемъ въ руцѣ Бога живаго. \textit{Страшно бо есть, еже впасти въ руцѣ Бога живаго}\footnote{Евр.~10,~31.}. Кто не вѣритъ словамъ Божіимъ, тотъ на себѣ дѣло праведнаго Божія суда дознаетъ. \textit{Богъ бо поругаемъ не бываетъ: еже бо аще сѣетъ человѣкъ, тожде пожнетъ}, глаголетъ апостолъ\footnote{Гал.~6,~7.}. \textit{День Господень, якоже тать въ нощи, тако пріидетъ. Егда бо рекутъ, миръ и утвержденіе, тогда внезапу нападетъ на нихъ всегубительство, якоже и болѣзнь во чревѣ имущей, и не имутъ избѣжати}\footnote{1~Сол.~5,~2 и 3.}.



\chapter*{часть вторая.\\О ПОКАЯНІИ И ПЛОДАХЪ ПОКАЯНІЯ, ИЛИ ДОБРЫХЪ ДѢЛАХЪ.}
\addcontentsline{toc}{chapter}{Часть 2-я. О покаяніи и плодахъ покаянія, или добрыхъ дѣлахъ.}

\begin{quotation}\textit{Уклонися отъ зла, и сотвори благо. Временная сласть, но вѣчная мука; временный крестъ и скорбь, но вѣчное утѣшеніе}.\end{quotation}
\begin{quotation}\textit{Человѣче, избирай что хощешь}!\end{quotation}


\section[Статья 1-я. О покаяніи.]{статья первая.\\\bfseries О покаяніи.}
\subsection[Глава 1-я. Како Богъ призываетъ грѣшника на покаяніе.]{глава первая.\\\bfseries Како Богъ призываетъ грѣшника на покаяніе.}

\begin{quotation}\textit{Пріидите ко Мнѣ вси труждающіися и обремененніи, и Азъ упокою вы. Возмите иго Мое на себе, и научитеся отъ Мене, яко кротокъ есмь и смиренъ сердцемъ: и обрящете покой душамъ вашимъ}\footnote{Матѳ.~11,~28 и 29.}.\end{quotation}


\paragraph*{§\:140.} Доколь человѣкъ стоитъ предъ кѣмъ, лицемъ своимъ къ тому стоитъ; а когда отъ него отъити хощетъ, отвращается отъ него лицемъ своимъ, и обращается къ нему хребтомъ своимъ, и тако отвратившися, идетъ въ намѣренный путь; и чимъ болѣе идетъ, тѣмъ болѣе отходитъ и удаляется отъ него. Тако, доколь человѣкъ тщится святую волю Божію исполнять, и слушаетъ того, что Богъ въ святомъ Своемъ словѣ повелѣваетъ, и готовымъ на повелѣніе Божіе себе показуетъ, "--- какъ бы лицемъ своимъ предъ Богомъ стоитъ, и почтеніе свое Ему отдаетъ, съ Нимъ пребываетъ и не отлучается отъ Него; но когда доброе свое произволеніе перемѣняетъ, и зло вмѣсто добра избираетъ, грѣхъ вмѣсто добродѣтели творитъ, "--- какъ бы отвращается лицемъ своимъ отъ Бога, и хребетъ Ему обращаетъ, якоже о беззаконныхъ Израильтянахъ глаголетъ чрезъ пророка: \textit{обратиша хребетъ ко Мнѣ, а не лице}\footnote{Іер.~32,~33.}. Тако отвратившися человѣкъ отъ Бога, яко безконечнаго добра, чимъ болѣе грѣховъ и беззаконій творитъ и въ нераскаяніи пребываетъ, тѣмъ болѣе отходитъ и удаляется отъ Него. Отвращается же въ самой вещи не лицемъ, но сердцемъ, и обращаетъ не хребетъ, но непослушаніе и нераскаянное сердце; и удаляется не ногами, но волею, не премѣненіемъ мѣста, но премѣненіемъ нравовъ злыхъ въ злѣйшіе. Куды бо отвратить лице можемъ отъ Того, Который вездѣ есть? И куды отъ Него пойдемъ, Который на всякомъ мѣстѣ обрѣтается? Якоже Псаломникъ глаголетъ: \textit{камо пойду отъ Духа Твоего, и отъ лица Твоего камо бѣжу? Аще взыду на небо, Ты тамо еси, аще сниду во адъ, тамо еси}\footnote{Пс.~138,~7 и 8.}. И сіе"=то есть отвратиться, отступить и удалиться отъ Бога. Сіе отвращеніе и удаленіе грѣшника отъ Бога изобразилося намъ притчею блуднаго сына, который, взявъ достойную часть имѣнія отъ благоутробнаго своего отца, съ бѣдою своею отсталъ отъ него, и, отшедъ на страну далече, расточилъ имѣніе свое, живый блудно, и послѣди началъ лишатися, и желалъ насытиться рожцами, ихже ядяху свинія. Тако человѣкъ, отступивши отъ Бога непослушаніемъ, погубляетъ данное ему отъ Бога имѣніе духовное, то"=есть, святость, правду, непорочность и прочее; и тако въ тяжкую страстей и грѣха работу попадается, и въ сей бѣдственной работѣ ни отъ чего не можетъ себѣ сыскать удовольствія и отрады. Душа бо человѣческая, яко духъ отъ Бога созданный, ни въ чемъ иномъ удовольствія, покоя, мира, утѣшенія и отрады сыскать не можетъ, какъ только въ Бозѣ, отъ Котораго по образу Его и по подобію создана; а когда отъ Него отлучится, принуждена искать себѣ удовольствія въ созданіяхъ, и страстями различными, какъ рожцами, себе питать, но надлежащаго упокоенія и отрады не обрѣтаетъ; и такъ отъ гладу слѣдуетъ умрети. Духу бо духовная пища потребна есть.

\paragraph*{§\:141.} Тако отвратившагося грѣшника и удаляющагося премилосердый Богъ не отвращается, но какъ бы въ слѣдъ отеческимъ гласомъ вопіетъ: чадо, куды ты отъ Мене удаляешься? Гдѣ ты съищешь безъ Мене и кромѣ Мене блаженство? Вездѣ тебе срящетъ бѣда. И тако вопія, призываетъ его къ Себѣ. Призываетъ же различно: 1)~Внутрь, чрезъ благодать Свою, которою ударяетъ въ двери сердца нашего. Сіе бываетъ, сколько разъ совѣсть, грѣхами раздраженная, обличаетъ и ударяетъ грѣшника, представляетъ ему судъ Божій и вѣчную муку за грѣхи. "--- 2)~Призываетъ чрезъ спасительное смотреніе, страданіе, смерть и воскресеніе Сына Своего, Іисуса Хріста, представляя ему, яко учинилося тое великое дѣло ради его. Не инаго бо кого ради пришедъ Хрістосъ въ міръ, какъ ради грѣшника. \textit{Пріиде бо сынъ человѣческій взыскати и спасти погибшаго}\footnote{Лук.~19,~10.}. \textit{Не пріидохъ, рече, призвати праведники, но грѣшники на покаяніе}\footnote{Матѳ.~9,~13.}, "--- Который такожде какъ бы вопіетъ въ слѣдъ грѣшника, и увѣщаваетъ возвратитися покаяніемъ къ Себѣ: «Почто ты, грѣшниче, Мене оставилъ? Почто тебе Возлюбившаго удаляешься? Помни, яко тебе ради Я родился отъ Дѣвы, и \textit{рабій зракъ пріялъ}\footnote{Филип.~2,~7.}; тебе ради младенствовалъ, обнищалъ, смирился, на земли пожилъ, плакалъ, трудился, гоненіе, злословіе, укореніе, безчестіе, поруганіе, раны, заплеваніе, насмѣяніе и всякое злостраданіе претерпѣлъ; наконецъ поносною смертію, смертію же крестною умеръ. Все сіе спасенія ради твоего учинилъ. Сошелъ съ небесъ, чтобы тебе на небо вознести; смирился, чтобы тебе возвысить уничтоженнаго; обнищалъ, чтобы тебе обогатить обнищавшаго; обезчестился, чтобы тебе прославить обезчещеннаго; уязвился, чтобы исцѣлить тебе уязвленнаго; умеръ, чтобы тебе оживить умершаго. Ты согрѣшилъ, а Я грѣхъ твой на Себе взялъ; ты виноватъ, а Я казнь претерпѣлъ; ты должникъ, а Я за тебе долгъ платилъ; ты на смерть осужденъ, а Я за тебе умеръ. И къ сему Моему за тебе страданію не иное что, какъ любовь Моя къ тебѣ привлекла Мене. Почто убо самовольно сію Мою любовь, труды, подвиги, страданія, тебе ради подъятыя, пренебрегаешь, и самъ себе хощешь погубить, не \textit{истлѣннымъ сребромъ или златомъ избавльшися} отъ грѣха, діавола, смерти и ада, но \textit{честною Моею кровію}\footnote{1~Петр.~1,~19.}?! Помяни сіе, коликая, коль драгая за тебе цѣна дана: кровію Моею и смертію Моею искупленъ еси! Помяни сіи всѣ Мои заслуги, которыми тебѣ заслужилъ отпущеніе грѣховъ и жизнь вѣчную, и покайся: и тако спасешися. \textit{Се стою при дверехъ, и толку: аще кто услышитъ гласъ Мой, и отверзетъ двери, вниду къ нему, и вечеряю съ нимъ, и той со Мною}»\footnote{Апок.~3,~20.}. О божественнаго, о любезнаго, о сладчайшаго гласа "--- гласа, спасенія нашего ищущаго! О преблагій и милосердый Іисусе, милостивый нашъ Искупитель! не отступи отъ насъ грѣшныхъ, ихже святою Твоею кровію искупилъ еси; но ударяй, ударяй въ двери каменныхъ сердецъ нашихъ, ударяй крѣпко пресладкимъ и спасительнымъ гласомъ Твоимъ, да отъ глубокаго грѣховнаго сна пробудимся, и услышимъ пресладкій и прелюбезный гласъ Твой; яко гласъ Твой сладокъ, и образъ Твой красенъ: и тако начнемъ сами просити, искати и толкати, толкати въ двери милосердія Твоего\footnote{Матѳ.~7,~7 и 8.}. Тако сынъ Божій зоветъ въ слѣдъ заблуждающаго грѣшника, и обращаетъ къ покаянію! И Отецъ съ небесе о томъ свидѣтельствуетъ: \textit{Того послушайте}\footnote{17,~5.}. И Духъ Святый во псалмѣ глаголетъ намъ: \textit{днесь, аще гласъ Его услышите, не ожесточите сердецъ вашихъ}\footnote{Пс.~94,~8; Евр.~3,~7.}. Тако святый и милостивый нашъ Богъ, Отецъ, Сынъ и Святый Духъ, зоветъ насъ Своею благодатію на покаяніе! "--- 3)~Зоветъ насъ на покаяніе чрезъ пророковъ и апостоловъ, яко посланниковъ Своихъ и рабовъ, которыхъ какъ бы въ слѣдъ насъ посылаетъ, и чрезъ нихъ обращаетъ насъ къ Себѣ, которые увѣщаваютъ насъ и молятъ: \textit{по Хрістѣ убо молимъ, яко Богу молящу нами, молимъ по Хрістѣ, примиритеся съ Богомъ}\footnote{2~Кор.~5,~20.}. И паки: \textit{обратитеся ко Мнѣ, глаголетъ Господь силъ, и обращуся къ вамъ, глаголетъ Господь силъ}\footnote{Зах.~1,~5; Мал.~3,~7.}. И паки: \textit{обратися, Израилю, ко Господу Богу твоему: зане изнемоглъ еси въ неправдахъ твоихъ}\footnote{Ос.~14,~2.}. И паки: \textit{обратися ко Мнѣ, доме Израилевъ, рече Господь, и не утвержду лица Моего на васъ: яко милостивъ Азъ есмь, рече Господь, и не прогнѣваюся на васъ во вѣки}\footnote{Іерем.~3,~12.}. И паки: \textit{и нынѣ глаголетъ Господь Богъ вашъ: обратитеся ко Мнѣ всѣмъ сердцемъ вашимъ, въ постѣ и въ плачи и въ рыданіи, и расторгните сердца ваша, а не ризы ваша, и обратитеся ко Господу Богу вашему; яко милостивъ и щедръ есть, долготерпѣливъ и многомилостивъ, и каяйся о злобахъ}\footnote{Іоил.~2,~12 и 13.}. И паки: \textit{обратитеся, и отвержитеся отъ всѣхъ нечестій вашихъ, и не будутъ вамъ неправды въ мученіе; отвержите отъ себе вся нечестія ваша, имиже нечествовасте ко Мнѣ, и сотворите себѣ сердце ново, и духъ новъ, и сотворите вся заповѣди Моя}\footnote{Іезек.~18,~30 и 31.}. Тако, якоже видиши здѣ, и на прочіихъ Писанія мѣстахъ чрезъ гласы пророческіе и апостольскіе обращаетъ Богъ грѣшника, и призываетъ къ покаянію. "--- 4)~Зоветъ чрезъ обѣщаніе временныхъ и вѣчныхъ благъ, которыми, какъ отецъ человѣколюбивый отроча малое яблокомъ, къ Себѣ привлекаетъ, да тако благими возбудимся и обратимся къ Нему; о чемъ въ пророческихъ книгахъ и апостольскихъ посланіяхъ довольно находимъ. Тако призываетъ къ Себѣ насъ Хрістосъ, и обѣщаетъ пресладкій покой душамъ нашимъ: \textit{пріидите ко Мнѣ вси труждающіися и обремененніи, и Азъ упокою вы. Возмите иго Мое на себе и научитеся отъ Мене, яко кротокъ есмь и смиренъ сердцемъ: и обрящете покой душамъ вашимъ}. Какъ бы сказалъ: «о бѣдные и утружденные суетами міра грѣшники, полно вамъ трудить себе и обременять замыслами, начинаніями, тщаніями, попеченіями и печальми житейскими. Нигдѣ вы не съищете истиннаго покоя и блаженства, кромѣ Мене. Къ чему вы ни обратитеся, никакой истинной себѣ пользы отъ того не сыщете. Вездѣ васъ, кромѣ Мене, срѣтаетъ истинная бѣда и зло. Богатство, честь, слава и сласть міра сего, которыхъ ищете, болѣе васъ обременятъ, нежели облегчатъ, болѣе утрудятъ, нежели упокоятъ. Добра ли себѣ хощете? всякое добро у Мене и отъ Мене происходитъ. Блаженства желаете? нѣтъ нигдѣ, кромѣ Мене. Красоты ищете? кто краснѣйшій паче Мене и кромѣ Мене? Благородія желаете? кто благороднѣе Сына Божія? Высоты хощете? Кто выше Царя небесъ? Славы ли ищете? кто славнѣе паче Мене? Богатства желаете? у Мене и въ Моей рукѣ все сокровище нетлѣнное. Премудрости ищете? Я Премудрость Божія. Дружества ищете? кто любезнѣе и любительнѣе паче Мене, иже душу мою за всѣхъ положилъ? Помощи, защищенія хощете? кто поможетъ и защититъ, кромѣ Мене? Врача ищете? кто исцѣлитъ, кромѣ Мене "--- душъ и тѣлесъ Врача? Утѣшенія, веселія и радости ищете? кто утѣшитъ и увеселитъ, кромѣ Мене? Покоя и мира себѣ хощете? Я миръ и покой душевный, и нигдѣ не сыщете, кромѣ Мене. Свѣта желаете? \textit{Азъ есмь свѣтъ міру: ходяй по Мнѣ не имать ходити во тмѣ}\footnote{Іоан.~8,~12.}. Не хощете прельститися и заблудити? \textit{Азъ есмь истина}. Хощете пріити къ Богу? \textit{Азъ есмь путь}. Не хощете умереть, но жить во вѣки? \textit{Азъ есмь животъ. Азъ есмь путь, истина и животъ}\footnote{14,~6.}. Приступить ко Мнѣ не смѣете? къ кому удобнѣйшій приступъ? всѣхъ зову и пріемлю: \textit{пріидите ко Мнѣ}. Просить опасаетеся? кому Я просящему съ вѣрою отказалъ? Грѣхи отвращаютъ васъ отъ Мене? Я за грѣшниковъ душу Мою положилъ. Множество грѣховъ смущаетъ васъ? болѣе у Мене милосердія паче всѣхъ всего міра грѣховъ. \textit{Пріидите ко Мнѣ вси труждающіися и обремененніи, и Азъ упокою вы}». Сего благопріятнаго и сладкаго гласа не должно намъ, грѣшниче, преслушать, но обратиться и пріитить къ милостивому Искупителю нашему, \textit{да обрящемъ покой душамъ нашимъ}. "--- 5)~Зоветъ чрезъ прещеніе временныхъ и вѣчныхъ наказаній, какъ о томъ такожде на многихъ святаго Писанія мѣстахъ видимъ. \textit{Аще не обратитеся, оружіе свое очиститъ, лукъ свой напряже, и уготова его, и въ немъ уготова сосуды смертныя, стрѣлы своя сгараемымъ содѣла}\footnote{Пс.~7,~13 и 14.}. \textit{О человѣче! или о богатствѣ благости Его и кротости и долготерпѣніи нерадиши, не вѣдый, яко благость Божія на покаяніе тя ведетъ? По жестокости же твоей и непокаянному сердцу, собираеши себѣ гнѣвъ въ день гнѣва и откровенія праведнаго суда Божія}\footnote{Римл.~2,~4 и 5.}. "--- 6)~Далѣе отеческій Божій промыслъ посылаетъ наказанія на грѣшниковъ, которыми какъ бы пресѣкаетъ имъ путь погибельный, дабы остановилися и обратилися къ Богу. И сіе"=то значатъ болѣзни, глады, пожары и прочія напасти. "--- 7)~Призываемся, наконецъ, чрезъ повседневные смертные случаи, когда видимъ братію нашу, отсюду отходящую, и сами себѣ того со дня на день ожидаемъ; всякому бо отъ насъ глаголетъ Богъ: \textit{земля еси, и въ землю возвратишися}, что сказано праотцу Адаму\footnote{Быт.~3,~19.}. Покайся убо, да возъимѣеши часть въ воскресеніи праведныхъ, которые востанутъ во славѣ.

\paragraph*{§\:142.} Причины, которыя могутъ и должны возбудить грѣшника на покаяніе, суть: 1)~Великое Божіе хотѣніе, желаніе и какъ бы алчба и жажда нѣкая спасенія нашего. \textit{Всѣмъ хощетъ спастися, и въ разумъ истины пріити} Богъ нашъ, чего видишь изъ прешедшаго параграфа, и прочіихъ пророческихъ и апостольскихъ увѣщаній примѣчается, которыя отеческое Божіе сердце и великое нашего спасенія желаніе открываютъ, и различными словами изображаютъ, а наипаче изъ того, что и Сына Своего единороднаго насъ ради не пощадилъ, но за насъ предалъ Его, да мы спасемся. Того ради различнымъ образомъ привлекаетъ насъ къ Себѣ, увѣщаетъ, молитъ чрезъ посланниковъ Своихъ, обѣщаетъ благая, угрожаетъ злыми, обѣщаетъ помощь и благодать подать обращающимся, обѣщаетъ отпущеніе грѣховъ содѣянныхъ и не помянуть ихъ, устрашаетъ вѣчнымъ огнемъ, и вѣчное блаженство открываетъ; и обѣщаетъ, да тако подвигнемся къ покаянію и спасеніе получимъ. Не подобаетъ убо намъ такъ великаго милосерднаго Бога хотѣніе презирать, но обратиться и каяться. "--- 2)~Требуетъ того отъ насъ чудное Сына Божія о насъ смотреніе, воплощеніе, смиренное на земли пожитіе, страданіе и смерть, которое все нашего ради спасенія совершилъ. Но все тое безполезно бываетъ тѣмъ, которые безъ покаянія пребываютъ, и не хотятъ себе исправить и плодовъ покаянія показать; но паче болѣе ихъ осудитъ, яко толикую благодать презрѣвшихъ. Слышишь и читаешь въ святомъ Евангеліи, грѣшниче, какъ смиреннымъ образомъ Избавитель нашъ пришелъ къ намъ спасти насъ; ясли и вертепъ, гдѣ родился, показуютъ тое: ты же хощешь въ гордости жити. Въ какой нищетѣ на землѣ пожилъ ради тебе, такъ, что Сынъ человѣческій не имѣлъ гдѣ главы подклонити: ты же ненасытно имѣній и богатства желаеши. Колико лѣтъ на землѣ трудился ради души твоея: ты же не печешися и не трудишися о спасеніи твоемъ, но все тое презираеши. Колико плакалъ о погибшей душе твоей, Иже во днехъ плоти Своея моленія же и молитвы къ Могущему спасти Его отъ смерти съ воплемъ крѣпкимъ и со слезами принеслъ: ты же не хощеши воздохнути и пролити слезъ ради души своея, жалѣти о грѣхахъ своихъ. Какія хулы и поношенія претерпѣлъ ради тебе: ты же не хощешь претерпѣть обиды отъ ближняго твоего, но злобишься на него и духомъ отмщенія дышеши. Помяни, како скорбѣлъ, тужилъ, ужасался, кровавымъ обливался потомъ: ты же не хощешь потужити, поскорбѣти и поболѣти о грѣхахъ твоихъ, которыми величество Божіе оскорбилъ; но безумно провождаешь въ веселостяхъ мірскихъ дни твоя. Како страшно поруганъ, посмѣянъ, оплеванъ, заушенъ и уязвленъ былъ ради тебе: ты же чести и славы въ мірѣ семъ ищеши и хощеши прославитися, \textit{земля и пепелъ}, бѣдный и отверженный. Како ужасно мученъ, распятъ, посредѣ разбойниковъ повѣшенъ, ко кресту пригвожденъ, и безчестно тебе ради умеръ: ты же въ сластяхъ и роскошахъ плотскихъ не престаеши утѣшатися. Сія неизреченная Его любовь и благодать вопіетъ намъ, да обратимся и оставимъ беззаконное житіе, и покаемся, \textit{и тако со страхомъ житія нашего время жительствуемъ, вѣдяще, яко не истлѣннымъ сребромъ или златомъ избавихомся отъ суетнаго житія нашего, отцы преданнаго, но честною кровію, яко Агнца непорочна и пречиста Хріста}\footnote{1~Петр.~1,~17--19.}. Намъ ли тѣ грѣхи творить, за которые Хрістосъ Сынъ Божій такъ язвленъ и мученъ былъ? Самая кровь Его вопіетъ намъ, да престанемъ отъ грѣховъ и покаемся. Аще же не покаемся, то таяжде кровь вопити будетъ на насъ, да суду Божію подпадемъ, и по дѣломъ нашимъ воспріимемъ. Страшно сіе слышати, но неотмѣнно будетъ тое! \textit{Богъ бо поругаемъ не бываетъ}\footnote{Гал.~6,~7.}. "--- 3)~Можетъ и должно возбудить насъ къ покаянію прещеніе Божіе. Богъ претитъ казнію некающимся. \textit{Аще не обратитеся, оружіе Свое очиститъ, лукъ Свой напряже, и уготова его, и въ немъ уготова сосуды смертныя}, и проч. Божіе же прещеніе, человѣче, не есть пустое и суетное, но непремѣнно оно будетъ самымъ дѣломъ, когда человѣкъ его пренебрежетъ и не исправитъ себе. И сокрыться отъ Него нигдѣ невозможно; вездѣ срѣтаетъ отмщеніе Его некающихся, якоже чрезъ пророка глаголетъ: \textit{аще скрыются во адъ, то и оттуду рука Моя исторгнетъ я; аще взыдутъ на небо, то и оттуду свергу я; и аще скрыются на версѣ кармила, то и оттуду взыщу и возму я; и аще погрузятся отъ очію Моею во глубинахъ морскихъ, то и тамо повелю зміеви, и угрызетъ я}\footnote{Ам.~9,~2,~3.}, и проч. Едино только убѣжище отъ гнѣва Божія есть "--- покаяніе; оно сокрываетъ грѣшника отъ мщенія и казни. \textit{Возглаголю на языкъ и на царство, да искореню ихъ, и разорю, и расточу я. И аще обратится языкъ той отъ всѣхъ лукавствъ своихъ, то раскаюся о озлобленіяхъ, яже помыслихъ сотворити имъ}\footnote{Іер.~18,~7 и 8.}. Къ сему безопасному убѣжищу прибѣгли Ниневитяне, послышавше грядущій гнѣвъ Божій, и сокрылися, и тако спаслися. Къ сему и нынѣ многіе прибѣгаютъ, и спасаются. Подобаетъ убо намъ, человѣче, къ томужде убѣжища граду бѣжать, да въ немъ сокрыемся отъ казни и мщенія Божія: покаяніе бо есть гора оная, въ которой Лотъ отъ губительнаго гнѣва Божія, Содомъ и Гоморръ попалившаго, сокрылся. Въ сію тихую и безопасную гору неотмѣнно должно убѣжать и сокрыться въ ней, кто не хощетъ съ беззаконнымъ міромъ, яко съ Содомомъ и Гоморромъ, погибнуть. Бѣжи убо, бѣжи, грѣшниче, въ сію гору, пока еще гнѣвъ Божій, какъ огнь отъ Господа съ небесе, не спалъ, о не озирайся вспять. \textit{Поминай жену Лотову}, якоже увѣщаваетъ насъ Хрістосъ\footnote{Лук.~17,~32.}. "--- 4)~Поощряетъ насъ къ покаянію неизвѣстная кончина житія нашего, которая неотмѣнно будетъ, но когда будетъ, неизвѣстно. Часъ сей неизвѣстенъ намъ опредѣлилъ Богъ, дабы всегда его ожидали. Богъ обѣщаетъ насъ принять кающихся, подать отпущеніе грѣховъ, но утрешняго дня не обѣщаетъ. Внимай сему, бѣдный грѣшникъ, и утрешняго дня себѣ не полагай къ покаянію. Богъ всегда велитъ готовымъ быть къ исходу; и каковыхъ застанетъ насъ смертный часъ, таковыми и предъ судомъ Его явимся. И потому каковымъ хощешь умрети, такимъ долженъ еси и во всемъ житіи быти. Хощеши ли умрети такъ, какъ благочестивые хрістіане умираютъ? Убо и жити такъ тебе должно, какъ благочестивые живутъ. Всякъ бо человѣкъ есть рабъ Божій. Рабъ же всегда долженъ въ готовности быти, когда ни позоветъ его къ себѣ господинъ его. Тако и хрістіанинъ, яко рабъ Божій, всегда готовымъ долженъ быть, когда позоветъ его Богъ. А призываетъ всякаго къ Себѣ Богъ чрезъ кончину житія сего. \textit{Блажени суть раби тіи, ихже пришедъ Господь обрящетъ бдящихъ}\footnote{Лук.~12,~37.}! Непремѣнно же тотъ бдитъ, кто во всегдашнемъ находится покаяніи. Окаянніи и бѣдніи суть, которыхъ обрящетъ Господь любовію міра сего упоенныхъ и сномъ грѣховнымъ спящихъ! Помяни, гдѣ богачъ оный, который \textit{облачашеся въ порфиру и виссоннъ, веселяся на вся дни свѣтло}? Вопіетъ изъ ада, сый въ мукахъ, и безъ конца вопить будетъ. Гдѣ прочіе грѣшники непокаявшіися и смертію восхищенные? Пребываютъ въ своихъ опредѣленныхъ мѣстахъ, ожидаютъ всемірнаго суда Божія и послѣдняго опредѣленія, и совершеннаго по дѣломъ воздаянія: желали бы возвратиться въ міръ сей, и всякое учинить покаяніе, но не дается имъ. Мы еще, слава Богу, на землѣ живемъ, еще смерть насъ не восхитила, еще не ушло намъ время покаятися, еще можемъ спастися Божіею благодатію, еще Богъ призываетъ насъ къ покаянію и обѣщаетъ милость: нужъ, покаемся, человѣче, и покажемъ плоды покаянія, и утвердимъ себе въ томъ новомъ житіи, да, когда ни пріидетъ къ намъ часъ смертный, застанетъ насъ, яко хрістіанъ, которые блаженно умираютъ, яко умираютъ о Господѣ, и тако сею смертію, какъ дверьми, внидемъ въ животъ лучшій и вѣчное блаженство. "--- 5)~Страшный судъ Хрістовъ да подвигнетъ насъ къ покаянію, на которомъ всѣмъ намъ "--- всѣмъ, праведнымъ, глаголю, и грѣшнымъ, явитися подобаетъ и воздати о содѣянныхъ отвѣть праведному Судіи. Нынѣ Богъ на покаяніе зоветъ; тогда къ отвѣту позоветъ. Нынѣ глаголетъ: \textit{покайтеся}; тогда возглаголетъ: отвѣщайте Мнѣ. Нынѣ призываетъ: \textit{пріидите ко Мнѣ вси труждающіися и обремененніи, и Азъ упокою вы}, тогда возглаголетъ: \textit{идите отъ Мене, проклятіи, во огнь вѣчный}. Убоимся убо суда Божія, паче же Самаго Бога, который судитъ не яко человѣкъ внѣшнія только дѣла, но и самыя помышленія испытуетъ, и не стыдится лица человѣческаго, но всякому по своимъ дѣламъ воздаетъ, и не требуетъ свидѣтелей, но Самъ всему "--- слову, дѣлу и помышленію всякому, свидѣтель вѣрный, \textit{яко сердца и утробы испытуяй Онъ есть}\footnote{Іер.~17,~10; Пс.~7,~10.}. "--- 6)~Вѣчная смерть или адское мученіе сильно есть подвигнуть человѣка къ обращенію. Тогда престанетъ все милосердіе, но правда Божія вступитъ въ свое дѣло; тогда раскаяніе безполезно; плачь и слезы суетны тогда. Тамо услышится отвѣтъ: \textit{чадо! помяни, яко воспріялъ еси благая твоя въ животѣ твоемъ}\footnote{Лук.~16,~25.}. Нынѣ несносенъ тебѣ жаръ, въ движеніи крови учинившійся: како стерпишь тотъ пламень, котораго и демоны трепещутъ, который мучитъ, а не снѣдаетъ!.. "--- 7)~Вѣчное блаженство, избраннымъ Божіимъ уготованное, да привлечетъ насъ къ покаянію и прилѣжному тщанію въ дѣлѣ благочестія. Какое же оно будетъ, умъ понять и слово изъяснить не можетъ. Будетъ оно въ зрѣніи Божія лица: тогда увидятъ Бога лицемъ къ лицу. \textit{Возлюбленніи}, глаголетъ апостолъ святый, \textit{нынѣ чада Божія есмы, и не у явися, что будемъ: вѣмы же, яко егда явится, подобни Ему будемъ, ибо узримъ Его, якоже есть}\footnote{1~Іоан.~3,~2.}. Но входятъ въ тое чистые и святіи, покаяніемъ и вѣрою очищенные. Хотящему туды внити надобно \textit{убѣлить ризы Своя въ крови Агнчей}\footnote{Апок.~7,~14.}. Ибо \textit{внѣ псы и чародѣи, и любодѣи, и убійцы и идолослужители, и всякъ любяй и творяй лжу}\footnote{22,~15.}. Не безумное ли убо дѣло ради временныя и мнимыя сладости погубить вѣчную радость?! "--- 8)~Наконецъ увѣщаваетъ и убѣждаетъ насъ къ покаянію совѣсти грызеніе, которое отъ грѣховъ бываетъ, отъ котораго грызенія ничимъ инымъ, какъ истиннымъ покаяніемъ избавляемся, и получаемъ тишину, покой и утѣшеніе.

\paragraph*{§\:143.} Какъ человѣкъ, идучи по пути, когда усматриваетъ, что не тотъ путь, по которому должно ему итить, или неполезенъ ему путь тотъ, тако размышляя, останавливается и обращается взадъ, и идетъ, куды ему надобно: тако грѣшникъ, идучи по пути погибельному, который въ вѣчную ведетъ пагубу, когда благодатію Божіею приходитъ въ познаніе своего заблужденія и окаянства, и, размышляя о своемъ заблужденіи, глаголетъ: что я это дѣлаю? къ какому концу путь мой ведетъ мене? Тако усумнѣваяся, какъ бы останавливается, и не идетъ далѣе. О коль доброе размышленіе, которое приводитъ въ сіе спасительное сумнѣніе! Коль блаженное сумнѣніе, которое воспящаетъ волю, злое зачинающую, и удерживаетъ ноги на зло идущія, и пресѣкаетъ путь, къ погибели ведущій! Ибо въ такомъ недоумѣніи находится первый къ спасенію степень. Познать себе и свое окаянство начало спасенія есть: понеже познаніе своея бѣдности приводитъ къ исканію способа, дабы избавиться отъ бѣды. Тако сотворилъ блудный сынъ, удалившійся отъ отца своего и находясь въ бѣдѣ глада, который, \textit{въ себе пришедъ, рече: колико наемникомъ отца моего избываютъ хлѣбы, азъ же гладомъ гиблю? И воставъ иде ко отцу своему}\footnote{Лук.~15,~19 и 20.}. Тако грѣшный человѣкъ, размышляя о своей бѣдности и благости небеснаго Отца, которою вси работающіи Ему наслаждаются, приходитъ въ себе и глаголетъ. «Коль многіе у Бога получаютъ милость истиннымъ покаяніемъ: я же той лишаюся своимъ небреженіемъ. Воставъ пойду и я къ Богу, возвращуся къ Нему покаяніемъ, отъ Котораго удалился беззаконнымъ нравомъ; повергу себе предъ Нимъ со смиреніемъ и со слезами, отъ Котораго отступилъ гордостію и сластолюбіемъ; признаю предъ Нимъ себе недостойнымъ ни неба, ни земли, который себе за немало почиталъ; объявлю Ему съ жалѣніемъ грѣхъ мой, которымъ Его оскорблялъ». Воставъ иду ко Отцу моему, Котораго доброты ангели и святіи наслаждаются; иду ко Отцу моему, отъ Котораго самовольно и безумно отлучился; \textit{иду ко Отцу моему, и реку Ему: Отче мой, согрѣшихъ на небо и предъ Тобою}!»\footnote{18.}. Тако, или подобно сему размышляя, бѣдный грѣшникъ, когда сердце свое и произволеніе на доброе премѣняетъ, какъ бы взадъ обращается; и, когда въ произволеніи, добрѣ начатомъ, стоитъ, и плоды обращенія показываетъ, усматривая пороки и недостатки въ совѣсти своей, тщится ихъ молитвою и сокрушеніемъ сердца истребить, какъ бы идетъ къ Богу, отъ Котораго злымъ житіемъ удалился; и чимъ усерднѣе исправляетъ себе, тѣмъ скорѣе идетъ, и какъ бы бѣжитъ къ дому Отца небеснаго. Тако сотворилъ помянутый блудный сынъ, \textit{и воставъ иде ко отцу своему}. Тако обратившагося и приходящаго грѣшника, когда видитъ Богъ, Отецъ небесный, "--- о коль благопріятно \textit{смотритъ} на него! какъ \textit{милъ} бываетъ святымъ и милосерднымъ очесамъ Его! съ какою охотою приближающагося срѣтаетъ его! какъ любезно пріемлетъ его! коль горячимъ любве Своея божественныя лобзаніемъ \textit{облобызаетъ} его\footnote{Лук.~15,~20.}! Повелѣваетъ \textit{первую изнести одежду и облещи его; даетъ перстень на руку его и сапоги на нозѣ его}\footnote{21.}. Созываетъ ангеловъ и праведныхъ души, аки други и сосѣды Своя, и велитъ имъ радоватися, веселитися и ликовати, глаголя: \textit{радуйтеся со Мною, яко сынъ Мой сей мертвъ бѣ, и оживе; и изгиблъ бѣ, и обрѣтеся}\footnote{9,~10 и 24.}. Сея радости, веселія, ликованія участникомъ хощетъ грѣшника сотворити Хрістосъ, единородный Сынъ Божій. \textit{Се стою при дверехъ, и толку: аще кто услышитъ гласъ Мой, и отверзетъ двери, вниду къ нему, и вечеряю съ нимъ, и той со Мною}\footnote{Апок.~3,~20.}. Еже значитъ благодатное Его, увеселительное въ сердцѣ кающагося пребываніе.

\paragraph*{§\:144.} Послушай убо, грѣшниче, Божія гласа; исполни хотѣніе Отца небеснаго, жаждущее спасенія твоего. Воззри на смерть и кровь Хрістову, изліянную тебе ради. Цѣна за тебе великая дана "--- кровь Хрістова, дражайшая всего міра! Покайся, пока время есть; обратися пока пріемлетъ Господь. Се кровь Хрістова вопіетъ къ тебѣ о семъ. Се Отецъ небесный ожидаетъ тебе. Ангели готовы радоватися о покаяніи твоемъ. Праведницы ждутъ тебе, дондеже покаешися, и о кающемся возрадуются. Се смерть настоитъ, отворяющая безконечную вѣчность! Се судъ Хрістовъ устрашаетъ, геенна ужасаетъ, блаженство вѣчное ободряетъ, Евангеліе увѣряетъ о семъ! Великая вещь есть спасеніе, ради котораго Сынъ Божій кровь Свою проліялъ: то \textit{едино есть на потребу}\footnote{Лук.~10,~42.}. Какъ единожды сыщешь, никогда не потеряешь: а какъ потеряешь, никогда не сыщешь. Здѣ его должно искать, и искать болѣе, нежели пищи, питія и одежды, и живота временнаго: а по смерти не сыщешь, хотя съ воздыханіемъ и слезами поищешь. Тѣло алчетъ, жаждетъ, и питаешь и напояешь его: душа алчетъ, жаждетъ, "--- и небрежеши! Тѣло нагое одѣваешь, грязное омываешь, немощное врачуешь: душа нага, пороками замарана, немощна, и уже мертва, "--- и нерадиши! Домъ горитъ, и тщишься угасить: домъ душевный горитъ, и уже сгараетъ, "--- и нерадиши! Богатство похищенное искать стараешься: сатана все богатство душевное похитилъ, "--- и нерадиши! О бѣдные и окаянные грѣшники, что мы спимъ?.. Се сатана крадетъ неоцѣненное сокровище наше "--- вѣчное спасеніе, которое Хрістосъ кровію Своею пріобрѣлъ намъ! Се гласъ Божій въ совѣсти нашей ударяетъ, и возбуждаетъ насъ: \textit{востани спяй, и воскресни отъ мертвыхъ}\footnote{Еф.~5,~14.}. Апостоли и проповѣдники Божія слова молятъ насъ: \textit{молимъ по Хрістѣ, примиритеся съ Богомъ}. Что мы не пробудимся, пока время не ушло, пока Богъ зоветъ и пріемлетъ кающихся?! Будетъ время, когда не звать, но судить будетъ грѣшниковъ.

\paragraph*{§\:145.} Худо и несмысленно дѣлаемъ, когда отлагаемъ покаяніе или до болѣзни, или до старости, или до кончины, или день отъ дне: понеже сколько жить, и когда умереть намъ, не въ нашей состоитъ власти, но въ Божіей. Да и тогда ли хощемъ Богу жить, когда оканчиваемъ житіе? Тогда ли престать отъ грѣховъ, когда уже не можемъ грѣшить? Тогда ли обращаться къ Богу, когда Богъ Самъ и нехотящихъ насъ къ Себѣ зоветъ? Тогда ли исправлять житіе, когда оканчиваемъ житіе? Діавольская се кознь есть! Онъ сію мысль намъ влагаетъ, чтобы тако удобнѣе моглъ погубить насъ; его то дѣло есть "--- едино милосердіе Божіе человѣку представлять, минуя правосудіе Его, чтобы тако удобнѣе человѣка къ грѣхамъ приводить и въ грѣхахъ содержать моглъ. Есть Богъ милостивъ, какъ въ слѣдующемъ параграфѣ увидишь; но есть и праведенъ. Посему должно всегда единымъ окомъ на милосердіе Божіе, а другимъ на правду Его смотрѣть. Тако разсужденіе Божія милосердія согрѣшившему отчаяться, а о правдѣ Его размышленіе болѣе грѣшить не попуститъ. «Чтобы, глаголетъ Василій Великій, не согрѣшить, нуженъ есть страхъ суда Божія; чтобы согрѣшившему не отчаяться, полезна есть надежда милосердія Божія».

Отъ вышеписанныхъ такое послѣдуетъ разсужденіе: 1)~Сколь великая человѣческая слѣпота "--- отвращаться отъ Бога ко грѣху, отъ свѣта ко тьмѣ, отъ живота къ смерти, отъ блаженства къ къ бѣдствію! 2)~Сколь тяжко грѣшитъ и безстыдствуетъ человѣкъ, когда, вмѣсто послушанія, ослушаніе Богу показуетъ, и тако вмѣсто лица своего, хребетъ обращаетъ, какъ глаголетъ пророкъ\footnote{Іер.~32,~35.}. Сіе человѣческое безстудство и страшная дерзость и на другихъ святаго Писанія мѣстахъ изображается, якоже пишется о беззаконникахъ: \textit{не предложиша Бога предъ собою}\footnote{Пс.~53,~5.}; и паки: \textit{и не предложиша Тебе предъ собою}, глаголетъ Псаломникъ къ Богу\footnote{85,~14.}. Бога"=де не предложили предъ собою, Котораго должны предъ собою имѣть, и смотрѣть, что повелѣваетъ, и тако почтеніе, послушаніе и покореніе Ему показывать; но они не учинили того: \textit{не предложиша Бога предъ собою}. И паки ко грѣшнику Богъ глаголетъ: \textit{отверглъ еси словеса Моя вспять}\footnote{Пс.~49,~17.}. Которые"=де словеса долженъ ты имѣть предъ собою, и, на нихъ смотря, исполнять ихъ; но ты, грѣшниче, взадъ себе поверглъ ихъ, чтобы тебѣ ихъ не видѣть и не соблюдать, какъ ненужныхъ. И сіе есть крайнее безстыдство и неистовство грѣшника, хотя онъ того и не примѣчаетъ. "--- 3)~Видно отсюду, что есть безбожное и нечестивое житіе, "--- то"=есть житіе, грѣхами противу совѣсти исполненное и нераскаянное. "--- 4)~Отрицается Бога и Хріста Сына Божія не токмо тотъ, который устами Его не исповѣдуетъ, но и тотъ, который не слушаетъ Его, ученію Его не послѣдуетъ; или кратко, отвергается человѣкъ Бога и Хріста Сына Божія не токмо устами, но и беззаконнымъ житіемъ. Разсуди всякъ сіе; Богъ повелѣваетъ: \textit{не убій}, гнѣвъ говоритъ: убій. Богъ запрещаетъ: \textit{не прелюбодѣйствуй}; плоть глаголетъ: дерзай. Богъ глаголетъ: \textit{не укради}; а мамона говоритъ: укради, и проч. Грѣшникъ, оставивши Бога, гнѣву, плоти и мамонѣ повинуется, и такъ самымъ дѣломъ показуетъ, что не Бога, но гнѣвъ, похоть и мамону за Господа своего имѣетъ, которыхъ слушаетъ, хотя устами и исповѣдуетъ Бога Господа своего быти. "--- 5)~Что обращеніе не въ томъ только состоитъ, чтобы внѣшніе грѣхи, какъ"=то: блудъ, хищеніе, піянство и прочая, оставить, но чтобы и сердце перемѣнить и новое житіе, противное первому, начать. Богъ бо новаго сердца и новаго духа отъ насъ требуетъ, якоже глаголетъ: \textit{сотворите себѣ сердце ново и духъ новъ}\footnote{Іез.~18,~31.}. Отъ внѣшнихъ бо грѣховъ и лицемѣры воздерживаются то ради стыда, то ради страха человѣческаго, то ради своего прибытка временнаго: не убиваютъ, но внутрь злобы и убійства исполнены; не блудодѣйствуютъ, но сердце похотію нечистою наполненное имѣютъ; не крадутъ, но желаютъ чужаго добра. Богъ бо по сердцу судитъ. И потому сердца новаго и духа новаго не имѣетъ тотъ, который не токмо грѣшитъ, но и хощетъ грѣшить. Духъ бо новый имѣяй не токмо не крадетъ, но и не хощетъ красти, хотя бы и случай былъ ему на тое. А кто отъ внѣшняго удерживается, напр. отъ хищенія, а внутрь непремѣнное сердце имѣетъ, тотъ въ случаѣ открываетъ себе, напр. хищникъ въ случаѣ похищаетъ. "--- 6)~Видно еще, коль жестокое и нераскаянное сердце имѣетъ грѣшникъ, который гласа Божія, столько кратъ въ сердце ударяющаго, не слушаетъ, столько увѣщаній пренебрегаетъ, о толикой благости, кротости и долготерпѣніи Божіи нерадитъ. "--- 7)~Коль великая и чудная Божія къ намъ благость, которая различными образы ищетъ обращенія нашего, и насъ отвращающихся не отвращается, и необращающихся призываетъ и не престаетъ призывать. "--- 8)~Всякъ не обращающійся и нераскаянный грѣшникъ самъ своей погибели виновенъ бываетъ, который о толикой благости Божіей, на покаяніе его ведущей, нерадитъ.

\subsection[Глава 2-я. О надеждѣ или утѣшеніи хотящимъ каятися.]{глава вторая.\\\bfseries О надеждѣ или утѣшеніи хотящимъ каятися.}

\begin{quotation}\textit{Беззаконникъ, аще обратится отъ всѣхъ беззаконій своихъ, яже сотворилъ, и сохранитъ вся заповѣди Моя, и сотворитъ судъ и правду и милость, жизнію поживетъ, и не умретъ: вся согрѣшенія, елика сотворилъ, не помянутся ему, но о правдѣ своей, юже сотворилъ, живъ будетъ. Еда хотѣніемъ восхощу смерти грѣшника, глаголетъ Адонаи Господь, а не еже обратитися ему отъ пути зла, и живу быти ему}\footnote{Іез.~18,~21--23.}?\end{quotation}


\paragraph*{§\:146.} Какъ то есть діавольская кознь, что онъ, прежде грѣха, милостиваго Бога представляетъ намъ, чтобы тако удобнѣе моглъ насъ въ грѣхъ привести; такъ и то его хитрость есть, что по согрѣшеніи правосудіе Божіе предлагаетъ, чтобы въ отчаяніе согрѣшившаго низринуть. И тотъ и сей совѣтъ есть пагубный злаго и враждебнаго духа. Сего ради, чтобы согрѣшившимъ въ пагубную сію сѣть не впасть, предлагаются причины нѣкія, которыя подаютъ надежду и утѣшеніе хотящимъ каятися къ полученію Божія милосердія. 1)~Самъ Богъ различнымъ образомъ на покаяніе призываетъ насъ, какъ въ прешедшей главѣ сказано. Убо тѣмъ самымъ хощетъ кающихся принять и помиловать. "--- 2)~Богъ, ожидая на покаяніе, \textit{долготерпитъ о насъ, не хотя, да кто погибнетъ, но да вси въ покаяніе пріидутъ}, какъ Петръ апостолъ учитъ\footnote{2~Петр.~3,~19.}. Какъ убо не пріиметъ кающагося, Который ожидаетъ на покаяніе? Како отвратится обращающагося, Который отвратившемуся долготерпитъ, чтобы въ чувство пришелъ и обратился? "--- 3)~Богъ есть весьма милосердъ; и милосердіе Его толикое, коликое величество Его. Величество же Его безконечно, убо и милосердіе безконечно. Не могутъ убо быть такъ многіе и такъ великіе грѣхи наши, которыхъ бы не превышало Божіе милосердіе. Есть же милосердый тотъ, который соболѣзнуетъ о бѣдствіи другаго, болѣзнь и состраданіе въ сердцѣ своемъ ощущаетъ. Такъ отецъ и матерь о бѣдствіи дѣтей своихъ соболѣзнуютъ. Такую болѣзнь почувствовалъ въ сердцѣ своемъ Давидъ святый, когда услышалъ о убіеніи сына своего Авессалома, и плакася тако: \textit{сыне мой Авессаломе, сыне мой, сыне мой Авессаломе! кто дастъ смерть мнѣ вмѣсто тебе}\footnote{2~Цар.~18,~33.}? Такое отческое и сострадательное сердце, или паче горячайшее, Богъ надъ бѣдствіемъ и окаянствомъ нашимъ имѣетъ. Сіе Божіе благоутробіе показалъ Сынъ Божій, когда за насъ, благоволеніемъ Отца Своего небеснаго, волею пострадалъ. И какъ Давидъ желалъ вмѣсто сына своего умрети, такъ Хрістосъ самою вещію вмѣсто насъ, сыновъ беззаконныхъ, умеръ, и тако насъ мертвыхъ Своею смертію оживилъ и отъ бѣдствія избавилъ. Како убо милосердый Богъ не умилосердится надъ нами, когда со слезами къ нему обратимся, и бѣдность и окаянство наше Ему объявимъ и признаемъ? "--- 4)~\textit{Тако возлюби Богъ міръ, яко и Сына Своего единороднаго далъ есть, да всякъ вѣруяй въ Онь, не погибнетъ, но имать животъ вѣчный}, глаголетъ Хрістосъ\footnote{Іоан.~3,~16.}. Како убо не отпуститъ тебѣ грѣховъ твоихъ ради тогожде Сына Своего, какіе и коликіе бы они ни были, Который ради тебе Сына Своего не пощадѣлъ? Како не проститъ грѣховъ твоихъ тебѣ кающемуся, Который за грѣхи твои Сына Своего предалъ на смерть? \textit{Иже убо Своего Сына не пощадѣ, но за насъ всѣхъ предалъ есть Его: како убо не и съ Нимъ вся намъ дарствуетъ}\footnote{Римл.~8,~32.}? Великое подалъ Онъ намъ, то есть, Сына Своего: велико ли Ему отпустить грѣхи наши намъ кающимся? "--- 5)~Благопріятно и радостно Богу милостивому отпущать грѣхи кающемуся грѣшнику. Понеже тогда смерти Хрістовой плодъ показуется, что грѣшникъ кающійся спасается. \textit{Хрістосъ бо пріиде въ міръ грѣшники спасти}\footnote{1~Тим.~1,~13.}. Тогда радость бываетъ предъ Отцемъ небеснымъ и ангелами Его: яко кровь Хрістова не всуе проліянна бысть ради того грѣшника. Тогда воля небеснаго Отца исполняется, Который \textit{всѣмъ хощетъ спастися и въ разумъ истины пріити}\footnote{1,~4.}. \textit{Радость бо бываетъ на небеси о единомъ грѣшницѣ кающемся}, глаголетъ Хрістосъ\footnote{Лук.~15,~7.}. "--- 6)~Богъ не хощетъ смерти грѣшника, но обратитися ему и живу быти, якоже глаголетъ чрезъ пророка: \textit{еда хотѣніемъ восхощу смерти грѣшника, глаголетъ Адонаи Господь, а не еже обратитися ему отъ пути зла, и живу быти ему}? Убо хощетъ обратившагося помиловать, и животъ вѣчный подать ему. "--- 7)~Богъ обѣщается и грѣховъ кающагося не помянуть, какъ чрезъ тогожде пророка глаголетъ: \textit{беззаконникъ, аще обратится отъ всѣхъ беззаконій своихъ, яже сотворилъ, и сохранитъ вся заповѣди Моя, и сотворитъ судъ и правду и милость, жизнію поживетъ, и не умретъ: вся согрѣшенія, елика сотворилъ, не помянутся ему, но въ правдѣ своей, юже сотворилъ, живъ будетъ}. "--- 8)~Богъ нехотѣніе Свое, которымъ не хощетъ смерти грѣшника, и клятвою утвердилъ, чтобъ мы не сумнились къ Нему съ покаяніемъ приходить. \textit{Живу Азъ, глаголетъ Адонаи Господь, не хощу смерти грѣшника, но еже обратитися нечестивому отъ пути своего, и живу быти ему}\footnote{Іез.~33,~11.}. О, милости неизреченныя Бога нашего, Который бѣднаго ради грѣшника клянется! О, блаженства нашего, ради которыхъ Богъ великій и Создатель нашъ клянется! О, окаянства тѣхъ, которые и клятвѣ сей не вѣрятъ!.. "--- 9)~Богъ въ томъ Писаніи Своемъ объявилъ, что Сынъ Его Іисусъ Хрістосъ будетъ судить всему міру, и воздастъ комуждо по дѣломъ его. Убо тѣмъ самымъ милостивно предостерегаетъ и поощряетъ къ покаянію, чтобы, покаявшеся, страшнаго по дѣломъ осужденія избѣгли. "--- 10)~Богъ грозитъ казнію нехотящимъ каятися. \textit{Аще не обратитеся, оружіе Свое очиститъ}, и проч.\footnote{Пс.~7,~13.} Убо тѣмъ самымъ хощетъ кающагося помиловать. "--- 11)~Богъ и казнь опредѣленную отвращаетъ, когда видитъ кающихся грѣшниковъ. Тако отвратилъ казнь отъ Ниневитянъ кающихся. \textit{И видѣ Богъ дѣла ихъ, яко обратишася отъ путей своихъ лукавыхъ: и раскаяся Богъ о злѣ, еже глаголаше сотворити имъ, и не сотвори}\footnote{Іон.~3,~10.}. "--- 12)~Хрістосъ Сынъ Божій того ради на землю пришелъ, чтобы грѣшниковъ призвать на покаяніе, какъ Самъ объявляетъ: \textit{пріиде Сынъ человѣческій взыскати и спасти погибшаго}\footnote{Лук.~19,~10.}. Звалъ прежде чрезъ пророковъ, какъ посланниковъ Своихъ; а послѣди Самъ въ тое великое и чудное дѣло вступилъ. Самъ \textit{на земли явися и съ человѣки поживе}, и призвалъ блудниковъ, разбойниковъ, мытарей и прочіихъ, и имъ кающимся царствіе Свое небесное отворилъ. Не могли мы къ Нему пріитить: того ради Самъ Онъ къ намъ пришелъ \textit{взыскати и спасти насъ}. Великое отсюду утѣшеніе грѣшникамъ проистекаетъ: Сынъ Божій пришелъ ихъ призвать на покаяніе!.. Великая ихъ слѣпота, когда не чувствуютъ такъ великой Божіей благодати! "--- 13)~Когда Богъ казнитъ некающагося грѣшника, не съ благоволеніемъ казнитъ его. Смотри, съ какимъ плачемъ смотритъ Хрістосъ на погибель, имѣющую быть Іерусалиму, котораго никакимъ образомъ не моглъ привести къ покаянію, якоже пишетъ евангелистъ святый: \textit{и видѣвъ градъ, плакася о немъ}\footnote{Лук.~19,~41.}. Чуждое бо Тому дѣло есть казнити, Котораго самое естество есть, чтобы благотворить; чуждое Ему погублять, Которому свойственно есть спасати. \textit{Богъ бо нашъ Богъ еже спасати}, глаголетъ Псаломникъ\footnote{Пс.~67,~21.}. "--- 14)~Хотя и казнитъ Богъ, но не такъ, какъ грѣхи наши заслужили, но съ милостію великою. На сію милость уповая, Давидъ царь, когда предложены были ему отъ Бога три казни, то"=есть, или три года гладу быть на земли его, или три мѣсяца бѣгати ему предъ враги своими, или три дни мору быти въ земли его, и отъ сихъ трехъ едину избралъ бы, сказалъ пророку объявившему: \textit{да впаду въ руцѣ Господни, яко многи суть щедроты Его зѣло}\footnote{2~Цар.~24,~14.}. Сими щедротами Онъ, преклоняемый, хотя и наказуетъ, но купно и милуетъ, "--- опечаляетъ, но и утѣшаетъ. \textit{За грѣхъ мало что опечалихъ его}, глаголетъ чрезъ пророка, \textit{и поразихъ его, и отвратихъ лице Мое отъ него: и опечалися, и пойде дряхлъ въ путехъ своихъ. Пути его видѣхъ, и исцѣлихъ его, и утѣшихъ его, и дахъ ему утѣшеніе истинно}\footnote{Ис.~5,~17 и 18.}. "--- 15)~Наказуетъ не ради погибели, но ради спасенія, чтобы исправившися грѣшникъ спасеніе получилъ. \textit{Судими отъ Господа наказуемся, да не съ міромъ осудимся}\footnote{1~Кор.~11,~32.}. Тако посылаются отъ Бога глады, пожары, болѣзни и прочее, дабы сими мы понудилися искать вѣчныхъ и небесныхъ благъ, видя, что нѣтъ ничего въ мірѣ семъ постояннаго. "--- 16)~Какъ прочитаешь со вниманіемъ Евангеліе, не сыщешь ни единаго, кто бы ко Хрісту съ вѣрою пришелъ и не получилъ желаемаго. Мытарь оправдается паче фарисеа; блудница слышитъ: \textit{отпущаются тебѣ грѣси}; разбойнику рай отверзается; слѣпые прозрѣніе, глухіе слышаніе, и нѣмые глаголаніе, прокаженные очищеніе, больные исцѣленіе, бѣснуемые освобожденіе и прочіе бѣдствующіе получаютъ "--- всякъ себѣ приличное утѣшеніе. Тебѣ ли единому откажетъ отпущеніе грѣховъ, когда съ вѣрою будешь просить? \textit{Іисусъ} бо \textit{Хрістосъ вчера и днесь, Тойже и во вѣки}\footnote{Евр.~13,~8.}; и лица Онъ не пріемлетъ, но всѣхъ равно кающихся милуетъ. Не видишь Его на земли, но видишь и слышишь Евангеліе Его, которое обѣщаетъ кающимся отпущеніе грѣховъ; видишь служителей Его, которые именемъ Его объявляютъ тебѣ тоежде отпущеніе. "--- 17)~Самое имя "--- \textit{Іисусъ} радость и утѣшеніе подаетъ намъ, которое значитъ \textit{Спасителя, спасающаго люди Своя отъ грѣхъ ихъ}\footnote{Матѳ.~1,~21.}. "--- 18)~Во всемъ священномъ Писаніи не сыщешь почти ни единыя главы, ни единой страницы, которая бы милосердія Божія не представляла намъ. Вѣдаетъ бо Богъ слабость естества нашего, и ради того толь часто поминаетъ намъ о милости Своей въ Словѣ Своемъ святомъ, дабы мы не отчаявалися, но паче искали ея покаяніемъ. Примѣчай, о бѣдный и окаянный грѣшниче, какъ милостивъ Богъ къ грѣшникамъ кающимся, и како Духъ Святый чрезъ пророковъ и апостоловъ милость Его объявляетъ. \textit{Яко у Господа милость и многое у Него избавленіе; и Той избавитъ Израиля отъ всѣхъ беззаконій его}\footnote{Пс.~129,~6.}. И паки: \textit{милостивъ Господь и праведенъ, и Богъ нашъ милуетъ}, глаголетъ Псаломникъ\footnote{114,~4.}. И паки: \textit{хвалите Господа вси языцы, похвалите Его вси людіе: яко утвердися милость Его на насъ, и истина Господня пребываетъ во вѣкъ}\footnote{116,~1 и 2.}. И паки: \textit{яко Ты, Господи, благъ, кротокъ и многомилостивъ всѣмъ призывающимъ Тя}\footnote{Пс.~85,~5.}. И паки: \textit{и Ты, Господи Боже мой, щедрый и милостивый, долготерпѣливый и многомилостивый, и истинный}\footnote{ст.~15.}. И паки: \textit{вси путіе Господни милость и истина взыскующимъ завѣта Его и свидѣнія Его}\footnote{24,~10.}. И паки: \textit{щедроты Твоя многи, Господи}\footnote{118,~56.}! И паки: \textit{Азъ есмь, Азъ есмь заглаждаяй беззаконія твоя Мене ради, и грѣхи твоя, и не помяну. Глаголи ты беззаконія твоя прежде, да оправдишися}\footnote{Ис.~43,~25 и 26.}. И паки: \textit{рече Сіонъ: остави мя Господь, и Богъ забы мя. Еда забудетъ жена отроча свое, еже не помиловати исчадія чрева своего? Аще же и забудетъ сихъ жена, но Азъ не забуду тебе, глаголетъ Господь}\footnote{49,~14 и 15.}. И паки: \textit{разумѣхъ, яко милостивъ Ты еси и щедръ, долготерпѣливъ, и многомилостивъ, и кайся о злобахъ человѣческихъ} (Іона пророкъ къ Богу). И рече Господь \textit{ко Іонѣ: ты оскорбился еси о тыквѣ, о нейже не трудился еси, не воскормилъ еси, яже родися обнощь, и обнощь погибе. Азъ же не пощажду ли Ниневіи града великаго, въ немже живутъ множайшіи, неже дванадесять темъ человѣкъ}\footnote{Іон.~4,~2,~10 и 11.}? И паки: \textit{не требуютъ здравіи врача, но болящіи. Шедше же научитеся, что есть: милости хощу, а не жертвы: не пріидохъ бо призвати праведники, но грѣшники на покаяніе}, глаголетъ Хрістосъ\footnote{Матѳ.~9,~12 и 13.}. И паки: \textit{пріиде Сынъ человѣческій взыскати и спасти погибшаго}\footnote{Лук.~19,~10.}. И паки: \textit{кая жена имущи десять драхмъ, аще погубитъ драхму едину, не вжигаетъ ли свѣтильника, и помететъ храмину, и ищетъ прилѣжно, дондеже обрящетъ? И обрѣтши созываетъ другини и сосѣды, глаголющи: радуйтеся со мною: яко обрѣтохъ драхму погибшую. Тако, глаголю вамъ, радость бываетъ предъ ангелы Божіими о единомъ грѣшницѣ кающемся}\footnote{Лук.~15,~8--10.}. И паки: \textit{аще кто согрѣшитъ, Ходатая имамы ко Отцу, Іисуса Хріста праведника}\footnote{1~Іоан.~2,~1.}. И паки: \textit{Иже убо сына Своего не пощадѣ, но за насъ всѣхъ предалъ есть Его: како убо не и съ Нимъ вся намъ дарствуетъ}\footnote{Римл.~8,~32.}? И паки: \textit{Тако возлюби Богъ міръ, яко и Сына Своего единороднаго далъ есть, да всякъ вѣруй въ Онь, не погибнетъ, но имать животъ вѣчный. Не посла бо Богъ Сына Своего въ міръ, да судитъ мірови, но да спасется Имъ міръ}, глаголетъ Хрістосъ\footnote{Іоан.~3,~16 и 17.}. Читай еще Псаломъ 102"~й, который весь проповѣдаетъ великое Божіе милосердіе; притчу о \textit{овцѣ заблуждшей и обрѣтенной}\footnote{Лук.~15,~4--7.}, притчу \textit{о блудномъ сынѣ}\footnote{ст.~11--24.}, притчу о \textit{должникѣ}, которому царь тму талантъ отпустилъ\footnote{Матѳ.~18,~27.}, и прочія святаго Писанія мѣста, и разсуждая прилагай елей милости Божіей, какъ живительный пластырь, къ уязвленному печалію сердцу твоему.

\paragraph*{§\:147.} Когда сія о великомъ Божіемъ милосердіи предлагаются, не подается случай или поводъ къ безстрашію и лѣности, но пресѣкается путь къ отчаянію; не къ излишнему на милость Божію упованію безъ покаянія пребывающимъ, но къ утѣшенію хотящимъ каятися глаголется. Есть Богъ милостивъ, но кающимся; а безстрашныхъ и нераскаянныхъ судъ Божій постигаетъ, и непремѣнно постигнетъ, какъ увидишь.

\subsection[Глава 3-я. Что есть покаяніе?]{глава третія.\\\bfseries Что есть покаяніе?}

\begin{quotation}\textit{Печаль, яже по Бозѣ, покаяніе нераскаянно во спасеніе содѣловаетъ}\footnote{2~Кор.~7,~10.}.\end{quotation}
\begin{quotation}\textit{Беззаконіе мое познахъ, и грѣха моего не покрыхъ; рѣхъ: исповѣмъ на мя беззаконіе мое Господеви: и Ты оставилъ еси нечестіе сердца моего}\footnote{Пс.~31,~5.}.\end{quotation}


\paragraph*{§\:148.} Видимъ въ святомъ Писаніи, что то есть покаяніе. Петръ апостолъ, отвергшися Хріста, \textit{изшедъ вонъ плакася горько}\footnote{Матѳ.~26,~75.}, и тако въ первый ликъ апостольскій принятъ. Блудница умываетъ слезами нозѣ Хрістовы, и власами главы своея отираетъ, и лобызаетъ нозѣ Его, и тако слышитъ отъ Хріста: \textit{отпущаются тебѣ грѣси}\footnote{Лук.~7,~37--48.}. Ниневитяне, облекошася во вретище, еже есть знакъ печали и сѣтованія, и возопиша прилѣжно къ Богу, и возвратися кійждо отъ пути своего лукаваго и отъ неправды сущія въ рукахъ ихъ: \textit{и раскаяся Богъ о злѣ, еже глаголаше сотворити имъ, и не сотвори}\footnote{Іоны 3,~8 и 10.}. Блудный сынъ признаетъ предъ отцемъ своимъ себе недостойнымъ нарещися сыномъ: \textit{Отче! согрѣшихъ на небо и предъ тобою, и уже нѣсмь достоинъ нарещися сынъ твой: сотвори мя яко единаго отъ наемникъ твоихъ}\footnote{Лук.~15,~18 и 21.}; уповательно, что и не безъ слезъ тое смиренное признаніе было: \textit{и облекается въ первую сыновнюю одежду}\footnote{ст.~22.}. Давидъ глаголетъ: \textit{исходища водная}, то"=есть, слезы, \textit{изведостѣ очи мои, понеже не сохранихъ закона Твоего, Господи}\footnote{Пс.~118,~136.}; и паки: \textit{беззаконіе мое познахъ, и грѣха моего не покрыхъ; рѣхъ: исповѣмъ на мя беззаконіе мое Господеви}. И придаетъ: \textit{и Ты оставилъ еси нечестіе сердца моего}\footnote{Пс.~31,~5.}. Мытарь же не дерзаетъ и очесъ возвести на небо, но біетъ въ перси своя, глаголя: \textit{Боже, милостивъ буди мнѣ грѣшнику! И сниде въ домъ оправданъ паче фарисеа}\footnote{Лук.~18,~13 и 14.}. Іеремія пророкъ, въ лицѣ всѣхъ людей своихъ согрѣшившихъ, плачетъ и сѣтуетъ, глаголя: \textit{горе намъ, яко согрѣшихомъ! О семъ смутися сердце наше! О семъ померкнуша очи наша}\footnote{Плач. Іер.~5,~6 и 17.}. Даніилъ глаголетъ: \textit{Тебѣ, Господи, есть правда, намъ же стыдѣніе лица, и царемъ нашимъ, и княземъ нашимъ, и отцемъ нашимъ, иже согрѣшихомъ Тебѣ}\footnote{Дан.~9,~8.}. Ездра стыдится и срамляется воздвигнути лице свое къ Богу. \textit{Господи Боже мой! стыждуся и срамляюся воздвигнути лице мое къ Тебѣ: яко беззаконія наша умножишася паче власъ главъ нашихъ, и прегрѣшенія наша возрастоша даже до небесе}\footnote{1~Ездр.~9,~6.}. И апостолъ глаголетъ грѣшникамъ: \textit{очистите руцѣ, грѣшницы, исправите сердца ваша, двоедушніи, постраждите и слезите и плачитеся; смѣхъ вашъ въ плачь да обратится, и радость въ сѣтованіе; смиритеся предъ Господемъ: и вознесетъ вы}\footnote{Іак.~4,~8--10.}.

Покаяніе убо есть \textit{печаль по Бозѣ}, которая \textit{покаяніе нераскаянно во спасеніе содѣловаетъ}, по словеси апостола. Слезы, плачь и воздыханіе суть знаки печали сердечныя, которою сердце, какъ стрѣлою, уязвляется, и тако слезы испущаетъ. Какъ бо губа, напоенная водою, когда стискивается и сжимается, издаетъ изъ себе воду: такъ сердце, наполненное печалію, когда печалію большею какъ бы стискивается, извергаетъ изъ себе слезы, и съ слезами отъ печали облегчевается. Плачь бо и слезы облегчеваютъ печаль, и, какъ дождемъ воздухъ, такъ слезами сѣтующая и скорбящая душа прохлаждается.

\paragraph*{§\:149.} Печаль сію сердечную содѣловаетъ разсужденіе и размышленіе о Бозѣ, грѣхами прогнѣванномъ. Егда бо грѣшникъ, пришедъ въ себе, разсудитъ, что Бога великаго, преблагаго, милосердаго, святаго, Котораго долженъ боятися, безстрашіемъ и преступленіемъ святаго Его закона прогнѣвалъ; Котораго долженъ любить, раздражилъ; Котораго долженъ почитать, злыми дѣлами безчестилъ: не можетъ не печалитися. Когда сынъ отцу, отъ котораго рожденъ и воспитанъ и всякимъ добромъ снабдѣнъ, вмѣсто любви должныя ненависть, вмѣсто благодаренія неблагодарность, и вмѣсто почтенія безчестіе покажетъ, и о сей своей такъ великой грубости и неблагодарствіи размыслитъ: приходитъ въ сожалѣніе, плачется, и самаго себе окаяваетъ и стыдится, что такъ безумно съ родителемъ своимъ поступилъ. Много паче намъ жалѣть, плакать и стыдиться должно, когда Богу, Который создалъ насъ, тѣло и душу и жизнь подалъ намъ, питаетъ, одѣваетъ, сохраняетъ и всякими благими снабдѣваетъ насъ, неблагодарными являемся; и Котораго ангели со страхомъ и благоговѣніемъ поютъ, почитаютъ и покланяются, мы, \textit{земля и пепелъ}, не почитаемъ; и Его, Иже есть вѣчная любовь и благостыня, не любимъ; и грѣхами нашими прогнѣвляемъ Его, Иже есть вѣчная правда. Когда грѣшникъ сіе разсуждаетъ, весьма уязвляется сердце его, самъ на себе гнѣвается, самъ собою мерзѣетъ. Но къ болѣзни болѣзнь и къ печали страхъ придается, егда разсуждаетъ и о своемъ окаянствѣ, что отъ прогнѣваннаго Бога не инаго чего, какъ суда ожидать должно. Итакъ, то печалію, то страхомъ объятъ, не знаетъ, куды обратиться. Въ такой тѣснотѣ изъ глубины сердца воздыхаетъ, и, не смѣя очесъ на небо возвести, но бія въ перси своя, вопіетъ съ мытаремъ: \textit{Боже, милостивъ буди мнѣ грѣшному}! Утѣшается же неизреченнымъ Божіимъ милосердіемъ, которое всѣмъ кающимся отверзается, помышляя, что какъ прочіимъ грѣшникамъ кающимся милость Свою Богъ являлъ и являетъ, такъ и ему непремѣнно явитъ. Въ сей надеждѣ утвердившися, воставъ, по подобію блуднаго сына, идетъ къ Отцу своему, и глаголетъ: \textit{Отче, согрѣшихъ на небо и предъ Тобою, и уже нѣсмь достоинъ нарещися сынъ Твой; сотвори мя, яко единаго отъ наемникъ Твоихъ}! Такъ смиренное и съ сокрушеніемъ сердца и вѣрою къ Отцу небесному грѣшника пришествіе не безплодно бываетъ, но непремѣнно получаетъ такую милость, какой блудный сынъ сподобился отъ благоутробнаго своего отца.

\paragraph*{§\:150.} Хотящему убо каятися должно учинить сія. 1)~Гласа Божія зовущаго послушать и отъ пути беззаконнаго къ Богу обратиться всѣмъ сердцемъ. 2)~Грѣхи, въ которыхъ находился, возненавидѣть, гнушаться и противу ихъ съ помощію Божіею подвизаться, и отъ прочіихъ берещися. 3)~Что Бога "--- Творца и Отца своего преблагаго прогнѣвалъ, жалѣть до кончины жизни своея. 4)~Самаго себе, что такъ безстыдно поступалъ, нарушая Божій законъ, стыдиться. 5)~Признавать себе недостойна никакого Божія благодѣянія, но паче достойнымъ всякаго наказанія. 6)~Утѣшать себе неизреченнымъ Божіимъ милосердіемъ, которое всѣмъ кающимся обѣщается о Хрістѣ Іисусѣ Господѣ нашемъ. "--- Образъ покаянія изрядно представляется во Псалмахъ, такожде въ Минеяхъ Четіихъ.

\subsection[Глава 4-я. О плодахъ покаянія.]{глава четвертая.\\\bfseries О плодахъ покаянія.}

\begin{quotation}\textit{Сотворите плоды достойны покаянія. И вопрошаху его народы, глаголюще: что убо сотворимъ? Отвѣщавъ же Іоаннъ, глагола имъ: имѣяй двѣ ризы, да подастъ неимущему и имѣяй брашна, такожде да творитъ. Пріидоша же и мытари креститися отъ него, и рѣша къ нему: учителю! что сотворимъ? Онъ же рече къ нимъ: ничтоже болѣе отъ повелѣннаго вамъ творите. Вопрошаху же его и воини, глаголюще: и мы что сотворимъ? И рече къ нимъ: никого же обидите, ни оклеветайте, и довольни будите оброки вашими}\footnote{Лук.~3,~8,~10--14.}.\quad Предтеча къ приходящимъ людемъ.\end{quotation}


\paragraph*{§\:151.} Двоякій плодъ истинному обращенію и покаянію послѣдуетъ. \textit{Первый плодъ} есть отпущеніе грѣховъ, котораго кающійся грѣшникъ отъ преблагаго Бога сподобляется, ради Ходатая всѣхъ Іисуса Хріста, Господа нашего. \textit{Аще кто согрѣшитъ, Ходатая имамы ко Отцу, Іисуса Хріста Праведника}, глаголетъ апостолъ\footnote{1~Іоан.~2,~1.}. А гдѣ отпущеніе грѣховъ, тамо вся благая, смертію Хрістовою пріобрѣтенная, послѣдуютъ; тамо вмѣсто клятвы благословеніе Божіе, вмѣсто гнѣва благодать и милость Божія подается. О, чуднаго и воистину желаемаго премѣненія! Окаянный грѣшникъ въ число праведныхъ пріемлется, изъ сына тьмы сыномъ свѣта, изъ чада діавольскаго чадомъ Божіимъ, изъ наслѣдника вѣчныя смерти и ада наслѣдникомъ вѣчнаго живота, вѣчнаго блаженства и царствія Божія дѣлается. Удивленія воистину достойное дѣло было бы, когда бы монархъ земный согрѣшившаго раба, и по законамъ къ смерти приговореннаго, а потомъ со смиреніемъ просящаго прощенія, не токмо простилъ, но и наслѣдникомъ царствія своего учинилъ. Всѣ вѣки удивлялися бы такъ необыкновенной милости. Несравненно большей милости истинно кающійся грѣшникъ сподобляется отъ Бога, когда не только отпущеніе грѣховъ отъ Него получаетъ, но и участникомъ вѣчнаго Его царствія дѣлается, \textit{наслѣдникъ Богу, снаслѣдникъ же Хрісту бываетъ}\footnote{Римл.~8,~17.}. Умъ человѣческій не можетъ постигнуть сея такъ высокія Божія милости и благодати! Не видимъ мы того нынѣ бренными сими очесами; но вѣра святая паче всякаго видѣнія увѣряетъ насъ, что непремѣнно будетъ такъ, \textit{егда тлѣнное сіе}, какъ учитъ Павелъ святый, \textit{облечется въ нетлѣніе, и смертное сіе облечется въ безсмертіе}\footnote{1~Кор.~15,~54.}. \textit{Возлюбленніи}, глаголетъ Богословъ, \textit{нынѣ чада Божіи есмы и не у явися, что будемъ; вѣмы же, яко, егда явится, подобни Ему будемъ, ибо узримъ Его, якоже есть}\footnote{1~Іоан.~3,~2.}. "--- \textit{Вторый плодъ} истиннаго обращенія и покаянія есть новое сердце и духъ новый, котораго Богъ отъ насъ требуетъ: \textit{сотворите себѣ сердце новое и духъ новъ}\footnote{Іез.~18,~31.}. Которое сердце и духъ Самъ Богъ обѣщается истинно кающимся подать ради святаго имени Своего. \textit{Дамъ вамъ сердце ново, и духъ новъ дамъ вамъ, и отъиму сердце каменное отъ плоти вашея, и дамъ вамъ сердце плотяно: и духъ Мой дамъ вамъ, и сотворю, да въ заповѣдехъ Моихъ ходите, и суды Моя сохраните и сотворите я}\footnote{36,~26 и 27.}. Ибо въ обратившемся всѣмъ сердцемъ и кающемся истинно все иное является "--- иныя мысли, начинанія, намѣренія, иныя тщанія и дѣла, какъ прежде были: 1)~Слово Божіе которое ему прежде какъ мертво было и нечувствительно, и какъ идола глухаго внѣ только ударяло, "--- уже дѣйствительно бываетъ, получаетъ плодъ свой, какъ сѣмя, на землѣ сердца его посѣянное. Тогда онъ имѣетъ \textit{уши слышати}; тогда слышитъ не токмо внѣ, но и внутрь гласъ Божій: \textit{Азъ есмь Господь Богъ твой}, и проч. Особенную бо нѣкую охоту чувствуетъ таковый къ слушанію Божія слова, \textit{къ обученію, къ обличенію, къ исправленію, къ наказанію, еже въ правдѣ, да совершенъ будетъ Божій человѣкъ, на всякое дѣло благое уготованъ}\footnote{2~Тим.~3,~16 и 17.}. Какъ бо тотъ, который отъ болѣзни возставъ, ищетъ отвсюду подкрѣпленія, дабы силы ослабѣвшія собрать и укрѣпить: такъ отъ болѣзни грѣховной свободившійся, души своея силы, которыя грѣхомъ, какъ лютою болѣзнію, разслаблены были, всякимъ образомъ тщится укрѣпить, чтобы дѣла Божія дѣлать и по пути заповѣдей Божіихъ тещи моглъ. И которому прежде обличеніе, наставленіе, наказаніе, какъ раны, несносно было, тому послѣ, какъ елей язвы облегчающій; и какъ пчела отъ различныхъ зелій и цвѣтовъ медъ, такъ онъ отъ различныхъ полезныхъ книгъ духовную себѣ собираетъ пользу. "--- 2)~Страсти Хрістовы весьма дорого и высоко почитаетъ, помышляя, что какъ за всѣхъ, такъ и за него единородный Сынъ Божій такъ горестную страданія чашу испилъ. И того ради часто поминаетъ о нихъ, покланяется имъ, сердцемъ и устами благодаритъ пострадавшему, и къ взаимной любви, сколько возможно, нудитъ себе. "--- 3)~Къ причастію святѣйшей Евхаристіи не такъ, какъ прежде, безстрашно, но со страхомъ и великимъ благоговѣніемъ приступаетъ, разсуждая высочество тайны и свое недостоинство, и за велико и за самый верхъ благополучія почитая со Хрістомъ соединитися и единъ духъ съ Нимъ быти. "--- 4)~Хрістіанское достоинство, и быть истиннымъ хрістіаниномъ такожде высоко почитаетъ, такъ, что всю славу міра сего, какъ гной и сметіе противу того вмѣняетъ, и всѣми силами тщится о томъ, чтобы не лишиться того; о чемъ Бога всеусердно молитъ, чтобы въ числѣ избранныхъ Своихъ имѣлъ его. "--- 5)~Какъ прежде покаянія умъ занятъ въ земномъ, временномъ и суетномъ имѣлъ: такъ по обращеніи во всѣхъ замыслахъ и дѣлахъ къ единому вѣчному и небесному блаженству стремится. \textit{Идѣже бо сокровище его, тамо и сердце его} обращается\footnote{Матѳ.~6,~21.}. "--- 6)~Какъ прежде плоти своей угодія творилъ въ похотяхъ ея, такъ потомъ волѣ Божіей всякимъ образомъ угождать тщится; почему отъ всякаго зла уклоняется, и противу всякаго грѣха подвизается. "--- 7)~Какъ прежде покаянія легко было ему грѣшить, такъ уже покаявшемуся тяжко, такъ что лучше изволяетъ бѣдствовать, всякую напасть терпѣть и умереть, нежели согрѣшить. Едину бо себѣ вмѣняетъ быти бѣду гнѣвъ Божій, который грѣху послѣдуетъ; едину почитаетъ напасть "--- Божіей отпасти милости. "--- 8)~Аще когда въ чемъ отъ немощи поткнется или падетъ, окаеваетъ себе, болѣзнуетъ, сокрушается, сѣтуетъ, какъ бы что великое потерялъ и съ глубочайшимъ смиреніемъ и жалѣніемъ повергаетъ себе предъ Богомъ и проситъ прощенія, взирая на милостиваго Ходатая Іисуса Хріста, и глаголетъ: \textit{отврати, Господи, лице Твое отъ грѣхъ моихъ}\footnote{Пс.~50,~11.}, обрати на лице Хріста Твоего, "--- и тако утѣшается надеждою милости Божіей. "--- 9)~Когда приходятъ ему на умъ прежніе грѣхи, весьма стыдится и срамляется ихъ и жалѣетъ, что ихъ творилъ, и Богу благодаритъ, что его въ такомъ нечестіи не поразилъ праведнымъ Своимъ судомъ, но потерпѣлъ благодатію Своею, ожидая на покаяніе. "--- 10)~Дѣла любве къ ближнему являетъ безъ всякаго лицемѣрія: откуду бываетъ милостивъ, милосердъ, сострадателенъ, терпѣливъ, кротокъ, отпущаетъ согрѣшенія съ радостію, и проч. "--- 11)~Отъ случаевъ, которые ко грѣху его приводили, убѣгаетъ, опасаяся, чтобы въ пагубу не впасть по дѣйствію діавольскому. "--- 12)~Таковый когда видитъ ближняго неисправнымъ, не осуждаетъ его, какъ прочіимъ есть обычай, но соболѣзнуетъ ему усердно, желаетъ и молится Богу, чтобы его въ чувство привелъ, поминая, что и самъ въ такомжде находился окаянствѣ и можетъ такожде согрѣшить. Общее бо человѣческое окаянство "--- падать. "--- 13)~Въ церковь ходитъ не ради церемоніи, но ради того, чтобы обще съ вѣрными, братіею своею, Богу молитвы принести, благодареніе воздать и святое Его имя славословить. "--- 14)~Пищи и питія и прочаго употребляетъ не ради сладострастія, но ради нужды, чтобы моглъ Богу работать и званія своего дѣла проходить. Словомъ сказать: у такого человѣка все иное, какъ внутреннее, такъ и внѣшнее дѣйствіе и тщаніе, какъ прежде было; все же творитъ добровольно, безъ всякаго принужденія. Вѣра бо и благодать Божія, внутрь живущая, поощряетъ его къ тому. Павелъ, по обращеніи къ вѣрѣ Хрістовой, Хріста, Котораго гналъ и спѣшилъ гнать, проповѣдуетъ въ Дамаскѣ такъ, что дивилися вси слышащіи и глаголаху: \textit{не сей ли есть гонивый во Іерусалимѣ нарицающія имя сіе}? и проч.\footnote{Дѣян.~9,~21.}; а потомъ и по всей вселеннѣй проноситъ имя Его; и ради Котораго другихъ вязалъ и на смерть предавалъ, ради Того самъ всеохотно и многократно связанъ быть и умереть желаетъ; а потомъ связуется, біется, ранится, уязвляется и умираетъ. Закхей, мытарь кающійся, полъ имѣнія своего нищимъ дать, и кого обидѣлъ, четверицею возвратить обѣщается; почему и слышитъ отъ Сердцевѣдца Хріста: \textit{днесь спасеніе дому сему бысть}\footnote{Лук.~19,~8 и 9.}. Тако истинное обращеніе и покаяніе премѣняетъ всего человѣка! Отъ сего видно, что гдѣ нѣтъ плодовъ покаянія, тамо нѣтъ истиннаго покаянія, но ложное, не иное что есть, какъ прельщеніе совѣсти; потому ничего и не пользуетъ кающемуся, пока грѣховъ не оставитъ и не начнетъ новаго житія. Покаяніе бо не иное что, какъ воскресеніе духовное. Ибо пока человѣкъ въ грѣхахъ пребываетъ, хотя тѣломъ и живетъ, но духомъ мертвъ есть: не имѣетъ бо въ души своей Бога, Иже есть животъ и живота источникъ. Понеже что тѣлу нашему душа, тое души нашей Богъ. Тѣло дотоль живетъ, доколь душа въ немъ находится: душа дотоль живетъ, доколь Богъ благодатію Своею въ ней обитаетъ. Тѣло умретъ, когда душа изыдетъ изъ него: душа умираетъ, когда Богъ ее оставляетъ. Оставляетъ же душу Богъ не ради инаго чего, какъ ради грѣха. Богъ бо и грѣхъ купно пребывати не могутъ: \textit{грѣси ваши разлучаютъ между вами и между Богомъ}, глаголетъ Исаія пророкъ\footnote{Ис.~59,~2.}. Душа убо, лишившися благодатнаго Божія присутствія, яко живота своего, есть мертва не иначе, какъ тѣло, лишившееся души, остается мертво. Такую душу возбуждаетъ Богъ: \textit{востани спяй, и воскресни отъ мертвыхъ}\footnote{Еф.~5,~14.}. «Спящаго и мертваго находящагося во грѣхахъ глаголетъ, какъ бесѣдуетъ Златоустъ: ибо злосмрадіе дыхаетъ якоже мертвый, и недѣйственъ есть якоже спяй, и ничтоже зритъ якоже оный, но сонія и мечтанія привидитъ»\footnote{\textit{Бес.~18"~я на посл. къ Еф}.}. Какъ бо тѣло мертвое не движется, не чувствуетъ и не дѣлаетъ ничего: такъ душа, лишившаяся благодати Божіей, не движется, не чувствуетъ и не дѣлаетъ духовно. Но какъ тѣло оживетъ, начинаетъ дѣйствовать все: такъ грѣшникъ, когда силою и благодатію Божіею оживетъ, начинаетъ духовное движеніе имѣть, духовно чувствовать, духовно дѣлать. Божіею бо благодатію душа оживляема производитъ духовныя дѣйствія: видитъ Бога вѣрою, осязаетъ вѣрою, слышитъ вѣрою Бога глаголющаго, вкушаетъ и обоняетъ любовію, и дѣла Ему угодныя тщится творить. Тако подобаетъ кающемуся чрезъ покаяніе новое житіе начати, и какъ бы вновь родитися, питатися, расти и въ мужа совершенна приходити! Согласно сему святый Златоустъ научаетъ, глаголя: «Самъ апостолъ Павелъ, предлежащу будущему, иного отъ насъ требуетъ воскресенія, то"=есть, новаго жительства, еже въ настоящемъ житіи отъ преложенія нравовъ бываетъ. Егда бо блудникъ будетъ цѣломудръ, и лихоимецъ милостивъ, и жестокій кротокъ: и здѣ воскресеніе бысть, которое есть начало онаго. И како есть воскресеніе? Понеже, грѣху умершу, правда воскресла, и, древнему житію загладившуся, новое сіе и ангельское жительствуетъ»\footnote{\textit{Бес.~10"~я на посл. къ Римл}.}. Тако, кто здѣ воскреснетъ, тотъ непремѣнно, аще и умретъ, живъ будетъ. Воскреснетъ бо во общее воскресеніе въ животъ вѣчный, по неложному обѣщанію Господа нашего: \textit{изыдутъ сотворшіи благая въ воскрешеніе живота}\footnote{Іоан.~5,~29.}. \textit{Блаженъ и святъ, иже имать часть въ воскресеніи первомъ}\footnote{Апок.~20,~6.}. Сіи знаки и плоды истинно обратившагося и кающагося примѣчаются какъ изъ святаго Писанія, такъ и изъ церковной Исторіи, которые и нынѣ примѣчаются въ обратившихся и кающихся чистосердечно.

\subsection[Глава 5-я. Прещенія некающимся.]{глава пятая.\\\bfseries Прещенія некающимся.}

\begin{quotation}\textit{Азъ предамъ васъ подъ мечь, вси закланіемъ падете: яко звахъ васъ, и не послушасте; глаголахъ, и преслушасте, и сотвористе лукавое предо Мною; и еже не хотѣхъ, избирасте. Сего ради тако глаголетъ Господь: се работающіи Ми ясти будутъ, вы же взалчете; се работающіи Ми возрадуются, вы же посрамитеся; се работающіи Ми возвеселятся въ веселіи сердца, вы же возопіете въ болѣзни сердца вашего, и отъ сокрушенія духа восплачетеся}\footnote{Ис.~65,~12--14.}.\end{quotation}
\begin{quotation}\textit{И пріиду къ вамъ съ судомъ, и буду Свидѣтель скоръ на чародѣи, и на прелюбодѣйцы, и на кленущіяся именемъ Моимъ во лжу; и на лишающія мзды наемника, и на насильствующія вдовицъ и пхающія сирыя, и на уклоняющія судъ пришельца, и на небоящіяся Мене, глаголетъ Господь Вседержитель}\footnote{Мал.~3,~5.}.\end{quotation}
\begin{quotation}\textit{Пріидоша же нѣцыи въ то время, повѣдающе Ему о Галилеехъ, ихже кровь смѣси Пилатъ съ жертвами ихъ. И отвѣщавъ Іисусъ рече имъ: мните ли, яко Галилеане сіи грѣшнѣйши паче всѣхъ Галилеанъ бяху, яко тако пострадаша? Ни, глаголю вамъ: но аще не покаетеся, вси такожде погибнете}\footnote{Лук.~13,~1--3.}.\end{quotation}
\begin{quotation}\textit{Аще не обратитеся, оружіе Свое очиститъ, лукъ Свой напряже, и уготова его, и въ немъ уготова сосуды смертныя, стрѣлы Своя сгараемымъ содѣла}\footnote{Пс.~7,~13 и 14.}.\end{quotation}
\begin{quotation}\textit{Или о богатствѣ благости Его и кротости и долготерпѣніи нерадиши, невѣдый, яко благость Божія на покаяніе тя ведетъ! По жестокости же твоей и непокаянному сердцу, собираеши себѣ гнѣвъ въ день гнѣва и откровенія праведнаго суда Божія, Иже воздастъ коемуждо по дѣломъ его}\footnote{Рим.~2,~4--6.}.\end{quotation}


\paragraph*{§\:152.} Самое прещеніе Божіе не ино что показуетъ намъ, какъ Его къ намъ милосердіе. Какъ бо будущая благая ради того въ Словѣ Своемъ объявилъ, чтобы мы, взирая на нихъ вѣрою, стремилися къ нимъ и искали ихъ: такъ и имѣющая быть злая, адъ, геенну, огнь вѣчный, червь неусыпающій, тму кромѣшнюю, скрежетъ зубный и прочая, которыми грѣшники мучими будутъ, въ томжде Словѣ ради того представляетъ, чтобы, послушавше о нихъ, тщались убѣжать ихъ. На сей конецъ и прежде бывшія казни Божіи, какъ"=то: всемірный потопъ, Содомъ и Гомморръ сожженный и прочія въ святомъ Писаніи написаны, чтобы мы, взирая на нихъ, береглись того дѣлать, что оные грѣшники дѣлали, и тако бы тѣмъ страшнымъ наказаніямъ, какія они пострадали, или подобнымъ, не подпали.

\paragraph*{§\:153.} Весьма опасно Божія прещенія презирать. Ибо кто презираетъ ихъ, тотъ самымъ дѣломъ дознаетъ ихъ на себѣ. Божіи бо прещенія не суть словеса праздна, но силу и дѣйствіе свое имѣютъ, и непремѣнно самымъ дѣломъ исполнятся, когда мы себе не исправимъ. Презрѣли прародители наши Божіе прещеніе сіе: \textit{въ оньже аще день снѣсте отъ него} "--- заповѣднаго древа, \textit{смертію умрете}\footnote{Быт.~2,~17.}, и вкусили отъ того: и смертію умерли. Презрѣли при Ноѣ бывшіе беззаконники: и потопомъ погибли. Презрѣли Содомляне: и не избѣжали страшной огненной казни. \textit{Богъ бо поругаемъ не бываетъ}\footnote{Гал.~6,~7.}. Что не казнитъ грѣшниковъ, но долготерпитъ имъ, то благости Его дѣло есть, которая ихъ на покаяніе ведетъ, якоже апостолъ глаголетъ: \textit{или о богатствѣ благости Его и кротости и долготерпѣніи нерадиши, невѣдый, яко благость Божія на покаяніе тя ведетъ}? Но когда въ нераскаяніи и ожесточеніи пребудутъ, тогда правда Божія вступитъ въ свое дѣло; и столько почувствуютъ на себѣ гнѣва Божія, сколько о богатствѣ благости Его и кротости и долготерпѣніи нерадятъ, якоже тойжде апостолъ глаголетъ: \textit{по жестокости же твоей и непокаянному сердцу, собираеши себѣ гнѣвъ въ день гнѣва и откровенія праведнаго суда Божія, Иже воздастъ коемуждо по дѣломъ его}. Напротивъ того, кто убоится прещеній Божіихъ и покается, тотъ избѣжитъ ихъ. Ниневитяне убоялись возвѣщенныя имъ чрезъ пророка казни, и обратившеся отъ пути лукаваго, спаслися. Но, что прежде, тоежъ и нынѣ бываетъ; какъ прежде кающіися милость Божію получали, и некающіися судъ Божій на себѣ дознавали: такъ и нынѣ кающимся таяжде Божія милость является, и нераскаяннымъ свой достается жребій. Единъ бо и Тойжде есть Богъ и нынѣ, Который и прежде былъ, и во вѣки будетъ. Онъ и нынѣ всѣмъ съ высоты святыя Своея гремитъ: \textit{аще не покаетеся, вси такожде погибнете}. Страшный сей громъ, какъ прежде насъ бывшихъ, такъ и въ наша, и послѣ насъ будущихъ ушеса ударяетъ. Блажени суть, которые не токмо въ ушесахъ, но и въ сердцахъ ударъ его чувствуютъ!

\paragraph*{§\:154.} Которые слѣпому нѣкоему случаю какъ все, такъ и казни бываемыя приписуютъ, якоже сердечные ихъ совѣты въ книзѣ Премудрости описаны, и потому не вѣрятъ и будущимъ, имѣющимъ быть по смерти, и, симъ безбожнымъ мнѣніемъ помрачившеся, устремляются на всякія беззаконія и неправды, къ которымъ похоть и злоба влечетъ ихъ: непремѣнно дознаютъ на себѣ тое, что случаю безумно и нечестиво приписуютъ. Тогда они признаютъ, что есть Богъ, Который какъ за благочестіе награждаетъ, такъ и за нечестіе казнитъ, и что все не по случаю, но по Его премудрому бываетъ Промыслу, когда прочіихъ, яко боящихся и почитающихъ Бога, въ вѣчной Его милости увидятъ, а на себѣ вѣчный гнѣвъ Его почувствуютъ. Тогда будутъ каяться истинно, но поздно, и въ тѣснотѣ духа воздыхать, но безполезно. Тогда самымъ дѣломъ узнаютъ, что есть геенна, вѣчный огонь, когда будутъ вкушать горести его. Тогда будутъ признавать свое заблужденіе. \textit{Убо заблудихомъ отъ пути истиннаго, и правды свѣтъ не облиста намъ, и солнце не возсія намъ}. \textit{Беззаконныхъ исполнихомся стезь и погибели, и ходихомъ въ пустыни непроходимыя; пути же Господня не увѣдѣхомъ. Что пользова намъ гордыня? и богатство съ величаніемъ что воздаде намъ? Преидоша вся она, яко сѣнь и яко вѣсть протекающая; яко корабль преходяй волнующуюся воду, егоже проходу нѣсть стопы обрѣсти, ниже стези шествія его въ волнахъ}\footnote{Прем.~5,~6--10.} и проч.


\section[Статья 2-я. О четырехъ послѣднихъ.]{статья вторая.\\\bfseries О четырехъ послѣднихъ.}
\subsection[Глава 1-я. О смерти.]{глава первая.\\\bfseries О смерти.}

\begin{quotation}\textit{Лежитъ человѣкомъ единою умрети}\footnote{Евр.~9,~27.}.\end{quotation}
\begin{quotation}\textit{Слышахъ гласъ съ небеси глаголющъ ми: напиши: блажени мертвіи, умирающіи о Господѣ отнынѣ. Ей, глаголетъ Духъ, да почіютъ отъ трудовъ своихъ}\footnote{Апок.~14,~13.}.\end{quotation}


\paragraph*{§\:155.} Житіе наше подобно пути, которымъ отъ самаго рожденія до смерти непрестанно идемъ. Бдимъ ли, или спимъ, дѣлаемъ, или почиваемъ: безпрерывно путь сей умаляется, и всякъ къ концу своему приближается. Когда раждаемся, на путь сей вступаемъ; а когда умираемъ, путь сей оканчиваемъ. И чимъ болѣе живемъ, тѣмъ далѣе отъ термина рожденія отходимъ, и болѣе къ термину скончанія приближаемся. И такъ житіе наше на земли не иное что, какъ непрестанное и безпрерывное къ смерти приближеніе.

\paragraph*{§\:156.} Путь сей иному должайшій, иному кратчайшій опредѣлилъ Богъ. Иный бо въ младенчествѣ, иный въ отрочествѣ, иный въ юношествѣ, иный въ мужескомъ возрастѣ, а иный въ старости оканчиваетъ его.

\paragraph*{§\:157.} Какъ извѣстно, что путь сей начали мы, такъ извѣстно, что его и окончаемъ. Но когда окончаемъ его, тое промыслъ Божій скрылъ отъ насъ, дабы всегда конца того ожидали мы, и такими бы всегда тщались быть, каковыми хощемъ быть при скончаніи. «Того ради, глаголетъ Златоустъ святый, Богъ постановилъ намъ неизвѣстный день смерти нашея, дабы мы ради неизвѣстнаго конца житія нашего всегда себе содержали въ добродѣтели»\footnote{Бес.~4"~я на посл. къ Евр.}.

\paragraph*{§\:158.} Какъ время, такъ и образъ скончанія нашего неизвѣстенъ намъ. Единымъ образомъ вси раждаемся, а многоразличными умираемъ, и столько почти смертей, сколько людей. Иный въ водѣ погрязаетъ и кончаетъ житіе свое, иный землею пожирается, иный удавленіемъ кончается, иный громомъ пораженный оставляетъ міръ сей, иный на древѣ повѣшенный преходитъ на оный свѣтъ, инаго разбойническія руки посѣкаютъ, иный почивать возлегши не востаетъ, иный ходя, иный сидя, иный стоя, падаетъ бездыханенъ; многіе виномъ и сномъ отягощени не пробуждаются и спать будутъ до общаго востанія, иный въ объятіяхъ блудницъ, иный въ хищеніи, разбоѣ, убійствѣ, иные въ прочіихъ беззаконіяхъ праведнымъ Божіимъ судомъ поражаются. Но что прочіимъ случается, или случилося, тое тебѣ и всякому случиться можетъ. Они, коихъ такая постигла смерть, не надѣялись себѣ того, однакожъ дознали на себѣ тое: такъ и тебѣ, хотя и не надѣешься, можетъ подобное приключиться. Того ради отъ чужаго бѣдствія учися осторожнымъ быть, и всегда къ часу оному покаяніемъ приготовляй себе, дабы, какая ни застанетъ тебе кончина, неготоваго не восхитила.

\paragraph*{§\:159.} При смерти всякъ познаетъ совершенно, какое мнѣніе должно имѣть о мірскихъ вещахъ здоровому. Тогда всякъ праведно о богатствѣ, чести, славѣ и сладострастіи разсуждаетъ. Праведно, говорю, разсуждаетъ, что ими, какъ во снѣ, любовался, и что, какъ соніе востающаго, образъ ихъ прешелъ. Ибо какъ нагъ въ міръ сей вошелъ, такъ нагъ и исходитъ; не выноситъ съ собою ничего; все мірское міру оставляетъ. Тогда подлинно съ Соломономъ признаетъ, что \textit{суета суетствій и всяческая суета}\footnote{Еккл.~1,~2.}.

\paragraph*{§\:160.} Хотя мы и часто видимъ на другихъ позднее сіе раскаяніе, однакожъ такъ міръ прелестный ослѣпляетъ насъ, что дотоль сего не чувствуемъ, доколь сами на себѣ дѣломъ самымъ не узнаемъ. Иный мыслитъ старыя житницы разорить и создать новыя, чтобы собрать жита своя, и душу свою чрезъ многія лѣта весело учреждать; но страшнаго Божія гласа не слушаетъ: \textit{безумне! въ сію нощь душу твою истяжутъ отъ тебе: а яже уготовалъ еси, кому будутъ}\footnote{Лук.~12,~20.}? Иный земли, вотчины, палаты расширяетъ, но не видитъ, что единъ гробъ въ три аршина въ земли готовится ему, "--- но и того, можетъ быть, не сподобится; ибо или утроба звѣриная, или пучина морская, или иное что гробомъ можетъ быть. Иный старается, какъ бы на высоту чести и славы подняться; но не примѣчаетъ, что равно славные, сановитые и подлые землѣ предаются. Иный утучняетъ тѣло свое, но не усматриваетъ, что большую пищу заготовляетъ червямъ. Но въ такихъ замыслахъ всякаго постигаетъ конецъ, и такъ \textit{погибаютъ вся помышленія его}\footnote{Пс.~145,~4.}. Познаетъ себе въ такихъ же обстоятельствахъ, въ каковыхъ другихъ прежде видѣлъ. Тогда и самъ все тое познаетъ, какъ суету, или какъ препятствіе къ истинному блаженству; тогда кается, жалѣетъ, что въ такихъ суетахъ умъ его занятъ былъ; тогда и самъ праведно, но поздно, о всемъ томъ разсуждаетъ.

\paragraph*{§\:161.} Когда раждаемся, не чувствуемъ, какъ раждаемся, хотя и знаемъ, что единъ образъ рожденія всѣмъ; и родившеся \textit{первый гласъ испущаемъ плача}\footnote{Прем.~7,~3.}. Но не тако умираемъ: тогда чувствуемъ, коль горька есть смерть; тогда познаемъ, коль тяжкое разлученіе духа отъ плоти, которыхъ соединеніе нечувствительно намъ было. Горесть сію смертную примѣчаемъ на тѣхъ, которые при глазахъ нашихъ отходятъ отъ насъ. Слышимъ, коль тяжко воздыхаютъ и стонутъ; примѣчаемъ, коль горестно болѣзнуютъ, какъ трудно разрывается союзъ души и тѣла, какъ великій подвигъ востаетъ въ совѣсти противу отчаянія, какъ нестерпимый страхъ осужденія, ужасъ геенны объемлетъ, и проч. Тогда разумъ, совѣтъ и мудрость оскудѣваетъ; *помощь человѣческая ничего не можетъ*; единъ человѣкъ остается, хотя и многими другами и сродниками обстоимъ; единъ въ жестокой той брани подвизается, борется противу враговъ, аще сильная помощь милостиваго Іисуса не приспѣетъ. Чего ради предстоящимъ въ такомъ случаѣ должно усердную пролить молитву, чтобы помощь Его пришла и избавила отъ скорби, одержащія брата.

\paragraph*{§\:162.} Какъ двоякій родъ есть по пути міра сего идущихъ, то"=есть, благочестивыхъ и нечестивыхъ, такъ двоякая и различная есть кончина, то"=есть, блаженная и неблагополучная. Блаженная есть кончина благочестивыхъ, когда въ благочестіи о Господѣ умираютъ, какъ небесный гласъ объявляетъ: \textit{блаженны мертвіи умирающіи о Господѣ}! Неблагополучна кончина есть нечестивыхъ, которые въ нечестіи безъ истиннаго покаяніи и живой вѣры оканчиваютъ житіе свое. Ибо какъ онымъ смерть "--- упокоеніе отъ трудовъ и прешествіе отъ многоболѣзненной къ вѣчной и радостной жизни, такъ симъ по временной смерти слѣдуетъ вѣчная смерть. Обоихъ воспріемлетъ вѣчность, которая начало имѣетъ, *но конца не имѣетъ*; но не равная вѣчность: тѣхъ блаженная, вся благая содержащая; сихъ злополучная, плачевная, мучительнѣйшая и вся злая заключающая въ себѣ.

\paragraph*{§\:163.} Понеже страхъ смертный всякаго, и благочестиваго смущаетъ, и хотя духъ въ такомъ случаѣ ободряетъ, однакожъ плоть немощная трепещетъ: того ради нѣкая утѣшенія противу страха того благочестивымъ и истинно кающимся здѣ предлагаются. 1)~Утѣшаетъ насъ неповинная Хріста Сына Божія смерть, Который \textit{смертію Своею упразднилъ державу имущаго смерти, сирѣчь, діавола}\footnote{Евр.~2,~14.}, и избавилъ вѣрующихъ во имя Его отъ ада. Умеръ Безсмертный, дабы мы смертные не убоялись смерти, "--- о чемъ изрядно святый Златоустъ глаголетъ тако: «Якоже врачъ, не имѣя нужды вкусить брашна, недужному уготованнаго, промышляя о немъ, первѣе самъ вкушаетъ, дабы небоязненно недужный снѣдь вкушалъ: тако и Хрістосъ, понеже вси человѣцы боялись смерти, увѣщавая ихъ не боятися смерти, Самъ вкусилъ смерть, не имѣя нужды умрети»\footnote{Бес.~4"~я на послан. къ Евр.}. Тако ободряясь благочестивые, прежде насъ бывшіе, небоязненно срѣтали приходящую смерть свою, и симъ упованіемъ, яко щитомъ, защищали себе противу страха ея, и горесть ея сладкимъ дражайшія смерти Хрістовой воспоминаніемъ и размышленіемъ растворяли, и тако какъ блаженнымъ упокоилися сномъ. "--- 2)~Утѣшеніе подаетъ намъ востаніе тѣлесъ нашихъ изъ мертвыхъ. Хотя и разрѣшается союзъ души съ тѣломъ, и тѣло мертвое предается землѣ и истлѣваетъ, но паки Божіею силою, которою вся изъ ничего быша, оживотворится, соединившися съ душею, \textit{егда мертвіи услышатъ гласъ Сына Божія, и услышавше оживутъ}\footnote{Іоан.~5,~25.}. Утверждаетъ насъ въ томъ вѣра наша обѣщаніями Божіими, яко котвою укрѣпляема. \textit{Вѣруяй въ Мя}, глаголетъ Хрістосъ, \textit{аще и умретъ, оживетъ}\footnote{11,~25.}. \textit{Нынѣ Хрістосъ}, говоритъ апостолъ, \textit{воста отъ мертвыхъ, начатокъ умершимъ бысть. Понеже бо человѣкомъ смерть бысть, и Человѣкомъ воскресеніе мертвыхъ}\footnote{1~Кор.~15,~20 и 21.}. \textit{Воскреснутъ мертвіи и востанутъ, иже во гробѣхъ}, увѣряетъ насъ Самъ Богъ устами пророка\footnote{Ис.~26,~19.}. \textit{Се, глаголетъ Адонаи Господь костемъ человѣческимъ: се Азъ введу въ васъ духъ животенъ, и возведу на васъ плоть, и простру по вамъ кожу, и дамъ духъ Мой въ васъ, и оживете, и увѣсте, яко Азъ есмь Господь}\footnote{Іез.~37,~5 и 6.}. И далѣе тамже: \textit{се Азъ отверзу гробы ваша, и изведу васъ отъ гробъ вашихъ, людіе Мои, и дамъ духъ Мой въ васъ, и живи будете, и увѣсте, яко Азъ Господь: глаголахъ и сотворю, глаголетъ Адонаи Господь}\footnote{ст.~12--14.}. И въ книгѣ пророка Даніила глаголетъ: \textit{мнози} (вси) \textit{отъ спящихъ въ земнѣй персти востанутъ, "--- сіи въ жизнь вѣчную, а оніи во укоризну и въ стыдѣніе вѣчное}\footnote{Дан.~12,~2.}. \textit{Грядетъ часъ}, говоритъ Спаситель, \textit{въ оньже вси сущіи во гробѣхъ услышатъ гласъ Сына Божія, и изыдутъ сотворшіи благая въ воскрешеніе живота, а сотворшіи злая въ воскрешеніе суда}\footnote{Іоан.~5,~28 и 29.}. "--- Утверждаетъ всемогущество Божіе: яко аще изъ ничего вся сотворилъ, кольми паче \textit{тлѣнное облечетъ въ нетлѣніе, и мертвенное облечетъ въ безсмертіе}\footnote{1~Кор.~15,~54.}. Утверждаетъ востаніе мертвыхъ бывшихъ и воставшихъ въ житіи семъ отъ мертвыхъ, какъ"=то: \textit{дщери Іаировой}\footnote{Матѳ.~9,~25; Марк.~5,~42.}, \textit{сына вдовицы во градѣ Наинѣ}\footnote{Лук.~7,~14 и 15.}, \textit{Лазаря}\footnote{Іоан.~11,~43 и 44.}, \textit{Тавиѳы}\footnote{Дѣян.~9,~40--42.}. Услышали они гласъ Сына Божія, и ожили. Услышатъ и въ послѣдній день мертвіи Тогожде Сына Божія гласъ, и услышавше оживутъ. "--- Показуетъ тое и самое естество. Зерно, посѣянное въ землю и умершее, востаетъ и многъ плодъ приноситъ\footnote{Іоан.~12,~24; 1~Кор.~15,~36.}. Показуетъ прозябеніе во время весны всѣхъ зелій, которыя изъ земли тогда, какъ умершіе изъ гробовъ, востаютъ и въ новый прекрасный видъ одѣваются, и различный подаютъ плодъ. "--- 3)~Умирающимъ о Господѣ присутствуютъ ангели святіи и преносятъ души ихъ въ лоно Авраамле, якоже Самъ Хрістосъ научаетъ о семъ\footnote{Лук.~16,~22.}. "--- 4)~Умершіе о Господѣ начинаютъ новую лучшую жизнь и блаженную вѣчность, какъ сказано. Здѣ живучи чаяніемъ по вся дни смерти не иное что дѣлаемъ, какъ по вся дни умираемъ, и день отъ дня къ смерти приближаемся. Но по смерти того не будетъ: едино непрестанное въ блаженной вѣчности пребываніе будетъ. "--- 5)~Сими грѣшными и тлѣнными глазами не можемъ видѣть славы Божія и Царя славы Іисуса Хріста Сына Божія, но \textit{тогда узримъ лицемъ къ лицу}\footnote{1~Кор.~13,~12.} \textit{и узримъ Его якоже есть}\footnote{1~Іоан.~3,~2.}. "--- 6)~По смерти умершіи о Господѣ пріидутъ въ любезное пребываніе со святыми ангелами, патріархами, пророками, апостолами, мучениками и всѣми святыми, которые ждутъ нынѣ вѣрныхъ, дондеже изведетъ ихъ Господь изъ темницы смертныя плоти. "--- 7)~Настоящее житіе многими бѣдами и болѣзньми исполнено есть. Нѣсть ни единаго, кто бы въ постоянномъ счастіи прожилъ до конца. Непрестанную видимъ перемѣну: изъ богатаго дѣлается нищій, изъ славнаго безчестный, изъ здороваго немощный; свободнаго темница воспріемлетъ, страхъ отъ непріятелей, страхъ отъ злыхъ людей, страхъ отъ огня, страхъ отъ плоти, прелестнаго міра, страхъ отъ супостата діавола, который, какъ \textit{левъ, ходитъ, искій кого поглотити}. Съ плачемъ раждаемся, въ бѣдахъ, скорбехъ, немощахъ, страсѣ живемъ; въ страсѣ и тѣснотѣ кончаемъ житіе. Такъ бѣдное и плачевное въ мірѣ семъ житіе наше! Но когда о Господѣ умираемъ, всѣхъ сихъ бѣдъ свобождаемся. "--- 8)~Житіе сіе исполнено есть грѣхами. Ибо всегда то претыкаемся, то падаемъ; то словомъ, то дѣломъ, то умомъ являемся неисправными и предъ судомъ Божіимъ виноватыми. Всякъ сіе дознаетъ на себѣ, кто примѣчаетъ и разсмотряетъ совѣсть свою. Праведникъ седмижды на день падаетъ и востаетъ. Младенцы самые не безъ скверны предъ очесами Божіими. \textit{Кто чистъ будетъ отъ скверны? никтоже, аще и единъ день житіе его на земли}, глаголетъ Іовъ\footnote{Іов.~14,~4 и 5.}. Сосудъ Божій избранный, Павелъ святый, жалуется о семъ окаянствѣ и воздыхаетъ: \textit{вижду инъ законъ во удѣхъ моихъ, противу воюющъ закону ума моего, и плѣняющъ мя закономъ грѣховнымъ сущимъ во удѣхъ моихъ. Окаяненъ азъ человѣкъ, кто мя избавитъ отъ тѣла смерти сея}\footnote{Римл.~7,~23 и 24.}? Но кто блаженною упокоевается кончиною, свобождается отъ окаянства сего. "--- 9)~Сколько соблазновъ имѣется въ прелестномъ мірѣ семъ, которыхъ то видѣть, то слышать понуждаемся! Сколько ересей, суевѣрій, различныхъ и пагубныхъ ученій возникаетъ, которыя вси тщатся насъ отвести отъ Хріста! И чѣмъ болѣе вѣкъ сей приближается къ концу, тѣмъ болѣе ихъ умножается, \textit{яко прельстити, аще возможно, и избранныя}\footnote{Матѳ.~24,~24.}. О коль блаженъ есть, который всѣхъ сихъ вражіихъ антихрістовыхъ стрѣлъ блаженною уклоняется кончиною! Сего ради благій Промыслитель Господь \textit{восхищаетъ} избранныхъ Своихъ \textit{отъ среды лукавствія, да не злоба измѣнитъ разумы ихъ, ни лесть прельститъ души ихъ}\footnote{Прем.~4,~11.}. "--- 10)~Житіе на земли не ино что, какъ путь, якоже сказано, странствованіе, брань непрестанная. Кто же убо пожелалъ бы всегда на пути быть, странствовать, всегда на брани быть, со врагами сражаться, уязвляться, въ безпокойствіи и страсѣ быть? Блаженная смерть окончеваетъ вся сія, и приводитъ къ желанному отечеству. Почто убо не хощемъ окончать пути странствованія и брани сея? "--- 11)~Всѣмъ людямъ общее есть умирать. Что праотцу нашему отъ Бога сказано: \textit{земля еси, и въ землю пойдеши}\footnote{Быт.~3,~19.}, "--- тое и всякому глаголется: \textit{земля еси, и въ землю пойдеши}. Предѣлъ сей никому неминуемый: сегодня тотъ, утро другій, а далѣе третій, и такъ единъ за другимъ слѣдуетъ и сокрывается въ нѣдрахъ земли. Бойся, страшися; не миновать того, что всѣмъ общее и необходимо. "--- 12)~Коль бѣдные и окаяннѣйшіе паче всѣхъ человѣкъ хрістіане были бы хрістіане, въ непрестанныхъ бѣдахъ, скорбѣхъ, подвизехъ находящіися, ежели бы имъ всегда въ мірѣ семъ повелѣно было жить! \textit{Аще въ животѣ семъ}, глаголетъ Павелъ святый, \textit{уповающе есмы во Хріста точію, окаяннѣйши всѣхъ человѣкъ есмы}\footnote{1~Кор.~15,~10.}. Хрістіанину бо всегда въ мірѣ семъ жить не ино что есть, какъ всегда въ бѣдахъ и скорбѣхъ находиться. Истинный бо хрістіанинъ безъ напастей быть не можетъ. "--- Но скажетъ кто: не боюся смерти, да смущаютъ мене грѣхи? "--- \textit{Отвѣтъ}. 1)~Когда грѣхи сотворилъ ты и отъ нихъ благодатію Божіею отсталъ, и покаянія плоды показуешь: нечего смущаться, но должно себе надеждою Божія милосердія утвердить. Ибо истинно кающемуся Богъ грѣхи оставляетъ и въ милость Свою пріемлетъ его. "--- 2)~Когда грѣхи творишь, должно отстать отъ нихъ и покаяться, и такожде надеждою Божіей милости себе утвердить, и тако излишній страхъ отъидетъ. "--- 3)~Когда не хочешь отъ грѣховъ отстать, подлинно должно трепетать смерти. Ибо по временной сей смерти слѣдуетъ вторая, горчайшая, вѣчная смерть, которою безъ конца грѣшники будутъ умирать. И хотя страхъ сей и благочестивыхъ смущаетъ, однакожъ чувствуютъ противу того утѣшающую Божію благодать, которая немощную подкрѣпляетъ плоть. Но грѣшникамъ некающимся поистиннѣ должно трепетать. Некающимся бо заключаетъ святое Божіе слово дверь небеснаго царствія. \textit{Не льстите себе: ни блудницы, ни идолослужители, ни прелюбодѣи, ни сквернители, ни малакіи, ни мужеложницы, ни лихоимцы, ни татіе, ни піяницы, ни досадители, ни хищницы, царствія Божія не наслѣдятъ}, глаголетъ апостолъ\footnote{1~Кор.~6,~9 и 10.}. Но вмѣсто того опредѣляется имъ озеро огненное. \textit{Страшливымъ и невѣрнымъ, и сквернымъ и убійцамъ, и блудъ творящимъ, и чары творящимъ, идоложерцемъ и всѣмъ лживымъ, часть имъ въ озерѣ горящемъ огнемъ и жупеломъ, еже есть смерть вторая}\footnote{Апок.~21,~8.}. "--- 4)~Когда не хочешь и грѣховъ оставить, и жить долѣе хочешь въ мірѣ семъ: не инаго чего желаешь, какъ болѣе грѣшить, и тако болѣе гнѣва Божія собирать; чимъ бо болѣе кто грѣшитъ, тѣмъ болѣе гнѣва Божія собираетъ. \textit{По жестокости твоей}, глаголетъ апостолъ, \textit{и непокаянному сердцу, собираеши себѣ гнѣвъ въ день гнѣва и откровенія праведнаго суда Божія}\footnote{Римл.~2,~5.}.

\paragraph*{§\:164.} Тѣмъ, которые по смерти умершихъ сѣтуютъ, предлагается такожде краткое утѣшеніе. 1)~Апостолъ святый о всѣхъ умершихъ хрістіанехъ всѣмъ подаетъ утѣшеніе. \textit{Не хощу васъ, братіе, не вѣдѣти о умершихъ, да не скорбите, якоже и прочіи не имущіи упованія. Аще бо вѣруемъ, яко Іисусъ умре и воскресе: тако и Богъ умершія о Іисусѣ приведетъ съ Нимъ}\footnote{1~Сол.~4,~13 и 14.}. (Смотри еще вышереченныя утѣшенія причины въ параграфѣ 2"~мъ). "--- 2)~Аще дѣтей, которыя малыми или младенцами кончаются, лишаешися: благодари Бога, что восхищаетъ ихъ, \textit{да не злоба измѣнитъ разумы ихъ, или лесть прельститъ души ихъ}. И тако плоды свои предпосылаешь Хрісту во царствіе Его небесное. Такожде, что не будешь о нихъ имѣть попеченія въ воспитаніи и наказаніи хрістіанскомъ: ибо родители за дѣтей, которыхъ по"=хрістіански не наказуютъ, сильно предъ судомъ Божіимъ истязани будутъ. Когда возрастныхъ лишаешься дѣтей и богатство имѣешь: что имѣлъ дѣтямъ въ наслѣдіе оставить, тое въ руки нищихъ "--- братіи Хрістовой "--- съ пользою душевною можешь вложить; и такъ не единаго, или трехъ, но многихъ наслѣдниковъ будешь имѣть, и обрящешь тое наслѣдіе со многимъ прибыткомъ въ будущемъ вѣкѣ. Когда нищъ остаешься по смерти возрастныхъ дѣтей: отдай себе промыслу Хріста Сына Божія, который всѣхъ питаетъ и о всѣхъ печется. "--- 3)~Когда отца лишаешься: имѣешь Отца Бога, отъ Котораго духовно родился. Той какъ о отцѣ твоемъ промышлялъ, такъ и о тебѣ промышляетъ. Когда отъ матери остаешься сирымъ: имѣешь такожде вмѣсто матери тогожде Бога, Иже есть \textit{Отецъ сирыхъ}\footnote{Пс.~67,~6.}, Который чрезъ пророка всѣхъ таковыхъ утѣшаетъ: \textit{еда забудетъ жена отроча свое, еже не помиловати исчадія чрева своего? Аще же и забудетъ сихъ жена, но Азъ не забуду тебе}, глаголетъ Господь\footnote{Ис.~49,~15.}. "--- 4)~Когда отъ брата или друга смертію разлучился: имѣешь столько друговъ и братій, сколько хрістіанъ, которые \textit{вси о Хрістѣ Іисусѣ едино суть}\footnote{Гал.~3,~28.}. Плотское дружество временное есть, и смертію пресѣкается; но союзъ духовнаго дружества, который бываетъ между Хрістомъ и вѣрными Его, есть нерушимый, твердый и вѣчный, и потому самою смертію не пресѣкаемый. Сіе благословенное дружество здѣ въ мірѣ семъ начинается, а въ будущемъ вѣкѣ совершится, и во вѣки безконечныя нерушимо будетъ, гдѣ едина другъ ко другу любовь будетъ, другъ о другѣ радоваться, утѣшаться, веселиться будутъ, Богъ о избранныхъ Своихъ радоваться будетъ: \textit{и будетъ, якоже радуется женихъ о невѣстѣ, тако возрадуется Господь о тебѣ, избранное, спасенное и благословенное собраніе}\footnote{Ис.~62,~5.}! Избранніи о Господѣ Бозѣ своемъ веселитися и утѣшатися будутъ. Тогда всякъ отъ нихъ радостнымъ духомъ возопіетъ: \textit{да возрадуется душа моя о Господѣ: облече бо мя въ ризу спасенія и одеждою веселія одѣя мя}\footnote{61,~10.}, и проч. "--- 5)~Когда мужъ жены лишился: свободился отъ закона брачнаго, которымъ жена мужу, и мужъ женѣ обязуется, и удобнѣе можетъ пещися, \textit{како угодити Господеви}. Такожде и жена, лишившаяся мужа, удобнѣе можетъ пещися, \textit{како угодити Господеви}\footnote{1~Кор.~7,~32--34.}. "--- 6)~Хотя какъ ни будешь скорбѣть, сѣтовать, плакать, рыдать и рваться, уже не можешь возвратить умершаго. Въ твоей воли состоитъ плакать, или не плакать, но возвратить не въ твоей воли и власти. Ты за умершимъ будешь слѣдовать непремѣнно, а онъ къ тебѣ не возвратится, и ради того тщетны слезы и плачь. "--- 7)~Отъ скорби излишней многіе до того приходятъ, что говорятъ: обидѣлъ"=де мене Богъ, что того"=то любезнаго отъ мене взялъ, и тако страшно хулятъ Бога. Богъ бо никого не обидитъ, но все дѣлаетъ по премудрому Своему о насъ промыслу. "--- 8)~Что Богу угодно, яко верховному Господу и Творцу, тое и дѣлаетъ. Убо, когда скорбишь и сѣтуешь о умершемъ, который волею Божіею умре, показуешь свое неблаговоленіе, и тѣмъ самымъ воли Божіей противишься, въ чемъ грѣшишь предъ Создателемъ своимъ. "--- 9)~Господь тебѣ далъ сына, или дочь, или жену, "--- или тебѣ, жено, мужа: Той отъ тебе и поемлетъ сына, или дочь, и проч.; и потому Свое поемлетъ, что тебѣ далъ, а не твое собственное. Вси бо Божіи есмы, и отцы и матери, сыны и дщери, мужіе и жены, братіе, друзи, господіе и раби; и потому нѣтъ, чего скорбѣть и сѣтовать неутѣшно, но должно всякому съ Іовомъ святымъ въ таковомъ случаѣ говорить: \textit{Господь даде, Господь и отъятъ; яко Господеви изволися, тако и бысть. Буди имя Господне благословено во вѣки}\footnote{Іов.~1,~21.}.

\subsection[Глава 2-я. О второмъ Хрістовомъ пришествіи, или о судѣ страшномъ.]{глава вторая.\\\bfseries О второмъ Хрістовомъ пришествіи, или о судѣ страшномъ.}

\begin{quotation}\textit{Господь во вѣкъ пребываетъ: уготова на судъ престолъ Свой, и Той судити имать вселеннѣй въ правду}\footnote{Пс.~9,~8 и 9.}.\end{quotation}
\begin{quotation}\textit{Мужіе Галилейстіи! что стоите зряще на небо? Сей Іисусъ, вознесыйся отъ васъ на небо, такожде пріидетъ, имже образомъ видѣсте Его идуща на небо}, ангели апостоламъ при вознесеніи Хрістовомъ глаголютъ\footnote{Дѣян.~1,~11.}.\end{quotation}
\begin{quotation}\textit{Всѣмъ явитися намъ подобаетъ предъ судищемъ Хрістовымъ, да пріиметъ кійждо, яже съ тѣломъ содѣла, или блага или зла}\footnote{2~Кор.~5,~10.}.\end{quotation}


\paragraph*{§\:165.} Какъ вѣра святая и, святое Божіе слово научаетъ насъ, что Іисусъ Сынъ Божій на земли явился нашего ради спасенія и съ человѣками пожилъ, и спасеніе вѣчное содѣлалъ вѣрующимъ въ Него, умре и воскресе, вознесеся на небо и сѣде одесную Бога: такъ таяжде вѣра утверждаетъ насъ, что паки пріидетъ на землю. Первое пришествіе Его было смиренное: второе славное. Въ первомъ пришелъ взыскати и спасти погибшаго: во второмъ пріидетъ судити и воздати коемуждо по дѣломъ его. Въ первомъ пришелъ призвати грѣшныхъ на покаяніе: во второмъ пріидетъ судити грѣшныхъ непокаявшихся. Въ первомъ милость Свою и человѣколюбіе показалъ намъ: во второмъ объявитъ правду Свою.

\paragraph*{§\:166.} Образъ или производство суда Хрістова изображается у Матѳеа евангелиста въ главѣ 25"~й. Что соберутся на судъ оный вси языцы, и разлучатся другъ отъ друга, якоже пастырь разлучаетъ овцы отъ козлищъ; и поставятся овцы одесную праведнаго Судіи, козлища же ошуюю. Сущіи одесную услышатъ вожделѣнный гласъ отъ Царя славы: \textit{пріидите, благословенніи Отца Моего, наслѣдуйте уготованное вамъ царствіе отъ сложенія міра. Взалкахся бо, и дасте Ми ясти; возжадахся, и напоисте Мя; страненъ бѣхъ, и введосте Мене; нагъ, и одѣясте Мя! боленъ, и посѣтисте Мене; въ темницѣ бѣхъ, и пріидосте ко Мнѣ}. Сущіи же ошуюю услышатъ страшное опредѣленіе: \textit{идите отъ Мене, проклятіи, во огнь вѣчный, уготованный діаволу и аггеломъ его. Взалкахся бо, и не дасте Ми ясти; возжадахся, и не напоисте Мене; страненъ бѣхъ, и не введосте Мене; нагъ, и не одѣясте Мене; боленъ и въ темницѣ, и не посѣтисте Мене. "--- И идутъ сіи въ муку вѣчную: праведницы же въ животъ вѣчный}\footnote{ст.~31--46.}.

\paragraph*{§\:167.} На судѣ ономъ не только за злая дѣла, но и за оставленіе добрыхъ дѣлъ будемъ судими. Глаголетъ бо къ сущимъ ошуюю: \textit{взалкахся, и не дасте Ми ясти}, и проч. Богъ бо повелѣваетъ какъ отъ зла уклоняться, такъ и благое творить: того ради какъ твореніемъ зла, такъ и оставленіемъ добра заповѣдь Божія преступается. Отъ преступленія же заповѣди Божія бываетъ непослушаніе Богу. Въ чемъ человѣкъ предъ Богомъ виноватъ находится.

\paragraph*{§\:168.} Доносителей и свидѣтелей, какъ на судѣ человѣческомъ бываетъ, на судѣ ономъ не потребно будетъ. Ибо Судія, яко сердцевѣдецъ, знаетъ все "--- не токмо слово и дѣло, но и помышленіе и умышленіе и намѣреніе всякаго, да и книги совѣсти разгнутся, въ которыхъ всякъ увидитъ дѣла, слова, умышленія, начинанія и намѣренія своя. «Якоже бо за тѣломъ тѣнь, глаголетъ Василій Великій, тако за душами грѣхи послѣдуютъ на послѣдній судъ, явно дѣянія показующіи»\footnote{\textit{Въ словѣ къ богатымъ}.}. Горе намъ тогда отъ сихъ нашихъ соперниковъ будетъ, когда отъ нихъ слезами и покаяніемъ здѣ не свободимся. И не только сами увидимъ ихъ, но и весь свѣтъ, ангели и вси человѣцы увидятъ дѣла наши явныя и сокровенныя. «Благая и злая, глаголетъ тойжде отецъ святый, явная и тайная, дѣла, слова и помышленія, вся сія явно въ слухъ всѣмъ, ангеломъ и человѣкомъ открыются»\footnote{\textit{Въ посл. къ падшей дѣвѣ}.}.

\paragraph*{§\:169.} На судѣ ономъ не будетъ различія между княземъ и подданнымъ его, между господиномъ и рабомъ, между благороднымъ и худороднымъ, какъ на судѣ человѣческомъ бываетъ. Ибо судія нелицепріемлющій есть, и судитъ по дѣламъ, а не по лицамъ: \textit{паче же сильніи сильнѣ истязани будутъ; не щадитъ бо лица всѣхъ Владыка, ниже усрамится вельможи}\footnote{Прем.~6,~6 и 7.}. \textit{И емуже дано будетъ много, много взыщется отъ него}\footnote{Лук.~12,~48.}. Отъ пастыря взыщутся овцы словесныя, кровію Хрістовою стяжанныя и ему въ храненіе порученныя; отъ судіи взыщется храненіе правды, и клятвы, предъ свидѣтелемъ Богомъ учиненныя; отъ богатаго взыщется, какъ и на что отъ Бога данное богатство употребилъ; отъ разумнаго и ученаго взыщется, на что даръ разума, отъ Бога данный, употребилъ; такъ и \textit{всякъ за свой талантъ данный отвѣтъ воздастъ}\footnote{Матѳ.~25,~14--30.}.

\paragraph*{§\:170.} Тамо и мнимые праведники, и почитаемые грѣшники объявятся. \textit{Тогда станетъ въ дерзновеніи мнозѣ праведникъ предъ лицемъ оскорбившихъ его, и отметающихъ труды его}. \textit{И} котораго \textit{имѣлъ} міръ \textit{въ посмѣхъ и притчу поношенія, и житіе вмѣнилъ неистово и кончину безчестну: того вмѣнившагося въ сынѣхъ Божіихъ и во святыхъ жребій его} увидитъ\footnote{Прем.~5,~1,~3--5.}.

\paragraph*{§\:171.} Обычай есть позываемымъ на судъ міра сего заранѣе приготовляться, что и какъ отвѣщать на судѣ, чтобы не быть осужденными. Но приготовленіе къ суду Хрістову противно сему. Здѣсь много помогаетъ хитрость и краснорѣчіе: тамо ничего. Здѣсь часто злато и сребро защищаетъ и правды уста заграждаетъ; а тамо ничего. Здѣсь предстатели многимъ помогаютъ, какъ"=то: сильные сильнымъ, благородные благороднымъ, богатые богатымъ: тамо ничего. И ради того хотящіи на судъ міра сего явиться, сихъ и подобныхъ пособій ищутъ, чтобы на судѣ не посрамиться: а къ суду оному какое приготовленіе? Весьма противное!.. Какое же? "--- Нынѣ заранѣе обнажить всего себе должно предъ Судіею онымъ, и то самовольно; прежде суда того самаго себе осудить, самаго себе виноватымъ объявить, признать себе чистосердечно достойнымъ всякія казни. А притомъ со смиреніемъ и надеждою молить Его, чтобы не судилъ по дѣломъ: \textit{не вниди въ судъ съ рабомъ Твоимъ}\footnote{Пс.~142,~2.}; и такъ отъ правды Его къ милости Его, отъ гнѣва Его праведнаго къ милосердію Его прибѣгать, и часто мытаревъ смиренный гласъ повторять: \textit{Боже, милостивъ буди мнѣ грѣшному}\footnote{Лук.~18,~13.}. И чинить тое нынѣ въ семъ вѣцѣ, пока милость Божія всѣмъ ищущимъ ея отверста. А тамо всякое смиреніе и моленіе безполезно, ибо правда тогда вступитъ въ свое дѣйствіе.

\paragraph*{§\:172.} \textit{Господь во вѣкъ пребываетъ: уготова на судъ престолъ Свой}. Готовъ архангелъ \textit{вострубити и возбудити} отъ начала міра \textit{спящія}\footnote{Сол.~4,~16; 1~Кор.~15,~32.}; готовы ангели \textit{собрати избранныя Его, отъ четырехъ вѣтръ, отъ конецъ небесъ до конецъ ихъ}\footnote{Матѳ.~24,~31.}, "--- готовы, но единаго повелѣнія Судіи ожидаютъ. Готова геенна мучити грѣшныя; готово царство небесное упокоити и увеселяти праведныя. \textit{Уготова на судъ престолъ Свой Судія праведный, воздати комуждо по дѣломъ его}. А когда будетъ день тотъ судный, когда пріидетъ Судія на судъ Свой, когда поставятся престоли на судъ, когда вострубитъ архангелъ и возбудитъ мертвыя, "--- сокрылъ отъ насъ промыслъ Божій, чтобы всегда Его ожидали. Пророцы вопіютъ, апостоли проповѣдуютъ, отцы и пастыри научаютъ, что \textit{уготова на судъ престолъ Свой Господь, и Той судити имать вселеннѣй въ правду}, но ожидаетъ, чтобы и мы готовилися. Готовися убо и ты, грѣшниче! Судія \textit{уготова уже на судъ престолъ Свой. Се Судія при дверехъ есть}\footnote{Іак.~5,~9.}\textit{, пріидетъ, и не закоснитъ}\footnote{Евр.~10,~37.}. Готовися и ты, чтобы неготоваго не засталъ тебе; очищай совѣсть твою теплымъ покаяніемъ; смирись, моли умильно Судію, пока не зоветъ тебе на судъ Свой. Нынѣ моли Его, пока время даетъ тебѣ; нынѣ исправляйся, пока \textit{время благопріятно и день спасенія}\footnote{2~Кор.~6,~2.}; нынѣ плачи, пока полезны слезы, нынѣ кайся, пока Онъ пріемлетъ кающихся. Тогда бо не будетъ мѣста покаянію, ни слезамъ, ни плачу; но единое строгое дѣлъ испытаніе будетъ.

\paragraph*{§\:173.} Особливо примѣчанія достойны оныя слова Хрістовы, которыя на судѣ Своемъ къ сущимъ ошуюю глаголетъ: \textit{идите отъ Мене, проклятіи, во огнь вѣчный, уготованный діаволу и аггеломъ его. Взалкахся бо, и не дасте Ми ясти}, и проч. Ежели тѣхъ, которые отъ своихъ имѣній не удѣляли и прочіихъ дѣлъ милости къ ближнему не показывали, отсылаетъ Хрістосъ въ вѣчный огнь: что уже будетъ тѣмъ, которые не токмо своего не давали, но и чуждая грабили? Что ворамъ, хищникамъ, мздоимцамъ, лихоимцамъ? Что тѣмъ, которые по должности своей не только не отирали слезъ плачущихъ, но ради лакомства своего и умножали? Что тѣмъ, которые наемниковъ своихъ безъ надлежащаго награжденія отсылали? Что тѣмъ, которые беззаконною питались лихвою? Что тѣмъ, которые съ подчиненныхъ своихъ послѣднюю копѣйку сдирали, и тѣмъ себѣ чести искали, или роскоши служили, или сынамъ богатое наслѣдіе заготовляли, или дочерей своихъ, яко подобіе храма, украшали и имъ знатное приданое давали? Строгости гнѣва Божія и тяжести будущія вѣчныя казни имъ слѣдующія изобразить невозможно!.. Самимъ имъ оставляю на разсужденіе. "--- А истинное покаяніе, о которомъ выше сказано, все заглаждаетъ.

\subsection[Глава 3-я. О мукѣ вѣчной и о животѣ вѣчномъ.]{глава третія.\\\bfseries О мукѣ вѣчной и о животѣ вѣчномъ.}

\begin{quotation}\textit{Идутъ сіи въ муку вѣчную: праведницы же въ животъ вѣчный}\footnote{Матѳ.~25,~46.}.\end{quotation}


\paragraph*{§\:174.} Вѣчность, въ которой по востаніи отъ мертвыхъ и окончаніи праведнаго Хрістова суда будемъ пребывать, не ино что есть, какъ едино начало безъ конца: она всегда начинаться будетъ, но никогда не скончается. Воображай въ умѣ, что пройдетъ сто тысящей лѣтъ: нѣтъ конца вѣчности. Пройдетъ сто тысящей вѣковъ (вѣкъ въ себѣ заключаетъ сто годовъ): нѣтъ конца вѣчности. Пройдетъ тысяща тысящей вѣковъ и милліонъ милліоновъ: нѣтъ конца вѣчности, но только начинается. И хотя далѣе умомъ поступать будешь въ вѣчность, едино увидишь начало безъ конца. Сколько ни собери милліоновъ вѣковъ въ умъ твой, то вси они такъ имѣются противу вѣчности, какъ едина капля воды противъ цѣлаго океана, или какъ маковое зерно противу всего свѣта, или, лучше сказать, какъ ничто. Ибо все можетъ мало"=по"=малу умалиться и скончаться, но вѣчность не можетъ, ибо безконечна. Чимъ далѣе простираешься умомъ въ вѣчность, тѣмъ большая долгота ея открывается, ибо безконечна. Добрѣ нѣкоторые изображаютъ вѣчность сими словами: вѣчность есть \textit{всегда и никогда}, то"=есть, всегда будетъ, но никогда не кончится.

\paragraph*{§\:175.} Святое Божіе слово объявляетъ намъ двоякую вѣчность: \textit{блаженную и неблагополучную}. Блаженная вѣчность, или вѣчное блаженство будетъ на небеси, по оному апостола слову: \textit{наше житіе на небесѣхъ есть}\footnote{Филип.~3,~20.}: неблагополучная во адѣ и гееннѣ огненной. Въ блаженной вѣчности будетъ житіе безсмертное: въ неблагополучной будетъ смерть безсмертная. Въ блаженной вѣчности будетъ пресладкое Божія лица зрѣніе: въ неблагополучной удаленіе отъ лица Божія. Въ блаженной вѣчности будетъ царствованіе со Хрістомъ и любезное дружество со святыми ангелами Его: въ неблагополучной съ діаволомъ и злыми его демонами пребываніе. Въ блаженной вѣчности будетъ свѣтъ неизреченный: въ неблагополучной тьма кромѣшняя. Въ блаженной вѣчности будетъ радость и веселіе непрестанное, слышаться будетъ \textit{гласъ радости и веселія}: въ неблагополучной скорбь и печаль непрестанная, и \textit{гласъ воздыханія и плача} услышится. Въ блаженной вѣчности совокупятся вся \textit{благая, ихже око не видѣ, и ухо не слыша, и на сердце человѣку не взыдоша}\footnote{2~Кор.~2,~9.}: въ неблагополучной снидутся вся злая, которыхъ такожде языкъ изрещи и умъ понять не можетъ.

\paragraph*{§\:176.} Изъясняется блаженная вѣчность сравненіемъ временнаго благополучія, которое хотя и ничтоже противу оныя, однакожъ показуетъ ее намъ такъ, какъ тѣнь и подобіе самую вещь. Всякъ знаетъ сіе, какъ люди благополучіемъ свѣта сего уловляются, которое наипаче состоитъ въ чести, славѣ, богатствѣ, мирѣ, покоѣ, дружествѣ съ добрыми и почтенными людьми, и проч., и какъ ублажаются тѣ, которые въ тѣхъ процвѣтаютъ. Но однакожъ всякое нынѣшнее благополучіе ничто въ сравненіи съ вѣчнымъ блаженствомъ. 1)~Всякое временное благополучіе не можетъ быть такъ великое, которое бы не было растворено съ какою нибудь противностію. Многіе богатство имѣютъ, но не имѣютъ здравія, мира, покоя, имени добраго, и прочая. Иные въ мирѣ и покоѣ живутъ, но нищенствуютъ и въ презрѣніи находятся. Дружество безъ подозрѣнія не бываетъ, ибо нѣтъ совершенной любви. Нѣтъ мира безъ опасности, ибо многіе окружаютъ враги. Нѣтъ здравія безъ немощи въ тлѣнномъ и смертномъ тѣлеси. Нѣтъ славы, которую бы зависть клеветою и злословіемъ не помрачала. Такъ и о прочемъ разумѣть должно. Но блаженная вѣчность всякую выключаетъ противность: тамо царствіе безъ страха, безъ трудовъ и попеченія; тамо слава всегда сіяетъ; тамо здравіе безъ всякой болѣзни; тамо радость безъ скорби; тамо миръ и покой безъ опасности; тамо веселіе безъ печали; тамо дружество нелицемѣрное; тамо премудрость безъ \textit{буйства}, тамо животъ безъ смерти; тамо всякое добро непричастно никакого зла. Ибо Богъ, Источникъ всѣхъ благъ, \textit{будетъ всяческая во всѣхъ}\footnote{Кор.~15,~28.}. "--- 2)~Сердце человѣческое никакимъ временнымъ благополучіемъ удовольствоваться не можетъ. Чимъ большее богатство, слава, честь, тѣмъ болѣе желанія къ богатству, славѣ, чести растетъ. А гдѣ желаніе и попеченіе, тамо безпокойствіе и мятежъ. Но въ блаженной вѣчности всякое кончится желаніе, ибо совершенное въ ней блаженство. Тамо слава, честь выше подняться не можетъ; тамо богатство не растетъ; тамо здравіе, миръ, покой, мудрость не умножается: ибо получится высочайшее Добро, которое есть Богъ. Тамо всякъ своимъ жребіемъ доволенъ. \textit{Насыщуся}, глаголетъ пророкъ, \textit{внегда явитимися славѣ Твоей}\footnote{Пс.~16,~15.}. Оную славу, честь, богатство, радость, миръ, покой получившіе болѣе ничего не будутъ желать. Ибо кончится желаніе, когда совершенное блаженство, каковаго болѣе быть не можетъ, постигается. "--- 3)~Хотя бы кто, временное благополучіе имѣя, и не желалъ болѣе, но страхомъ смущается, чтобы того не потерять. Ибо что на другихъ видитъ, того и себѣ чаетъ благополучіемъ цвѣтущій. Видитъ, что за богатствомъ ходитъ нищета, когда его \textit{тля тлитъ, и татіе подкопываютъ и крадутъ}\footnote{Матѳ.~6,~19.}, "--- за славою безславіе, за честію безчестіе. Который нынѣ ѣздитъ на колесницѣ, утро затворяется въ темницѣ, котораго нынѣ хвалятъ, утро проклинаютъ; которому нынѣ покланяются, утро попираютъ ногами. И мало есть такихъ, которые бы въ единомъ постоянномъ счастіи пребывали. Но хотя кому и служитъ рѣдкая сія фортуна, такъ страхъ смертный, съ часа на часъ ожидаемый, безпокойствуетъ, и такъ все благополучіе, какъ мгла солнечное сіяніе, помрачаетъ. Но блаженная вѣчность все выключаетъ противное. Оное добро, единожды полученное, никогда не потеряется; оное богатство не боится нищеты; оная слава не опасается безславія; оная честь не блюдется безчестія; оное здравіе, миръ, покой, веселіе не страшится немощи, безпокойства, печали. Ибо все тамо безопасно, все мирно и покойно. "--- 4)~Наконецъ какъ жизнь сія, такъ и всякое благополучіе смертію оканчивается, и такъ все бываетъ, какъ соніе востающаго, какъ пара, вмалѣ являющаяся и исчезающая. И ради того все временное благополучіе есть какъ ничто. Умираетъ человѣкъ: умираетъ и благополучіе его. Оставляетъ богатый міръ: оставляетъ и богатство свое. Отходитъ славный отъ свѣта сего: отходитъ отъ него и слава его. Кончится мудрый: кончится и мудрость его. Умираетъ сильный: умираетъ и сила его. Гдѣ нынѣ страхъ побѣдителей міра бывшихъ? гдѣ славныхъ слава? гдѣ богатыхъ богатство? гдѣ сильныхъ сила? Гдѣ премудрыхъ премудрость? гдѣ роскошныхъ утѣшеніе и сласть? Все отошло отъ нихъ, какъ сами отсюду отошли. \textit{Преидоша вся она, яко сѣнь, и яко вѣсть претекающая}\footnote{Прем.~5,~9.}. Но хотя такъ непостоянно благополучіе сіе, однакожъ имъ сыны вѣка утѣшаются. А какъ утѣшатся избранніи Божіи, когда истинное оное блаженство, истинную оную славу, честь, миръ, покой, радость и веселіе отъ руки Господни воспріимутъ, и все богатство благости Божіей наслѣдятъ, и твердо увѣрены будутъ, что въ ономъ высочайшемъ блаженствѣ во вѣки безконечные пребывати будутъ! Котораго какъ себѣ, такъ и всякому усердно желаю. Аминь!

\paragraph*{§\:177.} Какъ вѣчное блаженство отъ временнаго благополучія, такъ и вѣчное злополучіе отъ временнаго нѣсколько познавается и ощущается. Нѣтъ такъ великаго въ мірѣ семъ бѣдствія, которое бы не имѣло какого утѣшенія и прохлажденія. Скучна нищета, но получаетъ утѣшеніе отъ Хрістова имени и милости добрыхъ людей. Мучительна жестокая болѣзнь, но прохлаждается врачевствомъ, или другимъ чимъ. За великое бѣдствіе и злополучіе поставляется не только чести, имѣнія, жены, дѣтей лишиться, но и ввержену быть въ темницу; однакожъ тако бѣдствующій не безъ утѣшенія: свѣта и воздуха всѣмъ общаго наслаждается, хотя не равно, какъ свободные; и пищею и питіемъ прохлаждается; скуку спя забываетъ; посѣщеніемъ добрыхъ людей утѣшается; надеждою свобожденія питается: можетъ быть, что ранъ и біенія не терпитъ. Когда неправедно страждетъ, надеждою небесныя мзды увеселяется; когда праведно смиряется предъ Богомъ, признаетъ грѣхъ, кается, и тако надеждою милости Божіей прохлаждается, глаголя: \textit{благо мнѣ, яко смирилъ мя еси, яко да научуся оправданіямъ Твоимъ}\footnote{Пс.~118,~71.}. Сверхъ того смерть все тое бѣдствіе окончитъ. Но колико есть, коль тяжко, коль нестерпимо лишиться той славы, предъ которою вся слава міра сего, въ едино собранная, какъ гной; лишиться зрѣнія того прекраснаго Лица, Котораго красота, любезность и благопріятіе все сердце и духъ весь въ желаніе и любовь претворяетъ, Котораго блаженніи дуси отъ начала насытитися не могутъ, и во вѣки безъ сытости насыщатися будутъ; отринуться отъ Царя славы Хріста, Котораго красота солнечное сіяніе превосходитъ, Котораго сладости мало вкусивъ на горѣ, святый Петръ возопилъ: \textit{добро намъ здѣ быти}\footnote{Матѳ.~17,~4.}! удалиться отъ града онаго прекраснаго, небеснаго Іерусалима, который \textit{не требуетъ солнца и луны, да свѣтятъ въ немъ: слава бо Божія просвѣти его, и свѣтильникъ его Агнецъ}\footnote{Апок.~21,~23.}, изгнаться отъ славной и сладкой оной вечери, на которой отъ начала міра до конца званніи и избранніи Божіи \textit{пити, веселитися} и отъ радости восклицати будутъ\footnote{Ис.~65,~13,~14 и 18.}; отчуждиться любезнаго дружества блаженныхъ духовъ и всѣхъ святыхъ; сверхъ того причислиться къ темнымъ и лукавымъ духамъ, заключену быть въ темницу, свѣта непричастну, страдать безъ ослабленія въ геенскомъ огнѣ, который жжетъ, но не сжигаетъ, мучитъ, но не умерщвляетъ. Временное бѣдствіе и страданіе, какъ сказано, коликое бы ни было, не безъ утѣшенія бываетъ, и коль долго ни продолжается, смертію заключается: но оное бѣдствіе и злополучіе безконечное. Не будетъ возвращенія къ славѣ Божіей; не будетъ избавленія отъ узъ адовыхъ и страданія огненнаго; словомъ, вѣчное отчаяніе милости Божіей, вѣчное чувствованіе праведнаго гнѣва Божія. Тамо услышится отвѣтъ: \textit{помяни, яко воспріялъ еси благая твоя въ животѣ твоемъ}\footnote{Лук.~16,~25.}. Здѣ убѣгаютъ люди отъ смерти: но тамо пожелаютъ ея, и \textit{бѣжитъ отъ нихъ}; пожелаютъ въ ничтоже обратитися, но не возмогутъ, и тако всегда будутъ умирать, но никогда не умрутъ. И се есть \textit{вторая смерть}\footnote{Апок.~9,~6; 21,~8.}! Зѣльное страданіе умножается тѣмъ, что страждущему не остается никакой надежды возвратить того добра, которое нерадѣніемъ потерялъ, и свободиться отъ того страшнаго зла, въ которое самоизвольно попался. Отъ сего воспослѣдуетъ тоска превеликая, страхъ и ужасъ превеликій. Востанетъ совѣсть, которая чрезъ всю вѣчность будетъ обличать и грызть грѣшника, что такъ высокую Божію благодать отринулъ, и такъ самовольно высочайшей Божіей милости на вѣки лишился, и въ такъ мучительное состояніе пришелъ. Богъ, Творецъ небеси и земли, Сына Своего ради тебе не пощадѣлъ, но на смерть и такъ лютыя страданія предалъ Его, какъ тебѣ Евангеліе объявляло и учители церковные проповѣдывали; и сіе Богъ учинилъ того ради, чтобы тебе отъ бѣды сея избавить: но ты сію неизреченную Божію любовь презрѣлъ; ради временнаго малаго услажденія Бога оставилъ; исполнялъ свои прихоти, а не волю Божію, которая хотѣла тебѣ спастися; работалъ и угождалъ сатанѣ, а не Хрісту. Вотъ прочіе, подобные тебѣ люди, нынѣ утѣшаются, радуются, насыщаются вѣчныхъ благъ, понеже вѣровали, и вѣрою послѣдовали Хрісту Спасителю своему, а ты совѣтъ Божій отринулъ, и такъ погиблъ. Сколько тебе Богъ призывалъ на покаяніе; сколько увѣщавали проповѣдники: ты все тое презрѣлъ. Не хотѣлъ ты послушать Бога въ животѣ твоемъ: не услышитъ и тебе нынѣ Богъ; не ожидай болѣе ничего, кромѣ сего, въ которомъ находишься, бѣдствія. Сію чашу горести всегда будешь пить, но никогда не испіешь. Кайся нынѣ, но поздно и безполезно, когда не хотѣлъ въ мірѣ полезно каяться; стени, воздыхай, плачь, рыдай, но безполезно, когда не хотѣлъ плакать тамо, гдѣ полезны слезы были. Тако осужденнаго мучить и, какъ червь, грызть будетъ безъ конца совѣсть, и къ болѣзни болѣзнь, къ мученію мученіе придавать будетъ!.. Въ такой тѣснотѣ и отчаяніи не иное что послѣдуетъ, какъ что осужденный самъ себе будетъ ненавидѣть, самъ на себе гнѣваться, самъ собою мерзѣть, гнушаться, самаго себе окаевать, осуждать, часъ зачатія и часъ рожденія проклинать, проклинать время, въ которое грѣшилъ, проклинать случаи, которые ко грѣху приводили, проклинать тѣ лица, которыя ко грѣху случай подавали, воспоминать тѣхъ, которые отъ грѣха увѣщавали. О вѣчность, вѣчность, злополучная вѣчность! Какъ и самое воспоминаніе твое ужасаетъ духъ нашъ! Сколько увеселяетъ и восхищаетъ духъ размышляющаго блаженная, столько въ печаль и страхъ приводитъ злополучная вѣчность. Не возможно о сей помыслить безъ воздыханія и ужаса. Лучше здѣсь всякое бѣдствіе терпѣть; лучше во всякой болѣзни тяжчайшей чрезъ все житіе страдать, лучше оковану въ смрадной темницѣ чрезъ весь вѣкъ сидѣть, біеніе и раны пріимать, по вся дни умирать, въ огнѣ горѣть, когда сіе возможно будетъ; лучше наконецъ всѣ, сколько на свѣтѣ могутъ быть, бѣды, въ едино собравшіяся, съ благодареніемъ терпѣть, когда воли Божіей угодно будетъ, нежели блаженной лишитися, и попасться въ неблагополучную вѣчность. Ибо здѣсь, какое бы не приключилося страданіе, какое нибудь утѣшеніе имѣетъ, и смертію окончевается: тамо страданіе лютое, страданіе безъ всякаго утѣшенія, страданіе не токмо словомъ, но и умомъ непостижимое страданіе, которое всегда будетъ, но никогда не скончается. \textit{Червь ихъ не скончается, и огнь ихъ не угаснетъ, глаголетъ пророкъ}\footnote{Ис.~66,~24.}. Кто вѣритъ сему, тотъ, размышляя и часто поминая о страшномъ злополучіи ономъ, можетъ къ усерднѣйшему возбудиться покаянію, чтобы отъ бѣдствія того избавиться: а кто не вѣритъ, тотъ самымъ дѣломъ горести его вкуситъ, и безъ конца, какъ сказано, вкушать будетъ.

\paragraph*{§\:178.} Въ блаженную вѣчность входятъ благочестивые, вѣрою во Хріста Сына Божія оправданные, и въ вѣрѣ живой житіе сіе окончавшіе, которыхъ Самъ Хрістосъ, ублажая, призываетъ въ блаженство оное: \textit{пріидите, благословенніи Отца Моего, наслѣдуйте уготованное вамъ царствіе отъ сложенія міра. Взалкахся, и дасте Ми ясти; возжадахся, и напоисте Мя; страненъ бѣхъ, и введосте Мене; нагъ и одѣясте мя; боленъ, и посѣтисте Мене: въ темницѣ бѣхъ, и пріидосте ко Мнѣ}\footnote{Матѳ.~25,~34--36.}. Въ неблагополучную ввергнутся грѣшницы нераскаянные, которыхъ Хрістосъ отошлетъ отъ Себе, яко неблагодарныхъ, тако: \textit{идите отъ Мене проклятіи во огнь вѣчный, уготованный діаволу и аггеломъ его. Взалкахся бо, и не дасте Ми ясти; возжадахся, и не напоисте Мене; страненъ бѣхъ, и не введосте Мене; нагъ, и не одѣясте Мене; боленъ и въ темницѣ, и не посѣтисте Мене. "--- И идутъ сіи въ муку вѣчную: праведницы же въ животъ вѣчный}\footnote{25--41,~43 и 46.}. Идутъ обои, но не равно: одни съ плачемъ и рыданіемъ неутѣшнымъ, другіе съ веселіемъ и ликованіемъ неизреченнымъ. Ибо въ неравная мѣста идутъ: одни въ обители Отца небеснаго, въ украшенные чертоги Царя славы, на велію вечерю, безконечно увеселяющую, и пресладкій бракъ Агнчій; другіе въ темницу вѣчную и мѣсто плача, рыданія и мученія вселютѣйшаго. Однихъ воспріемлетъ небо со всѣми благими, \textit{ихже око не видѣ, и ухо не слыша, и на сердце человѣку не взыдоша}; другихъ заключаетъ адъ со всѣми злыми, которыхъ ни словомъ изобразить, ни умомъ понять невозможно.


\section[Статія 3-я. О добродѣтеляхъ христіанскихъ.]{статія третья.\\\bfseries О добродѣтеляхъ христіанскихъ.}
\subsection[Глава 1-я. О страсѣ Божіи.]{глава первая.\\\bfseries О страсѣ Божіи.}

\begin{quotation}\textit{Аще Отца называете нелицемѣрно судяща комуждо по дѣлу, со страхомъ житія вашего время жительствуйте, вѣдяще, яко не истлѣннымъ сребромъ или златомъ избавистеся отъ суетнаго вашего житія, отцы преданнаго, но честною кровію яко Агнца непорочна и пречиста Хріста, предувѣдѣннаго убо прежде сложенія міра, явльшагося же въ послѣдняя лѣта васъ ради}\footnote{1~Петр.~1,~17--20.}.\end{quotation}
\begin{quotation}\textit{Начало премудрости страхъ Господень; разумъ же благъ всѣмъ творящимъ и: хвала его пребываетъ въ вѣкъ вѣка}\footnote{Пс.~110,~10.}.\end{quotation}
\begin{quotation}\textit{Блаженъ мужъ бояйся Господа, въ заповѣдехъ Его восхощетъ зѣло}\footnote{111,~1.}.\end{quotation}


\paragraph*{§\:179.} Коль великое есть дарованіе Божіе страхъ Божій, и какъ блажени суть имѣющіи его, Божіе слово свидѣтельствуетъ. \textit{Страхъ Господень слава и похвала, и веселіе, и вѣнецъ радости; страхъ Господень возвеселитъ сердце и дастъ веселіе и радость и долгоденствіе. Боящемуся Господа благо будетъ напослѣдокъ, и въ день скончанія своего обрящетъ благодать. Страхъ Господень даръ отъ Господа, и на стезяхъ любленія поставляетъ. Любленіе Господа преславная премудрость; и имже является, раздѣляетъ себе въ вѣдѣніе его. Начало премудрости боятися Господа, и съ вѣрными въ ложеснѣхъ создася имъ: съ человѣки основаніе вѣка угнѣзди, и съ сѣменемъ ихъ увѣрится. Исполненіе премудрости, еже боятися Господа, и упоитъ ихъ отъ плодовъ ея; весь домъ ихъ исполнитъ желаній своихъ, и сосуды отъ житъ ея. Вѣнецъ мудрости страхъ Господень, возцвѣтаяй миръ и здравіе исцѣленія: обоя же суть дары Божія, и разширяетъ веселіе любящимъ Его. И видѣ и сочте ю, художество и вѣдѣніе разума одожди, и славу держащихъ ю вознесе. Корень премудрости, еже боятися Господа, и вѣтви ея долгоденствіе: Страхъ Господень отрѣяетъ грѣхи}\footnote{Сир.~1,~11--21.}. И паки: \textit{вельможа и судія и сильный славни будутъ: и нѣсть отъ нихъ ни единъ вящшій боящагося Господа}\footnote{10,~27.}. И паки: \textit{ничтоже лучше есть страха Господня и ничтоже сладчае, токмо внимати заповѣдемъ Господнимъ}\footnote{23,~36.}. И паки: \textit{коль великъ, иже мудрость обрѣте, но нѣсть паче боящагося Господа: страхъ Господень паче всего предуспѣ}\footnote{25,~13 и 14.}. И паки: \textit{боящийся Господа ничего не убоится и не устрашится: Той бо надежда ему. Боящемуся Господа блаженна душа}\footnote{34,~14 и 15.}. И паки глаголетъ Богъ: \textit{возсіяетъ вамъ боящимся имене Моего Солнце правды, и исцѣленіе въ крилѣхъ его: и изыдете, и взыграете якоже тельцы отъ узъ разрѣшени}\footnote{Мал.~4,~2.}. И паки: \textit{страхъ Господень источникъ жизни! творите же уклонитися отъ сѣти смертныя}\footnote{Притч.~14,~27.}. И паки, Псаломникъ: \textit{кто есть человѣкъ бояйся Господа, законоположитъ ему на пути, егоже изволи. Душа его во благихъ водворится, и сѣмя его наслѣдитъ землю. Держава Господь боящихся Его, и завѣтъ Его явитъ имъ}\footnote{Пс.~24,~12--14.}. Страхъ бо Божій есть, какъ стражъ вѣрный, который всегда бдитъ и хранитъ душевный домъ отъ всякаго зла, или паче, какъ стѣна крѣпкая и высокая, которая воспящаетъ мещемыя стрѣлы вражія и недѣйствительными дѣлаетъ. Страхъ Божій не попущаетъ прельщатися красотами міра сего. Страхъ Божій честь, славу, богатство, сладость и роскошь міра за подозрѣніе учитъ имѣти, яко подающія случаи ко всякому злу. Страхъ Божій не дозволяетъ свободно уму разсѣваться, языку много говорить, глазамъ всего смотрѣть, ушамъ всего слушать; но всегда и отъ всего берещися и, по подобію птицы, осматриваться научаетъ. Страхъ Божій и позволеннаго, какъ"=то: пищи, питія, одежды, не къ сласти и роскоши, но къ нуждѣ употреблять; страхъ Божій закону Божію и всякому полезному наставленію, обличенію, наказанію прилѣжно внимать, страхъ Божій совѣсть разсматривать и теплымъ покаяніемъ очищать научаетъ. Страхъ Божій всегда молиться и воздыхать къ Богу наставляетъ, чтобы въ грѣхъ не впасть. Страхъ Божій отъ случаевъ уклоняться, которые ко грѣху приводятъ; страхъ Божій обидѣвшему прощать, и на злобствующаго не злобиться, злословящаго не злословить, укоряющаго не укорять увѣщаваетъ. Страхъ Божій наконецъ и самый страхъ человѣческій смертный презирать учитъ, чтобы Бога не прогнѣвать, "--- и на смерть тѣлесную предаваться, чтобы смерти душевныя избѣжать. Страхъ Божій есть истинное училище смиренія, которое привлекаетъ Божію благодать. Воистину ничтоже крѣпчайше и твердѣйше того, кто имѣетъ страхъ Божій. \textit{Вельможа и судія и сильный славни будутъ, но нѣсть отъ нихъ ни единъ вящшій боящагося Господа}, глаголетъ Сирахъ.

\paragraph*{§\:180.} Знаки страха Божія примѣчаются сіи: 1)~Имѣющій страхъ Божій явно и тайно, то"=есть, предъ людьми и безъ людей бережется всякихъ грѣховъ, понеже вездѣ предъ собою Бога зритъ, Котораго опасается прогнѣвать. Слѣдственно кто предъ людьми не грѣшитъ, но тайно грѣшитъ, тотъ страха Божія не имѣетъ, но стыдъ и страхъ человѣческій имѣетъ; людей стыдится и боится, а не Бога; людямъ угождаетъ, а не Богу, "--- и такой человѣкъ есть лицемѣръ. "--- 2)~Не смотритъ чужихъ грѣховъ, но въ свою совѣсть приникаетъ, и въ ней находящіеся грѣхи усматривая, очищаетъ усерднымъ покаяніемъ и молитвою. "--- 3)~Охотно слушаетъ Божіе слово и всякое наставленіе, наказаніе, обличеніе, и ищетъ того, чтобы или грѣхи находящіеся въ совѣсти своей усмотрѣть, и тако о нихъ каяться, или впредь удобнѣе берещися съ помощію Божіею возмоглъ. "--- 4)~Непрестанно и усердно молится и воздыхаетъ, чтобы Самъ Богъ наставилъ его и сохранилъ отъ грѣховъ и всего того, что ко грѣху приводитъ. "--- 5)~Отъ случаевъ, которые ко грѣху приводятъ, какъ"=то: собраній, увеселеній, игръ, бесѣдъ злыхъ, разговоровъ о людяхъ, и прочіихъ, убѣгаетъ. "--- 6)~Видѣніе и слухъ бережетъ, вѣдая, что тѣми, какъ дверьми, всякое зло входитъ въ храмину сердечную и колеблетъ оную. "--- 7)~Языку не попущаетъ много говорить, но болѣе молчать, помышляя, что и за праздное слово слѣдуетъ воздать отвѣтъ въ день судный, и что симъ малымъ удомъ болѣе всего согрѣшаемъ вси. "--- 8)~Отъ чести, богатства и утѣхи міра сего уклоняется, яко подаютъ случай ко грѣху и отворяютъ дверь ко всякому нечестію. "--- 9)~Пищи и питія не къ сласти и роскоши, но къ нуждѣ употребляетъ,чтобы только тѣлесныя укрѣпить силы и тако дѣла званія своего дѣлать и Богу работать моглъ. "--- 10)~Не только никакой временной бѣды или страха человѣческаго, но и самой смерти не боится. Ежели гдѣ слѣдуетъ или въ бѣду впасть и умереть, или согрѣшить: лучше изволяетъ бѣдствовать и умереть, нежели согрѣшить; и въ страхѣ Божіи всякую бѣду и самый страхъ смертный премогаетъ, не иначе, какъ одержимый жестокою болѣзнію не чувствуетъ другія легкія болѣзни, или какъ слышащій великій шумъ малаго гласа не слышитъ. Іосифъ святый, сынъ Іаковль, изволяетъ лучше въ темницѣ сидѣть, нежели съ женою египетскою смѣшаться и согрѣшить\footnote{Быт.~39,~20.}. Бабы еврейскія презираютъ беззаконное повелѣніе Фараона, и не убиваютъ мужескаго пола: \textit{яко убояшася Бога}\footnote{Исх.~1,~17 и 21.}. Тріе отроки разженной пещи\footnote{Дан.~3,~21.}, Сусанна смерти не боятся ради страха Божія\footnote{13,~23.}, и проч.

\paragraph*{§\:181.} Страхъ Божій раждается и умножается при помощи Божіей: 1)~отъ размышленія страшныхъ Его бывшихъ судебъ, которыя на грѣшникахъ, какъ"=то: на Каинѣ\footnote{Быт.~4"~я.}, на погибшихъ потопомъ\footnote{7"~я.}, на Содомлянахъ, огнемъ и жупеломъ сгорѣвшихъ\footnote{19"~я.}, на Фараонѣ, съ воинствомъ въ морѣ погрязнувшемъ\footnote{Исх.~14"~я.}, на Дафанѣ и Авиронѣ, землею пожертыхъ\footnote{Числ.~16"~я.}, на Авессаломѣ, между небомъ и землею повисшемъ и погибшемъ\footnote{2~Цар.~17"~я.}, и на прочіихъ явилися. Но Который Богъ прежде за грѣхи казнилъ, Тойжде и нынѣ на беззаконныхъ посылаетъ казни. Слышимъ ужасныя трясенія земли, и грады въ единомъ гробѣ съ жителями погребаемые; слышимъ и видимъ плачевныя государствъ сраженія, ужасныя кровопролитія, толико тысящей народа, безполезно огнемъ и мечемъ падающаго; видимъ и слышимъ ужасные пожары, которыми веси, села и грады погибаютъ. Сія вся показуютъ, что есть Богъ, Который ненавидитъ беззаконія и наказуетъ беззаконниковъ. А временныя наказанія показуютъ, что будутъ и вѣчныя неисправнымъ и ожесточеннымъ грѣшникамъ: будетъ геенна, червь неусыпающій, скрежетъ зубный, тьма кромѣшная, беззаконныхъ мучащая. "--- 2)~Раждается отъ размышленія Божіихъ свойствъ, наипаче о вездѣсущіи и всевѣдѣніи Божіи разсужденіе и память страхъ Божій содѣловаетъ. Вездѣсущіе Божіе показуетъ тебѣ, что Богъ вездѣ съ тобою есть и надъ тобою, и предъ тобою. \textit{Въ Немъ бо живемъ, движемся и есмы}\footnote{Дѣян.~17,~28.}. Ходиши или почиваеши, говориши или молчиши, единъ ли еси, или съ кѣмъ бесѣдуеши? Богъ съ тобою есть. Дѣлаеши что? видитъ Богъ дѣло твое. Говориши что? слышитъ Богъ слово твое. Мыслиши ли что? проницаетъ Богъ помышленіе твое. Гордишися ли? Богъ смотритъ на гордость твою. Отступаетъ ли сердце твое отъ Бога и обращается къ твари? смотритъ Онъ на отступленіе твое. Преступаеши ли законъ Его святый? смотритъ Онъ на преступленіе твое. Дѣлаеши ли неправду? хищеніе, воровство? смотритъ Онъ на неправду твою, хищеніе и воровство твое. *Блудодѣйствуеши ли? смотритъ Онъ на блудодѣйство твое*. Гнѣваешися ли, или злобишься, или убиваеши ближняго твоего? смотритъ Онъ на гнѣвъ твой, злобу твою и убійство твое. Злословиши, хулиши, проклинаеши, оклеветаеши ли ближняго твоего? слышитъ Онъ злословіе, хулу, клятву и клевету твою. Мыслиши ли обидѣть, повредить, оклеветать, обмануть, оскорбить, опорочить, убить ближняго твоего? хочешь ли нечистоту совершить? видитъ Онъ злое умышленіе и начинаніе твое, и претитъ тебѣ въ совѣсти твоей; видитъ, и оскорбляется. Величество Его оскорбляется твоею гордостію, что ты, \textit{земля и пепелъ}, надымаешься; правда Его оскорбляется твоею неправдою; истина Его оскорбляется твоею лжею; святость Его оскорбляется твоею нечистотою; долготерпѣніе Его твоимъ нетерпѣніемъ; благость Его твоею злобою; милость Его твоимъ жестокосердіемъ; щедрота Его твоею скупостію; любовь Его твоею ненавистію и завистію оскорбляется. Убойся убо предъ величествомъ Его гордиться, предъ правдою Его неправду дѣлать, предъ святостію Его въ нечистотѣ валяться, предъ истиною Его лгать, предъ благостію Его злобиться, предъ долготерпѣніемъ Его гнѣваться и яриться, предъ милосердіемъ Его свирѣпствовать, предъ кротостію Его памятозлобствовать. Присутствуетъ Онъ тебѣ съ своимъ величествомъ и всемогуществомъ, правдою, истиною, святостію, милосердіемъ, щедротами, кротостію, благостію и долготерпѣніемъ, хотя сими очесами и не видишь Его, да и видѣть неможно. Великъ Онъ и страшенъ зѣло: како не убоишися величества Его, и дерзнеши возноситися предъ великимъ и страшнымъ? Праведенъ Онъ: како не убоишися правду нарушать предъ праведнымъ? Истиненъ Онъ: како не убоишися лгать предъ истиннымъ? Святъ Онъ, Которому ангели со страхомъ предстоятъ и поютъ: \textit{святъ, святъ, святъ Господь Саваоѳъ}!\footnote{Ис.~6,~3.}: како не убоишися предъ толикою святостію безчинствовать? Милосердъ Онъ и многомилостивъ: како не убоишися предъ милосердымъ свирѣпо поступать? Щедръ Онъ: како не убоишися затворять утробу твою требующему предъ щедрымъ? Благъ Онъ, кротокъ и долготерпѣливъ: како не убоишися предъ благимъ, кроткимъ и долготерпѣливымъ гнѣваться, яриться и памятозлобствовать? Наконецъ Создатель Онъ твой, высочайшій Господь и Царь твой, Отецъ твой: како не убоишися и не устыдишися предъ Создателемъ твоимъ, Царемъ и Господемъ страшнымъ и Отцемъ безстрашно поступать? Вседержитель Онъ, Который въ руцѣ Своей весь свѣтъ содержитъ\footnote{Пс.~94,~3--7.}: како не устрашишися Его, Который съ свѣтомъ и тебе въ руцѣ Своей содержитъ, прогнѣвлять? Предъ царемъ земнымъ, Государемъ твоимъ, или предъ господиномъ твоимъ, или предъ отцемъ твоимъ не дерзаешь безстрашія и безчинія показывать. Предъ очима Бога твоего, Который есть верховный Царь и Господь всего свѣта и Отецъ вѣчный, всегда и вездѣ, гдѣ ни бываеши, находишися: и како не убоишися безстрашія предъ Нимъ показывать? Какъ ни великъ монархъ земный, но человѣкъ есть подобный тебѣ; какъ ни высокъ господинъ твой, но человѣкъ есть подобный тебѣ; какъ ни честенъ тебѣ отецъ твой, но такожде есть человѣкъ подобный тебѣ. Вси такожде раждаются и умираютъ, какъ и ты, но обаче всякое имъ являеши почтеніе и благоговѣинство. Богъ, предъ Которымъ всегда находишися, такъ великъ есть, что весь свѣтъ предъ Нимъ, аки малѣйшій прахъ, или едина \textit{капля} воды\footnote{Ис.~40,~15.}; такъ силенъ, что все изъ ничего творитъ словомъ единымъ; и есть Царь царей и Господь господей, предъ Которымъ и ангели трепещутъ, трясутся, благоговѣютъ и со страхомъ покланяются: како убо не устрашишися согрѣшать предъ Нимъ, и такого Ему почтенія не показывать, какое земному царю, господину и отцу плотскому, подобнымъ человѣкомъ, показуешь? Всякаго почитанія достоинъ царь земный, господинъ твой и отецъ, и Самъ Богъ велитъ ихъ почитать; но Богъ несравненно большаго почитанія, и такого, какого не можетъ быть большее, такъ великаго, какъ Онъ Самъ великъ, достоинъ есть почитанія. Къ сему высокому почитанію и пророкъ увѣщаваетъ насъ: \textit{работайте Господеви со страхомъ, и радуйтеся Ему съ трепетомъ}\footnote{Пс.~2,~11.}. И на другомъ мѣстѣ созываетъ насъ къ томужде: \textit{пріидите, возрадуемся Господеви, воскликнемъ Богу Спасителю нашему, предваримъ лице Его во исповѣданіи, и во псалмѣхъ воскликнемъ Ему: яко Богъ велій Господь и Царь велій по всей земли; яко въ руцѣ Его вси концы земли, и высоты горъ Того суть; яко Того есть море, и той сотвори е, и сушу руцѣ Его создастѣ}\footnote{Пс.~94,~1--5.}, и проч. "--- 3)~Раждается и умножается страхъ Божій отъ разсужденія повелѣнія Божія. Яко Богъ, Который законъ издалъ и повелѣваетъ его хранить, есть великій и страшенъ и \textit{Богъ отмщеній}\footnote{3,~1.}, и повелѣваетъ его хранить зѣло: \textit{Ты заповѣдалъ еси заповѣди Твоя сохранити зѣло}\footnote{118,~4.}, "--- и за преступленіе святаго закона Своего временною и вѣчною казнію претитъ. Царя земнаго указовъ нарушать боимся, котораго гнѣвъ тѣло только наше умертвить можетъ: какъ Божія повелѣнія нарушать не бояться и праведнаго Его гнѣва, Который \textit{и тѣло и душу можетъ погубить въ гееннѣ огненнѣй}\footnote{Матѳ.~10,~28.}? "--- 4)~Раждается страхъ Божій отъ разсужденія Бога повелѣвающаго, яко Отца благоутробнаго и насъ любящаго. Отцу плотскому послушаніе показываемъ и опасаемся его прогнѣвать: какъ не убоимся ослушанія являть Богу, небесному и вѣчному Отцу, Который создалъ насъ, и въ Сынѣ Своемъ единородномъ отродилъ насъ, и толико благодѣяній являетъ намъ, которыхъ и умомъ понять не можемъ? Который отецъ такъ любитъ сына, какъ любитъ Богъ насъ? Скоро человѣческая любовь претворяется въ гнѣвъ и ненависть, но Божія любовь не тако; она не престаетъ никогда. Мы престаемъ и измѣняемъ, оскудѣваемъ и не хранимъ вѣрности и любве къ Нему: а Онъ всегда \textit{тойжде есть}\footnote{Евр.~1,~12.}, всегда \textit{сіяетъ солнце Свое на злыя и благія, и дождитъ на праведныя и на неправедныя}\footnote{Матѳ.~5,~45.}. "--- 5)~Понеже истинный \textit{страхъ Божій есть даръ отъ Господа}, якоже Сирахъ глаголетъ\footnote{Сир.~1,~13.}: того ради должно его намъ усердною молитвою у Бога просить, чтобы сердце наше и вся чувства страхомъ Своимъ оградилъ.

\paragraph*{§\:182.} Какъ страхъ Божій отвращаетъ человѣка отъ всякаго зла, такъ безстрашіе отворяетъ путь ко всякому злу и беззаконію. Оттуду хулы, клятвопреступленія, мздоиманія, лихоиманія, насилія, хищенія, воровство, сквернословія, безчинные кличи, насмѣшки, осужденія, клеветы, коварства, лести, обманы, всякая нечистота, безчинія, "--- словомъ, всякое беззаконіе и безбожіе. Какъ бо конь свирѣпый, не имѣя правителя, стремится и бѣжитъ, куды хощетъ и глаза смотрятъ: такъ человѣкъ, отъ природы ко всякому злу склонный, когда лишится страха Божія, аки добраго правителя, на вся злая устремляется, и куды злая воля ни повелѣваетъ ити ему, туды и идетъ, и что ни похощетъ, тое и дѣлаетъ, отъ зла во зло, отъ грѣха въ грѣхъ, и отъ беззаконія въ беззаконіе падаетъ; мыслитъ зло, и говоритъ зло; начинаетъ зло, и совершаетъ зло; и, какъ по ступенямъ, по различнымъ беззаконіямъ во глубину золъ низходитъ, и тако, \textit{пришедъ во глубину золъ, нерадитъ}. Тогда уже ни увѣщательное, ни угрозительное слово ему недѣйствительно: на едино тое только мыслію и дѣломъ стремится, на что злая воля и привычка позываетъ. Бѣдное и плача достойное таковаго человѣка, какого бы онъ званія ни былъ, состояніе! Ибо отъ сего не ино что слѣдуетъ, какъ вѣчная погибель, аще особливый милосердаго и нехотящаго смерти грѣшнику Бога промыслъ оттуду, какъ Іону отъ чрева китова, не извлечетъ.

\subsection[Глава 2-я. О смиреніи.]{глава вторая.\\\bfseries О смиреніи.}

\begin{quotation}\textit{Всякъ возносяйся, смирится: смиряяй же себе вознесется}\footnote{Лук.~18,~14.}.\end{quotation}
\begin{quotation}\textit{Смиритеся предъ Господемъ, и вознесетъ вы}\footnote{Іак.~4,~10.}.\end{quotation}
\begin{quotation}\textit{Научитеся отъ Мене, яко кротокъ есмь и смиренъ сердцемъ}, глаголетъ Хрістосъ\footnote{Матѳ.~11,~29.}.\end{quotation}
\begin{quotation}Страху Божію послѣдуетъ истинное смиреніе, \textit{яко} богобоящійся \textit{смиряется подъ крѣпкую руку Божію}\footnote{1~Петр.~5,~6.}.\end{quotation}


\paragraph*{§\:183.} Коликою гордостію сердце человѣческое заражено, всякъ можетъ лучше примѣтить на другомъ, нежели на себѣ. Ибо пороки удобѣе познаемъ въ ближнихъ нашихъ, нежели въ себѣ; и болѣе очи наши обращаемъ на чужіе, нежели на свои грѣхи. Но что въ ближнемъ человѣкъ видитъ, тое имѣетъ и внутрь себе. Видимъ, какъ"=то человѣкъ ищетъ надъ другимъ господствовать, приказывать, повелѣвать, отъ другихъ лучшимъ показаться, прославиться и превознестися, отъ всѣхъ почитаемымъ и покланяемымъ быть: къ сему колико посредствій употребляетъ, изчислить почти неможно. Иный самовольно подвергаетъ себе подъ трескъ и звукъ смертоноснаго непріятельскаго оружія, чтобы побѣдителемъ нарицаться и въ рангъ доступить. Иный школы и различныя науки проходитъ, дабы или разумнымъ и премудрымъ почитаться, или на высокомъ мѣстѣ сидѣть. Иный изъ млада указамъ, правамъ и всякимъ приказнымъ дѣламъ навыкаетъ, чтобы судіею слыть. Иный богатства ищетъ, и тѣмъ хощетъ отличнымъ себе паче прочіихъ показать. Другій созидаетъ богатые покои, дабы съ почтенными и высокими лицами знаться, и тако самому какъ бы такимъ именоваться. Инаго попеченіе о платьѣ и убранствѣ цвѣтномъ, чтобы тако въ собраніи не на послѣднемъ мѣстѣ сидѣть и отъ срѣтающихся почтеніе и поклонъ имѣть. Иному на умъ приходитъ, высокія и богатыя кареты, избранные и убранные кони со множествомъ слугъ приготовлять, дабы знатнымъ и чиновнымъ показаться. У инаго богатый столъ часто набирается, чтобы отъ питающихся честь и похвалу получать. Иный иная замышляетъ къ прославленію имене своего. Всѣмъ симъ неполезнымъ, паче же вреднымъ замысламъ корень есть гордость житейская. Но кто отъ смертоноснаго сего зла начнетъ благодатію Божіею уклоняться, и, оставивши тварь, Творца искать, сего другое злѣйшее зло срѣтаетъ "--- духовная гордость и фарисейское высокоуміе. Сія всепагубная язва въ тѣхъ наипаче обыкаетъ гнѣздиться, которые много постятся, много даютъ милостыни, какъ"=то фарисей оный на себѣ показалъ\footnote{Лук.~18,~10 и 11.}; такожде которые удаляются *въ пустыни, заключаются* въ монастыри, одѣваются мантіями, часто и много молятся, и прочія, по видимому, нехудыя дѣла дѣлать тщатся. Тако бѣдному человѣку вездѣ сія ехидна присѣдитъ и ищетъ ядомъ своимъ его умертвить! "--- Противу сего зла врачевство предлагаетъ намъ Сынъ Божій "--- смиреніе Свое, и велитъ отъ Себе того учитися: \textit{научитеся отъ Мене, яко кротокъ есмь и смиренъ сердцемъ}.

\paragraph*{§\:184.} Смиреніе раждается 1)~отъ разсужденія своего недостоинства и другаго достоинства. Такое смиреніе показалъ на себѣ сотникъ въ Капернаумѣ, который ко Хрісту, Спасителю міра, сказалъ\textit{: Господи, нѣсмь достоинъ, да подъ кровъ мой внидеши}\footnote{Матѳ.~8,~7 и 8.}, "--- такожде Предтеча святый, который, Духомъ Святымъ провидя Хрістово величество, и свою предъ Нимъ подлость разсуждая, исповѣдалъ: \textit{грядетъ крѣплій мене, Емуже нѣсмь достоинъ отрѣшити ремень сапогу Его}\footnote{Лук.~3,~16.}. "--- 2)~Бываетъ отъ разсужденія бѣдности и окаянства своего. Тако смирился мытарь, который, множествомъ грѣховъ отягчаемый, не хотѣлъ и очесъ возвести на небо, но біяше въ перси своя, глаголя: \textit{Боже, милостивъ буди мнѣ грѣшнику}\footnote{18,~13.}. "--- 3)~Происходитъ еще смиреніе отъ страха, когда предъ кѣмъ либо разгнѣваннымъ и огорченнымъ смиряемся, боясь послѣдующія казни. Тако смирились Ниневитяне, услышавше отъ проповѣди пророка Іоны грядущую на градъ ихъ Божію казнь. \textit{И вѣроваша мужіе Ниневійстіи Богови, и заповѣдаша постъ, и облекошася во вретища отъ велика ихъ даже до мала ихъ}\footnote{Іон.~3,~5.}. Тако и прочіе смиряются грѣшники, когда сердецъ ихъ коснется Богъ страхомъ Своимъ; и такъ бываетъ \textit{начало премудрости страхъ Господень}\footnote{Пс.~110,~10.}. "--- 4)~Бываетъ смиреніе отъ любви, когда единъ предъ другимъ смиряется, не боясь суда и казни, но жалѣя, что его или самъ оскорбилъ, или другой кто, за котораго смиряющійся ходатайствуетъ, и тѣмъ своимъ смиреніемъ желая его удовольствовать и оскорбленіе наградить. Тако и самыя высокія лица, страху человѣческому неподлежащія, предъ подлыми смиряются, которыхъ напрасно оскорбили. Пріятное и любезное зрѣлище, когда высочество такимъ духомъ внизъ низходитъ, и достоинство предъ подлостію склоняется! Князь и вельможа не стыдится просить прощенія у раба! Плотскимъ людямъ и сынамъ вѣка сего сіе снисхожденіе буйство быть кажется; но хрістіанамъ истиннымъ есть истинная премудрость, которые подражаютъ въ семъ дѣлѣ Хрісту, Спасителю своему, \textit{Божіей силѣ и Божіей премудрости}\footnote{1~Кор.~1,~24.}. Онъ, будучи Сынъ Вышняго и Господь славы, не устыдился \textit{зракъ раба пріяти, смиритися, послушливымъ быть даже до смерти, смерти же крестныя}\footnote{Филип.~2,~7 и 8.}. Къ сему глубочайшему и нашимъ умомъ непостижимому смиренію убѣдило Его не иное что, какъ \textit{любовь} къ небесному Своему Отцу, Который гордостію и непослушаніемъ человѣческимъ разгнѣванъ былъ и оскорбленъ, "--- и \textit{сожалѣніе} къ человѣку, который ради преслушанія отъ лица Божія отверженъ былъ. Такимъ самопроизвольнымъ смиреніемъ правду Божію раздраженную удовольствовалъ, и человѣка, отъ милости Божіей отпадшаго, паки въ высочайшую Его милость привелъ. Въ сіе ясное смиренія зерцало намъ всегда взирать должно, и предъ Богомъ, Котораго грѣхами оскорбили мы, смириться, и съ пророкомъ сердцемъ и устами исповѣдаться: \textit{Тебѣ, Господи, правда, намъ же стыдѣніе лица}\footnote{Дан.~9,~7.}, хотя бы геенны и муки не было. Ибо болѣе всякія муки быть должно намъ тое едино, когда видимъ отъ насъ прогнѣваннаго Того, Который есть вѣчная любовь, Который насъ по образу Своему создалъ, и такъ чудно о насъ промышляетъ.

\paragraph*{§\:185.} Причины, приводящія человѣка къ смиренію. 1)~Бѣдность и окаянство наше убѣждаетъ насъ смирятися. Возми всякъ себе въ разсужденіе, и возъимѣешь довольную причину смирятися; приникни во глубину сердца твоего, и увидишь, какое тамо зло и бѣдствіе крыется. Сіе зло злыми помыслами, какъ злое древо злыми своими плодами, себе оказываетъ. \textit{Извнутрь бо, отъ сердца человѣческа помышленія злая исходятъ, прелюбодѣянія, любодѣянія, убійства, татьбы, лихоимства, лукавствія, лесть, студодѣянія, око лукаво, хула, гордыня, безумство}, глаголетъ Хрістосъ\footnote{Марк.~7,~21 и 22.}. А что внутрь, въ сердцѣ кроется, тое внѣ, какъ изъ котла кипящаго, что вмѣститься не можетъ, чрезъ внѣшнія дѣянія извергается. Отсюду происходятъ ссоры, вражды, проклинанія, злословія, хулы, лести, обманы, насмѣшки, укоренія; отсюду гнѣва плоды: мщеніе, зловоздаяніе, кровопролитіе, убійство; отсюду хищеніе, лихоиманіе, клятвопреступленіе, обида бѣдныхъ, проліяніе слезъ вдовицъ и сиротъ. Сердце извергаетъ смрадные запахи и плоды свои "--- студодѣянія, блудодѣянія, прелюбодѣянія, сквернословія, срамословія, слова и дѣянія студныя, шутки безстыдныя, сладострастія скотскія; словомъ сказать, всякая мерзость и всякое зло отъ сердца исходитъ, какъ изъ земли посѣянное сѣмя возникаетъ; и дѣломъ самымъ исполняется, когда силою страха Божія не воспящается. Но, "--- что человѣческое окаянство умножаетъ, "--- человѣкъ сего зла и не познаетъ, пока Богъ благодатію Своею не просвѣтитъ его. Многіе нѣчто о себѣ мечтаютъ, противу которыхъ апостолъ глаголетъ: \textit{аще кто мнитъ себе быти что, ничтоже сый, умомъ льститъ себе}\footnote{Гал.~6,~3.}. Но таковые, когда благодатію Божіею осмотрятся, увидятъ, что ничтоже суть, и признаетъ всякъ: \textit{заблудихъ, яко овча погибшее}\footnote{Пс.~118,~176.}. И сіе бѣдствіе и окаянство всѣмъ человѣкамъ общее есть. Сіе должно намъ познавать и признавать, и тако смиряться. "--- 2)~Къ сему бѣдствію другое не меньшее присовокупляется "--- многоразличная хитрость и вражда діавольская. Сей врагъ, какъ \textit{человѣкоубійца бѣ искони}\footnote{Іоан.~8,~44.}, и на сердцѣ прародителя нашего злое свое сѣмя (о которомъ въ 1"~мъ пунктѣ сказано), посѣялъ, не престаетъ и нынѣ плевелы свои разсѣвать. Вездѣ онъ съ коварствомъ своимъ окружаетъ насъ: что ни дѣлаемъ, что ни бесѣдуемъ, примѣчаетъ, и во всемъ хощетъ запяти и озлобити насъ. Спимъ ли, или почиваемъ? онъ не спитъ, но бодрствуетъ на погибель нашу. Ходимъ ли, или сѣдимъ? онъ наблюдаетъ хожденіе, сѣданіе и востаніе наше. Ядимъ ли, или піемъ? подлагаетъ козни свои. Къ молитвѣ ли обращаемся? и тутъ не оставляетъ насъ, и тутъ препятствуетъ намъ, тщится разслабити насъ, отвратити умъ нашъ и сердце отъ Бога. Отъ зла ли уклоняемся? тщится паки въ тое вринути насъ. Доброе ли дѣлаемъ? подвизается и тутъ, чтобы не съ добрымъ намѣреніемъ и не на добрый конецъ дѣлалось отъ насъ. «Се, глаголетъ Августинъ, распростерлъ предъ ногами нашими безчисленныя сѣти, и всѣ пути наши различными наполнилъ обманами, дабы уловить души наша. Сѣти положилъ въ богатствѣ, сѣти положилъ въ нищетѣ, сѣти простерлъ въ пищѣ, въ питіи, въ роскоши, во снѣ и бдѣніи; сѣти простерлъ въ словѣ, въ дѣлѣ и во всемъ житіи нашемъ»\footnote{Soliloqu. гл.~16"~я.}. Сего ради апостолъ святый предостерегаетъ насъ, глаголя: \textit{трезвитеся, бодрствуйте, зане супостатъ вашъ діаволъ, яко левъ рыкая, ходитъ, искій кого поглотити}\footnote{1~Петр.~5,~8.}. Сего злоковарнаго врага сѣти и козни усматриваетъ и отъ нихъ уклоняется едино смиреніе. Ибо \textit{смиреннымъ Богъ даетъ благодать}\footnote{Іак.~4,~6.}, которая ихъ предостерегаетъ, просвѣщаетъ и сохраняетъ отъ коварныхъ сѣтей его. "--- 3)~Безъ благодати Божіей ничего благоугоднаго творити не можетъ человѣкъ. \textit{Богъ есть дѣйствуяй въ насъ, и еже хотѣти и еже дѣяти о благоволеніи}, глаголетъ Апостолъ\footnote{Филип.~3,~13.}. И Хрістосъ глаголетъ: \textit{безъ Мене не можете творити ничесоже}\footnote{Іоан.~15,~5.} (что и вышереченныя въ 1"~мъ и во 2"~мъ пунктѣ причины показуютъ). Человѣкъ бо въ духовныхъ дѣлахъ безъ Божіей благодати есть какъ изсохшая вѣтвь, которая никакого плода творити не можетъ. \textit{Якоже розга}, глаголетъ Хрістосъ, \textit{не можетъ плода сотворити о себѣ, аще не будетъ на лозѣ: тако и вы, аще во Мнѣ не пребудете}\footnote{ст.~4.}. Богу убо, дѣйствующему и хотѣніе и дѣланіе доброе, подобаетъ всякая хвала и слава; а человѣку стыдѣніе лица и смиреніе приличествуетъ, ибо не токмо дѣлать, но и хотѣть безъ Бога ничего добраго не можетъ. «Аще какое добро малое, или великое есть, глаголетъ помянутый Августинъ, Твое дарованіе, Господи; наше же только зло. О чемъ убо похвалится всякая плоть? развѣ о злѣ? Сіе не есть похвала, но бѣдность. Твое, Господи, есть добро, Твоя есть слава! Понеже кто отъ Твоего добра славы себѣ ищетъ, сей тать есть и разбойникъ, и подобенъ діаволу, который хотѣлъ похитить славу Твою. Ибо кто похвалы ищетъ отъ Твоего дарованія, и не ищетъ отъ того Твоей славы, но своей, "--- сей хотя ради Твоего дарованія и хвалится отъ человѣкъ, но отъ Тебе охуждается, понеже отъ Твоего дарованія не Твоей, но своей славы искалъ. А кто отъ человѣкъ хвалится, отъ Тебе же охуждается, тотъ не защитится отъ человѣкъ, когда будешь судить, и не избавится, когда Ты осудишь». И мало ниже: «Господи! исповѣдую, якоже научилъ Ты мене, яко не иное что есмь, какъ всякая суета, сѣнь смертная, и бездна нѣкая темная, и земля праздная, которая безъ Твоего благословенія ничего не прозябаетъ, и плода не творитъ, кромѣ стыдѣнія, грѣха и смерти. Аще какое добро имѣлъ я когда, отъ Тебе пріялъ; какое добро ни имѣю, *Твое есть и отъ Тебе имѣю*. Аще когда стоялъ я, Тобою стоялъ; аще когда падалъ, собою падалъ, и всегда бы въ блатѣ лежалъ, аще бы Ты мене не воздвиглъ; всегда бы слѣпъ былъ, когда бы Ты мене не просвѣтилъ. Когда палъ я, никогда бы не восталъ, аще бы Ты мнѣ руки Своея не простерлъ; но и когда Ты мене воздвиглъ, всегда бы падалъ, ежели бы Ты мене не поддерживалъ; частѣе бы погибалъ, ежели бы Ты мене не управлялъ. Тако всегда, Господи, тако всегда благодать Твоя и милость предваряетъ мя, которая избавляетъ мя отъ всѣхъ золъ, спасаетъ отъ прешедшихъ, возставляетъ отъ настоящихъ, и предваряетъ отъ будущихъ, пресѣкаетъ и сѣти грѣховныя предо мною, отнимаетъ случаи и причины. Понеже ежели бы Ты сего добра мнѣ не сотворилъ, то бы я вси грѣхи міра сотворилъ»\footnote{Soliloqu. гл.~15"~я.}. Тако блаженный Августинъ о себѣ исповѣдуетъ; тако научаетъ премудрый учитель и насъ о себѣ исповѣдовать и со смиреніемъ признавать немощь свою и окаянство, и "--- отъ зла ли когда уклоняемся, или доброе дѣлаемъ, благодати и милости Божіей тое все восписовать! "--- 4)~\textit{Смиреннымъ}, какъ Писаніе глаголетъ, \textit{Господь даетъ благодать}. Какъ воды обыкновенно съ высокихъ горъ на низкія мѣста стекаютъ, тако рѣки дарованій Божіихъ на удолія смиренныхъ сердецъ низпущаются. И какъ сосудъ праздный все удобно вмѣщаетъ, тако сердце отъ суеты мірской и отъ гордости изпраздненное, удобно есть къ воспріятію дарованій Божіихъ. Богъ бо милостивно и человѣколюбно зритъ на смиренное сердце, якоже Пресвятая Богородица глаголетъ: \textit{призрѣ на смиреніе рабы Своея}\footnote{Лук.~1,~48.}. Богъ, яко богатъ въ милости и щедротахъ, всѣмъ хощетъ подать благодать Свою; но, понеже не во всѣхъ находитъ удобное къ воспріятію сердце, того ради не всѣмъ подаетъ, но тѣмъ только, которые \textit{нищіи} суть \textit{духомъ} и признаютъ нищету свою, \textit{алчутъ и жаждутъ правды Его}\footnote{Матѳ.~5,~5 и 6.}. Какъ въ началѣ Богъ сотворилъ небо и землю, такъ дивная дѣла \textit{изъ ничего}, такъ тойжде Богъ и нынѣ все \textit{изъ ничего} творитъ. Кто мнитъ себе ничтоже, изъ того нѣчто, яко всемогущій, содѣловаетъ. Отсюду бываетъ, что смиренный невѣжда и земледѣлецъ "--- разумнѣйшій и мудрѣйшій въ званіи хрістіанскомъ, хотя красныхъ рѣчей и писменъ не знаетъ, нежели гордый міра сего мудрецъ. Сего ради апостолъ глаголетъ: \textit{никтоже себе да прельщаетъ: аще кто мнится мудръ быти въ васъ въ вѣцѣ семъ, буй да бываетъ, яко да премудръ будетъ}\footnote{1~Кор.~3,~18.}. "--- 5)~Якоже на смиренное сердце изливаются дарованія Божія, тако въ смиренномъ сердцѣ, яко въ сокровищи нѣкоемъ, сохраняются. Ибо таковое сердце благодать и страхъ Божій окружаетъ, и, какъ вѣрный стражъ, хранитъ его. Не тако надменная гордость, но, что и имѣетъ, погубляетъ. Тако прародитель нашъ въ раи, когда восхотѣлъ богомъ быть, потерялъ и тое, что имѣлъ отъ Бога данное; восхотѣлъ богомъ быть, но \textit{приложися скотомъ несмысленнымъ и уподобися имъ}\footnote{Пс.~48,~21.}, котораго зла и сыны его причастившеся, оплакати довольно не можемъ. Смиреніе есть какъ основаніе нѣкое, на которомъ духовную добродѣтелей храмину должно намъ созидать. Смиренный безъ сумнѣнія уподобляется мужу мудру, \textit{иже созда храмину свою на камени; и сниде дождь, и пріидоша рѣки, и возвѣяша вѣтри, и нападоша на храмину ту, и не падеся: основана бо на камени}\footnote{Матѳ.~7,~24 и 25.}. Смиреніе бо придержится вѣрою Того, Который все въ руцѣ Своей содержитъ; и того ради твердое есть основаніе. Безъ сего основанія, все, что ни созиждется духовное, падется и въ прахъ обратится. Фарисей думалъ, яко онъ создалъ храмину свою, но, понеже твердаго сего основанія не положилъ, обманулся, какъ пишется во Евангеліи\footnote{Лук.~18,~11,~12 и 14.}. Паче же безъ смиренія и наздать ничего духовнаго невозможно, ибо, какъ сказано выше, безъ благодати Божіей ничего духовнаго сотворить не можемъ. \textit{Зане гордымъ Богъ противится: смиреннымъ же даетъ благодать}\footnote{1~Петр.~5,~5.}. "--- 6)~Безъ смиренія истинное покаяніе быть не можетъ, но есть притворное и ложное, которое только на устахъ, а не на сердцѣ имѣется, ибо какъ больному, который хощетъ исцѣлитися, должно первѣе признать свою немощь и лѣкарю объявить: тако грѣшнику, который душею немоществуетъ, должно вопервыхъ душевную свою признать немощь, бѣдность и окаянство, признать себе, яко святаго закона Божія преступника, предъ судомъ Божіимъ виноватымъ, временнаго и вѣчнаго Божія наказанія достойнымъ, и съ таковымъ исповѣданіемъ прибѣгать вѣрою ко Хрісту, душъ и тѣлесъ Врачу, и повергать себе духовно предъ пречистыми Его ногами, по подобію евангельскія блудницы\footnote{Лук.~1,~37 и 38.}, что безъ смиренія быть не можетъ. Ибо самое таковое сердечное бѣдности и окаянства признаніе есть знакъ смиренія. Такое смиреніе показалъ блудный сынъ, когда призналъ свое недостоинство: \textit{Отче! согрѣшихъ на небо и предъ Тобою, и уже нѣсмь достоинъ нарещися сынъ Твой}\footnote{15,~21.}. Тако смирился мытарь, когда стоялъ издалеча, когда не хотѣлъ ни очесъ возвести на небо, но біяше въ перси своя, глаголя: \textit{Боже, милостивъ буди мнѣ грѣшнику}\footnote{18,~15.}. "--- 7)~Безъ смиренія неполезна молитва бываетъ, ибо \textit{гордымъ Богъ противится}. Напротивъ того, на смиренныхъ милостивно призираетъ Богъ: \textit{призрѣ на молитву смиренныхъ, и не уничижи моленія ихъ}, глаголетъ Псаломникъ\footnote{Пс.~101,~18.}. Тако призрѣлъ на молитву смиреннаго мытаря, хотя грѣхами и обремененъ былъ, "--- якоже отринулъ гордое самохвальство фарисейское. \textit{Яко, всякъ возносяйся, смирится: смиряяй же себе, вознесется}\footnote{Лук.~18,~14.}. Тако сотникъ въ Капернаумѣ, который вмѣнялъ недостойна себе, чтобы въ домъ его Хрістосъ вошелъ, получилъ желаемое, да еще съ похвалою: \textit{ни во Израили толики вѣры обрѣтохъ}, глаголетъ Хрістосъ\footnote{Матѳ.~8,~10.}. Тако вѣрная Хананея, которая не отреклась подобна быть псомъ, ядущимъ отъ крупицъ, падающихъ отъ трапезы господей своихъ, слышитъ отъ Хріста: \textit{о жено! велія вѣра твоя: буди тебѣ, якоже хощеши: и исцѣлѣ дщи ея отъ того часа}\footnote{15,~28.}. Ибо гдѣ вѣра истинная, тамо и смиреніе есть, смиреніе бо отъ вѣры неотлучно. Съ такою вѣрою и смиреніемъ и намъ должно приступать къ величеству Божію, когда хощемъ чего просить и просимое получить, "--- помнить, кто мы и къ кому съ прошеніемъ приступаемъ. "--- 8)~Къ смиренію приводитъ насъ начало и конецъ нашъ: \textit{изъ земли взяты, въ землю и обращаемся, яко земля есмы, и въ землю отъидемъ}\footnote{Быт.~2,~7; 3,~19.}. Аще бо и утѣшаетъ насъ вѣра наша, яко паки душа съ тѣломъ въ свое время соединится Божіею силою, и въ иный краснѣйшій видъ Божіею благодатію облечемся: обаче всякому слѣдуетъ \textit{единою умрети}, въ нѣдрѣ земли, отъ неяже взяты, сокрытися и до общаго воскресенія почити\footnote{Евр.~9,~27.}. А сія мертвость и тлѣніе приводитъ намъ на память и тое, что насъ въ сіе плачевное состояніе грѣхъ привелъ, хотя надежда несумнѣнная питаетъ и укрѣпляетъ сердца наша, яко \textit{во Хрістѣ животъ нашъ сокровенъ есть}\footnote{Гал.~3,~3.}. О семъ концѣ, разрѣшеніи, тлѣніи и разсыпаніи тѣла нашего размышленіе не допуститъ насъ надыматися и паче прочіихъ возноситися: яко всѣмъ, высокимъ и низкимъ, господамъ и рабамъ, богатымъ и нищимъ, слѣдуетъ по кончинѣ единъ домъ "--- земля, и въ смрадъ и въ землю тѣломъ всякому обратитися и разсыпатися. "--- 9)~Кто смиренія не имѣетъ и не тщится имѣть, тому опасаться должно, чтобы съ діаволомъ, начальникомъ гордости, не пасть, и съ нимъ вѣчно не быть отверженнымъ отъ милости Божіей. \textit{Гордымъ бо Богъ противится}. Какъ бо смиренныхъ хотя низкій путь есть, но къ высокому отечеству "--- небу ведетъ: тако гордые, хотя высоко подымаются и летаютъ, но наконецъ внизъ, то"=есть, во адъ низвергаются. Сего низверженія всякому высокоумному должно боятися. "--- 10)~Хрістосъ Сынъ Божій, хотя и всѣхъ добродѣтелей есть образъ намъ и зерцало, однакожъ смиренія и кротости у Себе учитися намъ повелѣваетъ: \textit{научитеся отъ Мене, яко кротокъ есмь и смиренъ сердцемъ}\footnote{Матѳ.~11,~29.}. Отсюду видимъ, коль великая есть добродѣтель "--- смиреніе: яко начало свое имѣетъ не отъ инаго кого, но отъ Хріста, Царя небесе и земли. \textit{Научитеся отъ Мене}, глаголетъ, не мертвыхъ воскрешать и прочія чудеса творить; но чего? \textit{яко кротокъ есмь и смиренъ сердцемъ}. Аще убо Самъ Господь небесе и земли былъ \textit{смиренъ сердцемъ}, какъ исповѣдуетъ; аще \textit{смирилъ Себе до смерти крестныя}\footnote{Филип.~2,~8.}; аще не устыдился \textit{ноги ученикамъ умыть}\footnote{Іоан.~13,~5.}; аще свидѣтельствуетъ Самъ о Себѣ, яко \textit{Сынъ человѣческій не пріиде, да послужатъ Ему, но послужити}\footnote{Матѳ.~20,~28.}; аще глаголетъ: \textit{Азъ посредѣ васъ} (учениковъ) \textit{есмь служай}\footnote{Лук.~22,~27.}: не много ли паче намъ "--- рабамъ, по примѣру Господа нашего, подобаетъ смиритися и не стыдитися братіи своей служити, и съ ними, какіе бы ни были, обращатися дружески; на сей образъ взирали святіи апостоли и вси святіи, и отъ него учились, и тако низкимъ смиренія путемъ въ высокое отечество "--- небо вошли.

\paragraph*{§\:186.} Смиреніе не токмо внѣ показывать, но наипаче внутрь должно стараться имѣть. Суть такіе, которые внѣ показуются смиренными, но внутрь того не имѣютъ. Многіе отлагаютъ чины и титулы міра сего, но не хотятъ отложить высокаго о себѣ мнѣнія; отрицаются чести и сана мірскаго, но хотятъ почитаться ради святости. Многіе не стыдятся называть себе предъ людьми грѣшниками, или, что еще болѣе, паче всѣхъ грѣшнѣйшими, но отъ другихъ того слышать не хотятъ, и потому устами только таковыми себе нарицаютъ. Иные, какъ серпъ, сляченную выю носятъ, но внутрь умъ возносятъ. Другіе низкіе поклоны братіи своей отдаютъ, но сердцемъ непреклонны бываютъ. Иный въ раздранномъ рубищѣ ходитъ, но сердца раздрать не хочетъ. Многіе мало и тихо, а иные и совсѣмъ не говорятъ, но сердцами безпрестанно ближнихъ порочатъ. Иные черною рясою и мантіею тѣло покрываютъ, но сердца покрыть не хотятъ. Тако и прочіе знаки смиренія показуютъ!.. Вси таковые смиренія на сердцѣ не имѣютъ. Могутъ сіи знаки быть примѣтами смиренія; но когда нѣтъ того, что значатъ, то не иное что суть, какъ лицемѣріе. И суть подобны таковые мѣху, воздухомъ надутому, который кажется, что нѣчимъ наполненъ, но, когда воздухъ выпустится, показуется, что пусть; или паче, по словеси Хрістову, подобни \textit{гробамъ повапленнымъ, иже внѣуду убо являются красны, внутрьуду же полны суть костей мертвыхъ и всякія нечистоты}\footnote{Матѳ.~23,~27.}. Сего ради смиреніе, какъ и всякое благочестіе въ сердцѣ должно имѣть. Ибо Богъ судитъ по \textit{совѣту сердечному}\footnote{1~Кор.~4,~5.}, а не по наружности, какъ предъ людьми являемся.

\paragraph*{§\:187.} Знаки нѣкоторые истиннаго смиренія здѣ полагаются, которое не токмо внѣ, но и внутрь, въ сердцѣ имѣется. 1)~Познать истинное свое окаянство и бѣдность, и сердцемъ признавать, которое окаянство всѣмъ намъ общее, хотя то мало кто познаваетъ и признаваетъ; почему немногіе и смиреніе истинное имѣютъ. "--- 2)~Не презирать никого, и самаго подлѣйшаго, не осуждать не токмо словомъ, но ниже мыслію никого, но свое окаянство и зло смотрѣть и оплакивать, и тако милости у всемогущаго Бога просить. "--- 3)~Чести и славы не искать, и, хотя подавается, недостойна себе вмѣнять не токмо устами, но и самою вещію; и, ежели нужно быть въ чести, жалѣти о томъ и съ подвластными обходитися, какъ съ братіею своею, не лучша себе паче ихъ имѣти, пользы ихъ искати, а не почтенія и поклоновъ отъ нихъ, и, хотя нужно будетъ кого наказывать и къ страху приводить, въ сердцѣ мнить, что и самъ не лучшій отъ нихъ. "--- 4)~Терпѣть безъ роптанія и охотно всякое презрѣніе и безчестіе; смиреніе бо безъ терпѣнія быть не можетъ, и гдѣ терпѣніе истинное, тамо и смиреніе. Ибо кто не терпитъ презрѣнія, тотъ любитъ почитаніе и похвалу, что знакъ есть гордости. "--- 5)~Доброхотно и усердно повиноваться и слушать не токмо высшихъ, но и равныхъ и меньшихъ въ нуждахъ ихъ и требованіяхъ: смиреніе бо всѣмъ склоняется, какъ и любовь. "--- 6)~Съ низшими обходиться такъ, какъ съ равными себѣ, не почитать себе за высшаго и лучшаго отъ нихъ, но помнить, что вси тогожде естества люди, и сановитые и подлые. "--- 7)~Окаяннѣйшаго себе имѣть и грѣшнѣйшаго паче прочіихъ. Ибо смиреніе когда свои только пороки смотритъ, то и находитъ себе достойнѣйшаго осужденія паче другихъ. "--- 8)~Въ истинномъ смиреніи примѣчается непрестанная алчба и жажда Божіей благодати: смиреніе бо не смотритъ на тое, что имѣетъ, но разсуждаетъ и ищетъ, чего не имѣетъ. Какъ бо учащіися отъ книгъ и учителей, чимъ болѣе учатся и навыкаютъ, тѣмъ болѣе невѣжество свое видятъ, потому что далеко болѣе не знаютъ, нежели научилися: тако учащіися въ школѣ премудрости Божіей, тѣмъ убогшими духовно себе познаютъ, чѣмъ болѣе причащаются дарованій Божіихъ, ибо видятъ, что многаго не имѣютъ, почему и ищутъ со смиреніемъ и воздыханіемъ.

\paragraph*{§\:188.} Какимъ образомъ смиреніе искать, здѣ вкратцѣ прилагается. 1)~Тщаться познать себе, свою бѣдность, немощь и окаянство, и тую немощь почаще предъ душевными очесами полагать. 2)~О величествѣ Божіемъ и о своей подлости разсуждать. 3)~О смиреніи Хрістовомъ, Котораго какъ любовь къ намъ, такъ и смиреніе ради насъ такъ велико, что и умомъ понять невозможно, размышлять съ прилѣжаніемъ, "--- что Евангеліе святое тебѣ представляетъ. 4)~Не смотрѣть того, какое имѣешь добро, но какого еще не имѣешь. 5)~Помнить преждебывшіе грѣхи. «Воспоминаніе, глаголетъ святый Златоустъ, прежнихъ грѣховъ есть довольная узда усмирить и уцѣломудрить насъ»\footnote{Бес.~31"~я на Быт.}. А притомъ и тое рассуждать, что еще болѣе и тяжчае можешь согрѣшить, когда благодать Божія не поможетъ тебѣ, какъ Августинъ исповѣдуетъ. 6)~Какое добро сдѣлалъ, Богу тое приписывать и благодарить, а не за свое вмѣнять. 7)~Когда видишь или слышишь брата согрѣшающа, то не грѣхъ только его разсуждай, но и что прочее дѣлаетъ или дѣлалъ, добрѣ внимай, и такъ увидишь, что онъ лучшій тебе, глаголетъ Василій \textit{(въ словѣ о смиреніи)}. Ктомужъ отъ его падежа своего берегись, и свои грѣхи на умъ приведи. Ибо общее зло есть человѣческое падать. Читай еще главу о гордости, и сноси едино съ другимъ ради лучшаго понятія. Ибо что отъ гордости отводитъ, тое приводитъ къ смиренію, которое всегда противно есть гордости.

\subsection[Глава 3-я. О презрѣніи и отрицаніи міра.]{глава третія.\\\bfseries О презрѣніи и отрицаніи міра.}

\begin{quotation}\textit{Кая польза человѣку, аще міръ весь пріобрящетъ, душу же свою отщетитъ? или что дастъ человѣкъ измѣну за душу свою}? глаголетъ Хрістосъ\footnote{Матѳ.~16,~26.}.\end{quotation}
\begin{quotation}\textit{Не сообразуйтеся вѣку сему, но преобразуйтеся обновленіемъ ума вашего, во еже искушати вамъ, что есть воля Божія, благая и угодная и совершенная}\footnote{Римл.~12,~2.}.\end{quotation}


\paragraph*{§\:189.} Міръ въ святомъ Писаніи не въ единомъ разумѣ берется. 1)~Міръ разумѣется за весь небесный и земный кругъ, въ которомъ всѣ созданныя вещи заключаются. Тако разумѣется оное слово: \textit{въ мірѣ бѣ, и міръ тѣмъ бысть}\footnote{Іоан.~1,~10.}. (О созданіи сего міра пишется въ главѣ 1 и 2"~й книги Бытія). 2)~Міръ означается за лучшую міра сего часть, то"=есть, человѣка. О семъ мірѣ глаголетъ Хрістосъ: \textit{тако возлюби Богъ міръ}\footnote{3,~16 и 17.} и проч. 3)~Міръ берется за все, что въ мірѣ семъ услаждаетъ и увеселяетъ нашу плоть, разжигаетъ похоть ея, и отъ Хріста отводитъ и ведетъ къ вѣчной погибели. О семъ глаголетъ апостолъ: \textit{все, еже въ мірѣ, похоть плотская, и похоть очесъ, и гордость житейская, нѣсть отъ Отца, но отъ міра сего есть}\footnote{1~Іоан.~2,~16.}. 4)~Міръ пріемлется за людей, которые, своимъ прихотямъ работая, не хотятъ истины евангельской принять и послѣдовать. Сей міръ означаетъ Хрістосъ, глаголя апостоламъ: \textit{аще міръ васъ ненавидитъ, вѣдите, яко Мене прежде васъ возненавидѣ}\footnote{Іоан.~15,~18.}.

\paragraph*{§\:190.} Міра отрещися 1)~не тое, чтобы всѣхъ вещей отстать. Ибо безъ нихъ ни на едину минуту быть не можемъ; и ради насъ онѣ созданы, чтобы ими пользовалися и Богу Создателю благодарили; да и удалиться отъ нихъ не можемъ, ибо вездѣ онѣ насъ окружаютъ, срѣтаютъ и послѣдуютъ намъ. 2)~Не тое паки, чтобы людей ненавидѣть и ими гнушаться. Ибо повелѣно намъ другъ друга любить, и любить не токмо добрыхъ, но и злыхъ, не токмо друговъ, но и враговъ нашихъ, по словеси Господню, подражая въ томъ небесному Отцу, \textit{Который солнце Свое сіяетъ на злыя и благія}\footnote{Матѳ.~5,~44 и 45.}. 3)~Не тое паки, чтобы заключиться въ монастырѣ или пустынѣ. Ибо міръ тотъ, котораго берещися должны мы, внутрь насъ носимъ, и потому, куды ни пойдемъ, не убѣжимъ отъ него. И какъ не вси, живущіи въ монастыряхъ или пустыняхъ, его отрекаются, такъ и во градѣхъ и селѣхъ пребывающіи, не вси работаютъ ему, какъ изъ слѣдующихъ сіе увидимъ. "--- Чтожъ убо есть отрещися міра? Есть не иное что, какъ отъ всего того, что насъ отъ любви Божіей отводитъ и до вѣчнаго неблагополучія доводитъ, сердце и любовь отвратить. Сюды принадлежитъ плоть наша со страстьми и похотьми, которую должно \textit{распинать}, когда хощемъ \textit{Хрістовыми быть}, а не міра сего чадами\footnote{Гал.~5,~24.}. Въ семъ мірѣ заключается самолюбіе, самоволіе, сребролюбіе, славолюбіе, любочестіе, лесть богатства, славы и чести суетныя, которыми надымается и возносится плоть наша. Къ сему причисляется все, что увеселяетъ и услаждаетъ чувства наши, и ударяетъ въ сердце наше и помрачаетъ душевное око, которымъ Свѣтъ вѣчный "--- Богъ видится; здѣ мѣсто имѣетъ богопротивное чреву угожденіе. До сего міра надлежатъ братія и друзи наши, жена и дѣти, отецъ и мать наши и прочіе, когда они насъ отъ Хрістовой отводятъ любви. Самаго здравія нашего, членовъ нашихъ, живота нашего отрещися должны мы; безчестія, поруганія, узъ, темницы, смерти не ужасатися, когда того честь Хрістова требуетъ, и любовь Божія иначе сохраниться не можетъ. \textit{Всякъ, иже не отречется всего своего имѣнія, не можетъ быти Мой ученикъ}, глаголетъ Хрістосъ\footnote{Лук.~14,~33.}. Самолюбіе, самоволіе, сребролюбіе и любославіе есть имѣніе плоти нашея: о сихъ она тщится, сихъ ищетъ; а когда видитъ препятствіе не терпитъ, гнѣвается, злобится и враждуетъ. Отъ сихъ безполезныхъ, паче же вредныхъ трудовъ отзываетъ насъ Хрістосъ, и призываетъ къ Своему покою: \textit{пріидите ко мнѣ вси труждающіися и обремененніи, и Азъ упокою вы}\footnote{Матѳ.~11,~28.}.

\paragraph*{§\:191.} Знаменія отрекшагося міра примѣчаются сіи. 1)~Таковый человѣкъ волю свою воли Божіей во всемъ тщится покорять. 2)~Самолюбіе свое обуздываетъ и усмиряетъ. 3)~Богатства, славы и чести не желаетъ, но, хотя и даются ему сія, сердца къ нимъ не прилагаетъ. 4)~Лишившися богатства, славы и чести и всего, что въ мірѣ семъ имѣется, не жалѣетъ, но съ Іовомъ глаголетъ: \textit{Господь даде, Господь и отъятъ}\footnote{Іов.~1,~21.}. 5)~Пищи, питія, одежды, покоя и прочіихъ мірскихъ вещей не ради роскоши, но ради нужды и требованія немощныя плоти употребляетъ, и всякой сладости убѣгаетъ. 6)~Какъ похвалою не превозносится и не утѣшается, такъ укореніемъ и поношеніемъ не оскорбляется. 7)~Всякія обиды терпѣливо сноситъ, и на обидящихъ не жалуется никому. 8)~Отъ міра, то"=есть, злыхъ людей, ненависть терпитъ, но ненавидящихъ любитъ. \textit{Аще отъ міра бысте были, міръ убо свое любилъ бы: якоже отъ міра нѣсте, но Азъ избрахъ вы отъ міра, сего ради ненавидитъ васъ міръ}, глаголетъ Хрістосъ\footnote{Іоан.~15,~19.}. Сіе извѣстнѣйшее знаменіе есть отрекшагося міра, что \textit{міръ его ненавидитъ}: ненавидитъ бо, яко не своего. Сказуютъ, что море содержитъ въ себѣ живыхъ только животныхъ, а мертвыхъ извергаетъ вонъ изъ себе: тако бываетъ въ морѣ міра сего. Кто міру и прихотямъ міра живетъ, сей міру любезное есть чадо: а кто отреклся и умеръ ему, сего извергаетъ и изгоняетъ вонъ изъ себе. 9)~Нищеты, безчестія и поруганія, въ мірѣ семъ случающихся, не боится, понеже богатства и славы не ищетъ. 10)~Всегда поминаетъ небесное отечество, и къ нему сердцемъ и мыслію стремится: \textit{идѣже бо сокровище его, тамо и сердце его, по словеси Хрістову}\footnote{Матѳ.~6,~21.}. Ибо которые знаки показуютъ любовь Божію, тѣ доказуютъ отверженіе міра: ибо кто Бога любитъ, тотъ не любитъ міра, и кто міра не любитъ, тотъ любитъ Бога.

\paragraph*{§\:192.} Отъ вышереченныхъ слѣдуютъ сія. 1)~Кто не тщится воли Божіей творить, но свою исполняетъ: самолюбецъ есть и міролюбецъ. 2)~Когда уклоняешися отъ зла не ради Бога, но ради стыда, или суда гражданскаго, или иныя какія временныя корысти: политикъ еси и міролюбецъ, а не хрістіанинъ. 3)~Когда ищешь чести, славы, богатства въ мірѣ семъ; хощешь, чтобы тебе хвалили, почитали, прославляли: міру работаешь сердцемъ твоимъ. 4)~Попался ли въ какую напасть, и прибѣгаешь къ сребру или злату, къ защитникамъ твоимъ, чтобы избавиться: на міръ надѣешься, а не на Бога; міру прилѣпился еси, а отъ Бога отступилъ. 5)~Расширяешь домъ и строенія богатыя, убираешь слугъ, коней, кареты, облекаешься одеждами шелковыми и разноцвѣтными ради пріобрѣтенія тщетныя славы: похоти плотской, похоти очесъ и гордости житейской служишь. 6)~Набираешь столы богатые съ драгими винами, и въ томъ увеселеніе полагаешь: чреву рабъ ты, \textit{богъ тебѣ чрево}, а не Господь\footnote{Филип.~3,~19.}. 7)~Дѣлаешь какое добро, то есть, или милостыню даешь, или въ церковь ходишь, или постишися, или храмы Божіи строишь и украшаешь, или иное что, чтобы \textit{явиться предъ людьми}: міру угождаешь, а не Богу. 8)~Когда лишаешься отца, или матери, или жены, или дѣтей, или братіи, или друзей, или богатства, или славы, или чести и прочаго, и болѣе скорбишь, нежели когда предъ Богомъ согрѣшишь, и тако чрезъ грѣхъ Бога лишаешься: болѣе любишь міръ, плоть свою и кровь, нежели Хріста. Ибо чимъ болѣе кого любимъ, тѣмъ болѣе скорбимъ, когда его лишаемся. 9)~Слышишь поносное слово отъ ближняго твоего, и гнѣваешься на него: еще міръ имѣетъ мѣсто свое въ сердцѣ твоемъ. 10)~Ищешь зло за зло воздать, отмстить за обиду: съ міромъ едино мыслишь, который не дѣлаетъ ничего, кромѣ зла. 11)~Боишися безчестія, изгнанія, темницы, ссылки, смерти, и ради того правды не свидѣтельствуешь: міра боишися, а не Бога; міру угождаешь, а не Богу. 12)~Ищешь чести и ранга, и ради того князьямъ и вельможамъ угождаешь, ласкаешь и покланяешься: у міра на подножіи лежишь, и міру, якоже \textit{образу} оному \textit{златому}, отъ Навуходоносора поставленному\footnote{Дан.~3,~1--7.}, безстыдно покланяешися. 13)~Домы богатые, увеселительные пруды, сады, галлереи и прочія забавы строишь, дочери приданое, или сыну наслѣдіе богатое уготовляешь; а ради имени Хрістова нагаго одѣть, или неимущему гдѣ главы подклонить, хижины построить, или сѣдящаго въ темницѣ за долгъ свободить, или плѣннаго искупить, или немогущаго прокормиться питать, или въ долгахъ увязшаго свободить не хощешь: знай, что плоть и кровь свою и мірскую суету любишь только, а ко Хрісту никакой любви не имѣешь. А когда съ людей сдираешь богатство твое, которое въ такъ непотребныя забавы употребляешь: то не токмо не любишь Хріста, но и гонишь Его безстудно съ злымъ міромъ. "--- Отъ сихъ примѣчаній и прочіихъ къ міру пристрастіяхъ можешь, любезный хрістіанине, разсуждать, и вѣрою съ помощію Божіею отъ мірской суеты себе отлучать. О клятвопреступникахъ, лихоимцахъ, хищникахъ, ворахъ, разбойникахъ, убійцахъ, пьяницахъ, блудникахъ, прелюбодѣяхъ и прочіихъ сквернителяхъ, о тѣхъ, кои ненависть за возлюбленіе, зло за добро, клевету и злословіе вмѣсто благодарности воздаютъ, о ругателяхъ и хульникахъ, о тѣхъ, кои судъ Божій за сребро попираютъ, правду съ престола на землю низвергаютъ и попираютъ, и тако плачущимъ вмѣсто утѣшенія большія слезы дѣлаютъ, здѣсь нѣтъ слова: ибо сіи явные міра, или паче міродержца и князя тмы плѣнники и невольники суть, и, хотя имя хрістіанское имѣютъ, худшіи самыхъ язычниковъ, которые, естественнымъ только закономъ водимы, таковыхъ пороковъ берегутся. Не поминаю и о тѣхъ, которые въ томъ поучаются, какъ бы въ сердце человѣческое вкрасться, какъ бы обмануть, прельстить, уловить, осмѣять, единаго предъ другимъ оклеветать, единаго съ другимъ въ ссору и вражду привести; ибо сіи такожде суть явные служители и раби князя міра сего, который чрезъ нихъ дѣло свое совершаетъ; его бо дѣло есть обманывать, лгать, прельщать, любовь искоренять, ненависть, ссору, вражду всѣвать, и проч.

\paragraph*{§\:193.} Причины, которыя возбуждаютъ насъ отъ любви міра сего отстать и къ Богу обратиться, слѣдующія примѣчаются: 1)~Богъ Самъ насъ отъ міра отзываетъ къ Себѣ. \textit{Не любите міра, ни яже въ мірѣ. Аще кто любитъ міръ, нѣсть любве Отчи въ немъ: яко все, еже въ мірѣ, похоть плотская, и похоть очесъ, и гордость житейская, нѣсть отъ Отца, но отъ міра сего есть. И міръ преходитъ, и похоть его; а творяй волю Божію, пребываетъ во вѣки}, глаголетъ апостолъ\footnote{1~Іоан.~2,~15--17.}. И Хрістосъ: \textit{пріидите ко Мнѣ вси труждающіися и обремененніи, и Азъ упокою вы}. Доколѣ вамъ въ суетѣ сей напрасно и безполезно трудиться? Не сыщете, кромѣ Мене, истиннаго утѣшенія и покоя: у Мене единаго \textit{обрящете покой душамъ вашимъ}\footnote{Матѳ.~11,~28 и 29.}. "--- 2)~Апостолъ глаголетъ: \textit{любы міра сего, вражда Богу есть; иже бо восхощетъ другъ быти міру, врагъ Божій бываетъ}\footnote{Іак.~4,~4.}. Коль же страшно есть врагомъ Божіимъ быть, и какъ жалостно дружества Божія отпасть, не требуетъ доказательства. Съ Богомъ бо и міромъ дружбу имѣть, Бога и міръ любить невозможно. \textit{Никтоже бо можетъ двѣма господинома работати: любо единаго возлюбитъ, а другаго возненавидитъ}, глаголетъ Хрістосъ\footnote{Матѳ.~6,~24.}. "--- 3)~Хрістосъ, Сынъ Божій, дражайшею Своею кровію искупилъ души наша къ вѣчному животу. Какъ несмысленно и безумно дѣлаемъ, когда злату, сребру, тлѣннымъ сокровищамъ, чести и сладострастію порабощаемъ ее, и толикую Его благодать презираемъ!.. Коликое неблагодарствіе показуемъ Ему, когда небесная благая, кровію Его неоцѣненною снисканная, оставляемъ, и ищемъ тлѣнныхъ земныхъ!.. "--- 4)~Великое паки неблагодарствіе и презрѣніе Ему бываетъ отъ насъ, когда Его, яко Создателя, Любителя и Избавителя, оставивше, сердцемъ къ злату, сребру и гордости житейской, какъ идоламъ, обращаемся и прилѣпляемся. Здѣ оное пророка Божія Моисея слово, которымъ неблагодарность ветхаго Израиля обличаетъ, приличествуетъ: \textit{роде строптивый и развращенный, сія ли Господеви воздаете}\footnote{Второз.~32,~5 и 6.}? Господь такъ тебе возлюбилъ, что ради тебе рабій зракъ пріялъ, пострадалъ горько, и поносною умеръ смертію: а ты, оставивши Его, любишь безчувственное созданіе Его, и тако уклоняяся отъ Бога къ міру, какъ бы вмѣсто благодарнаго почитанія и поклоненія хребетъ Ему обращаешь, якоже о Израильтянахъ глаголетъ Господь: \textit{обратиша хребетъ ко Мнѣ, а не лице}\footnote{Іер.~32,~33.}. Любить бо Бога и міръ, Богу прилѣпляться и міру купно не можно, какъ сказано. "--- 5)~Коль великое безуміе есть, любить несмысленную и безчувственную вещь, отъ которой взаимно любимы быть не можемъ!.. Вещь бо безчувственная и мертвая, какъ то: злато, сребро, любить насъ не можетъ, но токмо прельщаетъ, ослѣпляетъ, плѣняетъ и погубляетъ насъ: убо и намъ не подобаетъ любити его, но токмо единаго \textit{живаго и безсмертнаго} Бога, любящаго насъ, "--- и подобнаго намъ человѣка. "--- 6)~Высокое хрістіанское благородіе возбраняетъ намъ къ тлѣннымъ и суетнымъ вещамъ сердцемъ прилѣплятися, которое не въ чести, славѣ, титулахъ, знатной фамиліи міра сего, состоитъ, но во внутренней душевной добротѣ, по образу Божію созданной, и благодатію Хріста Сына Божія возобновленной. Сего благородія суетная честь и слава недостойна есть, хотя бы она и отъ всего міра во едино собрана была. И какъ царскому высокому титулу низшая почесть, и порфирѣ его рубище, такъ хрістіанскому достоинству, яко небесному, земная честь не приличествуетъ. Какъ убо, суще \textit{родъ избранъ, царское священіе, языкъ святъ, людіе обновленія}\footnote{1~Петр.~2,~9.}, такъ высокопочтенную фамилію раболѣпнымъ міра сего любочестіемъ безчестить, и, какъ солнца сіяніе, славу ея мглою славы суетныя помрачать дерзаемъ? Почто ищемъ внѣ насъ чести и славы, когда внутрь насъ толикое имѣемъ достоинство? \textit{Се бо царствіе Божіе внутрь васъ есть}, глаголетъ Хрістосъ\footnote{Лук.~17,~21.}. Почто тщимся собирать внѣ насъ сокровище тлѣнное, которое \textit{тля тлитъ и татіе подкопываютъ и крадутъ}, когда внутрь насъ дано намъ имѣть таковое сокровище, котораго \textit{ни тля тлитъ, ни татіе подкопываютъ, ни крадутъ}\footnote{Матѳ.~6,~19 и 20.}? Почто гоняемся за увеселеніемъ суетнаго міра, когда Самъ Сынъ Божій хощетъ утѣшить насъ и увеселить? \textit{Се стою при дверехъ и толку: аще кто услышитъ гласъ Мой и отверзетъ двери, вниду къ нему, и вечеряю съ нимъ, и той со Мною}\footnote{Апок.~3,~20.}. Сія пресладкая вечеря не въ яденіи сладкихъ и искусно приготовленныхъ снѣдей, ни въ питіи драгихъ, сердце увеселяющихъ винъ состоитъ, но въ благопріятномъ любве и благости Божіей вкушеніи. Сея вечери чимъ кто болѣе будетъ причащаться, тѣмъ болѣе ему мерзѣть будутъ сладости міра сего. "--- 7)~\textit{Нѣсмы свои: куплени есмы цѣною}\footnote{1~Кор.~6,~19 и 20.}. \textit{Не истлѣннымъ бо сребромъ или златомъ избавихомся отъ суетнаго нашего житія, отцы преданнаго, но честною кровію яко Агнца непорочна и пречиста Хріста}\footnote{1~Петр.~1,~18 и 19.}. Убо не иному должно жить и работать намъ, но Тому, Который искупилъ насъ такъ чуднымъ образомъ. Рабъ купленный тому господину служитъ и работаетъ, который его купилъ, а инаго господина не знаетъ: такъ и мы искупившему насъ Хрісту должны жить и угождать, а не міру, ни себѣ. \textit{Хрістосъ за всѣхъ умре, да живущіи не ктому себѣ живутъ, но Умершему за нихъ и Воскресшему}, поучаетъ апостолъ\footnote{2~Кор.~5,~15.}. "--- 8)~Пристрастившіися міру сытости не знаютъ. Чего ради всякому должно заблаговременно отъ сѣтей сихъ пагубныхъ отстать, дабы, въ нихъ запутавшися, тако не окончать житія своего: тогда уже міръ его и нехотящаго оставитъ. "--- 9)~Пристрастіе къ міру многихъ золъ виновно бываетъ. "--- 10)~Мірскія вещи такъ плѣняютъ человѣческое сердце, что не допускаютъ его о Бозѣ помыслить и о душѣ своей пещися. Богачъ сребролюбивый тамо всегда сердце свое имѣетъ, гдѣ сокровище его: сегодня приходы, утро расходы считаетъ, на третій день о новыхъ приходахъ думаетъ; въ домѣ сидя печется, чтобъ товары въ торгъ пущенные не пропали; внѣ дома боится, чтобы сокровище въ домѣ не окрадено было. Въ церкви стоитъ, и тутъ не покоится сердце его, и тогда то къ сундукамъ, то къ лавкамъ, то къ прочіимъ мѣстамъ, гдѣ товары ходятъ, мыслію бросается; днемъ о томъ печется, ночью печется, и во снѣ печется. Мучительная и безполезная забота! Пишется въ Апокалипсисѣ, что \textit{не имутъ покоя день и нощь покланяющіися звѣрю и образу его}\footnote{14,~11.}: подобно не имѣютъ покоя, но непрестанно въ сердцѣ своемъ мучатся, то страхомъ, то печалію снѣдаяся, которые богатству міра сего, аки звѣриному образу, сердца, какъ колѣна, преклоняютъ. "--- Что о сребролюбіи, тое о любочестіи и славолюбіи должно разумѣть. "--- 11)~Работающіи міру непремѣнно будутъ при смерти жалѣть и совѣстію мучиться, что міръ сей любили. Плѣнники, которые попадаются въ варварскія и мучительныя руки, день цѣлый работая, въ вечеру не иное что отъ нихъ, какъ посмѣяніе и біеніе пріемлютъ: тако міра сего плѣнники, приближившеся къ концу житія своего, не иное что отъ него чувствуютъ, какъ ударенія и язвы грѣховъ, которыми совѣсти ихъ уязвляются, что міру, а не Богу, работали; пріемлютъ и посмѣяніе отъ князя міра сего, какъ безчеловѣчнаго мучителя, который сею ихъ суетою прельщалъ и обманывалъ. Спроси у умирающаго міролюбца, котораго умъ и сердце заняты были суетою мірскою: что онъ тогда чувствуетъ внутрь себе? Сильное удареніе грѣховъ, печаль, страхъ и ужасъ поражаютъ совѣсть его и къ конечному преклоняютъ отчаянію. Врагъ душевный предстоитъ и смѣется: «въ руки наши пришелъ, и намъ преданъ еси!» Тогда бѣдная душа мятется, кается, жалѣетъ, сокрушается, что въ такъ непотребныхъ суетахъ житіе провождала; тогда праведно, но поздно о всемъ разсуждаетъ; тогда признаетъ, что \textit{суета суетствій и всяческая суета}\footnote{Екк.~1,~2.}. Убо учися всякъ и старайся міръ сей оставить, пока міръ тебе не оставитъ, и не опечалитъ и не поругается тебѣ. "--- 12)~Міръ сей непостояненъ. Какъ на воздухѣ различная перемѣна бываетъ, то день, то нощь, то свѣтло, то мрачно, то тепло, то холодно: такъ и человѣку въ мірѣ семъ случается; то богатство, то нищета, то слава, то безславіе, то честь, то безчестіе послѣдуетъ. Бываетъ, что кто сегодня богатъ, утро нищъ; кто нынѣ ѣздитъ на колесницѣ, утро заключается въ темницѣ; кого вчера хвалили, сегодня ругаютъ; кому покланялись, того попираютъ. И едва сыщется, кто бы въ единомъ постоянномъ счастіи до конца житія своего дошелъ. Почто убо за такою суетою гоняться, которая, какъ дымъ, вмалѣ является и исчезаетъ? "--- 13)~Апостолъ глаголетъ: \textit{ничтоже внесохомъ въ міръ сей, явѣ, яко ниже изнести что можемъ; имѣюще же пищу и одѣяніе, сими довольни да будемъ}\footnote{1~Тим.~6,~7 и 8.}. И Іовъ праведный тоежде: \textit{нагъ изыдохъ отъ чрева матере моея, нагъ паки и отъиду}\footnote{Іов.~1,~21.}. Какъ, входя въ міръ сей, ничего, кромѣ нагаго и немощнаго тѣла, не вносимъ, которое, кромѣ нужныя пищи и одѣянія, не требуетъ ничего: такъ и отъ міра сего исходя, ничего не износимъ; злато, сребро, славу, честь, титулы, отчины, домы, рабы, рабыни, кони, кареты, виссоны и прочая, яко мірская, въ мірѣ оставляемъ; и самаго тѣла въ день скончанія нашего совлекаемся. Съ едиными душами на оный свѣтъ отходимъ. Коль же бѣдно умираетъ наипаче тотъ, который \textit{себѣ собираетъ, а не въ Бога богатѣетъ}\footnote{Лук.~12,~21.}! Сего ради такого безуміе Самъ Богъ обличаетъ: \textit{безумне! въ сію нощь душу твою истяжутъ отъ тебе: а яже уготовалъ еси, кому будутъ}\footnote{ст.~20.}? "--- 14)~Хрістіане въ мірѣ семъ суть пришельцы и странники. \textit{Не имамы здѣ пребывающаго града, но грядущаго взыскуемъ}, глаголетъ апостолъ\footnote{Евр.~13,~14.}. И Давидъ ко Господу молится: \textit{пресельникъ азъ есмь у Тебе и пришлецъ, якоже вси отцы мои}\footnote{Пс.~38,~13.}. Аще пресельники и пришлецы есмы на земли, якоже и отцы наши, убо на иномъ мѣстѣ отечество наше и домъ нашъ есть. Почто убо такъ отягощаемъ странствующую душу нашу? Почто толико богатствъ собираемъ ради единаго смертнаго тѣла, которое можетъ единымъ укрухомъ хлѣба и чашею воды довольствоваться, и которое съ свѣтомъ симъ имѣемъ оставить? Не безуміе ли, ради такъ малаго и бѣднаго тѣла, которое малымъ довольствуется, и въ землю, отъ неяже взято бысть, имѣетъ обратитися, толико строить домовъ, толико собирать сокровищъ, толико приготовлять одѣяній, уборовъ, толико заготовлять снѣдей, толико имѣть слугъ? Ей, прелесть одна суетнаго міра и обманъ князя тмы, который помрачаетъ душевныя очи людей, чтобы сей суеты не могли усмотрѣть! "--- 15)~Хрістіанское отечество есть на небеси, гдѣ и Отецъ ихъ, къ Которому молятся тако: \textit{Отче нашъ, Иже еси на небесѣхъ}! Убо къ отечеству оному должны всегда сердца свои возводить, а не къ земнымъ прилѣпляться. Купцы, странствующіи по чужимъ сторонамъ, товары тамо собираютъ, но предпосылаютъ ихъ, или сами привозятъ въ отечество свое: тако хрістіанамъ, странствующимъ въ семъ мірѣ, должно товары свои духовные, яже суть добродѣтели, собирать и предпосылать въ небесное свое отечество; \textit{скрывать сокровище свое на небеси, идѣже ни червь, ни тля тлитъ, и идѣже татіе не подкопываютъ, ни крадутъ}. И тако \textit{идѣже сокровище ихъ будетъ, ту будетъ и сердце ихъ}, по словеси Хрістову\footnote{Матѳ.~6,~20 и 21.}. "--- 16)~Хрістіанамъ наслѣдіе, богатство, честь, слава, вѣнцы, царство и вся благая на небеси уготована суть. Почто убо ищемъ въ мірѣ и съ міромъ царствовать, прославитися и веселитися? Развѣ хощемъ и здѣ, и тамо царствовати? Но быть сіе никакъ не можетъ, ибо \textit{тѣсный}, а не пространный \textit{путь} къ оному царствію приводитъ\footnote{7,~14.}. Прельщаются убо тіи, и надежда ихъ обманетъ, которые и въ мірѣ семъ хотятъ пространно жить, и со Хрістомъ во вѣки царствовать. "--- 17)~Хрістосъ Сынъ Божій, какъ словомъ, такъ и житіемъ Своимъ научилъ насъ суету міра сего презирать, и тако входить въ небесное царствіе, которое Онъ горькимъ Своимъ страданіемъ и смертію отворилъ вѣрующимъ во имя Его. Онъ въ нищетѣ пожилъ и \textit{не имѣлъ гдѣ главы подклонити}\footnote{Матѳ.~8,~20.}: намъ ли за богатствомъ гоняться и домы расширять? Онъ \textit{смирилъ себе, послушливъ бывъ даже до смерти}\footnote{Филип.~2,~8.}: намъ ли, червямъ, гордиться и превозноситься? Онъ презиралъ славу и хвалу человѣческую: намъ ли искать прославленія? Онъ прощалъ врагомъ Своимъ: намъ ли искать отмщенія? Онъ горькую страданія чашу пилъ: намъ ли въ сластяхъ валяться? Весьма прельщаемся, когда, тако живучи, хощемъ быть хрістіанами! Не тое Онъ въ Евангеліи хрістіанамъ Своимъ предлагаетъ; не богатство, не честь, не славу, не роскошь, не пространство въ мірѣ семъ обѣщаетъ, "--- но что? Скорби, тѣсный путь и узкая врата, любовь ко врагамъ, отверженіе себе и свой всякому крестъ, якоже Самъ Себе во образъ далъ намъ. Аще убо мы хрістіане, то не токмо имя носить, но и Духъ Его имѣть должны. \textit{Аще бо кто Духа Хрістова не имать, сей нѣсть Его}, глаголетъ апостолъ\footnote{Римл.~8,~9.}. Аще Онъ Учитель нашъ, то не токмо слово Его слушать, но и тому, чему слово Его учитъ насъ, учиться должно намъ. Аще житіе Свое во образъ далъ намъ и сказалъ: \textit{научитеся отъ Мене}\footnote{Матѳ.~11,~29.}, то не токмо смотрѣть на нищету, смиреніе, терпѣніе, кротость и страданіе Его, но и учиться сихъ отъ Него должно. Аще Онъ Вождь намъ къ небеси, то должно намъ за Нимъ итить тѣмъ путемъ, Какимъ Онъ Самъ шелъ. \textit{Аще кто хощетъ по Мнѣ ити, да отвержется себе, и возметъ крестъ свой и послѣдуетъ Ми}\footnote{Матѳ.~16,~24; Марк.~8,~34; Лук.~9,~23.}. Аще Пастырь Онъ нашъ, то намъ, овцамъ Его, должно гласа Его слушать. \textit{Овцы Моя}, глаголетъ, \textit{гласа Моего слушаютъ}\footnote{Іоан.~10,~27.}. Аще Онъ Глава наша, то намъ, яко удамъ Его, должно Ему повиноваться. Аще Онъ Женихъ нашъ, то должно намъ вѣру и любовь къ Нему цѣло хранить и не обращаться къ нечистой міра сего любви. Сего отъ насъ требуетъ Хрістосъ нашъ, о хрістіане, когда хощемъ не токмо именоваться, но и быть хрістіанами. Сего ради разсмотри себе всякъ, хрістіанинъ ли еси, хотя хрістіаниномъ и называешися; и разсмотрѣвши, всякимъ образомъ старайся, чтобы и вещію быть хрістіаниномъ, дабы не услышать страшнаго онаго гласа отъ Хріста: \textit{не вѣмъ васъ, откуду есте}\footnote{Лук.~13,~26 и 27.}. "--- 18)~Хрістосъ глаголетъ: \textit{кая польза человѣку, аще міръ весь пріобрящетъ, душу же свою отщетитъ}\footnote{Матѳ.~16,~26.}? Что тебѣ пользы отъ того, что ты всю славу и все богатство міра сего имѣть будешь, и лишишися вѣчныхъ благъ, противъ которыхъ вся благая временная какъ ничто, и попадешися въ горестное и мучительное состояніе, въ которомъ вѣчно безъ конца страдать будеши? Какая тамо польза, гдѣ души пагуба? Слава сія какъ дымъ исчезнетъ; богатство въ мірѣ останется; сладости и утѣхи престанутъ, но вмѣсто того вѣчное поношеніе, печаль, воздыханіе и мученіе воспріиметъ тебе. О коликое обыметъ тогда раскаяніе! Къ сему бѣдствію и горести міръ любителей своихъ приводитъ и за любовь ихъ воздаетъ имъ вѣчную погибель. Ибо \textit{любы міра сего вражда Богу есть}, глаголетъ апостолъ\footnote{Іак.~4,~3.}. "--- 19)~Оставившіи міръ вѣрою, любви ради Хрістовой, Хріста со всѣмъ небеснымъ и вѣчнымъ сокровищемъ обрѣтаютъ; и, какъ Ему въ мірѣ семъ сообразны были въ смиреніи, любви, терпѣніи, кротости и прочемъ, такъ въ будущемъ вѣцѣ сообразны будутъ славѣ Его\footnote{Римл.~8,~13,~18 и 29.}. "--- 20)~Оставившіи міръ вѣрою и временная своя сокровища имѣютъ лучшая, нежели міролюбцы. Не ищутъ богатства, но живутъ какъ и богачи. Не имѣютъ богатства, но не имѣютъ излишняго и попеченія, печали, смущенія, страха. Довольствуются, что Богъ имъ подалъ, и потому всегдашнимъ наслаждаются внутреннимъ покоемъ. Сего сынове вѣка сего не имѣютъ, которые имѣній своихъ ищутъ съ многими трудами, хранятъ съ великимъ страхомъ, лишаются съ зѣльною печалію. "--- Убѣгаютъ чести и славы, но за ними слава бѣжитъ. Какъ бо къ солнцу идущіи, сколько ни убѣгаютъ отъ тѣни своей, убѣжати не могутъ: такъ къ вѣчному Солнцу "--- Богу приближающіися, чимъ болѣе убѣгаютъ славы суетной, какъ тѣни, тѣмъ болѣе слава за ними слѣдуетъ. И какъ свѣтъ во тьмѣ, такъ они сокрытися не могутъ; и хотя злый міръ свѣтъ сей помрачить тщится, однакожъ помрачить не можетъ: паче же большее сіяніе издаетъ. Уклоняются они сладости и веселостей внѣшнихъ; но внутрь далеко лучшую и истинную сладость и веселіе имѣютъ, которымъ утѣшаются паче всѣхъ увеселеній міра сего, "--- которое веселіе состоитъ \textit{въ чистой совѣсти и радости о Дусѣ Святѣ}\footnote{16,~17.}. Хощеши ли убо быть богатымъ? Не ищи богатства. Ибо не тотъ богатъ, кто много имѣетъ, но кто тѣмъ, что имѣетъ, довольствуется, и болѣе не желаетъ; якоже называемъ сытымъ не того, кто много ѣстъ и пьетъ, но того, кто болѣе не хощетъ ясти. Хощеши ли имѣть славу? Убѣгай славы, и будешь имѣть славу, хотя и не желаешь ея. Хощеши ли имѣть сладость? Уклоняйся сладости, и будешь имѣть истинную сладость. Каковыхъ сокровищъ міръ имѣти не можетъ, хотя и много старается.

\paragraph*{§\:194.} Можетъ здѣсь кто помыслить и противу сказать: созданія вся ради человѣка сотворена суть: ради чего ихъ отрицаться? "--- \textit{Отвѣщаю}: 1)~Не созданія отрицаться должно, но любви къ созданію. Извѣстно, что все ради человѣка сотворено, убо тѣмъ самымъ все должно человѣку служить, а не человѣкъ ему. Созданіе человѣку должно работать по повелѣнію Божію, а не человѣкъ созданію. Работаетъ же человѣкъ созданію, когда сердцемъ ему прилѣпляется, и которою любовію Богу долженъ, тую созданію посвящаетъ. "--- 2)~Должно созданія употреблять умѣренно, а не излишне, къ нуждѣ, а не къ сладострастію. 3)~Созданія суть какъ слѣды и свидѣтельства, которыя показуютъ Создателя, и отъ нихъ научаемся и увѣщаваемся любить и почитать Создателя. 4)~Созданія служатъ намъ, чтобы мы Создателю служили; а когда не служимъ, то и ихъ служеніе намъ тщетно бываетъ, и Богу оттуду великое неблагодарствіе послѣдуетъ. Человѣкъ бо, какъ разумное созданіе, есть ближайшій слуга Божій, и есть аки посредственникъ между Богомъ и созданіями, которыхъ служенія употребляя, Богу благодарить за нихъ и служить долженъ. Какъ раби господину своему служатъ того ради, чтобы онъ монарху и обществу служилъ; а когда сію должность господинъ оставляетъ, то и ихъ служеніе нерадѣніемъ его безполезно бываетъ: такъ и созданія человѣку служатъ, чтобы онъ Богу служилъ, и въ своемъ лицѣ за всѣхъ *ихъ Богу "--- Создателю всѣхъ* благодарилъ и хвалилъ. А когда того не исполняетъ, то и созданія напрасно употребляетъ, и потому Создателю своему неблагодаренъ является и дѣлаетъ обиду; обида бо бываетъ, когда должное не воздается.

\paragraph*{§\:195.} Къ тому, чтобы отъ любви суетнаго міра отвратилось сердце, во первыхъ нужна есть вѣра нелицемѣрная (которая имѣетъ мѣсто свое на сердцѣ, а не токмо на языкѣ), такъ, что безъ нея тое отвращеніе быть не можетъ. Ибо свойство вѣры сіе примѣчается, что она единаго Бога ищетъ, Ему единому прилѣпляется, на Него единаго надѣется, уповаетъ; защищенія, помощи, избавленія, спасенія отъ Него единаго чаетъ, и Его воли послѣдуетъ; слѣдственно удаляется отъ твари, отъ всего видимаго, земнаго; отъ богатства, чести, славы, всякаго пристрастія, сердце человѣческое, въ которомъ находится, отвращаетъ и къ единымъ невидимымъ и вѣчнымъ привлекаетъ. Какъ бо плоть и чувства наши наклоняютъ и влекутъ сердце наше къ земнымъ и видимымъ; такъ, напротивъ того, вѣра отвращаетъ отъ сихъ и обращаетъ къ Богу и Его вѣчнымъ обѣщаннымъ благимъ, во основаніе полагая истину Божію, въ словѣ Его святомъ явленную и утвержденную. "--- 2)~Нужно есть поученіе усердное въ словѣ Божіемъ, которое съ помощію Божіею вѣру вкореняетъ и умножаетъ, и суету міра сего показуетъ. 3)~Размышленіе о настоящей и будущей жизни, о настоящемъ и будущемъ вѣкѣ, о временныхъ и вѣчныхъ благихъ. Тако бо человѣкъ внутренними глазами усмотрѣть можетъ суету міра сего и пріитить въ познаніе истиннаго блаженства. Сіе бо паче всего нужно есть, чтобы познать суету, и въ чемъ состоитъ истинное блаженство. "--- 4)~Понеже плоть наша похотствуетъ противу духа, и чрезъ чувства, какъ орудія свои, влечетъ насъ къ земнымъ и видимымъ, нужно есть подкрѣпленіе отъ помощи Божіей. Чего ради должно съ пророкомъ усердно молитися всегда: \textit{отврати очи мои, еже не видѣти суеты; въ пути Твоемъ живи мя}, Господи\footnote{Пс.~118,~37.}!

\subsection[Глава 4-я. О любви Божіей.]{глава четвертая.\\\bfseries О любви Божіей.}

\begin{quotation}\textit{Аще любите Мя, заповѣди Моя соблюдите}, глаголетъ Хрістосъ\footnote{Іоан.~14,~15.}.\end{quotation}
\begin{quotation}\textit{Имѣяй заповѣди Моя, и соблюдаяй ихъ, той есть любяй Мя; а любяй Мя, возлюбленъ будетъ Отцемъ Моимъ; и Азъ возлюблю его, и явлюся ему Самъ}, глаголетъ Хрістосъ\footnote{ст.~21.}.\end{quotation}


\paragraph*{§\:196.} Что есть любовь Божія и какая сладость ея, слово изобразить не можетъ; едины только тіи познаютъ ее, которые вкушаютъ сладости ея: понеже любовь сія есть духовная и дѣло Святаго Духа, \textit{яко плодъ духовный есть любы}\footnote{Гал.~5,~22.}. Однакожъ плодами своими, какъ солнце лучами, оказываетъ себе, и въ познаніе подаетъ себе и другимъ. 1)~Истинный Божій любитель тщится волю Божію исполнять не ради страха наказаній, но ради того, чтобы Любимаго не оскорбить. Откуду послѣдуетъ тщательное соблюденіе Божіихъ заповѣдей, въ которыхъ воля Божія изображается; о чемъ Самъ Хрістосъ глаголетъ: \textit{имѣяй заповѣди Моя, и соблюдаяй ихъ, той есть любяй Мя}. Такъ сынъ добрый волю отца своего, жена добрая волю мужа своего тщится исполнять, чтобы любимаго не опечалить: опечаленіе бо противно есть любви и любовь разоряетъ. "--- 2)~Истинный Божій любитель ради Бога любитъ всякаго человѣка, вѣдая, что всякъ человѣкъ есть Божій, и Богъ его любитъ. Любитель бо любитъ и того, кого любимый его любитъ. Напр., понеже любишь друга своего, любишь ради его и того, кого другъ твой любитъ. Откуду апостолъ заключаетъ, что нѣтъ въ томъ и къ Богу любви, кто ненавидитъ брата своего, то"=есть, всякаго человѣка. \textit{Аще кто речетъ, яко люблю Бога, а брата своего ненавидитъ, ложь есть: ибо не любяй брата своего, егоже видѣ, Бога, Егоже не видѣ, како можетъ любити}? И придаетъ: \textit{и сію заповѣдь имамы отъ Него, да любяй Бога, любитъ и брата своего}\footnote{1~Іоан.~4,~20 и 21.}. "--- 3)~Истинный любитель отъ всего того уклоняется, чимъ любимый его оскорбляется, оскорбленіе бо противно любви. А понеже Богъ всякимъ грѣхомъ оскорбляется: убо истинный Божій любитель отъ всякаго грѣха бережется, какъ пророкъ увѣщаваетъ: \textit{любящіи Господа, ненавидите злая}\footnote{Пс.~96,~10.}. "--- 4)~Истинный Божій любитель Бога въ сердцѣ всегда объемлетъ и носитъ, любовь бо истинная въ сердцѣ свое мѣсто имѣетъ: почему всегда съ любовію и почтеніемъ святое имя Его поминаетъ, радуется о Немъ, благодаритъ Ему, хвалитъ, поетъ и прославляетъ Его съ радостію, безъ лицемѣрія. Такъ и сынъ добрый матерь или отца своего, понеже сердечно любитъ, часто поминаетъ, когда не видитъ ихъ или удаленъ отъ нихъ имѣется. "--- 5)~Истинный Божій любитель желаетъ всеусердно съ любимымъ соединитися, почему часто молится, воздыхаетъ, плачетъ, съ пророкомъ сердцемъ вопія: \textit{имже образомъ желаетъ елень на источники водныя, сице желаетъ душа моя къ Тебѣ, Боже}\footnote{41,~2.}. Таковому смерть не страшна, но желаема, понеже чрезъ нее къ лицу любимаго Бога прейдетъ. Таковъ былъ Павелъ, который \textit{желалъ разрѣшитися и со Хрістомъ жити}\footnote{Филип.~1,~23.}. "--- 6)~Истинный Божій любитель тщится подражать Богу во нравахъ Его; старается быть кроткимъ, терпѣливымъ, незлобивымъ, милостивымъ, милосердымъ, щедрымъ не ради инаго чего, но ради того только единаго, что Любимый его таковъ есть. Къ сему апостолъ насъ увѣщаваетъ: \textit{бывайте подражатели Богу, якоже чада возлюбленная}\footnote{Еф.~5,~1.}. Истинное бо Божіе чадо не можетъ не любить Бога, яко Отца своего. "--- 7)~Истинный Божій любитель уклоняется отъ любви міра сего; понеже любовь Божія съ любовію мірскою помѣститься не можетъ, якоже глаголетъ апостолъ: \textit{аще кто любитъ міръ, нѣсть любве Отчи въ немъ}\footnote{1 Iоан.~2,~15.}. "--- 8)~Истинный Божій любитель во всемъ ищетъ славы Божіей, а не своей, и молится о томъ, чтобы имя Божіе славилося: \textit{да святится имя Твое, Отче небесный}, какъ Хрістосъ научилъ\footnote{Матѳ.~6,~9.}. Чего для отъ всякаго зла уклоняться, и всякое добро дѣлать тщится не ради суетныя своея славы, но въ славу и честь имене Божія. Отсюду бываетъ, что за честь Божію во всякую бѣду и самую смерть себе повергаетъ, и желаетъ лучше умереть, нежели безчестіе видѣть или слышать имене Божія, "--- каковы были мученики святые. "--- 9)~Истинный Божій любитель безропотно терпитъ всякую бѣду и напасть, вѣдая, что не безъ воли Божіей все бываетъ; и хотя въ такомъ случаѣ немощная плоть и начинаетъ смущаться, однако духомъ терпѣнія усмиряетъ ее. "--- 10)~Истинный Божій любитель, когда отъ немощи что согрѣшитъ и почувствуетъ въ совѣсти удареніе, вельми жалѣетъ о томъ, скорбитъ, окаеваетъ себе, гнѣвается на себе, смиряется и повергаетъ себе съ любовнымъ смиреніемъ предъ Создателемъ и Отцемъ своимъ небеснымъ. Тако сотворилъ искренній и теплѣйшій любитель Хріста Бога Петръ святый, который, отрекшися Любимаго, \textit{изшедъ вонъ, плакася горько}\footnote{Матѳ.~26,~75.}. "--- 11)~Чимъ большая и горячайшая въ комъ любовь сія, тѣмъ большія являются дѣйствія ея; которыя дѣйствія нѣсколько изъясняются и отъ человѣческой любви, которая бываетъ между родителями и дѣтями, между мужемъ и женою, между любезными и вѣрными другами. О чемъ всякому оставляю разсуждать; а я здѣсь свое разсужденіе предлагаю о поощреніи къ сладкой сей любви.

\paragraph*{§\:197.} Причины, которыя при помощи благодати Божіей возбуждаютъ любовь Божію, примѣчаются сіи: 1)~Богъ есть благость высочайшая, естественная, вѣчная и безконечная. И кто ни есть отъ человѣкъ благъ, не самъ въ себѣ благъ, но поелику благости Божіей участникъ есть. Богъ же Самъ въ Себѣ по своему естеству благъ. Откуду Хрістосъ глаголетъ: \textit{никтоже благъ, токмо единъ Богъ}\footnote{19,~17.}. Самая убо благость Божія привлекаетъ всякаго къ любви Божіей. И хотя люди любятъ и зло, однакожъ подъ видомъ добра любятъ, какъ добро любятъ, а не какъ зло: зла, поелику зло есть, никто не любитъ, но уклоняется отъ него. Аще убо созданное добро и несовершенное любимъ, кольми паче естественное и совершенное добро, которое есть единъ Богъ, должно любить. Къ сему пророкъ увѣщаваетъ: \textit{вкусите и видите, яко благъ Господь}\footnote{Пс.~33,~9.}. "--- 2)~\textit{Богъ есть любы}, какъ апостолъ глаголетъ\footnote{1~Іоан.~4,~16.}, и есть любы вѣчная и непремѣняемая. Аще въ созданіяхъ, напр. въ матеряхъ къ своимъ дѣтямъ горячую насадилъ любовь, то несравненно большую и превосходную Самъ имѣетъ. Самая бо божественная Его любовь къ любленію привлекаетъ сердце человѣческое. И человѣкъ бо ничимъ такъ, какъ любовію своею къ любленію себе другихъ привлекаетъ, ибо безъ любви ничто намъ не пріятно. Любовь и самыхъ жестокосердыхъ, какъ магнитъ желѣзо, влечетъ къ себѣ и привлекаетъ. Хотя и не знаемъ кого, а слышимъ, что любительный человѣкъ есть, сердце наше возбуждается къ любленію его. Мы же въ Божіей любви заключены есмы, яко \textit{въ Немъ живемъ, движемся и есмы}\footnote{Дѣян.~17,~28.}. И куды ни обратимся, Божія любовь вездѣ срѣтаетъ насъ; и столько ея свидѣтелей и проповѣдниковъ, сколько Божіихъ къ намъ благодѣяній, такъ что и на малѣйшее время безъ нея не можемъ быть. Какъ убо такая любовь не подвигнетъ сердца нашего къ взаимной любви? "--- 3)~Богъ есть красота всѣхъ красотъ, которой ангели святіи и вси Божіи угодники насытитися не могутъ. Красное солнце, луну, звѣзды сотворилъ Онъ: то несравненно превосходную красоту имѣетъ Самъ. \textit{Во исповѣданіе и въ велелѣпоту облеклся еси, одѣяйся свѣтомъ яко ризою}, глаголетъ къ Нему Псаломникъ, Духомъ Божіимъ просвѣщаемъ и восхищаемъ\footnote{Пс.~103,~1 и 2.}. Красота же сія разумѣется не тѣлесная, но духовная нѣкая любезность и благопріятіе, всю красоту тѣлесную несравненно превосходящее, и духи святыхъ неизреченно веселящее. \textit{Богъ бо есть Духъ}\footnote{Іоан.~4,~24.}, и что ни есть въ Бозѣ, духовное есть, и есть Самъ Богъ. Сего сладкаго и увеселительнаго благопріятія нѣкую каплю и нынѣ любящіи Бога въ сердцахъ своихъ ощущаютъ, когда со Псаломникомъ иногда восклицаютъ тако: \textit{Боже, Боже мой, къ Тебѣ утреннюю: возжада Тебе душа моя}\footnote{Пс.~62,~2.}! Иногда тако: \textit{имже образомъ желаетъ елень на источники водныя: сице желаетъ душа моя къ Тебѣ, Боже. Возжада душа моя къ Богу крѣпкому, живому: когда пріиду и явлюся лицу Божію}\footnote{41,~2 и 3.}? Иногда тако: \textit{что ми есть на небеси? и отъ Тебе что восхотѣхъ на земли? Исчезе сердце мое и плоть моя, Боже сердца моего, и часть моя Боже во вѣкъ}\footnote{72,~25 и 26.}! Но тогда насытятся совершенно того, когда явятся лицу Божію, и \textit{узрятъ Его лицемъ къ лицу}\footnote{1~Кор.~13,~12.}, \textit{якоже есть}\footnote{1~Іоан.~3,~2.}. "--- 4)~Богъ человѣка по единой Своей благости, безъ всякой нужды и пользы Своей создалъ и изъ небытія въ бытіе привелъ\footnote{Быт.~2,~7.}. Сіе едино коликія любве и благодарности нашей требуетъ къ Создателю, всякъ можетъ удобно познать. "--- 5)~Богъ человѣка создалъ не такъ, какъ прочія вещи, но особеннымъ совѣтомъ. Весь міръ созидая, преблагій Создатель нашъ и Богъ глаголетъ: \textit{да будетъ! "--- и бысть тако}\footnote{1,~3,~6,~9,~11,~14,~15,~20 и 24.}; \textit{рече, и быша; повелѣ, и создашася}\footnote{Ис.~148,~5.}. А когда человѣка хотѣлъ создать, аки нѣкое великое и преславное дѣло имѣя создать, глаголалъ тако: \textit{сотворимъ человѣка по образу Нашему и по подобію}\footnote{Быт.~1,~26.}. О коль великія чести сподобился человѣкъ въ созданіи отъ Создателя своего! "--- 6)~Великое почтеніе человѣку, что онъ такимъ Божіимъ совѣтомъ сотворенъ, но большее есть, что по образу Божію сотворенъ. Всѣ прочія созданныя вещи, небо и земля и все украшеніе ихъ, суть свидѣтельства всемогущества, премудрости и благости Божіей; но человѣкъ есть образъ Божій. Умъ не можетъ постигнуть сего Божія къ человѣку благоволенія. О коль высоко почтенъ отъ Бога человѣкъ! какъ много одолженъ въ семъ благости и любви Божіей человѣкъ! "--- 7)~Весь свѣтъ человѣку въ службу опредѣлилъ Богъ. Небо, солнце, луна, звѣзды, воздухъ и земля съ украшеніемъ своимъ единому человѣку служатъ. Богъ бо ради Себе не требуетъ сихъ. Что бо ни создалъ, ради человѣка создалъ. "--- 8)~Богъ падшаго человѣка такъ чуднымъ и уму непонятнымъ образомъ возстановилъ и обновилъ и въ первое состояніе, паче же въ лучшее, чрезъ единороднаго Сына Своего Іисуса Хріста привелъ, такъ, что \textit{елицы пріяша Его, даде имъ область чадомъ Божіимъ быти, вѣрующымъ во имя Его}\footnote{Іоан.~1,~12.}. Небо вмѣсто рая со всѣми благими, \textit{ихже око не видѣ, ухо не слыша и на сердце человѣку не взыдоша}\footnote{1~Кор.~2,~9.}, человѣколюбно отворилъ имъ, и того жителями и царствія Его вѣчнаго участниками учинилъ. "--- 9)~Духа Святаго Утѣшителя, Просвѣтителя, Наставника и Хранителя просящимъ имъ подаетъ, Который вопіетъ въ сердцахъ ихъ: \textit{Авва Отче}\footnote{Гал.~4,~6.}. "--- 10)~Заблуждшихъ и отвратившихся со всякимъ желаніемъ призываетъ и ожидаетъ на покаяніе; кающихся съ радостію пріемлетъ. "--- 11)~Вся сія и прочая недовѣдомая благая отъ единой любви дѣлаетъ намъ. Истинное бо благодѣяніе не отъ инаго чего, какъ отъ истинной и горячей любви, происходитъ. Достойно убо и праведно любить Того, Который \textit{первѣе возлюбилъ насъ}\footnote{1~Іоан.~4,~10 и 19.}. Иначе безъ обиды и оскорбленія любителю быть не можетъ, ибо долгъ любве ничимъ инымъ, какъ любовію, платится. "--- 12)~Богъ есть \textit{Отецъ нашъ}\footnote{1~Кор.~8,~6; 2~Кор.~6,~18; Еф.~4,~6.}. Сіе едино имя "--- Отецъ, можетъ и должно во всякомъ огнь любве къ Нему возбудить. Какъ сынамъ отца не любить? За срамное бы и гнусное чудовище отъ всѣхъ почитался тотъ сынъ, который бы отца родившаго не любилъ. Аще Богъ отецъ нашъ есть, то и любитель и промыслитель, хранитель, помощникъ, заступникъ нашъ, и проч. \textit{Аще Отцемъ Его нелицемѣрно называемъ}\footnote{1~Петр.~1,~17.}, и какъ отца призываемъ: \textit{Отче нашъ, Иже еси на небесѣхъ}\footnote{Матѳ.~6,~9.}, "--- то какъ Отца должно намъ и любить, и съ любовію имя Его помнить и призывать. "--- 13)~Богъ хощетъ отъ человѣка любимъ быти: любитъ человѣка и отъ него хощетъ любитися, и тако въ дружество съ нимъ войдти. Дружество бо не иное что есть, какъ взаимная любовь, то"=есть, любить и любиму быть. За велико почитаемъ подданному рабу съ царемъ земнымъ дружество имѣть, коль несравненно больше есть человѣку съ Богомъ, убогому созданію съ Создателемъ, земному и перстному съ небеснымъ Царемъ имѣть дружество! Чести сея не токмо словомъ изобразить, но и умомъ понять невозможно. Къ сему такъ высокому достоинству любовію Своею призываетъ насъ Богъ; и благодѣяніями, какъ вѣстниками и свидѣтелями любве Своея, привлекаетъ и убѣждаетъ. Здѣ достойно съ пророкомъ удивиться и воскликнуть: \textit{Господи, что есть человѣкъ, яко помниши его}\footnote{Пс.~8,~5.}? О, коль слѣпы и нечувственны мы, когда отъ такъ высокаго и сладкаго дружества уклоняемся, и отъ живаго и безсмертнаго Бога къ безчувственному созданію обращаемся! Купно и неблагодарни есмы, когда Любителя нашего любить не хощемъ! Сего ради \textit{Тебѣ, Господи, правда: намъ же стыдѣніе лица}\footnote{Дан.~9,~7.}, "--- тако съ пророкомъ признавать и исповѣдаться должно.

\paragraph*{§\:198.} Какъ и какимъ образомъ должно намъ Бога любить, святое Божіе слово показуетъ. \textit{Возлюбиши Господа Бога твоего всѣмъ сердцемъ твоимъ, и всею душею твоею, и всею мыслію твоею}\footnote{Матѳ.~22,~37; Втор.~6,~3.}. Понеже человѣкъ отъ Бога все воспріялъ, бытіе свое, тѣло и душу, жизнь и дыханіе, и безъ Бога жить не можетъ, "--- \textit{въ Немъ бо живемъ и движемся и есмы}\footnote{Дѣян.~17,~28.}, "--- и весь свѣтъ ему повелѣніемъ Божіимъ служитъ; и чрезъ единороднаго Сына Божія такъ чудно возстановленъ падшій и возобновленъ, и къ такъ высокому благородію возведенъ, и потому такою любовію отъ Бога почтенъ, какой большая быть не можетъ; и самую тую любительную силу, которою можетъ Бога любить, съ прочіими душевными силами отъ Бога принялъ: того ради такую любовь одолженъ Богу показывать, какой большая быть не можетъ. Слѣдовательно "--- 1)~долженъ Его любить не токмо паче *всего созданія, паче всѣхъ человѣкъ, паче* брата и друга, паче жены своей и дѣтей, паче отца и матери, но и паче самаго себе. Ибо человѣкъ всѣмъ собою Богу долженъ, тѣло и душу, временную и вѣчную жизнь, и все сіе по любви единой отъ Бога пріялъ\footnote{1~Кор.~4,~7.}. И того ради человѣкъ всего себе любви Божіей предать долженъ, душу и тѣло, сердце все, и разумъ, память и волю, намѣреніе, начинаніе, слово, дѣло и помышленіе Богу въ любовь посвятить, яко вся сія отъ Бога воспріялъ. "--- 2)~Понеже Богъ туне, безъ всякой пользы, человѣка такъ сильно возлюбилъ, то и человѣку должно просто, безъ всякой своей пользы, Бога любить. Ибо, когда кого любимъ ради пользы нашей, не такъ самаго его любимъ, какъ его благодѣяніе и нашу пользу, и потому самихъ себе болѣе любимъ, и тѣмъ показуемъ, что не любили бы его мы, ежели бы не надѣялись отъ него добра какого. "--- 3)~Долженъ Божію волю своей воли предполагать; вмѣсто своей чести, славы, похвалы, "--- Божіей чести, славы и похвалы искать во всякихъ случаяхъ, во всякомъ дѣлѣ, словѣ и помышленіи. "--- 4)~Не токмо честь и славу свою и всякое благополучіе, жену и дѣтей, отца и матерь, друга и брата, но и самый животъ свой презрѣть и оставить долженъ, когда того требуетъ честь и слава Божія. Въ семъ, кажется, разумѣ Хрістосъ глаголетъ\textit{: аще кто грядетъ ко мнѣ, и не возненавидитъ отца своего и матерь, и жену, и чадъ, и братію и сестръ, еще же и душу свою, не можетъ Мой быти ученикъ}\footnote{Лук.~14,~26.}.

\paragraph*{§\:199.} Хотя Богъ и хощетъ, чтобы мы Его любили, однакожъ не ради Своей какой пользы хощетъ того, но ради нашей пользы. Богъ бо, какъ совершенно Самъ въ Себѣ блаженъ есть, такъ и ничего не требуетъ отъ насъ, но намъ все подаетъ, и ради насъ все дѣлаетъ, и ради того, когда любимъ Его истинно и нелицемѣрно, Ему отсюду никакой не бываетъ пользы, но только намъ самимъ любовь сія пользу приноситъ. И отсюду можно всякому видѣть, коль великую Онъ имѣетъ къ человѣку любовь, что, когда и любимъ быть отъ Него хощетъ, не ради Себе того хощетъ, но ради единаго человѣка. Польза же двоякая послѣдуетъ отъ любви Божіей: 1)~Въ семъ вѣцѣ радость духовная, сердечная, утѣшеніе и восклицаніе сердечное, какъ сказано. Любители міра сего радуются о чести, славѣ, богатствѣ, златѣ, сребрѣ, пищѣ и питіи, сладострастіи и роскоши, \textit{идѣже бо сокровище ихъ, тамо и сердце ихъ}\footnote{Матѳ.~6,~21.}. Но боголюбцы не тако. Понеже все сокровище ихъ Богъ единъ, *честь, слава и богатство Богъ единъ*, то о Немъ единомъ утѣшаются и веселятся. Радость же сія бываетъ не отъинуды, какъ отъ благодатнаго въ сердцѣ любящемъ Божія обитанія. \textit{Аще кто любитъ Мя, слово Мое соблюдетъ, и Отецъ Мой возлюбитъ его, и къ нему пріидемъ, и обитель у него сотворимъ}, глаголетъ Хрістосъ\footnote{Іоан.~14,~23.}. Не можетъ бо тамо не быть радость, гдѣ любовь, ибо любовь радость содѣловаетъ, Богъ же есть любовь\footnote{1~Іоан.~4,~8 и 16.}; и потому гдѣ Богъ съ Своею благодатію, тамо и радость. А понеже сокровище сіе "--- радость, глаголю, духовную внутрь имѣютъ, и всегда и вездѣ носятъ ее въ себѣ; то ничто ея отняти не можетъ, ни счастіе, ни несчастіе міра сего, ни честь, ни безчестіе, ни богатство, ни нищета, ни болѣзнь, ни раны, ни скорбь, ни узы, ни темница, ниже самая смерть. Паче же истинному боголюбцу и страдать ради любимаго радостно. Тако \textit{апостоли идяху, радующеся отъ лица собора, яко за имя Господа Іисуса сподобишася безчестіе пріяти}\footnote{Дѣян.~5,~41.}. Тако апостолъ Павелъ \textit{не токмо связанъ быти хотѣлъ, но и умрети во Іерусалимѣ за имя Господа Іисуса готовъ былъ}\footnote{21,~13.}. Тако мученицы святіи на мученіе и на смерть за имя сладчайшаго Іисуса Господа, яко на сладкій духовный пиръ, съ радостію себе предавали. И чимъ кто большую *имѣетъ любовь, тѣмъ большую чувствуетъ* въ себѣ радость, тѣмъ безбоязненнѣе за имя Хрістово подвизается. "--- А понеже не можетъ совершенна здѣ быть любовь ради немощи нашей, то и радость не можетъ быть совершенна, но токмо капля нѣкая, и, какъ малая солнца луча, сквозь облака проходящая, любителей сердца сладко ударяетъ. \textit{Видимъ бо нынѣ яко зерцаломъ въ гаданіи}\footnote{1~Кор.~13,~12.}. "--- 2)~Радость сія совершится въ будущемъ вѣцѣ, гдѣ прекрасное оное и вѣчное Солнце все открыется, и неизреченно увеселитъ любителей и зрителей Своихъ, егда Его не яко зерцаломъ въ гаданіи, но \textit{лицемъ къ лицу увидятъ}, и сладкаго того лицезрѣнія безъ конца и сытости насыщаться будутъ\footnote{ст.~12.}, и наслѣдятъ вся благая оная, \textit{ихже око не видѣ, и ухо не слыша, и на сердце человѣку не взыдоша, яже уготова Богъ любящимъ Его}\footnote{2,~9.}.

\paragraph*{§\:200.} Когда человѣкъ любовь, Богу единому должную, обращаетъ къ себѣ: сіе называется \textit{самолюбіе}, которое не иное что, какъ самаго себе неумѣренное любленіе. И гдѣ самолюбіе имѣется, тамо нѣтъ любви Божіей. Знаки самолюбія примѣчаются сіи: 1)~Знаменіе самолюбія есть, когда кто, оставивши волю Божію, свою исполняетъ, и не дѣлаетъ того, что воля Божія хощетъ, и дѣлаетъ тое, чего воля Божія не хощетъ: отъ чего послѣдуетъ всего закона Божія разореніе. Самолюбіемъ же сіе потому называется, понеже человѣкъ \textit{любитъ себе}, а не Бога*, и угождаетъ себѣ, а не Богу*. А любовь должна угождать любимому, а не себѣ, какъ выше сказано. "--- 2)~Знаменіе самолюбія есть, когда кто уклоняется отъ зла не ради воли Божіей, не хотящей зла, но или ради страха человѣческаго, или ради своей похвалы, или ради иной какой корысти своей; или кто предъ людьми не грѣшитъ, но *грѣшитъ* тайно, что лицемѣрію собственно. "--- 3)~Знаменіе самолюбія есть, когда кто добро дѣлаетъ ради похвалы или славы своей, или иной какой корысти, ибо таковый своей славы, а не Божіей ищетъ. "--- 4)~Хотя кто и не ради тщеславія дѣлаетъ добро, но тое своимъ силамъ приписуетъ, а не Богу, аки бы онъ самъ собою дѣлалъ тое: самолюбецъ есть. Ибо отъемлетъ у Бога славу, которая за всякое добро \textit{Ему единому} должна отдаваться; и привлекаетъ себѣ, который, кромѣ зла, самъ собою \textit{ничего не можетъ дѣлать}\footnote{Іоан.~15,~5; Филип.~2,~13.}. Всякое бо добро отъ Бога происходитъ\footnote{Іак.~1,~17; Римл.~11,~36.}, потому \textit{Ему единому} и приписывать, и во славу Его обращать должно. "--- 5)~Кто какое нибудь дарованіе имѣетъ, напр. богатство, премудрость, разумъ, здравіе и прочая, и тое своему тщанію и трудамъ приписуетъ, а не Богу: такожде самолюбецъ есть. Ибо \textit{все} кромѣ грѣха \textit{отъ Бога} имѣемъ, и самую \textit{душу и тѣло, и жизнь и дыханіе отъ Него имѣемъ}\footnote{Быт.~2,~7; Дѣян.~17,~25.}. Богу же должно и приписывать и благодарить за тое, и прославлять Его, а не себе. Откуду всякая неблагодарность есть знакъ самолюбія. "--- 6)~Знаменіе самолюбія есть, когда кто дарованіе, отъ Бога данное, или сокрываетъ, или не въ славу Божію и пользу ближняго употребляетъ. Таковые суть, которые богатство или хранятъ безъ всякаго употребленія, или на свои прихоти иждиваютъ, какъ"=то на великолѣпныя строенія, на роскошь, тщеславіе, украшеніе, и проч.; такожде кто имѣетъ разумъ и сокрываетъ его, или на зло употребляетъ, напр. на безполезныя сочиненія, на коварныя клеветы, и проч.; кто имѣетъ здравіе и не хощетъ трудитися, и проч. "--- 7)~Знаменіе самолюбія, когда кто, въ бѣдѣ какой и несчастіи находясь, не терпитъ, но ропщетъ. Ибо таковый волю свою Божіей воли предпочитаетъ, безъ которой ничто намъ приключитися не можетъ. "--- 8)~Знаменіе самолюбія есть, когда кто отъ бѣды какой, или и самой смерти съ нарушеніемъ святаго Божія закона ищетъ избавиться; напр., кто въ болѣзни прибѣгаетъ къ чародѣямъ и шептунамъ; такожде кто посредствомъ денегъ, или защитниковъ, ищетъ избавленія: таковый болѣе почитаетъ себе, нежели Божію заповѣдь и Самаго Заповѣдавшаго. "--- 9)~Хотя кто и не ищетъ отъ бѣды избавленія и терпитъ, но ради того терпитъ, чтобы похвалу отъ людей имѣть: такожде самолюбіемъ недугуетъ, ибо не ради Бога терпитъ. "--- 10)~Наконецъ кто что ни дѣлаетъ ради прославленія имене своего, "--- напр. или богатыя строенія созидаетъ, или въ богатое платье одѣвается, или богатство собираетъ, или богатые столы поставляетъ, или рѣчь украшаетъ, или разумъ свой оказываетъ, или что нибудь подобное симъ дѣлаетъ ради похвалы и снисканія славы себѣ; такожде кто нищетствуетъ, ханжею ходитъ, въ рубище, или черную рясу одѣвается, или въ монастырь затворяется, или иный какій смиренія и міра отрицанія образъ показуетъ ради того, чтобы его люди за святаго почитали: самолюбецъ есть и міролюбецъ, а не боголюбецъ. Ибо \textit{Богъ на сердце смотритъ}, а не на внѣшній видъ\footnote{1~Цар.~16,~7.}.

\paragraph*{§\:201.} Отъ вышереченныхъ видно 1)~что какъ боголюбіе есть корень и источникъ всѣхъ благъ, душевныхъ и тѣлесныхъ, мира, покоя, согласія и прочіихъ, такъ самолюбіе есть начало всѣхъ золъ и бѣдъ на свѣтѣ. Какъ бо отъ боголюбія послѣдуетъ тщательное исполненіе воли Божіей, которая всѣхъ благъ намъ желаетъ, такъ отъ самолюбія бываетъ закона Божія презрѣніе и святой воли Его отрицаніе, отъ чего вся злая происходятъ. "--- 2)~Какъ боголюбіе раждаетъ истинную сердечную и неотъемлемую радость, такъ самолюбіе дѣлаетъ ложную, прелестную и мнимую утѣху, которая подобна есть сновидѣнію, вмалѣ явившемуся и исчезнувшему, и вмѣсто того вводитъ истинную сердечную печаль, грызеніе совѣсти, и въ будущемъ вѣкѣ адское мученіе, тѣмъ жесточайшее и страшнѣйшее, чѣмъ болѣе себе грѣшникъ здѣ любилъ и угождалъ. Увидитъ бо тогда, что не иное здѣ любилъ онъ, какъ истинное зло, и потому тѣмъ болѣе себе тогда возненавидитъ, самимъ собою возгнушается, омерзѣетъ, чѣмъ болѣе самому себѣ здѣ угождалъ. "--- 3)~Видно отсюду, коль тяжко грѣшитъ человѣкъ, когда любовь, которую долженъ Богу отдавать, къ себѣ обращаетъ; когда вмѣсто Божіей воли свою исполняетъ; когда послушаніе, которое долженъ Богу, какъ верховному своему Господу, показывать, своей плоти подаетъ; когда славу, похвалу, честь, прославленіе долженъ имени Божію искать и восписывать во всемъ, но вмѣсто того самъ хощетъ прославлятися, "--- и такимъ образомъ, на которомъ мѣстѣ великаго Бога, Господа и Создателя своего долженъ имѣть, на томъ себе, какъ идола, поставляетъ и почитаетъ: еже не иное что, какъ великая неправда и безстыдная вражда противу Бога. \textit{Иже бо восхощетъ другъ быти міру, врагъ Божій бываетъ}\footnote{Іак.~4,~4.}. Міръ бо, котораго ненавидѣть должны мы, внутрь насъ, а не внѣ насъ есть. "--- 4)~Паки видно, коль много согрѣшаемъ вси, такъ, что \textit{грѣхопаденія кто разумѣетъ}? Почему и молиться и воздыхать должно со Псаломникомъ: \textit{отъ тайныхъ моихъ очисти мя}, Господи\footnote{Пс.~18,~13.}! Тайная наша грѣхопаденія суть, которыхъ мы въ совѣсти своей не усматриваемъ. "--- 5)~Отсюду научаемся всю спасенія нашего надежду полагать на единомъ основаніи великаго милосердія Божія и неизчерпаемой благодати единороднаго Сына Его, Господа нашего Іисуса Хріста, а на наше благочестіе не уповать, и всегда къ благости Божіей вопить: \textit{не вниди въ судъ съ рабомъ Твоимъ, яко не оправдится предъ Тобою всякъ живый}\footnote{Пс.~142,~2.}!

\paragraph*{§\:202.} Ежели бы кто сказалъ: какая можетъ быть боголюбцамъ радость, когда многи скорби ихъ окружаютъ, какъ пророкъ глаголетъ: \textit{многи скорби праведнымъ}\footnote{33,~20.}? То сіе истинно есть, что много бѣдъ и скорбей благочестивыя души терпятъ; но тыя скорби извнѣ только ихъ ударяютъ, а душъ ихъ боголюбивыхъ не касаются: напротивъ того, самолюбцевъ и внѣ и внутрь скорби смущаютъ и погружаютъ. Почему далеко болѣе скорбей самолюбцамъ, нежели боголюбцамъ бываетъ. Боголюбцевъ, хотя и окружаютъ скорби и бѣды, но не погружаютъ; бодутъ, какъ терніе розу, но не прободаютъ; покрываютъ, какъ мгла солнце, но не помрачаютъ; біютъ, какъ волны морскія камень, но не разбиваютъ: понеже духовное сокровище и царствіе Божіе \textit{внутрь ихъ есть}\footnote{Лук.~17,~21.}, которое, какъ котва волнующійся корабль, души ихъ содержитъ и благонадежными во всемъ дѣлаетъ. Не тако самолюбцы, но, какъ трость, и малымъ неблагополучія вѣтромъ колеблются и сокрушаются: понеже, хотя внѣ и показуются нѣчто быти, какъ пузырь на водѣ, но внутрь никакой крѣпости не имѣютъ; и ради того, какъ пузырь, исчезаютъ, когда вѣтръ противности повѣетъ на нихъ. Боголюбцы лишаются чести, славы, богатства мірскаго, но не лишаются внутренняго своего сокровища: понеже они честію, славою и богатствомъ не утѣшаются, но далеко лучшее имѣютъ утѣшеніе; и какъ не ищутъ чести и богатства, такъ и потерявше ихъ не скорбятъ. Течетъ имъ богатство? они \textit{не прилагаютъ сердца} къ нему, но единымъ внутреннимъ сокровищемъ довольствуются. Дается имъ честь? Они не такъ съ желаніемъ, какъ съ послушаніемъ пріемлютъ тую; и пріемлютъ не какъ честь, но какъ иго, отъ Бога наложенное, которое должно носить во славу Его и ближняго пользу. Славятъ ли ихъ? они того не пріемлютъ, яко всякая слава единому Богу подобаетъ. И такъ не для чего имъ и скорбѣть, когда отнимается у нихъ тое, чего не искали. Лишаются и самолюбцы, но не тако; но съ скорбію и плачемъ лишаются. Съ великимъ тщаніемъ они ищутъ сокровищъ своихъ сихъ; со страхомъ и опасеніемъ содержатъ и хранятъ ихъ; съ болѣзнію и сѣтованіемъ разстаются съ ними. Какая здѣ утѣха и радость можетъ быть, гдѣ непрестанное попеченіе, страхъ и печаль? Что за радость "--- внѣ златомъ блистать, но внутрь душею мракомъ страха покрываться; внѣ на высокомъ мѣстѣ сидѣть, но внутрь душею на подножіи міра лежать; внѣ славиться, но внутрь отъ совѣсти безчеститься? Воистину прелесть и обманъ и видъ только утѣхи, а не самая утѣха! Радость бо не можетъ быть, какъ на сердцѣ, якоже и печаль не бываетъ, кромѣ сердца. А хотя нынѣ и не лишаются они своихъ сокровищъ, какъ боголюбцы, однакожъ съ большею печалію принуждаются ихъ оставлять, оставляя міръ сей. Терпятъ боголюбцы клеветы, злословія, поношенія отъ людей, но \textit{отъ своей совѣсти} защищаются \textit{и похваляются}\footnote{2~Кор.~1,~12.}; осуждаетъ ихъ злорѣчивый міръ, но \textit{Богъ ихъ оправдаетъ}\footnote{Римл.~8,~33 и 34.}, и тако \textit{хулими утѣшаются}\footnote{1~Кор.~4,~13.}. Терпятъ поношенія и самолюбцы, но не тако, какъ боголюбцы: ибо терпятъ не токмо отъ людей, но и отъ своей совѣсти, которое поношеніе тяжкое есть. А хотя внѣшняго порицанія и злословія и убѣгаютъ, что весьма рѣдко бываетъ, но не могутъ убѣжать внутренняго. Сей обличитель и поноситель вездѣ съ ними есть; нигдѣ не престаетъ ихъ обличать, укорять, осуждать, поносить за преступленіе закона Божія и устрашать праведнымъ судомъ Божіимъ; какъ они ни сокрываютъ свои злодѣянія, но отъ сего назирателя сокрыть не могутъ: ибо внутрь себе имѣютъ его, который и тайные ихъ, сердечные совѣты видитъ и обличаетъ ихъ. "--- Имѣютъ боголюбцы враговъ своихъ, но не имѣютъ къ нимъ вражды, злобы, мщенія, и ради того \textit{внутрь миръ} и покой \textit{имѣютъ}, хотя отвнѣ и безпокойствуются\footnote{Іоан.~14,~27; Римл.~14,~17.}. Имѣютъ и самолюбцы враговъ, паче же между собою другъ съ другомъ враждуютъ; но какъ внѣ, такъ и внутрь безпокойствуются, и болѣе сами себе, нежели враждующіи ихъ безпокойствуютъ. Не можно словомъ изобразить, сколько полагаютъ тщанія и трудовъ, сколько теряютъ суммы, сколько отираютъ пороговъ у своихъ защитниковъ и судей, сколько имъ угожденій показуютъ, чтобы зло за зло воздать и обиду за обиду сдѣлать; сколько другъ на друга клеветъ соплетаютъ, судебныя о томъ свидѣтельствуютъ мѣста, которыя злохитрыми ихъ клеветами отягчены, и такимъ образомъ не токмо сами себе, но и другихъ въ безпокойствіе немалое приводятъ. Тако они ищутъ славы, но болѣе обезславляются, ищутъ чести, но болѣе безчестятся. Кто бо когда злобнаго и клеветника похвалилъ? Сами они такими гнушаются, но въ себѣ того не усматриваютъ. Какое сего бѣднѣйшее можетъ быть состояніе? Аще бы кому въ души ихъ посмотрѣть можно было, увидѣлъ бы, что болѣе они смущаются различными мысльми, нежели море волнами. Тако отторгшемуся отъ воли и любви Божіей, какъ тихаго пристанища, слѣдуетъ непремѣнно различными и опасными на морѣ міра волнами обуреватися! Не тако боголюбцы, не тако; но тихи, покойны, мирны. И хотя ненавидятъ, враждуютъ и озлобляютъ ихъ враги ихъ, однакожъ душами своими, яко \textit{чада мира} и покоя, безмятежно и сладко на пресладкомъ любве Божіей лонѣ почиваютъ, и всепріятнаго своего покоя не хотятъ оставить. Боголюбцы, понеже \textit{единаго} Бога боятся, никого кромѣ Его не боятся, хотя и всѣхъ любятъ и почитаютъ о томжде Бозѣ, и тако подъ всесильнымъ кровомъ крилъ Его безопасны пребываютъ. Самолюбцы не тако: но, понеже Божій страхъ отринули, принуждены всего опасаться. Надобно бо тому созданія бояться, который Создателя не боится. \textit{Бѣжитъ нечестивый, никомуже гонящу}\footnote{Притч.~28,~1.}. Совѣсть едина паче всякаго гонителя гонитъ его: гдѣ бы ни былъ, что бы ни дѣлалъ, вездѣ надъ нимъ гремитъ и устрашаетъ его. Судія неправедный, который указы монаршіи дерзаетъ нарушать, кого не боится? Самые слуги его и рабы дѣлаютъ страхъ ему; и други не безъ подозрѣнія. Тако всякъ законопреступникъ \textit{тамо боится, гдѣ нѣтъ страха}\footnote{Пс.~13,~5; 52,~6.}. Страхъ отъ домашнихъ, страхъ отъ внѣшнихъ, страхъ отъ враговъ, подозрѣніе и страхъ отъ друговъ; страхъ, чтобы не потерять чести; страхъ, чтобы не лишиться богатства; страхъ, чтобы не подпасть гнѣву цареву; страхъ, чтобы не пострадать зла отъ злыхъ; страхъ, чтобы не посрамиться предъ добрыми; страхъ отъ совѣсти, страхъ отъ суда Божія, страхъ отъ геенны, страхъ отъ діавола. Тако небоящагося Бога вездѣ срѣтаетъ страхъ! Какое се житіе, когда въ такой тѣснотѣ находится бѣдная душа?! Какой радости и утѣшенію быть тамо, гдѣ такое смущеніе и волнованіе совѣсти?! Не тако боголюбецъ; онъ всегда и вездѣ съ пророкомъ глаголетъ: \textit{Господь просвѣщеніе мое и Спаситель мой, кого убоюся? Господь защититель живота моего, отъ кого устрашуся}\footnote{Пс.~26,~1.}? \textit{На Бога уповахъ: не убоюся, что сотворитъ мнѣ человѣкъ}\footnote{55,~12; 117,~6.}. \textit{Аще и пойду посредѣ сѣни смертныя, не убоюся зла: яко Ты со мною еси}, Боже\footnote{22,~4.}. Что вожделѣннѣе можетъ быть, какъ и посредѣ самой сѣни смертной не бояться! Заключаются въ темницѣ боголюбцы, облагаются узами; но не лишаются духовной своей свободы, которая всегда съ ними есть, духа бо вязати никто не можетъ. Лишаются общаго сего свѣта, но не лишаются внутренняго просвѣщенія. Уязвляются ранами на тѣлѣ, но сладкимъ чистыя совѣсти свидѣтельствомъ облегчаются отъ болѣзни: ибо вся сія не ради инаго чего, какъ \textit{ради правды страждутъ}, и ради того радуются и веселятся духомъ\footnote{1~Петр.~3,~14.}. Заключаются и уязвляются и самолюбцы, но не тако, якоже боголюбцы. Понеже заключаются за свои злодѣянія, воровство, хищеніе, лихоимство и проч., того ради не токмо тѣломъ, но и душею страждутъ, совѣстью какъ мечемъ прободаеми. И хотя многіе хитростію своею и избѣгаютъ сего заключенія и біенія; но совѣстнаго біенія, яко домашняго своего и внутренняго мучителя, и вѣчныя темницы не избѣгнутъ, аще покаяніемъ истиннымъ не очистятся. Не вси бо злодѣи по недовѣдомымъ судьбамъ Божіимъ здѣ наказуются. "--- Удаляются боголюбцы отъ отца и матери, братіи и друзей; но не удаляются отъ Бога, Который \textit{вездѣ съ ними есть} "--- и въ темницѣ и ссылкѣ; и паче отца и матери, братіи и друзей, милостивнымъ Своимъ присутствіемъ \textit{утѣшаетъ ихъ}\footnote{Ис.~41,~10; 43,~1--3; 2~Кор.~1,~4.}. Удаляются и самолюбцы друзей и братіи своей, но купно и утѣхи своей лишаются. "--- Изгоняются боголюбцы на чужую страну; но понеже они \textit{не имѣютъ здѣ пребывающаго града, но грядущаго взыскуютъ}\footnote{Евр.~13,~14.}, то имъ странствовать вездѣ равно есть: ибо имъ житіе въ мірѣ семъ не иное что, какъ ссылка, изъ которой всегда къ отечеству своему "--- горнему Сіону зрятъ и \textit{воздыхаютъ, въ жилище небесное облещися желающе}\footnote{2~Кор.~5,~2.}. Изгоняются и самолюбцы; но, понеже они не грядущаго, но настоящаго упокоенія взыскуютъ, того ради, лишившеся того, неутѣшно и безполезно сѣтуютъ, и чѣмъ болѣе воспоминаютъ тое, тѣмъ болѣе сокрушаются. "--- Плачутъ въ мірѣ семъ боголюбцы; но и плачь ихъ радостію бываетъ растворенъ: ибо плачутъ или ради того, что \textit{пришельствіе ихъ продолжися} въ семъ мірѣ\footnote{Пс.~119,~5.}, или ради того, что за немощь плоти не могутъ достойныя любви человѣколюбцу Богу воздать; или ради того, что видятъ законъ Божій отъ беззаконныхъ въ попраніи, якоже пророкъ глаголетъ: \textit{печаль пріятъ мя отъ грѣшникъ, оставляющихъ законъ Твой}\footnote{118,~53.}, "--- и тако Законодавца презираема и презирающихъ погибель видятъ. Но понеже печаль сія есть \textit{печаль по Бозѣ}, того ради есть радости виновна, и, какъ послѣ дождя воздухъ, такъ сердца ихъ послѣ слезъ прохлаждаются и утѣшенія божественнаго сподобляются. Плачутъ и самолюбцы, но понеже о погубленіи богатства, или чести, или славы, или иныхъ своихъ утѣшеній плачутъ, то чѣмъ болѣе плачутъ, тѣмъ болѣе умножаютъ печаль свою и къ печали печаль пріобрѣтаютъ. Ибо сія ихъ печаль есть \textit{печаль міра сего}, которая, по апостолову ученію, \textit{смерть содѣловаетъ}\footnote{2~Кор.~7,~10.}. Тако видимъ, что не токмо боголюбцы, но и самолюбцы въ мірѣ семъ страждутъ. Лишаются чести и имѣнія боголюбцы: лишаются и самолюбцы. Терпятъ поношеніе и злословіе боголюбцы: терпятъ и самолюбцы. Затворяются въ темницахъ боголюбцы: затворяются и самолюбцы. Уязвляются боголюбцы: уязвляются и самолюбцы. Посылаются въ заточенія, отчуждаются домовъ, родителей, друзей и братій боголюбцы: страждутъ тоежъ и самолюбцы. Умерщвляются боголюбцы: умерщвляются и самолюбцы. Общая страданія, но не общая вина страданій: тіи \textit{правды ради}, а сіи злодѣяній ради. И понеже тіи безвинно страждутъ, то не только не скорбятъ, но и \textit{радуются въ страданіяхъ}\footnote{Кол.~1,~24.}, и ради того хотя тѣломъ страждутъ, но духомъ веселятся. Сіи, понеже по дѣломъ своимъ страждутъ, и тѣломъ и душею страждутъ; совѣстію своею паче всякаго мучителя уязвляеми, и ради того двоякое терпятъ страданіе "--- тѣлесное и совѣстное. Откуду всякъ можетъ заключить, что самолюбцамъ и въ семъ вѣкѣ множайшіи скорби, нежели боголюбцамъ; а отсюду и тое признать должно, что какъ боголюбіе радости сердечныя, такъ и самолюбіе печали виновно бываетъ. Боголюбцы, наконецъ, оканчивая многобѣдное сіе житіе, оканчиваютъ и скорби своя, и къ совершенной преходятъ радости, и, \textit{мало нынѣ прискорбни бывше въ различныхъ напастехъ, великое получаютъ веселіе}\footnote{1~Петр.~1,~6--8.}, и \textit{еже нынѣ легкое печали ихъ, тяготу вѣчныя славы содѣловаетъ имъ}\footnote{2~Кор.~4,~17.}. Самолюбцы отъ мнимыхъ утѣшеній своихъ къ истиннымъ скорбямъ, и отъ временныхъ къ вѣчнымъ бѣдствіямъ при окончаніи житія своего преселяются. И тако какъ отъ боголюбія всякое блаженство, такъ отъ самолюбія всякое окаянство послѣдуетъ.

\subsection[Глава 5-я. О узкомъ пути.]{глава пятая.\\\bfseries О узкомъ пути.}

\begin{quotation}\textit{Внидите узкими враты: яко пространная врата и широкій путь вводяй въ пагубу, и мнози суть входящіи имъ. Что узкая врата и тѣсный путь вводяй въ животъ, и мало ихъ есть, иже обрѣтаютъ его}, глаголетъ Хрістосъ\footnote{Матѳ.~7,~13 и 14.}.\end{quotation}
\begin{quotation}\textit{Въ мірѣ скорбни будете}, глаголетъ Хрістосъ\footnote{Іоан.~16,~33.}.\end{quotation}
\begin{quotation}\textit{Многими скорбьми подобаетъ намъ внити въ царствіе небесное}, глаголетъ апостолъ\footnote{Дѣян.~14,~22.}.\end{quotation}
\begin{quotation}\textit{Вси, хотящіи благочестно жити о Хрістѣ Іисусѣ, гоними будутъ. Лукавіи же человѣцы и чародѣи преуспѣютъ на горшее, прельщающе и прельщаеми}\footnote{2~Тим.~3,~12 и 13.}.\end{quotation}


\paragraph*{§\:203.} Житіе человѣческое есть путь отъ самаго часа рожденія до часа окончанія. Сей путь двоякій "--- \textit{тѣсный и пространный}. Тѣсный путь предѣломъ закона Божія огражденъ, и не попущающій намъ ни на десно, ни на шуее уклоняться, и по своей воли скитаться: пространный путь есть, который ограду закона Божія разрушилъ и даетъ свободу по своему изволенію бродить идущимъ по нему. Тѣсный путь окружаютъ скорби, бѣды, изгнанія, безчестія и поруганія: пространный вся сія отвергаетъ, и распространяетъ себе веселостьми міра сего. На тѣсномъ пути при самомъ входѣ стоитъ крестъ, и съ нимъ смиреніе, самаго себе отверженіе, терпѣніе, кротость и всякая добродѣтель, внѣ презрѣнна и умаленна, но внутрь великолѣпна, какъ царица ризою позлащенною одѣта: на пространномъ самолюбіе, и при немъ славолюбіе, любочестіе, сребролюбіе, злоба и всякій грѣхъ, внѣ сладокъ и надменъ, но внутрь горекъ и смраденъ. «На тѣсномъ пути, глаголетъ Василій великій, умерщвленіе плоти: на пространномъ угожденіе плоти. На тѣсномъ постъ: на пространномъ піянство. На тѣсномъ слезы: на пространномъ смѣхъ неумѣренный. На тѣсномъ молитва: на пространномъ танцованіе. На тѣсномъ воздыханіе: на пространномъ веселіе и лики. На тѣсномъ чистота: на пространномъ блудъ»\footnote{На пс.~1"~й.}.

\paragraph*{§\:204.} На тѣсномъ пути предводитель Самъ Хрістосъ, Сынъ Божій, крестъ понесшій и на крестѣ распятый, Который глаголетъ всѣмъ хотящимъ быть хрістіанами: \textit{иже хощетъ по мнѣ ити, да отвержется себе и возметъ крестъ свой и послѣдуетъ Ми}\footnote{Лук.~9,~23.}. Но сему пути идутъ и послѣдуютъ Ему вѣрніи Его, міра сего и его похотей отрекшіися, и крестъ свой вземшіи, и Ему, яко вождю своему и Начальнику, смиреніемъ, любовію, терпѣніемъ, кротостію и послушаніемъ послѣдующіи. На пространномъ міродержецъ, князь міра сего и тмы, по которому гуляютъ міру сему работающіи, и ему послѣдуютъ гордостію и величаніемъ.

\paragraph*{§\:205.} Причины, ради которыхъ благочестивые, идучи по пути Хрістову, скорби терпятъ, примѣчаются сіи: 1)~Сатана, котораго ига благодатію Божіею свободились, брань на нихъ возставляетъ, различнымъ образомъ искушаетъ ихъ, и такъ хощетъ отъ Хріста отторгнути, и паки подъ свою власть темную покорити; и когда не можетъ того учинить, озлобляетъ ихъ чрезъ своихъ служителей "--- злыхъ людей. Противу сихъ \textit{змій} оный \textit{великій, змій древній}, подвизается, \textit{иже заповѣди Божіи соблюдаютъ, имѣютъ свидѣтельство Іисусъ Хрістово}\footnote{Апок.~12,~17.}. "--- 2)~Сатана паки, не могучи сдѣлать ничего Хрісту, Который царство его темное разрушилъ, устремляется на рабовъ Его, и въ рабахъ Господа хощетъ озлобить. «Діаволъ, глаголетъ Василій великій, не могучи Богу Самому обиды сдѣлать, на образъ Его, то"=есть, человѣка, ненависть свою обратилъ»\footnote{Сл. въ Лакизахъ.}. "--- 3)~Тойжде злый духъ завидуетъ вѣчному благочестивыхъ блаженству *и великой славѣ*, и тщится ихъ отъ того отторгнуть всякими напастьми, якоже прародителей нашихъ благополучію позавидѣлъ и низринулъ въ бѣдственное состояніе\footnote{Быт.~3"~я.}. "--- 4)~Паки тойжде лукавый духъ, не могучи душъ благочестивыхъ искушеніемъ повредить, тщится хотя тѣло ихъ озлобить: ибо ему, яко злобному духу, радость бываетъ, какъ нибудь человѣка, а паче вѣрнаго озлобить. "--- 5)~Благочестивые не суть отъ міра сего, но отечество ихъ небо: того ради міръ ихъ ненавидитъ и гонитъ, яко не своихъ. \textit{Аще отъ міра бысте были}, глаголетъ Хрістосъ, \textit{міръ убо свое любилъ бы: якоже отъ міра нѣсте, но Азъ избрахъ вы отъ міра, сего ради ненавидитъ васъ міръ}\footnote{Іоан.~15,~19.}. Якоже море умершихъ животныхъ вонъ извергаетъ: тако міръ сей, который, какъ море, бѣдами и напастьми волнуется, всякаго умершаго похоти плотской, похоти очесъ и гордости житейской, еже все въ мірѣ есть, изгоняетъ. "--- 6)~Добрые злымъ не сообщаются, но злобу ихъ обличаютъ святымъ своимъ житіемъ: того ради злые добрыхъ гонятъ. Отсюду бываетъ, что въ единомъ дому злый мужъ добрую жену, злый отецъ добраго сына, злый братъ добраго брата, злая сестра добрую сестру ненавидитъ и гонитъ. И сіе"=то есть, что Хрістосъ глаголетъ: \textit{мните ли, яко мира пріидохъ дати на землю? ни, глаголю вамъ, но раздѣленіе. Будутъ бо отселѣ пять во единомъ дому раздѣлены, тріе на два, и два на три. Раздѣлится отецъ на сына, и сынъ на отца, мать на дщерь, и дщерь на матерь, свекры на невѣсту свою, и невѣста на свекровь свою}\footnote{Лук.~12,~51--53.}. "--- 7)~Богъ попускаетъ на нихъ напасти, чтобы они памятовали, что они здѣ странники суть и пришельцы (отечество же ихъ не въ мірѣ семъ, но на небеси), и тако бы отъ суеты удалялися. \textit{Егоже бо любитъ Господь, наказуетъ}\footnote{Евр.~12,~6.}. "--- 8)~Сими бѣдами они смиряются, познаютъ немощь свою. Какъ бо счастіе возноситъ, такъ бѣда смиряетъ человѣка и въ познаніе себе приводитъ. "--- 9)~Тако научаются изрядной добродѣтели терпѣнія, которая вѣнецъ есть благочестія, и которой безъ искушенія и скорби научиться намъ невозможно. \textit{Скорбь бо терпѣніе содѣловаетъ, терпѣніе же искусство, искусство же упованіе; упованіе же не посрамитъ}, глаголетъ апостолъ\footnote{Римл.~5,~3--5.}. «Смотри, глаголетъ Василій великій, къ чему скорбь приводитъ? Къ упованію, еже не посрамитъ. Немоществуешь? радуйся, понеже \textit{егоже любитъ Господь, наказуетъ}. Нищъ еси? веселися, яко съ Лазаремъ благая наслѣдиши. Ради Хрістова имени поношеніе терпишь? Блаженъ еси, яко сіе тебѣ поношеніе въ ангельскую славу обратится»\footnote{На пс.~59"~й.}. "--- 10)~Чѣмъ болѣе люди благочестивые здѣсь бѣдъ и искушеній терпятъ, тѣмъ болѣе въ вѣчномъ животѣ прославятся. \textit{Вмалѣ наказани бывше, великими благодѣтельствовани будутъ: яко Богъ искуси ихъ, и обрѣте ихъ достойны Себѣ. Яко злато въ горниле искуси ихъ, яко всеплодіе жертвенное пріятъ я. И во время посѣщенія ихъ возсіяютъ, и яко искры по стеблію потекутъ}\footnote{Прем.~3,~5--7.}. Тогда они \textit{вознепщуютъ, что недостойны страсти нынѣшняго вѣка къ хотящей явитися славѣ}\footnote{Римл.~8,~18.}. "--- 11)~Тако они \textit{сообразны дѣлаются Хрісту Сыну Божію}, яко духовніи уды Главѣ, Который всякія бѣды, скорби и страданія претерпѣлъ нашего ради спасенія. "--- 12)~Тако познаютъ \textit{силу грѣха, жало} оное \textit{смерти} познаютъ, коль великое зло грѣхъ есть, который толико бѣдъ и напастей міру навелъ, да отъ него берещися тщатся. "--- 13)~Тако познаютъ гнѣвъ Божій противу грѣха и исповѣдаютъ правду Божію, якоже Давидъ глаголетъ: \textit{праведенъ еси, Господи, и прави суди Твои}\footnote{Пс.~118,~137.}! "--- 14)~Тако научаются познавать самымъ дѣломъ страданіе Хрістово и его дорого и высоко почитать. Какъ сладость меда тогда познаемъ добрѣ, когда его вкушаемъ; и горесть бѣдствій тогда наипаче чувствуемъ, когда сами ими искусимся: тако горесть страданій Хрістовыхъ тогда лучше познаемъ, когда сами горести бѣдъ и напастей вкусимъ. А отъ сего научаемся вкусить и видѣть, \textit{коль благъ Господь есть}\footnote{Пс.~33,~9.}, Который толикую горести чашу испилъ за непотребныя рабы своя! "--- 15)~Бѣды и скорби подвигаютъ и возбуждаютъ насъ къ истинному покаянію, къ отвращенію отъ прелести міра и исканію истиннаго благополучія отъ Бога, которое никакими бѣдами отъятися не можетъ; подвигаютъ къ истинной молитвѣ, и проч.

\paragraph*{§\:206.} Тѣсный путь, понеже тѣсенъ есть и скорбенъ, того ради мало такихъ есть, по словеси Хрістову, которые обрѣтаютъ его. Ибо сласти и веселости міра сего чувствамъ нашимъ подлежатъ и плоти нашей угодны суть: вѣчное же блаженство, къ которому тѣсный путь ведетъ, понеже не плотскими, но душевныма очима, вѣрою просвѣщенными, зрится, того ради не всякъ видитъ его, развѣ вѣрою просвѣщенный. А кто не видитъ его, тотъ и не ищетъ его. И скорби терпѣть плоти нашей горестно. Почему многіе, послѣдуя плоти страстной и похотствующей, оставляютъ путь тѣсный, и на пространный всходятъ, и идутъ по нему, какъ плоти угодно.

\paragraph*{§\:207.} Тѣсный путь, хотя и тѣсенъ и немногіе по нему идутъ, но въ пространство вѣчнаго веселія и утѣхи приводитъ: пространный, хотя и широкъ и многіе по нему идутъ, но въ тѣсноту вѣчныхъ скорбей, печалей и воздыханій вводитъ. Лазарь, во Евангеліи упоминаемый, тѣснымъ болѣзни и нищеты путемъ шелъ; но по смерти несенъ бысть ангелы на лоно Авраамле: напротивъ того богачъ, таможде положенный, \textit{облачашеся въ порфиру и виссонъ, веселяся на вся дни свѣтло}, и нищему и многоболѣзненному Лазарю не хотѣлъ милости показать; но погребенный во адѣ возводитъ очи свои, сый въ мукахъ, и по многихъ веселостяхъ многую скорбь терпитъ, и по драгихъ винахъ капли воды проситъ, но не получаетъ\footnote{Лук.~16,~19--26.}. Чего ради Хрістосъ глаголетъ: \textit{внидите узкими враты: яко пространная врата и широкій путь вводяй въ пагубу, и мнози суть входящіи имъ. Что узкая врата и тѣсный путь вводяй въ животъ, и мало ихъ есть, иже обрѣтаютъ его}.

\paragraph*{§\:208.} Страшенъ путь сей плоти нашей. Ибо она хощетъ въ пространствѣ и во всякихъ веселостяхъ, чувствамъ подлежащихъ, быть; она любитъ отъ всѣхъ почитаема, похваляема и покланяема быть; она ищетъ во всемъ по воли своей жить. Хотящему же на путь сей взойти, должно отрещися всего сего: должно отсѣщи волю свою; должно оставить прихоти свои; должно не ужасаться безчестія, руганія, нищеты, изгнанія и прочіихъ противностей; не должно помышлять объ отмщеніи, и проч. На вся сія она смотритъ, а того не видитъ, чего чувства понять и разумъ постигнуть не можетъ. Какъ немощный не искусившійся боится лѣкарства принимать, не зная въ немъ крыющіяся пользы; или какъ младенцы убѣгаютъ отъ бани и плачутъ, когда матери ихъ омываютъ, понеже видятъ только настоящее, а дальняго не усматриваютъ: тако немощная плоть наша на тое только смотритъ, что чувствамъ подлежитъ, но не видитъ того, что отъ чувствъ удаленное есть. Видитъ крестъ и горесть его, но пользы и утѣшенія, горесть тую услаждающія, не видитъ: того ради смущается и ужасается. Но чтобы смущеніе и страхъ сей уничтожился, или умалился, хотящему узкими вратами внити и тѣснымъ путемъ итить должно не токмо на видимыя, но и на невидимыя его обстоятельства смотрѣть; не токмо на едины скорби и бѣды, которыя показуютъ чувства и тѣми смущаютъ насъ, но и на тое, что представляетъ вѣра хрістіанская, обращать разсужденіе и вниманіе. Она представляетъ намъ Бога, яко Отца, Который чадамъ Своимъ налагаетъ крестъ отъ единой любви, а не отъ гнѣва. \textit{Егоже любитъ Господь, наказуетъ; біетъ же всякаго сына, егоже пріемлетъ. Аще наказаніе терпите, якоже сыновомъ обрѣтается вамъ Богъ}\footnote{Евр.~12,~6 и 7.}. Она утѣшаетъ, что тойжде милосердый Отецъ присутствуетъ и помогаетъ чадомъ Своимъ, подъ крестомъ находящимся и воздыхающимъ. \textit{Съ нимъ есмь въ скорби и изму его}\footnote{Пс.~90,~15.}. \textit{Аще преходиши сквозѣ воду, съ тобою есмь, и рѣки не покрыютъ тебе, и аще сквозѣ огнь пройдеши, не сожжешися, и пламень не опалитъ тебе}\footnote{Ис.~63,~2.}. Она указуетъ на Начальника и Предводителя Іисуса Хріста Сына Божія, Который путемъ крестнымъ предшелъ, и Своими прекрасными ногами жестокость его умягчилъ, и слѣдующимъ по Себѣ удобнымъ сотворилъ. Она показуетъ желаемый конецъ, къ которому крестный путь приводитъ; она утверждаетъ, что по сей горести послѣдуетъ вѣчная сладость, *по временной скорби настанетъ вѣчная радость* и веселіе, по временномъ безчестіи и поношеніи воспріиметъ вѣчная слава, по временной смерти будетъ вѣчная блаженная жизнь "--- и тако горесть креста услаждаетъ или растворяетъ. Надежда бо и самые неудобные облегчеваетъ труды, и горькое услаждаетъ. Сіе примѣчаемъ и въ надеждѣ временныхъ благъ. Воина надежда побѣды и славы вызываетъ на подвигъ противу непріятеля; купца надежда богатства поощряетъ скитаться по чужимъ странамъ и принимать всякія бѣды; земледѣльца надежда плодовъ нелѣностнымъ дѣлаетъ въ поднятіи трудовъ; больнаго надежда здравія ободряетъ горькое и жестокое принимать врачевство. Кольми паче твердая вѣчныхъ благъ надежда подвигнетъ сердце къ подъятію всякаго бѣдствія и злополучія; а притомъ и \textit{всесильную} обѣщаетъ Божію помощь. (О семъ смотри пространнѣе въ слѣдующемъ параграфѣ).

\subsection[Глава 6-я. О терпѣніи.]{глава шестая.\\\bfseries О терпѣніи.}

\begin{quotation}\textit{Въ терпѣніи вашемъ стяжите души ваша}\footnote{Лук.~21,~19.}.\end{quotation}
\begin{quotation}\textit{Се есть угодно предъ Богомъ, аще совѣсти ради Божія терпитъ кто скорби, стражда безъ правды. Кая бо похвала, аще согрѣшающе мучими терпите? Но аще добро творяще и страждуще терпите, сіе угодно предъ Богомъ. На сіе бо и звани бысте}\footnote{1~Петр.~2,~19--21.}.\end{quotation}
\begin{quotation}\textit{Долготерпите убо, братіе моя, до пришествія Господня. Се земледѣлецъ ждетъ честнаго плода отъ земли, долготерпя о немъ, дондеже пріиметъ дождь ранъ и позденъ. Долготерпите убо и вы, утвердите сердца ваша: яко пришествіе Господне приближися. Не воздыхайте другъ на друга, братіе, да не осуждени будете: се Судія предъ дверьми стоитъ! Образъ пріимите, братіе моя, злостраданія и долготерпѣнія "--- пророки, иже глаголаша именемъ Господнимъ. Се блажимъ терпящія! Терпѣніе Іовле слышасте, и кончину Господню видѣсте, яко многомилостивъ есть Господь и щедръ}\footnote{Іак.~5,~7--11.}.\end{quotation}
\begin{quotation}\textit{Терпѣніемъ да течемъ на предлежащій намъ подвигъ, взирающе на Начальника вѣры и Совершителя Іисуса, Иже, вмѣсто предлежащія Ему радости, претерпѣ крестъ, о срамотѣ нерадивъ, одесную же престола Божія сѣдѣ}\footnote{Евр.~12,~1 и 2.}.\end{quotation}


\paragraph*{§\:209.} Житіе человѣческое отъ самаго рожденія до окончанія всякими бѣдами исполнено: раждаемся со слезами; живемъ въ бѣдахъ и слезахъ; окончиваемъ житіе съ воздыханіемъ и страхомъ. И хотя многіе, по мнѣнію нашему, въ благополучіи живутъ, однакожъ нѣтъ такого благополучія, которое бы горестію неблагополучія не растворено было. Противу всякаго бѣдствія есть изрядное врачевство "--- терпѣніе.

\paragraph*{§\:210.} Какъ храбрость воина не познается, какъ только во время сраженія, такъ истиннаго терпѣнія не познаемъ, какъ во время приключившихся бѣдъ и противностей. Многіе мнятъ о себѣ, аки бы имѣютъ терпѣніе; но приключившееся бѣдствіе показываетъ, терпѣливы ли они суть. Не тотъ искусенъ навклиръ, который во время тишины и благополучнаго вѣтра добрѣ управляетъ кораблемъ; но тотъ, который во время непогоды и бури дѣло свое добрѣ исправляетъ: тако не тотъ терпѣливъ, который во время благополучія тихо и кротко себе ведетъ; но тотъ, который во время бѣдствія сердце свое, аки кораблецъ, бѣдствіемъ возмущаемъ, укрощаетъ и усмиряетъ. Терпѣніе бо бѣдствія терпѣніе есть, а не благополучія.

\paragraph*{§\:211.} Терпѣніе не въ томъ состоитъ, чтобы много дознавать и терпѣть бѣдъ; но въ томъ, чтобы ихъ великодушно срѣтать и не ропотно, но терпѣливно сносить, и гнѣвъ востающій укрощать, усмирять и побѣждать, отъ роптанія, негодованія и гнѣва сердце свое удерживать, и сдаваться во всемъ на волю Божію. Бѣды бо и благочестивымъ и нечестивымъ приключаются; но едины только благочестивые терпѣливо и великодушно сносятъ ихъ. Добрые и злые лишаются имѣній; "--- добрые съ Іовомъ глаголютъ\textit{: Господь даде, Господь и отъятъ: буди воля Господня}!\footnote{Іов.~1,~21.}: злые ропщутъ и негодуютъ, а часто и хулятъ. Добрые и злые пріемлютъ обиды отъ другихъ; "--- добрые обидѣвшимъ прощаютъ, и, хотя могутъ мстить, не отмщеваютъ и не хотятъ отмщевать: злые не терпятъ того, но обиду обидою тщатся отвратить, и, когда не могутъ дѣломъ, словомъ язвительнымъ и клеветою отмщеваютъ. Терпѣніе бо истинное есть "--- не токмо не отмщевать, но и не хотѣть отмщевать обидѣвшему, хотя бы сердце и поощряло къ тому. Большее и высочайшее терпѣніе есть не чувствовать обиды и болѣзни въ сердцѣ своемъ. Таковый подлинно міру, плоти и грѣху \textit{умре}, а \textit{живетъ} Хрісту. Таковый ни похвалою возносится, ни поношеніемъ оскорбляется; какъ въ благополучіи, такъ и въ злополучіи равенъ, постояненъ, тихъ и кротокъ. Преблагополучное есть такое сердца состояніе! Лучшее есть отъ всѣхъ міра сего увеселеній! Сладкая есть сія тишина! Преблагопріятный покой, въ которомъ такъ сердце упокоевается! Къ сему покою призываетъ насъ Хрістосъ: \textit{возмите иго Мое на себе, и научитеся отъ Мене, яко кротокъ есмь и смиренъ сердцемъ: и обрящете покой душамъ вашимъ}\footnote{Матѳ.~11,~29.}.

\paragraph*{§\:212.} Истинное терпѣніе хрістіанское отъ вѣры происходитъ, и безъ вѣры быть не можетъ. Понеже плоть наша всегда хощетъ воли своей послѣдовать, въ пріятностяхъ и увеселеніяхъ міра сего быть: почему всякой противности ужасается и, когда приключится, смущается и негодуетъ. Но вѣра, въ сердцѣ живущая, сіе ея смущеніе укрощаетъ и усмиряетъ, представляя, что все по Божію промыслу бываетъ; что Богъ посылаетъ наказаніе не отъ гнѣва, но отъ любви: \textit{егоже бо любитъ Господь, наказуетъ}\footnote{Евр.~12,~6.}; что скоро сему бѣдствію конецъ будетъ; что временной скорби послѣдуетъ радость вѣчная; что нетерпѣніемъ и негодованіемъ благость Божія оскорбляется, "--- и тако смущаемое и волнующееся сердце укрощаетъ и утѣшаетъ. Тако Давидъ святый въ печали и скорби душу свою вѣрою утѣшалъ. \textit{Вскую прискорбна еси, душе моя? и вскую смущаеши мя? Уповай на Бога, яко исповѣмся Ему, спасеніе лица моего и Богъ мой}\footnote{Пс.~61,~6.}.

\paragraph*{§\:213.} Терпѣніе убо есть добродѣтель, которая во всякой противности, печали и скорби приключающейся, предается воли Божіей, и изволяетъ все страдати паче, нежели предъ Богомъ согрѣшити. Или, терпѣніе есть добродѣтель, укрѣпляющая сердце въ подвигѣ крестномъ, и сохраняющая отъ негодованія и роптанія, и научающая на волю Божію во всемъ предаватися; и не иное что есть, какъ самая вѣра, подвизающаяся противу самолюбія плоти и \textit{покоряющая ее въ послушаніе воли Божіей}. И хотя вѣрный, немощію плоти убѣждаемый, молится, и проситъ избавленія отъ скорби: \textit{духъ бо бодръ, плоть же немощна}\footnote{Матѳ.~26,~41.}; однакожъ, послѣдуя Начальнику вѣры и Совершителю Іисусу Сыну Божію, прилагаетъ слово: \textit{не моя воля, но Твоя да будетъ, Отче}\footnote{ст.~42; Лук.~22,~42; Марк.~14,~36.}. Тако Давидъ святый, отецъ Хрістовъ по плоти, въ гоненіи и печали своей глаголаше: \textit{аще обрящу благодать предъ очима Господнима, и возвратитъ мя, и покажетъ ми его} (кивотъ Божій) \textit{и лѣпоту его. И аще тако речетъ: не благоволихъ въ тебѣ: се азъ есмь, да сотворитъ ми по благому предъ очима Своима}\footnote{2~Цар.~15,~25 и 26.}?

\paragraph*{§\:214.} Причины, къ терпѣнію возбуждающія: 1)~Все, что ни бываетъ, по Божію святому промыслу бываетъ. \textit{Благая и злая, животъ и смерть, нищета и богатство отъ Господа суть}, глаголетъ Сирахъ\footnote{Сир.~11,~14.}. Убо тѣмъ должно удоволяться и укрѣплять себе въ терпѣніи, что \textit{все отъ Господа} происходитъ, и тако святой Его воли предаваться, и въ бѣдахъ глаголати съ Іовомъ: \textit{аще благая пріяхомъ отъ руки Господни, злыхъ ли не стерпимъ}\footnote{Іов.~2,~10.}? Отъ Бога бо, яко благаго и Источника благихъ, ничтоже произойти можетъ, токмо благое. И хотя плотскому нашему разсужденію бѣдствіе кажется быти зло: но, понеже отъ Бога \textit{благаго} посылается ради пользы нашей, благое есть, ибо оно хощетъ насъ духовно благополучными сдѣлать. "--- 2)~Наказаніе, чрезъ бѣды и напасти бываемое, есть знаменіе Божія милосердія къ намъ. \textit{Егоже бо любитъ Господь, наказуетъ}, апостолъ глаголетъ. "--- 3)~Грѣхи наши, которыми мы благость Божію оскорбили, далеко большіе, нежели Божіе наказаніе, какое бы оно ни было; всегда бо Богъ, по велицѣй Своей милости, менѣе насъ наказуетъ, нежели мы заслужили. "--- 4)~Богъ, по богатству благости и кротости и долготерпѣнія, на покаяніе насъ ведущаго, \textit{терпитъ намъ}, ожидая насъ на покаяніе\footnote{Римл.~2,~4; 2~Петр.~3,~9.}. Убо и намъ должно терпѣть, когда наказуетъ насъ за грѣхи, и благодарить Ему за отеческое наказаніе, яко тѣмъ наказаніемъ ищетъ спасенія нашего, и терпя милости Его ожидать, и что насъ во грѣхахъ нашихъ не погубилъ. "--- 5)~Аще бы кто обѣщалъ тебѣ великое богатство и славу въ мірѣ семъ, и повелѣлъ бы все терпѣть, что ни приключится скорбное: уповаю, что все терпѣливо и великодушно понесешь, ежели обѣщанное желаешь получить. Тако пріемлютъ больные жестокое врачевство, и терпятъ, взирая на обѣщанное отъ лѣкарей здравіе; тако подвизаются воины, и подвергаютъ себе въ великія опасности, надеждою обѣщаннаго сана поощряеми; тако и прочіе сынове вѣка страждутъ приключенія жестокая ради надежды временныхъ благъ, хотя часто и обманываются. Богъ обѣщаетъ намъ не временное богатство, но вѣчное, не земную славу, но небесную; и обѣщаетъ неложно, и велитъ терпѣть приключающаяся злая: какъ убо не терпѣть, когда хощемъ получить обѣщанная? Онъ обѣщаетъ печаль нашу въ радость обратить намъ: \textit{сѣющіи слезами радостію пожнутъ; ходящіи хождаху и плакахуся, метающе сѣмена своя; грядуще же пріидутъ радостію, вземлюще рукояти своя}\footnote{Пс.~125,~5 и 6.}. Скорбнымъ путемъ и терпѣніемъ доходятъ до вѣчной радости: \textit{яко многими скорбьми подобаетъ намъ внити въ царствіе небесное}, глаголетъ апостолъ\footnote{Дѣян.~14,~22.}. \textit{Сіи суть, иже пріидоша отъ скорби великія}, глаголетъ гласъ небесный о избранныхъ Божіихъ святыхъ\footnote{Апок.~7,~14.}. "--- 6)~Требуетъ Божія честь того, чтобы мы ради ея всякое приключающееся бѣдствіе терпѣли. Тріе отроки въ Вавилонѣ разжженной пещи не убоялися, когда отъ нечестиваго царя образу златому повелѣно было имъ кланятися, и тако изволили лучше жертвою огня быть, нежели жертву истукану принести, и честь Творца воздать твари\footnote{Дан.~3,~16--18.}. Тоежъ сотворили и святіи мученицы въ новой благодати. Разжигаетъ и нынѣ міродержецъ, князь тьмы, пещь искушенія, и устрашаетъ ею воли его злочестивой не повинующихся, но, ради Божіей чести, лучше намъ въ пещь тую вверженнымъ быть и терпѣть, нежели, оставивше волю Божію, его злому хотѣнію повиноваться. И сіе"=то есть отрещися себе, душу свою возненавидѣть. Тако терпящему и страждущему приспѣетъ Божія помощь, прохлаждающая и ликовати научающая\footnote{Дан.~3,~49 и 50.}. "--- 7)~Требуетъ того Хріста Сына Божія любовь въ намъ, чтобы мы случающееся бѣдствіе ради любви Его безропотно, паче же съ радостію претерпѣвали. За кого Хрістосъ такъ ужасное мученіе претерпѣлъ? За тебе и за мене. Что Его понудило къ тому? Едина любовь къ тебѣ и мнѣ. Для чего? Чтобы мы спасеніе получили. О, коль неблагодарни явимся раби, когда Господь нашъ ради насъ толико страдалъ, а мы не хочемъ и мало ради Его потерпѣть, паче же ради себе! Ибо и терпѣніе наше намъ, а не Ему пользуетъ, и нетерпѣніе наше намъ, а не Ему вредитъ. "--- 8)~Требуетъ того правда Божія, дабы грѣшникъ за грѣхи наказанъ былъ. Аще убо нужно есть грѣшнику наказаннымъ быть, лучше здѣ наказаннымъ быть, и съ благодареніемъ терпѣть и съ пророкомъ правду Божію исповѣдовать и прославлять: \textit{праведенъ еси, Господи, и прави суди Твои}\footnote{Пс.~118,~137.}, нежели нетерпѣть, и во ономъ вѣцѣ безъ конца мучиму быть: къ тому бо нетерпѣніе приводитъ. Здѣ наказуетъ Богъ, и утѣшаетъ, а тамо нѣтъ утѣшенія. Здѣ наказаніе легкое, отеческое, а тамо жестокое; здѣ маловременное, а тамо вѣчное. Ибо сто лѣтъ здѣ всякое страданіе терпѣть "--- ничтоже есть противу онаго. Смотри, куды богача евангельскаго пространный путь довелъ, который \textit{облачашеся въ порфиру и виссонъ, веселяся на вся дни свѣтло}? До ада!.. Изъ ада возводитъ очи свои, сый въ мукахъ, изъ пламени геенскаго вопіетъ: \textit{отче Аврааме! помилуй мя, яко стражду во пламени семъ}\footnote{Лук.~16,~23 и 24.}. Вопіетъ, но безполезно, и вовѣки вопить будетъ. О, коль великое Божіе милосердіе къ тѣмъ, которыхъ здѣ наказуетъ, да тамо помилуетъ! Коль блаженніи суть, кои сіе отеческое наказаніе терпятъ! \textit{Судими, отъ Господа наказуются, да не съ міромъ осудятся}\footnote{1~Кор.~11,~32.}. Коль окаянніи, кои сего наказанія убѣгаютъ! Они подобны дѣтямъ несмысленнымъ, которыя личинъ малеванныхъ боятся, а за горячее уголье руками хватать не боятся. Убѣгай, или не убѣгай, какъ хощешь, человѣче: надобно или здѣ, или тамо терпѣть. Нѣтъ инаго пути къ небу, кромѣ пути крестнаго, пути тѣснаго, а пространный путь \textit{вводяй въ пагубу}, по словеси Хрістову\footnote{Матѳ.~7,~13.}. Лучше убо намъ молить Бога о томъ, чтобы здѣ насъ наказалъ, а тамо помиловалъ: \textit{накажи насъ, Господи, обаче въ судѣ, а не въ ярости}, глаголетъ пророкъ\footnote{Іер.~10,~24.}, и здѣ Его наказаніе съ благодареніемъ и съ пользою нашею терпѣть, нежели тамо съ вѣчнымъ воздыханіемъ и безполезно страдать. Ежели, любезный хрістіанине, разсудишь добрѣ: что то есть вѣчность, и вообразишь въ умѣ, коль тяжко есть лишиться славы вѣчной и попасться во огнь геенскій, котораго и демони трепещутъ: изберешь всякое бѣдствіе и страданіе здѣ терпѣть, и благодарить будешь Богу за милостивое Его отеческое наказаніе. "--- 9)~Терпѣть, или не терпѣть въ страданіи, однакожъ не миновать того, и нетерпѣніемъ страданія не отвратить, которое промыслъ Божій намъ опредѣлилъ. А отъ нетерпѣнія не иное что, какъ вредъ и пагуба послѣдуетъ. "--- 10)~Терпѣніемъ облегчается всякое страданіе. Посмотри на тѣхъ, которые въ долговременной находятся болѣзни, или которые долговременно сидятъ въ темницѣ; такъ они къ тому бѣдствію терпѣніемъ привыкли, что аки бы не чувствовали того. \textit{Скорбь бо терпѣніе содѣловаетъ}\footnote{Римл.~5,~3.}. Напротивъ того, нетерпѣніемъ умножается болѣзнь, какъ самая вещь показуетъ, и къ тому приводитъ, что многіе сами себе смерти предаютъ, и тако временной и вѣчной жизни лишаются. "--- 11)~Всякое страданіе жестокое, или легкое есть: ежели жестокое, скоро прекратится смертію, ежели легкое, сносно есть и удобно можно терпѣть. И жестокое убо и легкое должно терпѣть: первое, что скоро окончится; другое, что легкое и удобное. "--- 12)~Кто страждетъ что нибудь, такъ можно въ себѣ помышлять: «До сего времени терпѣлъ я, убо и далѣе такожде могу терпѣть. Вчера терпѣлъ, убо и сегодня и утро можно терпѣть». Такое размышленіе утвердитъ въ терпѣніи съ помощію Божіею. Кто въ прошедшіе дни терпѣлъ, тотъ и въ слѣдующіе можетъ терпѣть. Вчера и третьяго дня болѣзнь, печаль и иную бѣду несъ, убо и далѣе нести можно. Когда несъ бѣду, и не исчезнулъ, убо сносная бѣда, убо стерпима, убо можно терпѣть. Хотящему бо и укрѣпившему сердце свое ничто невозможно, наипаче, когда Богъ тщащимся \textit{помогаетъ}\footnote{Пс.~53,~6; 36,~40.}. "--- 13)~Нетерпѣніе и роптаніе въ напастяхъ не иное что, какъ хуленіе. Ибо ропщущій непремѣнно думаетъ, что онъ неправедно напасти терпитъ, которыя на него посылаются отъ Бога, хотя бы онъ устами того и не произносилъ. Сынъ, наказуемый отъ отца, когда ропщетъ и негодуетъ противу отца, показуетъ тѣмъ негодованіемъ, что отецъ его или неправедно, или не по мѣрѣ согрѣшенія біетъ, и тако болѣе отца на гнѣвъ подвигаетъ. Такъ, когда въ бѣдахъ, отъ Бога намъ посылаемыхъ, негодуемъ, тѣмъ самымъ показуемъ, что неправедно страждемъ, "--- что благости и правдѣ Божіей весьма противно; и того ради болѣе Его раздражаемъ, и тако большимъ бѣдамъ себе подвергаемъ. «Которые наказанія Божія, глаголетъ Златоустъ, пріемлютъ не съ благодареніемъ, но со гнѣвомъ и негодованіемъ, тіи превеликимъ себе подвергаютъ бѣдамъ»\footnote{\textit{1~къ Статирію}.}. Кто бо безгрѣшенъ, и кто оправдится предъ Нимъ? Часто бываетъ, что предъ человѣкомъ, который гонитъ насъ, не согрѣшили мы, но предъ Богомъ согрѣшили, и часто въ томъ, въ чемъ поносятъ намъ, не виноваты, но въ другомъ виноваты, и такъ во всемъ вина наша. Аще бы грѣховъ не было, не было бы и бѣдъ, вси бо бѣды отъ грѣховъ произошли. Чего ради въ бѣдствіяхъ правду Божію, а нашу винность признавать, и благодарно терпѣть наносимое бѣдствіе должно. "--- 14)~Міръ сей подобенъ морю. Море всегда вѣтромъ волнуется: житіе наше всегда бѣдами смущается. На морѣ волна волнѣ послѣдуетъ: въ житіи нашемъ бѣда за бѣдою идетъ; едина прошла, другая наступаетъ, и такъ безпрестанно кораблецъ нашъ то къ верху, то къ низу бросаетъ. Терпѣніе же есть, какъ якорь, который, во глубину Божія милосердія съ надеждою поверженный, не попуститъ кораблецу нашему разрушитися; или какъ тихое пристанище, въ которое убѣгаемъ отъ потопленія. «Терпѣніе, глаголетъ Златоустъ святый, есть радости пристанище»\footnote{На пс.~14"~й.}. "--- 15)~Чѣмъ болѣе приближается день послѣдній, тѣмъ болѣе умножаются злобы, и, \textit{умножающуся беззаконію, исчезаетъ любовь}\footnote{Матѳ.~24,~12.}. Въ такихъ обстоятельствахъ нечего ожидать, кромѣ насилія и озлобленія. Слѣдственно всякому, хотящему благочестно жить о Хрістѣ Іисусѣ, должно себе терпѣніемъ, какъ мечемъ духовнымъ, вооружать и защищать противу нападенія злобы. Злоба бо не инымъ чимъ, какъ только терпѣніемъ, побѣждается. "--- 16)~Существо и сила подвига хрістіанскаго не въ иномъ состоитъ, какъ въ постоянномъ терпѣніи искушеній и бѣдъ. Подвигъ сыновъ вѣка сего состоитъ въ храбромъ сопротивленіи противу непріятеля, и побѣда въ прогнаніи и покореніи его: но хрістіане тогда добрѣ подвизаются, когда находящія бѣды терпѣливно, благодарно и великодушно сносятъ; и тогда преславно побѣждаютъ, когда гонящимъ ихъ уступаютъ, и благимъ злое побѣждаютъ, за ненависть любовь и за клятву благословеніе воздаютъ. О, коль знатная есть сія побѣда! воистину знатнѣйшая есть, нежели покорять народы! Гнѣвъ бо свой побѣдить и негодованіе "--- нѣтъ большей побѣды. \textit{Лучше мужъ долготерпѣливъ паче крѣпкаго, удержаваяй же гнѣвъ паче вземлющаго градъ}, глаголетъ Соломонъ\footnote{Притч.~16,~32.}. "--- 17)~Сколько Богъ Самъ долготерпитъ беззаконникамъ сего міра, Который во мгновеніе всѣхъ погубить можетъ! Сколько имѣется страшныхъ хульниковъ, которые на имя Его святое страшныя отрыгаютъ хулы? Сколько идолопоклонниковъ, которые честь Его восписуютъ бездушной твари! Сколько безбожниковъ, которые отвергаютъ присносущное Его бытіе! Сколько гонителей и еретиковъ, которые церковь Его истребить тщатся! Сколько прочіихъ грѣшниковъ, которые явно и безстрашно законъ Его святый нарушать не опасаются! Но благость Божія всѣмъ терпитъ. Симъ великимъ долготерпѣніемъ Его научаемся и мы врагамъ нашимъ терпѣть. Всѣмъ бо будетъ конецъ; вси свое воспріимутъ, озлобляющіи и озлобляеміи: \textit{се Судія предъ дверьми стоитъ}\footnote{Іак.~5,~9.}! \textit{уготова на судъ престолъ Свой}\footnote{Пс.~9,~8.}. Явны предъ Нимъ и насилующіи и насилуемые, гонящіи и гонимые, лишающіи и лишаемые, поношающіи и поношаемые, біющіи и біемые, изгоняющіи и изгоняемые; изочтены у Него слезы плачущихъ и воздыханія убогихъ: все тогда въ явленіе пріидетъ. \textit{Всѣмъ бо явитися намъ подобаетъ предъ судищемъ Хрістовымъ, да пріиметъ кійждо, яже съ тѣломъ содѣла, или блага, или зла}\footnote{2~Кор.~5,~10.}. "--- 18)~Терпѣніе истинное есть такая добродѣтель, на которую милосердый Богъ благопріятно смотритъ, и благодать Свою на сердце терпѣливаго низпосылаетъ. \textit{На кого воззрю? токмо на кроткаго и молчаливаго и трепещущаго словесъ Моихъ}, глаголетъ Господь\footnote{Ис.~66,~2.}. "--- 19)~Терпѣніе мужественнымъ и непобѣдимымъ дѣлаетъ человѣка. Можетъ терпѣливый лишенъ быти всего, можетъ изгнанъ быти, можетъ біенъ быти, можетъ въ темницѣ заключенъ быти, можетъ убіенъ быти, но побѣдимъ быть не можетъ. Ибо крѣпость его не тѣлесная, но духовная есть: тѣломъ побѣждается, но духомъ непобѣдимъ бываетъ; на тѣлѣ пріемлетъ раны, *но духомъ не пріемлетъ ранъ*; тѣломъ заключается, но духомъ свободенъ; тѣломъ умерщвляется, но духомъ умертвитися не можетъ, и тако надъ всѣми своими врагами духомъ торжествуетъ. Тако побѣждали мученицы святіи, которыхъ ни узы, ни темницы, ни раны, ни огнь, ни вода, ни зубы звѣрей, ни смерть, ни животъ, побѣдить не могли, но надъ всѣми свободнымъ торжествовали духомъ. Чего не дѣлали мучители имъ? Какого мученій рода не изобрѣтали? Но отходили съ позорища посрамленны и побѣжденны отъ нихъ, хотя плоть ихъ, какъ звѣри, терзали. Діоклитіанъ мучитель, когда ничего не успѣлъ святому Виту мученику своимъ мучительствомъ сотворить, побѣжалъ съ позорища, бія себе по лицу, и взывая: горе мнѣ, яко отъ такого отрока мала побѣжденъ есмь\footnote{Июн.~15"~го дня житіе его.}! "--- 20)~Страждущій, смотри на тѣхъ, кои большую имѣютъ скорбь, и болѣзнь терпятъ. Ежели находишься въ долговременной болѣзни, и имѣешь какое нибудь утѣшеніе отъ служащихъ тебѣ: посмотри на тѣхъ, кои большую твоей имѣютъ болѣзнь, кои внутрь огнемъ скорби и печали жегомы суть, извнѣ вси ранами осыпаны, ктомужъ не имѣютъ, кто бы имъ послужилъ, кто бы ихъ накормилъ, напоилъ, поднялъ, омылъ отъ ранъ; но терпятъ. Ежели терпишь изгнаніе: приведи на умъ каторжныхъ, которые въ кандалахъ, въ рубищахъ, полунаги, отъ дома и отечества удаленные, по всякъ день біеніе и раны пріемлющіи, днемъ на тяжкой работѣ, нощію въ темницахъ, нечистоты и смрада наполненныхъ, заключенные, безъ всякаго утѣшенія пребываютъ, коимъ смерть пріятнѣйшая, нежели животъ; но терпятъ. Ежели терпишь нищету: помысли о тѣхъ, которые прежде были богаты и славны, но до того пришли, что они ни себе, ни жену, ни дѣтей не имѣютъ чимъ питать, одѣвать, гдѣ главы приклонить, скитаются по чужимъ дворамъ; ктомужъ долгами обремененные, отвсюду тѣсноту, печаль, скорбь несносную имѣютъ, такъ какъ въ пещи горятъ. Ты, хотя своего къ потребѣ не имѣешь, можешь испросить именемъ Хрістовымъ; а они и просить стыдятся ради того, что прежде славны и богаты были. Посмотри еще на бѣдныхъ крестьянъ, нищихъ, полунагихъ, больныхъ, движенія неимѣющихъ, съ которыхъ подати и оброки требуются; а они не токмо дать, но и сами требуютъ, кто бы имъ далъ, да еще и послужилъ ради крайней нищеты и болѣзни. "--- Ежели терпишь поношеніе и клевету: приведи на память на высокомъ мѣстѣ сидящихъ, сколько они отъ подкомандныхъ роптанія, поношенія, злословія, руганія, навѣтовъ, коварствъ, лукавствія, проклинанія, насмѣшекъ, язвительныхъ укореній терпятъ, по подобію древа, на высокомъ мѣстѣ стоящаго, которое отъ всякаго и малѣйшаго вѣтра колеблемо бываетъ. Такъ и отъ прочіихъ пріемли себѣ подкрѣпленіе терпѣнія. Они большая и жесточайшая терпятъ: тебѣ ли малаго не терпѣть? "--- 21)~Низходи умомъ твоимъ во адъ, разсуди, какъ"=то осужденники мучатся, и во вѣки будутъ мучитися, которые, ежели бы можно было, желали бы здѣ хотя до скончанія міра во огнѣ горѣть, только бы отъ вѣчнаго свободиться мученія; но не дается имъ. "--- 22)~Возведи умныя очи въ небесныя селенія, и осмотри тамо всѣхъ жителей: ни единаго не сыщешь, кто бы терпѣнія путемъ отсюду не пришелъ туды. "--- 23)~\textit{Недостойны страсти нынѣшняго вѣка къ хотящей славѣ явитися въ насъ}, глаголетъ апостолъ Павелъ\footnote{Римл.~8,~18.}. Убо какую бѣду ни будешь терпѣть здѣ, однакожъ терпѣніе тое недостойно будущія славы, которая терпящимъ уготована: ибо \textit{око не видѣ, и ухо не слыша, и на сердце человѣку не взыдоша, яже уготова Богъ любящимъ Его}\footnote{1~Кор.~2,~9.}. "--- 24)~Въ страданіи своемъ помяни о ужасномъ страданіи мучениковъ святыхъ. Отъ нихъ иные палицами біени, инымъ зубы и глаза исторжены, у иныхъ языкъ, руки, ноги и сосцы отсѣчены, иные почти всѣ раздроблены, ко крестамъ пригвождаемы были, иные на снѣденіе звѣрямъ бросаемы, иные въ водѣ потопляемы, иные огнемъ сожигаемы, иные живы въ землю зарываемы, иные въ раскаленныхъ мѣдныхъ волахъ затворяемы были, съ иныхъ кожа и плоть до костей сдираема была, инымъ смола, олово растопленное въ уста вливаемы были, и прочая несказанная мученія терпѣли; но вся такъ великодушно терпѣли, что и смѣялись и ругались мучителямъ. Правда, что они терпѣли вся сія помощію Хрістовою: но таяжде помощь Хрістова и нынѣ всѣмъ готова терпящимъ: \textit{Іисусъ Хрістосъ вчера и днесь, Тойже и во вѣки}\footnote{Евр.~1,~3,~8.}. Хрістосъ, Сынъ Божій, \textit{безгрѣшный}, неповинно ради насъ терпѣлъ ужасныя и умомъ непостижимыя страданія, и всѣмъ намъ во образъ себе представляетъ: \textit{аще кто хощетъ по Мнѣ ити, да отвержется себе и возметъ крестъ свой, и послѣдуетъ Мнѣ}\footnote{Матѳ.~16,~24; Лук.~9,~23.}. На сей образъ вѣрою издалеча взирали пророцы, и \textit{изволяли паче страдать, нежели временную грѣха имѣти сладость}\footnote{Евр.~11"~я.}. Сему послѣдовали святіи апостоли, мученицы и прочіи святіи, \textit{иже проидоша сквозѣ огнь и воду} всякихъ искушеній, и \textit{внидоша въ покой вѣчный}\footnote{Пс.~65,~12.}. "--- \textit{Тѣмже убо и мы, толикъ имуще облежащъ насъ облакъ свидѣтелей, гордость всяку отложше и удобь обстоятельный грѣхъ, терпѣніемъ да течемъ на предлежащій намъ подвигъ, взирающе на Начальника вѣры и Совершителя Іисуса, Иже, вмѣсто предлежащія Ему радости, претерпѣ крестъ, о срамотѣ нерадивъ, одесную же престола Божія сѣде. Помыслите убо таковое Пострадавшаго отъ грѣшникъ на Себе прекословіе, да не стужаете, душами своими ослабляеми}. Тако увѣщаваетъ апостолъ святый всѣхъ страждущихъ\footnote{Евр.~12,~1--3.}! Хрістосъ глаголетъ: \textit{Азъ есмь путь и истина и животъ}\footnote{Іоан.~14,~6.}. Аще хощемъ къ Богу пріити, симъ путемъ вѣрою ити должно намъ. Аще не хощемъ прельститися, сей Истинѣ вѣровать должно намъ, Которая глаголетъ: \textit{аще кто Мнѣ служитъ, Мнѣ да послѣдствуетъ}\footnote{12,~26.}, то есть, Моему смиренію, терпѣнію и кротости. Аще хощемъ вѣчный получити животъ, сему Животу, кромѣ Котораго нѣтъ живота, вѣрою и любовію прилѣплятися должно намъ. Сей есть Путь, приводящій къ Богу! Сія есть Истина, прельстити не знающая! Сей есть Животъ, оживляющій тѣхъ, которые Ему вѣрою прилѣпляются! Аще убо кто симъ Путемъ не идетъ, сей заблуждаетъ. Аще кто сей Истинѣ не вѣруетъ, сей прельщается. Аще кто сего Живота не держится, сей мертвъ есть, и во вѣки мертвъ будетъ. Плоть, какъ ни ласкаетъ, обманетъ; міръ прельститъ насъ, и къ тому приведетъ, что каяться и жалѣть будемъ безполезно: но сія Истина прельстить не можетъ. "--- 25)~Истиннымъ терпѣніемъ сообразными дѣлаемся Хрісту Сыну Божію, яко уды Главѣ, Который толико скорбей и страданій насъ ради претерпѣлъ. А кто въ терпѣніи сообразенъ Ему есть, тотъ и славы Его причастится въ воскресеніи. \textit{Понеже съ Нимъ страждемъ}, глаголетъ апостолъ, \textit{да и съ Нимъ прославимся}\footnote{Римл.~8,~17.}. Колико же есть Божію Сыну быть сообразнымъ въ терпѣніи и славѣ, и преблагословенной оной Главы быть истиннымъ духовнымъ удомъ, "--- не токмо языкъ сказать, но и умъ нашъ того постигнуть не можетъ. "--- 26)~Требуетъ того Божія любовь, чтобы мы безропотно, паче же благодарно крестъ, намъ отъ Него налагаемый, понесли. Онъ насъ отъ любви, а не отъ гнѣва наказуетъ, какъ сказано выше: убо и намъ должно не съ гнѣвомъ и роптаніемъ, хотя бы сердце и смущалося, но съ любовію и благодареніемъ отеческое Его наказаніе терпѣть. Ничто бо такъ не показуетъ любве Божія, какъ благодарное бѣдъ и скорбей терпѣніе. Многіе мнятъ о себѣ, что Бога любятъ; но нашедшая противность показуетъ, истинное ли мнѣніе ихъ. Въ благоденствіи и злые мнятся Богу благодарить, но въ злоденствіи ропщутъ. Въ благополучіи же и злополучіи быть Богу благодарнымъ, единаго боголюбиваго сердца дѣло есть. "--- 27)~Коль великую пользу терпѣніе приноситъ въ обществѣ и всякомъ званіи! Терпѣніемъ сохраняется любовь и согласіе между властями и подвластными, между родителями и дѣтьми, между господами и рабами, между братьями, между другами, между сосѣдами, между купующими и продающими, такъ что безъ терпѣнія никакого добра быть не можетъ. Отъ нетерпѣнія мужъ съ женою, братъ съ братомъ, другъ съ другомъ, гдѣ должно миру и согласію быть, ссорится и враждуетъ. Отъ нетерпѣнія господинъ раба, отецъ сына, мужъ жену, властелинъ подвластнаго мучительски біетъ. Отъ нетерпѣнія въ біемыхъ возстаетъ злое умышленіе на біющихъ: отсюду бываетъ, что рабъ господина, жена мужа, подвластный властелина, сынъ злаго отца ищетъ умертвить, и умерщвляетъ, якоже сего зла много случается. Терпѣніе же все сіе зло пресѣкаетъ. Нетерпѣніе разоряетъ домы, села, грады и государства, ибо отъ нетерпѣнія несогласіе, отъ несогласія ссора и брань, отъ брани кровопролитіе и убійство въ людяхъ, которые общество составляютъ. Терпѣніе все сіе зло отвращаетъ. Ибо гдѣ терпѣніе, тамо нѣтъ ссоры и брани. Какой быть тамо ссорѣ, когда единъ обидѣлъ, а другій уступилъ и простилъ обидѣвшему? Отъ нетерпѣнія человѣческаго и самое неповинное естество страждетъ. Такъ злобные сжигаютъ домы, житницы, лавки, убиваютъ или вредятъ скотъ. Отъ нетерпѣнія самъ себе человѣкъ біетъ, свои власы терзаетъ, и дѣлается подобнымъ бѣсноватому, и страшнѣйшимъ звѣря; часто и себе убиваетъ, и тако душею и тѣломъ умираетъ. Терпѣніе же сему бѣдствію быти не допускаетъ. О! блажени тѣ домы, грады, веси, села и общества, въ которыхъ терпѣніе обитаетъ: оно бо болѣе сохраняетъ общество, нежели оружіе, болѣе защищаетъ градъ, нежели стѣны. О! любезное видѣніе и слышаніе терпѣнія "--- пристанище обуреваемыхъ, источникъ мира, крѣпость, дружество, забрало и хранилище добродѣтелей, вѣнецъ благочестія, извѣстное знаменіе вѣры, истинныя радости вина, плодъ смиренія, покой совѣсти, гербъ хрістіанскій, знамя Хрістовыхъ воиновъ, печать избранныхъ Божіихъ, путь къ вѣчному животу, лѣствица къ небеси, предтеча къ вѣчной славѣ, побѣда на враговъ, язва діаволу и аггеломъ его, міру поруганіе, торжество надъ самимъ собою, и проч.

\paragraph*{§\:215.} Терпѣніе укрѣпляется: 1)~молитвою, которая помощь Божію испрашиваетъ въ понесеніи наложеннаго креста. Какъ бо дѣти страждущіи объявляютъ печаль свою родителямъ, и тако отъ нихъ утѣшеніе получаютъ; или какъ другъ вѣрному своему другу сообщаетъ скорбь сердца, и такъ нѣкую отраду чувствуетъ на сердцѣ: тако наипаче облегченіе печали нашей чувствуемъ, когда Богу, \textit{иже есть Отецъ щедротъ и Богъ всякія утѣхи}\footnote{2~Кор.~1,~3 и 4.}, печаль нашу чрезъ молитву сообщаемъ. "--- 2)~Укрѣпляется исповѣданіемъ грѣховъ. Ибо въ исповѣданіи грѣховъ познаетъ человѣкъ, что онъ чрезъ собственные свои грѣхи въ такое бѣдствіе пришелъ, и потому ни на кого не жалуется, какъ токмо на себе и на свои грѣхи. Тако исповѣдующемуся и кающемуся посылается Божіе утѣшеніе. "--- 3)~Облегчается скорбь и печаль духовными пѣснями. Тако путники скуку и тугу отгоняютъ пѣснями различными; тако мастеровые люди растворяютъ печаль и скуку пѣніемъ. Пѣніе бо печальному, какъ пластырь ранѣ приложенный, или какъ жаждущему студеная вода. "--- 4)~Укрѣпляется терпѣніе ожиданіемъ избавленія, которое, хотя не до смерти, то съ смертію послѣдуетъ. Смертію бо всякое бѣдствіе сокращается. "--- 5)~Воззрѣніемъ на образъ страданія Хрістова, святыхъ мучениковъ и прочіихъ святыхъ, такожде и тѣхъ, кои нынѣ всякое бѣдствіе терпятъ, какъ сказано. "--- 6)~Ожиданіемъ будущія славы утѣшеніе пріобрѣтается. Тако утѣшаются земледѣльцы надеждою плодовъ, путники надеждою покоя, воины надеждою побѣды и славы, купцы надеждою пріобрѣтенія богатства. Аще сихъ временныхъ благъ надежда ободряетъ, кольми паче вѣрныхъ и неизреченныхъ благъ надежда ободритъ и поощритъ къ терпѣнію бѣдствія и печали. "--- 7)~Святое Писаніе различно подаетъ утѣшеніе скорбящимъ и сѣтующимъ душамъ. Какъ въ богатой и изобильной аптекѣ различныя имѣются лѣкарства, которыми различныя исцѣляются тѣлесныя болѣзни: такъ въ священномъ Писаніи различныя предлагаются намъ посредствія, которыми можемъ въ терпѣніи утѣшитися и укрѣпитися. На сіе бо и дано оно отъ Бога, \textit{да терпѣніемъ и утѣшеніемъ Писаній упованіе имамы}\footnote{Римл.~15,~4.}.

\subsection[Глава 7-я. О благодареніи Богу.]{глава седьмая.\\\bfseries О благодареніи Богу.}

\begin{quotation}\textit{О всемъ благодарите: сія бо есть воля Божія о Хрістѣ Іисусѣ въ васъ}, глаголетъ апостолъ\footnote{1~Сол.~5,~18.}.\end{quotation}
\begin{quotation}\textit{Благодарни бывайте. Слово Хрістово да вселяется въ васъ богатно, во всякой премудрости учаще и вразумляюще себе самѣхъ, во псалмѣхъ и пѣніихъ и пѣснехъ духовныхъ во благодати поюще въ сердцахъ вашихъ Господеви}\footnote{Колос.~3,~15 и 16.}.\end{quotation}


\paragraph*{§\:216.} Благодареніе Богу есть внутреннее, сердечное и радостное чувствованіе благости Его и милости, намъ недостойнымъ отъ Него туне показанныя и показуемыя, "--- и сердцемъ и устами засвидѣтельствованіе. Сердечнаго сего благодаренія образъ показалъ намъ Самъ Хрістосъ Сынъ Божій, \textit{егда возрадовася духомъ и рече: исповѣдаютися, Отче, Господи небесе и земли}! и проч.\footnote{Лук.~10,~21.} Показала пресвятая Матерь Его пречистая Дѣва Марія, егда радостнымъ духомъ воспѣла пѣснь сію: \textit{величитъ душа моя Господа, и возрадовася духъ мой о Бозѣ Спасѣ моемъ; яко призрѣ на смиреніе рабы Своея}, и проч.\footnote{1,~46--55.} Давидъ святый во многихъ псалмахъ радостнымъ духомъ и благодарнымъ сердцемъ и устами поетъ благость Божію, и другихъ къ пресладкому тому ликованію созываетъ: \textit{возвеличите Господа со мною, и вознесемъ имя Его вкупѣ}. И паки: \textit{вкусите и видите, яко благъ Господь}\footnote{Пс.~33,~4 и 9.}. Благодарнаго бо сердца есть, не только самому благодѣтеля своего прославлять, но тщаться, чтобы и другіе славили его.

\paragraph*{§\:217.} Раждается сіе чувствованіе отъ размышленія благодѣяній Божіихъ, которыя намъ показалъ и показуетъ. Благодѣянія Божія къ намъ не только неисчислимы, но и умомъ непостижимы суть. 1)~Изъ небытія въ бытіе привелъ насъ, за что едино никто не доволенъ возблагодарити. 2)~Душею разумною и образомъ Своимъ божественнымъ насъ почтилъ. 3)~Плодъ земный въ снѣдь намъ подалъ. 4)~Всякое животное на службу намъ опредѣлилъ. 5)~Солнце, луну и звѣзды къ освѣщенію нашему на небеси устроилъ. 6)~Воздухъ проліялъ ради сохраненія живота нашего. 7)~Облакамъ повелѣлъ орошать воздухъ ради прохлажденія нашего, землю для прозябенія и плодотворенія, дабы мы и скоты наши имѣли пищу. 8)~Древеса и травы различныя насадилъ къ различной нашей потребѣ и нуждѣ. 9)~Птицы различныя, аки музыку пресладкую, насъ увеселяющую, произвелъ. 10)~Огнь ради согрѣянія и освѣщенія въ нощномъ времени даровалъ намъ. 11)~Воды, моря, озера, источники и рѣки съ различными рыбами и прочіими животными къ пользѣ нашей и нуждѣ проліялъ. 12)~Показалъ путь плаванія по морямъ, дабы одна страна съ другою, востокъ съ западомъ, сѣверъ съ югомъ, имѣли удобнѣйшее сообщеніе, и благая его между собою сообщали. 13)~День и нощь, зиму и лѣто, весну и осень устроилъ ради насъ. 14)~Ангеламъ Своимъ окрестъ насъ ополчатися, и, аки стражамъ, хранити насъ повелѣлъ\footnote{Пс.~33,~8.}. 15)~Согрѣшающихъ насъ не абіе казнитъ, но ожидаетъ, призываетъ и увѣщаваетъ къ обращенію и покаянію. 16)~Предостерегаетъ и сохраняетъ насъ отъ навѣтовъ и козней врага нашего діавола. 17)~Но всѣхъ благодѣяній верхъ есть, что чрезъ Сына Своего, Господа нашего Іисуса Хріста, устроилъ намъ путь къ вѣчному блаженству. \textit{Иже во образѣ Божіи сый, не восхищеніемъ непщева быти равенъ Богу, но себе умалилъ, зракъ раба пріимъ, въ подобіи человѣчестѣмъ бывъ, и образомъ обрѣтеся якоже человѣкъ. Смирилъ Себе, послушливъ бывъ даже до смерти, смерти же крестныя}, якоже учитъ апостолъ\footnote{Филип.~2,~6--8.}. И горстію содержай вселенную, \textit{пеленами въ яслехъ повился}\footnote{Лук.~2,~7.}. И размѣряяй пядію небо, \textit{не имѣлъ гдѣ главы подклонити}\footnote{Матѳ.~8,~20.}. И богатъ сый \textit{обнища, да мы нищетою Его обогатимся}\footnote{2~Кор.~8,~9.}. И хотяй пріити на облацѣхъ судити живыхъ и мертвыхъ, \textit{неправедно отъ беззаконныхъ судимъ былъ}\footnote{Лук.~23,~1--24.}. И на Негоже \textit{не смѣютъ чини ангельстіи взирати, заушаемъ и оплеваемъ былъ}\footnote{Матѳ.~26,~67.}. И Одѣваяй небо облаки, \textit{въ ризу поруганія облеченъ былъ}\footnote{Лук.~23,~11; Марк.~15,~17.}. И Одѣяйся свѣтомъ, яко ризою, \textit{обнаженъ былъ}\footnote{Матѳ.~27,~27 и 31.}. И Емуже служатъ херувими и серафими \textit{человѣкамъ служити изволилъ}\footnote{Іоан.~13,~5.}. И руцѣ, которыми чудеса сотворилъ, \textit{пригвоздити ко кресту попустилъ}\footnote{Лук.~23,~33.}. И Иже никому никакого зла не сотворилъ, \textit{біенъ былъ}\footnote{Марк.~15,~19.}. И Иже маніемъ мертвыя воскресилъ, \textit{волею Своею смерть крестную претерпѣлъ}\footnote{Іоан.~19,~30; 10,~18.}. Вся же сія ради вѣчнаго нашего спасенія благій Владыка нашъ пострадалъ. Но когда толь многая и великая благодѣянія показалъ намъ, не требуетъ отъ насъ ничего инаго, какъ только единаго благодарнаго сердца. Туне бо, отъ единыя къ намъ недостойнымъ любве сія сотворилъ благодѣянія Благодѣтель нашъ Господь. Ибо естествомъ благъ есть, и не можетъ не благотворити. Благости бо свойство есть себе другимъ сообщать и удѣлять. Вся убо сія благодѣянія Божія: бытіе наше, которое мы пріяли отъ Него; душа разумная, которою мы почтены отъ Него; земля, которая держитъ и питаетъ насъ; вода, которая напаяетъ и омываетъ насъ; скоты, звѣри, рыбы, древеса и травы, которыя служатъ нуждамъ нашимъ; птицы, которыя увеселяютъ насъ; солнце, луна и звѣзды, которыя освѣщаютъ насъ; воздухъ, который сохраняетъ жизнь нашу; огнь, который согрѣваетъ насъ; облака, которыя дождятъ на насъ и на нивы наши; день, который для дѣланія нашего свѣтитъ намъ; нощь, которая ради упокоенія немощной плоти нашей устроена намъ; несумнѣнная будущаго живота надежда, такъ чуднымъ образомъ устроенная, надежда "--- всѣхъ благихъ забрало и конецъ, которую о Хрістѣ Іисусѣ Господѣ нашемъ имѣемъ мы, "--- вся сія, глаголю, и прочая недовѣдомая Его благодѣянія возбуждаютъ и аки понуждаютъ насъ со усердіемъ Богу Благодѣтелю благодарить и благодарными быть.

\paragraph*{§\:218.} Обдолжаютъ насъ неизреченныя Божія благодѣянія Богу благодѣтелю, и весьма благодарнаго сердца требуютъ отъ насъ: \textit{но что воздамы Господеви о всѣхъ, яже воздаде намъ}\footnote{Пс.~115,~3.}? За тое едино, что отъ Него созданы, изъ небытія въ бытіе приведены и разумною душею почтены мы, по образу и по подобію сотворены, "--- всѣхъ себе, то"=есть, душу и тѣло, отдать Ему должны мы. А что уже дать за тое сыщемъ, что согрѣшившіе помилованы, погибшіе взысканы, падшіе возставлены такъ чуднымъ образомъ? Что за тое, что ради насъ зракъ раба пріялъ, на земли пожилъ, алкалъ, жаждалъ, трудился, плакалъ, болѣзновалъ, печалился, безчестіе, поруганіе, оклеветаніе, осужденіе, оплеваніе, заушеніе, біеніе, раны, распятіе и смерть поносную пріялъ? Что, глаголю, за сію такъ чудную любовь воздадимъ? Ничего сыскать не можемъ! Что за тое, что \textit{чадами Божіими быти даде область вѣрующимъ во имя Его}\footnote{Іоан.~1,~12.}, гражданами вышняго Іерусалима быть, лику святыхъ ангелъ причтенными быть, вѣчнаго и небеснаго царствія участниками быть? Ничего! "--- Помяни, хрістіанине, колико ты сверхъ того Господу Богу одолженъ! Разсуди, колико кратъ отъ юности твоея предъ Господомъ Богомъ согрѣшилъ ты; колико кратъ по крещеніи законъ Его святый нарушилъ ты; колико кратъ обѣты твои, которые при вступленіи въ хрістіанство учинилъ, разорилъ ты; колико кратъ отъ Господа твоего сердцемъ уклонился! Всякое преступленіе чинилъ ты предъ лицемъ Божіимъ; всевидящее Его око смотрѣло на преступленія твоя. Богъ бо вездѣ есть, и что ни дѣлаемъ "--- предъ Богомъ дѣлаемъ. Грѣшилъ ты, и Богъ смотрѣлъ на тебе; смотрѣлъ и видѣлъ, но терпѣлъ тебѣ \textit{по богатству благости Своея}\footnote{Римл.~2,~14.}. Который царь такъ благъ, такъ кротокъ, такъ милостивъ можетъ быть, который бы видѣлъ раба своего предъ глазами своими преступающа законъ свой, и кротко терпѣлъ ему? Скоро человѣческая кротость въ гнѣвъ обращается, и терпѣніе въ негодованіе претворяется. Но царь такойжде человѣкъ, какъ подданные его, часто и самъ законъ преступаетъ, какъ человѣкъ: однакожъ не терпитъ, когда видитъ законъ свой отъ подданнаго разоряемый. Богъ, единъ безгрѣшный, Богъ Царь небесе и земли, единъ праведный и вѣчная правда, не тако. Хотя и видитъ грѣхъ человѣка, но по богатству благости Своея терпитъ ему, не казнитъ его, не воздаетъ ему тогда по дѣломъ его; и не токмо терпитъ, но и хранитъ согрѣшившаго, дабы не погиблъ, ожидая покаянія его. Помяни апостольское слово, яко \textit{супостатъ нашъ діаволъ, яко левъ рыкая ходитъ, искій кого поглотити}\footnote{1~Петр.~5,~8.}. Сей врагъ нашъ, сколько кратъ отъ юности твоея согрѣшилъ ты столько кратъ готовъ былъ погубить тебе; столько кратъ, яко исполинъ, устремлялся на тя, \textit{яко левъ готовъ на ловъ}, хотѣлъ похитить душу твою и свести во адъ: но благость Божія возбраняла ему тое. Ты заповѣдь Божію разорялъ, но Богъ милости Своея не разорилъ отъ тебе. Ты Бога прогнѣвлялъ, но благость Божія защищала тебе отъ врага твоего, который стоялъ при тебѣ согрѣшающемъ, и хотѣлъ погубити тебе. Ты отъ Бога отступалъ сердцемъ своимъ, и приступалъ ко врагу твоему; но благость Божія возбраняла ему и запрещала, да не похититъ душу твою. Сколько времени, согрѣшивши, въ непокаяніи пребывалъ ты, чрезъ все то время подкладалъ онъ сѣти ногамъ твоимъ, искалъ способа умертвить тя и вѣчной погибели предать душу твою; но благость Божія пресѣкала злыя умышленія его. Помяни, колико такихъ есть, которые въ любодѣяніи, въ разбоѣ, въ хищеніи, воровствѣ, въ пьянствѣ и снѣ погруженные, похищены безъ покаянія; но тебе благость Божія сохранила отъ того. О, колико въ сей части благости Божіей долженъ ты! Что живешь, что еще не погиблъ, что не низринутъ во адъ, что можеши покаятися и спастися, "--- все тое благости Божіей благодарно приписать долженъ ты. Въ чувство ли пришелъ ты, и пребываешь въ истинномъ покаяніи? Благодати Божіей дѣло есть сіе. Пребываеши ли въ нераскаяніи и неисправленіи? Благость Божія есть, что ты еще не погиблъ. Сатана наблюдаетъ всѣ стези твои, входы и исходы твои, хощетъ запять тя, и душу твою неисправную восхитити отъ тебе и свести во адъ; но благость Божія еще не дозволяетъ ему погубити тя, еще терпитъ тебѣ, еще ожидаетъ обращенія твоего, ведетъ на покаяніе тебе. Но слыши, что апостолъ къ нерадивому и неисправному глаголетъ: \textit{или о богатствѣ благости Его и кротости и долготерпѣніи нерадиши, невѣдый, яко благость Божія на покаяніе тя ведетъ? По жестокости же твоей и непокаянному сердцу, собираеши себѣ гнѣвъ въ день гнѣва и откровенія праведнаго суда Божія}\footnote{Римл.~2,~4 и 5.}. Видиши ли, возлюбленный хрістіанине, коль удивительная есть благость Божія къ намъ, которая не токмо создала насъ, и создала не такъ, какъ прочія вещи несмысленны и безчувственны, но чувствами и разумомъ почтила насъ, созданныхъ питаетъ, одѣваетъ, согрѣваетъ, сохраняетъ и освѣщаетъ не токмо согрѣшившихъ помиловала насъ, но и согрѣшающимъ терпитъ намъ; и не токмо терпитъ, но и сохраняетъ насъ отъ козней врага нашего, погибели нашей ищущаго, какъ выше видѣлъ ты. Но сколь великая Божія благость къ намъ, сколь великая любовь Его къ намъ, сколько мы одолжены Ему. Умъ нашъ, какъ постигнуть сея Божія къ намъ любве, о человѣцы, такъ и изобрѣсти, чимъ бы воздать, не доволенъ. Едино сіе можемъ знать и признать отъ сердца, что ничего не можемъ воздать, что бы ради Его ни дѣлали мы. \textit{Что воздамы Господеви о всѣхъ, яже воздаде намъ}? Тако всегда съ пророкомъ исповѣдаться со смиреніемъ должно. Все бо, что ни дѣлалъ или дѣлаетъ намъ Господь нашъ, туне, отъ единыя любве, дѣлаетъ. Но хотя и недостаетъ въ насъ силъ къ достойному благодаренію, однакожъ намъ должно, сколько силъ нашихъ есть, о томъ тщаться, чтобы всегда сердцемъ и устами благодарными Ему быть.

\paragraph*{§\:219.} Долгъ любве ничимъ инымъ, какъ любовію платится. Любовь бо ничимъ инымъ, какъ взаимною любовію удоволяется. Ибо любителю ничто не пріятно, что бы ни дѣлалъ, что бы ни приносилъ ему любимый, когда взаимной отъ него не видитъ любви. Кольми паче Богу, Любителю и Благодѣтелю нашему, Господу нашему, Который \textit{благихъ нашихъ не требуетъ}\footnote{Пс.~15,~2.}, и Котораго благая суть, какія мы ни имѣемъ, ничто не пріятно, что ни принесемъ, когда любовнаго и благодарнаго сердца не принесемъ. За долгъ убо любве Божія, которою мы несказанно отъ Него обязаны, ничего принести не можемъ, какъ взаимную любовь и благодарное сердце. Сей долгъ всегда и всякимъ образомъ должны мы платить, но никакъ и ничимъ не можемъ заплатить. Сей долгъ требуетъ отъ насъ, чтобы мы всякое Ему показывали послушаніе, и не свою, но Его волю исполняли. Сей долгъ требуетъ отъ насъ, чтобы мы, когда Ему не можемъ ничего дать, рабамъ Его, требующимъ ради имени Его, давали, алчущихъ питали, нагихъ одѣвали, безкровныхъ въ домы наши вводили, печальныхъ утѣшали, странствующихъ упокоевали, болящихъ и въ темницахъ сѣдящихъ посѣщали, и въ прочіихъ требованіяхъ и нуждахъ имъ служили\footnote{Матѳ.~25,~35 и 36.}. Сей долгъ требуетъ отъ насъ, чтобы мы согрѣшившей братіи нашей отъ сердца \textit{оставляли согрѣшенія ихъ}\footnote{6,~14 и 15.}. Сей долгъ требуетъ отъ насъ, чтобы мы \textit{любили враги наша, благословили кленущія насъ, и молилися за творящихъ намъ напасть, и изгонящія насъ}\footnote{5,~44.}. Сей долгъ требуетъ отъ насъ, чтобы мы представляли \textit{тѣлеса наша жертву живу, святу, благоугодну Богови, словесное служеніе наше}\footnote{Римл.~12,~1.}; \textit{умерщвляли уды наша сущія на земли, блудъ, нечистоту, страсть, похоть злую, лихоиманіе, еже есть идолослуженіе}\footnote{Кол.~3,~5.}. Сей долгъ требуетъ, чтобы мы благость сію Божію на сердцахъ нашихъ написали, устами свидѣтельствовали, со другами нашими бесѣдовали, предъ врагами не молчали о ней. Сей долгъ требуетъ, чтобы и душъ нашихъ не щадили, гдѣ честь имене Его требуетъ, "--- словомъ: все, что волѣ Его святой угодно, со усердіемъ и радостію творить не отрицалися\footnote{Еф.~5,~10.}.

\paragraph*{§\:220.} И въ противностяхъ, которыя намъ посылаются, должно Богу и паче благодарить, къ чему слѣдующія причины возбуждаютъ: 1)~Отъ \textit{благаго Бога} не можетъ быть намъ ничего кромѣ \textit{блага}, хотя намъ кажется и противно. Часто бо, по неразумію и слѣпотѣ нашей, вмѣняемъ тое быть добро, что намъ зло; и что намъ истинное добро, почитаемъ за зло. Но преблагій и премудрый Отецъ небесный иначе усматриваетъ, и \textit{вмѣсто хлѣба камня, и вмѣсто рыбы зміи подать не хощетъ просящимъ у Него}\footnote{Лук.~11,~13; Матѳ.~7,~11.}. "--- 2)~Отъятіемъ благихъ научаемся познавать благая и благихъ Дателя "--- Бога. Ибо доколь имѣемъ добро и насыщаемся добромъ, не познаемъ добра, а тако и Бога, добра всякаго Дателя, забываемъ: но тогда истинно познаемъ добро, когда его лишимся. Коль великое добро здравіе, въ болѣзни познаемъ; коль великое дарованіе Божіе хлѣбъ, во время глада примѣчаемъ; пользу свѣта во тьмѣ или въ слѣпотѣ, пользу мира и покоя во время нашествія враговъ, пользу дождя во время суши, пользу сіянія солнечнаго во время безведрія, пользу свободы въ неволѣ, пользу воды во время жажды, пользу огня во время хлада и зимы узнаемъ. А добро узнавше, отчасти вкушаемъ и видимъ, \textit{коль благъ Господь}, Который благая намъ даруетъ. "--- 3)~Когда отнимаетъ Богъ добро тѣлесное отъ насъ, хощетъ подать добро душевное. Когда отнимаетъ богатство тѣлесное, хощетъ подать душевное; когда отнимаетъ здравіе тѣлесное, хощетъ подать здравіе душевное; когда лишаетъ славы временныя, хощетъ подать славу вѣчную. А что дѣлаетъ такъ, то знать усмотрѣлъ, что добро тѣлесное съ душевнымъ не можетъ въ насъ помѣститься. И потому дѣлаетъ какъ искусный врачъ, который едину часть тѣла отсѣкаетъ, чтобы все тѣло сохранить: такъ и премудрый Врачъ душъ нашихъ, Господь, отсѣкаетъ тѣлесное добро, чтобы обоя, и душа и тѣло, въ вѣчномъ животѣ сохранилися. На лѣкаря не гнѣваемся, когда такое о насъ тщаніе являетъ, но паче благодаримъ ему, любимъ его и мзду ему даемъ: много паче Богу, такъ милостивно и человѣколюбиво о насъ промышляющему, должно благодарить. Ибо \textit{егоже любитъ Господь, наказуетъ}\footnote{Евр.~12,~6.}. Егда бо подаетъ намъ благополучіе преблагій Богъ, и тѣмъ насъ къ Себѣ привлекаетъ: растлѣнное наше естество тѣмъ возносится паче, нежели къ любви Божіей возбуждается. Того ради отеческій Божій промыслъ попущеніемъ бѣдъ смиряетъ насъ, и тѣмъ исправляетъ. И тако, что намъ кажется быть неблагополучіемъ, тое намъ истинное, прямое, и какого должно желать, есть благополучіе; плоти нашей есть бѣда, но духу счастіе; плоти изнеможеніе, но духу подкрѣпленіе; плоть увядаетъ, но духъ процвѣтаетъ. Бѣды убо и наказанія суть, какъ посланники Божіи, которые возвѣщаютъ намъ, благодарно терпящимъ ихъ, послѣдующую Божію благодать. Ибо бѣды къ Богу приводятъ насъ. \textit{Въ скорби своей утреневати будутъ ко Мнѣ, глаголюще: идемъ, и обратимся ко Господу Богу нашему; яко Той поби ны, и исцѣлитъ ны, уязвитъ и уврачуетъ ны}, глаголетъ пророкъ\footnote{Ос.~6,~1 и 2.} и проч. "--- 4)~Когда тѣлесное добро отнимаетъ Богъ, не все отнимаетъ, но оставляетъ нѣкая, смотря на силу духа нашего. Вѣдаетъ бо премудрый и преблагій Создатель нашъ немощь нашу, а потому, когда одного лишаетъ насъ, другое оставляетъ, чтобы возмогли искушеніе понести безъ изнеможенія; а притомъ искушаемымъ руку помощи подаетъ. Такъ, когда отнимаетъ богатство, оставляетъ здравіе; отнимая здравіе, оставляетъ другое какое утѣшеніе. И тако, ежели терпишь нищету, а здоровъ, благодари, что здравіе имѣешь и можешь трудами себѣ снискать хлѣбъ. Ежели нищъ и нездоровъ еси, благодари, что имя Хрістово питаетъ тя. Ежели терпишь болѣзнь, а имѣешь богатство, благодари, что имѣешь утѣшеніе отъ довольствія. Ежели пожитки похищены, благодари, что домъ цѣлъ и есть гдѣ покой имѣть. Ежели имѣніе и домъ сгорѣлъ, благодари, что самъ цѣлъ остался; ибо многіе и сами съ домами и имѣніемъ сгараютъ. Ежели терпишь злословіе и клевету, благодари, что еще не біютъ, въ ссылку не посылаютъ, не заключаютъ въ темницу. Ежели въ темницѣ сидишь просто, благодари, что не окованъ. Ежели окованъ, благодари что видишь свѣтъ, имѣешь пищу, и проч. Ежели чести лишился, благодари, что съ нею многихъ хлопотъ, суетъ, зависти, проклинанія, злословія, негодованія, злобы, навѣтовъ и прочіихъ золъ, окружающихъ честь, свободился, и что еще имѣнія не лишился. Ежели чести и имѣнія лишенъ, благодари, что въ заточеніе не посланъ. Ежели въ заточеніе посланъ, благодари, что смерти не преданъ. Ежели на смерть осужденъ, глаголетъ Василій Великій, и неправедно: благодари, что тебѣ за тое вѣнецъ преславный на небеси уготованъ; "--- ежели праведно, глаголетъ тойжде отецъ, благодари, что временною смертію отъ вѣчныя избавишися смерти\footnote{\textit{Въ сл. о Іулиттѣ мученицѣ}.}. Тако по примѣру сему и въ прочей противности поступай. А притомъ знай, что мы никакого добра у Бога не заслужили, но напротивъ того всякаго наказанія достойны; и какое бы наказаніе ни было, грѣхи наши большаго еще достойны.

\paragraph*{§\:221.} Какимъ образомъ преблагому Богу, Благодателю нашему и Промыслителю, за вся Его къ намъ благодѣянія должно намъ благодарить, здѣ представляется. Куды ни посмотримъ, куды ни обратимъ очи и умъ нашъ, вездѣ имѣемъ довольные случаи благость Божію прославлять. Въ нощи смотришь на небо чистое, звѣздами аки бисеромъ украшенное, и между звѣздъ луну сіяющую, "--- тебѣ сія служатъ: благодари \textit{Сотворшему луну и звѣзды во область нощи}\footnote{Пс.~135,~9.}. Возсіялъ день; солнце лучи свои по всей вселенной распустило, "--- тебѣ свѣтъ его сіяетъ: благодари \textit{Сотворшему солнце во область дне}\footnote{135,~8.}. Облака дождь кропятъ, "--- тебѣ кропятъ: \textit{благодари Одѣвающему небо облаки и Уготовляющему земли дождь}\footnote{146,~8.}. Возсталъ вѣтръ и началъ облака прогонять и очищать небо, "--- тебѣ онъ служитъ: благодари \textit{Изводящему вѣтры отъ сокровищъ Своихъ}\footnote{134,~7.}. Видишь поля, различными плодами исполненныя, луга и лѣса зеленѣющіеся, садовная древеса изобилующая плодами, "--- благости Божіей тое дѣло есть; тебѣ тое отъ Бога посылается: благодари Даровавшему, глаголя: \textit{благослови, душе моя, Господа}\footnote{102,~1.}. Пришла зима, одѣялась земля снѣгомъ, мразъ связалъ озера, рѣки и болота, и такъ вездѣ свободный учинился путь, нѣтъ нужды въ мостахъ и прочіихъ къ переправамъ надобностяхъ, "--- Божіе то благодѣяніе есть; твоей потребѣ служитъ сіе: благослови \textit{Дающаго снѣгъ Свой яко волну}\footnote{147,~5.}. Проходитъ зима, и весна наступаетъ, "--- тѣмъ тебѣ возвѣщается приближающееся какъ бы нѣкое воскресеніе всея твари, мразомъ умершія: благослови Благоволившаго тако. Наступила весна; тутъ новое открывается Божіихъ дарованій сокровище: солнце благопріятно сіяетъ и грѣетъ, чувствуется благорастворенный воздухъ, земля изъ нѣдръ своихъ издаетъ сокровища своя, сѣменъ и кореней плоды являются, и всѣмъ себе подаютъ въ употребленіе; луга, нивы, поля, лѣса одѣваются и зеленѣютъ, украшаются цвѣтами и издаютъ всякое благоуханіе, протекаютъ источники и рѣчная устремленія не токмо видѣніе, но и слухъ веселятъ; всюду слышится различный различныхъ птицъ гласъ, какъ сладкая нѣкая музыка; расходятся по полямъ и степямъ скоты, не требуютъ отъ насъ корма, питаются и насыщаются съ довольствіемъ, что рука Божія отверзла имъ, довольствуются, ядятъ и играютъ, и тако какъ бы благости Божіей благодарятъ; словомъ: вся поднебесная въ новый прекрасный и веселый премѣняется видъ; безчувственная и чувствомъ одаренная тварь какъ бы вновь раждается. Тебѣ, разумной твари, сіе все богатство благости Божіей открылося: глаголи благодарно: \textit{благослови, душе моя, Господа}\footnote{Пс.~103,~1.}. Принесла земля различные плоды къ насыщенію и утѣшенію твоему: благодари Устроившему тако, глаголя радостнымъ духомъ со пророкомъ: \textit{возвеселилъ мя еси, Господи, въ твореніи Твоемъ}\footnote{91,~5.}. Вкушаешь пищу: *благодари Дающему пищу всякой плоти. Одѣваешися одеждою: благость Божія одѣваетъ тя*: благодари Благодателю. Грѣется покой твой или варится пища огнемъ, "--- Божій даръ огнь есть, который тебѣ работаетъ и служитъ: благослови Сотворшаго его. День прешедъ, пой благодарно, яко сподобилъ тя Господь безъ вреда и погибели проводити его; пой сердцемъ и устнами: \textit{пѣснословлю Тя, Господи}! Уснулъ въ нощи, и спавъ восталъ, "--- глаголи: \textit{благословлю Тя, Господи}! глаголи со пророкомъ: \textit{азъ уснухъ и спахъ, востахъ, яко Господь заступитъ мя}\footnote{3,~6.}. Чувствуешь удареніе совѣсти за грѣхи: благодареніе Богу принеси, потерпѣвшему согрѣшенія твоя и не предавшему тя въ руки врага твоего; исповѣдайся благодарно со пророкомъ: \textit{исповѣмся Тебѣ, Господи, всѣмъ сердцемъ моимъ, и прославлю имя Твое во вѣкъ: яко милость Твоя велія на мнѣ, и избавилъ еси душу мою отъ ада преисподнѣйшаго}\footnote{85,~12 и 13.}. Ощущаешь страхъ суда Божія, или желаніе вѣчнаго живота Божія, "--- то благодать зоветъ тя на покаяніе, "--- глаголи: \textit{благословенъ Господь, Иже хощетъ всѣмъ спастися и въ разумъ истины пріити}\footnote{1~Тим.~2,~4.}; повинися званію Божія благодати, дондеже время благопріятно и день спасенія, дондеже слушаетъ и помогаетъ Господь. \textit{Глаголетъ бо: во время благопріятно послушахъ тебе, и въ день спасенія помогохъ тебѣ: се нынѣ время благопріятно! се нынѣ день спасенія}\footnote{2~Кор.~6,~2.}! Пришла скорбь, печаль, болѣзнь или иная какая противность, "--- спасенія твоего тако ищетъ Господь твой: благодари Хотящему и Ищущему спасенія твоего; глаголи со Псаломникомъ: \textit{благо мнѣ, яко смирилъ мя еси, яко да научуся оправданіемъ Твоимъ}\footnote{Пс.~118,~71.}. Помышляешь о спасительномъ Сына Божія къ роду человѣческому смотреніи (всегда же долженъ помышлять о такъ великомъ и важномъ дѣлѣ): съ радостію и глубокимъ смиреніемъ воспой: \textit{благослови, душе моя, Господа, и вся внутренняя моя имя святое Его! благослови, душе моя, Господа, и не забывай всѣхъ воздаяній Его}\footnote{102,~1,~2.} и проч.; или тако съ пророкомъ Захаріею благослови Господа: \textit{благословенъ Господь Богъ Израилевъ! яко посѣти и сотвори избавленіе людемъ Своимъ, и воздвиже рогъ спасенія намъ въ дому Давида отрока Своего}\footnote{Лук.~1,~68 и 69.}. Слышишь гласы пророческіе и апостольскіе, слышишь проповѣдающихъ Божіе слово, разумѣй, яко тебѣ сіи посланники Божіи служатъ, \textit{иже благовѣствуютъ миръ, благовѣствуютъ благая}\footnote{Римл.~10,~15.}, проповѣдуютъ отпущеніе грѣховъ, приближившееся царствіе небесное, научаютъ правдѣ, наставляютъ на путь правый, ведутъ къ вѣчному блаженству: глаголи благодарнымъ духомъ: \textit{слава Богу, Благодѣтелю нашему, во вѣки, аминь}! "--- Отъ вышереченныхъ слѣдуетъ, что какъ въ благоденствіи, такъ и злоденствіи должны мы Богу благодарить. Въ благоденствіи: яко благая Его недостойные туне получаемъ отъ Него, и тѣми утѣшаемся. Въ злоденствіи: яко исправляемся отъ Него, въ чувство приходимъ, познаемъ себе, свою подлость, недостоинство и окаянство, научаемся благая Божія за велико имѣть и благихъ Дателя почитать.

\paragraph*{§\:222.} Благодарность, понеже благопріятна есть, большихъ и множайшихъ благодѣяній сподобляется отъ Бога. И сіе, кажется, являетъ Хрістосъ, когда глаголетъ: \textit{иже имать, дастся ему и преизбудетъ ему}\footnote{Матѳ.~13,~12.}; и паки: \textit{имущему вездѣ дано будетъ, и преизбудетъ}\footnote{25,~29.}. Благодарность бо въ даръ Себѣ поставляетъ преблагій Владыка нашъ, и за сей даръ новыя воздастъ радованія. «Благодарность, глаголетъ Василій Великій, пріемлющихъ въ благодѣяніе вмѣняетъ Богъ; и Который имѣнія тебѣ даетъ, Той проситъ отъ тебе милостыни чрезъ руки нищихъ; и хотя Свое пріемлетъ отъ тебе, обаче, какъ за собственное твое добро, воздаетъ тебѣ истинную Свою благодать»\footnote{Сл. о Іулит. мученицѣ.}. Тоежде и Хрістосъ къ сущимъ одесную на судѣ Своемъ праведномъ глаголетъ: \textit{понеже сотвористе единому сихъ братій Моихъ меньшихъ, Мнѣ сотвористе}\footnote{Матѳ.~25,~40.}. Что бо имѣемъ свое, кромѣ грѣховъ? и что можемъ дать Тому, отъ Котораго все имѣемъ, какъ выше сказано? Но съ благодареніемъ къ Богу Божіе добро дающій ближнему, какъ бы свое даетъ; и Богъ тое дарованіе Себѣ вмѣняетъ, и потому новымъ благодѣяніемъ, \textit{яко богатъ сый въ милости и щедротахъ}, дающаго награждаетъ, и благодать за благодать воздаетъ.

\paragraph*{§\:223.} Неблагодарность человѣческую на многихъ мѣстахъ Божіе слово обличаетъ, изъ которыхъ нѣкоторыя здѣсь предлагаю ради того, чтобы показать, какъ"=то тяжкій есть грѣхъ неблагодарность. Святый пророкъ Моисей, хотя обличить неблагодарность людей Израилевыхъ, небо и землю призываетъ во свидѣтельство словесъ своихъ, глаголя: \textit{вонми небо, и возглаголю, и да слышитъ земля глаголы устъ моихъ. Да чается яко дождь вѣщаніе мое, и да снидутъ яко роса глаголи мои, яко туча на троскотъ, и яко иней на сѣно. Яко имя Господне призвахъ, дадите величіе Богу нашему. Богъ, истинна дѣла Его, и вси путіе Его, судъ: Богъ вѣренъ, и нѣсть неправды въ Немъ; праведенъ и преподобенъ Господь. Согрѣшиша, не Того чада порочная: роде строптивый и развращенный! сія ли Господеви воздаете? сіи людіе буіи и немудри! не Самъ ли Сей отецъ твой стяжа тя, и сотвори тя, и созда тя? Помяните дни вѣчныя, разумѣйте лѣта рода родовъ; вопроси отца твоего и возвѣститъ тебѣ, старцы твоя, и рекутъ тебѣ. Егда раздѣляше Вышній языки, яко разсѣя сыны Адамовы, постави предѣлы языковъ по числу ангелъ Божіихъ. И бысть часть Господня, людіе Его Іаковъ, уже наслѣдія Его Израиль. Удовли его въ пустыни, въ жажди зноя въ безводнѣ, обыде его, и наказа его, и сохрани его яко зѣницу ока. Яко орелъ покры гнѣздо свое, и на птенцы своя возжелѣ: простеръ крылѣ свои, и пріятъ ихъ, и подъятъ ихъ на раму Своею. Господь единъ вождаше ихъ, и не бѣ съ ними богъ чуждь. Возведе я на силу земли, насыти ихъ житъ сельныхъ; ссаша медъ изъ камене, и елей отъ тверда камене. Масло кравіе, и млеко овчее, съ тукомъ агнчимъ и овнимъ, сыновъ юнчихъ и козлихъ, съ тукомъ пшеничнымъ и кровь гроздову піяху вино. И яде Іаковъ, и насытися, и отвержеся возлюбленный; уты, утолстѣ, разширѣ: и остави Бога сотворшаго его, и отступи отъ Бога Спаса своего. Прогнѣваша Мя о чуждихъ, и въ мерзостехъ своихъ преогорчиша Мя: пожроша бѣсовомъ, а не Богу, богомъ, ихже не вѣдѣша, нови и секрати} (недавни) \textit{пріидоша, ихже не вѣдѣша отцы ихъ. Бога рождшаго тя оставилъ еси, и забылъ еси Бога питающаго тя}\footnote{Второз.~32,~1--18.}, "--- и проч. Исаія пророкъ такожде призываетъ небо и землю во свидѣтельство жалобы Божіей на неблагодарность людей Израилевыхъ: \textit{слыши небо, и внуши земле, яко Господь возглагола: сыны родихъ и возвысихъ, тіи же отвергошася Мене. Позна волъ стяжавшаго его, и оселъ ясли господина своего: Израиль же Мене не позна, и людіе Мои не разумѣша Мене}. А далѣе пророкъ, съ великою по Бозѣ ревностію укоряя ихъ, глаголетъ: \textit{увы языкъ грѣшный, людіе исполнени грѣховъ, сѣмя лукавое, сынове беззаконніи, остависте Господа, и разгнѣвасте Святаго Израилева: отвратистеся вспять}, и прочая\footnote{Ис.~1,~1--4.}. Чрезъ Малахію пророка глаголетъ Господь о непочитающихъ и небоящихся Его, чего корень есть неблагодарность: \textit{сынъ славитъ отца, и рабъ господина своего убоится: и аще Отецъ есмь Азъ, то гдѣ слава Моя? и аще Господь есмь Азъ, то гдѣ есть страхъ Мой? глаголетъ Господь Вседержитель}\footnote{Мал.~1,~6.}. Псаломникъ на неблагодарность людей Израилевыхъ жалуется, глаголя: \textit{забыша благодѣянія Его, и чудеса Его, яже сотвори имъ}; и паки: \textit{возлюбиша Его усты своими, и языкомъ своимъ солгаша Ему: сердце же ихъ не бѣ право съ Нимъ, ниже увѣришася въ завѣтѣ Его}\footnote{Пс.~77,~11,~36 и 37.}. Хрістосъ, когда десять прокаженныхъ очистилъ, а единъ только отъ тѣхъ очистившихся, Самарянинъ, возвратился къ Нему со благодареніемъ, глаголетъ: \textit{не десять ли очистишася, да девять гдѣ? како не обрѣтошася, возвращшеся, дати славу Богу, токмо иноплеменникъ сей}\footnote{Лук.~17,~17 и 18.}. И ко Іудѣ, предателю своему, обличая его неблагодарность, глаголетъ: \textit{Іудо! лобзаніемъ ли Сына человѣческаго предаеши}\footnote{22,~48.}? "--- Отъ сихъ приведенныхъ и прочіихъ святаго Писанія мѣстъ можемъ видѣть: 1)~Коль тяжкая есть предъ Богомъ неблагодарность. Тяжко людямъ терпѣть неблагодарность тѣхъ, коимъ благодѣяніе какое либо сдѣлали; кольми паче Богу. Люди бо какое ни дѣлаютъ добро ближнимъ своимъ, Божіе есть добро, а не ихъ собственное; отъ Бога имѣютъ, а не отъ себе, добро всякое. Ибо что ни имѣемъ, кромѣ грѣховъ, Божіе есть, а не наше, и потому, когда дѣлаемъ кому добро, Божія добра, намъ даннаго, удѣляемъ. Однакожъ, хотя и не своимъ добромъ пользуютъ кого и отъ него не видятъ благодарности, не безъ болѣзни претерпѣваютъ тое, какъ сіе всякому безсумнительно. Какъ несравненно тягчайшая есть наша неблагодарность предъ Богомъ, Который не чужая, но Своя благая, благая безчисленная, умомъ непостижимая, вѣдомая намъ и невѣдомая, показуетъ намъ, какъ сказано выше, и показуетъ не токмо на всякій день, но и на всякій часъ и минуту, такъ что и минуты безъ благихъ Его жить не можемъ! "--- 2)~Чимъ кто большихъ благихъ сподобляется отъ Бога, тѣмъ болѣе и тяжчае грѣшитъ, когда Ему не показуетъ благодарности. Тягчайшая неблагодарность предъ Богомъ была неблагодарныхъ Іудеевъ, нежели язычниковъ, незнающихъ Его. Ибо Іудеи, кромѣ общихъ благодѣяній Божіихъ, которыя изливаются благимъ и злымъ, праведнымъ и неправеднымъ, какъ"=то суть временная и земная благая, особливыя милости паче язычниковъ отъ Бога сподобилися. \textit{И бысть часть Господня, людіе Его Іаковъ, уже наслѣдія Его Израиль. Удовли его въ пустыни}, и прочая\footnote{См. выше въ приведен. Писаніи.}. \textit{Ихъ всыновленіе и слава, и завѣти и законоположеніе, и служеніе и обѣтованія; ихъ отцы, и отъ нихже Хрістосъ по плоти, сый надъ всѣми Богъ, благословенъ во вѣки, аминь}, воздыхаетъ о нихъ Павелъ\footnote{Римл.~9,~4 и 5.}. \textit{Не сотвори тако всякому языку, и судьбы Своя не яви имъ}, глаголетъ пророкъ, то"=есть, какъ сотворилъ Іудеямъ\footnote{Пс.~147,~9.}. Потому и неблагодарствіе ихъ предъ Богомъ тягчайшее есть паче язычниковъ, якоже они на многихъ мѣстахъ святаго Писанія обличаются. \textit{Согрѣшиша, не Того чада порочная: роде строптивый и развращенный, сія ли Господеви воздаете}? глаголетъ Моѵсей святый къ нимъ въ Пѣсни своей. \textit{Сыны родихъ и возвысихъ, тіи же отвергошася Мене}, глаголетъ Богъ чрезъ пророка. \textit{И забыша благодѣянія и чудеса Его, яже показа имъ}. "--- Великая и тяжкая была неблагодарность неблагодарныхъ Іудеевъ; но большая еще и тягчайшая неблагодарность есть неблагодарныхъ хрістіанъ. Ибо Іудеи оные не видѣли Бога во плоти: хрістіане видятъ. Не видѣли они ходяща по земли во образѣ раба; не видѣли *проповѣдующа, не видѣли* чудодѣйствующа; не видѣли плотію страждуща, плотію умирающа за грѣшниковъ, плотію востающа отъ мертвыхъ, возносящася на небо, и съ прославленною плотію сѣдяща одесную Отца: хрістіане видятъ. Іудеи не видѣли Духа Святаго, сшедшаго во огненныхъ языцѣхъ: хрістіане видятъ. Іудеи не слышали приближившагося царствія небеснаго: хрістіане слышатъ. Іудеи избавлени отъ работы Египетскія: хрістіане отъ работы и мученія адскаго. Іудеевъ Моѵсей избавилъ: хрістіанъ Сынъ Божій. Іудеи введены въ землю обѣтованную: хрістіанамъ отверста дверь царствія небеснаго. Іудеи не причащалися тѣла и крови Хрістовой: хрістіане причащаются. Іудеи не слышали Бога съ небесе глаголюща: \textit{Сей есть Сынъ Мой возлюбленный, о Немже благоволихъ}\footnote{Матѳ.~1,~3,~17.}: хрістіане слышать. \textit{Мнози пророцы и праведницы вожделѣша видѣти, яже видятъ} хрістіане, \textit{и не видѣша, и слышати, яже слышатъ, и не слышаша}\footnote{13,~17.}: хрістіане видятъ и слышатъ. Большія убо милости, чести и преимущества сподобилися отъ Бога хрістіане, нежели Іудеи. Ради того болѣе благость Божію огорчаютъ, когда являются неблагодарными, нежели Іудеи. Чимъ бо большее благодѣяніе есть, тѣмъ болѣе обдолжаетъ къ благодаренію, тѣмъ безстуднѣйшій бываетъ благодѣтельствуемый, когда благодарнаго почтенія не воздаетъ благодѣтелю своему. Іудеевъ неблагодарныхъ обличали пророки: хрістіанъ неблагодарныхъ обличитъ Самъ Хрістосъ, праведный Судія, въ страшномъ Своемъ пришествіи: \textit{взалкахся, и не дасте Ми ясти; возжадахся, и не напоисте Мене; страненъ бѣхъ, и не введосте Мене; нагъ, и не одѣясте Мене; боленъ, и въ темницѣ, и не посѣтисте Мене}\footnote{Матѳ.~25,~42 и 43.}. Хотя то и всѣмъ грѣшникамъ страшенъ будетъ громъ сей, но паче хрістіанамъ неблагодарнымъ, познавшимъ истину, и во истинѣ не ходившимъ. Весьма стыдно и страшно будетъ выговоръ сей слышать неблагодарнымъ грѣшникамъ; слышать предъ всѣмъ свѣтомъ, предъ святыми ангелами и избранными Божіими; слышать отъ Хріста, ради всѣхъ сшедшаго съ небесъ и воплотившагося, ради всѣхъ пожившаго на земли, ради всѣхъ страдавшаго, распеншагося, умершаго, погребеннаго и изъ мертвыхъ воскресшаго, такъ всѣхъ возлюбившаго, "--- страшно: яко лишатся всея Божія милости на вѣки безконечныя.

\paragraph*{§\:224.} Знаки неблагодарности примѣчаются сіи: 1)~Благодѣяніе Божіе забывать. И забыша благодѣянія и чудеса Его, яже показа имъ, о Іудеяхъ неблагодарныхъ пишется. Благодарный бо всегда поминаетъ благодѣяніе, и, смотря умомъ на благодѣяніе, благодаритъ благодѣтелю. Тако Давидъ святый всегда помнилъ и благословилъ благодѣтеля своего Господа: \textit{благословлю Господа на всякое время, выну хвала Его во устѣхъ моихъ}\footnote{Пс.~33,~2.}, "--- и къ тому душу свою возбуждалъ: \textit{благослови, душе моя, Господа, и не забывай всѣхъ воздаяній Его}; "--- но и другихъ къ тому возбуждалъ: \textit{возвеличите Господа со мною, и вознесемъ имя Его вкупѣ}\footnote{ст.~4.}. Тако тріе отроки, спасшіися въ пещи Халдейстѣй, не токмо сами пѣли спасшаго Господа, но и всю тварь къ пѣнію созывали: \textit{благословите вся дѣла Господня Господа, пойте и превозносите Его во вѣки}\footnote{Дан.~3,~57,~26--90.}. Неблагодарный не тако: но тогда только помнитъ, когда получаетъ благодѣяніе; тогда только хвалитъ благодѣтеля, когда утѣшается благодѣяніемъ его; а какъ утѣшеніе отъидетъ, забываетъ благодѣяніе и благодѣтеля. Тако Іудеи, изшедшіе изъ Египта, какъ только чрезъ Чермное море перешли и воспѣли хвалу Божію, вскорѣ забыли такъ великое Божіе дѣло, якоже объявляетъ о нихъ Псаломникъ: \textit{и воспѣша хвалу Его, ускориша, забыша дѣла Его}\footnote{Пс.~105,~12--14.}. Тако и нынѣ многіе хрістіане, или отъ болѣзни, или отъ смерти, или отъ темницы, или отъ иной какой бѣды избавившеся, забываютъ сіе Божіе благодѣяніе. Такожде питаются, одѣваются и прочіимъ добромъ Божіимъ снабдѣваются; но притомъ не поминаютъ благодѣтеля. Чаютъ воскресенія мертвыхъ и жизни будущаго вѣка, но Умершему за нихъ и Воскресшему отъ сердца не благодарятъ, и дѣла сего великаго не тщатся въ памяти содержать, и такъ являются Богу, Благодѣтелю своему, неблагодарни. "--- 2)~Знакъ великія неблагодарности есть Божіе благодѣяніе себѣ приписывать и тѣмъ хвалитися, или немощной твари: какую неблагодарность показали Израильтяне оные, которые великое оное своего изъ Египта избавленія Божіе дѣло тельцу златому приписали: \textit{се бози твои, Израилю, изведшіи тя изъ земли Егѵпетскія}\footnote{Исх.~32,~4.}?! Тако и нынѣ многіе, хотя не приписуютъ дѣла Божія бездушному идолу, но себе вмѣсто идола поставляютъ; и что Богу единому приличествуетъ, себѣ приписуютъ: я"=де тое и тое сдѣлалъ; я того и того научилъ; я того и того исцѣлилъ, того и того отъ смерти избавилъ; я того и того обогатилъ, "--- и прочія безумныя рѣчи. Что глаголешь, о человѣче суетный? Что ты можешь добра сдѣлать безъ Бога? Откуду ты взялъ разумъ? Откуду богатство, *откуду искусство*? Что имѣешь свое собственное, кромѣ грѣховъ? \textit{Что имаши, егоже нѣси пріялъ? Аще же и пріялъ еси, что хвалишися, яко не пріемъ}\footnote{1~Кор.~4,~7.}? Како можешь другихъ научить добру, самъ золъ и невѣжда? Како можешь другихъ просвѣтить, самъ слѣпъ? Како можешь другихъ обогатить, самъ нищъ? Како можешь другихъ избавить, самъ на всякую минуту помощи требуяй? Богъ единъ всѣхъ просвѣщаетъ, единъ научаетъ, единъ обогащаетъ, единъ избавляетъ; отъ Того единаго разумъ, помощь, избавленіе, богатство, исцѣленіе; Тому единому подобаетъ честь и похвала о всемъ. Разумъ твой, искусство, богатство, крѣпость Божіе дарованіе есть: почто убо тое, что Божіе есть, себѣ, человѣку немощному, скудному, бѣдному и ничтоже сущему, присвояешь? Какъ отъиметъ Богъ Свое, то останутся тебѣ только твои грѣхи, которые тебе должны смирять, а не дарованіе Божіе возносить? "--- 3)~Знаменіе неблагодарности есть дарованіе Божіе скрывать, или употреблять не въ славу Божію, но на свою корысть и прихоти. Таковые суть, которые имѣютъ разумъ, но не пользуютъ имъ ближнихъ своихъ, или, что хуждшее, употребляютъ разумъ на вредъ братіи своей; составляютъ язвительныя писанія, ложныя клеветы, которыми тщатся истину низложить, ложь поставить. Коль же сіе великое зло, всякъ можетъ видѣть. Такожде, которые богатство міра сего имѣютъ и сокрываютъ его, или на непотребные расходы употребляютъ, а требующихъ ради Хрістова имени презираютъ. Такожде которые, на чести будучи, не ищутъ чести и славы Божіей и пользы ближнихъ, "--- ради чего на честь всякъ зовется, "--- а своего сквернаго прибытка. Злоупотребленіе же дарованія Божія потому за неблагодарствіе признается, понеже дарованіе ради того намъ дается, дабы мы и сами имъ пользовалися, и благодарили Бога, и другихъ тѣмъ пользовали въ честь давшаго намъ добро Свое. Люди бо, пользуяся добромъ Божіимъ, отъ насъ подаемымъ, убѣждаются прославлять Бога, всѣхъ благъ Дателя. Злоупотребляющій же дарованіе пресѣкаетъ прославленіе сіе, и отворяетъ путь къ негодованію и хуленію. Сіе"=то значитъ выговоръ оный Хрістовъ къ сущимъ ошуюю Его на страшномъ судѣ: \textit{взалкахся, и не дасте Ми ясти; возжадахся, и не напоисте Мене; страненъ бѣхъ, и не введосте Мене; нагъ, и не одѣясте Мене; боленъ, и въ темницѣ, и не посѣтисте Мене}. «Я вамъ далъ благая Моя ради вашей и ближнихъ вашихъ пользы; а вы не хотѣли ради имене Моего требующимъ ихъ сообщать, но ими прихотямъ и страстямъ своимъ работали. Не требовалъ Я отъ васъ Себѣ пищи, одежды, покрова, посѣщенія, утѣшенія, но требовали братія Моя \textit{меньшая}, и \textit{понеже имъ не сотвористе} ради Мене, то \textit{ни Мнѣ сотвористе}»\footnote{Матѳ.~25,~42--45.}. "--- 4)~Знаменіе неблагодарности есть нетерпѣніе и роптаніе въ бѣдахъ, потому что бѣдами насъ хощетъ Богъ исправить и къ Себѣ привлещи, "--- за что подобало бы отъ сердца благодарити Богу, такъ милостивно о насъ промышляющему. Сколько такихъ есть, которые бѣдами подвиглися къ истинному покаянію, и такъ къ Богу возвратилися, не токмо святое Писаніе, но и церковная исторія свидѣтельствуетъ. Ниневитяне слѣдующею бѣдою обратилися отъ злыхъ своихъ дѣлъ и начали каятися\footnote{Іоны 3.}. Манассія, царь Іудинъ, бѣдами подвигнулся, и взыскалъ Господа смиреніемъ и покаяніемъ\footnote{2~Пар.~33,~12 и 13.}. Бѣдами Израильтяне подвигшеся, обращалися къ Богу, какъ о томъ наипаче книга Судей свидѣтельствуетъ. Бѣдою понуждаема, жена хананейская вопіетъ Хрісту: \textit{помилуй мя, Господи Сыне Давидовъ}\footnote{Матѳ.~15,~22.}, и проч. Павелъ апостолъ согрѣшившаго въ Коринѳѣ \textit{предаетъ сатанѣ во изможденіе плоти, да духъ спасется въ день Господа нашего Іисуса Хріста}\footnote{1~Кор.~5,~5.}. О коль многіе и нынѣ въ бѣдахъ ищутъ Бога, которые въ благополучіи оставляли Его! Коль многихъ нищета смиряетъ, которыхъ богатство возносило! Коль многихъ скорбь и печаль подвигаетъ къ истинной молитвѣ и сердечному воздыханію, которые въ мірской радости не поминали Бога! Коль многихъ болѣзнь убѣждаетъ презирать славу, честь, богатство, сладость міра сего, которые здоровые въ тѣхъ утѣшались, "--- и искать царствіе Божіе и правду Его, которые въ здравіи не думали о томъ! Искусивыйся вѣсть сія. О, противность "--- горькое вкусомъ врачевство, но спасительное, жезлъ отеческаго Божія наказанія, отъ сна грѣховнаго возбуждающій, искорененіе плотскаго пристрастія, возбужденіе къ духовному житію, низложеніе гордости, училище терпѣнія, воспитаніе смиренія, начало духовныя мудрости, вождь къ молитвѣ, приведеніе къ Богу, ходатай упованія, которое не посрамитъ! \textit{Скорбь бо терпѣніе содѣловаетъ, терпѣніе же искусство, искусство же упованіе, упованіе же не посрамитъ}\footnote{Римл.~5,~3--5.}. Благословенъ еси Господи Боже отецъ нашихъ, яко различнымъ образомъ приводиши насъ къ Себѣ и вѣчному Твоему блаженству. Явный убо знакъ, возлюбленный хрістіанине, неблагодарности къ Богу, кто въ бѣдахъ не имѣетъ терпѣнія: бѣдами бо какъ бы убѣждаетъ насъ Богъ входити въ блаженство Его вѣчное. Помяни, сколько погибло пожившихъ въ благополучіи и веселостяхъ міра сего. Лучше убо намъ молить Бога, чтобы отеческимъ Своимъ жезломъ наказывалъ, и милости Своя не разорялъ отъ насъ: \textit{накажи насъ, Господи, обаче въ судѣ, а не въ ярости}, съ пророкомъ глаголати\footnote{Іер.~10,~24.}, нежели съ роптаніемъ и негодованіемъ наказаніе Его пріимати, "--- что и Его благости противно, и намъ вредно. Всѣ вѣрные Его отъ начала міра съ благодареніемъ наказаніе Его пріимали и ожидали милости отъ Него, и получили. Тоежъ и намъ должно дѣлать, когда хочемъ милость Его получить. "--- 5)~Знакъ неблагодарности есть немилосердіе и суровость къ ближнему, когда человѣкъ по всякъ день и часъ отъ Бога великія милости сподобляется, а самъ не хощетъ ради Бога подобнаго себѣ человѣка помиловать. Ибо ради милосердія Божія, которое человѣку по всякъ день дѣлается, сколько бы обиды ни принялъ отъ ближняго, вси отпустить отъ сердца долженъ, такъ что хотя бы нѣсколько разъ убитъ былъ и ожилъ (ежели бы тое могло статься), простить долженъ убившему. Понеже какая бы обида ни сдѣлалась человѣку отъ человѣка, и сколько бы ни сдѣлалось обиды, то ничтоже есть противу обиды, которою величество Божіе подлый человѣкъ, \textit{земля и пепелъ}, оскорбляетъ, преступая законъ Его, и милости Его не лишается. Тогда сію, такъ великую Божію милость, не помнитъ человѣкъ, когда ближнему согрѣшившему не хощетъ оставить. И сіе есть великое человѣческое самолюбіе и лукавство, что самъ онъ отъ Бога получаетъ милость, а ближняго не хощетъ миловать. Такъ поступилъ съ клевретомъ своимъ жестокосердый оный должникъ, въ притчѣ евангельской упоминаемый, которому царь, по милости своей, тму талантовъ (десять тысящь) отпустилъ, а онъ самъ клеврету своему не хотѣлъ и ста пѣнязей, такъ малаго числа, отпустить. Чего ради и милости, которой сподобился"=было отъ милостиваго господина своего, лишился; и гнѣвъ отъ него съ выговоромъ почувствовалъ: \textit{рабе лукавый! весь долгъ оный отпустихъ тебѣ, понеже умолилъ мя еси: не подобаше ли и тебѣ помиловати клеврета твоего, якоже азъ тя помиловахъ? И разгнѣвася господь его, предаде его мучителемъ, дондеже воздастъ весь долгъ свой}\footnote{Матѳ.~18,~24--33.}. Такойжде гнѣвъ отъ Бога, Царя небесе и земли, почувствуютъ и вси тѣ, которые отъ Него всегда толикія милости сподобляются, толико Его оскорбляя не пріемлютъ казни; а и сами малѣйшія милости ближнему явить не хотятъ, но за малое поносное слово влекутъ на судъ и предаютъ темницѣ. \textit{Тако и Отецъ Мой небесный сотворитъ вамъ, аще не отпустите кійждо брату своему отъ сердецъ вашихъ прегрѣшенія ихъ}, заключаетъ Хрістосъ притчу оную\footnote{ст.~35.}. Сію немилость являютъ господа рабамъ своимъ и командиры подкоманднымъ, когда или за свою обиду, или не по мѣрѣ преступленія съ гнѣву жестоко мучатъ ихъ; такожде прочіе, которые за обиду и погрѣшеніе или судомъ, или инымъ образомъ отмщеніе дѣлаютъ ближнимъ своимъ, не взирая на тое, какъ милосердо съ ними Богъ поступаетъ, Который \textit{аще бы беззаконія назрилъ, кто бы постоялъ}\footnote{Пс.~129,~3.}? "--- 6)~Знакъ неблагодарности есть устами Богу благодарить, но сердце съ устами несогласное имѣть, устами почитать, но сердцемъ и житіемъ своимъ безчестить Его. О таковыхъ глаголетъ Псаломникъ: \textit{возлюбиша Его усты своими, и языкомъ своимъ солгаша Ему: сердце же ихъ не бѣ право съ Нимъ, ниже увѣришася въ завѣтѣ Его}. И индѣ Богъ глаголетъ: \textit{приближаются Мнѣ людіе сіи усты своими, и устнами чтутъ мя: сердце же ихъ далече отстоитъ отъ Мене}\footnote{Матѳ.~15,~8; Ис.~29,~13.}. Таковые суть, которые въ церковь на славословіе ходятъ, но дѣлами хулятъ Бога; языкомъ поютъ, но преступленіемъ святаго закона Его безчестятъ. Таковые суть, которые созидаютъ храмы Божіи каменные или деревянные, но разоряютъ храмы Божіи одушевленные; украшаютъ образъ Хрістовъ и святыхъ Его, но человѣки, по образу Божію и по подобію бывшіе, обнажаютъ; облагаютъ святое Евангеліе сребромъ, златомъ и драгимъ, по мнѣнію человѣческому, каменіемъ, но написаннаго въ Евангеліи и коснуться не хотятъ. Таковые паки суть, которые строятъ богадѣльни, обильныя раздаютъ милостыни, но съ другихъ, подобной себѣ братіи, сдираютъ, "--- что не милость есть, но безчеловѣчіе, не жертва, но мерзость есть предъ Богомъ. \textit{Якоже жряй сына предъ отцемъ его: тако приносяй жертвы отъ имѣнія убогихъ}, глаголетъ Писаніе\footnote{Сир.~34,~20.}. Коль пріятно смотрѣть отцу, когда сынъ предъ нимъ закалается, толь пріятна Богу жертва твоя, которую ты приносишь отъ слезъ убогихъ! Да развѣ ты хочешь утаить предъ Богомъ, что чужое приносишь, и толико слезъ проливъ приносишь! Но \textit{Насаждей ухо, не слышитъ ли} плача рабовъ Своихъ? И \textit{Создавый око, не сматрятъ ли} слезъ ихъ, которыя ты пролилъ\footnote{Пс.~93,~9.}? Ты приносишь имѣнія ихъ, а они плачъ и слезы проливаютъ предъ Богомъ: и обоихъ дѣла, твоя неправда, а ихъ стенаніе, до Бога доходятъ. И такъ Онъ и твою неправду видитъ, и ихъ плачъ и стенаніе слышитъ, и воздастъ тебѣ и имъ \textit{свое}. И ты, что приносишь, то не твое, но бѣдныхъ, тобою обиженныхъ. Что ты озлобилъ ихъ, то твое собственное, и вошло уже во уши Господа Саваоѳа, и уже записано въ книзѣ Его, и объявитъ тебѣ въ день суда Своего. И тако какъ осмотришься, то увидишь, что жертва твоя не иное что, какъ грѣхъ предъ Богомъ тяжкій, къ Нему вопіющій и отмщенія на тя просящій, и потому \textit{мерзость есть}. А хотя кто и безъ обиды ближняго созидаетъ и украшаетъ храмы, даетъ милостыню и прочая подобная симъ дѣлаетъ, но хощетъ прославитися отъ человѣкъ, "--- къ вышеписанному же числу людей принадлежитъ. Ибо не съ благодарностію Богу, но со тщеславіемъ дѣлаетъ; не Божіей славы ищетъ, но своея корысти; не Бога почитаетъ, но себе; не Бога любитъ, но себе. Безъ любви бо къ Богу и почитанія, отъ любви происходящаго, благодарность быть не можетъ. Ибо и человѣку благодѣтелю безъ любви и истиннаго почтенія ничто непріятно. Любовь бо ничимъ инымъ довольствоваться не можетъ, какъ любовію взаимною. "--- 7)~Превеликія неблагодарности знаменіе есть хула, когда человѣкъ или не признаетъ бытія Божія, или другія какія на имя Его святое и страшное отрыгаетъ хулы. Сего рода люди, или паче выродки рода человѣческаго, и, какъ правду признать, чудовища всего свѣта, послѣднюю разума искру погасили, хотя бы они и звѣзды считали, или землю размѣряли. Они подобно дѣлаютъ тому, который отца, отъ котораго рожденъ, не признаетъ: я"=де не имѣю отца. Они хуждше самыхъ скотовъ, которые хозяевъ своихъ и питающихъ ихъ познаютъ и почитаютъ. Они на зданія, домы, мастерства смотрятъ, а архитектора и хозяина не признаютъ; они пищу сваренную ядятъ, а повара не признаютъ; они письма читаютъ, а писателя не признаютъ, и думаютъ, что то само собою дѣлается. Когда бы кто заключилъ таковыхъ въ темницѣ и не далъ бы куска хлѣба имъ, пусть бы тогда самобытнымъ своимъ довольствовалися хлѣбомъ и свѣтомъ, которыхъ Дателя не хотятъ признавать и почитать. "--- 8)~Наконецъ всякое беззаконное житіе и вѣрѣ хрістіанской противное знакъ есть неблагодарности. Ибо къ Богу благодарность быть не можетъ безъ любви Божіей, которая познается отъ соблюденія заповѣдей Божіихъ, якоже глаголетъ Хрістосъ: \textit{имѣяй заповѣди Моя, и соблюдаяй ихъ, той есть любяй Мя}\footnote{Іоан.~14,~21.}.

\subsection[Глава 8-я. О надеждѣ.]{глава осьмая.\\\bfseries О надеждѣ.}

\begin{quotation}\textit{Сія глаголетъ Господь: проклятъ человѣкъ, иже надѣется на человѣка, и утвердитъ плоть мышцы своея на немъ, и отъ Господа отступитъ сердце его. И благословенъ человѣкъ, иже надѣется на Господа, и будетъ Господь упованіе его: и будетъ яко древо насажденное при водахъ, и во влагѣ пуститъ кореніе свое: не убоится, егда пріидетъ зной, и будетъ на немъ стебліе зелено, и во время бездождія не устрашится, и не престанетъ творити плода}\footnote{Іер.~17,~5,~7 и 8.}.\end{quotation}
\begin{quotation}\textit{Не надѣйтеся на князи, на сыны человѣческія, въ нихже нѣсть спасенія}\footnote{Пс.~145,~3.}.\end{quotation}
\begin{quotation}\textit{Благо есть надѣятися на Господа, нежели надѣятися на человѣка. Благо есть уповати на Господа, нежели уповати на князи}\footnote{Пс.~117,~8 и 9.}.\end{quotation}
\begin{quotation}\textit{Надѣющійся на Господа, яко гора Сіонъ}\footnote{124,~1.}.\end{quotation}


\paragraph*{§\:225.} Не вси едину имѣютъ надежду: иной надѣется на князей и сыновъ человѣческихъ, иной на богатство, иной на честь, иной на разумъ свой, иной на силу свою, иной на другое. Едины \textit{истинные хрістіане}, все прочее оставивше, на единаго Бога надежду свою полагаютъ. И какъ сіи едины истинную надежду имѣютъ, такъ прочіе ложную, и потому заблуждаютъ. Какъ бо кто съ истиннаго пути совратится, заблуждаетъ по различнымъ стезямъ: такъ кто отъ Бога отстанетъ, принужденъ бываетъ искать себѣ помощи отъ различныхъ вещей; но не меньше заблуждаетъ, какъ тотъ, который истинный путь потерялъ, или кто, свѣта очесъ своихъ лишився, не видитъ, куды идетъ.

\paragraph*{§\:226.} Какъ терпѣніе, такъ и надежда въ бѣдствіи познается. Многіе думаютъ, что они на Бога надежду имѣютъ, но бѣда пришедшая открываетъ надежду ихъ и показуетъ, на кого они надѣются. Отъ кого кто ищетъ въ напасти помощи и избавленія, въ томъ и надежду свою полагаетъ. Кто къ человѣку и прочему созданію въ бѣдствіи прибѣгаетъ, тотъ на человѣка и надѣется. Кто ни къ чему иному, какъ только къ Богу единому возводитъ очи свои и помощи отъ Него единаго неуклонно ожидаетъ, хотя и медлитъ помощь Его, тотъ показуетъ, что онъ и въ счастіи и въ несчастіи на единаго Бога имѣетъ упованіе.

\paragraph*{§\:227.} Причины, которыя возбраняютъ на человѣка и прочее созданіе надежду полагать, сіи суть: 1)~Надѣющійся на созданіе грѣшитъ противу заповѣди первой: \textit{Азъ есмь Господь Богъ твой}, которая заповѣдь повелѣваетъ, чтобы мы единаго Бога знали, почитали, любили, боялися, на Него надѣялися, къ Нему въ нуждахъ нашихъ прибѣгали и помощи просили. Ибо таковый отступилъ сердцемъ своимъ отъ Бога, какъ пророкъ глаголетъ: \textit{и отъ Господа отступитъ сердце его}, "--- и не вѣруетъ Богу, хотя устами и исповѣдуетъ Его. Вѣровать бо Богу и на созданіе Его надѣяться невозможно. Ибо надежда съ вѣрою совокупна, и одна безъ другой быть не можетъ. Того ради, кто надеждою отступилъ отъ Бога и приложился къ созданію, тотъ и вѣрою отступилъ отъ Бога. Вѣра же здѣ разумѣется живая, а не мертвая, которая не токмо на языкѣ, но и на сердцѣ имѣется. Таковая вѣра къ единому Богу, какъ малое отроча къ матери своей прилѣпляется, на Него единаго надѣется, отъ Него единаго помощи и избавленія чаетъ. "--- 2)~Подъ клятвою таковые находятся, какъ пророкъ святый глаголетъ: \textit{проклятъ человѣкъ, иже надѣется на человѣка, и утвердитъ плоть мышцы своея на немъ}. Напротивъ того, \textit{благословенъ человѣкъ, иже надѣется на Господа, и будетъ Господь упованіе его}, какъ тойжде пророкъ глаголетъ въ приведенномъ Писаніи. Коль же страшно есть быть подъ проклятіемъ! Что въ такомъ случаѣ пользуетъ имя хрістіанское? весьма ничего! Коль блаженно быть отъ Бога благословенну! Къ \textit{оному} состоянію надежда на человѣка и всякое созданіе: къ \textit{сему} упованіе на Бога приводитъ. "--- 3)~Таковому, который надѣется на созданіе, и молиться правильно невозможно. Вѣра и надежда весьма нужна къ молитвѣ; и безъ вѣры и надежды молиться неможно. Како и съ какою надеждою возведешь очи свои къ Богу, а самъ надеждою отъ Него отторгнулся и приложился къ человѣку, или другому созданію? Молитва бо истинная \textit{духомъ и истиною} совершается, а не токмо языкомъ и словомъ. И Богъ половины сердца, которое и къ созданію и къ Богу хощетъ прилѣпляться, и потому на двое раздѣляется, не пріемлетъ; но \textit{всего} сердца отъ насъ требуетъ. Како убо простретъ руки своя къ Богу той, который простираетъ къ человѣку? Како безъ зазрѣнія совѣсти скажетъ: \textit{Господи, помилуй}; а самъ у человѣка немощнаго милости ищетъ? Како исповѣсть: \textit{на Тя, Господи уповахъ, спаси мя}; а самъ уповаетъ на князей и на сыновъ человѣческихъ? Лицемѣрная у такого молитва, а не истинная, понеже иное на языкѣ, иное на сердцѣ имѣетъ. "--- 4)~Такой человѣкъ въ покоѣ быть не можетъ. Ибо всегда будетъ боятися, скорбь и печаль имѣть, не постояннымъ быть, туды и сюды колебаться и всего опасаться, по подобію дома, который не на твердомъ основаніи, но на пескѣ созданъ, всякаго вѣтра боится. Ибо всякое созданіе премѣненію подлежитъ, и потому непостоянно, того ради и надѣющійся на него не можетъ быть безъ боязни противнаго случая. Такъ надѣющійся на князя боится, чтобы его милости не лишиться (ибо скоро милость человѣческая обращается въ жестокость), или самому ему смерти или другаго несчастія не приключилося. Богатый опасается, чтобы богатства не потерять. Едина надежда на Бога ничего не боится. Ибо Богъ непремѣненъ и во вѣки пребываетъ. \textit{Надѣющійся на Господа, яко гора Сіонъ, не подвижится во вѣкъ}, глаголетъ Псаломникъ. "--- 5)~Никакой надежды не имѣетъ, кто надеждою возлагается на созданіе. Понеже всякое созданіе само помощи, подкрѣпленія, храненія Божія требуетъ: \textit{въ Немъ живемъ, движемся и есмы}\footnote{Дѣян.~17,~28.}. Убо надежда на всякое созданіе есть суетная и прелестная. "--- 6)~На кого хощемъ надѣятися въ день смерти нашея, на того и нынѣ во время живота нашего должно всю надежду полагать, къ нему прибѣгать и прилѣпляться. Тогда насъ все оставитъ: честь, богатство въ мірѣ останется; тогда исчезнетъ сила, разумъ, хитрость и премудрость; тогда ни друзи, ни братія, ни пріятели наши не помогутъ намъ; вси оставятъ насъ тогда. Единъ Хрістосъ, Искупитель нашъ, ежели въ Него нынѣ истинно вѣруемъ и надѣемся на Него, не оставитъ насъ тогда. Онъ сохранитъ насъ тогда; Онъ ангеламъ Своимъ повелитъ спутешествовать намъ, нести души наши въ лоно Авраамле, и тамо упокоитъ насъ. Къ сему убо единому Помощнику вѣрою нынѣ прилѣпляться, и на Него единаго все упованіе возлагать должно намъ, и тако сіе упованіе и во время смерти и по смерти не посрамитъ. Сего ради и святое Писаніе такъ сильно запрещаетъ намъ \textit{надѣятися на князи и на сыны человѣческія} и прочее созданіе, понеже надежда сія отъ Бога отвращаетъ и прельщаетъ человѣка, къ пагубѣ ведетъ, хотя того онъ и не примѣчаетъ.

\paragraph*{§\:228.} Понеже всякому человѣку природна немощь сія и слѣпота, что, чего сими глазами не усматриваетъ, того и не надѣется; и что чувства показываютъ, къ тому и стремится; и врагъ не спитъ, но мечтанія въ мысли наши влагаетъ различныя, и такъ отъ Бога тщится отвести насъ: того ради не неприлично надежда назватися можетъ такая добродѣтель, которая подвизается противъ діавольскихъ козней, мечтаній, противу упованія на самого себе, на свою силу, разумъ, благочестіе, честь, богатство, предстателей, князей, сыновъ человѣческихъ, словомъ: противу упованія на все тое, что кромѣ Бога есть, "--- и въ единомъ токмо человѣколюбіи, всемогуществѣ, истинѣ Божіей утверждаться тщится; или есть вѣра, отводящая сердце наше отъ надежды на насъ самихъ и на прочія созданія, и къ единому Богу приводящая, и отъ Него терпѣливо ожидать милости насъ наставляющая и убѣждающая.

\paragraph*{§\:229.} Хотящему убо истинную и непоколебимую имѣть надежду должно отъ всего созданія сердце свое отвратить, и ни на что тое не надѣяться; но на единаго Бога все упованіе возложить въ счастіи и несчастіи, и отъ Него единаго искать и ожидать безъ сумнѣнія милости, "--- къ чему сіи причины, отъ Божіихъ свойствъ взятыя возбуждаютъ: 1)~Богъ есть вѣчный и непремѣнный, живый и безсмертный, убо и надежда на Него тверда и непоколебима. 2)~Богъ есть всемогущій, Который какъ изъ ничего все создалъ, такъ и все можетъ сдѣлать, хотя чего разумъ нашъ и не понимаетъ. 3)~Богъ есть премудрый, Который знаетъ какъ помощи, избавить и спасти. Гдѣ способа спасенія нѣтъ, тамо Богъ способъ находитъ, и гдѣ пути нѣтъ, тамо путь обрѣтаетъ. 4)~Богъ есть преблагій и милосердый, Который не можетъ не благотворить намъ, и хощетъ насъ помиловать и избавить. 5)~Богъ есть истинный, Который солгать не можетъ въ Своемъ обѣщаніи. Обѣщался же призывающимъ Его помощи и \textit{спасати}, якоже о томъ на многихъ святаго Писанія мѣстахъ обрѣтается. Убо безсумнительно надежда на Него тверда и извѣстна, и не можетъ пасти, хотя бы вѣтры и рѣки ей приразилися, \textit{ибо основана на камени}\footnote{Матѳ.~7,~25.}.

\paragraph*{§\:230.} Надежда сія питается, укрѣпляется съ Божіею помощію: 1)~Чтеніемъ, или слушаніемъ прилѣжнымъ святаго Писанія. \textit{Елика бо преднаписана быша, въ наше наказаніе преднаписашася, да терпѣніемъ и утѣшеніемъ писаній упованіе имамы}, глаголетъ апостолъ\footnote{Римл.~15,~4.}. "--- 2)~Размышленіемъ прежде бывшихъ Божіихъ благодѣяній, отцамъ нашимъ показанныхъ. Пріими во умъ всѣхъ, отъ начала міра бывшихъ отцевъ, которые на Бога уповали, и получили милость. Уповалъ Ной, Авраамъ, Ісаакъ, Іаковъ, и избавилъ ихъ. Уповалъ Іосифъ, и избавленъ бысть. Уповалъ Израиль во Египтѣ, и чудесно отъ работы египетскія свободился, и проч. \textit{На тя уповаша отцы наши: уповаша, и избавилъ еси я. Къ Тебѣ воззваша, и спасошася; на Тя уповаша, и не постыдѣшася}\footnote{Пс.~21,~5 и 6.}, воспоминаетъ Псаломникъ Богу благодѣянія Его, и тѣмъ свою укрѣпляетъ надежду. И паки, прежнюю Божію милость воспоминая, молится: \textit{благоволилъ еси, Господи, землю Твою, возвратилъ еси плѣнъ Іаковль. Оставилъ еси беззаконія людей Твоихъ, покрылъ еси вся грѣхи ихъ. Укротилъ еси весь гнѣвъ Твой, возвратился еси отъ гнѣва ярости Твоея. Возврати насъ, Боже спасеній нашихъ}\footnote{84,~2--5.}, и проч. И симъ воспоминаніемъ укрѣпляетъ себе въ надеждѣ: \textit{услышу, что речетъ о мнѣ Господь Богъ: яко речетъ миръ на люди Своя, и на преподобныя Своя, и на обращающія сердца къ нему}\footnote{ст.~9.}. Тако, воспоминая прежде бывшія Божія благодѣянія, и намъ должно надежду нашу укрѣплять и не отлагать дерзновенія. Ихъ молящихся помиловалъ Богъ: и насъ помилуетъ. Они на Бога уповали, и избавилъ ихъ: и насъ уповающихъ избавитъ. Ихъ упованіе не посрамило: и насъ не посрамитъ. \textit{Нѣсть бо на лица зрѣнія у Бога}\footnote{Римл.~2,~11.}; всѣхъ равно милуетъ Онъ, которые милости ищутъ у Него, всѣхъ равно пріемлетъ, приходящихъ къ Нему.

\paragraph*{§\:231.} Надежда сія безъ терпѣнія быть не можетъ. И гдѣ истинная надежда, тамо и терпѣніе, и гдѣ терпѣніе, тамо и надежда. Понеже надежда подвержена многому искушенію, якоже и вѣра. Искушается отъятіемъ временныхъ благъ, когда лишаемся здравія, чести, богатства, благопріятія человѣческаго, мира, тишины, покоя, и попадаемся во всякое неблагополучіе. Въ семъ бѣдственномъ состояніи потребно есть терпѣніе, чтобы непозволеннымъ образомъ не искать отъ бѣдъ избавленія, но паче сдаться на волю Божію, и ожидать отъ Него милости или \textit{помогающей} въ терпѣніи, или \textit{избавляющей} отъ бѣдъ, якоже вѣсть Самъ. Нѣтъ большаго искушенія надеждѣ, какъ когда помыслы возстающіи въ совѣсти глаголютъ: \textit{нѣсть спасенія ему въ Бозѣ его}\footnote{Пс.~3,~3.}, когда страхъ суда Божія, ужасъ геенны и отчаяніе смущаютъ и колеблютъ душу и утѣсняютъ совѣсть. Удобнѣе человѣку всякое претерпѣвати внѣшнее искушеніе, нежели сіе совѣсти утѣсненіе. Отъ сего бываетъ, что человѣкъ никогда не можетъ веселитися, всегда мракомъ печали покровенъ; и что прочіихъ веселитъ, тое ему большую къ печали причину подаетъ. Въ такъ тяжкомъ искушеніи паче прочіихъ потребно есть терпѣніе, молчаніе, изъ глубины сердца воздыханіе, донелѣже сильная сія непогода пройдетъ или облегчится. Здѣ нужно есть надѣятися паче надежды, и уповати паче упованія, якоже о Авраамѣ пишется: \textit{иже паче упованія во упованіе вѣрова}\footnote{Римл.~4,~18.}. Не слушать, что помыслъ глаголетъ, но что Богъ обѣщаетъ: \textit{хотѣніемъ не хощу смерти грѣшника}\footnote{Іез.~18,~23 и 32.}, и паки: \textit{идѣже умножися грѣхъ, преизбыточествуетъ благодать}\footnote{Римл.~5,~20.}, "--- и прочіимъ утѣшительнымъ Божіимъ обѣщаніямъ внимать. Къ сему подвигу и терпѣливой надеждѣ увѣщаваетъ пророкъ: \textit{потерпи Господа, мужайся, и да крѣпится сердце твое, и потерпи Господа}\footnote{Пс.~26,~14.}, и милостію Божіею обнадеживаетъ: \textit{терпѣніе убогихъ не погибнетъ до конца}\footnote{9,~19.}, а себе во образъ представляетъ: \textit{терпя потерпѣхъ Господа, и внятъ ми, и услыша молитву мою}\footnote{39,~2.}, и паки: \textit{вѣрую видѣти благая Господня на земли живыхъ}\footnote{26,~13.}.

\paragraph*{§\:232.} При семъ должно примѣчать оное Псаломника слово: \textit{уповай на Господа, и твори благостыню}\footnote{36,~3.}, которое слово показуетъ намъ, что уповающему на Бога должно благое творить, послѣдовать воли Божіей, а не своей. Напрасно тотъ надѣется на Бога, который противится Богу; напрасно чаетъ отъ Бога милости, который не престаетъ Его раздражать нераскаяннымъ нравомъ; напрасно руки простираетъ и очи возводитъ къ Богу, который сердцемъ отъ Него отвращается и обращается къ мамонѣ, сребру и злату, нечистотѣ и прочіимъ бездушнымъ божкамъ. Богъ бо есть Избавитель \textit{своимъ}, а не чужимъ, то"=есть, противникамъ Своимъ. \textit{Господь крѣпость людемъ Своимъ дастъ}, глаголетъ Псаломникъ, а не чужимъ\footnote{28,~11.}, \textit{волю боящихся Его сотворитъ}, а не небоящихся, \textit{и молитву ихъ услышитъ боящихся, и спасетъ я}, а не небоящихся\footnote{144,~19.}. Не боятся же Бога, которые безстрашно законъ Его нарушать дерзаютъ. Таковыхъ \textit{воли не сотворитъ}, понеже сами они воли Его не творятъ; \textit{и молитвы не услышитъ}, понеже сами Его не хотятъ слушать и каяться: \textit{но паче лице Господне на творящія злая, еже потребити отъ земли память ихъ}\footnote{33,~17.}. Услышалъ въ Вавилонѣ тріехъ отроковъ, и отъ огня избавилъ, но почитающихъ Его и не отдавшихъ чести Его образу златому. Слушаетъ и нынѣ, но тѣхъ, которые почитаютъ Его устами и сердцемъ, и не покланяются гордости міра сего, какъ высокому и позлащенному идолу, котораго князь тмы и міра сего поставляетъ къ безчестію имене Божія и погибели человѣческой. Таковыхъ, глаголю, которые не страсти своя, ни злато и сребро, созданіе Божіе, но Создателя почитаютъ и покланяются, слушаетъ и въ разжженной искушенія пещи прохлаждаетъ, утѣшаетъ и увеселяетъ благодатію Своею; влагаетъ во уста ихъ пресладкую благодаренія пѣснь: \textit{благословенъ еси, Господи Боже отецъ нашихъ, и препѣтый и превозносимый во вѣки}\footnote{Дан.~3,~52.}! и проч.

\paragraph*{§\:233.} Слушаетъ и грѣшниковъ, но престающихъ грѣшить и кающихся. Услышалъ Манассію царя Іудина, но со смиреніемъ исповѣдующаго грѣхи своя и оставившаго мерзости своя\footnote{2~Пар.~33.}. Услышалъ Ниневитянъ, но проповѣдію Іониною покаявшихся\footnote{Іон.~3.}. Услышалъ Закхея, но смирившагося и кающагося\footnote{Лук.~19,~2--10.}. Услышалъ блудницу, но плакавшую и омывшую слезами нозѣ Его\footnote{7,~37 и 38.}. Принялъ блуднаго сына, но оставившаго чуждую беззаконную страну и возвратившагося къ Нему со смиреніемъ и покаяніемъ: \textit{Отче, согрѣшихъ на небо и предъ Тобою, и уже нѣсмь достоинъ нарещися сынъ Твой; сотвори мя, яко единаго отъ наемникъ Твоихъ}\footnote{15,~17--24.}. Грѣшникъ бо дотоль грѣшникъ есть, доколь грѣшить не престаетъ и живетъ въ безстрашіи; а когда отстанетъ отъ грѣховъ и о грѣхахъ кается, уже Божіею благодатію къ числу праведныхъ присоединяется. Чего ради и грѣшникамъ таковымъ не должно упованія своего отлагать, но безъ сумнѣнія ожидать милости Божіей о Хрістѣ Іисусѣ, который \textit{пріиде въ міръ грѣшныя спасти}\footnote{1~Тим.~1,~15; Лук.~19,~10.}.

\subsection[Глава 9-я. О молитвѣ.]{глава девятая.\\\bfseries О молитвѣ.}

\begin{quotation}\textit{Азъ вамъ глаголю, просите, и дастся вамъ, ищите, и обрящете; толцыте, и отверзется вамъ. Всякъ бо просяй пріемлетъ, и ищай обрѣтаетъ, и толкущему отверзется. Котораго же васъ отца воспроситъ сынъ хлѣба, еда камень подастъ ему? Или рыбы, еда въ рыбы мѣсто змію подастъ ему? Или аще попроситъ яица, еда подастъ ему скорпію? Аще убо вы, зли суще, умѣете даянія блага даяти чадомъ вашимъ, кольми паче Отецъ, Иже съ небесе, дастъ Духа Святаго просящимъ у Него}, глаголетъ Хрістосъ\footnote{11,~9--13.}.\end{quotation}
\begin{quotation}\textit{Молю прежде всѣхъ творити молитвы, моленія, прошенія, благодаренія, за вся человѣки, за царя и за вся, иже во власти суть: да тихое и безмолвное житіе поживемъ во всякомъ благочестіи и чистотѣ}\footnote{1~Тим.~2,~1 и 2.}.\end{quotation}


\paragraph*{§\:234.} Молитва истинная отъ надежды происходитъ. Какъ бо отъ человѣка ничего не просимъ, когда не надѣемся отъ него желаемаго получить; напр. отъ нищаго и скупаго милостыни, отъ скудоумнаго совѣта, отъ немощнаго вспоможенія не просимъ, вѣдая, что не могутъ намъ дать, чего желаемъ: такъ кто не надѣется отъ Бога желаемаго получить, не прибѣгаетъ къ Нему съ прошеніемъ, но уклоняется къ немощной твари. Напротивъ того, какъ къ доброму и щедрому мужу многіе прибѣгаютъ, понеже надѣются отъ него помощь въ нуждѣ своей получить: такъ кто твердую и непоколебимую имѣетъ надежду на Бога, богатаго въ милости и щедротахъ, проситъ отъ Него милости и помощи. И отъ сего видно: 1)~Что кто на созданіе надѣется и отъ него помощи и избавленія ищетъ: показуется, что сердце его отъ Бога отступило, милости и помощи Его отчаялось, и приложилось къ созданію, "--- что грѣхъ есть тяжкій, подобный идолопоклоненію, хотя бы устами и исповѣдалъ Бога таковый человѣкъ. "--- 2)~Хотя не устами, но сердцемъ хулитъ таковый Бога, ибо мнитъ Его немилосерда, или не всемогуща, или неистинна въ обѣщаніяхъ Своихъ, или иное что противное благости Божіей въ сердцѣ имѣетъ.

\paragraph*{§\:235.} Молитва есть прошеніе добра, отъ благочестивыхъ людей къ Богу бываемое. Такъ описуетъ святый Василій великій молитву \textit{въ словѣ о Іулиттѣ мученицѣ}.

\paragraph*{§\:236.} Понеже молитва есть прошеніе добра у Бога, благихъ Источника и Подателя просящимъ у Него; и \textit{всякое даяніе благо и всякъ даръ совершенъ свыше есть, сходяй отъ Отца свѣтовъ}, *какъ апостолъ учитъ\footnote{Іак.~1,~17.}: молитва убо есть всякаго добра*, какъ душевнаго, такъ тѣлеснаго и временнаго виновна, и показуетъ \textit{истиннаго хрістіанина}. 1)~Молитвою исповѣдуемъ тое, что Богъ Самъ глаголетъ: \textit{Азъ есмь Господь Богъ твой}, то"=есть, показуемъ молитвою о себѣ, что мы инаго Бога не знаемъ, кромѣ Его, и на Него единаго всю надежду нашу полагаемъ, Ему единому покланяемся, Его единаго почитаемъ, отъ Него единаго милости ищемъ; и тѣмъ исповѣдуемъ, что Онъ есть Богъ живый, вѣчный, вездѣсущій, всевѣдущій, всемогущій, о всѣхъ промышляющій и единъ всякое добро подающій, и \textit{есть Богъ нашъ, Господь нашъ}, Промыслитель нашъ, \textit{и мы людіе Его и овцы пажити Его}\footnote{Пс.~99,~3.}. "--- 2)~Молитвою показуемъ, что мы Божіе обѣщаніе за велико почитаемъ, и подаемъ извѣстіе, что мы вѣримъ тому, что Богъ обѣщалъ, и что Богъ есть истинный, неложный въ Своихъ обѣщаніяхъ, ибо Онъ самъ велѣлъ намъ молитися, и обѣщалъ услышати насъ. "--- 3)~Молитвою утверждается и умножается вѣра, по подобію древа, которое чѣмъ болѣе орошается, тѣмъ болѣе растетъ. Божія бо благодать, какъ дождь тихій, снисходитъ на молящагося, и орошаетъ сердце его, и плодоноснымъ творитъ къ творенію добрыхъ дѣлъ. "--- 4)~Молитвою убѣгаемъ безстрашія. Молитвою бо исповѣдуемъ Бога вездѣсущаго, вся назирающаго, Котораго, яко вездѣсущаго, должно боятися и почитать. "--- 5)~Молитвою вооружаемся противу искушенія діавола, грѣха и всякаго неблагополучія. Молитва бо есть оружіе хрістіанское, которымъ противимся діаволу и служителямъ его, и сохраняемъ себе и градъ нашъ душевный. "--- 6)~Молитвою прогоняемъ печаль и скорбь. Какъ бо отраду получаемъ нѣкую, когда вѣрному нашему другу сообщаемъ нашу скорбь: такъ, или много болѣе получаемъ утѣшеніе, когда скорбь нашу преблагому и милосердому Богу объявляемъ, и просимъ отъ Него утѣшенія. Тако молитвою Давидъ святый скорбь и печаль свою сообщалъ Богу и получалъ утѣшеніе, какъ о томъ во многихъ псалмахъ показуется. "--- 7)~Молитва есть бесѣда съ Богомъ. Коль же великое дѣло есть человѣку тлѣнному съ Богомъ великимъ и безсмертнымъ бесѣдовать! За велико почитаемъ съ царемъ земнымъ бесѣдовать, кольми паче съ Царемъ небеснымъ и вѣчнымъ бесѣдовать великое и желаемое дѣло есть! "--- 8)~Отъ оставленія молитвы все противное послѣдуетъ. Въ оставляющемъ молитву оскудѣваетъ вѣра и исчезаетъ; ибо человѣкъ самъ собою не можетъ противитися искушенію, и такъ падаетъ въ безстрашіе и всякій грѣхъ: отъ сего послѣдуетъ развращенное и безбожное житіе, а далѣе отчаяніе, наконецъ явная погибель.

\paragraph*{§\:237.} Причины, которыя подаютъ надежду къ молитвѣ: 1)~Богъ всѣхъ возбуждаетъ къ молитвѣ и призываетъ, какъ о томъ на многихъ святаго Писанія мѣстахъ обрѣтается. 2)~Молящагося услышати милостивно обѣщалъ\textit{: просите, и дастся вамъ}\footnote{Матѳ.~7,~7.}. 3)~Самъ Богъ научилъ, какъ и молитися *намъ, якоже молитва*: \textit{Отче нашъ}\footnote{Лук.~11,~1--4.} и псалмы и прочія молитвы пророческія показуютъ, ибо мы сами отъ себе не знаемъ, какъ и о чемъ молитися, того ради милосердый Богъ и тому насъ научилъ.

\paragraph*{§\:238.} Понеже Богъ на всякомъ мѣстѣ есть, "--- \textit{Богъ нашъ на небеси и на земли}, глаголетъ Псаломникъ\footnote{Пс.~113,~11.}, "--- убо на всякомъ мѣстѣ и молитву слушаетъ. Того ради на всякомъ мѣстѣ молитися возможно. Къ царемъ и княземъ надобно далеко итить, трудиться ради нужды, а къ Богу не тако. Гдѣ имѣешься, тамо и Богъ съ тобою; тамо и молись Ему; тамо и слышитъ тебе. Тако апостолъ святый \textit{хощетъ, да молитвы творятъ мужіе на всякомъ мѣстѣ}\footnote{1~Тим.~2,~8.}. И Хрістосъ глаголетъ: \textit{яко пріидетъ часъ, егда ни въ горѣ сей, ни во Іерусалимѣ поклонитеся Отцу}, но на всякомъ мѣстѣ\footnote{Іоан.~4,~21.}.

\paragraph*{§\:239.} Всякое время къ молитвѣ удобное есть, день и нощь, утро и вечеръ, полдень и полночь. Къ человѣку съ прошеніемъ не всегда можно приступать, понеже или нуждами, или немощію, или сномъ, или инымъ чимъ занять *бываетъ*, а къ Богу не тако, но всегда, и всякое время свободное есть; всегда двери отверсты намъ, когда хощемъ приступать къ Нему, пока въ семъ вѣцѣ живемъ; всегда готовъ слушать прошеніе наше\footnote{Матѳ.~7,~7 и 8; Ис.~65,~24.}; всегда готовъ, яко благъ, \textit{благодать подать просящимъ}\footnote{Лук.~11,~13.}.

\paragraph*{§\:240.} Понеже Богъ \textit{Сердцевѣдецъ} есть, то и сердечное наше желаніе, воздыханіе слышитъ и знаетъ, хотя бы уста и молчали. \textit{Желаніе убогихъ услышалъ еси, Господи, уготованію сердца ихъ внятъ ухо Твое}\footnote{Пс.~9,~38.}. Богъ бо не требуетъ отъ насъ словъ. Человѣкъ человѣку не можетъ прошенія своего иначе объявить, какъ только устами и посредствомъ словъ: а Богъ безъ слова и самое помышленіе наше знаетъ; и равно предъ Нимъ какъ слово, такъ и помышленіе наше. Но еще и тое, что имѣемъ помышлять, знаетъ. \textit{Ты разумѣлъ еси помышленія моя издалеча}, глаголетъ Псаломникъ\footnote{138,~2.}. Убо молитва самымъ умомъ и безъ словъ можетъ совершатися. Тако молилася святая Анна, мати Самуилова, какъ о ней Писаніе глаголетъ: \textit{и та глаголаше въ сердцѣ своемъ, токмо устнѣ ея двизастѣся, а гласъ ея не слышашеся}\footnote{1~Цар.~1,~13.}.

\paragraph*{§\:241.} Какъ сердечная молитва безъ гласа и словъ можетъ быть, такъ уста безъ сердца, и слова безъ разума, и гласъ внѣшній безъ внутренняго сердечнаго усердія ничего не пользуетъ. Ибо гласъ внѣшній и слово должно быть согласно съ внутреннимъ помышленіемъ; и слово не иное что, какъ извѣщеніе внутренняго состоянія. Богъ же \textit{проницаетъ сердце наше, и смотритъ на сердце}, а не на слова\footnote{1~Пар.~28,~9; 1~Цар.~16,~7.}. Слѣдственно слова безъ сердечнаго согласія ничтоже суть. Тогда же сіе бываетъ, когда иное глаголетъ языкъ, иное помышляетъ умъ. Сего ради должно тщаться, чтобы и сердце молилося, когда молится языкъ, и умъ помышлялъ тое, что уста глаголютъ; чтобы сердце согласно было слову, и слово сердцу, и помышленіе въ томъ упражнялося, что гласомъ произносится.

\paragraph*{§\:242.} Какъ праведнику, не смотря на свою правду, должно молитися: такъ и грѣшнику, ради грѣховъ сотворенныхъ, не должно оставлять молитвы, но всякому со смиреніемъ на едино милосердіе Божіе взирать. Какъ бо праведникъ отъ Бога, а не отъ себе, имѣетъ правду: такъ и грѣшнику должно тояжде правды отъ Бога молитвою, прошеніемъ, сокрушеніемъ сердца и истиннымъ покаяніемъ и вѣрою о Хрістѣ Іисусѣ искать. Фарисей, понеже на правду свою взиралъ въ молитвѣ, отверженъ: а мытарь, обремененный грѣхами, понеже едино милосердіе Божіе полагалъ предъ собою въ молитвѣ, услышанъ. \textit{И сниде сей оправданъ въ домъ свой паче онаго}, глаголетъ Хрістосъ\footnote{Лук.~18,~14.}.

\paragraph*{§\:243.} Какъ прилѣжно должно намъ молитися, Хрістосъ научаетъ насъ притчею о \textit{другѣ}, который въ полунощи пришелъ къ другу своему просить хлѣба\footnote{11,~5--8.}, "--- притчею о \textit{вдовицѣ}, которая неотступно приходила къ судіи нечестивому и просила отмщенія отъ соперника своего\footnote{18,~1--8.}. И паки: \textit{просите, и дастся вамъ; ищите, и обрящете; толцыте, и отверзется вамъ}\footnote{Матѳ.~7,~7.}. Должно бо не токмо просить, но искать; не токмо искать, но и толкать, то"=есть, неотступно просить и очи свои душевныя къ милосердію Божію возводить, по примѣру Псаломника: \textit{къ Тебѣ возведохъ очи мои, живущему на небеси. Се яко очи рабъ въ руку господій своихъ, яко очи рабыни въ руку госпожи своея, тако очи наши ко Господу Богу нашему, дондеже ущедритъ ны}\footnote{Пс.~122,~1 и 2.}.

\paragraph*{§\:244.} Молитва прилѣжная съ вѣрою все получаетъ, чему примѣръ подаетъ жена оная хананейская, во Евангеліи упоминаемая: \textit{Помилуй мя Господи, Сыне Давидовъ}, вопіетъ ко Хрісту: \textit{дщи моя злѣ бѣснуется}. Хрістосъ отвѣта ей не даетъ. Апостоли, приступивше, \textit{моляху Его глаголюще: отпусти ю, яко вопіетъ въ слѣдъ насъ}. Но Хрістосъ отвѣщаетъ: \textit{нѣсмь посланъ, токмо ко овцамъ погибшимъ дому Израилева}. Жена не отступаетъ, но большее усугубляетъ моленіе, кланяется, показуя смиреніе, и вопіетъ паки: \textit{Господи, помози ми}! Хрістосъ же \textit{отвѣщавъ рече: нѣсть добро отъяти хлѣба чадомъ и поврещи псомъ}. Жена и симъ, такъ жестокимъ отвѣтомъ не отходитъ; но, о псахъ услышавъ, за благо пріемля то и призная себе псомъ, паки сердечное желаніе и горячесть показуетъ: \textit{ей, Господи! ибо и пси ядятъ отъ крупицъ, падающихъ отъ трапезы господей своихъ}. Тогда видя жены постоянную и твердую вѣру и неотступное прошеніе, отвѣщавъ Іисусъ рече ей: \textit{о жено! велія вѣра твоя: буди тебѣ, якоже хощеши}\footnote{Матѳ.~15,~22--28.}. Вотъ что молитва прилѣжная, вѣрою подкрѣпляемая, получила: \textit{буди тебѣ, якоже хощеши}! И не токмо желаемое получила, но и похвалу вѣры своея услышала отъ Господа, сердца и утробы испытующаго. «Что же такъ не скоро учинилъ Хрістосъ по прошенію жены тоя? Хотѣлъ показать великую ея вѣру». Тако Златоустъ святый \textit{(въ Бесѣдѣ 38"~й на Быт.)} учитъ. "--- По подобію жены тоя, должно и намъ съ вѣрою толкать въ двери милосердія Божія, и вмѣнять себе аки псовъ, которые ядятъ отъ крупицъ, падающихъ отъ трапезы господей своихъ, да тако призритъ и на наше смиреніе Іисусъ Хрістосъ Господь нашъ, Который призрѣлъ на смиренную и вѣрную молитву жены хананейскія. Ибо \textit{Іисусъ Хрістосъ вчера и днесь, Тойжде и во вѣки}\footnote{Евр.~13,~8.}. Аще бо нынѣ и невидимъ есть нами, но \textit{Духомъ пребываетъ неотступно съ вѣрными Своими}\footnote{Матѳ.~28,~20.}, милуетъ съ вѣрою вопіющихъ къ Нему, и исполняетъ прошеніе просящихъ.

\paragraph*{§\:245.} Понеже не пользуетъ тая молитва, въ который языкъ молится, а умъ празденъ; языкъ глаголетъ, а умъ молчитъ; языкъ Бога призываетъ, а умъ разсѣвается по созданіямъ, какъ сказано: того ради должно съ помощію Божіею о томъ стараться, дабы и умъ тое помышлялъ и дѣлалъ, что языкъ глаголетъ, и словеса въ молитвѣ не иное что были бы, какъ толкованіе и свидѣтельство совѣта и желанія сердечнаго; \textit{отъ избытка} бы \textit{сердца уста глаголали}, а не токмо отъ памяти; и что сердце зачнетъ и породитъ, тое бы на устахъ являлось, по подобію воды въ котлѣ кипящей, которая на верхъ издаетъ пузыри, когда на днѣ котла горячѣетъ и кипитъ. Къ сему пособствуетъ: 1)~Тщаться о томъ, что Псаломникъ глаголетъ: \textit{предзрѣхъ Господа предо мною выну}\footnote{Пс.~16,~8.}, дабы хотя и всегда, но наипаче въ молитвѣ Бога предъ собою зрѣть, вѣрою предъ Нимъ съ раболѣпнымъ смиреніемъ стоять, Ему нужды свои предлагать, и съ сердечнымъ преклоненіемъ главу и колѣна преклонять, и тако ожидать милости отъ Него: якоже дѣлаемъ, егда предстоимъ земному монарху и отъ него милости просимъ. Такое устроеніе не попуститъ уму нашему разсѣваться въ молитвѣ. Того ради, когда хочемъ молитися, прежде начала молитвы помышлять и въ умѣ положить должны мы тое, что предъ Богомъ хочемъ стать и о своей нуждѣ Его просить. "--- 2)~Истинная молитва отъ размышленія бываетъ, того ради молитву съ размышленіемъ и начинать и творить должно. 3)~Понеже \textit{истиною и духомъ} молитися должно, и того сами отъ себе чинить не можемъ мы ради немощи нашей: того ради должно и о томъ молитися *Богу, чтобы научилъ насъ молитися* и призывати имя Его духомъ и истиною. \textit{О чесомъ бо помолимся, якоже подобаетъ, не вѣмы}, глаголетъ апостолъ\footnote{Римл.~8,~26.}. Должно и намъ просити Хріста: \textit{Господи, научи насъ молитися}\footnote{Лук.~11,~1.}.

\paragraph*{§\:246.} Молитися должно не токмо самому за себе, но 1)~другъ за друга; Ибо всѣ вѣрные, по всему міру разсѣянные, суть едино духовное тѣло, едину преблагословенную Главу имѣющіи "--- Хріста, и единымъ Духомъ Божіимъ просвѣщаемые и наставляемые\footnote{Римл.~12,~5; 1~Кор.~10,~17; 12,~12,~13,~20 и 27.}. Убо, какъ духовные уды, едино духовное тѣло составляющіе, должны другъ другу помогать молитвою. Якоже въ вещественномъ тѣлѣ члены другъ о другѣ пекутся, такъ и мы, другъ за друга молящеся, аки единъ гласъ къ небесному Отцу должны испущати съ вѣрою и любовію: \textit{Отче нашъ, Иже еси на небесѣхъ}! и проч. "--- 2)~Подданнымъ \textit{молитися за царя и за всѣхъ, иже во власти суть: да тихое и безмолвное житіе поживемъ во всякомъ благочестіи и чистотѣ}, якоже апостолъ учитъ\footnote{1~Тим.~2,~2.}. 3)~За пастырей, людямъ въ паству имъ порученнымъ, и пастырямъ за людей своихъ. 4)~Должно, по словеси Хрістову, \textit{молитися и за творящихъ намъ напасть и изгонящихъ насъ}\footnote{Матѳ.~5,~44.}.

\paragraph*{§\:247.} Молитвѣ препятствіе чинятъ слѣдующая: 1)~Грѣхъ содѣваемый, когда кто отъ грѣха отстать не хощетъ; напр., прелюбодѣяніе и всякая нечистота, хищеніе, мздоимство, пьянство и проч. \textit{Егда прострете руки ваша ко Мнѣ, отвращу очи Мои отъ васъ; и аще умножите моленіе, не услышу васъ: руки бо ваши исполнены крове}, глаголетъ Богъ чрезъ пророка Исаію\footnote{Ис.~1,~15.}. И паки: \textit{грѣси ваши разлучаютъ между вами и между Богомъ, и грѣхъ ради вашихъ отврати лице Свое отъ васъ, еже не помиловати}, и проч.\footnote{59,~2.} Апостолъ глаголетъ: \textit{да отступитъ отъ неправды всякъ именуяй имя Господне}\footnote{2~Тим.~2,~19.}. Сего ради должно оставить грѣхъ содѣваемый и примиритися съ Богомъ хотящему молитися, и тако съ надеждою приступати къ Богу. "--- 2)~Гнѣвъ и злоба, на ближняго въ сердцѣ питаемая. \textit{Егда стоите молящеся, отпущайте, аще что имате на кого: да и Отецъ вашъ, Иже есть на небесѣхъ, отпуститъ вамъ согрѣшенія ваша. Аще ли же вы не отпущаете, ни Отецъ вашъ, Иже есть на небесѣхъ, отпуститъ вамъ согрѣшеній вашихъ}, глаголетъ Хрістосъ\footnote{Марк.~11,~25 и 26.}. "--- 3)~Немилосердіе къ ближнему. \textit{Иже затыкаетъ ушеса своя, еже не послушати немощнаго, и той призоветъ, и не будетъ послушающаго его}, глаголетъ Писаніе\footnote{Притч.~21,~13.}. «Безъ милостыни безплодна есть молитва», глаголетъ Златоустъ\footnote{Бес.~6"~я на 2"~е посл. къ Тим.}. Убо должно самимъ намъ милость съ ближнимъ дѣлать, когда хощемъ милость у Бога получить. "--- 4)~Препятствуетъ обида ближнему сотворенная и ненагражденная. \textit{Аще принесеши даръ твой ко олтарю, и ту помянеши, яко братъ твой имать нѣчто на тя: остави ту даръ твой предъ олтаремъ, и шедъ прежде смирися съ братомъ твоимъ, и тогда пришедъ принеси даръ твой}, глаголетъ Хрістосъ\footnote{Матѳ.~5,~23 и 24.}. Убо должно съ обиженнымъ ближнимъ примиритися хотящему молитися. \textit{Единому молящуся, а другому проклинающу, коего гласъ услышитъ Владыка}? глаголетъ Писаніе\footnote{Сир.~34,~24.}. Ты молишься, а братъ твой, обиженный тобою, плачетъ; Хрістосъ же глаголетъ къ тебѣ: \textit{шедъ прежде смирися съ братомъ твоимъ}. Сего ради должно смиритися съ братомъ, и тако молитися.

\paragraph*{§\:248.} Здѣ прилично и нужно воспомянуть о беззаконныхъ и богопротивныхъ имене Божія призываніяхъ, которыя не ино что суть, какъ діавольскія козни и вымыслы, къ безчестію имене Божія и погибели человѣческой ухищренно изобрѣтенныя. 1)~Многимъ обычай есть во всякихъ разговорахъ о подлыхъ вещахъ и ко всякому почти слову приговаривать: \textit{ей Богу}! или, \textit{на то Богъ}! или, \textit{на то Хрістосъ}! или, \textit{свидѣтель Богъ}! или, \textit{видитъ Богъ}! и прочее. Страшное и съ глубокимъ смиреніемъ отъ самыхъ ангелъ почитаемое имя \textit{Богъ}, при воспоминаніи Котораго трепетать и главу съ сердцемъ преклонять должно! Но безстрашное сердце того не разсуждаетъ, и языкомъ злымъ, какъ подлой (что и говорить страшно) вещи, касается. Монарха земнаго, человѣка, имя почитается отъ своихъ подкомандныхъ, какъ и должно, и съ почтеніемъ въ нужныхъ случаяхъ поминается; а имя Божіе, великое и превознесенное, отъ хрістіанъ и такого почтенія не имѣетъ! Странное и жалости достойное дѣло! Такъ сатана ослѣпляетъ бѣднаго человѣка! Но что страшнѣе и горше, многіе въ шуткахъ, многіе для прельщенія и обмана ближняго, поминаютъ великое и страшное имя Божіе. О долготерпѣнія Твоего, преблагій Владыко нашъ! \textit{Благословенъ еси, Господи Боже отецъ нашихъ, и благословено имя славы Твоея святое и препѣтое и превозносимое во вѣки}\footnote{Дан.~3,~52.}! О, когда бы симъ людямъ открылися внутреннія очи, и увидѣли бы, что Богъ есть существо тое, у Котораго въ руцѣ весь свѣтъ, и они сами, и смерть и животъ ихъ: увидѣли бы, какъ въ бѣдственномъ и опасномъ состояніи находятся, и впредь опасалися бы поминать Божіе имя безъ почтенія и страха! "--- 2)~У многихъ обычай есть къ суду Божію отсылати обидѣвшаго: \textit{судитъ}"=де \textit{ему Богъ}! А за что? \textit{обидѣлъ}"=де \textit{меня онъ}. "--- Что говоришь, несмысленный человѣче, \textit{судитъ ему Богъ}? Ты говоришь: \textit{судитъ ему Богъ}; а Хрістосъ глаголетъ: \textit{молитеся за творящихъ вамъ напасть}. Такъ разсуди, какимъ ты духомъ здѣ имя Божіе призываешь и суда просишь у Бога на брата твоего? Онъ тебе обидѣлъ вчера, а ты его сегодня, а можетъ быть и прежде, какъ случается часто въ обществѣ живущимъ; ты на него суда просишь, а онъ на тебе. И что было бы, когда бы по желанію нашему дѣлалъ Богъ? Едва бы кто на свѣтѣ остался въ живыхъ; ибо много другъ другу и предъ Богомъ согрѣшаемъ. Такъ"=то несмысленно дѣлаемъ мы, когда другъ друга къ суду Божію позываемъ! "--- 3)~Сюды принадлежатъ, которые псалмы святые и прочія молитвы писанныя скоро и, какъ можетъ исправиться языкъ, читаютъ, что бываетъ и въ церковныхъ собраніяхъ отъ клириковъ. Всякъ можетъ разсудить, что сія молитва не молитва есть, но чтеніе словъ писанныхъ, или паче шумъ и звукъ, въ воздухъ ударяющій, и потому какъ чтецы, такъ и слышащіи чтенія таковыя безплодны бываютъ; и каковыми приходятъ, таковыми, и когда бы еще не горшими, отходятъ отъ молитвы. "--- 4)~Хуждше еще того дѣлаютъ, которые въ два голоса отправляютъ чтенія свои, когда иные поютъ, а другіе читаютъ вмѣстѣ, въ одномъ собраніи единъ тое, другій другое читаетъ вслухъ вмѣстѣ. Отъ таковаго чтенія и пѣнія не иное что послѣдуетъ, какъ токмо шумъ, которымъ храмы и слухи слышащихъ наполняются, "--- лѣнивымъ угожденіе, добрымъ печаль сердечная и воздыханіе, всѣмъ приходящимъ соблазнъ и вредъ. Что въ такомъ шумѣ понять можетъ пришедшій въ церковь человѣкъ? Что онъ будетъ слушать: пѣніе или чтеніе? Того, или другаго чтеца? Хотя два уха имѣемъ, но единъ слухъ, едину душу, которая слушаетъ; единъ умъ, который разсуждаетъ; едино сердце, которое чтеніемъ должно пользоватися. Но на два голоса раздѣленный слухъ ни того, ни другаго добрѣ слушать, понять и въ сердце допустить не можетъ, и тако стоитъ всякъ празденъ, ничимъ отъ чтенія таковаго не пользуется, какъ сіе всякъ можетъ видѣть. Жалости достойная вещь! Народъ простый, которые сами читать не знаютъ, приходятъ въ церковь ради того, чтобы послушать душеполезнаго чтенія и пѣнія, но обманываются бѣдные, лишаются желаемаго, и тако возвращаются въ домы свои. Всему тому причиною есть лѣность неисправныхъ поповъ и клириковъ, которые сами, закрывше глаза, спѣшно идутъ во адъ, и порученныхъ себѣ за собою тудыже влекутъ "--- 5)~Есть нѣкоторымъ обычай изъ святыхъ псалмовъ пѣть между бокалами и поздравленіями. Но коль неприлично, или паче богопротивно есть священную пѣснь къ плотоугодію присовокуплять, сами они видѣть могутъ. Псалмы святые суть дѣло Духа Святаго, и содержатъ высокія и великія тайны Божія, и преданы намъ къ душевной нашей пользѣ, которые должно пѣть трезвеннымъ и цѣломудреннымъ духомъ, сердцемъ и устами: піянство же есть діавольское дѣло, и многихъ соблазновъ виновно, которое отъ повтореній бокаловъ и поздравленій бываетъ. \textit{Кое убо общеніе свѣту ко тмѣ? Кое же согласіе Хрісту съ веліаромъ}\footnote{2~Кор.~6,~14 и 15.}? \textit{Грѣшнику бо рече Богъ: вскую ты повѣдаеши оправданія Моя, и воспріемлеши завѣтъ Мой усты твоими? Ты же возненавидѣлъ еси наказаніе, и отверглъ еси словеса Моя вспять}\footnote{Пс.~49,~16 и 17.}. Должно таковымъ опасаться, чтобы не быть подражателями Валтасару, царю вавилонскому, который приказалъ священные церковные сосуды на піянственный пиръ принести, и съ своими вельможами и наложницами пилъ ими\footnote{Дан.~5.}. Тамо сосуды были Богу посвященные: здѣ псалмы суть органы духовные, которыми пророкъ Божій, Духомъ Святымъ движимый, Божію хвалу воспѣлъ. Можно пѣть хрістіанамъ и при столѣ, и кромѣ стола, и вездѣ и на всякомъ мѣстѣ псалмы и прочія пѣсни Божіи; и долгъ хрістіанскій того требуетъ, чтобы вездѣ и на всякомъ мѣстѣ благость Божію хвалить и благодарить, "--- но съ трезвостію и цѣломудріемъ. Псалмы бо святые, какъ и прочее святое Писаніе, яко Божіе слово, требуютъ высокаго почтенія.

\subsection[Глава 10-я. О любви къ ближнему.]{глава десятая.\\\bfseries О любви къ ближнему.}

\begin{quotation}\textit{Возлюбиши искренняго твоего, яко самъ себе}\footnote{Матѳ.~22,~39.}.\end{quotation}
\begin{quotation}\textit{Якоже хощете, да творятъ вамъ человѣцы, и вы творите имъ такожде}\footnote{Лук.~6,~31.}.\end{quotation}
\begin{quotation}\textit{Заповѣдь новую даю вамъ, да любите другъ друга: якоже возлюбихъ вы, да и вы любите себе. О семъ разумѣютъ вси, яко Мои ученицы есте, аще любовь имате между собою}, глаголетъ Хрістосъ\footnote{Іоан.~13,~34.}.\end{quotation}
\begin{quotation}\textit{Любы долготерпитъ, милосердствуетъ; любы не завидитъ; любы не превозносится, не гордится, не безчинствуетъ, не ищетъ своихъ си, не раздражается, не мыслитъ зла; не радуется о неправдѣ, радуется же о истинѣ; вся любитъ, всему вѣру емлетъ, вся уповаетъ, вся терпитъ. Любы николиже отпадаетъ}\footnote{1~Кор.~13,~4--8.}.\end{quotation}


\paragraph*{§\:249.} Всякій человѣкъ другъ другу есть ближній: я тебѣ, а ты мнѣ ближній. Когда я помощи отъ тебе требую, то ты мнѣ ближній долженъ быть; а когда ты помощи отъ мене требуешь, то я тебѣ ближній долженъ быть. Тако богатый убогому, разумный невѣждѣ, молодый и здоровый старому, здоровый немощному, сильный беззаступному, свободный заключенному въ темницѣ, имѣющій домъ странному ближній долженъ быть. Тако впадшему въ разбойники и отъ нихъ уязвленному ближній былъ милосердый оный, во Евангеліи упоминаемый, самарянинъ, который \textit{сотворилъ милость съ нимъ, и приступивъ обвяза струпы его, возливая масло и вино}, и проч.\footnote{Лук.~10,~30--37.} Слѣдственно 1)~не родство, но нужда и бѣдность, не близость крове и плоти, но союзъ любве и милосердіе дѣлаетъ насъ ближними. И тако мы съ помощію нашею близкими должны быть, гдѣ бѣдность какая нибудь есть, "--- будетъ ли сродникъ нашъ по плоти или несродникъ, знаемый или незнаемый, любитель нашъ или нелюбитель, единоплеменникъ или иноплеменникъ, требующій помощи. Тако, когда сатана, яко разбойникъ, обнажилъ насъ и смертоносными беззаконій ранами уязвилъ, Хрістосъ сынъ Божій, видя нашу бѣдность, сотворился \textit{ближнимъ нашимъ}; сошедъ съ небесе, приступилъ къ намъ; Вышній, великій, всякія чести большій къ намъ нижнимъ, подлымъ, безчестнымъ, отверженнымъ, склонился, не смотря на то, что мы иноплеменники, самовольно отлучившіися отъ Него, враги Его были. Наша бѣдность и окаянство были причиною, Ему высокому смиритися, Богу человѣкомъ стать, Господу рабій зракъ принять, духу плотію быть, невидимому видимымъ, неосязаемому осязаемымъ, сильному немощнымъ быть, Господу славы обезчеститися, поругатися, Судіи живыхъ и мертвыхъ судитися, праведному со беззаконными вмѣнитися, безсмертному умрети. Тако \textit{ближнимъ нашимъ} сдѣлавшися, обвязалъ струпы наши, возливая масло и вино, и такимъ образомъ исцѣлилъ насъ, и отходя на небо, предалъ насъ гостинникамъ "--- апостоламъ и преемникамъ ихъ пастырямъ, дабы прилѣжали намъ. Тако научилъ преблагій Господь нашъ насъ, рабовъ Своихъ, къ подобнымъ намъ быть ближними. "--- 2)~Имени сего "--- \textit{ближній} "--- отрекаются тѣ, которые, видя бѣдность человѣка, хладно, или паче безстыдно говорятъ: \textit{что мнѣ до его нужды}? И тако мимоидутъ, никакой не дѣлая ему помощи, какъ учинили оные упоминаемые священникъ и левитъ, которые впадшему въ разбойники и уязвленному, не подавше руки помощи, мимоидоша. "--- 3)~Не токмо имени сего "--- \textit{ближній} "--- отрекаются, но и уподобляются разбойникамъ тѣ, которые не токмо не помогаютъ бѣднымъ, но и безчеловѣчно ихъ послѣдняго лишаютъ, и тако къ слезамъ слезы, къ бѣдности бѣдствіе придаютъ. Таковые суть вси грабители, насильники, мздоимцы, судіи, хищники, господа, крестьянъ своихъ излишне обременяющіи, удерживающіи мзду наемничу, съ пожара похищающіи и прочіе симъ подобные.

\paragraph*{§\:250.} Плоды любве хрістіанскія: 1)~\textit{Терпѣніе}. Кто любитъ ближняго, тотъ не отмщеваетъ ему за обиду нанесенную, но великодушно сноситъ; и не токмо не отмщеваетъ и сноситъ, но и молится за обидѣвшаго, причитая тую обиду главной причинѣ "--- общему врагу діаволу, который подстрекаетъ насъ другъ друга обижать, "--- а о человѣкѣ соболѣзнуетъ, видя его неисправность. Въ семъ онъ подражаетъ Хрісту молящемуся: \textit{Отче! отпусти имъ: не вѣдятъ бо что творятъ}\footnote{Лук.~23,~34.}. Къ сему апостолъ увѣщаваетъ: \textit{не побѣжденъ бывай отъ зла, но побѣждай благимъ злое}\footnote{Римл.~12,~21.}. "--- 2)~\textit{Милосердіе}. Любовь, видя неблагополучіе ближняго, соболѣзнуетъ ему и аки свое вмѣняетъ, страждущему состраждетъ, съ бѣдствующимъ бѣдствуетъ и тщится помощи бѣдствію его, не щадитъ себе, чтобы ближняго бѣдствію пособить, и такъ бѣдствіе его, а свое благополучіе, съ нимъ пополамъ дѣлитъ. Такъ дѣлаютъ тѣ, которые убогимъ свое истощаютъ богатство, нищихъ награждаютъ, отъ себе отнимая, и такъ своего временнаго благополучія убавляютъ, а тѣмъ у бѣдныхъ бѣдствія убавляютъ. Таковыхъ ублажаетъ Хрістосъ: \textit{блажени милостивіи, яко тіи помиловани будутъ}; и къ сему всѣхъ насъ увѣщаваетъ: \textit{будите милосерди, якоже Отецъ вашъ небесный милосердъ есть}\footnote{Матѳ.~5,~7 и 48.}. "--- 3)~\textit{Любовь не завидитъ}. Зависти дѣло есть о благополучіи ближняго скорбѣти и о радости унывати. Въ любви язва сія душевная не имѣетъ мѣста: она счастіе ближняго и несчастіе за свое вмѣняетъ, и потому какъ о несчастіи ближняго болѣзнуетъ, такъ о счастіи радуется; съ плачущими плачетъ, съ радующимися радуется. Тако \textit{радоватися съ радующимися и плакати съ плачущими} увѣщаваетъ апостолъ Павелъ\footnote{Римл.~12,~15.}. "--- 4)~\textit{Любовь не превозносится, ни гордится}. Гордости дѣло есть ближняго презирать, уничтожать, себе возвышать. Любы не тако: она себе уничтожаетъ, другихъ выше себе ставитъ, всѣхъ почитаетъ, предъ всѣми смиряется, высшимъ покорна и послушна, равнымъ учтива и благопріятна, низшимъ снисходительна и сообщительна; прежде себе, нежели другихъ, осуждаетъ, себе укоряетъ, а не другихъ; всякому мѣсто уступаетъ. Къ сему апостолъ насъ поощряетъ: \textit{другъ друга честію больша себе творите}\footnote{Филип.~2,~3.}. "--- 5)~\textit{Любы не безчинствуетъ}; но и срама не боится ради любимаго. «Не вѣсть любовь, глаголетъ святый Златоустъ, что когда есть срамъ»\footnote{Бес.~33"~я на 1"~е посл. къ Кор.}. Гдѣ прочіимъ стыдъ, тамо ей стыда нѣтъ; гдѣ прочіе гнушаются, тамо она не гнушается; гдѣ прочіе отвращаются и убѣгаютъ, тамо она приступаетъ и пристаетъ. Въ семъ дѣлѣ она уподобляется слѣпому, который думаетъ, что какъ онъ самъ не видитъ, такъ и другіе его не видятъ: такъ она помышляетъ, что какъ ей, такъ и прочіимъ нѣтъ тамо стыда и срама, гдѣ нужда и бѣдность ближняго помощи требуетъ. Тако она не стыдится одѣяннаго въ рубище, хотя сама порфирою и виссономъ украшена; тако не стыдится лежащему на гноищѣ склоняться, хотя сама высокою честію почтенна; тако не стыдится въ темницу смрадную войтить, хотя сама въ чертогахъ обитаетъ; не стыдится въ домъ свой ввести страннаго и упокоить, хотя бы нищій и ранами смердѣлъ; не стыдится утѣшать печальнаго, хотя бы онъ и весьма подлый былъ: она тамо отлагаетъ преимущество титула своего, гдѣ нужда бѣднаго требуетъ. Тако Хрістосъ, Сынъ Божій, сокрылъ величество Божества Своего подъ плотію человѣческою, чтобы тако удобнѣе моглъ бѣдствію нашему помощи; не устыдѣлся нищеты нашея, чтобы насъ обогатить; не устыдѣлся въ дѣвическомъ чревѣ обитати, \textit{имѣяй престолъ небо и подножіе землю}; не устыдѣлся пеленами младенческими повитися и въ убогомъ вертепѣ положитися, \textit{одѣяйся свѣтомъ яко ризою}; не устыдѣлся въ рабіемъ образѣ и съ рабами на земли пожити, \textit{Господь и Царь небесе}; не постыдѣлся съ мытарями ясти, съ блудницами бесѣдовати и съ прочіими грѣшниками обращатися, \textit{безконечная святыня и правда}; не постыдѣлся на судѣ стоять и судимъ быти, \textit{Судія всѣхъ}; не постыдѣлся безчестіе, поруганіе, заплеваніе пріяти, обнажитися, между разбойниками повѣшенъ быти, \textit{Которому серафими со страхомъ и трепетомъ предстоятъ}. А такъ глубоко смиритися не ино что Его убѣдило, какъ любовь Его къ намъ, не постижимая нами, чтобы тако насъ отъ бѣды избавить. Такимъ образомъ и насъ научилъ не стыдитися тамо, гдѣ ближняго нужда требуетъ отъ насъ помощи. На сей любве образъ должны смотрѣть начальники, и съ подлыми своими подкомандными обходитися любовно, не гнушаться ими, слушать предложенія ихъ, исполнять и помогать нуждамъ ихъ; пастыри "--- не стыдиться ходить въ слѣдъ погибшихъ и погибающихъ словесныхъ овецъ, и искать ихъ; богатые и славные "--- не стыдиться въ домы свои вводить убогихъ, учреждать и упокоевать ихъ: господа "--- не стыдиться рабовъ своихъ братіею называть; вси хрістіане "--- не стыдиться посѣщать сѣдящихъ въ темницѣ, окованныхъ кандалами, посѣщать лежащихъ на одрѣ болѣзни, помогать немощнымъ и престарѣлымъ и прочіимъ бѣднымъ. "--- 6)~\textit{Любовь не ищетъ своихъ}. Истинная любовь тщится съ радостію и веселіемъ любимому благотворить, и благотворить безъ всякія своея пользы. Она въ семъ уподобляется плодовитому древу, которое плодами своими не себе, но другихъ питаетъ, уподобляется землѣ, которая не ради себе, но ради насъ плоды прозябаетъ; уподобляется солнцу, которое не себѣ, но намъ свѣтитъ и насъ согрѣваетъ; или паче послѣдуетъ вѣчной оной и несозданной любви и благости, которая вся благая намъ подаетъ безъ всякія Своея корысти. Тако Моисей святый \textit{отвержеся нарицатися сынъ дщере Фараоновы; паче же изволи страдати съ людьми Божіими, нежели имѣти временную грѣха сладость}\footnote{Евр.~11,~24 и 25.}. Тако Павелъ святый о себѣ написалъ: \textit{сего ради вся терплю избранныхъ ради, да и тіи спасеніе улучатъ, еже о Хрістѣ Іисусѣ, со славою вѣчною}\footnote{Тим.~2,~10.}. Тако благочестивые царіе не себѣ, но отечеству своему царствуютъ; тако добрые судіи не себѣ, но людямъ судятъ; тако хрістолюбивые пастыри не ради себе, но ради овецъ Хрістовыхъ митрою покрываются и жезлъ въ рукахъ носятъ; тако боголюбивые воины не ради себе, но ради вѣры святой и отечества противу врага подвизаются: тако \textit{любовь не ищетъ своихъ}, но ближняго. Къ сей должности увѣщаваетъ Павелъ: \textit{никтоже своего си да ищетъ, но еже ближняго кійждо}\footnote{1~Кор.~10,~24.}. "--- 7)~Истинная любовь \textit{не раздражается}, ни гнѣвается на ближняго, хотя отъ него и обиду пріиметъ. Прочіе обиду за обиду воздать и злословіе за злословіе учинить тщатся: она не токмо того не дѣлаетъ, но и въ сердцѣ гнѣва не имѣетъ на обидѣвшаго\footnote{Златоустъ на сіе мѣсто.}. И такъ не токмо не дѣлаетъ, но \textit{ниже мыслитъ зла}. И хотя иногда и являетъ гнѣвъ свой, но гнѣвъ тотъ стремится на грѣхи, а не на человѣка; грѣхи гонитъ и тщится искоренить, а не согрѣшившихъ: каковъ гнѣвъ бываетъ наипаче отъ благочестивыхъ начальниковъ и пастырей. Таковъ гнѣвъ есть праведный, и великую любовь въ сердцѣ гнѣвающагося тако являетъ, которая всякимъ образомъ ищетъ спасенія братняго. Таковые подражаютъ доброму и искусному лѣкарю, который иногда даетъ жестокое лѣкарство немощному, чтобы тако удобнѣе выгнать изъ него немощь. Таковый гнѣвъ показалъ Павелъ святый, любовію къ Богу и ближнему горящая душа, когда тако къ Галатамъ согрѣшившимъ написалъ: \textit{о несмысленніи Галате! кто васъ прельстилъ есть не покоритися истинѣ}\footnote{Гал.~3,~1 и проч.}? Таковый гнѣвъ пастырямъ и начальникамъ есть нужный, которымъ злость и злонравіе подчиненныхъ, какъ моровую язву огнемъ, прогонять и искоренять должно. Ихъ дѣло есть собственную свою обиду кротко претерпѣвать: а когда законъ Божій нарушается и ближнему обида наносится, за тое крѣпко стоять, не молчать, и насильниковъ усмирять. "--- 8)~\textit{Любовь не радуется}, но болѣзнуетъ, \textit{о неправдѣ}, когда ближнему обида дѣлается. Понеже она ближняго благополучіе и злополучіе за свое вмѣняетъ, то и обидою, ближнему наносимою, не менѣе, какъ своимъ неблагополучіемъ, трогается. Тако болѣзнуетъ другъ, когда другъ его неправду терпитъ; болѣзнуетъ мать и отецъ чадолюбивый, когда сынъ ихъ насиліе страждетъ. «Любовь, глаголетъ святый Златоустъ на сіе слово, не услаждается злостраждущими». "--- 9)~\textit{Любовь радуется о истинѣ}, когда все благочинно и въ благомъ состояніи видитъ. Тако радуются ангели святіи и вси угодники Божіи, когда грѣшники каются о грѣхахъ своихъ, и добрая дѣла творятъ\footnote{Лук.~15,~10.}. "--- 10)~\textit{Любовь вся любитъ}, добрыхъ и злыхъ не выключаетъ отъ благоволенія и благотворенія своего. Въ чемъ подражаетъ она небесному Отцу, \textit{Иже солнце Свое сіяетъ на злыя и благія, и дождитъ на праведныя, и на неправедныя}\footnote{Матѳ.~5,~45.}; подобится солнцу, которое и хулящія и хвалящія его освѣщаетъ и согрѣваетъ; подобится плодовитому древу, которое и хозяина своего и чуждаго питаетъ плодами своими; подобится источнику, который и засоряющихъ и вычищающихъ его напаяетъ; подобится скоту, который и кормящихъ его и біющихъ возитъ; подобится землѣ, которая и дѣлающимъ ее и плюющимъ на ню подаетъ плоды своя. Таковъ нравъ любве есть. Она не смотритъ на лица, не разбираетъ чина и родства, близости и дальности, пріязни и непріязни; не спрашиваетъ, кто онъ есть, братъ или не братъ, сродникъ или несродникъ, одноземецъ или иноземецъ, пріятель или непріятель, добрый или злый, кто требуетъ любве плода; тому являетъ дѣйствіе теплоты своея, кто хощетъ и требуетъ того; тотъ ей и сродникъ, кто бѣденъ; тамо она близко приступаетъ, гдѣ нужда. Бѣдствіе и нужда человѣческая, какъ родство ей есть, и ее къ себѣ привлекаетъ. Такъ учинилъ упомянутый оный милостивый самарянинъ. "--- 11)~\textit{Любовь всему вѣру емлетъ}. Какъ сама простосердечна, такъ и о прочіихъ мнитъ; и какъ сама никого не обманываетъ, такъ и о прочіихъ думаетъ и потому всякому вѣруетъ. Любовь бо есть чистосердечна, не лукава, не лестна, и ради того, какова сама, тако и о прочіихъ надѣется и подозрѣнія не имѣетъ. Отъ сего"=то бываетъ, что добрые часто отъ лукавыхъ обманываются и много бѣдъ претерпѣваютъ, понеже со всѣми чистосердечно обходятся. Сихъ людей нынѣшній свѣтъ дураками безумно называетъ, потому что съ нимъ лукавновать не знаютъ. "--- 12)~\textit{Любовь вся уповаетъ, вся терпитъ}. О чемъ тако святый Златоустъ: «Что есть "--- \textit{вся уповаетъ}? Всѣхъ благихъ, рече, не отчаевается любимаго; но, аще и золъ будетъ, пребываетъ исправляющи, промышляющи, прилѣжащи; аще и не сбудутся по упованію благая сія, но и еще тяжчайшій онъ будетъ, и сія терпитъ»\footnote{Бес.~33"~я на 1"~е посл. къ Кор.}. "--- 12)~\textit{Любы николиже отпадаетъ}. «Вѣра и надежда, глаголетъ Златоустъ святый, когда вѣруемая и надѣемая благая пріидутъ, престаетъ: любы же тогда наипаче возгарается и бываетъ зѣльнѣйшая»\footnote{Бес.~34"~я.}. Нынѣ вѣрные вѣруютъ, надѣются и любятъ, но въ будущемъ вѣцѣ только будутъ любить. Увидятъ бо, что вѣруютъ; получатъ, чего надѣются, и для того вѣра и надежда престанетъ, но любовь во вѣки непрестанетъ. Ибо Бога будутъ видѣть во вѣки, Котораго будутъ любить; и другъ съ другомъ совокупно будутъ во вѣки, и такъ другъ друга будутъ любить. Паче же тогда совершится любовь, ибо \textit{лицемъ къ лицу увидятъ Бога}\footnote{1~Кор.~13,~12.}, Котораго нынѣ вѣрою видятъ; и увидятъ другъ въ другѣ совершенную взаимную любовь, другъ въ другѣ совершенный образъ Божій сіяющій и совершенное блаженство, о которомъ будутъ радоватися. "--- Сими плодами Бога и ближняго любящая душа изобилуетъ! Сею утварію, аки дщерь царская, украшается! Въ себѣ смиренна, но отъ Бога превознесенна, внѣ убога, но внутрь богата; внѣ подла, но внутрь благородна; внѣ презрѣнна, но внутрь почтенна; отъ міра умаленна, но отъ Бога возвеличена!

\paragraph*{§\:251.} Образъ любве къ ближнему объявляетъ святое Писаніе: \textit{возлюбиши искренняго твоего яко самъ себе}. Какъ любиши себе, такъ и ближняго твоего люби. Какъ не хощешь, чтобы кто тебе обидѣлъ, отнялъ что у тебе, оклеветалъ, оболгалъ, прельстилъ, обманулъ, опорочилъ, похитилъ что, осудилъ тебе, укорилъ и прочее зло сдѣлалъ, такъ не желай и не твори того и ближнему. Какъ не хощешь, чтобы кто тебе въ нуждѣ оставилъ, такъ не оставляй и искренняго твоего. Какъ хощешь, чтобы кто тебѣ въ бѣдности помоглъ, такъ желай и помогай самъ бѣдствующему. Какъ хощешь, чтобы кто тебе алчущаго напиталъ, жаждущаго напоилъ, нагаго одѣлъ, страннаго въ домъ ввелъ и упокоилъ, въ темницѣ сѣдящаго и болящаго посѣтилъ, печальнаго утѣшилъ, сумнящагося вразумилъ, незнающаго научилъ, заключеннаго свободилъ, плѣннаго искупилъ, немощному послужилъ и прочія дѣла любве дѣлалъ, такъ желай и твори самъ ближнему. \textit{Якоже хощете, да творятъ вамъ человѣцы и вы творите имъ такожде}, глаголетъ Хрістосъ. А богословъ святый, любимый Хрістомъ и любитель Хріста и ближняго своего, еще придаетъ къ любви сей: \textit{Онъ} (Хрістосъ) \textit{по насъ душу Свою положилъ: и мы должни есмы по братіи души полагати}\footnote{1~Іоан.~3,~16.}. Хотя всѣмъ, которые хотятъ Хрісту, такъ насъ возлюбившему, подражать, должность сія предлежитъ; однакожъ наипаче тѣмъ, которымъ поручилъ пасти словесныхъ Своихъ овецъ стадо. Они тогда наипаче съ жезломъ своимъ выходить и предъ овцами своими стать должны, когда видятъ находящихъ волковъ и хотящихъ расхитити и распудити стадо Хрістово, и тако отгонять или поражать ихъ, или души своя полагать, а не бѣжать, якоже наемники дѣлаютъ\footnote{Іоан.~10,~1--16.}.

\paragraph*{§\:252.} Причины, поощряющія насъ къ любви взаимной: 1)~Вси есмы братія между собою. Отъ единаго Бога созданы, вѣрные и невѣрные, добрые и злые. Единаго Господа на небесѣхъ имѣемъ Бога, Который есть Царь небесе и земли, Который всѣми добрыми и злыми, мертвыми и живыми, обладаетъ. Единаго имѣемъ Промыслителя Бога, Который всѣхъ питаетъ, одѣваетъ и сохраняетъ; \textit{солнце Свое сіяетъ на злыя и благія, и дождитъ на праведныя и на неправедныя}. Единаго Отца имѣемъ Бога, Который \textit{тако возлюбилъ насъ, яко и Сына Своего единороднаго далъ есть, да всякъ, вѣруяй въ Онь, не погибнетъ, но имать животъ вѣчный}\footnote{3,~16.}. Единаго имѣемъ Законоположника Бога, Который велитъ намъ другъ друга любить, мнѣ тебе, а тебѣ мене: \textit{возлюбиши искренняго твоего яко самъ себе}. Единаго наконецъ имѣемъ Судію тогожде Бога, Который всѣмъ будетъ судить и воздастъ всѣмъ по дѣломъ. Сей убо союзъ единости да увѣщаваетъ насъ къ любви взаимной. "--- 2)~Единаго родителя по плоти признаемъ Адама, отъ котораго вси начало свое и поколѣніе ведемъ; единымъ образомъ вси раждаемся; едино естество вси имѣемъ; единою душею вси оживляемся; единою плотію обложени есмы; вси единымъ и тѣмжде немощамъ подлежимъ и другъ отъ друга взаимнаго вспоможенія требуемъ. Сіе убо братство убѣждаетъ насъ другъ друга любить. "--- 3)~Въ единомъ государствѣ живущіи еще большее братство имѣютъ, и потому важнѣйшую причину къ любви взаимной. Въ единомъ отечествѣ, какъ въ лонѣ матернемъ, почиваютъ, упокоеваются и сохраняются; единаго на землѣ монарха, какъ уды главу, управляющаго, повелѣвающаго, промышляющаго признаютъ; единымъ законамъ подчинены, и ими, какъ союзомъ какимъ, связаны; едино общество, яко части цѣлость тѣла, составляютъ; едино имя отъ единости отечества имѣютъ, напр. въ нашемъ отечествѣ живущіи \textit{Россіане}, въ Италіи "--- \textit{Италіане}, въ Греціи "--- \textit{Греки}, такъ и о прочіихъ. Потому цѣлость общества и союзъ состава его союзомъ любве соблюдать должны, безъ котораго союза оно паденію подлежитъ. "--- 4)~Хрістіане, кромѣ вышеписанныхъ причинъ, еще важнѣйшія имѣютъ къ любви взаимной. Единаго Отца имѣютъ Бога, отъ Котораго духовно рождены, въ Котораго вѣруютъ, Котораго призываютъ, благодарятъ, славятъ и поютъ; единаго Хріста, Сына Божія, избавителя и Спасителя своего, надежду и упованіе свое, "--- единаго Духа Святаго просвѣтителя, наставника, оживотворителя и освятителя имѣютъ; единаго Тріѵпостаснаго Бога помощника, защитителя, сохранителя и промыслителя почитаютъ. Единымъ Тайнамъ причащаются; къ единой таинственной трапезѣ тѣла и крови Хрістовой приступаютъ; единаго Божія слова слушаютъ; къ единому вѣчному животу на велію оную вечерю позваны. Въ единой церкви святой, какъ въ домѣ Божіемъ, вѣрномъ Сіонѣ и градѣ Царя небеснаго, живутъ. Едину главу, какъ духовные уды, Хріста признаютъ, и отъ Нея управляются. Видиши ли, любезный хрістіанине, коль крѣпко мы другъ съ другомъ связаны. А отъ сего признать должно, коль крѣпокъ союзъ любве между нами долженъ быть, "--- такъ крѣпокъ, какъ въ естественномъ и вещественномъ тѣлѣ между членами. "--- 5)~Апостолъ глаголетъ: \textit{пребываяй въ любви, въ Бозѣ пребываетъ}\footnote{1~Іоан.~4,~16.}. О коль великое сіе дѣло есть "--- \textit{въ Бозѣ пребывати}, "--- что истинная хрістіанская любовь получаетъ. "--- 6)~Кто любитъ нелицемѣрно ближняго, тотъ истинный Хрістовъ ученикъ есть, какъ Самъ Хрістосъ свидѣтельствуетъ: \textit{о семъ разумѣютъ вси, яко Мои ученицы есте, аще любовь имате между собою}\footnote{Іоан.~13,~35.}. \textit{Истинный хрістіанинъ} есть, кто истинно, а не ложно Хрістово имя носитъ, кто ближняго любитъ. Коль же великое есть достоинство истиннымъ быть хрістіаниномъ!.. "--- 7)~Кто любитъ ближняго, тотъ во свѣтѣ пребываетъ, какъ глаголетъ апостолъ: \textit{любяй брата своего, во свѣтѣ пребываетъ}\footnote{1~Іоан.~2,~10.}. Свѣтъ же здѣ разумѣется душевный, а не тѣлесный. Вси бо въ свѣтѣ тѣлесномъ, то"=есть чувственномъ, пребываютъ, "--- вси, глаголю, то"=есть, праведные и грѣшные, любящіи и нелюбящіи; но въ душевномъ свѣтѣ едины любящіи пребываютъ, едины любящіи того наслаждаются. Какъ бо \textit{ненавидящихъ тма} грѣховная \textit{ослѣпляетъ очи} душевныя, \textit{и не вѣдятъ, камо идутъ}, не вѣдятъ, яко въ ровъ погибели имѣютъ впасти\footnote{ст.~11.}: такъ хрістіанскую имѣющіи любовь свѣтомъ Божіей благодати просвѣщаются, и тако все предусматриваютъ и остерегаются. Въ семъ свѣтѣ они зрятъ Свѣтъ вѣчный и незаходимый, Котораго нынѣ отчасти наслаждаются, но тогда \textit{узрятъ Его, якоже есть}\footnote{1~Іоан.~2,~2.}, \textit{егда еже отчасти упразднится}\footnote{1~Кор.~13,~10.}. "--- 8)~Истинная хрістіанская любовь есть предвкушеніе вѣчнаго живота, въ которомъ едина только другъ къ другу любовь будетъ, едина другъ о другѣ радость и веселіе. "--- 9)~Прочія дарованія, какъ"=то: языки глаголати, чудеса творити и прочая, безъ любви не пользуютъ намъ, какъ апостолъ учитъ: \textit{аще языки человѣческими глаголю и ангельскими, любве же не имамъ, быхъ яко мѣдь звѣнящи или кѵмвалъ звяцаяй. И аще имамъ пророчество, и вѣмъ тайны вся, и весь разумъ, и аще имамъ всю вѣру, яко и горы преставляти, любве же не имамъ, ничтоже есмь}\footnote{ст.~1,~2 и проч.}. И Хрістосъ глаголетъ: \textit{мнози рекутъ Мнѣ въ день онъ: Господи, Господи! не въ Твое ли имя пророчествовахомъ, и Твоимъ именемъ бѣсы изгонихомъ, и Твоимъ именемъ силы многи сотворихомъ? И тогда исповѣмъ имъ: яко николиже знахъ васъ; отъидите отъ Мене дѣлающіи беззаконіе}\footnote{Матѳ.~7,~22 и 23.}. И Златоустъ святый глаголетъ: «безъ любви всѣ дарованія не пользуютъ тѣмъ, которые имѣютъ ихъ»\footnote{Бес.~32"~я на 1~посл. къ Кор.}. "--- 10)~«Любовь болѣе всѣхъ жертвъ и приношеній, и никакой жертвы не пріемлетъ Богъ безъ любви», глаголетъ тойжде учитель святый\footnote{Бес.~10"~я на Матѳ.}. \textit{Шедше}, глаголетъ Хрістосъ, \textit{научитеся, что есть: милости хощу, а не жертвы}\footnote{Матѳ.~9,~13.}. Такъ высоко почитаетъ Богъ любовь къ ближнему, что и жертвы безъ любви не хощетъ: \textit{милости хощу, а не жертвы}. Милость бо есть плодъ любве. Что пользуетъ молитва безъ любви? что славословіе и пѣніе? что созиданіе и украшеніе храмовъ? что умерщвленіе плоти, когда ближній не любится? Весьма ничтоже!.. Къ такимъ Хрістосъ съ высоты отвѣщаетъ: \textit{шедше научитеся, что есть: милости хощу, а не жертвы}. "--- 11)~Не любитъ тотъ и Бога, кто ближняго не любитъ. \textit{Аще кто речетъ, яко люблю Бога, а брата своего ненавидитъ, ложь есть}, глаголетъ апостолъ Іоаннъ\footnote{1~Іоан.~4,~20.}. Ближній бо нашъ представленъ есть намъ ко испытанію и искушенію любве Божія. Аще ближняго любимъ, то и Бога любимъ: аще ближняго ненавидимъ, то и Бога не любимъ. "--- 12)~\textit{Ненавидяй брата своего, человѣкоубійца есть}, глаголетъ апостолъ святый Іоаннъ\footnote{3,~15.}. Отъ ненависти бо послѣдуетъ убійство. Какъ бо любители ближнихъ сохраняютъ ихъ животъ, когда не токмо не вредятъ имъ, но и отъ вреда предостерегаютъ ихъ, и въ нуждахъ помогаютъ имъ: такъ ненавистники противнымъ образомъ или явный вредъ наносятъ имъ, или отнимаютъ нужное, чѣмъ животъ сохраняется, или не помогаютъ въ бѣдствіи, которымъ животъ отнимается, или опечаляютъ ихъ, что животу вредительно, и тако животъ ихъ сокращаютъ. Не токмо бо тотъ убійца, который руками, мечемъ, или какимъ другимъ оружіемъ убиваетъ, но и который путь къ смерти стелетъ, или не избавляетъ отъ смерти, когда можетъ. Не помогаешь утопающему въ водѣ, когда можешь: убійца еси. Не избавляешь отъ рукъ убійцъ брата твоего, когда можешь: убійца еси. Не пущаешь въ домъ твой отъ мраза трясущагося: убійца еси. Презираешь уязвленнаго и лежащаго на пути: убійца еси. Отнимаешь у ближняго твоего одежду, которою одѣвается; отнимаешь пищу, которою насыщается: убійца еси. Не питаешь гладомъ погибающаго: убійца еси. Не одѣваешь отъ хлада трясущагося: убійца еси. Опечаляешь брата своего злобою своею: убійца еси. Клевещешь и злословишь ближняго твоего: какъ мечемъ, уязвляешь его языкомъ своимъ. Слыши, что Псаломникъ поетъ: \textit{сынове человѣчестіи, зубы ихъ оружія и стрѣлы, и языкъ ихъ мечь остръ}\footnote{Пс.~56,~5.}. Словомъ сказать, отнимаешь у ближняго нужныя къ житію: убиваешь его. Не помогаешь бѣдствующему, а можешь: убиваешь брата твоего, ибо не отнимаешь способа, чрезъ который смерть приходитъ. Ты не помогаешь ему, другой того отрицается, и тако безъ помощи оставшійся братъ погибаетъ. Ты и другій и третій, не помогши ему въ бѣдствіи его, погибели его виновны, ибо могли отвратить погибель его помощію своею, но не хотѣли; не хотѣли, понеже любви не имѣли. "--- 13)~\textit{Не любяй брата, пребываетъ въ смерти}, глаголетъ апостолъ\footnote{1~Іоан.~3,~14.}. Таковый хотя тѣломъ и живетъ, но душею мертвъ есть. Ибо какъ тѣло душею, такъ душа Духомъ Хрістовымъ оживляется. А гдѣ нѣтъ хрістіанской братской любви, тамо нѣтъ Духа Хрістова. \textit{Яко всякъ, не творяй правды, нѣсть отъ Бога, и не любяй брата своего}\footnote{ст.~10.}. Но тамо, вмѣсто того, духъ непріязни: ибо душа или находится въ благодати Божіей, и благодатію оживляется; или не имѣетъ благодати, и лишается жизни духовныя. Едино бо изъ сихъ двухъ непремѣнно послѣдуетъ: гдѣ нѣтъ живота духовнаго, тамо смерть *духовная, какъ гдѣ нѣтъ живота тѣлеснаго, тамо смерть тѣлесная*. Смерти духовной послѣдуетъ смерть вѣчная, когда душа покаяніемъ истиннымъ не воскреснетъ. "--- 14)~Общее благополучіе отъ взаимной любви процвѣтаетъ. О, коль благополучное имѣли бы житіе вси, ежели бы взаимно другъ друга любили! Не было бы тогда браней, и такъ ужаснаго и воистину плача достойнаго кровопролитія; не было бы такъ много въ едино малое время плачущихъ вдовъ по своихъ мужахъ, сиротъ по своихъ отцахъ, отцевъ и матерей по своихъ дѣтяхъ, которыхъ толь много въ краткое время оружіе бранное поражаетъ: ссора бо и брань есть знаменіе разорванной любви. Не было бы разбоевъ, воровства, хищенія, убійства, насилія, грабленія, лукавства, льсти, укоренія, злословія, поношенія, ругательства и прочіихъ смертоносныхъ хрістіанства язвъ, отъ которыхъ о коль многіе страждутъ, плачутъ и безвременно живота лишаются! Вся бо сія суть извѣстнѣйшіе знаки далеко прогнанной любве. Не укрѣпляемы бы были клѣти, лавки, житницы; не потребны бы были сторожи, запоры, замки; не было бы плачущихъ и кровавыя слезы проливающихъ, насилованныхъ людей; не слышалися бы жалобные гласы вдовъ, сиротъ и прочіихъ беззаступныхъ на небо вопіющихъ; не скиталися бы по улицамъ и стогнамъ алчущая братія; не тряслися бы отъ хлада и мраза полунагіе уды Хрістовы; не были бы наполнены темницы за долги, недоимки и вексели сѣдящими узниками: ибо любовь не допустила бы до сихъ и подобныхъ бѣдствій. Не было бы нища и убога: ибо любовь требуетъ, чтобы все общее было всѣмъ, нищему богатство богатаго, а богатому недостатки нищаго, и тако изобиліемъ богатаго недостатки нищаго исполнены бы были. Откуду пишется въ Дѣяніяхъ Апостольскихъ, что въ то время въ хрістіанехъ \textit{не бяше нищъ ни единъ}: понеже, какъ таможде пишется, \textit{народу вѣровавшему бѣ сердце и душа едина; и ни единъ же что отъ имѣній своихъ глаголаше свое быти, но бяху имъ вся обща}\footnote{Дѣян.~4,~34 и 32.}. Сказать невозможно, коль великое благополучіе любовь приноситъ во всякомъ чинѣ, между начальникомъ и подначальными, между родителями и дѣтьми; между мужемъ и женою, между братіею, сестрами, между рабами и господами; въ сосѣдствѣ между гражданами, между поселянами, между мастеровыми людьми, между воинами; въ духовенствѣ между іереями и клириками и прочіими. Безъ любви нѣтъ нигдѣ радости и утѣхи: гдѣ любовь, тамо всегдашній духовный пиръ и ликованіе. Любовію связаннымъ душамъ и въ темницѣ сидѣть пріятно, слезы другъ о другѣ проливать сладостно; безъ любви и красные чертоги не разнствуютъ отъ темницы. Съ любовію хлѣбъ и вода сладко вкушается; безъ любви и сладкая пища горькою бываетъ. Съ любовію и неволя пріятнѣйшая свободы; безъ любви и свобода горчайшая неволи. Любовію домы, грады, государства стоятъ: безъ любви падаютъ. Любовію присутственныя мѣста украшаются, совѣты во благое успѣваютъ. Любовь воинство укрѣпляетъ, врагу страшнымъ и непобѣдимымъ дѣлаетъ. О, блаженно тое общество, тотъ градъ, тотъ домъ, въ которомъ взаимная процвѣтаетъ любовь! Раю земному, радости и сладости исполненному, подобно мѣсто, въ которомъ любовь, какъ древо сладкими плодами обилующее, пребываетъ. О любы "--- любы, неоцѣненное сокровище, любы! всѣхъ благъ мати, любы! извѣстное живыя вѣры знаменіе хрістіанскія, любы! Коль много бѣдъ, напастей и золъ претерпѣваемъ, понеже не имѣемъ любве! Подтверждаетъ сіе святый Златоустъ, златымъ своимъ языкомъ тако глаголя: «Ежели бы вси въ свѣтѣ другъ друга взаимно любили, не надобно бы было ни законовъ, ни судовъ, ни казней, и ничего симъ подобнаго: понеже всякое бы зло такъ удаленно было, что и самое имя грѣха неизвѣстно бы было»\footnote{\textit{Бесѣд.~32"~я на 1"~е посл. къ Кор}.}.

\paragraph*{§\:253.} Когда человѣкъ тую любовь, которую ближнему долженъ оказывать, къ себѣ обращаетъ, тогда сіе его дѣло называется \textit{самолюбіе}. Самолюбія, которое противится любви ближняго, знаки сіи суть: 1)~Самъ хощетъ отъ прочіихъ любимъ быть; а другимъ того самъ не дѣлаетъ, и не хощетъ дѣлать. "--- 2)~Самъ хощетъ всякое добро имѣть и отъ всякаго зла и злополучія избыть, а о прочіихъ нерадитъ. "--- 3)~Хотя кто и дѣлаетъ кому добро, однакожъ не туне, но ради своей корысти. Таковые суть, которые на обѣдъ зовутъ ближнихъ, чтобы или самимъ отъ нихъ позванными быть, или отъ нихъ любимыми и почитаемыми быть; такожде которые подаютъ бѣднымъ помощь, чтобы отъ нихъ какое нибудь услуженіе имѣть. Сюды принадлежатъ и тѣ, которые ради того даютъ милостыню, чтобы пріемлющіи Богу о нихъ молилися. Любовь бо истинная \textit{туне и доброхотно} дѣлаетъ добро, туне надъ всякимъ бѣднымъ умилостивляется. "--- 4)~Самолюбецъ обиды нанесенныя не терпитъ, но отмщеваетъ, или хощетъ отмщевати. Самолюбецъ же есть таковый потому, что самъ хощетъ прощеніе согрѣшеній отъ Бога получить, и хощетъ, чтобы ближній ему обиду оставилъ (никто бо не хощетъ, чтобы обиженный отъ него на него злобился и отмщевалъ); но самъ того не хощетъ ближнему дѣлать, и такъ любитъ себе болѣе, нежели Бога и ближняго. "--- 5)~Сюды принадлежатъ богачи, которые или домы богатые и прочія увеселенія созидаютъ, или богатые столы набираютъ, или богатыми одеждами украшаются и прочія роскоши чинятъ; а ближнихъ бѣдныхъ презираютъ, или подаютъ имъ, но мало, чимъ нужда требующихъ не исправится. Таковый былъ евангельскій богачъ, который \textit{облачашеся въ порфиру и виссонъ, на вся дни веселяся свѣтло}, а Лазаря нищаго и немощнаго презиралъ\footnote{Лук.~16,~19--21.}. Такожде которые богатое наслѣдіе сынамъ, или богатое приданое дочерямъ готовятъ; а ближнимъ бѣднымъ или ничего не даютъ, или копѣйкою и деньгою награждаютъ: ибо плоть свою и кровь таковые любятъ, а не ближнихъ своихъ. Должно не оставлять и своихъ сродниковъ по плоти; но истинная любовь и не надъ сродниками, какъ надъ сродниками, умилостивляется. "--- 6)~Тягчайшее самолюбіе есть, когда кто ищетъ своей корысти съ обидою ближняго. Таковые суть, которые крадутъ, похищаютъ отнимаютъ чужое добро. Сюды принадлежатъ мздоимцы и неправедные судіи, которые правду за сребро продаютъ ради своего сквернаго прибытка; богачи, которые сребро свое даютъ въ лихву; удерживающіи мзду наемничу и прочіи, чужимъ добромъ неправедно обогащающіися. "--- 7)~Тяжкое паки самолюбіе есть, когда ради какой нибудь корысти не хощемъ научить ближняго, что сами знаемъ. Такъ многіе дѣлаютъ люди мастеровые, какъ"=то: живописцы, портные, сапожники и прочіи, которые ради того не хотятъ скорѣе учениковъ своихъ научить мастерству, чтобы имъ долѣе работали и служили. "--- 8)~Тяжкое паки самолюбіе есть, когда кто не хощетъ и тщится не допустить, чтобы ближній его имѣлъ такое благополучіе, пользу, похвалу, славу и честь, каковыя онъ имѣетъ. И сіе есть зависть злая. Тако дѣлаютъ такожде люди мастеровые, которые ради того не показуютъ мастерства ученикамъ своимъ, чтобы они такую честь и похвалу не получили, сдѣлавшеся искусными въ мастерствѣ, какую они имѣютъ. Тако мудрецы вѣка сего, которые ради того не открываютъ тайнъ естества учащимся у нихъ, чтобы они въ славу и честь не произошли, въ какой они находятся; такожде и прочіи, которые не хотятъ ближняго вразумлять, чтобы не былъ равенъ имъ въ благополучіи. "--- 9)~Тяжкое паки самолюбіе есть, когда кто ради того ближняго предъ людьми порочитъ, чтобы люди думали, что онъ такихъ пороковъ на себѣ не имѣетъ. И сіе есть великое и несносное человѣческаго сердца лукавство и хитрость, хотѣть отъ бѣды ближняго себѣ благополучіе сыскать, отъ поношенія его прославиться. Тако дѣлаютъ всѣ тѣ, которые ближняго тайно или явно осуждаютъ, а о себѣ или молчатъ, или безстыдно велерѣчатъ и злость языка своего тако прикрываютъ: \textit{я"=де не къ осужденію его говорю}. Когда не къ осужденію зло о ближнемъ проносишь, то ради чего о себѣ ничего \textit{не къ осужденію} не говоришь? развѣ никакого грѣха не имѣешь? Такое"=то зло, лукавство и слѣпота въ сердцѣ человѣческомъ крыется. Самъ золъ, а другихъ судитъ. "--- 10)~Весьма тяжкаго самолюбія знаменіе "--- за любовь ненависть и за благодѣяніе злодѣяніе воздавать благодѣтелю. Таковые суть, которые, забывше благодѣяніе благодѣтелей своихъ, поносятъ ихъ, клевещутъ на нихъ и ищутъ надъ ними злобу свою совершить. Сіи люди горшіи самыхъ скотовъ и звѣрей, которые своихъ кормителей знаютъ и никакого зла имъ не дѣлаютъ.

\paragraph*{§\:254.} Отъ вышереченныхъ видно, что какъ любовь къ ближнему всякихъ благъ духовныхъ и временнаго благополучія источникъ есть, такъ самолюбіе всякихъ золъ, грѣховъ и бѣдствія въ свѣтѣ бываемаго виновно. 1)~Отъ самолюбія не терпѣніе, но злоба и мщеніе; ибо \textit{любы долготерпитъ}. 2)~Отъ самолюбія жестокосердіе, немилосердіе, мучительство, лютость; ибо \textit{любы милосердствуетъ}. 3)~Отъ самолюбія зависть мучительная, и смѣха достойная о блаженствѣ ближняго печаль; понеже \textit{любы не завидитъ}. 4)~Самолюбіе кичливо, возносливо, гордо, презорчиво; ибо \textit{любы не превозносится, ни гордится}. 5)~При самолюбіи гнѣвъ и ярость, кличь и хула; ибо \textit{любы не раздражается}. 6)~Самолюбію сопряженно злопомнѣніе; \textit{любовь бо не мыслитъ зла}. 7)~Въ самолюбіи заключается всякая неправда, всякая ложь; понеже любовь не токмо не дѣлаетъ неправды, но и \textit{не радуется о неправдѣ, радуется же о истинѣ}. 8)~Словомъ сказать, сколько ни есть согрѣшеній, обидъ, которыя мы ближнему показуемъ, всѣ отъ самолюбія, какъ отъ корня злаго вѣтви прорастаютъ. 9)~Отъ самолюбія всякое въ мірѣ бѣдствіе. Отъ самолюбія брани, отъ самолюбія толико человѣческой проливается крове, толико остается вдовъ плачущихъ по мужахъ своихъ, толико матерей по сынѣхъ своихъ, толико дѣтей по отцахъ своихъ сѣтующихъ, толико запустѣваетъ государствъ и градовъ, толико иждивается казны, толико дѣлается безпокойствія, смятенія, страха въ обоихъ сражающихся государствахъ. Самолюбіе производитъ разбои, и тѣмъ послѣдующія убійства, насилія и грабленія. Самолюбіе дѣлаетъ судебныя мѣста торжищемъ, научаетъ попирать правду, и неповинною кровію обагрять судейскіе мечи. Самолюбіе наполняетъ темницы праведными и неправедными узниками. Самолюбію приписать должно, что много скитается бѣдныхъ безъ призрѣнія, покрова, одежды, пропитанія; ибо вмѣсто того, чтобы другъ друга снабдѣвать отъ избытка, самолюбіе иныхъ отнимать и послѣднее у ближнихъ, иныхъ хранить у себе богатство отъ Бога данное и не смотрѣть на требованія братіи, иныхъ роскоши чинить неумѣренныя научаетъ. И тако, когда иной на чужое добро руку свою беззаконно простираетъ, иной добро, отъ Бога данное на общую пользу, своимъ называетъ и при себѣ держитъ, не сообщая требующимъ, или на непотребные расходы употребляетъ: оттуду многіе нищіе и убогіе являются. Отъ самолюбія убо нищета и убожество. Тотъ домы богатые, увеселительныя галлереи, сады, пруды, мызы, статуи строитъ и украшаетъ; другій столы съ винами и прочіими снѣдьми богато набираетъ; третій кареты, кони и слуги великолѣпно убираетъ; иной себе, жену, дочерей и сыновъ драгоцѣнными утварьми украшаетъ; иной иныя увеселенія затѣваетъ; и такъ, что должно было нуждѣ и требованію ближняго служить, тое роскошь пожираетъ; и чимъ ближній долженъ былъ довольствоваться, тое самолюбивая душа къ себѣ привлекаетъ. Тако самолюбіе виною есть, что иной нищетствуетъ, иной преизлишне изобилуетъ, иной гладствуетъ, иной пресыщается. Отъ самолюбія и прочія въ мірѣ бѣдствія происходятъ, которыхъ и исчислить невозможно. Плоды убо самолюбія, да вкратцѣ скажу, сіи суть: 1)~всякъ грѣхъ и беззаконіе; 2)~всякое бѣдствіе и злополучіе, въ мірѣ семъ случающееся; 3)~самолюбцу самому злая совѣсть, печаль и отчаяніе вѣчныя Божія милости; 4)~по смерти въ ономъ вѣцѣ вѣчное во адѣ и огнѣ геенскомъ мученіе. Тогда онъ безмѣрно себе возненавидитъ, который себе нынѣ безмѣрно любилъ, самъ собою возгнушается, омерзѣетъ, который себе здѣ безмѣрно почиталъ; возжелаетъ ни во что обратитися, который себе здѣ за велико имѣлъ. Тако самолюбіе всякаго зла виновно.

\paragraph*{§\:255.} Понеже любовь къ ближнему съ любовію Божіею сопряженна: не можетъ любовь къ ближнему разоритися безъ нарушенія любве Божіей. Слѣдственно 1)~кто грѣшитъ противу ближняго, тотъ грѣшитъ и противу Бога, и потому таковый дважды грѣшитъ: противу Бога, заповѣдавшаго: \textit{возлюбиши искренняго твоего, яко самъ себе}, "--- и противу ближняго, котораго ненавидитъ. 2)~Кто хощетъ покаяться и примириться съ Богомъ, Котораго законъ разорилъ, тому первѣе должно примириться съ ближнимъ, котораго обидѣлъ. Тако бо научаетъ Хрістосъ: \textit{аще принесеши даръ твой ко олтарю, и ту помянеши, яко братъ твой имать нѣчто на тя: остави ту даръ твой предъ олтаремъ, и шедъ прежде смирися съ братомъ твоимъ, и тогда пришедъ принеси даръ твой}\footnote{Матѳ.~5,~23 и 24.}.

\subsection[Глава 11-я. О милости къ ближнему.]{глава перваянадесять.\\\bfseries О милости къ ближнему.}

\begin{quotation}\textit{Раздробляй алчущимъ хлѣбъ твой, и нищія безкровныя введи въ домъ твой: аще видиши нага, одѣй, и отъ свойственныхъ племене твоего не презри. Тогда разверзется рано свѣтъ твой, и исцѣленія твоя скоро возсіяютъ: и предъидетъ предъ тобою правда твоя, и слава Божія объиметъ тя. Тогда воззовеши, и Богъ услышитъ тя, и еще глаголющу ти, речетъ: се пріидохъ}\footnote{Ис.~58,~7--9.}.\end{quotation}


\paragraph*{§\:256.} Истинная милость есть плодъ хрістіанскія любве. Коль же высоко милость превозносится въ святомъ Писаніи, и какъ ублажаются милостивые и творящіе дѣла милости, и тѣмъ самымъ какъ сильно къ сей добродѣтели поощряемся, читающему Писаніе святое видно, такъ что и Хрістосъ, праведный Судія, добродѣтели сея плоды превознесетъ на всемірномъ ономъ судѣ, и какъ сокровища драгая, въ избранныхъ Своихъ сокровенная, всему міру въ явленіе произнесетъ, глаголя къ нимъ: \textit{пріидите, благословенніи Отца Моего, наслѣдуйте уготованное вамъ царствіе отъ сложенія міра. Взалкахся бо, и дасте Ми ясти; возжадахся, и напоисте мя; страненъ быхъ, и введосте Мене; нагъ, и одѣясте Мя; боленъ, и посѣтисте Мене; въ темницѣ бѣхъ, и пріидосте ко Мнѣ}\footnote{Матѳ.~25,~34--36.}. Ибо какую милость во имя Хрістово дѣлаемъ нищимъ, тую Хрістосъ \textit{Себѣ} приписуетъ\footnote{ст.~40.}.

\paragraph*{§\:257.} Какъ человѣкъ изъ двухъ частей, тѣла и души, состоитъ, такъ и дѣла милости, которыя ближнему являются, двоякія суть: душевныя и тѣлесныя. Дѣла милости, которыя тѣлу показуемъ, суть: алчущаго напитать, жаждущаго напоить, нагаго одѣть, и проч. Дѣла милости душевныя суть: скорбящаго утѣшить, заблуждшаго на путь истины наставить, сумнящагося вразумить, и проч. И къ той и другой милости одолжаетъ и поощряетъ насъ святое Божіе Слово.

\paragraph*{§\:258.} Причины, поощряющія насъ къ душевной милости: 1)~Требуетъ того отъ насъ Хріста Бога нашего спасительная воля, и какъ бы жажда спасенія нашего, чтобы мы какъ своего, такъ и ближняго спасенія искали, и другъ друга къ покаянію и богоугожденію поощряли. Аще убо кто хощетъ со Хрістомъ едино мыслить, долженъ какъ о своемъ, такъ и о ближняго спасеніи пещися. "--- 2)~Требуетъ того честь Хрістова, чтобы мы другъ друга исправляли. Братъ брата по плоти увѣщаваетъ, когда видитъ его безчинствующаго, дабы безчестія отцу и дому ихъ не было: тако должно и духовнымъ братіямъ, то"=есть хрістіанамъ, другъ друга увѣщавать и отводить отъ безчиннаго житія, чтобы имя Божіе и хрістіанское ученіе не терпѣло поношенія отъ язычниковъ. Коль бо \textit{добрымъ} хрістіанъ \textit{житіемъ славится}\footnote{5,~16.}, такъ \textit{злымъ безчестится имя Его}\footnote{Римл.~2,~23 и 24.}. Слыши, что апостолъ пишетъ къ Тимоѳею: \textit{елицы суть подъ игомъ раби, своихъ господій всякія чести да сподобляютъ, да имя Божіе не хулится и ученіе}\footnote{1~Тим.~6,~1.}. 3)~Требуетъ того братская любовь, дабы братъ заблуждающій не погиблъ, но на путь истины обратился, въ покаяніе пришелъ, и тако бы спасеніе получилъ. "--- 4)~Отъ того и любовь ко Хрісту познается, когда не токмо о своемъ, но и о ближняго спасеніи печемся. Когда Хрістосъ по воскресеніи вопрошалъ Петра святаго: \textit{Симоне Іонинъ, любиши ли Мя паче сихъ}? и отвѣщалъ Ему Петръ: \textit{ей, Господи, Ты вѣси, яко люблю Тя}, "--- придалъ Хрістосъ: \textit{паси овцы Моя}\footnote{Іоан.~21,~15--17.}. Отъ сего видѣть всякъ можетъ, какъ хощетъ Хрістосъ, чтобы другъ о другѣ старалися и другъ другу спасеніе искали; что, когда хощемъ любить Хріста, должны какъ себѣ, такъ и ближнему желать спасенія. И хотя вышереченное слово до пастырей касается, однакожъ кровь Хрістова, за грѣшниковъ изліянная, ко всѣмъ вопіетъ, чтобы не туне какъ ради насъ, такъ и ради братіи нашей проліянна была, но плодъ бы свой спасеніе наше возымѣло. Алчетъ и жаждетъ Онъ спасенія нашего безъ сумнѣнія; но кто какъ себе, такъ и ближняго исправляетъ и къ покаянію приводитъ, тотъ какъ бы алчущаго Его питаетъ и жаждущаго напояетъ. Не малую любовь ко Хрісту показуетъ, кто во имя Его милость тѣлесную ближнему дѣлаетъ; но большая любовь есть, когда духовно ближняго созидаетъ. Чимъ бо большая есть душа отъ тѣла, тѣмъ большая милость къ ближнему и ко Хрісту любовь, когда душа созидается. Хрістосъ бо весьма души человѣческія любитъ, такъ что и умереть за нихъ благоизволилъ. И потому ничто не можетъ благопріятнѣе Ему быть, какъ спасеніе человѣческое; и никто не можетъ болѣе Его любить, какъ тотъ, который спасенія ближняго ищетъ. "--- 5)~Когда въ вещественномъ тѣлѣ единъ удъ немоществуетъ, тогда другій немощному помогаетъ; напр. рука другую руку или ногу больную врачуетъ: такъ и въ духовномъ тѣлѣ, то"=есть, хрістіанствѣ, единъ здоровый другаго немощнаго долженъ лѣчить и тщаться о здравіи его. "--- 6)~Общество хрістіанское или церковь, странствующая на земли, подобна есть воинству, подвизающемуся противу непріятеля. На брани міра сего единъ воинъ другаго изнемогающаго поощряетъ и подкрѣпляетъ: тако и въ брани хрістіанской, которая \textit{не есть къ крови и плоти, но къ началомъ и ко властемъ и къ міродержителемъ тмы вѣка сего, къ духовомъ злобы поднебеснымъ}\footnote{Еф.~6,~12.}, должно единому другаго изнемогающаго и падающаго укрѣплять и возставлять. Діаволъ и аггели его единодушно на хрістіанъ вооружаются и погибели ихъ ищутъ: такъ должно и хрістіанамъ съ помощію Божіею противу ихъ стоять и другъ друга побуждать. "--- \textit{Пастырское}, говоришь, \textit{дѣло есть исправлять и къ покаянію приводить людей}? "--- \textit{Отвѣтъ}. Пастырское дѣло есть публично и вездѣ \textit{слово Божіе проповѣдывать, настоять благовременнѣ и безвременнѣ, обличать, запрещать, умолять}\footnote{2~Тим.~4,~2.}; а твое не публично, но тайно, дружески между тобою и братомъ, гдѣ благодать Божія подастъ тебе случай. "--- Иный отвѣщаетъ: \textit{я"=де невѣжда; какъ мнѣ ближняго увѣщавать и учить}? "--- \textit{Отвѣтъ}. Любовь сыщетъ слова, коими можешь созидать ближняго; она представитъ тебѣ способъ, и умъ и языкъ твой направитъ. И дѣло сіе не требуетъ красныхъ рѣчей; единаго напоминанія требуетъ. Часто грубое слово, благодатію Божіею и любовію растворенное, тое дѣлаетъ, чего многіе краснорѣчивые риторы не могутъ сдѣлать. Когда сына, или друга твоего уговаривать хощешь, "--- не ищешь краснорѣчія: такъ поступай съ ближнимъ твоимъ, который требуетъ того. А когда такъ скудоуменъ ты, что подлинно не можешь ближнему совѣта подать полезнаго: можешь другій употребить способъ. Когда отецъ твой, или мать, или жена, или сынъ, или братъ по плоти болитъ "--- ищешь такого человѣка, ктобъ имъ въ болѣзни помоглъ: такъ поступай съ духовнымъ твоимъ братомъ, который душею немоществуетъ. Поищи такого, который бы моглъ ему пособить; объяви ему болѣзнь брата твоего; болѣзнь объяви, а не оклеветай его, "--- съ сожалѣніемъ и любовію къ нему, а не съ ненавистію и злобою, какъ многіе обыкли дѣлать, чтобы онъ, узнавши немощь его, приличное немощи лѣкарство подалъ. Тако пріобрящешь брата твоего; а когда не пріобрящешь, самъ благодатію Божіею лучшій будеши. Ибо Богъ какъ дѣло, такъ и намѣреніе и тщаніе доброе милостиво пріемлетъ и вѣнчаетъ. "--- Иный отзывается: \textit{я"=де самъ грѣшникъ: какъ мнѣ другихъ увѣщавать}? "--- Правда, никто безъ грѣха; но ты въ такомъ случаѣ подражай доброму лѣкарю, который иногда самъ недомогаетъ, но другаго больнаго лѣчитъ. А когда такую любовь и милость покажешь ближнему твоему, то и самъ отъ Бога получишь, по оному: \textit{блажени милостивіи: яко тіи помиловани будутъ}\footnote{Матѳ.~5,~7.}. "--- Иный думаетъ: \textit{мнѣ"=де стыдно ближняго обличать}? "--- \textit{Отвѣтъ}. Любовь стыда и срама не знаетъ. Да и стыдиться тамо худо и неполезно, гдѣ погибель явная. Грѣшить стыдно, а не грѣхъ обличать и отъ грѣха отвращать. "--- Иный помышляетъ: \textit{я бы"=де и хотѣлъ ближняго обличить, но боюсь гнѣва и вражды его}, ибо не всякому обличеніе пріятно. "--- Правда, что не вси обличеніе любятъ, что и Соломонъ означаетъ, глаголя: \textit{не обличай злыхъ, да не возненавидятъ тебе: обличай премудра, и возлюбитъ тя}\footnote{Притч.~9,~8.}. На сіе тако отвѣщаю: 1)~Неизвѣстно тебѣ, золъ ли онъ, котораго хощешь обличить, или доброе и благосклонное сердце имѣетъ, потому и вражда его отъ обличенія неизвѣстна. "--- 2)~Часто бываетъ, что Божія благодать и злаго въ добраго премѣняетъ чрезъ полезное и любовное совѣтованіе и представленіе, о чемъ многіе примѣры въ церковной исторіи имѣются. "--- 3)~\textit{Любы вся уповаетъ}, какъ сказано\footnote{1~Кор.~13,~7.}. Она и тамо надѣется, гдѣ нѣтъ надежды. «Любы, глаголетъ Златоустъ святый, не отчаявается любимаго; но аще и золъ будетъ, пребываетъ исправляющи, промышляющи, прилѣжащи»\footnote{Бес.~33"~я на 1"~е посл. къ Кор.}. Сколько труждаешься около сына своего неисправнаго, котораго любишь и хощешь исправить; колико способовъ къ тому употребляешь, колико роптанія и негодованія отъ него терпишь, "--- и хотя видишь неисправна, однакожъ не оставляешь его увѣщавать, и паки новыхъ способовъ къ исправленію ищешь, ибо любовь не отчаявается, хотя и надежды нѣтъ: тако и ты, когда любишь ближняго твоего, не оставишь никакого способа, чтобы исправить его, хотя и непреклоненъ является; ибо \textit{любы вся уповаетъ}, "--- и хотя гнѣваться будетъ, за благо пріимешь, ибо \textit{любы вся терпитъ}. "--- 4)~Подобаетъ паче боятися Божія гнѣва, нежели человѣческаго. Ибо Богъ Который повелѣваетъ намъ \textit{любить ближняго какъ себе}, Тойжде велитъ намъ и о спасеніи его пещися, какъ о своемъ. Любовь бо не токмо состоитъ въ томъ, чтобы о тѣлесной брата пользѣ промышлять, но въ томъ паче, чтобы о души его пещися. Ибо тѣло и польза тѣлесная престаетъ, а душевное спасеніе во вѣки пребываетъ, и ради того болѣе о души какъ нашей, такъ и ближняго попеченіе имѣть должны мы, когда хощемъ ближняго нашего любить, какъ себе. "--- 5)~Пріиметъ ли онъ твой совѣть, или не пріиметъ, возлюбитъ ли онъ тебе за обличеніе, или возненавидитъ: что къ тебѣ? ты свою должность исполни. Когда ему не воспользуетъ совѣтъ твой: тебѣ пользу принесетъ; понеже любве долгъ брату покажешь. Но любовь и въ семъ случаѣ не престаетъ: она другій способъ къ исправленію ближняго находитъ. Она тогда теплыя проливаетъ молитвы къ человѣколюбцу Богу, чтобы, какими вѣсть, судьбами обратилъ заблуждающаго брата. "--- Суть еще люди такіе, которые, видя погибель ближняго, говорятъ: \textit{что мнѣ до него нужды}? "--- Сіи рѣчи не любительнаго, ниже хрістіанскаго духа суть. Таковые да слышатъ, что Златоустъ о семъ дѣлѣ глаголетъ: «Не могу вѣрить, чтобы тотъ спасеніе получилъ, кто о спасеніи ближняго нерадитъ»\footnote{Кн.~6 о священствѣ, гл.~10.}. Когда ты отрицаешься брата твоего, за котораго Хрістосъ умеръ: берегись, чтобы Хрістосъ тебе самаго не отреклся. Ибо, когда говоришь: \textit{что мнѣ до него нужды}? "--- показуешь, что ни малѣйшей искры любве и сожалѣнія не имѣешь, безъ чего хрістіанинъ быть не можетъ.

\paragraph*{§\:259.} Причины, къ тѣлесной милости возбуждающія. 1)~Созданіе Божіе повелѣніемъ Божіимъ всѣмъ равно служитъ. Солнце, мѣсяцъ, звѣзды богатому и нищему равно сіяютъ; облака богатому и нищему равно дождятъ; воздухъ богатаго и нищаго жизнь равно сохраняетъ; огнь богатаго и нищаго равно грѣетъ; вода богатаго и нищаго равно напаяетъ; земля богатому и нищему равно плодъ подаетъ. Убо тѣмъ научаемся, что благими міра сего равно богатые и нищіе довольствоваться должны. Слѣдственно худо дѣлаютъ, которые богатствомъ міра сего сами довольствуются, а нищимъ не сообщаютъ его; хуже того дѣлаютъ тѣ, которые не токмо не сообщаютъ своего, но и отнимаютъ у другихъ и къ себѣ привлекаютъ. "--- 2)~Какое ни имѣемъ добро, не свое имѣемъ, но отъ Бога данное намъ имѣемъ, которое дано намъ не ради насъ единыхъ, но и ради ближнихъ нашихъ: чего ради ввѣренное намъ добро не должны мы при себѣ удерживать, но вѣрной братіи нашей требующей въ славу Божію раздѣлять. Слѣдственно воздадятъ Богу отвѣтъ, которые данное отъ Бога добро не раздѣляютъ въ честь Даровавшаго, но или при себѣ удерживаютъ, или расточаютъ на роскоши и прочіе непотребные расходы. "--- 3)~Требуетъ того хрістіанская любовь, чтобы надъ бѣднымъ умилосердитися, и брату въ нуждѣ помощь подать, нагаго одѣть и алчущаго напитать, жаждущаго напоить, и проч.\footnote{Матѳ.~15,~35 и 36.} "--- 4)~\textit{Милостивые помилованы будутъ}, по Хрістову обѣщанію. «Нищій, глаголетъ Василій Великій, который проситъ у тебе, взаимъ проситъ, представляя тебѣ Богатаго на небесѣхъ, Который отдастъ тебѣ долгъ. \textit{Милуяй бо, рече, нищаго, взаимъ даетъ Богу}»\footnote{На пс.~14"~й.}. "--- 5)~\textit{Судъ безъ милости не сотворшему милости}\footnote{Іак.~2,~13.}. И немилостивые во огнь вѣчный отсылаются, какъ глаголетъ Хрістосъ: \textit{идите отъ Мене проклятіи во огнь вѣчный, уготованный діаволу и аггеломъ его. Взалкахся бо, и не дасте Ми ясти; возжадахся, и не напоисте Мене; страненъ бѣхъ, и не введосте Мене; нагъ, и не одѣясте Мене; боленъ, и въ темницѣ, и не посѣтисте Мене}\footnote{Матѳ.~25,~41--43.}. "--- 6)~Требуетъ того законъ естественный, чтобы дѣлать ближнему нашему тое, чего хощемъ себѣ. Сами хощемъ ѣсть, пить, одѣваться, упокоеваться: убо должны и братіи нашей тоежъ дѣлать. Тоеждо бо естество, плоть и немощь имѣютъ, какую и мы; такожде хотятъ ясти, пить, одѣваться, какъ и мы, и жить безъ того не могутъ. Аще убо кто требующему въ нуждѣ отрицаетъ, а имѣетъ, сей и послѣднюю закона естественнаго искру потерялъ, ибо никакого къ подобному себѣ естеству сожалѣнія не имѣетъ, и есть не человѣкъ, но точію внѣшній образъ человѣческій носитъ, самою же вещію какъ звѣрь, или какъ камень, который никому никакого плода не приноситъ, и ни жалостію, ни гласомъ плачевнымъ не подвигается. Понеже самые язычники, естественнымъ закономъ движимые, надъ бѣдными милосердствуютъ: хрістіанъ же не токмо родство естественное, то"=есть, что вси человѣцы есмы, но и родство духовное къ сожалѣнію и милосердію подвигнуть должно, а паче всего имя Хрістово, которое какъ велико, почтенно и сладко знающимъ Его должно быть, хрістіанамъ не безъизвѣстно, "--- имя, глаголю, Хрістово, отъ бѣднаго упоминаемое и къ умилостивленію предлагаемое, всякую милость и помощь отъ хрістіанъ просящему получить должно. Аще бо ради имени любимаго друга или благодѣтеля упоминаемаго не щадимъ ничего, но прошенія исполняемъ, кольми паче ради Хріста, Котораго не можетъ быть большій благодѣтель намъ, всякое прошеніе, только бы непротивное было воли Его, исполнять должны. Разсуди всякъ, велико ли есть неимущему покрова хижину соорудить, или нагаго одѣть, или изъ плѣна выкупить, или долгами обремененнаго отъ той тягости свободить, или за подати и оброки мучимаго искупить, или въ темницѣ сѣдящему, или болящему послужить, или иному какому бѣдствующему руку помощи подать тебѣ ради Того, Который, такъ великъ будучи, что никто Ему не равенъ, восхотѣлъ такъ смиритися и горчайшую страданія и смерти чашу ради тебе и твоего ради спасенія испить? Не токмо имѣній нашихъ о Хрістѣ, но и насъ самихъ ради имени Его щадить не должно намъ. Вѣдущимъ, что есть имя Хрістово, говорю здѣ. "--- 7)~Съ какою надеждою молитися будешь Богу, когда самъ молитвы подобныхъ тебѣ людей не слушаешь? Какъ скажешь: \textit{Господи помилуй}, когда самъ не милуешь? Какъ будешь просить съ прочіими: \textit{подай Господи}, когда самъ не подаешь, а можешь подать? Какими устами скажешь: \textit{услыши мя, Господи}, когда самъ не слышишь бѣднаго или паче въ бѣдномъ Самаго Хріста вопіющаго къ тебѣ? Съ какимъ упованіемъ прострешь руцѣ твои къ Создателю твоему, когда самъ подобнаго себѣ простирающаго руцѣ отвращаешися? Милостивъ Богъ и преклоняется естественнымъ милосердіемъ на молитву, но на молитву милостивыхъ. Ибо \textit{таковыми жертвами}, то"=есть страннолюбіемъ и общеніемъ, \textit{благоугождается Богъ}, глаголетъ апостолъ\footnote{Евр.~13,~16.}. "--- 8)~Нынѣ время благопріятно благотворити въ честь и славу благаго Бога, расточати, давати убогимъ, \textit{творити други себѣ отъ мамоны неправды, да, егда оскудѣемъ, пріимутъ насъ въ вѣчныя кровы}\footnote{Лук.~16,~9.}. Нынѣ сіи добрыя сѣмена сѣяти благопріятно, да въ воскресеніе общее пожнемъ съ радостію, несравненно множайшій и лучшій и во вѣки пребывающій плодъ. \textit{Егда тлѣнное сіе облечется въ нетлѣніе, и смертное сіе облечется въ безсмертіе}\footnote{1~Кор.~15,~54.}: тогда всякъ увидитъ своихъ добрыхъ трудовъ плодъ; и что нынѣ сѣялъ съ надеждою, тогда съ веселіемъ неизреченнымъ пожнетъ; тогда увидитъ свое \textit{сокровище, собранное на небеси}, сокровище нетлѣнное, \textit{котораго ни червь, ни тля тлитъ, ни татіе подкопываютъ, ни крадутъ}\footnote{Матѳ.~6,~20.}, которое здѣсь на земли во имя Хрістово \textit{расточи, даде убогимъ}\footnote{Пс.~111,~9; 2~Кор.~9,~9.}. "--- 9)~Какъ любезно, желательно и радостно будетъ явитися милостивому предъ лицемъ \textit{милостиваго и кроткаго} Сына Божія со многою братіею Его, которыхъ во имя Его здѣ питалъ, напоялъ, одѣвалъ, упокоевалъ, болящихъ и въ темницѣ сѣдящихъ посѣщалъ! Какъ преславно будетъ тамо нарицатися отцемъ многихъ бѣдныхъ, здѣ помилованныхъ и снабдѣнныхъ! "--- 10)~Коль желательнѣе того еще слышать отъ Судіи праведнаго благопріятный гласъ оный: \textit{пріидите, благословенніи Отца моего, наслѣдуйте уготованное вамъ царствіе отъ сложенія міра. Взалкахся бо, и дасте Ми ясти; возжадахся, и напоисте Мя; страненъ бѣхъ введосте Мене; нагъ, и одѣясте Мя; боленъ, и посѣтисте Мене; въ темницѣ бѣхъ, и пріидосте ко мнѣ}! "--- 11)~Напротивъ того, какъ ужасно будетъ немилостивымъ слышать страшный оный праведнаго Судіи гласъ: \textit{отъидите отъ Мене проклятіи во огнь вѣчный, уготованный діаволу и аггеломъ его. Взалкахся бо, и не дасте Ми ясти; возжадахся, и не напоисте Мене; страненъ бѣхъ, и не введосте Мене; нагъ, и не одѣясте Мене; боленъ и въ темницѣ, и не посѣтисте Мене}! "--- 12)~Богатства никто съ собою не возметъ, отходя отсюда, но какъ нагъ пришелъ въ міръ, такъ и отъидетъ нагъ, единою душею. \textit{Ничтоже внесохомъ въ міръ сей, явѣ, яко ниже изнести что можемъ}, глаголетъ апостолъ\footnote{1~Тим.~6,~7.}. Богатство же, которое здѣ оставляемъ, по смерти погибаетъ и намъ никакой пользы не принесетъ; а которое здѣ во имя Хрістово расточится, тое на небесѣхъ съ великимъ ростомъ обрящется. Ибо вмѣсто тлѣннаго нетлѣнное, вмѣсто земнаго небесное, вмѣсто временнаго вѣчное, вмѣсто малаго несравненно большее сокровище отъ щедрой руки Господни воспріиметъ расточаяй и даяй убогимъ имѣніе во имя Его: \textit{милуяй бо нищаго, взаимъ даетъ Богови, по даянію же воздастся ему}, глаголетъ Соломонъ\footnote{Притч.~19,~17.}. Тако бо милостивъ и щедръ Господь нашъ, что и тотъ, \textit{который чашу студеныя воды подастъ во имя Его, не погубитъ мзды своея}\footnote{Матѳ.~10,~42; Марк.~9,~41.}.

\paragraph*{§\:260.} Вызывается богачъ и представляетъ резонъ, что убогимъ отказываетъ: \textit{я"=де имѣю дѣтей; надобно дѣтямъ оставитъ наслѣдіе}. "--- Правда, не должно дѣтей оставлять. Требуетъ того естественная къ дѣтямъ отеческая любовь. Ближайшія суть дѣти твои тебѣ, нежели другіе. И худо и безчеловѣчно дѣлаетъ тотъ, который, оставивши домашнихъ безъ хлѣба, питаетъ другихъ. Не оставляй убо дѣтей, но не оставляй и убогихъ, которые просятъ у тебе милости ради Хріста, за всѣхъ и за тебе умершаго. Люби дѣтей, но Хріста не токмо болѣе дѣтей, но и болѣе себе люби, Который у тебе чрезъ убогихъ проситъ. Но притомъ внимай слѣдующему предложенію: 1)~Когда хощешь оставить наслѣдіе дѣтямъ, "--- неизвѣстно, дѣтямъ ли оно достанется, или врагамъ твоимъ. Ибо часто бываетъ, что имѣніе чужимъ достается по смерти, а не тѣмъ, которымъ желаемъ при животѣ нашемъ. "--- 2)~«Неизвѣстно, глаголетъ Василій Великій, будетъ ли имъ полезно оставленное богатство твое»\footnote{Въ сл. къ богатымъ.}. Понеже многія дѣти отеческимъ имѣніемъ оставльшимся развращаются и погибаютъ временно и вѣчно. "--- 3)~Душа твоя ближайшая тебѣ и дражайшая должна быть, нежели дѣти твои: должно тебѣ болѣе о ней пещися, нежели о дѣтяхъ. Ибо съ нею ты во вѣки будешь, или въ животѣ вѣчномъ, или въ мукѣ вѣчной: а дѣти твои дотоль твои, доколь въ свѣтѣ семъ съ ними живешь, хотя"=то и при животѣ твоемъ Божіи суть паче, нежели твои, какъ и самъ ты. "--- 4)~Дѣти, и безъ наслѣдія оставшеся, могутъ нажить имѣніе, какъ и ты нажилъ. Ибо Богъ, Который тебѣ далъ, можетъ и имъ дать: а душа, отъ тебе оставленная, у кого сыщетъ милость, когда и заповѣдь Божію презираешь, и милости къ ближнему не показуешь? "--- 5)~Дѣти твои имѣютъ Промыслителя общаго "--- Бога, Который о всѣхъ промышляетъ. Его промыслу какъ себѣ, такъ и дѣтей ввѣряй, а не на богатство погибающее надѣйся, отъ чего апостолъ отвращаетъ, и дѣтей тогожде учи, чтобы \textit{уповали не на богатство погибающее, но на Бога живаго, дающаго намъ вся обильно въ наслажденіе}\footnote{1~Тим.~6,~17.}. "--- 6)~Нѣтъ лучшаго наслѣдія, какъ чтобы дѣти твои наслѣдили твою добродѣтель и нравы добрые, чтобы отъ тебе "--- отца научились милость творить ближнему; видя тебе, родителя своего милостиваго, и сами бы милостивы были. Сіе наслѣдіе тщись имъ оставить. Потщись \textit{ихъ воспитать въ наказаніи и ученіи Господни}, какъ апостолъ учить\footnote{Еф.~6,~4.}. А когда безъ наказанія сего оставишь ихъ, то и богатство твое, имъ въ наслѣдіе оставленное, мечь паки будетъ, которымъ сами себе погубятъ, и самъ ты предъ судомъ Божіимъ повиненъ явишися. "--- 7)~Когда дѣтямъ оставить хощешь имѣніе, а убогихъ презираешь: никакой любви хрістіанской не имѣешь, кромѣ плотской; ибо хрістіанская любовь всѣхъ объятіями своими объемлетъ. Съ какою убо надеждою безъ хрістіанской любви умирать будешь? Ибо гдѣ любви хрістіанской нѣтъ, тамо и вѣры нѣтъ, безъ которой \textit{спастися невозможно}\footnote{11,~6.}: \textit{вѣра бо, любовію поспѣшествуема, спасаетъ}\footnote{Гал.~6,~6.}; и \textit{вѣра безъ дѣлъ мертва есть}, глаголетъ святое Писаніе\footnote{Іак.~2,~20 и 26.}. "--- Иный отзывается: \textit{я"=де при смерти имѣніе роздамъ}. "--- \textit{Отвѣтъ}. 1)~Неизвѣстно, какъ скончаешь житіе; неизвѣстно, возможеши ли слово сказать въ день смерти твоея; а хотя и напишешь духовную о имѣніи своемъ, но учинится ли по духовной твоей, такожде неизвѣстно. "--- 2)~«Тако смерти твоей, глаголетъ Василій Великій, приписуется милость, а не тебѣ»\footnote{Въ сл. къ богатымъ.}. Ибо ежели бы смерть не побудила тебе къ тому, не дѣлалъ бы ты милости. "--- 3)~Тогда ли хощешь раздавать богатство, когда уже и не можешь болѣе его имѣть, и не въ твоей власти остается? "--- 4)~Тогда ли хощешь милостивымъ и щедрымъ быть, когда престаешь жить и отъ людей отлучаешься? Тогда ли добро дѣлать, когда жизнь, дыханіе, движеніе престаетъ и всякое дѣйствіе кончится, и время испытанія и суда настоитъ? "--- 5)~«Ежели не дерзаешь знатныхъ людей останками трапезы твоея подчивать, глаголетъ Василій великій: како дерзаешь Бога останками твоими на милость преклонять?»\footnote{Въ сл. къ богатымъ.} Отложи убо всѣ сіи мысли, которыя тебе отвлекаютъ отъ добродѣтели, и связуютъ сердце твое скупостію и сребролюбіемъ; пресѣки узы сіи мечемъ вѣры и упованія на Бога, и отдай имѣніе твое руками убогихъ въ руки Хрісту, Который, \textit{богатъ сый, насъ ради обнища, да мы нищетою Его обогатимся}\footnote{2~Кор.~8,~9.}. "--- Иный объявляетъ: \textit{я"=де въ своемъ имѣніи воленъ}. "--- \textit{Отвѣтъ}. Ничего, кромѣ грѣховъ, своего не имѣешь, который \textit{нагъ изъ чрева матери твоея изшелъ, нагъ и отъидешь}\footnote{Іов.~1,~21.}. Скупость и лакомство "--- твое, а имѣніе "--- Божіе добро есть, котораго расходъ долженъ ты чинить по воли Божіей, а не твоей. А когда по своей воли и своимъ прихотямъ Божіе добро будешь расточать, или при себѣ держать, не употребляя на законные и богоугодные расходы: непремѣнно, какъ невѣрный и лукавый рабъ, суда Божія не избѣжишь\footnote{Лук.~12,~45 и 46.}. "--- Иный думаетъ и говоритъ: \textit{я"=де хощу на свѣтѣ жить, и ради того имѣніе запасаю и берегу}. "--- \textit{Отвѣтъ}. Такъ"=то думалъ и евангельскій богачъ жити на свѣтѣ, ясти, пити и веселитися; но слыши, что \textit{рече ему Богъ: безумне! въ сію нощь душу твою истяжутъ отъ тебе: а яже уготовалъ еси, кому будутъ}\footnote{Лук.~12,~20.}? Тако и ты обѣщаешь себѣ тое, что не въ твоей власти состоитъ, жить хощешь на свѣтѣ, но смерть предъ тобою невидимо стоитъ, судъ Божій при дверехъ, и Богъ съ небесе глаголетъ: \textit{безумне! въ сію нощь душу твою истяжутъ отъ тебе, а яже уготовалъ еси, кому будутъ}? "--- Иный мыслитъ: \textit{ежели де мнѣ отъ имѣнія убогимъ давать, то чимъ мнѣ жить}? "--- \textit{Отвѣтъ}. Не жить тебѣ нечѣмъ будетъ, но нечѣмъ будетъ роскошно жить; нечѣмъ будетъ расширять и украшать зданій, палатъ, стѣнъ; нечѣмъ набирать богатыхъ столовъ, нечѣмъ запасать драгихъ винъ, нечѣмъ принимать частыхъ гостей; нечѣмъ шить различной и драгой одежды; не на что покупать бисера, алмазовъ, яхонтовъ и прочей прелести и суеты; нечѣмъ доставать, украшать и содержать избранныхъ коней, имъ приличныхъ наборовъ и каретъ, богато одѣтыхъ слугъ; нечѣмъ въ карты играть, нечѣмъ собачью охоту содержатъ, нечѣмъ увеселительныхъ галлерей строить и прочіихъ прихотей исполнять: понеже на всѣ сіи суетные и непотребные расходы много иждивенія потребно. Не до милостыни, не до нищихъ, когда такъ пространно себе хочешь содержать. А когда отъимется излишество, прекратится роскошь, то и тебѣ и требующимъ у тебе довольно будетъ, ибо естество немногимъ довольствуется; а прихоть и роскошь пагубная и всѣмъ свѣтомъ недовольна. "--- Наконецъ, самолюбіе, скупость, сребролюбіе и немилосердіе много вымышляетъ причинъ и извиненій, которыхъ и исчислить невозможно. Сихъ"=то ради причинъ \textit{не удобно богатые входятъ въ царствіе небесное}\footnote{Матѳ.~19,~23.}. Понеже 1)~уповаютъ на богатство свое, а не на Бога живаго, что есть идолослуженіе; 2)~въ богатыхъ скупость и сребролюбіе гнѣздится; 3)~гордость, и тоя дщерь "--- презрѣніе бѣдныхъ и убогихъ; 4)~немилосердіе къ страждущей подобной братіи; 5)~пагубная роскошь и проч. А всему корень самолюбіе. Не богатство погибели богатыхъ виновно бываетъ, "--- ибо богатство Божіе есть дарованіе, и многіе богатые были, но были благочестивые, какъ"=то: Авраамъ, Исаакъ, Іаковъ и прочіи праотцы наши, якоже и нынѣ многіе суть, "--- но чтожъ? Сердце самолюбивое и прилѣпляющееся къ богатству и отъ Бога живаго отвращающееся. Чего ради Псаломникъ глаголетъ: \textit{богатство аще течетъ, не прилагайте сердца}\footnote{Пс.~61,~11.}. Примѣчай здѣ, любезный читатель, что здѣ предлагается о богатствѣ праведномъ, а не о томъ, которое собрано неправдою, хищеніемъ, беззаконною лихвою, лестію и коварствомъ. Сіе бо богатство есть пагубное, и стяжателя своего вѣчно погубитъ, когда злѣ собраннаго добрѣ не расточитъ, и истиннаго покаянія и плодовъ его не сотворить (о чемъ выше въ статьѣ первой части сея сказано).

\paragraph*{§\:261.} Здѣ прилично и нужно воспомянуть о словахъ многихъ хрістіанъ, которыя слова не токмо хладныя и удаленныя отъ любви, но и пустыя и, какъ истину сказать, безчеловѣчныя. 1)~Многіе, видя бѣдность ближняго и не хотя помощи ему подать, отрицаются безчеловѣчно и въ сердцѣ крыющееся немилосердіе объявляютъ: \textit{что"=де мнѣ до него нужды}? Язычники естественнымъ закономъ преклоняются на бѣдность, бѣдному состраждутъ и милосердіе показуютъ, какъ довольно исторіи о томъ повѣствуютъ; а хрістіане, исповѣдающіи Хріста, Хріста "--- Человѣколюбца и милость ублажающа, законъ богописанный и Евангеліе слышащіи, и чающіи воскресенія мертвыхъ и жизни будущаго вѣка, отрицаются сроднаго своего естества, и отрыгаютъ слова: \textit{что мнѣ до него нужды!.}. Хрістосъ, Котораго исповѣдаешь, до тебе и до всѣхъ насъ нужду возъимѣлъ, видя насъ въ великомъ бѣдствіи, и тако надъ всѣми нами умилосердился, и такъ чудно отъ бѣдствія избавилъ: а ты о подобномъ себѣ ближнемъ, котораго Онъ повелѣлъ тебѣ любить и милость показывать, "--- ближнемъ бѣдствующемъ, произносишь безчеловѣчныя слова: \textit{что мнѣ до него нужды}?! Богъ на всякъ день о тебѣ промышляетъ, питаетъ и одѣваетъ тя, даетъ тебѣ здравіе и крѣпость, силу и прочая благая, чтобы ты и самъ тѣми довольствовался и требующей братіи служилъ; а ты бѣдствующему отказываешь: \textit{что мнѣ до него нужды}?! Самъ милости сподобляешься, но другихъ миловать не хощешь и безстыдно отрицаешь: \textit{что мнѣ до него нужды}! Не токмо ослѣпленнаго, но и неблагодарнаго и лукаваго сердца слово есть: \textit{что мнѣ до него нужды}? Брать въ бѣдствіи погибаетъ, а ты помощи ему не подаешь, въ бѣдствіи погибающаго оставляешь, отъ погибели не избавляешь: нѣтъ сожалѣнія о сродномъ страждущемъ естествѣ! Самъ, уповаю, многократно отъ ближняго въ нуждахъ помощь дозналъ, и нынѣ требуешь, ибо никто на свѣтѣ не живетъ безъ помощи другаго. Вси другъ отъ друга помощи требуютъ: нищій отъ богатаго, невѣжда отъ разумнаго, немощный отъ здороваго, въ темницѣ заключенный отъ свободнаго, странствующій отъ имущаго домъ ищетъ милости. Сея"=то ради причины и Богъ, милосердуя о насъ, заповѣдалъ намъ: \textit{возлюбиши искренняго твоего, яко самъ себе}\footnote{Матѳ.~22,~39.}. \textit{Вся, елика аще хощете, да творятъ вамъ человѣцы, и вы творите имъ такожде: се бо есть законъ и пророцы}, глаголетъ Хрістосъ\footnote{7,~12.}. Хощешь, чтобы тебѣ ближній помогалъ въ нуждѣ твоей: безспорно сіе. А когда и хощешь, и дозналъ и дознаешь помощь себѣ отъ другаго, какъ сказано, и впредь безъ того не можешь жить: такъ самъ разсуди, какъ безумно отказываешь требующему у тебе вспоможенія, самъ отъ другаго получая помощь! Видишь, какъ лукаво сердце твое, которое отрыгаетъ слово: \textit{что мнѣ до него нужды}! Притомъ внимай, что изъ того слѣдуетъ, когда въ нуждѣ отказываешь ближнему? Тебѣ до него нужды нѣтъ; другій подобный тебѣ тоежде скажетъ; третій такожде отречется; а онъ въ томъ бѣдствіи погибнетъ: чтожъ оттуду? Берегись, чтобы какъ тебѣ, такъ и другимъ, отрекшимся его, погибель его праведнымъ судомъ Божіимъ не причтена была. О сколько на свѣтѣ убійцъ, которые не руками, но своимъ немилосердіемъ, жестокосердіемъ и такъ безчеловѣчнымъ отказываніемъ убиваютъ людей, но думаютъ о себѣ, что они чисты въ томъ! Не токмо бо тотъ убійца судится, кто какъ нибудь ближняго убиваетъ, но и тотъ, кто бѣдствующаго отъ смерти не избавляетъ, а можетъ избавить: понеже ежели бы онъ подалъ руку помощи бѣдствующему, не постигла бы его смерть. Ибо когда запрещается не убивать намъ брата нашего, то тѣмъ самымъ повелѣвается сохранять животъ его, который сохраняется помощію нашею. Бѣдствуешь убо самъ душею, когда брату бѣдствующему не подаешь помощи, а можешь. Но притомъ еще осмотрись, надлежишь ли до духовнаго онаго и благословеннаго тѣла Хрістова, которое есть церковь святая, когда такое о себѣ и ближнемъ мнѣніе имѣешь, хотя устами и исповѣдуешь Хріста? Въ единомъ тѣлѣ членъ члену помогаетъ: рука руку моетъ, глазъ глазу, ухо уху, нога ногѣ и прочіи другъ другу помогаютъ и охраняютъ: тако и хрістіанамъ, которые духовное Хрістово тѣло составляютъ, должно не токмо другъ друга не отрицаться, но и другъ другу помогать. "--- 2)~Есть многимъ обычай просящаго отсылать къ Богу: \textit{Богъ"=де дастъ}. Сіе, по видимому, кажется, не противно, понеже Богъ все даетъ, и даетъ туне. Но какъ возмешь, хрістіанине, въ разсужденіе слово сіе, то увидишь, что оно тоежъ значитъ, что и вышепомянутое слово: \textit{что мнѣ до него нужды}? Ибо отсылаешь просящаго къ Богу не ради того, чтобы Богъ ему подалъ, но ради того, что самъ не хощешь подать, и потому отъ себе отсылаешь, а не къ Богу посылаешь. А хотя и подлинно отсылаешь къ Богу, то и тогда неправильно дѣлаешь, ибо Богъ его посылаетъ къ тебѣ, и тебѣ велитъ ему подать: \textit{просящему у тебе дай}\footnote{Матѳ.~5,~42.}; а ты обратно его отсылаешь къ Богу: \textit{Богъ дастъ}, и тако заповѣдь Божію преступаешь. Богъ благая Своя подалъ тебѣ, чтобы ты и самъ нужды свои исправлялъ и требующимъ удѣлялъ во славу Его. Но ты, самъ только довольствуяся, требующихъ отсылаешь: \textit{Богъ дастъ}. И такъ Богу, благихъ Дателю, неблагодарнымъ себе показуешь. "--- Ежели бы кто сказалъ: \textit{я бы"=де подалъ просящему, да самъ не имѣю}. "--- Таковому \textit{отвѣщаю}: 1)~Имѣешь ли что подать, или не имѣешь, о томъ Богъ сердцевѣдецъ знаетъ, предъ Которымъ имѣніе твое и неимѣніе извѣстно. 2)~Здѣ рѣчь о такихъ, которые имѣютъ что подать, но отсылаютъ отъ себе требующихъ праздныхъ, а не о такихъ, кои сами нищи. 3)~Милостыня, какъ любве плодъ, не токмо въ дѣйствительномъ подаяніи состоитъ, но и въ святомъ къ подаянію желаніи, сердечномъ сожалѣніи и состраданіи бѣдному (какъ въ параграфѣ слѣдующемъ увидишь), и во всякомъ служеніи, какъ"=то: во введеніи страннаго въ домъ, въ посѣщеніи больнаго и въ темницѣ сидящаго, и въ послуженіи имъ, и проч. Наконецъ заключаю сіе мое разсужденіе ученіемъ святаго Василія великаго, который тако о немилостивыхъ глаголетъ: «Знаю я нѣкоторыхъ, которые постились, молились, воздыхали и прочую добродѣтель показали, но ни единаго пѣнязя нищимъ не подали: какая отъ того польза?»\footnote{Сл. къ богатымъ.}.

\paragraph*{§\:262.} Что есть милость, и какъ ее должно дѣлать? "--- 1)~Милость есть, когда дѣлается добро недостойнымъ, то"=есть, тѣмъ, которые никакъ у насъ того не заслужили: иначе не милость, но воздаяніе будетъ. Тако не называемъ милостію того, что дается наемникамъ за трудъ, но мздою, развѣ что сверхъ заслуженнаго дается; напр., трудъ наемника недостоинъ болѣе какъ пятидесяти копѣекъ, а дастся сто копѣекъ: тогда такое даяніе милость есть. "--- 2)~Когда дѣлается добро безъ всякія чаемыя корысти отъ того, кто пріемлетъ оное. Ибо \textit{любы}, которыя есть плодъ милость, \textit{не ищетъ своихъ}\footnote{1~Кор.~13,~5.}: иначе будетъ торгъ, или взаимодаяніе. Отдаяніе бо не въ томъ только состоитъ, чтобы тѣмъ самымъ возвратилъ должникъ заимодавцу, чѣмъ отъ него взялъ, но и въ томъ, чѣмъ бы ни возвращено было взятое. Такъ, когда кто дѣлаетъ кому добро, а чаетъ отъ него чего нибудь, тѣмъ самымъ какъ бы взаимъ ему даетъ. "--- 3)~Милость есть истинная, которая дѣлается не съ принужденія, но отъ любовнаго произволенія, усердія и сердечнаго сожалѣнія надъ требующимъ милости. \textit{Доброхотна бо дателя любитъ Богъ}, глаголетъ апостолъ\footnote{2~Кор.~9,~7.}. Тако не почитается за милость хозяину, у котораго солдатъ квартирою въ силу указа монаршаго упокоевается, ибо указъ ему даетъ квартиру, а не самъ онъ добровольно; понеже ежели бы онъ указа не боялся, то можетъ быть и не допустилъ бы его въ домъ свой. Кто отъ добраго произволенія пріемлетъ странныя, тотъ, хотя бы и указа не было, доброхотно принялъ бы, что называется страннолюбіе и истинная милость. "--- 4)~Милость судится не отъ количества подаваемыхъ, но отъ качества дающаго, то"=есть, съ какимъ кто усердіемъ и произволеніемъ подаетъ. «Не мѣрою даемыхъ, но силою и произволеніемъ дающихъ, глаголетъ Златоустъ святый, судится милостыни величество»\footnote{Бес.~1"~я на посл. къ Евр.}. И паки тойжде отецъ: «Не сущу усердію, ниже тмы талантъ злата сіе возмутъ, еже двѣ лептѣ со усердіемъ»\footnote{Бес.~15"~я на посл. къ Филип.}. "--- 5)~За милость почитается не токмо дѣло милости сотворенное или творимое, но и самое усердіе и желаніе къ творенію милости. Милость бо, какъ и всякая добродѣтель, на сердцѣ имѣетъ мѣсто, и потому хотя въ дѣло и не производится, однакожъ почитается такъ, какъ бы дѣломъ самымъ исполнена была. Какъ бо не токмо тотъ только хищникъ, который самымъ дѣломъ похищаетъ; но и тотъ, который хощетъ похитить, но не можетъ: такъ не токмо тотъ милостивъ, который дѣлаетъ милость, но и тотъ, который хощетъ дѣлать милости, но не можетъ, "--- хощетъ подать просящему, но не имѣетъ что дать, "--- хощетъ помощи бѣдствующему, но не можетъ. По сердцу бо и намѣренію всякое дѣло судится. «Аще и ничесоже имаши, глаголетъ Златоустъ святый, соболѣзнующую же имаши душу: и се тебѣ мзда соблюдена будетъ. Сего ради и Павелъ, завѣщая плакати съ плачущими, и съ юзниками аки связаннымъ быти повелѣлъ». И ниже: «и Хрістосъ, егда ублажаетъ милостивыя, не точію имѣніями милующія, но и произволеніемъ, то творящія ублажаетъ и похваляетъ. Иже бо человѣколюбивъ нравъ имѣяй и милостивъ, аще имѣнія имать, произнесетъ; аще въ напастехъ увидитъ кого, восплачется и возрыдаетъ; аще обидимаго повидитъ, заступитъ», и проч.\footnote{Бес.~19"~я на посл. къ Римл.} Милость бо, какъ сказано, есть плодъ любве неотлучный и всегда съ нею сопряженный, какъ теплота съ огнемъ, и какъ огонь теплоту, такъ любовь милость издаетъ. Сія теплота любве въ случаяхъ себе оказываетъ. Милостивый не отречется бѣдствующему помощи, когда можетъ; когда не можетъ, соболѣзнуетъ бѣдствію его. Милостивый не отречется нагаго одѣть, когда имѣетъ чимъ; не отречется страннаго въ домъ пріять и удовольствовать, печальнаго утѣшить, больнаго посѣтить, бѣдствующему руку помощи подать; а когда не силенъ того учинить, сердечное желаніе и усердіе къ тому имѣетъ. Не всякъ бо все можетъ. Кто можетъ помощи, но не хощетъ, тотъ подлинно милости не имѣетъ; можетъ просящему дать, но не хощетъ дать, и проч.; тутъ милость мѣста своего не имѣетъ, но немилосердіе. "--- 6)~Милость должно дѣлать не ради того, чтобы похвалу отъ человѣкъ имѣть, но во имя Хрістово, во славу Божію и пользы ради ближняго, чтобы имя Божіе, а не подающаго славилося, и ближній требующій милости созидался. Сіе намѣреніе должно быть дающаго милостыню. "--- 7)~Милостыню должно подавать не отъ похищенія и неправды, но отъ своихъ имѣній, Богомъ данныхъ намъ. Ибо у единаго отнимать и другому давать не милость есть, но безчеловѣчіе и прельщеніе совѣсти или умягченіе, которая, неправдою раздраженна, не престаетъ обличать хищника. Такая милостыня есть мерзость предъ Богомъ: \textit{якоже жряй сына предъ отцемъ его, тако приносяй жертвы отъ имѣнія убогихъ}, глаголетъ Сирахъ\footnote{Гл.~34,~20.}. Таковому должно учинить, что Закхей мытарь кающійся обѣщался предъ Хрістомъ учинить\footnote{Лук.~19,~8.}. То"=есть: что злѣ собралъ, добрѣ расточить, обиженныхъ наградить, впредь неправды не дѣлать, и что неправдовалъ и законъ Божій разорялъ, каяться и впредь благословенными трудами питаться. "--- 8)~Милость должно дѣлать всѣмъ, то"=есть, сродникамъ и несродникамъ, другамъ и врагамъ, всякому, кто ни требуетъ милости. Истинная бо любовь и надъ врагами бѣдствующими умилостивляется: она не на лице, но на бѣдность смотритъ. "--- 9)~Милость должно дѣлать, смотря на нужду требующаго. У кого большая нужда, большую должно и милость показать: алчущаго напитать копѣйка или двѣ могутъ: нагаго одѣть, хижину построить неимущему гдѣ главу подклонить "--- не двухъ копѣекъ требуется. Однакожъ кто сколько можетъ, смотря по своимъ достаткамъ. У кого болѣе прихода и достатка, болѣе долженъ и удѣлять требующимъ. "--- 10)~Милостію сотворенною не должно возноситься. Понеже, что ни имѣемъ, не свое имѣемъ, но Божіе, и ради того, что даемъ, Божіе даемъ, а не наше собственное. Строители бо есмы, которые повѣреннаго отъ Него намъ добра по воли Его должны расходъ чинить; раби есмы, которые должны дѣлать тое, что Господь нашъ хощетъ и повелѣваетъ. Горе же намъ, когда не слушаемъ Господа нашего и не дѣлаемъ повелѣннаго! Тако рогъ высокоумія можетъ сломиться. "--- 11)~Сдѣланное добро, сколько возможно, забывать должно, дабы не восхитила \textit{шуица, что творила десница}. Ибо едина о грѣхахъ память намъ полезна, понеже смиряетъ насъ. А добродѣтелей память не полезна, яко въ высокоуміе фарисейское приводитъ, что показуетъ притча о мытарѣ и фарисеѣ\footnote{Лук.~18,~10--14.}. Призываетъ Хрістосъ въ вѣчное Свое царствіе благословенныхъ Своихъ, и ублажаетъ ихъ, глаголя: \textit{взалкахся, и дасте Ми ясти}, и проч.; но они добродѣтелей своихъ не помнятъ и признаются, какъ бы ихъ и не дѣлали: \textit{Господи, когда Тя видѣхомъ алчуща, и напитахомъ; или жаждуща, и напоихомъ? когда же Тя видѣхомъ странна, и введохомъ; или нага, и одѣяхомъ? когда же Тя видѣхомъ боляща, или въ темницѣ, и пріидохомъ къ Тебѣ}\footnote{Матѳ.~25,~34--39.}? Тѣмъ показуется, что и намъ добрая дѣла должно дѣлать, но содѣланныя забывать. Ктомужъ должно твердо держать, что мы сами собою безъ Хрістова Духа не можемъ никакого добра дѣлать: \textit{безъ Мене}, глаголетъ Хрістосъ, \textit{не можете творити ничесоже}\footnote{Іоан.~15,~5.}. Почему всякое добро истинное отъ Бога происходитъ, и доброе дѣло есть Божіе дѣло, Богу его и приписывать должно: Тому единому, какъ благихъ Источнику и Дателю благодареніе, честь и слава подобаетъ.

\paragraph*{§\:263.} Понеже нищета немалую скорбь наипаче тѣмъ, которые съ женами и дѣтьми въ нищетѣ живутъ и подушныя и оброки платятъ, приноситъ: того ради должно имъ въ семъ бѣдствіи не инымъ чимъ, какъ терпѣніемъ себе укрѣпить, на волю Божію отдаться, и тако милости Его ожидать. Однакожъ да внимаютъ слѣдующему предложенію: 1)~Утѣшеніе нищимъ можетъ быть отъ сравненія богатаго съ нищимъ. Хотя сыны вѣка сего и полагаютъ блаженство въ богатствѣ; но когда на состояніе богатаго и нищаго посмотримъ, то увидимъ, что сей блаженнѣшій есть паче онаго. Блаженство бо состоитъ не во многомъ стяжаніи, но въ покоѣ душевномъ; не въ томъ, чтобы много имѣть, но въ томъ, чтобы тѣмъ довольствоваться, что имѣемъ. Смотри на богатаго и нищаго: богатый проживетъ дни свои, и нищій проживетъ. Богатый живетъ въ богатыхъ покояхъ, богатымъ столомъ насыщается, богатою одеждою одѣвается: нищій въ хижинѣ, хлѣбомъ съ водою питается, въ рубищахъ ходитъ, но такожде дни проводитъ, какъ и богатый. У богатаго и нищаго вчерашній день прошелъ, тоежъ и впредь будетъ. Богатый имѣетъ сокровище, но съ сокровищемъ имѣетъ и попеченіе, страхъ и боязнь, чтобы его не лишиться: нищій не имѣетъ сокровища, но не имѣетъ и страха. Богатый боится воровъ и разбойниковъ: нищій отъ того свободенъ. Богатый съ богатствомъ часто и живота отъ злодѣевъ лишается: нищій того не боится, ибо на нищаго не нападаетъ разбойникъ. Богатому многіе завидятъ, какъ самое дѣло показуетъ; а кому завидятъ многіе, тому много и навѣтовъ бываетъ, зависть бо безъ того не бываетъ: нищій того не опасается; кто бо нищетѣ позавидитъ? Богатый осужденію и злословію подлежитъ ради скупости и немилосердія: нищій того не знаетъ; ибо кто чего не имѣетъ, того и не даетъ. Богатый долженъ расходъ богатства по воли Божіей чинить, иначе предъ судомъ Божіимъ будетъ виноватъ: нищій отъ того свободенъ. При богатствѣ почти всегда неотлучна гордость, начало всякаго грѣха: нищета учитъ смиренію. Богатство или скупостію связуетъ, или роскошію и сластолюбіемъ разслабляетъ сердца богатыхъ: въ нищетѣ того нѣтъ; что бо нищему давать, когда ничего самъ не имѣетъ? или чимъ роскошно жить, когда не достаетъ? Богатому слѣдуетъ отвѣтъ дать предъ Богомъ о расходѣ богатства: нищій о томъ не думаетъ. Богатый предъ смертію разсуждаетъ, кому и какъ оставить богатство: нищій о томъ не печется. Богатый, умирая, оставляетъ богатство: нищій какъ не имѣлъ, такъ и не оставляетъ. Богатому и при смерти не попускаетъ богатство пещися о души своей: но нищій о единой душе своей подвизается. О богатыхъ глаголетъ Хрістосъ: \textit{аминь глаголю вамъ, яко неудобь богатый внидетъ въ царствіе небесное}\footnote{Матѳ.~19,~23.}. О нищихъ не тое; но что? \textit{Слышите}, глаголетъ апостолъ, \textit{братія моя возлюбленная, не Богъ ли избра нищія міра сего, богаты въ вѣрѣ и наслѣдники царствія, еже обѣща любящимъ Его}\footnote{Іак.~2,~5.}. "--- Да не помыслитъ кто, что богатство само собою погубляетъ, и нищета сама собою спасаетъ! Но какъ невѣріе съ плодами своими къ пагубѣ приводитъ, такъ \textit{вѣра, любовію поспѣшествуема}, дверь къ вѣчному животу отворяетъ. Всѣмъ Богъ хощетъ спастися, богатымъ и нищимъ; но по большей части нищіе сей спасительной Божіей воли повинуются; а богатые, прилѣпившеся суетѣ міра сего, отрицаются ея. Отсюду слышимъ, что званніи на вечерю велію \textit{начаша вкупѣ отрицатися}, вси житейскими попеченіями запутанніи. Иной сказалъ: \textit{село купихъ, и имамъ нужду изыти и видѣти е: молютися, имѣй мя отречена}. Другій рече: \textit{супругъ воловъ купихъ пять, и гряду искусити ихъ: молю тя, имѣй мя отречена}. И другій рече: \textit{жену пояхъ, и сего ради не могу пріити}\footnote{Лук.~14,~16--20.}. Чѣмъ не иное что означается, какъ пристрастіе къ земнымъ и тлѣннымъ вещамъ, презрѣніе слова Божія и преслушаніе воли Божіей, хотящія всѣмъ спастися. Напротивъ того, нищіи хотя и воспящаются отъ спасительнаго пути отъ общихъ враговъ, діавола, плоти и міра, но много имъ нищета противу ихъ помогаетъ, какъ выше сказано. Видиши, коль блаженнѣйшая нищета отъ богатства, хотя люди за симъ гоняются, а оныя убѣгаютъ! "--- 2)~Утѣшеніе нищіе въ нищетѣ своей получаютъ отъ имени Хрістова, которое, хотя многіе богатые, скупостію плѣненные, презираютъ, однакожъ многіе съ радостію и усердіемъ въ слухъ и сердце пріемлютъ, и тако отворяютъ утробы щедротъ просящимъ ради Его. Сіе имя имъ паче всякаго сокровища міра сего; сіе имя ихъ питаетъ, одѣваетъ, напояетъ и упокоеваетъ; съ симъ именемъ вездѣ имъ домъ, трапеза, одежда, утѣшеніе, гдѣ только ни имѣются хрістолюбивыя сердца. Богатый часто приходитъ въ скудость; но нищій съ именемъ Хрістовымъ всегда равенъ, доволенъ и богатъ. На сіе дражайшее сокровище надѣятися, симъ утѣшатися должны нищіи. "--- 3)~Хотя Богъ, яко щедръ и милостивъ, о всѣхъ промышляетъ, но о нищихъ, яко безпомощныхъ, наипаче. Ибо называется \textit{Отецъ сирыхъ и Судія вдовицъ}\footnote{Пс.~67,~6; Іоан.~14,~18.}. И паки: \textit{сира и вдову пріиметъ}\footnote{145,~9.}. Почему на многихъ Писанія мѣстахъ читаемъ, что о нихъ намъ пещися, имъ помогать, имъ просящимъ подавать повелѣваетъ, и вмѣняетъ тое Себѣ, что нищимъ доброхотно подается. \textit{Милуяй нищаго взаимъ даетъ Богови}\footnote{Притч.~19,~17.}. И Хрістосъ глаголетъ: \textit{понеже сотвористе единому сихъ братій Моихъ меншихъ, Мнѣ сотвористе}\footnote{Матѳ.~25,~40.}. "--- 4)~\textit{Хрістосъ Сынъ Божій, богатъ сый, обнища, да мы нищетою Его обогатимся}\footnote{2~Кор.~8,~9.}. Отъ сего великое утѣшеніе нищимъ проистекаетъ, что Хрістосъ, у Котораго весь міръ въ руцѣ, Царь небесе и земли, \textit{волею обнища}, такъ что \textit{Сынъ человѣческій}, какъ Самъ о Себѣ сказалъ, \textit{не имѣлъ гдѣ главы подклонити}\footnote{Матѳ.~8,~20.}. Ибо нищетою своею Ему сообразуются, когда ее съ благодареніемъ терпятъ. "--- 5)~Какъ богатые, такъ и нищіе не по случаю бываютъ, но какъ \textit{богатство}, такъ и \textit{нищета отъ Господа суть}\footnote{Сир.~11,~14.}. Убо, что отъ Бога есть, не можетъ быть зло, но добро. "--- 6)~Какъ всякое бѣдствіе, такъ и нищету можетъ Богъ Своимъ человѣколюбіемъ или прекратить, или облегчить, "--- что и самымъ дѣломъ бываетъ, когда \textit{воздвизаетъ отъ земли нища, и отъ гноища возвышаетъ убога, посадити его съ князи, съ князи людей Своихъ}\footnote{Пс.~112,~7 и 8.}. Тако воздвиглъ Іосифа изъ темницы, и посадилъ выше князей египетскихъ. Тако \textit{избра Давида раба Своего, и восприятъ его отъ стадъ овчихъ}\footnote{77,~70.}. "--- 7)~Наконецъ какъ всякое бѣдствіе, такъ и нищета смертію окончается. Всему конецъ смерть "--- богатству и нищетѣ; и богатыхъ утѣшеніе, и нищихъ терпѣніе смертію заключается.

\paragraph*{§\:264.} Когда здѣ предлагается утѣшеніе нищимъ, то тѣ только разумѣются нищіи, которые подлинно нищетствуютъ по Божію, вся намъ на пользу строящему, промыслу; или которые расточаютъ имѣнія и даютъ убогимъ, и тако самопроизвольно изволяютъ ради имени Хрістова нищету терпѣть, и въ слѣдъ Хріста обнищавшаго насъ ради ходить. Сего ради отъ сего утѣшенія выключаются: 1)~Тѣ нищіи, которые могутъ трудитися и трудами себе съ домашними питать, но не хотятъ, и тако въ праздности и лѣности живучи, не имѣютъ чимъ себе довольствовать, и хотятъ чужими трудами питаться. Сіи сами своей нищетѣ виновны; сіи не утѣшенія, но суда Божія, яко тунеядцы, ожидать должны, когда не покаются: ибо противу заповѣди Божіей дѣлаютъ, которою повелѣвается намъ \textit{въ потѣ лица нашего искать хлѣба себѣ}\footnote{Быт.~3,~19.}. "--- 2)~Которые, имѣя чимъ довольствоваться, притворяютъ себѣ нищету, въ рубища одѣваются, чтобы человѣколюбіе къ себѣ привлещи. Сіи обманщики, воры и хищники суть, а не нищіи; понеже и подающихъ обманываютъ, и подлинно нищихъ окрадываютъ; ибо которая милостыня истинно нищетствующимъ имѣла быть подана, тую они ухищреннымъ вымысломъ похищаютъ. Погрѣшаютъ они противу заповѣди Божіей: \textit{не укради}, и тако, яко татіе, суду Божію подлежатъ, когда не исправятся. "--- 3)~Которые поданную ради имени Хрістова милостыню издерживаютъ на піянство и прочіе пагубные расходы. Сіи такожде, какъ обманщики и воры и піяницы, суда Божія не избѣгнутъ, когда не очувствуются. "--- 4)~Сюды надлежатъ и тѣ нищіи, которые, не терпя нищеты, устремляются на воровство, хищеніе, разбой и прочія, закономъ Божіимъ запрещенныя, дѣла. "--- 5)~Въ семъ числѣ заключаются и тѣ нищіи, которые, не терпя, ропщутъ противу Бога и хулятъ. И прочіе симъ подобные отъ вышеписаннаго утѣшенія исключаются.

\subsection[Глава 12-я. О отпущеніи согрѣшеній ближнему.]{глава втораянадесять.\\\bfseries О отпущеніи согрѣшеній ближнему.}

\begin{quotation}\textit{Отпущайте, и отпустятъ вамъ}, глаголетъ Хрістосъ\footnote{Лук.~6,~37.}.\end{quotation}
\begin{quotation}\textit{Аще отпущаете человѣкомъ согрѣшенія ихъ, отпуститъ и вамъ Отецъ вашъ небесный. Аще ли не отпущаете человѣкомъ согрѣшенія ихъ, ни Отецъ вашъ отпуститъ вамъ согрѣшеній вашихъ}, глаголетъ Хрістосъ\footnote{Матѳ.~6,~14 и 15.}.\end{quotation}


\paragraph*{§\:265.} Живущимъ въ обществѣ трудно другъ къ другу не согрѣшить какъ нибудь, или словомъ, или дѣломъ или инымъ какимъ образомъ. Ибо сатана всегда и вездѣ плевелы свои пагубныя сѣетъ, и подстрекаетъ насъ другъ на друга, и тщится миръ и любовь братскую между нами нарушить, а ненависть и вражду всѣять. Сего ради нужно есть намъ другъ другу \textit{оставлять согрѣшенія}.

\paragraph*{§\:266.} Причины, возбуждающія къ отпущенію согрѣшеній ближнему. 1)~Богъ повелѣваетъ оставлять ближнему обиды; убо одолжаемся, по повелѣнію Божію, чинить тое. Иначе грѣшимъ противу повелѣвающаго Бога, яко не слушаемъ Его и противимся Ему, что тяжко и страшно. "--- 2)~Требуетъ того хрістіанская любовь, дабы брату нашему, который немощію естественною убѣжденъ, и діавольскимъ дѣйствомъ поощренъ, согрѣшилъ намъ, "--- не мстить, но умилосердившися простить, чтобы отъ мщенія зла не пострадалъ, и мы бы о томъ послѣ не жалѣли, что брату бѣду причинили. Ибо часто бываетъ, что какъ обидѣвшій, такъ и отмстившій жалѣетъ о содѣянномъ дѣлѣ, но уже содѣяннаго не возвратитъ. Чего ради напредь тое предусматривать должно, и не попущать гнѣву въ ненависть и злобу возрастать; но тотчасъ начинающее куриться зло угашать духомъ кротости и человѣколюбія. "--- 3)~Да подвигнетъ насъ ко взаимному прощенію и другъ друга помилованію милосердіе Божіе, которое на всякъ часъ намъ согрѣшающимъ является, которое такъ велико мы въ себѣ дознаемъ, что, хотя бы на всякій день и часъ ближній нашъ какъ ни согрѣшилъ намъ, должны мы ему оставлять ради того человѣколюбія Божія, намъ показуемаго. Ибо какое бы ни было согрѣшеніе ближняго нашего къ намъ, то какъ капля противу моря, или ничто въ сравненіи съ нашимъ предъ Богомъ согрѣшеніемъ, которымъ мы всякій день и часъ предъ Нимъ виновны находимся. Вси бо мы, кто бы ни были, человѣцы есмы, грѣшницы есмы, земля и пепелъ есмы; въ сравненіи предъ Богомъ, предъ Которымъ весь свѣтъ какъ капля, мы аки ничтоже есмы. Велико ли убо есть, когда человѣкъ человѣку согрѣшаетъ, и грѣшникъ грѣшнику прощаетъ? Велико воистину и страшно дѣло есть человѣку, рабу, грѣшнику, персти, предъ великимъ, безконечнымъ, святымъ и праведнымъ Богомъ согрѣшать, и отъ Него прощеніе кающемуся получать, *и не токмо прощеніе получать*, но и милости несказанныя сподобляться! Сего милосердія Божія не токмо слово изрещи, но и умъ понять не можетъ. Да постыдится убо и умолкнетъ гордость и злоба человѣческая, за слово поносное брата погубить ищущая, когда Богъ великій и святый такъ кротокъ, долготерпѣливъ и милостивъ согрѣшающимъ намъ! "--- 4)~Да подвигнетъ къ тому насъ великая польза наша, которую оставляющіи ближнему обиды пріобрѣтаютъ. Ибо кто оставляетъ согрѣшенія ближнему, тому удобный приступъ къ благости Божіей открывается просить во имя Хрістово своимъ согрѣшеніямъ прощенія: \textit{аще отпущаете}, глаголетъ Хрістосъ, \textit{человѣкомъ согрѣшенія ихъ, отпуститъ и вамъ Отецъ вашъ небесный}. Напротивъ того, заключается тому дверь къ полученію милосердія Божія и прощенія своихъ согрѣшеній, кто ближнему согрѣшеній не оставляетъ, какъ Хрістосъ Господь нашъ явно о семъ объявляетъ: \textit{аще не отпущаете человѣкомъ согрѣшенія ихъ, ни Отецъ вашъ отпуститъ вамъ согрѣшеній вашихъ}. А что отсюду послѣдуетъ тому, кому грѣхи не отпустятся? Не иное что, какъ строгій Божій судъ и праведный, вѣчное осужденіе, вѣчное во адѣ мученіе, которымъ за преступленіе закона Божія безъ конца будутъ платить осужденные, но никогда не заплатятъ правосудію Божію. Читай притчу у Матѳея въ главѣ 18"~й. Что дѣлается тамо съ тѣмъ неблагодарнымъ и лукавымъ рабомъ, которому царь его тму талантовъ отпустилъ, а онъ клеврету своему и ста пѣнязей отпустить не хотѣлъ? \textit{Тогда, призвавъ его господинъ его, глагола ему: рабе лукавый! весь долгъ оный отпустихъ тебѣ; понеже умолилъ мя еси: не подобаше ли и тебѣ помиловати клеврета твоего, якоже азъ тя помиловахъ? И разгнѣвався господь его, предаде его мучителемъ, дондеже воздастъ весь долгъ свой}\footnote{ст.~32 и 34.}. Такое воспослѣдуетъ отъ правды Божіей опредѣленіе и тѣмъ лукавымъ людемъ, которые, Богу безчисленнымъ согрѣшеній долгомъ виноваты, ближнему не хотятъ оставить и малаго. \textit{Тако и Отецъ Мой небесный}, заключаетъ Господь притчу, \textit{сотворитъ вамъ, аще не отпустите кійждо брату своему отъ сердецъ вашихъ прегрѣшенія ихъ}\footnote{ст.~35.}. А какъ тако? Вотъ какимъ образомъ: не оставляешь ты брату твоему, то"=есть, всякому человѣку, малаго долга согрѣшеній, какъ \textit{ста пѣнязей}: не оставитъ и тебѣ Богъ \textit{тмы}, то"=есть, десяти тысящей \textit{талантовъ}, противу котораго долга сто пѣнязей долгъ, какъ ничто; то"=есть, согрѣшенія, которыми ближній твой тебе оскорбилъ, въ сравненіи съ грѣхами, которыми ты величество Божіе прогнѣвалъ, за ничто вмѣняться должны. Ибо ты въ сравненіи съ величествомъ Божіимъ какъ ничто; и не токмо единъ ты, но и весь свѣтъ предъ Нимъ, какъ капля воды противу океана: потому и оскорбленіе твое въ сравненіи съ оскорбленіемъ Божіимъ какъ ничтоже. Аще убо ты такъ малаго долга должнику твоему не отпущаешь: не отпуститъ и тебѣ Богъ великаго твоего долга. Ты человѣкъ человѣку подобному не прощаешь: и какого прощенія отъ великаго и праведнаго Бога чаешь? Ближній твой тебе, подобнаго себѣ, оскорбилъ, и ты не прощаешь: ты безконечное величество Божіе грѣхами оскорбилъ и оскорбляешь, и хощешь получить прощеніе? Влечешь на судъ ближняго твоего: будетъ судить и тебе Богъ. Мстишь ты ближнему твоему: будетъ мститъ и тебѣ Богъ. Предаешь въ темницу клеврета твоего за малый его долгъ: предастъ Богъ и тебе за великій твой долгъ въ темницу, гдѣ чрезъ всю вѣчность будешь платить долгъ грѣховъ твоихъ, но никогда не заплатишь. Откуду святый Златоустъ глаголетъ: «Нѣтъ ничего безопаснѣйшаго, какъ врагу простить; и нѣтъ ничего опаснѣйшаго, какъ оному мстить»\footnote{Бес.~2"~я о Давидѣ и Саулѣ.}. Страшно убо злобиться на ближняго, страшно мстить, страшно суду предавать! Разбойникамъ, убійцамъ, блудникамъ, мытарямъ и всякимъ грѣшникамъ кающимся отворяются двери Божія милосердія, а злобнымъ заключаются: ибо нѣтъ въ нихъ истиннаго покаянія, безъ котораго къ престолу благодати приступу нѣтъ. Злоба бо есть великій грѣхъ, который ими обладаетъ и покаяніе ихъ недѣйствительнымъ дѣлаетъ. Понеже покаяніе не есть истинное, но притворное, ложное, и не иное что, какъ прельщеніе, или паче умягченіе грызущія совѣсти, когда отъ грѣха кающійся чистосердечно отстать не хощетъ. "--- 5)~Не можно тому и молиться Богу, кто ближнему не оставляетъ согрѣшеній. Какъ бо скажетъ: \textit{Отче! остави намъ долги наша, якоже и мы оставляемъ должникомъ нашимъ}, "--- что въ молитвѣ Господней заключается, а самъ не оставляетъ? Богъ бо знаетъ сердце человѣческое и на сердце смотритъ, а не на слова. И тако какое сердце, такая и молитва его. Когда сердце злобы исполнено, убо и молитва есть пустая. Паче же таковый, хотя проситъ Бога устами, чтобы оставилъ ему согрѣшенія, но сердцемъ не хощетъ того оставленія, ибо самъ не оставляетъ. Богъ бо сердца слушаетъ, а не словъ. И потому злобный съ тѣмижде грѣхами отходитъ отъ молитвы, съ каковыми приходитъ къ молитвѣ, и когда еще не съ множайшими. Ибо пишется: \textit{и молитва его да будетъ въ грѣхъ}\footnote{Пс.~108,~7.}. Устрашить сіе должно тѣхъ, которые въ церковь на молитву ходятъ, которые къ олтарю приступаютъ и даръ приносятъ, и мнятся не токмо о себѣ, но и о другихъ молитвы возсылать, а злобы на ближняго оставить не хотятъ. Должно имъ опасаться, чтобы приносы ихъ и молитвы не были имъ въ грѣхъ. "--- 6)~Другъ другу согрѣшаемъ вси. Тебѣ сегодня или вчера согрѣшилъ братъ твой; а ты ему заутра или уже и прежде согрѣшилъ, что въ обществѣ живущимъ удобно приключается. Ради того другъ другу должны и прощать, яко другъ другу согрѣшающіи, а тако вражды не будетъ, и миръ нерушимъ будетъ. "--- 7)~Ежели бы вси другъ другу мстили, общество бы стоять не могло, ибо вси бы другъ друга погубили. Отъ взаимной бо вражды слѣдуетъ взаимная злоба и мщеніе, а отъ мщенія взаимная пагуба, какъ то на брани бываетъ. \textit{Аще другъ друга угрызаете и снѣдаете, блюдитеся, да не другъ отъ друга истреблени будете}, глаголетъ апостолъ\footnote{Гал.~5,~15.}. И тако злобные вредятъ и погубляютъ общество: кроткіе же пользуютъ и сохраняютъ своимъ терпѣніемъ. "--- 8)~Оскорбленія сдѣланнаго не возвратить, а мстить не малаго труда требуетъ; да и то можетъ быть не удастся. Ибо часто бываетъ, что хотящіи мстить не токмо желаемаго не получаютъ, но и въ большія бѣды впадаютъ, и въ которую яму ближняго хотятъ вринуть, въ тую сами нечаянно падаютъ. Тако Аманъ на томъ древѣ самъ погиблъ, которое неповинному Мардохею приготовилъ"=было\footnote{Есѳ.~6,~10.}. \textit{(Смотри еще главу о злобѣ и терпѣніи)}.

\subsection[Глава 13-я. О любви ко врагамъ.]{глава третіянадесять.\\\bfseries О любви ко врагамъ.}

\begin{quotation}\textit{Аще алчетъ врагъ твой, ухлѣби его; аще ли жаждетъ, напой его: сіе бо творя, угліе огненное собираеши на главу его; Господь же воздастъ тебѣ благая}\footnote{Притч.~25,~22 и 23.}.\end{quotation}
\begin{quotation}Тоежде и апостолъ: \textit{не побѣжденъ бывай отъ зла, но побѣждай благимъ злое}\footnote{Рим.~12,~21.}.\end{quotation}
\begin{quotation}\textit{Азъ глаголю вамъ: любите враги ваша, благословите кленущія вы, добро творите ненавидящимъ васъ, и молитеся за творящихъ вамъ напасть и изгонящія вы: яко да будете сынове Отца вашего, Иже есть на небесѣхъ, яко солнце Свое сіяетъ на злыя и благія, и дождитъ на праведныя и на неправедныя. Аще бо любите любящихъ васъ, кую мзду имате? не и мытари ли тожде творятъ? И аще цѣлуете други ваша токмо, что лишше творите? не и язычницы ли такожде творятъ? Будите убо вы совершени, якоже и Отецъ вашъ небесный совершенъ есть}, глаголетъ Хрістосъ\footnote{Матѳ.~5,~44--48; Лук.~6,~27--36.}.\end{quotation}


\paragraph*{§\:267.} Враги здѣ разумѣются тѣ люди, которые намъ словомъ или дѣломъ какія либо являютъ озлобленія, которые кленутъ насъ, ненавидятъ насъ, творятъ намъ напасти и изгоняютъ насъ. Сихъ намъ Хрістосъ велитъ \textit{любить}.

Понеже въ главѣ о любви къ ближнему, то"=есть всякому человѣку, сказано, что есть любовь хрістіанская, и какіе ея плоды, того ради здѣ о томъ не предлагается; тамо смотри, читатель. Ибо между ближними нашими не токмо пріятели и други наши, но и враги наши разумѣются, которыхъ Хрістосъ намъ повелѣваетъ любить: \textit{любите враги ваша}. Но предлагаются здѣ токмо причины, которыя возбуждаютъ насъ къ любленію враговъ нашихъ: 1)~Хотя бы мы и не знали другой какой причины, для коей любить намъ враговъ нашихъ, однакожъ сіе едино, что Хрістосъ повелѣваетъ намъ любить ихъ, возбудить къ тому должно насъ, "--- что и о всякой Его святой заповѣди разумѣть должно. Хрістосъ хощетъ и велитъ намъ любить враговъ нашихъ. Хрістосъ "--- \textit{вѣчная Истина}, волю небеснаго Отца намъ открывшая, Котораго небесный Отецъ повелѣваетъ намъ слушать: \textit{Того послушайте}\footnote{17,~5.}. Убо, что Хрістосъ Сынъ Божій повелѣваетъ, того воля небеснаго Отца отъ насъ хощетъ, вѣчная Божія правда требуетъ, тое намъ полезно. "--- 2)~Ежели возмемъ въ разсужденіе, что"=то есть Хрістосъ, Который намъ глаголетъ: \textit{любите враги ваша}; то не токмо на сіе Его повелѣніе, но и на самую смерть за имя Его готовыхъ себе всеохотно показать должны мы. Хрістіанамъ должно знать, что Хрістосъ есть высочайшій Любитель нашъ, какого большій быть не можетъ, Избавитель, Искупитель, Ходатай и Примиритель нашъ къ Богу, вѣчная надежда и упованіе наше. Сей нашъ такъ высокій Благодѣтель повелѣваетъ: \textit{любите враги ваша}. Ежели бы ты попался въ такое несчастіе, что по законамъ и монаршему опредѣленію слѣдовала тебѣ смертная казнь; а сыскался бы такой добрый человѣкъ, который бы тебе не токмо отъ той казни избавилъ своимъ ходатайствомъ и различнымъ своимъ бѣдствіемъ, но и въ высокую монаршую милость привелъ: ты бы все, что тебѣ ни повелѣно отъ него было, по воли его безъ сумнѣнія исполнялъ; развѣ бы крайне неблагодарнымъ восхотѣлъ явитися къ такъ великому благодѣтелю. Но сія любовь, сіе благодѣяніе, хотя и великое есть, однакожъ въ сравненіи съ Хрістовою любовію, которую намъ показалъ, какъ ничтоже есть, яко временное, и смерти сей по естественному закону никому миновать невозможно. Ибо Хрістосъ не отъ временной, но отъ вѣчной смерти \textit{избавилъ насъ: не истлѣннымъ сребромъ или златомъ, но честною Своею кровію искупилъ насъ отъ той вѣчной горести}\footnote{Петр.~1,~18 и 19.}. \textit{Хрістосъ умре грѣхъ ради нашихъ}\footnote{1~Кор.~15,~3.}. И не токмо отъ того бѣдствія избавилъ, но \textit{елицы пріяша Его, даде имъ область чадомъ Божіимъ быти, вѣрующимъ во имя Его}\footnote{Іоан.~1,~12.}, и наслѣдіе вѣчнаго царствія отворилъ. Сія высочайшая Его къ намъ любовь требуетъ отъ насъ, чтобы не токмо друговъ, но и враговъ нашихъ по заповѣди Его любить тщались, когда хощемъ Ему благодарны быть. Открый Евангеліе Его святое, и тое представитъ тебе любовь Его сію. Оно покажетъ тебѣ, что Онъ насъ ради зракъ раба принялъ, обнищалъ, не имѣлъ гдѣ главы подклонить, трудился, отъ града во градъ, отъ веси въ весь ходилъ, проповѣдуя Евангеліе царствія, плакалъ, болѣзновалъ, гоненіе, поношеніе, хулы, заушеніе, оплеваніе, посмѣяніе, терновое вѣнчаніе претерпѣлъ, страдалъ, умеръ и погребенъ былъ. Вся сія неповинно и ради единой къ намъ любви претерпѣлъ, и тако насъ отъ смерти, ада и діавола искупилъ, и небесному Своему Отцу въ милость привелъ. Сія Его толикая любовь можетъ и должна умягчить и убѣдить сердце наше къ любленію враговъ нашихъ. Человѣка"=благодѣтеля, ежели бы отъ временной смерти избавилъ насъ, во всемъ бы слушали мы, ежели бы хотѣли ему благодарни быть: не много ли паче достоинъ того Сынъ Божій, Который насъ смертію Своею отъ вѣчной смерти искупилъ и къ вѣчному блаженству дверь намъ отворилъ? Кто любить хощетъ Хріста, Искупителя своего, тотъ слово Его соблюдать будетъ. \textit{Аще кто любитъ Мя}, глаголетъ Онъ, \textit{слово Мое соблюдетъ}\footnote{Іоан.~14,~23.}. Иначе никакой любви ко Хрісту не имѣетъ, что бы ни дѣлалъ и исповѣдывалъ о себѣ. Аще убо любви ко Хрісту не имѣетъ, кто враговъ своихъ не любитъ: что уже сказать о тѣхъ, которые не любятъ никакого зла недѣлающихъ имъ? О какъ далеко отстоятъ отъ хрістіанства и Хріста таковые! "--- 3)~Самъ Онъ насъ, враговъ Своихъ, такъ возлюбилъ, что и умеръ за насъ, чтобы тако небесному Отцу насъ примирить и въ вѣчное блаженство привести. Посмотри въ книгу святаго Евангелія, и увидишь, что Хрістосъ за грѣшниковъ и нечестивыхъ умре. \textit{Составляетъ Свою любовь къ намъ Богъ яко, еще грѣшникомъ сущимъ намъ, Хрістосъ за ны умре}. И паки: \textit{врази бывше, примирихомся Богу смертію Сына Его}\footnote{Римл.~5,~8 и 10.}. Убо и намъ должно, по примѣру Его, враговъ нашихъ любить, когда хощемъ учениками Его, то"=есть, хрістіанами быть. Хрістіанинъ бо не иное что есть, какъ ученикъ Хрістовъ. "--- 4)~Кто любитъ не токмо друговъ, но и враговъ, тѣмъ показуетъ, что онъ есть \textit{сынъ Божій по благодати}. Ибо таковый нравъ небеснаго Отца на себѣ изобразуетъ, \textit{Который солнце Свое сіяетъ на злыя и благія, и дождитъ на праведныя и на неправедныя}, то"=есть, всѣмъ благодѣтельствуетъ отъ единой любви Своей, добрымъ и злымъ, всѣмъ хощетъ спастися, праведнымъ и грѣшнымъ. Коль же великое дѣло есть быть сыномъ Божіимъ, слово изъяснить и умъ понять того не можетъ человѣческій. Аще убо кто любитъ друговъ и враговъ, сей послѣдуетъ Богу нравомъ своимъ, яко сынъ отцу своему; \textit{бываетъ подражатель Богу, якоже чадо возлюбленное}\footnote{Еф.~5,~1.}. "--- 5)~Единыя хрістіанскія души свойство есть враговъ любить. Любящихъ бо себе любятъ грѣшницы, мытари и язычницы. Хрістіанамъ же въ высшій любве степень должно восходить, то"=есть, не токмо любящихъ любить, но и ненавидящихъ; не токмо благотворящимъ благодарить, но и злотворящимъ; не токмо благословящихъ, но и кленущихъ благословлять; не токмо молитися за добро творящихъ, но и за творящихъ напасть и изгонящихъ. Аще убо кто любящихъ себе любитъ токмо, сей не хрістіанское дѣло показуетъ, но дѣлаетъ тое, что и язычникамъ, незнающимъ Хріста, и великимъ грѣшникамъ общее есть, и въ томъ отъ нихъ не разнствуетъ. Ибо истинная хрістіанская любовь \textit{не ищетъ своихъ}, но любитъ всякаго безъ своея пользы\footnote{1~Кор.~13,~5.}. А кто любящаго себе любитъ, сей равное воздаетъ, и добрымъ за добро платитъ, и любитъ себе паче, нежели ближняго. Ибо любитъ его ради того, что отъ него любится, добро и пользу пріемлетъ, и потому пользу и корысть свою любитъ паче, нежели ближняго; иначе, когда бы пользы и корысти отъ него не видѣлъ, не любилъ бы его, "--- что язычники и прочіи грѣшники творятъ. Откуду и Хрістосъ глаголетъ: \textit{аще любите любящихъ васъ, кую мзду имате? Не и мытари ли тожде творятъ? И аще цѣлуете други ваша токмо, что лишше творите? Не и язычницы ли такожде творятъ}? Хрістіанская убо добродѣтель не въ томъ состоитъ, чтобы токмо любящихъ любить, но всѣхъ безъ своея пользы. Любящихъ себе не любить есть грѣхъ такой, которымъ и язычники гнушаются. Ибо таковыхъ любить и природный разумъ велитъ; и большій грѣхъ творитъ, который любящихъ не любитъ, нежели тотъ, который враговъ не любитъ. Таковый бо и естественнаго лишился разума, и горшій невѣрнаго, который, закономъ естественнымъ наставляемъ, любящихъ себе любитъ: \textit{аще кто о своихъ, паче же о домашнихъ, не промышляетъ, вѣры отвергся, и невѣрнаго горшій есть}, глаголетъ апостолъ\footnote{1~Тим.~5,~8.}. О таковомъ глаголется: \textit{иже воздаетъ злая за благая, не подвигнутся злая изъ дому его}\footnote{Притч.~17,~13.}. "--- 6)~Высокая есть добродѣтель любить враговъ и добро творить ненавидящимъ. Тако бо человѣкъ самого себе побѣждаетъ, которая побѣда преславная и славнѣйшая есть, нежели грады и государства побѣждать. Тако духомъ смиренія духъ гордости низлагается; тако благостыня и кротость злобу гонитъ; тако ветхій Адамъ новому уступаетъ; тако міръ и міродержецъ попирается; тако хрістіанинъ подвигомъ добрымъ подвизается. На сей подвигъ подвигоположникъ Іисусъ благопріятно съ небесе взираетъ, и подвизающемуся неувядаемый вѣнецъ готовитъ. Знатная побѣда "--- благимъ злое побѣждать! Преславное торжество "--- надъ самимъ собою торжествовать! Благопріятное позорище "--- ненавидящихъ любовію объимать, зло творящимъ благотворить! "--- 7)~Какъ огнь огнемъ не гасится, тако гнѣвъ гнѣвомъ не побѣждается, но паче разжигается. Отсюду возстаютъ ссоры, брани, драки, кровопролитія, убійства и прочая злая. А кротостію и любовію часто и самые свирѣпые враги преклоняются и примиряются. Извѣстно изъ первой книги Царствъ, како гналъ Саулъ кроткаго Давида. Но когда Давидъ въ пустынѣ взялъ копіе отъ возглавія спящаго его, и не простерлъ руки своея на него, "--- слыши, что пробудившійся Саулъ къ нему глаголетъ, како признаетъ свою винность предъ нимъ: \textit{и рече Саулъ: согрѣшихъ, возвратися, чадо Давиде, яко ктому не сотворю ти зла: зане честна душа моя предъ очима твоима въ день сей: безумно сотворихъ, и погрѣшихъ много зѣло}\footnote{1~Цар.~26,~21.}. Тако любовь и кротость и лютыхъ враговъ въ сокрушеніе приводитъ. "--- 8)~Любовь ко врагамъ дѣлаетъ дерзновеніе въ молитвѣ. Ибо когда съ любовію оставляемъ имъ обиды, безъ зазрѣнія совѣсти глаголемъ небесному Отцу: \textit{остави намъ долги наша, якоже и мы оставляемъ должникомъ нашимъ}, "--- чему напротивъ вражда препятствуетъ. "--- 9)~Враги наши насъ смиряютъ, гордость, тщеславіе и самомнѣніе наше низлагаютъ, приводятъ насъ въ познаніе себе самихъ и немощи нашея, возбуждаютъ насъ къ молитвѣ усердной, и тако, хотя не съ симъ добрымъ намѣреніемъ гонятъ насъ, душевными благими обогащаютъ насъ, когда злобу ихъ любовію побѣждаемъ. Тако гонимый отъ враговъ своихъ Давидъ и отъ нихъ озлобляемый къ Богу, какъ елень отъ ловцовъ утружденный съ жаждою къ источнику водному, съ воздыханіемъ и молитвою прибѣгалъ. Любить убо ихъ, а не гнѣваться на нихъ должно намъ, что они, хотя и не съ добрымъ намѣреніемъ, дѣлаютъ сіе добро намъ. Тако \textit{любящимъ Бога} и, по заповѣди Его, враговъ своихъ, \textit{вся поспѣшествуютъ во благое}\footnote{Римл.~8,~28.}. Имъ и вражда, злоба, гоненіе враговъ великую приноситъ пользу, хотя они сами не знаютъ и не желаютъ тоя пользы имъ. "--- 10)~Хрістосъ глаголетъ: \textit{якоже хощете, да творятъ вамъ человѣцы, и вы творите имъ такожде}\footnote{Лук.~6,~31.}. Никто не хощетъ отъ враговъ ненавидимъ быть: и самъ не долженъ ихъ ненавидѣть. Но паче хощетъ всякъ въ нужномъ случаѣ и отъ враговъ помощь получить. Напр. попался бы кто или разбойникамъ въ руки, или самымъ врагомъ своимъ, или въ водѣ погрязлъ: въ такой тѣснотѣ непремѣнно и отъ враговъ своихъ искалъ бы милости, "--- что дѣлается во время сраженія воинскаго, когда единъ воинъ противной стороны, побѣжденный отъ другаго, у побѣдителя милости и живота проситъ. А когда самъ отъ враговъ своихъ въ нужномъ случаѣ милости желаетъ, того и имъ желать и самымъ дѣломъ показывать, по закону естественному, самъ долженъ. Тако Давидъ святый Саула, врага своего, который искалъ души его, пощадѣлъ и не отнялъ живота его; почему и Саулъ приписать принужденъ былъ похвалу Давиду, и признать любовь его къ себѣ: \textit{честна}, рече, \textit{душа моя предъ очима твоима въ день сей}, какъ выше сказано. "--- 11)~Врагъ нашъ истинный есть діаволъ, который какъ душу, такъ и тѣло наше ищетъ и тщится озлобить и погубить. Сей лукавый и враждебный духъ и людей научаетъ, чтобы насъ гнали и озлобляли. И такъ по большей части онъ нашего озлобленія причиною. Онъ насъ чрезъ людей гонитъ, и тѣмъ тщится отвести отъ терпѣнія и любви Божіей, привести насъ во вражду и ссору, о чемъ онъ, яко духъ злобы, радуется. На него убо должно намъ всю вражду нашу обратить, яко истаго и всегдашняго врага, ему противитися твердою вѣрою, воли его не попущать надъ нами совершатися, "--- что терпѣніемъ и кротостію бываетъ, не воздаяніемъ зла за зло и досажденія за досажденіе, но любовію за ненавидѣніе и благотвореніемъ за злодѣяніе; а людямъ, которые хитростію его прельщаеми гонятъ, духомъ любве соболѣзновать, что его слушаютъ, злый и пагубный совѣтъ его исполняютъ и злой его воли повинуются. Ибо отъ сего ихъ озлобленія познаемъ, что подъ темною его находятся властію, работаютъ ему какъ плѣнники. \textit{Всякъ бо ненавидяй брата своего} (всякаго человѣка), \textit{человѣкоубійца есть: и творяй грѣхъ отъ діавола есть, яко діаволъ исперва согрѣшаетъ}, глаголетъ апостолъ Хрістовъ\footnote{1~Іоан.~3,~15 и 8.}. Ктомужъ часто бываетъ, что враги наши жалѣютъ и каются о семъ, что гнали и озлобляли насъ. Тако вышепомянутый Саулъ каялся и признавалъ грѣхъ свой, что гналъ Давида: \textit{безумно сотворихъ, и погрѣшихъ много зѣло}. Ибо Богъ, яко человѣколюбецъ и милосердый, Своею благодатію преклоняетъ и умягчаетъ человѣческія сердца, о чемъ многія свидѣтельствуютъ исторіи. Жалѣть убо должно о врагахъ, а не гнѣваться на нихъ. Ибо болѣе гонятъ себе, а не насъ, болѣе губятъ себе, нежели насъ. Понеже наше тѣло озлобляютъ, а свою душу, которая далеко честнѣйшая и дражайшая есть паче тѣла. Не можетъ бо человѣкъ ближняго своего обидѣть безъ собственной своей душевной пагубы, яко заповѣди Божіей нарушитель, и потому предъ судомъ Божіимъ повинный. "--- 12)~Поощряютъ насъ къ тому примѣры святыхъ Божіихъ. Хрістосъ, Сынъ Божій, во днехъ плоти Своея и сожитія съ человѣки, какъ враговъ Своихъ любилъ, Евангеліе святое свидѣтельствуетъ. Сколько Его фарисеи хулили, гнали, но Онъ имъ усердно желалъ обращенія и спасенія. Горячія пролилъ слезы надъ враждебнымъ Ему и нераскаяннымъ Іерусалимомъ\footnote{Лук.~19,~41.}, провидя его погибель имѣющую быть. О погибели бо вражіей плакать всякъ признаетъ за извѣстнѣйшій ко врагамъ любве знакъ: якоже о погибели ихъ радоватися есть доказательство ненависти. О распинателяхъ Своихъ молился: \textit{Отче! отпусти имъ: не вѣдятъ бо, что творятъ}\footnote{23,~34.}. Святый Стефанъ первомученикъ, каменіемъ побиваемый, за побивающихъ молился: \textit{преклонь колѣна, возопи гласомъ веліимъ: Господи! не постави имъ грѣха сего}\footnote{Дѣян.~7,~60.}. Давидъ святый плакалъ и рыдалъ не токмо о сынѣ своемъ Авессаломѣ погибшемъ, который искалъ живота отца своего\footnote{2~Цар.~18,~33.}, но и о Саулѣ, врагѣ своемъ, убіенномъ\footnote{1,~11 и 12.}. Плачъ бо по врагѣ погибшемъ показуетъ плачущаго любовь ко врагу. Милосерды бо, сострадательны и любительны суть святіи Божіи, яко любительнымъ Духомъ Божіимъ водимые, и не о своей обидѣ, но о погибели гонителей своихъ разсуждаютъ и болѣзнуютъ. Тоежде учинили апостоли, которые, свидѣтельствуя о себѣ, и намъ примѣръ представляютъ: \textit{укоряеми благословляемъ}\footnote{1~Кор.~4,~12.}, и тако всю вселенную не оружіемъ, но любовнымъ терпѣніемъ и словомъ Божіимъ Хрісту покорили, \textit{Господу поспѣшствующу, и слово утверждающу послѣдствующими знаменьми}\footnote{Марк.~16,~20.}. Тоежъ показали мученицы святіи, которые злобу гонителей своихъ любовію побѣждали и за мучителей своихъ молились, и тако многихъ съ помощію Божіею въ вѣру привели. "--- 13)~Послушаемъ еще, какъ Златоустъ святый златымъ языкомъ о любви ко врагамъ бесѣдуетъ намъ: «можетъ быть, глаголетъ, мстить хощеши и озлобившему тебе равная или большая воздать тщишися. И что тебѣ въ томъ нужды, когда никакой послѣ не будетъ пользы? А притомъ и на страшномъ судѣ ономъ муку терпѣть будешь, яко данныхъ отъ него законовъ преступникъ. Скажи мнѣ, ежели бы царь нѣкій земный законъ тебѣ положилъ, дабы или о врагахъ твоихъ попеченіе имѣлъ, или смертію казненъ былъ: не всѣ ли, боясь тѣлесныя сея смерти, прибѣжали бы къ исполненію закона онаго? Коликаго убо осужденія достоинъ тотъ, который, тѣлесныя боясь смерти (а смерти и безъ того самый естества долгъ требуетъ отъ насъ), все терпѣти готовъ; а не боясь такія смерти, въ которой никакого невозможно найдти утѣшенія, нерадитъ о законѣ, данномъ отъ Владыки всяческихъ!»\footnote{Бес.~4"~я на Быт.}

\paragraph*{§\:268.} Слѣдственно 1)~противу заповѣди сей Хрістовой погрѣшаютъ, которые отъ враговъ своихъ любовь отъемлютъ; которые не благословятъ, но кленутъ кленущихъ себе; не добро, но зло творятъ ненавидящимъ ихъ, молятся не за творящихъ имъ напасть, но на творящихъ: \textit{суди}"=де \textit{имъ Богъ}, или \textit{судитъ}"=де \textit{имъ} или \textit{ему Богъ}, "--- а только любятъ тѣхъ, отъ которыхъ любовь къ себѣ дознаютъ, "--- въ чемъ они отъ язычниковъ, мытарей и грѣшниковъ ничимъ не разнствуютъ, какъ Хрістосъ учитъ, яко и тіи любящихъ себе любятъ. "--- 2)~Симъ показуютъ, что они къ числу сыновъ Божіихъ не надлежатъ, которые послѣдуютъ Отцу своему небесному, и, \textit{какъ Онъ солнце Свое сіяетъ на злыя и благія, и дождитъ на праведныя и на неправедныя}, объятіемъ любве своея обымать тщатся не токмо друговъ, но и враговъ своихъ, и благотворить не токмо благотворящимъ, но и злотворящимъ. "--- 3)~Ежели отъ язычниковъ, мытарей и грѣшниковъ не разнствуютъ, которые только любящихъ себе любятъ: въ какомъ классѣ уже находятся тѣ, которые любящихъ себе не любятъ, которые проливаютъ слезы неповинныхъ, которыя отирать должны, озлобляютъ никакого зла имъ не сотворшихъ, которыхъ заступать должны; обнажаютъ вдовицъ, сиротъ и прочіихъ бѣдныхъ, отъ которыхъ никакой напасти не пріемлютъ, и которыхъ ради бѣдности ихъ покрывать одолжаются? Кажется, да и всякъ можетъ изъ вышереченныхъ примѣтить, что они и самыхъ язычниковъ горшіи, которые сими безчеловѣчными пороками гнушаются и законами своими казнятъ, "--- и потому послѣднюю разума искру погасили, и звѣрскій, а не человѣческій нравъ имѣютъ, хотя бы они какъ ни показывались предъ людьми. "--- 4)~Которые благодѣтелямъ своимъ злодѣяніе за благодѣяніе воздаютъ, и вмѣсто благодарности, которую имъ должны воздавать, имя ихъ ядовитымъ языкомъ терзаютъ и самихъ ихъ попрать и погубить умышляютъ, и самыхъ безразумныхъ скотовъ *кажется* хуждшіи. Ибо скотъ познаетъ господина своего питающаго, служитъ и работаетъ ему, песъ ласкается предъ кормителемъ своимъ, охраняетъ и въ случаѣ защищаетъ его, сохраняетъ стадо овчее отъ находящихъ звѣрей; неблагодарные же сея должности не показуютъ своимъ благодѣтелямъ, и для того въ сей части безумнѣйшіи суть отъ скотовъ. Къ сему числу надлежатъ всѣ тѣ, которые не воздаютъ чести должныя монарху своему, о цѣлости общества, въ которомъ и они заключаются, пекущемуся; которые злословятъ и поносятъ пастырей своихъ, о стадѣ своемъ, въ которомъ и они мнятся быть, пекущихся; которые учителей своихъ и наставниковъ, у которыхъ слово истины слышали, имя помрачаютъ, или инако озлобляютъ; сыны, которые родителей своихъ не чтутъ, оставляютъ ихъ, и, что горше есть, ругаютъ, озлобляютъ и біютъ; и прочіи, благодѣяніе получившіи, но благодѣтелей ненавидящіи и злая за благая воздающіи. Таковые всѣ да внимаютъ, что Соломонъ глаголетъ: \textit{иже воздастъ злая за благая, не подвигнутся злая изъ дому его}.

\paragraph*{§\:269.} \textit{Какъ"=де мнѣ враговъ любить}? "--- \textit{Отвѣтъ}. 1)~Знаю я, что любить враговъ плоти нашей есть горестно; но того должность хрістіанская отъ насъ требуетъ, чтобы намъ не тое дѣлать, что сердце наше хощетъ, но что законъ Божій повелѣваетъ. Въ семъ бо и подвигъ хрістіанскій состоитъ, чтобы намъ противу себе самихъ, то"=есть, противу страстей плоти нашей подвизаться, \textit{распинать ее со страстьми и похотьми, и покоряти духу}\footnote{Гал.~5,~24 и 25.}. Что убо? Плоти ли хощемъ повиноватися, которая хощетъ насъ погубить, не повинуяся закону Божію, "--- или заповѣди Хрістовой, которая хощетъ оживить? \textit{Аще по плоти живете, имате умрети: аще ли духомъ дѣянія плотская умерщвляете, живи будете}, глаголетъ апостолъ\footnote{Римл.~8,~13.}. "--- 2)~Понеже сердце наше не хощетъ дѣлать того, что заповѣдь Хрістова повелѣваетъ, тѣмъ самымъ обличается немощь наша, и какъ бы убѣждаемся молиться Самому Тому, Который бы далъ намъ \textit{сердце новое и духъ новъ}, дабы мы усердно и съ любовію заповѣдь Его исполнять, хотѣть и дѣлать благое возмогли. "--- 3)~Разсуждай ученіе святаго Златоустаго о семъ и прочія причины прописанныя, когда злоба на врага твоего подстрекаетъ тя. "--- 4)~Вражда и злоба во врагѣ твоемъ достойна ненависти, яко діавольское дѣло; а онъ самъ, поелику человѣкъ и созданіе Божіе, якоже и ты, самъ тогожде рода, естества, достоинъ любве. Устремляйся убо на ненависть его, а не на него самого; и гони злобу его, которая, какъ хладъ огнемъ, любовію и благосклонностію изгоняется, "--- и тако или его лучшимъ сдѣлаешь, или когда не то, то самъ лучшимъ будеши.



\chapter*{ЗАКЛЮЧЕНІЯ КНИГИ СЕЯ.}
\addcontentsline{toc}{chapter}{Заключенія книги сея.}
\section[Заключеніе 1-е. О концѣ добрыхъ дѣлъ.]{первое заключеніе.\\\bfseries О концѣ добрыхъ дѣлъ.}

\begin{quotation}\textit{Тако да просвѣтится свѣтъ вашъ предъ человѣки, яко да видятъ ваша добрая дѣла, и прославятъ Отца вашего, Иже на небесѣхъ}, глаголетъ Хрістосъ\footnote{Матѳ.~5,~16.}.\end{quotation}
\begin{quotation}\textit{Кая польза, братіе моя, аще вѣру глаголетъ кто имѣти, дѣлъ же не имать? Еда можетъ вѣра спасти его}? и проч. \textit{Якоже бо тѣло безъ души мертво есть: тако и вѣра безъ дѣлъ мертва есть}\footnote{Іак.~2,~14--26.}.\end{quotation}
\begin{quotation}\textit{Молю убо васъ азъ юзникъ о Господѣ, достойно ходити званія, въ неже звани бысте, со всякимъ смиренномудріемъ и кротостію, съ долготерпѣніемъ, терпяще другъ друга любовію}, и проч.\footnote{Еф.~4,~1 и слѣд.}\end{quotation}


\paragraph*{§\:270.} Не все то доброе дѣло, что очесамъ человѣческимъ кажется доброе быть. Доброе дѣло отъ добраго конца быть судится. Почему, хотя и добро дѣлается, но не на добрый конецъ, доброе быть не можетъ. Творятъ часто и лицемѣры добрыя своя дѣла; часто молятся, воздыхаютъ, еще и плачутъ, даютъ обильныя милостыни, показуютъ знаки смиренія, облекаются въ рубища, преклоняютъ главы свои, тихо говорятъ, грѣшниками себе называютъ и, по видимому образу, благочестіе показуютъ: но \textit{силы его отвергаются}. Ибо творятъ дѣла свои, да \textit{видимы будутъ отъ человѣкъ}; почему дѣла ихъ отвергаются отъ Бога, Который сердце и намѣреніе всякаго знаетъ, и по тому всякое дѣло судитъ. Сего ради тое только дѣло доброе, подлинно доброе, которое добрѣ дѣлается. Добрѣ же дѣлается тогда, когда отъ вѣры и на добрый конецъ дѣлается (какъ въ слѣдующемъ параграфѣ увидимъ). \textit{Злое бо древо плода добраго творити не можетъ}, глаголетъ Хрістосъ\footnote{Матѳ.~7,~18.}. Чего ради не токмо доброе дѣлать должно, но и добрѣ. Откуду Василій Великій глаголетъ: «не токмо тое, что заповѣдано, но и такъ, какъ заповѣдано, должно дѣлать»\footnote{Въ кн.~2"~й о крещеніи.}.

\paragraph*{§\:271.} \textit{Конецъ добрыхъ дѣлъ}: 1)~Должно намъ творить добрыя дѣла, да покажемъ повиновеніе и послушаніе Богу нашему, Который намъ глаголетъ: \textit{уклонися отъ зла, и сотвори благо}\footnote{Пс.~33,~15.}. Аще бо Его называемъ Отцемъ: \textit{Отче нашъ, Иже еси на небесѣхъ}, якоже въ Господней молитвѣ на всякій день молимся, то, какъ сынамъ, отъ любви сыновнее Ему послушаніе должно показывать. Аще Онъ Господь нашъ есть, то, какъ раби, должны мы Ему со страхомъ работать, и не быть рабами грѣха, не попущать, дабы онъ нами обладалъ. Который сынъ отца своего, какій рабъ господина своего не слушаетъ? Когда сынъ отцу плотскому, рабъ господину своему "--- подобнымъ себѣ человѣкамъ, повинуются, кольми паче Богу сіе повиновеніе должно показывать, Который отцевъ и дѣтей, господъ и рабовъ есть Верховнѣйшій Господь. И хотя мы, ради немощи нашей, и не можемъ таковаго повиновенія Ему показывать, каковаго святый и праведный Его законъ отъ насъ требуетъ, потому что \textit{плоть похотствуетъ на духа}\footnote{Гал.~5,~17.}, однакожъ должно намъ о томъ отъ всего сердца тщаться, чтобы не попущать плоти надъ нами господствовать, но паче плоть духу покорять, и \textit{духомъ дѣянія плотская умерщвлять}, какъ учитъ апостолъ\footnote{Римл.~8,~13.}. Сіе повиновеніе или послушаніе обѣщались мы показывать Богу, Господу и Отцу нашему, при вступленіи въ хрістіанство, когда отрекались сатаны и всѣхъ дѣлъ его, и обѣщались \textit{служить Ему преподобіемъ и правдою предъ Нимъ вся дни живота нашего}\footnote{Лук.~1,~75.}. А чтобы сіе послушаніе наше отъ усердія и свободнаго духа происходило, обѣщалъ Онъ Самъ въ томъ намъ помогать; чего ради и молиться повелѣлъ: \textit{просите, и дастся вамъ}\footnote{Матѳ.~7,~7.}. Обѣщанное послушаніе наше столько разъ повторяемъ мы, сколько разъ каемся предъ Нимъ о согрѣшеніяхъ нашихъ. Слѣдственно, кто не тщится послушанія сего Богу показывать, напрасно Ему глаголетъ: \textit{Господи, Господи}! якоже Хрістосъ таковымъ отвѣщаетъ: что Мя зовете: \textit{Господи, Господи, и не творите, яже глаголю}?\footnote{Лук.~6,~46.}; ибо имѣетъ другаго господа, которому работаетъ, то"=есть мамонѣ, или похоти злой и нечистой, или иному подобному симъ. Такожде напрасно съ сынами Божіими сообщаетъ гласъ свой къ Богу: \textit{Отче нашъ, Иже еси на небесѣхъ}; понеже имѣетъ другаго отца, котораго нравами своими изображаетъ и волю творитъ. Хвалились нѣкогда Іудеи злобствующіи на Хріста: \textit{единаго Отца имамы Бога}; но услышали отъ Него страшный отвѣтъ: \textit{вы отца вашего діавола есте, и похоти отца вашего хощете творити}\footnote{Іоан.~8,~41 и 44.}. Надобно опасаться и хрістіанамъ, \textit{которые Бога исповѣдуютъ, но дѣлы отмещутся Его}\footnote{Тит.~1,~16.}, чтобы сего человѣкоубійцы и отца лжи не быть истыми сынами. Страшно сіе всякому слово, но, какъ взять въ разсужденіе свойства сына и отца, должно признать за истину. Сынъ бо отцу непремѣнно подобится, ибо каково сѣмя, таковъ и плодъ его. И апостолъ глаголетъ: \textit{творяй грѣхъ, отъ діавола есть, яко исперва діаволъ согрѣшаетъ}\footnote{1~Іоан.~3,~8.}. Сей страшный титулъ приличествуетъ тѣмъ, которые не отъ немощи, но отъ произволенія согрѣшаютъ, и за мало, или, что горше того, за ничто ставятъ законъ Божій нарушать. И дотоль они отъ сего не избавятся, доколь чистосердечно не обратятся къ Богу, и истиннымъ покаяніемъ не очистятъ себе. "--- 2)~Одолжаемся уклоняться отъ зла и творить благое къ прославленію святаго имене Божія. Не такъ должно хрістіанамъ творить добрыя дѣла, какъ язычники нѣкогда и нынѣ лицемѣры творятъ "--- къ похвалѣ и славѣ своей, "--- но къ похвалѣ и славѣ Того, отъ Котораго они происходятъ, то"=есть Божіей. Ибо безъ Бога ничего добраго сдѣлать не можемъ\footnote{Іоан.~15,~5.}. Убо какъ добро отъ Бога происходитъ, такъ отъ насъ въ Божію славу и обращаться должно, по Хрістову словеси: \textit{тако да просвѣтится свѣтъ вашъ предъ человѣки, яко да видятъ ваша добрая дѣла, и прославятъ Отца вашего, Иже есть на небесѣхъ}. Имя Божіе какъ свято, такъ и славно есть само въ себѣ, и не требуетъ прославленія: но отъ насъ или прославляется, когда Его исповѣдуемъ такъ, какъ святое Его Слово открыло, и живемъ такъ, какъ рабомъ Его, имя Его нарицающимъ и Его за Господа своего признающимъ, должно, "--- или безчестится и хулится, когда не такъ Его исповѣдуемъ, какъ откровенно, и не такъ обращаемся и обходимся, какъ рабамъ Божіимъ прилично, якоже апостолъ приводитъ изъ пророковъ о Іудеяхъ: \textit{имя Божіе вами хулится во языцѣхъ}\footnote{Римл.~2,~24; Ис.~52,~5.}. Слава бо отцу бываетъ, когда сыны его постоянно живутъ, и господина похвала, когда раби его въ своемъ званіи исправны. Истина сія безспорна. Такъ слава Отцу небесному бываетъ, когда хрістіане, которые Его за Отца и Господа своего признаютъ, живутъ по правилу святаго Божія Слова, и какъ сынамъ такъ великаго и благаго Отца, и рабамъ такъ страшнаго Господа прилично. Къ прославленію имене Божія подаетъ намъ случай божественныхъ свойствъ познаніе, и о нихъ прилѣжное размышленіе. Имя Божіе заключаетъ въ себѣ божественныя его свойства, въ святомъ Писаніи откровенныя, яже суть: святый, праведный, присносущный, вездѣсущій, великій, всемогущій, благій, щедрый, милостивый, истинный, и прочая. Тако прославляемъ \textit{присносущное бытіе} Его, когда отъ сердца признаемъ и исповѣдуемъ, что Онъ есть, есть единъ, и конца и начала не имѣетъ. Прославляемъ \textit{истину Его}, когда вѣруемъ сердечно неложнымъ обѣщаніямъ Его. Тако Авраамъ, праотецъ нашъ, \textit{далъ славу Богу}, якоже апостолъ учитъ, когда \textit{во обѣтованіи Божіи не усумнѣся невѣрованіемъ, но возможе вѣрою}\footnote{Римл.~4,~20.}. Прославляемъ \textit{правду Его}, которая всѣмъ воздаетъ по дѣломъ, когда признаемъ въ наказаніи съ пророкомъ: \textit{праведенъ еси, Господи, и прави суди Твои}\footnote{Пс.~118,~137.}! "--- и грѣхи наши исповѣдуемъ и каемся, и глаголемъ тако: \textit{Тебѣ, Господи, правда, намъ же стыдѣніе лица}\footnote{Дан.~9,~7 и 8.}. Прославляемъ \textit{величество Его и всемогущество}, когда, позная и призная подлость и ничтожество наше, смиряемся предъ Нимъ. Прославляемъ высочайшее \textit{господство и власть} Его, когда воли и повелѣнію Его со усердіемъ повинуемся. Прославляемъ \textit{вездѣсущіе Его и всевѣдѣніе}, когда вездѣ и на всякомъ мѣстѣ уклоняемся отъ зла и творимъ благое; отдаемъ Ему, яко вездѣсущему, Который слово наше слышитъ, дѣло и помышленіе видитъ, замыслъ и намѣреніе проницаетъ, "--- отдаемъ Ему, глаголю, словомъ, дѣломъ и помышленіемъ почтеніе высокое, яко высочайшему Господину и Царю, не токмо не дѣлая, но ни помышляя и не произнося ничего, святѣйшимъ Его очесамъ и ушесамъ непристойнаго. Прославляемъ \textit{святость Его}, когда сами тщимся хранить святыню, хранимся \textit{отъ всякія скверны плоти и духа}\footnote{2~Кор.~7,~1.}, по словеси Его: \textit{будете святи, якоже Азъ святъ есмь}\footnote{1~Петр.~1,~16; Лев.~11,~44; 19,~2.}. Прославляемъ \textit{премудрость Его}, когда вѣримъ чудному Его промыслу, которымъ все къ надлежащему концу порядочно управляетъ, хотя нашему слѣпому уму и непонятно, и \textit{высшихъ себе не испытуемъ}, но со смиреніемъ и молчаніемъ тое оставляемъ премудрости Его и всемогуществу, что намъ не дано знать и понять. Прославляемъ \textit{благость Его}, когда Его единаго, яко высочайшее добро, любимъ и благодаримъ Ему. Прославляемъ \textit{милость и щедроты Его}, которыя на насъ всегда обильно изливаетъ, когда усердно отъ сердца благодаримъ Ему, и, въ знакъ благодарности, сами милостивыми и щедрыми къ ближнимъ нашимъ тщимся быть. Прославляемъ \textit{невещественное и духовное Его существо}, когда почитаемъ Его \textit{духомъ и истиною}\footnote{Іоан.~4,~24.}; не полагаемъ того почитанія во внѣшнихъ только обрядахъ, но въ сердечномъ усердіи, страсѣ и любви; и что \textit{сердцемъ вѣруемъ}, тое и \textit{усты} небоязненно, гдѣ должно, \textit{исповѣдуемъ}. И сіе не токмо сами дѣлаемъ, но и другихъ къ тому словомъ и дѣломъ возбуждаемъ. Тако святится и славится въ насъ святое и страшное имя Божіе. А понеже сея должности, ради слабости нашей, исполнять сами не можемъ, то, когда исполняемъ, благодати Его, дѣйствующей въ насъ, должно восписывать, дабы Тому похвала и слава была, отъ Кого добро происходитъ. И тако прославляемъ Бога въ сей части, когда себе не прославляемъ въ дѣлахъ, не нашему имени чести и похвалы, но Божіему ищемъ, по оному: \textit{не намъ, Господи, не намъ, но имени Твоему даждь славу о милости Твоей и истинѣ Твоей}\footnote{Пс.~113,~9.}! Прославляемъ правду Его, когда нашу неправду предъ Нимъ исповѣдуемъ съ пророкомъ: \textit{Тебѣ Господи, правда, намъ же стыдѣніе лица}, "--- и отъ Него просимъ, вѣруемъ и надѣемся оправдатися \textit{правдою единороднаго Сына Его}, за грѣшниковъ умершаго\footnote{Римл.~3,~21--28.}. Прославляемъ святость Его, когда нашу предъ Нимъ исповѣдуемъ нечистоту, и \textit{Его святыни} ищемъ \textit{пріобщитися}\footnote{Евр.~12,~10.}. Прославляемъ истину Его, когда предъ Нимъ нашу признаемъ лжу, яко \textit{всякъ человѣкъ ложь}\footnote{Пс.~115,~2.}. Прославляемъ благость, милость, долготерпѣніе, премудрость и крѣпость Его, когда предъ Нимъ нашу злость, немилосердіе, нетерпѣніе, слѣпоту и немощь чистосердечно признаемъ, и себѣ отъ Него просимъ сихъ Его благихъ. Тако имя Божіе прославляется въ насъ, когда имя наше, слава и похвала увядаетъ, \textit{яко да не похвалится всяка плоть предъ Богомъ}\footnote{1~Кор.~1,~29.}, когда \textit{сладцѣ хвалимся} со апостоломъ \textit{въ немощехъ нашихъ, яко да вселится и въ насъ сила Хрістова}\footnote{2~Кор.~12,~8.}, вся дѣйствующая, и немощная укрѣпляющая и \textit{оскудѣвающая восполняющая}, "--- да Егоже сила и крѣпость въ насъ дѣйствуетъ, Тому похвала и слава отъ насъ будетъ. Слѣдственно, которые въ своихъ добрыхъ дѣлахъ славы Божіей не ищутъ, но своей, и которые соблазнительно и слову Божію противно житіе свое провождаютъ, славы Богу не отдаютъ; того ради и сами той славы, которою \textit{обѣщается прославляющихъ Его} милостивно \textit{прославить}\footnote{1~Цар.~2,~30.}, не сподобятся; но услышатъ, что сказано отъ пророка: \textit{да возмется нечестивый, да не видитъ славы Господни}\footnote{Ис.~26,~10.}. "--- 3)~Одолжаемся добрыя дѣла творить, да засвидѣтельствуемъ Богу, Благодѣтелю нашему, сердечную нашу благодарность \textit{(о семъ смотри главу о благодареніи Богу)}. Благодарность истинная требуетъ не токмо устнаго исповѣданія и прославленія благодѣтеля, но и сердечныя къ нему любве. Любовь же познается отъ творенія заповѣдей Его святыхъ. \textit{Аще кто любитъ Мя, глаголетъ Хрістосъ, слово Мое соблюдетъ}\footnote{Іоан.~14,~23.}. И благодареніе устное безъ сердечнаго любленія не иное что, какъ лицемѣріе есть, о каковыхъ Богъ чрезъ пророка глаголетъ: \textit{приближаются Мнѣ людіе сіи усты своими, и устнами чтутъ Мя, сердце же ихъ далече отстоитъ отъ Мене}\footnote{Ис.~29,~13.}. Сего ради не токмо устами, но и сердцемъ и дѣлами добрыми, отъ сердца происходящими, благодарность Богу оказывать должно. Слѣдственно сея должности не исполняющіи хрістіане къ Богу, Благодѣтелю своему, суть неблагодарни, хотя устами своими и приближаются Ему, ибо \textit{дѣлы отмещутся Его}. А понеже они Бога оставляютъ, то и сами судомъ праведнымъ отъ Него оставляются. Откуду бываетъ, что таковые люди во всякое заблужденіе и отъ грѣха въ грѣхъ падаютъ. Таковый человѣкъ, яко безъ Божіей благодати находящійся, есть какъ свирѣпый конь безъ узды, которому слѣдуетъ самовольная пагуба. Отсюду \textit{именующимся хрістіанами} нѣтъ грѣха ближняго обидѣть, клятву преступить, судъ Божій превращать, судебное мѣсто торжищемъ и корчемницею дѣлать, сиру и вдовицѣ насиліе дѣлать, оправдать нечестиваго и неповиннаго осуждать, лгать, обманывать, прельщать и прочія беззаконныя дѣла, каковыми язычники гнушаются, дѣлать. Сіи суть истинныя безбожнаго сердца примѣты, хотя и исповѣдуютъ устами имя Его. Должно сего всякому хрістіанину опасаться, и благодарность свою любовію и послушаніемъ оказывать Богу, чтобы въ вышереченное не пріитить заблужденіе. "--- 4)~Требуетъ того отъ насъ вѣра, которую имѣемъ въ Бога и Сына Его Іисуса Хріста, да покажемъ вѣру нашу отъ дѣлъ нашихъ, якоже апостолъ глаголетъ: \textit{покажи ми вѣру твою отъ дѣлъ твоихъ}\footnote{Іак.~2,~18.}. И хотя едина вѣра во Хріста Сына Божія, умершаго за насъ и воскресшаго, оправдаетъ насъ безъ взгляду дѣлъ нашихъ, однакожъ вѣра сія праздна быть не можетъ, но любовь себѣ сопряженну имѣетъ. Любовь же есть добрыхъ дѣлъ источникъ. Откуду апостоли святіи, проповѣдуя Хріста всей твари, въ основаніе полагаютъ вѣру во Хріста; но тую вѣру засвидѣтельствовать любовію и прочіими добрыми дѣлами поучаютъ насъ и увѣщаваютъ, якоже изъ посланій ихъ видно. \textit{Подадите въ вѣрѣ вашей добродѣтель, въ добродѣтели же разумъ, въ разумѣ же воздержаніе, въ воздержаніи же терпѣніе, въ терпѣніи же благочестіе, въ благочестіи же братолюбіе, въ братолюбіи же любовь}, глаголетъ Петръ святый\footnote{2~Петр.~1,~5--7.}. Вѣра бо истинная есть какъ искра, отъ Духа Святаго въ сердцѣ человѣческомъ возжженная, которая теплоту любве издаетъ. Сія божественная искра возгрѣвается и раздувается съ Божіею помощію чтеніемъ или слушаніемъ Божія слова, размышленіемъ прежде бывшихъ дѣлъ Божіихъ, молитвою, причащеніемъ святыхъ Хрістовыхъ Таинъ; и внѣ оказываетъ себе, какъ доброе древо, сладкими любве плодами: терпѣніемъ, кротостію, милосердіемъ, вѣрностію, воздержаніемъ, братолюбіемъ, миролюбіемъ и прочіими хрістіанскими добродѣтелями. Сіи суть добраго древа плоды, сіи суть добраго сердца, вѣрою Хрістовою очищеннаго, примѣты; сіи суть свидѣтели \textit{не по плоти ходящаго, но по духу}\footnote{Рим.~8,~1.}. Сіи свидѣтели предстанутъ на всемірномъ судѣ Хрістовомъ и засвидѣтельствуютъ въ избранныхъ Божіихъ истинную и нелицемѣрную вѣру, которою оправданы, и которою сіи прекрасные прозябли плоды. Сихъ, аки сокровища нѣкая, произнесетъ Самъ Судія оный праведный, и объявитъ ихъ предъ всѣмъ свѣтомъ: \textit{взалкахся, и дасте Ми ясти; возжадахся, и напоисте Мя; страненъ бѣхъ, и введосте Мене; нагъ, и одѣясте Мя; боленъ, и посѣтисте Мене; въ темницѣ бѣхъ, и пріидосте ко Мнѣ}. Якоже невѣрнымъ свидѣтели будутъ невѣрствія ихъ злая ихъ дѣла: \textit{взалкахся, и не дасте Ми ясти; возжадахся, и не напоисте Мене; страненъ бѣхъ, и не введосте Мене; нагъ, и не одѣясте Мене; боленъ, и въ темницѣ, и не посѣтисте Мене}\footnote{Матѳ.~25,~35,~36,~42 и 43.}; "--- и паки: \textit{обличу тя, и представлю предъ лицемъ твоимъ грѣхи твоя}\footnote{Пс.~49,~21.}. Сего ради искушай себе, возлюбленный хрістіанине, находишься ли въ вѣрѣ, хотя Бога Хріста Сына Божія устами исповѣдуешь. Къ сему искушенію увѣщаваетъ апостолъ: \textit{себе искушайте, аще есте въ вѣрѣ, себе искушайте}\footnote{2~Кор.~13,~5.}. Знай точно, что вѣры спасающей въ томъ нѣтъ, въ комъ нѣтъ добрыхъ дѣлъ, но есть нѣкая устная, лицемѣрная, или, какъ апостолъ учитъ, \textit{мертвая: якоже бо тѣло безъ духа мертво есть, тако и вѣра безъ дѣлъ мертва есть}\footnote{Іак.~2,~26.}. Сего ради и Хрістосъ глаголетъ: \textit{не всякъ, глаголяй Ми: Господи Господи, внидетъ въ царствіе небесное, но творяй волю Отца Моего, Иже есть на небесѣхъ}\footnote{Матѳ.~7,~21.}. Воля же небеснаго Отца въ томъ состоитъ, дабы намъ о грѣхахъ каяться, вѣровать во Хріста и слушать святаго ученія Его, какъ Самъ Онъ объявляетъ: \textit{Сей есть Сынъ Мой возлюбленный, о Немже благоволихъ: Того послушайте}\footnote{17,~5.}. А чему учитъ насъ сей небесный и всему міру проповѣданный Учитель, святое Его Евангеліе представляетъ намъ. "--- 5)~Должно намъ уклонятися отъ зла и творить благое, да не порочимъ и безчестимъ высокаго хрістіанскаго званія и имене. Хрістіанамъ должно разсуждать и помнить, что апостолъ пишетъ: \textit{не призва насъ Богъ на нечистоту, но во святость}\footnote{1~Сол.~4,~7.}; помнить и внимать, что Петръ святый имъ приписуетъ: \textit{вы родъ избранъ, царское священіе, языкъ святъ, люди обновленія, яко да добродѣтели возвѣстите изъ тмы васъ Призвавшаго въ чудный Свой свѣтъ, иже иногда не людіе, нынѣ же людіе Божіи}, и проч.\footnote{1~Петр.~2,~9,~10 и слѣд.}; помнить, что они Бога небеснаго призываютъ, чтутъ и молятся: \textit{Отче нашъ, Иже еси на небесѣхъ}\footnote{Матѳ.~6,~9.}; прилѣжно внимать тому, что Богословъ пишетъ: \textit{видите, какову любовь далъ есть Отецъ намъ, да чада Божія наречемся, и есмы}\footnote{1~Іоан.~3,~1.}; и паки: \textit{общеніе наше со Отцемъ и съ Сыномъ Его Іисусомъ Хрістомъ}\footnote{1,~3.}; помнить и размышлять, что они, по апостолу, \textit{храмъ Божій суть, и Духъ Божій живетъ въ нихъ; храмъ бо Божій святъ есть}\footnote{1~Кор.~3,~16 и 17.}. Великое, высокое и совсѣмъ небесное есть званіе хрістіанское, почему должно быть отъ земныхъ и мірскихъ сквернъ и нечистотъ удаленное. Царская міра сего фамилія въ рубищахъ не терпитъ ходить, и отъ пороковъ, помрачающихъ славу ея, бережется, но покрывается порфирою и виссономъ и прочею утварію, царству земному приличною. Хрістіанство есть фамилія небеснаго Царя, Который рубища мірскаго смрада и скверны грѣховныя крайне неприличны; но какъ одѣяна порфирою и чистымъ виссономъ \textit{правды Хрістовой}, такъ и ходить въ томъ царскомъ украшеніи должна, и благодатію Божіею украшать себе утварію добродѣтелей хрістіанскихъ. Сего ради апостоли святіи, разсуждая сіе высокое хрістіанское достоинство, вездѣ въ посланіяхъ своихъ сіе намъ предъ душевныя очи полагаютъ и поучаютъ насъ достойно того жительствовать и увѣщаваютъ: \textit{молю вы азъ, юзникъ о Господѣ, достойно ходити званія, въ неже звани бысте}, и проч.\footnote{Еф.~4,~1 и слѣд.} И паки: \textit{явися благодать Божія спасительная всѣмъ человѣкомъ, наказующи насъ, да, отвергшеся нечестія и мірскихъ похотей, цѣломудренно и праведно и благочестно поживемъ въ нынѣшнемъ вѣцѣ}\footnote{Тит.~2,~11 и 12.}. Сюды надлежитъ, что поучаютъ насъ \textit{распинать плоть нашу}, которая похотствуетъ на духа, распинать \textit{со страстьми и похотьми, совлекаться ветхаго человѣка и облекаться въ новаго}\footnote{Гал.~5,~24; Кол.~3,~9 и 10; Еф.~4,~22--24.}, и проч. О, коль убо много званіе сіе небесное безчестятъ, которые не токмо достойныхъ того дѣлъ не творятъ, но и пороками грѣховными, какъ смрадными рубищами, себе оскверняютъ! Тако они явно показуютъ, что отъ того удалились и обратились въ тую тму, изъ которой благодатію Божіею призваны были въ \textit{чудный свѣтъ}. Сего ради должно имъ, пока \textit{время благопріятно} пока \textit{день спасенія} есть, признавши свое заблужденіе, паки возвратиться съ чистымъ сердцемъ къ позвавшему ихъ и нынѣ всѣхъ зовущему Богу, и молить Его, чтобы пріялъ въ первую Свою милость. "--- 6)~Должно хрістіанамъ примѣчать и помнить и тое, что Хрістосъ глаголетъ: \textit{всяко древо, еже не творитъ плода добра, посѣкаютъ е, и во огнь вметаютъ}\footnote{Матѳ.~7,~19.}, "--- и что апостолъ: \textit{аще по плоти живете, имате умрети}\footnote{Римл.~8,~13.}. Древо есть человѣкъ, духовно разсуждаемый, который, хотя имя Хрістово исповѣдуетъ, и тѣмъ исповѣданіемъ, якоже листвіемъ древо, красится и величается, но исповѣданія своего достойныхъ плодовъ, то"=есть, добрыхъ дѣлъ, не творитъ: почему не иному чему, какъ вѣчному геенскому огню, якоже древо безплодное на сожженіе, праведнымъ судомъ Божіимъ предастся. Почему и апостолъ глаголетъ: \textit{аще по плоти живете}, то"=есть, плотскимъ похотямъ послѣдуете, \textit{имате умрети}, "--- гдѣ смерть не временная, но вѣчная разумѣется, ибо и праведные и грѣшные временною смертію умираютъ: но тіи \textit{воскреснутъ въ воскрешеніе живота} (вѣчнаго), а сіи \textit{въ воскрешеніе суда} (вѣчнаго осужденія)\footnote{Іоан.~5,~29.}. Сей страшный конецъ да устрашитъ насъ отъ злыхъ и горькихъ плодовъ грѣховныхъ, которые вѣчному огню предаютъ, и да подвигнетъ къ покаянію и плодамъ его сладкимъ. "--- Но когда и на временное бѣдствіе посмотримъ, увидимъ, что не отъинуды, какъ отъ беззаконнаго и нераскаяннаго житія человѣческаго происходитъ. Отсюду пожары, глады, оскудѣнія хлѣба и прочіихъ житейскихъ благихъ; отсюду моровыя язвы, трясенія земли; отсюду брани, ужасныя сраженія, кровопролитія, многотысящныхъ людей паденія, и отъ того послѣдующіе женъ, родителей, дѣтей плачевные вопли, разоренія и опустошенія градовъ и государствъ, и проч. Тако правда Божія, беззаконіями человѣческими раздраженная, посылаетъ страшный суда Своего мечъ и поражаетъ беззаконниковъ! Да убоимся убо страшнаго меча сего, и \textit{сотворимъ плоды достойные покаянія}\footnote{Матѳ.~3,~8; Лук.~3,~8.}. "--- 7)~Апостолъ глаголетъ: \textit{благочестіе на все полезно есть, обѣтованіе имѣюще живота нынѣшняго и грядущаго}\footnote{1~Тим.~4,~8.}. Благочестіе Богъ обѣщаетъ милостивно и благодатно не токмо въ будущемъ, но и въ нынѣшнемъ вѣцѣ наградить Своимъ благословеніемъ. \textit{Аще въ повелѣніихъ Моихъ ходите, и заповѣди Моя сохраните, и сотворите я: и дамъ дождь вамъ во время свое, и земля дастъ плоды своя, и древеса сельная отдадятъ плодъ свой; и постигнетъ вамъ млаченіе житъ, обраніе вина, и обраніе вина постигнетъ сѣятву, и снѣсте хлѣбъ вашъ въ сытость, и вселитеся съ твердостію на земли вашей, и рать не пройдетъ сквозѣ землю вашу; и дамъ миръ въ земли вашей, и уснете, и не будетъ устрашаяй васъ: и погублю звѣри лютыя отъ земли вашея}, и проч.\footnote{Лев.~26,~3--6 и слѣд.} Изрядно представляетъ очесамъ нашимъ псаломъ 36"~й благословеніе Божіе, которое благочестивымъ подается: якоже нечестивые хотя и возносятся и высятся, яко кедры Ливанстіи, однакожъ такъ исчезаютъ, что и мѣсто ихъ не обрѣтается, хотя и поищеши. Благочестіе и въ день смерти благочестивому помогаетъ. \textit{Боящемуся Господа благо будетъ, и въ день скончанія своего обрящетъ благодать}\footnote{Сир.~1,~13.}. Въ будущемъ же вѣцѣ \textit{вся благая, ихже око не видѣ, и ухо не слыша, и на сердце человѣку не взыдоша, любящихъ Бога} ожидаютъ, по неложному обѣщанію Его\footnote{1~Кор.~2,~9.}. Однакожъ сіе не такъ должно разумѣть, что аки бы дѣла наши заслужили что у Бога; понеже и дѣла наши добрыя Богу должно приписывать, Который въ насъ благодатію Своею дѣйствуетъ тая; наши же называются того ради, что мы благодати Божіей \textit{зовущей и дѣйствующей} повинуемся, и, тою благодатію просвѣщаеми и укрѣпляеми, дѣянія плотская умерщвляемъ, и плоды вѣры, яже есть даръ Божій, творимъ, боимся и любимъ Бога, и ближнему дѣла любве показуемъ. Богъ бо никому не долженъ, и, что намъ благое даетъ, \textit{туне даетъ}, по единой Своей благости. Того ради, какъ обѣщаетъ благая Своя, которымъ обѣщаніемъ возбуждаетъ вѣру въ насъ и охоту къ благочестію, такъ и обѣщанная послушавшимъ Его и вѣрующимъ даетъ \textit{туне}, по единой Своей благости. Надобно помнить въ семъ случаѣ намъ Хрістово слово: \textit{егда сотворите вся повелѣнная вамъ, глаголите, яко раби неключими есмы: яко, еже должни бѣхомъ сотворити, сотворихомъ}\footnote{Лук.~17,~10.}. Но кто есть, который бы вся повелѣнная исполнилъ? Единъ Хрістосъ, исполненіе закона и пророковъ, вся исполнилъ и заслужилъ всѣмъ благодать и славу у Бога, небеснаго Своего Отца. Тому слава и похвала подобаетъ, намъ же достоитъ всегда признаватися со смиреніемъ: \textit{Тебѣ, Господи, правда, намъ же стыдѣніе лица}\footnote{Дан.~9,~7.}, и молитися: \textit{не намъ, Господи, не намъ, но имени Твоему даждь славу о милости Твоей и истинѣ Твоей}\footnote{Пс.~113,~9.}. Однакожъ отсюду не должно заключать по плотскому мудрованію, что когда туне единою вѣрою, а не добрыми дѣлами, получаются благая отъ Бога, то непотребны и дѣла добрыя. Мнѣніе такое погрѣшительно. Кто такое мнѣніе имѣетъ, въ томъ и вѣры нѣтъ, понеже гдѣ добрыхъ дѣлъ нѣтъ, тамо и вѣры нѣтъ, которая отъ добрыхъ дѣлъ, какъ древо отъ плодовъ, познается (якоже сказано подъ числомъ 4"~мъ сего параграфа, и ниже во второй книгѣ предложится). Того ради должно имѣть вѣру, и тую добрыми дѣлами засвидѣтельствовать, да вѣрою получимъ Божіе обѣщаніе. "--- 8)~Требуетъ того братская хрістіанская любовь, дабы мы ближнему нашему въ нуждахъ его помогали, какъ сказано о томъ въ главахъ о любви, милости и прочіихъ, и духовно его созидали примѣромъ хрістіанскаго житія и благочиннаго обращенія. Сіе глаголется паки не на такій конецъ, дабы мы добрыя дѣла дѣлали того ради, \textit{да видимы будемъ отъ человѣкъ} (сей конецъ и намѣреніе, яко фарисейское лицемѣріе, святымъ Божіимъ словомъ охуждается), но что доброе единаго житіе можетъ другому подать случай или къ ревнительнѣйшему въ хрістіанскомъ дѣлѣ подвигу, или къ востанію отъ унынія и пренемоганія изнемогающимъ. Ибо истинное благочестіе есть какъ свѣтъ, въ нощи міра сего сіяющій, который хотя и таится, однакожъ сокрытися не можетъ, но издалеча видится. Симъ благочестиваго житія примѣромъ не менѣе люди созидаются, какъ и словомъ. И хотя сей благочестія примѣръ необходимо нуженъ родителямъ ради дѣтей, пастырямъ, то есть, епископамъ и іереямъ ради словесныхъ Хрістовыхъ овецъ, которыхъ словомъ Божіимъ пасутъ, властямъ ради подвластныхъ, учителямъ ради учениковъ, старымъ всѣмъ ради юныхъ; однакожъ и всѣмъ хрістіанамъ не должно о томъ пренебрегать, но другъ другу въ подвигѣ хрістіанскомъ противу плоти и діавола примѣромъ своимъ помогать и поощрять. Коль силенъ добраго обращенія примѣръ, видимъ изъ исторіи о древнихъ хрістіанахъ. Во время гоненій на хрістіанъ, какъ хрістіане другъ другу послѣдовали въ небоязненномъ предъ мучителями исповѣданіи имене Хрістова, и какъ другъ друга взаимнымъ примѣромъ поощряли къ страданію за Хріста, "--- церковная исторія свидѣтельствуетъ, такъ что не токмо мужи и жены, совершенные въ возрастѣ, но и самые отроки и отроковицы, и самый почти младенческій вѣкъ (возрастъ) усердно за Хріста подклоняли свои главы подъ мечь. И хотя благочестивая оная ревность была даръ Божій, однакожъ спасительный тотъ огонь силу и дѣйствіе свое большее воспріялъ отъ взаимнаго примѣра, не иначе, какъ естественный возженный огонь въ большій пламень обращается, когда ему другій приложится и присоединится огонь и отъ вѣтра раздувается. "--- 9)~Къ благочестивому житію и благочинному обращенію долженъ насъ возбуждать злый конецъ непостояннаго житія, то"=есть, соблазнъ. Добрыхъ дѣлъ правило есть Божіе слово. \textit{Свѣтильникъ ногама моима законъ Твой, Господи, и свѣтъ стезямъ моимъ}, глаголетъ Псаломникъ\footnote{Пс.~118,~105.}. Сей свѣтильникъ свѣтитъ намъ, во тмѣ міра сего ходящимъ, и показуетъ, какъ ходить, жить и обращаться, что дѣлать, и отъ чего уклоняться. Почему соблазнъ двоякимъ наипаче образомъ бываетъ: \textit{первое}, развращеннымъ и слову Божію противнымъ ученіемъ, какъ то дѣлаютъ еретики и прочіи суевѣры; \textit{второе}, подаетъ соблазнъ и тотъ, который противно слову Божію живетъ и развращенно между братіею своею обходится. Ибо какъ добрый, такъ и злый примѣръ единаго ударяетъ въ сердце, другаго чрезъ слухъ, или видѣніе; оный къ добру, а сей къ злу побуждаетъ видящаго, или слышащаго. Соблазнъ подобенъ язвѣ моровой, которая, заченшися въ одномъ человѣкѣ, многихъ близъ живущихъ заражаетъ и умерщвляетъ: такъ и развращенное ученіе въ единомъ ересеначальникѣ зачинается, но безчисленный народъ поражаетъ и погубляетъ; такъ и развращенное единое житіе часто подаетъ случай многимъ беззаконновать, и тое, что видѣли или слышали, дѣлать, и не иначе, какъ домъ отъ сосѣдняго дома горящаго загорается, беззаконнымъ нечестиваго примѣра пламенемъ разжигаться и загараться. Отсюду видимъ, что иный градъ и село тѣмъ, иный другимъ беззаконіемъ изобилуетъ паче другаго; инымъ мамона и сребролюбіе наипаче обладаетъ; въ иномъ роскошь и всякое плотское угодіе преизобильно разливается, въ другомъ гордость житейская паче прочіихъ себе оказываетъ. Тоежъ примѣчается по домамъ и фамиліямъ: ибо единъ отъ другаго видитъ, или слышитъ зло и перенимаетъ, и такъ развращается; а отъ того другій тогожде научается, и такъ расширяется смрадный беззаконнаго житія запахъ и поядаетъ души, по подобію язвы усилившейся. Хотя и всякъ другъ другу можетъ подать соблазнъ, однакожъ наипаче подаютъ родители дѣтямъ, пастыри людямъ, начальники подвластнымъ, господа рабамъ своимъ, учители ученикамъ, старые младымъ. Ибо высшихъ житіе и примѣръ низшимъ есть какъ правило, или какъ зерцало, въ которое смотрятъ, и по тому себе исправляютъ. Къ разсѣянію соблазновъ много служатъ и помогаютъ клеветники, которые разсѣваютъ чужіе грѣхи и беззаконія, и тѣми слухи слушателей своихъ наполняютъ. \textit{Горе міру отъ соблазнъ}, глаголетъ Хрістосъ\footnote{Матѳ.~18,~7.}. \textit{Горе же тому, егоже ради приходитъ! Уне ему было бы аще жерновъ осельскій облежалъ бы о выи его, и вверженъ въ море, неже да соблазнитъ малыхъ сихъ единаго}\footnote{Лук.~17,~1 и 2.}. Изъ сего видимъ, коль великое зло соблазнъ, который, какъ язва, разливается и многихъ умерщвляетъ. Сіе зло да убѣдитъ насъ уклоняться отъ беззаконія и тщаться о добродѣтельномъ житіи, дабы насъ тое \textit{горе}, которое соблазняющимъ, по словеси Хрістову, слѣдуетъ, не постигло. "--- 10)~О добродѣтельномъ житіи и того ради должно намъ тщаться, да заградимъ уста противникамъ нашимъ и врагамъ, которые изощряютъ ядовитый языкъ свой на поношеніе и злословіе наше, и назираютъ, да обрящутъ рѣчь на насъ: тако они, увидѣвше житіе наше не таково, каково мнятъ и разносятъ, посрамятся и заградятъ уста свои и умолкнутъ. И хотя свойство клеветниковъ есть и тамо находить причину порицанія и злословія, гдѣ нѣтъ причины; однакожъ клевета ихъ, яко ложная и неправедная, тщетна будетъ, егда \textit{изведетъ Господь, яко свѣтъ, правду неповинныхъ, и судьбу ихъ яко полудне}, и тогда \textit{мечъ ихъ}, который противу неповинныхъ извлекаютъ, \textit{внидетъ въ сердца ихъ}\footnote{Пс.~36,~6,~14 и 15.}. "--- 11)~Житіе наше благочестиво должно проводить намъ на такій конецъ, да, видѣвше житіе наше доброе, незнающіи и неисповѣдающіи имени Хрістова обратятся ко Хрісту, и съ нами едино будутъ, \textit{и прославятъ Отца нашего, Иже есть на небесѣхъ}\footnote{Матѳ.~5,~16.}. Тако древніе хрістіане многихъ пріобрѣтали Хрісту, какъ читаемъ въ исторіи церковной.

\paragraph*{§\:272.} Разсужденіе о добрыхъ дѣлахъ подаетъ случай воспомянуть о древнихъ хрістіанахъ (о которыхъ мало въ параграфѣ прешедшемъ сказано) и о состояніи нынѣшняго вѣка хрістіанъ, на такій конецъ, дабы представить всякому предъ умныя очи, коль далеко нынѣшніи хрістіане отъ хрістіанской должности отступили. Сіе можетъ читатель примѣтить и отъ разсужденія, въ которомъ предлагалось о плодахъ вѣры, о послушаніи усердномъ Богу, о прославленіи имени Его, о благодарности къ Нему, и о прочемъ. На такій бо конецъ и предложено оно, да \textit{искушаетъ себе всякъ, аще есть въ вѣрѣ}\footnote{2~Кор.~13,~5.}. Въ Апокалипсисѣ глаголетъ Хрістосъ ангелу (епископу) Ефесскія церкви: \textit{имамъ на тя, яко любовь твою первую оставилъ еси. Помяни убо, откуду спалъ еси, и покайся; и первая дѣла сотвори. Аще ли ни, гряду ти скоро}, и прочая\footnote{2,~4 и 5.}. Сіе увѣщательное Хріста Господа нашего слово хрістіанамъ нынѣшняго вѣка приличествуетъ и ихъ касается, \textit{яко любовь}, которую первые оные хрістіане имѣли, \textit{оставили}. У древнихъ хрістіанъ сердце и душа бѣ едина, какъ апостолъ свидѣтельствуетъ: \textit{народу вѣровавшему бѣ сердце и душа едина}\footnote{Дѣян.~4,~32.}, то"=есть, былъ между ними миръ, согласіе и любовь, всѣхъ благихъ мати. Сіе есть знакъ \textit{истиннаго хрістіанства}, какъ о семъ блаженномъ согласіи пророкъ предвозвѣстилъ: \textit{раскуютъ мечи своя на орала, и копія своя на серпы, и не возметъ языкъ на языкъ меча, и не навыкнутъ ктому ратоватися}\footnote{Ис.~2,~4.}; "--- и Хрістосъ глаголетъ: \textit{о семъ разумѣютъ вси, яко Мои ученицы есте, аще любовь имате между собою}\footnote{Іоан.~13,~35.}. У нынѣшнихъ того нѣтъ, но сердце противу сердца и душа противу души востаетъ; другъ на друга умышляютъ злая въ сердцахъ своихъ и душахъ. А что въ сердцахъ крыется, тое и внѣ является чрезъ ссоры, проклинанія, злословія, клеветы, ябеды коварныя, брани, біенія и прочіе злые плоды. Истины сея не токмо всякій градъ, село, мѣсто и страна, но и всякій почти домъ свидѣтель есть. Едва бо сыщемъ какій домъ, въ которомъ бы жалобы другъ на друга не происходило. Но, что дивнѣе, или паче сожалительнѣе, братъ съ братомъ, сестра съ сестрою, мужъ съ женою, которыхъ естественная любовь должна соединять, въ несогласіи и ссорѣ пребываютъ. "--- У древнихъ хрістіанъ \textit{не былъ нищъ ни единъ въ нихъ}\footnote{Дѣян.~4,~34.}, понеже были человѣколюбивы, милостивы и щедры, подражая Отцу своему, Иже есть на небесѣхъ. У нынѣшнихъ не тако; но нищими наполнены улицы, домы; наполнены темницы обнищавшихъ должниковъ, сидящихъ за долги, недоимки оброчныя, за вексели и прочія нужды; многіе скитаются отъ пожарныхъ случаевъ безъ домовъ съ женами и малыми дѣтьми; многіе лишаются дневныя пищи, многіе ходятъ въ рубищахъ, многіе полунаги. Причина тому самолюбіе и скупость богатыхъ, которые сами хотятъ довольно и преизлишно насыщаться, упокоеваться, украшаться шелками и виссонами, жить въ богатыхъ палатахъ, представлять богатые столы, пить дорогія вина, довольно имѣть слугъ, проѣзжаться цугами на избранныхъ коняхъ, держать собачью охоту, играть въ карты и прочія забавы; а о своей, подобной себѣ братіи, бѣдствующей небрегутъ и смотрѣть не хотятъ. "--- Древніи хрістіане продавали села и домы, и цѣну продаемыхъ приносили во едино на общую *вѣрныхъ* пользу\footnote{4,~34--37 и 32.}. Нынѣшніи не тако; но другъ у друга отнимаютъ села, земли, поля, нивы, суды и прочія житейскія потребности; изгоняютъ изъ домовъ, проливаютъ слезы бѣдныхъ, вдовицъ, сиротъ и прочіихъ беззаступныхъ людей. "--- Древніи хрістіане были простосердечны, другъ друга не обманывали, но истину другъ ко другу объявляли, и потому другъ другу вѣрили. Нынѣшніи не тако; самая вещь показуетъ противное, когда другъ другу льстятъ, ласкаютъ, но въ сердцахъ помышляютъ злая. Языкомъ глаголютъ, какъ Іуда Хрісту: \textit{радуйся}! а въ сердцѣ на преданіе и убіеніе поучаются. Устами ласкаютъ: \textit{братецъ}, а въ сердцѣ какъ врага имѣютъ. Словомъ обѣщаютъ, а сердцемъ иное думаютъ. "--- У древнихъ хрістіанъ цѣломудріе и воздержаніе процвѣтало. Нынѣ не тако: піянство, роскошь, похоть плотская, и симъ послѣдующія неподобныя игры, пѣсни, кличи, вездѣ почти оказываются. Древніи хрістіане были кротки, другъ другу прощали отъ сердца погрѣшности. У нынѣшнихъ не то; но за противное слово, или малую обиду, хотятъ обидѣвшаго искоренить. Свидѣтельствуютъ о семъ судебныя мѣста, которыя язвительными другъ на друга клеветами преисполнены. "--- У древнихъ хрістіанъ коликое усердіе было къ славословію Божію, къ молитвѣ, къ слушанію Божія слова, отсюду видно, что пока не имѣли храмовъ Божіихъ молитвенныхъ ради гоненія, собирались по домамъ ради того богоугоднаго дѣла, и тако домы ихъ были какъ церкви; инымъ пустыни и вертепы были молитвенные домы. Нынѣ не тако, но хотя храмы и суть, и всегда отверзаются, и всегда звономъ колокольнымъ, какъ гласомъ благопріятнымъ, церковь зоветъ хрістіанъ на славословіе Божіе и молитву общую, однакожъ мало кто спѣшитъ. Многіе хотя и приходятъ, но ради церемоніи, а иные, что горше того, на единый соблазнъ, со смѣхами, разговорами, и прочіими непотребными забавами. Многіе изволяютъ по улицамъ скитаться, по домамъ питейнымъ и шинкамъ шумѣть и безчинствовать, нежели на славословіе Божіе и молитву въ храмы итить. О словѣ Божіемъ и объявлять нечего, которое хотя мало проповѣдуется и рѣдко, однакожъ, какъ насыщенные, или паче, какъ лихорадкою немоществующіи пищу пріемлютъ, премногіе изволяютъ забавными и соблазнительными книжками утѣшать *и увеселять* плоть свою, нежели отъ слова Божія пользы душевной, на что оно дано, искать. "--- У древнихъ хрістіанъ какая любовь и усердіе ко Хрісту было, оттуду явно, что мужи, жены, отроки, отроковицы и недорослый юношескій вѣкъ съ радостію предавали себе на смерть за имя Его: нынѣ того нѣтъ. Оныхъ мучители "--- цари и князи жесточайшими орудіями не могли отвратить отъ Хріста: нынѣшнихъ сребро, злато, похоть плотская, роскошь, пьянство и прочія страсти отвращаютъ. "--- Древніи хрістіане часто пріобщались Хрістовыхъ Таинъ, яко безсмертія виновныя пищи; откуду и нынѣ повседневно призываетъ церковь святая: \textit{со страхомъ Божіимъ и вѣрою приступите}! У нынѣшнихъ нѣтъ того, какъ самое дѣло показуетъ; но въ годъ единожды, и то почти съ принужденіемъ, приступаютъ къ безсмертной той трапезѣ, а многіе до болѣзни, иные до самой почти смерти отлагаютъ. Жалости достойная вещь! На банкеты съ радостію спѣшатъ люди, но къ духовной оной и святѣйшей трапезѣ, къ которой Хрістосъ призываетъ, почти съ нуждою идутъ! "--- У древнихъ хрістіанъ страннолюбіе такъ сильно было, что другъ друга предваряли въ пріятіи странныхъ, послѣдуя вѣрѣ и любви праотца своего Авраама. У нынѣшнихъ не видно того; но или совсѣмъ не отворяютъ дверей странствующимъ, или отворяютъ, но съ договоромъ мзды. "--- У древнихъ хрістіанъ лжи, лукавства не слышно было: у нынѣшнихъ солгать, обмануть, прельстить за грѣхъ не почитается. "--- Древнихъ хрістіанъ пастыри какое попеченіе и тщаніе о стадѣ Хрістовомъ имѣли, дивиться должно: о нынѣшнихъ не говорю; вси о томъ знаютъ (пастырей разумѣю епископовъ и священниковъ). "--- Древніи хрістіане, засѣдающіи въ судебныхъ мѣстахъ, какъ усердно наблюдали правду, и отирали бѣдныхъ слезы! Нынѣ не слышится тое: но вмѣсто того клятвопреступленіе, попраніе правды, или паче, выгнана она вонъ, и судебныя мѣста торжищами и корчемницами сдѣлались: вмѣсто Божія суда купля дѣлается; вмѣсто Бога мамона почитается, и вмѣсто правды злато и сребро голосъ свой издаетъ. Всякъ сію истину узнаетъ, когда безъ денегъ къ суду пріидетъ. "--- Древніи хрістіане дѣтей своихъ обучали въ законѣ Господни, страсѣ Божіи и всякомъ благочестіи: у нынѣшнихъ нѣтъ того; но вмѣсто того обучаютъ дѣтей танцамъ, политикѣ свѣтской и прочей суетѣ. Откуду бываетъ, что въ отцахъ нѣтъ, а въ дѣтяхъ и внукахъ и искать благочестія нечего. Тако нынѣшніи хрістіане первыхъ любовь оставили, и съ любовію вѣру и благочестіе истинное. И хотя многіе отъ хрістіанъ нынѣшнихъ созидаютъ храмы Божіи, какъ видимъ, каменные или деревянные, но одушевленные разоряютъ. Украшаютъ Евангеліе златомъ и сребромъ, но, что Евангеліе учитъ, того и перстомъ коснуться не хотятъ, или паче, и не знаютъ. Иные даютъ милостыню, но у другихъ отнимаютъ. И какая изъ того польза? Ибо сіе не богоугодное, но богопротивное дѣло есть. И тако въ хрістіанство нынѣшнее языческое состояніе вошло. Ибо хотя не устами, но дѣлами отъ нынѣшнихъ хрістіанъ отвергается Хрістосъ, Котораго древніи хрістіане, и муками принуждаеми, не хотѣли отрещись. \textit{Бога исповѣдуютъ вѣдѣти, а дѣлы отмещутся Его}, глаголетъ апостолъ\footnote{Тит.~1,~16.}; "--- и паки Хрістосъ: \textit{приближаются Мнѣ людіе сіи усты своими, и устнами чтутъ Мя: сердце же ихъ далече отстоитъ отъ Мене}\footnote{Матѳ.~15,~8.}.

\paragraph*{§\:273.} Сего ради внимать намъ должно, что далѣе Хрістосъ Ефесскія церкви епископу глаголетъ: \textit{помяни, откуду спалъ еси, и покайся}. Сей сладкій милосердаго нашего Избавителя гласъ во уши всѣхъ ударяетъ. О, да ударитъ и въ сердца божественною своею крѣпостію, и возбудитъ души, тяжкимъ грѣха сномъ уснувшія! Ударяетъ пастырей: \textit{помяните званіе ваше, и пасите еже въ васъ стадо Божіе} (Мое), \textit{посѣщающе не нуждею, но волею, и по Бозѣ; ниже неправедными прибытки, но усердно; не яко обладающе причту, но образи бывайте стаду}, еже стяжахъ кровію Моею\footnote{1~Петр.~5,~2 и 3.}. Ударяютъ судей: \textit{помяните}, на какомъ сѣдите мѣстѣ; \textit{Азъ рѣхъ: бози есте. Доколѣ судите неправду, и лица грѣшниковъ пріемлете}\footnote{Пс.~81,~6 и 2.}? Ударяетъ родителей: помяните о чадѣхъ вашихъ; \textit{воспитовайте ихъ въ наказаніи и ученіи Господни}\footnote{Еф.~6,~4.}. Ударяетъ всѣхъ вообще и всякаго чина людей: \textit{помяните, откуду спали вы, и покайтеся}! Духъ же Святый во псалмѣ глаголетъ: \textit{днесь, аще гласъ Его услышите, не ожесточите сердецъ вашихъ}\footnote{Пс.~94,~7 и 8.}. \textit{Блюдите, братіе}, глаголетъ апостолъ, \textit{да не когда будетъ въ нѣкоемъ отъ васъ сердце лукаво, исполнено невѣрія, во еже отступити отъ Бога жива. Но утѣшайте себе на всякъ день, дондеже днесь нарицается, да не ожесточится нѣкто отъ васъ лестію грѣховною. Причастницы бо быхомъ Хрісту, аще точію начатокъ состава даже до конца извѣстенъ удержимъ}\footnote{Евр.~3,~12--14.}. Аще же ожесточимъ сердца наша, то послушаемъ еще, что паки глаголетъ Хрістосъ помянутому епископу: \textit{гряду ти скоро}, и прочая. Сіе уже и приходитъ къ намъ различными наказаніями, нашествіемъ иноплеменниковъ, враговъ нашихъ, моровою язвою, частыми пожарами, отъятіемъ благоплодія отъ нивъ нашихъ, болѣзнями и скорбьми и различными напастьми, и какъ бы убѣждаетъ насъ послушати своего гласа и покаятися. Аще же и тако не послушаемъ: \textit{се уготова на судъ престолъ Свой}\footnote{Пс.~9,~8.}. \textit{Се грядетъ Господь во тмахъ святыхъ ангелъ Своихъ, сотворити судъ о всѣхъ и изобличити всѣхъ нечестивыхъ о всѣхъ дѣлѣхъ нечестія ихъ}, и проч.\footnote{Іуд.~1,~14,~15 и слѣд.} \textit{Се}, глаголетъ Самъ, \textit{гряду яко тать: блаженъ бдяй и блюдый ризы своя, да Не нагъ ходитъ, и узрятъ срамоту его}\footnote{Апок.~16,~15.}. \textit{Якоже бо бѣху во дни прежде потопа, ядуще и піюще, женящеся и посягающе, до негоже дне вниде Ное въ ковчегъ, и не увѣдѣша, дондеже пріиде вода и взятъ вся: тако будетъ и пришествіе Сына человѣческаго}\footnote{Матѳ.~24,~38 и 39.}. Сего ради пріидите, пока день оный, страшный грѣшникамъ, не пріидетъ. Пріидите вси. Возвратимся къ пастырю и посѣтителю душъ нашихъ, Хрісту Сыну Божію. Зоветъ Онъ насъ, какъ слышали мы, и ожидаетъ; готовъ святѣйшими любве Своея объятіями пріяти всѣхъ, какъ отецъ блуднаго сына. Никого да не смущаетъ множество и величество грѣховъ! Всѣхъ Онъ, малыхъ и великихъ, грѣшниковъ призываетъ, и обѣщаетъ вѣчный душамъ покой: \textit{пріидите ко мнѣ вси труждающіися и обремененніи, и Азъ упокою вы}\footnote{11,~28.}. Никто да не усумнѣвается о превосходящей сей милости и человѣколюбіи Его! Евангеліе Его святое и въ немъ человѣколюбіе Его изображенное, которое грѣшникамъ, въ святѣйшей плоти странствуя на земли, показалъ, крѣпко увѣряетъ насъ. Какъ усердно привлекаетъ жену Самаряныню, блудницу бывшую! Обѣщаетъ ей \textit{источникъ воды текущія въ животъ вѣчный}\footnote{Іоан.~4,~14.}. Какъ благопріятнымъ лицемъ зритъ на грѣшницу, омывающую слезами нозѣ Его, и любезный и утѣшительный гласъ испущаетъ ей: \textit{отпущаются тебѣ грѣси}\footnote{Лук.~7,~48.}. Какъ человѣколюбно разрѣшаетъ прелюбодѣйцу! \textit{Ни Азъ тебе осуждаю; иди, и отселѣ ктому не согрѣшай}\footnote{Іоан.~8,~11.}. Какъ благосклонно и милостиво пріемлетъ всѣхъ приближающихся Ему мытарей и грѣшниковъ\footnote{Лук.~15,~1.}. Какъ живо изображаетъ Свое тщательное и усердное ко взысканію грѣшниковъ милосердіе во образѣ хозяина, который, \textit{имый сто овецъ, и погубль едину отъ нихъ, не оставитъ ли девятидесяти и девяти въ пустыни, и идетъ въ слѣдъ погибшія, дондеже обрящетъ ю? И обрѣтъ возлагаетъ на рамѣ свои радуяся}, и прочая\footnote{Таможде ст.~4,~5 и слѣд.}. Какъ милостивый Свой взоръ въ самомъ страданіи обращаетъ на ученика отвергшагося\footnote{Лук.~22,~61.}, и со взоромъ милость и благодать Свою подаетъ ему въ утѣшеніе и укрѣпленіе его, \textit{да не оскудѣетъ вѣра его}\footnote{ст.~32.}! Какъ милостивую руку помощи, уже на крестѣ висящъ, подаетъ разбойнику, и почти изъ самыхъ адовыхъ вратъ восхищаетъ его, и отверзаетъ двери небеснаго царствія! \textit{Аминь глаголю тебѣ, днесь со Мною будеши въ раи}\footnote{23,~43.}. Како наконецъ съ высоты славы Своея милостивымъ и человѣколюбнымъ гласомъ ударяетъ во уши и сердце Савла, бывшаго хульника и гонителя и досадителя! \textit{Савле, Савле, что Мя гониши}\footnote{Дѣян.~9,~4.}? И изъ гонителя проповѣдникомъ имене Своего, и всея вселенныя учителемъ дѣлаетъ\footnote{1~Тим.~1,~13.}. Такъ кротко, такъ милостиво, такъ человѣколюбиво поступаетъ со грѣшниками Господь нашъ! \textit{Яко хотѣніемъ не хощетъ смерти грѣшника, но обратитися ему отъ пути зла, и живу быти ему}\footnote{Іез.~18,~23.}. Когда убо такъ \textit{милосердаго} имѣемъ Господа нашего: пріидите, отложимъ темный противныхъ помышленій облакъ, который покрываетъ сердца наша, и, \textit{отложивше дѣла темная, облечемся во оружіе свѣта}\footnote{Римл.~13,12.}, воспомянемъ, \textit{откуду спали мы}, и воспомянувше восплачемся предъ Господемъ, сотворшимъ насъ и искупившимъ кровію Своею, \textit{яко Той есть Богъ нашъ, и мы людіе Его и овцы руки Его}\footnote{Пс.~94,~7.}. Возвысимъ вси единодушно отъ сердца гласъ нашъ: \textit{скоро да предварятъ ны щедроты Твоя, Господи, яко обнищахомъ зѣло! Помози намъ, Боже Спасителю нашъ, славы ради имене Твоего. Господи, избави ны, и очисти грѣхи наша, имене ради Твоего}\footnote{78,~8--10.}. И паки: \textit{Господи Боже силъ! обрати ны, и просвѣти лице Твое и спасемся}\footnote{79,~8.}. И паки: \textit{возврати насъ, Боже спасеній нашихъ, и отврати ярость Твою отъ насъ. Еда во вѣки прогнѣваешися на ны? или простреши гнѣвъ Твой отъ рода въ родъ? Боже! Ты обращся оживиши ны, и людіе Твои возвеселятся о Тебѣ. Яви намъ, Господи, милость Твою, и спасеніе Твое даждь намъ}\footnote{84,~5--8.}. И паки: \textit{услыши ны, Боже Спасителю нашъ, упованіе всѣхъ концевъ земли, и сущихъ въ мори далече}\footnote{64,~6.}. И паки: \textit{помяни насъ, Господи, во благоволеніи людей Твоихъ, посѣти насъ спасеніемъ Твоимъ, видѣти во благости избранныя Твоя, возвеселитися въ веселіи языка Твоего, хвалитися съ достояніемъ Твоимъ. Согрѣшихомъ со отцы нашими, беззаконновахомъ, неправдовахомъ}\footnote{105,~4--6.}.

Сіе мое разсужденіе не касается тѣхъ, которые благодатію Божіею состоятъ неподвижно въ вѣрѣ, и вѣры плоды показуютъ. Знаю я и твердо держу, что имѣетъ Господь Своихъ рабовъ почитающихъ Его, и знаетъ ихъ. \textit{Твердое основаніе Божіе стоитъ, имущее печать сію: позна Господь сущія Его}, глаголетъ апостолъ\footnote{2~Тим.~2,~19.}. \textit{Онъ знаетъ Своихъ, и Его знаютъ сущіи Его}\footnote{Іоан.~10,~14.}. Тѣмъ желаю въ той благодати Божіей пребывать, и день отъ дне успѣвать, отъ силы въ силу приходить, и до конца въ томъ подвигѣ подвизаться, и своей братіи, забывшей и забывающей хрістіанскую должность, усердною молитвою помоществовать. Таковымъ глаголетъ Хрістосъ: \textit{буди вѣренъ даже до смерти, и дамъ ти вѣнецъ живота}\footnote{Апок.~2,~10.}. А только здѣсь слово мое до тѣхъ надлежитъ, которые благочестіемъ хвалятся, но силы его, которая состоитъ въ вѣрѣ, страсѣ и любви Божіей, \textit{отвергаются}\footnote{2~Тим.~3,~5.}, \textit{Бога исповѣдуютъ вѣдѣти, но дѣлы отмещутся Его}\footnote{Тит.~1,~16.}, на языкѣ вѣру, а въ сердцѣ невѣріе и безбожіе имѣютъ\footnote{Іак.~2,~14.}.

\section[Заключеніе 2-е. О постоянствѣ и конечномъ пребываніи въ вѣрѣ.]{заключеніе второе:\\\bfseries О постоянствѣ и конечномъ пребываніи въ вѣрѣ.}

\begin{quotation}\textit{Претерпѣвый до конца, той спасенъ будетъ}, глаголетъ Хрістосъ\footnote{Матѳ.~10,~22.}.\end{quotation}
\begin{quotation}\textit{Никтоже возложь руку свою на рало и зря вспять, управленъ есть въ царствіи Божіи}, глаголетъ Хрістосъ\footnote{Лук.~9,~62.}\end{quotation}
\begin{quotation}\textit{Буди вѣренъ даже до смерти, и дамъ ти вѣнецъ живота}\footnote{Апок.~2,~10.}.\end{quotation}
\begin{quotation}\textit{Аще совратится праведникъ отъ правды своея, и сотворитъ неправду, по всѣмъ беззаконіямъ, яже сотворилъ беззаконникъ, вся правды его, яже сотворилъ есть, не помянутся: въ преступленіи своемъ, имъже преступи, и во грѣсѣхъ своихъ, имиже согрѣши, въ нихъ умретъ}\footnote{Іез.~18,~24.}.\end{quotation}
\begin{quotation}\textit{Не вѣсте ли, яко текущіи въ позорищи, вси убо текутъ, единъ же пріемлетъ почесть? Тако тецыте, да постигнете}\footnote{1~Кор.~9,~24.}.\end{quotation}


\paragraph*{§\:274.} Не довольно ко спасенію начать благочестіе, но должно и кончать житіе въ подвигѣ благочестія. Многіе начинаютъ, но не вси кончаютъ, какъ самая показываетъ вещь. Таковый подобенъ есть наченшему путь отъ единаго мѣста къ другому, и взадъ паки съ пути возвращающемуся; подобенъ наченшему строить храмину, но оставляющему безъ совершенія; подобенъ есть наченшему созидати столпъ, и не совершающему. Таковый отдаетъ себе въ посмѣяніе и поруганіе врагамъ своимъ, діаволу и слугамъ его, которые, видѣвше, \textit{начнутъ ругатися ему, глаголюще: яко сей человѣкъ начатъ здати, и не може совершити}\footnote{Лук.~14,~28--30.}. Откуду святое Божіе слово тѣмъ только вѣрнымъ приписуетъ спасеніе, которые до конца въ вѣрѣ пребываютъ, \textit{до конца претерпѣваютъ}\footnote{Матѳ.~10,~22.}; которые \textit{начатокъ состава даже до конца извѣстенъ удерживаютъ}\footnote{Евр.~3,~14.}, \textit{не возвращаются вспять отъ преданныя святыя заповѣди}\footnote{2~Петр.~2,~21.}, до конца \textit{подвигомъ добрымъ подвизаться будутъ, теченіе скончаютъ, вѣру соблюдутъ}\footnote{2~Тим.~4,~7.}. Симъ \textit{соблюдается вѣнецъ правды}\footnote{ст.~8.}. Сего ради наченшему добрѣ, добрѣ должно и окончать, чтобы обѣтованіе Божіе "--- вѣчный животъ получить.

\paragraph*{§\:275.} Препятствія къ совершенному спасенію и полученію вѣчнаго блаженства. 1)~Сатана, врагъ душъ нашихъ, со ангелами злыми различныя подлагаетъ намъ козни. \textit{Нѣсть наша брань къ крови и плоти, но къ началомъ, и ко властемъ, и къ міродержителемъ, тмы вѣка сего, къ духовомъ злобы поднебеснымъ}, глаголетъ апостолъ\footnote{Еф.~6,~12.}. И паки: \textit{супостатъ вашъ діаволъ, яко левъ рыкая, ходитъ, искій кого поглотити}, глаголетъ Петръ святый\footnote{1~Петр.~5,~8.}. "--- 2)~Плоть наша со страстьми и похотьми возстаетъ противу насъ: \textit{плоть бо похотствуетъ на духа}, глаголетъ апостолъ\footnote{Гал.~5,~17.}. \textit{Чувствуемъ инъ законъ во удѣхъ нашихъ, противу воюющъ закону ума нашего, и плѣняющъ насъ закономъ грѣховнымъ, сущимъ во удѣхъ нашихъ}\footnote{Римл.~7,~23.}. "--- 3)~Многія прелести и соблазны міра сего, которыми немощная плоть наша разжизается и часто біется совѣсть наша, такъ что часто и нехотя принуждаемся видѣти ихъ, или слышати. \textit{Горе міру отъ соблазнъ: нужда бо есть пріити соблазномъ}, глаголетъ Хрістосъ\footnote{Матѳ.~18,~7.}. И чѣмъ болѣе приближается кончина вѣка, тѣмъ болѣе умножается ихъ. \textit{Сіе вѣждь}, глаголетъ апостолъ Павелъ къ Тимоѳею, \textit{яко въ послѣднія дни настанутъ времена люта. Будутъ бо человѣцы самолюбцы, сребролюбцы, величавы, горди, хульницы, родителемъ противящіися, неблагодарни, неправедни, нелюбовни, непримирительни, продерзиви, возносливи, прелагатае, клеветницы, невоздержницы, некротцы, неблаголюбцы, предателе, нагли, напыщени, сластолюбцы паче, нежели боголюбцы}\footnote{2~Тим.~3,~1--5.}. "--- Сіи суть благочестія запинатели! Съ сими врагами хрістіанамъ всегдашняя предлежитъ брань! Противу сихъ вѣрные всегда, днемъ и нощію, бодрымъ духомъ и неусыпнымъ тщаніемъ подвизаться имѣютъ нужду! Сіи супостаты часто отвлекаютъ душу отъ пути благочестивыхъ и при концѣ. Съ плачемъ уподобляетъ святый Василій Великій хрістіанина купцу, который по морю міра сего, волнами искушеній смущаемому, пловетъ, и часто при самомъ пристанищѣ корабль свой разбиваемый видитъ\footnote{Въ словѣ на начало Притчей.}. Сія на такій конецъ здѣ предлагаются, да, хотящіи вѣчное спасеніе получить, оттрясутъ сонъ лѣности, не смотрятъ на начало, но на конецъ, "--- конецъ бо всякое совершаетъ дѣло, "--- да не надѣются на свою силу, благочестіе, труды подъятые, но на Бога, начинающаго и совершающаго. Сея ради причины святое Божіе слово вездѣ насъ къ тщанію, осторожности и бдѣнію поощряетъ. \textit{Поминайте жену Лотову}, глаголетъ Хрістосъ\footnote{Лук.~17,~32.}, которая \textit{озрѣся вспять, и бысть столпъ сланъ}\footnote{Быт.~19,~26.}. \textit{Буди вѣренъ даже до смерти}\footnote{Апок.~2,~10.}. \textit{Мняйся стояти, да блюдется, да не падетъ}\footnote{1~Кор.~10,~12.}. \textit{Со страхомъ и трепетомъ свое спасеніе содѣвайте}\footnote{Филип.~2,~12.}. \textit{Трезвитеся, бодрствуйте}\footnote{1~Петр.~5,~8 и проч.}.

\paragraph*{§\:276.} Тщаніе человѣческое само собою ничего не можетъ; ни начать, ни дѣлать, ни совершить не можетъ человѣкъ безъ Бога. \textit{Безъ Мене не можете творити ничесоже}, глаголетъ Хрістосъ\footnote{Іоан.~15,~5.}. Богъ начинаетъ дѣло спасенія нашего, дѣлаетъ и совершаетъ, когда Ему не противимся и дѣлаемъ тое, что благодать Его дѣйствуетъ. Вездѣ убо нужна есть намъ Божія помощь. Обѣщается же Божія помощь тѣмъ, которые вѣрою просятъ отъ Него: \textit{просите, и дастся вамъ; ищите, и обрящете; толцыте, и отверзется вамъ}\footnote{Лук.~11,~9.}. И которые вѣрою просятъ, ищутъ и толкутъ въ двери небесе, всегда вѣрою просящимъ отверстыя, тѣ по обѣщанію пріемлютъ, обрѣтаютъ тое, и отверзается имъ. \textit{Всякъ бо просяй пріемлетъ, и ищай обрѣтаетъ, и толкущему отверзается}, глаголетъ Хрістосъ\footnote{Лук.~11,~10.}. \textit{Аще бо} человѣцы, \textit{зли суще, умѣютъ даянія благая даяти чадомъ своимъ, кольми паче Отецъ, Иже съ небесе}, естествомъ благъ и источникъ благостыни, \textit{дастъ Духа Святаго просящимъ у Него}\footnote{ст.~13.}. И родители, хотя желаютъ помощи чадамъ своимъ, но часто бываетъ, что не могутъ, ибо не все человѣкъ можетъ что хощетъ. Богъ же не тако, но все можетъ. Все хощетъ подать намъ, яко \textit{благъ}; и все можетъ, яко \textit{всемогущій}. Сего ради должно намъ въ такъ тѣсныхъ обстоятельствахъ, въ такъ трудномъ и опасномъ подвигѣ нашемъ противу враговъ нашихъ не унывать, не отчаяваться, но къ Нему, яко многомилостивому Отцу, поборнику, защитнику и помощнику нашему, прибѣгать съ вѣрою, просить Его, искать у Него, всегда толкать воздыханіемъ и вѣрою въ двери милосердія Его: \textit{да начный дѣло благое въ насъ, совершитъ е, даже до дне Іисусъ Хрістова}\footnote{Филип.~1,~6.}; да \textit{силою Своею божественною соблюдетъ насъ вѣрою во спасеніе, готовое явитися во время послѣднее}\footnote{1~Петр.~1,~5.}; да \textit{призвавый насъ въ вѣчную Свою славу о Хрістѣ Іисусѣ, мало пострадавшія, Той совершитъ насъ, да утвердитъ, да укрѣпитъ, да оснуетъ}\footnote{5,~10.}; да \textit{утвердитъ насъ и сохранитъ отъ лукаваго}\footnote{2~Сол.~3,~3.}; да \textit{не оставитъ насъ искуситися паче, еже можемъ, но сотворитъ со искушеніемъ и избытіе, яко возмощи намъ понести}\footnote{1~Кор.~10,~13.}. «Обычай есть, глаголетъ Василій Великій, плавающимъ по морю на небо смотрѣть, и оттуду плаванія теченіе пріимать, дни отъ солнца, нощію же отъ сѣверныя или иныя какія являющіяся звѣзды, и отъ сихъ всегда правое теченіе плаванія примѣчать. И ты убо, то есть, по морю міра сего плавая, очеса на небо возведи, по примѣру рекшаго: \textit{къ Тебѣ возведохъ очи мои, живущему на небеси}\footnote{Пс.~122,~1.}. Смотри на Солнце правды, то"=есть, Хріста, и, аки звѣздамъ нѣкіимъ свѣтлымъ, заповѣдямъ Господнимъ въ правленіе себе отдай, и бодрствующее око имѣй. \textit{Не даждь сна очима твоима, и вѣждома твоима дреманія}, да всегдашнее возъимѣешь отъ заповѣдей теченіе. \textit{Свѣтильникъ} бо, рече, \textit{ногама моима законъ Твой и свѣтъ стезямъ моимъ}\footnote{Пс.~118,~105.}. Аще бо никогда не будешь дремать при кормилѣ, пока въ семъ житіи находишься: помощь Духа содѣйствующаго получишь, Который тебе всегда далѣе будетъ вести, и легкимъ и тихимъ дыханіемъ даже до спасительнаго, тихаго и веселаго пристанища онаго доведетъ»\footnote{Въ словѣ на начало Притчей.}.

Отцу и Сыну и Святому Духу, Единому Тріѵпостасному Богу, Творцу, Спасителю и Промыслителю нашему, честь и слава во вѣки вѣковъ. Аминь!

\begin{center}\small\textsc{Конецъ втораго тома.}\end{center}
